\setdate{19~января~2022}
\setclass{11«БА»}

\addpersonalvariant{Михаил Бурмистров}

\tasknumber{1}%
\task{%
    Запишите известные вам виды классификации изображений.
}
\solutionspace{60pt}

\tasknumber{2}%
\task{%
    В каких линзах можно получить прямое изображение объекта?
}
\answer{%
    $\text{ собирающие и рассеивающие }$
}
\solutionspace{40pt}

\tasknumber{3}%
\task{%
    Какое изображение называют мнимым?
}
\solutionspace{40pt}

\tasknumber{4}%
\task{%
    Есть две линзы, обозначим их 1 и 2.
    Известно что фокусное расстояние линзы 1 больше, чем у линзы 2.
    Какая линза сильнее преломляет лучи?
}
\answer{%
    $2$
}
\solutionspace{40pt}

\tasknumber{5}%
\task{%
    Предмет находится на расстоянии $20\,\text{см}$ от собирающей линзы с фокусным расстоянием $40\,\text{см}$.
    Определите тип изображения, расстояние между предметом и его изображением, увеличение предмета.
    Сделайте схематичный рисунок (не обязательно в масштабе, но с сохранением свойств линзы и изображения).
}
\solutionspace{100pt}

\tasknumber{6}%
\task{%
    Объект находится на расстоянии $45\,\text{см}$ от линзы, а его мнимое изображение — в $40\,\text{см}$ от неё.
    Определите увеличение предмета, фокусное расстояние линзы, оптическую силу линзы и её тип.
}
\solutionspace{80pt}

\tasknumber{7}%
\task{%
    Известно, что из формулы тонкой линзы $\cbr{\frac 1F = \frac 1a + \frac 1b}$
    и определения увеличения $\cbr{\Gamma_y = \frac ba}$ можно получить выражение
    для увеличения: $\Gamma_y = \frac {aF}{a - F} \cdot \frac 1a = \frac {F}{a - F}.$
    Назовём такое увеличение «поперечным»: поперёк главной оптической оси (поэтому и ${}_y$).
    Получите формулу для «продольного» увеличения $\Gamma_x$ небольшого предмета, находящегося на главной оптической оси.
    Можно ли применить эту формулу для предмета, не лежащего на главной оптической оси, почему?
}
\answer{%
    \begin{align*}
    \frac 1F &= \frac 1a + \frac 1b \implies b = \frac {aF}{a - F} \\
    \frac 1F &= \frac 1{a + x} + \frac 1c \implies c = \frac {(a+x)F}{a + x - F} \\
    x' &= \abs{b - c} = \frac {aF}{a - F} - \frac {(a+x)F}{a + x - F} = F\cbr{\frac {a}{a - F} - \frac {a+x}{a + x - F}} =  \\
    &= F \cdot \frac {a^2 + ax - aF - a^2 - ax + aF + xF}{(a - F)(a + x - F)} = F \cdot \frac {xF}{(a - F)(a + x - F)} \\
    \Gamma_x &= \frac{x'}x = \frac{F^2}{(a - F)(a + x - F)} \to \frac{F^2}{\sqr{a - F}}.
    \\
    &\text{Нельзя: изображение по-разному растянет по осям $x$ и $y$ и понадобится теорема Пифагора}
    \end{align*}
}
\solutionspace{150pt}

\tasknumber{8}%
\task{%
    Доказать формулу тонкой линзы для рассеивающей линзы.
}
\solutionspace{120pt}

\tasknumber{9}%
\task{%
    Постройте ход луча $CL$ в тонкой линзе.
    Известно положение линзы и оба её фокуса (см.
    рис.
    на доске).
    Рассмотрите оба типа линзы, сделав 2 рисунка: собирающую и рассеивающую.
}
\solutionspace{120pt}

\tasknumber{10}%
\task{%
    На экране, расположенном иа расстоянии $60\,\text{см}$ от собирающей линзы,
    получено изображение точечного источника, расположенного на главной оптической оси линзы.
    На какое расстояние переместится изображение на экране,
    если при неподвижном источнике переместить линзу на $3\,\text{см}$ в плоскости, перпендикулярной главной оптической оси?
    Фокусное расстояние линзы равно $20\,\text{см}$.
}
\answer{%
    \begin{align*}
    &\frac 1F = \frac 1a + \frac 1b \implies a = \frac{bF}{b-F} \implies \Gamma = \frac ba = \frac{b-F}F \\
    &y = x \cdot \Gamma = x \cdot \frac{b-F}F \implies d = x + y = 9\,\text{см}.
    \end{align*}
}
\solutionspace{120pt}

\tasknumber{11}%
\task{%
    Оптическая сила двояковыпуклой линзы в воздухе $5{,}5\,\text{дптр}$, а в воде $1{,}5\,\text{дптр}$.
    Определить показатель преломления $n$ материала, из которого изготовлена линза.
    Показатель преломления воды равен $1{,}33$.
}
\answer{%
    \begin{align*}
    D_1 &=\cbr{\frac n{n_1} - 1}\cbr{\frac 1{R_1} + \frac 1{R_2}}, \\
    D_2 &=\cbr{\frac n{n_2} - 1}\cbr{\frac 1{R_1} + \frac 1{R_2}}, \\
    \frac {D_2}{D_1} &=\frac{\frac n{n_2} - 1}{\frac n{n_1} - 1} \implies {D_2}\cbr{\frac n{n_1} - 1} = {D_1}\cbr{\frac n{n_2} - 1}  \implies n\cbr{\frac{D_2}{n_1} - \frac{D_1}{n_2}} = D_2 - D_1, \\
    n &= \frac{D_2 - D_1}{\frac{D_2}{n_1} - \frac{D_1}{n_2}} = \frac{n_1 n_2 (D_2 - D_1)}{D_2n_2 - D_1n_1} \approx 1{,}518.
    \end{align*}
}
\solutionspace{120pt}

\tasknumber{12}%
\task{%
    На каком расстоянии от собирающей линзы с фокусным расстоянием $40\,\text{дптр}$
    следует надо поместить предмет, чтобы расстояние
    от предмета до его действительного изображения было наименьшим?
}
\answer{%
    \begin{align*}
    \frac 1a &+ \frac 1b = D \implies b = \frac 1{D - \frac 1a} \implies \ell = a + b = a + \frac a{Da - 1} = \frac{ Da^2 }{Da - 1} \implies \\
    \implies \ell'_a &= \frac{ 2Da \cdot (Da - 1) - Da^2 \cdot D }{\sqr{Da - 1}}= \frac{ D^2a^2 - 2Da}{\sqr{Da - 1}} = \frac{ Da(Da - 2)}{\sqr{Da - 1}}\implies a_{\min} = \frac 2D \approx 50\,\text{мм}.
    \end{align*}
}
\solutionspace{120pt}

\tasknumber{13}%
\task{%
    Даны точечный источник света $S$, его изображение $S_1$, полученное с помощью собирающей линзы,
    и ближайший к источнику фокус линзы $F$ (см.
    рис.
    на доске).
    Расстояния $SF = \ell$ и $SS_1 = L$.
    Определить положение линзы и её фокусное расстояние.
}
\answer{%
    \begin{align*}
    \frac 1a + \frac 1b &= \frac 1F, \ell = a - F, L = a + b \implies a = \ell + F, b = L - a = L - \ell - F \\
    \frac 1{\ell + F} + \frac 1{L - \ell - F} &= \frac 1F \\
    F\ell + F^2 + LF - F\ell - F^2 &= L\ell - \ell^2 - F\ell + LF - F\ell - F^2 \\
    0 &= L\ell - \ell^2 - 2F\ell - F^2 \\
    0 &=  F^2 + 2F\ell - L\ell + \ell^2 \\
    F &= -\ell \pm \sqrt{\ell^2 +  L\ell - \ell^2} = -\ell \pm \sqrt{L\ell} \implies F = \sqrt{L\ell} - \ell \\
    a &= \ell + F = \ell + \sqrt{L\ell} - \ell = \sqrt{L\ell}.
    \end{align*}
}
\solutionspace{120pt}

\tasknumber{14}%
\task{%
    Расстояние от освещённого предмета до экрана $80\,\text{см}$.
    Линза, помещенная между ними, даёт чёткое изображение предмета на
    экране при двух положениях, расстояние между которыми $40\,\text{см}$.
    Найти фокусное расстояние линзы.
}
\answer{%
    \begin{align*}
    \frac 1a + \frac 1b &= \frac 1F, \frac 1{a-\ell} + \frac 1{b+\ell} = \frac 1F, a + b = L \\
    \frac 1a + \frac 1b &= \frac 1{a-\ell} + \frac 1{b+\ell}\implies \frac{a + b}{ab} = \frac{(a-\ell) + (b+\ell)}{(a-\ell)(b+\ell)} \\
    ab  &= (a - \ell)(b+\ell) \implies 0  = -b\ell + a\ell - \ell^2 \implies 0 = -b + a - \ell \implies b = a - \ell \\
    a + (a - \ell) &= L \implies a = \frac{L + \ell}2 \implies b = \frac{L - \ell}2 \\
    F &= \frac{ab}{a + b} = \frac{L^2 -\ell^2}{4L} \approx 15\,\text{см}.
    \end{align*}
}
\solutionspace{120pt}

\tasknumber{15}%
\task{%
    Предмет находится на расстоянии $80\,\text{см}$ от экрана.
    Между предметом и экраном помещают линзу, причём при одном
    положении линзы на экране получается увеличенное изображение предмета,
    а при другом — уменьшенное.
    Каково фокусное расстояние линзы, если
    линейные размеры первого изображения в два раза больше второго?
}
\answer{%
    \begin{align*}
    \frac 1a + \frac 1{L-a} &= \frac 1F, h_1 = h \cdot \frac{L-a}a, \\
    \frac 1b + \frac 1{L-b} &= \frac 1F, h_2 = h \cdot \frac{L-b}b, \\
    \frac{h_1}{h_2} &= 2 \implies \frac{(L-a)b}{(L-b)a} = 2, \\
    \frac 1F &= \frac{ L }{a(L-a)} = \frac{ L }{b(L-b)} \implies \frac{L-a}{L-b} = \frac b a \implies \frac {b^2}{a^2} = 2.
    \\
    \frac 1a + \frac 1{L-a} &= \frac 1b + \frac 1{L-b} \implies \frac L{a(L-a)} = \frac L{b(L-b)} \implies \\
    \implies aL - a^2 &= bL - b^2 \implies (a-b)L = (a-b)(a+b) \implies b = L - a, \\
    \frac{\sqr{L-a}}{a^2} &= 2 \implies \frac La - 1 = \sqrt{2} \implies a = \frac{ L }{\sqrt{2} + 1} \\
    F &= \frac{a(L-a)}L = \frac 1L \cdot \frac L{\sqrt{2} + 1} \cdot \frac {L\sqrt{2}}{\sqrt{2} + 1}= \frac { L\sqrt{2} }{ \sqr{\sqrt{2} + 1} } \approx 19{,}4\,\text{см}.
    \end{align*}
}
\solutionspace{120pt}

\tasknumber{16}%
\task{%
    (Задача-«гроб»: решать на обратной стороне) Квадрат со стороной $d = 2\,\text{см}$ расположен так,
    что 2 его стороны параллельны главной оптической оси собирающей линзы,
    его центр удален на $h = 4\,\text{см}$ от этой оси и на $a = 15\,\text{см}$ от плоскости линзы.
    Определите площадь изображения квадрата, если фокусное расстояние линзы составляет $F = 20\,\text{см}$.
    % (и сравните с площадью объекта, умноженной на квадрат увеличения центра квадрата).
}
\answer{%
    \begin{align*}
    &\text{Все явные вычисления — в см и $\text{см}^2$,} \\
    \frac 1 F &= \frac 1{a + \frac d2} + \frac 1b \implies b = \frac 1{\frac 1 F - \frac 1{a + \frac d2}} = \frac{F(a + \frac d2)}{a + \frac d2 - F} = -80, \\
    \frac 1 F &= \frac 1{a - \frac d2} + \frac 1c \implies c = \frac 1{\frac 1 F - \frac 1{a - \frac d2}} = \frac{F(a - \frac d2)}{a - \frac d2 - F} = -\frac{140}3, \\
    c - b &= \frac{F(a - \frac d2)}{a - \frac d2 - F} - \frac{F(a + \frac d2)}{a + \frac d2 - F} = F\cbr{ \frac{a - \frac d2}{a - \frac d2 - F} - \frac{a + \frac d2}{a + \frac d2 - F} } =  \\
    &= F \cdot \frac{a^2 + \frac {ad}2 - aF - \frac{ad}2 - \frac{d^2}4 + \frac{dF}2 - a^2 + \frac {ad}2 + aF - \frac{ad}2 + \frac{d^2}4 + \frac{dF}2}{\cbr{a + \frac d2 - F}\cbr{a - \frac d2 - F}}= F \cdot \frac {dF}{\cbr{a + \frac d2 - F}\cbr{a - \frac d2 - F}} = \frac{100}3.
    \\
    \Gamma_b &= \frac b{a + \frac d2} = \frac{ F }{a + \frac d2 - F} = -5, \\
    \Gamma_c &= \frac c{a - \frac d2} = \frac{ F }{a - \frac d2 - F} = -\frac{10}3, \\
    &\text{ тут интересно отметить, что } \Gamma_x = \frac{ c - b}{ d } = \frac{ F^2 }{\cbr{a + \frac d2 - F}\cbr{a - \frac d2 - F}} \ne \Gamma_b \text{ или } \Gamma_c \text{ даже при малых $d$}.
    \\
    S' &= \frac{d \cdot \Gamma_b + d \cdot \Gamma_c}2 \cdot (c - b) = \frac d2 \cbr{\frac{ F }{a + \frac d2 - F} + \frac{ F }{a - \frac d2 - F}} \cdot \cbr{c - b} =  \\
    &=\frac {dF}2 \cbr{\frac 1{a + \frac d2 - F} + \frac 1{a - \frac d2 - F}} \cdot \frac {dF^2}{\cbr{a + \frac d2 - F}\cbr{a - \frac d2 - F}} =  \\
    &=\frac {dF}2 \cdot \frac{a - \frac d2 - F + a + \frac d2 - F}{\cbr{a + \frac d2 - F}\cbr{a - \frac d2 - F}} \cdot \frac {dF^2}{\cbr{a + \frac d2 - F}\cbr{a - \frac d2 - F}} =  \\
    &= \frac {d^2F^3}{2\sqr{a + \frac d2 - F}\sqr{a - \frac d2 - F}} \cdot (2a - 2F) = \frac {d^2F^3(a - F)}{ \sqr{\sqr{a - F} - \frac{d^2}4} } = -\frac{2500}9.
    \end{align*}
}

\variantsplitter

\addpersonalvariant{Ирина Ан}

\tasknumber{1}%
\task{%
    Запишите формулу тонкой линзы и сделайте рисунок, указав на нём физические величины из этой формулы.
}
\solutionspace{60pt}

\tasknumber{2}%
\task{%
    В каких линзах можно получить мнимое изображение объекта?
}
\answer{%
    $\text{ собирающие и рассеивающие }$
}
\solutionspace{40pt}

\tasknumber{3}%
\task{%
    Какое изображение называют мнимым?
}
\solutionspace{40pt}

\tasknumber{4}%
\task{%
    Есть две линзы, обозначим их 1 и 2.
    Известно что фокусное расстояние линзы 2 меньше, чем у линзы 1.
    Какая линза сильнее преломляет лучи?
}
\answer{%
    $2$
}
\solutionspace{40pt}

\tasknumber{5}%
\task{%
    Предмет находится на расстоянии $30\,\text{см}$ от собирающей линзы с фокусным расстоянием $15\,\text{см}$.
    Определите тип изображения, расстояние между предметом и его изображением, увеличение предмета.
    Сделайте схематичный рисунок (не обязательно в масштабе, но с сохранением свойств линзы и изображения).
}
\solutionspace{100pt}

\tasknumber{6}%
\task{%
    Объект находится на расстоянии $25\,\text{см}$ от линзы, а его действительное изображение — в $40\,\text{см}$ от неё.
    Определите увеличение предмета, фокусное расстояние линзы, оптическую силу линзы и её тип.
}
\solutionspace{80pt}

\tasknumber{7}%
\task{%
    Известно, что из формулы тонкой линзы $\cbr{\frac 1F = \frac 1a + \frac 1b}$
    и определения увеличения $\cbr{\Gamma_y = \frac ba}$ можно получить выражение
    для увеличения: $\Gamma_y = \frac {aF}{a - F} \cdot \frac 1a = \frac {F}{a - F}.$
    Назовём такое увеличение «поперечным»: поперёк главной оптической оси (поэтому и ${}_y$).
    Получите формулу для «продольного» увеличения $\Gamma_x$ небольшого предмета, находящегося на главной оптической оси.
    Можно ли применить эту формулу для предмета, не лежащего на главной оптической оси, почему?
}
\answer{%
    \begin{align*}
    \frac 1F &= \frac 1a + \frac 1b \implies b = \frac {aF}{a - F} \\
    \frac 1F &= \frac 1{a + x} + \frac 1c \implies c = \frac {(a+x)F}{a + x - F} \\
    x' &= \abs{b - c} = \frac {aF}{a - F} - \frac {(a+x)F}{a + x - F} = F\cbr{\frac {a}{a - F} - \frac {a+x}{a + x - F}} =  \\
    &= F \cdot \frac {a^2 + ax - aF - a^2 - ax + aF + xF}{(a - F)(a + x - F)} = F \cdot \frac {xF}{(a - F)(a + x - F)} \\
    \Gamma_x &= \frac{x'}x = \frac{F^2}{(a - F)(a + x - F)} \to \frac{F^2}{\sqr{a - F}}.
    \\
    &\text{Нельзя: изображение по-разному растянет по осям $x$ и $y$ и понадобится теорема Пифагора}
    \end{align*}
}
\solutionspace{150pt}

\tasknumber{8}%
\task{%
    Доказать формулу тонкой линзы для рассеивающей линзы.
}
\solutionspace{120pt}

\tasknumber{9}%
\task{%
    Постройте ход луча $BL$ в тонкой линзе.
    Известно положение линзы и оба её фокуса (см.
    рис.
    на доске).
    Рассмотрите оба типа линзы, сделав 2 рисунка: собирающую и рассеивающую.
}
\solutionspace{120pt}

\tasknumber{10}%
\task{%
    На экране, расположенном иа расстоянии $120\,\text{см}$ от собирающей линзы,
    получено изображение точечного источника, расположенного на главной оптической оси линзы.
    На какое расстояние переместится изображение на экране,
    если при неподвижной линзе переместить источник на $1\,\text{см}$ в плоскости, перпендикулярной главной оптической оси?
    Фокусное расстояние линзы равно $40\,\text{см}$.
}
\answer{%
    \begin{align*}
    &\frac 1F = \frac 1a + \frac 1b \implies a = \frac{bF}{b-F} \implies \Gamma = \frac ba = \frac{b-F}F \\
    &y = x \cdot \Gamma = x \cdot \frac{b-F}F \implies d = y = 2\,\text{см}.
    \end{align*}
}
\solutionspace{120pt}

\tasknumber{11}%
\task{%
    Оптическая сила двояковыпуклой линзы в воздухе $5\,\text{дптр}$, а в воде $1{,}4\,\text{дптр}$.
    Определить показатель преломления $n$ материала, из которого изготовлена линза.
    Показатель преломления воды равен $1{,}33$.
}
\answer{%
    \begin{align*}
    D_1 &=\cbr{\frac n{n_1} - 1}\cbr{\frac 1{R_1} + \frac 1{R_2}}, \\
    D_2 &=\cbr{\frac n{n_2} - 1}\cbr{\frac 1{R_1} + \frac 1{R_2}}, \\
    \frac {D_2}{D_1} &=\frac{\frac n{n_2} - 1}{\frac n{n_1} - 1} \implies {D_2}\cbr{\frac n{n_1} - 1} = {D_1}\cbr{\frac n{n_2} - 1}  \implies n\cbr{\frac{D_2}{n_1} - \frac{D_1}{n_2}} = D_2 - D_1, \\
    n &= \frac{D_2 - D_1}{\frac{D_2}{n_1} - \frac{D_1}{n_2}} = \frac{n_1 n_2 (D_2 - D_1)}{D_2n_2 - D_1n_1} \approx 1{,}526.
    \end{align*}
}
\solutionspace{120pt}

\tasknumber{12}%
\task{%
    На каком расстоянии от собирающей линзы с фокусным расстоянием $30\,\text{дптр}$
    следует надо поместить предмет, чтобы расстояние
    от предмета до его действительного изображения было наименьшим?
}
\answer{%
    \begin{align*}
    \frac 1a &+ \frac 1b = D \implies b = \frac 1{D - \frac 1a} \implies \ell = a + b = a + \frac a{Da - 1} = \frac{ Da^2 }{Da - 1} \implies \\
    \implies \ell'_a &= \frac{ 2Da \cdot (Da - 1) - Da^2 \cdot D }{\sqr{Da - 1}}= \frac{ D^2a^2 - 2Da}{\sqr{Da - 1}} = \frac{ Da(Da - 2)}{\sqr{Da - 1}}\implies a_{\min} = \frac 2D \approx 66{,}7\,\text{мм}.
    \end{align*}
}
\solutionspace{120pt}

\tasknumber{13}%
\task{%
    Даны точечный источник света $S$, его изображение $S_1$, полученное с помощью собирающей линзы,
    и ближайший к источнику фокус линзы $F$ (см.
    рис.
    на доске).
    Расстояния $SF = \ell$ и $SS_1 = L$.
    Определить положение линзы и её фокусное расстояние.
}
\answer{%
    \begin{align*}
    \frac 1a + \frac 1b &= \frac 1F, \ell = a - F, L = a + b \implies a = \ell + F, b = L - a = L - \ell - F \\
    \frac 1{\ell + F} + \frac 1{L - \ell - F} &= \frac 1F \\
    F\ell + F^2 + LF - F\ell - F^2 &= L\ell - \ell^2 - F\ell + LF - F\ell - F^2 \\
    0 &= L\ell - \ell^2 - 2F\ell - F^2 \\
    0 &=  F^2 + 2F\ell - L\ell + \ell^2 \\
    F &= -\ell \pm \sqrt{\ell^2 +  L\ell - \ell^2} = -\ell \pm \sqrt{L\ell} \implies F = \sqrt{L\ell} - \ell \\
    a &= \ell + F = \ell + \sqrt{L\ell} - \ell = \sqrt{L\ell}.
    \end{align*}
}
\solutionspace{120pt}

\tasknumber{14}%
\task{%
    Расстояние от освещённого предмета до экрана $100\,\text{см}$.
    Линза, помещенная между ними, даёт чёткое изображение предмета на
    экране при двух положениях, расстояние между которыми $40\,\text{см}$.
    Найти фокусное расстояние линзы.
}
\answer{%
    \begin{align*}
    \frac 1a + \frac 1b &= \frac 1F, \frac 1{a-\ell} + \frac 1{b+\ell} = \frac 1F, a + b = L \\
    \frac 1a + \frac 1b &= \frac 1{a-\ell} + \frac 1{b+\ell}\implies \frac{a + b}{ab} = \frac{(a-\ell) + (b+\ell)}{(a-\ell)(b+\ell)} \\
    ab  &= (a - \ell)(b+\ell) \implies 0  = -b\ell + a\ell - \ell^2 \implies 0 = -b + a - \ell \implies b = a - \ell \\
    a + (a - \ell) &= L \implies a = \frac{L + \ell}2 \implies b = \frac{L - \ell}2 \\
    F &= \frac{ab}{a + b} = \frac{L^2 -\ell^2}{4L} \approx 21\,\text{см}.
    \end{align*}
}
\solutionspace{120pt}

\tasknumber{15}%
\task{%
    Предмет находится на расстоянии $60\,\text{см}$ от экрана.
    Между предметом и экраном помещают линзу, причём при одном
    положении линзы на экране получается увеличенное изображение предмета,
    а при другом — уменьшенное.
    Каково фокусное расстояние линзы, если
    линейные размеры первого изображения в пять раз больше второго?
}
\answer{%
    \begin{align*}
    \frac 1a + \frac 1{L-a} &= \frac 1F, h_1 = h \cdot \frac{L-a}a, \\
    \frac 1b + \frac 1{L-b} &= \frac 1F, h_2 = h \cdot \frac{L-b}b, \\
    \frac{h_1}{h_2} &= 5 \implies \frac{(L-a)b}{(L-b)a} = 5, \\
    \frac 1F &= \frac{ L }{a(L-a)} = \frac{ L }{b(L-b)} \implies \frac{L-a}{L-b} = \frac b a \implies \frac {b^2}{a^2} = 5.
    \\
    \frac 1a + \frac 1{L-a} &= \frac 1b + \frac 1{L-b} \implies \frac L{a(L-a)} = \frac L{b(L-b)} \implies \\
    \implies aL - a^2 &= bL - b^2 \implies (a-b)L = (a-b)(a+b) \implies b = L - a, \\
    \frac{\sqr{L-a}}{a^2} &= 5 \implies \frac La - 1 = \sqrt{5} \implies a = \frac{ L }{\sqrt{5} + 1} \\
    F &= \frac{a(L-a)}L = \frac 1L \cdot \frac L{\sqrt{5} + 1} \cdot \frac {L\sqrt{5}}{\sqrt{5} + 1}= \frac { L\sqrt{5} }{ \sqr{\sqrt{5} + 1} } \approx 12{,}8\,\text{см}.
    \end{align*}
}
\solutionspace{120pt}

\tasknumber{16}%
\task{%
    (Задача-«гроб»: решать на обратной стороне) Квадрат со стороной $d = 3\,\text{см}$ расположен так,
    что 2 его стороны параллельны главной оптической оси рассеивающей линзы,
    его центр удален на $h = 5\,\text{см}$ от этой оси и на $a = 10\,\text{см}$ от плоскости линзы.
    Определите площадь изображения квадрата, если фокусное расстояние линзы составляет $F = 25\,\text{см}$.
    % (и сравните с площадью объекта, умноженной на квадрат увеличения центра квадрата).
}
\answer{%
    \begin{align*}
    &\text{Все явные вычисления — в см и $\text{см}^2$,} \\
    \frac 1 F &= \frac 1{a + \frac d2} + \frac 1b \implies b = \frac 1{\frac 1 F - \frac 1{a + \frac d2}} = \frac{F(a + \frac d2)}{a + \frac d2 - F} = -\frac{575}{73}, \\
    \frac 1 F &= \frac 1{a - \frac d2} + \frac 1c \implies c = \frac 1{\frac 1 F - \frac 1{a - \frac d2}} = \frac{F(a - \frac d2)}{a - \frac d2 - F} = -\frac{425}{67}, \\
    c - b &= \frac{F(a - \frac d2)}{a - \frac d2 - F} - \frac{F(a + \frac d2)}{a + \frac d2 - F} = F\cbr{ \frac{a - \frac d2}{a - \frac d2 - F} - \frac{a + \frac d2}{a + \frac d2 - F} } =  \\
    &= F \cdot \frac{a^2 + \frac {ad}2 - aF - \frac{ad}2 - \frac{d^2}4 + \frac{dF}2 - a^2 + \frac {ad}2 + aF - \frac{ad}2 + \frac{d^2}4 + \frac{dF}2}{\cbr{a + \frac d2 - F}\cbr{a - \frac d2 - F}}= F \cdot \frac {dF}{\cbr{a + \frac d2 - F}\cbr{a - \frac d2 - F}} = \frac{7500}{4891}.
    \\
    \Gamma_b &= \frac b{a + \frac d2} = \frac{ F }{a + \frac d2 - F} = -\frac{50}{73}, \\
    \Gamma_c &= \frac c{a - \frac d2} = \frac{ F }{a - \frac d2 - F} = -\frac{50}{67}, \\
    &\text{ тут интересно отметить, что } \Gamma_x = \frac{ c - b}{ d } = \frac{ F^2 }{\cbr{a + \frac d2 - F}\cbr{a - \frac d2 - F}} \ne \Gamma_b \text{ или } \Gamma_c \text{ даже при малых $d$}.
    \\
    S' &= \frac{d \cdot \Gamma_b + d \cdot \Gamma_c}2 \cdot (c - b) = \frac d2 \cbr{\frac{ F }{a + \frac d2 - F} + \frac{ F }{a - \frac d2 - F}} \cdot \cbr{c - b} =  \\
    &=\frac {dF}2 \cbr{\frac 1{a + \frac d2 - F} + \frac 1{a - \frac d2 - F}} \cdot \frac {dF^2}{\cbr{a + \frac d2 - F}\cbr{a - \frac d2 - F}} =  \\
    &=\frac {dF}2 \cdot \frac{a - \frac d2 - F + a + \frac d2 - F}{\cbr{a + \frac d2 - F}\cbr{a - \frac d2 - F}} \cdot \frac {dF^2}{\cbr{a + \frac d2 - F}\cbr{a - \frac d2 - F}} =  \\
    &= \frac {d^2F^3}{2\sqr{a + \frac d2 - F}\sqr{a - \frac d2 - F}} \cdot (2a - 2F) = \frac {d^2F^3(a - F)}{ \sqr{\sqr{a - F} - \frac{d^2}4} } = -\frac{78750000}{23921881}.
    \end{align*}
}

\variantsplitter

\addpersonalvariant{Софья Андрианова}

\tasknumber{1}%
\task{%
    Запишите известные вам виды классификации изображений.
}
\solutionspace{60pt}

\tasknumber{2}%
\task{%
    В каких линзах можно получить обратное изображение объекта?
}
\answer{%
    $\text{ собирающие }$
}
\solutionspace{40pt}

\tasknumber{3}%
\task{%
    Какое изображение называют действительным?
}
\solutionspace{40pt}

\tasknumber{4}%
\task{%
    Есть две линзы, обозначим их 1 и 2.
    Известно что фокусное расстояние линзы 1 меньше, чем у линзы 2.
    Какая линза сильнее преломляет лучи?
}
\answer{%
    $1$
}
\solutionspace{40pt}

\tasknumber{5}%
\task{%
    Предмет находится на расстоянии $10\,\text{см}$ от рассеивающей линзы с фокусным расстоянием $25\,\text{см}$.
    Определите тип изображения, расстояние между предметом и его изображением, увеличение предмета.
    Сделайте схематичный рисунок (не обязательно в масштабе, но с сохранением свойств линзы и изображения).
}
\solutionspace{100pt}

\tasknumber{6}%
\task{%
    Объект находится на расстоянии $115\,\text{см}$ от линзы, а его действительное изображение — в $40\,\text{см}$ от неё.
    Определите увеличение предмета, фокусное расстояние линзы, оптическую силу линзы и её тип.
}
\solutionspace{80pt}

\tasknumber{7}%
\task{%
    Известно, что из формулы тонкой линзы $\cbr{\frac 1F = \frac 1a + \frac 1b}$
    и определения увеличения $\cbr{\Gamma_y = \frac ba}$ можно получить выражение
    для увеличения: $\Gamma_y = \frac {aF}{a - F} \cdot \frac 1a = \frac {F}{a - F}.$
    Назовём такое увеличение «поперечным»: поперёк главной оптической оси (поэтому и ${}_y$).
    Получите формулу для «продольного» увеличения $\Gamma_x$ небольшого предмета, находящегося на главной оптической оси.
    Можно ли применить эту формулу для предмета, не лежащего на главной оптической оси, почему?
}
\answer{%
    \begin{align*}
    \frac 1F &= \frac 1a + \frac 1b \implies b = \frac {aF}{a - F} \\
    \frac 1F &= \frac 1{a + x} + \frac 1c \implies c = \frac {(a+x)F}{a + x - F} \\
    x' &= \abs{b - c} = \frac {aF}{a - F} - \frac {(a+x)F}{a + x - F} = F\cbr{\frac {a}{a - F} - \frac {a+x}{a + x - F}} =  \\
    &= F \cdot \frac {a^2 + ax - aF - a^2 - ax + aF + xF}{(a - F)(a + x - F)} = F \cdot \frac {xF}{(a - F)(a + x - F)} \\
    \Gamma_x &= \frac{x'}x = \frac{F^2}{(a - F)(a + x - F)} \to \frac{F^2}{\sqr{a - F}}.
    \\
    &\text{Нельзя: изображение по-разному растянет по осям $x$ и $y$ и понадобится теорема Пифагора}
    \end{align*}
}
\solutionspace{150pt}

\tasknumber{8}%
\task{%
    Доказать формулу тонкой линзы для собирающей линзы.
}
\solutionspace{120pt}

\tasknumber{9}%
\task{%
    Постройте ход луча $BL$ в тонкой линзе.
    Известно положение линзы и оба её фокуса (см.
    рис.
    на доске).
    Рассмотрите оба типа линзы, сделав 2 рисунка: собирающую и рассеивающую.
}
\solutionspace{120pt}

\tasknumber{10}%
\task{%
    На экране, расположенном иа расстоянии $80\,\text{см}$ от собирающей линзы,
    получено изображение точечного источника, расположенного на главной оптической оси линзы.
    На какое расстояние переместится изображение на экране,
    если при неподвижной линзе переместить источник на $1\,\text{см}$ в плоскости, перпендикулярной главной оптической оси?
    Фокусное расстояние линзы равно $30\,\text{см}$.
}
\answer{%
    \begin{align*}
    &\frac 1F = \frac 1a + \frac 1b \implies a = \frac{bF}{b-F} \implies \Gamma = \frac ba = \frac{b-F}F \\
    &y = x \cdot \Gamma = x \cdot \frac{b-F}F \implies d = y = 1{,}67\,\text{см}.
    \end{align*}
}
\solutionspace{120pt}

\tasknumber{11}%
\task{%
    Оптическая сила двояковыпуклой линзы в воздухе $5{,}5\,\text{дптр}$, а в воде $1{,}5\,\text{дптр}$.
    Определить показатель преломления $n$ материала, из которого изготовлена линза.
    Показатель преломления воды равен $1{,}33$.
}
\answer{%
    \begin{align*}
    D_1 &=\cbr{\frac n{n_1} - 1}\cbr{\frac 1{R_1} + \frac 1{R_2}}, \\
    D_2 &=\cbr{\frac n{n_2} - 1}\cbr{\frac 1{R_1} + \frac 1{R_2}}, \\
    \frac {D_2}{D_1} &=\frac{\frac n{n_2} - 1}{\frac n{n_1} - 1} \implies {D_2}\cbr{\frac n{n_1} - 1} = {D_1}\cbr{\frac n{n_2} - 1}  \implies n\cbr{\frac{D_2}{n_1} - \frac{D_1}{n_2}} = D_2 - D_1, \\
    n &= \frac{D_2 - D_1}{\frac{D_2}{n_1} - \frac{D_1}{n_2}} = \frac{n_1 n_2 (D_2 - D_1)}{D_2n_2 - D_1n_1} \approx 1{,}518.
    \end{align*}
}
\solutionspace{120pt}

\tasknumber{12}%
\task{%
    На каком расстоянии от собирающей линзы с фокусным расстоянием $40\,\text{дптр}$
    следует надо поместить предмет, чтобы расстояние
    от предмета до его действительного изображения было наименьшим?
}
\answer{%
    \begin{align*}
    \frac 1a &+ \frac 1b = D \implies b = \frac 1{D - \frac 1a} \implies \ell = a + b = a + \frac a{Da - 1} = \frac{ Da^2 }{Da - 1} \implies \\
    \implies \ell'_a &= \frac{ 2Da \cdot (Da - 1) - Da^2 \cdot D }{\sqr{Da - 1}}= \frac{ D^2a^2 - 2Da}{\sqr{Da - 1}} = \frac{ Da(Da - 2)}{\sqr{Da - 1}}\implies a_{\min} = \frac 2D \approx 50\,\text{мм}.
    \end{align*}
}
\solutionspace{120pt}

\tasknumber{13}%
\task{%
    Даны точечный источник света $S$, его изображение $S_1$, полученное с помощью собирающей линзы,
    и ближайший к источнику фокус линзы $F$ (см.
    рис.
    на доске).
    Расстояния $SF = \ell$ и $SS_1 = L$.
    Определить положение линзы и её фокусное расстояние.
}
\answer{%
    \begin{align*}
    \frac 1a + \frac 1b &= \frac 1F, \ell = a - F, L = a + b \implies a = \ell + F, b = L - a = L - \ell - F \\
    \frac 1{\ell + F} + \frac 1{L - \ell - F} &= \frac 1F \\
    F\ell + F^2 + LF - F\ell - F^2 &= L\ell - \ell^2 - F\ell + LF - F\ell - F^2 \\
    0 &= L\ell - \ell^2 - 2F\ell - F^2 \\
    0 &=  F^2 + 2F\ell - L\ell + \ell^2 \\
    F &= -\ell \pm \sqrt{\ell^2 +  L\ell - \ell^2} = -\ell \pm \sqrt{L\ell} \implies F = \sqrt{L\ell} - \ell \\
    a &= \ell + F = \ell + \sqrt{L\ell} - \ell = \sqrt{L\ell}.
    \end{align*}
}
\solutionspace{120pt}

\tasknumber{14}%
\task{%
    Расстояние от освещённого предмета до экрана $100\,\text{см}$.
    Линза, помещенная между ними, даёт чёткое изображение предмета на
    экране при двух положениях, расстояние между которыми $40\,\text{см}$.
    Найти фокусное расстояние линзы.
}
\answer{%
    \begin{align*}
    \frac 1a + \frac 1b &= \frac 1F, \frac 1{a-\ell} + \frac 1{b+\ell} = \frac 1F, a + b = L \\
    \frac 1a + \frac 1b &= \frac 1{a-\ell} + \frac 1{b+\ell}\implies \frac{a + b}{ab} = \frac{(a-\ell) + (b+\ell)}{(a-\ell)(b+\ell)} \\
    ab  &= (a - \ell)(b+\ell) \implies 0  = -b\ell + a\ell - \ell^2 \implies 0 = -b + a - \ell \implies b = a - \ell \\
    a + (a - \ell) &= L \implies a = \frac{L + \ell}2 \implies b = \frac{L - \ell}2 \\
    F &= \frac{ab}{a + b} = \frac{L^2 -\ell^2}{4L} \approx 21\,\text{см}.
    \end{align*}
}
\solutionspace{120pt}

\tasknumber{15}%
\task{%
    Предмет находится на расстоянии $70\,\text{см}$ от экрана.
    Между предметом и экраном помещают линзу, причём при одном
    положении линзы на экране получается увеличенное изображение предмета,
    а при другом — уменьшенное.
    Каково фокусное расстояние линзы, если
    линейные размеры первого изображения в два раза больше второго?
}
\answer{%
    \begin{align*}
    \frac 1a + \frac 1{L-a} &= \frac 1F, h_1 = h \cdot \frac{L-a}a, \\
    \frac 1b + \frac 1{L-b} &= \frac 1F, h_2 = h \cdot \frac{L-b}b, \\
    \frac{h_1}{h_2} &= 2 \implies \frac{(L-a)b}{(L-b)a} = 2, \\
    \frac 1F &= \frac{ L }{a(L-a)} = \frac{ L }{b(L-b)} \implies \frac{L-a}{L-b} = \frac b a \implies \frac {b^2}{a^2} = 2.
    \\
    \frac 1a + \frac 1{L-a} &= \frac 1b + \frac 1{L-b} \implies \frac L{a(L-a)} = \frac L{b(L-b)} \implies \\
    \implies aL - a^2 &= bL - b^2 \implies (a-b)L = (a-b)(a+b) \implies b = L - a, \\
    \frac{\sqr{L-a}}{a^2} &= 2 \implies \frac La - 1 = \sqrt{2} \implies a = \frac{ L }{\sqrt{2} + 1} \\
    F &= \frac{a(L-a)}L = \frac 1L \cdot \frac L{\sqrt{2} + 1} \cdot \frac {L\sqrt{2}}{\sqrt{2} + 1}= \frac { L\sqrt{2} }{ \sqr{\sqrt{2} + 1} } \approx 17{,}0\,\text{см}.
    \end{align*}
}
\solutionspace{120pt}

\tasknumber{16}%
\task{%
    (Задача-«гроб»: решать на обратной стороне) Квадрат со стороной $d = 3\,\text{см}$ расположен так,
    что 2 его стороны параллельны главной оптической оси рассеивающей линзы,
    его центр удален на $h = 4\,\text{см}$ от этой оси и на $a = 12\,\text{см}$ от плоскости линзы.
    Определите площадь изображения квадрата, если фокусное расстояние линзы составляет $F = 20\,\text{см}$.
    % (и сравните с площадью объекта, умноженной на квадрат увеличения центра квадрата).
}
\answer{%
    \begin{align*}
    &\text{Все явные вычисления — в см и $\text{см}^2$,} \\
    \frac 1 F &= \frac 1{a + \frac d2} + \frac 1b \implies b = \frac 1{\frac 1 F - \frac 1{a + \frac d2}} = \frac{F(a + \frac d2)}{a + \frac d2 - F} = -\frac{540}{67}, \\
    \frac 1 F &= \frac 1{a - \frac d2} + \frac 1c \implies c = \frac 1{\frac 1 F - \frac 1{a - \frac d2}} = \frac{F(a - \frac d2)}{a - \frac d2 - F} = -\frac{420}{61}, \\
    c - b &= \frac{F(a - \frac d2)}{a - \frac d2 - F} - \frac{F(a + \frac d2)}{a + \frac d2 - F} = F\cbr{ \frac{a - \frac d2}{a - \frac d2 - F} - \frac{a + \frac d2}{a + \frac d2 - F} } =  \\
    &= F \cdot \frac{a^2 + \frac {ad}2 - aF - \frac{ad}2 - \frac{d^2}4 + \frac{dF}2 - a^2 + \frac {ad}2 + aF - \frac{ad}2 + \frac{d^2}4 + \frac{dF}2}{\cbr{a + \frac d2 - F}\cbr{a - \frac d2 - F}}= F \cdot \frac {dF}{\cbr{a + \frac d2 - F}\cbr{a - \frac d2 - F}} = \frac{4800}{4087}.
    \\
    \Gamma_b &= \frac b{a + \frac d2} = \frac{ F }{a + \frac d2 - F} = -\frac{40}{67}, \\
    \Gamma_c &= \frac c{a - \frac d2} = \frac{ F }{a - \frac d2 - F} = -\frac{40}{61}, \\
    &\text{ тут интересно отметить, что } \Gamma_x = \frac{ c - b}{ d } = \frac{ F^2 }{\cbr{a + \frac d2 - F}\cbr{a - \frac d2 - F}} \ne \Gamma_b \text{ или } \Gamma_c \text{ даже при малых $d$}.
    \\
    S' &= \frac{d \cdot \Gamma_b + d \cdot \Gamma_c}2 \cdot (c - b) = \frac d2 \cbr{\frac{ F }{a + \frac d2 - F} + \frac{ F }{a - \frac d2 - F}} \cdot \cbr{c - b} =  \\
    &=\frac {dF}2 \cbr{\frac 1{a + \frac d2 - F} + \frac 1{a - \frac d2 - F}} \cdot \frac {dF^2}{\cbr{a + \frac d2 - F}\cbr{a - \frac d2 - F}} =  \\
    &=\frac {dF}2 \cdot \frac{a - \frac d2 - F + a + \frac d2 - F}{\cbr{a + \frac d2 - F}\cbr{a - \frac d2 - F}} \cdot \frac {dF^2}{\cbr{a + \frac d2 - F}\cbr{a - \frac d2 - F}} =  \\
    &= \frac {d^2F^3}{2\sqr{a + \frac d2 - F}\sqr{a - \frac d2 - F}} \cdot (2a - 2F) = \frac {d^2F^3(a - F)}{ \sqr{\sqr{a - F} - \frac{d^2}4} } = -\frac{36864000}{16703569}.
    \end{align*}
}

\variantsplitter

\addpersonalvariant{Владимир Артемчук}

\tasknumber{1}%
\task{%
    Запишите известные вам виды классификации изображений.
}
\solutionspace{60pt}

\tasknumber{2}%
\task{%
    В каких линзах можно получить прямое изображение объекта?
}
\answer{%
    $\text{ собирающие и рассеивающие }$
}
\solutionspace{40pt}

\tasknumber{3}%
\task{%
    Какое изображение называют мнимым?
}
\solutionspace{40pt}

\tasknumber{4}%
\task{%
    Есть две линзы, обозначим их 1 и 2.
    Известно что оптическая сила линзы 2 больше, чем у линзы 1.
    Какая линза сильнее преломляет лучи?
}
\answer{%
    $2$
}
\solutionspace{40pt}

\tasknumber{5}%
\task{%
    Предмет находится на расстоянии $30\,\text{см}$ от рассеивающей линзы с фокусным расстоянием $8\,\text{см}$.
    Определите тип изображения, расстояние между предметом и его изображением, увеличение предмета.
    Сделайте схематичный рисунок (не обязательно в масштабе, но с сохранением свойств линзы и изображения).
}
\solutionspace{100pt}

\tasknumber{6}%
\task{%
    Объект находится на расстоянии $25\,\text{см}$ от линзы, а его действительное изображение — в $40\,\text{см}$ от неё.
    Определите увеличение предмета, фокусное расстояние линзы, оптическую силу линзы и её тип.
}
\solutionspace{80pt}

\tasknumber{7}%
\task{%
    Известно, что из формулы тонкой линзы $\cbr{\frac 1F = \frac 1a + \frac 1b}$
    и определения увеличения $\cbr{\Gamma_y = \frac ba}$ можно получить выражение
    для увеличения: $\Gamma_y = \frac {aF}{a - F} \cdot \frac 1a = \frac {F}{a - F}.$
    Назовём такое увеличение «поперечным»: поперёк главной оптической оси (поэтому и ${}_y$).
    Получите формулу для «продольного» увеличения $\Gamma_x$ небольшого предмета, находящегося на главной оптической оси.
    Можно ли применить эту формулу для предмета, не лежащего на главной оптической оси, почему?
}
\answer{%
    \begin{align*}
    \frac 1F &= \frac 1a + \frac 1b \implies b = \frac {aF}{a - F} \\
    \frac 1F &= \frac 1{a + x} + \frac 1c \implies c = \frac {(a+x)F}{a + x - F} \\
    x' &= \abs{b - c} = \frac {aF}{a - F} - \frac {(a+x)F}{a + x - F} = F\cbr{\frac {a}{a - F} - \frac {a+x}{a + x - F}} =  \\
    &= F \cdot \frac {a^2 + ax - aF - a^2 - ax + aF + xF}{(a - F)(a + x - F)} = F \cdot \frac {xF}{(a - F)(a + x - F)} \\
    \Gamma_x &= \frac{x'}x = \frac{F^2}{(a - F)(a + x - F)} \to \frac{F^2}{\sqr{a - F}}.
    \\
    &\text{Нельзя: изображение по-разному растянет по осям $x$ и $y$ и понадобится теорема Пифагора}
    \end{align*}
}
\solutionspace{150pt}

\tasknumber{8}%
\task{%
    Доказать формулу тонкой линзы для рассеивающей линзы.
}
\solutionspace{120pt}

\tasknumber{9}%
\task{%
    Постройте ход луча $AL$ в тонкой линзе.
    Известно положение линзы и оба её фокуса (см.
    рис.
    на доске).
    Рассмотрите оба типа линзы, сделав 2 рисунка: собирающую и рассеивающую.
}
\solutionspace{120pt}

\tasknumber{10}%
\task{%
    На экране, расположенном иа расстоянии $120\,\text{см}$ от собирающей линзы,
    получено изображение точечного источника, расположенного на главной оптической оси линзы.
    На какое расстояние переместится изображение на экране,
    если при неподвижном источнике переместить линзу на $3\,\text{см}$ в плоскости, перпендикулярной главной оптической оси?
    Фокусное расстояние линзы равно $40\,\text{см}$.
}
\answer{%
    \begin{align*}
    &\frac 1F = \frac 1a + \frac 1b \implies a = \frac{bF}{b-F} \implies \Gamma = \frac ba = \frac{b-F}F \\
    &y = x \cdot \Gamma = x \cdot \frac{b-F}F \implies d = x + y = 9\,\text{см}.
    \end{align*}
}
\solutionspace{120pt}

\tasknumber{11}%
\task{%
    Оптическая сила двояковыпуклой линзы в воздухе $5{,}5\,\text{дптр}$, а в воде $1{,}6\,\text{дптр}$.
    Определить показатель преломления $n$ материала, из которого изготовлена линза.
    Показатель преломления воды равен $1{,}33$.
}
\answer{%
    \begin{align*}
    D_1 &=\cbr{\frac n{n_1} - 1}\cbr{\frac 1{R_1} + \frac 1{R_2}}, \\
    D_2 &=\cbr{\frac n{n_2} - 1}\cbr{\frac 1{R_1} + \frac 1{R_2}}, \\
    \frac {D_2}{D_1} &=\frac{\frac n{n_2} - 1}{\frac n{n_1} - 1} \implies {D_2}\cbr{\frac n{n_1} - 1} = {D_1}\cbr{\frac n{n_2} - 1}  \implies n\cbr{\frac{D_2}{n_1} - \frac{D_1}{n_2}} = D_2 - D_1, \\
    n &= \frac{D_2 - D_1}{\frac{D_2}{n_1} - \frac{D_1}{n_2}} = \frac{n_1 n_2 (D_2 - D_1)}{D_2n_2 - D_1n_1} \approx 1{,}538.
    \end{align*}
}
\solutionspace{120pt}

\tasknumber{12}%
\task{%
    На каком расстоянии от собирающей линзы с фокусным расстоянием $40\,\text{дптр}$
    следует надо поместить предмет, чтобы расстояние
    от предмета до его действительного изображения было наименьшим?
}
\answer{%
    \begin{align*}
    \frac 1a &+ \frac 1b = D \implies b = \frac 1{D - \frac 1a} \implies \ell = a + b = a + \frac a{Da - 1} = \frac{ Da^2 }{Da - 1} \implies \\
    \implies \ell'_a &= \frac{ 2Da \cdot (Da - 1) - Da^2 \cdot D }{\sqr{Da - 1}}= \frac{ D^2a^2 - 2Da}{\sqr{Da - 1}} = \frac{ Da(Da - 2)}{\sqr{Da - 1}}\implies a_{\min} = \frac 2D \approx 50\,\text{мм}.
    \end{align*}
}
\solutionspace{120pt}

\tasknumber{13}%
\task{%
    Даны точечный источник света $S$, его изображение $S_1$, полученное с помощью собирающей линзы,
    и ближайший к источнику фокус линзы $F$ (см.
    рис.
    на доске).
    Расстояния $SF = \ell$ и $SS_1 = L$.
    Определить положение линзы и её фокусное расстояние.
}
\answer{%
    \begin{align*}
    \frac 1a + \frac 1b &= \frac 1F, \ell = a - F, L = a + b \implies a = \ell + F, b = L - a = L - \ell - F \\
    \frac 1{\ell + F} + \frac 1{L - \ell - F} &= \frac 1F \\
    F\ell + F^2 + LF - F\ell - F^2 &= L\ell - \ell^2 - F\ell + LF - F\ell - F^2 \\
    0 &= L\ell - \ell^2 - 2F\ell - F^2 \\
    0 &=  F^2 + 2F\ell - L\ell + \ell^2 \\
    F &= -\ell \pm \sqrt{\ell^2 +  L\ell - \ell^2} = -\ell \pm \sqrt{L\ell} \implies F = \sqrt{L\ell} - \ell \\
    a &= \ell + F = \ell + \sqrt{L\ell} - \ell = \sqrt{L\ell}.
    \end{align*}
}
\solutionspace{120pt}

\tasknumber{14}%
\task{%
    Расстояние от освещённого предмета до экрана $80\,\text{см}$.
    Линза, помещенная между ними, даёт чёткое изображение предмета на
    экране при двух положениях, расстояние между которыми $30\,\text{см}$.
    Найти фокусное расстояние линзы.
}
\answer{%
    \begin{align*}
    \frac 1a + \frac 1b &= \frac 1F, \frac 1{a-\ell} + \frac 1{b+\ell} = \frac 1F, a + b = L \\
    \frac 1a + \frac 1b &= \frac 1{a-\ell} + \frac 1{b+\ell}\implies \frac{a + b}{ab} = \frac{(a-\ell) + (b+\ell)}{(a-\ell)(b+\ell)} \\
    ab  &= (a - \ell)(b+\ell) \implies 0  = -b\ell + a\ell - \ell^2 \implies 0 = -b + a - \ell \implies b = a - \ell \\
    a + (a - \ell) &= L \implies a = \frac{L + \ell}2 \implies b = \frac{L - \ell}2 \\
    F &= \frac{ab}{a + b} = \frac{L^2 -\ell^2}{4L} \approx 17{,}2\,\text{см}.
    \end{align*}
}
\solutionspace{120pt}

\tasknumber{15}%
\task{%
    Предмет находится на расстоянии $80\,\text{см}$ от экрана.
    Между предметом и экраном помещают линзу, причём при одном
    положении линзы на экране получается увеличенное изображение предмета,
    а при другом — уменьшенное.
    Каково фокусное расстояние линзы, если
    линейные размеры первого изображения в три раза больше второго?
}
\answer{%
    \begin{align*}
    \frac 1a + \frac 1{L-a} &= \frac 1F, h_1 = h \cdot \frac{L-a}a, \\
    \frac 1b + \frac 1{L-b} &= \frac 1F, h_2 = h \cdot \frac{L-b}b, \\
    \frac{h_1}{h_2} &= 3 \implies \frac{(L-a)b}{(L-b)a} = 3, \\
    \frac 1F &= \frac{ L }{a(L-a)} = \frac{ L }{b(L-b)} \implies \frac{L-a}{L-b} = \frac b a \implies \frac {b^2}{a^2} = 3.
    \\
    \frac 1a + \frac 1{L-a} &= \frac 1b + \frac 1{L-b} \implies \frac L{a(L-a)} = \frac L{b(L-b)} \implies \\
    \implies aL - a^2 &= bL - b^2 \implies (a-b)L = (a-b)(a+b) \implies b = L - a, \\
    \frac{\sqr{L-a}}{a^2} &= 3 \implies \frac La - 1 = \sqrt{3} \implies a = \frac{ L }{\sqrt{3} + 1} \\
    F &= \frac{a(L-a)}L = \frac 1L \cdot \frac L{\sqrt{3} + 1} \cdot \frac {L\sqrt{3}}{\sqrt{3} + 1}= \frac { L\sqrt{3} }{ \sqr{\sqrt{3} + 1} } \approx 18{,}6\,\text{см}.
    \end{align*}
}
\solutionspace{120pt}

\tasknumber{16}%
\task{%
    (Задача-«гроб»: решать на обратной стороне) Квадрат со стороной $d = 1\,\text{см}$ расположен так,
    что 2 его стороны параллельны главной оптической оси рассеивающей линзы,
    его центр удален на $h = 6\,\text{см}$ от этой оси и на $a = 12\,\text{см}$ от плоскости линзы.
    Определите площадь изображения квадрата, если фокусное расстояние линзы составляет $F = 25\,\text{см}$.
    % (и сравните с площадью объекта, умноженной на квадрат увеличения центра квадрата).
}
\answer{%
    \begin{align*}
    &\text{Все явные вычисления — в см и $\text{см}^2$,} \\
    \frac 1 F &= \frac 1{a + \frac d2} + \frac 1b \implies b = \frac 1{\frac 1 F - \frac 1{a + \frac d2}} = \frac{F(a + \frac d2)}{a + \frac d2 - F} = -\frac{25}3, \\
    \frac 1 F &= \frac 1{a - \frac d2} + \frac 1c \implies c = \frac 1{\frac 1 F - \frac 1{a - \frac d2}} = \frac{F(a - \frac d2)}{a - \frac d2 - F} = -\frac{575}{73}, \\
    c - b &= \frac{F(a - \frac d2)}{a - \frac d2 - F} - \frac{F(a + \frac d2)}{a + \frac d2 - F} = F\cbr{ \frac{a - \frac d2}{a - \frac d2 - F} - \frac{a + \frac d2}{a + \frac d2 - F} } =  \\
    &= F \cdot \frac{a^2 + \frac {ad}2 - aF - \frac{ad}2 - \frac{d^2}4 + \frac{dF}2 - a^2 + \frac {ad}2 + aF - \frac{ad}2 + \frac{d^2}4 + \frac{dF}2}{\cbr{a + \frac d2 - F}\cbr{a - \frac d2 - F}}= F \cdot \frac {dF}{\cbr{a + \frac d2 - F}\cbr{a - \frac d2 - F}} = \frac{100}{219}.
    \\
    \Gamma_b &= \frac b{a + \frac d2} = \frac{ F }{a + \frac d2 - F} = -\frac23, \\
    \Gamma_c &= \frac c{a - \frac d2} = \frac{ F }{a - \frac d2 - F} = -\frac{50}{73}, \\
    &\text{ тут интересно отметить, что } \Gamma_x = \frac{ c - b}{ d } = \frac{ F^2 }{\cbr{a + \frac d2 - F}\cbr{a - \frac d2 - F}} \ne \Gamma_b \text{ или } \Gamma_c \text{ даже при малых $d$}.
    \\
    S' &= \frac{d \cdot \Gamma_b + d \cdot \Gamma_c}2 \cdot (c - b) = \frac d2 \cbr{\frac{ F }{a + \frac d2 - F} + \frac{ F }{a - \frac d2 - F}} \cdot \cbr{c - b} =  \\
    &=\frac {dF}2 \cbr{\frac 1{a + \frac d2 - F} + \frac 1{a - \frac d2 - F}} \cdot \frac {dF^2}{\cbr{a + \frac d2 - F}\cbr{a - \frac d2 - F}} =  \\
    &=\frac {dF}2 \cdot \frac{a - \frac d2 - F + a + \frac d2 - F}{\cbr{a + \frac d2 - F}\cbr{a - \frac d2 - F}} \cdot \frac {dF^2}{\cbr{a + \frac d2 - F}\cbr{a - \frac d2 - F}} =  \\
    &= \frac {d^2F^3}{2\sqr{a + \frac d2 - F}\sqr{a - \frac d2 - F}} \cdot (2a - 2F) = \frac {d^2F^3(a - F)}{ \sqr{\sqr{a - F} - \frac{d^2}4} } = -\frac{14800}{47961}.
    \end{align*}
}

\variantsplitter

\addpersonalvariant{Софья Белянкина}

\tasknumber{1}%
\task{%
    Запишите известные вам виды классификации изображений.
}
\solutionspace{60pt}

\tasknumber{2}%
\task{%
    В каких линзах можно получить мнимое изображение объекта?
}
\answer{%
    $\text{ собирающие и рассеивающие }$
}
\solutionspace{40pt}

\tasknumber{3}%
\task{%
    Какое изображение называют мнимым?
}
\solutionspace{40pt}

\tasknumber{4}%
\task{%
    Есть две линзы, обозначим их 1 и 2.
    Известно что оптическая сила линзы 1 меньше, чем у линзы 2.
    Какая линза сильнее преломляет лучи?
}
\answer{%
    $2$
}
\solutionspace{40pt}

\tasknumber{5}%
\task{%
    Предмет находится на расстоянии $30\,\text{см}$ от собирающей линзы с фокусным расстоянием $40\,\text{см}$.
    Определите тип изображения, расстояние между предметом и его изображением, увеличение предмета.
    Сделайте схематичный рисунок (не обязательно в масштабе, но с сохранением свойств линзы и изображения).
}
\solutionspace{100pt}

\tasknumber{6}%
\task{%
    Объект находится на расстоянии $45\,\text{см}$ от линзы, а его действительное изображение — в $20\,\text{см}$ от неё.
    Определите увеличение предмета, фокусное расстояние линзы, оптическую силу линзы и её тип.
}
\solutionspace{80pt}

\tasknumber{7}%
\task{%
    Известно, что из формулы тонкой линзы $\cbr{\frac 1F = \frac 1a + \frac 1b}$
    и определения увеличения $\cbr{\Gamma_y = \frac ba}$ можно получить выражение
    для увеличения: $\Gamma_y = \frac {aF}{a - F} \cdot \frac 1a = \frac {F}{a - F}.$
    Назовём такое увеличение «поперечным»: поперёк главной оптической оси (поэтому и ${}_y$).
    Получите формулу для «продольного» увеличения $\Gamma_x$ небольшого предмета, находящегося на главной оптической оси.
    Можно ли применить эту формулу для предмета, не лежащего на главной оптической оси, почему?
}
\answer{%
    \begin{align*}
    \frac 1F &= \frac 1a + \frac 1b \implies b = \frac {aF}{a - F} \\
    \frac 1F &= \frac 1{a + x} + \frac 1c \implies c = \frac {(a+x)F}{a + x - F} \\
    x' &= \abs{b - c} = \frac {aF}{a - F} - \frac {(a+x)F}{a + x - F} = F\cbr{\frac {a}{a - F} - \frac {a+x}{a + x - F}} =  \\
    &= F \cdot \frac {a^2 + ax - aF - a^2 - ax + aF + xF}{(a - F)(a + x - F)} = F \cdot \frac {xF}{(a - F)(a + x - F)} \\
    \Gamma_x &= \frac{x'}x = \frac{F^2}{(a - F)(a + x - F)} \to \frac{F^2}{\sqr{a - F}}.
    \\
    &\text{Нельзя: изображение по-разному растянет по осям $x$ и $y$ и понадобится теорема Пифагора}
    \end{align*}
}
\solutionspace{150pt}

\tasknumber{8}%
\task{%
    Доказать формулу тонкой линзы для рассеивающей линзы.
}
\solutionspace{120pt}

\tasknumber{9}%
\task{%
    Постройте ход луча $AL$ в тонкой линзе.
    Известно положение линзы и оба её фокуса (см.
    рис.
    на доске).
    Рассмотрите оба типа линзы, сделав 2 рисунка: собирающую и рассеивающую.
}
\solutionspace{120pt}

\tasknumber{10}%
\task{%
    На экране, расположенном иа расстоянии $60\,\text{см}$ от собирающей линзы,
    получено изображение точечного источника, расположенного на главной оптической оси линзы.
    На какое расстояние переместится изображение на экране,
    если при неподвижном источнике переместить линзу на $2\,\text{см}$ в плоскости, перпендикулярной главной оптической оси?
    Фокусное расстояние линзы равно $20\,\text{см}$.
}
\answer{%
    \begin{align*}
    &\frac 1F = \frac 1a + \frac 1b \implies a = \frac{bF}{b-F} \implies \Gamma = \frac ba = \frac{b-F}F \\
    &y = x \cdot \Gamma = x \cdot \frac{b-F}F \implies d = x + y = 6\,\text{см}.
    \end{align*}
}
\solutionspace{120pt}

\tasknumber{11}%
\task{%
    Оптическая сила двояковыпуклой линзы в воздухе $5{,}5\,\text{дптр}$, а в воде $1{,}6\,\text{дптр}$.
    Определить показатель преломления $n$ материала, из которого изготовлена линза.
    Показатель преломления воды равен $1{,}33$.
}
\answer{%
    \begin{align*}
    D_1 &=\cbr{\frac n{n_1} - 1}\cbr{\frac 1{R_1} + \frac 1{R_2}}, \\
    D_2 &=\cbr{\frac n{n_2} - 1}\cbr{\frac 1{R_1} + \frac 1{R_2}}, \\
    \frac {D_2}{D_1} &=\frac{\frac n{n_2} - 1}{\frac n{n_1} - 1} \implies {D_2}\cbr{\frac n{n_1} - 1} = {D_1}\cbr{\frac n{n_2} - 1}  \implies n\cbr{\frac{D_2}{n_1} - \frac{D_1}{n_2}} = D_2 - D_1, \\
    n &= \frac{D_2 - D_1}{\frac{D_2}{n_1} - \frac{D_1}{n_2}} = \frac{n_1 n_2 (D_2 - D_1)}{D_2n_2 - D_1n_1} \approx 1{,}538.
    \end{align*}
}
\solutionspace{120pt}

\tasknumber{12}%
\task{%
    На каком расстоянии от собирающей линзы с фокусным расстоянием $30\,\text{дптр}$
    следует надо поместить предмет, чтобы расстояние
    от предмета до его действительного изображения было наименьшим?
}
\answer{%
    \begin{align*}
    \frac 1a &+ \frac 1b = D \implies b = \frac 1{D - \frac 1a} \implies \ell = a + b = a + \frac a{Da - 1} = \frac{ Da^2 }{Da - 1} \implies \\
    \implies \ell'_a &= \frac{ 2Da \cdot (Da - 1) - Da^2 \cdot D }{\sqr{Da - 1}}= \frac{ D^2a^2 - 2Da}{\sqr{Da - 1}} = \frac{ Da(Da - 2)}{\sqr{Da - 1}}\implies a_{\min} = \frac 2D \approx 66{,}7\,\text{мм}.
    \end{align*}
}
\solutionspace{120pt}

\tasknumber{13}%
\task{%
    Даны точечный источник света $S$, его изображение $S_1$, полученное с помощью собирающей линзы,
    и ближайший к источнику фокус линзы $F$ (см.
    рис.
    на доске).
    Расстояния $SF = \ell$ и $SS_1 = L$.
    Определить положение линзы и её фокусное расстояние.
}
\answer{%
    \begin{align*}
    \frac 1a + \frac 1b &= \frac 1F, \ell = a - F, L = a + b \implies a = \ell + F, b = L - a = L - \ell - F \\
    \frac 1{\ell + F} + \frac 1{L - \ell - F} &= \frac 1F \\
    F\ell + F^2 + LF - F\ell - F^2 &= L\ell - \ell^2 - F\ell + LF - F\ell - F^2 \\
    0 &= L\ell - \ell^2 - 2F\ell - F^2 \\
    0 &=  F^2 + 2F\ell - L\ell + \ell^2 \\
    F &= -\ell \pm \sqrt{\ell^2 +  L\ell - \ell^2} = -\ell \pm \sqrt{L\ell} \implies F = \sqrt{L\ell} - \ell \\
    a &= \ell + F = \ell + \sqrt{L\ell} - \ell = \sqrt{L\ell}.
    \end{align*}
}
\solutionspace{120pt}

\tasknumber{14}%
\task{%
    Расстояние от освещённого предмета до экрана $100\,\text{см}$.
    Линза, помещенная между ними, даёт чёткое изображение предмета на
    экране при двух положениях, расстояние между которыми $40\,\text{см}$.
    Найти фокусное расстояние линзы.
}
\answer{%
    \begin{align*}
    \frac 1a + \frac 1b &= \frac 1F, \frac 1{a-\ell} + \frac 1{b+\ell} = \frac 1F, a + b = L \\
    \frac 1a + \frac 1b &= \frac 1{a-\ell} + \frac 1{b+\ell}\implies \frac{a + b}{ab} = \frac{(a-\ell) + (b+\ell)}{(a-\ell)(b+\ell)} \\
    ab  &= (a - \ell)(b+\ell) \implies 0  = -b\ell + a\ell - \ell^2 \implies 0 = -b + a - \ell \implies b = a - \ell \\
    a + (a - \ell) &= L \implies a = \frac{L + \ell}2 \implies b = \frac{L - \ell}2 \\
    F &= \frac{ab}{a + b} = \frac{L^2 -\ell^2}{4L} \approx 21\,\text{см}.
    \end{align*}
}
\solutionspace{120pt}

\tasknumber{15}%
\task{%
    Предмет находится на расстоянии $80\,\text{см}$ от экрана.
    Между предметом и экраном помещают линзу, причём при одном
    положении линзы на экране получается увеличенное изображение предмета,
    а при другом — уменьшенное.
    Каково фокусное расстояние линзы, если
    линейные размеры первого изображения в пять раз больше второго?
}
\answer{%
    \begin{align*}
    \frac 1a + \frac 1{L-a} &= \frac 1F, h_1 = h \cdot \frac{L-a}a, \\
    \frac 1b + \frac 1{L-b} &= \frac 1F, h_2 = h \cdot \frac{L-b}b, \\
    \frac{h_1}{h_2} &= 5 \implies \frac{(L-a)b}{(L-b)a} = 5, \\
    \frac 1F &= \frac{ L }{a(L-a)} = \frac{ L }{b(L-b)} \implies \frac{L-a}{L-b} = \frac b a \implies \frac {b^2}{a^2} = 5.
    \\
    \frac 1a + \frac 1{L-a} &= \frac 1b + \frac 1{L-b} \implies \frac L{a(L-a)} = \frac L{b(L-b)} \implies \\
    \implies aL - a^2 &= bL - b^2 \implies (a-b)L = (a-b)(a+b) \implies b = L - a, \\
    \frac{\sqr{L-a}}{a^2} &= 5 \implies \frac La - 1 = \sqrt{5} \implies a = \frac{ L }{\sqrt{5} + 1} \\
    F &= \frac{a(L-a)}L = \frac 1L \cdot \frac L{\sqrt{5} + 1} \cdot \frac {L\sqrt{5}}{\sqrt{5} + 1}= \frac { L\sqrt{5} }{ \sqr{\sqrt{5} + 1} } \approx 17{,}1\,\text{см}.
    \end{align*}
}
\solutionspace{120pt}

\tasknumber{16}%
\task{%
    (Задача-«гроб»: решать на обратной стороне) Квадрат со стороной $d = 3\,\text{см}$ расположен так,
    что 2 его стороны параллельны главной оптической оси собирающей линзы,
    его центр удален на $h = 6\,\text{см}$ от этой оси и на $a = 10\,\text{см}$ от плоскости линзы.
    Определите площадь изображения квадрата, если фокусное расстояние линзы составляет $F = 18\,\text{см}$.
    % (и сравните с площадью объекта, умноженной на квадрат увеличения центра квадрата).
}
\answer{%
    \begin{align*}
    &\text{Все явные вычисления — в см и $\text{см}^2$,} \\
    \frac 1 F &= \frac 1{a + \frac d2} + \frac 1b \implies b = \frac 1{\frac 1 F - \frac 1{a + \frac d2}} = \frac{F(a + \frac d2)}{a + \frac d2 - F} = -\frac{414}{13}, \\
    \frac 1 F &= \frac 1{a - \frac d2} + \frac 1c \implies c = \frac 1{\frac 1 F - \frac 1{a - \frac d2}} = \frac{F(a - \frac d2)}{a - \frac d2 - F} = -\frac{306}{19}, \\
    c - b &= \frac{F(a - \frac d2)}{a - \frac d2 - F} - \frac{F(a + \frac d2)}{a + \frac d2 - F} = F\cbr{ \frac{a - \frac d2}{a - \frac d2 - F} - \frac{a + \frac d2}{a + \frac d2 - F} } =  \\
    &= F \cdot \frac{a^2 + \frac {ad}2 - aF - \frac{ad}2 - \frac{d^2}4 + \frac{dF}2 - a^2 + \frac {ad}2 + aF - \frac{ad}2 + \frac{d^2}4 + \frac{dF}2}{\cbr{a + \frac d2 - F}\cbr{a - \frac d2 - F}}= F \cdot \frac {dF}{\cbr{a + \frac d2 - F}\cbr{a - \frac d2 - F}} = \frac{3888}{247}.
    \\
    \Gamma_b &= \frac b{a + \frac d2} = \frac{ F }{a + \frac d2 - F} = -\frac{36}{13}, \\
    \Gamma_c &= \frac c{a - \frac d2} = \frac{ F }{a - \frac d2 - F} = -\frac{36}{19}, \\
    &\text{ тут интересно отметить, что } \Gamma_x = \frac{ c - b}{ d } = \frac{ F^2 }{\cbr{a + \frac d2 - F}\cbr{a - \frac d2 - F}} \ne \Gamma_b \text{ или } \Gamma_c \text{ даже при малых $d$}.
    \\
    S' &= \frac{d \cdot \Gamma_b + d \cdot \Gamma_c}2 \cdot (c - b) = \frac d2 \cbr{\frac{ F }{a + \frac d2 - F} + \frac{ F }{a - \frac d2 - F}} \cdot \cbr{c - b} =  \\
    &=\frac {dF}2 \cbr{\frac 1{a + \frac d2 - F} + \frac 1{a - \frac d2 - F}} \cdot \frac {dF^2}{\cbr{a + \frac d2 - F}\cbr{a - \frac d2 - F}} =  \\
    &=\frac {dF}2 \cdot \frac{a - \frac d2 - F + a + \frac d2 - F}{\cbr{a + \frac d2 - F}\cbr{a - \frac d2 - F}} \cdot \frac {dF^2}{\cbr{a + \frac d2 - F}\cbr{a - \frac d2 - F}} =  \\
    &= \frac {d^2F^3}{2\sqr{a + \frac d2 - F}\sqr{a - \frac d2 - F}} \cdot (2a - 2F) = \frac {d^2F^3(a - F)}{ \sqr{\sqr{a - F} - \frac{d^2}4} } = -\frac{6718464}{61009}.
    \end{align*}
}

\variantsplitter

\addpersonalvariant{Варвара Егиазарян}

\tasknumber{1}%
\task{%
    Запишите известные вам виды классификации изображений.
}
\solutionspace{60pt}

\tasknumber{2}%
\task{%
    В каких линзах можно получить действительное изображение объекта?
}
\answer{%
    $\text{ собирающие }$
}
\solutionspace{40pt}

\tasknumber{3}%
\task{%
    Какое изображение называют действительным?
}
\solutionspace{40pt}

\tasknumber{4}%
\task{%
    Есть две линзы, обозначим их 1 и 2.
    Известно что оптическая сила линзы 2 больше, чем у линзы 1.
    Какая линза сильнее преломляет лучи?
}
\answer{%
    $2$
}
\solutionspace{40pt}

\tasknumber{5}%
\task{%
    Предмет находится на расстоянии $20\,\text{см}$ от собирающей линзы с фокусным расстоянием $50\,\text{см}$.
    Определите тип изображения, расстояние между предметом и его изображением, увеличение предмета.
    Сделайте схематичный рисунок (не обязательно в масштабе, но с сохранением свойств линзы и изображения).
}
\solutionspace{100pt}

\tasknumber{6}%
\task{%
    Объект находится на расстоянии $115\,\text{см}$ от линзы, а его мнимое изображение — в $30\,\text{см}$ от неё.
    Определите увеличение предмета, фокусное расстояние линзы, оптическую силу линзы и её тип.
}
\solutionspace{80pt}

\tasknumber{7}%
\task{%
    Известно, что из формулы тонкой линзы $\cbr{\frac 1F = \frac 1a + \frac 1b}$
    и определения увеличения $\cbr{\Gamma_y = \frac ba}$ можно получить выражение
    для увеличения: $\Gamma_y = \frac {aF}{a - F} \cdot \frac 1a = \frac {F}{a - F}.$
    Назовём такое увеличение «поперечным»: поперёк главной оптической оси (поэтому и ${}_y$).
    Получите формулу для «продольного» увеличения $\Gamma_x$ небольшого предмета, находящегося на главной оптической оси.
    Можно ли применить эту формулу для предмета, не лежащего на главной оптической оси, почему?
}
\answer{%
    \begin{align*}
    \frac 1F &= \frac 1a + \frac 1b \implies b = \frac {aF}{a - F} \\
    \frac 1F &= \frac 1{a + x} + \frac 1c \implies c = \frac {(a+x)F}{a + x - F} \\
    x' &= \abs{b - c} = \frac {aF}{a - F} - \frac {(a+x)F}{a + x - F} = F\cbr{\frac {a}{a - F} - \frac {a+x}{a + x - F}} =  \\
    &= F \cdot \frac {a^2 + ax - aF - a^2 - ax + aF + xF}{(a - F)(a + x - F)} = F \cdot \frac {xF}{(a - F)(a + x - F)} \\
    \Gamma_x &= \frac{x'}x = \frac{F^2}{(a - F)(a + x - F)} \to \frac{F^2}{\sqr{a - F}}.
    \\
    &\text{Нельзя: изображение по-разному растянет по осям $x$ и $y$ и понадобится теорема Пифагора}
    \end{align*}
}
\solutionspace{150pt}

\tasknumber{8}%
\task{%
    Доказать формулу тонкой линзы для собирающей линзы.
}
\solutionspace{120pt}

\tasknumber{9}%
\task{%
    Постройте ход луча $CL$ в тонкой линзе.
    Известно положение линзы и оба её фокуса (см.
    рис.
    на доске).
    Рассмотрите оба типа линзы, сделав 2 рисунка: собирающую и рассеивающую.
}
\solutionspace{120pt}

\tasknumber{10}%
\task{%
    На экране, расположенном иа расстоянии $80\,\text{см}$ от собирающей линзы,
    получено изображение точечного источника, расположенного на главной оптической оси линзы.
    На какое расстояние переместится изображение на экране,
    если при неподвижной линзе переместить источник на $1\,\text{см}$ в плоскости, перпендикулярной главной оптической оси?
    Фокусное расстояние линзы равно $20\,\text{см}$.
}
\answer{%
    \begin{align*}
    &\frac 1F = \frac 1a + \frac 1b \implies a = \frac{bF}{b-F} \implies \Gamma = \frac ba = \frac{b-F}F \\
    &y = x \cdot \Gamma = x \cdot \frac{b-F}F \implies d = y = 3\,\text{см}.
    \end{align*}
}
\solutionspace{120pt}

\tasknumber{11}%
\task{%
    Оптическая сила двояковыпуклой линзы в воздухе $5{,}5\,\text{дптр}$, а в воде $1{,}6\,\text{дптр}$.
    Определить показатель преломления $n$ материала, из которого изготовлена линза.
    Показатель преломления воды равен $1{,}33$.
}
\answer{%
    \begin{align*}
    D_1 &=\cbr{\frac n{n_1} - 1}\cbr{\frac 1{R_1} + \frac 1{R_2}}, \\
    D_2 &=\cbr{\frac n{n_2} - 1}\cbr{\frac 1{R_1} + \frac 1{R_2}}, \\
    \frac {D_2}{D_1} &=\frac{\frac n{n_2} - 1}{\frac n{n_1} - 1} \implies {D_2}\cbr{\frac n{n_1} - 1} = {D_1}\cbr{\frac n{n_2} - 1}  \implies n\cbr{\frac{D_2}{n_1} - \frac{D_1}{n_2}} = D_2 - D_1, \\
    n &= \frac{D_2 - D_1}{\frac{D_2}{n_1} - \frac{D_1}{n_2}} = \frac{n_1 n_2 (D_2 - D_1)}{D_2n_2 - D_1n_1} \approx 1{,}538.
    \end{align*}
}
\solutionspace{120pt}

\tasknumber{12}%
\task{%
    На каком расстоянии от собирающей линзы с фокусным расстоянием $50\,\text{дптр}$
    следует надо поместить предмет, чтобы расстояние
    от предмета до его действительного изображения было наименьшим?
}
\answer{%
    \begin{align*}
    \frac 1a &+ \frac 1b = D \implies b = \frac 1{D - \frac 1a} \implies \ell = a + b = a + \frac a{Da - 1} = \frac{ Da^2 }{Da - 1} \implies \\
    \implies \ell'_a &= \frac{ 2Da \cdot (Da - 1) - Da^2 \cdot D }{\sqr{Da - 1}}= \frac{ D^2a^2 - 2Da}{\sqr{Da - 1}} = \frac{ Da(Da - 2)}{\sqr{Da - 1}}\implies a_{\min} = \frac 2D \approx 40\,\text{мм}.
    \end{align*}
}
\solutionspace{120pt}

\tasknumber{13}%
\task{%
    Даны точечный источник света $S$, его изображение $S_1$, полученное с помощью собирающей линзы,
    и ближайший к источнику фокус линзы $F$ (см.
    рис.
    на доске).
    Расстояния $SF = \ell$ и $SS_1 = L$.
    Определить положение линзы и её фокусное расстояние.
}
\answer{%
    \begin{align*}
    \frac 1a + \frac 1b &= \frac 1F, \ell = a - F, L = a + b \implies a = \ell + F, b = L - a = L - \ell - F \\
    \frac 1{\ell + F} + \frac 1{L - \ell - F} &= \frac 1F \\
    F\ell + F^2 + LF - F\ell - F^2 &= L\ell - \ell^2 - F\ell + LF - F\ell - F^2 \\
    0 &= L\ell - \ell^2 - 2F\ell - F^2 \\
    0 &=  F^2 + 2F\ell - L\ell + \ell^2 \\
    F &= -\ell \pm \sqrt{\ell^2 +  L\ell - \ell^2} = -\ell \pm \sqrt{L\ell} \implies F = \sqrt{L\ell} - \ell \\
    a &= \ell + F = \ell + \sqrt{L\ell} - \ell = \sqrt{L\ell}.
    \end{align*}
}
\solutionspace{120pt}

\tasknumber{14}%
\task{%
    Расстояние от освещённого предмета до экрана $100\,\text{см}$.
    Линза, помещенная между ними, даёт чёткое изображение предмета на
    экране при двух положениях, расстояние между которыми $30\,\text{см}$.
    Найти фокусное расстояние линзы.
}
\answer{%
    \begin{align*}
    \frac 1a + \frac 1b &= \frac 1F, \frac 1{a-\ell} + \frac 1{b+\ell} = \frac 1F, a + b = L \\
    \frac 1a + \frac 1b &= \frac 1{a-\ell} + \frac 1{b+\ell}\implies \frac{a + b}{ab} = \frac{(a-\ell) + (b+\ell)}{(a-\ell)(b+\ell)} \\
    ab  &= (a - \ell)(b+\ell) \implies 0  = -b\ell + a\ell - \ell^2 \implies 0 = -b + a - \ell \implies b = a - \ell \\
    a + (a - \ell) &= L \implies a = \frac{L + \ell}2 \implies b = \frac{L - \ell}2 \\
    F &= \frac{ab}{a + b} = \frac{L^2 -\ell^2}{4L} \approx 22{,}8\,\text{см}.
    \end{align*}
}
\solutionspace{120pt}

\tasknumber{15}%
\task{%
    Предмет находится на расстоянии $70\,\text{см}$ от экрана.
    Между предметом и экраном помещают линзу, причём при одном
    положении линзы на экране получается увеличенное изображение предмета,
    а при другом — уменьшенное.
    Каково фокусное расстояние линзы, если
    линейные размеры первого изображения в три раза больше второго?
}
\answer{%
    \begin{align*}
    \frac 1a + \frac 1{L-a} &= \frac 1F, h_1 = h \cdot \frac{L-a}a, \\
    \frac 1b + \frac 1{L-b} &= \frac 1F, h_2 = h \cdot \frac{L-b}b, \\
    \frac{h_1}{h_2} &= 3 \implies \frac{(L-a)b}{(L-b)a} = 3, \\
    \frac 1F &= \frac{ L }{a(L-a)} = \frac{ L }{b(L-b)} \implies \frac{L-a}{L-b} = \frac b a \implies \frac {b^2}{a^2} = 3.
    \\
    \frac 1a + \frac 1{L-a} &= \frac 1b + \frac 1{L-b} \implies \frac L{a(L-a)} = \frac L{b(L-b)} \implies \\
    \implies aL - a^2 &= bL - b^2 \implies (a-b)L = (a-b)(a+b) \implies b = L - a, \\
    \frac{\sqr{L-a}}{a^2} &= 3 \implies \frac La - 1 = \sqrt{3} \implies a = \frac{ L }{\sqrt{3} + 1} \\
    F &= \frac{a(L-a)}L = \frac 1L \cdot \frac L{\sqrt{3} + 1} \cdot \frac {L\sqrt{3}}{\sqrt{3} + 1}= \frac { L\sqrt{3} }{ \sqr{\sqrt{3} + 1} } \approx 16{,}2\,\text{см}.
    \end{align*}
}
\solutionspace{120pt}

\tasknumber{16}%
\task{%
    (Задача-«гроб»: решать на обратной стороне) Квадрат со стороной $d = 2\,\text{см}$ расположен так,
    что 2 его стороны параллельны главной оптической оси рассеивающей линзы,
    его центр удален на $h = 4\,\text{см}$ от этой оси и на $a = 15\,\text{см}$ от плоскости линзы.
    Определите площадь изображения квадрата, если фокусное расстояние линзы составляет $F = 25\,\text{см}$.
    % (и сравните с площадью объекта, умноженной на квадрат увеличения центра квадрата).
}
\answer{%
    \begin{align*}
    &\text{Все явные вычисления — в см и $\text{см}^2$,} \\
    \frac 1 F &= \frac 1{a + \frac d2} + \frac 1b \implies b = \frac 1{\frac 1 F - \frac 1{a + \frac d2}} = \frac{F(a + \frac d2)}{a + \frac d2 - F} = -\frac{400}{41}, \\
    \frac 1 F &= \frac 1{a - \frac d2} + \frac 1c \implies c = \frac 1{\frac 1 F - \frac 1{a - \frac d2}} = \frac{F(a - \frac d2)}{a - \frac d2 - F} = -\frac{350}{39}, \\
    c - b &= \frac{F(a - \frac d2)}{a - \frac d2 - F} - \frac{F(a + \frac d2)}{a + \frac d2 - F} = F\cbr{ \frac{a - \frac d2}{a - \frac d2 - F} - \frac{a + \frac d2}{a + \frac d2 - F} } =  \\
    &= F \cdot \frac{a^2 + \frac {ad}2 - aF - \frac{ad}2 - \frac{d^2}4 + \frac{dF}2 - a^2 + \frac {ad}2 + aF - \frac{ad}2 + \frac{d^2}4 + \frac{dF}2}{\cbr{a + \frac d2 - F}\cbr{a - \frac d2 - F}}= F \cdot \frac {dF}{\cbr{a + \frac d2 - F}\cbr{a - \frac d2 - F}} = \frac{1250}{1599}.
    \\
    \Gamma_b &= \frac b{a + \frac d2} = \frac{ F }{a + \frac d2 - F} = -\frac{25}{41}, \\
    \Gamma_c &= \frac c{a - \frac d2} = \frac{ F }{a - \frac d2 - F} = -\frac{25}{39}, \\
    &\text{ тут интересно отметить, что } \Gamma_x = \frac{ c - b}{ d } = \frac{ F^2 }{\cbr{a + \frac d2 - F}\cbr{a - \frac d2 - F}} \ne \Gamma_b \text{ или } \Gamma_c \text{ даже при малых $d$}.
    \\
    S' &= \frac{d \cdot \Gamma_b + d \cdot \Gamma_c}2 \cdot (c - b) = \frac d2 \cbr{\frac{ F }{a + \frac d2 - F} + \frac{ F }{a - \frac d2 - F}} \cdot \cbr{c - b} =  \\
    &=\frac {dF}2 \cbr{\frac 1{a + \frac d2 - F} + \frac 1{a - \frac d2 - F}} \cdot \frac {dF^2}{\cbr{a + \frac d2 - F}\cbr{a - \frac d2 - F}} =  \\
    &=\frac {dF}2 \cdot \frac{a - \frac d2 - F + a + \frac d2 - F}{\cbr{a + \frac d2 - F}\cbr{a - \frac d2 - F}} \cdot \frac {dF^2}{\cbr{a + \frac d2 - F}\cbr{a - \frac d2 - F}} =  \\
    &= \frac {d^2F^3}{2\sqr{a + \frac d2 - F}\sqr{a - \frac d2 - F}} \cdot (2a - 2F) = \frac {d^2F^3(a - F)}{ \sqr{\sqr{a - F} - \frac{d^2}4} } = -\frac{2500000}{2556801}.
    \end{align*}
}

\variantsplitter

\addpersonalvariant{Владислав Емелин}

\tasknumber{1}%
\task{%
    Запишите формулу тонкой линзы и сделайте рисунок, указав на нём физические величины из этой формулы.
}
\solutionspace{60pt}

\tasknumber{2}%
\task{%
    В каких линзах можно получить действительное изображение объекта?
}
\answer{%
    $\text{ собирающие }$
}
\solutionspace{40pt}

\tasknumber{3}%
\task{%
    Какое изображение называют действительным?
}
\solutionspace{40pt}

\tasknumber{4}%
\task{%
    Есть две линзы, обозначим их 1 и 2.
    Известно что фокусное расстояние линзы 1 больше, чем у линзы 2.
    Какая линза сильнее преломляет лучи?
}
\answer{%
    $2$
}
\solutionspace{40pt}

\tasknumber{5}%
\task{%
    Предмет находится на расстоянии $10\,\text{см}$ от рассеивающей линзы с фокусным расстоянием $15\,\text{см}$.
    Определите тип изображения, расстояние между предметом и его изображением, увеличение предмета.
    Сделайте схематичный рисунок (не обязательно в масштабе, но с сохранением свойств линзы и изображения).
}
\solutionspace{100pt}

\tasknumber{6}%
\task{%
    Объект находится на расстоянии $25\,\text{см}$ от линзы, а его мнимое изображение — в $40\,\text{см}$ от неё.
    Определите увеличение предмета, фокусное расстояние линзы, оптическую силу линзы и её тип.
}
\solutionspace{80pt}

\tasknumber{7}%
\task{%
    Известно, что из формулы тонкой линзы $\cbr{\frac 1F = \frac 1a + \frac 1b}$
    и определения увеличения $\cbr{\Gamma_y = \frac ba}$ можно получить выражение
    для увеличения: $\Gamma_y = \frac {aF}{a - F} \cdot \frac 1a = \frac {F}{a - F}.$
    Назовём такое увеличение «поперечным»: поперёк главной оптической оси (поэтому и ${}_y$).
    Получите формулу для «продольного» увеличения $\Gamma_x$ небольшого предмета, находящегося на главной оптической оси.
    Можно ли применить эту формулу для предмета, не лежащего на главной оптической оси, почему?
}
\answer{%
    \begin{align*}
    \frac 1F &= \frac 1a + \frac 1b \implies b = \frac {aF}{a - F} \\
    \frac 1F &= \frac 1{a + x} + \frac 1c \implies c = \frac {(a+x)F}{a + x - F} \\
    x' &= \abs{b - c} = \frac {aF}{a - F} - \frac {(a+x)F}{a + x - F} = F\cbr{\frac {a}{a - F} - \frac {a+x}{a + x - F}} =  \\
    &= F \cdot \frac {a^2 + ax - aF - a^2 - ax + aF + xF}{(a - F)(a + x - F)} = F \cdot \frac {xF}{(a - F)(a + x - F)} \\
    \Gamma_x &= \frac{x'}x = \frac{F^2}{(a - F)(a + x - F)} \to \frac{F^2}{\sqr{a - F}}.
    \\
    &\text{Нельзя: изображение по-разному растянет по осям $x$ и $y$ и понадобится теорема Пифагора}
    \end{align*}
}
\solutionspace{150pt}

\tasknumber{8}%
\task{%
    Доказать формулу тонкой линзы для собирающей линзы.
}
\solutionspace{120pt}

\tasknumber{9}%
\task{%
    Постройте ход луча $AL$ в тонкой линзе.
    Известно положение линзы и оба её фокуса (см.
    рис.
    на доске).
    Рассмотрите оба типа линзы, сделав 2 рисунка: собирающую и рассеивающую.
}
\solutionspace{120pt}

\tasknumber{10}%
\task{%
    На экране, расположенном иа расстоянии $120\,\text{см}$ от собирающей линзы,
    получено изображение точечного источника, расположенного на главной оптической оси линзы.
    На какое расстояние переместится изображение на экране,
    если при неподвижном источнике переместить линзу на $2\,\text{см}$ в плоскости, перпендикулярной главной оптической оси?
    Фокусное расстояние линзы равно $40\,\text{см}$.
}
\answer{%
    \begin{align*}
    &\frac 1F = \frac 1a + \frac 1b \implies a = \frac{bF}{b-F} \implies \Gamma = \frac ba = \frac{b-F}F \\
    &y = x \cdot \Gamma = x \cdot \frac{b-F}F \implies d = x + y = 6\,\text{см}.
    \end{align*}
}
\solutionspace{120pt}

\tasknumber{11}%
\task{%
    Оптическая сила двояковыпуклой линзы в воздухе $4{,}5\,\text{дптр}$, а в воде $1{,}4\,\text{дптр}$.
    Определить показатель преломления $n$ материала, из которого изготовлена линза.
    Показатель преломления воды равен $1{,}33$.
}
\answer{%
    \begin{align*}
    D_1 &=\cbr{\frac n{n_1} - 1}\cbr{\frac 1{R_1} + \frac 1{R_2}}, \\
    D_2 &=\cbr{\frac n{n_2} - 1}\cbr{\frac 1{R_1} + \frac 1{R_2}}, \\
    \frac {D_2}{D_1} &=\frac{\frac n{n_2} - 1}{\frac n{n_1} - 1} \implies {D_2}\cbr{\frac n{n_1} - 1} = {D_1}\cbr{\frac n{n_2} - 1}  \implies n\cbr{\frac{D_2}{n_1} - \frac{D_1}{n_2}} = D_2 - D_1, \\
    n &= \frac{D_2 - D_1}{\frac{D_2}{n_1} - \frac{D_1}{n_2}} = \frac{n_1 n_2 (D_2 - D_1)}{D_2n_2 - D_1n_1} \approx 1{,}563.
    \end{align*}
}
\solutionspace{120pt}

\tasknumber{12}%
\task{%
    На каком расстоянии от собирающей линзы с фокусным расстоянием $40\,\text{дптр}$
    следует надо поместить предмет, чтобы расстояние
    от предмета до его действительного изображения было наименьшим?
}
\answer{%
    \begin{align*}
    \frac 1a &+ \frac 1b = D \implies b = \frac 1{D - \frac 1a} \implies \ell = a + b = a + \frac a{Da - 1} = \frac{ Da^2 }{Da - 1} \implies \\
    \implies \ell'_a &= \frac{ 2Da \cdot (Da - 1) - Da^2 \cdot D }{\sqr{Da - 1}}= \frac{ D^2a^2 - 2Da}{\sqr{Da - 1}} = \frac{ Da(Da - 2)}{\sqr{Da - 1}}\implies a_{\min} = \frac 2D \approx 50\,\text{мм}.
    \end{align*}
}
\solutionspace{120pt}

\tasknumber{13}%
\task{%
    Даны точечный источник света $S$, его изображение $S_1$, полученное с помощью собирающей линзы,
    и ближайший к источнику фокус линзы $F$ (см.
    рис.
    на доске).
    Расстояния $SF = \ell$ и $SS_1 = L$.
    Определить положение линзы и её фокусное расстояние.
}
\answer{%
    \begin{align*}
    \frac 1a + \frac 1b &= \frac 1F, \ell = a - F, L = a + b \implies a = \ell + F, b = L - a = L - \ell - F \\
    \frac 1{\ell + F} + \frac 1{L - \ell - F} &= \frac 1F \\
    F\ell + F^2 + LF - F\ell - F^2 &= L\ell - \ell^2 - F\ell + LF - F\ell - F^2 \\
    0 &= L\ell - \ell^2 - 2F\ell - F^2 \\
    0 &=  F^2 + 2F\ell - L\ell + \ell^2 \\
    F &= -\ell \pm \sqrt{\ell^2 +  L\ell - \ell^2} = -\ell \pm \sqrt{L\ell} \implies F = \sqrt{L\ell} - \ell \\
    a &= \ell + F = \ell + \sqrt{L\ell} - \ell = \sqrt{L\ell}.
    \end{align*}
}
\solutionspace{120pt}

\tasknumber{14}%
\task{%
    Расстояние от освещённого предмета до экрана $80\,\text{см}$.
    Линза, помещенная между ними, даёт чёткое изображение предмета на
    экране при двух положениях, расстояние между которыми $40\,\text{см}$.
    Найти фокусное расстояние линзы.
}
\answer{%
    \begin{align*}
    \frac 1a + \frac 1b &= \frac 1F, \frac 1{a-\ell} + \frac 1{b+\ell} = \frac 1F, a + b = L \\
    \frac 1a + \frac 1b &= \frac 1{a-\ell} + \frac 1{b+\ell}\implies \frac{a + b}{ab} = \frac{(a-\ell) + (b+\ell)}{(a-\ell)(b+\ell)} \\
    ab  &= (a - \ell)(b+\ell) \implies 0  = -b\ell + a\ell - \ell^2 \implies 0 = -b + a - \ell \implies b = a - \ell \\
    a + (a - \ell) &= L \implies a = \frac{L + \ell}2 \implies b = \frac{L - \ell}2 \\
    F &= \frac{ab}{a + b} = \frac{L^2 -\ell^2}{4L} \approx 15\,\text{см}.
    \end{align*}
}
\solutionspace{120pt}

\tasknumber{15}%
\task{%
    Предмет находится на расстоянии $80\,\text{см}$ от экрана.
    Между предметом и экраном помещают линзу, причём при одном
    положении линзы на экране получается увеличенное изображение предмета,
    а при другом — уменьшенное.
    Каково фокусное расстояние линзы, если
    линейные размеры первого изображения в пять раз больше второго?
}
\answer{%
    \begin{align*}
    \frac 1a + \frac 1{L-a} &= \frac 1F, h_1 = h \cdot \frac{L-a}a, \\
    \frac 1b + \frac 1{L-b} &= \frac 1F, h_2 = h \cdot \frac{L-b}b, \\
    \frac{h_1}{h_2} &= 5 \implies \frac{(L-a)b}{(L-b)a} = 5, \\
    \frac 1F &= \frac{ L }{a(L-a)} = \frac{ L }{b(L-b)} \implies \frac{L-a}{L-b} = \frac b a \implies \frac {b^2}{a^2} = 5.
    \\
    \frac 1a + \frac 1{L-a} &= \frac 1b + \frac 1{L-b} \implies \frac L{a(L-a)} = \frac L{b(L-b)} \implies \\
    \implies aL - a^2 &= bL - b^2 \implies (a-b)L = (a-b)(a+b) \implies b = L - a, \\
    \frac{\sqr{L-a}}{a^2} &= 5 \implies \frac La - 1 = \sqrt{5} \implies a = \frac{ L }{\sqrt{5} + 1} \\
    F &= \frac{a(L-a)}L = \frac 1L \cdot \frac L{\sqrt{5} + 1} \cdot \frac {L\sqrt{5}}{\sqrt{5} + 1}= \frac { L\sqrt{5} }{ \sqr{\sqrt{5} + 1} } \approx 17{,}1\,\text{см}.
    \end{align*}
}
\solutionspace{120pt}

\tasknumber{16}%
\task{%
    (Задача-«гроб»: решать на обратной стороне) Квадрат со стороной $d = 2\,\text{см}$ расположен так,
    что 2 его стороны параллельны главной оптической оси собирающей линзы,
    его центр удален на $h = 6\,\text{см}$ от этой оси и на $a = 12\,\text{см}$ от плоскости линзы.
    Определите площадь изображения квадрата, если фокусное расстояние линзы составляет $F = 20\,\text{см}$.
    % (и сравните с площадью объекта, умноженной на квадрат увеличения центра квадрата).
}
\answer{%
    \begin{align*}
    &\text{Все явные вычисления — в см и $\text{см}^2$,} \\
    \frac 1 F &= \frac 1{a + \frac d2} + \frac 1b \implies b = \frac 1{\frac 1 F - \frac 1{a + \frac d2}} = \frac{F(a + \frac d2)}{a + \frac d2 - F} = -\frac{260}7, \\
    \frac 1 F &= \frac 1{a - \frac d2} + \frac 1c \implies c = \frac 1{\frac 1 F - \frac 1{a - \frac d2}} = \frac{F(a - \frac d2)}{a - \frac d2 - F} = -\frac{220}9, \\
    c - b &= \frac{F(a - \frac d2)}{a - \frac d2 - F} - \frac{F(a + \frac d2)}{a + \frac d2 - F} = F\cbr{ \frac{a - \frac d2}{a - \frac d2 - F} - \frac{a + \frac d2}{a + \frac d2 - F} } =  \\
    &= F \cdot \frac{a^2 + \frac {ad}2 - aF - \frac{ad}2 - \frac{d^2}4 + \frac{dF}2 - a^2 + \frac {ad}2 + aF - \frac{ad}2 + \frac{d^2}4 + \frac{dF}2}{\cbr{a + \frac d2 - F}\cbr{a - \frac d2 - F}}= F \cdot \frac {dF}{\cbr{a + \frac d2 - F}\cbr{a - \frac d2 - F}} = \frac{800}{63}.
    \\
    \Gamma_b &= \frac b{a + \frac d2} = \frac{ F }{a + \frac d2 - F} = -\frac{20}7, \\
    \Gamma_c &= \frac c{a - \frac d2} = \frac{ F }{a - \frac d2 - F} = -\frac{20}9, \\
    &\text{ тут интересно отметить, что } \Gamma_x = \frac{ c - b}{ d } = \frac{ F^2 }{\cbr{a + \frac d2 - F}\cbr{a - \frac d2 - F}} \ne \Gamma_b \text{ или } \Gamma_c \text{ даже при малых $d$}.
    \\
    S' &= \frac{d \cdot \Gamma_b + d \cdot \Gamma_c}2 \cdot (c - b) = \frac d2 \cbr{\frac{ F }{a + \frac d2 - F} + \frac{ F }{a - \frac d2 - F}} \cdot \cbr{c - b} =  \\
    &=\frac {dF}2 \cbr{\frac 1{a + \frac d2 - F} + \frac 1{a - \frac d2 - F}} \cdot \frac {dF^2}{\cbr{a + \frac d2 - F}\cbr{a - \frac d2 - F}} =  \\
    &=\frac {dF}2 \cdot \frac{a - \frac d2 - F + a + \frac d2 - F}{\cbr{a + \frac d2 - F}\cbr{a - \frac d2 - F}} \cdot \frac {dF^2}{\cbr{a + \frac d2 - F}\cbr{a - \frac d2 - F}} =  \\
    &= \frac {d^2F^3}{2\sqr{a + \frac d2 - F}\sqr{a - \frac d2 - F}} \cdot (2a - 2F) = \frac {d^2F^3(a - F)}{ \sqr{\sqr{a - F} - \frac{d^2}4} } = -\frac{256000}{3969}.
    \end{align*}
}

\variantsplitter

\addpersonalvariant{Артём Жичин}

\tasknumber{1}%
\task{%
    Запишите формулу тонкой линзы и сделайте рисунок, указав на нём физические величины из этой формулы.
}
\solutionspace{60pt}

\tasknumber{2}%
\task{%
    В каких линзах можно получить уменьшенное изображение объекта?
}
\answer{%
    $\text{ собирающие и рассеивающие }$
}
\solutionspace{40pt}

\tasknumber{3}%
\task{%
    Какое изображение называют мнимым?
}
\solutionspace{40pt}

\tasknumber{4}%
\task{%
    Есть две линзы, обозначим их 1 и 2.
    Известно что оптическая сила линзы 2 больше, чем у линзы 1.
    Какая линза сильнее преломляет лучи?
}
\answer{%
    $2$
}
\solutionspace{40pt}

\tasknumber{5}%
\task{%
    Предмет находится на расстоянии $30\,\text{см}$ от рассеивающей линзы с фокусным расстоянием $25\,\text{см}$.
    Определите тип изображения, расстояние между предметом и его изображением, увеличение предмета.
    Сделайте схематичный рисунок (не обязательно в масштабе, но с сохранением свойств линзы и изображения).
}
\solutionspace{100pt}

\tasknumber{6}%
\task{%
    Объект находится на расстоянии $45\,\text{см}$ от линзы, а его мнимое изображение — в $20\,\text{см}$ от неё.
    Определите увеличение предмета, фокусное расстояние линзы, оптическую силу линзы и её тип.
}
\solutionspace{80pt}

\tasknumber{7}%
\task{%
    Известно, что из формулы тонкой линзы $\cbr{\frac 1F = \frac 1a + \frac 1b}$
    и определения увеличения $\cbr{\Gamma_y = \frac ba}$ можно получить выражение
    для увеличения: $\Gamma_y = \frac {aF}{a - F} \cdot \frac 1a = \frac {F}{a - F}.$
    Назовём такое увеличение «поперечным»: поперёк главной оптической оси (поэтому и ${}_y$).
    Получите формулу для «продольного» увеличения $\Gamma_x$ небольшого предмета, находящегося на главной оптической оси.
    Можно ли применить эту формулу для предмета, не лежащего на главной оптической оси, почему?
}
\answer{%
    \begin{align*}
    \frac 1F &= \frac 1a + \frac 1b \implies b = \frac {aF}{a - F} \\
    \frac 1F &= \frac 1{a + x} + \frac 1c \implies c = \frac {(a+x)F}{a + x - F} \\
    x' &= \abs{b - c} = \frac {aF}{a - F} - \frac {(a+x)F}{a + x - F} = F\cbr{\frac {a}{a - F} - \frac {a+x}{a + x - F}} =  \\
    &= F \cdot \frac {a^2 + ax - aF - a^2 - ax + aF + xF}{(a - F)(a + x - F)} = F \cdot \frac {xF}{(a - F)(a + x - F)} \\
    \Gamma_x &= \frac{x'}x = \frac{F^2}{(a - F)(a + x - F)} \to \frac{F^2}{\sqr{a - F}}.
    \\
    &\text{Нельзя: изображение по-разному растянет по осям $x$ и $y$ и понадобится теорема Пифагора}
    \end{align*}
}
\solutionspace{150pt}

\tasknumber{8}%
\task{%
    Доказать формулу тонкой линзы для рассеивающей линзы.
}
\solutionspace{120pt}

\tasknumber{9}%
\task{%
    Постройте ход луча $BM$ в тонкой линзе.
    Известно положение линзы и оба её фокуса (см.
    рис.
    на доске).
    Рассмотрите оба типа линзы, сделав 2 рисунка: собирающую и рассеивающую.
}
\solutionspace{120pt}

\tasknumber{10}%
\task{%
    На экране, расположенном иа расстоянии $120\,\text{см}$ от собирающей линзы,
    получено изображение точечного источника, расположенного на главной оптической оси линзы.
    На какое расстояние переместится изображение на экране,
    если при неподвижном источнике переместить линзу на $1\,\text{см}$ в плоскости, перпендикулярной главной оптической оси?
    Фокусное расстояние линзы равно $20\,\text{см}$.
}
\answer{%
    \begin{align*}
    &\frac 1F = \frac 1a + \frac 1b \implies a = \frac{bF}{b-F} \implies \Gamma = \frac ba = \frac{b-F}F \\
    &y = x \cdot \Gamma = x \cdot \frac{b-F}F \implies d = x + y = 6\,\text{см}.
    \end{align*}
}
\solutionspace{120pt}

\tasknumber{11}%
\task{%
    Оптическая сила двояковыпуклой линзы в воздухе $5{,}5\,\text{дптр}$, а в воде $1{,}6\,\text{дптр}$.
    Определить показатель преломления $n$ материала, из которого изготовлена линза.
    Показатель преломления воды равен $1{,}33$.
}
\answer{%
    \begin{align*}
    D_1 &=\cbr{\frac n{n_1} - 1}\cbr{\frac 1{R_1} + \frac 1{R_2}}, \\
    D_2 &=\cbr{\frac n{n_2} - 1}\cbr{\frac 1{R_1} + \frac 1{R_2}}, \\
    \frac {D_2}{D_1} &=\frac{\frac n{n_2} - 1}{\frac n{n_1} - 1} \implies {D_2}\cbr{\frac n{n_1} - 1} = {D_1}\cbr{\frac n{n_2} - 1}  \implies n\cbr{\frac{D_2}{n_1} - \frac{D_1}{n_2}} = D_2 - D_1, \\
    n &= \frac{D_2 - D_1}{\frac{D_2}{n_1} - \frac{D_1}{n_2}} = \frac{n_1 n_2 (D_2 - D_1)}{D_2n_2 - D_1n_1} \approx 1{,}538.
    \end{align*}
}
\solutionspace{120pt}

\tasknumber{12}%
\task{%
    На каком расстоянии от собирающей линзы с фокусным расстоянием $30\,\text{дптр}$
    следует надо поместить предмет, чтобы расстояние
    от предмета до его действительного изображения было наименьшим?
}
\answer{%
    \begin{align*}
    \frac 1a &+ \frac 1b = D \implies b = \frac 1{D - \frac 1a} \implies \ell = a + b = a + \frac a{Da - 1} = \frac{ Da^2 }{Da - 1} \implies \\
    \implies \ell'_a &= \frac{ 2Da \cdot (Da - 1) - Da^2 \cdot D }{\sqr{Da - 1}}= \frac{ D^2a^2 - 2Da}{\sqr{Da - 1}} = \frac{ Da(Da - 2)}{\sqr{Da - 1}}\implies a_{\min} = \frac 2D \approx 66{,}7\,\text{мм}.
    \end{align*}
}
\solutionspace{120pt}

\tasknumber{13}%
\task{%
    Даны точечный источник света $S$, его изображение $S_1$, полученное с помощью собирающей линзы,
    и ближайший к источнику фокус линзы $F$ (см.
    рис.
    на доске).
    Расстояния $SF = \ell$ и $SS_1 = L$.
    Определить положение линзы и её фокусное расстояние.
}
\answer{%
    \begin{align*}
    \frac 1a + \frac 1b &= \frac 1F, \ell = a - F, L = a + b \implies a = \ell + F, b = L - a = L - \ell - F \\
    \frac 1{\ell + F} + \frac 1{L - \ell - F} &= \frac 1F \\
    F\ell + F^2 + LF - F\ell - F^2 &= L\ell - \ell^2 - F\ell + LF - F\ell - F^2 \\
    0 &= L\ell - \ell^2 - 2F\ell - F^2 \\
    0 &=  F^2 + 2F\ell - L\ell + \ell^2 \\
    F &= -\ell \pm \sqrt{\ell^2 +  L\ell - \ell^2} = -\ell \pm \sqrt{L\ell} \implies F = \sqrt{L\ell} - \ell \\
    a &= \ell + F = \ell + \sqrt{L\ell} - \ell = \sqrt{L\ell}.
    \end{align*}
}
\solutionspace{120pt}

\tasknumber{14}%
\task{%
    Расстояние от освещённого предмета до экрана $80\,\text{см}$.
    Линза, помещенная между ними, даёт чёткое изображение предмета на
    экране при двух положениях, расстояние между которыми $30\,\text{см}$.
    Найти фокусное расстояние линзы.
}
\answer{%
    \begin{align*}
    \frac 1a + \frac 1b &= \frac 1F, \frac 1{a-\ell} + \frac 1{b+\ell} = \frac 1F, a + b = L \\
    \frac 1a + \frac 1b &= \frac 1{a-\ell} + \frac 1{b+\ell}\implies \frac{a + b}{ab} = \frac{(a-\ell) + (b+\ell)}{(a-\ell)(b+\ell)} \\
    ab  &= (a - \ell)(b+\ell) \implies 0  = -b\ell + a\ell - \ell^2 \implies 0 = -b + a - \ell \implies b = a - \ell \\
    a + (a - \ell) &= L \implies a = \frac{L + \ell}2 \implies b = \frac{L - \ell}2 \\
    F &= \frac{ab}{a + b} = \frac{L^2 -\ell^2}{4L} \approx 17{,}2\,\text{см}.
    \end{align*}
}
\solutionspace{120pt}

\tasknumber{15}%
\task{%
    Предмет находится на расстоянии $90\,\text{см}$ от экрана.
    Между предметом и экраном помещают линзу, причём при одном
    положении линзы на экране получается увеличенное изображение предмета,
    а при другом — уменьшенное.
    Каково фокусное расстояние линзы, если
    линейные размеры первого изображения в пять раз больше второго?
}
\answer{%
    \begin{align*}
    \frac 1a + \frac 1{L-a} &= \frac 1F, h_1 = h \cdot \frac{L-a}a, \\
    \frac 1b + \frac 1{L-b} &= \frac 1F, h_2 = h \cdot \frac{L-b}b, \\
    \frac{h_1}{h_2} &= 5 \implies \frac{(L-a)b}{(L-b)a} = 5, \\
    \frac 1F &= \frac{ L }{a(L-a)} = \frac{ L }{b(L-b)} \implies \frac{L-a}{L-b} = \frac b a \implies \frac {b^2}{a^2} = 5.
    \\
    \frac 1a + \frac 1{L-a} &= \frac 1b + \frac 1{L-b} \implies \frac L{a(L-a)} = \frac L{b(L-b)} \implies \\
    \implies aL - a^2 &= bL - b^2 \implies (a-b)L = (a-b)(a+b) \implies b = L - a, \\
    \frac{\sqr{L-a}}{a^2} &= 5 \implies \frac La - 1 = \sqrt{5} \implies a = \frac{ L }{\sqrt{5} + 1} \\
    F &= \frac{a(L-a)}L = \frac 1L \cdot \frac L{\sqrt{5} + 1} \cdot \frac {L\sqrt{5}}{\sqrt{5} + 1}= \frac { L\sqrt{5} }{ \sqr{\sqrt{5} + 1} } \approx 19{,}2\,\text{см}.
    \end{align*}
}
\solutionspace{120pt}

\tasknumber{16}%
\task{%
    (Задача-«гроб»: решать на обратной стороне) Квадрат со стороной $d = 3\,\text{см}$ расположен так,
    что 2 его стороны параллельны главной оптической оси собирающей линзы,
    его центр удален на $h = 5\,\text{см}$ от этой оси и на $a = 10\,\text{см}$ от плоскости линзы.
    Определите площадь изображения квадрата, если фокусное расстояние линзы составляет $F = 18\,\text{см}$.
    % (и сравните с площадью объекта, умноженной на квадрат увеличения центра квадрата).
}
\answer{%
    \begin{align*}
    &\text{Все явные вычисления — в см и $\text{см}^2$,} \\
    \frac 1 F &= \frac 1{a + \frac d2} + \frac 1b \implies b = \frac 1{\frac 1 F - \frac 1{a + \frac d2}} = \frac{F(a + \frac d2)}{a + \frac d2 - F} = -\frac{414}{13}, \\
    \frac 1 F &= \frac 1{a - \frac d2} + \frac 1c \implies c = \frac 1{\frac 1 F - \frac 1{a - \frac d2}} = \frac{F(a - \frac d2)}{a - \frac d2 - F} = -\frac{306}{19}, \\
    c - b &= \frac{F(a - \frac d2)}{a - \frac d2 - F} - \frac{F(a + \frac d2)}{a + \frac d2 - F} = F\cbr{ \frac{a - \frac d2}{a - \frac d2 - F} - \frac{a + \frac d2}{a + \frac d2 - F} } =  \\
    &= F \cdot \frac{a^2 + \frac {ad}2 - aF - \frac{ad}2 - \frac{d^2}4 + \frac{dF}2 - a^2 + \frac {ad}2 + aF - \frac{ad}2 + \frac{d^2}4 + \frac{dF}2}{\cbr{a + \frac d2 - F}\cbr{a - \frac d2 - F}}= F \cdot \frac {dF}{\cbr{a + \frac d2 - F}\cbr{a - \frac d2 - F}} = \frac{3888}{247}.
    \\
    \Gamma_b &= \frac b{a + \frac d2} = \frac{ F }{a + \frac d2 - F} = -\frac{36}{13}, \\
    \Gamma_c &= \frac c{a - \frac d2} = \frac{ F }{a - \frac d2 - F} = -\frac{36}{19}, \\
    &\text{ тут интересно отметить, что } \Gamma_x = \frac{ c - b}{ d } = \frac{ F^2 }{\cbr{a + \frac d2 - F}\cbr{a - \frac d2 - F}} \ne \Gamma_b \text{ или } \Gamma_c \text{ даже при малых $d$}.
    \\
    S' &= \frac{d \cdot \Gamma_b + d \cdot \Gamma_c}2 \cdot (c - b) = \frac d2 \cbr{\frac{ F }{a + \frac d2 - F} + \frac{ F }{a - \frac d2 - F}} \cdot \cbr{c - b} =  \\
    &=\frac {dF}2 \cbr{\frac 1{a + \frac d2 - F} + \frac 1{a - \frac d2 - F}} \cdot \frac {dF^2}{\cbr{a + \frac d2 - F}\cbr{a - \frac d2 - F}} =  \\
    &=\frac {dF}2 \cdot \frac{a - \frac d2 - F + a + \frac d2 - F}{\cbr{a + \frac d2 - F}\cbr{a - \frac d2 - F}} \cdot \frac {dF^2}{\cbr{a + \frac d2 - F}\cbr{a - \frac d2 - F}} =  \\
    &= \frac {d^2F^3}{2\sqr{a + \frac d2 - F}\sqr{a - \frac d2 - F}} \cdot (2a - 2F) = \frac {d^2F^3(a - F)}{ \sqr{\sqr{a - F} - \frac{d^2}4} } = -\frac{6718464}{61009}.
    \end{align*}
}

\variantsplitter

\addpersonalvariant{Дарья Кошман}

\tasknumber{1}%
\task{%
    Запишите известные вам виды классификации изображений.
}
\solutionspace{60pt}

\tasknumber{2}%
\task{%
    В каких линзах можно получить увеличенное изображение объекта?
}
\answer{%
    $\text{ рассеивающие }$
}
\solutionspace{40pt}

\tasknumber{3}%
\task{%
    Какое изображение называют действительным?
}
\solutionspace{40pt}

\tasknumber{4}%
\task{%
    Есть две линзы, обозначим их 1 и 2.
    Известно что фокусное расстояние линзы 1 больше, чем у линзы 2.
    Какая линза сильнее преломляет лучи?
}
\answer{%
    $2$
}
\solutionspace{40pt}

\tasknumber{5}%
\task{%
    Предмет находится на расстоянии $10\,\text{см}$ от рассеивающей линзы с фокусным расстоянием $8\,\text{см}$.
    Определите тип изображения, расстояние между предметом и его изображением, увеличение предмета.
    Сделайте схематичный рисунок (не обязательно в масштабе, но с сохранением свойств линзы и изображения).
}
\solutionspace{100pt}

\tasknumber{6}%
\task{%
    Объект находится на расстоянии $25\,\text{см}$ от линзы, а его действительное изображение — в $10\,\text{см}$ от неё.
    Определите увеличение предмета, фокусное расстояние линзы, оптическую силу линзы и её тип.
}
\solutionspace{80pt}

\tasknumber{7}%
\task{%
    Известно, что из формулы тонкой линзы $\cbr{\frac 1F = \frac 1a + \frac 1b}$
    и определения увеличения $\cbr{\Gamma_y = \frac ba}$ можно получить выражение
    для увеличения: $\Gamma_y = \frac {aF}{a - F} \cdot \frac 1a = \frac {F}{a - F}.$
    Назовём такое увеличение «поперечным»: поперёк главной оптической оси (поэтому и ${}_y$).
    Получите формулу для «продольного» увеличения $\Gamma_x$ небольшого предмета, находящегося на главной оптической оси.
    Можно ли применить эту формулу для предмета, не лежащего на главной оптической оси, почему?
}
\answer{%
    \begin{align*}
    \frac 1F &= \frac 1a + \frac 1b \implies b = \frac {aF}{a - F} \\
    \frac 1F &= \frac 1{a + x} + \frac 1c \implies c = \frac {(a+x)F}{a + x - F} \\
    x' &= \abs{b - c} = \frac {aF}{a - F} - \frac {(a+x)F}{a + x - F} = F\cbr{\frac {a}{a - F} - \frac {a+x}{a + x - F}} =  \\
    &= F \cdot \frac {a^2 + ax - aF - a^2 - ax + aF + xF}{(a - F)(a + x - F)} = F \cdot \frac {xF}{(a - F)(a + x - F)} \\
    \Gamma_x &= \frac{x'}x = \frac{F^2}{(a - F)(a + x - F)} \to \frac{F^2}{\sqr{a - F}}.
    \\
    &\text{Нельзя: изображение по-разному растянет по осям $x$ и $y$ и понадобится теорема Пифагора}
    \end{align*}
}
\solutionspace{150pt}

\tasknumber{8}%
\task{%
    Доказать формулу тонкой линзы для собирающей линзы.
}
\solutionspace{120pt}

\tasknumber{9}%
\task{%
    Постройте ход луча $BK$ в тонкой линзе.
    Известно положение линзы и оба её фокуса (см.
    рис.
    на доске).
    Рассмотрите оба типа линзы, сделав 2 рисунка: собирающую и рассеивающую.
}
\solutionspace{120pt}

\tasknumber{10}%
\task{%
    На экране, расположенном иа расстоянии $120\,\text{см}$ от собирающей линзы,
    получено изображение точечного источника, расположенного на главной оптической оси линзы.
    На какое расстояние переместится изображение на экране,
    если при неподвижной линзе переместить источник на $3\,\text{см}$ в плоскости, перпендикулярной главной оптической оси?
    Фокусное расстояние линзы равно $40\,\text{см}$.
}
\answer{%
    \begin{align*}
    &\frac 1F = \frac 1a + \frac 1b \implies a = \frac{bF}{b-F} \implies \Gamma = \frac ba = \frac{b-F}F \\
    &y = x \cdot \Gamma = x \cdot \frac{b-F}F \implies d = y = 6\,\text{см}.
    \end{align*}
}
\solutionspace{120pt}

\tasknumber{11}%
\task{%
    Оптическая сила двояковыпуклой линзы в воздухе $5\,\text{дптр}$, а в воде $1{,}4\,\text{дптр}$.
    Определить показатель преломления $n$ материала, из которого изготовлена линза.
    Показатель преломления воды равен $1{,}33$.
}
\answer{%
    \begin{align*}
    D_1 &=\cbr{\frac n{n_1} - 1}\cbr{\frac 1{R_1} + \frac 1{R_2}}, \\
    D_2 &=\cbr{\frac n{n_2} - 1}\cbr{\frac 1{R_1} + \frac 1{R_2}}, \\
    \frac {D_2}{D_1} &=\frac{\frac n{n_2} - 1}{\frac n{n_1} - 1} \implies {D_2}\cbr{\frac n{n_1} - 1} = {D_1}\cbr{\frac n{n_2} - 1}  \implies n\cbr{\frac{D_2}{n_1} - \frac{D_1}{n_2}} = D_2 - D_1, \\
    n &= \frac{D_2 - D_1}{\frac{D_2}{n_1} - \frac{D_1}{n_2}} = \frac{n_1 n_2 (D_2 - D_1)}{D_2n_2 - D_1n_1} \approx 1{,}526.
    \end{align*}
}
\solutionspace{120pt}

\tasknumber{12}%
\task{%
    На каком расстоянии от собирающей линзы с фокусным расстоянием $40\,\text{дптр}$
    следует надо поместить предмет, чтобы расстояние
    от предмета до его действительного изображения было наименьшим?
}
\answer{%
    \begin{align*}
    \frac 1a &+ \frac 1b = D \implies b = \frac 1{D - \frac 1a} \implies \ell = a + b = a + \frac a{Da - 1} = \frac{ Da^2 }{Da - 1} \implies \\
    \implies \ell'_a &= \frac{ 2Da \cdot (Da - 1) - Da^2 \cdot D }{\sqr{Da - 1}}= \frac{ D^2a^2 - 2Da}{\sqr{Da - 1}} = \frac{ Da(Da - 2)}{\sqr{Da - 1}}\implies a_{\min} = \frac 2D \approx 50\,\text{мм}.
    \end{align*}
}
\solutionspace{120pt}

\tasknumber{13}%
\task{%
    Даны точечный источник света $S$, его изображение $S_1$, полученное с помощью собирающей линзы,
    и ближайший к источнику фокус линзы $F$ (см.
    рис.
    на доске).
    Расстояния $SF = \ell$ и $SS_1 = L$.
    Определить положение линзы и её фокусное расстояние.
}
\answer{%
    \begin{align*}
    \frac 1a + \frac 1b &= \frac 1F, \ell = a - F, L = a + b \implies a = \ell + F, b = L - a = L - \ell - F \\
    \frac 1{\ell + F} + \frac 1{L - \ell - F} &= \frac 1F \\
    F\ell + F^2 + LF - F\ell - F^2 &= L\ell - \ell^2 - F\ell + LF - F\ell - F^2 \\
    0 &= L\ell - \ell^2 - 2F\ell - F^2 \\
    0 &=  F^2 + 2F\ell - L\ell + \ell^2 \\
    F &= -\ell \pm \sqrt{\ell^2 +  L\ell - \ell^2} = -\ell \pm \sqrt{L\ell} \implies F = \sqrt{L\ell} - \ell \\
    a &= \ell + F = \ell + \sqrt{L\ell} - \ell = \sqrt{L\ell}.
    \end{align*}
}
\solutionspace{120pt}

\tasknumber{14}%
\task{%
    Расстояние от освещённого предмета до экрана $100\,\text{см}$.
    Линза, помещенная между ними, даёт чёткое изображение предмета на
    экране при двух положениях, расстояние между которыми $30\,\text{см}$.
    Найти фокусное расстояние линзы.
}
\answer{%
    \begin{align*}
    \frac 1a + \frac 1b &= \frac 1F, \frac 1{a-\ell} + \frac 1{b+\ell} = \frac 1F, a + b = L \\
    \frac 1a + \frac 1b &= \frac 1{a-\ell} + \frac 1{b+\ell}\implies \frac{a + b}{ab} = \frac{(a-\ell) + (b+\ell)}{(a-\ell)(b+\ell)} \\
    ab  &= (a - \ell)(b+\ell) \implies 0  = -b\ell + a\ell - \ell^2 \implies 0 = -b + a - \ell \implies b = a - \ell \\
    a + (a - \ell) &= L \implies a = \frac{L + \ell}2 \implies b = \frac{L - \ell}2 \\
    F &= \frac{ab}{a + b} = \frac{L^2 -\ell^2}{4L} \approx 22{,}8\,\text{см}.
    \end{align*}
}
\solutionspace{120pt}

\tasknumber{15}%
\task{%
    Предмет находится на расстоянии $80\,\text{см}$ от экрана.
    Между предметом и экраном помещают линзу, причём при одном
    положении линзы на экране получается увеличенное изображение предмета,
    а при другом — уменьшенное.
    Каково фокусное расстояние линзы, если
    линейные размеры первого изображения в пять раз больше второго?
}
\answer{%
    \begin{align*}
    \frac 1a + \frac 1{L-a} &= \frac 1F, h_1 = h \cdot \frac{L-a}a, \\
    \frac 1b + \frac 1{L-b} &= \frac 1F, h_2 = h \cdot \frac{L-b}b, \\
    \frac{h_1}{h_2} &= 5 \implies \frac{(L-a)b}{(L-b)a} = 5, \\
    \frac 1F &= \frac{ L }{a(L-a)} = \frac{ L }{b(L-b)} \implies \frac{L-a}{L-b} = \frac b a \implies \frac {b^2}{a^2} = 5.
    \\
    \frac 1a + \frac 1{L-a} &= \frac 1b + \frac 1{L-b} \implies \frac L{a(L-a)} = \frac L{b(L-b)} \implies \\
    \implies aL - a^2 &= bL - b^2 \implies (a-b)L = (a-b)(a+b) \implies b = L - a, \\
    \frac{\sqr{L-a}}{a^2} &= 5 \implies \frac La - 1 = \sqrt{5} \implies a = \frac{ L }{\sqrt{5} + 1} \\
    F &= \frac{a(L-a)}L = \frac 1L \cdot \frac L{\sqrt{5} + 1} \cdot \frac {L\sqrt{5}}{\sqrt{5} + 1}= \frac { L\sqrt{5} }{ \sqr{\sqrt{5} + 1} } \approx 17{,}1\,\text{см}.
    \end{align*}
}
\solutionspace{120pt}

\tasknumber{16}%
\task{%
    (Задача-«гроб»: решать на обратной стороне) Квадрат со стороной $d = 1\,\text{см}$ расположен так,
    что 2 его стороны параллельны главной оптической оси рассеивающей линзы,
    его центр удален на $h = 5\,\text{см}$ от этой оси и на $a = 10\,\text{см}$ от плоскости линзы.
    Определите площадь изображения квадрата, если фокусное расстояние линзы составляет $F = 20\,\text{см}$.
    % (и сравните с площадью объекта, умноженной на квадрат увеличения центра квадрата).
}
\answer{%
    \begin{align*}
    &\text{Все явные вычисления — в см и $\text{см}^2$,} \\
    \frac 1 F &= \frac 1{a + \frac d2} + \frac 1b \implies b = \frac 1{\frac 1 F - \frac 1{a + \frac d2}} = \frac{F(a + \frac d2)}{a + \frac d2 - F} = -\frac{420}{61}, \\
    \frac 1 F &= \frac 1{a - \frac d2} + \frac 1c \implies c = \frac 1{\frac 1 F - \frac 1{a - \frac d2}} = \frac{F(a - \frac d2)}{a - \frac d2 - F} = -\frac{380}{59}, \\
    c - b &= \frac{F(a - \frac d2)}{a - \frac d2 - F} - \frac{F(a + \frac d2)}{a + \frac d2 - F} = F\cbr{ \frac{a - \frac d2}{a - \frac d2 - F} - \frac{a + \frac d2}{a + \frac d2 - F} } =  \\
    &= F \cdot \frac{a^2 + \frac {ad}2 - aF - \frac{ad}2 - \frac{d^2}4 + \frac{dF}2 - a^2 + \frac {ad}2 + aF - \frac{ad}2 + \frac{d^2}4 + \frac{dF}2}{\cbr{a + \frac d2 - F}\cbr{a - \frac d2 - F}}= F \cdot \frac {dF}{\cbr{a + \frac d2 - F}\cbr{a - \frac d2 - F}} = \frac{1600}{3599}.
    \\
    \Gamma_b &= \frac b{a + \frac d2} = \frac{ F }{a + \frac d2 - F} = -\frac{40}{61}, \\
    \Gamma_c &= \frac c{a - \frac d2} = \frac{ F }{a - \frac d2 - F} = -\frac{40}{59}, \\
    &\text{ тут интересно отметить, что } \Gamma_x = \frac{ c - b}{ d } = \frac{ F^2 }{\cbr{a + \frac d2 - F}\cbr{a - \frac d2 - F}} \ne \Gamma_b \text{ или } \Gamma_c \text{ даже при малых $d$}.
    \\
    S' &= \frac{d \cdot \Gamma_b + d \cdot \Gamma_c}2 \cdot (c - b) = \frac d2 \cbr{\frac{ F }{a + \frac d2 - F} + \frac{ F }{a - \frac d2 - F}} \cdot \cbr{c - b} =  \\
    &=\frac {dF}2 \cbr{\frac 1{a + \frac d2 - F} + \frac 1{a - \frac d2 - F}} \cdot \frac {dF^2}{\cbr{a + \frac d2 - F}\cbr{a - \frac d2 - F}} =  \\
    &=\frac {dF}2 \cdot \frac{a - \frac d2 - F + a + \frac d2 - F}{\cbr{a + \frac d2 - F}\cbr{a - \frac d2 - F}} \cdot \frac {dF^2}{\cbr{a + \frac d2 - F}\cbr{a - \frac d2 - F}} =  \\
    &= \frac {d^2F^3}{2\sqr{a + \frac d2 - F}\sqr{a - \frac d2 - F}} \cdot (2a - 2F) = \frac {d^2F^3(a - F)}{ \sqr{\sqr{a - F} - \frac{d^2}4} } = -\frac{3840000}{12952801}.
    \end{align*}
}

\variantsplitter

\addpersonalvariant{Анна Кузьмичёва}

\tasknumber{1}%
\task{%
    Запишите формулу тонкой линзы и сделайте рисунок, указав на нём физические величины из этой формулы.
}
\solutionspace{60pt}

\tasknumber{2}%
\task{%
    В каких линзах можно получить действительное изображение объекта?
}
\answer{%
    $\text{ собирающие }$
}
\solutionspace{40pt}

\tasknumber{3}%
\task{%
    Какое изображение называют действительным?
}
\solutionspace{40pt}

\tasknumber{4}%
\task{%
    Есть две линзы, обозначим их 1 и 2.
    Известно что оптическая сила линзы 1 больше, чем у линзы 2.
    Какая линза сильнее преломляет лучи?
}
\answer{%
    $1$
}
\solutionspace{40pt}

\tasknumber{5}%
\task{%
    Предмет находится на расстоянии $10\,\text{см}$ от рассеивающей линзы с фокусным расстоянием $12\,\text{см}$.
    Определите тип изображения, расстояние между предметом и его изображением, увеличение предмета.
    Сделайте схематичный рисунок (не обязательно в масштабе, но с сохранением свойств линзы и изображения).
}
\solutionspace{100pt}

\tasknumber{6}%
\task{%
    Объект находится на расстоянии $115\,\text{см}$ от линзы, а его мнимое изображение — в $30\,\text{см}$ от неё.
    Определите увеличение предмета, фокусное расстояние линзы, оптическую силу линзы и её тип.
}
\solutionspace{80pt}

\tasknumber{7}%
\task{%
    Известно, что из формулы тонкой линзы $\cbr{\frac 1F = \frac 1a + \frac 1b}$
    и определения увеличения $\cbr{\Gamma_y = \frac ba}$ можно получить выражение
    для увеличения: $\Gamma_y = \frac {aF}{a - F} \cdot \frac 1a = \frac {F}{a - F}.$
    Назовём такое увеличение «поперечным»: поперёк главной оптической оси (поэтому и ${}_y$).
    Получите формулу для «продольного» увеличения $\Gamma_x$ небольшого предмета, находящегося на главной оптической оси.
    Можно ли применить эту формулу для предмета, не лежащего на главной оптической оси, почему?
}
\answer{%
    \begin{align*}
    \frac 1F &= \frac 1a + \frac 1b \implies b = \frac {aF}{a - F} \\
    \frac 1F &= \frac 1{a + x} + \frac 1c \implies c = \frac {(a+x)F}{a + x - F} \\
    x' &= \abs{b - c} = \frac {aF}{a - F} - \frac {(a+x)F}{a + x - F} = F\cbr{\frac {a}{a - F} - \frac {a+x}{a + x - F}} =  \\
    &= F \cdot \frac {a^2 + ax - aF - a^2 - ax + aF + xF}{(a - F)(a + x - F)} = F \cdot \frac {xF}{(a - F)(a + x - F)} \\
    \Gamma_x &= \frac{x'}x = \frac{F^2}{(a - F)(a + x - F)} \to \frac{F^2}{\sqr{a - F}}.
    \\
    &\text{Нельзя: изображение по-разному растянет по осям $x$ и $y$ и понадобится теорема Пифагора}
    \end{align*}
}
\solutionspace{150pt}

\tasknumber{8}%
\task{%
    Доказать формулу тонкой линзы для собирающей линзы.
}
\solutionspace{120pt}

\tasknumber{9}%
\task{%
    Постройте ход луча $AL$ в тонкой линзе.
    Известно положение линзы и оба её фокуса (см.
    рис.
    на доске).
    Рассмотрите оба типа линзы, сделав 2 рисунка: собирающую и рассеивающую.
}
\solutionspace{120pt}

\tasknumber{10}%
\task{%
    На экране, расположенном иа расстоянии $120\,\text{см}$ от собирающей линзы,
    получено изображение точечного источника, расположенного на главной оптической оси линзы.
    На какое расстояние переместится изображение на экране,
    если при неподвижной линзе переместить источник на $1\,\text{см}$ в плоскости, перпендикулярной главной оптической оси?
    Фокусное расстояние линзы равно $20\,\text{см}$.
}
\answer{%
    \begin{align*}
    &\frac 1F = \frac 1a + \frac 1b \implies a = \frac{bF}{b-F} \implies \Gamma = \frac ba = \frac{b-F}F \\
    &y = x \cdot \Gamma = x \cdot \frac{b-F}F \implies d = y = 5\,\text{см}.
    \end{align*}
}
\solutionspace{120pt}

\tasknumber{11}%
\task{%
    Оптическая сила двояковыпуклой линзы в воздухе $5\,\text{дптр}$, а в воде $1{,}4\,\text{дптр}$.
    Определить показатель преломления $n$ материала, из которого изготовлена линза.
    Показатель преломления воды равен $1{,}33$.
}
\answer{%
    \begin{align*}
    D_1 &=\cbr{\frac n{n_1} - 1}\cbr{\frac 1{R_1} + \frac 1{R_2}}, \\
    D_2 &=\cbr{\frac n{n_2} - 1}\cbr{\frac 1{R_1} + \frac 1{R_2}}, \\
    \frac {D_2}{D_1} &=\frac{\frac n{n_2} - 1}{\frac n{n_1} - 1} \implies {D_2}\cbr{\frac n{n_1} - 1} = {D_1}\cbr{\frac n{n_2} - 1}  \implies n\cbr{\frac{D_2}{n_1} - \frac{D_1}{n_2}} = D_2 - D_1, \\
    n &= \frac{D_2 - D_1}{\frac{D_2}{n_1} - \frac{D_1}{n_2}} = \frac{n_1 n_2 (D_2 - D_1)}{D_2n_2 - D_1n_1} \approx 1{,}526.
    \end{align*}
}
\solutionspace{120pt}

\tasknumber{12}%
\task{%
    На каком расстоянии от собирающей линзы с фокусным расстоянием $30\,\text{дптр}$
    следует надо поместить предмет, чтобы расстояние
    от предмета до его действительного изображения было наименьшим?
}
\answer{%
    \begin{align*}
    \frac 1a &+ \frac 1b = D \implies b = \frac 1{D - \frac 1a} \implies \ell = a + b = a + \frac a{Da - 1} = \frac{ Da^2 }{Da - 1} \implies \\
    \implies \ell'_a &= \frac{ 2Da \cdot (Da - 1) - Da^2 \cdot D }{\sqr{Da - 1}}= \frac{ D^2a^2 - 2Da}{\sqr{Da - 1}} = \frac{ Da(Da - 2)}{\sqr{Da - 1}}\implies a_{\min} = \frac 2D \approx 66{,}7\,\text{мм}.
    \end{align*}
}
\solutionspace{120pt}

\tasknumber{13}%
\task{%
    Даны точечный источник света $S$, его изображение $S_1$, полученное с помощью собирающей линзы,
    и ближайший к источнику фокус линзы $F$ (см.
    рис.
    на доске).
    Расстояния $SF = \ell$ и $SS_1 = L$.
    Определить положение линзы и её фокусное расстояние.
}
\answer{%
    \begin{align*}
    \frac 1a + \frac 1b &= \frac 1F, \ell = a - F, L = a + b \implies a = \ell + F, b = L - a = L - \ell - F \\
    \frac 1{\ell + F} + \frac 1{L - \ell - F} &= \frac 1F \\
    F\ell + F^2 + LF - F\ell - F^2 &= L\ell - \ell^2 - F\ell + LF - F\ell - F^2 \\
    0 &= L\ell - \ell^2 - 2F\ell - F^2 \\
    0 &=  F^2 + 2F\ell - L\ell + \ell^2 \\
    F &= -\ell \pm \sqrt{\ell^2 +  L\ell - \ell^2} = -\ell \pm \sqrt{L\ell} \implies F = \sqrt{L\ell} - \ell \\
    a &= \ell + F = \ell + \sqrt{L\ell} - \ell = \sqrt{L\ell}.
    \end{align*}
}
\solutionspace{120pt}

\tasknumber{14}%
\task{%
    Расстояние от освещённого предмета до экрана $80\,\text{см}$.
    Линза, помещенная между ними, даёт чёткое изображение предмета на
    экране при двух положениях, расстояние между которыми $40\,\text{см}$.
    Найти фокусное расстояние линзы.
}
\answer{%
    \begin{align*}
    \frac 1a + \frac 1b &= \frac 1F, \frac 1{a-\ell} + \frac 1{b+\ell} = \frac 1F, a + b = L \\
    \frac 1a + \frac 1b &= \frac 1{a-\ell} + \frac 1{b+\ell}\implies \frac{a + b}{ab} = \frac{(a-\ell) + (b+\ell)}{(a-\ell)(b+\ell)} \\
    ab  &= (a - \ell)(b+\ell) \implies 0  = -b\ell + a\ell - \ell^2 \implies 0 = -b + a - \ell \implies b = a - \ell \\
    a + (a - \ell) &= L \implies a = \frac{L + \ell}2 \implies b = \frac{L - \ell}2 \\
    F &= \frac{ab}{a + b} = \frac{L^2 -\ell^2}{4L} \approx 15\,\text{см}.
    \end{align*}
}
\solutionspace{120pt}

\tasknumber{15}%
\task{%
    Предмет находится на расстоянии $80\,\text{см}$ от экрана.
    Между предметом и экраном помещают линзу, причём при одном
    положении линзы на экране получается увеличенное изображение предмета,
    а при другом — уменьшенное.
    Каково фокусное расстояние линзы, если
    линейные размеры первого изображения в пять раз больше второго?
}
\answer{%
    \begin{align*}
    \frac 1a + \frac 1{L-a} &= \frac 1F, h_1 = h \cdot \frac{L-a}a, \\
    \frac 1b + \frac 1{L-b} &= \frac 1F, h_2 = h \cdot \frac{L-b}b, \\
    \frac{h_1}{h_2} &= 5 \implies \frac{(L-a)b}{(L-b)a} = 5, \\
    \frac 1F &= \frac{ L }{a(L-a)} = \frac{ L }{b(L-b)} \implies \frac{L-a}{L-b} = \frac b a \implies \frac {b^2}{a^2} = 5.
    \\
    \frac 1a + \frac 1{L-a} &= \frac 1b + \frac 1{L-b} \implies \frac L{a(L-a)} = \frac L{b(L-b)} \implies \\
    \implies aL - a^2 &= bL - b^2 \implies (a-b)L = (a-b)(a+b) \implies b = L - a, \\
    \frac{\sqr{L-a}}{a^2} &= 5 \implies \frac La - 1 = \sqrt{5} \implies a = \frac{ L }{\sqrt{5} + 1} \\
    F &= \frac{a(L-a)}L = \frac 1L \cdot \frac L{\sqrt{5} + 1} \cdot \frac {L\sqrt{5}}{\sqrt{5} + 1}= \frac { L\sqrt{5} }{ \sqr{\sqrt{5} + 1} } \approx 17{,}1\,\text{см}.
    \end{align*}
}
\solutionspace{120pt}

\tasknumber{16}%
\task{%
    (Задача-«гроб»: решать на обратной стороне) Квадрат со стороной $d = 2\,\text{см}$ расположен так,
    что 2 его стороны параллельны главной оптической оси собирающей линзы,
    его центр удален на $h = 6\,\text{см}$ от этой оси и на $a = 15\,\text{см}$ от плоскости линзы.
    Определите площадь изображения квадрата, если фокусное расстояние линзы составляет $F = 20\,\text{см}$.
    % (и сравните с площадью объекта, умноженной на квадрат увеличения центра квадрата).
}
\answer{%
    \begin{align*}
    &\text{Все явные вычисления — в см и $\text{см}^2$,} \\
    \frac 1 F &= \frac 1{a + \frac d2} + \frac 1b \implies b = \frac 1{\frac 1 F - \frac 1{a + \frac d2}} = \frac{F(a + \frac d2)}{a + \frac d2 - F} = -80, \\
    \frac 1 F &= \frac 1{a - \frac d2} + \frac 1c \implies c = \frac 1{\frac 1 F - \frac 1{a - \frac d2}} = \frac{F(a - \frac d2)}{a - \frac d2 - F} = -\frac{140}3, \\
    c - b &= \frac{F(a - \frac d2)}{a - \frac d2 - F} - \frac{F(a + \frac d2)}{a + \frac d2 - F} = F\cbr{ \frac{a - \frac d2}{a - \frac d2 - F} - \frac{a + \frac d2}{a + \frac d2 - F} } =  \\
    &= F \cdot \frac{a^2 + \frac {ad}2 - aF - \frac{ad}2 - \frac{d^2}4 + \frac{dF}2 - a^2 + \frac {ad}2 + aF - \frac{ad}2 + \frac{d^2}4 + \frac{dF}2}{\cbr{a + \frac d2 - F}\cbr{a - \frac d2 - F}}= F \cdot \frac {dF}{\cbr{a + \frac d2 - F}\cbr{a - \frac d2 - F}} = \frac{100}3.
    \\
    \Gamma_b &= \frac b{a + \frac d2} = \frac{ F }{a + \frac d2 - F} = -5, \\
    \Gamma_c &= \frac c{a - \frac d2} = \frac{ F }{a - \frac d2 - F} = -\frac{10}3, \\
    &\text{ тут интересно отметить, что } \Gamma_x = \frac{ c - b}{ d } = \frac{ F^2 }{\cbr{a + \frac d2 - F}\cbr{a - \frac d2 - F}} \ne \Gamma_b \text{ или } \Gamma_c \text{ даже при малых $d$}.
    \\
    S' &= \frac{d \cdot \Gamma_b + d \cdot \Gamma_c}2 \cdot (c - b) = \frac d2 \cbr{\frac{ F }{a + \frac d2 - F} + \frac{ F }{a - \frac d2 - F}} \cdot \cbr{c - b} =  \\
    &=\frac {dF}2 \cbr{\frac 1{a + \frac d2 - F} + \frac 1{a - \frac d2 - F}} \cdot \frac {dF^2}{\cbr{a + \frac d2 - F}\cbr{a - \frac d2 - F}} =  \\
    &=\frac {dF}2 \cdot \frac{a - \frac d2 - F + a + \frac d2 - F}{\cbr{a + \frac d2 - F}\cbr{a - \frac d2 - F}} \cdot \frac {dF^2}{\cbr{a + \frac d2 - F}\cbr{a - \frac d2 - F}} =  \\
    &= \frac {d^2F^3}{2\sqr{a + \frac d2 - F}\sqr{a - \frac d2 - F}} \cdot (2a - 2F) = \frac {d^2F^3(a - F)}{ \sqr{\sqr{a - F} - \frac{d^2}4} } = -\frac{2500}9.
    \end{align*}
}

\variantsplitter

\addpersonalvariant{Алёна Куприянова}

\tasknumber{1}%
\task{%
    Запишите известные вам виды классификации изображений.
}
\solutionspace{60pt}

\tasknumber{2}%
\task{%
    В каких линзах можно получить увеличенное изображение объекта?
}
\answer{%
    $\text{ рассеивающие }$
}
\solutionspace{40pt}

\tasknumber{3}%
\task{%
    Какое изображение называют действительным?
}
\solutionspace{40pt}

\tasknumber{4}%
\task{%
    Есть две линзы, обозначим их 1 и 2.
    Известно что оптическая сила линзы 2 меньше, чем у линзы 1.
    Какая линза сильнее преломляет лучи?
}
\answer{%
    $1$
}
\solutionspace{40pt}

\tasknumber{5}%
\task{%
    Предмет находится на расстоянии $20\,\text{см}$ от собирающей линзы с фокусным расстоянием $8\,\text{см}$.
    Определите тип изображения, расстояние между предметом и его изображением, увеличение предмета.
    Сделайте схематичный рисунок (не обязательно в масштабе, но с сохранением свойств линзы и изображения).
}
\solutionspace{100pt}

\tasknumber{6}%
\task{%
    Объект находится на расстоянии $45\,\text{см}$ от линзы, а его мнимое изображение — в $10\,\text{см}$ от неё.
    Определите увеличение предмета, фокусное расстояние линзы, оптическую силу линзы и её тип.
}
\solutionspace{80pt}

\tasknumber{7}%
\task{%
    Известно, что из формулы тонкой линзы $\cbr{\frac 1F = \frac 1a + \frac 1b}$
    и определения увеличения $\cbr{\Gamma_y = \frac ba}$ можно получить выражение
    для увеличения: $\Gamma_y = \frac {aF}{a - F} \cdot \frac 1a = \frac {F}{a - F}.$
    Назовём такое увеличение «поперечным»: поперёк главной оптической оси (поэтому и ${}_y$).
    Получите формулу для «продольного» увеличения $\Gamma_x$ небольшого предмета, находящегося на главной оптической оси.
    Можно ли применить эту формулу для предмета, не лежащего на главной оптической оси, почему?
}
\answer{%
    \begin{align*}
    \frac 1F &= \frac 1a + \frac 1b \implies b = \frac {aF}{a - F} \\
    \frac 1F &= \frac 1{a + x} + \frac 1c \implies c = \frac {(a+x)F}{a + x - F} \\
    x' &= \abs{b - c} = \frac {aF}{a - F} - \frac {(a+x)F}{a + x - F} = F\cbr{\frac {a}{a - F} - \frac {a+x}{a + x - F}} =  \\
    &= F \cdot \frac {a^2 + ax - aF - a^2 - ax + aF + xF}{(a - F)(a + x - F)} = F \cdot \frac {xF}{(a - F)(a + x - F)} \\
    \Gamma_x &= \frac{x'}x = \frac{F^2}{(a - F)(a + x - F)} \to \frac{F^2}{\sqr{a - F}}.
    \\
    &\text{Нельзя: изображение по-разному растянет по осям $x$ и $y$ и понадобится теорема Пифагора}
    \end{align*}
}
\solutionspace{150pt}

\tasknumber{8}%
\task{%
    Доказать формулу тонкой линзы для собирающей линзы.
}
\solutionspace{120pt}

\tasknumber{9}%
\task{%
    Постройте ход луча $BK$ в тонкой линзе.
    Известно положение линзы и оба её фокуса (см.
    рис.
    на доске).
    Рассмотрите оба типа линзы, сделав 2 рисунка: собирающую и рассеивающую.
}
\solutionspace{120pt}

\tasknumber{10}%
\task{%
    На экране, расположенном иа расстоянии $60\,\text{см}$ от собирающей линзы,
    получено изображение точечного источника, расположенного на главной оптической оси линзы.
    На какое расстояние переместится изображение на экране,
    если при неподвижной линзе переместить источник на $3\,\text{см}$ в плоскости, перпендикулярной главной оптической оси?
    Фокусное расстояние линзы равно $20\,\text{см}$.
}
\answer{%
    \begin{align*}
    &\frac 1F = \frac 1a + \frac 1b \implies a = \frac{bF}{b-F} \implies \Gamma = \frac ba = \frac{b-F}F \\
    &y = x \cdot \Gamma = x \cdot \frac{b-F}F \implies d = y = 6\,\text{см}.
    \end{align*}
}
\solutionspace{120pt}

\tasknumber{11}%
\task{%
    Оптическая сила двояковыпуклой линзы в воздухе $4{,}5\,\text{дптр}$, а в воде $1{,}6\,\text{дптр}$.
    Определить показатель преломления $n$ материала, из которого изготовлена линза.
    Показатель преломления воды равен $1{,}33$.
}
\answer{%
    \begin{align*}
    D_1 &=\cbr{\frac n{n_1} - 1}\cbr{\frac 1{R_1} + \frac 1{R_2}}, \\
    D_2 &=\cbr{\frac n{n_2} - 1}\cbr{\frac 1{R_1} + \frac 1{R_2}}, \\
    \frac {D_2}{D_1} &=\frac{\frac n{n_2} - 1}{\frac n{n_1} - 1} \implies {D_2}\cbr{\frac n{n_1} - 1} = {D_1}\cbr{\frac n{n_2} - 1}  \implies n\cbr{\frac{D_2}{n_1} - \frac{D_1}{n_2}} = D_2 - D_1, \\
    n &= \frac{D_2 - D_1}{\frac{D_2}{n_1} - \frac{D_1}{n_2}} = \frac{n_1 n_2 (D_2 - D_1)}{D_2n_2 - D_1n_1} \approx 1{,}626.
    \end{align*}
}
\solutionspace{120pt}

\tasknumber{12}%
\task{%
    На каком расстоянии от собирающей линзы с фокусным расстоянием $30\,\text{дптр}$
    следует надо поместить предмет, чтобы расстояние
    от предмета до его действительного изображения было наименьшим?
}
\answer{%
    \begin{align*}
    \frac 1a &+ \frac 1b = D \implies b = \frac 1{D - \frac 1a} \implies \ell = a + b = a + \frac a{Da - 1} = \frac{ Da^2 }{Da - 1} \implies \\
    \implies \ell'_a &= \frac{ 2Da \cdot (Da - 1) - Da^2 \cdot D }{\sqr{Da - 1}}= \frac{ D^2a^2 - 2Da}{\sqr{Da - 1}} = \frac{ Da(Da - 2)}{\sqr{Da - 1}}\implies a_{\min} = \frac 2D \approx 66{,}7\,\text{мм}.
    \end{align*}
}
\solutionspace{120pt}

\tasknumber{13}%
\task{%
    Даны точечный источник света $S$, его изображение $S_1$, полученное с помощью собирающей линзы,
    и ближайший к источнику фокус линзы $F$ (см.
    рис.
    на доске).
    Расстояния $SF = \ell$ и $SS_1 = L$.
    Определить положение линзы и её фокусное расстояние.
}
\answer{%
    \begin{align*}
    \frac 1a + \frac 1b &= \frac 1F, \ell = a - F, L = a + b \implies a = \ell + F, b = L - a = L - \ell - F \\
    \frac 1{\ell + F} + \frac 1{L - \ell - F} &= \frac 1F \\
    F\ell + F^2 + LF - F\ell - F^2 &= L\ell - \ell^2 - F\ell + LF - F\ell - F^2 \\
    0 &= L\ell - \ell^2 - 2F\ell - F^2 \\
    0 &=  F^2 + 2F\ell - L\ell + \ell^2 \\
    F &= -\ell \pm \sqrt{\ell^2 +  L\ell - \ell^2} = -\ell \pm \sqrt{L\ell} \implies F = \sqrt{L\ell} - \ell \\
    a &= \ell + F = \ell + \sqrt{L\ell} - \ell = \sqrt{L\ell}.
    \end{align*}
}
\solutionspace{120pt}

\tasknumber{14}%
\task{%
    Расстояние от освещённого предмета до экрана $80\,\text{см}$.
    Линза, помещенная между ними, даёт чёткое изображение предмета на
    экране при двух положениях, расстояние между которыми $40\,\text{см}$.
    Найти фокусное расстояние линзы.
}
\answer{%
    \begin{align*}
    \frac 1a + \frac 1b &= \frac 1F, \frac 1{a-\ell} + \frac 1{b+\ell} = \frac 1F, a + b = L \\
    \frac 1a + \frac 1b &= \frac 1{a-\ell} + \frac 1{b+\ell}\implies \frac{a + b}{ab} = \frac{(a-\ell) + (b+\ell)}{(a-\ell)(b+\ell)} \\
    ab  &= (a - \ell)(b+\ell) \implies 0  = -b\ell + a\ell - \ell^2 \implies 0 = -b + a - \ell \implies b = a - \ell \\
    a + (a - \ell) &= L \implies a = \frac{L + \ell}2 \implies b = \frac{L - \ell}2 \\
    F &= \frac{ab}{a + b} = \frac{L^2 -\ell^2}{4L} \approx 15\,\text{см}.
    \end{align*}
}
\solutionspace{120pt}

\tasknumber{15}%
\task{%
    Предмет находится на расстоянии $80\,\text{см}$ от экрана.
    Между предметом и экраном помещают линзу, причём при одном
    положении линзы на экране получается увеличенное изображение предмета,
    а при другом — уменьшенное.
    Каково фокусное расстояние линзы, если
    линейные размеры первого изображения в пять раз больше второго?
}
\answer{%
    \begin{align*}
    \frac 1a + \frac 1{L-a} &= \frac 1F, h_1 = h \cdot \frac{L-a}a, \\
    \frac 1b + \frac 1{L-b} &= \frac 1F, h_2 = h \cdot \frac{L-b}b, \\
    \frac{h_1}{h_2} &= 5 \implies \frac{(L-a)b}{(L-b)a} = 5, \\
    \frac 1F &= \frac{ L }{a(L-a)} = \frac{ L }{b(L-b)} \implies \frac{L-a}{L-b} = \frac b a \implies \frac {b^2}{a^2} = 5.
    \\
    \frac 1a + \frac 1{L-a} &= \frac 1b + \frac 1{L-b} \implies \frac L{a(L-a)} = \frac L{b(L-b)} \implies \\
    \implies aL - a^2 &= bL - b^2 \implies (a-b)L = (a-b)(a+b) \implies b = L - a, \\
    \frac{\sqr{L-a}}{a^2} &= 5 \implies \frac La - 1 = \sqrt{5} \implies a = \frac{ L }{\sqrt{5} + 1} \\
    F &= \frac{a(L-a)}L = \frac 1L \cdot \frac L{\sqrt{5} + 1} \cdot \frac {L\sqrt{5}}{\sqrt{5} + 1}= \frac { L\sqrt{5} }{ \sqr{\sqrt{5} + 1} } \approx 17{,}1\,\text{см}.
    \end{align*}
}
\solutionspace{120pt}

\tasknumber{16}%
\task{%
    (Задача-«гроб»: решать на обратной стороне) Квадрат со стороной $d = 2\,\text{см}$ расположен так,
    что 2 его стороны параллельны главной оптической оси собирающей линзы,
    его центр удален на $h = 6\,\text{см}$ от этой оси и на $a = 12\,\text{см}$ от плоскости линзы.
    Определите площадь изображения квадрата, если фокусное расстояние линзы составляет $F = 25\,\text{см}$.
    % (и сравните с площадью объекта, умноженной на квадрат увеличения центра квадрата).
}
\answer{%
    \begin{align*}
    &\text{Все явные вычисления — в см и $\text{см}^2$,} \\
    \frac 1 F &= \frac 1{a + \frac d2} + \frac 1b \implies b = \frac 1{\frac 1 F - \frac 1{a + \frac d2}} = \frac{F(a + \frac d2)}{a + \frac d2 - F} = -\frac{325}{12}, \\
    \frac 1 F &= \frac 1{a - \frac d2} + \frac 1c \implies c = \frac 1{\frac 1 F - \frac 1{a - \frac d2}} = \frac{F(a - \frac d2)}{a - \frac d2 - F} = -\frac{275}{14}, \\
    c - b &= \frac{F(a - \frac d2)}{a - \frac d2 - F} - \frac{F(a + \frac d2)}{a + \frac d2 - F} = F\cbr{ \frac{a - \frac d2}{a - \frac d2 - F} - \frac{a + \frac d2}{a + \frac d2 - F} } =  \\
    &= F \cdot \frac{a^2 + \frac {ad}2 - aF - \frac{ad}2 - \frac{d^2}4 + \frac{dF}2 - a^2 + \frac {ad}2 + aF - \frac{ad}2 + \frac{d^2}4 + \frac{dF}2}{\cbr{a + \frac d2 - F}\cbr{a - \frac d2 - F}}= F \cdot \frac {dF}{\cbr{a + \frac d2 - F}\cbr{a - \frac d2 - F}} = \frac{625}{84}.
    \\
    \Gamma_b &= \frac b{a + \frac d2} = \frac{ F }{a + \frac d2 - F} = -\frac{25}{12}, \\
    \Gamma_c &= \frac c{a - \frac d2} = \frac{ F }{a - \frac d2 - F} = -\frac{25}{14}, \\
    &\text{ тут интересно отметить, что } \Gamma_x = \frac{ c - b}{ d } = \frac{ F^2 }{\cbr{a + \frac d2 - F}\cbr{a - \frac d2 - F}} \ne \Gamma_b \text{ или } \Gamma_c \text{ даже при малых $d$}.
    \\
    S' &= \frac{d \cdot \Gamma_b + d \cdot \Gamma_c}2 \cdot (c - b) = \frac d2 \cbr{\frac{ F }{a + \frac d2 - F} + \frac{ F }{a - \frac d2 - F}} \cdot \cbr{c - b} =  \\
    &=\frac {dF}2 \cbr{\frac 1{a + \frac d2 - F} + \frac 1{a - \frac d2 - F}} \cdot \frac {dF^2}{\cbr{a + \frac d2 - F}\cbr{a - \frac d2 - F}} =  \\
    &=\frac {dF}2 \cdot \frac{a - \frac d2 - F + a + \frac d2 - F}{\cbr{a + \frac d2 - F}\cbr{a - \frac d2 - F}} \cdot \frac {dF^2}{\cbr{a + \frac d2 - F}\cbr{a - \frac d2 - F}} =  \\
    &= \frac {d^2F^3}{2\sqr{a + \frac d2 - F}\sqr{a - \frac d2 - F}} \cdot (2a - 2F) = \frac {d^2F^3(a - F)}{ \sqr{\sqr{a - F} - \frac{d^2}4} } = -\frac{203125}{7056}.
    \end{align*}
}

\variantsplitter

\addpersonalvariant{Ярослав Лавровский}

\tasknumber{1}%
\task{%
    Запишите формулу тонкой линзы и сделайте рисунок, указав на нём физические величины из этой формулы.
}
\solutionspace{60pt}

\tasknumber{2}%
\task{%
    В каких линзах можно получить обратное изображение объекта?
}
\answer{%
    $\text{ собирающие }$
}
\solutionspace{40pt}

\tasknumber{3}%
\task{%
    Какое изображение называют действительным?
}
\solutionspace{40pt}

\tasknumber{4}%
\task{%
    Есть две линзы, обозначим их 1 и 2.
    Известно что оптическая сила линзы 1 больше, чем у линзы 2.
    Какая линза сильнее преломляет лучи?
}
\answer{%
    $1$
}
\solutionspace{40pt}

\tasknumber{5}%
\task{%
    Предмет находится на расстоянии $20\,\text{см}$ от собирающей линзы с фокусным расстоянием $25\,\text{см}$.
    Определите тип изображения, расстояние между предметом и его изображением, увеличение предмета.
    Сделайте схематичный рисунок (не обязательно в масштабе, но с сохранением свойств линзы и изображения).
}
\solutionspace{100pt}

\tasknumber{6}%
\task{%
    Объект находится на расстоянии $45\,\text{см}$ от линзы, а его действительное изображение — в $20\,\text{см}$ от неё.
    Определите увеличение предмета, фокусное расстояние линзы, оптическую силу линзы и её тип.
}
\solutionspace{80pt}

\tasknumber{7}%
\task{%
    Известно, что из формулы тонкой линзы $\cbr{\frac 1F = \frac 1a + \frac 1b}$
    и определения увеличения $\cbr{\Gamma_y = \frac ba}$ можно получить выражение
    для увеличения: $\Gamma_y = \frac {aF}{a - F} \cdot \frac 1a = \frac {F}{a - F}.$
    Назовём такое увеличение «поперечным»: поперёк главной оптической оси (поэтому и ${}_y$).
    Получите формулу для «продольного» увеличения $\Gamma_x$ небольшого предмета, находящегося на главной оптической оси.
    Можно ли применить эту формулу для предмета, не лежащего на главной оптической оси, почему?
}
\answer{%
    \begin{align*}
    \frac 1F &= \frac 1a + \frac 1b \implies b = \frac {aF}{a - F} \\
    \frac 1F &= \frac 1{a + x} + \frac 1c \implies c = \frac {(a+x)F}{a + x - F} \\
    x' &= \abs{b - c} = \frac {aF}{a - F} - \frac {(a+x)F}{a + x - F} = F\cbr{\frac {a}{a - F} - \frac {a+x}{a + x - F}} =  \\
    &= F \cdot \frac {a^2 + ax - aF - a^2 - ax + aF + xF}{(a - F)(a + x - F)} = F \cdot \frac {xF}{(a - F)(a + x - F)} \\
    \Gamma_x &= \frac{x'}x = \frac{F^2}{(a - F)(a + x - F)} \to \frac{F^2}{\sqr{a - F}}.
    \\
    &\text{Нельзя: изображение по-разному растянет по осям $x$ и $y$ и понадобится теорема Пифагора}
    \end{align*}
}
\solutionspace{150pt}

\tasknumber{8}%
\task{%
    Доказать формулу тонкой линзы для собирающей линзы.
}
\solutionspace{120pt}

\tasknumber{9}%
\task{%
    Постройте ход луча $CL$ в тонкой линзе.
    Известно положение линзы и оба её фокуса (см.
    рис.
    на доске).
    Рассмотрите оба типа линзы, сделав 2 рисунка: собирающую и рассеивающую.
}
\solutionspace{120pt}

\tasknumber{10}%
\task{%
    На экране, расположенном иа расстоянии $80\,\text{см}$ от собирающей линзы,
    получено изображение точечного источника, расположенного на главной оптической оси линзы.
    На какое расстояние переместится изображение на экране,
    если при неподвижной линзе переместить источник на $2\,\text{см}$ в плоскости, перпендикулярной главной оптической оси?
    Фокусное расстояние линзы равно $40\,\text{см}$.
}
\answer{%
    \begin{align*}
    &\frac 1F = \frac 1a + \frac 1b \implies a = \frac{bF}{b-F} \implies \Gamma = \frac ba = \frac{b-F}F \\
    &y = x \cdot \Gamma = x \cdot \frac{b-F}F \implies d = y = 2\,\text{см}.
    \end{align*}
}
\solutionspace{120pt}

\tasknumber{11}%
\task{%
    Оптическая сила двояковыпуклой линзы в воздухе $5{,}5\,\text{дптр}$, а в воде $1{,}5\,\text{дптр}$.
    Определить показатель преломления $n$ материала, из которого изготовлена линза.
    Показатель преломления воды равен $1{,}33$.
}
\answer{%
    \begin{align*}
    D_1 &=\cbr{\frac n{n_1} - 1}\cbr{\frac 1{R_1} + \frac 1{R_2}}, \\
    D_2 &=\cbr{\frac n{n_2} - 1}\cbr{\frac 1{R_1} + \frac 1{R_2}}, \\
    \frac {D_2}{D_1} &=\frac{\frac n{n_2} - 1}{\frac n{n_1} - 1} \implies {D_2}\cbr{\frac n{n_1} - 1} = {D_1}\cbr{\frac n{n_2} - 1}  \implies n\cbr{\frac{D_2}{n_1} - \frac{D_1}{n_2}} = D_2 - D_1, \\
    n &= \frac{D_2 - D_1}{\frac{D_2}{n_1} - \frac{D_1}{n_2}} = \frac{n_1 n_2 (D_2 - D_1)}{D_2n_2 - D_1n_1} \approx 1{,}518.
    \end{align*}
}
\solutionspace{120pt}

\tasknumber{12}%
\task{%
    На каком расстоянии от собирающей линзы с фокусным расстоянием $50\,\text{дптр}$
    следует надо поместить предмет, чтобы расстояние
    от предмета до его действительного изображения было наименьшим?
}
\answer{%
    \begin{align*}
    \frac 1a &+ \frac 1b = D \implies b = \frac 1{D - \frac 1a} \implies \ell = a + b = a + \frac a{Da - 1} = \frac{ Da^2 }{Da - 1} \implies \\
    \implies \ell'_a &= \frac{ 2Da \cdot (Da - 1) - Da^2 \cdot D }{\sqr{Da - 1}}= \frac{ D^2a^2 - 2Da}{\sqr{Da - 1}} = \frac{ Da(Da - 2)}{\sqr{Da - 1}}\implies a_{\min} = \frac 2D \approx 40\,\text{мм}.
    \end{align*}
}
\solutionspace{120pt}

\tasknumber{13}%
\task{%
    Даны точечный источник света $S$, его изображение $S_1$, полученное с помощью собирающей линзы,
    и ближайший к источнику фокус линзы $F$ (см.
    рис.
    на доске).
    Расстояния $SF = \ell$ и $SS_1 = L$.
    Определить положение линзы и её фокусное расстояние.
}
\answer{%
    \begin{align*}
    \frac 1a + \frac 1b &= \frac 1F, \ell = a - F, L = a + b \implies a = \ell + F, b = L - a = L - \ell - F \\
    \frac 1{\ell + F} + \frac 1{L - \ell - F} &= \frac 1F \\
    F\ell + F^2 + LF - F\ell - F^2 &= L\ell - \ell^2 - F\ell + LF - F\ell - F^2 \\
    0 &= L\ell - \ell^2 - 2F\ell - F^2 \\
    0 &=  F^2 + 2F\ell - L\ell + \ell^2 \\
    F &= -\ell \pm \sqrt{\ell^2 +  L\ell - \ell^2} = -\ell \pm \sqrt{L\ell} \implies F = \sqrt{L\ell} - \ell \\
    a &= \ell + F = \ell + \sqrt{L\ell} - \ell = \sqrt{L\ell}.
    \end{align*}
}
\solutionspace{120pt}

\tasknumber{14}%
\task{%
    Расстояние от освещённого предмета до экрана $100\,\text{см}$.
    Линза, помещенная между ними, даёт чёткое изображение предмета на
    экране при двух положениях, расстояние между которыми $30\,\text{см}$.
    Найти фокусное расстояние линзы.
}
\answer{%
    \begin{align*}
    \frac 1a + \frac 1b &= \frac 1F, \frac 1{a-\ell} + \frac 1{b+\ell} = \frac 1F, a + b = L \\
    \frac 1a + \frac 1b &= \frac 1{a-\ell} + \frac 1{b+\ell}\implies \frac{a + b}{ab} = \frac{(a-\ell) + (b+\ell)}{(a-\ell)(b+\ell)} \\
    ab  &= (a - \ell)(b+\ell) \implies 0  = -b\ell + a\ell - \ell^2 \implies 0 = -b + a - \ell \implies b = a - \ell \\
    a + (a - \ell) &= L \implies a = \frac{L + \ell}2 \implies b = \frac{L - \ell}2 \\
    F &= \frac{ab}{a + b} = \frac{L^2 -\ell^2}{4L} \approx 22{,}8\,\text{см}.
    \end{align*}
}
\solutionspace{120pt}

\tasknumber{15}%
\task{%
    Предмет находится на расстоянии $70\,\text{см}$ от экрана.
    Между предметом и экраном помещают линзу, причём при одном
    положении линзы на экране получается увеличенное изображение предмета,
    а при другом — уменьшенное.
    Каково фокусное расстояние линзы, если
    линейные размеры первого изображения в два раза больше второго?
}
\answer{%
    \begin{align*}
    \frac 1a + \frac 1{L-a} &= \frac 1F, h_1 = h \cdot \frac{L-a}a, \\
    \frac 1b + \frac 1{L-b} &= \frac 1F, h_2 = h \cdot \frac{L-b}b, \\
    \frac{h_1}{h_2} &= 2 \implies \frac{(L-a)b}{(L-b)a} = 2, \\
    \frac 1F &= \frac{ L }{a(L-a)} = \frac{ L }{b(L-b)} \implies \frac{L-a}{L-b} = \frac b a \implies \frac {b^2}{a^2} = 2.
    \\
    \frac 1a + \frac 1{L-a} &= \frac 1b + \frac 1{L-b} \implies \frac L{a(L-a)} = \frac L{b(L-b)} \implies \\
    \implies aL - a^2 &= bL - b^2 \implies (a-b)L = (a-b)(a+b) \implies b = L - a, \\
    \frac{\sqr{L-a}}{a^2} &= 2 \implies \frac La - 1 = \sqrt{2} \implies a = \frac{ L }{\sqrt{2} + 1} \\
    F &= \frac{a(L-a)}L = \frac 1L \cdot \frac L{\sqrt{2} + 1} \cdot \frac {L\sqrt{2}}{\sqrt{2} + 1}= \frac { L\sqrt{2} }{ \sqr{\sqrt{2} + 1} } \approx 17{,}0\,\text{см}.
    \end{align*}
}
\solutionspace{120pt}

\tasknumber{16}%
\task{%
    (Задача-«гроб»: решать на обратной стороне) Квадрат со стороной $d = 3\,\text{см}$ расположен так,
    что 2 его стороны параллельны главной оптической оси собирающей линзы,
    его центр удален на $h = 4\,\text{см}$ от этой оси и на $a = 10\,\text{см}$ от плоскости линзы.
    Определите площадь изображения квадрата, если фокусное расстояние линзы составляет $F = 18\,\text{см}$.
    % (и сравните с площадью объекта, умноженной на квадрат увеличения центра квадрата).
}
\answer{%
    \begin{align*}
    &\text{Все явные вычисления — в см и $\text{см}^2$,} \\
    \frac 1 F &= \frac 1{a + \frac d2} + \frac 1b \implies b = \frac 1{\frac 1 F - \frac 1{a + \frac d2}} = \frac{F(a + \frac d2)}{a + \frac d2 - F} = -\frac{414}{13}, \\
    \frac 1 F &= \frac 1{a - \frac d2} + \frac 1c \implies c = \frac 1{\frac 1 F - \frac 1{a - \frac d2}} = \frac{F(a - \frac d2)}{a - \frac d2 - F} = -\frac{306}{19}, \\
    c - b &= \frac{F(a - \frac d2)}{a - \frac d2 - F} - \frac{F(a + \frac d2)}{a + \frac d2 - F} = F\cbr{ \frac{a - \frac d2}{a - \frac d2 - F} - \frac{a + \frac d2}{a + \frac d2 - F} } =  \\
    &= F \cdot \frac{a^2 + \frac {ad}2 - aF - \frac{ad}2 - \frac{d^2}4 + \frac{dF}2 - a^2 + \frac {ad}2 + aF - \frac{ad}2 + \frac{d^2}4 + \frac{dF}2}{\cbr{a + \frac d2 - F}\cbr{a - \frac d2 - F}}= F \cdot \frac {dF}{\cbr{a + \frac d2 - F}\cbr{a - \frac d2 - F}} = \frac{3888}{247}.
    \\
    \Gamma_b &= \frac b{a + \frac d2} = \frac{ F }{a + \frac d2 - F} = -\frac{36}{13}, \\
    \Gamma_c &= \frac c{a - \frac d2} = \frac{ F }{a - \frac d2 - F} = -\frac{36}{19}, \\
    &\text{ тут интересно отметить, что } \Gamma_x = \frac{ c - b}{ d } = \frac{ F^2 }{\cbr{a + \frac d2 - F}\cbr{a - \frac d2 - F}} \ne \Gamma_b \text{ или } \Gamma_c \text{ даже при малых $d$}.
    \\
    S' &= \frac{d \cdot \Gamma_b + d \cdot \Gamma_c}2 \cdot (c - b) = \frac d2 \cbr{\frac{ F }{a + \frac d2 - F} + \frac{ F }{a - \frac d2 - F}} \cdot \cbr{c - b} =  \\
    &=\frac {dF}2 \cbr{\frac 1{a + \frac d2 - F} + \frac 1{a - \frac d2 - F}} \cdot \frac {dF^2}{\cbr{a + \frac d2 - F}\cbr{a - \frac d2 - F}} =  \\
    &=\frac {dF}2 \cdot \frac{a - \frac d2 - F + a + \frac d2 - F}{\cbr{a + \frac d2 - F}\cbr{a - \frac d2 - F}} \cdot \frac {dF^2}{\cbr{a + \frac d2 - F}\cbr{a - \frac d2 - F}} =  \\
    &= \frac {d^2F^3}{2\sqr{a + \frac d2 - F}\sqr{a - \frac d2 - F}} \cdot (2a - 2F) = \frac {d^2F^3(a - F)}{ \sqr{\sqr{a - F} - \frac{d^2}4} } = -\frac{6718464}{61009}.
    \end{align*}
}

\variantsplitter

\addpersonalvariant{Анастасия Ламанова}

\tasknumber{1}%
\task{%
    Запишите известные вам виды классификации изображений.
}
\solutionspace{60pt}

\tasknumber{2}%
\task{%
    В каких линзах можно получить мнимое изображение объекта?
}
\answer{%
    $\text{ собирающие и рассеивающие }$
}
\solutionspace{40pt}

\tasknumber{3}%
\task{%
    Какое изображение называют мнимым?
}
\solutionspace{40pt}

\tasknumber{4}%
\task{%
    Есть две линзы, обозначим их 1 и 2.
    Известно что оптическая сила линзы 2 меньше, чем у линзы 1.
    Какая линза сильнее преломляет лучи?
}
\answer{%
    $1$
}
\solutionspace{40pt}

\tasknumber{5}%
\task{%
    Предмет находится на расстоянии $20\,\text{см}$ от собирающей линзы с фокусным расстоянием $15\,\text{см}$.
    Определите тип изображения, расстояние между предметом и его изображением, увеличение предмета.
    Сделайте схематичный рисунок (не обязательно в масштабе, но с сохранением свойств линзы и изображения).
}
\solutionspace{100pt}

\tasknumber{6}%
\task{%
    Объект находится на расстоянии $115\,\text{см}$ от линзы, а его действительное изображение — в $10\,\text{см}$ от неё.
    Определите увеличение предмета, фокусное расстояние линзы, оптическую силу линзы и её тип.
}
\solutionspace{80pt}

\tasknumber{7}%
\task{%
    Известно, что из формулы тонкой линзы $\cbr{\frac 1F = \frac 1a + \frac 1b}$
    и определения увеличения $\cbr{\Gamma_y = \frac ba}$ можно получить выражение
    для увеличения: $\Gamma_y = \frac {aF}{a - F} \cdot \frac 1a = \frac {F}{a - F}.$
    Назовём такое увеличение «поперечным»: поперёк главной оптической оси (поэтому и ${}_y$).
    Получите формулу для «продольного» увеличения $\Gamma_x$ небольшого предмета, находящегося на главной оптической оси.
    Можно ли применить эту формулу для предмета, не лежащего на главной оптической оси, почему?
}
\answer{%
    \begin{align*}
    \frac 1F &= \frac 1a + \frac 1b \implies b = \frac {aF}{a - F} \\
    \frac 1F &= \frac 1{a + x} + \frac 1c \implies c = \frac {(a+x)F}{a + x - F} \\
    x' &= \abs{b - c} = \frac {aF}{a - F} - \frac {(a+x)F}{a + x - F} = F\cbr{\frac {a}{a - F} - \frac {a+x}{a + x - F}} =  \\
    &= F \cdot \frac {a^2 + ax - aF - a^2 - ax + aF + xF}{(a - F)(a + x - F)} = F \cdot \frac {xF}{(a - F)(a + x - F)} \\
    \Gamma_x &= \frac{x'}x = \frac{F^2}{(a - F)(a + x - F)} \to \frac{F^2}{\sqr{a - F}}.
    \\
    &\text{Нельзя: изображение по-разному растянет по осям $x$ и $y$ и понадобится теорема Пифагора}
    \end{align*}
}
\solutionspace{150pt}

\tasknumber{8}%
\task{%
    Доказать формулу тонкой линзы для рассеивающей линзы.
}
\solutionspace{120pt}

\tasknumber{9}%
\task{%
    Постройте ход луча $BM$ в тонкой линзе.
    Известно положение линзы и оба её фокуса (см.
    рис.
    на доске).
    Рассмотрите оба типа линзы, сделав 2 рисунка: собирающую и рассеивающую.
}
\solutionspace{120pt}

\tasknumber{10}%
\task{%
    На экране, расположенном иа расстоянии $80\,\text{см}$ от собирающей линзы,
    получено изображение точечного источника, расположенного на главной оптической оси линзы.
    На какое расстояние переместится изображение на экране,
    если при неподвижной линзе переместить источник на $2\,\text{см}$ в плоскости, перпендикулярной главной оптической оси?
    Фокусное расстояние линзы равно $20\,\text{см}$.
}
\answer{%
    \begin{align*}
    &\frac 1F = \frac 1a + \frac 1b \implies a = \frac{bF}{b-F} \implies \Gamma = \frac ba = \frac{b-F}F \\
    &y = x \cdot \Gamma = x \cdot \frac{b-F}F \implies d = y = 6\,\text{см}.
    \end{align*}
}
\solutionspace{120pt}

\tasknumber{11}%
\task{%
    Оптическая сила двояковыпуклой линзы в воздухе $5{,}5\,\text{дптр}$, а в воде $1{,}6\,\text{дптр}$.
    Определить показатель преломления $n$ материала, из которого изготовлена линза.
    Показатель преломления воды равен $1{,}33$.
}
\answer{%
    \begin{align*}
    D_1 &=\cbr{\frac n{n_1} - 1}\cbr{\frac 1{R_1} + \frac 1{R_2}}, \\
    D_2 &=\cbr{\frac n{n_2} - 1}\cbr{\frac 1{R_1} + \frac 1{R_2}}, \\
    \frac {D_2}{D_1} &=\frac{\frac n{n_2} - 1}{\frac n{n_1} - 1} \implies {D_2}\cbr{\frac n{n_1} - 1} = {D_1}\cbr{\frac n{n_2} - 1}  \implies n\cbr{\frac{D_2}{n_1} - \frac{D_1}{n_2}} = D_2 - D_1, \\
    n &= \frac{D_2 - D_1}{\frac{D_2}{n_1} - \frac{D_1}{n_2}} = \frac{n_1 n_2 (D_2 - D_1)}{D_2n_2 - D_1n_1} \approx 1{,}538.
    \end{align*}
}
\solutionspace{120pt}

\tasknumber{12}%
\task{%
    На каком расстоянии от собирающей линзы с фокусным расстоянием $30\,\text{дптр}$
    следует надо поместить предмет, чтобы расстояние
    от предмета до его действительного изображения было наименьшим?
}
\answer{%
    \begin{align*}
    \frac 1a &+ \frac 1b = D \implies b = \frac 1{D - \frac 1a} \implies \ell = a + b = a + \frac a{Da - 1} = \frac{ Da^2 }{Da - 1} \implies \\
    \implies \ell'_a &= \frac{ 2Da \cdot (Da - 1) - Da^2 \cdot D }{\sqr{Da - 1}}= \frac{ D^2a^2 - 2Da}{\sqr{Da - 1}} = \frac{ Da(Da - 2)}{\sqr{Da - 1}}\implies a_{\min} = \frac 2D \approx 66{,}7\,\text{мм}.
    \end{align*}
}
\solutionspace{120pt}

\tasknumber{13}%
\task{%
    Даны точечный источник света $S$, его изображение $S_1$, полученное с помощью собирающей линзы,
    и ближайший к источнику фокус линзы $F$ (см.
    рис.
    на доске).
    Расстояния $SF = \ell$ и $SS_1 = L$.
    Определить положение линзы и её фокусное расстояние.
}
\answer{%
    \begin{align*}
    \frac 1a + \frac 1b &= \frac 1F, \ell = a - F, L = a + b \implies a = \ell + F, b = L - a = L - \ell - F \\
    \frac 1{\ell + F} + \frac 1{L - \ell - F} &= \frac 1F \\
    F\ell + F^2 + LF - F\ell - F^2 &= L\ell - \ell^2 - F\ell + LF - F\ell - F^2 \\
    0 &= L\ell - \ell^2 - 2F\ell - F^2 \\
    0 &=  F^2 + 2F\ell - L\ell + \ell^2 \\
    F &= -\ell \pm \sqrt{\ell^2 +  L\ell - \ell^2} = -\ell \pm \sqrt{L\ell} \implies F = \sqrt{L\ell} - \ell \\
    a &= \ell + F = \ell + \sqrt{L\ell} - \ell = \sqrt{L\ell}.
    \end{align*}
}
\solutionspace{120pt}

\tasknumber{14}%
\task{%
    Расстояние от освещённого предмета до экрана $80\,\text{см}$.
    Линза, помещенная между ними, даёт чёткое изображение предмета на
    экране при двух положениях, расстояние между которыми $30\,\text{см}$.
    Найти фокусное расстояние линзы.
}
\answer{%
    \begin{align*}
    \frac 1a + \frac 1b &= \frac 1F, \frac 1{a-\ell} + \frac 1{b+\ell} = \frac 1F, a + b = L \\
    \frac 1a + \frac 1b &= \frac 1{a-\ell} + \frac 1{b+\ell}\implies \frac{a + b}{ab} = \frac{(a-\ell) + (b+\ell)}{(a-\ell)(b+\ell)} \\
    ab  &= (a - \ell)(b+\ell) \implies 0  = -b\ell + a\ell - \ell^2 \implies 0 = -b + a - \ell \implies b = a - \ell \\
    a + (a - \ell) &= L \implies a = \frac{L + \ell}2 \implies b = \frac{L - \ell}2 \\
    F &= \frac{ab}{a + b} = \frac{L^2 -\ell^2}{4L} \approx 17{,}2\,\text{см}.
    \end{align*}
}
\solutionspace{120pt}

\tasknumber{15}%
\task{%
    Предмет находится на расстоянии $70\,\text{см}$ от экрана.
    Между предметом и экраном помещают линзу, причём при одном
    положении линзы на экране получается увеличенное изображение предмета,
    а при другом — уменьшенное.
    Каково фокусное расстояние линзы, если
    линейные размеры первого изображения в два раза больше второго?
}
\answer{%
    \begin{align*}
    \frac 1a + \frac 1{L-a} &= \frac 1F, h_1 = h \cdot \frac{L-a}a, \\
    \frac 1b + \frac 1{L-b} &= \frac 1F, h_2 = h \cdot \frac{L-b}b, \\
    \frac{h_1}{h_2} &= 2 \implies \frac{(L-a)b}{(L-b)a} = 2, \\
    \frac 1F &= \frac{ L }{a(L-a)} = \frac{ L }{b(L-b)} \implies \frac{L-a}{L-b} = \frac b a \implies \frac {b^2}{a^2} = 2.
    \\
    \frac 1a + \frac 1{L-a} &= \frac 1b + \frac 1{L-b} \implies \frac L{a(L-a)} = \frac L{b(L-b)} \implies \\
    \implies aL - a^2 &= bL - b^2 \implies (a-b)L = (a-b)(a+b) \implies b = L - a, \\
    \frac{\sqr{L-a}}{a^2} &= 2 \implies \frac La - 1 = \sqrt{2} \implies a = \frac{ L }{\sqrt{2} + 1} \\
    F &= \frac{a(L-a)}L = \frac 1L \cdot \frac L{\sqrt{2} + 1} \cdot \frac {L\sqrt{2}}{\sqrt{2} + 1}= \frac { L\sqrt{2} }{ \sqr{\sqrt{2} + 1} } \approx 17{,}0\,\text{см}.
    \end{align*}
}
\solutionspace{120pt}

\tasknumber{16}%
\task{%
    (Задача-«гроб»: решать на обратной стороне) Квадрат со стороной $d = 3\,\text{см}$ расположен так,
    что 2 его стороны параллельны главной оптической оси рассеивающей линзы,
    его центр удален на $h = 5\,\text{см}$ от этой оси и на $a = 12\,\text{см}$ от плоскости линзы.
    Определите площадь изображения квадрата, если фокусное расстояние линзы составляет $F = 18\,\text{см}$.
    % (и сравните с площадью объекта, умноженной на квадрат увеличения центра квадрата).
}
\answer{%
    \begin{align*}
    &\text{Все явные вычисления — в см и $\text{см}^2$,} \\
    \frac 1 F &= \frac 1{a + \frac d2} + \frac 1b \implies b = \frac 1{\frac 1 F - \frac 1{a + \frac d2}} = \frac{F(a + \frac d2)}{a + \frac d2 - F} = -\frac{54}7, \\
    \frac 1 F &= \frac 1{a - \frac d2} + \frac 1c \implies c = \frac 1{\frac 1 F - \frac 1{a - \frac d2}} = \frac{F(a - \frac d2)}{a - \frac d2 - F} = -\frac{126}{19}, \\
    c - b &= \frac{F(a - \frac d2)}{a - \frac d2 - F} - \frac{F(a + \frac d2)}{a + \frac d2 - F} = F\cbr{ \frac{a - \frac d2}{a - \frac d2 - F} - \frac{a + \frac d2}{a + \frac d2 - F} } =  \\
    &= F \cdot \frac{a^2 + \frac {ad}2 - aF - \frac{ad}2 - \frac{d^2}4 + \frac{dF}2 - a^2 + \frac {ad}2 + aF - \frac{ad}2 + \frac{d^2}4 + \frac{dF}2}{\cbr{a + \frac d2 - F}\cbr{a - \frac d2 - F}}= F \cdot \frac {dF}{\cbr{a + \frac d2 - F}\cbr{a - \frac d2 - F}} = \frac{144}{133}.
    \\
    \Gamma_b &= \frac b{a + \frac d2} = \frac{ F }{a + \frac d2 - F} = -\frac47, \\
    \Gamma_c &= \frac c{a - \frac d2} = \frac{ F }{a - \frac d2 - F} = -\frac{12}{19}, \\
    &\text{ тут интересно отметить, что } \Gamma_x = \frac{ c - b}{ d } = \frac{ F^2 }{\cbr{a + \frac d2 - F}\cbr{a - \frac d2 - F}} \ne \Gamma_b \text{ или } \Gamma_c \text{ даже при малых $d$}.
    \\
    S' &= \frac{d \cdot \Gamma_b + d \cdot \Gamma_c}2 \cdot (c - b) = \frac d2 \cbr{\frac{ F }{a + \frac d2 - F} + \frac{ F }{a - \frac d2 - F}} \cdot \cbr{c - b} =  \\
    &=\frac {dF}2 \cbr{\frac 1{a + \frac d2 - F} + \frac 1{a - \frac d2 - F}} \cdot \frac {dF^2}{\cbr{a + \frac d2 - F}\cbr{a - \frac d2 - F}} =  \\
    &=\frac {dF}2 \cdot \frac{a - \frac d2 - F + a + \frac d2 - F}{\cbr{a + \frac d2 - F}\cbr{a - \frac d2 - F}} \cdot \frac {dF^2}{\cbr{a + \frac d2 - F}\cbr{a - \frac d2 - F}} =  \\
    &= \frac {d^2F^3}{2\sqr{a + \frac d2 - F}\sqr{a - \frac d2 - F}} \cdot (2a - 2F) = \frac {d^2F^3(a - F)}{ \sqr{\sqr{a - F} - \frac{d^2}4} } = -\frac{34560}{17689}.
    \end{align*}
}

\variantsplitter

\addpersonalvariant{Виктория Легонькова}

\tasknumber{1}%
\task{%
    Запишите известные вам виды классификации изображений.
}
\solutionspace{60pt}

\tasknumber{2}%
\task{%
    В каких линзах можно получить обратное изображение объекта?
}
\answer{%
    $\text{ собирающие }$
}
\solutionspace{40pt}

\tasknumber{3}%
\task{%
    Какое изображение называют действительным?
}
\solutionspace{40pt}

\tasknumber{4}%
\task{%
    Есть две линзы, обозначим их 1 и 2.
    Известно что оптическая сила линзы 1 меньше, чем у линзы 2.
    Какая линза сильнее преломляет лучи?
}
\answer{%
    $2$
}
\solutionspace{40pt}

\tasknumber{5}%
\task{%
    Предмет находится на расстоянии $30\,\text{см}$ от рассеивающей линзы с фокусным расстоянием $25\,\text{см}$.
    Определите тип изображения, расстояние между предметом и его изображением, увеличение предмета.
    Сделайте схематичный рисунок (не обязательно в масштабе, но с сохранением свойств линзы и изображения).
}
\solutionspace{100pt}

\tasknumber{6}%
\task{%
    Объект находится на расстоянии $115\,\text{см}$ от линзы, а его мнимое изображение — в $20\,\text{см}$ от неё.
    Определите увеличение предмета, фокусное расстояние линзы, оптическую силу линзы и её тип.
}
\solutionspace{80pt}

\tasknumber{7}%
\task{%
    Известно, что из формулы тонкой линзы $\cbr{\frac 1F = \frac 1a + \frac 1b}$
    и определения увеличения $\cbr{\Gamma_y = \frac ba}$ можно получить выражение
    для увеличения: $\Gamma_y = \frac {aF}{a - F} \cdot \frac 1a = \frac {F}{a - F}.$
    Назовём такое увеличение «поперечным»: поперёк главной оптической оси (поэтому и ${}_y$).
    Получите формулу для «продольного» увеличения $\Gamma_x$ небольшого предмета, находящегося на главной оптической оси.
    Можно ли применить эту формулу для предмета, не лежащего на главной оптической оси, почему?
}
\answer{%
    \begin{align*}
    \frac 1F &= \frac 1a + \frac 1b \implies b = \frac {aF}{a - F} \\
    \frac 1F &= \frac 1{a + x} + \frac 1c \implies c = \frac {(a+x)F}{a + x - F} \\
    x' &= \abs{b - c} = \frac {aF}{a - F} - \frac {(a+x)F}{a + x - F} = F\cbr{\frac {a}{a - F} - \frac {a+x}{a + x - F}} =  \\
    &= F \cdot \frac {a^2 + ax - aF - a^2 - ax + aF + xF}{(a - F)(a + x - F)} = F \cdot \frac {xF}{(a - F)(a + x - F)} \\
    \Gamma_x &= \frac{x'}x = \frac{F^2}{(a - F)(a + x - F)} \to \frac{F^2}{\sqr{a - F}}.
    \\
    &\text{Нельзя: изображение по-разному растянет по осям $x$ и $y$ и понадобится теорема Пифагора}
    \end{align*}
}
\solutionspace{150pt}

\tasknumber{8}%
\task{%
    Доказать формулу тонкой линзы для собирающей линзы.
}
\solutionspace{120pt}

\tasknumber{9}%
\task{%
    Постройте ход луча $BL$ в тонкой линзе.
    Известно положение линзы и оба её фокуса (см.
    рис.
    на доске).
    Рассмотрите оба типа линзы, сделав 2 рисунка: собирающую и рассеивающую.
}
\solutionspace{120pt}

\tasknumber{10}%
\task{%
    На экране, расположенном иа расстоянии $80\,\text{см}$ от собирающей линзы,
    получено изображение точечного источника, расположенного на главной оптической оси линзы.
    На какое расстояние переместится изображение на экране,
    если при неподвижном источнике переместить линзу на $3\,\text{см}$ в плоскости, перпендикулярной главной оптической оси?
    Фокусное расстояние линзы равно $40\,\text{см}$.
}
\answer{%
    \begin{align*}
    &\frac 1F = \frac 1a + \frac 1b \implies a = \frac{bF}{b-F} \implies \Gamma = \frac ba = \frac{b-F}F \\
    &y = x \cdot \Gamma = x \cdot \frac{b-F}F \implies d = x + y = 6\,\text{см}.
    \end{align*}
}
\solutionspace{120pt}

\tasknumber{11}%
\task{%
    Оптическая сила двояковыпуклой линзы в воздухе $5\,\text{дптр}$, а в воде $1{,}6\,\text{дптр}$.
    Определить показатель преломления $n$ материала, из которого изготовлена линза.
    Показатель преломления воды равен $1{,}33$.
}
\answer{%
    \begin{align*}
    D_1 &=\cbr{\frac n{n_1} - 1}\cbr{\frac 1{R_1} + \frac 1{R_2}}, \\
    D_2 &=\cbr{\frac n{n_2} - 1}\cbr{\frac 1{R_1} + \frac 1{R_2}}, \\
    \frac {D_2}{D_1} &=\frac{\frac n{n_2} - 1}{\frac n{n_1} - 1} \implies {D_2}\cbr{\frac n{n_1} - 1} = {D_1}\cbr{\frac n{n_2} - 1}  \implies n\cbr{\frac{D_2}{n_1} - \frac{D_1}{n_2}} = D_2 - D_1, \\
    n &= \frac{D_2 - D_1}{\frac{D_2}{n_1} - \frac{D_1}{n_2}} = \frac{n_1 n_2 (D_2 - D_1)}{D_2n_2 - D_1n_1} \approx 1{,}575.
    \end{align*}
}
\solutionspace{120pt}

\tasknumber{12}%
\task{%
    На каком расстоянии от собирающей линзы с фокусным расстоянием $30\,\text{дптр}$
    следует надо поместить предмет, чтобы расстояние
    от предмета до его действительного изображения было наименьшим?
}
\answer{%
    \begin{align*}
    \frac 1a &+ \frac 1b = D \implies b = \frac 1{D - \frac 1a} \implies \ell = a + b = a + \frac a{Da - 1} = \frac{ Da^2 }{Da - 1} \implies \\
    \implies \ell'_a &= \frac{ 2Da \cdot (Da - 1) - Da^2 \cdot D }{\sqr{Da - 1}}= \frac{ D^2a^2 - 2Da}{\sqr{Da - 1}} = \frac{ Da(Da - 2)}{\sqr{Da - 1}}\implies a_{\min} = \frac 2D \approx 66{,}7\,\text{мм}.
    \end{align*}
}
\solutionspace{120pt}

\tasknumber{13}%
\task{%
    Даны точечный источник света $S$, его изображение $S_1$, полученное с помощью собирающей линзы,
    и ближайший к источнику фокус линзы $F$ (см.
    рис.
    на доске).
    Расстояния $SF = \ell$ и $SS_1 = L$.
    Определить положение линзы и её фокусное расстояние.
}
\answer{%
    \begin{align*}
    \frac 1a + \frac 1b &= \frac 1F, \ell = a - F, L = a + b \implies a = \ell + F, b = L - a = L - \ell - F \\
    \frac 1{\ell + F} + \frac 1{L - \ell - F} &= \frac 1F \\
    F\ell + F^2 + LF - F\ell - F^2 &= L\ell - \ell^2 - F\ell + LF - F\ell - F^2 \\
    0 &= L\ell - \ell^2 - 2F\ell - F^2 \\
    0 &=  F^2 + 2F\ell - L\ell + \ell^2 \\
    F &= -\ell \pm \sqrt{\ell^2 +  L\ell - \ell^2} = -\ell \pm \sqrt{L\ell} \implies F = \sqrt{L\ell} - \ell \\
    a &= \ell + F = \ell + \sqrt{L\ell} - \ell = \sqrt{L\ell}.
    \end{align*}
}
\solutionspace{120pt}

\tasknumber{14}%
\task{%
    Расстояние от освещённого предмета до экрана $80\,\text{см}$.
    Линза, помещенная между ними, даёт чёткое изображение предмета на
    экране при двух положениях, расстояние между которыми $40\,\text{см}$.
    Найти фокусное расстояние линзы.
}
\answer{%
    \begin{align*}
    \frac 1a + \frac 1b &= \frac 1F, \frac 1{a-\ell} + \frac 1{b+\ell} = \frac 1F, a + b = L \\
    \frac 1a + \frac 1b &= \frac 1{a-\ell} + \frac 1{b+\ell}\implies \frac{a + b}{ab} = \frac{(a-\ell) + (b+\ell)}{(a-\ell)(b+\ell)} \\
    ab  &= (a - \ell)(b+\ell) \implies 0  = -b\ell + a\ell - \ell^2 \implies 0 = -b + a - \ell \implies b = a - \ell \\
    a + (a - \ell) &= L \implies a = \frac{L + \ell}2 \implies b = \frac{L - \ell}2 \\
    F &= \frac{ab}{a + b} = \frac{L^2 -\ell^2}{4L} \approx 15\,\text{см}.
    \end{align*}
}
\solutionspace{120pt}

\tasknumber{15}%
\task{%
    Предмет находится на расстоянии $60\,\text{см}$ от экрана.
    Между предметом и экраном помещают линзу, причём при одном
    положении линзы на экране получается увеличенное изображение предмета,
    а при другом — уменьшенное.
    Каково фокусное расстояние линзы, если
    линейные размеры первого изображения в три раза больше второго?
}
\answer{%
    \begin{align*}
    \frac 1a + \frac 1{L-a} &= \frac 1F, h_1 = h \cdot \frac{L-a}a, \\
    \frac 1b + \frac 1{L-b} &= \frac 1F, h_2 = h \cdot \frac{L-b}b, \\
    \frac{h_1}{h_2} &= 3 \implies \frac{(L-a)b}{(L-b)a} = 3, \\
    \frac 1F &= \frac{ L }{a(L-a)} = \frac{ L }{b(L-b)} \implies \frac{L-a}{L-b} = \frac b a \implies \frac {b^2}{a^2} = 3.
    \\
    \frac 1a + \frac 1{L-a} &= \frac 1b + \frac 1{L-b} \implies \frac L{a(L-a)} = \frac L{b(L-b)} \implies \\
    \implies aL - a^2 &= bL - b^2 \implies (a-b)L = (a-b)(a+b) \implies b = L - a, \\
    \frac{\sqr{L-a}}{a^2} &= 3 \implies \frac La - 1 = \sqrt{3} \implies a = \frac{ L }{\sqrt{3} + 1} \\
    F &= \frac{a(L-a)}L = \frac 1L \cdot \frac L{\sqrt{3} + 1} \cdot \frac {L\sqrt{3}}{\sqrt{3} + 1}= \frac { L\sqrt{3} }{ \sqr{\sqrt{3} + 1} } \approx 13{,}9\,\text{см}.
    \end{align*}
}
\solutionspace{120pt}

\tasknumber{16}%
\task{%
    (Задача-«гроб»: решать на обратной стороне) Квадрат со стороной $d = 2\,\text{см}$ расположен так,
    что 2 его стороны параллельны главной оптической оси рассеивающей линзы,
    его центр удален на $h = 4\,\text{см}$ от этой оси и на $a = 10\,\text{см}$ от плоскости линзы.
    Определите площадь изображения квадрата, если фокусное расстояние линзы составляет $F = 20\,\text{см}$.
    % (и сравните с площадью объекта, умноженной на квадрат увеличения центра квадрата).
}
\answer{%
    \begin{align*}
    &\text{Все явные вычисления — в см и $\text{см}^2$,} \\
    \frac 1 F &= \frac 1{a + \frac d2} + \frac 1b \implies b = \frac 1{\frac 1 F - \frac 1{a + \frac d2}} = \frac{F(a + \frac d2)}{a + \frac d2 - F} = -\frac{220}{31}, \\
    \frac 1 F &= \frac 1{a - \frac d2} + \frac 1c \implies c = \frac 1{\frac 1 F - \frac 1{a - \frac d2}} = \frac{F(a - \frac d2)}{a - \frac d2 - F} = -\frac{180}{29}, \\
    c - b &= \frac{F(a - \frac d2)}{a - \frac d2 - F} - \frac{F(a + \frac d2)}{a + \frac d2 - F} = F\cbr{ \frac{a - \frac d2}{a - \frac d2 - F} - \frac{a + \frac d2}{a + \frac d2 - F} } =  \\
    &= F \cdot \frac{a^2 + \frac {ad}2 - aF - \frac{ad}2 - \frac{d^2}4 + \frac{dF}2 - a^2 + \frac {ad}2 + aF - \frac{ad}2 + \frac{d^2}4 + \frac{dF}2}{\cbr{a + \frac d2 - F}\cbr{a - \frac d2 - F}}= F \cdot \frac {dF}{\cbr{a + \frac d2 - F}\cbr{a - \frac d2 - F}} = \frac{800}{899}.
    \\
    \Gamma_b &= \frac b{a + \frac d2} = \frac{ F }{a + \frac d2 - F} = -\frac{20}{31}, \\
    \Gamma_c &= \frac c{a - \frac d2} = \frac{ F }{a - \frac d2 - F} = -\frac{20}{29}, \\
    &\text{ тут интересно отметить, что } \Gamma_x = \frac{ c - b}{ d } = \frac{ F^2 }{\cbr{a + \frac d2 - F}\cbr{a - \frac d2 - F}} \ne \Gamma_b \text{ или } \Gamma_c \text{ даже при малых $d$}.
    \\
    S' &= \frac{d \cdot \Gamma_b + d \cdot \Gamma_c}2 \cdot (c - b) = \frac d2 \cbr{\frac{ F }{a + \frac d2 - F} + \frac{ F }{a - \frac d2 - F}} \cdot \cbr{c - b} =  \\
    &=\frac {dF}2 \cbr{\frac 1{a + \frac d2 - F} + \frac 1{a - \frac d2 - F}} \cdot \frac {dF^2}{\cbr{a + \frac d2 - F}\cbr{a - \frac d2 - F}} =  \\
    &=\frac {dF}2 \cdot \frac{a - \frac d2 - F + a + \frac d2 - F}{\cbr{a + \frac d2 - F}\cbr{a - \frac d2 - F}} \cdot \frac {dF^2}{\cbr{a + \frac d2 - F}\cbr{a - \frac d2 - F}} =  \\
    &= \frac {d^2F^3}{2\sqr{a + \frac d2 - F}\sqr{a - \frac d2 - F}} \cdot (2a - 2F) = \frac {d^2F^3(a - F)}{ \sqr{\sqr{a - F} - \frac{d^2}4} } = -\frac{960000}{808201}.
    \end{align*}
}

\variantsplitter

\addpersonalvariant{Семён Мартынов}

\tasknumber{1}%
\task{%
    Запишите известные вам виды классификации изображений.
}
\solutionspace{60pt}

\tasknumber{2}%
\task{%
    В каких линзах можно получить уменьшенное изображение объекта?
}
\answer{%
    $\text{ собирающие и рассеивающие }$
}
\solutionspace{40pt}

\tasknumber{3}%
\task{%
    Какое изображение называют мнимым?
}
\solutionspace{40pt}

\tasknumber{4}%
\task{%
    Есть две линзы, обозначим их 1 и 2.
    Известно что оптическая сила линзы 2 меньше, чем у линзы 1.
    Какая линза сильнее преломляет лучи?
}
\answer{%
    $1$
}
\solutionspace{40pt}

\tasknumber{5}%
\task{%
    Предмет находится на расстоянии $10\,\text{см}$ от собирающей линзы с фокусным расстоянием $6\,\text{см}$.
    Определите тип изображения, расстояние между предметом и его изображением, увеличение предмета.
    Сделайте схематичный рисунок (не обязательно в масштабе, но с сохранением свойств линзы и изображения).
}
\solutionspace{100pt}

\tasknumber{6}%
\task{%
    Объект находится на расстоянии $25\,\text{см}$ от линзы, а его действительное изображение — в $40\,\text{см}$ от неё.
    Определите увеличение предмета, фокусное расстояние линзы, оптическую силу линзы и её тип.
}
\solutionspace{80pt}

\tasknumber{7}%
\task{%
    Известно, что из формулы тонкой линзы $\cbr{\frac 1F = \frac 1a + \frac 1b}$
    и определения увеличения $\cbr{\Gamma_y = \frac ba}$ можно получить выражение
    для увеличения: $\Gamma_y = \frac {aF}{a - F} \cdot \frac 1a = \frac {F}{a - F}.$
    Назовём такое увеличение «поперечным»: поперёк главной оптической оси (поэтому и ${}_y$).
    Получите формулу для «продольного» увеличения $\Gamma_x$ небольшого предмета, находящегося на главной оптической оси.
    Можно ли применить эту формулу для предмета, не лежащего на главной оптической оси, почему?
}
\answer{%
    \begin{align*}
    \frac 1F &= \frac 1a + \frac 1b \implies b = \frac {aF}{a - F} \\
    \frac 1F &= \frac 1{a + x} + \frac 1c \implies c = \frac {(a+x)F}{a + x - F} \\
    x' &= \abs{b - c} = \frac {aF}{a - F} - \frac {(a+x)F}{a + x - F} = F\cbr{\frac {a}{a - F} - \frac {a+x}{a + x - F}} =  \\
    &= F \cdot \frac {a^2 + ax - aF - a^2 - ax + aF + xF}{(a - F)(a + x - F)} = F \cdot \frac {xF}{(a - F)(a + x - F)} \\
    \Gamma_x &= \frac{x'}x = \frac{F^2}{(a - F)(a + x - F)} \to \frac{F^2}{\sqr{a - F}}.
    \\
    &\text{Нельзя: изображение по-разному растянет по осям $x$ и $y$ и понадобится теорема Пифагора}
    \end{align*}
}
\solutionspace{150pt}

\tasknumber{8}%
\task{%
    Доказать формулу тонкой линзы для рассеивающей линзы.
}
\solutionspace{120pt}

\tasknumber{9}%
\task{%
    Постройте ход луча $CK$ в тонкой линзе.
    Известно положение линзы и оба её фокуса (см.
    рис.
    на доске).
    Рассмотрите оба типа линзы, сделав 2 рисунка: собирающую и рассеивающую.
}
\solutionspace{120pt}

\tasknumber{10}%
\task{%
    На экране, расположенном иа расстоянии $120\,\text{см}$ от собирающей линзы,
    получено изображение точечного источника, расположенного на главной оптической оси линзы.
    На какое расстояние переместится изображение на экране,
    если при неподвижной линзе переместить источник на $3\,\text{см}$ в плоскости, перпендикулярной главной оптической оси?
    Фокусное расстояние линзы равно $20\,\text{см}$.
}
\answer{%
    \begin{align*}
    &\frac 1F = \frac 1a + \frac 1b \implies a = \frac{bF}{b-F} \implies \Gamma = \frac ba = \frac{b-F}F \\
    &y = x \cdot \Gamma = x \cdot \frac{b-F}F \implies d = y = 15\,\text{см}.
    \end{align*}
}
\solutionspace{120pt}

\tasknumber{11}%
\task{%
    Оптическая сила двояковыпуклой линзы в воздухе $5\,\text{дптр}$, а в воде $1{,}5\,\text{дптр}$.
    Определить показатель преломления $n$ материала, из которого изготовлена линза.
    Показатель преломления воды равен $1{,}33$.
}
\answer{%
    \begin{align*}
    D_1 &=\cbr{\frac n{n_1} - 1}\cbr{\frac 1{R_1} + \frac 1{R_2}}, \\
    D_2 &=\cbr{\frac n{n_2} - 1}\cbr{\frac 1{R_1} + \frac 1{R_2}}, \\
    \frac {D_2}{D_1} &=\frac{\frac n{n_2} - 1}{\frac n{n_1} - 1} \implies {D_2}\cbr{\frac n{n_1} - 1} = {D_1}\cbr{\frac n{n_2} - 1}  \implies n\cbr{\frac{D_2}{n_1} - \frac{D_1}{n_2}} = D_2 - D_1, \\
    n &= \frac{D_2 - D_1}{\frac{D_2}{n_1} - \frac{D_1}{n_2}} = \frac{n_1 n_2 (D_2 - D_1)}{D_2n_2 - D_1n_1} \approx 1{,}549.
    \end{align*}
}
\solutionspace{120pt}

\tasknumber{12}%
\task{%
    На каком расстоянии от собирающей линзы с фокусным расстоянием $30\,\text{дптр}$
    следует надо поместить предмет, чтобы расстояние
    от предмета до его действительного изображения было наименьшим?
}
\answer{%
    \begin{align*}
    \frac 1a &+ \frac 1b = D \implies b = \frac 1{D - \frac 1a} \implies \ell = a + b = a + \frac a{Da - 1} = \frac{ Da^2 }{Da - 1} \implies \\
    \implies \ell'_a &= \frac{ 2Da \cdot (Da - 1) - Da^2 \cdot D }{\sqr{Da - 1}}= \frac{ D^2a^2 - 2Da}{\sqr{Da - 1}} = \frac{ Da(Da - 2)}{\sqr{Da - 1}}\implies a_{\min} = \frac 2D \approx 66{,}7\,\text{мм}.
    \end{align*}
}
\solutionspace{120pt}

\tasknumber{13}%
\task{%
    Даны точечный источник света $S$, его изображение $S_1$, полученное с помощью собирающей линзы,
    и ближайший к источнику фокус линзы $F$ (см.
    рис.
    на доске).
    Расстояния $SF = \ell$ и $SS_1 = L$.
    Определить положение линзы и её фокусное расстояние.
}
\answer{%
    \begin{align*}
    \frac 1a + \frac 1b &= \frac 1F, \ell = a - F, L = a + b \implies a = \ell + F, b = L - a = L - \ell - F \\
    \frac 1{\ell + F} + \frac 1{L - \ell - F} &= \frac 1F \\
    F\ell + F^2 + LF - F\ell - F^2 &= L\ell - \ell^2 - F\ell + LF - F\ell - F^2 \\
    0 &= L\ell - \ell^2 - 2F\ell - F^2 \\
    0 &=  F^2 + 2F\ell - L\ell + \ell^2 \\
    F &= -\ell \pm \sqrt{\ell^2 +  L\ell - \ell^2} = -\ell \pm \sqrt{L\ell} \implies F = \sqrt{L\ell} - \ell \\
    a &= \ell + F = \ell + \sqrt{L\ell} - \ell = \sqrt{L\ell}.
    \end{align*}
}
\solutionspace{120pt}

\tasknumber{14}%
\task{%
    Расстояние от освещённого предмета до экрана $80\,\text{см}$.
    Линза, помещенная между ними, даёт чёткое изображение предмета на
    экране при двух положениях, расстояние между которыми $20\,\text{см}$.
    Найти фокусное расстояние линзы.
}
\answer{%
    \begin{align*}
    \frac 1a + \frac 1b &= \frac 1F, \frac 1{a-\ell} + \frac 1{b+\ell} = \frac 1F, a + b = L \\
    \frac 1a + \frac 1b &= \frac 1{a-\ell} + \frac 1{b+\ell}\implies \frac{a + b}{ab} = \frac{(a-\ell) + (b+\ell)}{(a-\ell)(b+\ell)} \\
    ab  &= (a - \ell)(b+\ell) \implies 0  = -b\ell + a\ell - \ell^2 \implies 0 = -b + a - \ell \implies b = a - \ell \\
    a + (a - \ell) &= L \implies a = \frac{L + \ell}2 \implies b = \frac{L - \ell}2 \\
    F &= \frac{ab}{a + b} = \frac{L^2 -\ell^2}{4L} \approx 18{,}8\,\text{см}.
    \end{align*}
}
\solutionspace{120pt}

\tasknumber{15}%
\task{%
    Предмет находится на расстоянии $70\,\text{см}$ от экрана.
    Между предметом и экраном помещают линзу, причём при одном
    положении линзы на экране получается увеличенное изображение предмета,
    а при другом — уменьшенное.
    Каково фокусное расстояние линзы, если
    линейные размеры первого изображения в пять раз больше второго?
}
\answer{%
    \begin{align*}
    \frac 1a + \frac 1{L-a} &= \frac 1F, h_1 = h \cdot \frac{L-a}a, \\
    \frac 1b + \frac 1{L-b} &= \frac 1F, h_2 = h \cdot \frac{L-b}b, \\
    \frac{h_1}{h_2} &= 5 \implies \frac{(L-a)b}{(L-b)a} = 5, \\
    \frac 1F &= \frac{ L }{a(L-a)} = \frac{ L }{b(L-b)} \implies \frac{L-a}{L-b} = \frac b a \implies \frac {b^2}{a^2} = 5.
    \\
    \frac 1a + \frac 1{L-a} &= \frac 1b + \frac 1{L-b} \implies \frac L{a(L-a)} = \frac L{b(L-b)} \implies \\
    \implies aL - a^2 &= bL - b^2 \implies (a-b)L = (a-b)(a+b) \implies b = L - a, \\
    \frac{\sqr{L-a}}{a^2} &= 5 \implies \frac La - 1 = \sqrt{5} \implies a = \frac{ L }{\sqrt{5} + 1} \\
    F &= \frac{a(L-a)}L = \frac 1L \cdot \frac L{\sqrt{5} + 1} \cdot \frac {L\sqrt{5}}{\sqrt{5} + 1}= \frac { L\sqrt{5} }{ \sqr{\sqrt{5} + 1} } \approx 14{,}9\,\text{см}.
    \end{align*}
}
\solutionspace{120pt}

\tasknumber{16}%
\task{%
    (Задача-«гроб»: решать на обратной стороне) Квадрат со стороной $d = 3\,\text{см}$ расположен так,
    что 2 его стороны параллельны главной оптической оси рассеивающей линзы,
    его центр удален на $h = 4\,\text{см}$ от этой оси и на $a = 10\,\text{см}$ от плоскости линзы.
    Определите площадь изображения квадрата, если фокусное расстояние линзы составляет $F = 20\,\text{см}$.
    % (и сравните с площадью объекта, умноженной на квадрат увеличения центра квадрата).
}
\answer{%
    \begin{align*}
    &\text{Все явные вычисления — в см и $\text{см}^2$,} \\
    \frac 1 F &= \frac 1{a + \frac d2} + \frac 1b \implies b = \frac 1{\frac 1 F - \frac 1{a + \frac d2}} = \frac{F(a + \frac d2)}{a + \frac d2 - F} = -\frac{460}{63}, \\
    \frac 1 F &= \frac 1{a - \frac d2} + \frac 1c \implies c = \frac 1{\frac 1 F - \frac 1{a - \frac d2}} = \frac{F(a - \frac d2)}{a - \frac d2 - F} = -\frac{340}{57}, \\
    c - b &= \frac{F(a - \frac d2)}{a - \frac d2 - F} - \frac{F(a + \frac d2)}{a + \frac d2 - F} = F\cbr{ \frac{a - \frac d2}{a - \frac d2 - F} - \frac{a + \frac d2}{a + \frac d2 - F} } =  \\
    &= F \cdot \frac{a^2 + \frac {ad}2 - aF - \frac{ad}2 - \frac{d^2}4 + \frac{dF}2 - a^2 + \frac {ad}2 + aF - \frac{ad}2 + \frac{d^2}4 + \frac{dF}2}{\cbr{a + \frac d2 - F}\cbr{a - \frac d2 - F}}= F \cdot \frac {dF}{\cbr{a + \frac d2 - F}\cbr{a - \frac d2 - F}} = \frac{1600}{1197}.
    \\
    \Gamma_b &= \frac b{a + \frac d2} = \frac{ F }{a + \frac d2 - F} = -\frac{40}{63}, \\
    \Gamma_c &= \frac c{a - \frac d2} = \frac{ F }{a - \frac d2 - F} = -\frac{40}{57}, \\
    &\text{ тут интересно отметить, что } \Gamma_x = \frac{ c - b}{ d } = \frac{ F^2 }{\cbr{a + \frac d2 - F}\cbr{a - \frac d2 - F}} \ne \Gamma_b \text{ или } \Gamma_c \text{ даже при малых $d$}.
    \\
    S' &= \frac{d \cdot \Gamma_b + d \cdot \Gamma_c}2 \cdot (c - b) = \frac d2 \cbr{\frac{ F }{a + \frac d2 - F} + \frac{ F }{a - \frac d2 - F}} \cdot \cbr{c - b} =  \\
    &=\frac {dF}2 \cbr{\frac 1{a + \frac d2 - F} + \frac 1{a - \frac d2 - F}} \cdot \frac {dF^2}{\cbr{a + \frac d2 - F}\cbr{a - \frac d2 - F}} =  \\
    &=\frac {dF}2 \cdot \frac{a - \frac d2 - F + a + \frac d2 - F}{\cbr{a + \frac d2 - F}\cbr{a - \frac d2 - F}} \cdot \frac {dF^2}{\cbr{a + \frac d2 - F}\cbr{a - \frac d2 - F}} =  \\
    &= \frac {d^2F^3}{2\sqr{a + \frac d2 - F}\sqr{a - \frac d2 - F}} \cdot (2a - 2F) = \frac {d^2F^3(a - F)}{ \sqr{\sqr{a - F} - \frac{d^2}4} } = -\frac{1280000}{477603}.
    \end{align*}
}

\variantsplitter

\addpersonalvariant{Варвара Минаева}

\tasknumber{1}%
\task{%
    Запишите известные вам виды классификации изображений.
}
\solutionspace{60pt}

\tasknumber{2}%
\task{%
    В каких линзах можно получить увеличенное изображение объекта?
}
\answer{%
    $\text{ рассеивающие }$
}
\solutionspace{40pt}

\tasknumber{3}%
\task{%
    Какое изображение называют действительным?
}
\solutionspace{40pt}

\tasknumber{4}%
\task{%
    Есть две линзы, обозначим их 1 и 2.
    Известно что оптическая сила линзы 1 больше, чем у линзы 2.
    Какая линза сильнее преломляет лучи?
}
\answer{%
    $1$
}
\solutionspace{40pt}

\tasknumber{5}%
\task{%
    Предмет находится на расстоянии $20\,\text{см}$ от рассеивающей линзы с фокусным расстоянием $40\,\text{см}$.
    Определите тип изображения, расстояние между предметом и его изображением, увеличение предмета.
    Сделайте схематичный рисунок (не обязательно в масштабе, но с сохранением свойств линзы и изображения).
}
\solutionspace{100pt}

\tasknumber{6}%
\task{%
    Объект находится на расстоянии $25\,\text{см}$ от линзы, а его действительное изображение — в $50\,\text{см}$ от неё.
    Определите увеличение предмета, фокусное расстояние линзы, оптическую силу линзы и её тип.
}
\solutionspace{80pt}

\tasknumber{7}%
\task{%
    Известно, что из формулы тонкой линзы $\cbr{\frac 1F = \frac 1a + \frac 1b}$
    и определения увеличения $\cbr{\Gamma_y = \frac ba}$ можно получить выражение
    для увеличения: $\Gamma_y = \frac {aF}{a - F} \cdot \frac 1a = \frac {F}{a - F}.$
    Назовём такое увеличение «поперечным»: поперёк главной оптической оси (поэтому и ${}_y$).
    Получите формулу для «продольного» увеличения $\Gamma_x$ небольшого предмета, находящегося на главной оптической оси.
    Можно ли применить эту формулу для предмета, не лежащего на главной оптической оси, почему?
}
\answer{%
    \begin{align*}
    \frac 1F &= \frac 1a + \frac 1b \implies b = \frac {aF}{a - F} \\
    \frac 1F &= \frac 1{a + x} + \frac 1c \implies c = \frac {(a+x)F}{a + x - F} \\
    x' &= \abs{b - c} = \frac {aF}{a - F} - \frac {(a+x)F}{a + x - F} = F\cbr{\frac {a}{a - F} - \frac {a+x}{a + x - F}} =  \\
    &= F \cdot \frac {a^2 + ax - aF - a^2 - ax + aF + xF}{(a - F)(a + x - F)} = F \cdot \frac {xF}{(a - F)(a + x - F)} \\
    \Gamma_x &= \frac{x'}x = \frac{F^2}{(a - F)(a + x - F)} \to \frac{F^2}{\sqr{a - F}}.
    \\
    &\text{Нельзя: изображение по-разному растянет по осям $x$ и $y$ и понадобится теорема Пифагора}
    \end{align*}
}
\solutionspace{150pt}

\tasknumber{8}%
\task{%
    Доказать формулу тонкой линзы для собирающей линзы.
}
\solutionspace{120pt}

\tasknumber{9}%
\task{%
    Постройте ход луча $AK$ в тонкой линзе.
    Известно положение линзы и оба её фокуса (см.
    рис.
    на доске).
    Рассмотрите оба типа линзы, сделав 2 рисунка: собирающую и рассеивающую.
}
\solutionspace{120pt}

\tasknumber{10}%
\task{%
    На экране, расположенном иа расстоянии $80\,\text{см}$ от собирающей линзы,
    получено изображение точечного источника, расположенного на главной оптической оси линзы.
    На какое расстояние переместится изображение на экране,
    если при неподвижном источнике переместить линзу на $1\,\text{см}$ в плоскости, перпендикулярной главной оптической оси?
    Фокусное расстояние линзы равно $20\,\text{см}$.
}
\answer{%
    \begin{align*}
    &\frac 1F = \frac 1a + \frac 1b \implies a = \frac{bF}{b-F} \implies \Gamma = \frac ba = \frac{b-F}F \\
    &y = x \cdot \Gamma = x \cdot \frac{b-F}F \implies d = x + y = 4\,\text{см}.
    \end{align*}
}
\solutionspace{120pt}

\tasknumber{11}%
\task{%
    Оптическая сила двояковыпуклой линзы в воздухе $5\,\text{дптр}$, а в воде $1{,}5\,\text{дптр}$.
    Определить показатель преломления $n$ материала, из которого изготовлена линза.
    Показатель преломления воды равен $1{,}33$.
}
\answer{%
    \begin{align*}
    D_1 &=\cbr{\frac n{n_1} - 1}\cbr{\frac 1{R_1} + \frac 1{R_2}}, \\
    D_2 &=\cbr{\frac n{n_2} - 1}\cbr{\frac 1{R_1} + \frac 1{R_2}}, \\
    \frac {D_2}{D_1} &=\frac{\frac n{n_2} - 1}{\frac n{n_1} - 1} \implies {D_2}\cbr{\frac n{n_1} - 1} = {D_1}\cbr{\frac n{n_2} - 1}  \implies n\cbr{\frac{D_2}{n_1} - \frac{D_1}{n_2}} = D_2 - D_1, \\
    n &= \frac{D_2 - D_1}{\frac{D_2}{n_1} - \frac{D_1}{n_2}} = \frac{n_1 n_2 (D_2 - D_1)}{D_2n_2 - D_1n_1} \approx 1{,}549.
    \end{align*}
}
\solutionspace{120pt}

\tasknumber{12}%
\task{%
    На каком расстоянии от собирающей линзы с фокусным расстоянием $50\,\text{дптр}$
    следует надо поместить предмет, чтобы расстояние
    от предмета до его действительного изображения было наименьшим?
}
\answer{%
    \begin{align*}
    \frac 1a &+ \frac 1b = D \implies b = \frac 1{D - \frac 1a} \implies \ell = a + b = a + \frac a{Da - 1} = \frac{ Da^2 }{Da - 1} \implies \\
    \implies \ell'_a &= \frac{ 2Da \cdot (Da - 1) - Da^2 \cdot D }{\sqr{Da - 1}}= \frac{ D^2a^2 - 2Da}{\sqr{Da - 1}} = \frac{ Da(Da - 2)}{\sqr{Da - 1}}\implies a_{\min} = \frac 2D \approx 40\,\text{мм}.
    \end{align*}
}
\solutionspace{120pt}

\tasknumber{13}%
\task{%
    Даны точечный источник света $S$, его изображение $S_1$, полученное с помощью собирающей линзы,
    и ближайший к источнику фокус линзы $F$ (см.
    рис.
    на доске).
    Расстояния $SF = \ell$ и $SS_1 = L$.
    Определить положение линзы и её фокусное расстояние.
}
\answer{%
    \begin{align*}
    \frac 1a + \frac 1b &= \frac 1F, \ell = a - F, L = a + b \implies a = \ell + F, b = L - a = L - \ell - F \\
    \frac 1{\ell + F} + \frac 1{L - \ell - F} &= \frac 1F \\
    F\ell + F^2 + LF - F\ell - F^2 &= L\ell - \ell^2 - F\ell + LF - F\ell - F^2 \\
    0 &= L\ell - \ell^2 - 2F\ell - F^2 \\
    0 &=  F^2 + 2F\ell - L\ell + \ell^2 \\
    F &= -\ell \pm \sqrt{\ell^2 +  L\ell - \ell^2} = -\ell \pm \sqrt{L\ell} \implies F = \sqrt{L\ell} - \ell \\
    a &= \ell + F = \ell + \sqrt{L\ell} - \ell = \sqrt{L\ell}.
    \end{align*}
}
\solutionspace{120pt}

\tasknumber{14}%
\task{%
    Расстояние от освещённого предмета до экрана $100\,\text{см}$.
    Линза, помещенная между ними, даёт чёткое изображение предмета на
    экране при двух положениях, расстояние между которыми $30\,\text{см}$.
    Найти фокусное расстояние линзы.
}
\answer{%
    \begin{align*}
    \frac 1a + \frac 1b &= \frac 1F, \frac 1{a-\ell} + \frac 1{b+\ell} = \frac 1F, a + b = L \\
    \frac 1a + \frac 1b &= \frac 1{a-\ell} + \frac 1{b+\ell}\implies \frac{a + b}{ab} = \frac{(a-\ell) + (b+\ell)}{(a-\ell)(b+\ell)} \\
    ab  &= (a - \ell)(b+\ell) \implies 0  = -b\ell + a\ell - \ell^2 \implies 0 = -b + a - \ell \implies b = a - \ell \\
    a + (a - \ell) &= L \implies a = \frac{L + \ell}2 \implies b = \frac{L - \ell}2 \\
    F &= \frac{ab}{a + b} = \frac{L^2 -\ell^2}{4L} \approx 22{,}8\,\text{см}.
    \end{align*}
}
\solutionspace{120pt}

\tasknumber{15}%
\task{%
    Предмет находится на расстоянии $90\,\text{см}$ от экрана.
    Между предметом и экраном помещают линзу, причём при одном
    положении линзы на экране получается увеличенное изображение предмета,
    а при другом — уменьшенное.
    Каково фокусное расстояние линзы, если
    линейные размеры первого изображения в три раза больше второго?
}
\answer{%
    \begin{align*}
    \frac 1a + \frac 1{L-a} &= \frac 1F, h_1 = h \cdot \frac{L-a}a, \\
    \frac 1b + \frac 1{L-b} &= \frac 1F, h_2 = h \cdot \frac{L-b}b, \\
    \frac{h_1}{h_2} &= 3 \implies \frac{(L-a)b}{(L-b)a} = 3, \\
    \frac 1F &= \frac{ L }{a(L-a)} = \frac{ L }{b(L-b)} \implies \frac{L-a}{L-b} = \frac b a \implies \frac {b^2}{a^2} = 3.
    \\
    \frac 1a + \frac 1{L-a} &= \frac 1b + \frac 1{L-b} \implies \frac L{a(L-a)} = \frac L{b(L-b)} \implies \\
    \implies aL - a^2 &= bL - b^2 \implies (a-b)L = (a-b)(a+b) \implies b = L - a, \\
    \frac{\sqr{L-a}}{a^2} &= 3 \implies \frac La - 1 = \sqrt{3} \implies a = \frac{ L }{\sqrt{3} + 1} \\
    F &= \frac{a(L-a)}L = \frac 1L \cdot \frac L{\sqrt{3} + 1} \cdot \frac {L\sqrt{3}}{\sqrt{3} + 1}= \frac { L\sqrt{3} }{ \sqr{\sqrt{3} + 1} } \approx 21\,\text{см}.
    \end{align*}
}
\solutionspace{120pt}

\tasknumber{16}%
\task{%
    (Задача-«гроб»: решать на обратной стороне) Квадрат со стороной $d = 3\,\text{см}$ расположен так,
    что 2 его стороны параллельны главной оптической оси собирающей линзы,
    его центр удален на $h = 4\,\text{см}$ от этой оси и на $a = 10\,\text{см}$ от плоскости линзы.
    Определите площадь изображения квадрата, если фокусное расстояние линзы составляет $F = 20\,\text{см}$.
    % (и сравните с площадью объекта, умноженной на квадрат увеличения центра квадрата).
}
\answer{%
    \begin{align*}
    &\text{Все явные вычисления — в см и $\text{см}^2$,} \\
    \frac 1 F &= \frac 1{a + \frac d2} + \frac 1b \implies b = \frac 1{\frac 1 F - \frac 1{a + \frac d2}} = \frac{F(a + \frac d2)}{a + \frac d2 - F} = -\frac{460}{17}, \\
    \frac 1 F &= \frac 1{a - \frac d2} + \frac 1c \implies c = \frac 1{\frac 1 F - \frac 1{a - \frac d2}} = \frac{F(a - \frac d2)}{a - \frac d2 - F} = -\frac{340}{23}, \\
    c - b &= \frac{F(a - \frac d2)}{a - \frac d2 - F} - \frac{F(a + \frac d2)}{a + \frac d2 - F} = F\cbr{ \frac{a - \frac d2}{a - \frac d2 - F} - \frac{a + \frac d2}{a + \frac d2 - F} } =  \\
    &= F \cdot \frac{a^2 + \frac {ad}2 - aF - \frac{ad}2 - \frac{d^2}4 + \frac{dF}2 - a^2 + \frac {ad}2 + aF - \frac{ad}2 + \frac{d^2}4 + \frac{dF}2}{\cbr{a + \frac d2 - F}\cbr{a - \frac d2 - F}}= F \cdot \frac {dF}{\cbr{a + \frac d2 - F}\cbr{a - \frac d2 - F}} = \frac{4800}{391}.
    \\
    \Gamma_b &= \frac b{a + \frac d2} = \frac{ F }{a + \frac d2 - F} = -\frac{40}{17}, \\
    \Gamma_c &= \frac c{a - \frac d2} = \frac{ F }{a - \frac d2 - F} = -\frac{40}{23}, \\
    &\text{ тут интересно отметить, что } \Gamma_x = \frac{ c - b}{ d } = \frac{ F^2 }{\cbr{a + \frac d2 - F}\cbr{a - \frac d2 - F}} \ne \Gamma_b \text{ или } \Gamma_c \text{ даже при малых $d$}.
    \\
    S' &= \frac{d \cdot \Gamma_b + d \cdot \Gamma_c}2 \cdot (c - b) = \frac d2 \cbr{\frac{ F }{a + \frac d2 - F} + \frac{ F }{a - \frac d2 - F}} \cdot \cbr{c - b} =  \\
    &=\frac {dF}2 \cbr{\frac 1{a + \frac d2 - F} + \frac 1{a - \frac d2 - F}} \cdot \frac {dF^2}{\cbr{a + \frac d2 - F}\cbr{a - \frac d2 - F}} =  \\
    &=\frac {dF}2 \cdot \frac{a - \frac d2 - F + a + \frac d2 - F}{\cbr{a + \frac d2 - F}\cbr{a - \frac d2 - F}} \cdot \frac {dF^2}{\cbr{a + \frac d2 - F}\cbr{a - \frac d2 - F}} =  \\
    &= \frac {d^2F^3}{2\sqr{a + \frac d2 - F}\sqr{a - \frac d2 - F}} \cdot (2a - 2F) = \frac {d^2F^3(a - F)}{ \sqr{\sqr{a - F} - \frac{d^2}4} } = -\frac{11520000}{152881}.
    \end{align*}
}

\variantsplitter

\addpersonalvariant{Леонид Никитин}

\tasknumber{1}%
\task{%
    Запишите известные вам виды классификации изображений.
}
\solutionspace{60pt}

\tasknumber{2}%
\task{%
    В каких линзах можно получить мнимое изображение объекта?
}
\answer{%
    $\text{ собирающие и рассеивающие }$
}
\solutionspace{40pt}

\tasknumber{3}%
\task{%
    Какое изображение называют мнимым?
}
\solutionspace{40pt}

\tasknumber{4}%
\task{%
    Есть две линзы, обозначим их 1 и 2.
    Известно что оптическая сила линзы 2 меньше, чем у линзы 1.
    Какая линза сильнее преломляет лучи?
}
\answer{%
    $1$
}
\solutionspace{40pt}

\tasknumber{5}%
\task{%
    Предмет находится на расстоянии $20\,\text{см}$ от рассеивающей линзы с фокусным расстоянием $6\,\text{см}$.
    Определите тип изображения, расстояние между предметом и его изображением, увеличение предмета.
    Сделайте схематичный рисунок (не обязательно в масштабе, но с сохранением свойств линзы и изображения).
}
\solutionspace{100pt}

\tasknumber{6}%
\task{%
    Объект находится на расстоянии $45\,\text{см}$ от линзы, а его действительное изображение — в $40\,\text{см}$ от неё.
    Определите увеличение предмета, фокусное расстояние линзы, оптическую силу линзы и её тип.
}
\solutionspace{80pt}

\tasknumber{7}%
\task{%
    Известно, что из формулы тонкой линзы $\cbr{\frac 1F = \frac 1a + \frac 1b}$
    и определения увеличения $\cbr{\Gamma_y = \frac ba}$ можно получить выражение
    для увеличения: $\Gamma_y = \frac {aF}{a - F} \cdot \frac 1a = \frac {F}{a - F}.$
    Назовём такое увеличение «поперечным»: поперёк главной оптической оси (поэтому и ${}_y$).
    Получите формулу для «продольного» увеличения $\Gamma_x$ небольшого предмета, находящегося на главной оптической оси.
    Можно ли применить эту формулу для предмета, не лежащего на главной оптической оси, почему?
}
\answer{%
    \begin{align*}
    \frac 1F &= \frac 1a + \frac 1b \implies b = \frac {aF}{a - F} \\
    \frac 1F &= \frac 1{a + x} + \frac 1c \implies c = \frac {(a+x)F}{a + x - F} \\
    x' &= \abs{b - c} = \frac {aF}{a - F} - \frac {(a+x)F}{a + x - F} = F\cbr{\frac {a}{a - F} - \frac {a+x}{a + x - F}} =  \\
    &= F \cdot \frac {a^2 + ax - aF - a^2 - ax + aF + xF}{(a - F)(a + x - F)} = F \cdot \frac {xF}{(a - F)(a + x - F)} \\
    \Gamma_x &= \frac{x'}x = \frac{F^2}{(a - F)(a + x - F)} \to \frac{F^2}{\sqr{a - F}}.
    \\
    &\text{Нельзя: изображение по-разному растянет по осям $x$ и $y$ и понадобится теорема Пифагора}
    \end{align*}
}
\solutionspace{150pt}

\tasknumber{8}%
\task{%
    Доказать формулу тонкой линзы для рассеивающей линзы.
}
\solutionspace{120pt}

\tasknumber{9}%
\task{%
    Постройте ход луча $BK$ в тонкой линзе.
    Известно положение линзы и оба её фокуса (см.
    рис.
    на доске).
    Рассмотрите оба типа линзы, сделав 2 рисунка: собирающую и рассеивающую.
}
\solutionspace{120pt}

\tasknumber{10}%
\task{%
    На экране, расположенном иа расстоянии $80\,\text{см}$ от собирающей линзы,
    получено изображение точечного источника, расположенного на главной оптической оси линзы.
    На какое расстояние переместится изображение на экране,
    если при неподвижной линзе переместить источник на $1\,\text{см}$ в плоскости, перпендикулярной главной оптической оси?
    Фокусное расстояние линзы равно $20\,\text{см}$.
}
\answer{%
    \begin{align*}
    &\frac 1F = \frac 1a + \frac 1b \implies a = \frac{bF}{b-F} \implies \Gamma = \frac ba = \frac{b-F}F \\
    &y = x \cdot \Gamma = x \cdot \frac{b-F}F \implies d = y = 3\,\text{см}.
    \end{align*}
}
\solutionspace{120pt}

\tasknumber{11}%
\task{%
    Оптическая сила двояковыпуклой линзы в воздухе $5{,}5\,\text{дптр}$, а в воде $1{,}6\,\text{дптр}$.
    Определить показатель преломления $n$ материала, из которого изготовлена линза.
    Показатель преломления воды равен $1{,}33$.
}
\answer{%
    \begin{align*}
    D_1 &=\cbr{\frac n{n_1} - 1}\cbr{\frac 1{R_1} + \frac 1{R_2}}, \\
    D_2 &=\cbr{\frac n{n_2} - 1}\cbr{\frac 1{R_1} + \frac 1{R_2}}, \\
    \frac {D_2}{D_1} &=\frac{\frac n{n_2} - 1}{\frac n{n_1} - 1} \implies {D_2}\cbr{\frac n{n_1} - 1} = {D_1}\cbr{\frac n{n_2} - 1}  \implies n\cbr{\frac{D_2}{n_1} - \frac{D_1}{n_2}} = D_2 - D_1, \\
    n &= \frac{D_2 - D_1}{\frac{D_2}{n_1} - \frac{D_1}{n_2}} = \frac{n_1 n_2 (D_2 - D_1)}{D_2n_2 - D_1n_1} \approx 1{,}538.
    \end{align*}
}
\solutionspace{120pt}

\tasknumber{12}%
\task{%
    На каком расстоянии от собирающей линзы с фокусным расстоянием $50\,\text{дптр}$
    следует надо поместить предмет, чтобы расстояние
    от предмета до его действительного изображения было наименьшим?
}
\answer{%
    \begin{align*}
    \frac 1a &+ \frac 1b = D \implies b = \frac 1{D - \frac 1a} \implies \ell = a + b = a + \frac a{Da - 1} = \frac{ Da^2 }{Da - 1} \implies \\
    \implies \ell'_a &= \frac{ 2Da \cdot (Da - 1) - Da^2 \cdot D }{\sqr{Da - 1}}= \frac{ D^2a^2 - 2Da}{\sqr{Da - 1}} = \frac{ Da(Da - 2)}{\sqr{Da - 1}}\implies a_{\min} = \frac 2D \approx 40\,\text{мм}.
    \end{align*}
}
\solutionspace{120pt}

\tasknumber{13}%
\task{%
    Даны точечный источник света $S$, его изображение $S_1$, полученное с помощью собирающей линзы,
    и ближайший к источнику фокус линзы $F$ (см.
    рис.
    на доске).
    Расстояния $SF = \ell$ и $SS_1 = L$.
    Определить положение линзы и её фокусное расстояние.
}
\answer{%
    \begin{align*}
    \frac 1a + \frac 1b &= \frac 1F, \ell = a - F, L = a + b \implies a = \ell + F, b = L - a = L - \ell - F \\
    \frac 1{\ell + F} + \frac 1{L - \ell - F} &= \frac 1F \\
    F\ell + F^2 + LF - F\ell - F^2 &= L\ell - \ell^2 - F\ell + LF - F\ell - F^2 \\
    0 &= L\ell - \ell^2 - 2F\ell - F^2 \\
    0 &=  F^2 + 2F\ell - L\ell + \ell^2 \\
    F &= -\ell \pm \sqrt{\ell^2 +  L\ell - \ell^2} = -\ell \pm \sqrt{L\ell} \implies F = \sqrt{L\ell} - \ell \\
    a &= \ell + F = \ell + \sqrt{L\ell} - \ell = \sqrt{L\ell}.
    \end{align*}
}
\solutionspace{120pt}

\tasknumber{14}%
\task{%
    Расстояние от освещённого предмета до экрана $80\,\text{см}$.
    Линза, помещенная между ними, даёт чёткое изображение предмета на
    экране при двух положениях, расстояние между которыми $20\,\text{см}$.
    Найти фокусное расстояние линзы.
}
\answer{%
    \begin{align*}
    \frac 1a + \frac 1b &= \frac 1F, \frac 1{a-\ell} + \frac 1{b+\ell} = \frac 1F, a + b = L \\
    \frac 1a + \frac 1b &= \frac 1{a-\ell} + \frac 1{b+\ell}\implies \frac{a + b}{ab} = \frac{(a-\ell) + (b+\ell)}{(a-\ell)(b+\ell)} \\
    ab  &= (a - \ell)(b+\ell) \implies 0  = -b\ell + a\ell - \ell^2 \implies 0 = -b + a - \ell \implies b = a - \ell \\
    a + (a - \ell) &= L \implies a = \frac{L + \ell}2 \implies b = \frac{L - \ell}2 \\
    F &= \frac{ab}{a + b} = \frac{L^2 -\ell^2}{4L} \approx 18{,}8\,\text{см}.
    \end{align*}
}
\solutionspace{120pt}

\tasknumber{15}%
\task{%
    Предмет находится на расстоянии $60\,\text{см}$ от экрана.
    Между предметом и экраном помещают линзу, причём при одном
    положении линзы на экране получается увеличенное изображение предмета,
    а при другом — уменьшенное.
    Каково фокусное расстояние линзы, если
    линейные размеры первого изображения в три раза больше второго?
}
\answer{%
    \begin{align*}
    \frac 1a + \frac 1{L-a} &= \frac 1F, h_1 = h \cdot \frac{L-a}a, \\
    \frac 1b + \frac 1{L-b} &= \frac 1F, h_2 = h \cdot \frac{L-b}b, \\
    \frac{h_1}{h_2} &= 3 \implies \frac{(L-a)b}{(L-b)a} = 3, \\
    \frac 1F &= \frac{ L }{a(L-a)} = \frac{ L }{b(L-b)} \implies \frac{L-a}{L-b} = \frac b a \implies \frac {b^2}{a^2} = 3.
    \\
    \frac 1a + \frac 1{L-a} &= \frac 1b + \frac 1{L-b} \implies \frac L{a(L-a)} = \frac L{b(L-b)} \implies \\
    \implies aL - a^2 &= bL - b^2 \implies (a-b)L = (a-b)(a+b) \implies b = L - a, \\
    \frac{\sqr{L-a}}{a^2} &= 3 \implies \frac La - 1 = \sqrt{3} \implies a = \frac{ L }{\sqrt{3} + 1} \\
    F &= \frac{a(L-a)}L = \frac 1L \cdot \frac L{\sqrt{3} + 1} \cdot \frac {L\sqrt{3}}{\sqrt{3} + 1}= \frac { L\sqrt{3} }{ \sqr{\sqrt{3} + 1} } \approx 13{,}9\,\text{см}.
    \end{align*}
}
\solutionspace{120pt}

\tasknumber{16}%
\task{%
    (Задача-«гроб»: решать на обратной стороне) Квадрат со стороной $d = 3\,\text{см}$ расположен так,
    что 2 его стороны параллельны главной оптической оси рассеивающей линзы,
    его центр удален на $h = 5\,\text{см}$ от этой оси и на $a = 15\,\text{см}$ от плоскости линзы.
    Определите площадь изображения квадрата, если фокусное расстояние линзы составляет $F = 25\,\text{см}$.
    % (и сравните с площадью объекта, умноженной на квадрат увеличения центра квадрата).
}
\answer{%
    \begin{align*}
    &\text{Все явные вычисления — в см и $\text{см}^2$,} \\
    \frac 1 F &= \frac 1{a + \frac d2} + \frac 1b \implies b = \frac 1{\frac 1 F - \frac 1{a + \frac d2}} = \frac{F(a + \frac d2)}{a + \frac d2 - F} = -\frac{825}{83}, \\
    \frac 1 F &= \frac 1{a - \frac d2} + \frac 1c \implies c = \frac 1{\frac 1 F - \frac 1{a - \frac d2}} = \frac{F(a - \frac d2)}{a - \frac d2 - F} = -\frac{675}{77}, \\
    c - b &= \frac{F(a - \frac d2)}{a - \frac d2 - F} - \frac{F(a + \frac d2)}{a + \frac d2 - F} = F\cbr{ \frac{a - \frac d2}{a - \frac d2 - F} - \frac{a + \frac d2}{a + \frac d2 - F} } =  \\
    &= F \cdot \frac{a^2 + \frac {ad}2 - aF - \frac{ad}2 - \frac{d^2}4 + \frac{dF}2 - a^2 + \frac {ad}2 + aF - \frac{ad}2 + \frac{d^2}4 + \frac{dF}2}{\cbr{a + \frac d2 - F}\cbr{a - \frac d2 - F}}= F \cdot \frac {dF}{\cbr{a + \frac d2 - F}\cbr{a - \frac d2 - F}} = \frac{7500}{6391}.
    \\
    \Gamma_b &= \frac b{a + \frac d2} = \frac{ F }{a + \frac d2 - F} = -\frac{50}{83}, \\
    \Gamma_c &= \frac c{a - \frac d2} = \frac{ F }{a - \frac d2 - F} = -\frac{50}{77}, \\
    &\text{ тут интересно отметить, что } \Gamma_x = \frac{ c - b}{ d } = \frac{ F^2 }{\cbr{a + \frac d2 - F}\cbr{a - \frac d2 - F}} \ne \Gamma_b \text{ или } \Gamma_c \text{ даже при малых $d$}.
    \\
    S' &= \frac{d \cdot \Gamma_b + d \cdot \Gamma_c}2 \cdot (c - b) = \frac d2 \cbr{\frac{ F }{a + \frac d2 - F} + \frac{ F }{a - \frac d2 - F}} \cdot \cbr{c - b} =  \\
    &=\frac {dF}2 \cbr{\frac 1{a + \frac d2 - F} + \frac 1{a - \frac d2 - F}} \cdot \frac {dF^2}{\cbr{a + \frac d2 - F}\cbr{a - \frac d2 - F}} =  \\
    &=\frac {dF}2 \cdot \frac{a - \frac d2 - F + a + \frac d2 - F}{\cbr{a + \frac d2 - F}\cbr{a - \frac d2 - F}} \cdot \frac {dF^2}{\cbr{a + \frac d2 - F}\cbr{a - \frac d2 - F}} =  \\
    &= \frac {d^2F^3}{2\sqr{a + \frac d2 - F}\sqr{a - \frac d2 - F}} \cdot (2a - 2F) = \frac {d^2F^3(a - F)}{ \sqr{\sqr{a - F} - \frac{d^2}4} } = -\frac{90000000}{40844881}.
    \end{align*}
}

\variantsplitter

\addpersonalvariant{Тимофей Полетаев}

\tasknumber{1}%
\task{%
    Запишите формулу тонкой линзы и сделайте рисунок, указав на нём физические величины из этой формулы.
}
\solutionspace{60pt}

\tasknumber{2}%
\task{%
    В каких линзах можно получить уменьшенное изображение объекта?
}
\answer{%
    $\text{ собирающие и рассеивающие }$
}
\solutionspace{40pt}

\tasknumber{3}%
\task{%
    Какое изображение называют мнимым?
}
\solutionspace{40pt}

\tasknumber{4}%
\task{%
    Есть две линзы, обозначим их 1 и 2.
    Известно что оптическая сила линзы 1 меньше, чем у линзы 2.
    Какая линза сильнее преломляет лучи?
}
\answer{%
    $2$
}
\solutionspace{40pt}

\tasknumber{5}%
\task{%
    Предмет находится на расстоянии $20\,\text{см}$ от рассеивающей линзы с фокусным расстоянием $50\,\text{см}$.
    Определите тип изображения, расстояние между предметом и его изображением, увеличение предмета.
    Сделайте схематичный рисунок (не обязательно в масштабе, но с сохранением свойств линзы и изображения).
}
\solutionspace{100pt}

\tasknumber{6}%
\task{%
    Объект находится на расстоянии $115\,\text{см}$ от линзы, а его действительное изображение — в $50\,\text{см}$ от неё.
    Определите увеличение предмета, фокусное расстояние линзы, оптическую силу линзы и её тип.
}
\solutionspace{80pt}

\tasknumber{7}%
\task{%
    Известно, что из формулы тонкой линзы $\cbr{\frac 1F = \frac 1a + \frac 1b}$
    и определения увеличения $\cbr{\Gamma_y = \frac ba}$ можно получить выражение
    для увеличения: $\Gamma_y = \frac {aF}{a - F} \cdot \frac 1a = \frac {F}{a - F}.$
    Назовём такое увеличение «поперечным»: поперёк главной оптической оси (поэтому и ${}_y$).
    Получите формулу для «продольного» увеличения $\Gamma_x$ небольшого предмета, находящегося на главной оптической оси.
    Можно ли применить эту формулу для предмета, не лежащего на главной оптической оси, почему?
}
\answer{%
    \begin{align*}
    \frac 1F &= \frac 1a + \frac 1b \implies b = \frac {aF}{a - F} \\
    \frac 1F &= \frac 1{a + x} + \frac 1c \implies c = \frac {(a+x)F}{a + x - F} \\
    x' &= \abs{b - c} = \frac {aF}{a - F} - \frac {(a+x)F}{a + x - F} = F\cbr{\frac {a}{a - F} - \frac {a+x}{a + x - F}} =  \\
    &= F \cdot \frac {a^2 + ax - aF - a^2 - ax + aF + xF}{(a - F)(a + x - F)} = F \cdot \frac {xF}{(a - F)(a + x - F)} \\
    \Gamma_x &= \frac{x'}x = \frac{F^2}{(a - F)(a + x - F)} \to \frac{F^2}{\sqr{a - F}}.
    \\
    &\text{Нельзя: изображение по-разному растянет по осям $x$ и $y$ и понадобится теорема Пифагора}
    \end{align*}
}
\solutionspace{150pt}

\tasknumber{8}%
\task{%
    Доказать формулу тонкой линзы для рассеивающей линзы.
}
\solutionspace{120pt}

\tasknumber{9}%
\task{%
    Постройте ход луча $AL$ в тонкой линзе.
    Известно положение линзы и оба её фокуса (см.
    рис.
    на доске).
    Рассмотрите оба типа линзы, сделав 2 рисунка: собирающую и рассеивающую.
}
\solutionspace{120pt}

\tasknumber{10}%
\task{%
    На экране, расположенном иа расстоянии $80\,\text{см}$ от собирающей линзы,
    получено изображение точечного источника, расположенного на главной оптической оси линзы.
    На какое расстояние переместится изображение на экране,
    если при неподвижной линзе переместить источник на $1\,\text{см}$ в плоскости, перпендикулярной главной оптической оси?
    Фокусное расстояние линзы равно $30\,\text{см}$.
}
\answer{%
    \begin{align*}
    &\frac 1F = \frac 1a + \frac 1b \implies a = \frac{bF}{b-F} \implies \Gamma = \frac ba = \frac{b-F}F \\
    &y = x \cdot \Gamma = x \cdot \frac{b-F}F \implies d = y = 1{,}67\,\text{см}.
    \end{align*}
}
\solutionspace{120pt}

\tasknumber{11}%
\task{%
    Оптическая сила двояковыпуклой линзы в воздухе $5\,\text{дптр}$, а в воде $1{,}5\,\text{дптр}$.
    Определить показатель преломления $n$ материала, из которого изготовлена линза.
    Показатель преломления воды равен $1{,}33$.
}
\answer{%
    \begin{align*}
    D_1 &=\cbr{\frac n{n_1} - 1}\cbr{\frac 1{R_1} + \frac 1{R_2}}, \\
    D_2 &=\cbr{\frac n{n_2} - 1}\cbr{\frac 1{R_1} + \frac 1{R_2}}, \\
    \frac {D_2}{D_1} &=\frac{\frac n{n_2} - 1}{\frac n{n_1} - 1} \implies {D_2}\cbr{\frac n{n_1} - 1} = {D_1}\cbr{\frac n{n_2} - 1}  \implies n\cbr{\frac{D_2}{n_1} - \frac{D_1}{n_2}} = D_2 - D_1, \\
    n &= \frac{D_2 - D_1}{\frac{D_2}{n_1} - \frac{D_1}{n_2}} = \frac{n_1 n_2 (D_2 - D_1)}{D_2n_2 - D_1n_1} \approx 1{,}549.
    \end{align*}
}
\solutionspace{120pt}

\tasknumber{12}%
\task{%
    На каком расстоянии от собирающей линзы с фокусным расстоянием $40\,\text{дптр}$
    следует надо поместить предмет, чтобы расстояние
    от предмета до его действительного изображения было наименьшим?
}
\answer{%
    \begin{align*}
    \frac 1a &+ \frac 1b = D \implies b = \frac 1{D - \frac 1a} \implies \ell = a + b = a + \frac a{Da - 1} = \frac{ Da^2 }{Da - 1} \implies \\
    \implies \ell'_a &= \frac{ 2Da \cdot (Da - 1) - Da^2 \cdot D }{\sqr{Da - 1}}= \frac{ D^2a^2 - 2Da}{\sqr{Da - 1}} = \frac{ Da(Da - 2)}{\sqr{Da - 1}}\implies a_{\min} = \frac 2D \approx 50\,\text{мм}.
    \end{align*}
}
\solutionspace{120pt}

\tasknumber{13}%
\task{%
    Даны точечный источник света $S$, его изображение $S_1$, полученное с помощью собирающей линзы,
    и ближайший к источнику фокус линзы $F$ (см.
    рис.
    на доске).
    Расстояния $SF = \ell$ и $SS_1 = L$.
    Определить положение линзы и её фокусное расстояние.
}
\answer{%
    \begin{align*}
    \frac 1a + \frac 1b &= \frac 1F, \ell = a - F, L = a + b \implies a = \ell + F, b = L - a = L - \ell - F \\
    \frac 1{\ell + F} + \frac 1{L - \ell - F} &= \frac 1F \\
    F\ell + F^2 + LF - F\ell - F^2 &= L\ell - \ell^2 - F\ell + LF - F\ell - F^2 \\
    0 &= L\ell - \ell^2 - 2F\ell - F^2 \\
    0 &=  F^2 + 2F\ell - L\ell + \ell^2 \\
    F &= -\ell \pm \sqrt{\ell^2 +  L\ell - \ell^2} = -\ell \pm \sqrt{L\ell} \implies F = \sqrt{L\ell} - \ell \\
    a &= \ell + F = \ell + \sqrt{L\ell} - \ell = \sqrt{L\ell}.
    \end{align*}
}
\solutionspace{120pt}

\tasknumber{14}%
\task{%
    Расстояние от освещённого предмета до экрана $100\,\text{см}$.
    Линза, помещенная между ними, даёт чёткое изображение предмета на
    экране при двух положениях, расстояние между которыми $20\,\text{см}$.
    Найти фокусное расстояние линзы.
}
\answer{%
    \begin{align*}
    \frac 1a + \frac 1b &= \frac 1F, \frac 1{a-\ell} + \frac 1{b+\ell} = \frac 1F, a + b = L \\
    \frac 1a + \frac 1b &= \frac 1{a-\ell} + \frac 1{b+\ell}\implies \frac{a + b}{ab} = \frac{(a-\ell) + (b+\ell)}{(a-\ell)(b+\ell)} \\
    ab  &= (a - \ell)(b+\ell) \implies 0  = -b\ell + a\ell - \ell^2 \implies 0 = -b + a - \ell \implies b = a - \ell \\
    a + (a - \ell) &= L \implies a = \frac{L + \ell}2 \implies b = \frac{L - \ell}2 \\
    F &= \frac{ab}{a + b} = \frac{L^2 -\ell^2}{4L} \approx 24\,\text{см}.
    \end{align*}
}
\solutionspace{120pt}

\tasknumber{15}%
\task{%
    Предмет находится на расстоянии $70\,\text{см}$ от экрана.
    Между предметом и экраном помещают линзу, причём при одном
    положении линзы на экране получается увеличенное изображение предмета,
    а при другом — уменьшенное.
    Каково фокусное расстояние линзы, если
    линейные размеры первого изображения в два раза больше второго?
}
\answer{%
    \begin{align*}
    \frac 1a + \frac 1{L-a} &= \frac 1F, h_1 = h \cdot \frac{L-a}a, \\
    \frac 1b + \frac 1{L-b} &= \frac 1F, h_2 = h \cdot \frac{L-b}b, \\
    \frac{h_1}{h_2} &= 2 \implies \frac{(L-a)b}{(L-b)a} = 2, \\
    \frac 1F &= \frac{ L }{a(L-a)} = \frac{ L }{b(L-b)} \implies \frac{L-a}{L-b} = \frac b a \implies \frac {b^2}{a^2} = 2.
    \\
    \frac 1a + \frac 1{L-a} &= \frac 1b + \frac 1{L-b} \implies \frac L{a(L-a)} = \frac L{b(L-b)} \implies \\
    \implies aL - a^2 &= bL - b^2 \implies (a-b)L = (a-b)(a+b) \implies b = L - a, \\
    \frac{\sqr{L-a}}{a^2} &= 2 \implies \frac La - 1 = \sqrt{2} \implies a = \frac{ L }{\sqrt{2} + 1} \\
    F &= \frac{a(L-a)}L = \frac 1L \cdot \frac L{\sqrt{2} + 1} \cdot \frac {L\sqrt{2}}{\sqrt{2} + 1}= \frac { L\sqrt{2} }{ \sqr{\sqrt{2} + 1} } \approx 17{,}0\,\text{см}.
    \end{align*}
}
\solutionspace{120pt}

\tasknumber{16}%
\task{%
    (Задача-«гроб»: решать на обратной стороне) Квадрат со стороной $d = 2\,\text{см}$ расположен так,
    что 2 его стороны параллельны главной оптической оси рассеивающей линзы,
    его центр удален на $h = 5\,\text{см}$ от этой оси и на $a = 12\,\text{см}$ от плоскости линзы.
    Определите площадь изображения квадрата, если фокусное расстояние линзы составляет $F = 18\,\text{см}$.
    % (и сравните с площадью объекта, умноженной на квадрат увеличения центра квадрата).
}
\answer{%
    \begin{align*}
    &\text{Все явные вычисления — в см и $\text{см}^2$,} \\
    \frac 1 F &= \frac 1{a + \frac d2} + \frac 1b \implies b = \frac 1{\frac 1 F - \frac 1{a + \frac d2}} = \frac{F(a + \frac d2)}{a + \frac d2 - F} = -\frac{234}{31}, \\
    \frac 1 F &= \frac 1{a - \frac d2} + \frac 1c \implies c = \frac 1{\frac 1 F - \frac 1{a - \frac d2}} = \frac{F(a - \frac d2)}{a - \frac d2 - F} = -\frac{198}{29}, \\
    c - b &= \frac{F(a - \frac d2)}{a - \frac d2 - F} - \frac{F(a + \frac d2)}{a + \frac d2 - F} = F\cbr{ \frac{a - \frac d2}{a - \frac d2 - F} - \frac{a + \frac d2}{a + \frac d2 - F} } =  \\
    &= F \cdot \frac{a^2 + \frac {ad}2 - aF - \frac{ad}2 - \frac{d^2}4 + \frac{dF}2 - a^2 + \frac {ad}2 + aF - \frac{ad}2 + \frac{d^2}4 + \frac{dF}2}{\cbr{a + \frac d2 - F}\cbr{a - \frac d2 - F}}= F \cdot \frac {dF}{\cbr{a + \frac d2 - F}\cbr{a - \frac d2 - F}} = \frac{648}{899}.
    \\
    \Gamma_b &= \frac b{a + \frac d2} = \frac{ F }{a + \frac d2 - F} = -\frac{18}{31}, \\
    \Gamma_c &= \frac c{a - \frac d2} = \frac{ F }{a - \frac d2 - F} = -\frac{18}{29}, \\
    &\text{ тут интересно отметить, что } \Gamma_x = \frac{ c - b}{ d } = \frac{ F^2 }{\cbr{a + \frac d2 - F}\cbr{a - \frac d2 - F}} \ne \Gamma_b \text{ или } \Gamma_c \text{ даже при малых $d$}.
    \\
    S' &= \frac{d \cdot \Gamma_b + d \cdot \Gamma_c}2 \cdot (c - b) = \frac d2 \cbr{\frac{ F }{a + \frac d2 - F} + \frac{ F }{a - \frac d2 - F}} \cdot \cbr{c - b} =  \\
    &=\frac {dF}2 \cbr{\frac 1{a + \frac d2 - F} + \frac 1{a - \frac d2 - F}} \cdot \frac {dF^2}{\cbr{a + \frac d2 - F}\cbr{a - \frac d2 - F}} =  \\
    &=\frac {dF}2 \cdot \frac{a - \frac d2 - F + a + \frac d2 - F}{\cbr{a + \frac d2 - F}\cbr{a - \frac d2 - F}} \cdot \frac {dF^2}{\cbr{a + \frac d2 - F}\cbr{a - \frac d2 - F}} =  \\
    &= \frac {d^2F^3}{2\sqr{a + \frac d2 - F}\sqr{a - \frac d2 - F}} \cdot (2a - 2F) = \frac {d^2F^3(a - F)}{ \sqr{\sqr{a - F} - \frac{d^2}4} } = -\frac{699840}{808201}.
    \end{align*}
}

\variantsplitter

\addpersonalvariant{Андрей Рожков}

\tasknumber{1}%
\task{%
    Запишите известные вам виды классификации изображений.
}
\solutionspace{60pt}

\tasknumber{2}%
\task{%
    В каких линзах можно получить обратное изображение объекта?
}
\answer{%
    $\text{ собирающие }$
}
\solutionspace{40pt}

\tasknumber{3}%
\task{%
    Какое изображение называют действительным?
}
\solutionspace{40pt}

\tasknumber{4}%
\task{%
    Есть две линзы, обозначим их 1 и 2.
    Известно что оптическая сила линзы 1 больше, чем у линзы 2.
    Какая линза сильнее преломляет лучи?
}
\answer{%
    $1$
}
\solutionspace{40pt}

\tasknumber{5}%
\task{%
    Предмет находится на расстоянии $30\,\text{см}$ от собирающей линзы с фокусным расстоянием $25\,\text{см}$.
    Определите тип изображения, расстояние между предметом и его изображением, увеличение предмета.
    Сделайте схематичный рисунок (не обязательно в масштабе, но с сохранением свойств линзы и изображения).
}
\solutionspace{100pt}

\tasknumber{6}%
\task{%
    Объект находится на расстоянии $45\,\text{см}$ от линзы, а его действительное изображение — в $50\,\text{см}$ от неё.
    Определите увеличение предмета, фокусное расстояние линзы, оптическую силу линзы и её тип.
}
\solutionspace{80pt}

\tasknumber{7}%
\task{%
    Известно, что из формулы тонкой линзы $\cbr{\frac 1F = \frac 1a + \frac 1b}$
    и определения увеличения $\cbr{\Gamma_y = \frac ba}$ можно получить выражение
    для увеличения: $\Gamma_y = \frac {aF}{a - F} \cdot \frac 1a = \frac {F}{a - F}.$
    Назовём такое увеличение «поперечным»: поперёк главной оптической оси (поэтому и ${}_y$).
    Получите формулу для «продольного» увеличения $\Gamma_x$ небольшого предмета, находящегося на главной оптической оси.
    Можно ли применить эту формулу для предмета, не лежащего на главной оптической оси, почему?
}
\answer{%
    \begin{align*}
    \frac 1F &= \frac 1a + \frac 1b \implies b = \frac {aF}{a - F} \\
    \frac 1F &= \frac 1{a + x} + \frac 1c \implies c = \frac {(a+x)F}{a + x - F} \\
    x' &= \abs{b - c} = \frac {aF}{a - F} - \frac {(a+x)F}{a + x - F} = F\cbr{\frac {a}{a - F} - \frac {a+x}{a + x - F}} =  \\
    &= F \cdot \frac {a^2 + ax - aF - a^2 - ax + aF + xF}{(a - F)(a + x - F)} = F \cdot \frac {xF}{(a - F)(a + x - F)} \\
    \Gamma_x &= \frac{x'}x = \frac{F^2}{(a - F)(a + x - F)} \to \frac{F^2}{\sqr{a - F}}.
    \\
    &\text{Нельзя: изображение по-разному растянет по осям $x$ и $y$ и понадобится теорема Пифагора}
    \end{align*}
}
\solutionspace{150pt}

\tasknumber{8}%
\task{%
    Доказать формулу тонкой линзы для собирающей линзы.
}
\solutionspace{120pt}

\tasknumber{9}%
\task{%
    Постройте ход луча $CK$ в тонкой линзе.
    Известно положение линзы и оба её фокуса (см.
    рис.
    на доске).
    Рассмотрите оба типа линзы, сделав 2 рисунка: собирающую и рассеивающую.
}
\solutionspace{120pt}

\tasknumber{10}%
\task{%
    На экране, расположенном иа расстоянии $60\,\text{см}$ от собирающей линзы,
    получено изображение точечного источника, расположенного на главной оптической оси линзы.
    На какое расстояние переместится изображение на экране,
    если при неподвижной линзе переместить источник на $2\,\text{см}$ в плоскости, перпендикулярной главной оптической оси?
    Фокусное расстояние линзы равно $20\,\text{см}$.
}
\answer{%
    \begin{align*}
    &\frac 1F = \frac 1a + \frac 1b \implies a = \frac{bF}{b-F} \implies \Gamma = \frac ba = \frac{b-F}F \\
    &y = x \cdot \Gamma = x \cdot \frac{b-F}F \implies d = y = 4\,\text{см}.
    \end{align*}
}
\solutionspace{120pt}

\tasknumber{11}%
\task{%
    Оптическая сила двояковыпуклой линзы в воздухе $5\,\text{дптр}$, а в воде $1{,}6\,\text{дптр}$.
    Определить показатель преломления $n$ материала, из которого изготовлена линза.
    Показатель преломления воды равен $1{,}33$.
}
\answer{%
    \begin{align*}
    D_1 &=\cbr{\frac n{n_1} - 1}\cbr{\frac 1{R_1} + \frac 1{R_2}}, \\
    D_2 &=\cbr{\frac n{n_2} - 1}\cbr{\frac 1{R_1} + \frac 1{R_2}}, \\
    \frac {D_2}{D_1} &=\frac{\frac n{n_2} - 1}{\frac n{n_1} - 1} \implies {D_2}\cbr{\frac n{n_1} - 1} = {D_1}\cbr{\frac n{n_2} - 1}  \implies n\cbr{\frac{D_2}{n_1} - \frac{D_1}{n_2}} = D_2 - D_1, \\
    n &= \frac{D_2 - D_1}{\frac{D_2}{n_1} - \frac{D_1}{n_2}} = \frac{n_1 n_2 (D_2 - D_1)}{D_2n_2 - D_1n_1} \approx 1{,}575.
    \end{align*}
}
\solutionspace{120pt}

\tasknumber{12}%
\task{%
    На каком расстоянии от собирающей линзы с фокусным расстоянием $40\,\text{дптр}$
    следует надо поместить предмет, чтобы расстояние
    от предмета до его действительного изображения было наименьшим?
}
\answer{%
    \begin{align*}
    \frac 1a &+ \frac 1b = D \implies b = \frac 1{D - \frac 1a} \implies \ell = a + b = a + \frac a{Da - 1} = \frac{ Da^2 }{Da - 1} \implies \\
    \implies \ell'_a &= \frac{ 2Da \cdot (Da - 1) - Da^2 \cdot D }{\sqr{Da - 1}}= \frac{ D^2a^2 - 2Da}{\sqr{Da - 1}} = \frac{ Da(Da - 2)}{\sqr{Da - 1}}\implies a_{\min} = \frac 2D \approx 50\,\text{мм}.
    \end{align*}
}
\solutionspace{120pt}

\tasknumber{13}%
\task{%
    Даны точечный источник света $S$, его изображение $S_1$, полученное с помощью собирающей линзы,
    и ближайший к источнику фокус линзы $F$ (см.
    рис.
    на доске).
    Расстояния $SF = \ell$ и $SS_1 = L$.
    Определить положение линзы и её фокусное расстояние.
}
\answer{%
    \begin{align*}
    \frac 1a + \frac 1b &= \frac 1F, \ell = a - F, L = a + b \implies a = \ell + F, b = L - a = L - \ell - F \\
    \frac 1{\ell + F} + \frac 1{L - \ell - F} &= \frac 1F \\
    F\ell + F^2 + LF - F\ell - F^2 &= L\ell - \ell^2 - F\ell + LF - F\ell - F^2 \\
    0 &= L\ell - \ell^2 - 2F\ell - F^2 \\
    0 &=  F^2 + 2F\ell - L\ell + \ell^2 \\
    F &= -\ell \pm \sqrt{\ell^2 +  L\ell - \ell^2} = -\ell \pm \sqrt{L\ell} \implies F = \sqrt{L\ell} - \ell \\
    a &= \ell + F = \ell + \sqrt{L\ell} - \ell = \sqrt{L\ell}.
    \end{align*}
}
\solutionspace{120pt}

\tasknumber{14}%
\task{%
    Расстояние от освещённого предмета до экрана $100\,\text{см}$.
    Линза, помещенная между ними, даёт чёткое изображение предмета на
    экране при двух положениях, расстояние между которыми $20\,\text{см}$.
    Найти фокусное расстояние линзы.
}
\answer{%
    \begin{align*}
    \frac 1a + \frac 1b &= \frac 1F, \frac 1{a-\ell} + \frac 1{b+\ell} = \frac 1F, a + b = L \\
    \frac 1a + \frac 1b &= \frac 1{a-\ell} + \frac 1{b+\ell}\implies \frac{a + b}{ab} = \frac{(a-\ell) + (b+\ell)}{(a-\ell)(b+\ell)} \\
    ab  &= (a - \ell)(b+\ell) \implies 0  = -b\ell + a\ell - \ell^2 \implies 0 = -b + a - \ell \implies b = a - \ell \\
    a + (a - \ell) &= L \implies a = \frac{L + \ell}2 \implies b = \frac{L - \ell}2 \\
    F &= \frac{ab}{a + b} = \frac{L^2 -\ell^2}{4L} \approx 24\,\text{см}.
    \end{align*}
}
\solutionspace{120pt}

\tasknumber{15}%
\task{%
    Предмет находится на расстоянии $70\,\text{см}$ от экрана.
    Между предметом и экраном помещают линзу, причём при одном
    положении линзы на экране получается увеличенное изображение предмета,
    а при другом — уменьшенное.
    Каково фокусное расстояние линзы, если
    линейные размеры первого изображения в пять раз больше второго?
}
\answer{%
    \begin{align*}
    \frac 1a + \frac 1{L-a} &= \frac 1F, h_1 = h \cdot \frac{L-a}a, \\
    \frac 1b + \frac 1{L-b} &= \frac 1F, h_2 = h \cdot \frac{L-b}b, \\
    \frac{h_1}{h_2} &= 5 \implies \frac{(L-a)b}{(L-b)a} = 5, \\
    \frac 1F &= \frac{ L }{a(L-a)} = \frac{ L }{b(L-b)} \implies \frac{L-a}{L-b} = \frac b a \implies \frac {b^2}{a^2} = 5.
    \\
    \frac 1a + \frac 1{L-a} &= \frac 1b + \frac 1{L-b} \implies \frac L{a(L-a)} = \frac L{b(L-b)} \implies \\
    \implies aL - a^2 &= bL - b^2 \implies (a-b)L = (a-b)(a+b) \implies b = L - a, \\
    \frac{\sqr{L-a}}{a^2} &= 5 \implies \frac La - 1 = \sqrt{5} \implies a = \frac{ L }{\sqrt{5} + 1} \\
    F &= \frac{a(L-a)}L = \frac 1L \cdot \frac L{\sqrt{5} + 1} \cdot \frac {L\sqrt{5}}{\sqrt{5} + 1}= \frac { L\sqrt{5} }{ \sqr{\sqrt{5} + 1} } \approx 14{,}9\,\text{см}.
    \end{align*}
}
\solutionspace{120pt}

\tasknumber{16}%
\task{%
    (Задача-«гроб»: решать на обратной стороне) Квадрат со стороной $d = 2\,\text{см}$ расположен так,
    что 2 его стороны параллельны главной оптической оси рассеивающей линзы,
    его центр удален на $h = 6\,\text{см}$ от этой оси и на $a = 12\,\text{см}$ от плоскости линзы.
    Определите площадь изображения квадрата, если фокусное расстояние линзы составляет $F = 18\,\text{см}$.
    % (и сравните с площадью объекта, умноженной на квадрат увеличения центра квадрата).
}
\answer{%
    \begin{align*}
    &\text{Все явные вычисления — в см и $\text{см}^2$,} \\
    \frac 1 F &= \frac 1{a + \frac d2} + \frac 1b \implies b = \frac 1{\frac 1 F - \frac 1{a + \frac d2}} = \frac{F(a + \frac d2)}{a + \frac d2 - F} = -\frac{234}{31}, \\
    \frac 1 F &= \frac 1{a - \frac d2} + \frac 1c \implies c = \frac 1{\frac 1 F - \frac 1{a - \frac d2}} = \frac{F(a - \frac d2)}{a - \frac d2 - F} = -\frac{198}{29}, \\
    c - b &= \frac{F(a - \frac d2)}{a - \frac d2 - F} - \frac{F(a + \frac d2)}{a + \frac d2 - F} = F\cbr{ \frac{a - \frac d2}{a - \frac d2 - F} - \frac{a + \frac d2}{a + \frac d2 - F} } =  \\
    &= F \cdot \frac{a^2 + \frac {ad}2 - aF - \frac{ad}2 - \frac{d^2}4 + \frac{dF}2 - a^2 + \frac {ad}2 + aF - \frac{ad}2 + \frac{d^2}4 + \frac{dF}2}{\cbr{a + \frac d2 - F}\cbr{a - \frac d2 - F}}= F \cdot \frac {dF}{\cbr{a + \frac d2 - F}\cbr{a - \frac d2 - F}} = \frac{648}{899}.
    \\
    \Gamma_b &= \frac b{a + \frac d2} = \frac{ F }{a + \frac d2 - F} = -\frac{18}{31}, \\
    \Gamma_c &= \frac c{a - \frac d2} = \frac{ F }{a - \frac d2 - F} = -\frac{18}{29}, \\
    &\text{ тут интересно отметить, что } \Gamma_x = \frac{ c - b}{ d } = \frac{ F^2 }{\cbr{a + \frac d2 - F}\cbr{a - \frac d2 - F}} \ne \Gamma_b \text{ или } \Gamma_c \text{ даже при малых $d$}.
    \\
    S' &= \frac{d \cdot \Gamma_b + d \cdot \Gamma_c}2 \cdot (c - b) = \frac d2 \cbr{\frac{ F }{a + \frac d2 - F} + \frac{ F }{a - \frac d2 - F}} \cdot \cbr{c - b} =  \\
    &=\frac {dF}2 \cbr{\frac 1{a + \frac d2 - F} + \frac 1{a - \frac d2 - F}} \cdot \frac {dF^2}{\cbr{a + \frac d2 - F}\cbr{a - \frac d2 - F}} =  \\
    &=\frac {dF}2 \cdot \frac{a - \frac d2 - F + a + \frac d2 - F}{\cbr{a + \frac d2 - F}\cbr{a - \frac d2 - F}} \cdot \frac {dF^2}{\cbr{a + \frac d2 - F}\cbr{a - \frac d2 - F}} =  \\
    &= \frac {d^2F^3}{2\sqr{a + \frac d2 - F}\sqr{a - \frac d2 - F}} \cdot (2a - 2F) = \frac {d^2F^3(a - F)}{ \sqr{\sqr{a - F} - \frac{d^2}4} } = -\frac{699840}{808201}.
    \end{align*}
}

\variantsplitter

\addpersonalvariant{Рената Таржиманова}

\tasknumber{1}%
\task{%
    Запишите формулу тонкой линзы и сделайте рисунок, указав на нём физические величины из этой формулы.
}
\solutionspace{60pt}

\tasknumber{2}%
\task{%
    В каких линзах можно получить прямое изображение объекта?
}
\answer{%
    $\text{ собирающие и рассеивающие }$
}
\solutionspace{40pt}

\tasknumber{3}%
\task{%
    Какое изображение называют мнимым?
}
\solutionspace{40pt}

\tasknumber{4}%
\task{%
    Есть две линзы, обозначим их 1 и 2.
    Известно что оптическая сила линзы 1 больше, чем у линзы 2.
    Какая линза сильнее преломляет лучи?
}
\answer{%
    $1$
}
\solutionspace{40pt}

\tasknumber{5}%
\task{%
    Предмет находится на расстоянии $20\,\text{см}$ от рассеивающей линзы с фокусным расстоянием $50\,\text{см}$.
    Определите тип изображения, расстояние между предметом и его изображением, увеличение предмета.
    Сделайте схематичный рисунок (не обязательно в масштабе, но с сохранением свойств линзы и изображения).
}
\solutionspace{100pt}

\tasknumber{6}%
\task{%
    Объект находится на расстоянии $25\,\text{см}$ от линзы, а его мнимое изображение — в $10\,\text{см}$ от неё.
    Определите увеличение предмета, фокусное расстояние линзы, оптическую силу линзы и её тип.
}
\solutionspace{80pt}

\tasknumber{7}%
\task{%
    Известно, что из формулы тонкой линзы $\cbr{\frac 1F = \frac 1a + \frac 1b}$
    и определения увеличения $\cbr{\Gamma_y = \frac ba}$ можно получить выражение
    для увеличения: $\Gamma_y = \frac {aF}{a - F} \cdot \frac 1a = \frac {F}{a - F}.$
    Назовём такое увеличение «поперечным»: поперёк главной оптической оси (поэтому и ${}_y$).
    Получите формулу для «продольного» увеличения $\Gamma_x$ небольшого предмета, находящегося на главной оптической оси.
    Можно ли применить эту формулу для предмета, не лежащего на главной оптической оси, почему?
}
\answer{%
    \begin{align*}
    \frac 1F &= \frac 1a + \frac 1b \implies b = \frac {aF}{a - F} \\
    \frac 1F &= \frac 1{a + x} + \frac 1c \implies c = \frac {(a+x)F}{a + x - F} \\
    x' &= \abs{b - c} = \frac {aF}{a - F} - \frac {(a+x)F}{a + x - F} = F\cbr{\frac {a}{a - F} - \frac {a+x}{a + x - F}} =  \\
    &= F \cdot \frac {a^2 + ax - aF - a^2 - ax + aF + xF}{(a - F)(a + x - F)} = F \cdot \frac {xF}{(a - F)(a + x - F)} \\
    \Gamma_x &= \frac{x'}x = \frac{F^2}{(a - F)(a + x - F)} \to \frac{F^2}{\sqr{a - F}}.
    \\
    &\text{Нельзя: изображение по-разному растянет по осям $x$ и $y$ и понадобится теорема Пифагора}
    \end{align*}
}
\solutionspace{150pt}

\tasknumber{8}%
\task{%
    Доказать формулу тонкой линзы для рассеивающей линзы.
}
\solutionspace{120pt}

\tasknumber{9}%
\task{%
    Постройте ход луча $AL$ в тонкой линзе.
    Известно положение линзы и оба её фокуса (см.
    рис.
    на доске).
    Рассмотрите оба типа линзы, сделав 2 рисунка: собирающую и рассеивающую.
}
\solutionspace{120pt}

\tasknumber{10}%
\task{%
    На экране, расположенном иа расстоянии $60\,\text{см}$ от собирающей линзы,
    получено изображение точечного источника, расположенного на главной оптической оси линзы.
    На какое расстояние переместится изображение на экране,
    если при неподвижном источнике переместить линзу на $1\,\text{см}$ в плоскости, перпендикулярной главной оптической оси?
    Фокусное расстояние линзы равно $30\,\text{см}$.
}
\answer{%
    \begin{align*}
    &\frac 1F = \frac 1a + \frac 1b \implies a = \frac{bF}{b-F} \implies \Gamma = \frac ba = \frac{b-F}F \\
    &y = x \cdot \Gamma = x \cdot \frac{b-F}F \implies d = x + y = 2\,\text{см}.
    \end{align*}
}
\solutionspace{120pt}

\tasknumber{11}%
\task{%
    Оптическая сила двояковыпуклой линзы в воздухе $5\,\text{дптр}$, а в воде $1{,}4\,\text{дптр}$.
    Определить показатель преломления $n$ материала, из которого изготовлена линза.
    Показатель преломления воды равен $1{,}33$.
}
\answer{%
    \begin{align*}
    D_1 &=\cbr{\frac n{n_1} - 1}\cbr{\frac 1{R_1} + \frac 1{R_2}}, \\
    D_2 &=\cbr{\frac n{n_2} - 1}\cbr{\frac 1{R_1} + \frac 1{R_2}}, \\
    \frac {D_2}{D_1} &=\frac{\frac n{n_2} - 1}{\frac n{n_1} - 1} \implies {D_2}\cbr{\frac n{n_1} - 1} = {D_1}\cbr{\frac n{n_2} - 1}  \implies n\cbr{\frac{D_2}{n_1} - \frac{D_1}{n_2}} = D_2 - D_1, \\
    n &= \frac{D_2 - D_1}{\frac{D_2}{n_1} - \frac{D_1}{n_2}} = \frac{n_1 n_2 (D_2 - D_1)}{D_2n_2 - D_1n_1} \approx 1{,}526.
    \end{align*}
}
\solutionspace{120pt}

\tasknumber{12}%
\task{%
    На каком расстоянии от собирающей линзы с фокусным расстоянием $40\,\text{дптр}$
    следует надо поместить предмет, чтобы расстояние
    от предмета до его действительного изображения было наименьшим?
}
\answer{%
    \begin{align*}
    \frac 1a &+ \frac 1b = D \implies b = \frac 1{D - \frac 1a} \implies \ell = a + b = a + \frac a{Da - 1} = \frac{ Da^2 }{Da - 1} \implies \\
    \implies \ell'_a &= \frac{ 2Da \cdot (Da - 1) - Da^2 \cdot D }{\sqr{Da - 1}}= \frac{ D^2a^2 - 2Da}{\sqr{Da - 1}} = \frac{ Da(Da - 2)}{\sqr{Da - 1}}\implies a_{\min} = \frac 2D \approx 50\,\text{мм}.
    \end{align*}
}
\solutionspace{120pt}

\tasknumber{13}%
\task{%
    Даны точечный источник света $S$, его изображение $S_1$, полученное с помощью собирающей линзы,
    и ближайший к источнику фокус линзы $F$ (см.
    рис.
    на доске).
    Расстояния $SF = \ell$ и $SS_1 = L$.
    Определить положение линзы и её фокусное расстояние.
}
\answer{%
    \begin{align*}
    \frac 1a + \frac 1b &= \frac 1F, \ell = a - F, L = a + b \implies a = \ell + F, b = L - a = L - \ell - F \\
    \frac 1{\ell + F} + \frac 1{L - \ell - F} &= \frac 1F \\
    F\ell + F^2 + LF - F\ell - F^2 &= L\ell - \ell^2 - F\ell + LF - F\ell - F^2 \\
    0 &= L\ell - \ell^2 - 2F\ell - F^2 \\
    0 &=  F^2 + 2F\ell - L\ell + \ell^2 \\
    F &= -\ell \pm \sqrt{\ell^2 +  L\ell - \ell^2} = -\ell \pm \sqrt{L\ell} \implies F = \sqrt{L\ell} - \ell \\
    a &= \ell + F = \ell + \sqrt{L\ell} - \ell = \sqrt{L\ell}.
    \end{align*}
}
\solutionspace{120pt}

\tasknumber{14}%
\task{%
    Расстояние от освещённого предмета до экрана $80\,\text{см}$.
    Линза, помещенная между ними, даёт чёткое изображение предмета на
    экране при двух положениях, расстояние между которыми $30\,\text{см}$.
    Найти фокусное расстояние линзы.
}
\answer{%
    \begin{align*}
    \frac 1a + \frac 1b &= \frac 1F, \frac 1{a-\ell} + \frac 1{b+\ell} = \frac 1F, a + b = L \\
    \frac 1a + \frac 1b &= \frac 1{a-\ell} + \frac 1{b+\ell}\implies \frac{a + b}{ab} = \frac{(a-\ell) + (b+\ell)}{(a-\ell)(b+\ell)} \\
    ab  &= (a - \ell)(b+\ell) \implies 0  = -b\ell + a\ell - \ell^2 \implies 0 = -b + a - \ell \implies b = a - \ell \\
    a + (a - \ell) &= L \implies a = \frac{L + \ell}2 \implies b = \frac{L - \ell}2 \\
    F &= \frac{ab}{a + b} = \frac{L^2 -\ell^2}{4L} \approx 17{,}2\,\text{см}.
    \end{align*}
}
\solutionspace{120pt}

\tasknumber{15}%
\task{%
    Предмет находится на расстоянии $80\,\text{см}$ от экрана.
    Между предметом и экраном помещают линзу, причём при одном
    положении линзы на экране получается увеличенное изображение предмета,
    а при другом — уменьшенное.
    Каково фокусное расстояние линзы, если
    линейные размеры первого изображения в три раза больше второго?
}
\answer{%
    \begin{align*}
    \frac 1a + \frac 1{L-a} &= \frac 1F, h_1 = h \cdot \frac{L-a}a, \\
    \frac 1b + \frac 1{L-b} &= \frac 1F, h_2 = h \cdot \frac{L-b}b, \\
    \frac{h_1}{h_2} &= 3 \implies \frac{(L-a)b}{(L-b)a} = 3, \\
    \frac 1F &= \frac{ L }{a(L-a)} = \frac{ L }{b(L-b)} \implies \frac{L-a}{L-b} = \frac b a \implies \frac {b^2}{a^2} = 3.
    \\
    \frac 1a + \frac 1{L-a} &= \frac 1b + \frac 1{L-b} \implies \frac L{a(L-a)} = \frac L{b(L-b)} \implies \\
    \implies aL - a^2 &= bL - b^2 \implies (a-b)L = (a-b)(a+b) \implies b = L - a, \\
    \frac{\sqr{L-a}}{a^2} &= 3 \implies \frac La - 1 = \sqrt{3} \implies a = \frac{ L }{\sqrt{3} + 1} \\
    F &= \frac{a(L-a)}L = \frac 1L \cdot \frac L{\sqrt{3} + 1} \cdot \frac {L\sqrt{3}}{\sqrt{3} + 1}= \frac { L\sqrt{3} }{ \sqr{\sqrt{3} + 1} } \approx 18{,}6\,\text{см}.
    \end{align*}
}
\solutionspace{120pt}

\tasknumber{16}%
\task{%
    (Задача-«гроб»: решать на обратной стороне) Квадрат со стороной $d = 3\,\text{см}$ расположен так,
    что 2 его стороны параллельны главной оптической оси рассеивающей линзы,
    его центр удален на $h = 6\,\text{см}$ от этой оси и на $a = 10\,\text{см}$ от плоскости линзы.
    Определите площадь изображения квадрата, если фокусное расстояние линзы составляет $F = 20\,\text{см}$.
    % (и сравните с площадью объекта, умноженной на квадрат увеличения центра квадрата).
}
\answer{%
    \begin{align*}
    &\text{Все явные вычисления — в см и $\text{см}^2$,} \\
    \frac 1 F &= \frac 1{a + \frac d2} + \frac 1b \implies b = \frac 1{\frac 1 F - \frac 1{a + \frac d2}} = \frac{F(a + \frac d2)}{a + \frac d2 - F} = -\frac{460}{63}, \\
    \frac 1 F &= \frac 1{a - \frac d2} + \frac 1c \implies c = \frac 1{\frac 1 F - \frac 1{a - \frac d2}} = \frac{F(a - \frac d2)}{a - \frac d2 - F} = -\frac{340}{57}, \\
    c - b &= \frac{F(a - \frac d2)}{a - \frac d2 - F} - \frac{F(a + \frac d2)}{a + \frac d2 - F} = F\cbr{ \frac{a - \frac d2}{a - \frac d2 - F} - \frac{a + \frac d2}{a + \frac d2 - F} } =  \\
    &= F \cdot \frac{a^2 + \frac {ad}2 - aF - \frac{ad}2 - \frac{d^2}4 + \frac{dF}2 - a^2 + \frac {ad}2 + aF - \frac{ad}2 + \frac{d^2}4 + \frac{dF}2}{\cbr{a + \frac d2 - F}\cbr{a - \frac d2 - F}}= F \cdot \frac {dF}{\cbr{a + \frac d2 - F}\cbr{a - \frac d2 - F}} = \frac{1600}{1197}.
    \\
    \Gamma_b &= \frac b{a + \frac d2} = \frac{ F }{a + \frac d2 - F} = -\frac{40}{63}, \\
    \Gamma_c &= \frac c{a - \frac d2} = \frac{ F }{a - \frac d2 - F} = -\frac{40}{57}, \\
    &\text{ тут интересно отметить, что } \Gamma_x = \frac{ c - b}{ d } = \frac{ F^2 }{\cbr{a + \frac d2 - F}\cbr{a - \frac d2 - F}} \ne \Gamma_b \text{ или } \Gamma_c \text{ даже при малых $d$}.
    \\
    S' &= \frac{d \cdot \Gamma_b + d \cdot \Gamma_c}2 \cdot (c - b) = \frac d2 \cbr{\frac{ F }{a + \frac d2 - F} + \frac{ F }{a - \frac d2 - F}} \cdot \cbr{c - b} =  \\
    &=\frac {dF}2 \cbr{\frac 1{a + \frac d2 - F} + \frac 1{a - \frac d2 - F}} \cdot \frac {dF^2}{\cbr{a + \frac d2 - F}\cbr{a - \frac d2 - F}} =  \\
    &=\frac {dF}2 \cdot \frac{a - \frac d2 - F + a + \frac d2 - F}{\cbr{a + \frac d2 - F}\cbr{a - \frac d2 - F}} \cdot \frac {dF^2}{\cbr{a + \frac d2 - F}\cbr{a - \frac d2 - F}} =  \\
    &= \frac {d^2F^3}{2\sqr{a + \frac d2 - F}\sqr{a - \frac d2 - F}} \cdot (2a - 2F) = \frac {d^2F^3(a - F)}{ \sqr{\sqr{a - F} - \frac{d^2}4} } = -\frac{1280000}{477603}.
    \end{align*}
}

\variantsplitter

\addpersonalvariant{Андрей Щербаков}

\tasknumber{1}%
\task{%
    Запишите известные вам виды классификации изображений.
}
\solutionspace{60pt}

\tasknumber{2}%
\task{%
    В каких линзах можно получить мнимое изображение объекта?
}
\answer{%
    $\text{ собирающие и рассеивающие }$
}
\solutionspace{40pt}

\tasknumber{3}%
\task{%
    Какое изображение называют мнимым?
}
\solutionspace{40pt}

\tasknumber{4}%
\task{%
    Есть две линзы, обозначим их 1 и 2.
    Известно что фокусное расстояние линзы 1 меньше, чем у линзы 2.
    Какая линза сильнее преломляет лучи?
}
\answer{%
    $1$
}
\solutionspace{40pt}

\tasknumber{5}%
\task{%
    Предмет находится на расстоянии $30\,\text{см}$ от рассеивающей линзы с фокусным расстоянием $25\,\text{см}$.
    Определите тип изображения, расстояние между предметом и его изображением, увеличение предмета.
    Сделайте схематичный рисунок (не обязательно в масштабе, но с сохранением свойств линзы и изображения).
}
\solutionspace{100pt}

\tasknumber{6}%
\task{%
    Объект находится на расстоянии $115\,\text{см}$ от линзы, а его действительное изображение — в $50\,\text{см}$ от неё.
    Определите увеличение предмета, фокусное расстояние линзы, оптическую силу линзы и её тип.
}
\solutionspace{80pt}

\tasknumber{7}%
\task{%
    Известно, что из формулы тонкой линзы $\cbr{\frac 1F = \frac 1a + \frac 1b}$
    и определения увеличения $\cbr{\Gamma_y = \frac ba}$ можно получить выражение
    для увеличения: $\Gamma_y = \frac {aF}{a - F} \cdot \frac 1a = \frac {F}{a - F}.$
    Назовём такое увеличение «поперечным»: поперёк главной оптической оси (поэтому и ${}_y$).
    Получите формулу для «продольного» увеличения $\Gamma_x$ небольшого предмета, находящегося на главной оптической оси.
    Можно ли применить эту формулу для предмета, не лежащего на главной оптической оси, почему?
}
\answer{%
    \begin{align*}
    \frac 1F &= \frac 1a + \frac 1b \implies b = \frac {aF}{a - F} \\
    \frac 1F &= \frac 1{a + x} + \frac 1c \implies c = \frac {(a+x)F}{a + x - F} \\
    x' &= \abs{b - c} = \frac {aF}{a - F} - \frac {(a+x)F}{a + x - F} = F\cbr{\frac {a}{a - F} - \frac {a+x}{a + x - F}} =  \\
    &= F \cdot \frac {a^2 + ax - aF - a^2 - ax + aF + xF}{(a - F)(a + x - F)} = F \cdot \frac {xF}{(a - F)(a + x - F)} \\
    \Gamma_x &= \frac{x'}x = \frac{F^2}{(a - F)(a + x - F)} \to \frac{F^2}{\sqr{a - F}}.
    \\
    &\text{Нельзя: изображение по-разному растянет по осям $x$ и $y$ и понадобится теорема Пифагора}
    \end{align*}
}
\solutionspace{150pt}

\tasknumber{8}%
\task{%
    Доказать формулу тонкой линзы для рассеивающей линзы.
}
\solutionspace{120pt}

\tasknumber{9}%
\task{%
    Постройте ход луча $BM$ в тонкой линзе.
    Известно положение линзы и оба её фокуса (см.
    рис.
    на доске).
    Рассмотрите оба типа линзы, сделав 2 рисунка: собирающую и рассеивающую.
}
\solutionspace{120pt}

\tasknumber{10}%
\task{%
    На экране, расположенном иа расстоянии $60\,\text{см}$ от собирающей линзы,
    получено изображение точечного источника, расположенного на главной оптической оси линзы.
    На какое расстояние переместится изображение на экране,
    если при неподвижной линзе переместить источник на $1\,\text{см}$ в плоскости, перпендикулярной главной оптической оси?
    Фокусное расстояние линзы равно $20\,\text{см}$.
}
\answer{%
    \begin{align*}
    &\frac 1F = \frac 1a + \frac 1b \implies a = \frac{bF}{b-F} \implies \Gamma = \frac ba = \frac{b-F}F \\
    &y = x \cdot \Gamma = x \cdot \frac{b-F}F \implies d = y = 2\,\text{см}.
    \end{align*}
}
\solutionspace{120pt}

\tasknumber{11}%
\task{%
    Оптическая сила двояковыпуклой линзы в воздухе $4{,}5\,\text{дптр}$, а в воде $1{,}5\,\text{дптр}$.
    Определить показатель преломления $n$ материала, из которого изготовлена линза.
    Показатель преломления воды равен $1{,}33$.
}
\answer{%
    \begin{align*}
    D_1 &=\cbr{\frac n{n_1} - 1}\cbr{\frac 1{R_1} + \frac 1{R_2}}, \\
    D_2 &=\cbr{\frac n{n_2} - 1}\cbr{\frac 1{R_1} + \frac 1{R_2}}, \\
    \frac {D_2}{D_1} &=\frac{\frac n{n_2} - 1}{\frac n{n_1} - 1} \implies {D_2}\cbr{\frac n{n_1} - 1} = {D_1}\cbr{\frac n{n_2} - 1}  \implies n\cbr{\frac{D_2}{n_1} - \frac{D_1}{n_2}} = D_2 - D_1, \\
    n &= \frac{D_2 - D_1}{\frac{D_2}{n_1} - \frac{D_1}{n_2}} = \frac{n_1 n_2 (D_2 - D_1)}{D_2n_2 - D_1n_1} \approx 1{,}593.
    \end{align*}
}
\solutionspace{120pt}

\tasknumber{12}%
\task{%
    На каком расстоянии от собирающей линзы с фокусным расстоянием $30\,\text{дптр}$
    следует надо поместить предмет, чтобы расстояние
    от предмета до его действительного изображения было наименьшим?
}
\answer{%
    \begin{align*}
    \frac 1a &+ \frac 1b = D \implies b = \frac 1{D - \frac 1a} \implies \ell = a + b = a + \frac a{Da - 1} = \frac{ Da^2 }{Da - 1} \implies \\
    \implies \ell'_a &= \frac{ 2Da \cdot (Da - 1) - Da^2 \cdot D }{\sqr{Da - 1}}= \frac{ D^2a^2 - 2Da}{\sqr{Da - 1}} = \frac{ Da(Da - 2)}{\sqr{Da - 1}}\implies a_{\min} = \frac 2D \approx 66{,}7\,\text{мм}.
    \end{align*}
}
\solutionspace{120pt}

\tasknumber{13}%
\task{%
    Даны точечный источник света $S$, его изображение $S_1$, полученное с помощью собирающей линзы,
    и ближайший к источнику фокус линзы $F$ (см.
    рис.
    на доске).
    Расстояния $SF = \ell$ и $SS_1 = L$.
    Определить положение линзы и её фокусное расстояние.
}
\answer{%
    \begin{align*}
    \frac 1a + \frac 1b &= \frac 1F, \ell = a - F, L = a + b \implies a = \ell + F, b = L - a = L - \ell - F \\
    \frac 1{\ell + F} + \frac 1{L - \ell - F} &= \frac 1F \\
    F\ell + F^2 + LF - F\ell - F^2 &= L\ell - \ell^2 - F\ell + LF - F\ell - F^2 \\
    0 &= L\ell - \ell^2 - 2F\ell - F^2 \\
    0 &=  F^2 + 2F\ell - L\ell + \ell^2 \\
    F &= -\ell \pm \sqrt{\ell^2 +  L\ell - \ell^2} = -\ell \pm \sqrt{L\ell} \implies F = \sqrt{L\ell} - \ell \\
    a &= \ell + F = \ell + \sqrt{L\ell} - \ell = \sqrt{L\ell}.
    \end{align*}
}
\solutionspace{120pt}

\tasknumber{14}%
\task{%
    Расстояние от освещённого предмета до экрана $80\,\text{см}$.
    Линза, помещенная между ними, даёт чёткое изображение предмета на
    экране при двух положениях, расстояние между которыми $40\,\text{см}$.
    Найти фокусное расстояние линзы.
}
\answer{%
    \begin{align*}
    \frac 1a + \frac 1b &= \frac 1F, \frac 1{a-\ell} + \frac 1{b+\ell} = \frac 1F, a + b = L \\
    \frac 1a + \frac 1b &= \frac 1{a-\ell} + \frac 1{b+\ell}\implies \frac{a + b}{ab} = \frac{(a-\ell) + (b+\ell)}{(a-\ell)(b+\ell)} \\
    ab  &= (a - \ell)(b+\ell) \implies 0  = -b\ell + a\ell - \ell^2 \implies 0 = -b + a - \ell \implies b = a - \ell \\
    a + (a - \ell) &= L \implies a = \frac{L + \ell}2 \implies b = \frac{L - \ell}2 \\
    F &= \frac{ab}{a + b} = \frac{L^2 -\ell^2}{4L} \approx 15\,\text{см}.
    \end{align*}
}
\solutionspace{120pt}

\tasknumber{15}%
\task{%
    Предмет находится на расстоянии $80\,\text{см}$ от экрана.
    Между предметом и экраном помещают линзу, причём при одном
    положении линзы на экране получается увеличенное изображение предмета,
    а при другом — уменьшенное.
    Каково фокусное расстояние линзы, если
    линейные размеры первого изображения в пять раз больше второго?
}
\answer{%
    \begin{align*}
    \frac 1a + \frac 1{L-a} &= \frac 1F, h_1 = h \cdot \frac{L-a}a, \\
    \frac 1b + \frac 1{L-b} &= \frac 1F, h_2 = h \cdot \frac{L-b}b, \\
    \frac{h_1}{h_2} &= 5 \implies \frac{(L-a)b}{(L-b)a} = 5, \\
    \frac 1F &= \frac{ L }{a(L-a)} = \frac{ L }{b(L-b)} \implies \frac{L-a}{L-b} = \frac b a \implies \frac {b^2}{a^2} = 5.
    \\
    \frac 1a + \frac 1{L-a} &= \frac 1b + \frac 1{L-b} \implies \frac L{a(L-a)} = \frac L{b(L-b)} \implies \\
    \implies aL - a^2 &= bL - b^2 \implies (a-b)L = (a-b)(a+b) \implies b = L - a, \\
    \frac{\sqr{L-a}}{a^2} &= 5 \implies \frac La - 1 = \sqrt{5} \implies a = \frac{ L }{\sqrt{5} + 1} \\
    F &= \frac{a(L-a)}L = \frac 1L \cdot \frac L{\sqrt{5} + 1} \cdot \frac {L\sqrt{5}}{\sqrt{5} + 1}= \frac { L\sqrt{5} }{ \sqr{\sqrt{5} + 1} } \approx 17{,}1\,\text{см}.
    \end{align*}
}
\solutionspace{120pt}

\tasknumber{16}%
\task{%
    (Задача-«гроб»: решать на обратной стороне) Квадрат со стороной $d = 1\,\text{см}$ расположен так,
    что 2 его стороны параллельны главной оптической оси собирающей линзы,
    его центр удален на $h = 6\,\text{см}$ от этой оси и на $a = 10\,\text{см}$ от плоскости линзы.
    Определите площадь изображения квадрата, если фокусное расстояние линзы составляет $F = 18\,\text{см}$.
    % (и сравните с площадью объекта, умноженной на квадрат увеличения центра квадрата).
}
\answer{%
    \begin{align*}
    &\text{Все явные вычисления — в см и $\text{см}^2$,} \\
    \frac 1 F &= \frac 1{a + \frac d2} + \frac 1b \implies b = \frac 1{\frac 1 F - \frac 1{a + \frac d2}} = \frac{F(a + \frac d2)}{a + \frac d2 - F} = -\frac{126}5, \\
    \frac 1 F &= \frac 1{a - \frac d2} + \frac 1c \implies c = \frac 1{\frac 1 F - \frac 1{a - \frac d2}} = \frac{F(a - \frac d2)}{a - \frac d2 - F} = -\frac{342}{17}, \\
    c - b &= \frac{F(a - \frac d2)}{a - \frac d2 - F} - \frac{F(a + \frac d2)}{a + \frac d2 - F} = F\cbr{ \frac{a - \frac d2}{a - \frac d2 - F} - \frac{a + \frac d2}{a + \frac d2 - F} } =  \\
    &= F \cdot \frac{a^2 + \frac {ad}2 - aF - \frac{ad}2 - \frac{d^2}4 + \frac{dF}2 - a^2 + \frac {ad}2 + aF - \frac{ad}2 + \frac{d^2}4 + \frac{dF}2}{\cbr{a + \frac d2 - F}\cbr{a - \frac d2 - F}}= F \cdot \frac {dF}{\cbr{a + \frac d2 - F}\cbr{a - \frac d2 - F}} = \frac{432}{85}.
    \\
    \Gamma_b &= \frac b{a + \frac d2} = \frac{ F }{a + \frac d2 - F} = -\frac{12}5, \\
    \Gamma_c &= \frac c{a - \frac d2} = \frac{ F }{a - \frac d2 - F} = -\frac{36}{17}, \\
    &\text{ тут интересно отметить, что } \Gamma_x = \frac{ c - b}{ d } = \frac{ F^2 }{\cbr{a + \frac d2 - F}\cbr{a - \frac d2 - F}} \ne \Gamma_b \text{ или } \Gamma_c \text{ даже при малых $d$}.
    \\
    S' &= \frac{d \cdot \Gamma_b + d \cdot \Gamma_c}2 \cdot (c - b) = \frac d2 \cbr{\frac{ F }{a + \frac d2 - F} + \frac{ F }{a - \frac d2 - F}} \cdot \cbr{c - b} =  \\
    &=\frac {dF}2 \cbr{\frac 1{a + \frac d2 - F} + \frac 1{a - \frac d2 - F}} \cdot \frac {dF^2}{\cbr{a + \frac d2 - F}\cbr{a - \frac d2 - F}} =  \\
    &=\frac {dF}2 \cdot \frac{a - \frac d2 - F + a + \frac d2 - F}{\cbr{a + \frac d2 - F}\cbr{a - \frac d2 - F}} \cdot \frac {dF^2}{\cbr{a + \frac d2 - F}\cbr{a - \frac d2 - F}} =  \\
    &= \frac {d^2F^3}{2\sqr{a + \frac d2 - F}\sqr{a - \frac d2 - F}} \cdot (2a - 2F) = \frac {d^2F^3(a - F)}{ \sqr{\sqr{a - F} - \frac{d^2}4} } = -\frac{82944}{7225}.
    \end{align*}
}

\variantsplitter

\addpersonalvariant{Михаил Ярошевский}

\tasknumber{1}%
\task{%
    Запишите формулу тонкой линзы и сделайте рисунок, указав на нём физические величины из этой формулы.
}
\solutionspace{60pt}

\tasknumber{2}%
\task{%
    В каких линзах можно получить действительное изображение объекта?
}
\answer{%
    $\text{ собирающие }$
}
\solutionspace{40pt}

\tasknumber{3}%
\task{%
    Какое изображение называют действительным?
}
\solutionspace{40pt}

\tasknumber{4}%
\task{%
    Есть две линзы, обозначим их 1 и 2.
    Известно что фокусное расстояние линзы 2 меньше, чем у линзы 1.
    Какая линза сильнее преломляет лучи?
}
\answer{%
    $2$
}
\solutionspace{40pt}

\tasknumber{5}%
\task{%
    Предмет находится на расстоянии $10\,\text{см}$ от собирающей линзы с фокусным расстоянием $15\,\text{см}$.
    Определите тип изображения, расстояние между предметом и его изображением, увеличение предмета.
    Сделайте схематичный рисунок (не обязательно в масштабе, но с сохранением свойств линзы и изображения).
}
\solutionspace{100pt}

\tasknumber{6}%
\task{%
    Объект находится на расстоянии $45\,\text{см}$ от линзы, а его мнимое изображение — в $30\,\text{см}$ от неё.
    Определите увеличение предмета, фокусное расстояние линзы, оптическую силу линзы и её тип.
}
\solutionspace{80pt}

\tasknumber{7}%
\task{%
    Известно, что из формулы тонкой линзы $\cbr{\frac 1F = \frac 1a + \frac 1b}$
    и определения увеличения $\cbr{\Gamma_y = \frac ba}$ можно получить выражение
    для увеличения: $\Gamma_y = \frac {aF}{a - F} \cdot \frac 1a = \frac {F}{a - F}.$
    Назовём такое увеличение «поперечным»: поперёк главной оптической оси (поэтому и ${}_y$).
    Получите формулу для «продольного» увеличения $\Gamma_x$ небольшого предмета, находящегося на главной оптической оси.
    Можно ли применить эту формулу для предмета, не лежащего на главной оптической оси, почему?
}
\answer{%
    \begin{align*}
    \frac 1F &= \frac 1a + \frac 1b \implies b = \frac {aF}{a - F} \\
    \frac 1F &= \frac 1{a + x} + \frac 1c \implies c = \frac {(a+x)F}{a + x - F} \\
    x' &= \abs{b - c} = \frac {aF}{a - F} - \frac {(a+x)F}{a + x - F} = F\cbr{\frac {a}{a - F} - \frac {a+x}{a + x - F}} =  \\
    &= F \cdot \frac {a^2 + ax - aF - a^2 - ax + aF + xF}{(a - F)(a + x - F)} = F \cdot \frac {xF}{(a - F)(a + x - F)} \\
    \Gamma_x &= \frac{x'}x = \frac{F^2}{(a - F)(a + x - F)} \to \frac{F^2}{\sqr{a - F}}.
    \\
    &\text{Нельзя: изображение по-разному растянет по осям $x$ и $y$ и понадобится теорема Пифагора}
    \end{align*}
}
\solutionspace{150pt}

\tasknumber{8}%
\task{%
    Доказать формулу тонкой линзы для собирающей линзы.
}
\solutionspace{120pt}

\tasknumber{9}%
\task{%
    Постройте ход луча $AM$ в тонкой линзе.
    Известно положение линзы и оба её фокуса (см.
    рис.
    на доске).
    Рассмотрите оба типа линзы, сделав 2 рисунка: собирающую и рассеивающую.
}
\solutionspace{120pt}

\tasknumber{10}%
\task{%
    На экране, расположенном иа расстоянии $80\,\text{см}$ от собирающей линзы,
    получено изображение точечного источника, расположенного на главной оптической оси линзы.
    На какое расстояние переместится изображение на экране,
    если при неподвижном источнике переместить линзу на $3\,\text{см}$ в плоскости, перпендикулярной главной оптической оси?
    Фокусное расстояние линзы равно $40\,\text{см}$.
}
\answer{%
    \begin{align*}
    &\frac 1F = \frac 1a + \frac 1b \implies a = \frac{bF}{b-F} \implies \Gamma = \frac ba = \frac{b-F}F \\
    &y = x \cdot \Gamma = x \cdot \frac{b-F}F \implies d = x + y = 6\,\text{см}.
    \end{align*}
}
\solutionspace{120pt}

\tasknumber{11}%
\task{%
    Оптическая сила двояковыпуклой линзы в воздухе $5\,\text{дптр}$, а в воде $1{,}4\,\text{дптр}$.
    Определить показатель преломления $n$ материала, из которого изготовлена линза.
    Показатель преломления воды равен $1{,}33$.
}
\answer{%
    \begin{align*}
    D_1 &=\cbr{\frac n{n_1} - 1}\cbr{\frac 1{R_1} + \frac 1{R_2}}, \\
    D_2 &=\cbr{\frac n{n_2} - 1}\cbr{\frac 1{R_1} + \frac 1{R_2}}, \\
    \frac {D_2}{D_1} &=\frac{\frac n{n_2} - 1}{\frac n{n_1} - 1} \implies {D_2}\cbr{\frac n{n_1} - 1} = {D_1}\cbr{\frac n{n_2} - 1}  \implies n\cbr{\frac{D_2}{n_1} - \frac{D_1}{n_2}} = D_2 - D_1, \\
    n &= \frac{D_2 - D_1}{\frac{D_2}{n_1} - \frac{D_1}{n_2}} = \frac{n_1 n_2 (D_2 - D_1)}{D_2n_2 - D_1n_1} \approx 1{,}526.
    \end{align*}
}
\solutionspace{120pt}

\tasknumber{12}%
\task{%
    На каком расстоянии от собирающей линзы с фокусным расстоянием $50\,\text{дптр}$
    следует надо поместить предмет, чтобы расстояние
    от предмета до его действительного изображения было наименьшим?
}
\answer{%
    \begin{align*}
    \frac 1a &+ \frac 1b = D \implies b = \frac 1{D - \frac 1a} \implies \ell = a + b = a + \frac a{Da - 1} = \frac{ Da^2 }{Da - 1} \implies \\
    \implies \ell'_a &= \frac{ 2Da \cdot (Da - 1) - Da^2 \cdot D }{\sqr{Da - 1}}= \frac{ D^2a^2 - 2Da}{\sqr{Da - 1}} = \frac{ Da(Da - 2)}{\sqr{Da - 1}}\implies a_{\min} = \frac 2D \approx 40\,\text{мм}.
    \end{align*}
}
\solutionspace{120pt}

\tasknumber{13}%
\task{%
    Даны точечный источник света $S$, его изображение $S_1$, полученное с помощью собирающей линзы,
    и ближайший к источнику фокус линзы $F$ (см.
    рис.
    на доске).
    Расстояния $SF = \ell$ и $SS_1 = L$.
    Определить положение линзы и её фокусное расстояние.
}
\answer{%
    \begin{align*}
    \frac 1a + \frac 1b &= \frac 1F, \ell = a - F, L = a + b \implies a = \ell + F, b = L - a = L - \ell - F \\
    \frac 1{\ell + F} + \frac 1{L - \ell - F} &= \frac 1F \\
    F\ell + F^2 + LF - F\ell - F^2 &= L\ell - \ell^2 - F\ell + LF - F\ell - F^2 \\
    0 &= L\ell - \ell^2 - 2F\ell - F^2 \\
    0 &=  F^2 + 2F\ell - L\ell + \ell^2 \\
    F &= -\ell \pm \sqrt{\ell^2 +  L\ell - \ell^2} = -\ell \pm \sqrt{L\ell} \implies F = \sqrt{L\ell} - \ell \\
    a &= \ell + F = \ell + \sqrt{L\ell} - \ell = \sqrt{L\ell}.
    \end{align*}
}
\solutionspace{120pt}

\tasknumber{14}%
\task{%
    Расстояние от освещённого предмета до экрана $80\,\text{см}$.
    Линза, помещенная между ними, даёт чёткое изображение предмета на
    экране при двух положениях, расстояние между которыми $20\,\text{см}$.
    Найти фокусное расстояние линзы.
}
\answer{%
    \begin{align*}
    \frac 1a + \frac 1b &= \frac 1F, \frac 1{a-\ell} + \frac 1{b+\ell} = \frac 1F, a + b = L \\
    \frac 1a + \frac 1b &= \frac 1{a-\ell} + \frac 1{b+\ell}\implies \frac{a + b}{ab} = \frac{(a-\ell) + (b+\ell)}{(a-\ell)(b+\ell)} \\
    ab  &= (a - \ell)(b+\ell) \implies 0  = -b\ell + a\ell - \ell^2 \implies 0 = -b + a - \ell \implies b = a - \ell \\
    a + (a - \ell) &= L \implies a = \frac{L + \ell}2 \implies b = \frac{L - \ell}2 \\
    F &= \frac{ab}{a + b} = \frac{L^2 -\ell^2}{4L} \approx 18{,}8\,\text{см}.
    \end{align*}
}
\solutionspace{120pt}

\tasknumber{15}%
\task{%
    Предмет находится на расстоянии $60\,\text{см}$ от экрана.
    Между предметом и экраном помещают линзу, причём при одном
    положении линзы на экране получается увеличенное изображение предмета,
    а при другом — уменьшенное.
    Каково фокусное расстояние линзы, если
    линейные размеры первого изображения в три раза больше второго?
}
\answer{%
    \begin{align*}
    \frac 1a + \frac 1{L-a} &= \frac 1F, h_1 = h \cdot \frac{L-a}a, \\
    \frac 1b + \frac 1{L-b} &= \frac 1F, h_2 = h \cdot \frac{L-b}b, \\
    \frac{h_1}{h_2} &= 3 \implies \frac{(L-a)b}{(L-b)a} = 3, \\
    \frac 1F &= \frac{ L }{a(L-a)} = \frac{ L }{b(L-b)} \implies \frac{L-a}{L-b} = \frac b a \implies \frac {b^2}{a^2} = 3.
    \\
    \frac 1a + \frac 1{L-a} &= \frac 1b + \frac 1{L-b} \implies \frac L{a(L-a)} = \frac L{b(L-b)} \implies \\
    \implies aL - a^2 &= bL - b^2 \implies (a-b)L = (a-b)(a+b) \implies b = L - a, \\
    \frac{\sqr{L-a}}{a^2} &= 3 \implies \frac La - 1 = \sqrt{3} \implies a = \frac{ L }{\sqrt{3} + 1} \\
    F &= \frac{a(L-a)}L = \frac 1L \cdot \frac L{\sqrt{3} + 1} \cdot \frac {L\sqrt{3}}{\sqrt{3} + 1}= \frac { L\sqrt{3} }{ \sqr{\sqrt{3} + 1} } \approx 13{,}9\,\text{см}.
    \end{align*}
}
\solutionspace{120pt}

\tasknumber{16}%
\task{%
    (Задача-«гроб»: решать на обратной стороне) Квадрат со стороной $d = 3\,\text{см}$ расположен так,
    что 2 его стороны параллельны главной оптической оси рассеивающей линзы,
    его центр удален на $h = 4\,\text{см}$ от этой оси и на $a = 12\,\text{см}$ от плоскости линзы.
    Определите площадь изображения квадрата, если фокусное расстояние линзы составляет $F = 18\,\text{см}$.
    % (и сравните с площадью объекта, умноженной на квадрат увеличения центра квадрата).
}
\answer{%
    \begin{align*}
    &\text{Все явные вычисления — в см и $\text{см}^2$,} \\
    \frac 1 F &= \frac 1{a + \frac d2} + \frac 1b \implies b = \frac 1{\frac 1 F - \frac 1{a + \frac d2}} = \frac{F(a + \frac d2)}{a + \frac d2 - F} = -\frac{54}7, \\
    \frac 1 F &= \frac 1{a - \frac d2} + \frac 1c \implies c = \frac 1{\frac 1 F - \frac 1{a - \frac d2}} = \frac{F(a - \frac d2)}{a - \frac d2 - F} = -\frac{126}{19}, \\
    c - b &= \frac{F(a - \frac d2)}{a - \frac d2 - F} - \frac{F(a + \frac d2)}{a + \frac d2 - F} = F\cbr{ \frac{a - \frac d2}{a - \frac d2 - F} - \frac{a + \frac d2}{a + \frac d2 - F} } =  \\
    &= F \cdot \frac{a^2 + \frac {ad}2 - aF - \frac{ad}2 - \frac{d^2}4 + \frac{dF}2 - a^2 + \frac {ad}2 + aF - \frac{ad}2 + \frac{d^2}4 + \frac{dF}2}{\cbr{a + \frac d2 - F}\cbr{a - \frac d2 - F}}= F \cdot \frac {dF}{\cbr{a + \frac d2 - F}\cbr{a - \frac d2 - F}} = \frac{144}{133}.
    \\
    \Gamma_b &= \frac b{a + \frac d2} = \frac{ F }{a + \frac d2 - F} = -\frac47, \\
    \Gamma_c &= \frac c{a - \frac d2} = \frac{ F }{a - \frac d2 - F} = -\frac{12}{19}, \\
    &\text{ тут интересно отметить, что } \Gamma_x = \frac{ c - b}{ d } = \frac{ F^2 }{\cbr{a + \frac d2 - F}\cbr{a - \frac d2 - F}} \ne \Gamma_b \text{ или } \Gamma_c \text{ даже при малых $d$}.
    \\
    S' &= \frac{d \cdot \Gamma_b + d \cdot \Gamma_c}2 \cdot (c - b) = \frac d2 \cbr{\frac{ F }{a + \frac d2 - F} + \frac{ F }{a - \frac d2 - F}} \cdot \cbr{c - b} =  \\
    &=\frac {dF}2 \cbr{\frac 1{a + \frac d2 - F} + \frac 1{a - \frac d2 - F}} \cdot \frac {dF^2}{\cbr{a + \frac d2 - F}\cbr{a - \frac d2 - F}} =  \\
    &=\frac {dF}2 \cdot \frac{a - \frac d2 - F + a + \frac d2 - F}{\cbr{a + \frac d2 - F}\cbr{a - \frac d2 - F}} \cdot \frac {dF^2}{\cbr{a + \frac d2 - F}\cbr{a - \frac d2 - F}} =  \\
    &= \frac {d^2F^3}{2\sqr{a + \frac d2 - F}\sqr{a - \frac d2 - F}} \cdot (2a - 2F) = \frac {d^2F^3(a - F)}{ \sqr{\sqr{a - F} - \frac{d^2}4} } = -\frac{34560}{17689}.
    \end{align*}
}

\variantsplitter

\addpersonalvariant{Алексей Алимпиев}

\tasknumber{1}%
\task{%
    Запишите известные вам виды классификации изображений.
}
\solutionspace{60pt}

\tasknumber{2}%
\task{%
    В каких линзах можно получить прямое изображение объекта?
}
\answer{%
    $\text{ собирающие и рассеивающие }$
}
\solutionspace{40pt}

\tasknumber{3}%
\task{%
    Какое изображение называют мнимым?
}
\solutionspace{40pt}

\tasknumber{4}%
\task{%
    Есть две линзы, обозначим их 1 и 2.
    Известно что фокусное расстояние линзы 2 больше, чем у линзы 1.
    Какая линза сильнее преломляет лучи?
}
\answer{%
    $1$
}
\solutionspace{40pt}

\tasknumber{5}%
\task{%
    Предмет находится на расстоянии $30\,\text{см}$ от собирающей линзы с фокусным расстоянием $6\,\text{см}$.
    Определите тип изображения, расстояние между предметом и его изображением, увеличение предмета.
    Сделайте схематичный рисунок (не обязательно в масштабе, но с сохранением свойств линзы и изображения).
}
\solutionspace{100pt}

\tasknumber{6}%
\task{%
    Объект находится на расстоянии $115\,\text{см}$ от линзы, а его мнимое изображение — в $30\,\text{см}$ от неё.
    Определите увеличение предмета, фокусное расстояние линзы, оптическую силу линзы и её тип.
}
\solutionspace{80pt}

\tasknumber{7}%
\task{%
    Известно, что из формулы тонкой линзы $\cbr{\frac 1F = \frac 1a + \frac 1b}$
    и определения увеличения $\cbr{\Gamma_y = \frac ba}$ можно получить выражение
    для увеличения: $\Gamma_y = \frac {aF}{a - F} \cdot \frac 1a = \frac {F}{a - F}.$
    Назовём такое увеличение «поперечным»: поперёк главной оптической оси (поэтому и ${}_y$).
    Получите формулу для «продольного» увеличения $\Gamma_x$ небольшого предмета, находящегося на главной оптической оси.
    Можно ли применить эту формулу для предмета, не лежащего на главной оптической оси, почему?
}
\answer{%
    \begin{align*}
    \frac 1F &= \frac 1a + \frac 1b \implies b = \frac {aF}{a - F} \\
    \frac 1F &= \frac 1{a + x} + \frac 1c \implies c = \frac {(a+x)F}{a + x - F} \\
    x' &= \abs{b - c} = \frac {aF}{a - F} - \frac {(a+x)F}{a + x - F} = F\cbr{\frac {a}{a - F} - \frac {a+x}{a + x - F}} =  \\
    &= F \cdot \frac {a^2 + ax - aF - a^2 - ax + aF + xF}{(a - F)(a + x - F)} = F \cdot \frac {xF}{(a - F)(a + x - F)} \\
    \Gamma_x &= \frac{x'}x = \frac{F^2}{(a - F)(a + x - F)} \to \frac{F^2}{\sqr{a - F}}.
    \\
    &\text{Нельзя: изображение по-разному растянет по осям $x$ и $y$ и понадобится теорема Пифагора}
    \end{align*}
}
\solutionspace{150pt}

\tasknumber{8}%
\task{%
    Доказать формулу тонкой линзы для рассеивающей линзы.
}
\solutionspace{120pt}

\tasknumber{9}%
\task{%
    Постройте ход луча $BM$ в тонкой линзе.
    Известно положение линзы и оба её фокуса (см.
    рис.
    на доске).
    Рассмотрите оба типа линзы, сделав 2 рисунка: собирающую и рассеивающую.
}
\solutionspace{120pt}

\tasknumber{10}%
\task{%
    На экране, расположенном иа расстоянии $80\,\text{см}$ от собирающей линзы,
    получено изображение точечного источника, расположенного на главной оптической оси линзы.
    На какое расстояние переместится изображение на экране,
    если при неподвижном источнике переместить линзу на $3\,\text{см}$ в плоскости, перпендикулярной главной оптической оси?
    Фокусное расстояние линзы равно $30\,\text{см}$.
}
\answer{%
    \begin{align*}
    &\frac 1F = \frac 1a + \frac 1b \implies a = \frac{bF}{b-F} \implies \Gamma = \frac ba = \frac{b-F}F \\
    &y = x \cdot \Gamma = x \cdot \frac{b-F}F \implies d = x + y = 8{,}0\,\text{см}.
    \end{align*}
}
\solutionspace{120pt}

\tasknumber{11}%
\task{%
    Оптическая сила двояковыпуклой линзы в воздухе $5\,\text{дптр}$, а в воде $1{,}5\,\text{дптр}$.
    Определить показатель преломления $n$ материала, из которого изготовлена линза.
    Показатель преломления воды равен $1{,}33$.
}
\answer{%
    \begin{align*}
    D_1 &=\cbr{\frac n{n_1} - 1}\cbr{\frac 1{R_1} + \frac 1{R_2}}, \\
    D_2 &=\cbr{\frac n{n_2} - 1}\cbr{\frac 1{R_1} + \frac 1{R_2}}, \\
    \frac {D_2}{D_1} &=\frac{\frac n{n_2} - 1}{\frac n{n_1} - 1} \implies {D_2}\cbr{\frac n{n_1} - 1} = {D_1}\cbr{\frac n{n_2} - 1}  \implies n\cbr{\frac{D_2}{n_1} - \frac{D_1}{n_2}} = D_2 - D_1, \\
    n &= \frac{D_2 - D_1}{\frac{D_2}{n_1} - \frac{D_1}{n_2}} = \frac{n_1 n_2 (D_2 - D_1)}{D_2n_2 - D_1n_1} \approx 1{,}549.
    \end{align*}
}
\solutionspace{120pt}

\tasknumber{12}%
\task{%
    На каком расстоянии от собирающей линзы с фокусным расстоянием $40\,\text{дптр}$
    следует надо поместить предмет, чтобы расстояние
    от предмета до его действительного изображения было наименьшим?
}
\answer{%
    \begin{align*}
    \frac 1a &+ \frac 1b = D \implies b = \frac 1{D - \frac 1a} \implies \ell = a + b = a + \frac a{Da - 1} = \frac{ Da^2 }{Da - 1} \implies \\
    \implies \ell'_a &= \frac{ 2Da \cdot (Da - 1) - Da^2 \cdot D }{\sqr{Da - 1}}= \frac{ D^2a^2 - 2Da}{\sqr{Da - 1}} = \frac{ Da(Da - 2)}{\sqr{Da - 1}}\implies a_{\min} = \frac 2D \approx 50\,\text{мм}.
    \end{align*}
}
\solutionspace{120pt}

\tasknumber{13}%
\task{%
    Даны точечный источник света $S$, его изображение $S_1$, полученное с помощью собирающей линзы,
    и ближайший к источнику фокус линзы $F$ (см.
    рис.
    на доске).
    Расстояния $SF = \ell$ и $SS_1 = L$.
    Определить положение линзы и её фокусное расстояние.
}
\answer{%
    \begin{align*}
    \frac 1a + \frac 1b &= \frac 1F, \ell = a - F, L = a + b \implies a = \ell + F, b = L - a = L - \ell - F \\
    \frac 1{\ell + F} + \frac 1{L - \ell - F} &= \frac 1F \\
    F\ell + F^2 + LF - F\ell - F^2 &= L\ell - \ell^2 - F\ell + LF - F\ell - F^2 \\
    0 &= L\ell - \ell^2 - 2F\ell - F^2 \\
    0 &=  F^2 + 2F\ell - L\ell + \ell^2 \\
    F &= -\ell \pm \sqrt{\ell^2 +  L\ell - \ell^2} = -\ell \pm \sqrt{L\ell} \implies F = \sqrt{L\ell} - \ell \\
    a &= \ell + F = \ell + \sqrt{L\ell} - \ell = \sqrt{L\ell}.
    \end{align*}
}
\solutionspace{120pt}

\tasknumber{14}%
\task{%
    Расстояние от освещённого предмета до экрана $100\,\text{см}$.
    Линза, помещенная между ними, даёт чёткое изображение предмета на
    экране при двух положениях, расстояние между которыми $20\,\text{см}$.
    Найти фокусное расстояние линзы.
}
\answer{%
    \begin{align*}
    \frac 1a + \frac 1b &= \frac 1F, \frac 1{a-\ell} + \frac 1{b+\ell} = \frac 1F, a + b = L \\
    \frac 1a + \frac 1b &= \frac 1{a-\ell} + \frac 1{b+\ell}\implies \frac{a + b}{ab} = \frac{(a-\ell) + (b+\ell)}{(a-\ell)(b+\ell)} \\
    ab  &= (a - \ell)(b+\ell) \implies 0  = -b\ell + a\ell - \ell^2 \implies 0 = -b + a - \ell \implies b = a - \ell \\
    a + (a - \ell) &= L \implies a = \frac{L + \ell}2 \implies b = \frac{L - \ell}2 \\
    F &= \frac{ab}{a + b} = \frac{L^2 -\ell^2}{4L} \approx 24\,\text{см}.
    \end{align*}
}
\solutionspace{120pt}

\tasknumber{15}%
\task{%
    Предмет находится на расстоянии $80\,\text{см}$ от экрана.
    Между предметом и экраном помещают линзу, причём при одном
    положении линзы на экране получается увеличенное изображение предмета,
    а при другом — уменьшенное.
    Каково фокусное расстояние линзы, если
    линейные размеры первого изображения в три раза больше второго?
}
\answer{%
    \begin{align*}
    \frac 1a + \frac 1{L-a} &= \frac 1F, h_1 = h \cdot \frac{L-a}a, \\
    \frac 1b + \frac 1{L-b} &= \frac 1F, h_2 = h \cdot \frac{L-b}b, \\
    \frac{h_1}{h_2} &= 3 \implies \frac{(L-a)b}{(L-b)a} = 3, \\
    \frac 1F &= \frac{ L }{a(L-a)} = \frac{ L }{b(L-b)} \implies \frac{L-a}{L-b} = \frac b a \implies \frac {b^2}{a^2} = 3.
    \\
    \frac 1a + \frac 1{L-a} &= \frac 1b + \frac 1{L-b} \implies \frac L{a(L-a)} = \frac L{b(L-b)} \implies \\
    \implies aL - a^2 &= bL - b^2 \implies (a-b)L = (a-b)(a+b) \implies b = L - a, \\
    \frac{\sqr{L-a}}{a^2} &= 3 \implies \frac La - 1 = \sqrt{3} \implies a = \frac{ L }{\sqrt{3} + 1} \\
    F &= \frac{a(L-a)}L = \frac 1L \cdot \frac L{\sqrt{3} + 1} \cdot \frac {L\sqrt{3}}{\sqrt{3} + 1}= \frac { L\sqrt{3} }{ \sqr{\sqrt{3} + 1} } \approx 18{,}6\,\text{см}.
    \end{align*}
}
\solutionspace{120pt}

\tasknumber{16}%
\task{%
    (Задача-«гроб»: решать на обратной стороне) Квадрат со стороной $d = 3\,\text{см}$ расположен так,
    что 2 его стороны параллельны главной оптической оси рассеивающей линзы,
    его центр удален на $h = 6\,\text{см}$ от этой оси и на $a = 12\,\text{см}$ от плоскости линзы.
    Определите площадь изображения квадрата, если фокусное расстояние линзы составляет $F = 25\,\text{см}$.
    % (и сравните с площадью объекта, умноженной на квадрат увеличения центра квадрата).
}
\answer{%
    \begin{align*}
    &\text{Все явные вычисления — в см и $\text{см}^2$,} \\
    \frac 1 F &= \frac 1{a + \frac d2} + \frac 1b \implies b = \frac 1{\frac 1 F - \frac 1{a + \frac d2}} = \frac{F(a + \frac d2)}{a + \frac d2 - F} = -\frac{675}{77}, \\
    \frac 1 F &= \frac 1{a - \frac d2} + \frac 1c \implies c = \frac 1{\frac 1 F - \frac 1{a - \frac d2}} = \frac{F(a - \frac d2)}{a - \frac d2 - F} = -\frac{525}{71}, \\
    c - b &= \frac{F(a - \frac d2)}{a - \frac d2 - F} - \frac{F(a + \frac d2)}{a + \frac d2 - F} = F\cbr{ \frac{a - \frac d2}{a - \frac d2 - F} - \frac{a + \frac d2}{a + \frac d2 - F} } =  \\
    &= F \cdot \frac{a^2 + \frac {ad}2 - aF - \frac{ad}2 - \frac{d^2}4 + \frac{dF}2 - a^2 + \frac {ad}2 + aF - \frac{ad}2 + \frac{d^2}4 + \frac{dF}2}{\cbr{a + \frac d2 - F}\cbr{a - \frac d2 - F}}= F \cdot \frac {dF}{\cbr{a + \frac d2 - F}\cbr{a - \frac d2 - F}} = \frac{7500}{5467}.
    \\
    \Gamma_b &= \frac b{a + \frac d2} = \frac{ F }{a + \frac d2 - F} = -\frac{50}{77}, \\
    \Gamma_c &= \frac c{a - \frac d2} = \frac{ F }{a - \frac d2 - F} = -\frac{50}{71}, \\
    &\text{ тут интересно отметить, что } \Gamma_x = \frac{ c - b}{ d } = \frac{ F^2 }{\cbr{a + \frac d2 - F}\cbr{a - \frac d2 - F}} \ne \Gamma_b \text{ или } \Gamma_c \text{ даже при малых $d$}.
    \\
    S' &= \frac{d \cdot \Gamma_b + d \cdot \Gamma_c}2 \cdot (c - b) = \frac d2 \cbr{\frac{ F }{a + \frac d2 - F} + \frac{ F }{a - \frac d2 - F}} \cdot \cbr{c - b} =  \\
    &=\frac {dF}2 \cbr{\frac 1{a + \frac d2 - F} + \frac 1{a - \frac d2 - F}} \cdot \frac {dF^2}{\cbr{a + \frac d2 - F}\cbr{a - \frac d2 - F}} =  \\
    &=\frac {dF}2 \cdot \frac{a - \frac d2 - F + a + \frac d2 - F}{\cbr{a + \frac d2 - F}\cbr{a - \frac d2 - F}} \cdot \frac {dF^2}{\cbr{a + \frac d2 - F}\cbr{a - \frac d2 - F}} =  \\
    &= \frac {d^2F^3}{2\sqr{a + \frac d2 - F}\sqr{a - \frac d2 - F}} \cdot (2a - 2F) = \frac {d^2F^3(a - F)}{ \sqr{\sqr{a - F} - \frac{d^2}4} } = -\frac{83250000}{29888089}.
    \end{align*}
}

\variantsplitter

\addpersonalvariant{Евгений Васин}

\tasknumber{1}%
\task{%
    Запишите известные вам виды классификации изображений.
}
\solutionspace{60pt}

\tasknumber{2}%
\task{%
    В каких линзах можно получить увеличенное изображение объекта?
}
\answer{%
    $\text{ рассеивающие }$
}
\solutionspace{40pt}

\tasknumber{3}%
\task{%
    Какое изображение называют действительным?
}
\solutionspace{40pt}

\tasknumber{4}%
\task{%
    Есть две линзы, обозначим их 1 и 2.
    Известно что фокусное расстояние линзы 2 больше, чем у линзы 1.
    Какая линза сильнее преломляет лучи?
}
\answer{%
    $1$
}
\solutionspace{40pt}

\tasknumber{5}%
\task{%
    Предмет находится на расстоянии $10\,\text{см}$ от рассеивающей линзы с фокусным расстоянием $6\,\text{см}$.
    Определите тип изображения, расстояние между предметом и его изображением, увеличение предмета.
    Сделайте схематичный рисунок (не обязательно в масштабе, но с сохранением свойств линзы и изображения).
}
\solutionspace{100pt}

\tasknumber{6}%
\task{%
    Объект находится на расстоянии $115\,\text{см}$ от линзы, а его мнимое изображение — в $20\,\text{см}$ от неё.
    Определите увеличение предмета, фокусное расстояние линзы, оптическую силу линзы и её тип.
}
\solutionspace{80pt}

\tasknumber{7}%
\task{%
    Известно, что из формулы тонкой линзы $\cbr{\frac 1F = \frac 1a + \frac 1b}$
    и определения увеличения $\cbr{\Gamma_y = \frac ba}$ можно получить выражение
    для увеличения: $\Gamma_y = \frac {aF}{a - F} \cdot \frac 1a = \frac {F}{a - F}.$
    Назовём такое увеличение «поперечным»: поперёк главной оптической оси (поэтому и ${}_y$).
    Получите формулу для «продольного» увеличения $\Gamma_x$ небольшого предмета, находящегося на главной оптической оси.
    Можно ли применить эту формулу для предмета, не лежащего на главной оптической оси, почему?
}
\answer{%
    \begin{align*}
    \frac 1F &= \frac 1a + \frac 1b \implies b = \frac {aF}{a - F} \\
    \frac 1F &= \frac 1{a + x} + \frac 1c \implies c = \frac {(a+x)F}{a + x - F} \\
    x' &= \abs{b - c} = \frac {aF}{a - F} - \frac {(a+x)F}{a + x - F} = F\cbr{\frac {a}{a - F} - \frac {a+x}{a + x - F}} =  \\
    &= F \cdot \frac {a^2 + ax - aF - a^2 - ax + aF + xF}{(a - F)(a + x - F)} = F \cdot \frac {xF}{(a - F)(a + x - F)} \\
    \Gamma_x &= \frac{x'}x = \frac{F^2}{(a - F)(a + x - F)} \to \frac{F^2}{\sqr{a - F}}.
    \\
    &\text{Нельзя: изображение по-разному растянет по осям $x$ и $y$ и понадобится теорема Пифагора}
    \end{align*}
}
\solutionspace{150pt}

\tasknumber{8}%
\task{%
    Доказать формулу тонкой линзы для собирающей линзы.
}
\solutionspace{120pt}

\tasknumber{9}%
\task{%
    Постройте ход луча $BM$ в тонкой линзе.
    Известно положение линзы и оба её фокуса (см.
    рис.
    на доске).
    Рассмотрите оба типа линзы, сделав 2 рисунка: собирающую и рассеивающую.
}
\solutionspace{120pt}

\tasknumber{10}%
\task{%
    На экране, расположенном иа расстоянии $60\,\text{см}$ от собирающей линзы,
    получено изображение точечного источника, расположенного на главной оптической оси линзы.
    На какое расстояние переместится изображение на экране,
    если при неподвижной линзе переместить источник на $2\,\text{см}$ в плоскости, перпендикулярной главной оптической оси?
    Фокусное расстояние линзы равно $30\,\text{см}$.
}
\answer{%
    \begin{align*}
    &\frac 1F = \frac 1a + \frac 1b \implies a = \frac{bF}{b-F} \implies \Gamma = \frac ba = \frac{b-F}F \\
    &y = x \cdot \Gamma = x \cdot \frac{b-F}F \implies d = y = 2\,\text{см}.
    \end{align*}
}
\solutionspace{120pt}

\tasknumber{11}%
\task{%
    Оптическая сила двояковыпуклой линзы в воздухе $5{,}5\,\text{дптр}$, а в воде $1{,}5\,\text{дптр}$.
    Определить показатель преломления $n$ материала, из которого изготовлена линза.
    Показатель преломления воды равен $1{,}33$.
}
\answer{%
    \begin{align*}
    D_1 &=\cbr{\frac n{n_1} - 1}\cbr{\frac 1{R_1} + \frac 1{R_2}}, \\
    D_2 &=\cbr{\frac n{n_2} - 1}\cbr{\frac 1{R_1} + \frac 1{R_2}}, \\
    \frac {D_2}{D_1} &=\frac{\frac n{n_2} - 1}{\frac n{n_1} - 1} \implies {D_2}\cbr{\frac n{n_1} - 1} = {D_1}\cbr{\frac n{n_2} - 1}  \implies n\cbr{\frac{D_2}{n_1} - \frac{D_1}{n_2}} = D_2 - D_1, \\
    n &= \frac{D_2 - D_1}{\frac{D_2}{n_1} - \frac{D_1}{n_2}} = \frac{n_1 n_2 (D_2 - D_1)}{D_2n_2 - D_1n_1} \approx 1{,}518.
    \end{align*}
}
\solutionspace{120pt}

\tasknumber{12}%
\task{%
    На каком расстоянии от собирающей линзы с фокусным расстоянием $40\,\text{дптр}$
    следует надо поместить предмет, чтобы расстояние
    от предмета до его действительного изображения было наименьшим?
}
\answer{%
    \begin{align*}
    \frac 1a &+ \frac 1b = D \implies b = \frac 1{D - \frac 1a} \implies \ell = a + b = a + \frac a{Da - 1} = \frac{ Da^2 }{Da - 1} \implies \\
    \implies \ell'_a &= \frac{ 2Da \cdot (Da - 1) - Da^2 \cdot D }{\sqr{Da - 1}}= \frac{ D^2a^2 - 2Da}{\sqr{Da - 1}} = \frac{ Da(Da - 2)}{\sqr{Da - 1}}\implies a_{\min} = \frac 2D \approx 50\,\text{мм}.
    \end{align*}
}
\solutionspace{120pt}

\tasknumber{13}%
\task{%
    Даны точечный источник света $S$, его изображение $S_1$, полученное с помощью собирающей линзы,
    и ближайший к источнику фокус линзы $F$ (см.
    рис.
    на доске).
    Расстояния $SF = \ell$ и $SS_1 = L$.
    Определить положение линзы и её фокусное расстояние.
}
\answer{%
    \begin{align*}
    \frac 1a + \frac 1b &= \frac 1F, \ell = a - F, L = a + b \implies a = \ell + F, b = L - a = L - \ell - F \\
    \frac 1{\ell + F} + \frac 1{L - \ell - F} &= \frac 1F \\
    F\ell + F^2 + LF - F\ell - F^2 &= L\ell - \ell^2 - F\ell + LF - F\ell - F^2 \\
    0 &= L\ell - \ell^2 - 2F\ell - F^2 \\
    0 &=  F^2 + 2F\ell - L\ell + \ell^2 \\
    F &= -\ell \pm \sqrt{\ell^2 +  L\ell - \ell^2} = -\ell \pm \sqrt{L\ell} \implies F = \sqrt{L\ell} - \ell \\
    a &= \ell + F = \ell + \sqrt{L\ell} - \ell = \sqrt{L\ell}.
    \end{align*}
}
\solutionspace{120pt}

\tasknumber{14}%
\task{%
    Расстояние от освещённого предмета до экрана $80\,\text{см}$.
    Линза, помещенная между ними, даёт чёткое изображение предмета на
    экране при двух положениях, расстояние между которыми $40\,\text{см}$.
    Найти фокусное расстояние линзы.
}
\answer{%
    \begin{align*}
    \frac 1a + \frac 1b &= \frac 1F, \frac 1{a-\ell} + \frac 1{b+\ell} = \frac 1F, a + b = L \\
    \frac 1a + \frac 1b &= \frac 1{a-\ell} + \frac 1{b+\ell}\implies \frac{a + b}{ab} = \frac{(a-\ell) + (b+\ell)}{(a-\ell)(b+\ell)} \\
    ab  &= (a - \ell)(b+\ell) \implies 0  = -b\ell + a\ell - \ell^2 \implies 0 = -b + a - \ell \implies b = a - \ell \\
    a + (a - \ell) &= L \implies a = \frac{L + \ell}2 \implies b = \frac{L - \ell}2 \\
    F &= \frac{ab}{a + b} = \frac{L^2 -\ell^2}{4L} \approx 15\,\text{см}.
    \end{align*}
}
\solutionspace{120pt}

\tasknumber{15}%
\task{%
    Предмет находится на расстоянии $80\,\text{см}$ от экрана.
    Между предметом и экраном помещают линзу, причём при одном
    положении линзы на экране получается увеличенное изображение предмета,
    а при другом — уменьшенное.
    Каково фокусное расстояние линзы, если
    линейные размеры первого изображения в пять раз больше второго?
}
\answer{%
    \begin{align*}
    \frac 1a + \frac 1{L-a} &= \frac 1F, h_1 = h \cdot \frac{L-a}a, \\
    \frac 1b + \frac 1{L-b} &= \frac 1F, h_2 = h \cdot \frac{L-b}b, \\
    \frac{h_1}{h_2} &= 5 \implies \frac{(L-a)b}{(L-b)a} = 5, \\
    \frac 1F &= \frac{ L }{a(L-a)} = \frac{ L }{b(L-b)} \implies \frac{L-a}{L-b} = \frac b a \implies \frac {b^2}{a^2} = 5.
    \\
    \frac 1a + \frac 1{L-a} &= \frac 1b + \frac 1{L-b} \implies \frac L{a(L-a)} = \frac L{b(L-b)} \implies \\
    \implies aL - a^2 &= bL - b^2 \implies (a-b)L = (a-b)(a+b) \implies b = L - a, \\
    \frac{\sqr{L-a}}{a^2} &= 5 \implies \frac La - 1 = \sqrt{5} \implies a = \frac{ L }{\sqrt{5} + 1} \\
    F &= \frac{a(L-a)}L = \frac 1L \cdot \frac L{\sqrt{5} + 1} \cdot \frac {L\sqrt{5}}{\sqrt{5} + 1}= \frac { L\sqrt{5} }{ \sqr{\sqrt{5} + 1} } \approx 17{,}1\,\text{см}.
    \end{align*}
}
\solutionspace{120pt}

\tasknumber{16}%
\task{%
    (Задача-«гроб»: решать на обратной стороне) Квадрат со стороной $d = 1\,\text{см}$ расположен так,
    что 2 его стороны параллельны главной оптической оси рассеивающей линзы,
    его центр удален на $h = 6\,\text{см}$ от этой оси и на $a = 12\,\text{см}$ от плоскости линзы.
    Определите площадь изображения квадрата, если фокусное расстояние линзы составляет $F = 25\,\text{см}$.
    % (и сравните с площадью объекта, умноженной на квадрат увеличения центра квадрата).
}
\answer{%
    \begin{align*}
    &\text{Все явные вычисления — в см и $\text{см}^2$,} \\
    \frac 1 F &= \frac 1{a + \frac d2} + \frac 1b \implies b = \frac 1{\frac 1 F - \frac 1{a + \frac d2}} = \frac{F(a + \frac d2)}{a + \frac d2 - F} = -\frac{25}3, \\
    \frac 1 F &= \frac 1{a - \frac d2} + \frac 1c \implies c = \frac 1{\frac 1 F - \frac 1{a - \frac d2}} = \frac{F(a - \frac d2)}{a - \frac d2 - F} = -\frac{575}{73}, \\
    c - b &= \frac{F(a - \frac d2)}{a - \frac d2 - F} - \frac{F(a + \frac d2)}{a + \frac d2 - F} = F\cbr{ \frac{a - \frac d2}{a - \frac d2 - F} - \frac{a + \frac d2}{a + \frac d2 - F} } =  \\
    &= F \cdot \frac{a^2 + \frac {ad}2 - aF - \frac{ad}2 - \frac{d^2}4 + \frac{dF}2 - a^2 + \frac {ad}2 + aF - \frac{ad}2 + \frac{d^2}4 + \frac{dF}2}{\cbr{a + \frac d2 - F}\cbr{a - \frac d2 - F}}= F \cdot \frac {dF}{\cbr{a + \frac d2 - F}\cbr{a - \frac d2 - F}} = \frac{100}{219}.
    \\
    \Gamma_b &= \frac b{a + \frac d2} = \frac{ F }{a + \frac d2 - F} = -\frac23, \\
    \Gamma_c &= \frac c{a - \frac d2} = \frac{ F }{a - \frac d2 - F} = -\frac{50}{73}, \\
    &\text{ тут интересно отметить, что } \Gamma_x = \frac{ c - b}{ d } = \frac{ F^2 }{\cbr{a + \frac d2 - F}\cbr{a - \frac d2 - F}} \ne \Gamma_b \text{ или } \Gamma_c \text{ даже при малых $d$}.
    \\
    S' &= \frac{d \cdot \Gamma_b + d \cdot \Gamma_c}2 \cdot (c - b) = \frac d2 \cbr{\frac{ F }{a + \frac d2 - F} + \frac{ F }{a - \frac d2 - F}} \cdot \cbr{c - b} =  \\
    &=\frac {dF}2 \cbr{\frac 1{a + \frac d2 - F} + \frac 1{a - \frac d2 - F}} \cdot \frac {dF^2}{\cbr{a + \frac d2 - F}\cbr{a - \frac d2 - F}} =  \\
    &=\frac {dF}2 \cdot \frac{a - \frac d2 - F + a + \frac d2 - F}{\cbr{a + \frac d2 - F}\cbr{a - \frac d2 - F}} \cdot \frac {dF^2}{\cbr{a + \frac d2 - F}\cbr{a - \frac d2 - F}} =  \\
    &= \frac {d^2F^3}{2\sqr{a + \frac d2 - F}\sqr{a - \frac d2 - F}} \cdot (2a - 2F) = \frac {d^2F^3(a - F)}{ \sqr{\sqr{a - F} - \frac{d^2}4} } = -\frac{14800}{47961}.
    \end{align*}
}

\variantsplitter

\addpersonalvariant{Вячеслав Волохов}

\tasknumber{1}%
\task{%
    Запишите известные вам виды классификации изображений.
}
\solutionspace{60pt}

\tasknumber{2}%
\task{%
    В каких линзах можно получить прямое изображение объекта?
}
\answer{%
    $\text{ собирающие и рассеивающие }$
}
\solutionspace{40pt}

\tasknumber{3}%
\task{%
    Какое изображение называют мнимым?
}
\solutionspace{40pt}

\tasknumber{4}%
\task{%
    Есть две линзы, обозначим их 1 и 2.
    Известно что оптическая сила линзы 2 больше, чем у линзы 1.
    Какая линза сильнее преломляет лучи?
}
\answer{%
    $2$
}
\solutionspace{40pt}

\tasknumber{5}%
\task{%
    Предмет находится на расстоянии $20\,\text{см}$ от рассеивающей линзы с фокусным расстоянием $15\,\text{см}$.
    Определите тип изображения, расстояние между предметом и его изображением, увеличение предмета.
    Сделайте схематичный рисунок (не обязательно в масштабе, но с сохранением свойств линзы и изображения).
}
\solutionspace{100pt}

\tasknumber{6}%
\task{%
    Объект находится на расстоянии $45\,\text{см}$ от линзы, а его мнимое изображение — в $40\,\text{см}$ от неё.
    Определите увеличение предмета, фокусное расстояние линзы, оптическую силу линзы и её тип.
}
\solutionspace{80pt}

\tasknumber{7}%
\task{%
    Известно, что из формулы тонкой линзы $\cbr{\frac 1F = \frac 1a + \frac 1b}$
    и определения увеличения $\cbr{\Gamma_y = \frac ba}$ можно получить выражение
    для увеличения: $\Gamma_y = \frac {aF}{a - F} \cdot \frac 1a = \frac {F}{a - F}.$
    Назовём такое увеличение «поперечным»: поперёк главной оптической оси (поэтому и ${}_y$).
    Получите формулу для «продольного» увеличения $\Gamma_x$ небольшого предмета, находящегося на главной оптической оси.
    Можно ли применить эту формулу для предмета, не лежащего на главной оптической оси, почему?
}
\answer{%
    \begin{align*}
    \frac 1F &= \frac 1a + \frac 1b \implies b = \frac {aF}{a - F} \\
    \frac 1F &= \frac 1{a + x} + \frac 1c \implies c = \frac {(a+x)F}{a + x - F} \\
    x' &= \abs{b - c} = \frac {aF}{a - F} - \frac {(a+x)F}{a + x - F} = F\cbr{\frac {a}{a - F} - \frac {a+x}{a + x - F}} =  \\
    &= F \cdot \frac {a^2 + ax - aF - a^2 - ax + aF + xF}{(a - F)(a + x - F)} = F \cdot \frac {xF}{(a - F)(a + x - F)} \\
    \Gamma_x &= \frac{x'}x = \frac{F^2}{(a - F)(a + x - F)} \to \frac{F^2}{\sqr{a - F}}.
    \\
    &\text{Нельзя: изображение по-разному растянет по осям $x$ и $y$ и понадобится теорема Пифагора}
    \end{align*}
}
\solutionspace{150pt}

\tasknumber{8}%
\task{%
    Доказать формулу тонкой линзы для рассеивающей линзы.
}
\solutionspace{120pt}

\tasknumber{9}%
\task{%
    Постройте ход луча $BK$ в тонкой линзе.
    Известно положение линзы и оба её фокуса (см.
    рис.
    на доске).
    Рассмотрите оба типа линзы, сделав 2 рисунка: собирающую и рассеивающую.
}
\solutionspace{120pt}

\tasknumber{10}%
\task{%
    На экране, расположенном иа расстоянии $80\,\text{см}$ от собирающей линзы,
    получено изображение точечного источника, расположенного на главной оптической оси линзы.
    На какое расстояние переместится изображение на экране,
    если при неподвижном источнике переместить линзу на $1\,\text{см}$ в плоскости, перпендикулярной главной оптической оси?
    Фокусное расстояние линзы равно $30\,\text{см}$.
}
\answer{%
    \begin{align*}
    &\frac 1F = \frac 1a + \frac 1b \implies a = \frac{bF}{b-F} \implies \Gamma = \frac ba = \frac{b-F}F \\
    &y = x \cdot \Gamma = x \cdot \frac{b-F}F \implies d = x + y = 2{,}7\,\text{см}.
    \end{align*}
}
\solutionspace{120pt}

\tasknumber{11}%
\task{%
    Оптическая сила двояковыпуклой линзы в воздухе $5{,}5\,\text{дптр}$, а в воде $1{,}5\,\text{дптр}$.
    Определить показатель преломления $n$ материала, из которого изготовлена линза.
    Показатель преломления воды равен $1{,}33$.
}
\answer{%
    \begin{align*}
    D_1 &=\cbr{\frac n{n_1} - 1}\cbr{\frac 1{R_1} + \frac 1{R_2}}, \\
    D_2 &=\cbr{\frac n{n_2} - 1}\cbr{\frac 1{R_1} + \frac 1{R_2}}, \\
    \frac {D_2}{D_1} &=\frac{\frac n{n_2} - 1}{\frac n{n_1} - 1} \implies {D_2}\cbr{\frac n{n_1} - 1} = {D_1}\cbr{\frac n{n_2} - 1}  \implies n\cbr{\frac{D_2}{n_1} - \frac{D_1}{n_2}} = D_2 - D_1, \\
    n &= \frac{D_2 - D_1}{\frac{D_2}{n_1} - \frac{D_1}{n_2}} = \frac{n_1 n_2 (D_2 - D_1)}{D_2n_2 - D_1n_1} \approx 1{,}518.
    \end{align*}
}
\solutionspace{120pt}

\tasknumber{12}%
\task{%
    На каком расстоянии от собирающей линзы с фокусным расстоянием $40\,\text{дптр}$
    следует надо поместить предмет, чтобы расстояние
    от предмета до его действительного изображения было наименьшим?
}
\answer{%
    \begin{align*}
    \frac 1a &+ \frac 1b = D \implies b = \frac 1{D - \frac 1a} \implies \ell = a + b = a + \frac a{Da - 1} = \frac{ Da^2 }{Da - 1} \implies \\
    \implies \ell'_a &= \frac{ 2Da \cdot (Da - 1) - Da^2 \cdot D }{\sqr{Da - 1}}= \frac{ D^2a^2 - 2Da}{\sqr{Da - 1}} = \frac{ Da(Da - 2)}{\sqr{Da - 1}}\implies a_{\min} = \frac 2D \approx 50\,\text{мм}.
    \end{align*}
}
\solutionspace{120pt}

\tasknumber{13}%
\task{%
    Даны точечный источник света $S$, его изображение $S_1$, полученное с помощью собирающей линзы,
    и ближайший к источнику фокус линзы $F$ (см.
    рис.
    на доске).
    Расстояния $SF = \ell$ и $SS_1 = L$.
    Определить положение линзы и её фокусное расстояние.
}
\answer{%
    \begin{align*}
    \frac 1a + \frac 1b &= \frac 1F, \ell = a - F, L = a + b \implies a = \ell + F, b = L - a = L - \ell - F \\
    \frac 1{\ell + F} + \frac 1{L - \ell - F} &= \frac 1F \\
    F\ell + F^2 + LF - F\ell - F^2 &= L\ell - \ell^2 - F\ell + LF - F\ell - F^2 \\
    0 &= L\ell - \ell^2 - 2F\ell - F^2 \\
    0 &=  F^2 + 2F\ell - L\ell + \ell^2 \\
    F &= -\ell \pm \sqrt{\ell^2 +  L\ell - \ell^2} = -\ell \pm \sqrt{L\ell} \implies F = \sqrt{L\ell} - \ell \\
    a &= \ell + F = \ell + \sqrt{L\ell} - \ell = \sqrt{L\ell}.
    \end{align*}
}
\solutionspace{120pt}

\tasknumber{14}%
\task{%
    Расстояние от освещённого предмета до экрана $80\,\text{см}$.
    Линза, помещенная между ними, даёт чёткое изображение предмета на
    экране при двух положениях, расстояние между которыми $20\,\text{см}$.
    Найти фокусное расстояние линзы.
}
\answer{%
    \begin{align*}
    \frac 1a + \frac 1b &= \frac 1F, \frac 1{a-\ell} + \frac 1{b+\ell} = \frac 1F, a + b = L \\
    \frac 1a + \frac 1b &= \frac 1{a-\ell} + \frac 1{b+\ell}\implies \frac{a + b}{ab} = \frac{(a-\ell) + (b+\ell)}{(a-\ell)(b+\ell)} \\
    ab  &= (a - \ell)(b+\ell) \implies 0  = -b\ell + a\ell - \ell^2 \implies 0 = -b + a - \ell \implies b = a - \ell \\
    a + (a - \ell) &= L \implies a = \frac{L + \ell}2 \implies b = \frac{L - \ell}2 \\
    F &= \frac{ab}{a + b} = \frac{L^2 -\ell^2}{4L} \approx 18{,}8\,\text{см}.
    \end{align*}
}
\solutionspace{120pt}

\tasknumber{15}%
\task{%
    Предмет находится на расстоянии $60\,\text{см}$ от экрана.
    Между предметом и экраном помещают линзу, причём при одном
    положении линзы на экране получается увеличенное изображение предмета,
    а при другом — уменьшенное.
    Каково фокусное расстояние линзы, если
    линейные размеры первого изображения в пять раз больше второго?
}
\answer{%
    \begin{align*}
    \frac 1a + \frac 1{L-a} &= \frac 1F, h_1 = h \cdot \frac{L-a}a, \\
    \frac 1b + \frac 1{L-b} &= \frac 1F, h_2 = h \cdot \frac{L-b}b, \\
    \frac{h_1}{h_2} &= 5 \implies \frac{(L-a)b}{(L-b)a} = 5, \\
    \frac 1F &= \frac{ L }{a(L-a)} = \frac{ L }{b(L-b)} \implies \frac{L-a}{L-b} = \frac b a \implies \frac {b^2}{a^2} = 5.
    \\
    \frac 1a + \frac 1{L-a} &= \frac 1b + \frac 1{L-b} \implies \frac L{a(L-a)} = \frac L{b(L-b)} \implies \\
    \implies aL - a^2 &= bL - b^2 \implies (a-b)L = (a-b)(a+b) \implies b = L - a, \\
    \frac{\sqr{L-a}}{a^2} &= 5 \implies \frac La - 1 = \sqrt{5} \implies a = \frac{ L }{\sqrt{5} + 1} \\
    F &= \frac{a(L-a)}L = \frac 1L \cdot \frac L{\sqrt{5} + 1} \cdot \frac {L\sqrt{5}}{\sqrt{5} + 1}= \frac { L\sqrt{5} }{ \sqr{\sqrt{5} + 1} } \approx 12{,}8\,\text{см}.
    \end{align*}
}
\solutionspace{120pt}

\tasknumber{16}%
\task{%
    (Задача-«гроб»: решать на обратной стороне) Квадрат со стороной $d = 3\,\text{см}$ расположен так,
    что 2 его стороны параллельны главной оптической оси рассеивающей линзы,
    его центр удален на $h = 4\,\text{см}$ от этой оси и на $a = 10\,\text{см}$ от плоскости линзы.
    Определите площадь изображения квадрата, если фокусное расстояние линзы составляет $F = 25\,\text{см}$.
    % (и сравните с площадью объекта, умноженной на квадрат увеличения центра квадрата).
}
\answer{%
    \begin{align*}
    &\text{Все явные вычисления — в см и $\text{см}^2$,} \\
    \frac 1 F &= \frac 1{a + \frac d2} + \frac 1b \implies b = \frac 1{\frac 1 F - \frac 1{a + \frac d2}} = \frac{F(a + \frac d2)}{a + \frac d2 - F} = -\frac{575}{73}, \\
    \frac 1 F &= \frac 1{a - \frac d2} + \frac 1c \implies c = \frac 1{\frac 1 F - \frac 1{a - \frac d2}} = \frac{F(a - \frac d2)}{a - \frac d2 - F} = -\frac{425}{67}, \\
    c - b &= \frac{F(a - \frac d2)}{a - \frac d2 - F} - \frac{F(a + \frac d2)}{a + \frac d2 - F} = F\cbr{ \frac{a - \frac d2}{a - \frac d2 - F} - \frac{a + \frac d2}{a + \frac d2 - F} } =  \\
    &= F \cdot \frac{a^2 + \frac {ad}2 - aF - \frac{ad}2 - \frac{d^2}4 + \frac{dF}2 - a^2 + \frac {ad}2 + aF - \frac{ad}2 + \frac{d^2}4 + \frac{dF}2}{\cbr{a + \frac d2 - F}\cbr{a - \frac d2 - F}}= F \cdot \frac {dF}{\cbr{a + \frac d2 - F}\cbr{a - \frac d2 - F}} = \frac{7500}{4891}.
    \\
    \Gamma_b &= \frac b{a + \frac d2} = \frac{ F }{a + \frac d2 - F} = -\frac{50}{73}, \\
    \Gamma_c &= \frac c{a - \frac d2} = \frac{ F }{a - \frac d2 - F} = -\frac{50}{67}, \\
    &\text{ тут интересно отметить, что } \Gamma_x = \frac{ c - b}{ d } = \frac{ F^2 }{\cbr{a + \frac d2 - F}\cbr{a - \frac d2 - F}} \ne \Gamma_b \text{ или } \Gamma_c \text{ даже при малых $d$}.
    \\
    S' &= \frac{d \cdot \Gamma_b + d \cdot \Gamma_c}2 \cdot (c - b) = \frac d2 \cbr{\frac{ F }{a + \frac d2 - F} + \frac{ F }{a - \frac d2 - F}} \cdot \cbr{c - b} =  \\
    &=\frac {dF}2 \cbr{\frac 1{a + \frac d2 - F} + \frac 1{a - \frac d2 - F}} \cdot \frac {dF^2}{\cbr{a + \frac d2 - F}\cbr{a - \frac d2 - F}} =  \\
    &=\frac {dF}2 \cdot \frac{a - \frac d2 - F + a + \frac d2 - F}{\cbr{a + \frac d2 - F}\cbr{a - \frac d2 - F}} \cdot \frac {dF^2}{\cbr{a + \frac d2 - F}\cbr{a - \frac d2 - F}} =  \\
    &= \frac {d^2F^3}{2\sqr{a + \frac d2 - F}\sqr{a - \frac d2 - F}} \cdot (2a - 2F) = \frac {d^2F^3(a - F)}{ \sqr{\sqr{a - F} - \frac{d^2}4} } = -\frac{78750000}{23921881}.
    \end{align*}
}

\variantsplitter

\addpersonalvariant{Герман Говоров}

\tasknumber{1}%
\task{%
    Запишите формулу тонкой линзы и сделайте рисунок, указав на нём физические величины из этой формулы.
}
\solutionspace{60pt}

\tasknumber{2}%
\task{%
    В каких линзах можно получить обратное изображение объекта?
}
\answer{%
    $\text{ собирающие }$
}
\solutionspace{40pt}

\tasknumber{3}%
\task{%
    Какое изображение называют действительным?
}
\solutionspace{40pt}

\tasknumber{4}%
\task{%
    Есть две линзы, обозначим их 1 и 2.
    Известно что фокусное расстояние линзы 2 меньше, чем у линзы 1.
    Какая линза сильнее преломляет лучи?
}
\answer{%
    $2$
}
\solutionspace{40pt}

\tasknumber{5}%
\task{%
    Предмет находится на расстоянии $30\,\text{см}$ от рассеивающей линзы с фокусным расстоянием $40\,\text{см}$.
    Определите тип изображения, расстояние между предметом и его изображением, увеличение предмета.
    Сделайте схематичный рисунок (не обязательно в масштабе, но с сохранением свойств линзы и изображения).
}
\solutionspace{100pt}

\tasknumber{6}%
\task{%
    Объект находится на расстоянии $115\,\text{см}$ от линзы, а его мнимое изображение — в $30\,\text{см}$ от неё.
    Определите увеличение предмета, фокусное расстояние линзы, оптическую силу линзы и её тип.
}
\solutionspace{80pt}

\tasknumber{7}%
\task{%
    Известно, что из формулы тонкой линзы $\cbr{\frac 1F = \frac 1a + \frac 1b}$
    и определения увеличения $\cbr{\Gamma_y = \frac ba}$ можно получить выражение
    для увеличения: $\Gamma_y = \frac {aF}{a - F} \cdot \frac 1a = \frac {F}{a - F}.$
    Назовём такое увеличение «поперечным»: поперёк главной оптической оси (поэтому и ${}_y$).
    Получите формулу для «продольного» увеличения $\Gamma_x$ небольшого предмета, находящегося на главной оптической оси.
    Можно ли применить эту формулу для предмета, не лежащего на главной оптической оси, почему?
}
\answer{%
    \begin{align*}
    \frac 1F &= \frac 1a + \frac 1b \implies b = \frac {aF}{a - F} \\
    \frac 1F &= \frac 1{a + x} + \frac 1c \implies c = \frac {(a+x)F}{a + x - F} \\
    x' &= \abs{b - c} = \frac {aF}{a - F} - \frac {(a+x)F}{a + x - F} = F\cbr{\frac {a}{a - F} - \frac {a+x}{a + x - F}} =  \\
    &= F \cdot \frac {a^2 + ax - aF - a^2 - ax + aF + xF}{(a - F)(a + x - F)} = F \cdot \frac {xF}{(a - F)(a + x - F)} \\
    \Gamma_x &= \frac{x'}x = \frac{F^2}{(a - F)(a + x - F)} \to \frac{F^2}{\sqr{a - F}}.
    \\
    &\text{Нельзя: изображение по-разному растянет по осям $x$ и $y$ и понадобится теорема Пифагора}
    \end{align*}
}
\solutionspace{150pt}

\tasknumber{8}%
\task{%
    Доказать формулу тонкой линзы для собирающей линзы.
}
\solutionspace{120pt}

\tasknumber{9}%
\task{%
    Постройте ход луча $CK$ в тонкой линзе.
    Известно положение линзы и оба её фокуса (см.
    рис.
    на доске).
    Рассмотрите оба типа линзы, сделав 2 рисунка: собирающую и рассеивающую.
}
\solutionspace{120pt}

\tasknumber{10}%
\task{%
    На экране, расположенном иа расстоянии $80\,\text{см}$ от собирающей линзы,
    получено изображение точечного источника, расположенного на главной оптической оси линзы.
    На какое расстояние переместится изображение на экране,
    если при неподвижной линзе переместить источник на $3\,\text{см}$ в плоскости, перпендикулярной главной оптической оси?
    Фокусное расстояние линзы равно $30\,\text{см}$.
}
\answer{%
    \begin{align*}
    &\frac 1F = \frac 1a + \frac 1b \implies a = \frac{bF}{b-F} \implies \Gamma = \frac ba = \frac{b-F}F \\
    &y = x \cdot \Gamma = x \cdot \frac{b-F}F \implies d = y = 5{,}0\,\text{см}.
    \end{align*}
}
\solutionspace{120pt}

\tasknumber{11}%
\task{%
    Оптическая сила двояковыпуклой линзы в воздухе $4{,}5\,\text{дптр}$, а в воде $1{,}6\,\text{дптр}$.
    Определить показатель преломления $n$ материала, из которого изготовлена линза.
    Показатель преломления воды равен $1{,}33$.
}
\answer{%
    \begin{align*}
    D_1 &=\cbr{\frac n{n_1} - 1}\cbr{\frac 1{R_1} + \frac 1{R_2}}, \\
    D_2 &=\cbr{\frac n{n_2} - 1}\cbr{\frac 1{R_1} + \frac 1{R_2}}, \\
    \frac {D_2}{D_1} &=\frac{\frac n{n_2} - 1}{\frac n{n_1} - 1} \implies {D_2}\cbr{\frac n{n_1} - 1} = {D_1}\cbr{\frac n{n_2} - 1}  \implies n\cbr{\frac{D_2}{n_1} - \frac{D_1}{n_2}} = D_2 - D_1, \\
    n &= \frac{D_2 - D_1}{\frac{D_2}{n_1} - \frac{D_1}{n_2}} = \frac{n_1 n_2 (D_2 - D_1)}{D_2n_2 - D_1n_1} \approx 1{,}626.
    \end{align*}
}
\solutionspace{120pt}

\tasknumber{12}%
\task{%
    На каком расстоянии от собирающей линзы с фокусным расстоянием $50\,\text{дптр}$
    следует надо поместить предмет, чтобы расстояние
    от предмета до его действительного изображения было наименьшим?
}
\answer{%
    \begin{align*}
    \frac 1a &+ \frac 1b = D \implies b = \frac 1{D - \frac 1a} \implies \ell = a + b = a + \frac a{Da - 1} = \frac{ Da^2 }{Da - 1} \implies \\
    \implies \ell'_a &= \frac{ 2Da \cdot (Da - 1) - Da^2 \cdot D }{\sqr{Da - 1}}= \frac{ D^2a^2 - 2Da}{\sqr{Da - 1}} = \frac{ Da(Da - 2)}{\sqr{Da - 1}}\implies a_{\min} = \frac 2D \approx 40\,\text{мм}.
    \end{align*}
}
\solutionspace{120pt}

\tasknumber{13}%
\task{%
    Даны точечный источник света $S$, его изображение $S_1$, полученное с помощью собирающей линзы,
    и ближайший к источнику фокус линзы $F$ (см.
    рис.
    на доске).
    Расстояния $SF = \ell$ и $SS_1 = L$.
    Определить положение линзы и её фокусное расстояние.
}
\answer{%
    \begin{align*}
    \frac 1a + \frac 1b &= \frac 1F, \ell = a - F, L = a + b \implies a = \ell + F, b = L - a = L - \ell - F \\
    \frac 1{\ell + F} + \frac 1{L - \ell - F} &= \frac 1F \\
    F\ell + F^2 + LF - F\ell - F^2 &= L\ell - \ell^2 - F\ell + LF - F\ell - F^2 \\
    0 &= L\ell - \ell^2 - 2F\ell - F^2 \\
    0 &=  F^2 + 2F\ell - L\ell + \ell^2 \\
    F &= -\ell \pm \sqrt{\ell^2 +  L\ell - \ell^2} = -\ell \pm \sqrt{L\ell} \implies F = \sqrt{L\ell} - \ell \\
    a &= \ell + F = \ell + \sqrt{L\ell} - \ell = \sqrt{L\ell}.
    \end{align*}
}
\solutionspace{120pt}

\tasknumber{14}%
\task{%
    Расстояние от освещённого предмета до экрана $80\,\text{см}$.
    Линза, помещенная между ними, даёт чёткое изображение предмета на
    экране при двух положениях, расстояние между которыми $40\,\text{см}$.
    Найти фокусное расстояние линзы.
}
\answer{%
    \begin{align*}
    \frac 1a + \frac 1b &= \frac 1F, \frac 1{a-\ell} + \frac 1{b+\ell} = \frac 1F, a + b = L \\
    \frac 1a + \frac 1b &= \frac 1{a-\ell} + \frac 1{b+\ell}\implies \frac{a + b}{ab} = \frac{(a-\ell) + (b+\ell)}{(a-\ell)(b+\ell)} \\
    ab  &= (a - \ell)(b+\ell) \implies 0  = -b\ell + a\ell - \ell^2 \implies 0 = -b + a - \ell \implies b = a - \ell \\
    a + (a - \ell) &= L \implies a = \frac{L + \ell}2 \implies b = \frac{L - \ell}2 \\
    F &= \frac{ab}{a + b} = \frac{L^2 -\ell^2}{4L} \approx 15\,\text{см}.
    \end{align*}
}
\solutionspace{120pt}

\tasknumber{15}%
\task{%
    Предмет находится на расстоянии $70\,\text{см}$ от экрана.
    Между предметом и экраном помещают линзу, причём при одном
    положении линзы на экране получается увеличенное изображение предмета,
    а при другом — уменьшенное.
    Каково фокусное расстояние линзы, если
    линейные размеры первого изображения в пять раз больше второго?
}
\answer{%
    \begin{align*}
    \frac 1a + \frac 1{L-a} &= \frac 1F, h_1 = h \cdot \frac{L-a}a, \\
    \frac 1b + \frac 1{L-b} &= \frac 1F, h_2 = h \cdot \frac{L-b}b, \\
    \frac{h_1}{h_2} &= 5 \implies \frac{(L-a)b}{(L-b)a} = 5, \\
    \frac 1F &= \frac{ L }{a(L-a)} = \frac{ L }{b(L-b)} \implies \frac{L-a}{L-b} = \frac b a \implies \frac {b^2}{a^2} = 5.
    \\
    \frac 1a + \frac 1{L-a} &= \frac 1b + \frac 1{L-b} \implies \frac L{a(L-a)} = \frac L{b(L-b)} \implies \\
    \implies aL - a^2 &= bL - b^2 \implies (a-b)L = (a-b)(a+b) \implies b = L - a, \\
    \frac{\sqr{L-a}}{a^2} &= 5 \implies \frac La - 1 = \sqrt{5} \implies a = \frac{ L }{\sqrt{5} + 1} \\
    F &= \frac{a(L-a)}L = \frac 1L \cdot \frac L{\sqrt{5} + 1} \cdot \frac {L\sqrt{5}}{\sqrt{5} + 1}= \frac { L\sqrt{5} }{ \sqr{\sqrt{5} + 1} } \approx 14{,}9\,\text{см}.
    \end{align*}
}
\solutionspace{120pt}

\tasknumber{16}%
\task{%
    (Задача-«гроб»: решать на обратной стороне) Квадрат со стороной $d = 2\,\text{см}$ расположен так,
    что 2 его стороны параллельны главной оптической оси собирающей линзы,
    его центр удален на $h = 4\,\text{см}$ от этой оси и на $a = 10\,\text{см}$ от плоскости линзы.
    Определите площадь изображения квадрата, если фокусное расстояние линзы составляет $F = 25\,\text{см}$.
    % (и сравните с площадью объекта, умноженной на квадрат увеличения центра квадрата).
}
\answer{%
    \begin{align*}
    &\text{Все явные вычисления — в см и $\text{см}^2$,} \\
    \frac 1 F &= \frac 1{a + \frac d2} + \frac 1b \implies b = \frac 1{\frac 1 F - \frac 1{a + \frac d2}} = \frac{F(a + \frac d2)}{a + \frac d2 - F} = -\frac{275}{14}, \\
    \frac 1 F &= \frac 1{a - \frac d2} + \frac 1c \implies c = \frac 1{\frac 1 F - \frac 1{a - \frac d2}} = \frac{F(a - \frac d2)}{a - \frac d2 - F} = -\frac{225}{16}, \\
    c - b &= \frac{F(a - \frac d2)}{a - \frac d2 - F} - \frac{F(a + \frac d2)}{a + \frac d2 - F} = F\cbr{ \frac{a - \frac d2}{a - \frac d2 - F} - \frac{a + \frac d2}{a + \frac d2 - F} } =  \\
    &= F \cdot \frac{a^2 + \frac {ad}2 - aF - \frac{ad}2 - \frac{d^2}4 + \frac{dF}2 - a^2 + \frac {ad}2 + aF - \frac{ad}2 + \frac{d^2}4 + \frac{dF}2}{\cbr{a + \frac d2 - F}\cbr{a - \frac d2 - F}}= F \cdot \frac {dF}{\cbr{a + \frac d2 - F}\cbr{a - \frac d2 - F}} = \frac{625}{112}.
    \\
    \Gamma_b &= \frac b{a + \frac d2} = \frac{ F }{a + \frac d2 - F} = -\frac{25}{14}, \\
    \Gamma_c &= \frac c{a - \frac d2} = \frac{ F }{a - \frac d2 - F} = -\frac{25}{16}, \\
    &\text{ тут интересно отметить, что } \Gamma_x = \frac{ c - b}{ d } = \frac{ F^2 }{\cbr{a + \frac d2 - F}\cbr{a - \frac d2 - F}} \ne \Gamma_b \text{ или } \Gamma_c \text{ даже при малых $d$}.
    \\
    S' &= \frac{d \cdot \Gamma_b + d \cdot \Gamma_c}2 \cdot (c - b) = \frac d2 \cbr{\frac{ F }{a + \frac d2 - F} + \frac{ F }{a - \frac d2 - F}} \cdot \cbr{c - b} =  \\
    &=\frac {dF}2 \cbr{\frac 1{a + \frac d2 - F} + \frac 1{a - \frac d2 - F}} \cdot \frac {dF^2}{\cbr{a + \frac d2 - F}\cbr{a - \frac d2 - F}} =  \\
    &=\frac {dF}2 \cdot \frac{a - \frac d2 - F + a + \frac d2 - F}{\cbr{a + \frac d2 - F}\cbr{a - \frac d2 - F}} \cdot \frac {dF^2}{\cbr{a + \frac d2 - F}\cbr{a - \frac d2 - F}} =  \\
    &= \frac {d^2F^3}{2\sqr{a + \frac d2 - F}\sqr{a - \frac d2 - F}} \cdot (2a - 2F) = \frac {d^2F^3(a - F)}{ \sqr{\sqr{a - F} - \frac{d^2}4} } = -\frac{234375}{12544}.
    \end{align*}
}

\variantsplitter

\addpersonalvariant{София Журавлёва}

\tasknumber{1}%
\task{%
    Запишите известные вам виды классификации изображений.
}
\solutionspace{60pt}

\tasknumber{2}%
\task{%
    В каких линзах можно получить действительное изображение объекта?
}
\answer{%
    $\text{ собирающие }$
}
\solutionspace{40pt}

\tasknumber{3}%
\task{%
    Какое изображение называют действительным?
}
\solutionspace{40pt}

\tasknumber{4}%
\task{%
    Есть две линзы, обозначим их 1 и 2.
    Известно что фокусное расстояние линзы 1 меньше, чем у линзы 2.
    Какая линза сильнее преломляет лучи?
}
\answer{%
    $1$
}
\solutionspace{40pt}

\tasknumber{5}%
\task{%
    Предмет находится на расстоянии $20\,\text{см}$ от рассеивающей линзы с фокусным расстоянием $8\,\text{см}$.
    Определите тип изображения, расстояние между предметом и его изображением, увеличение предмета.
    Сделайте схематичный рисунок (не обязательно в масштабе, но с сохранением свойств линзы и изображения).
}
\solutionspace{100pt}

\tasknumber{6}%
\task{%
    Объект находится на расстоянии $115\,\text{см}$ от линзы, а его действительное изображение — в $30\,\text{см}$ от неё.
    Определите увеличение предмета, фокусное расстояние линзы, оптическую силу линзы и её тип.
}
\solutionspace{80pt}

\tasknumber{7}%
\task{%
    Известно, что из формулы тонкой линзы $\cbr{\frac 1F = \frac 1a + \frac 1b}$
    и определения увеличения $\cbr{\Gamma_y = \frac ba}$ можно получить выражение
    для увеличения: $\Gamma_y = \frac {aF}{a - F} \cdot \frac 1a = \frac {F}{a - F}.$
    Назовём такое увеличение «поперечным»: поперёк главной оптической оси (поэтому и ${}_y$).
    Получите формулу для «продольного» увеличения $\Gamma_x$ небольшого предмета, находящегося на главной оптической оси.
    Можно ли применить эту формулу для предмета, не лежащего на главной оптической оси, почему?
}
\answer{%
    \begin{align*}
    \frac 1F &= \frac 1a + \frac 1b \implies b = \frac {aF}{a - F} \\
    \frac 1F &= \frac 1{a + x} + \frac 1c \implies c = \frac {(a+x)F}{a + x - F} \\
    x' &= \abs{b - c} = \frac {aF}{a - F} - \frac {(a+x)F}{a + x - F} = F\cbr{\frac {a}{a - F} - \frac {a+x}{a + x - F}} =  \\
    &= F \cdot \frac {a^2 + ax - aF - a^2 - ax + aF + xF}{(a - F)(a + x - F)} = F \cdot \frac {xF}{(a - F)(a + x - F)} \\
    \Gamma_x &= \frac{x'}x = \frac{F^2}{(a - F)(a + x - F)} \to \frac{F^2}{\sqr{a - F}}.
    \\
    &\text{Нельзя: изображение по-разному растянет по осям $x$ и $y$ и понадобится теорема Пифагора}
    \end{align*}
}
\solutionspace{150pt}

\tasknumber{8}%
\task{%
    Доказать формулу тонкой линзы для собирающей линзы.
}
\solutionspace{120pt}

\tasknumber{9}%
\task{%
    Постройте ход луча $CK$ в тонкой линзе.
    Известно положение линзы и оба её фокуса (см.
    рис.
    на доске).
    Рассмотрите оба типа линзы, сделав 2 рисунка: собирающую и рассеивающую.
}
\solutionspace{120pt}

\tasknumber{10}%
\task{%
    На экране, расположенном иа расстоянии $120\,\text{см}$ от собирающей линзы,
    получено изображение точечного источника, расположенного на главной оптической оси линзы.
    На какое расстояние переместится изображение на экране,
    если при неподвижном источнике переместить линзу на $3\,\text{см}$ в плоскости, перпендикулярной главной оптической оси?
    Фокусное расстояние линзы равно $30\,\text{см}$.
}
\answer{%
    \begin{align*}
    &\frac 1F = \frac 1a + \frac 1b \implies a = \frac{bF}{b-F} \implies \Gamma = \frac ba = \frac{b-F}F \\
    &y = x \cdot \Gamma = x \cdot \frac{b-F}F \implies d = x + y = 12\,\text{см}.
    \end{align*}
}
\solutionspace{120pt}

\tasknumber{11}%
\task{%
    Оптическая сила двояковыпуклой линзы в воздухе $5\,\text{дптр}$, а в воде $1{,}5\,\text{дптр}$.
    Определить показатель преломления $n$ материала, из которого изготовлена линза.
    Показатель преломления воды равен $1{,}33$.
}
\answer{%
    \begin{align*}
    D_1 &=\cbr{\frac n{n_1} - 1}\cbr{\frac 1{R_1} + \frac 1{R_2}}, \\
    D_2 &=\cbr{\frac n{n_2} - 1}\cbr{\frac 1{R_1} + \frac 1{R_2}}, \\
    \frac {D_2}{D_1} &=\frac{\frac n{n_2} - 1}{\frac n{n_1} - 1} \implies {D_2}\cbr{\frac n{n_1} - 1} = {D_1}\cbr{\frac n{n_2} - 1}  \implies n\cbr{\frac{D_2}{n_1} - \frac{D_1}{n_2}} = D_2 - D_1, \\
    n &= \frac{D_2 - D_1}{\frac{D_2}{n_1} - \frac{D_1}{n_2}} = \frac{n_1 n_2 (D_2 - D_1)}{D_2n_2 - D_1n_1} \approx 1{,}549.
    \end{align*}
}
\solutionspace{120pt}

\tasknumber{12}%
\task{%
    На каком расстоянии от собирающей линзы с фокусным расстоянием $30\,\text{дптр}$
    следует надо поместить предмет, чтобы расстояние
    от предмета до его действительного изображения было наименьшим?
}
\answer{%
    \begin{align*}
    \frac 1a &+ \frac 1b = D \implies b = \frac 1{D - \frac 1a} \implies \ell = a + b = a + \frac a{Da - 1} = \frac{ Da^2 }{Da - 1} \implies \\
    \implies \ell'_a &= \frac{ 2Da \cdot (Da - 1) - Da^2 \cdot D }{\sqr{Da - 1}}= \frac{ D^2a^2 - 2Da}{\sqr{Da - 1}} = \frac{ Da(Da - 2)}{\sqr{Da - 1}}\implies a_{\min} = \frac 2D \approx 66{,}7\,\text{мм}.
    \end{align*}
}
\solutionspace{120pt}

\tasknumber{13}%
\task{%
    Даны точечный источник света $S$, его изображение $S_1$, полученное с помощью собирающей линзы,
    и ближайший к источнику фокус линзы $F$ (см.
    рис.
    на доске).
    Расстояния $SF = \ell$ и $SS_1 = L$.
    Определить положение линзы и её фокусное расстояние.
}
\answer{%
    \begin{align*}
    \frac 1a + \frac 1b &= \frac 1F, \ell = a - F, L = a + b \implies a = \ell + F, b = L - a = L - \ell - F \\
    \frac 1{\ell + F} + \frac 1{L - \ell - F} &= \frac 1F \\
    F\ell + F^2 + LF - F\ell - F^2 &= L\ell - \ell^2 - F\ell + LF - F\ell - F^2 \\
    0 &= L\ell - \ell^2 - 2F\ell - F^2 \\
    0 &=  F^2 + 2F\ell - L\ell + \ell^2 \\
    F &= -\ell \pm \sqrt{\ell^2 +  L\ell - \ell^2} = -\ell \pm \sqrt{L\ell} \implies F = \sqrt{L\ell} - \ell \\
    a &= \ell + F = \ell + \sqrt{L\ell} - \ell = \sqrt{L\ell}.
    \end{align*}
}
\solutionspace{120pt}

\tasknumber{14}%
\task{%
    Расстояние от освещённого предмета до экрана $100\,\text{см}$.
    Линза, помещенная между ними, даёт чёткое изображение предмета на
    экране при двух положениях, расстояние между которыми $40\,\text{см}$.
    Найти фокусное расстояние линзы.
}
\answer{%
    \begin{align*}
    \frac 1a + \frac 1b &= \frac 1F, \frac 1{a-\ell} + \frac 1{b+\ell} = \frac 1F, a + b = L \\
    \frac 1a + \frac 1b &= \frac 1{a-\ell} + \frac 1{b+\ell}\implies \frac{a + b}{ab} = \frac{(a-\ell) + (b+\ell)}{(a-\ell)(b+\ell)} \\
    ab  &= (a - \ell)(b+\ell) \implies 0  = -b\ell + a\ell - \ell^2 \implies 0 = -b + a - \ell \implies b = a - \ell \\
    a + (a - \ell) &= L \implies a = \frac{L + \ell}2 \implies b = \frac{L - \ell}2 \\
    F &= \frac{ab}{a + b} = \frac{L^2 -\ell^2}{4L} \approx 21\,\text{см}.
    \end{align*}
}
\solutionspace{120pt}

\tasknumber{15}%
\task{%
    Предмет находится на расстоянии $90\,\text{см}$ от экрана.
    Между предметом и экраном помещают линзу, причём при одном
    положении линзы на экране получается увеличенное изображение предмета,
    а при другом — уменьшенное.
    Каково фокусное расстояние линзы, если
    линейные размеры первого изображения в пять раз больше второго?
}
\answer{%
    \begin{align*}
    \frac 1a + \frac 1{L-a} &= \frac 1F, h_1 = h \cdot \frac{L-a}a, \\
    \frac 1b + \frac 1{L-b} &= \frac 1F, h_2 = h \cdot \frac{L-b}b, \\
    \frac{h_1}{h_2} &= 5 \implies \frac{(L-a)b}{(L-b)a} = 5, \\
    \frac 1F &= \frac{ L }{a(L-a)} = \frac{ L }{b(L-b)} \implies \frac{L-a}{L-b} = \frac b a \implies \frac {b^2}{a^2} = 5.
    \\
    \frac 1a + \frac 1{L-a} &= \frac 1b + \frac 1{L-b} \implies \frac L{a(L-a)} = \frac L{b(L-b)} \implies \\
    \implies aL - a^2 &= bL - b^2 \implies (a-b)L = (a-b)(a+b) \implies b = L - a, \\
    \frac{\sqr{L-a}}{a^2} &= 5 \implies \frac La - 1 = \sqrt{5} \implies a = \frac{ L }{\sqrt{5} + 1} \\
    F &= \frac{a(L-a)}L = \frac 1L \cdot \frac L{\sqrt{5} + 1} \cdot \frac {L\sqrt{5}}{\sqrt{5} + 1}= \frac { L\sqrt{5} }{ \sqr{\sqrt{5} + 1} } \approx 19{,}2\,\text{см}.
    \end{align*}
}
\solutionspace{120pt}

\tasknumber{16}%
\task{%
    (Задача-«гроб»: решать на обратной стороне) Квадрат со стороной $d = 1\,\text{см}$ расположен так,
    что 2 его стороны параллельны главной оптической оси рассеивающей линзы,
    его центр удален на $h = 6\,\text{см}$ от этой оси и на $a = 15\,\text{см}$ от плоскости линзы.
    Определите площадь изображения квадрата, если фокусное расстояние линзы составляет $F = 25\,\text{см}$.
    % (и сравните с площадью объекта, умноженной на квадрат увеличения центра квадрата).
}
\answer{%
    \begin{align*}
    &\text{Все явные вычисления — в см и $\text{см}^2$,} \\
    \frac 1 F &= \frac 1{a + \frac d2} + \frac 1b \implies b = \frac 1{\frac 1 F - \frac 1{a + \frac d2}} = \frac{F(a + \frac d2)}{a + \frac d2 - F} = -\frac{775}{81}, \\
    \frac 1 F &= \frac 1{a - \frac d2} + \frac 1c \implies c = \frac 1{\frac 1 F - \frac 1{a - \frac d2}} = \frac{F(a - \frac d2)}{a - \frac d2 - F} = -\frac{725}{79}, \\
    c - b &= \frac{F(a - \frac d2)}{a - \frac d2 - F} - \frac{F(a + \frac d2)}{a + \frac d2 - F} = F\cbr{ \frac{a - \frac d2}{a - \frac d2 - F} - \frac{a + \frac d2}{a + \frac d2 - F} } =  \\
    &= F \cdot \frac{a^2 + \frac {ad}2 - aF - \frac{ad}2 - \frac{d^2}4 + \frac{dF}2 - a^2 + \frac {ad}2 + aF - \frac{ad}2 + \frac{d^2}4 + \frac{dF}2}{\cbr{a + \frac d2 - F}\cbr{a - \frac d2 - F}}= F \cdot \frac {dF}{\cbr{a + \frac d2 - F}\cbr{a - \frac d2 - F}} = \frac{2500}{6399}.
    \\
    \Gamma_b &= \frac b{a + \frac d2} = \frac{ F }{a + \frac d2 - F} = -\frac{50}{81}, \\
    \Gamma_c &= \frac c{a - \frac d2} = \frac{ F }{a - \frac d2 - F} = -\frac{50}{79}, \\
    &\text{ тут интересно отметить, что } \Gamma_x = \frac{ c - b}{ d } = \frac{ F^2 }{\cbr{a + \frac d2 - F}\cbr{a - \frac d2 - F}} \ne \Gamma_b \text{ или } \Gamma_c \text{ даже при малых $d$}.
    \\
    S' &= \frac{d \cdot \Gamma_b + d \cdot \Gamma_c}2 \cdot (c - b) = \frac d2 \cbr{\frac{ F }{a + \frac d2 - F} + \frac{ F }{a - \frac d2 - F}} \cdot \cbr{c - b} =  \\
    &=\frac {dF}2 \cbr{\frac 1{a + \frac d2 - F} + \frac 1{a - \frac d2 - F}} \cdot \frac {dF^2}{\cbr{a + \frac d2 - F}\cbr{a - \frac d2 - F}} =  \\
    &=\frac {dF}2 \cdot \frac{a - \frac d2 - F + a + \frac d2 - F}{\cbr{a + \frac d2 - F}\cbr{a - \frac d2 - F}} \cdot \frac {dF^2}{\cbr{a + \frac d2 - F}\cbr{a - \frac d2 - F}} =  \\
    &= \frac {d^2F^3}{2\sqr{a + \frac d2 - F}\sqr{a - \frac d2 - F}} \cdot (2a - 2F) = \frac {d^2F^3(a - F)}{ \sqr{\sqr{a - F} - \frac{d^2}4} } = -\frac{10000000}{40947201}.
    \end{align*}
}

\variantsplitter

\addpersonalvariant{Константин Козлов}

\tasknumber{1}%
\task{%
    Запишите формулу тонкой линзы и сделайте рисунок, указав на нём физические величины из этой формулы.
}
\solutionspace{60pt}

\tasknumber{2}%
\task{%
    В каких линзах можно получить действительное изображение объекта?
}
\answer{%
    $\text{ собирающие }$
}
\solutionspace{40pt}

\tasknumber{3}%
\task{%
    Какое изображение называют действительным?
}
\solutionspace{40pt}

\tasknumber{4}%
\task{%
    Есть две линзы, обозначим их 1 и 2.
    Известно что фокусное расстояние линзы 1 больше, чем у линзы 2.
    Какая линза сильнее преломляет лучи?
}
\answer{%
    $2$
}
\solutionspace{40pt}

\tasknumber{5}%
\task{%
    Предмет находится на расстоянии $30\,\text{см}$ от рассеивающей линзы с фокусным расстоянием $25\,\text{см}$.
    Определите тип изображения, расстояние между предметом и его изображением, увеличение предмета.
    Сделайте схематичный рисунок (не обязательно в масштабе, но с сохранением свойств линзы и изображения).
}
\solutionspace{100pt}

\tasknumber{6}%
\task{%
    Объект находится на расстоянии $25\,\text{см}$ от линзы, а его мнимое изображение — в $10\,\text{см}$ от неё.
    Определите увеличение предмета, фокусное расстояние линзы, оптическую силу линзы и её тип.
}
\solutionspace{80pt}

\tasknumber{7}%
\task{%
    Известно, что из формулы тонкой линзы $\cbr{\frac 1F = \frac 1a + \frac 1b}$
    и определения увеличения $\cbr{\Gamma_y = \frac ba}$ можно получить выражение
    для увеличения: $\Gamma_y = \frac {aF}{a - F} \cdot \frac 1a = \frac {F}{a - F}.$
    Назовём такое увеличение «поперечным»: поперёк главной оптической оси (поэтому и ${}_y$).
    Получите формулу для «продольного» увеличения $\Gamma_x$ небольшого предмета, находящегося на главной оптической оси.
    Можно ли применить эту формулу для предмета, не лежащего на главной оптической оси, почему?
}
\answer{%
    \begin{align*}
    \frac 1F &= \frac 1a + \frac 1b \implies b = \frac {aF}{a - F} \\
    \frac 1F &= \frac 1{a + x} + \frac 1c \implies c = \frac {(a+x)F}{a + x - F} \\
    x' &= \abs{b - c} = \frac {aF}{a - F} - \frac {(a+x)F}{a + x - F} = F\cbr{\frac {a}{a - F} - \frac {a+x}{a + x - F}} =  \\
    &= F \cdot \frac {a^2 + ax - aF - a^2 - ax + aF + xF}{(a - F)(a + x - F)} = F \cdot \frac {xF}{(a - F)(a + x - F)} \\
    \Gamma_x &= \frac{x'}x = \frac{F^2}{(a - F)(a + x - F)} \to \frac{F^2}{\sqr{a - F}}.
    \\
    &\text{Нельзя: изображение по-разному растянет по осям $x$ и $y$ и понадобится теорема Пифагора}
    \end{align*}
}
\solutionspace{150pt}

\tasknumber{8}%
\task{%
    Доказать формулу тонкой линзы для собирающей линзы.
}
\solutionspace{120pt}

\tasknumber{9}%
\task{%
    Постройте ход луча $CM$ в тонкой линзе.
    Известно положение линзы и оба её фокуса (см.
    рис.
    на доске).
    Рассмотрите оба типа линзы, сделав 2 рисунка: собирающую и рассеивающую.
}
\solutionspace{120pt}

\tasknumber{10}%
\task{%
    На экране, расположенном иа расстоянии $60\,\text{см}$ от собирающей линзы,
    получено изображение точечного источника, расположенного на главной оптической оси линзы.
    На какое расстояние переместится изображение на экране,
    если при неподвижном источнике переместить линзу на $3\,\text{см}$ в плоскости, перпендикулярной главной оптической оси?
    Фокусное расстояние линзы равно $30\,\text{см}$.
}
\answer{%
    \begin{align*}
    &\frac 1F = \frac 1a + \frac 1b \implies a = \frac{bF}{b-F} \implies \Gamma = \frac ba = \frac{b-F}F \\
    &y = x \cdot \Gamma = x \cdot \frac{b-F}F \implies d = x + y = 6\,\text{см}.
    \end{align*}
}
\solutionspace{120pt}

\tasknumber{11}%
\task{%
    Оптическая сила двояковыпуклой линзы в воздухе $5{,}5\,\text{дптр}$, а в воде $1{,}5\,\text{дптр}$.
    Определить показатель преломления $n$ материала, из которого изготовлена линза.
    Показатель преломления воды равен $1{,}33$.
}
\answer{%
    \begin{align*}
    D_1 &=\cbr{\frac n{n_1} - 1}\cbr{\frac 1{R_1} + \frac 1{R_2}}, \\
    D_2 &=\cbr{\frac n{n_2} - 1}\cbr{\frac 1{R_1} + \frac 1{R_2}}, \\
    \frac {D_2}{D_1} &=\frac{\frac n{n_2} - 1}{\frac n{n_1} - 1} \implies {D_2}\cbr{\frac n{n_1} - 1} = {D_1}\cbr{\frac n{n_2} - 1}  \implies n\cbr{\frac{D_2}{n_1} - \frac{D_1}{n_2}} = D_2 - D_1, \\
    n &= \frac{D_2 - D_1}{\frac{D_2}{n_1} - \frac{D_1}{n_2}} = \frac{n_1 n_2 (D_2 - D_1)}{D_2n_2 - D_1n_1} \approx 1{,}518.
    \end{align*}
}
\solutionspace{120pt}

\tasknumber{12}%
\task{%
    На каком расстоянии от собирающей линзы с фокусным расстоянием $30\,\text{дптр}$
    следует надо поместить предмет, чтобы расстояние
    от предмета до его действительного изображения было наименьшим?
}
\answer{%
    \begin{align*}
    \frac 1a &+ \frac 1b = D \implies b = \frac 1{D - \frac 1a} \implies \ell = a + b = a + \frac a{Da - 1} = \frac{ Da^2 }{Da - 1} \implies \\
    \implies \ell'_a &= \frac{ 2Da \cdot (Da - 1) - Da^2 \cdot D }{\sqr{Da - 1}}= \frac{ D^2a^2 - 2Da}{\sqr{Da - 1}} = \frac{ Da(Da - 2)}{\sqr{Da - 1}}\implies a_{\min} = \frac 2D \approx 66{,}7\,\text{мм}.
    \end{align*}
}
\solutionspace{120pt}

\tasknumber{13}%
\task{%
    Даны точечный источник света $S$, его изображение $S_1$, полученное с помощью собирающей линзы,
    и ближайший к источнику фокус линзы $F$ (см.
    рис.
    на доске).
    Расстояния $SF = \ell$ и $SS_1 = L$.
    Определить положение линзы и её фокусное расстояние.
}
\answer{%
    \begin{align*}
    \frac 1a + \frac 1b &= \frac 1F, \ell = a - F, L = a + b \implies a = \ell + F, b = L - a = L - \ell - F \\
    \frac 1{\ell + F} + \frac 1{L - \ell - F} &= \frac 1F \\
    F\ell + F^2 + LF - F\ell - F^2 &= L\ell - \ell^2 - F\ell + LF - F\ell - F^2 \\
    0 &= L\ell - \ell^2 - 2F\ell - F^2 \\
    0 &=  F^2 + 2F\ell - L\ell + \ell^2 \\
    F &= -\ell \pm \sqrt{\ell^2 +  L\ell - \ell^2} = -\ell \pm \sqrt{L\ell} \implies F = \sqrt{L\ell} - \ell \\
    a &= \ell + F = \ell + \sqrt{L\ell} - \ell = \sqrt{L\ell}.
    \end{align*}
}
\solutionspace{120pt}

\tasknumber{14}%
\task{%
    Расстояние от освещённого предмета до экрана $100\,\text{см}$.
    Линза, помещенная между ними, даёт чёткое изображение предмета на
    экране при двух положениях, расстояние между которыми $40\,\text{см}$.
    Найти фокусное расстояние линзы.
}
\answer{%
    \begin{align*}
    \frac 1a + \frac 1b &= \frac 1F, \frac 1{a-\ell} + \frac 1{b+\ell} = \frac 1F, a + b = L \\
    \frac 1a + \frac 1b &= \frac 1{a-\ell} + \frac 1{b+\ell}\implies \frac{a + b}{ab} = \frac{(a-\ell) + (b+\ell)}{(a-\ell)(b+\ell)} \\
    ab  &= (a - \ell)(b+\ell) \implies 0  = -b\ell + a\ell - \ell^2 \implies 0 = -b + a - \ell \implies b = a - \ell \\
    a + (a - \ell) &= L \implies a = \frac{L + \ell}2 \implies b = \frac{L - \ell}2 \\
    F &= \frac{ab}{a + b} = \frac{L^2 -\ell^2}{4L} \approx 21\,\text{см}.
    \end{align*}
}
\solutionspace{120pt}

\tasknumber{15}%
\task{%
    Предмет находится на расстоянии $60\,\text{см}$ от экрана.
    Между предметом и экраном помещают линзу, причём при одном
    положении линзы на экране получается увеличенное изображение предмета,
    а при другом — уменьшенное.
    Каково фокусное расстояние линзы, если
    линейные размеры первого изображения в пять раз больше второго?
}
\answer{%
    \begin{align*}
    \frac 1a + \frac 1{L-a} &= \frac 1F, h_1 = h \cdot \frac{L-a}a, \\
    \frac 1b + \frac 1{L-b} &= \frac 1F, h_2 = h \cdot \frac{L-b}b, \\
    \frac{h_1}{h_2} &= 5 \implies \frac{(L-a)b}{(L-b)a} = 5, \\
    \frac 1F &= \frac{ L }{a(L-a)} = \frac{ L }{b(L-b)} \implies \frac{L-a}{L-b} = \frac b a \implies \frac {b^2}{a^2} = 5.
    \\
    \frac 1a + \frac 1{L-a} &= \frac 1b + \frac 1{L-b} \implies \frac L{a(L-a)} = \frac L{b(L-b)} \implies \\
    \implies aL - a^2 &= bL - b^2 \implies (a-b)L = (a-b)(a+b) \implies b = L - a, \\
    \frac{\sqr{L-a}}{a^2} &= 5 \implies \frac La - 1 = \sqrt{5} \implies a = \frac{ L }{\sqrt{5} + 1} \\
    F &= \frac{a(L-a)}L = \frac 1L \cdot \frac L{\sqrt{5} + 1} \cdot \frac {L\sqrt{5}}{\sqrt{5} + 1}= \frac { L\sqrt{5} }{ \sqr{\sqrt{5} + 1} } \approx 12{,}8\,\text{см}.
    \end{align*}
}
\solutionspace{120pt}

\tasknumber{16}%
\task{%
    (Задача-«гроб»: решать на обратной стороне) Квадрат со стороной $d = 3\,\text{см}$ расположен так,
    что 2 его стороны параллельны главной оптической оси рассеивающей линзы,
    его центр удален на $h = 6\,\text{см}$ от этой оси и на $a = 12\,\text{см}$ от плоскости линзы.
    Определите площадь изображения квадрата, если фокусное расстояние линзы составляет $F = 25\,\text{см}$.
    % (и сравните с площадью объекта, умноженной на квадрат увеличения центра квадрата).
}
\answer{%
    \begin{align*}
    &\text{Все явные вычисления — в см и $\text{см}^2$,} \\
    \frac 1 F &= \frac 1{a + \frac d2} + \frac 1b \implies b = \frac 1{\frac 1 F - \frac 1{a + \frac d2}} = \frac{F(a + \frac d2)}{a + \frac d2 - F} = -\frac{675}{77}, \\
    \frac 1 F &= \frac 1{a - \frac d2} + \frac 1c \implies c = \frac 1{\frac 1 F - \frac 1{a - \frac d2}} = \frac{F(a - \frac d2)}{a - \frac d2 - F} = -\frac{525}{71}, \\
    c - b &= \frac{F(a - \frac d2)}{a - \frac d2 - F} - \frac{F(a + \frac d2)}{a + \frac d2 - F} = F\cbr{ \frac{a - \frac d2}{a - \frac d2 - F} - \frac{a + \frac d2}{a + \frac d2 - F} } =  \\
    &= F \cdot \frac{a^2 + \frac {ad}2 - aF - \frac{ad}2 - \frac{d^2}4 + \frac{dF}2 - a^2 + \frac {ad}2 + aF - \frac{ad}2 + \frac{d^2}4 + \frac{dF}2}{\cbr{a + \frac d2 - F}\cbr{a - \frac d2 - F}}= F \cdot \frac {dF}{\cbr{a + \frac d2 - F}\cbr{a - \frac d2 - F}} = \frac{7500}{5467}.
    \\
    \Gamma_b &= \frac b{a + \frac d2} = \frac{ F }{a + \frac d2 - F} = -\frac{50}{77}, \\
    \Gamma_c &= \frac c{a - \frac d2} = \frac{ F }{a - \frac d2 - F} = -\frac{50}{71}, \\
    &\text{ тут интересно отметить, что } \Gamma_x = \frac{ c - b}{ d } = \frac{ F^2 }{\cbr{a + \frac d2 - F}\cbr{a - \frac d2 - F}} \ne \Gamma_b \text{ или } \Gamma_c \text{ даже при малых $d$}.
    \\
    S' &= \frac{d \cdot \Gamma_b + d \cdot \Gamma_c}2 \cdot (c - b) = \frac d2 \cbr{\frac{ F }{a + \frac d2 - F} + \frac{ F }{a - \frac d2 - F}} \cdot \cbr{c - b} =  \\
    &=\frac {dF}2 \cbr{\frac 1{a + \frac d2 - F} + \frac 1{a - \frac d2 - F}} \cdot \frac {dF^2}{\cbr{a + \frac d2 - F}\cbr{a - \frac d2 - F}} =  \\
    &=\frac {dF}2 \cdot \frac{a - \frac d2 - F + a + \frac d2 - F}{\cbr{a + \frac d2 - F}\cbr{a - \frac d2 - F}} \cdot \frac {dF^2}{\cbr{a + \frac d2 - F}\cbr{a - \frac d2 - F}} =  \\
    &= \frac {d^2F^3}{2\sqr{a + \frac d2 - F}\sqr{a - \frac d2 - F}} \cdot (2a - 2F) = \frac {d^2F^3(a - F)}{ \sqr{\sqr{a - F} - \frac{d^2}4} } = -\frac{83250000}{29888089}.
    \end{align*}
}

\variantsplitter

\addpersonalvariant{Наталья Кравченко}

\tasknumber{1}%
\task{%
    Запишите формулу тонкой линзы и сделайте рисунок, указав на нём физические величины из этой формулы.
}
\solutionspace{60pt}

\tasknumber{2}%
\task{%
    В каких линзах можно получить увеличенное изображение объекта?
}
\answer{%
    $\text{ рассеивающие }$
}
\solutionspace{40pt}

\tasknumber{3}%
\task{%
    Какое изображение называют действительным?
}
\solutionspace{40pt}

\tasknumber{4}%
\task{%
    Есть две линзы, обозначим их 1 и 2.
    Известно что оптическая сила линзы 2 меньше, чем у линзы 1.
    Какая линза сильнее преломляет лучи?
}
\answer{%
    $1$
}
\solutionspace{40pt}

\tasknumber{5}%
\task{%
    Предмет находится на расстоянии $20\,\text{см}$ от рассеивающей линзы с фокусным расстоянием $25\,\text{см}$.
    Определите тип изображения, расстояние между предметом и его изображением, увеличение предмета.
    Сделайте схематичный рисунок (не обязательно в масштабе, но с сохранением свойств линзы и изображения).
}
\solutionspace{100pt}

\tasknumber{6}%
\task{%
    Объект находится на расстоянии $45\,\text{см}$ от линзы, а его действительное изображение — в $20\,\text{см}$ от неё.
    Определите увеличение предмета, фокусное расстояние линзы, оптическую силу линзы и её тип.
}
\solutionspace{80pt}

\tasknumber{7}%
\task{%
    Известно, что из формулы тонкой линзы $\cbr{\frac 1F = \frac 1a + \frac 1b}$
    и определения увеличения $\cbr{\Gamma_y = \frac ba}$ можно получить выражение
    для увеличения: $\Gamma_y = \frac {aF}{a - F} \cdot \frac 1a = \frac {F}{a - F}.$
    Назовём такое увеличение «поперечным»: поперёк главной оптической оси (поэтому и ${}_y$).
    Получите формулу для «продольного» увеличения $\Gamma_x$ небольшого предмета, находящегося на главной оптической оси.
    Можно ли применить эту формулу для предмета, не лежащего на главной оптической оси, почему?
}
\answer{%
    \begin{align*}
    \frac 1F &= \frac 1a + \frac 1b \implies b = \frac {aF}{a - F} \\
    \frac 1F &= \frac 1{a + x} + \frac 1c \implies c = \frac {(a+x)F}{a + x - F} \\
    x' &= \abs{b - c} = \frac {aF}{a - F} - \frac {(a+x)F}{a + x - F} = F\cbr{\frac {a}{a - F} - \frac {a+x}{a + x - F}} =  \\
    &= F \cdot \frac {a^2 + ax - aF - a^2 - ax + aF + xF}{(a - F)(a + x - F)} = F \cdot \frac {xF}{(a - F)(a + x - F)} \\
    \Gamma_x &= \frac{x'}x = \frac{F^2}{(a - F)(a + x - F)} \to \frac{F^2}{\sqr{a - F}}.
    \\
    &\text{Нельзя: изображение по-разному растянет по осям $x$ и $y$ и понадобится теорема Пифагора}
    \end{align*}
}
\solutionspace{150pt}

\tasknumber{8}%
\task{%
    Доказать формулу тонкой линзы для собирающей линзы.
}
\solutionspace{120pt}

\tasknumber{9}%
\task{%
    Постройте ход луча $AK$ в тонкой линзе.
    Известно положение линзы и оба её фокуса (см.
    рис.
    на доске).
    Рассмотрите оба типа линзы, сделав 2 рисунка: собирающую и рассеивающую.
}
\solutionspace{120pt}

\tasknumber{10}%
\task{%
    На экране, расположенном иа расстоянии $120\,\text{см}$ от собирающей линзы,
    получено изображение точечного источника, расположенного на главной оптической оси линзы.
    На какое расстояние переместится изображение на экране,
    если при неподвижной линзе переместить источник на $3\,\text{см}$ в плоскости, перпендикулярной главной оптической оси?
    Фокусное расстояние линзы равно $40\,\text{см}$.
}
\answer{%
    \begin{align*}
    &\frac 1F = \frac 1a + \frac 1b \implies a = \frac{bF}{b-F} \implies \Gamma = \frac ba = \frac{b-F}F \\
    &y = x \cdot \Gamma = x \cdot \frac{b-F}F \implies d = y = 6\,\text{см}.
    \end{align*}
}
\solutionspace{120pt}

\tasknumber{11}%
\task{%
    Оптическая сила двояковыпуклой линзы в воздухе $4{,}5\,\text{дптр}$, а в воде $1{,}5\,\text{дптр}$.
    Определить показатель преломления $n$ материала, из которого изготовлена линза.
    Показатель преломления воды равен $1{,}33$.
}
\answer{%
    \begin{align*}
    D_1 &=\cbr{\frac n{n_1} - 1}\cbr{\frac 1{R_1} + \frac 1{R_2}}, \\
    D_2 &=\cbr{\frac n{n_2} - 1}\cbr{\frac 1{R_1} + \frac 1{R_2}}, \\
    \frac {D_2}{D_1} &=\frac{\frac n{n_2} - 1}{\frac n{n_1} - 1} \implies {D_2}\cbr{\frac n{n_1} - 1} = {D_1}\cbr{\frac n{n_2} - 1}  \implies n\cbr{\frac{D_2}{n_1} - \frac{D_1}{n_2}} = D_2 - D_1, \\
    n &= \frac{D_2 - D_1}{\frac{D_2}{n_1} - \frac{D_1}{n_2}} = \frac{n_1 n_2 (D_2 - D_1)}{D_2n_2 - D_1n_1} \approx 1{,}593.
    \end{align*}
}
\solutionspace{120pt}

\tasknumber{12}%
\task{%
    На каком расстоянии от собирающей линзы с фокусным расстоянием $40\,\text{дптр}$
    следует надо поместить предмет, чтобы расстояние
    от предмета до его действительного изображения было наименьшим?
}
\answer{%
    \begin{align*}
    \frac 1a &+ \frac 1b = D \implies b = \frac 1{D - \frac 1a} \implies \ell = a + b = a + \frac a{Da - 1} = \frac{ Da^2 }{Da - 1} \implies \\
    \implies \ell'_a &= \frac{ 2Da \cdot (Da - 1) - Da^2 \cdot D }{\sqr{Da - 1}}= \frac{ D^2a^2 - 2Da}{\sqr{Da - 1}} = \frac{ Da(Da - 2)}{\sqr{Da - 1}}\implies a_{\min} = \frac 2D \approx 50\,\text{мм}.
    \end{align*}
}
\solutionspace{120pt}

\tasknumber{13}%
\task{%
    Даны точечный источник света $S$, его изображение $S_1$, полученное с помощью собирающей линзы,
    и ближайший к источнику фокус линзы $F$ (см.
    рис.
    на доске).
    Расстояния $SF = \ell$ и $SS_1 = L$.
    Определить положение линзы и её фокусное расстояние.
}
\answer{%
    \begin{align*}
    \frac 1a + \frac 1b &= \frac 1F, \ell = a - F, L = a + b \implies a = \ell + F, b = L - a = L - \ell - F \\
    \frac 1{\ell + F} + \frac 1{L - \ell - F} &= \frac 1F \\
    F\ell + F^2 + LF - F\ell - F^2 &= L\ell - \ell^2 - F\ell + LF - F\ell - F^2 \\
    0 &= L\ell - \ell^2 - 2F\ell - F^2 \\
    0 &=  F^2 + 2F\ell - L\ell + \ell^2 \\
    F &= -\ell \pm \sqrt{\ell^2 +  L\ell - \ell^2} = -\ell \pm \sqrt{L\ell} \implies F = \sqrt{L\ell} - \ell \\
    a &= \ell + F = \ell + \sqrt{L\ell} - \ell = \sqrt{L\ell}.
    \end{align*}
}
\solutionspace{120pt}

\tasknumber{14}%
\task{%
    Расстояние от освещённого предмета до экрана $80\,\text{см}$.
    Линза, помещенная между ними, даёт чёткое изображение предмета на
    экране при двух положениях, расстояние между которыми $40\,\text{см}$.
    Найти фокусное расстояние линзы.
}
\answer{%
    \begin{align*}
    \frac 1a + \frac 1b &= \frac 1F, \frac 1{a-\ell} + \frac 1{b+\ell} = \frac 1F, a + b = L \\
    \frac 1a + \frac 1b &= \frac 1{a-\ell} + \frac 1{b+\ell}\implies \frac{a + b}{ab} = \frac{(a-\ell) + (b+\ell)}{(a-\ell)(b+\ell)} \\
    ab  &= (a - \ell)(b+\ell) \implies 0  = -b\ell + a\ell - \ell^2 \implies 0 = -b + a - \ell \implies b = a - \ell \\
    a + (a - \ell) &= L \implies a = \frac{L + \ell}2 \implies b = \frac{L - \ell}2 \\
    F &= \frac{ab}{a + b} = \frac{L^2 -\ell^2}{4L} \approx 15\,\text{см}.
    \end{align*}
}
\solutionspace{120pt}

\tasknumber{15}%
\task{%
    Предмет находится на расстоянии $60\,\text{см}$ от экрана.
    Между предметом и экраном помещают линзу, причём при одном
    положении линзы на экране получается увеличенное изображение предмета,
    а при другом — уменьшенное.
    Каково фокусное расстояние линзы, если
    линейные размеры первого изображения в два раза больше второго?
}
\answer{%
    \begin{align*}
    \frac 1a + \frac 1{L-a} &= \frac 1F, h_1 = h \cdot \frac{L-a}a, \\
    \frac 1b + \frac 1{L-b} &= \frac 1F, h_2 = h \cdot \frac{L-b}b, \\
    \frac{h_1}{h_2} &= 2 \implies \frac{(L-a)b}{(L-b)a} = 2, \\
    \frac 1F &= \frac{ L }{a(L-a)} = \frac{ L }{b(L-b)} \implies \frac{L-a}{L-b} = \frac b a \implies \frac {b^2}{a^2} = 2.
    \\
    \frac 1a + \frac 1{L-a} &= \frac 1b + \frac 1{L-b} \implies \frac L{a(L-a)} = \frac L{b(L-b)} \implies \\
    \implies aL - a^2 &= bL - b^2 \implies (a-b)L = (a-b)(a+b) \implies b = L - a, \\
    \frac{\sqr{L-a}}{a^2} &= 2 \implies \frac La - 1 = \sqrt{2} \implies a = \frac{ L }{\sqrt{2} + 1} \\
    F &= \frac{a(L-a)}L = \frac 1L \cdot \frac L{\sqrt{2} + 1} \cdot \frac {L\sqrt{2}}{\sqrt{2} + 1}= \frac { L\sqrt{2} }{ \sqr{\sqrt{2} + 1} } \approx 14{,}6\,\text{см}.
    \end{align*}
}
\solutionspace{120pt}

\tasknumber{16}%
\task{%
    (Задача-«гроб»: решать на обратной стороне) Квадрат со стороной $d = 1\,\text{см}$ расположен так,
    что 2 его стороны параллельны главной оптической оси рассеивающей линзы,
    его центр удален на $h = 6\,\text{см}$ от этой оси и на $a = 12\,\text{см}$ от плоскости линзы.
    Определите площадь изображения квадрата, если фокусное расстояние линзы составляет $F = 25\,\text{см}$.
    % (и сравните с площадью объекта, умноженной на квадрат увеличения центра квадрата).
}
\answer{%
    \begin{align*}
    &\text{Все явные вычисления — в см и $\text{см}^2$,} \\
    \frac 1 F &= \frac 1{a + \frac d2} + \frac 1b \implies b = \frac 1{\frac 1 F - \frac 1{a + \frac d2}} = \frac{F(a + \frac d2)}{a + \frac d2 - F} = -\frac{25}3, \\
    \frac 1 F &= \frac 1{a - \frac d2} + \frac 1c \implies c = \frac 1{\frac 1 F - \frac 1{a - \frac d2}} = \frac{F(a - \frac d2)}{a - \frac d2 - F} = -\frac{575}{73}, \\
    c - b &= \frac{F(a - \frac d2)}{a - \frac d2 - F} - \frac{F(a + \frac d2)}{a + \frac d2 - F} = F\cbr{ \frac{a - \frac d2}{a - \frac d2 - F} - \frac{a + \frac d2}{a + \frac d2 - F} } =  \\
    &= F \cdot \frac{a^2 + \frac {ad}2 - aF - \frac{ad}2 - \frac{d^2}4 + \frac{dF}2 - a^2 + \frac {ad}2 + aF - \frac{ad}2 + \frac{d^2}4 + \frac{dF}2}{\cbr{a + \frac d2 - F}\cbr{a - \frac d2 - F}}= F \cdot \frac {dF}{\cbr{a + \frac d2 - F}\cbr{a - \frac d2 - F}} = \frac{100}{219}.
    \\
    \Gamma_b &= \frac b{a + \frac d2} = \frac{ F }{a + \frac d2 - F} = -\frac23, \\
    \Gamma_c &= \frac c{a - \frac d2} = \frac{ F }{a - \frac d2 - F} = -\frac{50}{73}, \\
    &\text{ тут интересно отметить, что } \Gamma_x = \frac{ c - b}{ d } = \frac{ F^2 }{\cbr{a + \frac d2 - F}\cbr{a - \frac d2 - F}} \ne \Gamma_b \text{ или } \Gamma_c \text{ даже при малых $d$}.
    \\
    S' &= \frac{d \cdot \Gamma_b + d \cdot \Gamma_c}2 \cdot (c - b) = \frac d2 \cbr{\frac{ F }{a + \frac d2 - F} + \frac{ F }{a - \frac d2 - F}} \cdot \cbr{c - b} =  \\
    &=\frac {dF}2 \cbr{\frac 1{a + \frac d2 - F} + \frac 1{a - \frac d2 - F}} \cdot \frac {dF^2}{\cbr{a + \frac d2 - F}\cbr{a - \frac d2 - F}} =  \\
    &=\frac {dF}2 \cdot \frac{a - \frac d2 - F + a + \frac d2 - F}{\cbr{a + \frac d2 - F}\cbr{a - \frac d2 - F}} \cdot \frac {dF^2}{\cbr{a + \frac d2 - F}\cbr{a - \frac d2 - F}} =  \\
    &= \frac {d^2F^3}{2\sqr{a + \frac d2 - F}\sqr{a - \frac d2 - F}} \cdot (2a - 2F) = \frac {d^2F^3(a - F)}{ \sqr{\sqr{a - F} - \frac{d^2}4} } = -\frac{14800}{47961}.
    \end{align*}
}

\variantsplitter

\addpersonalvariant{Матвей Кузьмин}

\tasknumber{1}%
\task{%
    Запишите известные вам виды классификации изображений.
}
\solutionspace{60pt}

\tasknumber{2}%
\task{%
    В каких линзах можно получить прямое изображение объекта?
}
\answer{%
    $\text{ собирающие и рассеивающие }$
}
\solutionspace{40pt}

\tasknumber{3}%
\task{%
    Какое изображение называют мнимым?
}
\solutionspace{40pt}

\tasknumber{4}%
\task{%
    Есть две линзы, обозначим их 1 и 2.
    Известно что оптическая сила линзы 1 больше, чем у линзы 2.
    Какая линза сильнее преломляет лучи?
}
\answer{%
    $1$
}
\solutionspace{40pt}

\tasknumber{5}%
\task{%
    Предмет находится на расстоянии $10\,\text{см}$ от рассеивающей линзы с фокусным расстоянием $6\,\text{см}$.
    Определите тип изображения, расстояние между предметом и его изображением, увеличение предмета.
    Сделайте схематичный рисунок (не обязательно в масштабе, но с сохранением свойств линзы и изображения).
}
\solutionspace{100pt}

\tasknumber{6}%
\task{%
    Объект находится на расстоянии $45\,\text{см}$ от линзы, а его мнимое изображение — в $50\,\text{см}$ от неё.
    Определите увеличение предмета, фокусное расстояние линзы, оптическую силу линзы и её тип.
}
\solutionspace{80pt}

\tasknumber{7}%
\task{%
    Известно, что из формулы тонкой линзы $\cbr{\frac 1F = \frac 1a + \frac 1b}$
    и определения увеличения $\cbr{\Gamma_y = \frac ba}$ можно получить выражение
    для увеличения: $\Gamma_y = \frac {aF}{a - F} \cdot \frac 1a = \frac {F}{a - F}.$
    Назовём такое увеличение «поперечным»: поперёк главной оптической оси (поэтому и ${}_y$).
    Получите формулу для «продольного» увеличения $\Gamma_x$ небольшого предмета, находящегося на главной оптической оси.
    Можно ли применить эту формулу для предмета, не лежащего на главной оптической оси, почему?
}
\answer{%
    \begin{align*}
    \frac 1F &= \frac 1a + \frac 1b \implies b = \frac {aF}{a - F} \\
    \frac 1F &= \frac 1{a + x} + \frac 1c \implies c = \frac {(a+x)F}{a + x - F} \\
    x' &= \abs{b - c} = \frac {aF}{a - F} - \frac {(a+x)F}{a + x - F} = F\cbr{\frac {a}{a - F} - \frac {a+x}{a + x - F}} =  \\
    &= F \cdot \frac {a^2 + ax - aF - a^2 - ax + aF + xF}{(a - F)(a + x - F)} = F \cdot \frac {xF}{(a - F)(a + x - F)} \\
    \Gamma_x &= \frac{x'}x = \frac{F^2}{(a - F)(a + x - F)} \to \frac{F^2}{\sqr{a - F}}.
    \\
    &\text{Нельзя: изображение по-разному растянет по осям $x$ и $y$ и понадобится теорема Пифагора}
    \end{align*}
}
\solutionspace{150pt}

\tasknumber{8}%
\task{%
    Доказать формулу тонкой линзы для рассеивающей линзы.
}
\solutionspace{120pt}

\tasknumber{9}%
\task{%
    Постройте ход луча $BL$ в тонкой линзе.
    Известно положение линзы и оба её фокуса (см.
    рис.
    на доске).
    Рассмотрите оба типа линзы, сделав 2 рисунка: собирающую и рассеивающую.
}
\solutionspace{120pt}

\tasknumber{10}%
\task{%
    На экране, расположенном иа расстоянии $120\,\text{см}$ от собирающей линзы,
    получено изображение точечного источника, расположенного на главной оптической оси линзы.
    На какое расстояние переместится изображение на экране,
    если при неподвижном источнике переместить линзу на $3\,\text{см}$ в плоскости, перпендикулярной главной оптической оси?
    Фокусное расстояние линзы равно $40\,\text{см}$.
}
\answer{%
    \begin{align*}
    &\frac 1F = \frac 1a + \frac 1b \implies a = \frac{bF}{b-F} \implies \Gamma = \frac ba = \frac{b-F}F \\
    &y = x \cdot \Gamma = x \cdot \frac{b-F}F \implies d = x + y = 9\,\text{см}.
    \end{align*}
}
\solutionspace{120pt}

\tasknumber{11}%
\task{%
    Оптическая сила двояковыпуклой линзы в воздухе $4{,}5\,\text{дптр}$, а в воде $1{,}4\,\text{дптр}$.
    Определить показатель преломления $n$ материала, из которого изготовлена линза.
    Показатель преломления воды равен $1{,}33$.
}
\answer{%
    \begin{align*}
    D_1 &=\cbr{\frac n{n_1} - 1}\cbr{\frac 1{R_1} + \frac 1{R_2}}, \\
    D_2 &=\cbr{\frac n{n_2} - 1}\cbr{\frac 1{R_1} + \frac 1{R_2}}, \\
    \frac {D_2}{D_1} &=\frac{\frac n{n_2} - 1}{\frac n{n_1} - 1} \implies {D_2}\cbr{\frac n{n_1} - 1} = {D_1}\cbr{\frac n{n_2} - 1}  \implies n\cbr{\frac{D_2}{n_1} - \frac{D_1}{n_2}} = D_2 - D_1, \\
    n &= \frac{D_2 - D_1}{\frac{D_2}{n_1} - \frac{D_1}{n_2}} = \frac{n_1 n_2 (D_2 - D_1)}{D_2n_2 - D_1n_1} \approx 1{,}563.
    \end{align*}
}
\solutionspace{120pt}

\tasknumber{12}%
\task{%
    На каком расстоянии от собирающей линзы с фокусным расстоянием $30\,\text{дптр}$
    следует надо поместить предмет, чтобы расстояние
    от предмета до его действительного изображения было наименьшим?
}
\answer{%
    \begin{align*}
    \frac 1a &+ \frac 1b = D \implies b = \frac 1{D - \frac 1a} \implies \ell = a + b = a + \frac a{Da - 1} = \frac{ Da^2 }{Da - 1} \implies \\
    \implies \ell'_a &= \frac{ 2Da \cdot (Da - 1) - Da^2 \cdot D }{\sqr{Da - 1}}= \frac{ D^2a^2 - 2Da}{\sqr{Da - 1}} = \frac{ Da(Da - 2)}{\sqr{Da - 1}}\implies a_{\min} = \frac 2D \approx 66{,}7\,\text{мм}.
    \end{align*}
}
\solutionspace{120pt}

\tasknumber{13}%
\task{%
    Даны точечный источник света $S$, его изображение $S_1$, полученное с помощью собирающей линзы,
    и ближайший к источнику фокус линзы $F$ (см.
    рис.
    на доске).
    Расстояния $SF = \ell$ и $SS_1 = L$.
    Определить положение линзы и её фокусное расстояние.
}
\answer{%
    \begin{align*}
    \frac 1a + \frac 1b &= \frac 1F, \ell = a - F, L = a + b \implies a = \ell + F, b = L - a = L - \ell - F \\
    \frac 1{\ell + F} + \frac 1{L - \ell - F} &= \frac 1F \\
    F\ell + F^2 + LF - F\ell - F^2 &= L\ell - \ell^2 - F\ell + LF - F\ell - F^2 \\
    0 &= L\ell - \ell^2 - 2F\ell - F^2 \\
    0 &=  F^2 + 2F\ell - L\ell + \ell^2 \\
    F &= -\ell \pm \sqrt{\ell^2 +  L\ell - \ell^2} = -\ell \pm \sqrt{L\ell} \implies F = \sqrt{L\ell} - \ell \\
    a &= \ell + F = \ell + \sqrt{L\ell} - \ell = \sqrt{L\ell}.
    \end{align*}
}
\solutionspace{120pt}

\tasknumber{14}%
\task{%
    Расстояние от освещённого предмета до экрана $80\,\text{см}$.
    Линза, помещенная между ними, даёт чёткое изображение предмета на
    экране при двух положениях, расстояние между которыми $40\,\text{см}$.
    Найти фокусное расстояние линзы.
}
\answer{%
    \begin{align*}
    \frac 1a + \frac 1b &= \frac 1F, \frac 1{a-\ell} + \frac 1{b+\ell} = \frac 1F, a + b = L \\
    \frac 1a + \frac 1b &= \frac 1{a-\ell} + \frac 1{b+\ell}\implies \frac{a + b}{ab} = \frac{(a-\ell) + (b+\ell)}{(a-\ell)(b+\ell)} \\
    ab  &= (a - \ell)(b+\ell) \implies 0  = -b\ell + a\ell - \ell^2 \implies 0 = -b + a - \ell \implies b = a - \ell \\
    a + (a - \ell) &= L \implies a = \frac{L + \ell}2 \implies b = \frac{L - \ell}2 \\
    F &= \frac{ab}{a + b} = \frac{L^2 -\ell^2}{4L} \approx 15\,\text{см}.
    \end{align*}
}
\solutionspace{120pt}

\tasknumber{15}%
\task{%
    Предмет находится на расстоянии $70\,\text{см}$ от экрана.
    Между предметом и экраном помещают линзу, причём при одном
    положении линзы на экране получается увеличенное изображение предмета,
    а при другом — уменьшенное.
    Каково фокусное расстояние линзы, если
    линейные размеры первого изображения в два раза больше второго?
}
\answer{%
    \begin{align*}
    \frac 1a + \frac 1{L-a} &= \frac 1F, h_1 = h \cdot \frac{L-a}a, \\
    \frac 1b + \frac 1{L-b} &= \frac 1F, h_2 = h \cdot \frac{L-b}b, \\
    \frac{h_1}{h_2} &= 2 \implies \frac{(L-a)b}{(L-b)a} = 2, \\
    \frac 1F &= \frac{ L }{a(L-a)} = \frac{ L }{b(L-b)} \implies \frac{L-a}{L-b} = \frac b a \implies \frac {b^2}{a^2} = 2.
    \\
    \frac 1a + \frac 1{L-a} &= \frac 1b + \frac 1{L-b} \implies \frac L{a(L-a)} = \frac L{b(L-b)} \implies \\
    \implies aL - a^2 &= bL - b^2 \implies (a-b)L = (a-b)(a+b) \implies b = L - a, \\
    \frac{\sqr{L-a}}{a^2} &= 2 \implies \frac La - 1 = \sqrt{2} \implies a = \frac{ L }{\sqrt{2} + 1} \\
    F &= \frac{a(L-a)}L = \frac 1L \cdot \frac L{\sqrt{2} + 1} \cdot \frac {L\sqrt{2}}{\sqrt{2} + 1}= \frac { L\sqrt{2} }{ \sqr{\sqrt{2} + 1} } \approx 17{,}0\,\text{см}.
    \end{align*}
}
\solutionspace{120pt}

\tasknumber{16}%
\task{%
    (Задача-«гроб»: решать на обратной стороне) Квадрат со стороной $d = 3\,\text{см}$ расположен так,
    что 2 его стороны параллельны главной оптической оси рассеивающей линзы,
    его центр удален на $h = 6\,\text{см}$ от этой оси и на $a = 15\,\text{см}$ от плоскости линзы.
    Определите площадь изображения квадрата, если фокусное расстояние линзы составляет $F = 25\,\text{см}$.
    % (и сравните с площадью объекта, умноженной на квадрат увеличения центра квадрата).
}
\answer{%
    \begin{align*}
    &\text{Все явные вычисления — в см и $\text{см}^2$,} \\
    \frac 1 F &= \frac 1{a + \frac d2} + \frac 1b \implies b = \frac 1{\frac 1 F - \frac 1{a + \frac d2}} = \frac{F(a + \frac d2)}{a + \frac d2 - F} = -\frac{825}{83}, \\
    \frac 1 F &= \frac 1{a - \frac d2} + \frac 1c \implies c = \frac 1{\frac 1 F - \frac 1{a - \frac d2}} = \frac{F(a - \frac d2)}{a - \frac d2 - F} = -\frac{675}{77}, \\
    c - b &= \frac{F(a - \frac d2)}{a - \frac d2 - F} - \frac{F(a + \frac d2)}{a + \frac d2 - F} = F\cbr{ \frac{a - \frac d2}{a - \frac d2 - F} - \frac{a + \frac d2}{a + \frac d2 - F} } =  \\
    &= F \cdot \frac{a^2 + \frac {ad}2 - aF - \frac{ad}2 - \frac{d^2}4 + \frac{dF}2 - a^2 + \frac {ad}2 + aF - \frac{ad}2 + \frac{d^2}4 + \frac{dF}2}{\cbr{a + \frac d2 - F}\cbr{a - \frac d2 - F}}= F \cdot \frac {dF}{\cbr{a + \frac d2 - F}\cbr{a - \frac d2 - F}} = \frac{7500}{6391}.
    \\
    \Gamma_b &= \frac b{a + \frac d2} = \frac{ F }{a + \frac d2 - F} = -\frac{50}{83}, \\
    \Gamma_c &= \frac c{a - \frac d2} = \frac{ F }{a - \frac d2 - F} = -\frac{50}{77}, \\
    &\text{ тут интересно отметить, что } \Gamma_x = \frac{ c - b}{ d } = \frac{ F^2 }{\cbr{a + \frac d2 - F}\cbr{a - \frac d2 - F}} \ne \Gamma_b \text{ или } \Gamma_c \text{ даже при малых $d$}.
    \\
    S' &= \frac{d \cdot \Gamma_b + d \cdot \Gamma_c}2 \cdot (c - b) = \frac d2 \cbr{\frac{ F }{a + \frac d2 - F} + \frac{ F }{a - \frac d2 - F}} \cdot \cbr{c - b} =  \\
    &=\frac {dF}2 \cbr{\frac 1{a + \frac d2 - F} + \frac 1{a - \frac d2 - F}} \cdot \frac {dF^2}{\cbr{a + \frac d2 - F}\cbr{a - \frac d2 - F}} =  \\
    &=\frac {dF}2 \cdot \frac{a - \frac d2 - F + a + \frac d2 - F}{\cbr{a + \frac d2 - F}\cbr{a - \frac d2 - F}} \cdot \frac {dF^2}{\cbr{a + \frac d2 - F}\cbr{a - \frac d2 - F}} =  \\
    &= \frac {d^2F^3}{2\sqr{a + \frac d2 - F}\sqr{a - \frac d2 - F}} \cdot (2a - 2F) = \frac {d^2F^3(a - F)}{ \sqr{\sqr{a - F} - \frac{d^2}4} } = -\frac{90000000}{40844881}.
    \end{align*}
}

\variantsplitter

\addpersonalvariant{Сергей Малышев}

\tasknumber{1}%
\task{%
    Запишите известные вам виды классификации изображений.
}
\solutionspace{60pt}

\tasknumber{2}%
\task{%
    В каких линзах можно получить прямое изображение объекта?
}
\answer{%
    $\text{ собирающие и рассеивающие }$
}
\solutionspace{40pt}

\tasknumber{3}%
\task{%
    Какое изображение называют мнимым?
}
\solutionspace{40pt}

\tasknumber{4}%
\task{%
    Есть две линзы, обозначим их 1 и 2.
    Известно что оптическая сила линзы 1 меньше, чем у линзы 2.
    Какая линза сильнее преломляет лучи?
}
\answer{%
    $2$
}
\solutionspace{40pt}

\tasknumber{5}%
\task{%
    Предмет находится на расстоянии $30\,\text{см}$ от собирающей линзы с фокусным расстоянием $12\,\text{см}$.
    Определите тип изображения, расстояние между предметом и его изображением, увеличение предмета.
    Сделайте схематичный рисунок (не обязательно в масштабе, но с сохранением свойств линзы и изображения).
}
\solutionspace{100pt}

\tasknumber{6}%
\task{%
    Объект находится на расстоянии $25\,\text{см}$ от линзы, а его мнимое изображение — в $10\,\text{см}$ от неё.
    Определите увеличение предмета, фокусное расстояние линзы, оптическую силу линзы и её тип.
}
\solutionspace{80pt}

\tasknumber{7}%
\task{%
    Известно, что из формулы тонкой линзы $\cbr{\frac 1F = \frac 1a + \frac 1b}$
    и определения увеличения $\cbr{\Gamma_y = \frac ba}$ можно получить выражение
    для увеличения: $\Gamma_y = \frac {aF}{a - F} \cdot \frac 1a = \frac {F}{a - F}.$
    Назовём такое увеличение «поперечным»: поперёк главной оптической оси (поэтому и ${}_y$).
    Получите формулу для «продольного» увеличения $\Gamma_x$ небольшого предмета, находящегося на главной оптической оси.
    Можно ли применить эту формулу для предмета, не лежащего на главной оптической оси, почему?
}
\answer{%
    \begin{align*}
    \frac 1F &= \frac 1a + \frac 1b \implies b = \frac {aF}{a - F} \\
    \frac 1F &= \frac 1{a + x} + \frac 1c \implies c = \frac {(a+x)F}{a + x - F} \\
    x' &= \abs{b - c} = \frac {aF}{a - F} - \frac {(a+x)F}{a + x - F} = F\cbr{\frac {a}{a - F} - \frac {a+x}{a + x - F}} =  \\
    &= F \cdot \frac {a^2 + ax - aF - a^2 - ax + aF + xF}{(a - F)(a + x - F)} = F \cdot \frac {xF}{(a - F)(a + x - F)} \\
    \Gamma_x &= \frac{x'}x = \frac{F^2}{(a - F)(a + x - F)} \to \frac{F^2}{\sqr{a - F}}.
    \\
    &\text{Нельзя: изображение по-разному растянет по осям $x$ и $y$ и понадобится теорема Пифагора}
    \end{align*}
}
\solutionspace{150pt}

\tasknumber{8}%
\task{%
    Доказать формулу тонкой линзы для рассеивающей линзы.
}
\solutionspace{120pt}

\tasknumber{9}%
\task{%
    Постройте ход луча $BM$ в тонкой линзе.
    Известно положение линзы и оба её фокуса (см.
    рис.
    на доске).
    Рассмотрите оба типа линзы, сделав 2 рисунка: собирающую и рассеивающую.
}
\solutionspace{120pt}

\tasknumber{10}%
\task{%
    На экране, расположенном иа расстоянии $60\,\text{см}$ от собирающей линзы,
    получено изображение точечного источника, расположенного на главной оптической оси линзы.
    На какое расстояние переместится изображение на экране,
    если при неподвижной линзе переместить источник на $2\,\text{см}$ в плоскости, перпендикулярной главной оптической оси?
    Фокусное расстояние линзы равно $40\,\text{см}$.
}
\answer{%
    \begin{align*}
    &\frac 1F = \frac 1a + \frac 1b \implies a = \frac{bF}{b-F} \implies \Gamma = \frac ba = \frac{b-F}F \\
    &y = x \cdot \Gamma = x \cdot \frac{b-F}F \implies d = y = 1\,\text{см}.
    \end{align*}
}
\solutionspace{120pt}

\tasknumber{11}%
\task{%
    Оптическая сила двояковыпуклой линзы в воздухе $4{,}5\,\text{дптр}$, а в воде $1{,}5\,\text{дптр}$.
    Определить показатель преломления $n$ материала, из которого изготовлена линза.
    Показатель преломления воды равен $1{,}33$.
}
\answer{%
    \begin{align*}
    D_1 &=\cbr{\frac n{n_1} - 1}\cbr{\frac 1{R_1} + \frac 1{R_2}}, \\
    D_2 &=\cbr{\frac n{n_2} - 1}\cbr{\frac 1{R_1} + \frac 1{R_2}}, \\
    \frac {D_2}{D_1} &=\frac{\frac n{n_2} - 1}{\frac n{n_1} - 1} \implies {D_2}\cbr{\frac n{n_1} - 1} = {D_1}\cbr{\frac n{n_2} - 1}  \implies n\cbr{\frac{D_2}{n_1} - \frac{D_1}{n_2}} = D_2 - D_1, \\
    n &= \frac{D_2 - D_1}{\frac{D_2}{n_1} - \frac{D_1}{n_2}} = \frac{n_1 n_2 (D_2 - D_1)}{D_2n_2 - D_1n_1} \approx 1{,}593.
    \end{align*}
}
\solutionspace{120pt}

\tasknumber{12}%
\task{%
    На каком расстоянии от собирающей линзы с фокусным расстоянием $40\,\text{дптр}$
    следует надо поместить предмет, чтобы расстояние
    от предмета до его действительного изображения было наименьшим?
}
\answer{%
    \begin{align*}
    \frac 1a &+ \frac 1b = D \implies b = \frac 1{D - \frac 1a} \implies \ell = a + b = a + \frac a{Da - 1} = \frac{ Da^2 }{Da - 1} \implies \\
    \implies \ell'_a &= \frac{ 2Da \cdot (Da - 1) - Da^2 \cdot D }{\sqr{Da - 1}}= \frac{ D^2a^2 - 2Da}{\sqr{Da - 1}} = \frac{ Da(Da - 2)}{\sqr{Da - 1}}\implies a_{\min} = \frac 2D \approx 50\,\text{мм}.
    \end{align*}
}
\solutionspace{120pt}

\tasknumber{13}%
\task{%
    Даны точечный источник света $S$, его изображение $S_1$, полученное с помощью собирающей линзы,
    и ближайший к источнику фокус линзы $F$ (см.
    рис.
    на доске).
    Расстояния $SF = \ell$ и $SS_1 = L$.
    Определить положение линзы и её фокусное расстояние.
}
\answer{%
    \begin{align*}
    \frac 1a + \frac 1b &= \frac 1F, \ell = a - F, L = a + b \implies a = \ell + F, b = L - a = L - \ell - F \\
    \frac 1{\ell + F} + \frac 1{L - \ell - F} &= \frac 1F \\
    F\ell + F^2 + LF - F\ell - F^2 &= L\ell - \ell^2 - F\ell + LF - F\ell - F^2 \\
    0 &= L\ell - \ell^2 - 2F\ell - F^2 \\
    0 &=  F^2 + 2F\ell - L\ell + \ell^2 \\
    F &= -\ell \pm \sqrt{\ell^2 +  L\ell - \ell^2} = -\ell \pm \sqrt{L\ell} \implies F = \sqrt{L\ell} - \ell \\
    a &= \ell + F = \ell + \sqrt{L\ell} - \ell = \sqrt{L\ell}.
    \end{align*}
}
\solutionspace{120pt}

\tasknumber{14}%
\task{%
    Расстояние от освещённого предмета до экрана $100\,\text{см}$.
    Линза, помещенная между ними, даёт чёткое изображение предмета на
    экране при двух положениях, расстояние между которыми $40\,\text{см}$.
    Найти фокусное расстояние линзы.
}
\answer{%
    \begin{align*}
    \frac 1a + \frac 1b &= \frac 1F, \frac 1{a-\ell} + \frac 1{b+\ell} = \frac 1F, a + b = L \\
    \frac 1a + \frac 1b &= \frac 1{a-\ell} + \frac 1{b+\ell}\implies \frac{a + b}{ab} = \frac{(a-\ell) + (b+\ell)}{(a-\ell)(b+\ell)} \\
    ab  &= (a - \ell)(b+\ell) \implies 0  = -b\ell + a\ell - \ell^2 \implies 0 = -b + a - \ell \implies b = a - \ell \\
    a + (a - \ell) &= L \implies a = \frac{L + \ell}2 \implies b = \frac{L - \ell}2 \\
    F &= \frac{ab}{a + b} = \frac{L^2 -\ell^2}{4L} \approx 21\,\text{см}.
    \end{align*}
}
\solutionspace{120pt}

\tasknumber{15}%
\task{%
    Предмет находится на расстоянии $90\,\text{см}$ от экрана.
    Между предметом и экраном помещают линзу, причём при одном
    положении линзы на экране получается увеличенное изображение предмета,
    а при другом — уменьшенное.
    Каково фокусное расстояние линзы, если
    линейные размеры первого изображения в два раза больше второго?
}
\answer{%
    \begin{align*}
    \frac 1a + \frac 1{L-a} &= \frac 1F, h_1 = h \cdot \frac{L-a}a, \\
    \frac 1b + \frac 1{L-b} &= \frac 1F, h_2 = h \cdot \frac{L-b}b, \\
    \frac{h_1}{h_2} &= 2 \implies \frac{(L-a)b}{(L-b)a} = 2, \\
    \frac 1F &= \frac{ L }{a(L-a)} = \frac{ L }{b(L-b)} \implies \frac{L-a}{L-b} = \frac b a \implies \frac {b^2}{a^2} = 2.
    \\
    \frac 1a + \frac 1{L-a} &= \frac 1b + \frac 1{L-b} \implies \frac L{a(L-a)} = \frac L{b(L-b)} \implies \\
    \implies aL - a^2 &= bL - b^2 \implies (a-b)L = (a-b)(a+b) \implies b = L - a, \\
    \frac{\sqr{L-a}}{a^2} &= 2 \implies \frac La - 1 = \sqrt{2} \implies a = \frac{ L }{\sqrt{2} + 1} \\
    F &= \frac{a(L-a)}L = \frac 1L \cdot \frac L{\sqrt{2} + 1} \cdot \frac {L\sqrt{2}}{\sqrt{2} + 1}= \frac { L\sqrt{2} }{ \sqr{\sqrt{2} + 1} } \approx 22\,\text{см}.
    \end{align*}
}
\solutionspace{120pt}

\tasknumber{16}%
\task{%
    (Задача-«гроб»: решать на обратной стороне) Квадрат со стороной $d = 1\,\text{см}$ расположен так,
    что 2 его стороны параллельны главной оптической оси собирающей линзы,
    его центр удален на $h = 6\,\text{см}$ от этой оси и на $a = 12\,\text{см}$ от плоскости линзы.
    Определите площадь изображения квадрата, если фокусное расстояние линзы составляет $F = 18\,\text{см}$.
    % (и сравните с площадью объекта, умноженной на квадрат увеличения центра квадрата).
}
\answer{%
    \begin{align*}
    &\text{Все явные вычисления — в см и $\text{см}^2$,} \\
    \frac 1 F &= \frac 1{a + \frac d2} + \frac 1b \implies b = \frac 1{\frac 1 F - \frac 1{a + \frac d2}} = \frac{F(a + \frac d2)}{a + \frac d2 - F} = -\frac{450}{11}, \\
    \frac 1 F &= \frac 1{a - \frac d2} + \frac 1c \implies c = \frac 1{\frac 1 F - \frac 1{a - \frac d2}} = \frac{F(a - \frac d2)}{a - \frac d2 - F} = -\frac{414}{13}, \\
    c - b &= \frac{F(a - \frac d2)}{a - \frac d2 - F} - \frac{F(a + \frac d2)}{a + \frac d2 - F} = F\cbr{ \frac{a - \frac d2}{a - \frac d2 - F} - \frac{a + \frac d2}{a + \frac d2 - F} } =  \\
    &= F \cdot \frac{a^2 + \frac {ad}2 - aF - \frac{ad}2 - \frac{d^2}4 + \frac{dF}2 - a^2 + \frac {ad}2 + aF - \frac{ad}2 + \frac{d^2}4 + \frac{dF}2}{\cbr{a + \frac d2 - F}\cbr{a - \frac d2 - F}}= F \cdot \frac {dF}{\cbr{a + \frac d2 - F}\cbr{a - \frac d2 - F}} = \frac{1296}{143}.
    \\
    \Gamma_b &= \frac b{a + \frac d2} = \frac{ F }{a + \frac d2 - F} = -\frac{36}{11}, \\
    \Gamma_c &= \frac c{a - \frac d2} = \frac{ F }{a - \frac d2 - F} = -\frac{36}{13}, \\
    &\text{ тут интересно отметить, что } \Gamma_x = \frac{ c - b}{ d } = \frac{ F^2 }{\cbr{a + \frac d2 - F}\cbr{a - \frac d2 - F}} \ne \Gamma_b \text{ или } \Gamma_c \text{ даже при малых $d$}.
    \\
    S' &= \frac{d \cdot \Gamma_b + d \cdot \Gamma_c}2 \cdot (c - b) = \frac d2 \cbr{\frac{ F }{a + \frac d2 - F} + \frac{ F }{a - \frac d2 - F}} \cdot \cbr{c - b} =  \\
    &=\frac {dF}2 \cbr{\frac 1{a + \frac d2 - F} + \frac 1{a - \frac d2 - F}} \cdot \frac {dF^2}{\cbr{a + \frac d2 - F}\cbr{a - \frac d2 - F}} =  \\
    &=\frac {dF}2 \cdot \frac{a - \frac d2 - F + a + \frac d2 - F}{\cbr{a + \frac d2 - F}\cbr{a - \frac d2 - F}} \cdot \frac {dF^2}{\cbr{a + \frac d2 - F}\cbr{a - \frac d2 - F}} =  \\
    &= \frac {d^2F^3}{2\sqr{a + \frac d2 - F}\sqr{a - \frac d2 - F}} \cdot (2a - 2F) = \frac {d^2F^3(a - F)}{ \sqr{\sqr{a - F} - \frac{d^2}4} } = -\frac{559872}{20449}.
    \end{align*}
}

\variantsplitter

\addpersonalvariant{Алина Полканова}

\tasknumber{1}%
\task{%
    Запишите формулу тонкой линзы и сделайте рисунок, указав на нём физические величины из этой формулы.
}
\solutionspace{60pt}

\tasknumber{2}%
\task{%
    В каких линзах можно получить прямое изображение объекта?
}
\answer{%
    $\text{ собирающие и рассеивающие }$
}
\solutionspace{40pt}

\tasknumber{3}%
\task{%
    Какое изображение называют мнимым?
}
\solutionspace{40pt}

\tasknumber{4}%
\task{%
    Есть две линзы, обозначим их 1 и 2.
    Известно что фокусное расстояние линзы 2 меньше, чем у линзы 1.
    Какая линза сильнее преломляет лучи?
}
\answer{%
    $2$
}
\solutionspace{40pt}

\tasknumber{5}%
\task{%
    Предмет находится на расстоянии $30\,\text{см}$ от рассеивающей линзы с фокусным расстоянием $8\,\text{см}$.
    Определите тип изображения, расстояние между предметом и его изображением, увеличение предмета.
    Сделайте схематичный рисунок (не обязательно в масштабе, но с сохранением свойств линзы и изображения).
}
\solutionspace{100pt}

\tasknumber{6}%
\task{%
    Объект находится на расстоянии $115\,\text{см}$ от линзы, а его мнимое изображение — в $20\,\text{см}$ от неё.
    Определите увеличение предмета, фокусное расстояние линзы, оптическую силу линзы и её тип.
}
\solutionspace{80pt}

\tasknumber{7}%
\task{%
    Известно, что из формулы тонкой линзы $\cbr{\frac 1F = \frac 1a + \frac 1b}$
    и определения увеличения $\cbr{\Gamma_y = \frac ba}$ можно получить выражение
    для увеличения: $\Gamma_y = \frac {aF}{a - F} \cdot \frac 1a = \frac {F}{a - F}.$
    Назовём такое увеличение «поперечным»: поперёк главной оптической оси (поэтому и ${}_y$).
    Получите формулу для «продольного» увеличения $\Gamma_x$ небольшого предмета, находящегося на главной оптической оси.
    Можно ли применить эту формулу для предмета, не лежащего на главной оптической оси, почему?
}
\answer{%
    \begin{align*}
    \frac 1F &= \frac 1a + \frac 1b \implies b = \frac {aF}{a - F} \\
    \frac 1F &= \frac 1{a + x} + \frac 1c \implies c = \frac {(a+x)F}{a + x - F} \\
    x' &= \abs{b - c} = \frac {aF}{a - F} - \frac {(a+x)F}{a + x - F} = F\cbr{\frac {a}{a - F} - \frac {a+x}{a + x - F}} =  \\
    &= F \cdot \frac {a^2 + ax - aF - a^2 - ax + aF + xF}{(a - F)(a + x - F)} = F \cdot \frac {xF}{(a - F)(a + x - F)} \\
    \Gamma_x &= \frac{x'}x = \frac{F^2}{(a - F)(a + x - F)} \to \frac{F^2}{\sqr{a - F}}.
    \\
    &\text{Нельзя: изображение по-разному растянет по осям $x$ и $y$ и понадобится теорема Пифагора}
    \end{align*}
}
\solutionspace{150pt}

\tasknumber{8}%
\task{%
    Доказать формулу тонкой линзы для рассеивающей линзы.
}
\solutionspace{120pt}

\tasknumber{9}%
\task{%
    Постройте ход луча $CK$ в тонкой линзе.
    Известно положение линзы и оба её фокуса (см.
    рис.
    на доске).
    Рассмотрите оба типа линзы, сделав 2 рисунка: собирающую и рассеивающую.
}
\solutionspace{120pt}

\tasknumber{10}%
\task{%
    На экране, расположенном иа расстоянии $80\,\text{см}$ от собирающей линзы,
    получено изображение точечного источника, расположенного на главной оптической оси линзы.
    На какое расстояние переместится изображение на экране,
    если при неподвижной линзе переместить источник на $2\,\text{см}$ в плоскости, перпендикулярной главной оптической оси?
    Фокусное расстояние линзы равно $40\,\text{см}$.
}
\answer{%
    \begin{align*}
    &\frac 1F = \frac 1a + \frac 1b \implies a = \frac{bF}{b-F} \implies \Gamma = \frac ba = \frac{b-F}F \\
    &y = x \cdot \Gamma = x \cdot \frac{b-F}F \implies d = y = 2\,\text{см}.
    \end{align*}
}
\solutionspace{120pt}

\tasknumber{11}%
\task{%
    Оптическая сила двояковыпуклой линзы в воздухе $5\,\text{дптр}$, а в воде $1{,}6\,\text{дптр}$.
    Определить показатель преломления $n$ материала, из которого изготовлена линза.
    Показатель преломления воды равен $1{,}33$.
}
\answer{%
    \begin{align*}
    D_1 &=\cbr{\frac n{n_1} - 1}\cbr{\frac 1{R_1} + \frac 1{R_2}}, \\
    D_2 &=\cbr{\frac n{n_2} - 1}\cbr{\frac 1{R_1} + \frac 1{R_2}}, \\
    \frac {D_2}{D_1} &=\frac{\frac n{n_2} - 1}{\frac n{n_1} - 1} \implies {D_2}\cbr{\frac n{n_1} - 1} = {D_1}\cbr{\frac n{n_2} - 1}  \implies n\cbr{\frac{D_2}{n_1} - \frac{D_1}{n_2}} = D_2 - D_1, \\
    n &= \frac{D_2 - D_1}{\frac{D_2}{n_1} - \frac{D_1}{n_2}} = \frac{n_1 n_2 (D_2 - D_1)}{D_2n_2 - D_1n_1} \approx 1{,}575.
    \end{align*}
}
\solutionspace{120pt}

\tasknumber{12}%
\task{%
    На каком расстоянии от собирающей линзы с фокусным расстоянием $40\,\text{дптр}$
    следует надо поместить предмет, чтобы расстояние
    от предмета до его действительного изображения было наименьшим?
}
\answer{%
    \begin{align*}
    \frac 1a &+ \frac 1b = D \implies b = \frac 1{D - \frac 1a} \implies \ell = a + b = a + \frac a{Da - 1} = \frac{ Da^2 }{Da - 1} \implies \\
    \implies \ell'_a &= \frac{ 2Da \cdot (Da - 1) - Da^2 \cdot D }{\sqr{Da - 1}}= \frac{ D^2a^2 - 2Da}{\sqr{Da - 1}} = \frac{ Da(Da - 2)}{\sqr{Da - 1}}\implies a_{\min} = \frac 2D \approx 50\,\text{мм}.
    \end{align*}
}
\solutionspace{120pt}

\tasknumber{13}%
\task{%
    Даны точечный источник света $S$, его изображение $S_1$, полученное с помощью собирающей линзы,
    и ближайший к источнику фокус линзы $F$ (см.
    рис.
    на доске).
    Расстояния $SF = \ell$ и $SS_1 = L$.
    Определить положение линзы и её фокусное расстояние.
}
\answer{%
    \begin{align*}
    \frac 1a + \frac 1b &= \frac 1F, \ell = a - F, L = a + b \implies a = \ell + F, b = L - a = L - \ell - F \\
    \frac 1{\ell + F} + \frac 1{L - \ell - F} &= \frac 1F \\
    F\ell + F^2 + LF - F\ell - F^2 &= L\ell - \ell^2 - F\ell + LF - F\ell - F^2 \\
    0 &= L\ell - \ell^2 - 2F\ell - F^2 \\
    0 &=  F^2 + 2F\ell - L\ell + \ell^2 \\
    F &= -\ell \pm \sqrt{\ell^2 +  L\ell - \ell^2} = -\ell \pm \sqrt{L\ell} \implies F = \sqrt{L\ell} - \ell \\
    a &= \ell + F = \ell + \sqrt{L\ell} - \ell = \sqrt{L\ell}.
    \end{align*}
}
\solutionspace{120pt}

\tasknumber{14}%
\task{%
    Расстояние от освещённого предмета до экрана $80\,\text{см}$.
    Линза, помещенная между ними, даёт чёткое изображение предмета на
    экране при двух положениях, расстояние между которыми $20\,\text{см}$.
    Найти фокусное расстояние линзы.
}
\answer{%
    \begin{align*}
    \frac 1a + \frac 1b &= \frac 1F, \frac 1{a-\ell} + \frac 1{b+\ell} = \frac 1F, a + b = L \\
    \frac 1a + \frac 1b &= \frac 1{a-\ell} + \frac 1{b+\ell}\implies \frac{a + b}{ab} = \frac{(a-\ell) + (b+\ell)}{(a-\ell)(b+\ell)} \\
    ab  &= (a - \ell)(b+\ell) \implies 0  = -b\ell + a\ell - \ell^2 \implies 0 = -b + a - \ell \implies b = a - \ell \\
    a + (a - \ell) &= L \implies a = \frac{L + \ell}2 \implies b = \frac{L - \ell}2 \\
    F &= \frac{ab}{a + b} = \frac{L^2 -\ell^2}{4L} \approx 18{,}8\,\text{см}.
    \end{align*}
}
\solutionspace{120pt}

\tasknumber{15}%
\task{%
    Предмет находится на расстоянии $90\,\text{см}$ от экрана.
    Между предметом и экраном помещают линзу, причём при одном
    положении линзы на экране получается увеличенное изображение предмета,
    а при другом — уменьшенное.
    Каково фокусное расстояние линзы, если
    линейные размеры первого изображения в пять раз больше второго?
}
\answer{%
    \begin{align*}
    \frac 1a + \frac 1{L-a} &= \frac 1F, h_1 = h \cdot \frac{L-a}a, \\
    \frac 1b + \frac 1{L-b} &= \frac 1F, h_2 = h \cdot \frac{L-b}b, \\
    \frac{h_1}{h_2} &= 5 \implies \frac{(L-a)b}{(L-b)a} = 5, \\
    \frac 1F &= \frac{ L }{a(L-a)} = \frac{ L }{b(L-b)} \implies \frac{L-a}{L-b} = \frac b a \implies \frac {b^2}{a^2} = 5.
    \\
    \frac 1a + \frac 1{L-a} &= \frac 1b + \frac 1{L-b} \implies \frac L{a(L-a)} = \frac L{b(L-b)} \implies \\
    \implies aL - a^2 &= bL - b^2 \implies (a-b)L = (a-b)(a+b) \implies b = L - a, \\
    \frac{\sqr{L-a}}{a^2} &= 5 \implies \frac La - 1 = \sqrt{5} \implies a = \frac{ L }{\sqrt{5} + 1} \\
    F &= \frac{a(L-a)}L = \frac 1L \cdot \frac L{\sqrt{5} + 1} \cdot \frac {L\sqrt{5}}{\sqrt{5} + 1}= \frac { L\sqrt{5} }{ \sqr{\sqrt{5} + 1} } \approx 19{,}2\,\text{см}.
    \end{align*}
}
\solutionspace{120pt}

\tasknumber{16}%
\task{%
    (Задача-«гроб»: решать на обратной стороне) Квадрат со стороной $d = 1\,\text{см}$ расположен так,
    что 2 его стороны параллельны главной оптической оси собирающей линзы,
    его центр удален на $h = 5\,\text{см}$ от этой оси и на $a = 12\,\text{см}$ от плоскости линзы.
    Определите площадь изображения квадрата, если фокусное расстояние линзы составляет $F = 25\,\text{см}$.
    % (и сравните с площадью объекта, умноженной на квадрат увеличения центра квадрата).
}
\answer{%
    \begin{align*}
    &\text{Все явные вычисления — в см и $\text{см}^2$,} \\
    \frac 1 F &= \frac 1{a + \frac d2} + \frac 1b \implies b = \frac 1{\frac 1 F - \frac 1{a + \frac d2}} = \frac{F(a + \frac d2)}{a + \frac d2 - F} = -25, \\
    \frac 1 F &= \frac 1{a - \frac d2} + \frac 1c \implies c = \frac 1{\frac 1 F - \frac 1{a - \frac d2}} = \frac{F(a - \frac d2)}{a - \frac d2 - F} = -\frac{575}{27}, \\
    c - b &= \frac{F(a - \frac d2)}{a - \frac d2 - F} - \frac{F(a + \frac d2)}{a + \frac d2 - F} = F\cbr{ \frac{a - \frac d2}{a - \frac d2 - F} - \frac{a + \frac d2}{a + \frac d2 - F} } =  \\
    &= F \cdot \frac{a^2 + \frac {ad}2 - aF - \frac{ad}2 - \frac{d^2}4 + \frac{dF}2 - a^2 + \frac {ad}2 + aF - \frac{ad}2 + \frac{d^2}4 + \frac{dF}2}{\cbr{a + \frac d2 - F}\cbr{a - \frac d2 - F}}= F \cdot \frac {dF}{\cbr{a + \frac d2 - F}\cbr{a - \frac d2 - F}} = \frac{100}{27}.
    \\
    \Gamma_b &= \frac b{a + \frac d2} = \frac{ F }{a + \frac d2 - F} = -2, \\
    \Gamma_c &= \frac c{a - \frac d2} = \frac{ F }{a - \frac d2 - F} = -\frac{50}{27}, \\
    &\text{ тут интересно отметить, что } \Gamma_x = \frac{ c - b}{ d } = \frac{ F^2 }{\cbr{a + \frac d2 - F}\cbr{a - \frac d2 - F}} \ne \Gamma_b \text{ или } \Gamma_c \text{ даже при малых $d$}.
    \\
    S' &= \frac{d \cdot \Gamma_b + d \cdot \Gamma_c}2 \cdot (c - b) = \frac d2 \cbr{\frac{ F }{a + \frac d2 - F} + \frac{ F }{a - \frac d2 - F}} \cdot \cbr{c - b} =  \\
    &=\frac {dF}2 \cbr{\frac 1{a + \frac d2 - F} + \frac 1{a - \frac d2 - F}} \cdot \frac {dF^2}{\cbr{a + \frac d2 - F}\cbr{a - \frac d2 - F}} =  \\
    &=\frac {dF}2 \cdot \frac{a - \frac d2 - F + a + \frac d2 - F}{\cbr{a + \frac d2 - F}\cbr{a - \frac d2 - F}} \cdot \frac {dF^2}{\cbr{a + \frac d2 - F}\cbr{a - \frac d2 - F}} =  \\
    &= \frac {d^2F^3}{2\sqr{a + \frac d2 - F}\sqr{a - \frac d2 - F}} \cdot (2a - 2F) = \frac {d^2F^3(a - F)}{ \sqr{\sqr{a - F} - \frac{d^2}4} } = -\frac{5200}{729}.
    \end{align*}
}

\variantsplitter

\addpersonalvariant{Сергей Пономарёв}

\tasknumber{1}%
\task{%
    Запишите формулу тонкой линзы и сделайте рисунок, указав на нём физические величины из этой формулы.
}
\solutionspace{60pt}

\tasknumber{2}%
\task{%
    В каких линзах можно получить прямое изображение объекта?
}
\answer{%
    $\text{ собирающие и рассеивающие }$
}
\solutionspace{40pt}

\tasknumber{3}%
\task{%
    Какое изображение называют мнимым?
}
\solutionspace{40pt}

\tasknumber{4}%
\task{%
    Есть две линзы, обозначим их 1 и 2.
    Известно что оптическая сила линзы 2 больше, чем у линзы 1.
    Какая линза сильнее преломляет лучи?
}
\answer{%
    $2$
}
\solutionspace{40pt}

\tasknumber{5}%
\task{%
    Предмет находится на расстоянии $30\,\text{см}$ от собирающей линзы с фокусным расстоянием $12\,\text{см}$.
    Определите тип изображения, расстояние между предметом и его изображением, увеличение предмета.
    Сделайте схематичный рисунок (не обязательно в масштабе, но с сохранением свойств линзы и изображения).
}
\solutionspace{100pt}

\tasknumber{6}%
\task{%
    Объект находится на расстоянии $25\,\text{см}$ от линзы, а его действительное изображение — в $50\,\text{см}$ от неё.
    Определите увеличение предмета, фокусное расстояние линзы, оптическую силу линзы и её тип.
}
\solutionspace{80pt}

\tasknumber{7}%
\task{%
    Известно, что из формулы тонкой линзы $\cbr{\frac 1F = \frac 1a + \frac 1b}$
    и определения увеличения $\cbr{\Gamma_y = \frac ba}$ можно получить выражение
    для увеличения: $\Gamma_y = \frac {aF}{a - F} \cdot \frac 1a = \frac {F}{a - F}.$
    Назовём такое увеличение «поперечным»: поперёк главной оптической оси (поэтому и ${}_y$).
    Получите формулу для «продольного» увеличения $\Gamma_x$ небольшого предмета, находящегося на главной оптической оси.
    Можно ли применить эту формулу для предмета, не лежащего на главной оптической оси, почему?
}
\answer{%
    \begin{align*}
    \frac 1F &= \frac 1a + \frac 1b \implies b = \frac {aF}{a - F} \\
    \frac 1F &= \frac 1{a + x} + \frac 1c \implies c = \frac {(a+x)F}{a + x - F} \\
    x' &= \abs{b - c} = \frac {aF}{a - F} - \frac {(a+x)F}{a + x - F} = F\cbr{\frac {a}{a - F} - \frac {a+x}{a + x - F}} =  \\
    &= F \cdot \frac {a^2 + ax - aF - a^2 - ax + aF + xF}{(a - F)(a + x - F)} = F \cdot \frac {xF}{(a - F)(a + x - F)} \\
    \Gamma_x &= \frac{x'}x = \frac{F^2}{(a - F)(a + x - F)} \to \frac{F^2}{\sqr{a - F}}.
    \\
    &\text{Нельзя: изображение по-разному растянет по осям $x$ и $y$ и понадобится теорема Пифагора}
    \end{align*}
}
\solutionspace{150pt}

\tasknumber{8}%
\task{%
    Доказать формулу тонкой линзы для рассеивающей линзы.
}
\solutionspace{120pt}

\tasknumber{9}%
\task{%
    Постройте ход луча $AL$ в тонкой линзе.
    Известно положение линзы и оба её фокуса (см.
    рис.
    на доске).
    Рассмотрите оба типа линзы, сделав 2 рисунка: собирающую и рассеивающую.
}
\solutionspace{120pt}

\tasknumber{10}%
\task{%
    На экране, расположенном иа расстоянии $120\,\text{см}$ от собирающей линзы,
    получено изображение точечного источника, расположенного на главной оптической оси линзы.
    На какое расстояние переместится изображение на экране,
    если при неподвижной линзе переместить источник на $3\,\text{см}$ в плоскости, перпендикулярной главной оптической оси?
    Фокусное расстояние линзы равно $30\,\text{см}$.
}
\answer{%
    \begin{align*}
    &\frac 1F = \frac 1a + \frac 1b \implies a = \frac{bF}{b-F} \implies \Gamma = \frac ba = \frac{b-F}F \\
    &y = x \cdot \Gamma = x \cdot \frac{b-F}F \implies d = y = 9\,\text{см}.
    \end{align*}
}
\solutionspace{120pt}

\tasknumber{11}%
\task{%
    Оптическая сила двояковыпуклой линзы в воздухе $5{,}5\,\text{дптр}$, а в воде $1{,}5\,\text{дптр}$.
    Определить показатель преломления $n$ материала, из которого изготовлена линза.
    Показатель преломления воды равен $1{,}33$.
}
\answer{%
    \begin{align*}
    D_1 &=\cbr{\frac n{n_1} - 1}\cbr{\frac 1{R_1} + \frac 1{R_2}}, \\
    D_2 &=\cbr{\frac n{n_2} - 1}\cbr{\frac 1{R_1} + \frac 1{R_2}}, \\
    \frac {D_2}{D_1} &=\frac{\frac n{n_2} - 1}{\frac n{n_1} - 1} \implies {D_2}\cbr{\frac n{n_1} - 1} = {D_1}\cbr{\frac n{n_2} - 1}  \implies n\cbr{\frac{D_2}{n_1} - \frac{D_1}{n_2}} = D_2 - D_1, \\
    n &= \frac{D_2 - D_1}{\frac{D_2}{n_1} - \frac{D_1}{n_2}} = \frac{n_1 n_2 (D_2 - D_1)}{D_2n_2 - D_1n_1} \approx 1{,}518.
    \end{align*}
}
\solutionspace{120pt}

\tasknumber{12}%
\task{%
    На каком расстоянии от собирающей линзы с фокусным расстоянием $50\,\text{дптр}$
    следует надо поместить предмет, чтобы расстояние
    от предмета до его действительного изображения было наименьшим?
}
\answer{%
    \begin{align*}
    \frac 1a &+ \frac 1b = D \implies b = \frac 1{D - \frac 1a} \implies \ell = a + b = a + \frac a{Da - 1} = \frac{ Da^2 }{Da - 1} \implies \\
    \implies \ell'_a &= \frac{ 2Da \cdot (Da - 1) - Da^2 \cdot D }{\sqr{Da - 1}}= \frac{ D^2a^2 - 2Da}{\sqr{Da - 1}} = \frac{ Da(Da - 2)}{\sqr{Da - 1}}\implies a_{\min} = \frac 2D \approx 40\,\text{мм}.
    \end{align*}
}
\solutionspace{120pt}

\tasknumber{13}%
\task{%
    Даны точечный источник света $S$, его изображение $S_1$, полученное с помощью собирающей линзы,
    и ближайший к источнику фокус линзы $F$ (см.
    рис.
    на доске).
    Расстояния $SF = \ell$ и $SS_1 = L$.
    Определить положение линзы и её фокусное расстояние.
}
\answer{%
    \begin{align*}
    \frac 1a + \frac 1b &= \frac 1F, \ell = a - F, L = a + b \implies a = \ell + F, b = L - a = L - \ell - F \\
    \frac 1{\ell + F} + \frac 1{L - \ell - F} &= \frac 1F \\
    F\ell + F^2 + LF - F\ell - F^2 &= L\ell - \ell^2 - F\ell + LF - F\ell - F^2 \\
    0 &= L\ell - \ell^2 - 2F\ell - F^2 \\
    0 &=  F^2 + 2F\ell - L\ell + \ell^2 \\
    F &= -\ell \pm \sqrt{\ell^2 +  L\ell - \ell^2} = -\ell \pm \sqrt{L\ell} \implies F = \sqrt{L\ell} - \ell \\
    a &= \ell + F = \ell + \sqrt{L\ell} - \ell = \sqrt{L\ell}.
    \end{align*}
}
\solutionspace{120pt}

\tasknumber{14}%
\task{%
    Расстояние от освещённого предмета до экрана $80\,\text{см}$.
    Линза, помещенная между ними, даёт чёткое изображение предмета на
    экране при двух положениях, расстояние между которыми $40\,\text{см}$.
    Найти фокусное расстояние линзы.
}
\answer{%
    \begin{align*}
    \frac 1a + \frac 1b &= \frac 1F, \frac 1{a-\ell} + \frac 1{b+\ell} = \frac 1F, a + b = L \\
    \frac 1a + \frac 1b &= \frac 1{a-\ell} + \frac 1{b+\ell}\implies \frac{a + b}{ab} = \frac{(a-\ell) + (b+\ell)}{(a-\ell)(b+\ell)} \\
    ab  &= (a - \ell)(b+\ell) \implies 0  = -b\ell + a\ell - \ell^2 \implies 0 = -b + a - \ell \implies b = a - \ell \\
    a + (a - \ell) &= L \implies a = \frac{L + \ell}2 \implies b = \frac{L - \ell}2 \\
    F &= \frac{ab}{a + b} = \frac{L^2 -\ell^2}{4L} \approx 15\,\text{см}.
    \end{align*}
}
\solutionspace{120pt}

\tasknumber{15}%
\task{%
    Предмет находится на расстоянии $70\,\text{см}$ от экрана.
    Между предметом и экраном помещают линзу, причём при одном
    положении линзы на экране получается увеличенное изображение предмета,
    а при другом — уменьшенное.
    Каково фокусное расстояние линзы, если
    линейные размеры первого изображения в два раза больше второго?
}
\answer{%
    \begin{align*}
    \frac 1a + \frac 1{L-a} &= \frac 1F, h_1 = h \cdot \frac{L-a}a, \\
    \frac 1b + \frac 1{L-b} &= \frac 1F, h_2 = h \cdot \frac{L-b}b, \\
    \frac{h_1}{h_2} &= 2 \implies \frac{(L-a)b}{(L-b)a} = 2, \\
    \frac 1F &= \frac{ L }{a(L-a)} = \frac{ L }{b(L-b)} \implies \frac{L-a}{L-b} = \frac b a \implies \frac {b^2}{a^2} = 2.
    \\
    \frac 1a + \frac 1{L-a} &= \frac 1b + \frac 1{L-b} \implies \frac L{a(L-a)} = \frac L{b(L-b)} \implies \\
    \implies aL - a^2 &= bL - b^2 \implies (a-b)L = (a-b)(a+b) \implies b = L - a, \\
    \frac{\sqr{L-a}}{a^2} &= 2 \implies \frac La - 1 = \sqrt{2} \implies a = \frac{ L }{\sqrt{2} + 1} \\
    F &= \frac{a(L-a)}L = \frac 1L \cdot \frac L{\sqrt{2} + 1} \cdot \frac {L\sqrt{2}}{\sqrt{2} + 1}= \frac { L\sqrt{2} }{ \sqr{\sqrt{2} + 1} } \approx 17{,}0\,\text{см}.
    \end{align*}
}
\solutionspace{120pt}

\tasknumber{16}%
\task{%
    (Задача-«гроб»: решать на обратной стороне) Квадрат со стороной $d = 2\,\text{см}$ расположен так,
    что 2 его стороны параллельны главной оптической оси рассеивающей линзы,
    его центр удален на $h = 5\,\text{см}$ от этой оси и на $a = 15\,\text{см}$ от плоскости линзы.
    Определите площадь изображения квадрата, если фокусное расстояние линзы составляет $F = 20\,\text{см}$.
    % (и сравните с площадью объекта, умноженной на квадрат увеличения центра квадрата).
}
\answer{%
    \begin{align*}
    &\text{Все явные вычисления — в см и $\text{см}^2$,} \\
    \frac 1 F &= \frac 1{a + \frac d2} + \frac 1b \implies b = \frac 1{\frac 1 F - \frac 1{a + \frac d2}} = \frac{F(a + \frac d2)}{a + \frac d2 - F} = -\frac{80}9, \\
    \frac 1 F &= \frac 1{a - \frac d2} + \frac 1c \implies c = \frac 1{\frac 1 F - \frac 1{a - \frac d2}} = \frac{F(a - \frac d2)}{a - \frac d2 - F} = -\frac{140}{17}, \\
    c - b &= \frac{F(a - \frac d2)}{a - \frac d2 - F} - \frac{F(a + \frac d2)}{a + \frac d2 - F} = F\cbr{ \frac{a - \frac d2}{a - \frac d2 - F} - \frac{a + \frac d2}{a + \frac d2 - F} } =  \\
    &= F \cdot \frac{a^2 + \frac {ad}2 - aF - \frac{ad}2 - \frac{d^2}4 + \frac{dF}2 - a^2 + \frac {ad}2 + aF - \frac{ad}2 + \frac{d^2}4 + \frac{dF}2}{\cbr{a + \frac d2 - F}\cbr{a - \frac d2 - F}}= F \cdot \frac {dF}{\cbr{a + \frac d2 - F}\cbr{a - \frac d2 - F}} = \frac{100}{153}.
    \\
    \Gamma_b &= \frac b{a + \frac d2} = \frac{ F }{a + \frac d2 - F} = -\frac59, \\
    \Gamma_c &= \frac c{a - \frac d2} = \frac{ F }{a - \frac d2 - F} = -\frac{10}{17}, \\
    &\text{ тут интересно отметить, что } \Gamma_x = \frac{ c - b}{ d } = \frac{ F^2 }{\cbr{a + \frac d2 - F}\cbr{a - \frac d2 - F}} \ne \Gamma_b \text{ или } \Gamma_c \text{ даже при малых $d$}.
    \\
    S' &= \frac{d \cdot \Gamma_b + d \cdot \Gamma_c}2 \cdot (c - b) = \frac d2 \cbr{\frac{ F }{a + \frac d2 - F} + \frac{ F }{a - \frac d2 - F}} \cdot \cbr{c - b} =  \\
    &=\frac {dF}2 \cbr{\frac 1{a + \frac d2 - F} + \frac 1{a - \frac d2 - F}} \cdot \frac {dF^2}{\cbr{a + \frac d2 - F}\cbr{a - \frac d2 - F}} =  \\
    &=\frac {dF}2 \cdot \frac{a - \frac d2 - F + a + \frac d2 - F}{\cbr{a + \frac d2 - F}\cbr{a - \frac d2 - F}} \cdot \frac {dF^2}{\cbr{a + \frac d2 - F}\cbr{a - \frac d2 - F}} =  \\
    &= \frac {d^2F^3}{2\sqr{a + \frac d2 - F}\sqr{a - \frac d2 - F}} \cdot (2a - 2F) = \frac {d^2F^3(a - F)}{ \sqr{\sqr{a - F} - \frac{d^2}4} } = -\frac{17500}{23409}.
    \end{align*}
}

\variantsplitter

\addpersonalvariant{Егор Свистушкин}

\tasknumber{1}%
\task{%
    Запишите известные вам виды классификации изображений.
}
\solutionspace{60pt}

\tasknumber{2}%
\task{%
    В каких линзах можно получить увеличенное изображение объекта?
}
\answer{%
    $\text{ рассеивающие }$
}
\solutionspace{40pt}

\tasknumber{3}%
\task{%
    Какое изображение называют действительным?
}
\solutionspace{40pt}

\tasknumber{4}%
\task{%
    Есть две линзы, обозначим их 1 и 2.
    Известно что оптическая сила линзы 1 больше, чем у линзы 2.
    Какая линза сильнее преломляет лучи?
}
\answer{%
    $1$
}
\solutionspace{40pt}

\tasknumber{5}%
\task{%
    Предмет находится на расстоянии $10\,\text{см}$ от рассеивающей линзы с фокусным расстоянием $25\,\text{см}$.
    Определите тип изображения, расстояние между предметом и его изображением, увеличение предмета.
    Сделайте схематичный рисунок (не обязательно в масштабе, но с сохранением свойств линзы и изображения).
}
\solutionspace{100pt}

\tasknumber{6}%
\task{%
    Объект находится на расстоянии $45\,\text{см}$ от линзы, а его действительное изображение — в $50\,\text{см}$ от неё.
    Определите увеличение предмета, фокусное расстояние линзы, оптическую силу линзы и её тип.
}
\solutionspace{80pt}

\tasknumber{7}%
\task{%
    Известно, что из формулы тонкой линзы $\cbr{\frac 1F = \frac 1a + \frac 1b}$
    и определения увеличения $\cbr{\Gamma_y = \frac ba}$ можно получить выражение
    для увеличения: $\Gamma_y = \frac {aF}{a - F} \cdot \frac 1a = \frac {F}{a - F}.$
    Назовём такое увеличение «поперечным»: поперёк главной оптической оси (поэтому и ${}_y$).
    Получите формулу для «продольного» увеличения $\Gamma_x$ небольшого предмета, находящегося на главной оптической оси.
    Можно ли применить эту формулу для предмета, не лежащего на главной оптической оси, почему?
}
\answer{%
    \begin{align*}
    \frac 1F &= \frac 1a + \frac 1b \implies b = \frac {aF}{a - F} \\
    \frac 1F &= \frac 1{a + x} + \frac 1c \implies c = \frac {(a+x)F}{a + x - F} \\
    x' &= \abs{b - c} = \frac {aF}{a - F} - \frac {(a+x)F}{a + x - F} = F\cbr{\frac {a}{a - F} - \frac {a+x}{a + x - F}} =  \\
    &= F \cdot \frac {a^2 + ax - aF - a^2 - ax + aF + xF}{(a - F)(a + x - F)} = F \cdot \frac {xF}{(a - F)(a + x - F)} \\
    \Gamma_x &= \frac{x'}x = \frac{F^2}{(a - F)(a + x - F)} \to \frac{F^2}{\sqr{a - F}}.
    \\
    &\text{Нельзя: изображение по-разному растянет по осям $x$ и $y$ и понадобится теорема Пифагора}
    \end{align*}
}
\solutionspace{150pt}

\tasknumber{8}%
\task{%
    Доказать формулу тонкой линзы для собирающей линзы.
}
\solutionspace{120pt}

\tasknumber{9}%
\task{%
    Постройте ход луча $CK$ в тонкой линзе.
    Известно положение линзы и оба её фокуса (см.
    рис.
    на доске).
    Рассмотрите оба типа линзы, сделав 2 рисунка: собирающую и рассеивающую.
}
\solutionspace{120pt}

\tasknumber{10}%
\task{%
    На экране, расположенном иа расстоянии $60\,\text{см}$ от собирающей линзы,
    получено изображение точечного источника, расположенного на главной оптической оси линзы.
    На какое расстояние переместится изображение на экране,
    если при неподвижной линзе переместить источник на $3\,\text{см}$ в плоскости, перпендикулярной главной оптической оси?
    Фокусное расстояние линзы равно $30\,\text{см}$.
}
\answer{%
    \begin{align*}
    &\frac 1F = \frac 1a + \frac 1b \implies a = \frac{bF}{b-F} \implies \Gamma = \frac ba = \frac{b-F}F \\
    &y = x \cdot \Gamma = x \cdot \frac{b-F}F \implies d = y = 3\,\text{см}.
    \end{align*}
}
\solutionspace{120pt}

\tasknumber{11}%
\task{%
    Оптическая сила двояковыпуклой линзы в воздухе $5\,\text{дптр}$, а в воде $1{,}5\,\text{дптр}$.
    Определить показатель преломления $n$ материала, из которого изготовлена линза.
    Показатель преломления воды равен $1{,}33$.
}
\answer{%
    \begin{align*}
    D_1 &=\cbr{\frac n{n_1} - 1}\cbr{\frac 1{R_1} + \frac 1{R_2}}, \\
    D_2 &=\cbr{\frac n{n_2} - 1}\cbr{\frac 1{R_1} + \frac 1{R_2}}, \\
    \frac {D_2}{D_1} &=\frac{\frac n{n_2} - 1}{\frac n{n_1} - 1} \implies {D_2}\cbr{\frac n{n_1} - 1} = {D_1}\cbr{\frac n{n_2} - 1}  \implies n\cbr{\frac{D_2}{n_1} - \frac{D_1}{n_2}} = D_2 - D_1, \\
    n &= \frac{D_2 - D_1}{\frac{D_2}{n_1} - \frac{D_1}{n_2}} = \frac{n_1 n_2 (D_2 - D_1)}{D_2n_2 - D_1n_1} \approx 1{,}549.
    \end{align*}
}
\solutionspace{120pt}

\tasknumber{12}%
\task{%
    На каком расстоянии от собирающей линзы с фокусным расстоянием $50\,\text{дптр}$
    следует надо поместить предмет, чтобы расстояние
    от предмета до его действительного изображения было наименьшим?
}
\answer{%
    \begin{align*}
    \frac 1a &+ \frac 1b = D \implies b = \frac 1{D - \frac 1a} \implies \ell = a + b = a + \frac a{Da - 1} = \frac{ Da^2 }{Da - 1} \implies \\
    \implies \ell'_a &= \frac{ 2Da \cdot (Da - 1) - Da^2 \cdot D }{\sqr{Da - 1}}= \frac{ D^2a^2 - 2Da}{\sqr{Da - 1}} = \frac{ Da(Da - 2)}{\sqr{Da - 1}}\implies a_{\min} = \frac 2D \approx 40\,\text{мм}.
    \end{align*}
}
\solutionspace{120pt}

\tasknumber{13}%
\task{%
    Даны точечный источник света $S$, его изображение $S_1$, полученное с помощью собирающей линзы,
    и ближайший к источнику фокус линзы $F$ (см.
    рис.
    на доске).
    Расстояния $SF = \ell$ и $SS_1 = L$.
    Определить положение линзы и её фокусное расстояние.
}
\answer{%
    \begin{align*}
    \frac 1a + \frac 1b &= \frac 1F, \ell = a - F, L = a + b \implies a = \ell + F, b = L - a = L - \ell - F \\
    \frac 1{\ell + F} + \frac 1{L - \ell - F} &= \frac 1F \\
    F\ell + F^2 + LF - F\ell - F^2 &= L\ell - \ell^2 - F\ell + LF - F\ell - F^2 \\
    0 &= L\ell - \ell^2 - 2F\ell - F^2 \\
    0 &=  F^2 + 2F\ell - L\ell + \ell^2 \\
    F &= -\ell \pm \sqrt{\ell^2 +  L\ell - \ell^2} = -\ell \pm \sqrt{L\ell} \implies F = \sqrt{L\ell} - \ell \\
    a &= \ell + F = \ell + \sqrt{L\ell} - \ell = \sqrt{L\ell}.
    \end{align*}
}
\solutionspace{120pt}

\tasknumber{14}%
\task{%
    Расстояние от освещённого предмета до экрана $80\,\text{см}$.
    Линза, помещенная между ними, даёт чёткое изображение предмета на
    экране при двух положениях, расстояние между которыми $30\,\text{см}$.
    Найти фокусное расстояние линзы.
}
\answer{%
    \begin{align*}
    \frac 1a + \frac 1b &= \frac 1F, \frac 1{a-\ell} + \frac 1{b+\ell} = \frac 1F, a + b = L \\
    \frac 1a + \frac 1b &= \frac 1{a-\ell} + \frac 1{b+\ell}\implies \frac{a + b}{ab} = \frac{(a-\ell) + (b+\ell)}{(a-\ell)(b+\ell)} \\
    ab  &= (a - \ell)(b+\ell) \implies 0  = -b\ell + a\ell - \ell^2 \implies 0 = -b + a - \ell \implies b = a - \ell \\
    a + (a - \ell) &= L \implies a = \frac{L + \ell}2 \implies b = \frac{L - \ell}2 \\
    F &= \frac{ab}{a + b} = \frac{L^2 -\ell^2}{4L} \approx 17{,}2\,\text{см}.
    \end{align*}
}
\solutionspace{120pt}

\tasknumber{15}%
\task{%
    Предмет находится на расстоянии $80\,\text{см}$ от экрана.
    Между предметом и экраном помещают линзу, причём при одном
    положении линзы на экране получается увеличенное изображение предмета,
    а при другом — уменьшенное.
    Каково фокусное расстояние линзы, если
    линейные размеры первого изображения в два раза больше второго?
}
\answer{%
    \begin{align*}
    \frac 1a + \frac 1{L-a} &= \frac 1F, h_1 = h \cdot \frac{L-a}a, \\
    \frac 1b + \frac 1{L-b} &= \frac 1F, h_2 = h \cdot \frac{L-b}b, \\
    \frac{h_1}{h_2} &= 2 \implies \frac{(L-a)b}{(L-b)a} = 2, \\
    \frac 1F &= \frac{ L }{a(L-a)} = \frac{ L }{b(L-b)} \implies \frac{L-a}{L-b} = \frac b a \implies \frac {b^2}{a^2} = 2.
    \\
    \frac 1a + \frac 1{L-a} &= \frac 1b + \frac 1{L-b} \implies \frac L{a(L-a)} = \frac L{b(L-b)} \implies \\
    \implies aL - a^2 &= bL - b^2 \implies (a-b)L = (a-b)(a+b) \implies b = L - a, \\
    \frac{\sqr{L-a}}{a^2} &= 2 \implies \frac La - 1 = \sqrt{2} \implies a = \frac{ L }{\sqrt{2} + 1} \\
    F &= \frac{a(L-a)}L = \frac 1L \cdot \frac L{\sqrt{2} + 1} \cdot \frac {L\sqrt{2}}{\sqrt{2} + 1}= \frac { L\sqrt{2} }{ \sqr{\sqrt{2} + 1} } \approx 19{,}4\,\text{см}.
    \end{align*}
}
\solutionspace{120pt}

\tasknumber{16}%
\task{%
    (Задача-«гроб»: решать на обратной стороне) Квадрат со стороной $d = 1\,\text{см}$ расположен так,
    что 2 его стороны параллельны главной оптической оси рассеивающей линзы,
    его центр удален на $h = 4\,\text{см}$ от этой оси и на $a = 15\,\text{см}$ от плоскости линзы.
    Определите площадь изображения квадрата, если фокусное расстояние линзы составляет $F = 20\,\text{см}$.
    % (и сравните с площадью объекта, умноженной на квадрат увеличения центра квадрата).
}
\answer{%
    \begin{align*}
    &\text{Все явные вычисления — в см и $\text{см}^2$,} \\
    \frac 1 F &= \frac 1{a + \frac d2} + \frac 1b \implies b = \frac 1{\frac 1 F - \frac 1{a + \frac d2}} = \frac{F(a + \frac d2)}{a + \frac d2 - F} = -\frac{620}{71}, \\
    \frac 1 F &= \frac 1{a - \frac d2} + \frac 1c \implies c = \frac 1{\frac 1 F - \frac 1{a - \frac d2}} = \frac{F(a - \frac d2)}{a - \frac d2 - F} = -\frac{580}{69}, \\
    c - b &= \frac{F(a - \frac d2)}{a - \frac d2 - F} - \frac{F(a + \frac d2)}{a + \frac d2 - F} = F\cbr{ \frac{a - \frac d2}{a - \frac d2 - F} - \frac{a + \frac d2}{a + \frac d2 - F} } =  \\
    &= F \cdot \frac{a^2 + \frac {ad}2 - aF - \frac{ad}2 - \frac{d^2}4 + \frac{dF}2 - a^2 + \frac {ad}2 + aF - \frac{ad}2 + \frac{d^2}4 + \frac{dF}2}{\cbr{a + \frac d2 - F}\cbr{a - \frac d2 - F}}= F \cdot \frac {dF}{\cbr{a + \frac d2 - F}\cbr{a - \frac d2 - F}} = \frac{1600}{4899}.
    \\
    \Gamma_b &= \frac b{a + \frac d2} = \frac{ F }{a + \frac d2 - F} = -\frac{40}{71}, \\
    \Gamma_c &= \frac c{a - \frac d2} = \frac{ F }{a - \frac d2 - F} = -\frac{40}{69}, \\
    &\text{ тут интересно отметить, что } \Gamma_x = \frac{ c - b}{ d } = \frac{ F^2 }{\cbr{a + \frac d2 - F}\cbr{a - \frac d2 - F}} \ne \Gamma_b \text{ или } \Gamma_c \text{ даже при малых $d$}.
    \\
    S' &= \frac{d \cdot \Gamma_b + d \cdot \Gamma_c}2 \cdot (c - b) = \frac d2 \cbr{\frac{ F }{a + \frac d2 - F} + \frac{ F }{a - \frac d2 - F}} \cdot \cbr{c - b} =  \\
    &=\frac {dF}2 \cbr{\frac 1{a + \frac d2 - F} + \frac 1{a - \frac d2 - F}} \cdot \frac {dF^2}{\cbr{a + \frac d2 - F}\cbr{a - \frac d2 - F}} =  \\
    &=\frac {dF}2 \cdot \frac{a - \frac d2 - F + a + \frac d2 - F}{\cbr{a + \frac d2 - F}\cbr{a - \frac d2 - F}} \cdot \frac {dF^2}{\cbr{a + \frac d2 - F}\cbr{a - \frac d2 - F}} =  \\
    &= \frac {d^2F^3}{2\sqr{a + \frac d2 - F}\sqr{a - \frac d2 - F}} \cdot (2a - 2F) = \frac {d^2F^3(a - F)}{ \sqr{\sqr{a - F} - \frac{d^2}4} } = -\frac{4480000}{24000201}.
    \end{align*}
}

\variantsplitter

\addpersonalvariant{Дмитрий Соколов}

\tasknumber{1}%
\task{%
    Запишите известные вам виды классификации изображений.
}
\solutionspace{60pt}

\tasknumber{2}%
\task{%
    В каких линзах можно получить мнимое изображение объекта?
}
\answer{%
    $\text{ собирающие и рассеивающие }$
}
\solutionspace{40pt}

\tasknumber{3}%
\task{%
    Какое изображение называют мнимым?
}
\solutionspace{40pt}

\tasknumber{4}%
\task{%
    Есть две линзы, обозначим их 1 и 2.
    Известно что оптическая сила линзы 2 больше, чем у линзы 1.
    Какая линза сильнее преломляет лучи?
}
\answer{%
    $2$
}
\solutionspace{40pt}

\tasknumber{5}%
\task{%
    Предмет находится на расстоянии $30\,\text{см}$ от собирающей линзы с фокусным расстоянием $15\,\text{см}$.
    Определите тип изображения, расстояние между предметом и его изображением, увеличение предмета.
    Сделайте схематичный рисунок (не обязательно в масштабе, но с сохранением свойств линзы и изображения).
}
\solutionspace{100pt}

\tasknumber{6}%
\task{%
    Объект находится на расстоянии $45\,\text{см}$ от линзы, а его мнимое изображение — в $20\,\text{см}$ от неё.
    Определите увеличение предмета, фокусное расстояние линзы, оптическую силу линзы и её тип.
}
\solutionspace{80pt}

\tasknumber{7}%
\task{%
    Известно, что из формулы тонкой линзы $\cbr{\frac 1F = \frac 1a + \frac 1b}$
    и определения увеличения $\cbr{\Gamma_y = \frac ba}$ можно получить выражение
    для увеличения: $\Gamma_y = \frac {aF}{a - F} \cdot \frac 1a = \frac {F}{a - F}.$
    Назовём такое увеличение «поперечным»: поперёк главной оптической оси (поэтому и ${}_y$).
    Получите формулу для «продольного» увеличения $\Gamma_x$ небольшого предмета, находящегося на главной оптической оси.
    Можно ли применить эту формулу для предмета, не лежащего на главной оптической оси, почему?
}
\answer{%
    \begin{align*}
    \frac 1F &= \frac 1a + \frac 1b \implies b = \frac {aF}{a - F} \\
    \frac 1F &= \frac 1{a + x} + \frac 1c \implies c = \frac {(a+x)F}{a + x - F} \\
    x' &= \abs{b - c} = \frac {aF}{a - F} - \frac {(a+x)F}{a + x - F} = F\cbr{\frac {a}{a - F} - \frac {a+x}{a + x - F}} =  \\
    &= F \cdot \frac {a^2 + ax - aF - a^2 - ax + aF + xF}{(a - F)(a + x - F)} = F \cdot \frac {xF}{(a - F)(a + x - F)} \\
    \Gamma_x &= \frac{x'}x = \frac{F^2}{(a - F)(a + x - F)} \to \frac{F^2}{\sqr{a - F}}.
    \\
    &\text{Нельзя: изображение по-разному растянет по осям $x$ и $y$ и понадобится теорема Пифагора}
    \end{align*}
}
\solutionspace{150pt}

\tasknumber{8}%
\task{%
    Доказать формулу тонкой линзы для рассеивающей линзы.
}
\solutionspace{120pt}

\tasknumber{9}%
\task{%
    Постройте ход луча $CK$ в тонкой линзе.
    Известно положение линзы и оба её фокуса (см.
    рис.
    на доске).
    Рассмотрите оба типа линзы, сделав 2 рисунка: собирающую и рассеивающую.
}
\solutionspace{120pt}

\tasknumber{10}%
\task{%
    На экране, расположенном иа расстоянии $60\,\text{см}$ от собирающей линзы,
    получено изображение точечного источника, расположенного на главной оптической оси линзы.
    На какое расстояние переместится изображение на экране,
    если при неподвижном источнике переместить линзу на $2\,\text{см}$ в плоскости, перпендикулярной главной оптической оси?
    Фокусное расстояние линзы равно $30\,\text{см}$.
}
\answer{%
    \begin{align*}
    &\frac 1F = \frac 1a + \frac 1b \implies a = \frac{bF}{b-F} \implies \Gamma = \frac ba = \frac{b-F}F \\
    &y = x \cdot \Gamma = x \cdot \frac{b-F}F \implies d = x + y = 4\,\text{см}.
    \end{align*}
}
\solutionspace{120pt}

\tasknumber{11}%
\task{%
    Оптическая сила двояковыпуклой линзы в воздухе $5{,}5\,\text{дптр}$, а в воде $1{,}4\,\text{дптр}$.
    Определить показатель преломления $n$ материала, из которого изготовлена линза.
    Показатель преломления воды равен $1{,}33$.
}
\answer{%
    \begin{align*}
    D_1 &=\cbr{\frac n{n_1} - 1}\cbr{\frac 1{R_1} + \frac 1{R_2}}, \\
    D_2 &=\cbr{\frac n{n_2} - 1}\cbr{\frac 1{R_1} + \frac 1{R_2}}, \\
    \frac {D_2}{D_1} &=\frac{\frac n{n_2} - 1}{\frac n{n_1} - 1} \implies {D_2}\cbr{\frac n{n_1} - 1} = {D_1}\cbr{\frac n{n_2} - 1}  \implies n\cbr{\frac{D_2}{n_1} - \frac{D_1}{n_2}} = D_2 - D_1, \\
    n &= \frac{D_2 - D_1}{\frac{D_2}{n_1} - \frac{D_1}{n_2}} = \frac{n_1 n_2 (D_2 - D_1)}{D_2n_2 - D_1n_1} \approx 1{,}499.
    \end{align*}
}
\solutionspace{120pt}

\tasknumber{12}%
\task{%
    На каком расстоянии от собирающей линзы с фокусным расстоянием $50\,\text{дптр}$
    следует надо поместить предмет, чтобы расстояние
    от предмета до его действительного изображения было наименьшим?
}
\answer{%
    \begin{align*}
    \frac 1a &+ \frac 1b = D \implies b = \frac 1{D - \frac 1a} \implies \ell = a + b = a + \frac a{Da - 1} = \frac{ Da^2 }{Da - 1} \implies \\
    \implies \ell'_a &= \frac{ 2Da \cdot (Da - 1) - Da^2 \cdot D }{\sqr{Da - 1}}= \frac{ D^2a^2 - 2Da}{\sqr{Da - 1}} = \frac{ Da(Da - 2)}{\sqr{Da - 1}}\implies a_{\min} = \frac 2D \approx 40\,\text{мм}.
    \end{align*}
}
\solutionspace{120pt}

\tasknumber{13}%
\task{%
    Даны точечный источник света $S$, его изображение $S_1$, полученное с помощью собирающей линзы,
    и ближайший к источнику фокус линзы $F$ (см.
    рис.
    на доске).
    Расстояния $SF = \ell$ и $SS_1 = L$.
    Определить положение линзы и её фокусное расстояние.
}
\answer{%
    \begin{align*}
    \frac 1a + \frac 1b &= \frac 1F, \ell = a - F, L = a + b \implies a = \ell + F, b = L - a = L - \ell - F \\
    \frac 1{\ell + F} + \frac 1{L - \ell - F} &= \frac 1F \\
    F\ell + F^2 + LF - F\ell - F^2 &= L\ell - \ell^2 - F\ell + LF - F\ell - F^2 \\
    0 &= L\ell - \ell^2 - 2F\ell - F^2 \\
    0 &=  F^2 + 2F\ell - L\ell + \ell^2 \\
    F &= -\ell \pm \sqrt{\ell^2 +  L\ell - \ell^2} = -\ell \pm \sqrt{L\ell} \implies F = \sqrt{L\ell} - \ell \\
    a &= \ell + F = \ell + \sqrt{L\ell} - \ell = \sqrt{L\ell}.
    \end{align*}
}
\solutionspace{120pt}

\tasknumber{14}%
\task{%
    Расстояние от освещённого предмета до экрана $80\,\text{см}$.
    Линза, помещенная между ними, даёт чёткое изображение предмета на
    экране при двух положениях, расстояние между которыми $30\,\text{см}$.
    Найти фокусное расстояние линзы.
}
\answer{%
    \begin{align*}
    \frac 1a + \frac 1b &= \frac 1F, \frac 1{a-\ell} + \frac 1{b+\ell} = \frac 1F, a + b = L \\
    \frac 1a + \frac 1b &= \frac 1{a-\ell} + \frac 1{b+\ell}\implies \frac{a + b}{ab} = \frac{(a-\ell) + (b+\ell)}{(a-\ell)(b+\ell)} \\
    ab  &= (a - \ell)(b+\ell) \implies 0  = -b\ell + a\ell - \ell^2 \implies 0 = -b + a - \ell \implies b = a - \ell \\
    a + (a - \ell) &= L \implies a = \frac{L + \ell}2 \implies b = \frac{L - \ell}2 \\
    F &= \frac{ab}{a + b} = \frac{L^2 -\ell^2}{4L} \approx 17{,}2\,\text{см}.
    \end{align*}
}
\solutionspace{120pt}

\tasknumber{15}%
\task{%
    Предмет находится на расстоянии $90\,\text{см}$ от экрана.
    Между предметом и экраном помещают линзу, причём при одном
    положении линзы на экране получается увеличенное изображение предмета,
    а при другом — уменьшенное.
    Каково фокусное расстояние линзы, если
    линейные размеры первого изображения в два раза больше второго?
}
\answer{%
    \begin{align*}
    \frac 1a + \frac 1{L-a} &= \frac 1F, h_1 = h \cdot \frac{L-a}a, \\
    \frac 1b + \frac 1{L-b} &= \frac 1F, h_2 = h \cdot \frac{L-b}b, \\
    \frac{h_1}{h_2} &= 2 \implies \frac{(L-a)b}{(L-b)a} = 2, \\
    \frac 1F &= \frac{ L }{a(L-a)} = \frac{ L }{b(L-b)} \implies \frac{L-a}{L-b} = \frac b a \implies \frac {b^2}{a^2} = 2.
    \\
    \frac 1a + \frac 1{L-a} &= \frac 1b + \frac 1{L-b} \implies \frac L{a(L-a)} = \frac L{b(L-b)} \implies \\
    \implies aL - a^2 &= bL - b^2 \implies (a-b)L = (a-b)(a+b) \implies b = L - a, \\
    \frac{\sqr{L-a}}{a^2} &= 2 \implies \frac La - 1 = \sqrt{2} \implies a = \frac{ L }{\sqrt{2} + 1} \\
    F &= \frac{a(L-a)}L = \frac 1L \cdot \frac L{\sqrt{2} + 1} \cdot \frac {L\sqrt{2}}{\sqrt{2} + 1}= \frac { L\sqrt{2} }{ \sqr{\sqrt{2} + 1} } \approx 22\,\text{см}.
    \end{align*}
}
\solutionspace{120pt}

\tasknumber{16}%
\task{%
    (Задача-«гроб»: решать на обратной стороне) Квадрат со стороной $d = 3\,\text{см}$ расположен так,
    что 2 его стороны параллельны главной оптической оси собирающей линзы,
    его центр удален на $h = 4\,\text{см}$ от этой оси и на $a = 15\,\text{см}$ от плоскости линзы.
    Определите площадь изображения квадрата, если фокусное расстояние линзы составляет $F = 18\,\text{см}$.
    % (и сравните с площадью объекта, умноженной на квадрат увеличения центра квадрата).
}
\answer{%
    \begin{align*}
    &\text{Все явные вычисления — в см и $\text{см}^2$,} \\
    \frac 1 F &= \frac 1{a + \frac d2} + \frac 1b \implies b = \frac 1{\frac 1 F - \frac 1{a + \frac d2}} = \frac{F(a + \frac d2)}{a + \frac d2 - F} = -198, \\
    \frac 1 F &= \frac 1{a - \frac d2} + \frac 1c \implies c = \frac 1{\frac 1 F - \frac 1{a - \frac d2}} = \frac{F(a - \frac d2)}{a - \frac d2 - F} = -54, \\
    c - b &= \frac{F(a - \frac d2)}{a - \frac d2 - F} - \frac{F(a + \frac d2)}{a + \frac d2 - F} = F\cbr{ \frac{a - \frac d2}{a - \frac d2 - F} - \frac{a + \frac d2}{a + \frac d2 - F} } =  \\
    &= F \cdot \frac{a^2 + \frac {ad}2 - aF - \frac{ad}2 - \frac{d^2}4 + \frac{dF}2 - a^2 + \frac {ad}2 + aF - \frac{ad}2 + \frac{d^2}4 + \frac{dF}2}{\cbr{a + \frac d2 - F}\cbr{a - \frac d2 - F}}= F \cdot \frac {dF}{\cbr{a + \frac d2 - F}\cbr{a - \frac d2 - F}} = 144.
    \\
    \Gamma_b &= \frac b{a + \frac d2} = \frac{ F }{a + \frac d2 - F} = -12, \\
    \Gamma_c &= \frac c{a - \frac d2} = \frac{ F }{a - \frac d2 - F} = -4, \\
    &\text{ тут интересно отметить, что } \Gamma_x = \frac{ c - b}{ d } = \frac{ F^2 }{\cbr{a + \frac d2 - F}\cbr{a - \frac d2 - F}} \ne \Gamma_b \text{ или } \Gamma_c \text{ даже при малых $d$}.
    \\
    S' &= \frac{d \cdot \Gamma_b + d \cdot \Gamma_c}2 \cdot (c - b) = \frac d2 \cbr{\frac{ F }{a + \frac d2 - F} + \frac{ F }{a - \frac d2 - F}} \cdot \cbr{c - b} =  \\
    &=\frac {dF}2 \cbr{\frac 1{a + \frac d2 - F} + \frac 1{a - \frac d2 - F}} \cdot \frac {dF^2}{\cbr{a + \frac d2 - F}\cbr{a - \frac d2 - F}} =  \\
    &=\frac {dF}2 \cdot \frac{a - \frac d2 - F + a + \frac d2 - F}{\cbr{a + \frac d2 - F}\cbr{a - \frac d2 - F}} \cdot \frac {dF^2}{\cbr{a + \frac d2 - F}\cbr{a - \frac d2 - F}} =  \\
    &= \frac {d^2F^3}{2\sqr{a + \frac d2 - F}\sqr{a - \frac d2 - F}} \cdot (2a - 2F) = \frac {d^2F^3(a - F)}{ \sqr{\sqr{a - F} - \frac{d^2}4} } = -3456.
    \end{align*}
}

\variantsplitter

\addpersonalvariant{Арсений Трофимов}

\tasknumber{1}%
\task{%
    Запишите формулу тонкой линзы и сделайте рисунок, указав на нём физические величины из этой формулы.
}
\solutionspace{60pt}

\tasknumber{2}%
\task{%
    В каких линзах можно получить мнимое изображение объекта?
}
\answer{%
    $\text{ собирающие и рассеивающие }$
}
\solutionspace{40pt}

\tasknumber{3}%
\task{%
    Какое изображение называют мнимым?
}
\solutionspace{40pt}

\tasknumber{4}%
\task{%
    Есть две линзы, обозначим их 1 и 2.
    Известно что фокусное расстояние линзы 2 меньше, чем у линзы 1.
    Какая линза сильнее преломляет лучи?
}
\answer{%
    $2$
}
\solutionspace{40pt}

\tasknumber{5}%
\task{%
    Предмет находится на расстоянии $10\,\text{см}$ от рассеивающей линзы с фокусным расстоянием $40\,\text{см}$.
    Определите тип изображения, расстояние между предметом и его изображением, увеличение предмета.
    Сделайте схематичный рисунок (не обязательно в масштабе, но с сохранением свойств линзы и изображения).
}
\solutionspace{100pt}

\tasknumber{6}%
\task{%
    Объект находится на расстоянии $115\,\text{см}$ от линзы, а его мнимое изображение — в $30\,\text{см}$ от неё.
    Определите увеличение предмета, фокусное расстояние линзы, оптическую силу линзы и её тип.
}
\solutionspace{80pt}

\tasknumber{7}%
\task{%
    Известно, что из формулы тонкой линзы $\cbr{\frac 1F = \frac 1a + \frac 1b}$
    и определения увеличения $\cbr{\Gamma_y = \frac ba}$ можно получить выражение
    для увеличения: $\Gamma_y = \frac {aF}{a - F} \cdot \frac 1a = \frac {F}{a - F}.$
    Назовём такое увеличение «поперечным»: поперёк главной оптической оси (поэтому и ${}_y$).
    Получите формулу для «продольного» увеличения $\Gamma_x$ небольшого предмета, находящегося на главной оптической оси.
    Можно ли применить эту формулу для предмета, не лежащего на главной оптической оси, почему?
}
\answer{%
    \begin{align*}
    \frac 1F &= \frac 1a + \frac 1b \implies b = \frac {aF}{a - F} \\
    \frac 1F &= \frac 1{a + x} + \frac 1c \implies c = \frac {(a+x)F}{a + x - F} \\
    x' &= \abs{b - c} = \frac {aF}{a - F} - \frac {(a+x)F}{a + x - F} = F\cbr{\frac {a}{a - F} - \frac {a+x}{a + x - F}} =  \\
    &= F \cdot \frac {a^2 + ax - aF - a^2 - ax + aF + xF}{(a - F)(a + x - F)} = F \cdot \frac {xF}{(a - F)(a + x - F)} \\
    \Gamma_x &= \frac{x'}x = \frac{F^2}{(a - F)(a + x - F)} \to \frac{F^2}{\sqr{a - F}}.
    \\
    &\text{Нельзя: изображение по-разному растянет по осям $x$ и $y$ и понадобится теорема Пифагора}
    \end{align*}
}
\solutionspace{150pt}

\tasknumber{8}%
\task{%
    Доказать формулу тонкой линзы для рассеивающей линзы.
}
\solutionspace{120pt}

\tasknumber{9}%
\task{%
    Постройте ход луча $CL$ в тонкой линзе.
    Известно положение линзы и оба её фокуса (см.
    рис.
    на доске).
    Рассмотрите оба типа линзы, сделав 2 рисунка: собирающую и рассеивающую.
}
\solutionspace{120pt}

\tasknumber{10}%
\task{%
    На экране, расположенном иа расстоянии $120\,\text{см}$ от собирающей линзы,
    получено изображение точечного источника, расположенного на главной оптической оси линзы.
    На какое расстояние переместится изображение на экране,
    если при неподвижном источнике переместить линзу на $2\,\text{см}$ в плоскости, перпендикулярной главной оптической оси?
    Фокусное расстояние линзы равно $40\,\text{см}$.
}
\answer{%
    \begin{align*}
    &\frac 1F = \frac 1a + \frac 1b \implies a = \frac{bF}{b-F} \implies \Gamma = \frac ba = \frac{b-F}F \\
    &y = x \cdot \Gamma = x \cdot \frac{b-F}F \implies d = x + y = 6\,\text{см}.
    \end{align*}
}
\solutionspace{120pt}

\tasknumber{11}%
\task{%
    Оптическая сила двояковыпуклой линзы в воздухе $5\,\text{дптр}$, а в воде $1{,}6\,\text{дптр}$.
    Определить показатель преломления $n$ материала, из которого изготовлена линза.
    Показатель преломления воды равен $1{,}33$.
}
\answer{%
    \begin{align*}
    D_1 &=\cbr{\frac n{n_1} - 1}\cbr{\frac 1{R_1} + \frac 1{R_2}}, \\
    D_2 &=\cbr{\frac n{n_2} - 1}\cbr{\frac 1{R_1} + \frac 1{R_2}}, \\
    \frac {D_2}{D_1} &=\frac{\frac n{n_2} - 1}{\frac n{n_1} - 1} \implies {D_2}\cbr{\frac n{n_1} - 1} = {D_1}\cbr{\frac n{n_2} - 1}  \implies n\cbr{\frac{D_2}{n_1} - \frac{D_1}{n_2}} = D_2 - D_1, \\
    n &= \frac{D_2 - D_1}{\frac{D_2}{n_1} - \frac{D_1}{n_2}} = \frac{n_1 n_2 (D_2 - D_1)}{D_2n_2 - D_1n_1} \approx 1{,}575.
    \end{align*}
}
\solutionspace{120pt}

\tasknumber{12}%
\task{%
    На каком расстоянии от собирающей линзы с фокусным расстоянием $30\,\text{дптр}$
    следует надо поместить предмет, чтобы расстояние
    от предмета до его действительного изображения было наименьшим?
}
\answer{%
    \begin{align*}
    \frac 1a &+ \frac 1b = D \implies b = \frac 1{D - \frac 1a} \implies \ell = a + b = a + \frac a{Da - 1} = \frac{ Da^2 }{Da - 1} \implies \\
    \implies \ell'_a &= \frac{ 2Da \cdot (Da - 1) - Da^2 \cdot D }{\sqr{Da - 1}}= \frac{ D^2a^2 - 2Da}{\sqr{Da - 1}} = \frac{ Da(Da - 2)}{\sqr{Da - 1}}\implies a_{\min} = \frac 2D \approx 66{,}7\,\text{мм}.
    \end{align*}
}
\solutionspace{120pt}

\tasknumber{13}%
\task{%
    Даны точечный источник света $S$, его изображение $S_1$, полученное с помощью собирающей линзы,
    и ближайший к источнику фокус линзы $F$ (см.
    рис.
    на доске).
    Расстояния $SF = \ell$ и $SS_1 = L$.
    Определить положение линзы и её фокусное расстояние.
}
\answer{%
    \begin{align*}
    \frac 1a + \frac 1b &= \frac 1F, \ell = a - F, L = a + b \implies a = \ell + F, b = L - a = L - \ell - F \\
    \frac 1{\ell + F} + \frac 1{L - \ell - F} &= \frac 1F \\
    F\ell + F^2 + LF - F\ell - F^2 &= L\ell - \ell^2 - F\ell + LF - F\ell - F^2 \\
    0 &= L\ell - \ell^2 - 2F\ell - F^2 \\
    0 &=  F^2 + 2F\ell - L\ell + \ell^2 \\
    F &= -\ell \pm \sqrt{\ell^2 +  L\ell - \ell^2} = -\ell \pm \sqrt{L\ell} \implies F = \sqrt{L\ell} - \ell \\
    a &= \ell + F = \ell + \sqrt{L\ell} - \ell = \sqrt{L\ell}.
    \end{align*}
}
\solutionspace{120pt}

\tasknumber{14}%
\task{%
    Расстояние от освещённого предмета до экрана $100\,\text{см}$.
    Линза, помещенная между ними, даёт чёткое изображение предмета на
    экране при двух положениях, расстояние между которыми $20\,\text{см}$.
    Найти фокусное расстояние линзы.
}
\answer{%
    \begin{align*}
    \frac 1a + \frac 1b &= \frac 1F, \frac 1{a-\ell} + \frac 1{b+\ell} = \frac 1F, a + b = L \\
    \frac 1a + \frac 1b &= \frac 1{a-\ell} + \frac 1{b+\ell}\implies \frac{a + b}{ab} = \frac{(a-\ell) + (b+\ell)}{(a-\ell)(b+\ell)} \\
    ab  &= (a - \ell)(b+\ell) \implies 0  = -b\ell + a\ell - \ell^2 \implies 0 = -b + a - \ell \implies b = a - \ell \\
    a + (a - \ell) &= L \implies a = \frac{L + \ell}2 \implies b = \frac{L - \ell}2 \\
    F &= \frac{ab}{a + b} = \frac{L^2 -\ell^2}{4L} \approx 24\,\text{см}.
    \end{align*}
}
\solutionspace{120pt}

\tasknumber{15}%
\task{%
    Предмет находится на расстоянии $60\,\text{см}$ от экрана.
    Между предметом и экраном помещают линзу, причём при одном
    положении линзы на экране получается увеличенное изображение предмета,
    а при другом — уменьшенное.
    Каково фокусное расстояние линзы, если
    линейные размеры первого изображения в три раза больше второго?
}
\answer{%
    \begin{align*}
    \frac 1a + \frac 1{L-a} &= \frac 1F, h_1 = h \cdot \frac{L-a}a, \\
    \frac 1b + \frac 1{L-b} &= \frac 1F, h_2 = h \cdot \frac{L-b}b, \\
    \frac{h_1}{h_2} &= 3 \implies \frac{(L-a)b}{(L-b)a} = 3, \\
    \frac 1F &= \frac{ L }{a(L-a)} = \frac{ L }{b(L-b)} \implies \frac{L-a}{L-b} = \frac b a \implies \frac {b^2}{a^2} = 3.
    \\
    \frac 1a + \frac 1{L-a} &= \frac 1b + \frac 1{L-b} \implies \frac L{a(L-a)} = \frac L{b(L-b)} \implies \\
    \implies aL - a^2 &= bL - b^2 \implies (a-b)L = (a-b)(a+b) \implies b = L - a, \\
    \frac{\sqr{L-a}}{a^2} &= 3 \implies \frac La - 1 = \sqrt{3} \implies a = \frac{ L }{\sqrt{3} + 1} \\
    F &= \frac{a(L-a)}L = \frac 1L \cdot \frac L{\sqrt{3} + 1} \cdot \frac {L\sqrt{3}}{\sqrt{3} + 1}= \frac { L\sqrt{3} }{ \sqr{\sqrt{3} + 1} } \approx 13{,}9\,\text{см}.
    \end{align*}
}
\solutionspace{120pt}

\tasknumber{16}%
\task{%
    (Задача-«гроб»: решать на обратной стороне) Квадрат со стороной $d = 3\,\text{см}$ расположен так,
    что 2 его стороны параллельны главной оптической оси собирающей линзы,
    его центр удален на $h = 6\,\text{см}$ от этой оси и на $a = 12\,\text{см}$ от плоскости линзы.
    Определите площадь изображения квадрата, если фокусное расстояние линзы составляет $F = 20\,\text{см}$.
    % (и сравните с площадью объекта, умноженной на квадрат увеличения центра квадрата).
}
\answer{%
    \begin{align*}
    &\text{Все явные вычисления — в см и $\text{см}^2$,} \\
    \frac 1 F &= \frac 1{a + \frac d2} + \frac 1b \implies b = \frac 1{\frac 1 F - \frac 1{a + \frac d2}} = \frac{F(a + \frac d2)}{a + \frac d2 - F} = -\frac{540}{13}, \\
    \frac 1 F &= \frac 1{a - \frac d2} + \frac 1c \implies c = \frac 1{\frac 1 F - \frac 1{a - \frac d2}} = \frac{F(a - \frac d2)}{a - \frac d2 - F} = -\frac{420}{19}, \\
    c - b &= \frac{F(a - \frac d2)}{a - \frac d2 - F} - \frac{F(a + \frac d2)}{a + \frac d2 - F} = F\cbr{ \frac{a - \frac d2}{a - \frac d2 - F} - \frac{a + \frac d2}{a + \frac d2 - F} } =  \\
    &= F \cdot \frac{a^2 + \frac {ad}2 - aF - \frac{ad}2 - \frac{d^2}4 + \frac{dF}2 - a^2 + \frac {ad}2 + aF - \frac{ad}2 + \frac{d^2}4 + \frac{dF}2}{\cbr{a + \frac d2 - F}\cbr{a - \frac d2 - F}}= F \cdot \frac {dF}{\cbr{a + \frac d2 - F}\cbr{a - \frac d2 - F}} = \frac{4800}{247}.
    \\
    \Gamma_b &= \frac b{a + \frac d2} = \frac{ F }{a + \frac d2 - F} = -\frac{40}{13}, \\
    \Gamma_c &= \frac c{a - \frac d2} = \frac{ F }{a - \frac d2 - F} = -\frac{40}{19}, \\
    &\text{ тут интересно отметить, что } \Gamma_x = \frac{ c - b}{ d } = \frac{ F^2 }{\cbr{a + \frac d2 - F}\cbr{a - \frac d2 - F}} \ne \Gamma_b \text{ или } \Gamma_c \text{ даже при малых $d$}.
    \\
    S' &= \frac{d \cdot \Gamma_b + d \cdot \Gamma_c}2 \cdot (c - b) = \frac d2 \cbr{\frac{ F }{a + \frac d2 - F} + \frac{ F }{a - \frac d2 - F}} \cdot \cbr{c - b} =  \\
    &=\frac {dF}2 \cbr{\frac 1{a + \frac d2 - F} + \frac 1{a - \frac d2 - F}} \cdot \frac {dF^2}{\cbr{a + \frac d2 - F}\cbr{a - \frac d2 - F}} =  \\
    &=\frac {dF}2 \cdot \frac{a - \frac d2 - F + a + \frac d2 - F}{\cbr{a + \frac d2 - F}\cbr{a - \frac d2 - F}} \cdot \frac {dF^2}{\cbr{a + \frac d2 - F}\cbr{a - \frac d2 - F}} =  \\
    &= \frac {d^2F^3}{2\sqr{a + \frac d2 - F}\sqr{a - \frac d2 - F}} \cdot (2a - 2F) = \frac {d^2F^3(a - F)}{ \sqr{\sqr{a - F} - \frac{d^2}4} } = -\frac{9216000}{61009}.
    \end{align*}
}
% autogenerated
