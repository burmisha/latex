\newcommand\rootpath{../../..}
\documentclass[12pt,a4paper]{amsart}%DVI-mode.
\usepackage{graphics,graphicx,epsfig}%DVI-mode.
% \documentclass[pdftex,12pt]{amsart} %PDF-mode.
% \usepackage[pdftex]{graphicx}       %PDF-mode.
% \usepackage[babel=true]{microtype}
% \usepackage[T1]{fontenc}
% \usepackage{lmodern}

\usepackage{cmap}
%\usepackage{a4wide}                 % Fit the text to A4 page tightly.
% \usepackage[utf8]{inputenc}
\usepackage[T2A]{fontenc}
\usepackage[english,russian]{babel} % Download Russian fonts.
\usepackage{amsmath,amsfonts,amssymb,amsthm,amscd,mathrsfs} % Use AMS symbols.
\usepackage{tikz}
\usetikzlibrary{circuits.ee.IEC}
\usetikzlibrary{shapes.geometric}
\usetikzlibrary{decorations.markings}
%\usetikzlibrary{dashs}
%\usetikzlibrary{info}


\textheight=28cm % высота текста
\textwidth=18cm % ширина текста
\topmargin=-2.5cm % отступ от верхнего края
\parskip=2pt % интервал между абзацами
\oddsidemargin=-1.5cm
\evensidemargin=-1.5cm 

\parindent=0pt % абзацный отступ
\tolerance=500 % терпимость к "жидким" строкам
\binoppenalty=10000 % штраф за перенос формул - 10000 - абсолютный запрет
\relpenalty=10000
\flushbottom % выравнивание высоты страниц
\pagenumbering{gobble}

\newcommand\bivec[2]{\begin{pmatrix} #1 \\ #2 \end{pmatrix}}

\newcommand\ol[1]{\overline{#1}}

\newcommand\p[1]{\Prob\!\left(#1\right)}
\newcommand\e[1]{\mathsf{E}\!\left(#1\right)}
\newcommand\disp[1]{\mathsf{D}\!\left(#1\right)}
%\newcommand\norm[2]{\mathcal{N}\!\cbr{#1,#2}}
\newcommand\sign{\text{ sign }}

\newcommand\al[1]{\begin{align*} #1 \end{align*}}
\newcommand\begcas[1]{\begin{cases}#1\end{cases}}
\newcommand\tab[2]{	\vspace{-#1pt}
						\begin{tabbing} 
						#2
						\end{tabbing}
					\vspace{-#1pt}
					}

\newcommand\maintext[1]{{\bfseries\sffamily{#1}}}
\newcommand\skipped[1]{ {\ensuremath{\text{\small{\sffamily{Пропущено:} #1} } } } }
\newcommand\simpletitle[1]{\begin{center} \maintext{#1} \end{center}}

\def\le{\leqslant}
\def\ge{\geqslant}
\def\Ell{\mathcal{L}}
\def\eps{{\varepsilon}}
\def\Rn{\mathbb{R}^n}
\def\RSS{\mathsf{RSS}}

\newcommand\foral[1]{\forall\,#1\:}
\newcommand\exist[1]{\exists\,#1\:\colon}

\newcommand\cbr[1]{\left(#1\right)} %circled brackets
\newcommand\fbr[1]{\left\{#1\right\}} %figure brackets
\newcommand\sbr[1]{\left[#1\right]} %square brackets
\newcommand\modul[1]{\left|#1\right|}

\newcommand\sqr[1]{\cbr{#1}^2}
\newcommand\inv[1]{\cbr{#1}^{-1}}

\newcommand\cdf[2]{\cdot\frac{#1}{#2}}
\newcommand\dd[2]{\frac{\partial#1}{\partial#2}}

\newcommand\integr[2]{\int\limits_{#1}^{#2}}
\newcommand\suml[2]{\sum\limits_{#1}^{#2}}
\newcommand\isum[2]{\sum\limits_{#1=#2}^{+\infty}}
\newcommand\idots[3]{#1_{#2},\ldots,#1_{#3}}
\newcommand\fdots[5]{#4{#1_{#2}}#5\ldots#5#4{#1_{#3}}}

\newcommand\obol[1]{O\!\cbr{#1}}
\newcommand\omal[1]{o\!\cbr{#1}}

\newcommand\addeps[2]{
	\begin{figure} [!ht] %lrp
		\centering
		\includegraphics[height=320px]{#1.eps}
		\vspace{-10pt}
		\caption{#2}
		\label{eps:#1}
	\end{figure}
}

\newcommand\addepssize[3]{
	\begin{figure} [!ht] %lrp hp
		\centering
		\includegraphics[height=#3px]{#1.eps}
		\vspace{-10pt}
		\caption{#2}
		\label{eps:#1}
	\end{figure}
}


\newcommand\norm[1]{\ensuremath{\left\|{#1}\right\|}}
\newcommand\ort{\bot}
\newcommand\theorem[1]{{\sffamily Теорема #1\ }}
\newcommand\lemma[1]{{\sffamily Лемма #1\ }}
\newcommand\difflim[2]{\frac{#1\cbr{#2 + \Delta#2} - #1\cbr{#2}}{\Delta #2}}
\renewcommand\proof[1]{\par\noindent$\square$ #1 \hfill$\blacksquare$\par}
\newcommand\defenition[1]{{\sffamilyОпределение #1\ }}

% \begin{document}
% %\raggedright
% \addclassdate{7}{20 апреля 2018}

\task 1
Площадь большого поршня гидравлического домкрата $S_1 = 20\units{см}^2$, а малого $S_2 = 0{,}5\units{см}^2.$ Груз какой максимальной массы можно поднять этим домкратом, если на малый поршень давить с силой не более $F=200\units{Н}?$ Силой трения от стенки цилиндров пренебречь.

\task 2
В сосуд налита вода. Расстояние от поверхности воды до дна $H = 0{,}5\units{м},$ площадь дна $S = 0{,}1\units{м}^2.$ Найти гидростатическое давление $P_1$ и полное давление $P_2$ вблизи дна. Найти силу давления воды на дно. Плотность воды \rhowater

\task 3
На лёгкий поршень площадью $S=900\units{см}^2,$ касающийся поверхности воды, поставили гирю массы $m=3\units{кг}$. Высота слоя воды в сосуде с вертикальными стенками $H = 20\units{см}$. Определить давление жидкости вблизи дна, если плотность воды \rhowater

\task 4
Давление газов в конце сгорания в цилиндре дизельного двигателя трактора $P = 9\units{МПа}.$ Диаметр цилиндра $d = 130\units{мм}.$ С какой силой газы давят на поршень в цилиндре? Площадь круга диаметром $D$ равна $S = \cfrac{\pi D^2}4.$

\task 5
Площадь малого поршня гидравлического подъёмника $S_1 = 0{,}8\units{см}^2$, а большого $S_2 = 40\units{см}^2.$ Какую силу $F$ надо приложить к малому поршню, чтобы поднять груз весом $P = 8\units{кН}?$

\task 6
Герметичный сосуд полностью заполнен водой и стоит на столе. На небольшой поршень площадью $S$ давят рукой с силой $F$. Поршень находится ниже крышки сосуда на $H_1$, выше дна на $H_2$ и может свободно перемещаться. Плотность воды $\rho$, атмосферное давление $P_A$. Найти давления $P_1$ и $P_2$ в воде вблизи крышки и дна сосуда.
\\ \\
\addclassdate{7}{20 апреля 2018}

\task 1
Площадь большого поршня гидравлического домкрата $S_1 = 20\units{см}^2$, а малого $S_2 = 0{,}5\units{см}^2.$ Груз какой максимальной массы можно поднять этим домкратом, если на малый поршень давить с силой не более $F=200\units{Н}?$ Силой трения от стенки цилиндров пренебречь.

\task 2
В сосуд налита вода. Расстояние от поверхности воды до дна $H = 0{,}5\units{м},$ площадь дна $S = 0{,}1\units{м}^2.$ Найти гидростатическое давление $P_1$ и полное давление $P_2$ вблизи дна. Найти силу давления воды на дно. Плотность воды \rhowater

\task 3
На лёгкий поршень площадью $S=900\units{см}^2,$ касающийся поверхности воды, поставили гирю массы $m=3\units{кг}$. Высота слоя воды в сосуде с вертикальными стенками $H = 20\units{см}$. Определить давление жидкости вблизи дна, если плотность воды \rhowater

\task 4
Давление газов в конце сгорания в цилиндре дизельного двигателя трактора $P = 9\units{МПа}.$ Диаметр цилиндра $d = 130\units{мм}.$ С какой силой газы давят на поршень в цилиндре? Площадь круга диаметром $D$ равна $S = \cfrac{\pi D^2}4.$

\task 5
Площадь малого поршня гидравлического подъёмника $S_1 = 0{,}8\units{см}^2$, а большого $S_2 = 40\units{см}^2.$ Какую силу $F$ надо приложить к малому поршню, чтобы поднять груз весом $P = 8\units{кН}?$

\task 6
Герметичный сосуд полностью заполнен водой и стоит на столе. На небольшой поршень площадью $S$ давят рукой с силой $F$. Поршень находится ниже крышки сосуда на $H_1$, выше дна на $H_2$ и может свободно перемещаться. Плотность воды $\rho$, атмосферное давление $P_A$. Найти давления $P_1$ и $P_2$ в воде вблизи крышки и дна сосуда.

\newpage

\adddate{8 класс. 20 апреля 2018}

\task 1
Между точками $A$ и $B$ электрической цепи подключены последовательно резисторы $R_1 = 10\units{Ом}$ и $R_2 = 20\units{Ом}$ и параллельно им $R_3 = 30\units{Ом}.$ Найдите эквивалентное сопротивление $R_{AB}$ этого участка цепи.

\task 2
Электрическая цепь состоит из последовательности $N$ одинаковых звеньев, в которых каждый резистор имеет сопротивление $r$. Последнее звено замкнуто резистором сопротивлением $R$. При каком соотношении $\cfrac{R}{r}$ сопротивление цепи не зависит от числа звеньев?

\task 3
Для измерения сопротивления $R$ проводника собрана электрическая цепь. Вольтметр $V$ показывает напряжение $U_V = 5\units{В},$ показание амперметра $A$ равно $I_A = 25\units{мА}.$ Найдите величину $R$ сопротивления проводника. Внутреннее сопротивление вольтметра $R_V = 1{,}0\units{кОм},$ внутреннее сопротивление амперметра $R_A = 2{,}0\units{Ом}.$

\task 4
Шкала гальванометра имеет $N=100$ делений, цена деления $\delta = 1\units{мкА}$. Внутреннее сопротивление гальванометра $R_G = 1{,}0\units{кОм}.$ Как из этого прибора сделать вольтметр для измерения напряжений до $U = 100\units{В}$ или амперметр для измерения токов силой до $I = 1\units{А}?$

\\ \\ \\ \\ \\ \\ \\ \\
\adddate{8 класс. 20 апреля 2018}

\task 1
Между точками $A$ и $B$ электрической цепи подключены последовательно резисторы $R_1 = 10\units{Ом}$ и $R_2 = 20\units{Ом}$ и параллельно им $R_3 = 30\units{Ом}.$ Найдите эквивалентное сопротивление $R_{AB}$ этого участка цепи.

\task 2
Электрическая цепь состоит из последовательности $N$ одинаковых звеньев, в которых каждый резистор имеет сопротивление $r$. Последнее звено замкнуто резистором сопротивлением $R$. При каком соотношении $\cfrac{R}{r}$ сопротивление цепи не зависит от числа звеньев?

\task 3
Для измерения сопротивления $R$ проводника собрана электрическая цепь. Вольтметр $V$ показывает напряжение $U_V = 5\units{В},$ показание амперметра $A$ равно $I_A = 25\units{мА}.$ Найдите величину $R$ сопротивления проводника. Внутреннее сопротивление вольтметра $R_V = 1{,}0\units{кОм},$ внутреннее сопротивление амперметра $R_A = 2{,}0\units{Ом}.$

\task 4
Шкала гальванометра имеет $N=100$ делений, цена деления $\delta = 1\units{мкА}$. Внутреннее сопротивление гальванометра $R_G = 1{,}0\units{кОм}.$ Как из этого прибора сделать вольтметр для измерения напряжений до $U = 100\units{В}$ или амперметр для измерения токов силой до $I = 1\units{А}?$


% % \begin{flushright}
\textsc{ГБОУ школа №554, 20 ноября 2018\,г.}
\end{flushright}

\begin{center}
\LARGE \textsc{Математический бой, 8 класс}
\end{center}

\problem{1} Есть тридцать карточек, на каждой написано по одному числу: на десяти карточках~–~$a$,  на десяти других~–~$b$ и на десяти оставшихся~–~$c$ (числа  различны). Известно, что к любым пяти карточкам можно подобрать ещё пять так, что сумма чисел на этих десяти карточках будет равна нулю. Докажите, что~одно из~чисел~$a, b, c$ равно нулю.

\problem{2} Вокруг стола стола пустили пакет с орешками. Первый взял один орешек, второй — 2, третий — 3 и так далее: каждый следующий брал на 1 орешек больше. Известно, что на втором круге было взято в сумме на 100 орешков больше, чем на первом. Сколько человек сидело за столом?

% \problem{2} Натуральное число разрешено увеличить на любое целое число процентов от 1 до 100, если при этом получаем натуральное число. Найдите наименьшее натуральное число, которое нельзя при помощи таких операций получить из~числа 1.

% \problem{3} Найти сумму $1^2 - 2^2 + 3^2 - 4^2 + 5^2 + \ldots - 2018^2$.

\problem{3} В кружке рукоделия, где занимается Валя, более 93\% участников~—~девочки. Какое наименьшее число участников может быть в таком кружке?

\problem{4} Произведение 2018 целых чисел равно 1. Может ли их сумма оказаться равной~0?

% \problem{4} Можно ли все натуральные числа от~1 до~9 записать в~клетки таблицы~$3\times3$ так, чтобы сумма в~любых двух соседних (по~вертикали или горизонтали) клетках равнялось простому числу?

\problem{5} На доске написано 2018 нулей и 2019 единиц. Женя стирает 2 числа и, если они были одинаковы, дописывает к оставшимся один ноль, а~если разные — единицу. Потом Женя повторяет эту операцию снова, потом ещё и~так далее. В~результате на~доске останется только одно число. Что это за~число?

\problem{6} Докажите, что в~любой компании людей найдутся 2~человека, имеющие равное число знакомых в этой компании (если $A$~знаком с~$B$, то~и $B$~знаком с~$A$).

\problem{7} Три колокола начинают бить одновременно. Интервалы между ударами колоколов соответственно составляют $\cfrac43$~секунды, $\cfrac53$~секунды и $2$~секунды. Совпавшие по времени удары воспринимаются за~один. Сколько ударов будет услышано за 1~минуту, включая первый и последний удары?

\problem{8} Восемь одинаковых момент расположены по кругу. Известно, что три из~них~— фальшивые, и они расположены рядом друг с~другом. Вес фальшивой монеты отличается от~веса настоящей. Все фальшивые монеты весят одинаково, но неизвестно, тяжелее или легче фальшивая монета настоящей. Покажите, что за~3~взвешивания на~чашечных весах без~гирь можно определить все фальшивые монеты.

% \end{document}

\begin{document}

\setdate{26~января~2022}
\setclass{11«БА»}

\addpersonalvariant{Михаил Бурмистров}

\tasknumber{1}%
\task{%
    Найти оптическую силу собирающей линзы, если действительное изображение предмета,
    помещённого в $15\,\text{см}$ от линзы, получается на расстоянии $30\,\text{см}$ от неё.
}
\answer{%
    $D = \frac 1F = \frac 1a + \frac 1b = \frac 1{15\,\text{см}} + \frac 1{30\,\text{см}} \approx 10\,\text{дптр}$
}
\solutionspace{80pt}

\tasknumber{2}%
\task{%
    Найти увеличение изображения, если изображение предмета, находящегося
    на расстоянии $20\,\text{см}$ от линзы, получается на расстоянии $30\,\text{см}$ от неё.
}
\answer{%
    $\Gamma = \frac ba = \frac {30\,\text{см}}{20\,\text{см}} \approx 1{,}50$
}
\solutionspace{80pt}

\tasknumber{3}%
\task{%
    Расстояние от предмета до линзы $10\,\text{см}$, а от линзы до мнимого изображения $30\,\text{см}$.
    Чему равно фокусное расстояние линзы?
}
\answer{%
    $\pm \frac 1F = \frac 1a - \frac 1b \implies F = \frac{a b}{\abs{b - a}} \approx 15\,\text{см}$
}
\solutionspace{80pt}

\tasknumber{4}%
\task{%
    Две тонкие собирающие линзы с фокусными расстояниями $25\,\text{см}$ и $30\,\text{см}$ сложены вместе.
    Чему равно фокусное расстояние такой оптической системы?
}
\answer{%
    $\frac 1{f_1} = \frac 1a + \frac 1b; \frac 1{f_2} = - \frac 1b + \frac 1c \implies \frac 1{f_1} + \frac 1{f_2} = \frac 1a + \frac 1c \implies f' = \frac 1{\frac 1{f_1} + \frac 1{f_2}} = \frac{f_1 f_2}{f_1 + f_2} \approx 13{,}6\,\text{см}$
}
\solutionspace{80pt}

\tasknumber{5}%
\task{%
    Линейные размеры прямого изображения предмета, полученного в собирающей линзе,
    в два раза больше линейных размеров предмета.
    Зная, что предмет находится на $40\,\text{см}$ ближе к линзе,
    чем его изображение, найти оптическую силу линзы.
}
\answer{%
    \begin{align*}
    &\text{Если изображение действительное:} \\
    D &= \frac 1F = \frac 1a + \frac 1b, \qquad \Gamma = \frac ba, \qquad b - a = \ell \implies b = \Gamma a \implies \Gamma a - a = \ell \implies  \\
    a &= \frac {\ell}{\Gamma - 1} \implies b = \frac {{\ell} \Gamma}{\Gamma - 1} \implies  \\
    D &= \frac {\Gamma - 1}\ell + \frac {\Gamma - 1}{\ell \Gamma} = \frac 1\ell \cdot \cbr{\Gamma - 1 + \frac {\Gamma - 1}{\Gamma} } =\frac 1\ell \cdot \cbr{\Gamma - \frac 1\Gamma} \approx 3{,}8\,\text{дптр}.
    \\
    &\text{Если изображение мнимое:} \\
    D &= \frac 1F = \frac 1a - \frac 1b, \qquad \Gamma = \frac ba, \qquad b - a = \ell \implies b = \Gamma a \implies \Gamma a - a = \ell \implies  \\
    a &= \frac {\ell}{\Gamma - 1} \implies b = \frac {{\ell} \Gamma}{\Gamma - 1} \implies  \\
    D &= \frac {\Gamma - 1}\ell - \frac {\Gamma - 1}{\ell \Gamma} = \frac 1\ell \cdot \cbr{\Gamma - 1 - \frac {\Gamma - 1}{\Gamma} } =\frac 1\ell \cdot \cbr{\Gamma + \frac 1\Gamma - 2} \approx 1{,}2\,\text{дптр}.
    \\
    &\text{В ответе надо указать оба значения.}
    \end{align*}
}
\solutionspace{120pt}

\tasknumber{6}%
\task{%
    Оптическая сила объектива фотоаппарата равна $6\,\text{дптр}$.
    При фотографировании чертежа с расстояния $1{,}2\,\text{м}$ площадь изображения
    чертежа на фотопластинке оказалась равной $16\,\text{см}^{2}$.
    Какова площадь самого чертежа? Ответ выразите в квадратных сантиметрах.
}
\answer{%
    \begin{align*}
    &\frac 1a + \frac 1b = \frac 1F = D \implies b = \frac{aF}{a - F} \\
    &\frac {S'}S = \Gamma^2 = \sqr{\frac ba} = \sqr{\frac F{a - F}} \implies \\
    &\implies S = S' \cdot \sqr{\frac{a - F}F} = S' \cdot \sqr{\frac aF - 1} = S' \cdot \sqr{aD - 1} \approx 600\,\text{см}^{2}.
    \end{align*}
}


\variantsplitter


\addpersonalvariant{Михаил Бурмистров}

\tasknumber{7}%
\task{%
    В каком месте на главной оптической оси двояковогнутой линзы
    нужно поместить точечный источник света,
    чтобы его изображение оказалось в главном фокусе линзы?
}
\answer{%
    $\text{на половине фокусного расстояния}$
}
\solutionspace{120pt}

\tasknumber{8}%
\task{%
    Предмет в виде отрезка длиной $\ell$ расположен вдоль оптической оси
    собирающей линзы с фокусным расстоянием $F$.
    Середина отрезка расположена
    на расстоянии $a$ от линзы, которая даёт действительное изображение
    всех точек предмета.
    Определить продольное увеличение предмета.
}
\answer{%
    \begin{align*}
    \frac 1{a + \frac \ell 2} &+ \frac 1b = \frac 1F \implies b = \frac{F\cbr{a + \frac \ell 2}}{a + \frac \ell 2 - F} \\
    \frac 1{a - \frac \ell 2} &+ \frac 1c = \frac 1F \implies c = \frac{F\cbr{a - \frac \ell 2}}{a - \frac \ell 2 - F} \\
    \abs{b - c} &= \abs{\frac{F\cbr{a + \frac \ell 2}}{a + \frac \ell 2 - F} - \frac{F\cbr{a - \frac \ell 2}}{a - \frac \ell 2 - F}}= F\abs{\frac{\cbr{a + \frac \ell 2}\cbr{a - \frac \ell 2 - F} - \cbr{a - \frac \ell 2}\cbr{a + \frac \ell 2 - F}}{ \cbr{a + \frac \ell 2 - F} \cbr{a - \frac \ell 2 - F} }} =  \\
    &= F\abs{\frac{a^2 - \frac {a\ell} 2 - Fa + \frac {a\ell} 2 - \frac {\ell^2} 4 - \frac {F\ell}2 - a^2 - \frac {a\ell}2 + aF + \frac {a\ell}2 + \frac {\ell^2} 4 - \frac {F\ell} 2}{\cbr{a + \frac \ell 2 - F} \cbr{a - \frac \ell 2 - F} }} = \\
    &= F\frac{F\ell}{\sqr{a-F} - \frac {\ell^2}4} = \frac{F^2\ell}{\sqr{a-F} - \frac {\ell^2}4}\implies \Gamma = \frac{\abs{b - c}}\ell = \frac{F^2}{\sqr{a-F} - \frac {\ell^2}4}.
    \end{align*}
}
\solutionspace{120pt}

\tasknumber{9}%
\task{%
    На экране с помощью тонкой линзы получено изображение предмета
    с увеличением $2$.
    Предмет передвинули на $4\,\text{см}$.
    Для того, чтобы получить резкое изображение, пришлось передвинуть экран.
    При этом увеличение оказалось равным $6$.
    На какое расстояние
    пришлось передвинуть экран?
}
\answer{%
    \begin{align*}
    &\frac 1a + \frac 1b = \frac 1F, \Gamma_1 = \frac ba = \frac{F}{a-F} \implies \Gamma_1(a-F) = F \implies a = F \cdot \frac{1 + \Gamma_1}{\Gamma_1} \\
    &\frac 1{a + x} + \frac 1{b + y} = \frac 1F, \Gamma_2 = \frac {b+y}{a+x} = \frac{F}{a+x-F} \implies a + x = F \cdot \frac{1 + \Gamma_2}{\Gamma_2} \\
    &1 + \frac xa = \frac{ \frac{1 + \Gamma_2}{\Gamma_2} }{ \frac{1 + \Gamma_1}{\Gamma_1} } = \frac{\Gamma_1(1 + \Gamma_2)}{\Gamma_2(1 + \Gamma_1)} \\
    &a = \frac x{ \frac{\Gamma_1(1 + \Gamma_2)}{\Gamma_2(1 + \Gamma_1)} - 1} = x \cdot \frac{\Gamma_2(1 + \Gamma_1)}{\Gamma_1 - \Gamma_2} \\
    &y = (a + x)\Gamma_2 - b = (a + x)\Gamma_2 - a\Gamma_1 = a(\Gamma_2 - \Gamma_1) + x\Gamma_2 = -x\Gamma_2(1 + \Gamma_1) + x\Gamma_2 = -x\Gamma_2\Gamma_1 = 48\,\text{см}, \\
    &\text{знаки разные, т.е.
    экран надо было подвинуть в другую сторону чем предмет: $x < 0, y > 0$.}
    \end{align*}
}
\solutionspace{120pt}

\tasknumber{10}%
\task{%
    Тонкая собирающая линза дает изображение предмета на экране при двух положениях линзы между предметом и экраном.
    Высота изображения при первом положении $20\,\text{см}$, во втором — $5\,\text{см}$.
    Расстояние между предметом и экранов постоянно.
    Чему равна высота предмета?
}
\answer{%
    \begin{align*}
    &\frac 1a + \frac 1b = \frac 1F, \frac 1c + \frac 1d = \frac 1F, a + b = c + d \implies \frac{a + b}{ab} = \frac 1F = \frac{c+d}{cd} \implies ab = cd, \\
    &\implies ab = c(a + b - c) \implies c^2 - ac - bc + ab = 0 \implies c = a \text{ или } c = b \implies c = b \implies d = a.
    \\
    &\Gamma_1 = \frac {H_1}H = \frac ba, \Gamma_2 = \frac {H_2}H = \frac dc = \frac ab \implies \frac {H_1}H \cdot \frac {H_2}H = \frac ba \cdot \frac ab = 1, \\
    &H = \sqrt{H_1 H_2} \approx 10\,\text{см}.
    \end{align*}
}
\solutionspace{120pt}

\tasknumber{11}%
\task{%
    Какие предметы можно рассмотреть на фотографии, сделанной со спутника,
    если разрешающая способность плёнки $0{,}010\,\text{мм}$? Каким должно быть
    время экспозиции $\tau$ чтобы полностью использовать возможности плёнки?
    Фокусное расстояние объектива используемого фотоаппарата $10\,\text{см}$,
    высота орбиты спутника $150\,\text{км}$.
}
\answer{%
    \begin{align*}
    &H \ll R \implies v = v_{\text{I}} = \sqrt{G R} \approx 7{,}9\,\frac{\text{км}}{\text{с}}.
    \\
    &F \ll H \implies b = F, a = H, \\
    &\Gamma = \frac \delta\ell = \frac ba \implies \ell = \frac{\delta a}b = \frac{\delta H}F \approx \frac{0{,}010\,\text{мм} \cdot 150\,\text{км}}{10\,\text{см}} \approx 15\,\text{м}, \\
    &\implies \tau = \frac \ell v = \frac{\delta H}{F v} = \frac{0{,}010\,\text{мм} \cdot 150\,\text{км}}{10\,\text{см} \cdot 7{,}9\,\frac{\text{км}}{\text{с}}} \approx 1{,}9\,\text{мс}.
    \end{align*}
}


\variantsplitter


\addpersonalvariant{Михаил Бурмистров}

\tasknumber{12}%
\task{%
    При аэрофотосъемках используется фотоаппарат, объектив которого
    имеет фокусиое расстояние $20\,\text{см}$.
    Разрешающая способность плёнки $0{,}015\,\text{мм}$.
    На какой высоте должен лететь самолет, чтобы на фотографии можно
    было различить следы размером $20\,\text{см}$?
    При какой скорости самолета изображение не будет размытым,
    если время экспозиции $1\,\text{мс}$?
}
\answer{%
    \begin{align*}
    &F \ll H \implies b = F, a = H, \\
    &\Gamma = \frac \delta\ell = \frac ba = \frac FH \implies H = \frac{\ell F}\delta = \frac{20\,\text{см} \cdot 20\,\text{см}}{0{,}015\,\text{мм}} \approx 3\,\text{км}, \\
    &\implies v = \frac l\tau = \frac{20\,\text{см}}{1\,\text{мс}} \approx 700\,\frac{\text{км}}{\text{ч}}.
    \end{align*}
}
\solutionspace{120pt}

\tasknumber{13}%
\task{%
    Две одинаковые собиращие линзы установлены так, что их главные оптические оси совпадают,
    а главный фокус первой находится там же, где главный фокус второй.
    Расстояние от первой линзы до предмета равно $10\,\text{см}$.
    Чему равно расстояние от изображения объекта во второй линзе до самого объекта?
    Определите также увеличение.
    Фокусное расстояние каждой линзы $30\,\text{см}$.
}
\answer{%
    \begin{align*}
    \frac 1a + \frac 1b &= \frac 1F \implies b = \frac{aF}{a - F} \implies 2F - b = \frac{2aF - 2F^2 - aF}{a - F} = \frac{F(a - 2F)}{a - F}.
    \\
    \frac 1{2F - b} + \frac 1c &= \frac 1F \implies c = \frac{F(2F-b)}{(2F - b) - F} = \frac{F \cdot \frac{F(a - 2F)}{a - F}}{\frac{F(a - 2F)}{a - F} - F}  = F \cdot \frac{ \frac{F(a - 2F)}{a - F} }{ \frac{F(a - 2F)}{a - F} - 1} = \\
     &= F \cdot \frac{a - 2F}{a - 2F - a + F} = 2F - a = 50\,\text{см}.
     \\
    \ell &= a + 2F + c = 4F = 120\,\text{см}.
    \\
    &\Gamma = \Gamma_1 \cdot \Gamma_2 = \frac ba \cdot \frac c{2F-b} = \frac F{a - F} \cdot \frac{2F - a}{\frac{F(a - 2F)}{a - F}} = -1.
    \end{align*}
}
\solutionspace{120pt}

\tasknumber{14}%
\task{%
    Собирающая линза с фокусным расстоянием $F_1 > 0$ и рассеивающая линза с фокусным расстоянием $F_2 < 0$
    установлены коаксиально на расстоянии $\ell$.
    Пучок параллельных лучей падает на рассеивающую линзу.
    Сделайте схематичное построение и определите, в какой точке система из этих линз соберёт пучок.
}
\answer{%
    \begin{align*}
    &\text{Если пучок падает на собирающую линзу:} \\
    \frac 1{\infty} + \frac 1b &= \frac 1{F_1} \implies b = F_1 \implies \ell - b = \ell - F_1 \\
    \frac 1{\ell - b} + \frac 1c &= \frac 1{F_2} \implies c = \frac{F_2(\ell - b)}{\ell - b - F_2} = \frac{F_2(\ell - F_1)}{\ell - F_1 - F_2}.
    \\
    &\text{Если же пучок падает на рассеивающую линзу:} \\
    \frac 1{\infty} + \frac 1b &= \frac 1{F_2} \implies b = F_2 \implies \ell - b = \ell - F_2 \\
    \frac 1{\ell - b} + \frac 1c &= \frac 1{F_1} \implies c = \frac{F_1(\ell - b)}{\ell - b - F_1} = \frac{F_1(\ell - F_2)}{\ell - F_2 - F_1}.
    \end{align*}
}
\solutionspace{120pt}

\tasknumber{15}%
\task{%
    Две собирающих линзы с фокусными расстояниями $20\,\text{см}$ и $45\,\text{см}$ расположены так,
    что их оптические оси совмещены.
    На первую линзу падает пучок параллельных лучей.
    Пройдя через вторую линзу, он остался параллельным.
    Найдите расстояние между линзами и сделайте рисунок.
}
\answer{%
    \begin{align*}
    \frac 1\infty + \frac 1b &= \frac 1{F_1} \implies b = F_1, \\
    \frac 1{\ell - b} + \frac 1{\infty} &= \frac 1{F_2} \implies \ell - b = F_2 \implies \ell = b + F_2 = F_1 + F_2 = 65\,\text{см}.
    \end{align*}
}

\end{document}
% autogenerated
