\setdate{20~января~2022}
\setclass{11«Б»}

\addpersonalvariant{Михаил Бурмистров}

\tasknumber{1}%
\task{%
    Запишите формулу тонкой линзы и сделайте рисунок, указав на нём физические величины из этой формулы.
}
\solutionspace{60pt}

\tasknumber{2}%
\task{%
    В каких линзах можно получить действительное изображение объекта?
}
\answer{%
    $\text{ собирающие }$
}
\solutionspace{40pt}

\tasknumber{3}%
\task{%
    Какое изображение называют действительным?
}
\solutionspace{40pt}

\tasknumber{4}%
\task{%
    Есть две линзы, обозначим их 1 и 2.
    Известно что оптическая сила линзы 1 больше, чем у линзы 2.
    Какая линза сильнее преломляет лучи?
}
\answer{%
    $1$
}
\solutionspace{40pt}

\tasknumber{5}%
\task{%
    Постройте ход луча $AM$ в тонкой линзе.
    Известно положение линзы и оба её фокуса (см.
    рис.
    на доске).
    Рассмотрите оба типа линзы, сделав 2 рисунка: собирающую и рассеивающую.
}
\solutionspace{120pt}

\tasknumber{6}%
\task{%
    Предмет находится на расстоянии $20\,\text{см}$ от рассеивающей линзы с фокусным расстоянием $12\,\text{см}$.
    Определите тип изображения, расстояние между предметом и его изображением, увеличение предмета.
    Сделайте схематичный рисунок (не обязательно в масштабе, но с сохранением свойств линзы и изображения).
}
\solutionspace{100pt}

\tasknumber{7}%
\task{%
    Объект находится на расстоянии $115\,\text{см}$ от линзы, а его действительное изображение — в $10\,\text{см}$ от неё.
    Определите увеличение предмета, фокусное расстояние линзы, оптическую силу линзы и её тип.
}

\variantsplitter

\addpersonalvariant{Снежана Авдошина}

\tasknumber{1}%
\task{%
    Запишите формулу тонкой линзы и сделайте рисунок, указав на нём физические величины из этой формулы.
}
\solutionspace{60pt}

\tasknumber{2}%
\task{%
    В каких линзах можно получить действительное изображение объекта?
}
\answer{%
    $\text{ собирающие }$
}
\solutionspace{40pt}

\tasknumber{3}%
\task{%
    Какое изображение называют действительным?
}
\solutionspace{40pt}

\tasknumber{4}%
\task{%
    Есть две линзы, обозначим их 1 и 2.
    Известно что фокусное расстояние линзы 2 меньше, чем у линзы 1.
    Какая линза сильнее преломляет лучи?
}
\answer{%
    $2$
}
\solutionspace{40pt}

\tasknumber{5}%
\task{%
    Постройте ход луча $BK$ в тонкой линзе.
    Известно положение линзы и оба её фокуса (см.
    рис.
    на доске).
    Рассмотрите оба типа линзы, сделав 2 рисунка: собирающую и рассеивающую.
}
\solutionspace{120pt}

\tasknumber{6}%
\task{%
    Предмет находится на расстоянии $20\,\text{см}$ от рассеивающей линзы с фокусным расстоянием $15\,\text{см}$.
    Определите тип изображения, расстояние между предметом и его изображением, увеличение предмета.
    Сделайте схематичный рисунок (не обязательно в масштабе, но с сохранением свойств линзы и изображения).
}
\solutionspace{100pt}

\tasknumber{7}%
\task{%
    Объект находится на расстоянии $115\,\text{см}$ от линзы, а его действительное изображение — в $20\,\text{см}$ от неё.
    Определите увеличение предмета, фокусное расстояние линзы, оптическую силу линзы и её тип.
}

\variantsplitter

\addpersonalvariant{Марьяна Аристова}

\tasknumber{1}%
\task{%
    Запишите известные вам виды классификации изображений.
}
\solutionspace{60pt}

\tasknumber{2}%
\task{%
    В каких линзах можно получить уменьшенное изображение объекта?
}
\answer{%
    $\text{ собирающие и рассеивающие }$
}
\solutionspace{40pt}

\tasknumber{3}%
\task{%
    Какое изображение называют мнимым?
}
\solutionspace{40pt}

\tasknumber{4}%
\task{%
    Есть две линзы, обозначим их 1 и 2.
    Известно что фокусное расстояние линзы 2 меньше, чем у линзы 1.
    Какая линза сильнее преломляет лучи?
}
\answer{%
    $2$
}
\solutionspace{40pt}

\tasknumber{5}%
\task{%
    Постройте ход луча $AK$ в тонкой линзе.
    Известно положение линзы и оба её фокуса (см.
    рис.
    на доске).
    Рассмотрите оба типа линзы, сделав 2 рисунка: собирающую и рассеивающую.
}
\solutionspace{120pt}

\tasknumber{6}%
\task{%
    Предмет находится на расстоянии $10\,\text{см}$ от рассеивающей линзы с фокусным расстоянием $12\,\text{см}$.
    Определите тип изображения, расстояние между предметом и его изображением, увеличение предмета.
    Сделайте схематичный рисунок (не обязательно в масштабе, но с сохранением свойств линзы и изображения).
}
\solutionspace{100pt}

\tasknumber{7}%
\task{%
    Объект находится на расстоянии $45\,\text{см}$ от линзы, а его действительное изображение — в $10\,\text{см}$ от неё.
    Определите увеличение предмета, фокусное расстояние линзы, оптическую силу линзы и её тип.
}

\variantsplitter

\addpersonalvariant{Никита Иванов}

\tasknumber{1}%
\task{%
    Запишите формулу тонкой линзы и сделайте рисунок, указав на нём физические величины из этой формулы.
}
\solutionspace{60pt}

\tasknumber{2}%
\task{%
    В каких линзах можно получить увеличенное изображение объекта?
}
\answer{%
    $\text{ рассеивающие }$
}
\solutionspace{40pt}

\tasknumber{3}%
\task{%
    Какое изображение называют действительным?
}
\solutionspace{40pt}

\tasknumber{4}%
\task{%
    Есть две линзы, обозначим их 1 и 2.
    Известно что фокусное расстояние линзы 1 больше, чем у линзы 2.
    Какая линза сильнее преломляет лучи?
}
\answer{%
    $2$
}
\solutionspace{40pt}

\tasknumber{5}%
\task{%
    Постройте ход луча $AK$ в тонкой линзе.
    Известно положение линзы и оба её фокуса (см.
    рис.
    на доске).
    Рассмотрите оба типа линзы, сделав 2 рисунка: собирающую и рассеивающую.
}
\solutionspace{120pt}

\tasknumber{6}%
\task{%
    Предмет находится на расстоянии $30\,\text{см}$ от собирающей линзы с фокусным расстоянием $15\,\text{см}$.
    Определите тип изображения, расстояние между предметом и его изображением, увеличение предмета.
    Сделайте схематичный рисунок (не обязательно в масштабе, но с сохранением свойств линзы и изображения).
}
\solutionspace{100pt}

\tasknumber{7}%
\task{%
    Объект находится на расстоянии $115\,\text{см}$ от линзы, а его действительное изображение — в $20\,\text{см}$ от неё.
    Определите увеличение предмета, фокусное расстояние линзы, оптическую силу линзы и её тип.
}

\variantsplitter

\addpersonalvariant{Анастасия Князева}

\tasknumber{1}%
\task{%
    Запишите известные вам виды классификации изображений.
}
\solutionspace{60pt}

\tasknumber{2}%
\task{%
    В каких линзах можно получить обратное изображение объекта?
}
\answer{%
    $\text{ собирающие }$
}
\solutionspace{40pt}

\tasknumber{3}%
\task{%
    Какое изображение называют действительным?
}
\solutionspace{40pt}

\tasknumber{4}%
\task{%
    Есть две линзы, обозначим их 1 и 2.
    Известно что оптическая сила линзы 1 больше, чем у линзы 2.
    Какая линза сильнее преломляет лучи?
}
\answer{%
    $1$
}
\solutionspace{40pt}

\tasknumber{5}%
\task{%
    Постройте ход луча $AK$ в тонкой линзе.
    Известно положение линзы и оба её фокуса (см.
    рис.
    на доске).
    Рассмотрите оба типа линзы, сделав 2 рисунка: собирающую и рассеивающую.
}
\solutionspace{120pt}

\tasknumber{6}%
\task{%
    Предмет находится на расстоянии $10\,\text{см}$ от собирающей линзы с фокусным расстоянием $50\,\text{см}$.
    Определите тип изображения, расстояние между предметом и его изображением, увеличение предмета.
    Сделайте схематичный рисунок (не обязательно в масштабе, но с сохранением свойств линзы и изображения).
}
\solutionspace{100pt}

\tasknumber{7}%
\task{%
    Объект находится на расстоянии $45\,\text{см}$ от линзы, а его мнимое изображение — в $10\,\text{см}$ от неё.
    Определите увеличение предмета, фокусное расстояние линзы, оптическую силу линзы и её тип.
}

\variantsplitter

\addpersonalvariant{Елизавета Кутумова}

\tasknumber{1}%
\task{%
    Запишите формулу тонкой линзы и сделайте рисунок, указав на нём физические величины из этой формулы.
}
\solutionspace{60pt}

\tasknumber{2}%
\task{%
    В каких линзах можно получить прямое изображение объекта?
}
\answer{%
    $\text{ собирающие и рассеивающие }$
}
\solutionspace{40pt}

\tasknumber{3}%
\task{%
    Какое изображение называют мнимым?
}
\solutionspace{40pt}

\tasknumber{4}%
\task{%
    Есть две линзы, обозначим их 1 и 2.
    Известно что фокусное расстояние линзы 2 больше, чем у линзы 1.
    Какая линза сильнее преломляет лучи?
}
\answer{%
    $1$
}
\solutionspace{40pt}

\tasknumber{5}%
\task{%
    Постройте ход луча $AL$ в тонкой линзе.
    Известно положение линзы и оба её фокуса (см.
    рис.
    на доске).
    Рассмотрите оба типа линзы, сделав 2 рисунка: собирающую и рассеивающую.
}
\solutionspace{120pt}

\tasknumber{6}%
\task{%
    Предмет находится на расстоянии $30\,\text{см}$ от собирающей линзы с фокусным расстоянием $15\,\text{см}$.
    Определите тип изображения, расстояние между предметом и его изображением, увеличение предмета.
    Сделайте схематичный рисунок (не обязательно в масштабе, но с сохранением свойств линзы и изображения).
}
\solutionspace{100pt}

\tasknumber{7}%
\task{%
    Объект находится на расстоянии $45\,\text{см}$ от линзы, а его мнимое изображение — в $40\,\text{см}$ от неё.
    Определите увеличение предмета, фокусное расстояние линзы, оптическую силу линзы и её тип.
}

\variantsplitter

\addpersonalvariant{Роксана Мехтиева}

\tasknumber{1}%
\task{%
    Запишите формулу тонкой линзы и сделайте рисунок, указав на нём физические величины из этой формулы.
}
\solutionspace{60pt}

\tasknumber{2}%
\task{%
    В каких линзах можно получить прямое изображение объекта?
}
\answer{%
    $\text{ собирающие и рассеивающие }$
}
\solutionspace{40pt}

\tasknumber{3}%
\task{%
    Какое изображение называют мнимым?
}
\solutionspace{40pt}

\tasknumber{4}%
\task{%
    Есть две линзы, обозначим их 1 и 2.
    Известно что оптическая сила линзы 1 больше, чем у линзы 2.
    Какая линза сильнее преломляет лучи?
}
\answer{%
    $1$
}
\solutionspace{40pt}

\tasknumber{5}%
\task{%
    Постройте ход луча $BK$ в тонкой линзе.
    Известно положение линзы и оба её фокуса (см.
    рис.
    на доске).
    Рассмотрите оба типа линзы, сделав 2 рисунка: собирающую и рассеивающую.
}
\solutionspace{120pt}

\tasknumber{6}%
\task{%
    Предмет находится на расстоянии $30\,\text{см}$ от рассеивающей линзы с фокусным расстоянием $12\,\text{см}$.
    Определите тип изображения, расстояние между предметом и его изображением, увеличение предмета.
    Сделайте схематичный рисунок (не обязательно в масштабе, но с сохранением свойств линзы и изображения).
}
\solutionspace{100pt}

\tasknumber{7}%
\task{%
    Объект находится на расстоянии $45\,\text{см}$ от линзы, а его мнимое изображение — в $20\,\text{см}$ от неё.
    Определите увеличение предмета, фокусное расстояние линзы, оптическую силу линзы и её тип.
}

\variantsplitter

\addpersonalvariant{Дилноза Нодиршоева}

\tasknumber{1}%
\task{%
    Запишите формулу тонкой линзы и сделайте рисунок, указав на нём физические величины из этой формулы.
}
\solutionspace{60pt}

\tasknumber{2}%
\task{%
    В каких линзах можно получить действительное изображение объекта?
}
\answer{%
    $\text{ собирающие }$
}
\solutionspace{40pt}

\tasknumber{3}%
\task{%
    Какое изображение называют действительным?
}
\solutionspace{40pt}

\tasknumber{4}%
\task{%
    Есть две линзы, обозначим их 1 и 2.
    Известно что оптическая сила линзы 2 меньше, чем у линзы 1.
    Какая линза сильнее преломляет лучи?
}
\answer{%
    $1$
}
\solutionspace{40pt}

\tasknumber{5}%
\task{%
    Постройте ход луча $AK$ в тонкой линзе.
    Известно положение линзы и оба её фокуса (см.
    рис.
    на доске).
    Рассмотрите оба типа линзы, сделав 2 рисунка: собирающую и рассеивающую.
}
\solutionspace{120pt}

\tasknumber{6}%
\task{%
    Предмет находится на расстоянии $30\,\text{см}$ от рассеивающей линзы с фокусным расстоянием $15\,\text{см}$.
    Определите тип изображения, расстояние между предметом и его изображением, увеличение предмета.
    Сделайте схематичный рисунок (не обязательно в масштабе, но с сохранением свойств линзы и изображения).
}
\solutionspace{100pt}

\tasknumber{7}%
\task{%
    Объект находится на расстоянии $115\,\text{см}$ от линзы, а его действительное изображение — в $20\,\text{см}$ от неё.
    Определите увеличение предмета, фокусное расстояние линзы, оптическую силу линзы и её тип.
}

\variantsplitter

\addpersonalvariant{Жаклин Пантелеева}

\tasknumber{1}%
\task{%
    Запишите формулу тонкой линзы и сделайте рисунок, указав на нём физические величины из этой формулы.
}
\solutionspace{60pt}

\tasknumber{2}%
\task{%
    В каких линзах можно получить действительное изображение объекта?
}
\answer{%
    $\text{ собирающие }$
}
\solutionspace{40pt}

\tasknumber{3}%
\task{%
    Какое изображение называют действительным?
}
\solutionspace{40pt}

\tasknumber{4}%
\task{%
    Есть две линзы, обозначим их 1 и 2.
    Известно что фокусное расстояние линзы 1 меньше, чем у линзы 2.
    Какая линза сильнее преломляет лучи?
}
\answer{%
    $1$
}
\solutionspace{40pt}

\tasknumber{5}%
\task{%
    Постройте ход луча $AK$ в тонкой линзе.
    Известно положение линзы и оба её фокуса (см.
    рис.
    на доске).
    Рассмотрите оба типа линзы, сделав 2 рисунка: собирающую и рассеивающую.
}
\solutionspace{120pt}

\tasknumber{6}%
\task{%
    Предмет находится на расстоянии $10\,\text{см}$ от рассеивающей линзы с фокусным расстоянием $15\,\text{см}$.
    Определите тип изображения, расстояние между предметом и его изображением, увеличение предмета.
    Сделайте схематичный рисунок (не обязательно в масштабе, но с сохранением свойств линзы и изображения).
}
\solutionspace{100pt}

\tasknumber{7}%
\task{%
    Объект находится на расстоянии $45\,\text{см}$ от линзы, а его действительное изображение — в $30\,\text{см}$ от неё.
    Определите увеличение предмета, фокусное расстояние линзы, оптическую силу линзы и её тип.
}

\variantsplitter

\addpersonalvariant{Артём Переверзев}

\tasknumber{1}%
\task{%
    Запишите формулу тонкой линзы и сделайте рисунок, указав на нём физические величины из этой формулы.
}
\solutionspace{60pt}

\tasknumber{2}%
\task{%
    В каких линзах можно получить обратное изображение объекта?
}
\answer{%
    $\text{ собирающие }$
}
\solutionspace{40pt}

\tasknumber{3}%
\task{%
    Какое изображение называют действительным?
}
\solutionspace{40pt}

\tasknumber{4}%
\task{%
    Есть две линзы, обозначим их 1 и 2.
    Известно что оптическая сила линзы 1 меньше, чем у линзы 2.
    Какая линза сильнее преломляет лучи?
}
\answer{%
    $2$
}
\solutionspace{40pt}

\tasknumber{5}%
\task{%
    Постройте ход луча $CM$ в тонкой линзе.
    Известно положение линзы и оба её фокуса (см.
    рис.
    на доске).
    Рассмотрите оба типа линзы, сделав 2 рисунка: собирающую и рассеивающую.
}
\solutionspace{120pt}

\tasknumber{6}%
\task{%
    Предмет находится на расстоянии $30\,\text{см}$ от рассеивающей линзы с фокусным расстоянием $8\,\text{см}$.
    Определите тип изображения, расстояние между предметом и его изображением, увеличение предмета.
    Сделайте схематичный рисунок (не обязательно в масштабе, но с сохранением свойств линзы и изображения).
}
\solutionspace{100pt}

\tasknumber{7}%
\task{%
    Объект находится на расстоянии $115\,\text{см}$ от линзы, а его мнимое изображение — в $50\,\text{см}$ от неё.
    Определите увеличение предмета, фокусное расстояние линзы, оптическую силу линзы и её тип.
}

\variantsplitter

\addpersonalvariant{Варвара Пранова}

\tasknumber{1}%
\task{%
    Запишите формулу тонкой линзы и сделайте рисунок, указав на нём физические величины из этой формулы.
}
\solutionspace{60pt}

\tasknumber{2}%
\task{%
    В каких линзах можно получить уменьшенное изображение объекта?
}
\answer{%
    $\text{ собирающие и рассеивающие }$
}
\solutionspace{40pt}

\tasknumber{3}%
\task{%
    Какое изображение называют мнимым?
}
\solutionspace{40pt}

\tasknumber{4}%
\task{%
    Есть две линзы, обозначим их 1 и 2.
    Известно что фокусное расстояние линзы 1 меньше, чем у линзы 2.
    Какая линза сильнее преломляет лучи?
}
\answer{%
    $1$
}
\solutionspace{40pt}

\tasknumber{5}%
\task{%
    Постройте ход луча $CM$ в тонкой линзе.
    Известно положение линзы и оба её фокуса (см.
    рис.
    на доске).
    Рассмотрите оба типа линзы, сделав 2 рисунка: собирающую и рассеивающую.
}
\solutionspace{120pt}

\tasknumber{6}%
\task{%
    Предмет находится на расстоянии $30\,\text{см}$ от собирающей линзы с фокусным расстоянием $25\,\text{см}$.
    Определите тип изображения, расстояние между предметом и его изображением, увеличение предмета.
    Сделайте схематичный рисунок (не обязательно в масштабе, но с сохранением свойств линзы и изображения).
}
\solutionspace{100pt}

\tasknumber{7}%
\task{%
    Объект находится на расстоянии $25\,\text{см}$ от линзы, а его мнимое изображение — в $20\,\text{см}$ от неё.
    Определите увеличение предмета, фокусное расстояние линзы, оптическую силу линзы и её тип.
}

\variantsplitter

\addpersonalvariant{Марьям Салимова}

\tasknumber{1}%
\task{%
    Запишите формулу тонкой линзы и сделайте рисунок, указав на нём физические величины из этой формулы.
}
\solutionspace{60pt}

\tasknumber{2}%
\task{%
    В каких линзах можно получить увеличенное изображение объекта?
}
\answer{%
    $\text{ рассеивающие }$
}
\solutionspace{40pt}

\tasknumber{3}%
\task{%
    Какое изображение называют действительным?
}
\solutionspace{40pt}

\tasknumber{4}%
\task{%
    Есть две линзы, обозначим их 1 и 2.
    Известно что фокусное расстояние линзы 2 больше, чем у линзы 1.
    Какая линза сильнее преломляет лучи?
}
\answer{%
    $1$
}
\solutionspace{40pt}

\tasknumber{5}%
\task{%
    Постройте ход луча $AL$ в тонкой линзе.
    Известно положение линзы и оба её фокуса (см.
    рис.
    на доске).
    Рассмотрите оба типа линзы, сделав 2 рисунка: собирающую и рассеивающую.
}
\solutionspace{120pt}

\tasknumber{6}%
\task{%
    Предмет находится на расстоянии $10\,\text{см}$ от собирающей линзы с фокусным расстоянием $12\,\text{см}$.
    Определите тип изображения, расстояние между предметом и его изображением, увеличение предмета.
    Сделайте схематичный рисунок (не обязательно в масштабе, но с сохранением свойств линзы и изображения).
}
\solutionspace{100pt}

\tasknumber{7}%
\task{%
    Объект находится на расстоянии $45\,\text{см}$ от линзы, а его мнимое изображение — в $10\,\text{см}$ от неё.
    Определите увеличение предмета, фокусное расстояние линзы, оптическую силу линзы и её тип.
}

\variantsplitter

\addpersonalvariant{Юлия Шевченко}

\tasknumber{1}%
\task{%
    Запишите формулу тонкой линзы и сделайте рисунок, указав на нём физические величины из этой формулы.
}
\solutionspace{60pt}

\tasknumber{2}%
\task{%
    В каких линзах можно получить действительное изображение объекта?
}
\answer{%
    $\text{ собирающие }$
}
\solutionspace{40pt}

\tasknumber{3}%
\task{%
    Какое изображение называют действительным?
}
\solutionspace{40pt}

\tasknumber{4}%
\task{%
    Есть две линзы, обозначим их 1 и 2.
    Известно что оптическая сила линзы 2 меньше, чем у линзы 1.
    Какая линза сильнее преломляет лучи?
}
\answer{%
    $1$
}
\solutionspace{40pt}

\tasknumber{5}%
\task{%
    Постройте ход луча $AL$ в тонкой линзе.
    Известно положение линзы и оба её фокуса (см.
    рис.
    на доске).
    Рассмотрите оба типа линзы, сделав 2 рисунка: собирающую и рассеивающую.
}
\solutionspace{120pt}

\tasknumber{6}%
\task{%
    Предмет находится на расстоянии $10\,\text{см}$ от рассеивающей линзы с фокусным расстоянием $25\,\text{см}$.
    Определите тип изображения, расстояние между предметом и его изображением, увеличение предмета.
    Сделайте схематичный рисунок (не обязательно в масштабе, но с сохранением свойств линзы и изображения).
}
\solutionspace{100pt}

\tasknumber{7}%
\task{%
    Объект находится на расстоянии $25\,\text{см}$ от линзы, а его действительное изображение — в $40\,\text{см}$ от неё.
    Определите увеличение предмета, фокусное расстояние линзы, оптическую силу линзы и её тип.
}
% autogenerated
