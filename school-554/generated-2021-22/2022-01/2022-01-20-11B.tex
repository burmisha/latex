\setdate{20~января~2022}
\setclass{11«Б»}

\addpersonalvariant{Михаил Бурмистров}

\tasknumber{1}%
\task{%
    Запишите формулу тонкой линзы и сделайте рисунок, указав на нём физические величины из этой формулы.
}
\solutionspace{60pt}

\tasknumber{2}%
\task{%
    В каких линзах можно получить действительное изображение объекта?
}
\answer{%
    $\text{ собирающие }$
}
\solutionspace{40pt}

\tasknumber{3}%
\task{%
    Какое изображение называют действительным?
}
\solutionspace{40pt}

\tasknumber{4}%
\task{%
    Есть две линзы, обозначим их 1 и 2.
    Известно что оптическая сила линзы 1 больше, чем у линзы 2.
    Какая линза сильнее преломляет лучи?
}
\answer{%
    $1$
}
\solutionspace{40pt}

\tasknumber{5}%
\task{%
    Постройте ход луча $AM$ в тонкой линзе.
    Известно положение линзы и оба её фокуса (см.
    рис.
    на доске).
    Рассмотрите оба типа линзы, сделав 2 рисунка: собирающую и рассеивающую.
}
\solutionspace{120pt}

\tasknumber{6}%
\task{%
    Предмет находится на расстоянии $20\,\text{см}$ от рассеивающей линзы с фокусным расстоянием $12\,\text{см}$.
    Определите тип изображения, расстояние между предметом и его изображением, увеличение предмета.
    Сделайте схематичный рисунок (не обязательно в масштабе, но с сохранением свойств линзы и изображения).
}
\answer{%
    $b = \frac{aF}{a - F} \approx -7{,}5\,\text{см}, l = \abs{a + b} = 12{,}5\,\text{см}, \Gamma = 0{,}38, \text{мнимое, прямое, уменьшенное}$
}
\solutionspace{100pt}

\tasknumber{7}%
\task{%
    Объект находится на расстоянии $115\,\text{см}$ от линзы, а его действительное изображение — в $10\,\text{см}$ от неё.
    Определите увеличение предмета, фокусное расстояние линзы, оптическую силу линзы и её тип.
}
\answer{%
    $\frac 1F = D = \frac 1a + \frac 1b, D \approx10{,}9\,\text{дптр}, F\approx9{,}2\,\text{см}, \Gamma\approx0{,}09,\text{собирающая}$
}

\variantsplitter

\addpersonalvariant{Снежана Авдошина}

\tasknumber{1}%
\task{%
    Запишите формулу тонкой линзы и сделайте рисунок, указав на нём физические величины из этой формулы.
}
\solutionspace{60pt}

\tasknumber{2}%
\task{%
    В каких линзах можно получить действительное изображение объекта?
}
\answer{%
    $\text{ собирающие }$
}
\solutionspace{40pt}

\tasknumber{3}%
\task{%
    Какое изображение называют действительным?
}
\solutionspace{40pt}

\tasknumber{4}%
\task{%
    Есть две линзы, обозначим их 1 и 2.
    Известно что фокусное расстояние линзы 2 меньше, чем у линзы 1.
    Какая линза сильнее преломляет лучи?
}
\answer{%
    $2$
}
\solutionspace{40pt}

\tasknumber{5}%
\task{%
    Постройте ход луча $BK$ в тонкой линзе.
    Известно положение линзы и оба её фокуса (см.
    рис.
    на доске).
    Рассмотрите оба типа линзы, сделав 2 рисунка: собирающую и рассеивающую.
}
\solutionspace{120pt}

\tasknumber{6}%
\task{%
    Предмет находится на расстоянии $20\,\text{см}$ от рассеивающей линзы с фокусным расстоянием $15\,\text{см}$.
    Определите тип изображения, расстояние между предметом и его изображением, увеличение предмета.
    Сделайте схематичный рисунок (не обязательно в масштабе, но с сохранением свойств линзы и изображения).
}
\answer{%
    $b = \frac{aF}{a - F} \approx -8{,}6\,\text{см}, l = \abs{a + b} = 11{,}4\,\text{см}, \Gamma = 0{,}43, \text{мнимое, прямое, уменьшенное}$
}
\solutionspace{100pt}

\tasknumber{7}%
\task{%
    Объект находится на расстоянии $115\,\text{см}$ от линзы, а его действительное изображение — в $20\,\text{см}$ от неё.
    Определите увеличение предмета, фокусное расстояние линзы, оптическую силу линзы и её тип.
}
\answer{%
    $\frac 1F = D = \frac 1a + \frac 1b, D \approx5{,}9\,\text{дптр}, F\approx17\,\text{см}, \Gamma\approx0{,}17,\text{собирающая}$
}

\variantsplitter

\addpersonalvariant{Марьяна Аристова}

\tasknumber{1}%
\task{%
    Запишите известные вам виды классификации изображений.
}
\solutionspace{60pt}

\tasknumber{2}%
\task{%
    В каких линзах можно получить уменьшенное изображение объекта?
}
\answer{%
    $\text{ собирающие и рассеивающие }$
}
\solutionspace{40pt}

\tasknumber{3}%
\task{%
    Какое изображение называют мнимым?
}
\solutionspace{40pt}

\tasknumber{4}%
\task{%
    Есть две линзы, обозначим их 1 и 2.
    Известно что фокусное расстояние линзы 2 меньше, чем у линзы 1.
    Какая линза сильнее преломляет лучи?
}
\answer{%
    $2$
}
\solutionspace{40pt}

\tasknumber{5}%
\task{%
    Постройте ход луча $AK$ в тонкой линзе.
    Известно положение линзы и оба её фокуса (см.
    рис.
    на доске).
    Рассмотрите оба типа линзы, сделав 2 рисунка: собирающую и рассеивающую.
}
\solutionspace{120pt}

\tasknumber{6}%
\task{%
    Предмет находится на расстоянии $10\,\text{см}$ от рассеивающей линзы с фокусным расстоянием $12\,\text{см}$.
    Определите тип изображения, расстояние между предметом и его изображением, увеличение предмета.
    Сделайте схематичный рисунок (не обязательно в масштабе, но с сохранением свойств линзы и изображения).
}
\answer{%
    $b = \frac{aF}{a - F} \approx -5{,}5\,\text{см}, l = \abs{a + b} = 4{,}5\,\text{см}, \Gamma = 0{,}55, \text{мнимое, прямое, уменьшенное}$
}
\solutionspace{100pt}

\tasknumber{7}%
\task{%
    Объект находится на расстоянии $45\,\text{см}$ от линзы, а его действительное изображение — в $10\,\text{см}$ от неё.
    Определите увеличение предмета, фокусное расстояние линзы, оптическую силу линзы и её тип.
}
\answer{%
    $\frac 1F = D = \frac 1a + \frac 1b, D \approx12{,}2\,\text{дптр}, F\approx8{,}2\,\text{см}, \Gamma\approx0{,}22,\text{собирающая}$
}

\variantsplitter

\addpersonalvariant{Никита Иванов}

\tasknumber{1}%
\task{%
    Запишите формулу тонкой линзы и сделайте рисунок, указав на нём физические величины из этой формулы.
}
\solutionspace{60pt}

\tasknumber{2}%
\task{%
    В каких линзах можно получить увеличенное изображение объекта?
}
\answer{%
    $\text{ рассеивающие }$
}
\solutionspace{40pt}

\tasknumber{3}%
\task{%
    Какое изображение называют действительным?
}
\solutionspace{40pt}

\tasknumber{4}%
\task{%
    Есть две линзы, обозначим их 1 и 2.
    Известно что фокусное расстояние линзы 1 больше, чем у линзы 2.
    Какая линза сильнее преломляет лучи?
}
\answer{%
    $2$
}
\solutionspace{40pt}

\tasknumber{5}%
\task{%
    Постройте ход луча $AK$ в тонкой линзе.
    Известно положение линзы и оба её фокуса (см.
    рис.
    на доске).
    Рассмотрите оба типа линзы, сделав 2 рисунка: собирающую и рассеивающую.
}
\solutionspace{120pt}

\tasknumber{6}%
\task{%
    Предмет находится на расстоянии $30\,\text{см}$ от собирающей линзы с фокусным расстоянием $15\,\text{см}$.
    Определите тип изображения, расстояние между предметом и его изображением, увеличение предмета.
    Сделайте схематичный рисунок (не обязательно в масштабе, но с сохранением свойств линзы и изображения).
}
\answer{%
    $b = \frac{aF}{a - F} \approx 30\,\text{см}, l = \abs{a + b} = 60\,\text{см}, \Gamma = 1{,}00, \text{действительное, прямое, равное}$
}
\solutionspace{100pt}

\tasknumber{7}%
\task{%
    Объект находится на расстоянии $115\,\text{см}$ от линзы, а его действительное изображение — в $20\,\text{см}$ от неё.
    Определите увеличение предмета, фокусное расстояние линзы, оптическую силу линзы и её тип.
}
\answer{%
    $\frac 1F = D = \frac 1a + \frac 1b, D \approx5{,}9\,\text{дптр}, F\approx17\,\text{см}, \Gamma\approx0{,}17,\text{собирающая}$
}

\variantsplitter

\addpersonalvariant{Анастасия Князева}

\tasknumber{1}%
\task{%
    Запишите известные вам виды классификации изображений.
}
\solutionspace{60pt}

\tasknumber{2}%
\task{%
    В каких линзах можно получить обратное изображение объекта?
}
\answer{%
    $\text{ собирающие }$
}
\solutionspace{40pt}

\tasknumber{3}%
\task{%
    Какое изображение называют действительным?
}
\solutionspace{40pt}

\tasknumber{4}%
\task{%
    Есть две линзы, обозначим их 1 и 2.
    Известно что оптическая сила линзы 1 больше, чем у линзы 2.
    Какая линза сильнее преломляет лучи?
}
\answer{%
    $1$
}
\solutionspace{40pt}

\tasknumber{5}%
\task{%
    Постройте ход луча $AK$ в тонкой линзе.
    Известно положение линзы и оба её фокуса (см.
    рис.
    на доске).
    Рассмотрите оба типа линзы, сделав 2 рисунка: собирающую и рассеивающую.
}
\solutionspace{120pt}

\tasknumber{6}%
\task{%
    Предмет находится на расстоянии $10\,\text{см}$ от собирающей линзы с фокусным расстоянием $50\,\text{см}$.
    Определите тип изображения, расстояние между предметом и его изображением, увеличение предмета.
    Сделайте схематичный рисунок (не обязательно в масштабе, но с сохранением свойств линзы и изображения).
}
\answer{%
    $b = \frac{aF}{a - F} \approx -12{,}5\,\text{см}, l = \abs{a + b} = 2{,}5\,\text{см}, \Gamma = 1{,}25, \text{мнимое, прямое, увеличенное}$
}
\solutionspace{100pt}

\tasknumber{7}%
\task{%
    Объект находится на расстоянии $45\,\text{см}$ от линзы, а его мнимое изображение — в $10\,\text{см}$ от неё.
    Определите увеличение предмета, фокусное расстояние линзы, оптическую силу линзы и её тип.
}
\answer{%
    $\frac 1F = D = \frac 1a + \frac 1b, D \approx-7{,}8\,\text{дптр}, F\approx-12{,}9\,\text{см}, \Gamma\approx0{,}22,\text{рассеивающая}$
}

\variantsplitter

\addpersonalvariant{Елизавета Кутумова}

\tasknumber{1}%
\task{%
    Запишите формулу тонкой линзы и сделайте рисунок, указав на нём физические величины из этой формулы.
}
\solutionspace{60pt}

\tasknumber{2}%
\task{%
    В каких линзах можно получить прямое изображение объекта?
}
\answer{%
    $\text{ собирающие и рассеивающие }$
}
\solutionspace{40pt}

\tasknumber{3}%
\task{%
    Какое изображение называют мнимым?
}
\solutionspace{40pt}

\tasknumber{4}%
\task{%
    Есть две линзы, обозначим их 1 и 2.
    Известно что фокусное расстояние линзы 2 больше, чем у линзы 1.
    Какая линза сильнее преломляет лучи?
}
\answer{%
    $1$
}
\solutionspace{40pt}

\tasknumber{5}%
\task{%
    Постройте ход луча $AL$ в тонкой линзе.
    Известно положение линзы и оба её фокуса (см.
    рис.
    на доске).
    Рассмотрите оба типа линзы, сделав 2 рисунка: собирающую и рассеивающую.
}
\solutionspace{120pt}

\tasknumber{6}%
\task{%
    Предмет находится на расстоянии $30\,\text{см}$ от собирающей линзы с фокусным расстоянием $15\,\text{см}$.
    Определите тип изображения, расстояние между предметом и его изображением, увеличение предмета.
    Сделайте схематичный рисунок (не обязательно в масштабе, но с сохранением свойств линзы и изображения).
}
\answer{%
    $b = \frac{aF}{a - F} \approx 30\,\text{см}, l = \abs{a + b} = 60\,\text{см}, \Gamma = 1{,}00, \text{действительное, прямое, равное}$
}
\solutionspace{100pt}

\tasknumber{7}%
\task{%
    Объект находится на расстоянии $45\,\text{см}$ от линзы, а его мнимое изображение — в $40\,\text{см}$ от неё.
    Определите увеличение предмета, фокусное расстояние линзы, оптическую силу линзы и её тип.
}
\answer{%
    $\frac 1F = D = \frac 1a + \frac 1b, D \approx-0{,}3\,\text{дптр}, F\approx-360\,\text{см}, \Gamma\approx0{,}89,\text{рассеивающая}$
}

\variantsplitter

\addpersonalvariant{Роксана Мехтиева}

\tasknumber{1}%
\task{%
    Запишите формулу тонкой линзы и сделайте рисунок, указав на нём физические величины из этой формулы.
}
\solutionspace{60pt}

\tasknumber{2}%
\task{%
    В каких линзах можно получить прямое изображение объекта?
}
\answer{%
    $\text{ собирающие и рассеивающие }$
}
\solutionspace{40pt}

\tasknumber{3}%
\task{%
    Какое изображение называют мнимым?
}
\solutionspace{40pt}

\tasknumber{4}%
\task{%
    Есть две линзы, обозначим их 1 и 2.
    Известно что оптическая сила линзы 1 больше, чем у линзы 2.
    Какая линза сильнее преломляет лучи?
}
\answer{%
    $1$
}
\solutionspace{40pt}

\tasknumber{5}%
\task{%
    Постройте ход луча $BK$ в тонкой линзе.
    Известно положение линзы и оба её фокуса (см.
    рис.
    на доске).
    Рассмотрите оба типа линзы, сделав 2 рисунка: собирающую и рассеивающую.
}
\solutionspace{120pt}

\tasknumber{6}%
\task{%
    Предмет находится на расстоянии $30\,\text{см}$ от рассеивающей линзы с фокусным расстоянием $12\,\text{см}$.
    Определите тип изображения, расстояние между предметом и его изображением, увеличение предмета.
    Сделайте схематичный рисунок (не обязательно в масштабе, но с сохранением свойств линзы и изображения).
}
\answer{%
    $b = \frac{aF}{a - F} \approx -8{,}6\,\text{см}, l = \abs{a + b} = 21{,}4\,\text{см}, \Gamma = 0{,}29, \text{мнимое, прямое, уменьшенное}$
}
\solutionspace{100pt}

\tasknumber{7}%
\task{%
    Объект находится на расстоянии $45\,\text{см}$ от линзы, а его мнимое изображение — в $20\,\text{см}$ от неё.
    Определите увеличение предмета, фокусное расстояние линзы, оптическую силу линзы и её тип.
}
\answer{%
    $\frac 1F = D = \frac 1a + \frac 1b, D \approx-2{,}8\,\text{дптр}, F\approx-36\,\text{см}, \Gamma\approx0{,}44,\text{рассеивающая}$
}

\variantsplitter

\addpersonalvariant{Дилноза Нодиршоева}

\tasknumber{1}%
\task{%
    Запишите формулу тонкой линзы и сделайте рисунок, указав на нём физические величины из этой формулы.
}
\solutionspace{60pt}

\tasknumber{2}%
\task{%
    В каких линзах можно получить действительное изображение объекта?
}
\answer{%
    $\text{ собирающие }$
}
\solutionspace{40pt}

\tasknumber{3}%
\task{%
    Какое изображение называют действительным?
}
\solutionspace{40pt}

\tasknumber{4}%
\task{%
    Есть две линзы, обозначим их 1 и 2.
    Известно что оптическая сила линзы 2 меньше, чем у линзы 1.
    Какая линза сильнее преломляет лучи?
}
\answer{%
    $1$
}
\solutionspace{40pt}

\tasknumber{5}%
\task{%
    Постройте ход луча $AK$ в тонкой линзе.
    Известно положение линзы и оба её фокуса (см.
    рис.
    на доске).
    Рассмотрите оба типа линзы, сделав 2 рисунка: собирающую и рассеивающую.
}
\solutionspace{120pt}

\tasknumber{6}%
\task{%
    Предмет находится на расстоянии $30\,\text{см}$ от рассеивающей линзы с фокусным расстоянием $15\,\text{см}$.
    Определите тип изображения, расстояние между предметом и его изображением, увеличение предмета.
    Сделайте схематичный рисунок (не обязательно в масштабе, но с сохранением свойств линзы и изображения).
}
\answer{%
    $b = \frac{aF}{a - F} \approx -10\,\text{см}, l = \abs{a + b} = 20\,\text{см}, \Gamma = 0{,}33, \text{мнимое, прямое, уменьшенное}$
}
\solutionspace{100pt}

\tasknumber{7}%
\task{%
    Объект находится на расстоянии $115\,\text{см}$ от линзы, а его действительное изображение — в $20\,\text{см}$ от неё.
    Определите увеличение предмета, фокусное расстояние линзы, оптическую силу линзы и её тип.
}
\answer{%
    $\frac 1F = D = \frac 1a + \frac 1b, D \approx5{,}9\,\text{дптр}, F\approx17\,\text{см}, \Gamma\approx0{,}17,\text{собирающая}$
}

\variantsplitter

\addpersonalvariant{Жаклин Пантелеева}

\tasknumber{1}%
\task{%
    Запишите формулу тонкой линзы и сделайте рисунок, указав на нём физические величины из этой формулы.
}
\solutionspace{60pt}

\tasknumber{2}%
\task{%
    В каких линзах можно получить действительное изображение объекта?
}
\answer{%
    $\text{ собирающие }$
}
\solutionspace{40pt}

\tasknumber{3}%
\task{%
    Какое изображение называют действительным?
}
\solutionspace{40pt}

\tasknumber{4}%
\task{%
    Есть две линзы, обозначим их 1 и 2.
    Известно что фокусное расстояние линзы 1 меньше, чем у линзы 2.
    Какая линза сильнее преломляет лучи?
}
\answer{%
    $1$
}
\solutionspace{40pt}

\tasknumber{5}%
\task{%
    Постройте ход луча $AK$ в тонкой линзе.
    Известно положение линзы и оба её фокуса (см.
    рис.
    на доске).
    Рассмотрите оба типа линзы, сделав 2 рисунка: собирающую и рассеивающую.
}
\solutionspace{120pt}

\tasknumber{6}%
\task{%
    Предмет находится на расстоянии $10\,\text{см}$ от рассеивающей линзы с фокусным расстоянием $15\,\text{см}$.
    Определите тип изображения, расстояние между предметом и его изображением, увеличение предмета.
    Сделайте схематичный рисунок (не обязательно в масштабе, но с сохранением свойств линзы и изображения).
}
\answer{%
    $b = \frac{aF}{a - F} \approx -6\,\text{см}, l = \abs{a + b} = 4\,\text{см}, \Gamma = 0{,}60, \text{мнимое, прямое, уменьшенное}$
}
\solutionspace{100pt}

\tasknumber{7}%
\task{%
    Объект находится на расстоянии $45\,\text{см}$ от линзы, а его действительное изображение — в $30\,\text{см}$ от неё.
    Определите увеличение предмета, фокусное расстояние линзы, оптическую силу линзы и её тип.
}
\answer{%
    $\frac 1F = D = \frac 1a + \frac 1b, D \approx5{,}6\,\text{дптр}, F\approx18\,\text{см}, \Gamma\approx0{,}67,\text{собирающая}$
}

\variantsplitter

\addpersonalvariant{Артём Переверзев}

\tasknumber{1}%
\task{%
    Запишите формулу тонкой линзы и сделайте рисунок, указав на нём физические величины из этой формулы.
}
\solutionspace{60pt}

\tasknumber{2}%
\task{%
    В каких линзах можно получить обратное изображение объекта?
}
\answer{%
    $\text{ собирающие }$
}
\solutionspace{40pt}

\tasknumber{3}%
\task{%
    Какое изображение называют действительным?
}
\solutionspace{40pt}

\tasknumber{4}%
\task{%
    Есть две линзы, обозначим их 1 и 2.
    Известно что оптическая сила линзы 1 меньше, чем у линзы 2.
    Какая линза сильнее преломляет лучи?
}
\answer{%
    $2$
}
\solutionspace{40pt}

\tasknumber{5}%
\task{%
    Постройте ход луча $CM$ в тонкой линзе.
    Известно положение линзы и оба её фокуса (см.
    рис.
    на доске).
    Рассмотрите оба типа линзы, сделав 2 рисунка: собирающую и рассеивающую.
}
\solutionspace{120pt}

\tasknumber{6}%
\task{%
    Предмет находится на расстоянии $30\,\text{см}$ от рассеивающей линзы с фокусным расстоянием $8\,\text{см}$.
    Определите тип изображения, расстояние между предметом и его изображением, увеличение предмета.
    Сделайте схематичный рисунок (не обязательно в масштабе, но с сохранением свойств линзы и изображения).
}
\answer{%
    $b = \frac{aF}{a - F} \approx -6{,}3\,\text{см}, l = \abs{a + b} = 23{,}7\,\text{см}, \Gamma = 0{,}21, \text{мнимое, прямое, уменьшенное}$
}
\solutionspace{100pt}

\tasknumber{7}%
\task{%
    Объект находится на расстоянии $115\,\text{см}$ от линзы, а его мнимое изображение — в $50\,\text{см}$ от неё.
    Определите увеличение предмета, фокусное расстояние линзы, оптическую силу линзы и её тип.
}
\answer{%
    $\frac 1F = D = \frac 1a + \frac 1b, D \approx-1{,}1\,\text{дптр}, F\approx-88{,}5\,\text{см}, \Gamma\approx0{,}43,\text{рассеивающая}$
}

\variantsplitter

\addpersonalvariant{Варвара Пранова}

\tasknumber{1}%
\task{%
    Запишите формулу тонкой линзы и сделайте рисунок, указав на нём физические величины из этой формулы.
}
\solutionspace{60pt}

\tasknumber{2}%
\task{%
    В каких линзах можно получить уменьшенное изображение объекта?
}
\answer{%
    $\text{ собирающие и рассеивающие }$
}
\solutionspace{40pt}

\tasknumber{3}%
\task{%
    Какое изображение называют мнимым?
}
\solutionspace{40pt}

\tasknumber{4}%
\task{%
    Есть две линзы, обозначим их 1 и 2.
    Известно что фокусное расстояние линзы 1 меньше, чем у линзы 2.
    Какая линза сильнее преломляет лучи?
}
\answer{%
    $1$
}
\solutionspace{40pt}

\tasknumber{5}%
\task{%
    Постройте ход луча $CM$ в тонкой линзе.
    Известно положение линзы и оба её фокуса (см.
    рис.
    на доске).
    Рассмотрите оба типа линзы, сделав 2 рисунка: собирающую и рассеивающую.
}
\solutionspace{120pt}

\tasknumber{6}%
\task{%
    Предмет находится на расстоянии $30\,\text{см}$ от собирающей линзы с фокусным расстоянием $25\,\text{см}$.
    Определите тип изображения, расстояние между предметом и его изображением, увеличение предмета.
    Сделайте схематичный рисунок (не обязательно в масштабе, но с сохранением свойств линзы и изображения).
}
\answer{%
    $b = \frac{aF}{a - F} \approx 150\,\text{см}, l = \abs{a + b} = 180\,\text{см}, \Gamma = 5{,}00, \text{действительное, прямое, увеличенное}$
}
\solutionspace{100pt}

\tasknumber{7}%
\task{%
    Объект находится на расстоянии $25\,\text{см}$ от линзы, а его мнимое изображение — в $20\,\text{см}$ от неё.
    Определите увеличение предмета, фокусное расстояние линзы, оптическую силу линзы и её тип.
}
\answer{%
    $\frac 1F = D = \frac 1a + \frac 1b, D \approx-1\,\text{дптр}, F\approx-100\,\text{см}, \Gamma\approx0{,}80,\text{рассеивающая}$
}

\variantsplitter

\addpersonalvariant{Марьям Салимова}

\tasknumber{1}%
\task{%
    Запишите формулу тонкой линзы и сделайте рисунок, указав на нём физические величины из этой формулы.
}
\solutionspace{60pt}

\tasknumber{2}%
\task{%
    В каких линзах можно получить увеличенное изображение объекта?
}
\answer{%
    $\text{ рассеивающие }$
}
\solutionspace{40pt}

\tasknumber{3}%
\task{%
    Какое изображение называют действительным?
}
\solutionspace{40pt}

\tasknumber{4}%
\task{%
    Есть две линзы, обозначим их 1 и 2.
    Известно что фокусное расстояние линзы 2 больше, чем у линзы 1.
    Какая линза сильнее преломляет лучи?
}
\answer{%
    $1$
}
\solutionspace{40pt}

\tasknumber{5}%
\task{%
    Постройте ход луча $AL$ в тонкой линзе.
    Известно положение линзы и оба её фокуса (см.
    рис.
    на доске).
    Рассмотрите оба типа линзы, сделав 2 рисунка: собирающую и рассеивающую.
}
\solutionspace{120pt}

\tasknumber{6}%
\task{%
    Предмет находится на расстоянии $10\,\text{см}$ от собирающей линзы с фокусным расстоянием $12\,\text{см}$.
    Определите тип изображения, расстояние между предметом и его изображением, увеличение предмета.
    Сделайте схематичный рисунок (не обязательно в масштабе, но с сохранением свойств линзы и изображения).
}
\answer{%
    $b = \frac{aF}{a - F} \approx -60\,\text{см}, l = \abs{a + b} = 50\,\text{см}, \Gamma = 6{,}00, \text{мнимое, прямое, увеличенное}$
}
\solutionspace{100pt}

\tasknumber{7}%
\task{%
    Объект находится на расстоянии $45\,\text{см}$ от линзы, а его мнимое изображение — в $10\,\text{см}$ от неё.
    Определите увеличение предмета, фокусное расстояние линзы, оптическую силу линзы и её тип.
}
\answer{%
    $\frac 1F = D = \frac 1a + \frac 1b, D \approx-7{,}8\,\text{дптр}, F\approx-12{,}9\,\text{см}, \Gamma\approx0{,}22,\text{рассеивающая}$
}

\variantsplitter

\addpersonalvariant{Юлия Шевченко}

\tasknumber{1}%
\task{%
    Запишите формулу тонкой линзы и сделайте рисунок, указав на нём физические величины из этой формулы.
}
\solutionspace{60pt}

\tasknumber{2}%
\task{%
    В каких линзах можно получить действительное изображение объекта?
}
\answer{%
    $\text{ собирающие }$
}
\solutionspace{40pt}

\tasknumber{3}%
\task{%
    Какое изображение называют действительным?
}
\solutionspace{40pt}

\tasknumber{4}%
\task{%
    Есть две линзы, обозначим их 1 и 2.
    Известно что оптическая сила линзы 2 меньше, чем у линзы 1.
    Какая линза сильнее преломляет лучи?
}
\answer{%
    $1$
}
\solutionspace{40pt}

\tasknumber{5}%
\task{%
    Постройте ход луча $AL$ в тонкой линзе.
    Известно положение линзы и оба её фокуса (см.
    рис.
    на доске).
    Рассмотрите оба типа линзы, сделав 2 рисунка: собирающую и рассеивающую.
}
\solutionspace{120pt}

\tasknumber{6}%
\task{%
    Предмет находится на расстоянии $10\,\text{см}$ от рассеивающей линзы с фокусным расстоянием $25\,\text{см}$.
    Определите тип изображения, расстояние между предметом и его изображением, увеличение предмета.
    Сделайте схематичный рисунок (не обязательно в масштабе, но с сохранением свойств линзы и изображения).
}
\answer{%
    $b = \frac{aF}{a - F} \approx -7{,}1\,\text{см}, l = \abs{a + b} = 2{,}9\,\text{см}, \Gamma = 0{,}71, \text{мнимое, прямое, уменьшенное}$
}
\solutionspace{100pt}

\tasknumber{7}%
\task{%
    Объект находится на расстоянии $25\,\text{см}$ от линзы, а его действительное изображение — в $40\,\text{см}$ от неё.
    Определите увеличение предмета, фокусное расстояние линзы, оптическую силу линзы и её тип.
}
\answer{%
    $\frac 1F = D = \frac 1a + \frac 1b, D \approx6{,}5\,\text{дптр}, F\approx15{,}4\,\text{см}, \Gamma\approx1{,}60,\text{собирающая}$
}
% autogenerated
