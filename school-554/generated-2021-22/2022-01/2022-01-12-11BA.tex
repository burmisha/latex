\setdate{12~января~2022}
\setclass{11«БА»}

\addpersonalvariant{Михаил Бурмистров}

\tasknumber{1}%
\task{%
    Для закона отражения:
    \begin{itemize}
        \item сделайте рисунок,
        \item отметьте все необходимые углы и подпишите их названия,
        \item запишите этот закон формулой.
    \end{itemize}
}
\solutionspace{80pt}

\tasknumber{2}%
\task{%
    Под каким углом (в градусах) к горизонту следует расположить плоское зеркало,
    чтобы осветить дно вертикального колодца отраженными от зеркала солнечными лучами,
    падающими под углом $36\degrees$ к горизонту?
}
\answer{%
    $\alpha = 36\degrees, (90\degrees - \beta + \alpha) + (90\degrees - \beta) = 90\degrees \implies \beta = 45\degrees + \frac \alpha 2 = 63\degrees$
}
\solutionspace{80pt}

\tasknumber{3}%
\task{%
    Два плоских зеркала располагаются под углом друг к другу
    и между ними помещается точечный источник света.
    Расстояние от этого источника до одного зеркала $3\,\text{см}$, до другого $6\,\text{см}$.
    Расстояние между первыми изображениями в зеркалах $17{,}46\,\text{см}$.
    Найдите угол (в градусах) между зеркалами.
}
\answer{%
    $\cos \alpha = \frac{c^2 - \sqr{2a} - \sqr{2b}}{2 \cdot 2a \cdot 2b} \approx 0{,}867 \implies \alpha = 29{,}9\degrees$
}
\solutionspace{80pt}

\tasknumber{4}%
\task{%
    Докажите, что тонкий клин с углом $\varphi$ при вершине из стекла с показателем преломления $n$
        отклонит луч на угол $(n-1)\varphi$ (в приближении малых углов).
}
\solutionspace{150pt}

\tasknumber{5}%
\task{%
    Солнце составляет с горизонтом угол, синус которого  0{,}5 .
    Шест высотой $160\,\text{см}$ вбит в дно водоёма глубиной $90\,\text{см}$.
    Найдите длину тени от этого шеста на дне водоёма, если показатель преломления воды 1{,}33.
}
\answer{%
    $
        n \sin \beta = 1 \cdot \cos \alpha \implies \beta \approx 40{,}6\degrees,
        L = (H - h)\ctg \alpha + h \tg \beta \approx 121{,}2\,\text{см} + 77{,}2\,\text{см} \approx 198{,}5\,\text{см}.
    $
}

\variantsplitter

\addpersonalvariant{Ирина Ан}

\tasknumber{1}%
\task{%
    Для закона преломления:
    \begin{itemize}
        \item сделайте рисунок,
        \item отметьте все необходимые углы и подпишите их названия,
        \item запишите этот закон формулой.
    \end{itemize}
}
\solutionspace{80pt}

\tasknumber{2}%
\task{%
    Под каким углом (в градусах) к горизонту следует расположить плоское зеркало,
    чтобы осветить дно вертикального колодца отраженными от зеркала солнечными лучами,
    падающими под углом $40\degrees$ к горизонту?
}
\answer{%
    $\alpha = 40\degrees, (90\degrees - \beta + \alpha) + (90\degrees - \beta) = 90\degrees \implies \beta = 45\degrees + \frac \alpha 2 = 65\degrees$
}
\solutionspace{80pt}

\tasknumber{3}%
\task{%
    Два плоских зеркала располагаются под углом друг к другу
    и между ними помещается точечный источник света.
    Расстояние от этого источника до одного зеркала $4\,\text{см}$, до другого $8\,\text{см}$.
    Расстояние между первыми изображениями в зеркалах $23{,}27\,\text{см}$.
    Найдите угол (в градусах) между зеркалами.
}
\answer{%
    $\cos \alpha = \frac{c^2 - \sqr{2a} - \sqr{2b}}{2 \cdot 2a \cdot 2b} \approx 0{,}865 \implies \alpha = 30{,}1\degrees$
}
\solutionspace{80pt}

\tasknumber{4}%
\task{%
    Докажите, что тонкий клин с углом $\varphi$ при вершине из стекла с показателем преломления $n$
        отклонит луч на угол $(n-1)\varphi$ (в приближении малых углов).
}
\solutionspace{150pt}

\tasknumber{5}%
\task{%
    Солнце составляет с горизонтом угол, синус которого  0{,}8 .
    Шест высотой $160\,\text{см}$ вбит в дно водоёма глубиной $80\,\text{см}$.
    Найдите длину тени от этого шеста на дне водоёма, если показатель преломления воды 1{,}33.
}
\answer{%
    $
        n \sin \beta = 1 \cdot \cos \alpha \implies \beta \approx 26{,}8\degrees,
        L = (H - h)\ctg \alpha + h \tg \beta \approx 60\,\text{см} + 40{,}4\,\text{см} \approx 100{,}4\,\text{см}.
    $
}

\variantsplitter

\addpersonalvariant{Софья Андрианова}

\tasknumber{1}%
\task{%
    Для закона отражения:
    \begin{itemize}
        \item сделайте рисунок,
        \item отметьте все необходимые углы и подпишите их названия,
        \item запишите этот закон формулой.
    \end{itemize}
}
\solutionspace{80pt}

\tasknumber{2}%
\task{%
    Под каким углом (в градусах) к горизонту следует расположить плоское зеркало,
    чтобы осветить дно вертикального колодца отраженными от зеркала солнечными лучами,
    падающими под углом $32\degrees$ к вертикали?
}
\answer{%
    $\alpha = 58\degrees, (90\degrees - \beta + \alpha) + (90\degrees - \beta) = 90\degrees \implies \beta = 45\degrees + \frac \alpha 2 = 74\degrees$
}
\solutionspace{80pt}

\tasknumber{3}%
\task{%
    Два плоских зеркала располагаются под углом друг к другу
    и между ними помещается точечный источник света.
    Расстояние от этого источника до одного зеркала $5\,\text{см}$, до другого $8\,\text{см}$.
    Расстояние между первыми изображениями в зеркалах $25{,}16\,\text{см}$.
    Найдите угол (в градусах) между зеркалами.
}
\answer{%
    $\cos \alpha = \frac{c^2 - \sqr{2a} - \sqr{2b}}{2 \cdot 2a \cdot 2b} \approx 0{,}866 \implies \alpha = 30{,}0\degrees$
}
\solutionspace{80pt}

\tasknumber{4}%
\task{%
    Докажите, что тонкий клин с углом $\varphi$ при вершине из стекла с показателем преломления $n$
        отклонит луч на угол $(n-1)\varphi$ (в приближении малых углов).
}
\solutionspace{150pt}

\tasknumber{5}%
\task{%
    Солнце составляет с горизонтом угол, синус которого  0{,}8 .
    Шест высотой $180\,\text{см}$ вбит в дно водоёма глубиной $90\,\text{см}$.
    Найдите длину тени от этого шеста на дне водоёма, если показатель преломления воды 1{,}33.
}
\answer{%
    $
        n \sin \beta = 1 \cdot \cos \alpha \implies \beta \approx 26{,}8\degrees,
        L = (H - h)\ctg \alpha + h \tg \beta \approx 67{,}5\,\text{см} + 45{,}5\,\text{см} \approx 113\,\text{см}.
    $
}

\variantsplitter

\addpersonalvariant{Владимир Артемчук}

\tasknumber{1}%
\task{%
    Для закона отражения:
    \begin{itemize}
        \item сделайте рисунок,
        \item отметьте все необходимые углы и подпишите их названия,
        \item запишите этот закон формулой.
    \end{itemize}
}
\solutionspace{80pt}

\tasknumber{2}%
\task{%
    Под каким углом (в градусах) к горизонту следует расположить плоское зеркало,
    чтобы осветить дно вертикального колодца отраженными от зеркала солнечными лучами,
    падающими под углом $50\degrees$ к горизонту?
}
\answer{%
    $\alpha = 50\degrees, (90\degrees - \beta + \alpha) + (90\degrees - \beta) = 90\degrees \implies \beta = 45\degrees + \frac \alpha 2 = 70\degrees$
}
\solutionspace{80pt}

\tasknumber{3}%
\task{%
    Два плоских зеркала располагаются под углом друг к другу
    и между ними помещается точечный источник света.
    Расстояние от этого источника до одного зеркала $3\,\text{см}$, до другого $9\,\text{см}$.
    Расстояние между первыми изображениями в зеркалах $23{,}39\,\text{см}$.
    Найдите угол (в градусах) между зеркалами.
}
\answer{%
    $\cos \alpha = \frac{c^2 - \sqr{2a} - \sqr{2b}}{2 \cdot 2a \cdot 2b} \approx 0{,}866 \implies \alpha = 30{,}0\degrees$
}
\solutionspace{80pt}

\tasknumber{4}%
\task{%
    Докажите, что мнимое изображение точечного источника света под поверхностью воды из воздуха
        видно на глубине в $n$ раз меньше его реальной глубины (в приближении малых углов).
}
\solutionspace{150pt}

\tasknumber{5}%
\task{%
    Солнце составляет с горизонтом угол, синус которого  0{,}7 .
    Шест высотой $180\,\text{см}$ вбит в дно водоёма глубиной $80\,\text{см}$.
    Найдите длину тени от этого шеста на дне водоёма, если показатель преломления воды 1{,}33.
}
\answer{%
    $
        n \sin \beta = 1 \cdot \cos \alpha \implies \beta \approx 32{,}5\degrees,
        L = (H - h)\ctg \alpha + h \tg \beta \approx 102\,\text{см} + 50{,}9\,\text{см} \approx 152{,}9\,\text{см}.
    $
}

\variantsplitter

\addpersonalvariant{Софья Белянкина}

\tasknumber{1}%
\task{%
    Для закона преломления:
    \begin{itemize}
        \item сделайте рисунок,
        \item отметьте все необходимые углы и подпишите их названия,
        \item запишите этот закон формулой.
    \end{itemize}
}
\solutionspace{80pt}

\tasknumber{2}%
\task{%
    Под каким углом (в градусах) к горизонту следует расположить плоское зеркало,
    чтобы осветить дно вертикального колодца отраженными от зеркала солнечными лучами,
    падающими под углом $28\degrees$ к вертикали?
}
\answer{%
    $\alpha = 62\degrees, (90\degrees - \beta + \alpha) + (90\degrees - \beta) = 90\degrees \implies \beta = 45\degrees + \frac \alpha 2 = 76\degrees$
}
\solutionspace{80pt}

\tasknumber{3}%
\task{%
    Два плоских зеркала располагаются под углом друг к другу
    и между ними помещается точечный источник света.
    Расстояние от этого источника до одного зеркала $5\,\text{см}$, до другого $7\,\text{см}$.
    Расстояние между первыми изображениями в зеркалах $23{,}21\,\text{см}$.
    Найдите угол (в градусах) между зеркалами.
}
\answer{%
    $\cos \alpha = \frac{c^2 - \sqr{2a} - \sqr{2b}}{2 \cdot 2a \cdot 2b} \approx 0{,}867 \implies \alpha = 29{,}9\degrees$
}
\solutionspace{80pt}

\tasknumber{4}%
\task{%
    Докажите, что тонкий клин с углом $\varphi$ при вершине из стекла с показателем преломления $n$
        отклонит луч на угол $(n-1)\varphi$ (в приближении малых углов).
}
\solutionspace{150pt}

\tasknumber{5}%
\task{%
    Солнце составляет с горизонтом угол, синус которого  0{,}8 .
    Шест высотой $170\,\text{см}$ вбит в дно водоёма глубиной $90\,\text{см}$.
    Найдите длину тени от этого шеста на дне водоёма, если показатель преломления воды 1{,}33.
}
\answer{%
    $
        n \sin \beta = 1 \cdot \cos \alpha \implies \beta \approx 26{,}8\degrees,
        L = (H - h)\ctg \alpha + h \tg \beta \approx 60\,\text{см} + 45{,}5\,\text{см} \approx 105{,}5\,\text{см}.
    $
}

\variantsplitter

\addpersonalvariant{Варвара Егиазарян}

\tasknumber{1}%
\task{%
    Для закона отражения:
    \begin{itemize}
        \item сделайте рисунок,
        \item отметьте все необходимые углы и подпишите их названия,
        \item запишите этот закон формулой.
    \end{itemize}
}
\solutionspace{80pt}

\tasknumber{2}%
\task{%
    Под каким углом (в градусах) к горизонту следует расположить плоское зеркало,
    чтобы осветить дно вертикального колодца отраженными от зеркала солнечными лучами,
    падающими под углом $64\degrees$ к вертикали?
}
\answer{%
    $\alpha = 26\degrees, (90\degrees - \beta + \alpha) + (90\degrees - \beta) = 90\degrees \implies \beta = 45\degrees + \frac \alpha 2 = 58\degrees$
}
\solutionspace{80pt}

\tasknumber{3}%
\task{%
    Два плоских зеркала располагаются под углом друг к другу
    и между ними помещается точечный источник света.
    Расстояние от этого источника до одного зеркала $4\,\text{см}$, до другого $7\,\text{см}$.
    Расстояние между первыми изображениями в зеркалах $21{,}31\,\text{см}$.
    Найдите угол (в градусах) между зеркалами.
}
\answer{%
    $\cos \alpha = \frac{c^2 - \sqr{2a} - \sqr{2b}}{2 \cdot 2a \cdot 2b} \approx 0{,}867 \implies \alpha = 29{,}9\degrees$
}
\solutionspace{80pt}

\tasknumber{4}%
\task{%
    Докажите, что мнимое изображение точечного источника света под поверхностью воды из воздуха
        видно на глубине в $n$ раз меньше его реальной глубины (в приближении малых углов).
}
\solutionspace{150pt}

\tasknumber{5}%
\task{%
    Солнце составляет с горизонтом угол, синус которого  0{,}8 .
    Шест высотой $130\,\text{см}$ вбит в дно водоёма глубиной $90\,\text{см}$.
    Найдите длину тени от этого шеста на дне водоёма, если показатель преломления воды 1{,}33.
}
\answer{%
    $
        n \sin \beta = 1 \cdot \cos \alpha \implies \beta \approx 26{,}8\degrees,
        L = (H - h)\ctg \alpha + h \tg \beta \approx 30\,\text{см} + 45{,}5\,\text{см} \approx 75{,}5\,\text{см}.
    $
}

\variantsplitter

\addpersonalvariant{Владислав Емелин}

\tasknumber{1}%
\task{%
    Для закона преломления:
    \begin{itemize}
        \item сделайте рисунок,
        \item отметьте все необходимые углы и подпишите их названия,
        \item запишите этот закон формулой.
    \end{itemize}
}
\solutionspace{80pt}

\tasknumber{2}%
\task{%
    Под каким углом (в градусах) к горизонту следует расположить плоское зеркало,
    чтобы осветить дно вертикального колодца отраженными от зеркала солнечными лучами,
    падающими под углом $62\degrees$ к горизонту?
}
\answer{%
    $\alpha = 62\degrees, (90\degrees - \beta + \alpha) + (90\degrees - \beta) = 90\degrees \implies \beta = 45\degrees + \frac \alpha 2 = 76\degrees$
}
\solutionspace{80pt}

\tasknumber{3}%
\task{%
    Два плоских зеркала располагаются под углом друг к другу
    и между ними помещается точечный источник света.
    Расстояние от этого источника до одного зеркала $4\,\text{см}$, до другого $8\,\text{см}$.
    Расстояние между первыми изображениями в зеркалах $23{,}27\,\text{см}$.
    Найдите угол (в градусах) между зеркалами.
}
\answer{%
    $\cos \alpha = \frac{c^2 - \sqr{2a} - \sqr{2b}}{2 \cdot 2a \cdot 2b} \approx 0{,}865 \implies \alpha = 30{,}1\degrees$
}
\solutionspace{80pt}

\tasknumber{4}%
\task{%
    Докажите, что тонкий клин с углом $\varphi$ при вершине из стекла с показателем преломления $n$
        отклонит луч на угол $(n-1)\varphi$ (в приближении малых углов).
}
\solutionspace{150pt}

\tasknumber{5}%
\task{%
    Солнце составляет с горизонтом угол, синус которого  0{,}5 .
    Шест высотой $130\,\text{см}$ вбит в дно водоёма глубиной $90\,\text{см}$.
    Найдите длину тени от этого шеста на дне водоёма, если показатель преломления воды 1{,}33.
}
\answer{%
    $
        n \sin \beta = 1 \cdot \cos \alpha \implies \beta \approx 40{,}6\degrees,
        L = (H - h)\ctg \alpha + h \tg \beta \approx 69{,}3\,\text{см} + 77{,}2\,\text{см} \approx 146{,}5\,\text{см}.
    $
}

\variantsplitter

\addpersonalvariant{Артём Жичин}

\tasknumber{1}%
\task{%
    Для закона отражения:
    \begin{itemize}
        \item сделайте рисунок,
        \item отметьте все необходимые углы и подпишите их названия,
        \item запишите этот закон формулой.
    \end{itemize}
}
\solutionspace{80pt}

\tasknumber{2}%
\task{%
    Под каким углом (в градусах) к горизонту следует расположить плоское зеркало,
    чтобы осветить дно вертикального колодца отраженными от зеркала солнечными лучами,
    падающими под углом $28\degrees$ к вертикали?
}
\answer{%
    $\alpha = 62\degrees, (90\degrees - \beta + \alpha) + (90\degrees - \beta) = 90\degrees \implies \beta = 45\degrees + \frac \alpha 2 = 76\degrees$
}
\solutionspace{80pt}

\tasknumber{3}%
\task{%
    Два плоских зеркала располагаются под углом друг к другу
    и между ними помещается точечный источник света.
    Расстояние от этого источника до одного зеркала $3\,\text{см}$, до другого $7\,\text{см}$.
    Расстояние между первыми изображениями в зеркалах $19{,}43\,\text{см}$.
    Найдите угол (в градусах) между зеркалами.
}
\answer{%
    $\cos \alpha = \frac{c^2 - \sqr{2a} - \sqr{2b}}{2 \cdot 2a \cdot 2b} \approx 0{,}866 \implies \alpha = 30{,}0\degrees$
}
\solutionspace{80pt}

\tasknumber{4}%
\task{%
    Докажите, что тонкий клин с углом $\varphi$ при вершине из стекла с показателем преломления $n$
        отклонит луч на угол $(n-1)\varphi$ (в приближении малых углов).
}
\solutionspace{150pt}

\tasknumber{5}%
\task{%
    Солнце составляет с горизонтом угол, синус которого  0{,}7 .
    Шест высотой $130\,\text{см}$ вбит в дно водоёма глубиной $90\,\text{см}$.
    Найдите длину тени от этого шеста на дне водоёма, если показатель преломления воды 1{,}33.
}
\answer{%
    $
        n \sin \beta = 1 \cdot \cos \alpha \implies \beta \approx 32{,}5\degrees,
        L = (H - h)\ctg \alpha + h \tg \beta \approx 40{,}8\,\text{см} + 57{,}3\,\text{см} \approx 98{,}1\,\text{см}.
    $
}

\variantsplitter

\addpersonalvariant{Дарья Кошман}

\tasknumber{1}%
\task{%
    Для закона отражения:
    \begin{itemize}
        \item сделайте рисунок,
        \item отметьте все необходимые углы и подпишите их названия,
        \item запишите этот закон формулой.
    \end{itemize}
}
\solutionspace{80pt}

\tasknumber{2}%
\task{%
    Под каким углом (в градусах) к горизонту следует расположить плоское зеркало,
    чтобы осветить дно вертикального колодца отраженными от зеркала солнечными лучами,
    падающими под углом $40\degrees$ к вертикали?
}
\answer{%
    $\alpha = 50\degrees, (90\degrees - \beta + \alpha) + (90\degrees - \beta) = 90\degrees \implies \beta = 45\degrees + \frac \alpha 2 = 70\degrees$
}
\solutionspace{80pt}

\tasknumber{3}%
\task{%
    Два плоских зеркала располагаются под углом друг к другу
    и между ними помещается точечный источник света.
    Расстояние от этого источника до одного зеркала $4\,\text{см}$, до другого $7\,\text{см}$.
    Расстояние между первыми изображениями в зеркалах $21{,}31\,\text{см}$.
    Найдите угол (в градусах) между зеркалами.
}
\answer{%
    $\cos \alpha = \frac{c^2 - \sqr{2a} - \sqr{2b}}{2 \cdot 2a \cdot 2b} \approx 0{,}867 \implies \alpha = 29{,}9\degrees$
}
\solutionspace{80pt}

\tasknumber{4}%
\task{%
    Докажите, что тонкий клин с углом $\varphi$ при вершине из стекла с показателем преломления $n$
        отклонит луч на угол $(n-1)\varphi$ (в приближении малых углов).
}
\solutionspace{150pt}

\tasknumber{5}%
\task{%
    Солнце составляет с горизонтом угол, синус которого  0{,}5 .
    Шест высотой $140\,\text{см}$ вбит в дно водоёма глубиной $70\,\text{см}$.
    Найдите длину тени от этого шеста на дне водоёма, если показатель преломления воды 1{,}33.
}
\answer{%
    $
        n \sin \beta = 1 \cdot \cos \alpha \implies \beta \approx 40{,}6\degrees,
        L = (H - h)\ctg \alpha + h \tg \beta \approx 121{,}2\,\text{см} + 60{,}1\,\text{см} \approx 181{,}3\,\text{см}.
    $
}

\variantsplitter

\addpersonalvariant{Анна Кузьмичёва}

\tasknumber{1}%
\task{%
    Для закона отражения:
    \begin{itemize}
        \item сделайте рисунок,
        \item отметьте все необходимые углы и подпишите их названия,
        \item запишите этот закон формулой.
    \end{itemize}
}
\solutionspace{80pt}

\tasknumber{2}%
\task{%
    Под каким углом (в градусах) к горизонту следует расположить плоское зеркало,
    чтобы осветить дно вертикального колодца отраженными от зеркала солнечными лучами,
    падающими под углом $64\degrees$ к горизонту?
}
\answer{%
    $\alpha = 64\degrees, (90\degrees - \beta + \alpha) + (90\degrees - \beta) = 90\degrees \implies \beta = 45\degrees + \frac \alpha 2 = 77\degrees$
}
\solutionspace{80pt}

\tasknumber{3}%
\task{%
    Два плоских зеркала располагаются под углом друг к другу
    и между ними помещается точечный источник света.
    Расстояние от этого источника до одного зеркала $3\,\text{см}$, до другого $9\,\text{см}$.
    Расстояние между первыми изображениями в зеркалах $21{,}63\,\text{см}$.
    Найдите угол (в градусах) между зеркалами.
}
\answer{%
    $\cos \alpha = \frac{c^2 - \sqr{2a} - \sqr{2b}}{2 \cdot 2a \cdot 2b} \approx 0{,}499 \implies \alpha = 60{,}0\degrees$
}
\solutionspace{80pt}

\tasknumber{4}%
\task{%
    Докажите, что тонкий клин с углом $\varphi$ при вершине из стекла с показателем преломления $n$
        отклонит луч на угол $(n-1)\varphi$ (в приближении малых углов).
}
\solutionspace{150pt}

\tasknumber{5}%
\task{%
    Солнце составляет с горизонтом угол, синус которого  0{,}8 .
    Шест высотой $130\,\text{см}$ вбит в дно водоёма глубиной $70\,\text{см}$.
    Найдите длину тени от этого шеста на дне водоёма, если показатель преломления воды 1{,}33.
}
\answer{%
    $
        n \sin \beta = 1 \cdot \cos \alpha \implies \beta \approx 26{,}8\degrees,
        L = (H - h)\ctg \alpha + h \tg \beta \approx 45\,\text{см} + 35{,}4\,\text{см} \approx 80{,}4\,\text{см}.
    $
}

\variantsplitter

\addpersonalvariant{Алёна Куприянова}

\tasknumber{1}%
\task{%
    Для закона преломления:
    \begin{itemize}
        \item сделайте рисунок,
        \item отметьте все необходимые углы и подпишите их названия,
        \item запишите этот закон формулой.
    \end{itemize}
}
\solutionspace{80pt}

\tasknumber{2}%
\task{%
    Под каким углом (в градусах) к горизонту следует расположить плоское зеркало,
    чтобы осветить дно вертикального колодца отраженными от зеркала солнечными лучами,
    падающими под углом $60\degrees$ к горизонту?
}
\answer{%
    $\alpha = 60\degrees, (90\degrees - \beta + \alpha) + (90\degrees - \beta) = 90\degrees \implies \beta = 45\degrees + \frac \alpha 2 = 75\degrees$
}
\solutionspace{80pt}

\tasknumber{3}%
\task{%
    Два плоских зеркала располагаются под углом друг к другу
    и между ними помещается точечный источник света.
    Расстояние от этого источника до одного зеркала $5\,\text{см}$, до другого $9\,\text{см}$.
    Расстояние между первыми изображениями в зеркалах $27{,}13\,\text{см}$.
    Найдите угол (в градусах) между зеркалами.
}
\answer{%
    $\cos \alpha = \frac{c^2 - \sqr{2a} - \sqr{2b}}{2 \cdot 2a \cdot 2b} \approx 0{,}867 \implies \alpha = 29{,}9\degrees$
}
\solutionspace{80pt}

\tasknumber{4}%
\task{%
    Докажите, что мнимое изображение точечного источника света под поверхностью воды из воздуха
        видно на глубине в $n$ раз меньше его реальной глубины (в приближении малых углов).
}
\solutionspace{150pt}

\tasknumber{5}%
\task{%
    Солнце составляет с горизонтом угол, синус которого  0{,}5 .
    Шест высотой $150\,\text{см}$ вбит в дно водоёма глубиной $90\,\text{см}$.
    Найдите длину тени от этого шеста на дне водоёма, если показатель преломления воды 1{,}33.
}
\answer{%
    $
        n \sin \beta = 1 \cdot \cos \alpha \implies \beta \approx 40{,}6\degrees,
        L = (H - h)\ctg \alpha + h \tg \beta \approx 103{,}9\,\text{см} + 77{,}2\,\text{см} \approx 181{,}1\,\text{см}.
    $
}

\variantsplitter

\addpersonalvariant{Ярослав Лавровский}

\tasknumber{1}%
\task{%
    Для закона преломления:
    \begin{itemize}
        \item сделайте рисунок,
        \item отметьте все необходимые углы и подпишите их названия,
        \item запишите этот закон формулой.
    \end{itemize}
}
\solutionspace{80pt}

\tasknumber{2}%
\task{%
    Под каким углом (в градусах) к горизонту следует расположить плоское зеркало,
    чтобы осветить дно вертикального колодца отраженными от зеркала солнечными лучами,
    падающими под углом $36\degrees$ к вертикали?
}
\answer{%
    $\alpha = 54\degrees, (90\degrees - \beta + \alpha) + (90\degrees - \beta) = 90\degrees \implies \beta = 45\degrees + \frac \alpha 2 = 72\degrees$
}
\solutionspace{80pt}

\tasknumber{3}%
\task{%
    Два плоских зеркала располагаются под углом друг к другу
    и между ними помещается точечный источник света.
    Расстояние от этого источника до одного зеркала $3\,\text{см}$, до другого $8\,\text{см}$.
    Расстояние между первыми изображениями в зеркалах $19{,}70\,\text{см}$.
    Найдите угол (в градусах) между зеркалами.
}
\answer{%
    $\cos \alpha = \frac{c^2 - \sqr{2a} - \sqr{2b}}{2 \cdot 2a \cdot 2b} \approx 0{,}500 \implies \alpha = 60{,}0\degrees$
}
\solutionspace{80pt}

\tasknumber{4}%
\task{%
    Докажите, что мнимое изображение точечного источника света под поверхностью воды из воздуха
        видно на глубине в $n$ раз меньше его реальной глубины (в приближении малых углов).
}
\solutionspace{150pt}

\tasknumber{5}%
\task{%
    Солнце составляет с горизонтом угол, синус которого  0{,}5 .
    Шест высотой $180\,\text{см}$ вбит в дно водоёма глубиной $70\,\text{см}$.
    Найдите длину тени от этого шеста на дне водоёма, если показатель преломления воды 1{,}33.
}
\answer{%
    $
        n \sin \beta = 1 \cdot \cos \alpha \implies \beta \approx 40{,}6\degrees,
        L = (H - h)\ctg \alpha + h \tg \beta \approx 190{,}5\,\text{см} + 60{,}1\,\text{см} \approx 250{,}6\,\text{см}.
    $
}

\variantsplitter

\addpersonalvariant{Анастасия Ламанова}

\tasknumber{1}%
\task{%
    Для закона преломления:
    \begin{itemize}
        \item сделайте рисунок,
        \item отметьте все необходимые углы и подпишите их названия,
        \item запишите этот закон формулой.
    \end{itemize}
}
\solutionspace{80pt}

\tasknumber{2}%
\task{%
    Под каким углом (в градусах) к горизонту следует расположить плоское зеркало,
    чтобы осветить дно вертикального колодца отраженными от зеркала солнечными лучами,
    падающими под углом $68\degrees$ к вертикали?
}
\answer{%
    $\alpha = 22\degrees, (90\degrees - \beta + \alpha) + (90\degrees - \beta) = 90\degrees \implies \beta = 45\degrees + \frac \alpha 2 = 56\degrees$
}
\solutionspace{80pt}

\tasknumber{3}%
\task{%
    Два плоских зеркала располагаются под углом друг к другу
    и между ними помещается точечный источник света.
    Расстояние от этого источника до одного зеркала $4\,\text{см}$, до другого $7\,\text{см}$.
    Расстояние между первыми изображениями в зеркалах $21{,}31\,\text{см}$.
    Найдите угол (в градусах) между зеркалами.
}
\answer{%
    $\cos \alpha = \frac{c^2 - \sqr{2a} - \sqr{2b}}{2 \cdot 2a \cdot 2b} \approx 0{,}867 \implies \alpha = 29{,}9\degrees$
}
\solutionspace{80pt}

\tasknumber{4}%
\task{%
    Докажите, что мнимое изображение точечного источника света под поверхностью воды из воздуха
        видно на глубине в $n$ раз меньше его реальной глубины (в приближении малых углов).
}
\solutionspace{150pt}

\tasknumber{5}%
\task{%
    Солнце составляет с горизонтом угол, синус которого  0{,}7 .
    Шест высотой $120\,\text{см}$ вбит в дно водоёма глубиной $70\,\text{см}$.
    Найдите длину тени от этого шеста на дне водоёма, если показатель преломления воды 1{,}33.
}
\answer{%
    $
        n \sin \beta = 1 \cdot \cos \alpha \implies \beta \approx 32{,}5\degrees,
        L = (H - h)\ctg \alpha + h \tg \beta \approx 51\,\text{см} + 44{,}6\,\text{см} \approx 95{,}6\,\text{см}.
    $
}

\variantsplitter

\addpersonalvariant{Виктория Легонькова}

\tasknumber{1}%
\task{%
    Для закона отражения:
    \begin{itemize}
        \item сделайте рисунок,
        \item отметьте все необходимые углы и подпишите их названия,
        \item запишите этот закон формулой.
    \end{itemize}
}
\solutionspace{80pt}

\tasknumber{2}%
\task{%
    Под каким углом (в градусах) к горизонту следует расположить плоское зеркало,
    чтобы осветить дно вертикального колодца отраженными от зеркала солнечными лучами,
    падающими под углом $34\degrees$ к горизонту?
}
\answer{%
    $\alpha = 34\degrees, (90\degrees - \beta + \alpha) + (90\degrees - \beta) = 90\degrees \implies \beta = 45\degrees + \frac \alpha 2 = 62\degrees$
}
\solutionspace{80pt}

\tasknumber{3}%
\task{%
    Два плоских зеркала располагаются под углом друг к другу
    и между ними помещается точечный источник света.
    Расстояние от этого источника до одного зеркала $3\,\text{см}$, до другого $7\,\text{см}$.
    Расстояние между первыми изображениями в зеркалах $19{,}43\,\text{см}$.
    Найдите угол (в градусах) между зеркалами.
}
\answer{%
    $\cos \alpha = \frac{c^2 - \sqr{2a} - \sqr{2b}}{2 \cdot 2a \cdot 2b} \approx 0{,}866 \implies \alpha = 30{,}0\degrees$
}
\solutionspace{80pt}

\tasknumber{4}%
\task{%
    Докажите, что тонкий клин с углом $\varphi$ при вершине из стекла с показателем преломления $n$
        отклонит луч на угол $(n-1)\varphi$ (в приближении малых углов).
}
\solutionspace{150pt}

\tasknumber{5}%
\task{%
    Солнце составляет с горизонтом угол, синус которого  0{,}6 .
    Шест высотой $180\,\text{см}$ вбит в дно водоёма глубиной $90\,\text{см}$.
    Найдите длину тени от этого шеста на дне водоёма, если показатель преломления воды 1{,}33.
}
\answer{%
    $
        n \sin \beta = 1 \cdot \cos \alpha \implies \beta \approx 37{,}0\degrees,
        L = (H - h)\ctg \alpha + h \tg \beta \approx 120\,\text{см} + 67{,}8\,\text{см} \approx 187{,}8\,\text{см}.
    $
}

\variantsplitter

\addpersonalvariant{Семён Мартынов}

\tasknumber{1}%
\task{%
    Для закона преломления:
    \begin{itemize}
        \item сделайте рисунок,
        \item отметьте все необходимые углы и подпишите их названия,
        \item запишите этот закон формулой.
    \end{itemize}
}
\solutionspace{80pt}

\tasknumber{2}%
\task{%
    Под каким углом (в градусах) к горизонту следует расположить плоское зеркало,
    чтобы осветить дно вертикального колодца отраженными от зеркала солнечными лучами,
    падающими под углом $50\degrees$ к горизонту?
}
\answer{%
    $\alpha = 50\degrees, (90\degrees - \beta + \alpha) + (90\degrees - \beta) = 90\degrees \implies \beta = 45\degrees + \frac \alpha 2 = 70\degrees$
}
\solutionspace{80pt}

\tasknumber{3}%
\task{%
    Два плоских зеркала располагаются под углом друг к другу
    и между ними помещается точечный источник света.
    Расстояние от этого источника до одного зеркала $4\,\text{см}$, до другого $8\,\text{см}$.
    Расстояние между первыми изображениями в зеркалах $22{,}38\,\text{см}$.
    Найдите угол (в градусах) между зеркалами.
}
\answer{%
    $\cos \alpha = \frac{c^2 - \sqr{2a} - \sqr{2b}}{2 \cdot 2a \cdot 2b} \approx 0{,}707 \implies \alpha = 45{,}0\degrees$
}
\solutionspace{80pt}

\tasknumber{4}%
\task{%
    Докажите, что мнимое изображение точечного источника света под поверхностью воды из воздуха
        видно на глубине в $n$ раз меньше его реальной глубины (в приближении малых углов).
}
\solutionspace{150pt}

\tasknumber{5}%
\task{%
    Солнце составляет с горизонтом угол, синус которого  0{,}5 .
    Шест высотой $180\,\text{см}$ вбит в дно водоёма глубиной $70\,\text{см}$.
    Найдите длину тени от этого шеста на дне водоёма, если показатель преломления воды 1{,}33.
}
\answer{%
    $
        n \sin \beta = 1 \cdot \cos \alpha \implies \beta \approx 40{,}6\degrees,
        L = (H - h)\ctg \alpha + h \tg \beta \approx 190{,}5\,\text{см} + 60{,}1\,\text{см} \approx 250{,}6\,\text{см}.
    $
}

\variantsplitter

\addpersonalvariant{Варвара Минаева}

\tasknumber{1}%
\task{%
    Для закона преломления:
    \begin{itemize}
        \item сделайте рисунок,
        \item отметьте все необходимые углы и подпишите их названия,
        \item запишите этот закон формулой.
    \end{itemize}
}
\solutionspace{80pt}

\tasknumber{2}%
\task{%
    Под каким углом (в градусах) к горизонту следует расположить плоское зеркало,
    чтобы осветить дно вертикального колодца отраженными от зеркала солнечными лучами,
    падающими под углом $28\degrees$ к вертикали?
}
\answer{%
    $\alpha = 62\degrees, (90\degrees - \beta + \alpha) + (90\degrees - \beta) = 90\degrees \implies \beta = 45\degrees + \frac \alpha 2 = 76\degrees$
}
\solutionspace{80pt}

\tasknumber{3}%
\task{%
    Два плоских зеркала располагаются под углом друг к другу
    и между ними помещается точечный источник света.
    Расстояние от этого источника до одного зеркала $5\,\text{см}$, до другого $6\,\text{см}$.
    Расстояние между первыми изображениями в зеркалах $21{,}26\,\text{см}$.
    Найдите угол (в градусах) между зеркалами.
}
\answer{%
    $\cos \alpha = \frac{c^2 - \sqr{2a} - \sqr{2b}}{2 \cdot 2a \cdot 2b} \approx 0{,}867 \implies \alpha = 29{,}9\degrees$
}
\solutionspace{80pt}

\tasknumber{4}%
\task{%
    Докажите, что тонкий клин с углом $\varphi$ при вершине из стекла с показателем преломления $n$
        отклонит луч на угол $(n-1)\varphi$ (в приближении малых углов).
}
\solutionspace{150pt}

\tasknumber{5}%
\task{%
    Солнце составляет с горизонтом угол, синус которого  0{,}5 .
    Шест высотой $120\,\text{см}$ вбит в дно водоёма глубиной $90\,\text{см}$.
    Найдите длину тени от этого шеста на дне водоёма, если показатель преломления воды 1{,}33.
}
\answer{%
    $
        n \sin \beta = 1 \cdot \cos \alpha \implies \beta \approx 40{,}6\degrees,
        L = (H - h)\ctg \alpha + h \tg \beta \approx 52\,\text{см} + 77{,}2\,\text{см} \approx 129{,}2\,\text{см}.
    $
}

\variantsplitter

\addpersonalvariant{Леонид Никитин}

\tasknumber{1}%
\task{%
    Для закона отражения:
    \begin{itemize}
        \item сделайте рисунок,
        \item отметьте все необходимые углы и подпишите их названия,
        \item запишите этот закон формулой.
    \end{itemize}
}
\solutionspace{80pt}

\tasknumber{2}%
\task{%
    Под каким углом (в градусах) к горизонту следует расположить плоское зеркало,
    чтобы осветить дно вертикального колодца отраженными от зеркала солнечными лучами,
    падающими под углом $22\degrees$ к горизонту?
}
\answer{%
    $\alpha = 22\degrees, (90\degrees - \beta + \alpha) + (90\degrees - \beta) = 90\degrees \implies \beta = 45\degrees + \frac \alpha 2 = 56\degrees$
}
\solutionspace{80pt}

\tasknumber{3}%
\task{%
    Два плоских зеркала располагаются под углом друг к другу
    и между ними помещается точечный источник света.
    Расстояние от этого источника до одного зеркала $4\,\text{см}$, до другого $6\,\text{см}$.
    Расстояние между первыми изображениями в зеркалах $19{,}35\,\text{см}$.
    Найдите угол (в градусах) между зеркалами.
}
\answer{%
    $\cos \alpha = \frac{c^2 - \sqr{2a} - \sqr{2b}}{2 \cdot 2a \cdot 2b} \approx 0{,}867 \implies \alpha = 29{,}9\degrees$
}
\solutionspace{80pt}

\tasknumber{4}%
\task{%
    Докажите, что тонкий клин с углом $\varphi$ при вершине из стекла с показателем преломления $n$
        отклонит луч на угол $(n-1)\varphi$ (в приближении малых углов).
}
\solutionspace{150pt}

\tasknumber{5}%
\task{%
    Солнце составляет с горизонтом угол, синус которого  0{,}5 .
    Шест высотой $120\,\text{см}$ вбит в дно водоёма глубиной $80\,\text{см}$.
    Найдите длину тени от этого шеста на дне водоёма, если показатель преломления воды 1{,}33.
}
\answer{%
    $
        n \sin \beta = 1 \cdot \cos \alpha \implies \beta \approx 40{,}6\degrees,
        L = (H - h)\ctg \alpha + h \tg \beta \approx 69{,}3\,\text{см} + 68{,}6\,\text{см} \approx 137{,}9\,\text{см}.
    $
}

\variantsplitter

\addpersonalvariant{Тимофей Полетаев}

\tasknumber{1}%
\task{%
    Для закона преломления:
    \begin{itemize}
        \item сделайте рисунок,
        \item отметьте все необходимые углы и подпишите их названия,
        \item запишите этот закон формулой.
    \end{itemize}
}
\solutionspace{80pt}

\tasknumber{2}%
\task{%
    Под каким углом (в градусах) к горизонту следует расположить плоское зеркало,
    чтобы осветить дно вертикального колодца отраженными от зеркала солнечными лучами,
    падающими под углом $46\degrees$ к горизонту?
}
\answer{%
    $\alpha = 46\degrees, (90\degrees - \beta + \alpha) + (90\degrees - \beta) = 90\degrees \implies \beta = 45\degrees + \frac \alpha 2 = 68\degrees$
}
\solutionspace{80pt}

\tasknumber{3}%
\task{%
    Два плоских зеркала располагаются под углом друг к другу
    и между ними помещается точечный источник света.
    Расстояние от этого источника до одного зеркала $5\,\text{см}$, до другого $6\,\text{см}$.
    Расстояние между первыми изображениями в зеркалах $21{,}26\,\text{см}$.
    Найдите угол (в градусах) между зеркалами.
}
\answer{%
    $\cos \alpha = \frac{c^2 - \sqr{2a} - \sqr{2b}}{2 \cdot 2a \cdot 2b} \approx 0{,}867 \implies \alpha = 29{,}9\degrees$
}
\solutionspace{80pt}

\tasknumber{4}%
\task{%
    Докажите, что мнимое изображение точечного источника света под поверхностью воды из воздуха
        видно на глубине в $n$ раз меньше его реальной глубины (в приближении малых углов).
}
\solutionspace{150pt}

\tasknumber{5}%
\task{%
    Солнце составляет с горизонтом угол, синус которого  0{,}8 .
    Шест высотой $120\,\text{см}$ вбит в дно водоёма глубиной $80\,\text{см}$.
    Найдите длину тени от этого шеста на дне водоёма, если показатель преломления воды 1{,}33.
}
\answer{%
    $
        n \sin \beta = 1 \cdot \cos \alpha \implies \beta \approx 26{,}8\degrees,
        L = (H - h)\ctg \alpha + h \tg \beta \approx 30\,\text{см} + 40{,}4\,\text{см} \approx 70{,}4\,\text{см}.
    $
}

\variantsplitter

\addpersonalvariant{Андрей Рожков}

\tasknumber{1}%
\task{%
    Для закона отражения:
    \begin{itemize}
        \item сделайте рисунок,
        \item отметьте все необходимые углы и подпишите их названия,
        \item запишите этот закон формулой.
    \end{itemize}
}
\solutionspace{80pt}

\tasknumber{2}%
\task{%
    Под каким углом (в градусах) к горизонту следует расположить плоское зеркало,
    чтобы осветить дно вертикального колодца отраженными от зеркала солнечными лучами,
    падающими под углом $32\degrees$ к вертикали?
}
\answer{%
    $\alpha = 58\degrees, (90\degrees - \beta + \alpha) + (90\degrees - \beta) = 90\degrees \implies \beta = 45\degrees + \frac \alpha 2 = 74\degrees$
}
\solutionspace{80pt}

\tasknumber{3}%
\task{%
    Два плоских зеркала располагаются под углом друг к другу
    и между ними помещается точечный источник света.
    Расстояние от этого источника до одного зеркала $3\,\text{см}$, до другого $8\,\text{см}$.
    Расстояние между первыми изображениями в зеркалах $21{,}41\,\text{см}$.
    Найдите угол (в градусах) между зеркалами.
}
\answer{%
    $\cos \alpha = \frac{c^2 - \sqr{2a} - \sqr{2b}}{2 \cdot 2a \cdot 2b} \approx 0{,}867 \implies \alpha = 29{,}9\degrees$
}
\solutionspace{80pt}

\tasknumber{4}%
\task{%
    Докажите, что мнимое изображение точечного источника света под поверхностью воды из воздуха
        видно на глубине в $n$ раз меньше его реальной глубины (в приближении малых углов).
}
\solutionspace{150pt}

\tasknumber{5}%
\task{%
    Солнце составляет с горизонтом угол, синус которого  0{,}8 .
    Шест высотой $130\,\text{см}$ вбит в дно водоёма глубиной $90\,\text{см}$.
    Найдите длину тени от этого шеста на дне водоёма, если показатель преломления воды 1{,}33.
}
\answer{%
    $
        n \sin \beta = 1 \cdot \cos \alpha \implies \beta \approx 26{,}8\degrees,
        L = (H - h)\ctg \alpha + h \tg \beta \approx 30\,\text{см} + 45{,}5\,\text{см} \approx 75{,}5\,\text{см}.
    $
}

\variantsplitter

\addpersonalvariant{Рената Таржиманова}

\tasknumber{1}%
\task{%
    Для закона преломления:
    \begin{itemize}
        \item сделайте рисунок,
        \item отметьте все необходимые углы и подпишите их названия,
        \item запишите этот закон формулой.
    \end{itemize}
}
\solutionspace{80pt}

\tasknumber{2}%
\task{%
    Под каким углом (в градусах) к горизонту следует расположить плоское зеркало,
    чтобы осветить дно вертикального колодца отраженными от зеркала солнечными лучами,
    падающими под углом $50\degrees$ к горизонту?
}
\answer{%
    $\alpha = 50\degrees, (90\degrees - \beta + \alpha) + (90\degrees - \beta) = 90\degrees \implies \beta = 45\degrees + \frac \alpha 2 = 70\degrees$
}
\solutionspace{80pt}

\tasknumber{3}%
\task{%
    Два плоских зеркала располагаются под углом друг к другу
    и между ними помещается точечный источник света.
    Расстояние от этого источника до одного зеркала $4\,\text{см}$, до другого $9\,\text{см}$.
    Расстояние между первыми изображениями в зеркалах $24{,}32\,\text{см}$.
    Найдите угол (в градусах) между зеркалами.
}
\answer{%
    $\cos \alpha = \frac{c^2 - \sqr{2a} - \sqr{2b}}{2 \cdot 2a \cdot 2b} \approx 0{,}706 \implies \alpha = 45{,}1\degrees$
}
\solutionspace{80pt}

\tasknumber{4}%
\task{%
    Докажите, что тонкий клин с углом $\varphi$ при вершине из стекла с показателем преломления $n$
        отклонит луч на угол $(n-1)\varphi$ (в приближении малых углов).
}
\solutionspace{150pt}

\tasknumber{5}%
\task{%
    Солнце составляет с горизонтом угол, синус которого  0{,}7 .
    Шест высотой $160\,\text{см}$ вбит в дно водоёма глубиной $80\,\text{см}$.
    Найдите длину тени от этого шеста на дне водоёма, если показатель преломления воды 1{,}33.
}
\answer{%
    $
        n \sin \beta = 1 \cdot \cos \alpha \implies \beta \approx 32{,}5\degrees,
        L = (H - h)\ctg \alpha + h \tg \beta \approx 81{,}6\,\text{см} + 50{,}9\,\text{см} \approx 132{,}5\,\text{см}.
    $
}

\variantsplitter

\addpersonalvariant{Андрей Щербаков}

\tasknumber{1}%
\task{%
    Для закона отражения:
    \begin{itemize}
        \item сделайте рисунок,
        \item отметьте все необходимые углы и подпишите их названия,
        \item запишите этот закон формулой.
    \end{itemize}
}
\solutionspace{80pt}

\tasknumber{2}%
\task{%
    Под каким углом (в градусах) к горизонту следует расположить плоское зеркало,
    чтобы осветить дно вертикального колодца отраженными от зеркала солнечными лучами,
    падающими под углом $44\degrees$ к горизонту?
}
\answer{%
    $\alpha = 44\degrees, (90\degrees - \beta + \alpha) + (90\degrees - \beta) = 90\degrees \implies \beta = 45\degrees + \frac \alpha 2 = 67\degrees$
}
\solutionspace{80pt}

\tasknumber{3}%
\task{%
    Два плоских зеркала располагаются под углом друг к другу
    и между ними помещается точечный источник света.
    Расстояние от этого источника до одного зеркала $3\,\text{см}$, до другого $7\,\text{см}$.
    Расстояние между первыми изображениями в зеркалах $18{,}73\,\text{см}$.
    Найдите угол (в градусах) между зеркалами.
}
\answer{%
    $\cos \alpha = \frac{c^2 - \sqr{2a} - \sqr{2b}}{2 \cdot 2a \cdot 2b} \approx 0{,}707 \implies \alpha = 45{,}0\degrees$
}
\solutionspace{80pt}

\tasknumber{4}%
\task{%
    Докажите, что тонкий клин с углом $\varphi$ при вершине из стекла с показателем преломления $n$
        отклонит луч на угол $(n-1)\varphi$ (в приближении малых углов).
}
\solutionspace{150pt}

\tasknumber{5}%
\task{%
    Солнце составляет с горизонтом угол, синус которого  0{,}6 .
    Шест высотой $170\,\text{см}$ вбит в дно водоёма глубиной $80\,\text{см}$.
    Найдите длину тени от этого шеста на дне водоёма, если показатель преломления воды 1{,}33.
}
\answer{%
    $
        n \sin \beta = 1 \cdot \cos \alpha \implies \beta \approx 37{,}0\degrees,
        L = (H - h)\ctg \alpha + h \tg \beta \approx 120\,\text{см} + 60{,}2\,\text{см} \approx 180{,}2\,\text{см}.
    $
}

\variantsplitter

\addpersonalvariant{Михаил Ярошевский}

\tasknumber{1}%
\task{%
    Для закона отражения:
    \begin{itemize}
        \item сделайте рисунок,
        \item отметьте все необходимые углы и подпишите их названия,
        \item запишите этот закон формулой.
    \end{itemize}
}
\solutionspace{80pt}

\tasknumber{2}%
\task{%
    Под каким углом (в градусах) к горизонту следует расположить плоское зеркало,
    чтобы осветить дно вертикального колодца отраженными от зеркала солнечными лучами,
    падающими под углом $44\degrees$ к горизонту?
}
\answer{%
    $\alpha = 44\degrees, (90\degrees - \beta + \alpha) + (90\degrees - \beta) = 90\degrees \implies \beta = 45\degrees + \frac \alpha 2 = 67\degrees$
}
\solutionspace{80pt}

\tasknumber{3}%
\task{%
    Два плоских зеркала располагаются под углом друг к другу
    и между ними помещается точечный источник света.
    Расстояние от этого источника до одного зеркала $3\,\text{см}$, до другого $6\,\text{см}$.
    Расстояние между первыми изображениями в зеркалах $17{,}46\,\text{см}$.
    Найдите угол (в градусах) между зеркалами.
}
\answer{%
    $\cos \alpha = \frac{c^2 - \sqr{2a} - \sqr{2b}}{2 \cdot 2a \cdot 2b} \approx 0{,}867 \implies \alpha = 29{,}9\degrees$
}
\solutionspace{80pt}

\tasknumber{4}%
\task{%
    Докажите, что тонкий клин с углом $\varphi$ при вершине из стекла с показателем преломления $n$
        отклонит луч на угол $(n-1)\varphi$ (в приближении малых углов).
}
\solutionspace{150pt}

\tasknumber{5}%
\task{%
    Солнце составляет с горизонтом угол, синус которого  0{,}5 .
    Шест высотой $180\,\text{см}$ вбит в дно водоёма глубиной $70\,\text{см}$.
    Найдите длину тени от этого шеста на дне водоёма, если показатель преломления воды 1{,}33.
}
\answer{%
    $
        n \sin \beta = 1 \cdot \cos \alpha \implies \beta \approx 40{,}6\degrees,
        L = (H - h)\ctg \alpha + h \tg \beta \approx 190{,}5\,\text{см} + 60{,}1\,\text{см} \approx 250{,}6\,\text{см}.
    $
}

\variantsplitter

\addpersonalvariant{Алексей Алимпиев}

\tasknumber{1}%
\task{%
    Для закона отражения:
    \begin{itemize}
        \item сделайте рисунок,
        \item отметьте все необходимые углы и подпишите их названия,
        \item запишите этот закон формулой.
    \end{itemize}
}
\solutionspace{80pt}

\tasknumber{2}%
\task{%
    Под каким углом (в градусах) к горизонту следует расположить плоское зеркало,
    чтобы осветить дно вертикального колодца отраженными от зеркала солнечными лучами,
    падающими под углом $62\degrees$ к горизонту?
}
\answer{%
    $\alpha = 62\degrees, (90\degrees - \beta + \alpha) + (90\degrees - \beta) = 90\degrees \implies \beta = 45\degrees + \frac \alpha 2 = 76\degrees$
}
\solutionspace{80pt}

\tasknumber{3}%
\task{%
    Два плоских зеркала располагаются под углом друг к другу
    и между ними помещается точечный источник света.
    Расстояние от этого источника до одного зеркала $4\,\text{см}$, до другого $7\,\text{см}$.
    Расстояние между первыми изображениями в зеркалах $21{,}31\,\text{см}$.
    Найдите угол (в градусах) между зеркалами.
}
\answer{%
    $\cos \alpha = \frac{c^2 - \sqr{2a} - \sqr{2b}}{2 \cdot 2a \cdot 2b} \approx 0{,}867 \implies \alpha = 29{,}9\degrees$
}
\solutionspace{80pt}

\tasknumber{4}%
\task{%
    Докажите, что мнимое изображение точечного источника света под поверхностью воды из воздуха
        видно на глубине в $n$ раз меньше его реальной глубины (в приближении малых углов).
}
\solutionspace{150pt}

\tasknumber{5}%
\task{%
    Солнце составляет с горизонтом угол, синус которого  0{,}6 .
    Шест высотой $180\,\text{см}$ вбит в дно водоёма глубиной $70\,\text{см}$.
    Найдите длину тени от этого шеста на дне водоёма, если показатель преломления воды 1{,}33.
}
\answer{%
    $
        n \sin \beta = 1 \cdot \cos \alpha \implies \beta \approx 37{,}0\degrees,
        L = (H - h)\ctg \alpha + h \tg \beta \approx 146{,}7\,\text{см} + 52{,}7\,\text{см} \approx 199{,}4\,\text{см}.
    $
}

\variantsplitter

\addpersonalvariant{Евгений Васин}

\tasknumber{1}%
\task{%
    Для закона отражения:
    \begin{itemize}
        \item сделайте рисунок,
        \item отметьте все необходимые углы и подпишите их названия,
        \item запишите этот закон формулой.
    \end{itemize}
}
\solutionspace{80pt}

\tasknumber{2}%
\task{%
    Под каким углом (в градусах) к горизонту следует расположить плоское зеркало,
    чтобы осветить дно вертикального колодца отраженными от зеркала солнечными лучами,
    падающими под углом $60\degrees$ к горизонту?
}
\answer{%
    $\alpha = 60\degrees, (90\degrees - \beta + \alpha) + (90\degrees - \beta) = 90\degrees \implies \beta = 45\degrees + \frac \alpha 2 = 75\degrees$
}
\solutionspace{80pt}

\tasknumber{3}%
\task{%
    Два плоских зеркала располагаются под углом друг к другу
    и между ними помещается точечный источник света.
    Расстояние от этого источника до одного зеркала $5\,\text{см}$, до другого $7\,\text{см}$.
    Расстояние между первыми изображениями в зеркалах $20{,}88\,\text{см}$.
    Найдите угол (в градусах) между зеркалами.
}
\answer{%
    $\cos \alpha = \frac{c^2 - \sqr{2a} - \sqr{2b}}{2 \cdot 2a \cdot 2b} \approx 0{,}500 \implies \alpha = 60{,}0\degrees$
}
\solutionspace{80pt}

\tasknumber{4}%
\task{%
    Докажите, что тонкий клин с углом $\varphi$ при вершине из стекла с показателем преломления $n$
        отклонит луч на угол $(n-1)\varphi$ (в приближении малых углов).
}
\solutionspace{150pt}

\tasknumber{5}%
\task{%
    Солнце составляет с горизонтом угол, синус которого  0{,}6 .
    Шест высотой $150\,\text{см}$ вбит в дно водоёма глубиной $80\,\text{см}$.
    Найдите длину тени от этого шеста на дне водоёма, если показатель преломления воды 1{,}33.
}
\answer{%
    $
        n \sin \beta = 1 \cdot \cos \alpha \implies \beta \approx 37{,}0\degrees,
        L = (H - h)\ctg \alpha + h \tg \beta \approx 93{,}3\,\text{см} + 60{,}2\,\text{см} \approx 153{,}6\,\text{см}.
    $
}

\variantsplitter

\addpersonalvariant{Вячеслав Волохов}

\tasknumber{1}%
\task{%
    Для закона отражения:
    \begin{itemize}
        \item сделайте рисунок,
        \item отметьте все необходимые углы и подпишите их названия,
        \item запишите этот закон формулой.
    \end{itemize}
}
\solutionspace{80pt}

\tasknumber{2}%
\task{%
    Под каким углом (в градусах) к горизонту следует расположить плоское зеркало,
    чтобы осветить дно вертикального колодца отраженными от зеркала солнечными лучами,
    падающими под углом $66\degrees$ к горизонту?
}
\answer{%
    $\alpha = 66\degrees, (90\degrees - \beta + \alpha) + (90\degrees - \beta) = 90\degrees \implies \beta = 45\degrees + \frac \alpha 2 = 78\degrees$
}
\solutionspace{80pt}

\tasknumber{3}%
\task{%
    Два плоских зеркала располагаются под углом друг к другу
    и между ними помещается точечный источник света.
    Расстояние от этого источника до одного зеркала $4\,\text{см}$, до другого $7\,\text{см}$.
    Расстояние между первыми изображениями в зеркалах $19{,}29\,\text{см}$.
    Найдите угол (в градусах) между зеркалами.
}
\answer{%
    $\cos \alpha = \frac{c^2 - \sqr{2a} - \sqr{2b}}{2 \cdot 2a \cdot 2b} \approx 0{,}500 \implies \alpha = 60{,}0\degrees$
}
\solutionspace{80pt}

\tasknumber{4}%
\task{%
    Докажите, что тонкий клин с углом $\varphi$ при вершине из стекла с показателем преломления $n$
        отклонит луч на угол $(n-1)\varphi$ (в приближении малых углов).
}
\solutionspace{150pt}

\tasknumber{5}%
\task{%
    Солнце составляет с горизонтом угол, синус которого  0{,}7 .
    Шест высотой $180\,\text{см}$ вбит в дно водоёма глубиной $70\,\text{см}$.
    Найдите длину тени от этого шеста на дне водоёма, если показатель преломления воды 1{,}33.
}
\answer{%
    $
        n \sin \beta = 1 \cdot \cos \alpha \implies \beta \approx 32{,}5\degrees,
        L = (H - h)\ctg \alpha + h \tg \beta \approx 112{,}2\,\text{см} + 44{,}6\,\text{см} \approx 156{,}8\,\text{см}.
    $
}

\variantsplitter

\addpersonalvariant{Герман Говоров}

\tasknumber{1}%
\task{%
    Для закона преломления:
    \begin{itemize}
        \item сделайте рисунок,
        \item отметьте все необходимые углы и подпишите их названия,
        \item запишите этот закон формулой.
    \end{itemize}
}
\solutionspace{80pt}

\tasknumber{2}%
\task{%
    Под каким углом (в градусах) к горизонту следует расположить плоское зеркало,
    чтобы осветить дно вертикального колодца отраженными от зеркала солнечными лучами,
    падающими под углом $30\degrees$ к вертикали?
}
\answer{%
    $\alpha = 60\degrees, (90\degrees - \beta + \alpha) + (90\degrees - \beta) = 90\degrees \implies \beta = 45\degrees + \frac \alpha 2 = 75\degrees$
}
\solutionspace{80pt}

\tasknumber{3}%
\task{%
    Два плоских зеркала располагаются под углом друг к другу
    и между ними помещается точечный источник света.
    Расстояние от этого источника до одного зеркала $5\,\text{см}$, до другого $6\,\text{см}$.
    Расстояние между первыми изображениями в зеркалах $21{,}26\,\text{см}$.
    Найдите угол (в градусах) между зеркалами.
}
\answer{%
    $\cos \alpha = \frac{c^2 - \sqr{2a} - \sqr{2b}}{2 \cdot 2a \cdot 2b} \approx 0{,}867 \implies \alpha = 29{,}9\degrees$
}
\solutionspace{80pt}

\tasknumber{4}%
\task{%
    Докажите, что мнимое изображение точечного источника света под поверхностью воды из воздуха
        видно на глубине в $n$ раз меньше его реальной глубины (в приближении малых углов).
}
\solutionspace{150pt}

\tasknumber{5}%
\task{%
    Солнце составляет с горизонтом угол, синус которого  0{,}6 .
    Шест высотой $130\,\text{см}$ вбит в дно водоёма глубиной $90\,\text{см}$.
    Найдите длину тени от этого шеста на дне водоёма, если показатель преломления воды 1{,}33.
}
\answer{%
    $
        n \sin \beta = 1 \cdot \cos \alpha \implies \beta \approx 37{,}0\degrees,
        L = (H - h)\ctg \alpha + h \tg \beta \approx 53{,}3\,\text{см} + 67{,}8\,\text{см} \approx 121{,}1\,\text{см}.
    $
}

\variantsplitter

\addpersonalvariant{София Журавлёва}

\tasknumber{1}%
\task{%
    Для закона преломления:
    \begin{itemize}
        \item сделайте рисунок,
        \item отметьте все необходимые углы и подпишите их названия,
        \item запишите этот закон формулой.
    \end{itemize}
}
\solutionspace{80pt}

\tasknumber{2}%
\task{%
    Под каким углом (в градусах) к горизонту следует расположить плоское зеркало,
    чтобы осветить дно вертикального колодца отраженными от зеркала солнечными лучами,
    падающими под углом $48\degrees$ к вертикали?
}
\answer{%
    $\alpha = 42\degrees, (90\degrees - \beta + \alpha) + (90\degrees - \beta) = 90\degrees \implies \beta = 45\degrees + \frac \alpha 2 = 66\degrees$
}
\solutionspace{80pt}

\tasknumber{3}%
\task{%
    Два плоских зеркала располагаются под углом друг к другу
    и между ними помещается точечный источник света.
    Расстояние от этого источника до одного зеркала $5\,\text{см}$, до другого $9\,\text{см}$.
    Расстояние между первыми изображениями в зеркалах $27{,}13\,\text{см}$.
    Найдите угол (в градусах) между зеркалами.
}
\answer{%
    $\cos \alpha = \frac{c^2 - \sqr{2a} - \sqr{2b}}{2 \cdot 2a \cdot 2b} \approx 0{,}867 \implies \alpha = 29{,}9\degrees$
}
\solutionspace{80pt}

\tasknumber{4}%
\task{%
    Докажите, что тонкий клин с углом $\varphi$ при вершине из стекла с показателем преломления $n$
        отклонит луч на угол $(n-1)\varphi$ (в приближении малых углов).
}
\solutionspace{150pt}

\tasknumber{5}%
\task{%
    Солнце составляет с горизонтом угол, синус которого  0{,}5 .
    Шест высотой $140\,\text{см}$ вбит в дно водоёма глубиной $90\,\text{см}$.
    Найдите длину тени от этого шеста на дне водоёма, если показатель преломления воды 1{,}33.
}
\answer{%
    $
        n \sin \beta = 1 \cdot \cos \alpha \implies \beta \approx 40{,}6\degrees,
        L = (H - h)\ctg \alpha + h \tg \beta \approx 86{,}6\,\text{см} + 77{,}2\,\text{см} \approx 163{,}8\,\text{см}.
    $
}

\variantsplitter

\addpersonalvariant{Константин Козлов}

\tasknumber{1}%
\task{%
    Для закона отражения:
    \begin{itemize}
        \item сделайте рисунок,
        \item отметьте все необходимые углы и подпишите их названия,
        \item запишите этот закон формулой.
    \end{itemize}
}
\solutionspace{80pt}

\tasknumber{2}%
\task{%
    Под каким углом (в градусах) к горизонту следует расположить плоское зеркало,
    чтобы осветить дно вертикального колодца отраженными от зеркала солнечными лучами,
    падающими под углом $50\degrees$ к вертикали?
}
\answer{%
    $\alpha = 40\degrees, (90\degrees - \beta + \alpha) + (90\degrees - \beta) = 90\degrees \implies \beta = 45\degrees + \frac \alpha 2 = 65\degrees$
}
\solutionspace{80pt}

\tasknumber{3}%
\task{%
    Два плоских зеркала располагаются под углом друг к другу
    и между ними помещается точечный источник света.
    Расстояние от этого источника до одного зеркала $3\,\text{см}$, до другого $7\,\text{см}$.
    Расстояние между первыми изображениями в зеркалах $18{,}73\,\text{см}$.
    Найдите угол (в градусах) между зеркалами.
}
\answer{%
    $\cos \alpha = \frac{c^2 - \sqr{2a} - \sqr{2b}}{2 \cdot 2a \cdot 2b} \approx 0{,}707 \implies \alpha = 45{,}0\degrees$
}
\solutionspace{80pt}

\tasknumber{4}%
\task{%
    Докажите, что тонкий клин с углом $\varphi$ при вершине из стекла с показателем преломления $n$
        отклонит луч на угол $(n-1)\varphi$ (в приближении малых углов).
}
\solutionspace{150pt}

\tasknumber{5}%
\task{%
    Солнце составляет с горизонтом угол, синус которого  0{,}8 .
    Шест высотой $140\,\text{см}$ вбит в дно водоёма глубиной $70\,\text{см}$.
    Найдите длину тени от этого шеста на дне водоёма, если показатель преломления воды 1{,}33.
}
\answer{%
    $
        n \sin \beta = 1 \cdot \cos \alpha \implies \beta \approx 26{,}8\degrees,
        L = (H - h)\ctg \alpha + h \tg \beta \approx 52{,}5\,\text{см} + 35{,}4\,\text{см} \approx 87{,}9\,\text{см}.
    $
}

\variantsplitter

\addpersonalvariant{Наталья Кравченко}

\tasknumber{1}%
\task{%
    Для закона отражения:
    \begin{itemize}
        \item сделайте рисунок,
        \item отметьте все необходимые углы и подпишите их названия,
        \item запишите этот закон формулой.
    \end{itemize}
}
\solutionspace{80pt}

\tasknumber{2}%
\task{%
    Под каким углом (в градусах) к горизонту следует расположить плоское зеркало,
    чтобы осветить дно вертикального колодца отраженными от зеркала солнечными лучами,
    падающими под углом $42\degrees$ к вертикали?
}
\answer{%
    $\alpha = 48\degrees, (90\degrees - \beta + \alpha) + (90\degrees - \beta) = 90\degrees \implies \beta = 45\degrees + \frac \alpha 2 = 69\degrees$
}
\solutionspace{80pt}

\tasknumber{3}%
\task{%
    Два плоских зеркала располагаются под углом друг к другу
    и между ними помещается точечный источник света.
    Расстояние от этого источника до одного зеркала $3\,\text{см}$, до другого $9\,\text{см}$.
    Расстояние между первыми изображениями в зеркалах $21{,}63\,\text{см}$.
    Найдите угол (в градусах) между зеркалами.
}
\answer{%
    $\cos \alpha = \frac{c^2 - \sqr{2a} - \sqr{2b}}{2 \cdot 2a \cdot 2b} \approx 0{,}499 \implies \alpha = 60{,}0\degrees$
}
\solutionspace{80pt}

\tasknumber{4}%
\task{%
    Докажите, что тонкий клин с углом $\varphi$ при вершине из стекла с показателем преломления $n$
        отклонит луч на угол $(n-1)\varphi$ (в приближении малых углов).
}
\solutionspace{150pt}

\tasknumber{5}%
\task{%
    Солнце составляет с горизонтом угол, синус которого  0{,}6 .
    Шест высотой $140\,\text{см}$ вбит в дно водоёма глубиной $90\,\text{см}$.
    Найдите длину тени от этого шеста на дне водоёма, если показатель преломления воды 1{,}33.
}
\answer{%
    $
        n \sin \beta = 1 \cdot \cos \alpha \implies \beta \approx 37{,}0\degrees,
        L = (H - h)\ctg \alpha + h \tg \beta \approx 66{,}7\,\text{см} + 67{,}8\,\text{см} \approx 134{,}4\,\text{см}.
    $
}

\variantsplitter

\addpersonalvariant{Матвей Кузьмин}

\tasknumber{1}%
\task{%
    Для закона преломления:
    \begin{itemize}
        \item сделайте рисунок,
        \item отметьте все необходимые углы и подпишите их названия,
        \item запишите этот закон формулой.
    \end{itemize}
}
\solutionspace{80pt}

\tasknumber{2}%
\task{%
    Под каким углом (в градусах) к горизонту следует расположить плоское зеркало,
    чтобы осветить дно вертикального колодца отраженными от зеркала солнечными лучами,
    падающими под углом $46\degrees$ к горизонту?
}
\answer{%
    $\alpha = 46\degrees, (90\degrees - \beta + \alpha) + (90\degrees - \beta) = 90\degrees \implies \beta = 45\degrees + \frac \alpha 2 = 68\degrees$
}
\solutionspace{80pt}

\tasknumber{3}%
\task{%
    Два плоских зеркала располагаются под углом друг к другу
    и между ними помещается точечный источник света.
    Расстояние от этого источника до одного зеркала $5\,\text{см}$, до другого $7\,\text{см}$.
    Расстояние между первыми изображениями в зеркалах $20{,}88\,\text{см}$.
    Найдите угол (в градусах) между зеркалами.
}
\answer{%
    $\cos \alpha = \frac{c^2 - \sqr{2a} - \sqr{2b}}{2 \cdot 2a \cdot 2b} \approx 0{,}500 \implies \alpha = 60{,}0\degrees$
}
\solutionspace{80pt}

\tasknumber{4}%
\task{%
    Докажите, что тонкий клин с углом $\varphi$ при вершине из стекла с показателем преломления $n$
        отклонит луч на угол $(n-1)\varphi$ (в приближении малых углов).
}
\solutionspace{150pt}

\tasknumber{5}%
\task{%
    Солнце составляет с горизонтом угол, синус которого  0{,}8 .
    Шест высотой $160\,\text{см}$ вбит в дно водоёма глубиной $70\,\text{см}$.
    Найдите длину тени от этого шеста на дне водоёма, если показатель преломления воды 1{,}33.
}
\answer{%
    $
        n \sin \beta = 1 \cdot \cos \alpha \implies \beta \approx 26{,}8\degrees,
        L = (H - h)\ctg \alpha + h \tg \beta \approx 67{,}5\,\text{см} + 35{,}4\,\text{см} \approx 102{,}9\,\text{см}.
    $
}

\variantsplitter

\addpersonalvariant{Сергей Малышев}

\tasknumber{1}%
\task{%
    Для закона преломления:
    \begin{itemize}
        \item сделайте рисунок,
        \item отметьте все необходимые углы и подпишите их названия,
        \item запишите этот закон формулой.
    \end{itemize}
}
\solutionspace{80pt}

\tasknumber{2}%
\task{%
    Под каким углом (в градусах) к горизонту следует расположить плоское зеркало,
    чтобы осветить дно вертикального колодца отраженными от зеркала солнечными лучами,
    падающими под углом $50\degrees$ к горизонту?
}
\answer{%
    $\alpha = 50\degrees, (90\degrees - \beta + \alpha) + (90\degrees - \beta) = 90\degrees \implies \beta = 45\degrees + \frac \alpha 2 = 70\degrees$
}
\solutionspace{80pt}

\tasknumber{3}%
\task{%
    Два плоских зеркала располагаются под углом друг к другу
    и между ними помещается точечный источник света.
    Расстояние от этого источника до одного зеркала $3\,\text{см}$, до другого $6\,\text{см}$.
    Расстояние между первыми изображениями в зеркалах $17{,}46\,\text{см}$.
    Найдите угол (в градусах) между зеркалами.
}
\answer{%
    $\cos \alpha = \frac{c^2 - \sqr{2a} - \sqr{2b}}{2 \cdot 2a \cdot 2b} \approx 0{,}867 \implies \alpha = 29{,}9\degrees$
}
\solutionspace{80pt}

\tasknumber{4}%
\task{%
    Докажите, что мнимое изображение точечного источника света под поверхностью воды из воздуха
        видно на глубине в $n$ раз меньше его реальной глубины (в приближении малых углов).
}
\solutionspace{150pt}

\tasknumber{5}%
\task{%
    Солнце составляет с горизонтом угол, синус которого  0{,}8 .
    Шест высотой $140\,\text{см}$ вбит в дно водоёма глубиной $80\,\text{см}$.
    Найдите длину тени от этого шеста на дне водоёма, если показатель преломления воды 1{,}33.
}
\answer{%
    $
        n \sin \beta = 1 \cdot \cos \alpha \implies \beta \approx 26{,}8\degrees,
        L = (H - h)\ctg \alpha + h \tg \beta \approx 45\,\text{см} + 40{,}4\,\text{см} \approx 85{,}4\,\text{см}.
    $
}

\variantsplitter

\addpersonalvariant{Алина Полканова}

\tasknumber{1}%
\task{%
    Для закона отражения:
    \begin{itemize}
        \item сделайте рисунок,
        \item отметьте все необходимые углы и подпишите их названия,
        \item запишите этот закон формулой.
    \end{itemize}
}
\solutionspace{80pt}

\tasknumber{2}%
\task{%
    Под каким углом (в градусах) к горизонту следует расположить плоское зеркало,
    чтобы осветить дно вертикального колодца отраженными от зеркала солнечными лучами,
    падающими под углом $60\degrees$ к вертикали?
}
\answer{%
    $\alpha = 30\degrees, (90\degrees - \beta + \alpha) + (90\degrees - \beta) = 90\degrees \implies \beta = 45\degrees + \frac \alpha 2 = 60\degrees$
}
\solutionspace{80pt}

\tasknumber{3}%
\task{%
    Два плоских зеркала располагаются под углом друг к другу
    и между ними помещается точечный источник света.
    Расстояние от этого источника до одного зеркала $3\,\text{см}$, до другого $9\,\text{см}$.
    Расстояние между первыми изображениями в зеркалах $23{,}39\,\text{см}$.
    Найдите угол (в градусах) между зеркалами.
}
\answer{%
    $\cos \alpha = \frac{c^2 - \sqr{2a} - \sqr{2b}}{2 \cdot 2a \cdot 2b} \approx 0{,}866 \implies \alpha = 30{,}0\degrees$
}
\solutionspace{80pt}

\tasknumber{4}%
\task{%
    Докажите, что тонкий клин с углом $\varphi$ при вершине из стекла с показателем преломления $n$
        отклонит луч на угол $(n-1)\varphi$ (в приближении малых углов).
}
\solutionspace{150pt}

\tasknumber{5}%
\task{%
    Солнце составляет с горизонтом угол, синус которого  0{,}8 .
    Шест высотой $160\,\text{см}$ вбит в дно водоёма глубиной $90\,\text{см}$.
    Найдите длину тени от этого шеста на дне водоёма, если показатель преломления воды 1{,}33.
}
\answer{%
    $
        n \sin \beta = 1 \cdot \cos \alpha \implies \beta \approx 26{,}8\degrees,
        L = (H - h)\ctg \alpha + h \tg \beta \approx 52{,}5\,\text{см} + 45{,}5\,\text{см} \approx 98\,\text{см}.
    $
}

\variantsplitter

\addpersonalvariant{Сергей Пономарёв}

\tasknumber{1}%
\task{%
    Для закона преломления:
    \begin{itemize}
        \item сделайте рисунок,
        \item отметьте все необходимые углы и подпишите их названия,
        \item запишите этот закон формулой.
    \end{itemize}
}
\solutionspace{80pt}

\tasknumber{2}%
\task{%
    Под каким углом (в градусах) к горизонту следует расположить плоское зеркало,
    чтобы осветить дно вертикального колодца отраженными от зеркала солнечными лучами,
    падающими под углом $46\degrees$ к горизонту?
}
\answer{%
    $\alpha = 46\degrees, (90\degrees - \beta + \alpha) + (90\degrees - \beta) = 90\degrees \implies \beta = 45\degrees + \frac \alpha 2 = 68\degrees$
}
\solutionspace{80pt}

\tasknumber{3}%
\task{%
    Два плоских зеркала располагаются под углом друг к другу
    и между ними помещается точечный источник света.
    Расстояние от этого источника до одного зеркала $4\,\text{см}$, до другого $6\,\text{см}$.
    Расстояние между первыми изображениями в зеркалах $17{,}44\,\text{см}$.
    Найдите угол (в градусах) между зеркалами.
}
\answer{%
    $\cos \alpha = \frac{c^2 - \sqr{2a} - \sqr{2b}}{2 \cdot 2a \cdot 2b} \approx 0{,}501 \implies \alpha = 59{,}9\degrees$
}
\solutionspace{80pt}

\tasknumber{4}%
\task{%
    Докажите, что мнимое изображение точечного источника света под поверхностью воды из воздуха
        видно на глубине в $n$ раз меньше его реальной глубины (в приближении малых углов).
}
\solutionspace{150pt}

\tasknumber{5}%
\task{%
    Солнце составляет с горизонтом угол, синус которого  0{,}7 .
    Шест высотой $140\,\text{см}$ вбит в дно водоёма глубиной $90\,\text{см}$.
    Найдите длину тени от этого шеста на дне водоёма, если показатель преломления воды 1{,}33.
}
\answer{%
    $
        n \sin \beta = 1 \cdot \cos \alpha \implies \beta \approx 32{,}5\degrees,
        L = (H - h)\ctg \alpha + h \tg \beta \approx 51\,\text{см} + 57{,}3\,\text{см} \approx 108{,}3\,\text{см}.
    $
}

\variantsplitter

\addpersonalvariant{Егор Свистушкин}

\tasknumber{1}%
\task{%
    Для закона отражения:
    \begin{itemize}
        \item сделайте рисунок,
        \item отметьте все необходимые углы и подпишите их названия,
        \item запишите этот закон формулой.
    \end{itemize}
}
\solutionspace{80pt}

\tasknumber{2}%
\task{%
    Под каким углом (в градусах) к горизонту следует расположить плоское зеркало,
    чтобы осветить дно вертикального колодца отраженными от зеркала солнечными лучами,
    падающими под углом $58\degrees$ к вертикали?
}
\answer{%
    $\alpha = 32\degrees, (90\degrees - \beta + \alpha) + (90\degrees - \beta) = 90\degrees \implies \beta = 45\degrees + \frac \alpha 2 = 61\degrees$
}
\solutionspace{80pt}

\tasknumber{3}%
\task{%
    Два плоских зеркала располагаются под углом друг к другу
    и между ними помещается точечный источник света.
    Расстояние от этого источника до одного зеркала $5\,\text{см}$, до другого $9\,\text{см}$.
    Расстояние между первыми изображениями в зеркалах $26{,}05\,\text{см}$.
    Найдите угол (в градусах) между зеркалами.
}
\answer{%
    $\cos \alpha = \frac{c^2 - \sqr{2a} - \sqr{2b}}{2 \cdot 2a \cdot 2b} \approx 0{,}707 \implies \alpha = 45{,}0\degrees$
}
\solutionspace{80pt}

\tasknumber{4}%
\task{%
    Докажите, что мнимое изображение точечного источника света под поверхностью воды из воздуха
        видно на глубине в $n$ раз меньше его реальной глубины (в приближении малых углов).
}
\solutionspace{150pt}

\tasknumber{5}%
\task{%
    Солнце составляет с горизонтом угол, синус которого  0{,}7 .
    Шест высотой $180\,\text{см}$ вбит в дно водоёма глубиной $90\,\text{см}$.
    Найдите длину тени от этого шеста на дне водоёма, если показатель преломления воды 1{,}33.
}
\answer{%
    $
        n \sin \beta = 1 \cdot \cos \alpha \implies \beta \approx 32{,}5\degrees,
        L = (H - h)\ctg \alpha + h \tg \beta \approx 91{,}8\,\text{см} + 57{,}3\,\text{см} \approx 149{,}1\,\text{см}.
    $
}

\variantsplitter

\addpersonalvariant{Дмитрий Соколов}

\tasknumber{1}%
\task{%
    Для закона отражения:
    \begin{itemize}
        \item сделайте рисунок,
        \item отметьте все необходимые углы и подпишите их названия,
        \item запишите этот закон формулой.
    \end{itemize}
}
\solutionspace{80pt}

\tasknumber{2}%
\task{%
    Под каким углом (в градусах) к горизонту следует расположить плоское зеркало,
    чтобы осветить дно вертикального колодца отраженными от зеркала солнечными лучами,
    падающими под углом $48\degrees$ к горизонту?
}
\answer{%
    $\alpha = 48\degrees, (90\degrees - \beta + \alpha) + (90\degrees - \beta) = 90\degrees \implies \beta = 45\degrees + \frac \alpha 2 = 69\degrees$
}
\solutionspace{80pt}

\tasknumber{3}%
\task{%
    Два плоских зеркала располагаются под углом друг к другу
    и между ними помещается точечный источник света.
    Расстояние от этого источника до одного зеркала $4\,\text{см}$, до другого $6\,\text{см}$.
    Расстояние между первыми изображениями в зеркалах $19{,}35\,\text{см}$.
    Найдите угол (в градусах) между зеркалами.
}
\answer{%
    $\cos \alpha = \frac{c^2 - \sqr{2a} - \sqr{2b}}{2 \cdot 2a \cdot 2b} \approx 0{,}867 \implies \alpha = 29{,}9\degrees$
}
\solutionspace{80pt}

\tasknumber{4}%
\task{%
    Докажите, что тонкий клин с углом $\varphi$ при вершине из стекла с показателем преломления $n$
        отклонит луч на угол $(n-1)\varphi$ (в приближении малых углов).
}
\solutionspace{150pt}

\tasknumber{5}%
\task{%
    Солнце составляет с горизонтом угол, синус которого  0{,}5 .
    Шест высотой $160\,\text{см}$ вбит в дно водоёма глубиной $90\,\text{см}$.
    Найдите длину тени от этого шеста на дне водоёма, если показатель преломления воды 1{,}33.
}
\answer{%
    $
        n \sin \beta = 1 \cdot \cos \alpha \implies \beta \approx 40{,}6\degrees,
        L = (H - h)\ctg \alpha + h \tg \beta \approx 121{,}2\,\text{см} + 77{,}2\,\text{см} \approx 198{,}5\,\text{см}.
    $
}

\variantsplitter

\addpersonalvariant{Арсений Трофимов}

\tasknumber{1}%
\task{%
    Для закона отражения:
    \begin{itemize}
        \item сделайте рисунок,
        \item отметьте все необходимые углы и подпишите их названия,
        \item запишите этот закон формулой.
    \end{itemize}
}
\solutionspace{80pt}

\tasknumber{2}%
\task{%
    Под каким углом (в градусах) к горизонту следует расположить плоское зеркало,
    чтобы осветить дно вертикального колодца отраженными от зеркала солнечными лучами,
    падающими под углом $40\degrees$ к вертикали?
}
\answer{%
    $\alpha = 50\degrees, (90\degrees - \beta + \alpha) + (90\degrees - \beta) = 90\degrees \implies \beta = 45\degrees + \frac \alpha 2 = 70\degrees$
}
\solutionspace{80pt}

\tasknumber{3}%
\task{%
    Два плоских зеркала располагаются под углом друг к другу
    и между ними помещается точечный источник света.
    Расстояние от этого источника до одного зеркала $3\,\text{см}$, до другого $7\,\text{см}$.
    Расстояние между первыми изображениями в зеркалах $18{,}73\,\text{см}$.
    Найдите угол (в градусах) между зеркалами.
}
\answer{%
    $\cos \alpha = \frac{c^2 - \sqr{2a} - \sqr{2b}}{2 \cdot 2a \cdot 2b} \approx 0{,}707 \implies \alpha = 45{,}0\degrees$
}
\solutionspace{80pt}

\tasknumber{4}%
\task{%
    Докажите, что мнимое изображение точечного источника света под поверхностью воды из воздуха
        видно на глубине в $n$ раз меньше его реальной глубины (в приближении малых углов).
}
\solutionspace{150pt}

\tasknumber{5}%
\task{%
    Солнце составляет с горизонтом угол, синус которого  0{,}8 .
    Шест высотой $130\,\text{см}$ вбит в дно водоёма глубиной $70\,\text{см}$.
    Найдите длину тени от этого шеста на дне водоёма, если показатель преломления воды 1{,}33.
}
\answer{%
    $
        n \sin \beta = 1 \cdot \cos \alpha \implies \beta \approx 26{,}8\degrees,
        L = (H - h)\ctg \alpha + h \tg \beta \approx 45\,\text{см} + 35{,}4\,\text{см} \approx 80{,}4\,\text{см}.
    $
}
% autogenerated
