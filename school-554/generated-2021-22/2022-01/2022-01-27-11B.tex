\setdate{27~января~2022}
\setclass{11«Б»}

\addpersonalvariant{Михаил Бурмистров}

\tasknumber{1}%
\task{%
    Под каким углом (в градусах) к горизонту следует расположить плоское зеркало,
    чтобы осветить дно вертикального колодца отраженными от зеркала солнечными лучами,
    падающими под углом $68\degrees$ к горизонту?
}
\answer{%
    $\alpha = 68\degrees, (90\degrees - \beta + \alpha) + (90\degrees - \beta) = 90\degrees \implies \beta = 45\degrees + \frac \alpha 2 = 79\degrees$
}
\solutionspace{80pt}

\tasknumber{2}%
\task{%
    Во сколько раз увеличится расстояние между предметом и его изображением
    в плоском зеркале, если зеркало переместить в то место, где было изображение? Предмет остаётся неподвижным.
}
\answer{%
    $2$
}
\solutionspace{80pt}

\tasknumber{3}%
\task{%
    Плоское зеркало движется по направлению к точечному источнику света со скоростью $18\,\frac{\text{см}}{\text{с}}$.
    Определите скорость движения изображения относительно источника света.
    Направление скорости зеркала перпендикулярно плоскости зеркала.
}
\answer{%
    $36\,\frac{\text{см}}{\text{с}}$
}
\solutionspace{80pt}

\tasknumber{4}%
\task{%
    Луч падает из вакуума на стекло с показателем преломления  1{,}35 .
    Сделайте рисунок (без рисунка и отмеченных углов задача не проверяется) и определите:
    \begin{itemize}
        \item угол отражения,
        \item угол преломления,
        \item угол между падающим и отраженным лучом,
        \item угол между падающим и преломленным лучом,
        \item угол отклонения луча при преломлении,
    \end{itemize}
    если угол падения равен $35\degrees$.
}
\answer{%
    \begin{align*}
    \alpha &= 35\degrees, \\
    1 \cdot \sin \alpha &= n \sin \beta \implies \beta = \arcsin\cbr{ \frac{\sin \alpha}{ n } } \approx 25{,}14\degrees, \\
    \varphi_1 &= \alpha \approx 35\degrees, \\
    \varphi_2 &= \beta \approx 25{,}14\degrees, \\
    \varphi_3 &= 2\alpha = 70\degrees, \\
    \varphi_4 &= 180\degrees - \alpha + \beta \approx 170{,}14\degrees, \\
    \varphi_5 &= \alpha - \beta \approx 9{,}86\degrees.
    \end{align*}
}
\solutionspace{100pt}

\tasknumber{5}%
\task{%
    На дне водоёма глубиной $2\,\text{м}$ лежит зеркало.
    Луч света, пройдя через воду, отражается от зеркала и выходит из воды.
    Найти расстояние между точкой входа луча в воду и точкой выхода луча из воды,
    если показатель преломления воды $1{,}33$, а угол падения луча $30\degrees$.
}
\answer{%
    \begin{align*}
    \ctg \beta &= \frac{h}{d:L:s} \implies d = \frac{h}{\ctg \beta} \\
    \frac 1{\sin^2 \beta} &= \ctg^2 \beta + 1 \implies \ctg \beta = \sqrt{\frac 1{\sin^2 \beta} - 1} \\
    \sin\alpha &= n\sin \beta \implies \sin \beta = \frac{\sin\alpha}{n} \\
    d &= \frac{h}{\sqrt{\frac 1{\sin^2 \beta} - 1}} = \frac{h}{\sqrt{\sqr{\frac{n}{\sin\alpha}} - 1}} \\
    2d &= \frac{2{h}}{\sqrt{\sqr{\frac{n}{\sin\alpha}} - 1}} \approx 162{,}3\,\text{см}
    \end{align*}
}

\variantsplitter

\addpersonalvariant{Снежана Авдошина}

\tasknumber{1}%
\task{%
    Под каким углом (в градусах) к горизонту следует расположить плоское зеркало,
    чтобы осветить дно вертикального колодца отраженными от зеркала солнечными лучами,
    падающими под углом $48\degrees$ к горизонту?
}
\answer{%
    $\alpha = 48\degrees, (90\degrees - \beta + \alpha) + (90\degrees - \beta) = 90\degrees \implies \beta = 45\degrees + \frac \alpha 2 = 69\degrees$
}
\solutionspace{80pt}

\tasknumber{2}%
\task{%
    Во сколько раз увеличится расстояние между предметом и его изображением
    в плоском зеркале, если зеркало переместить в то место, где было изображение? Предмет остаётся неподвижным.
}
\answer{%
    $2$
}
\solutionspace{80pt}

\tasknumber{3}%
\task{%
    Плоское зеркало движется по направлению к точечному источнику света со скоростью $15\,\frac{\text{см}}{\text{с}}$.
    Определите скорость движения изображения относительно источника света.
    Направление скорости зеркала перпендикулярно плоскости зеркала.
}
\answer{%
    $30\,\frac{\text{см}}{\text{с}}$
}
\solutionspace{80pt}

\tasknumber{4}%
\task{%
    Луч падает из вакуума на стекло с показателем преломления  1{,}65 .
    Сделайте рисунок (без рисунка и отмеченных углов задача не проверяется) и определите:
    \begin{itemize}
        \item угол отражения,
        \item угол преломления,
        \item угол между падающим и отраженным лучом,
        \item угол между падающим и преломленным лучом,
        \item угол отклонения луча при преломлении,
    \end{itemize}
    если между падающим лучом и границей раздела сред равен $50\degrees$.
}
\answer{%
    \begin{align*}
    \alpha &= 40\degrees, \\
    1 \cdot \sin \alpha &= n \sin \beta \implies \beta = \arcsin\cbr{ \frac{\sin \alpha}{ n } } \approx 22{,}93\degrees, \\
    \varphi_1 &= \alpha \approx 40\degrees, \\
    \varphi_2 &= \beta \approx 22{,}93\degrees, \\
    \varphi_3 &= 2\alpha = 80\degrees, \\
    \varphi_4 &= 180\degrees - \alpha + \beta \approx 162{,}93\degrees, \\
    \varphi_5 &= \alpha - \beta \approx 17{,}07\degrees.
    \end{align*}
}
\solutionspace{100pt}

\tasknumber{5}%
\task{%
    На дне водоёма глубиной $2\,\text{м}$ лежит зеркало.
    Луч света, пройдя через воду, отражается от зеркала и выходит из воды.
    Найти расстояние между точкой входа луча в воду и точкой выхода луча из воды,
    если показатель преломления воды $1{,}33$, а угол падения луча $35\degrees$.
}
\answer{%
    \begin{align*}
    \ctg \beta &= \frac{h}{d:L:s} \implies d = \frac{h}{\ctg \beta} \\
    \frac 1{\sin^2 \beta} &= \ctg^2 \beta + 1 \implies \ctg \beta = \sqrt{\frac 1{\sin^2 \beta} - 1} \\
    \sin\alpha &= n\sin \beta \implies \sin \beta = \frac{\sin\alpha}{n} \\
    d &= \frac{h}{\sqrt{\frac 1{\sin^2 \beta} - 1}} = \frac{h}{\sqrt{\sqr{\frac{n}{\sin\alpha}} - 1}} \\
    2d &= \frac{2{h}}{\sqrt{\sqr{\frac{n}{\sin\alpha}} - 1}} \approx 191{,}2\,\text{см}
    \end{align*}
}

\variantsplitter

\addpersonalvariant{Марьяна Аристова}

\tasknumber{1}%
\task{%
    Под каким углом (в градусах) к горизонту следует расположить плоское зеркало,
    чтобы осветить дно вертикального колодца отраженными от зеркала солнечными лучами,
    падающими под углом $24\degrees$ к горизонту?
}
\answer{%
    $\alpha = 24\degrees, (90\degrees - \beta + \alpha) + (90\degrees - \beta) = 90\degrees \implies \beta = 45\degrees + \frac \alpha 2 = 57\degrees$
}
\solutionspace{80pt}

\tasknumber{2}%
\task{%
    Во сколько раз увеличится расстояние между предметом и его изображением
    в плоском зеркале, если зеркало переместить в то место, где было изображение? Предмет остаётся неподвижным.
}
\answer{%
    $2$
}
\solutionspace{80pt}

\tasknumber{3}%
\task{%
    Плоское зеркало движется по направлению к точечному источнику света со скоростью $20\,\frac{\text{см}}{\text{с}}$.
    Определите скорость движения изображения относительно зеркала.
    Направление скорости зеркала перпендикулярно плоскости зеркала.
}
\answer{%
    $20\,\frac{\text{см}}{\text{с}}$
}
\solutionspace{80pt}

\tasknumber{4}%
\task{%
    Луч падает из вакуума на стекло с показателем преломления  1{,}65 .
    Сделайте рисунок (без рисунка и отмеченных углов задача не проверяется) и определите:
    \begin{itemize}
        \item угол отражения,
        \item угол преломления,
        \item угол между падающим и отраженным лучом,
        \item угол между падающим и преломленным лучом,
        \item угол отклонения луча при преломлении,
    \end{itemize}
    если между падающим лучом и границей раздела сред равен $35\degrees$.
}
\answer{%
    \begin{align*}
    \alpha &= 55\degrees, \\
    1 \cdot \sin \alpha &= n \sin \beta \implies \beta = \arcsin\cbr{ \frac{\sin \alpha}{ n } } \approx 29{,}77\degrees, \\
    \varphi_1 &= \alpha \approx 55\degrees, \\
    \varphi_2 &= \beta \approx 29{,}77\degrees, \\
    \varphi_3 &= 2\alpha = 110\degrees, \\
    \varphi_4 &= 180\degrees - \alpha + \beta \approx 154{,}77\degrees, \\
    \varphi_5 &= \alpha - \beta \approx 25{,}23\degrees.
    \end{align*}
}
\solutionspace{100pt}

\tasknumber{5}%
\task{%
    На дне водоёма глубиной $3\,\text{м}$ лежит зеркало.
    Луч света, пройдя через воду, отражается от зеркала и выходит из воды.
    Найти расстояние между точкой входа луча в воду и точкой выхода луча из воды,
    если показатель преломления воды $1{,}33$, а угол падения луча $35\degrees$.
}
\answer{%
    \begin{align*}
    \ctg \beta &= \frac{h}{d:L:s} \implies d = \frac{h}{\ctg \beta} \\
    \frac 1{\sin^2 \beta} &= \ctg^2 \beta + 1 \implies \ctg \beta = \sqrt{\frac 1{\sin^2 \beta} - 1} \\
    \sin\alpha &= n\sin \beta \implies \sin \beta = \frac{\sin\alpha}{n} \\
    d &= \frac{h}{\sqrt{\frac 1{\sin^2 \beta} - 1}} = \frac{h}{\sqrt{\sqr{\frac{n}{\sin\alpha}} - 1}} \\
    2d &= \frac{2{h}}{\sqrt{\sqr{\frac{n}{\sin\alpha}} - 1}} \approx 286{,}8\,\text{см}
    \end{align*}
}

\variantsplitter

\addpersonalvariant{Никита Иванов}

\tasknumber{1}%
\task{%
    Под каким углом (в градусах) к горизонту следует расположить плоское зеркало,
    чтобы осветить дно вертикального колодца отраженными от зеркала солнечными лучами,
    падающими под углом $36\degrees$ к горизонту?
}
\answer{%
    $\alpha = 36\degrees, (90\degrees - \beta + \alpha) + (90\degrees - \beta) = 90\degrees \implies \beta = 45\degrees + \frac \alpha 2 = 63\degrees$
}
\solutionspace{80pt}

\tasknumber{2}%
\task{%
    Во сколько раз увеличится расстояние между предметом и его изображением
    в плоском зеркале, если зеркало переместить в то место, где было изображение? Предмет остаётся неподвижным.
}
\answer{%
    $2$
}
\solutionspace{80pt}

\tasknumber{3}%
\task{%
    Плоское зеркало движется по направлению к точечному источнику света со скоростью $18\,\frac{\text{см}}{\text{с}}$.
    Определите скорость движения изображения относительно зеркала.
    Направление скорости зеркала перпендикулярно плоскости зеркала.
}
\answer{%
    $18\,\frac{\text{см}}{\text{с}}$
}
\solutionspace{80pt}

\tasknumber{4}%
\task{%
    Луч падает из воздуха на стекло с показателем преломления  1{,}35 .
    Сделайте рисунок (без рисунка и отмеченных углов задача не проверяется) и определите:
    \begin{itemize}
        \item угол отражения,
        \item угол преломления,
        \item угол между падающим и отраженным лучом,
        \item угол между падающим и преломленным лучом,
        \item угол отклонения луча при преломлении,
    \end{itemize}
    если между падающим лучом и границей раздела сред равен $55\degrees$.
}
\answer{%
    \begin{align*}
    \alpha &= 35\degrees, \\
    1 \cdot \sin \alpha &= n \sin \beta \implies \beta = \arcsin\cbr{ \frac{\sin \alpha}{ n } } \approx 25{,}14\degrees, \\
    \varphi_1 &= \alpha \approx 35\degrees, \\
    \varphi_2 &= \beta \approx 25{,}14\degrees, \\
    \varphi_3 &= 2\alpha = 70\degrees, \\
    \varphi_4 &= 180\degrees - \alpha + \beta \approx 170{,}14\degrees, \\
    \varphi_5 &= \alpha - \beta \approx 9{,}86\degrees.
    \end{align*}
}
\solutionspace{100pt}

\tasknumber{5}%
\task{%
    На дне водоёма глубиной $3\,\text{м}$ лежит зеркало.
    Луч света, пройдя через воду, отражается от зеркала и выходит из воды.
    Найти расстояние между точкой входа луча в воду и точкой выхода луча из воды,
    если показатель преломления воды $1{,}33$, а угол падения луча $30\degrees$.
}
\answer{%
    \begin{align*}
    \ctg \beta &= \frac{h}{d:L:s} \implies d = \frac{h}{\ctg \beta} \\
    \frac 1{\sin^2 \beta} &= \ctg^2 \beta + 1 \implies \ctg \beta = \sqrt{\frac 1{\sin^2 \beta} - 1} \\
    \sin\alpha &= n\sin \beta \implies \sin \beta = \frac{\sin\alpha}{n} \\
    d &= \frac{h}{\sqrt{\frac 1{\sin^2 \beta} - 1}} = \frac{h}{\sqrt{\sqr{\frac{n}{\sin\alpha}} - 1}} \\
    2d &= \frac{2{h}}{\sqrt{\sqr{\frac{n}{\sin\alpha}} - 1}} \approx 243{,}4\,\text{см}
    \end{align*}
}

\variantsplitter

\addpersonalvariant{Анастасия Князева}

\tasknumber{1}%
\task{%
    Под каким углом (в градусах) к горизонту следует расположить плоское зеркало,
    чтобы осветить дно вертикального колодца отраженными от зеркала солнечными лучами,
    падающими под углом $58\degrees$ к вертикали?
}
\answer{%
    $\alpha = 32\degrees, (90\degrees - \beta + \alpha) + (90\degrees - \beta) = 90\degrees \implies \beta = 45\degrees + \frac \alpha 2 = 61\degrees$
}
\solutionspace{80pt}

\tasknumber{2}%
\task{%
    Во сколько раз увеличится расстояние между предметом и его изображением
    в плоском зеркале, если зеркало переместить в то место, где было изображение? Предмет остаётся неподвижным.
}
\answer{%
    $2$
}
\solutionspace{80pt}

\tasknumber{3}%
\task{%
    Плоское зеркало движется по направлению к точечному источнику света со скоростью $10\,\frac{\text{см}}{\text{с}}$.
    Определите скорость движения изображения относительно источника света.
    Направление скорости зеркала перпендикулярно плоскости зеркала.
}
\answer{%
    $20\,\frac{\text{см}}{\text{с}}$
}
\solutionspace{80pt}

\tasknumber{4}%
\task{%
    Луч падает из вакуума на стекло с показателем преломления  1{,}65 .
    Сделайте рисунок (без рисунка и отмеченных углов задача не проверяется) и определите:
    \begin{itemize}
        \item угол отражения,
        \item угол преломления,
        \item угол между падающим и отраженным лучом,
        \item угол между падающим и преломленным лучом,
        \item угол отклонения луча при преломлении,
    \end{itemize}
    если между падающим лучом и границей раздела сред равен $55\degrees$.
}
\answer{%
    \begin{align*}
    \alpha &= 35\degrees, \\
    1 \cdot \sin \alpha &= n \sin \beta \implies \beta = \arcsin\cbr{ \frac{\sin \alpha}{ n } } \approx 20{,}34\degrees, \\
    \varphi_1 &= \alpha \approx 35\degrees, \\
    \varphi_2 &= \beta \approx 20{,}34\degrees, \\
    \varphi_3 &= 2\alpha = 70\degrees, \\
    \varphi_4 &= 180\degrees - \alpha + \beta \approx 165{,}34\degrees, \\
    \varphi_5 &= \alpha - \beta \approx 14{,}66\degrees.
    \end{align*}
}
\solutionspace{100pt}

\tasknumber{5}%
\task{%
    На дне водоёма глубиной $4\,\text{м}$ лежит зеркало.
    Луч света, пройдя через воду, отражается от зеркала и выходит из воды.
    Найти расстояние между точкой входа луча в воду и точкой выхода луча из воды,
    если показатель преломления воды $1{,}33$, а угол падения луча $25\degrees$.
}
\answer{%
    \begin{align*}
    \ctg \beta &= \frac{h}{d:L:s} \implies d = \frac{h}{\ctg \beta} \\
    \frac 1{\sin^2 \beta} &= \ctg^2 \beta + 1 \implies \ctg \beta = \sqrt{\frac 1{\sin^2 \beta} - 1} \\
    \sin\alpha &= n\sin \beta \implies \sin \beta = \frac{\sin\alpha}{n} \\
    d &= \frac{h}{\sqrt{\frac 1{\sin^2 \beta} - 1}} = \frac{h}{\sqrt{\sqr{\frac{n}{\sin\alpha}} - 1}} \\
    2d &= \frac{2{h}}{\sqrt{\sqr{\frac{n}{\sin\alpha}} - 1}} \approx 268{,}1\,\text{см}
    \end{align*}
}

\variantsplitter

\addpersonalvariant{Елизавета Кутумова}

\tasknumber{1}%
\task{%
    Под каким углом (в градусах) к горизонту следует расположить плоское зеркало,
    чтобы осветить дно вертикального колодца отраженными от зеркала солнечными лучами,
    падающими под углом $42\degrees$ к горизонту?
}
\answer{%
    $\alpha = 42\degrees, (90\degrees - \beta + \alpha) + (90\degrees - \beta) = 90\degrees \implies \beta = 45\degrees + \frac \alpha 2 = 66\degrees$
}
\solutionspace{80pt}

\tasknumber{2}%
\task{%
    Во сколько раз увеличится расстояние между предметом и его изображением
    в плоском зеркале, если зеркало переместить в то место, где было изображение? Предмет остаётся неподвижным.
}
\answer{%
    $2$
}
\solutionspace{80pt}

\tasknumber{3}%
\task{%
    Плоское зеркало движется по направлению к точечному источнику света со скоростью $15\,\frac{\text{см}}{\text{с}}$.
    Определите скорость движения изображения относительно источника света.
    Направление скорости зеркала перпендикулярно плоскости зеркала.
}
\answer{%
    $30\,\frac{\text{см}}{\text{с}}$
}
\solutionspace{80pt}

\tasknumber{4}%
\task{%
    Луч падает из воздуха на стекло с показателем преломления  1{,}55 .
    Сделайте рисунок (без рисунка и отмеченных углов задача не проверяется) и определите:
    \begin{itemize}
        \item угол отражения,
        \item угол преломления,
        \item угол между падающим и отраженным лучом,
        \item угол между падающим и преломленным лучом,
        \item угол отклонения луча при преломлении,
    \end{itemize}
    если между падающим лучом и границей раздела сред равен $65\degrees$.
}
\answer{%
    \begin{align*}
    \alpha &= 25\degrees, \\
    1 \cdot \sin \alpha &= n \sin \beta \implies \beta = \arcsin\cbr{ \frac{\sin \alpha}{ n } } \approx 15{,}82\degrees, \\
    \varphi_1 &= \alpha \approx 25\degrees, \\
    \varphi_2 &= \beta \approx 15{,}82\degrees, \\
    \varphi_3 &= 2\alpha = 50\degrees, \\
    \varphi_4 &= 180\degrees - \alpha + \beta \approx 170{,}82\degrees, \\
    \varphi_5 &= \alpha - \beta \approx 9{,}18\degrees.
    \end{align*}
}
\solutionspace{100pt}

\tasknumber{5}%
\task{%
    На дне водоёма глубиной $2\,\text{м}$ лежит зеркало.
    Луч света, пройдя через воду, отражается от зеркала и выходит из воды.
    Найти расстояние между точкой входа луча в воду и точкой выхода луча из воды,
    если показатель преломления воды $1{,}33$, а угол падения луча $25\degrees$.
}
\answer{%
    \begin{align*}
    \ctg \beta &= \frac{h}{d:L:s} \implies d = \frac{h}{\ctg \beta} \\
    \frac 1{\sin^2 \beta} &= \ctg^2 \beta + 1 \implies \ctg \beta = \sqrt{\frac 1{\sin^2 \beta} - 1} \\
    \sin\alpha &= n\sin \beta \implies \sin \beta = \frac{\sin\alpha}{n} \\
    d &= \frac{h}{\sqrt{\frac 1{\sin^2 \beta} - 1}} = \frac{h}{\sqrt{\sqr{\frac{n}{\sin\alpha}} - 1}} \\
    2d &= \frac{2{h}}{\sqrt{\sqr{\frac{n}{\sin\alpha}} - 1}} \approx 134{,}1\,\text{см}
    \end{align*}
}

\variantsplitter

\addpersonalvariant{Роксана Мехтиева}

\tasknumber{1}%
\task{%
    Под каким углом (в градусах) к горизонту следует расположить плоское зеркало,
    чтобы осветить дно вертикального колодца отраженными от зеркала солнечными лучами,
    падающими под углом $62\degrees$ к горизонту?
}
\answer{%
    $\alpha = 62\degrees, (90\degrees - \beta + \alpha) + (90\degrees - \beta) = 90\degrees \implies \beta = 45\degrees + \frac \alpha 2 = 76\degrees$
}
\solutionspace{80pt}

\tasknumber{2}%
\task{%
    Во сколько раз увеличится расстояние между предметом и его изображением
    в плоском зеркале, если зеркало переместить в то место, где было изображение? Предмет остаётся неподвижным.
}
\answer{%
    $2$
}
\solutionspace{80pt}

\tasknumber{3}%
\task{%
    Плоское зеркало движется по направлению к точечному источнику света со скоростью $12\,\frac{\text{см}}{\text{с}}$.
    Определите скорость движения изображения относительно источника света.
    Направление скорости зеркала перпендикулярно плоскости зеркала.
}
\answer{%
    $24\,\frac{\text{см}}{\text{с}}$
}
\solutionspace{80pt}

\tasknumber{4}%
\task{%
    Луч падает из воздуха на стекло с показателем преломления  1{,}45 .
    Сделайте рисунок (без рисунка и отмеченных углов задача не проверяется) и определите:
    \begin{itemize}
        \item угол отражения,
        \item угол преломления,
        \item угол между падающим и отраженным лучом,
        \item угол между падающим и преломленным лучом,
        \item угол отклонения луча при преломлении,
    \end{itemize}
    если между падающим лучом и границей раздела сред равен $28\degrees$.
}
\answer{%
    \begin{align*}
    \alpha &= 62\degrees, \\
    1 \cdot \sin \alpha &= n \sin \beta \implies \beta = \arcsin\cbr{ \frac{\sin \alpha}{ n } } \approx 37{,}51\degrees, \\
    \varphi_1 &= \alpha \approx 62\degrees, \\
    \varphi_2 &= \beta \approx 37{,}51\degrees, \\
    \varphi_3 &= 2\alpha = 124\degrees, \\
    \varphi_4 &= 180\degrees - \alpha + \beta \approx 155{,}51\degrees, \\
    \varphi_5 &= \alpha - \beta \approx 24{,}49\degrees.
    \end{align*}
}
\solutionspace{100pt}

\tasknumber{5}%
\task{%
    На дне водоёма глубиной $3\,\text{м}$ лежит зеркало.
    Луч света, пройдя через воду, отражается от зеркала и выходит из воды.
    Найти расстояние между точкой входа луча в воду и точкой выхода луча из воды,
    если показатель преломления воды $1{,}33$, а угол падения луча $30\degrees$.
}
\answer{%
    \begin{align*}
    \ctg \beta &= \frac{h}{d:L:s} \implies d = \frac{h}{\ctg \beta} \\
    \frac 1{\sin^2 \beta} &= \ctg^2 \beta + 1 \implies \ctg \beta = \sqrt{\frac 1{\sin^2 \beta} - 1} \\
    \sin\alpha &= n\sin \beta \implies \sin \beta = \frac{\sin\alpha}{n} \\
    d &= \frac{h}{\sqrt{\frac 1{\sin^2 \beta} - 1}} = \frac{h}{\sqrt{\sqr{\frac{n}{\sin\alpha}} - 1}} \\
    2d &= \frac{2{h}}{\sqrt{\sqr{\frac{n}{\sin\alpha}} - 1}} \approx 243{,}4\,\text{см}
    \end{align*}
}

\variantsplitter

\addpersonalvariant{Дилноза Нодиршоева}

\tasknumber{1}%
\task{%
    Под каким углом (в градусах) к горизонту следует расположить плоское зеркало,
    чтобы осветить дно вертикального колодца отраженными от зеркала солнечными лучами,
    падающими под углом $24\degrees$ к вертикали?
}
\answer{%
    $\alpha = 66\degrees, (90\degrees - \beta + \alpha) + (90\degrees - \beta) = 90\degrees \implies \beta = 45\degrees + \frac \alpha 2 = 78\degrees$
}
\solutionspace{80pt}

\tasknumber{2}%
\task{%
    Во сколько раз увеличится расстояние между предметом и его изображением
    в плоском зеркале, если зеркало переместить в то место, где было изображение? Предмет остаётся неподвижным.
}
\answer{%
    $2$
}
\solutionspace{80pt}

\tasknumber{3}%
\task{%
    Плоское зеркало движется по направлению к точечному источнику света со скоростью $15\,\frac{\text{см}}{\text{с}}$.
    Определите скорость движения изображения относительно источника света.
    Направление скорости зеркала перпендикулярно плоскости зеркала.
}
\answer{%
    $30\,\frac{\text{см}}{\text{с}}$
}
\solutionspace{80pt}

\tasknumber{4}%
\task{%
    Луч падает из вакуума на стекло с показателем преломления  1{,}45 .
    Сделайте рисунок (без рисунка и отмеченных углов задача не проверяется) и определите:
    \begin{itemize}
        \item угол отражения,
        \item угол преломления,
        \item угол между падающим и отраженным лучом,
        \item угол между падающим и преломленным лучом,
        \item угол отклонения луча при преломлении,
    \end{itemize}
    если между падающим лучом и границей раздела сред равен $50\degrees$.
}
\answer{%
    \begin{align*}
    \alpha &= 40\degrees, \\
    1 \cdot \sin \alpha &= n \sin \beta \implies \beta = \arcsin\cbr{ \frac{\sin \alpha}{ n } } \approx 26{,}31\degrees, \\
    \varphi_1 &= \alpha \approx 40\degrees, \\
    \varphi_2 &= \beta \approx 26{,}31\degrees, \\
    \varphi_3 &= 2\alpha = 80\degrees, \\
    \varphi_4 &= 180\degrees - \alpha + \beta \approx 166{,}31\degrees, \\
    \varphi_5 &= \alpha - \beta \approx 13{,}69\degrees.
    \end{align*}
}
\solutionspace{100pt}

\tasknumber{5}%
\task{%
    На дне водоёма глубиной $3\,\text{м}$ лежит зеркало.
    Луч света, пройдя через воду, отражается от зеркала и выходит из воды.
    Найти расстояние между точкой входа луча в воду и точкой выхода луча из воды,
    если показатель преломления воды $1{,}33$, а угол падения луча $25\degrees$.
}
\answer{%
    \begin{align*}
    \ctg \beta &= \frac{h}{d:L:s} \implies d = \frac{h}{\ctg \beta} \\
    \frac 1{\sin^2 \beta} &= \ctg^2 \beta + 1 \implies \ctg \beta = \sqrt{\frac 1{\sin^2 \beta} - 1} \\
    \sin\alpha &= n\sin \beta \implies \sin \beta = \frac{\sin\alpha}{n} \\
    d &= \frac{h}{\sqrt{\frac 1{\sin^2 \beta} - 1}} = \frac{h}{\sqrt{\sqr{\frac{n}{\sin\alpha}} - 1}} \\
    2d &= \frac{2{h}}{\sqrt{\sqr{\frac{n}{\sin\alpha}} - 1}} \approx 201{,}1\,\text{см}
    \end{align*}
}

\variantsplitter

\addpersonalvariant{Жаклин Пантелеева}

\tasknumber{1}%
\task{%
    Под каким углом (в градусах) к горизонту следует расположить плоское зеркало,
    чтобы осветить дно вертикального колодца отраженными от зеркала солнечными лучами,
    падающими под углом $56\degrees$ к горизонту?
}
\answer{%
    $\alpha = 56\degrees, (90\degrees - \beta + \alpha) + (90\degrees - \beta) = 90\degrees \implies \beta = 45\degrees + \frac \alpha 2 = 73\degrees$
}
\solutionspace{80pt}

\tasknumber{2}%
\task{%
    Во сколько раз увеличится расстояние между предметом и его изображением
    в плоском зеркале, если зеркало переместить в то место, где было изображение? Предмет остаётся неподвижным.
}
\answer{%
    $2$
}
\solutionspace{80pt}

\tasknumber{3}%
\task{%
    Плоское зеркало движется по направлению к точечному источнику света со скоростью $12\,\frac{\text{см}}{\text{с}}$.
    Определите скорость движения изображения относительно источника света.
    Направление скорости зеркала перпендикулярно плоскости зеркала.
}
\answer{%
    $24\,\frac{\text{см}}{\text{с}}$
}
\solutionspace{80pt}

\tasknumber{4}%
\task{%
    Луч падает из вакуума на стекло с показателем преломления  1{,}45 .
    Сделайте рисунок (без рисунка и отмеченных углов задача не проверяется) и определите:
    \begin{itemize}
        \item угол отражения,
        \item угол преломления,
        \item угол между падающим и отраженным лучом,
        \item угол между падающим и преломленным лучом,
        \item угол отклонения луча при преломлении,
    \end{itemize}
    если угол падения равен $55\degrees$.
}
\answer{%
    \begin{align*}
    \alpha &= 55\degrees, \\
    1 \cdot \sin \alpha &= n \sin \beta \implies \beta = \arcsin\cbr{ \frac{\sin \alpha}{ n } } \approx 34{,}40\degrees, \\
    \varphi_1 &= \alpha \approx 55\degrees, \\
    \varphi_2 &= \beta \approx 34{,}40\degrees, \\
    \varphi_3 &= 2\alpha = 110\degrees, \\
    \varphi_4 &= 180\degrees - \alpha + \beta \approx 159{,}40\degrees, \\
    \varphi_5 &= \alpha - \beta \approx 20{,}60\degrees.
    \end{align*}
}
\solutionspace{100pt}

\tasknumber{5}%
\task{%
    На дне водоёма глубиной $2\,\text{м}$ лежит зеркало.
    Луч света, пройдя через воду, отражается от зеркала и выходит из воды.
    Найти расстояние между точкой входа луча в воду и точкой выхода луча из воды,
    если показатель преломления воды $1{,}33$, а угол падения луча $35\degrees$.
}
\answer{%
    \begin{align*}
    \ctg \beta &= \frac{h}{d:L:s} \implies d = \frac{h}{\ctg \beta} \\
    \frac 1{\sin^2 \beta} &= \ctg^2 \beta + 1 \implies \ctg \beta = \sqrt{\frac 1{\sin^2 \beta} - 1} \\
    \sin\alpha &= n\sin \beta \implies \sin \beta = \frac{\sin\alpha}{n} \\
    d &= \frac{h}{\sqrt{\frac 1{\sin^2 \beta} - 1}} = \frac{h}{\sqrt{\sqr{\frac{n}{\sin\alpha}} - 1}} \\
    2d &= \frac{2{h}}{\sqrt{\sqr{\frac{n}{\sin\alpha}} - 1}} \approx 191{,}2\,\text{см}
    \end{align*}
}

\variantsplitter

\addpersonalvariant{Артём Переверзев}

\tasknumber{1}%
\task{%
    Под каким углом (в градусах) к горизонту следует расположить плоское зеркало,
    чтобы осветить дно вертикального колодца отраженными от зеркала солнечными лучами,
    падающими под углом $68\degrees$ к горизонту?
}
\answer{%
    $\alpha = 68\degrees, (90\degrees - \beta + \alpha) + (90\degrees - \beta) = 90\degrees \implies \beta = 45\degrees + \frac \alpha 2 = 79\degrees$
}
\solutionspace{80pt}

\tasknumber{2}%
\task{%
    Во сколько раз увеличится расстояние между предметом и его изображением
    в плоском зеркале, если зеркало переместить в то место, где было изображение? Предмет остаётся неподвижным.
}
\answer{%
    $2$
}
\solutionspace{80pt}

\tasknumber{3}%
\task{%
    Плоское зеркало движется по направлению к точечному источнику света со скоростью $15\,\frac{\text{см}}{\text{с}}$.
    Определите скорость движения изображения относительно зеркала.
    Направление скорости зеркала перпендикулярно плоскости зеркала.
}
\answer{%
    $15\,\frac{\text{см}}{\text{с}}$
}
\solutionspace{80pt}

\tasknumber{4}%
\task{%
    Луч падает из вакуума на стекло с показателем преломления  1{,}35 .
    Сделайте рисунок (без рисунка и отмеченных углов задача не проверяется) и определите:
    \begin{itemize}
        \item угол отражения,
        \item угол преломления,
        \item угол между падающим и отраженным лучом,
        \item угол между падающим и преломленным лучом,
        \item угол отклонения луча при преломлении,
    \end{itemize}
    если между падающим лучом и границей раздела сред равен $50\degrees$.
}
\answer{%
    \begin{align*}
    \alpha &= 40\degrees, \\
    1 \cdot \sin \alpha &= n \sin \beta \implies \beta = \arcsin\cbr{ \frac{\sin \alpha}{ n } } \approx 28{,}43\degrees, \\
    \varphi_1 &= \alpha \approx 40\degrees, \\
    \varphi_2 &= \beta \approx 28{,}43\degrees, \\
    \varphi_3 &= 2\alpha = 80\degrees, \\
    \varphi_4 &= 180\degrees - \alpha + \beta \approx 168{,}43\degrees, \\
    \varphi_5 &= \alpha - \beta \approx 11{,}57\degrees.
    \end{align*}
}
\solutionspace{100pt}

\tasknumber{5}%
\task{%
    На дне водоёма глубиной $2\,\text{м}$ лежит зеркало.
    Луч света, пройдя через воду, отражается от зеркала и выходит из воды.
    Найти расстояние между точкой входа луча в воду и точкой выхода луча из воды,
    если показатель преломления воды $1{,}33$, а угол падения луча $25\degrees$.
}
\answer{%
    \begin{align*}
    \ctg \beta &= \frac{h}{d:L:s} \implies d = \frac{h}{\ctg \beta} \\
    \frac 1{\sin^2 \beta} &= \ctg^2 \beta + 1 \implies \ctg \beta = \sqrt{\frac 1{\sin^2 \beta} - 1} \\
    \sin\alpha &= n\sin \beta \implies \sin \beta = \frac{\sin\alpha}{n} \\
    d &= \frac{h}{\sqrt{\frac 1{\sin^2 \beta} - 1}} = \frac{h}{\sqrt{\sqr{\frac{n}{\sin\alpha}} - 1}} \\
    2d &= \frac{2{h}}{\sqrt{\sqr{\frac{n}{\sin\alpha}} - 1}} \approx 134{,}1\,\text{см}
    \end{align*}
}

\variantsplitter

\addpersonalvariant{Варвара Пранова}

\tasknumber{1}%
\task{%
    Под каким углом (в градусах) к горизонту следует расположить плоское зеркало,
    чтобы осветить дно вертикального колодца отраженными от зеркала солнечными лучами,
    падающими под углом $64\degrees$ к горизонту?
}
\answer{%
    $\alpha = 64\degrees, (90\degrees - \beta + \alpha) + (90\degrees - \beta) = 90\degrees \implies \beta = 45\degrees + \frac \alpha 2 = 77\degrees$
}
\solutionspace{80pt}

\tasknumber{2}%
\task{%
    Во сколько раз увеличится расстояние между предметом и его изображением
    в плоском зеркале, если зеркало переместить в то место, где было изображение? Предмет остаётся неподвижным.
}
\answer{%
    $2$
}
\solutionspace{80pt}

\tasknumber{3}%
\task{%
    Плоское зеркало движется по направлению к точечному источнику света со скоростью $20\,\frac{\text{см}}{\text{с}}$.
    Определите скорость движения изображения относительно источника света.
    Направление скорости зеркала перпендикулярно плоскости зеркала.
}
\answer{%
    $40\,\frac{\text{см}}{\text{с}}$
}
\solutionspace{80pt}

\tasknumber{4}%
\task{%
    Луч падает из воздуха на стекло с показателем преломления  1{,}35 .
    Сделайте рисунок (без рисунка и отмеченных углов задача не проверяется) и определите:
    \begin{itemize}
        \item угол отражения,
        \item угол преломления,
        \item угол между падающим и отраженным лучом,
        \item угол между падающим и преломленным лучом,
        \item угол отклонения луча при преломлении,
    \end{itemize}
    если между падающим лучом и границей раздела сред равен $40\degrees$.
}
\answer{%
    \begin{align*}
    \alpha &= 50\degrees, \\
    1 \cdot \sin \alpha &= n \sin \beta \implies \beta = \arcsin\cbr{ \frac{\sin \alpha}{ n } } \approx 34{,}57\degrees, \\
    \varphi_1 &= \alpha \approx 50\degrees, \\
    \varphi_2 &= \beta \approx 34{,}57\degrees, \\
    \varphi_3 &= 2\alpha = 100\degrees, \\
    \varphi_4 &= 180\degrees - \alpha + \beta \approx 164{,}57\degrees, \\
    \varphi_5 &= \alpha - \beta \approx 15{,}43\degrees.
    \end{align*}
}
\solutionspace{100pt}

\tasknumber{5}%
\task{%
    На дне водоёма глубиной $3\,\text{м}$ лежит зеркало.
    Луч света, пройдя через воду, отражается от зеркала и выходит из воды.
    Найти расстояние между точкой входа луча в воду и точкой выхода луча из воды,
    если показатель преломления воды $1{,}33$, а угол падения луча $25\degrees$.
}
\answer{%
    \begin{align*}
    \ctg \beta &= \frac{h}{d:L:s} \implies d = \frac{h}{\ctg \beta} \\
    \frac 1{\sin^2 \beta} &= \ctg^2 \beta + 1 \implies \ctg \beta = \sqrt{\frac 1{\sin^2 \beta} - 1} \\
    \sin\alpha &= n\sin \beta \implies \sin \beta = \frac{\sin\alpha}{n} \\
    d &= \frac{h}{\sqrt{\frac 1{\sin^2 \beta} - 1}} = \frac{h}{\sqrt{\sqr{\frac{n}{\sin\alpha}} - 1}} \\
    2d &= \frac{2{h}}{\sqrt{\sqr{\frac{n}{\sin\alpha}} - 1}} \approx 201{,}1\,\text{см}
    \end{align*}
}

\variantsplitter

\addpersonalvariant{Марьям Салимова}

\tasknumber{1}%
\task{%
    Под каким углом (в градусах) к горизонту следует расположить плоское зеркало,
    чтобы осветить дно вертикального колодца отраженными от зеркала солнечными лучами,
    падающими под углом $54\degrees$ к горизонту?
}
\answer{%
    $\alpha = 54\degrees, (90\degrees - \beta + \alpha) + (90\degrees - \beta) = 90\degrees \implies \beta = 45\degrees + \frac \alpha 2 = 72\degrees$
}
\solutionspace{80pt}

\tasknumber{2}%
\task{%
    Во сколько раз увеличится расстояние между предметом и его изображением
    в плоском зеркале, если зеркало переместить в то место, где было изображение? Предмет остаётся неподвижным.
}
\answer{%
    $2$
}
\solutionspace{80pt}

\tasknumber{3}%
\task{%
    Плоское зеркало движется по направлению к точечному источнику света со скоростью $15\,\frac{\text{см}}{\text{с}}$.
    Определите скорость движения изображения относительно зеркала.
    Направление скорости зеркала перпендикулярно плоскости зеркала.
}
\answer{%
    $15\,\frac{\text{см}}{\text{с}}$
}
\solutionspace{80pt}

\tasknumber{4}%
\task{%
    Луч падает из вакуума на стекло с показателем преломления  1{,}35 .
    Сделайте рисунок (без рисунка и отмеченных углов задача не проверяется) и определите:
    \begin{itemize}
        \item угол отражения,
        \item угол преломления,
        \item угол между падающим и отраженным лучом,
        \item угол между падающим и преломленным лучом,
        \item угол отклонения луча при преломлении,
    \end{itemize}
    если угол падения равен $65\degrees$.
}
\answer{%
    \begin{align*}
    \alpha &= 65\degrees, \\
    1 \cdot \sin \alpha &= n \sin \beta \implies \beta = \arcsin\cbr{ \frac{\sin \alpha}{ n } } \approx 42{,}17\degrees, \\
    \varphi_1 &= \alpha \approx 65\degrees, \\
    \varphi_2 &= \beta \approx 42{,}17\degrees, \\
    \varphi_3 &= 2\alpha = 130\degrees, \\
    \varphi_4 &= 180\degrees - \alpha + \beta \approx 157{,}17\degrees, \\
    \varphi_5 &= \alpha - \beta \approx 22{,}83\degrees.
    \end{align*}
}
\solutionspace{100pt}

\tasknumber{5}%
\task{%
    На дне водоёма глубиной $2\,\text{м}$ лежит зеркало.
    Луч света, пройдя через воду, отражается от зеркала и выходит из воды.
    Найти расстояние между точкой входа луча в воду и точкой выхода луча из воды,
    если показатель преломления воды $1{,}33$, а угол падения луча $30\degrees$.
}
\answer{%
    \begin{align*}
    \ctg \beta &= \frac{h}{d:L:s} \implies d = \frac{h}{\ctg \beta} \\
    \frac 1{\sin^2 \beta} &= \ctg^2 \beta + 1 \implies \ctg \beta = \sqrt{\frac 1{\sin^2 \beta} - 1} \\
    \sin\alpha &= n\sin \beta \implies \sin \beta = \frac{\sin\alpha}{n} \\
    d &= \frac{h}{\sqrt{\frac 1{\sin^2 \beta} - 1}} = \frac{h}{\sqrt{\sqr{\frac{n}{\sin\alpha}} - 1}} \\
    2d &= \frac{2{h}}{\sqrt{\sqr{\frac{n}{\sin\alpha}} - 1}} \approx 162{,}3\,\text{см}
    \end{align*}
}

\variantsplitter

\addpersonalvariant{Юлия Шевченко}

\tasknumber{1}%
\task{%
    Под каким углом (в градусах) к горизонту следует расположить плоское зеркало,
    чтобы осветить дно вертикального колодца отраженными от зеркала солнечными лучами,
    падающими под углом $64\degrees$ к вертикали?
}
\answer{%
    $\alpha = 26\degrees, (90\degrees - \beta + \alpha) + (90\degrees - \beta) = 90\degrees \implies \beta = 45\degrees + \frac \alpha 2 = 58\degrees$
}
\solutionspace{80pt}

\tasknumber{2}%
\task{%
    Во сколько раз увеличится расстояние между предметом и его изображением
    в плоском зеркале, если зеркало переместить в то место, где было изображение? Предмет остаётся неподвижным.
}
\answer{%
    $2$
}
\solutionspace{80pt}

\tasknumber{3}%
\task{%
    Плоское зеркало движется по направлению к точечному источнику света со скоростью $20\,\frac{\text{см}}{\text{с}}$.
    Определите скорость движения изображения относительно источника света.
    Направление скорости зеркала перпендикулярно плоскости зеркала.
}
\answer{%
    $40\,\frac{\text{см}}{\text{с}}$
}
\solutionspace{80pt}

\tasknumber{4}%
\task{%
    Луч падает из воздуха на стекло с показателем преломления  1{,}55 .
    Сделайте рисунок (без рисунка и отмеченных углов задача не проверяется) и определите:
    \begin{itemize}
        \item угол отражения,
        \item угол преломления,
        \item угол между падающим и отраженным лучом,
        \item угол между падающим и преломленным лучом,
        \item угол отклонения луча при преломлении,
    \end{itemize}
    если между падающим лучом и границей раздела сред равен $22\degrees$.
}
\answer{%
    \begin{align*}
    \alpha &= 68\degrees, \\
    1 \cdot \sin \alpha &= n \sin \beta \implies \beta = \arcsin\cbr{ \frac{\sin \alpha}{ n } } \approx 36{,}74\degrees, \\
    \varphi_1 &= \alpha \approx 68\degrees, \\
    \varphi_2 &= \beta \approx 36{,}74\degrees, \\
    \varphi_3 &= 2\alpha = 136\degrees, \\
    \varphi_4 &= 180\degrees - \alpha + \beta \approx 148{,}74\degrees, \\
    \varphi_5 &= \alpha - \beta \approx 31{,}26\degrees.
    \end{align*}
}
\solutionspace{100pt}

\tasknumber{5}%
\task{%
    На дне водоёма глубиной $4\,\text{м}$ лежит зеркало.
    Луч света, пройдя через воду, отражается от зеркала и выходит из воды.
    Найти расстояние между точкой входа луча в воду и точкой выхода луча из воды,
    если показатель преломления воды $1{,}33$, а угол падения луча $35\degrees$.
}
\answer{%
    \begin{align*}
    \ctg \beta &= \frac{h}{d:L:s} \implies d = \frac{h}{\ctg \beta} \\
    \frac 1{\sin^2 \beta} &= \ctg^2 \beta + 1 \implies \ctg \beta = \sqrt{\frac 1{\sin^2 \beta} - 1} \\
    \sin\alpha &= n\sin \beta \implies \sin \beta = \frac{\sin\alpha}{n} \\
    d &= \frac{h}{\sqrt{\frac 1{\sin^2 \beta} - 1}} = \frac{h}{\sqrt{\sqr{\frac{n}{\sin\alpha}} - 1}} \\
    2d &= \frac{2{h}}{\sqrt{\sqr{\frac{n}{\sin\alpha}} - 1}} \approx 382{,}4\,\text{см}
    \end{align*}
}
% autogenerated
