\setdate{26~января~2022}
\setclass{11«Б»}

\addpersonalvariant{Михаил Бурмистров}

\tasknumber{1}%
\task{%
    Укажите, верны ли утверждения («да» или «нет» слева от каждого утверждения):
    \begin{itemize}
        \item  Изображение предмета в рассеивающей линзе всегда действительное.
        \item  Изображение предмета в рассеивающей линзе всегда перевёрнутое.
        \item  Изображение предмета в рассеивающей линзе всегда увеличенное.
        \item  Оптическая сила собирающей линзы положительна.
    \end{itemize}
}
\answer{%
    $\text{ нет, нет, нет, да }$
}

\tasknumber{2}%
\task{%
    Докажите формулу тонкой линзы для собирающей линзы.
}
\solutionspace{120pt}

\tasknumber{3}%
\task{%
    Найти оптическую силу собирающей линзы, если действительное изображение предмета,
    помещённого в $35\,\text{см}$ от линзы, получается на расстоянии $40\,\text{см}$ от неё.
}
\answer{%
    $D = \frac 1F = \frac 1a + \frac 1b = \frac 1{35\,\text{см}} + \frac 1{40\,\text{см}} \approx 5{,}36\,\text{дптр}$
}
\solutionspace{80pt}

\tasknumber{4}%
\task{%
    Найти увеличение изображения, если изображение предмета, находящегося
    на расстоянии $20\,\text{см}$ от линзы, получается на расстоянии $18\,\text{см}$ от неё.
}
\answer{%
    $\Gamma = \frac ba = \frac {18\,\text{см}}{20\,\text{см}} \approx 0{,}9$
}
\solutionspace{80pt}

\tasknumber{5}%
\task{%
    Расстояние от предмета до линзы $10\,\text{см}$, а от линзы до мнимого изображения $25\,\text{см}$.
    Чему равно фокусное расстояние линзы?
}
\answer{%
    $\pm \frac 1F = \frac 1a - \frac 1b \implies F = \frac{a b}{\abs{b - a}} \approx 16{,}7\,\text{см}$
}
\solutionspace{80pt}

\tasknumber{6}%
\task{%
    Две одинаковые собиращие линзы установлены так, что их главные оптические оси совпадают,
    а главный фокус первой находится там же, где главный фокус второй.
    Расстояние от первой линзы до предмета равно $7\,\text{см}$.
    Чему равно расстояние от изображения объекта во второй линзе до самого объекта?
    Определите также увеличение.
    Фокусное расстояние каждой линзы $25\,\text{см}$.
}
\answer{%
    \begin{align*}
    \frac 1a + \frac 1b &= \frac 1F \implies b = \frac{aF}{a - F} \implies 2F - b = \frac{2aF - 2F^2 - aF}{a - F} = \frac{F(a - 2F)}{a - F}.
    \\
    \frac 1{2F - b} + \frac 1c &= \frac 1F \implies c = \frac{F(2F-b)}{(2F - b) - F} = \frac{F \cdot \frac{F(a - 2F)}{a - F}}{\frac{F(a - 2F)}{a - F} - F}  = F \cdot \frac{ \frac{F(a - 2F)}{a - F} }{ \frac{F(a - 2F)}{a - F} - 1} = \\
     &= F \cdot \frac{a - 2F}{a - 2F - a + F} = 2F - a = 43\,\text{см}.
     \\
    \ell &= a + 2F + c = 4F = 100\,\text{см}.
    \\
    &\Gamma = \Gamma_1 \cdot \Gamma_2 = \frac ba \cdot \frac c{2F-b} = \frac F{a - F} \cdot \frac{2F - a}{\frac{F(a - 2F)}{a - F}} = -1.
    \end{align*}
}

\variantsplitter

\addpersonalvariant{Снежана Авдошина}

\tasknumber{1}%
\task{%
    Укажите, верны ли утверждения («да» или «нет» слева от каждого утверждения):
    \begin{itemize}
        \item  Изображение предмета в рассеивающей линзе всегда мнимое.
        \item  Изображение предмета в рассеивающей линзе всегда прямое.
        \item  Изображение предмета в рассеивающей линзе всегда увеличенное.
        \item  Оптическая сила собирающей линзы положительна.
    \end{itemize}
}
\answer{%
    $\text{ да, да, нет, да }$
}

\tasknumber{2}%
\task{%
    Докажите формулу тонкой линзы для рассеивающей линзы.
}
\solutionspace{120pt}

\tasknumber{3}%
\task{%
    Найти оптическую силу собирающей линзы, если действительное изображение предмета,
    помещённого в $15\,\text{см}$ от линзы, получается на расстоянии $40\,\text{см}$ от неё.
}
\answer{%
    $D = \frac 1F = \frac 1a + \frac 1b = \frac 1{15\,\text{см}} + \frac 1{40\,\text{см}} \approx 9{,}17\,\text{дптр}$
}
\solutionspace{80pt}

\tasknumber{4}%
\task{%
    Найти увеличение изображения, если изображение предмета, находящегося
    на расстоянии $25\,\text{см}$ от линзы, получается на расстоянии $30\,\text{см}$ от неё.
}
\answer{%
    $\Gamma = \frac ba = \frac {30\,\text{см}}{25\,\text{см}} \approx 1{,}20$
}
\solutionspace{80pt}

\tasknumber{5}%
\task{%
    Расстояние от предмета до линзы $12\,\text{см}$, а от линзы до мнимого изображения $30\,\text{см}$.
    Чему равно фокусное расстояние линзы?
}
\answer{%
    $\pm \frac 1F = \frac 1a - \frac 1b \implies F = \frac{a b}{\abs{b - a}} \approx 20\,\text{см}$
}
\solutionspace{80pt}

\tasknumber{6}%
\task{%
    Две одинаковые собиращие линзы установлены так, что их главные оптические оси совпадают,
    а главный фокус первой находится там же, где главный фокус второй.
    Расстояние от первой линзы до предмета равно $5\,\text{см}$.
    Чему равно расстояние от изображения объекта во второй линзе до самого объекта?
    Определите также увеличение.
    Фокусное расстояние каждой линзы $25\,\text{см}$.
}
\answer{%
    \begin{align*}
    \frac 1a + \frac 1b &= \frac 1F \implies b = \frac{aF}{a - F} \implies 2F - b = \frac{2aF - 2F^2 - aF}{a - F} = \frac{F(a - 2F)}{a - F}.
    \\
    \frac 1{2F - b} + \frac 1c &= \frac 1F \implies c = \frac{F(2F-b)}{(2F - b) - F} = \frac{F \cdot \frac{F(a - 2F)}{a - F}}{\frac{F(a - 2F)}{a - F} - F}  = F \cdot \frac{ \frac{F(a - 2F)}{a - F} }{ \frac{F(a - 2F)}{a - F} - 1} = \\
     &= F \cdot \frac{a - 2F}{a - 2F - a + F} = 2F - a = 45\,\text{см}.
     \\
    \ell &= a + 2F + c = 4F = 100\,\text{см}.
    \\
    &\Gamma = \Gamma_1 \cdot \Gamma_2 = \frac ba \cdot \frac c{2F-b} = \frac F{a - F} \cdot \frac{2F - a}{\frac{F(a - 2F)}{a - F}} = -1.
    \end{align*}
}

\variantsplitter

\addpersonalvariant{Марьяна Аристова}

\tasknumber{1}%
\task{%
    Укажите, верны ли утверждения («да» или «нет» слева от каждого утверждения):
    \begin{itemize}
        \item  Изображение предмета в собирающей линзе всегда мнимое.
        \item  Изображение предмета в собирающей линзе всегда перевёрнутое.
        \item  Изображение предмета в собирающей линзе всегда уменьшенное.
        \item  Оптическая сила рассеивающей линзы отрицательна.
    \end{itemize}
}
\answer{%
    $\text{ нет, нет, нет, да }$
}

\tasknumber{2}%
\task{%
    Докажите формулу тонкой линзы для собирающей линзы.
}
\solutionspace{120pt}

\tasknumber{3}%
\task{%
    Найти оптическую силу собирающей линзы, если действительное изображение предмета,
    помещённого в $35\,\text{см}$ от линзы, получается на расстоянии $40\,\text{см}$ от неё.
}
\answer{%
    $D = \frac 1F = \frac 1a + \frac 1b = \frac 1{35\,\text{см}} + \frac 1{40\,\text{см}} \approx 5{,}36\,\text{дптр}$
}
\solutionspace{80pt}

\tasknumber{4}%
\task{%
    Найти увеличение изображения, если изображение предмета, находящегося
    на расстоянии $20\,\text{см}$ от линзы, получается на расстоянии $12\,\text{см}$ от неё.
}
\answer{%
    $\Gamma = \frac ba = \frac {12\,\text{см}}{20\,\text{см}} \approx 0{,}6$
}
\solutionspace{80pt}

\tasknumber{5}%
\task{%
    Расстояние от предмета до линзы $10\,\text{см}$, а от линзы до мнимого изображения $20\,\text{см}$.
    Чему равно фокусное расстояние линзы?
}
\answer{%
    $\pm \frac 1F = \frac 1a - \frac 1b \implies F = \frac{a b}{\abs{b - a}} \approx 20\,\text{см}$
}
\solutionspace{80pt}

\tasknumber{6}%
\task{%
    Две одинаковые собиращие линзы установлены так, что их главные оптические оси совпадают,
    а главный фокус первой находится там же, где главный фокус второй.
    Расстояние от первой линзы до предмета равно $11\,\text{см}$.
    Чему равно расстояние от изображения объекта во второй линзе до второй линзы?
    Определите также увеличение.
    Фокусное расстояние каждой линзы $40\,\text{см}$.
}
\answer{%
    \begin{align*}
    \frac 1a + \frac 1b &= \frac 1F \implies b = \frac{aF}{a - F} \implies 2F - b = \frac{2aF - 2F^2 - aF}{a - F} = \frac{F(a - 2F)}{a - F}.
    \\
    \frac 1{2F - b} + \frac 1c &= \frac 1F \implies c = \frac{F(2F-b)}{(2F - b) - F} = \frac{F \cdot \frac{F(a - 2F)}{a - F}}{\frac{F(a - 2F)}{a - F} - F}  = F \cdot \frac{ \frac{F(a - 2F)}{a - F} }{ \frac{F(a - 2F)}{a - F} - 1} = \\
     &= F \cdot \frac{a - 2F}{a - 2F - a + F} = 2F - a = 69\,\text{см}.
     \\
    \ell &= a + 2F + c = 4F = 160\,\text{см}.
    \\
    &\Gamma = \Gamma_1 \cdot \Gamma_2 = \frac ba \cdot \frac c{2F-b} = \frac F{a - F} \cdot \frac{2F - a}{\frac{F(a - 2F)}{a - F}} = -1.
    \end{align*}
}

\variantsplitter

\addpersonalvariant{Никита Иванов}

\tasknumber{1}%
\task{%
    Укажите, верны ли утверждения («да» или «нет» слева от каждого утверждения):
    \begin{itemize}
        \item  Изображение предмета в собирающей линзе всегда действительное.
        \item  Изображение предмета в собирающей линзе всегда перевёрнутое.
        \item  Изображение предмета в собирающей линзе всегда уменьшенное.
        \item  Оптическая сила рассеивающей линзы положительна.
    \end{itemize}
}
\answer{%
    $\text{ нет, нет, нет, нет }$
}

\tasknumber{2}%
\task{%
    Докажите формулу тонкой линзы для рассеивающей линзы.
}
\solutionspace{120pt}

\tasknumber{3}%
\task{%
    Найти оптическую силу собирающей линзы, если действительное изображение предмета,
    помещённого в $35\,\text{см}$ от линзы, получается на расстоянии $20\,\text{см}$ от неё.
}
\answer{%
    $D = \frac 1F = \frac 1a + \frac 1b = \frac 1{35\,\text{см}} + \frac 1{20\,\text{см}} \approx 7{,}86\,\text{дптр}$
}
\solutionspace{80pt}

\tasknumber{4}%
\task{%
    Найти увеличение изображения, если изображение предмета, находящегося
    на расстоянии $25\,\text{см}$ от линзы, получается на расстоянии $18\,\text{см}$ от неё.
}
\answer{%
    $\Gamma = \frac ba = \frac {18\,\text{см}}{25\,\text{см}} \approx 0{,}7$
}
\solutionspace{80pt}

\tasknumber{5}%
\task{%
    Расстояние от предмета до линзы $12\,\text{см}$, а от линзы до мнимого изображения $25\,\text{см}$.
    Чему равно фокусное расстояние линзы?
}
\answer{%
    $\pm \frac 1F = \frac 1a - \frac 1b \implies F = \frac{a b}{\abs{b - a}} \approx 23{,}1\,\text{см}$
}
\solutionspace{80pt}

\tasknumber{6}%
\task{%
    Две одинаковые собиращие линзы установлены так, что их главные оптические оси совпадают,
    а главный фокус первой находится там же, где главный фокус второй.
    Расстояние от первой линзы до предмета равно $30\,\text{см}$.
    Чему равно расстояние от изображения объекта во второй линзе до второй линзы?
    Определите также увеличение.
    Фокусное расстояние каждой линзы $40\,\text{см}$.
}
\answer{%
    \begin{align*}
    \frac 1a + \frac 1b &= \frac 1F \implies b = \frac{aF}{a - F} \implies 2F - b = \frac{2aF - 2F^2 - aF}{a - F} = \frac{F(a - 2F)}{a - F}.
    \\
    \frac 1{2F - b} + \frac 1c &= \frac 1F \implies c = \frac{F(2F-b)}{(2F - b) - F} = \frac{F \cdot \frac{F(a - 2F)}{a - F}}{\frac{F(a - 2F)}{a - F} - F}  = F \cdot \frac{ \frac{F(a - 2F)}{a - F} }{ \frac{F(a - 2F)}{a - F} - 1} = \\
     &= F \cdot \frac{a - 2F}{a - 2F - a + F} = 2F - a = 50\,\text{см}.
     \\
    \ell &= a + 2F + c = 4F = 160\,\text{см}.
    \\
    &\Gamma = \Gamma_1 \cdot \Gamma_2 = \frac ba \cdot \frac c{2F-b} = \frac F{a - F} \cdot \frac{2F - a}{\frac{F(a - 2F)}{a - F}} = -1.
    \end{align*}
}

\variantsplitter

\addpersonalvariant{Анастасия Князева}

\tasknumber{1}%
\task{%
    Укажите, верны ли утверждения («да» или «нет» слева от каждого утверждения):
    \begin{itemize}
        \item  Изображение предмета в собирающей линзе всегда мнимое.
        \item  Изображение предмета в собирающей линзе всегда прямое.
        \item  Изображение предмета в собирающей линзе всегда уменьшенное.
        \item  Оптическая сила собирающей линзы положительна.
    \end{itemize}
}
\answer{%
    $\text{ нет, нет, нет, да }$
}

\tasknumber{2}%
\task{%
    Докажите формулу тонкой линзы для собирающей линзы.
}
\solutionspace{120pt}

\tasknumber{3}%
\task{%
    Найти оптическую силу собирающей линзы, если действительное изображение предмета,
    помещённого в $35\,\text{см}$ от линзы, получается на расстоянии $20\,\text{см}$ от неё.
}
\answer{%
    $D = \frac 1F = \frac 1a + \frac 1b = \frac 1{35\,\text{см}} + \frac 1{20\,\text{см}} \approx 7{,}86\,\text{дптр}$
}
\solutionspace{80pt}

\tasknumber{4}%
\task{%
    Найти увеличение изображения, если изображение предмета, находящегося
    на расстоянии $25\,\text{см}$ от линзы, получается на расстоянии $12\,\text{см}$ от неё.
}
\answer{%
    $\Gamma = \frac ba = \frac {12\,\text{см}}{25\,\text{см}} \approx 0{,}5$
}
\solutionspace{80pt}

\tasknumber{5}%
\task{%
    Расстояние от предмета до линзы $12\,\text{см}$, а от линзы до мнимого изображения $20\,\text{см}$.
    Чему равно фокусное расстояние линзы?
}
\answer{%
    $\pm \frac 1F = \frac 1a - \frac 1b \implies F = \frac{a b}{\abs{b - a}} \approx 30\,\text{см}$
}
\solutionspace{80pt}

\tasknumber{6}%
\task{%
    Две одинаковые собиращие линзы установлены так, что их главные оптические оси совпадают,
    а главный фокус первой находится там же, где главный фокус второй.
    Расстояние от первой линзы до предмета равно $10\,\text{см}$.
    Чему равно расстояние от изображения объекта во второй линзе до второй линзы?
    Определите также увеличение.
    Фокусное расстояние каждой линзы $40\,\text{см}$.
}
\answer{%
    \begin{align*}
    \frac 1a + \frac 1b &= \frac 1F \implies b = \frac{aF}{a - F} \implies 2F - b = \frac{2aF - 2F^2 - aF}{a - F} = \frac{F(a - 2F)}{a - F}.
    \\
    \frac 1{2F - b} + \frac 1c &= \frac 1F \implies c = \frac{F(2F-b)}{(2F - b) - F} = \frac{F \cdot \frac{F(a - 2F)}{a - F}}{\frac{F(a - 2F)}{a - F} - F}  = F \cdot \frac{ \frac{F(a - 2F)}{a - F} }{ \frac{F(a - 2F)}{a - F} - 1} = \\
     &= F \cdot \frac{a - 2F}{a - 2F - a + F} = 2F - a = 70\,\text{см}.
     \\
    \ell &= a + 2F + c = 4F = 160\,\text{см}.
    \\
    &\Gamma = \Gamma_1 \cdot \Gamma_2 = \frac ba \cdot \frac c{2F-b} = \frac F{a - F} \cdot \frac{2F - a}{\frac{F(a - 2F)}{a - F}} = -1.
    \end{align*}
}

\variantsplitter

\addpersonalvariant{Елизавета Кутумова}

\tasknumber{1}%
\task{%
    Укажите, верны ли утверждения («да» или «нет» слева от каждого утверждения):
    \begin{itemize}
        \item  Изображение предмета в собирающей линзе всегда действительное.
        \item  Изображение предмета в собирающей линзе всегда прямое.
        \item  Изображение предмета в собирающей линзе всегда увеличенное.
        \item  Оптическая сила рассеивающей линзы положительна.
    \end{itemize}
}
\answer{%
    $\text{ нет, нет, нет, нет }$
}

\tasknumber{2}%
\task{%
    Докажите формулу тонкой линзы для собирающей линзы.
}
\solutionspace{120pt}

\tasknumber{3}%
\task{%
    Найти оптическую силу собирающей линзы, если действительное изображение предмета,
    помещённого в $35\,\text{см}$ от линзы, получается на расстоянии $40\,\text{см}$ от неё.
}
\answer{%
    $D = \frac 1F = \frac 1a + \frac 1b = \frac 1{35\,\text{см}} + \frac 1{40\,\text{см}} \approx 5{,}36\,\text{дптр}$
}
\solutionspace{80pt}

\tasknumber{4}%
\task{%
    Найти увеличение изображения, если изображение предмета, находящегося
    на расстоянии $15\,\text{см}$ от линзы, получается на расстоянии $18\,\text{см}$ от неё.
}
\answer{%
    $\Gamma = \frac ba = \frac {18\,\text{см}}{15\,\text{см}} \approx 1{,}2$
}
\solutionspace{80pt}

\tasknumber{5}%
\task{%
    Расстояние от предмета до линзы $8\,\text{см}$, а от линзы до мнимого изображения $25\,\text{см}$.
    Чему равно фокусное расстояние линзы?
}
\answer{%
    $\pm \frac 1F = \frac 1a - \frac 1b \implies F = \frac{a b}{\abs{b - a}} \approx 11{,}8\,\text{см}$
}
\solutionspace{80pt}

\tasknumber{6}%
\task{%
    Две одинаковые собиращие линзы установлены так, что их главные оптические оси совпадают,
    а главный фокус первой находится там же, где главный фокус второй.
    Расстояние от первой линзы до предмета равно $30\,\text{см}$.
    Чему равно расстояние от изображения объекта во второй линзе до самого объекта?
    Определите также увеличение.
    Фокусное расстояние каждой линзы $20\,\text{см}$.
}
\answer{%
    \begin{align*}
    \frac 1a + \frac 1b &= \frac 1F \implies b = \frac{aF}{a - F} \implies 2F - b = \frac{2aF - 2F^2 - aF}{a - F} = \frac{F(a - 2F)}{a - F}.
    \\
    \frac 1{2F - b} + \frac 1c &= \frac 1F \implies c = \frac{F(2F-b)}{(2F - b) - F} = \frac{F \cdot \frac{F(a - 2F)}{a - F}}{\frac{F(a - 2F)}{a - F} - F}  = F \cdot \frac{ \frac{F(a - 2F)}{a - F} }{ \frac{F(a - 2F)}{a - F} - 1} = \\
     &= F \cdot \frac{a - 2F}{a - 2F - a + F} = 2F - a = 10\,\text{см}.
     \\
    \ell &= a + 2F + c = 4F = 80\,\text{см}.
    \\
    &\Gamma = \Gamma_1 \cdot \Gamma_2 = \frac ba \cdot \frac c{2F-b} = \frac F{a - F} \cdot \frac{2F - a}{\frac{F(a - 2F)}{a - F}} = -1.
    \end{align*}
}

\variantsplitter

\addpersonalvariant{Роксана Мехтиева}

\tasknumber{1}%
\task{%
    Укажите, верны ли утверждения («да» или «нет» слева от каждого утверждения):
    \begin{itemize}
        \item  Изображение предмета в собирающей линзе всегда мнимое.
        \item  Изображение предмета в собирающей линзе всегда прямое.
        \item  Изображение предмета в собирающей линзе всегда увеличенное.
        \item  Оптическая сила собирающей линзы положительна.
    \end{itemize}
}
\answer{%
    $\text{ нет, нет, нет, да }$
}

\tasknumber{2}%
\task{%
    Докажите формулу тонкой линзы для рассеивающей линзы.
}
\solutionspace{120pt}

\tasknumber{3}%
\task{%
    Найти оптическую силу собирающей линзы, если действительное изображение предмета,
    помещённого в $55\,\text{см}$ от линзы, получается на расстоянии $30\,\text{см}$ от неё.
}
\answer{%
    $D = \frac 1F = \frac 1a + \frac 1b = \frac 1{55\,\text{см}} + \frac 1{30\,\text{см}} \approx 5{,}15\,\text{дптр}$
}
\solutionspace{80pt}

\tasknumber{4}%
\task{%
    Найти увеличение изображения, если изображение предмета, находящегося
    на расстоянии $20\,\text{см}$ от линзы, получается на расстоянии $18\,\text{см}$ от неё.
}
\answer{%
    $\Gamma = \frac ba = \frac {18\,\text{см}}{20\,\text{см}} \approx 0{,}9$
}
\solutionspace{80pt}

\tasknumber{5}%
\task{%
    Расстояние от предмета до линзы $10\,\text{см}$, а от линзы до мнимого изображения $25\,\text{см}$.
    Чему равно фокусное расстояние линзы?
}
\answer{%
    $\pm \frac 1F = \frac 1a - \frac 1b \implies F = \frac{a b}{\abs{b - a}} \approx 16{,}7\,\text{см}$
}
\solutionspace{80pt}

\tasknumber{6}%
\task{%
    Две одинаковые собиращие линзы установлены так, что их главные оптические оси совпадают,
    а главный фокус первой находится там же, где главный фокус второй.
    Расстояние от первой линзы до предмета равно $18\,\text{см}$.
    Чему равно расстояние от изображения объекта во второй линзе до самого объекта?
    Определите также увеличение.
    Фокусное расстояние каждой линзы $25\,\text{см}$.
}
\answer{%
    \begin{align*}
    \frac 1a + \frac 1b &= \frac 1F \implies b = \frac{aF}{a - F} \implies 2F - b = \frac{2aF - 2F^2 - aF}{a - F} = \frac{F(a - 2F)}{a - F}.
    \\
    \frac 1{2F - b} + \frac 1c &= \frac 1F \implies c = \frac{F(2F-b)}{(2F - b) - F} = \frac{F \cdot \frac{F(a - 2F)}{a - F}}{\frac{F(a - 2F)}{a - F} - F}  = F \cdot \frac{ \frac{F(a - 2F)}{a - F} }{ \frac{F(a - 2F)}{a - F} - 1} = \\
     &= F \cdot \frac{a - 2F}{a - 2F - a + F} = 2F - a = 32\,\text{см}.
     \\
    \ell &= a + 2F + c = 4F = 100\,\text{см}.
    \\
    &\Gamma = \Gamma_1 \cdot \Gamma_2 = \frac ba \cdot \frac c{2F-b} = \frac F{a - F} \cdot \frac{2F - a}{\frac{F(a - 2F)}{a - F}} = -1.
    \end{align*}
}

\variantsplitter

\addpersonalvariant{Дилноза Нодиршоева}

\tasknumber{1}%
\task{%
    Укажите, верны ли утверждения («да» или «нет» слева от каждого утверждения):
    \begin{itemize}
        \item  Изображение предмета в рассеивающей линзе всегда действительное.
        \item  Изображение предмета в рассеивающей линзе всегда перевёрнутое.
        \item  Изображение предмета в рассеивающей линзе всегда уменьшенное.
        \item  Оптическая сила рассеивающей линзы положительна.
    \end{itemize}
}
\answer{%
    $\text{ нет, нет, да, нет }$
}

\tasknumber{2}%
\task{%
    Докажите формулу тонкой линзы для рассеивающей линзы.
}
\solutionspace{120pt}

\tasknumber{3}%
\task{%
    Найти оптическую силу собирающей линзы, если действительное изображение предмета,
    помещённого в $15\,\text{см}$ от линзы, получается на расстоянии $20\,\text{см}$ от неё.
}
\answer{%
    $D = \frac 1F = \frac 1a + \frac 1b = \frac 1{15\,\text{см}} + \frac 1{20\,\text{см}} \approx 11{,}67\,\text{дптр}$
}
\solutionspace{80pt}

\tasknumber{4}%
\task{%
    Найти увеличение изображения, если изображение предмета, находящегося
    на расстоянии $15\,\text{см}$ от линзы, получается на расстоянии $30\,\text{см}$ от неё.
}
\answer{%
    $\Gamma = \frac ba = \frac {30\,\text{см}}{15\,\text{см}} \approx 2$
}
\solutionspace{80pt}

\tasknumber{5}%
\task{%
    Расстояние от предмета до линзы $8\,\text{см}$, а от линзы до мнимого изображения $30\,\text{см}$.
    Чему равно фокусное расстояние линзы?
}
\answer{%
    $\pm \frac 1F = \frac 1a - \frac 1b \implies F = \frac{a b}{\abs{b - a}} \approx 10{,}9\,\text{см}$
}
\solutionspace{80pt}

\tasknumber{6}%
\task{%
    Две одинаковые собиращие линзы установлены так, что их главные оптические оси совпадают,
    а главный фокус первой находится там же, где главный фокус второй.
    Расстояние от первой линзы до предмета равно $10\,\text{см}$.
    Чему равно расстояние от изображения объекта во второй линзе до второй линзы?
    Определите также увеличение.
    Фокусное расстояние каждой линзы $25\,\text{см}$.
}
\answer{%
    \begin{align*}
    \frac 1a + \frac 1b &= \frac 1F \implies b = \frac{aF}{a - F} \implies 2F - b = \frac{2aF - 2F^2 - aF}{a - F} = \frac{F(a - 2F)}{a - F}.
    \\
    \frac 1{2F - b} + \frac 1c &= \frac 1F \implies c = \frac{F(2F-b)}{(2F - b) - F} = \frac{F \cdot \frac{F(a - 2F)}{a - F}}{\frac{F(a - 2F)}{a - F} - F}  = F \cdot \frac{ \frac{F(a - 2F)}{a - F} }{ \frac{F(a - 2F)}{a - F} - 1} = \\
     &= F \cdot \frac{a - 2F}{a - 2F - a + F} = 2F - a = 40\,\text{см}.
     \\
    \ell &= a + 2F + c = 4F = 100\,\text{см}.
    \\
    &\Gamma = \Gamma_1 \cdot \Gamma_2 = \frac ba \cdot \frac c{2F-b} = \frac F{a - F} \cdot \frac{2F - a}{\frac{F(a - 2F)}{a - F}} = -1.
    \end{align*}
}

\variantsplitter

\addpersonalvariant{Жаклин Пантелеева}

\tasknumber{1}%
\task{%
    Укажите, верны ли утверждения («да» или «нет» слева от каждого утверждения):
    \begin{itemize}
        \item  Изображение предмета в рассеивающей линзе всегда мнимое.
        \item  Изображение предмета в рассеивающей линзе всегда перевёрнутое.
        \item  Изображение предмета в рассеивающей линзе всегда увеличенное.
        \item  Оптическая сила собирающей линзы положительна.
    \end{itemize}
}
\answer{%
    $\text{ да, нет, нет, да }$
}

\tasknumber{2}%
\task{%
    Докажите формулу тонкой линзы для собирающей линзы.
}
\solutionspace{120pt}

\tasknumber{3}%
\task{%
    Найти оптическую силу собирающей линзы, если действительное изображение предмета,
    помещённого в $55\,\text{см}$ от линзы, получается на расстоянии $20\,\text{см}$ от неё.
}
\answer{%
    $D = \frac 1F = \frac 1a + \frac 1b = \frac 1{55\,\text{см}} + \frac 1{20\,\text{см}} \approx 6{,}82\,\text{дптр}$
}
\solutionspace{80pt}

\tasknumber{4}%
\task{%
    Найти увеличение изображения, если изображение предмета, находящегося
    на расстоянии $15\,\text{см}$ от линзы, получается на расстоянии $12\,\text{см}$ от неё.
}
\answer{%
    $\Gamma = \frac ba = \frac {12\,\text{см}}{15\,\text{см}} \approx 0{,}8$
}
\solutionspace{80pt}

\tasknumber{5}%
\task{%
    Расстояние от предмета до линзы $8\,\text{см}$, а от линзы до мнимого изображения $20\,\text{см}$.
    Чему равно фокусное расстояние линзы?
}
\answer{%
    $\pm \frac 1F = \frac 1a - \frac 1b \implies F = \frac{a b}{\abs{b - a}} \approx 13{,}3\,\text{см}$
}
\solutionspace{80pt}

\tasknumber{6}%
\task{%
    Две одинаковые собиращие линзы установлены так, что их главные оптические оси совпадают,
    а главный фокус первой находится там же, где главный фокус второй.
    Расстояние от первой линзы до предмета равно $26\,\text{см}$.
    Чему равно расстояние от изображения объекта во второй линзе до самого объекта?
    Определите также увеличение.
    Фокусное расстояние каждой линзы $25\,\text{см}$.
}
\answer{%
    \begin{align*}
    \frac 1a + \frac 1b &= \frac 1F \implies b = \frac{aF}{a - F} \implies 2F - b = \frac{2aF - 2F^2 - aF}{a - F} = \frac{F(a - 2F)}{a - F}.
    \\
    \frac 1{2F - b} + \frac 1c &= \frac 1F \implies c = \frac{F(2F-b)}{(2F - b) - F} = \frac{F \cdot \frac{F(a - 2F)}{a - F}}{\frac{F(a - 2F)}{a - F} - F}  = F \cdot \frac{ \frac{F(a - 2F)}{a - F} }{ \frac{F(a - 2F)}{a - F} - 1} = \\
     &= F \cdot \frac{a - 2F}{a - 2F - a + F} = 2F - a = 24\,\text{см}.
     \\
    \ell &= a + 2F + c = 4F = 100\,\text{см}.
    \\
    &\Gamma = \Gamma_1 \cdot \Gamma_2 = \frac ba \cdot \frac c{2F-b} = \frac F{a - F} \cdot \frac{2F - a}{\frac{F(a - 2F)}{a - F}} = -1.
    \end{align*}
}

\variantsplitter

\addpersonalvariant{Артём Переверзев}

\tasknumber{1}%
\task{%
    Укажите, верны ли утверждения («да» или «нет» слева от каждого утверждения):
    \begin{itemize}
        \item  Изображение предмета в рассеивающей линзе всегда действительное.
        \item  Изображение предмета в рассеивающей линзе всегда прямое.
        \item  Изображение предмета в рассеивающей линзе всегда увеличенное.
        \item  Оптическая сила рассеивающей линзы положительна.
    \end{itemize}
}
\answer{%
    $\text{ нет, да, нет, нет }$
}

\tasknumber{2}%
\task{%
    Докажите формулу тонкой линзы для собирающей линзы.
}
\solutionspace{120pt}

\tasknumber{3}%
\task{%
    Найти оптическую силу собирающей линзы, если действительное изображение предмета,
    помещённого в $15\,\text{см}$ от линзы, получается на расстоянии $30\,\text{см}$ от неё.
}
\answer{%
    $D = \frac 1F = \frac 1a + \frac 1b = \frac 1{15\,\text{см}} + \frac 1{30\,\text{см}} \approx 10\,\text{дптр}$
}
\solutionspace{80pt}

\tasknumber{4}%
\task{%
    Найти увеличение изображения, если изображение предмета, находящегося
    на расстоянии $20\,\text{см}$ от линзы, получается на расстоянии $30\,\text{см}$ от неё.
}
\answer{%
    $\Gamma = \frac ba = \frac {30\,\text{см}}{20\,\text{см}} \approx 1{,}50$
}
\solutionspace{80pt}

\tasknumber{5}%
\task{%
    Расстояние от предмета до линзы $10\,\text{см}$, а от линзы до мнимого изображения $30\,\text{см}$.
    Чему равно фокусное расстояние линзы?
}
\answer{%
    $\pm \frac 1F = \frac 1a - \frac 1b \implies F = \frac{a b}{\abs{b - a}} \approx 15\,\text{см}$
}
\solutionspace{80pt}

\tasknumber{6}%
\task{%
    Две одинаковые собиращие линзы установлены так, что их главные оптические оси совпадают,
    а главный фокус первой находится там же, где главный фокус второй.
    Расстояние от первой линзы до предмета равно $11\,\text{см}$.
    Чему равно расстояние от изображения объекта во второй линзе до второй линзы?
    Определите также увеличение.
    Фокусное расстояние каждой линзы $25\,\text{см}$.
}
\answer{%
    \begin{align*}
    \frac 1a + \frac 1b &= \frac 1F \implies b = \frac{aF}{a - F} \implies 2F - b = \frac{2aF - 2F^2 - aF}{a - F} = \frac{F(a - 2F)}{a - F}.
    \\
    \frac 1{2F - b} + \frac 1c &= \frac 1F \implies c = \frac{F(2F-b)}{(2F - b) - F} = \frac{F \cdot \frac{F(a - 2F)}{a - F}}{\frac{F(a - 2F)}{a - F} - F}  = F \cdot \frac{ \frac{F(a - 2F)}{a - F} }{ \frac{F(a - 2F)}{a - F} - 1} = \\
     &= F \cdot \frac{a - 2F}{a - 2F - a + F} = 2F - a = 39\,\text{см}.
     \\
    \ell &= a + 2F + c = 4F = 100\,\text{см}.
    \\
    &\Gamma = \Gamma_1 \cdot \Gamma_2 = \frac ba \cdot \frac c{2F-b} = \frac F{a - F} \cdot \frac{2F - a}{\frac{F(a - 2F)}{a - F}} = -1.
    \end{align*}
}

\variantsplitter

\addpersonalvariant{Варвара Пранова}

\tasknumber{1}%
\task{%
    Укажите, верны ли утверждения («да» или «нет» слева от каждого утверждения):
    \begin{itemize}
        \item  Изображение предмета в собирающей линзе всегда мнимое.
        \item  Изображение предмета в собирающей линзе всегда прямое.
        \item  Изображение предмета в собирающей линзе всегда уменьшенное.
        \item  Оптическая сила собирающей линзы отрицательна.
    \end{itemize}
}
\answer{%
    $\text{ нет, нет, нет, нет }$
}

\tasknumber{2}%
\task{%
    Докажите формулу тонкой линзы для собирающей линзы.
}
\solutionspace{120pt}

\tasknumber{3}%
\task{%
    Найти оптическую силу собирающей линзы, если действительное изображение предмета,
    помещённого в $35\,\text{см}$ от линзы, получается на расстоянии $30\,\text{см}$ от неё.
}
\answer{%
    $D = \frac 1F = \frac 1a + \frac 1b = \frac 1{35\,\text{см}} + \frac 1{30\,\text{см}} \approx 6{,}19\,\text{дптр}$
}
\solutionspace{80pt}

\tasknumber{4}%
\task{%
    Найти увеличение изображения, если изображение предмета, находящегося
    на расстоянии $25\,\text{см}$ от линзы, получается на расстоянии $18\,\text{см}$ от неё.
}
\answer{%
    $\Gamma = \frac ba = \frac {18\,\text{см}}{25\,\text{см}} \approx 0{,}7$
}
\solutionspace{80pt}

\tasknumber{5}%
\task{%
    Расстояние от предмета до линзы $12\,\text{см}$, а от линзы до мнимого изображения $25\,\text{см}$.
    Чему равно фокусное расстояние линзы?
}
\answer{%
    $\pm \frac 1F = \frac 1a - \frac 1b \implies F = \frac{a b}{\abs{b - a}} \approx 23{,}1\,\text{см}$
}
\solutionspace{80pt}

\tasknumber{6}%
\task{%
    Две одинаковые собиращие линзы установлены так, что их главные оптические оси совпадают,
    а главный фокус первой находится там же, где главный фокус второй.
    Расстояние от первой линзы до предмета равно $34\,\text{см}$.
    Чему равно расстояние от изображения объекта во второй линзе до самого объекта?
    Определите также увеличение.
    Фокусное расстояние каждой линзы $20\,\text{см}$.
}
\answer{%
    \begin{align*}
    \frac 1a + \frac 1b &= \frac 1F \implies b = \frac{aF}{a - F} \implies 2F - b = \frac{2aF - 2F^2 - aF}{a - F} = \frac{F(a - 2F)}{a - F}.
    \\
    \frac 1{2F - b} + \frac 1c &= \frac 1F \implies c = \frac{F(2F-b)}{(2F - b) - F} = \frac{F \cdot \frac{F(a - 2F)}{a - F}}{\frac{F(a - 2F)}{a - F} - F}  = F \cdot \frac{ \frac{F(a - 2F)}{a - F} }{ \frac{F(a - 2F)}{a - F} - 1} = \\
     &= F \cdot \frac{a - 2F}{a - 2F - a + F} = 2F - a = 6\,\text{см}.
     \\
    \ell &= a + 2F + c = 4F = 80\,\text{см}.
    \\
    &\Gamma = \Gamma_1 \cdot \Gamma_2 = \frac ba \cdot \frac c{2F-b} = \frac F{a - F} \cdot \frac{2F - a}{\frac{F(a - 2F)}{a - F}} = -1.
    \end{align*}
}

\variantsplitter

\addpersonalvariant{Марьям Салимова}

\tasknumber{1}%
\task{%
    Укажите, верны ли утверждения («да» или «нет» слева от каждого утверждения):
    \begin{itemize}
        \item  Изображение предмета в рассеивающей линзе всегда мнимое.
        \item  Изображение предмета в рассеивающей линзе всегда перевёрнутое.
        \item  Изображение предмета в рассеивающей линзе всегда увеличенное.
        \item  Оптическая сила собирающей линзы отрицательна.
    \end{itemize}
}
\answer{%
    $\text{ да, нет, нет, нет }$
}

\tasknumber{2}%
\task{%
    Докажите формулу тонкой линзы для собирающей линзы.
}
\solutionspace{120pt}

\tasknumber{3}%
\task{%
    Найти оптическую силу собирающей линзы, если действительное изображение предмета,
    помещённого в $35\,\text{см}$ от линзы, получается на расстоянии $20\,\text{см}$ от неё.
}
\answer{%
    $D = \frac 1F = \frac 1a + \frac 1b = \frac 1{35\,\text{см}} + \frac 1{20\,\text{см}} \approx 7{,}86\,\text{дптр}$
}
\solutionspace{80pt}

\tasknumber{4}%
\task{%
    Найти увеличение изображения, если изображение предмета, находящегося
    на расстоянии $20\,\text{см}$ от линзы, получается на расстоянии $30\,\text{см}$ от неё.
}
\answer{%
    $\Gamma = \frac ba = \frac {30\,\text{см}}{20\,\text{см}} \approx 1{,}50$
}
\solutionspace{80pt}

\tasknumber{5}%
\task{%
    Расстояние от предмета до линзы $10\,\text{см}$, а от линзы до мнимого изображения $30\,\text{см}$.
    Чему равно фокусное расстояние линзы?
}
\answer{%
    $\pm \frac 1F = \frac 1a - \frac 1b \implies F = \frac{a b}{\abs{b - a}} \approx 15\,\text{см}$
}
\solutionspace{80pt}

\tasknumber{6}%
\task{%
    Две одинаковые собиращие линзы установлены так, что их главные оптические оси совпадают,
    а главный фокус первой находится там же, где главный фокус второй.
    Расстояние от первой линзы до предмета равно $10\,\text{см}$.
    Чему равно расстояние от изображения объекта во второй линзе до второй линзы?
    Определите также увеличение.
    Фокусное расстояние каждой линзы $20\,\text{см}$.
}
\answer{%
    \begin{align*}
    \frac 1a + \frac 1b &= \frac 1F \implies b = \frac{aF}{a - F} \implies 2F - b = \frac{2aF - 2F^2 - aF}{a - F} = \frac{F(a - 2F)}{a - F}.
    \\
    \frac 1{2F - b} + \frac 1c &= \frac 1F \implies c = \frac{F(2F-b)}{(2F - b) - F} = \frac{F \cdot \frac{F(a - 2F)}{a - F}}{\frac{F(a - 2F)}{a - F} - F}  = F \cdot \frac{ \frac{F(a - 2F)}{a - F} }{ \frac{F(a - 2F)}{a - F} - 1} = \\
     &= F \cdot \frac{a - 2F}{a - 2F - a + F} = 2F - a = 30\,\text{см}.
     \\
    \ell &= a + 2F + c = 4F = 80\,\text{см}.
    \\
    &\Gamma = \Gamma_1 \cdot \Gamma_2 = \frac ba \cdot \frac c{2F-b} = \frac F{a - F} \cdot \frac{2F - a}{\frac{F(a - 2F)}{a - F}} = -1.
    \end{align*}
}

\variantsplitter

\addpersonalvariant{Юлия Шевченко}

\tasknumber{1}%
\task{%
    Укажите, верны ли утверждения («да» или «нет» слева от каждого утверждения):
    \begin{itemize}
        \item  Изображение предмета в собирающей линзе всегда действительное.
        \item  Изображение предмета в собирающей линзе всегда прямое.
        \item  Изображение предмета в собирающей линзе всегда уменьшенное.
        \item  Оптическая сила рассеивающей линзы отрицательна.
    \end{itemize}
}
\answer{%
    $\text{ нет, нет, нет, да }$
}

\tasknumber{2}%
\task{%
    Докажите формулу тонкой линзы для собирающей линзы.
}
\solutionspace{120pt}

\tasknumber{3}%
\task{%
    Найти оптическую силу собирающей линзы, если действительное изображение предмета,
    помещённого в $55\,\text{см}$ от линзы, получается на расстоянии $30\,\text{см}$ от неё.
}
\answer{%
    $D = \frac 1F = \frac 1a + \frac 1b = \frac 1{55\,\text{см}} + \frac 1{30\,\text{см}} \approx 5{,}15\,\text{дптр}$
}
\solutionspace{80pt}

\tasknumber{4}%
\task{%
    Найти увеличение изображения, если изображение предмета, находящегося
    на расстоянии $20\,\text{см}$ от линзы, получается на расстоянии $18\,\text{см}$ от неё.
}
\answer{%
    $\Gamma = \frac ba = \frac {18\,\text{см}}{20\,\text{см}} \approx 0{,}9$
}
\solutionspace{80pt}

\tasknumber{5}%
\task{%
    Расстояние от предмета до линзы $10\,\text{см}$, а от линзы до мнимого изображения $25\,\text{см}$.
    Чему равно фокусное расстояние линзы?
}
\answer{%
    $\pm \frac 1F = \frac 1a - \frac 1b \implies F = \frac{a b}{\abs{b - a}} \approx 16{,}7\,\text{см}$
}
\solutionspace{80pt}

\tasknumber{6}%
\task{%
    Две одинаковые собиращие линзы установлены так, что их главные оптические оси совпадают,
    а главный фокус первой находится там же, где главный фокус второй.
    Расстояние от первой линзы до предмета равно $30\,\text{см}$.
    Чему равно расстояние от изображения объекта во второй линзе до самого объекта?
    Определите также увеличение.
    Фокусное расстояние каждой линзы $25\,\text{см}$.
}
\answer{%
    \begin{align*}
    \frac 1a + \frac 1b &= \frac 1F \implies b = \frac{aF}{a - F} \implies 2F - b = \frac{2aF - 2F^2 - aF}{a - F} = \frac{F(a - 2F)}{a - F}.
    \\
    \frac 1{2F - b} + \frac 1c &= \frac 1F \implies c = \frac{F(2F-b)}{(2F - b) - F} = \frac{F \cdot \frac{F(a - 2F)}{a - F}}{\frac{F(a - 2F)}{a - F} - F}  = F \cdot \frac{ \frac{F(a - 2F)}{a - F} }{ \frac{F(a - 2F)}{a - F} - 1} = \\
     &= F \cdot \frac{a - 2F}{a - 2F - a + F} = 2F - a = 20\,\text{см}.
     \\
    \ell &= a + 2F + c = 4F = 100\,\text{см}.
    \\
    &\Gamma = \Gamma_1 \cdot \Gamma_2 = \frac ba \cdot \frac c{2F-b} = \frac F{a - F} \cdot \frac{2F - a}{\frac{F(a - 2F)}{a - F}} = -1.
    \end{align*}
}
% autogenerated
