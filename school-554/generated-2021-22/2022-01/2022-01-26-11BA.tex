\setdate{26~января~2022}
\setclass{11«БА»}

\addpersonalvariant{Михаил Бурмистров}

\tasknumber{1}%
\task{%
    Найти оптическую силу собирающей линзы, если действительное изображение предмета,
    помещённого в $15\,\text{см}$ от линзы, получается на расстоянии $30\,\text{см}$ от неё.
}
\answer{%
    $D = \frac 1F = \frac 1a + \frac 1b = \frac 1{15\,\text{см}} + \frac 1{30\,\text{см}} \approx 10\,\text{дптр}$
}
\solutionspace{80pt}

\tasknumber{2}%
\task{%
    Найти увеличение изображения, если изображение предмета, находящегося
    на расстоянии $20\,\text{см}$ от линзы, получается на расстоянии $30\,\text{см}$ от неё.
}
\answer{%
    $\Gamma = \frac ba = \frac {30\,\text{см}}{20\,\text{см}} \approx 1{,}50$
}
\solutionspace{80pt}

\tasknumber{3}%
\task{%
    Расстояние от предмета до линзы $10\,\text{см}$, а от линзы до мнимого изображения $30\,\text{см}$.
    Чему равно фокусное расстояние линзы?
}
\answer{%
    $\pm \frac 1F = \frac 1a - \frac 1b \implies F = \frac{a b}{\abs{b - a}} \approx 15\,\text{см}$
}
\solutionspace{80pt}

\tasknumber{4}%
\task{%
    Две тонкие собирающие линзы с фокусными расстояниями $25\,\text{см}$ и $30\,\text{см}$ сложены вместе.
    Чему равно фокусное расстояние такой оптической системы?
}
\answer{%
    $\frac 1{f_1} = \frac 1a + \frac 1b; \frac 1{f_2} = - \frac 1b + \frac 1c \implies \frac 1{f_1} + \frac 1{f_2} = \frac 1a + \frac 1c \implies f' = \frac 1{\frac 1{f_1} + \frac 1{f_2}} = \frac{f_1 f_2}{f_1 + f_2} \approx 13{,}6\,\text{см}$
}
\solutionspace{80pt}

\tasknumber{5}%
\task{%
    Линейные размеры прямого изображения предмета, полученного в собирающей линзе,
    в два раза больше линейных размеров предмета.
    Зная, что предмет находится на $40\,\text{см}$ ближе к линзе,
    чем его изображение, найти оптическую силу линзы.
}
\answer{%
    \begin{align*}
    &\text{Если изображение действительное:} \\
    D &= \frac 1F = \frac 1a + \frac 1b, \qquad \Gamma = \frac ba, \qquad b - a = \ell \implies b = \Gamma a \implies \Gamma a - a = \ell \implies  \\
    a &= \frac {\ell}{\Gamma - 1} \implies b = \frac {{\ell} \Gamma}{\Gamma - 1} \implies  \\
    D &= \frac {\Gamma - 1}\ell + \frac {\Gamma - 1}{\ell \Gamma} = \frac 1\ell \cdot \cbr{\Gamma - 1 + \frac {\Gamma - 1}{\Gamma} } =\frac 1\ell \cdot \cbr{\Gamma - \frac 1\Gamma} \approx 3{,}8\,\text{дптр}.
    \\
    &\text{Если изображение мнимое:} \\
    D &= \frac 1F = \frac 1a - \frac 1b, \qquad \Gamma = \frac ba, \qquad b - a = \ell \implies b = \Gamma a \implies \Gamma a - a = \ell \implies  \\
    a &= \frac {\ell}{\Gamma - 1} \implies b = \frac {{\ell} \Gamma}{\Gamma - 1} \implies  \\
    D &= \frac {\Gamma - 1}\ell - \frac {\Gamma - 1}{\ell \Gamma} = \frac 1\ell \cdot \cbr{\Gamma - 1 - \frac {\Gamma - 1}{\Gamma} } =\frac 1\ell \cdot \cbr{\Gamma + \frac 1\Gamma - 2} \approx 1{,}2\,\text{дптр}.
    \\
    &\text{В ответе надо указать оба значения.}
    \end{align*}
}
\solutionspace{120pt}

\tasknumber{6}%
\task{%
    Оптическая сила объектива фотоаппарата равна $6\,\text{дптр}$.
    При фотографировании чертежа с расстояния $1{,}2\,\text{м}$ площадь изображения
    чертежа на фотопластинке оказалась равной $16\,\text{см}^{2}$.
    Какова площадь самого чертежа? Ответ выразите в квадратных сантиметрах.
}
\answer{%
    \begin{align*}
    &\frac 1a + \frac 1b = \frac 1F = D \implies b = \frac{aF}{a - F} \\
    &\frac {S'}S = \Gamma^2 = \sqr{\frac ba} = \sqr{\frac F{a - F}} \implies \\
    &\implies S = S' \cdot \sqr{\frac{a - F}F} = S' \cdot \sqr{\frac aF - 1} = S' \cdot \sqr{aD - 1} \approx 600\,\text{см}^{2}.
    \end{align*}
}


\variantsplitter


\addpersonalvariant{Михаил Бурмистров}

\tasknumber{7}%
\task{%
    В каком месте на главной оптической оси двояковыгнутой линзы
    нужно поместить точечный источник света,
    чтобы его изображение оказалось в главном фокусе линзы?
}
\answer{%
    $\text{на половине фокусного расстояния}$
}
\solutionspace{120pt}

\tasknumber{8}%
\task{%
    Предмет в виде отрезка длиной $\ell$ расположен вдоль оптической оси
    собирающей линзы с фокусным расстоянием $F$.
    Середина отрезка расположена
    на расстоянии $a$ от линзы, которая даёт действительное изображение
    всех точек предмета.
    Определить продольное увеличение предмета.
}
\answer{%
    \begin{align*}
    \frac 1{a + \frac \ell 2} &+ \frac 1b = \frac 1F \implies b = \frac{F\cbr{a + \frac \ell 2}}{a + \frac \ell 2 - F} \\
    \frac 1{a - \frac \ell 2} &+ \frac 1c = \frac 1F \implies c = \frac{F\cbr{a - \frac \ell 2}}{a - \frac \ell 2 - F} \\
    \abs{b - c} &= \abs{\frac{F\cbr{a + \frac \ell 2}}{a + \frac \ell 2 - F} - \frac{F\cbr{a - \frac \ell 2}}{a - \frac \ell 2 - F}}= F\abs{\frac{\cbr{a + \frac \ell 2}\cbr{a - \frac \ell 2 - F} - \cbr{a - \frac \ell 2}\cbr{a + \frac \ell 2 - F}}{ \cbr{a + \frac \ell 2 - F} \cbr{a - \frac \ell 2 - F} }} =  \\
    &= F\abs{\frac{a^2 - \frac {a\ell} 2 - Fa + \frac {a\ell} 2 - \frac {\ell^2} 4 - \frac {F\ell}2 - a^2 - \frac {a\ell}2 + aF + \frac {a\ell}2 + \frac {\ell^2} 4 - \frac {F\ell} 2}{\cbr{a + \frac \ell 2 - F} \cbr{a - \frac \ell 2 - F} }} = \\
    &= F\frac{F\ell}{\sqr{a-F} - \frac {\ell^2}4} = \frac{F^2\ell}{\sqr{a-F} - \frac {\ell^2}4}\implies \Gamma = \frac{\abs{b - c}}\ell = \frac{F^2}{\sqr{a-F} - \frac {\ell^2}4}.
    \end{align*}
}
\solutionspace{120pt}

\tasknumber{9}%
\task{%
    На экране с помощью тонкой линзы получено изображение предмета
    с увеличением $2$.
    Предмет передвинули на $4\,\text{см}$.
    Для того, чтобы получить резкое изображение, пришлось передвинуть экран.
    При этом увеличение оказалось равным $6$.
    На какое расстояние
    пришлось передвинуть экран?
}
\answer{%
    \begin{align*}
    &\frac 1a + \frac 1b = \frac 1F, \Gamma_1 = \frac ba = \frac{F}{a-F} \implies \Gamma_1(a-F) = F \implies a = F \cdot \frac{1 + \Gamma_1}{\Gamma_1} \\
    &\frac 1{a + x} + \frac 1{b + y} = \frac 1F, \Gamma_2 = \frac {b+y}{a+x} = \frac{F}{a+x-F} \implies a + x = F \cdot \frac{1 + \Gamma_2}{\Gamma_2} \\
    &1 + \frac xa = \frac{ \frac{1 + \Gamma_2}{\Gamma_2} }{ \frac{1 + \Gamma_1}{\Gamma_1} } = \frac{\Gamma_1(1 + \Gamma_2)}{\Gamma_2(1 + \Gamma_1)} \\
    &a = \frac x{ \frac{\Gamma_1(1 + \Gamma_2)}{\Gamma_2(1 + \Gamma_1)} - 1} = x \cdot \frac{\Gamma_2(1 + \Gamma_1)}{\Gamma_1 - \Gamma_2} \\
    &y = (a + x)\Gamma_2 - b = (a + x)\Gamma_2 - a\Gamma_1 = a(\Gamma_2 - \Gamma_1) + x\Gamma_2 = -x\Gamma_2(1 + \Gamma_1) + x\Gamma_2 = -x\Gamma_2\Gamma_1 = 48\,\text{см}, \\
    &\text{знаки разные, т.е.
    экран надо было подвинуть в другую сторону чем предмет: $x < 0, y > 0$.}
    \end{align*}
}
\solutionspace{120pt}

\tasknumber{10}%
\task{%
    Тонкая собирающая линза дает изображение предмета на экране при двух положениях линзы между предметом и экраном.
    Высота изображения при первом положении $20\,\text{см}$, во втором — $5\,\text{см}$.
    Расстояние между предметом и экранов постоянно.
    Чему равна высота предмета?
}
\answer{%
    \begin{align*}
    &\frac 1a + \frac 1b = \frac 1F, \frac 1c + \frac 1d = \frac 1F, a + b = c + d \implies \frac{a + b}{ab} = \frac 1F = \frac{c+d}{cd} \implies ab = cd, \\
    &\implies ab = c(a + b - c) \implies c^2 - ac - bc + ab = 0 \implies c = a \text{ или } c = b \implies c = b \implies d = a.
    \\
    &\Gamma_1 = \frac {H_1}H = \frac ba, \Gamma_2 = \frac {H_2}H = \frac dc = \frac ab \implies \frac {H_1}H \cdot \frac {H_2}H = \frac ba \cdot \frac ab = 1, \\
    &H = \sqrt{H_1 H_2} \approx 10\,\text{см}.
    \end{align*}
}
\solutionspace{120pt}

\tasknumber{11}%
\task{%
    Какие предметы можно рассмотреть на фотографии, сделанной со спутника,
    если разрешающая способность плёнки $0{,}010\,\text{мм}$? Каким должно быть
    время экспозиции $\tau$ чтобы полностью использовать возможности плёнки?
    Фокусное расстояние объектива используемого фотоаппарата $10\,\text{см}$,
    высота орбиты спутника $150\,\text{км}$.
}
\answer{%
    \begin{align*}
    &H \ll R \implies v = v_{\text{I}} = \sqrt{G R} \approx 7{,}9\,\frac{\text{км}}{\text{с}}.
    \\
    &F \ll H \implies b = F, a = H, \\
    &\Gamma = \frac \delta\ell = \frac ba \implies \ell = \frac{\delta a}b = \frac{\delta H}F \approx \frac{0{,}010\,\text{мм} \cdot 150\,\text{км}}{10\,\text{см}} \approx 15\,\text{м}, \\
    &\implies \tau = \frac \ell v = \frac{\delta H}{F v} = \frac{0{,}010\,\text{мм} \cdot 150\,\text{км}}{10\,\text{см} \cdot 7{,}9\,\frac{\text{км}}{\text{с}}} \approx 1{,}9\,\text{мс}.
    \end{align*}
}


\variantsplitter


\addpersonalvariant{Михаил Бурмистров}

\tasknumber{12}%
\task{%
    При аэрофотосъемках используется фотоаппарат, объектив которого
    имеет фокусиое расстояние $20\,\text{см}$.
    Разрешающая способность плёнки $0{,}015\,\text{мм}$.
    На какой высоте должен лететь самолет, чтобы на фотографии можно
    было различить следы размером $20\,\text{см}$?
    При какой скорости самолета изображение не будет размытым,
    если время экспозиции $1\,\text{мс}$?
}
\answer{%
    \begin{align*}
    &F \ll H \implies b = F, a = H, \\
    &\Gamma = \frac \delta\ell = \frac ba = \frac FH \implies H = \frac{\ell F}\delta = \frac{20\,\text{см} \cdot 20\,\text{см}}{0{,}015\,\text{мм}} \approx 3\,\text{км}, \\
    &\implies v = \frac l\tau = \frac{20\,\text{см}}{1\,\text{мс}} \approx 700\,\frac{\text{км}}{\text{ч}}.
    \end{align*}
}
\solutionspace{120pt}

\tasknumber{13}%
\task{%
    Две одинаковые собиращие линзы установлены так, что их главные оптические оси совпадают,
    а главный фокус первой находится там же, где главный фокус второй.
    Расстояние от первой линзы до предмета равно $10\,\text{см}$.
    Чему равно расстояние от изображения объекта во второй линзе до самого объекта?
    Определите также увеличение.
    Фокусное расстояние каждой линзы $30\,\text{см}$.
}
\answer{%
    \begin{align*}
    \frac 1a + \frac 1b &= \frac 1F \implies b = \frac{aF}{a - F} \implies 2F - b = \frac{2aF - 2F^2 - aF}{a - F} = \frac{F(a - 2F)}{a - F}.
    \\
    \frac 1{2F - b} + \frac 1c &= \frac 1F \implies c = \frac{F(2F-b)}{(2F - b) - F} = \frac{F \cdot \frac{F(a - 2F)}{a - F}}{\frac{F(a - 2F)}{a - F} - F}  = F \cdot \frac{ \frac{F(a - 2F)}{a - F} }{ \frac{F(a - 2F)}{a - F} - 1} = \\
     &= F \cdot \frac{a - 2F}{a - 2F - a + F} = 2F - a = 50\,\text{см}.
     \\
    \ell &= a + 2F + c = 4F = 120\,\text{см}.
    \\
    &\Gamma = \Gamma_1 \cdot \Gamma_2 = \frac ba \cdot \frac c{2F-b} = \frac F{a - F} \cdot \frac{2F - a}{\frac{F(a - 2F)}{a - F}} = -1.
    \end{align*}
}
\solutionspace{120pt}

\tasknumber{14}%
\task{%
    Собирающая линза с фокусным расстоянием $F_1 > 0$ и рассеивающая линза с фокусным расстоянием $F_2 < 0$
    установлены коаксиально на расстоянии $\ell$.
    Пучок параллельных лучей падает на рассеивающую линзу.
    Сделайте схематичное построение и определите, в какой точке система из этих линз соберёт пучок.
}
\answer{%
    \begin{align*}
    &\text{Если пучок падает на собирающую линзу:} \\
    \frac 1{\infty} + \frac 1b &= \frac 1{F_1} \implies b = F_1 \implies \ell - b = \ell - F_1 \\
    \frac 1{\ell - b} + \frac 1c &= \frac 1{F_2} \implies c = \frac{F_2(\ell - b)}{\ell - b - F_2} = \frac{F_2(\ell - F_1)}{\ell - F_1 - F_2}.
    \\
    &\text{Если же пучок падает на рассеивающую линзу:} \\
    \frac 1{\infty} + \frac 1b &= \frac 1{F_2} \implies b = F_2 \implies \ell - b = \ell - F_2 \\
    \frac 1{\ell - b} + \frac 1c &= \frac 1{F_1} \implies c = \frac{F_1(\ell - b)}{\ell - b - F_1} = \frac{F_1(\ell - F_2)}{\ell - F_2 - F_1}.
    \end{align*}
}
\solutionspace{120pt}

\tasknumber{15}%
\task{%
    Две собирающих линзы с фокусными расстояниями $20\,\text{см}$ и $45\,\text{см}$ расположены так,
    что их оптические оси совмещены.
    На первую линзу падает пучок параллельных лучей.
    Пройдя через вторую линзу, он остался параллельным.
    Найдите расстояние между линзами и сделайте рисунок.
}
\answer{%
    \begin{align*}
    \frac 1\infty + \frac 1b &= \frac 1{F_1} \implies b = F_1, \\
    \frac 1{\ell - b} + \frac 1{\infty} &= \frac 1{F_2} \implies \ell - b = F_2 \implies \ell = b + F_2 = F_1 + F_2 = 65\,\text{см}.
    \end{align*}
}

\variantsplitter

\addpersonalvariant{Ирина Ан}

\tasknumber{1}%
\task{%
    Найти оптическую силу собирающей линзы, если действительное изображение предмета,
    помещённого в $15\,\text{см}$ от линзы, получается на расстоянии $30\,\text{см}$ от неё.
}
\answer{%
    $D = \frac 1F = \frac 1a + \frac 1b = \frac 1{15\,\text{см}} + \frac 1{30\,\text{см}} \approx 10\,\text{дптр}$
}
\solutionspace{80pt}

\tasknumber{2}%
\task{%
    Найти увеличение изображения, если изображение предмета, находящегося
    на расстоянии $20\,\text{см}$ от линзы, получается на расстоянии $18\,\text{см}$ от неё.
}
\answer{%
    $\Gamma = \frac ba = \frac {18\,\text{см}}{20\,\text{см}} \approx 0{,}9$
}
\solutionspace{80pt}

\tasknumber{3}%
\task{%
    Расстояние от предмета до линзы $10\,\text{см}$, а от линзы до мнимого изображения $25\,\text{см}$.
    Чему равно фокусное расстояние линзы?
}
\answer{%
    $\pm \frac 1F = \frac 1a - \frac 1b \implies F = \frac{a b}{\abs{b - a}} \approx 16{,}7\,\text{см}$
}
\solutionspace{80pt}

\tasknumber{4}%
\task{%
    Две тонкие собирающие линзы с фокусными расстояниями $25\,\text{см}$ и $30\,\text{см}$ сложены вместе.
    Чему равно фокусное расстояние такой оптической системы?
}
\answer{%
    $\frac 1{f_1} = \frac 1a + \frac 1b; \frac 1{f_2} = - \frac 1b + \frac 1c \implies \frac 1{f_1} + \frac 1{f_2} = \frac 1a + \frac 1c \implies f' = \frac 1{\frac 1{f_1} + \frac 1{f_2}} = \frac{f_1 f_2}{f_1 + f_2} \approx 13{,}6\,\text{см}$
}
\solutionspace{80pt}

\tasknumber{5}%
\task{%
    Линейные размеры прямого изображения предмета, полученного в собирающей линзе,
    в четыре раза больше линейных размеров предмета.
    Зная, что предмет находится на $20\,\text{см}$ ближе к линзе,
    чем его изображение, найти оптическую силу линзы.
}
\answer{%
    \begin{align*}
    &\text{Если изображение действительное:} \\
    D &= \frac 1F = \frac 1a + \frac 1b, \qquad \Gamma = \frac ba, \qquad b - a = \ell \implies b = \Gamma a \implies \Gamma a - a = \ell \implies  \\
    a &= \frac {\ell}{\Gamma - 1} \implies b = \frac {{\ell} \Gamma}{\Gamma - 1} \implies  \\
    D &= \frac {\Gamma - 1}\ell + \frac {\Gamma - 1}{\ell \Gamma} = \frac 1\ell \cdot \cbr{\Gamma - 1 + \frac {\Gamma - 1}{\Gamma} } =\frac 1\ell \cdot \cbr{\Gamma - \frac 1\Gamma} \approx 18{,}8\,\text{дптр}.
    \\
    &\text{Если изображение мнимое:} \\
    D &= \frac 1F = \frac 1a - \frac 1b, \qquad \Gamma = \frac ba, \qquad b - a = \ell \implies b = \Gamma a \implies \Gamma a - a = \ell \implies  \\
    a &= \frac {\ell}{\Gamma - 1} \implies b = \frac {{\ell} \Gamma}{\Gamma - 1} \implies  \\
    D &= \frac {\Gamma - 1}\ell - \frac {\Gamma - 1}{\ell \Gamma} = \frac 1\ell \cdot \cbr{\Gamma - 1 - \frac {\Gamma - 1}{\Gamma} } =\frac 1\ell \cdot \cbr{\Gamma + \frac 1\Gamma - 2} \approx 11{,}2\,\text{дптр}.
    \\
    &\text{В ответе надо указать оба значения.}
    \end{align*}
}
\solutionspace{120pt}

\tasknumber{6}%
\task{%
    Оптическая сила объектива фотоаппарата равна $6\,\text{дптр}$.
    При фотографировании чертежа с расстояния $0{,}9\,\text{м}$ площадь изображения
    чертежа на фотопластинке оказалась равной $16\,\text{см}^{2}$.
    Какова площадь самого чертежа? Ответ выразите в квадратных сантиметрах.
}
\answer{%
    \begin{align*}
    &\frac 1a + \frac 1b = \frac 1F = D \implies b = \frac{aF}{a - F} \\
    &\frac {S'}S = \Gamma^2 = \sqr{\frac ba} = \sqr{\frac F{a - F}} \implies \\
    &\implies S = S' \cdot \sqr{\frac{a - F}F} = S' \cdot \sqr{\frac aF - 1} = S' \cdot \sqr{aD - 1} \approx 300\,\text{см}^{2}.
    \end{align*}
}


\variantsplitter


\addpersonalvariant{Ирина Ан}

\tasknumber{7}%
\task{%
    В каком месте на главной оптической оси двояковыгнутой линзы
    нужно поместить точечный источник света,
    чтобы его изображение оказалось в главном фокусе линзы?
}
\answer{%
    $\text{на половине фокусного расстояния}$
}
\solutionspace{120pt}

\tasknumber{8}%
\task{%
    Предмет в виде отрезка длиной $\ell$ расположен вдоль оптической оси
    собирающей линзы с фокусным расстоянием $F$.
    Середина отрезка расположена
    на расстоянии $a$ от линзы, которая даёт действительное изображение
    всех точек предмета.
    Определить продольное увеличение предмета.
}
\answer{%
    \begin{align*}
    \frac 1{a + \frac \ell 2} &+ \frac 1b = \frac 1F \implies b = \frac{F\cbr{a + \frac \ell 2}}{a + \frac \ell 2 - F} \\
    \frac 1{a - \frac \ell 2} &+ \frac 1c = \frac 1F \implies c = \frac{F\cbr{a - \frac \ell 2}}{a - \frac \ell 2 - F} \\
    \abs{b - c} &= \abs{\frac{F\cbr{a + \frac \ell 2}}{a + \frac \ell 2 - F} - \frac{F\cbr{a - \frac \ell 2}}{a - \frac \ell 2 - F}}= F\abs{\frac{\cbr{a + \frac \ell 2}\cbr{a - \frac \ell 2 - F} - \cbr{a - \frac \ell 2}\cbr{a + \frac \ell 2 - F}}{ \cbr{a + \frac \ell 2 - F} \cbr{a - \frac \ell 2 - F} }} =  \\
    &= F\abs{\frac{a^2 - \frac {a\ell} 2 - Fa + \frac {a\ell} 2 - \frac {\ell^2} 4 - \frac {F\ell}2 - a^2 - \frac {a\ell}2 + aF + \frac {a\ell}2 + \frac {\ell^2} 4 - \frac {F\ell} 2}{\cbr{a + \frac \ell 2 - F} \cbr{a - \frac \ell 2 - F} }} = \\
    &= F\frac{F\ell}{\sqr{a-F} - \frac {\ell^2}4} = \frac{F^2\ell}{\sqr{a-F} - \frac {\ell^2}4}\implies \Gamma = \frac{\abs{b - c}}\ell = \frac{F^2}{\sqr{a-F} - \frac {\ell^2}4}.
    \end{align*}
}
\solutionspace{120pt}

\tasknumber{9}%
\task{%
    На экране с помощью тонкой линзы получено изображение предмета
    с увеличением $4$.
    Предмет передвинули на $8\,\text{см}$.
    Для того, чтобы получить резкое изображение, пришлось передвинуть экран.
    При этом увеличение оказалось равным $8$.
    На какое расстояние
    пришлось передвинуть экран?
}
\answer{%
    \begin{align*}
    &\frac 1a + \frac 1b = \frac 1F, \Gamma_1 = \frac ba = \frac{F}{a-F} \implies \Gamma_1(a-F) = F \implies a = F \cdot \frac{1 + \Gamma_1}{\Gamma_1} \\
    &\frac 1{a + x} + \frac 1{b + y} = \frac 1F, \Gamma_2 = \frac {b+y}{a+x} = \frac{F}{a+x-F} \implies a + x = F \cdot \frac{1 + \Gamma_2}{\Gamma_2} \\
    &1 + \frac xa = \frac{ \frac{1 + \Gamma_2}{\Gamma_2} }{ \frac{1 + \Gamma_1}{\Gamma_1} } = \frac{\Gamma_1(1 + \Gamma_2)}{\Gamma_2(1 + \Gamma_1)} \\
    &a = \frac x{ \frac{\Gamma_1(1 + \Gamma_2)}{\Gamma_2(1 + \Gamma_1)} - 1} = x \cdot \frac{\Gamma_2(1 + \Gamma_1)}{\Gamma_1 - \Gamma_2} \\
    &y = (a + x)\Gamma_2 - b = (a + x)\Gamma_2 - a\Gamma_1 = a(\Gamma_2 - \Gamma_1) + x\Gamma_2 = -x\Gamma_2(1 + \Gamma_1) + x\Gamma_2 = -x\Gamma_2\Gamma_1 = 256\,\text{см}, \\
    &\text{знаки разные, т.е.
    экран надо было подвинуть в другую сторону чем предмет: $x < 0, y > 0$.}
    \end{align*}
}
\solutionspace{120pt}

\tasknumber{10}%
\task{%
    Тонкая собирающая линза дает изображение предмета на экране при двух положениях линзы между предметом и экраном.
    Высота изображения при первом положении $15\,\text{см}$, во втором — $5\,\text{см}$.
    Расстояние между предметом и экранов постоянно.
    Чему равна высота предмета?
}
\answer{%
    \begin{align*}
    &\frac 1a + \frac 1b = \frac 1F, \frac 1c + \frac 1d = \frac 1F, a + b = c + d \implies \frac{a + b}{ab} = \frac 1F = \frac{c+d}{cd} \implies ab = cd, \\
    &\implies ab = c(a + b - c) \implies c^2 - ac - bc + ab = 0 \implies c = a \text{ или } c = b \implies c = b \implies d = a.
    \\
    &\Gamma_1 = \frac {H_1}H = \frac ba, \Gamma_2 = \frac {H_2}H = \frac dc = \frac ab \implies \frac {H_1}H \cdot \frac {H_2}H = \frac ba \cdot \frac ab = 1, \\
    &H = \sqrt{H_1 H_2} \approx 8{,}7\,\text{см}.
    \end{align*}
}
\solutionspace{120pt}

\tasknumber{11}%
\task{%
    Какие предметы можно рассмотреть на фотографии, сделанной со спутника,
    если разрешающая способность плёнки $0{,}010\,\text{мм}$? Каким должно быть
    время экспозиции $\tau$ чтобы полностью использовать возможности плёнки?
    Фокусное расстояние объектива используемого фотоаппарата $20\,\text{см}$,
    высота орбиты спутника $100\,\text{км}$.
}
\answer{%
    \begin{align*}
    &H \ll R \implies v = v_{\text{I}} = \sqrt{G R} \approx 7{,}9\,\frac{\text{км}}{\text{с}}.
    \\
    &F \ll H \implies b = F, a = H, \\
    &\Gamma = \frac \delta\ell = \frac ba \implies \ell = \frac{\delta a}b = \frac{\delta H}F \approx \frac{0{,}010\,\text{мм} \cdot 100\,\text{км}}{20\,\text{см}} \approx 5\,\text{м}, \\
    &\implies \tau = \frac \ell v = \frac{\delta H}{F v} = \frac{0{,}010\,\text{мм} \cdot 100\,\text{км}}{20\,\text{см} \cdot 7{,}9\,\frac{\text{км}}{\text{с}}} \approx 0{,}6\,\text{мс}.
    \end{align*}
}


\variantsplitter


\addpersonalvariant{Ирина Ан}

\tasknumber{12}%
\task{%
    При аэрофотосъемках используется фотоаппарат, объектив которого
    имеет фокусиое расстояние $10\,\text{см}$.
    Разрешающая способность плёнки $0{,}015\,\text{мм}$.
    На какой высоте должен лететь самолет, чтобы на фотографии можно
    было различить следы размером $15\,\text{см}$?
    При какой скорости самолета изображение не будет размытым,
    если время экспозиции $2\,\text{мс}$?
}
\answer{%
    \begin{align*}
    &F \ll H \implies b = F, a = H, \\
    &\Gamma = \frac \delta\ell = \frac ba = \frac FH \implies H = \frac{\ell F}\delta = \frac{15\,\text{см} \cdot 10\,\text{см}}{0{,}015\,\text{мм}} \approx 1\,\text{км}, \\
    &\implies v = \frac l\tau = \frac{15\,\text{см}}{2\,\text{мс}} \approx 270\,\frac{\text{км}}{\text{ч}}.
    \end{align*}
}
\solutionspace{120pt}

\tasknumber{13}%
\task{%
    Две одинаковые собиращие линзы установлены так, что их главные оптические оси совпадают,
    а главный фокус первой находится там же, где главный фокус второй.
    Расстояние от первой линзы до предмета равно $21\,\text{см}$.
    Чему равно расстояние от изображения объекта во второй линзе до второй линзы?
    Определите также увеличение.
    Фокусное расстояние каждой линзы $20\,\text{см}$.
}
\answer{%
    \begin{align*}
    \frac 1a + \frac 1b &= \frac 1F \implies b = \frac{aF}{a - F} \implies 2F - b = \frac{2aF - 2F^2 - aF}{a - F} = \frac{F(a - 2F)}{a - F}.
    \\
    \frac 1{2F - b} + \frac 1c &= \frac 1F \implies c = \frac{F(2F-b)}{(2F - b) - F} = \frac{F \cdot \frac{F(a - 2F)}{a - F}}{\frac{F(a - 2F)}{a - F} - F}  = F \cdot \frac{ \frac{F(a - 2F)}{a - F} }{ \frac{F(a - 2F)}{a - F} - 1} = \\
     &= F \cdot \frac{a - 2F}{a - 2F - a + F} = 2F - a = 19\,\text{см}.
     \\
    \ell &= a + 2F + c = 4F = 80\,\text{см}.
    \\
    &\Gamma = \Gamma_1 \cdot \Gamma_2 = \frac ba \cdot \frac c{2F-b} = \frac F{a - F} \cdot \frac{2F - a}{\frac{F(a - 2F)}{a - F}} = -1.
    \end{align*}
}
\solutionspace{120pt}

\tasknumber{14}%
\task{%
    Собирающая линза с фокусным расстоянием $F_1 > 0$ и рассеивающая линза с фокусным расстоянием $F_2 < 0$
    установлены коаксиально на расстоянии $\ell$.
    Пучок параллельных лучей падает на рассеивающую линзу.
    Сделайте схематичное построение и определите, в какой точке система из этих линз соберёт пучок.
}
\answer{%
    \begin{align*}
    &\text{Если пучок падает на собирающую линзу:} \\
    \frac 1{\infty} + \frac 1b &= \frac 1{F_1} \implies b = F_1 \implies \ell - b = \ell - F_1 \\
    \frac 1{\ell - b} + \frac 1c &= \frac 1{F_2} \implies c = \frac{F_2(\ell - b)}{\ell - b - F_2} = \frac{F_2(\ell - F_1)}{\ell - F_1 - F_2}.
    \\
    &\text{Если же пучок падает на рассеивающую линзу:} \\
    \frac 1{\infty} + \frac 1b &= \frac 1{F_2} \implies b = F_2 \implies \ell - b = \ell - F_2 \\
    \frac 1{\ell - b} + \frac 1c &= \frac 1{F_1} \implies c = \frac{F_1(\ell - b)}{\ell - b - F_1} = \frac{F_1(\ell - F_2)}{\ell - F_2 - F_1}.
    \end{align*}
}
\solutionspace{120pt}

\tasknumber{15}%
\task{%
    Две собирающих линзы с фокусными расстояниями $20\,\text{см}$ и $45\,\text{см}$ расположены так,
    что их оптические оси совмещены.
    На первую линзу падает пучок параллельных лучей.
    Пройдя через вторую линзу, он остался параллельным.
    Найдите расстояние между линзами и сделайте рисунок.
}
\answer{%
    \begin{align*}
    \frac 1\infty + \frac 1b &= \frac 1{F_1} \implies b = F_1, \\
    \frac 1{\ell - b} + \frac 1{\infty} &= \frac 1{F_2} \implies \ell - b = F_2 \implies \ell = b + F_2 = F_1 + F_2 = 65\,\text{см}.
    \end{align*}
}

\variantsplitter

\addpersonalvariant{Софья Андрианова}

\tasknumber{1}%
\task{%
    Найти оптическую силу собирающей линзы, если действительное изображение предмета,
    помещённого в $15\,\text{см}$ от линзы, получается на расстоянии $20\,\text{см}$ от неё.
}
\answer{%
    $D = \frac 1F = \frac 1a + \frac 1b = \frac 1{15\,\text{см}} + \frac 1{20\,\text{см}} \approx 11{,}67\,\text{дптр}$
}
\solutionspace{80pt}

\tasknumber{2}%
\task{%
    Найти увеличение изображения, если изображение предмета, находящегося
    на расстоянии $15\,\text{см}$ от линзы, получается на расстоянии $18\,\text{см}$ от неё.
}
\answer{%
    $\Gamma = \frac ba = \frac {18\,\text{см}}{15\,\text{см}} \approx 1{,}2$
}
\solutionspace{80pt}

\tasknumber{3}%
\task{%
    Расстояние от предмета до линзы $8\,\text{см}$, а от линзы до мнимого изображения $25\,\text{см}$.
    Чему равно фокусное расстояние линзы?
}
\answer{%
    $\pm \frac 1F = \frac 1a - \frac 1b \implies F = \frac{a b}{\abs{b - a}} \approx 11{,}8\,\text{см}$
}
\solutionspace{80pt}

\tasknumber{4}%
\task{%
    Две тонкие собирающие линзы с фокусными расстояниями $25\,\text{см}$ и $20\,\text{см}$ сложены вместе.
    Чему равно фокусное расстояние такой оптической системы?
}
\answer{%
    $\frac 1{f_1} = \frac 1a + \frac 1b; \frac 1{f_2} = - \frac 1b + \frac 1c \implies \frac 1{f_1} + \frac 1{f_2} = \frac 1a + \frac 1c \implies f' = \frac 1{\frac 1{f_1} + \frac 1{f_2}} = \frac{f_1 f_2}{f_1 + f_2} \approx 11{,}1\,\text{см}$
}
\solutionspace{80pt}

\tasknumber{5}%
\task{%
    Линейные размеры прямого изображения предмета, полученного в собирающей линзе,
    в три раза больше линейных размеров предмета.
    Зная, что предмет находится на $30\,\text{см}$ ближе к линзе,
    чем его изображение, найти оптическую силу линзы.
}
\answer{%
    \begin{align*}
    &\text{Если изображение действительное:} \\
    D &= \frac 1F = \frac 1a + \frac 1b, \qquad \Gamma = \frac ba, \qquad b - a = \ell \implies b = \Gamma a \implies \Gamma a - a = \ell \implies  \\
    a &= \frac {\ell}{\Gamma - 1} \implies b = \frac {{\ell} \Gamma}{\Gamma - 1} \implies  \\
    D &= \frac {\Gamma - 1}\ell + \frac {\Gamma - 1}{\ell \Gamma} = \frac 1\ell \cdot \cbr{\Gamma - 1 + \frac {\Gamma - 1}{\Gamma} } =\frac 1\ell \cdot \cbr{\Gamma - \frac 1\Gamma} \approx 8{,}9\,\text{дптр}.
    \\
    &\text{Если изображение мнимое:} \\
    D &= \frac 1F = \frac 1a - \frac 1b, \qquad \Gamma = \frac ba, \qquad b - a = \ell \implies b = \Gamma a \implies \Gamma a - a = \ell \implies  \\
    a &= \frac {\ell}{\Gamma - 1} \implies b = \frac {{\ell} \Gamma}{\Gamma - 1} \implies  \\
    D &= \frac {\Gamma - 1}\ell - \frac {\Gamma - 1}{\ell \Gamma} = \frac 1\ell \cdot \cbr{\Gamma - 1 - \frac {\Gamma - 1}{\Gamma} } =\frac 1\ell \cdot \cbr{\Gamma + \frac 1\Gamma - 2} \approx 4{,}4\,\text{дптр}.
    \\
    &\text{В ответе надо указать оба значения.}
    \end{align*}
}
\solutionspace{120pt}

\tasknumber{6}%
\task{%
    Оптическая сила объектива фотоаппарата равна $3\,\text{дптр}$.
    При фотографировании чертежа с расстояния $0{,}8\,\text{м}$ площадь изображения
    чертежа на фотопластинке оказалась равной $16\,\text{см}^{2}$.
    Какова площадь самого чертежа? Ответ выразите в квадратных сантиметрах.
}
\answer{%
    \begin{align*}
    &\frac 1a + \frac 1b = \frac 1F = D \implies b = \frac{aF}{a - F} \\
    &\frac {S'}S = \Gamma^2 = \sqr{\frac ba} = \sqr{\frac F{a - F}} \implies \\
    &\implies S = S' \cdot \sqr{\frac{a - F}F} = S' \cdot \sqr{\frac aF - 1} = S' \cdot \sqr{aD - 1} \approx 30\,\text{см}^{2}.
    \end{align*}
}


\variantsplitter


\addpersonalvariant{Софья Андрианова}

\tasknumber{7}%
\task{%
    В каком месте на главной оптической оси двояковыпуклой линзы
    нужно поместить точечный источник света,
    чтобы его изображение оказалось в главном фокусе линзы?
}
\answer{%
    $\text{для мнимого - на половине фокусного, для действительного - на бесконечности}$
}
\solutionspace{120pt}

\tasknumber{8}%
\task{%
    Предмет в виде отрезка длиной $\ell$ расположен вдоль оптической оси
    собирающей линзы с фокусным расстоянием $F$.
    Середина отрезка расположена
    на расстоянии $a$ от линзы, которая даёт действительное изображение
    всех точек предмета.
    Определить продольное увеличение предмета.
}
\answer{%
    \begin{align*}
    \frac 1{a + \frac \ell 2} &+ \frac 1b = \frac 1F \implies b = \frac{F\cbr{a + \frac \ell 2}}{a + \frac \ell 2 - F} \\
    \frac 1{a - \frac \ell 2} &+ \frac 1c = \frac 1F \implies c = \frac{F\cbr{a - \frac \ell 2}}{a - \frac \ell 2 - F} \\
    \abs{b - c} &= \abs{\frac{F\cbr{a + \frac \ell 2}}{a + \frac \ell 2 - F} - \frac{F\cbr{a - \frac \ell 2}}{a - \frac \ell 2 - F}}= F\abs{\frac{\cbr{a + \frac \ell 2}\cbr{a - \frac \ell 2 - F} - \cbr{a - \frac \ell 2}\cbr{a + \frac \ell 2 - F}}{ \cbr{a + \frac \ell 2 - F} \cbr{a - \frac \ell 2 - F} }} =  \\
    &= F\abs{\frac{a^2 - \frac {a\ell} 2 - Fa + \frac {a\ell} 2 - \frac {\ell^2} 4 - \frac {F\ell}2 - a^2 - \frac {a\ell}2 + aF + \frac {a\ell}2 + \frac {\ell^2} 4 - \frac {F\ell} 2}{\cbr{a + \frac \ell 2 - F} \cbr{a - \frac \ell 2 - F} }} = \\
    &= F\frac{F\ell}{\sqr{a-F} - \frac {\ell^2}4} = \frac{F^2\ell}{\sqr{a-F} - \frac {\ell^2}4}\implies \Gamma = \frac{\abs{b - c}}\ell = \frac{F^2}{\sqr{a-F} - \frac {\ell^2}4}.
    \end{align*}
}
\solutionspace{120pt}

\tasknumber{9}%
\task{%
    На экране с помощью тонкой линзы получено изображение предмета
    с увеличением $4$.
    Предмет передвинули на $4\,\text{см}$.
    Для того, чтобы получить резкое изображение, пришлось передвинуть экран.
    При этом увеличение оказалось равным $8$.
    На какое расстояние
    пришлось передвинуть экран?
}
\answer{%
    \begin{align*}
    &\frac 1a + \frac 1b = \frac 1F, \Gamma_1 = \frac ba = \frac{F}{a-F} \implies \Gamma_1(a-F) = F \implies a = F \cdot \frac{1 + \Gamma_1}{\Gamma_1} \\
    &\frac 1{a + x} + \frac 1{b + y} = \frac 1F, \Gamma_2 = \frac {b+y}{a+x} = \frac{F}{a+x-F} \implies a + x = F \cdot \frac{1 + \Gamma_2}{\Gamma_2} \\
    &1 + \frac xa = \frac{ \frac{1 + \Gamma_2}{\Gamma_2} }{ \frac{1 + \Gamma_1}{\Gamma_1} } = \frac{\Gamma_1(1 + \Gamma_2)}{\Gamma_2(1 + \Gamma_1)} \\
    &a = \frac x{ \frac{\Gamma_1(1 + \Gamma_2)}{\Gamma_2(1 + \Gamma_1)} - 1} = x \cdot \frac{\Gamma_2(1 + \Gamma_1)}{\Gamma_1 - \Gamma_2} \\
    &y = (a + x)\Gamma_2 - b = (a + x)\Gamma_2 - a\Gamma_1 = a(\Gamma_2 - \Gamma_1) + x\Gamma_2 = -x\Gamma_2(1 + \Gamma_1) + x\Gamma_2 = -x\Gamma_2\Gamma_1 = 128\,\text{см}, \\
    &\text{знаки разные, т.е.
    экран надо было подвинуть в другую сторону чем предмет: $x < 0, y > 0$.}
    \end{align*}
}
\solutionspace{120pt}

\tasknumber{10}%
\task{%
    Тонкая собирающая линза дает изображение предмета на экране при двух положениях линзы между предметом и экраном.
    Высота изображения при первом положении $15\,\text{см}$, во втором — $7\,\text{см}$.
    Расстояние между предметом и экранов постоянно.
    Чему равна высота предмета?
}
\answer{%
    \begin{align*}
    &\frac 1a + \frac 1b = \frac 1F, \frac 1c + \frac 1d = \frac 1F, a + b = c + d \implies \frac{a + b}{ab} = \frac 1F = \frac{c+d}{cd} \implies ab = cd, \\
    &\implies ab = c(a + b - c) \implies c^2 - ac - bc + ab = 0 \implies c = a \text{ или } c = b \implies c = b \implies d = a.
    \\
    &\Gamma_1 = \frac {H_1}H = \frac ba, \Gamma_2 = \frac {H_2}H = \frac dc = \frac ab \implies \frac {H_1}H \cdot \frac {H_2}H = \frac ba \cdot \frac ab = 1, \\
    &H = \sqrt{H_1 H_2} \approx 10{,}2\,\text{см}.
    \end{align*}
}
\solutionspace{120pt}

\tasknumber{11}%
\task{%
    Какие предметы можно рассмотреть на фотографии, сделанной со спутника,
    если разрешающая способность плёнки $0{,}02\,\text{мм}$? Каким должно быть
    время экспозиции $\tau$ чтобы полностью использовать возможности плёнки?
    Фокусное расстояние объектива используемого фотоаппарата $10\,\text{см}$,
    высота орбиты спутника $80\,\text{км}$.
}
\answer{%
    \begin{align*}
    &H \ll R \implies v = v_{\text{I}} = \sqrt{G R} \approx 7{,}9\,\frac{\text{км}}{\text{с}}.
    \\
    &F \ll H \implies b = F, a = H, \\
    &\Gamma = \frac \delta\ell = \frac ba \implies \ell = \frac{\delta a}b = \frac{\delta H}F \approx \frac{0{,}02\,\text{мм} \cdot 80\,\text{км}}{10\,\text{см}} \approx 16\,\text{м}, \\
    &\implies \tau = \frac \ell v = \frac{\delta H}{F v} = \frac{0{,}02\,\text{мм} \cdot 80\,\text{км}}{10\,\text{см} \cdot 7{,}9\,\frac{\text{км}}{\text{с}}} \approx 2\,\text{мс}.
    \end{align*}
}


\variantsplitter


\addpersonalvariant{Софья Андрианова}

\tasknumber{12}%
\task{%
    При аэрофотосъемках используется фотоаппарат, объектив которого
    имеет фокусиое расстояние $10\,\text{см}$.
    Разрешающая способность плёнки $0{,}010\,\text{мм}$.
    На какой высоте должен лететь самолет, чтобы на фотографии можно
    было различить следы размером $20\,\text{см}$?
    При какой скорости самолета изображение не будет размытым,
    если время экспозиции $1\,\text{мс}$?
}
\answer{%
    \begin{align*}
    &F \ll H \implies b = F, a = H, \\
    &\Gamma = \frac \delta\ell = \frac ba = \frac FH \implies H = \frac{\ell F}\delta = \frac{20\,\text{см} \cdot 10\,\text{см}}{0{,}010\,\text{мм}} \approx 2\,\text{км}, \\
    &\implies v = \frac l\tau = \frac{20\,\text{см}}{1\,\text{мс}} \approx 700\,\frac{\text{км}}{\text{ч}}.
    \end{align*}
}
\solutionspace{120pt}

\tasknumber{13}%
\task{%
    Две одинаковые собиращие линзы установлены так, что их главные оптические оси совпадают,
    а главный фокус первой находится там же, где главный фокус второй.
    Расстояние от первой линзы до предмета равно $18\,\text{см}$.
    Чему равно расстояние от изображения объекта во второй линзе до второй линзы?
    Определите также увеличение.
    Фокусное расстояние каждой линзы $30\,\text{см}$.
}
\answer{%
    \begin{align*}
    \frac 1a + \frac 1b &= \frac 1F \implies b = \frac{aF}{a - F} \implies 2F - b = \frac{2aF - 2F^2 - aF}{a - F} = \frac{F(a - 2F)}{a - F}.
    \\
    \frac 1{2F - b} + \frac 1c &= \frac 1F \implies c = \frac{F(2F-b)}{(2F - b) - F} = \frac{F \cdot \frac{F(a - 2F)}{a - F}}{\frac{F(a - 2F)}{a - F} - F}  = F \cdot \frac{ \frac{F(a - 2F)}{a - F} }{ \frac{F(a - 2F)}{a - F} - 1} = \\
     &= F \cdot \frac{a - 2F}{a - 2F - a + F} = 2F - a = 42\,\text{см}.
     \\
    \ell &= a + 2F + c = 4F = 120\,\text{см}.
    \\
    &\Gamma = \Gamma_1 \cdot \Gamma_2 = \frac ba \cdot \frac c{2F-b} = \frac F{a - F} \cdot \frac{2F - a}{\frac{F(a - 2F)}{a - F}} = -1.
    \end{align*}
}
\solutionspace{120pt}

\tasknumber{14}%
\task{%
    Собирающая линза с фокусным расстоянием $F_1 > 0$ и рассеивающая линза с фокусным расстоянием $F_2 < 0$
    установлены коаксиально на расстоянии $\ell$.
    Пучок параллельных лучей падает на собирающую линзу.
    Сделайте схематичное построение и определите, в какой точке система из этих линз соберёт пучок.
}
\answer{%
    \begin{align*}
    &\text{Если пучок падает на собирающую линзу:} \\
    \frac 1{\infty} + \frac 1b &= \frac 1{F_1} \implies b = F_1 \implies \ell - b = \ell - F_1 \\
    \frac 1{\ell - b} + \frac 1c &= \frac 1{F_2} \implies c = \frac{F_2(\ell - b)}{\ell - b - F_2} = \frac{F_2(\ell - F_1)}{\ell - F_1 - F_2}.
    \\
    &\text{Если же пучок падает на рассеивающую линзу:} \\
    \frac 1{\infty} + \frac 1b &= \frac 1{F_2} \implies b = F_2 \implies \ell - b = \ell - F_2 \\
    \frac 1{\ell - b} + \frac 1c &= \frac 1{F_1} \implies c = \frac{F_1(\ell - b)}{\ell - b - F_1} = \frac{F_1(\ell - F_2)}{\ell - F_2 - F_1}.
    \end{align*}
}
\solutionspace{120pt}

\tasknumber{15}%
\task{%
    Две собирающих линзы с фокусными расстояниями $50\,\text{см}$ и $45\,\text{см}$ расположены так,
    что их оптические оси совмещены.
    На первую линзу падает пучок параллельных лучей.
    Пройдя через вторую линзу, он остался параллельным.
    Найдите расстояние между линзами и сделайте рисунок.
}
\answer{%
    \begin{align*}
    \frac 1\infty + \frac 1b &= \frac 1{F_1} \implies b = F_1, \\
    \frac 1{\ell - b} + \frac 1{\infty} &= \frac 1{F_2} \implies \ell - b = F_2 \implies \ell = b + F_2 = F_1 + F_2 = 95\,\text{см}.
    \end{align*}
}

\variantsplitter

\addpersonalvariant{Владимир Артемчук}

\tasknumber{1}%
\task{%
    Найти оптическую силу собирающей линзы, если действительное изображение предмета,
    помещённого в $15\,\text{см}$ от линзы, получается на расстоянии $40\,\text{см}$ от неё.
}
\answer{%
    $D = \frac 1F = \frac 1a + \frac 1b = \frac 1{15\,\text{см}} + \frac 1{40\,\text{см}} \approx 9{,}17\,\text{дптр}$
}
\solutionspace{80pt}

\tasknumber{2}%
\task{%
    Найти увеличение изображения, если изображение предмета, находящегося
    на расстоянии $25\,\text{см}$ от линзы, получается на расстоянии $30\,\text{см}$ от неё.
}
\answer{%
    $\Gamma = \frac ba = \frac {30\,\text{см}}{25\,\text{см}} \approx 1{,}20$
}
\solutionspace{80pt}

\tasknumber{3}%
\task{%
    Расстояние от предмета до линзы $12\,\text{см}$, а от линзы до мнимого изображения $30\,\text{см}$.
    Чему равно фокусное расстояние линзы?
}
\answer{%
    $\pm \frac 1F = \frac 1a - \frac 1b \implies F = \frac{a b}{\abs{b - a}} \approx 20\,\text{см}$
}
\solutionspace{80pt}

\tasknumber{4}%
\task{%
    Две тонкие собирающие линзы с фокусными расстояниями $12\,\text{см}$ и $30\,\text{см}$ сложены вместе.
    Чему равно фокусное расстояние такой оптической системы?
}
\answer{%
    $\frac 1{f_1} = \frac 1a + \frac 1b; \frac 1{f_2} = - \frac 1b + \frac 1c \implies \frac 1{f_1} + \frac 1{f_2} = \frac 1a + \frac 1c \implies f' = \frac 1{\frac 1{f_1} + \frac 1{f_2}} = \frac{f_1 f_2}{f_1 + f_2} \approx 8{,}6\,\text{см}$
}
\solutionspace{80pt}

\tasknumber{5}%
\task{%
    Линейные размеры прямого изображения предмета, полученного в собирающей линзе,
    в четыре раза больше линейных размеров предмета.
    Зная, что предмет находится на $20\,\text{см}$ ближе к линзе,
    чем его изображение, найти оптическую силу линзы.
}
\answer{%
    \begin{align*}
    &\text{Если изображение действительное:} \\
    D &= \frac 1F = \frac 1a + \frac 1b, \qquad \Gamma = \frac ba, \qquad b - a = \ell \implies b = \Gamma a \implies \Gamma a - a = \ell \implies  \\
    a &= \frac {\ell}{\Gamma - 1} \implies b = \frac {{\ell} \Gamma}{\Gamma - 1} \implies  \\
    D &= \frac {\Gamma - 1}\ell + \frac {\Gamma - 1}{\ell \Gamma} = \frac 1\ell \cdot \cbr{\Gamma - 1 + \frac {\Gamma - 1}{\Gamma} } =\frac 1\ell \cdot \cbr{\Gamma - \frac 1\Gamma} \approx 18{,}8\,\text{дптр}.
    \\
    &\text{Если изображение мнимое:} \\
    D &= \frac 1F = \frac 1a - \frac 1b, \qquad \Gamma = \frac ba, \qquad b - a = \ell \implies b = \Gamma a \implies \Gamma a - a = \ell \implies  \\
    a &= \frac {\ell}{\Gamma - 1} \implies b = \frac {{\ell} \Gamma}{\Gamma - 1} \implies  \\
    D &= \frac {\Gamma - 1}\ell - \frac {\Gamma - 1}{\ell \Gamma} = \frac 1\ell \cdot \cbr{\Gamma - 1 - \frac {\Gamma - 1}{\Gamma} } =\frac 1\ell \cdot \cbr{\Gamma + \frac 1\Gamma - 2} \approx 11{,}2\,\text{дптр}.
    \\
    &\text{В ответе надо указать оба значения.}
    \end{align*}
}
\solutionspace{120pt}

\tasknumber{6}%
\task{%
    Оптическая сила объектива фотоаппарата равна $6\,\text{дптр}$.
    При фотографировании чертежа с расстояния $1{,}2\,\text{м}$ площадь изображения
    чертежа на фотопластинке оказалась равной $9\,\text{см}^{2}$.
    Какова площадь самого чертежа? Ответ выразите в квадратных сантиметрах.
}
\answer{%
    \begin{align*}
    &\frac 1a + \frac 1b = \frac 1F = D \implies b = \frac{aF}{a - F} \\
    &\frac {S'}S = \Gamma^2 = \sqr{\frac ba} = \sqr{\frac F{a - F}} \implies \\
    &\implies S = S' \cdot \sqr{\frac{a - F}F} = S' \cdot \sqr{\frac aF - 1} = S' \cdot \sqr{aD - 1} \approx 300\,\text{см}^{2}.
    \end{align*}
}


\variantsplitter


\addpersonalvariant{Владимир Артемчук}

\tasknumber{7}%
\task{%
    В каком месте на главной оптической оси двояковыпуклой линзы
    нужно поместить точечный источник света,
    чтобы его изображение оказалось в главном фокусе линзы?
}
\answer{%
    $\text{для мнимого - на половине фокусного, для действительного - на бесконечности}$
}
\solutionspace{120pt}

\tasknumber{8}%
\task{%
    Предмет в виде отрезка длиной $\ell$ расположен вдоль оптической оси
    собирающей линзы с фокусным расстоянием $F$.
    Середина отрезка расположена
    на расстоянии $a$ от линзы, которая даёт действительное изображение
    всех точек предмета.
    Определить продольное увеличение предмета.
}
\answer{%
    \begin{align*}
    \frac 1{a + \frac \ell 2} &+ \frac 1b = \frac 1F \implies b = \frac{F\cbr{a + \frac \ell 2}}{a + \frac \ell 2 - F} \\
    \frac 1{a - \frac \ell 2} &+ \frac 1c = \frac 1F \implies c = \frac{F\cbr{a - \frac \ell 2}}{a - \frac \ell 2 - F} \\
    \abs{b - c} &= \abs{\frac{F\cbr{a + \frac \ell 2}}{a + \frac \ell 2 - F} - \frac{F\cbr{a - \frac \ell 2}}{a - \frac \ell 2 - F}}= F\abs{\frac{\cbr{a + \frac \ell 2}\cbr{a - \frac \ell 2 - F} - \cbr{a - \frac \ell 2}\cbr{a + \frac \ell 2 - F}}{ \cbr{a + \frac \ell 2 - F} \cbr{a - \frac \ell 2 - F} }} =  \\
    &= F\abs{\frac{a^2 - \frac {a\ell} 2 - Fa + \frac {a\ell} 2 - \frac {\ell^2} 4 - \frac {F\ell}2 - a^2 - \frac {a\ell}2 + aF + \frac {a\ell}2 + \frac {\ell^2} 4 - \frac {F\ell} 2}{\cbr{a + \frac \ell 2 - F} \cbr{a - \frac \ell 2 - F} }} = \\
    &= F\frac{F\ell}{\sqr{a-F} - \frac {\ell^2}4} = \frac{F^2\ell}{\sqr{a-F} - \frac {\ell^2}4}\implies \Gamma = \frac{\abs{b - c}}\ell = \frac{F^2}{\sqr{a-F} - \frac {\ell^2}4}.
    \end{align*}
}
\solutionspace{120pt}

\tasknumber{9}%
\task{%
    На экране с помощью тонкой линзы получено изображение предмета
    с увеличением $2$.
    Предмет передвинули на $4\,\text{см}$.
    Для того, чтобы получить резкое изображение, пришлось передвинуть экран.
    При этом увеличение оказалось равным $8$.
    На какое расстояние
    пришлось передвинуть экран?
}
\answer{%
    \begin{align*}
    &\frac 1a + \frac 1b = \frac 1F, \Gamma_1 = \frac ba = \frac{F}{a-F} \implies \Gamma_1(a-F) = F \implies a = F \cdot \frac{1 + \Gamma_1}{\Gamma_1} \\
    &\frac 1{a + x} + \frac 1{b + y} = \frac 1F, \Gamma_2 = \frac {b+y}{a+x} = \frac{F}{a+x-F} \implies a + x = F \cdot \frac{1 + \Gamma_2}{\Gamma_2} \\
    &1 + \frac xa = \frac{ \frac{1 + \Gamma_2}{\Gamma_2} }{ \frac{1 + \Gamma_1}{\Gamma_1} } = \frac{\Gamma_1(1 + \Gamma_2)}{\Gamma_2(1 + \Gamma_1)} \\
    &a = \frac x{ \frac{\Gamma_1(1 + \Gamma_2)}{\Gamma_2(1 + \Gamma_1)} - 1} = x \cdot \frac{\Gamma_2(1 + \Gamma_1)}{\Gamma_1 - \Gamma_2} \\
    &y = (a + x)\Gamma_2 - b = (a + x)\Gamma_2 - a\Gamma_1 = a(\Gamma_2 - \Gamma_1) + x\Gamma_2 = -x\Gamma_2(1 + \Gamma_1) + x\Gamma_2 = -x\Gamma_2\Gamma_1 = 64\,\text{см}, \\
    &\text{знаки разные, т.е.
    экран надо было подвинуть в другую сторону чем предмет: $x < 0, y > 0$.}
    \end{align*}
}
\solutionspace{120pt}

\tasknumber{10}%
\task{%
    Тонкая собирающая линза дает изображение предмета на экране при двух положениях линзы между предметом и экраном.
    Высота изображения при первом положении $25\,\text{см}$, во втором — $9\,\text{см}$.
    Расстояние между предметом и экранов постоянно.
    Чему равна высота предмета?
}
\answer{%
    \begin{align*}
    &\frac 1a + \frac 1b = \frac 1F, \frac 1c + \frac 1d = \frac 1F, a + b = c + d \implies \frac{a + b}{ab} = \frac 1F = \frac{c+d}{cd} \implies ab = cd, \\
    &\implies ab = c(a + b - c) \implies c^2 - ac - bc + ab = 0 \implies c = a \text{ или } c = b \implies c = b \implies d = a.
    \\
    &\Gamma_1 = \frac {H_1}H = \frac ba, \Gamma_2 = \frac {H_2}H = \frac dc = \frac ab \implies \frac {H_1}H \cdot \frac {H_2}H = \frac ba \cdot \frac ab = 1, \\
    &H = \sqrt{H_1 H_2} \approx 15\,\text{см}.
    \end{align*}
}
\solutionspace{120pt}

\tasknumber{11}%
\task{%
    Какие предметы можно рассмотреть на фотографии, сделанной со спутника,
    если разрешающая способность плёнки $0{,}010\,\text{мм}$? Каким должно быть
    время экспозиции $\tau$ чтобы полностью использовать возможности плёнки?
    Фокусное расстояние объектива используемого фотоаппарата $15\,\text{см}$,
    высота орбиты спутника $100\,\text{км}$.
}
\answer{%
    \begin{align*}
    &H \ll R \implies v = v_{\text{I}} = \sqrt{G R} \approx 7{,}9\,\frac{\text{км}}{\text{с}}.
    \\
    &F \ll H \implies b = F, a = H, \\
    &\Gamma = \frac \delta\ell = \frac ba \implies \ell = \frac{\delta a}b = \frac{\delta H}F \approx \frac{0{,}010\,\text{мм} \cdot 100\,\text{км}}{15\,\text{см}} \approx 7\,\text{м}, \\
    &\implies \tau = \frac \ell v = \frac{\delta H}{F v} = \frac{0{,}010\,\text{мм} \cdot 100\,\text{км}}{15\,\text{см} \cdot 7{,}9\,\frac{\text{км}}{\text{с}}} \approx 0{,}8\,\text{мс}.
    \end{align*}
}


\variantsplitter


\addpersonalvariant{Владимир Артемчук}

\tasknumber{12}%
\task{%
    При аэрофотосъемках используется фотоаппарат, объектив которого
    имеет фокусиое расстояние $15\,\text{см}$.
    Разрешающая способность плёнки $0{,}010\,\text{мм}$.
    На какой высоте должен лететь самолет, чтобы на фотографии можно
    было различить следы размером $30\,\text{см}$?
    При какой скорости самолета изображение не будет размытым,
    если время экспозиции $1\,\text{мс}$?
}
\answer{%
    \begin{align*}
    &F \ll H \implies b = F, a = H, \\
    &\Gamma = \frac \delta\ell = \frac ba = \frac FH \implies H = \frac{\ell F}\delta = \frac{30\,\text{см} \cdot 15\,\text{см}}{0{,}010\,\text{мм}} \approx 5\,\text{км}, \\
    &\implies v = \frac l\tau = \frac{30\,\text{см}}{1\,\text{мс}} \approx 1080\,\frac{\text{км}}{\text{ч}}.
    \end{align*}
}
\solutionspace{120pt}

\tasknumber{13}%
\task{%
    Две одинаковые собиращие линзы установлены так, что их главные оптические оси совпадают,
    а главный фокус первой находится там же, где главный фокус второй.
    Расстояние от первой линзы до предмета равно $15\,\text{см}$.
    Чему равно расстояние от изображения объекта во второй линзе до второй линзы?
    Определите также увеличение.
    Фокусное расстояние каждой линзы $40\,\text{см}$.
}
\answer{%
    \begin{align*}
    \frac 1a + \frac 1b &= \frac 1F \implies b = \frac{aF}{a - F} \implies 2F - b = \frac{2aF - 2F^2 - aF}{a - F} = \frac{F(a - 2F)}{a - F}.
    \\
    \frac 1{2F - b} + \frac 1c &= \frac 1F \implies c = \frac{F(2F-b)}{(2F - b) - F} = \frac{F \cdot \frac{F(a - 2F)}{a - F}}{\frac{F(a - 2F)}{a - F} - F}  = F \cdot \frac{ \frac{F(a - 2F)}{a - F} }{ \frac{F(a - 2F)}{a - F} - 1} = \\
     &= F \cdot \frac{a - 2F}{a - 2F - a + F} = 2F - a = 65\,\text{см}.
     \\
    \ell &= a + 2F + c = 4F = 160\,\text{см}.
    \\
    &\Gamma = \Gamma_1 \cdot \Gamma_2 = \frac ba \cdot \frac c{2F-b} = \frac F{a - F} \cdot \frac{2F - a}{\frac{F(a - 2F)}{a - F}} = -1.
    \end{align*}
}
\solutionspace{120pt}

\tasknumber{14}%
\task{%
    Собирающая линза с фокусным расстоянием $F_1 > 0$ и рассеивающая линза с фокусным расстоянием $F_2 < 0$
    установлены коаксиально на расстоянии $\ell$.
    Пучок параллельных лучей падает на собирающую линзу.
    Сделайте схематичное построение и определите, в какой точке система из этих линз соберёт пучок.
}
\answer{%
    \begin{align*}
    &\text{Если пучок падает на собирающую линзу:} \\
    \frac 1{\infty} + \frac 1b &= \frac 1{F_1} \implies b = F_1 \implies \ell - b = \ell - F_1 \\
    \frac 1{\ell - b} + \frac 1c &= \frac 1{F_2} \implies c = \frac{F_2(\ell - b)}{\ell - b - F_2} = \frac{F_2(\ell - F_1)}{\ell - F_1 - F_2}.
    \\
    &\text{Если же пучок падает на рассеивающую линзу:} \\
    \frac 1{\infty} + \frac 1b &= \frac 1{F_2} \implies b = F_2 \implies \ell - b = \ell - F_2 \\
    \frac 1{\ell - b} + \frac 1c &= \frac 1{F_1} \implies c = \frac{F_1(\ell - b)}{\ell - b - F_1} = \frac{F_1(\ell - F_2)}{\ell - F_2 - F_1}.
    \end{align*}
}
\solutionspace{120pt}

\tasknumber{15}%
\task{%
    Две собирающих линзы с фокусными расстояниями $40\,\text{см}$ и $45\,\text{см}$ расположены так,
    что их оптические оси совмещены.
    На первую линзу падает пучок параллельных лучей.
    Пройдя через вторую линзу, он остался параллельным.
    Найдите расстояние между линзами и сделайте рисунок.
}
\answer{%
    \begin{align*}
    \frac 1\infty + \frac 1b &= \frac 1{F_1} \implies b = F_1, \\
    \frac 1{\ell - b} + \frac 1{\infty} &= \frac 1{F_2} \implies \ell - b = F_2 \implies \ell = b + F_2 = F_1 + F_2 = 85\,\text{см}.
    \end{align*}
}

\variantsplitter

\addpersonalvariant{Софья Белянкина}

\tasknumber{1}%
\task{%
    Найти оптическую силу собирающей линзы, если действительное изображение предмета,
    помещённого в $35\,\text{см}$ от линзы, получается на расстоянии $40\,\text{см}$ от неё.
}
\answer{%
    $D = \frac 1F = \frac 1a + \frac 1b = \frac 1{35\,\text{см}} + \frac 1{40\,\text{см}} \approx 5{,}36\,\text{дптр}$
}
\solutionspace{80pt}

\tasknumber{2}%
\task{%
    Найти увеличение изображения, если изображение предмета, находящегося
    на расстоянии $20\,\text{см}$ от линзы, получается на расстоянии $30\,\text{см}$ от неё.
}
\answer{%
    $\Gamma = \frac ba = \frac {30\,\text{см}}{20\,\text{см}} \approx 1{,}50$
}
\solutionspace{80pt}

\tasknumber{3}%
\task{%
    Расстояние от предмета до линзы $10\,\text{см}$, а от линзы до мнимого изображения $30\,\text{см}$.
    Чему равно фокусное расстояние линзы?
}
\answer{%
    $\pm \frac 1F = \frac 1a - \frac 1b \implies F = \frac{a b}{\abs{b - a}} \approx 15\,\text{см}$
}
\solutionspace{80pt}

\tasknumber{4}%
\task{%
    Две тонкие собирающие линзы с фокусными расстояниями $12\,\text{см}$ и $30\,\text{см}$ сложены вместе.
    Чему равно фокусное расстояние такой оптической системы?
}
\answer{%
    $\frac 1{f_1} = \frac 1a + \frac 1b; \frac 1{f_2} = - \frac 1b + \frac 1c \implies \frac 1{f_1} + \frac 1{f_2} = \frac 1a + \frac 1c \implies f' = \frac 1{\frac 1{f_1} + \frac 1{f_2}} = \frac{f_1 f_2}{f_1 + f_2} \approx 8{,}6\,\text{см}$
}
\solutionspace{80pt}

\tasknumber{5}%
\task{%
    Линейные размеры прямого изображения предмета, полученного в собирающей линзе,
    в четыре раза больше линейных размеров предмета.
    Зная, что предмет находится на $30\,\text{см}$ ближе к линзе,
    чем его изображение, найти оптическую силу линзы.
}
\answer{%
    \begin{align*}
    &\text{Если изображение действительное:} \\
    D &= \frac 1F = \frac 1a + \frac 1b, \qquad \Gamma = \frac ba, \qquad b - a = \ell \implies b = \Gamma a \implies \Gamma a - a = \ell \implies  \\
    a &= \frac {\ell}{\Gamma - 1} \implies b = \frac {{\ell} \Gamma}{\Gamma - 1} \implies  \\
    D &= \frac {\Gamma - 1}\ell + \frac {\Gamma - 1}{\ell \Gamma} = \frac 1\ell \cdot \cbr{\Gamma - 1 + \frac {\Gamma - 1}{\Gamma} } =\frac 1\ell \cdot \cbr{\Gamma - \frac 1\Gamma} \approx 12{,}5\,\text{дптр}.
    \\
    &\text{Если изображение мнимое:} \\
    D &= \frac 1F = \frac 1a - \frac 1b, \qquad \Gamma = \frac ba, \qquad b - a = \ell \implies b = \Gamma a \implies \Gamma a - a = \ell \implies  \\
    a &= \frac {\ell}{\Gamma - 1} \implies b = \frac {{\ell} \Gamma}{\Gamma - 1} \implies  \\
    D &= \frac {\Gamma - 1}\ell - \frac {\Gamma - 1}{\ell \Gamma} = \frac 1\ell \cdot \cbr{\Gamma - 1 - \frac {\Gamma - 1}{\Gamma} } =\frac 1\ell \cdot \cbr{\Gamma + \frac 1\Gamma - 2} \approx 7{,}5\,\text{дптр}.
    \\
    &\text{В ответе надо указать оба значения.}
    \end{align*}
}
\solutionspace{120pt}

\tasknumber{6}%
\task{%
    Оптическая сила объектива фотоаппарата равна $5\,\text{дптр}$.
    При фотографировании чертежа с расстояния $0{,}8\,\text{м}$ площадь изображения
    чертежа на фотопластинке оказалась равной $16\,\text{см}^{2}$.
    Какова площадь самого чертежа? Ответ выразите в квадратных сантиметрах.
}
\answer{%
    \begin{align*}
    &\frac 1a + \frac 1b = \frac 1F = D \implies b = \frac{aF}{a - F} \\
    &\frac {S'}S = \Gamma^2 = \sqr{\frac ba} = \sqr{\frac F{a - F}} \implies \\
    &\implies S = S' \cdot \sqr{\frac{a - F}F} = S' \cdot \sqr{\frac aF - 1} = S' \cdot \sqr{aD - 1} \approx 144\,\text{см}^{2}.
    \end{align*}
}


\variantsplitter


\addpersonalvariant{Софья Белянкина}

\tasknumber{7}%
\task{%
    В каком месте на главной оптической оси двояковыпуклой линзы
    нужно поместить точечный источник света,
    чтобы его изображение оказалось в главном фокусе линзы?
}
\answer{%
    $\text{для мнимого - на половине фокусного, для действительного - на бесконечности}$
}
\solutionspace{120pt}

\tasknumber{8}%
\task{%
    Предмет в виде отрезка длиной $\ell$ расположен вдоль оптической оси
    собирающей линзы с фокусным расстоянием $F$.
    Середина отрезка расположена
    на расстоянии $a$ от линзы, которая даёт действительное изображение
    всех точек предмета.
    Определить продольное увеличение предмета.
}
\answer{%
    \begin{align*}
    \frac 1{a + \frac \ell 2} &+ \frac 1b = \frac 1F \implies b = \frac{F\cbr{a + \frac \ell 2}}{a + \frac \ell 2 - F} \\
    \frac 1{a - \frac \ell 2} &+ \frac 1c = \frac 1F \implies c = \frac{F\cbr{a - \frac \ell 2}}{a - \frac \ell 2 - F} \\
    \abs{b - c} &= \abs{\frac{F\cbr{a + \frac \ell 2}}{a + \frac \ell 2 - F} - \frac{F\cbr{a - \frac \ell 2}}{a - \frac \ell 2 - F}}= F\abs{\frac{\cbr{a + \frac \ell 2}\cbr{a - \frac \ell 2 - F} - \cbr{a - \frac \ell 2}\cbr{a + \frac \ell 2 - F}}{ \cbr{a + \frac \ell 2 - F} \cbr{a - \frac \ell 2 - F} }} =  \\
    &= F\abs{\frac{a^2 - \frac {a\ell} 2 - Fa + \frac {a\ell} 2 - \frac {\ell^2} 4 - \frac {F\ell}2 - a^2 - \frac {a\ell}2 + aF + \frac {a\ell}2 + \frac {\ell^2} 4 - \frac {F\ell} 2}{\cbr{a + \frac \ell 2 - F} \cbr{a - \frac \ell 2 - F} }} = \\
    &= F\frac{F\ell}{\sqr{a-F} - \frac {\ell^2}4} = \frac{F^2\ell}{\sqr{a-F} - \frac {\ell^2}4}\implies \Gamma = \frac{\abs{b - c}}\ell = \frac{F^2}{\sqr{a-F} - \frac {\ell^2}4}.
    \end{align*}
}
\solutionspace{120pt}

\tasknumber{9}%
\task{%
    На экране с помощью тонкой линзы получено изображение предмета
    с увеличением $2$.
    Предмет передвинули на $6\,\text{см}$.
    Для того, чтобы получить резкое изображение, пришлось передвинуть экран.
    При этом увеличение оказалось равным $6$.
    На какое расстояние
    пришлось передвинуть экран?
}
\answer{%
    \begin{align*}
    &\frac 1a + \frac 1b = \frac 1F, \Gamma_1 = \frac ba = \frac{F}{a-F} \implies \Gamma_1(a-F) = F \implies a = F \cdot \frac{1 + \Gamma_1}{\Gamma_1} \\
    &\frac 1{a + x} + \frac 1{b + y} = \frac 1F, \Gamma_2 = \frac {b+y}{a+x} = \frac{F}{a+x-F} \implies a + x = F \cdot \frac{1 + \Gamma_2}{\Gamma_2} \\
    &1 + \frac xa = \frac{ \frac{1 + \Gamma_2}{\Gamma_2} }{ \frac{1 + \Gamma_1}{\Gamma_1} } = \frac{\Gamma_1(1 + \Gamma_2)}{\Gamma_2(1 + \Gamma_1)} \\
    &a = \frac x{ \frac{\Gamma_1(1 + \Gamma_2)}{\Gamma_2(1 + \Gamma_1)} - 1} = x \cdot \frac{\Gamma_2(1 + \Gamma_1)}{\Gamma_1 - \Gamma_2} \\
    &y = (a + x)\Gamma_2 - b = (a + x)\Gamma_2 - a\Gamma_1 = a(\Gamma_2 - \Gamma_1) + x\Gamma_2 = -x\Gamma_2(1 + \Gamma_1) + x\Gamma_2 = -x\Gamma_2\Gamma_1 = 72\,\text{см}, \\
    &\text{знаки разные, т.е.
    экран надо было подвинуть в другую сторону чем предмет: $x < 0, y > 0$.}
    \end{align*}
}
\solutionspace{120pt}

\tasknumber{10}%
\task{%
    Тонкая собирающая линза дает изображение предмета на экране при двух положениях линзы между предметом и экраном.
    Высота изображения при первом положении $30\,\text{см}$, во втором — $9\,\text{см}$.
    Расстояние между предметом и экранов постоянно.
    Чему равна высота предмета?
}
\answer{%
    \begin{align*}
    &\frac 1a + \frac 1b = \frac 1F, \frac 1c + \frac 1d = \frac 1F, a + b = c + d \implies \frac{a + b}{ab} = \frac 1F = \frac{c+d}{cd} \implies ab = cd, \\
    &\implies ab = c(a + b - c) \implies c^2 - ac - bc + ab = 0 \implies c = a \text{ или } c = b \implies c = b \implies d = a.
    \\
    &\Gamma_1 = \frac {H_1}H = \frac ba, \Gamma_2 = \frac {H_2}H = \frac dc = \frac ab \implies \frac {H_1}H \cdot \frac {H_2}H = \frac ba \cdot \frac ab = 1, \\
    &H = \sqrt{H_1 H_2} \approx 16{,}4\,\text{см}.
    \end{align*}
}
\solutionspace{120pt}

\tasknumber{11}%
\task{%
    Какие предметы можно рассмотреть на фотографии, сделанной со спутника,
    если разрешающая способность плёнки $0{,}010\,\text{мм}$? Каким должно быть
    время экспозиции $\tau$ чтобы полностью использовать возможности плёнки?
    Фокусное расстояние объектива используемого фотоаппарата $20\,\text{см}$,
    высота орбиты спутника $80\,\text{км}$.
}
\answer{%
    \begin{align*}
    &H \ll R \implies v = v_{\text{I}} = \sqrt{G R} \approx 7{,}9\,\frac{\text{км}}{\text{с}}.
    \\
    &F \ll H \implies b = F, a = H, \\
    &\Gamma = \frac \delta\ell = \frac ba \implies \ell = \frac{\delta a}b = \frac{\delta H}F \approx \frac{0{,}010\,\text{мм} \cdot 80\,\text{км}}{20\,\text{см}} \approx 4\,\text{м}, \\
    &\implies \tau = \frac \ell v = \frac{\delta H}{F v} = \frac{0{,}010\,\text{мм} \cdot 80\,\text{км}}{20\,\text{см} \cdot 7{,}9\,\frac{\text{км}}{\text{с}}} \approx 0{,}5\,\text{мс}.
    \end{align*}
}


\variantsplitter


\addpersonalvariant{Софья Белянкина}

\tasknumber{12}%
\task{%
    При аэрофотосъемках используется фотоаппарат, объектив которого
    имеет фокусиое расстояние $20\,\text{см}$.
    Разрешающая способность плёнки $0{,}015\,\text{мм}$.
    На какой высоте должен лететь самолет, чтобы на фотографии можно
    было различить следы размером $15\,\text{см}$?
    При какой скорости самолета изображение не будет размытым,
    если время экспозиции $2\,\text{мс}$?
}
\answer{%
    \begin{align*}
    &F \ll H \implies b = F, a = H, \\
    &\Gamma = \frac \delta\ell = \frac ba = \frac FH \implies H = \frac{\ell F}\delta = \frac{15\,\text{см} \cdot 20\,\text{см}}{0{,}015\,\text{мм}} \approx 2\,\text{км}, \\
    &\implies v = \frac l\tau = \frac{15\,\text{см}}{2\,\text{мс}} \approx 270\,\frac{\text{км}}{\text{ч}}.
    \end{align*}
}
\solutionspace{120pt}

\tasknumber{13}%
\task{%
    Две одинаковые собиращие линзы установлены так, что их главные оптические оси совпадают,
    а главный фокус первой находится там же, где главный фокус второй.
    Расстояние от первой линзы до предмета равно $16\,\text{см}$.
    Чему равно расстояние от изображения объекта во второй линзе до самого объекта?
    Определите также увеличение.
    Фокусное расстояние каждой линзы $35\,\text{см}$.
}
\answer{%
    \begin{align*}
    \frac 1a + \frac 1b &= \frac 1F \implies b = \frac{aF}{a - F} \implies 2F - b = \frac{2aF - 2F^2 - aF}{a - F} = \frac{F(a - 2F)}{a - F}.
    \\
    \frac 1{2F - b} + \frac 1c &= \frac 1F \implies c = \frac{F(2F-b)}{(2F - b) - F} = \frac{F \cdot \frac{F(a - 2F)}{a - F}}{\frac{F(a - 2F)}{a - F} - F}  = F \cdot \frac{ \frac{F(a - 2F)}{a - F} }{ \frac{F(a - 2F)}{a - F} - 1} = \\
     &= F \cdot \frac{a - 2F}{a - 2F - a + F} = 2F - a = 54\,\text{см}.
     \\
    \ell &= a + 2F + c = 4F = 140\,\text{см}.
    \\
    &\Gamma = \Gamma_1 \cdot \Gamma_2 = \frac ba \cdot \frac c{2F-b} = \frac F{a - F} \cdot \frac{2F - a}{\frac{F(a - 2F)}{a - F}} = -1.
    \end{align*}
}
\solutionspace{120pt}

\tasknumber{14}%
\task{%
    Собирающая линза с фокусным расстоянием $F_1 > 0$ и рассеивающая линза с фокусным расстоянием $F_2 < 0$
    установлены коаксиально на расстоянии $\ell$.
    Пучок параллельных лучей падает на собирающую линзу.
    Сделайте схематичное построение и определите, в какой точке система из этих линз соберёт пучок.
}
\answer{%
    \begin{align*}
    &\text{Если пучок падает на собирающую линзу:} \\
    \frac 1{\infty} + \frac 1b &= \frac 1{F_1} \implies b = F_1 \implies \ell - b = \ell - F_1 \\
    \frac 1{\ell - b} + \frac 1c &= \frac 1{F_2} \implies c = \frac{F_2(\ell - b)}{\ell - b - F_2} = \frac{F_2(\ell - F_1)}{\ell - F_1 - F_2}.
    \\
    &\text{Если же пучок падает на рассеивающую линзу:} \\
    \frac 1{\infty} + \frac 1b &= \frac 1{F_2} \implies b = F_2 \implies \ell - b = \ell - F_2 \\
    \frac 1{\ell - b} + \frac 1c &= \frac 1{F_1} \implies c = \frac{F_1(\ell - b)}{\ell - b - F_1} = \frac{F_1(\ell - F_2)}{\ell - F_2 - F_1}.
    \end{align*}
}
\solutionspace{120pt}

\tasknumber{15}%
\task{%
    Две собирающих линзы с фокусными расстояниями $50\,\text{см}$ и $35\,\text{см}$ расположены так,
    что их оптические оси совмещены.
    На первую линзу падает пучок параллельных лучей.
    Пройдя через вторую линзу, он остался параллельным.
    Найдите расстояние между линзами и сделайте рисунок.
}
\answer{%
    \begin{align*}
    \frac 1\infty + \frac 1b &= \frac 1{F_1} \implies b = F_1, \\
    \frac 1{\ell - b} + \frac 1{\infty} &= \frac 1{F_2} \implies \ell - b = F_2 \implies \ell = b + F_2 = F_1 + F_2 = 85\,\text{см}.
    \end{align*}
}

\variantsplitter

\addpersonalvariant{Варвара Егиазарян}

\tasknumber{1}%
\task{%
    Найти оптическую силу собирающей линзы, если действительное изображение предмета,
    помещённого в $35\,\text{см}$ от линзы, получается на расстоянии $40\,\text{см}$ от неё.
}
\answer{%
    $D = \frac 1F = \frac 1a + \frac 1b = \frac 1{35\,\text{см}} + \frac 1{40\,\text{см}} \approx 5{,}36\,\text{дптр}$
}
\solutionspace{80pt}

\tasknumber{2}%
\task{%
    Найти увеличение изображения, если изображение предмета, находящегося
    на расстоянии $15\,\text{см}$ от линзы, получается на расстоянии $12\,\text{см}$ от неё.
}
\answer{%
    $\Gamma = \frac ba = \frac {12\,\text{см}}{15\,\text{см}} \approx 0{,}8$
}
\solutionspace{80pt}

\tasknumber{3}%
\task{%
    Расстояние от предмета до линзы $8\,\text{см}$, а от линзы до мнимого изображения $20\,\text{см}$.
    Чему равно фокусное расстояние линзы?
}
\answer{%
    $\pm \frac 1F = \frac 1a - \frac 1b \implies F = \frac{a b}{\abs{b - a}} \approx 13{,}3\,\text{см}$
}
\solutionspace{80pt}

\tasknumber{4}%
\task{%
    Две тонкие собирающие линзы с фокусными расстояниями $18\,\text{см}$ и $30\,\text{см}$ сложены вместе.
    Чему равно фокусное расстояние такой оптической системы?
}
\answer{%
    $\frac 1{f_1} = \frac 1a + \frac 1b; \frac 1{f_2} = - \frac 1b + \frac 1c \implies \frac 1{f_1} + \frac 1{f_2} = \frac 1a + \frac 1c \implies f' = \frac 1{\frac 1{f_1} + \frac 1{f_2}} = \frac{f_1 f_2}{f_1 + f_2} \approx 11{,}2\,\text{см}$
}
\solutionspace{80pt}

\tasknumber{5}%
\task{%
    Линейные размеры прямого изображения предмета, полученного в собирающей линзе,
    в два раза больше линейных размеров предмета.
    Зная, что предмет находится на $30\,\text{см}$ ближе к линзе,
    чем его изображение, найти оптическую силу линзы.
}
\answer{%
    \begin{align*}
    &\text{Если изображение действительное:} \\
    D &= \frac 1F = \frac 1a + \frac 1b, \qquad \Gamma = \frac ba, \qquad b - a = \ell \implies b = \Gamma a \implies \Gamma a - a = \ell \implies  \\
    a &= \frac {\ell}{\Gamma - 1} \implies b = \frac {{\ell} \Gamma}{\Gamma - 1} \implies  \\
    D &= \frac {\Gamma - 1}\ell + \frac {\Gamma - 1}{\ell \Gamma} = \frac 1\ell \cdot \cbr{\Gamma - 1 + \frac {\Gamma - 1}{\Gamma} } =\frac 1\ell \cdot \cbr{\Gamma - \frac 1\Gamma} \approx 5\,\text{дптр}.
    \\
    &\text{Если изображение мнимое:} \\
    D &= \frac 1F = \frac 1a - \frac 1b, \qquad \Gamma = \frac ba, \qquad b - a = \ell \implies b = \Gamma a \implies \Gamma a - a = \ell \implies  \\
    a &= \frac {\ell}{\Gamma - 1} \implies b = \frac {{\ell} \Gamma}{\Gamma - 1} \implies  \\
    D &= \frac {\Gamma - 1}\ell - \frac {\Gamma - 1}{\ell \Gamma} = \frac 1\ell \cdot \cbr{\Gamma - 1 - \frac {\Gamma - 1}{\Gamma} } =\frac 1\ell \cdot \cbr{\Gamma + \frac 1\Gamma - 2} \approx 1{,}7\,\text{дптр}.
    \\
    &\text{В ответе надо указать оба значения.}
    \end{align*}
}
\solutionspace{120pt}

\tasknumber{6}%
\task{%
    Оптическая сила объектива фотоаппарата равна $6\,\text{дптр}$.
    При фотографировании чертежа с расстояния $1{,}1\,\text{м}$ площадь изображения
    чертежа на фотопластинке оказалась равной $4\,\text{см}^{2}$.
    Какова площадь самого чертежа? Ответ выразите в квадратных сантиметрах.
}
\answer{%
    \begin{align*}
    &\frac 1a + \frac 1b = \frac 1F = D \implies b = \frac{aF}{a - F} \\
    &\frac {S'}S = \Gamma^2 = \sqr{\frac ba} = \sqr{\frac F{a - F}} \implies \\
    &\implies S = S' \cdot \sqr{\frac{a - F}F} = S' \cdot \sqr{\frac aF - 1} = S' \cdot \sqr{aD - 1} \approx 130\,\text{см}^{2}.
    \end{align*}
}


\variantsplitter


\addpersonalvariant{Варвара Егиазарян}

\tasknumber{7}%
\task{%
    В каком месте на главной оптической оси двояковыпуклой линзы
    нужно поместить точечный источник света,
    чтобы его изображение оказалось в главном фокусе линзы?
}
\answer{%
    $\text{для мнимого - на половине фокусного, для действительного - на бесконечности}$
}
\solutionspace{120pt}

\tasknumber{8}%
\task{%
    Предмет в виде отрезка длиной $\ell$ расположен вдоль оптической оси
    собирающей линзы с фокусным расстоянием $F$.
    Середина отрезка расположена
    на расстоянии $a$ от линзы, которая даёт действительное изображение
    всех точек предмета.
    Определить продольное увеличение предмета.
}
\answer{%
    \begin{align*}
    \frac 1{a + \frac \ell 2} &+ \frac 1b = \frac 1F \implies b = \frac{F\cbr{a + \frac \ell 2}}{a + \frac \ell 2 - F} \\
    \frac 1{a - \frac \ell 2} &+ \frac 1c = \frac 1F \implies c = \frac{F\cbr{a - \frac \ell 2}}{a - \frac \ell 2 - F} \\
    \abs{b - c} &= \abs{\frac{F\cbr{a + \frac \ell 2}}{a + \frac \ell 2 - F} - \frac{F\cbr{a - \frac \ell 2}}{a - \frac \ell 2 - F}}= F\abs{\frac{\cbr{a + \frac \ell 2}\cbr{a - \frac \ell 2 - F} - \cbr{a - \frac \ell 2}\cbr{a + \frac \ell 2 - F}}{ \cbr{a + \frac \ell 2 - F} \cbr{a - \frac \ell 2 - F} }} =  \\
    &= F\abs{\frac{a^2 - \frac {a\ell} 2 - Fa + \frac {a\ell} 2 - \frac {\ell^2} 4 - \frac {F\ell}2 - a^2 - \frac {a\ell}2 + aF + \frac {a\ell}2 + \frac {\ell^2} 4 - \frac {F\ell} 2}{\cbr{a + \frac \ell 2 - F} \cbr{a - \frac \ell 2 - F} }} = \\
    &= F\frac{F\ell}{\sqr{a-F} - \frac {\ell^2}4} = \frac{F^2\ell}{\sqr{a-F} - \frac {\ell^2}4}\implies \Gamma = \frac{\abs{b - c}}\ell = \frac{F^2}{\sqr{a-F} - \frac {\ell^2}4}.
    \end{align*}
}
\solutionspace{120pt}

\tasknumber{9}%
\task{%
    На экране с помощью тонкой линзы получено изображение предмета
    с увеличением $4$.
    Предмет передвинули на $4\,\text{см}$.
    Для того, чтобы получить резкое изображение, пришлось передвинуть экран.
    При этом увеличение оказалось равным $6$.
    На какое расстояние
    пришлось передвинуть экран?
}
\answer{%
    \begin{align*}
    &\frac 1a + \frac 1b = \frac 1F, \Gamma_1 = \frac ba = \frac{F}{a-F} \implies \Gamma_1(a-F) = F \implies a = F \cdot \frac{1 + \Gamma_1}{\Gamma_1} \\
    &\frac 1{a + x} + \frac 1{b + y} = \frac 1F, \Gamma_2 = \frac {b+y}{a+x} = \frac{F}{a+x-F} \implies a + x = F \cdot \frac{1 + \Gamma_2}{\Gamma_2} \\
    &1 + \frac xa = \frac{ \frac{1 + \Gamma_2}{\Gamma_2} }{ \frac{1 + \Gamma_1}{\Gamma_1} } = \frac{\Gamma_1(1 + \Gamma_2)}{\Gamma_2(1 + \Gamma_1)} \\
    &a = \frac x{ \frac{\Gamma_1(1 + \Gamma_2)}{\Gamma_2(1 + \Gamma_1)} - 1} = x \cdot \frac{\Gamma_2(1 + \Gamma_1)}{\Gamma_1 - \Gamma_2} \\
    &y = (a + x)\Gamma_2 - b = (a + x)\Gamma_2 - a\Gamma_1 = a(\Gamma_2 - \Gamma_1) + x\Gamma_2 = -x\Gamma_2(1 + \Gamma_1) + x\Gamma_2 = -x\Gamma_2\Gamma_1 = 96\,\text{см}, \\
    &\text{знаки разные, т.е.
    экран надо было подвинуть в другую сторону чем предмет: $x < 0, y > 0$.}
    \end{align*}
}
\solutionspace{120pt}

\tasknumber{10}%
\task{%
    Тонкая собирающая линза дает изображение предмета на экране при двух положениях линзы между предметом и экраном.
    Высота изображения при первом положении $20\,\text{см}$, во втором — $9\,\text{см}$.
    Расстояние между предметом и экранов постоянно.
    Чему равна высота предмета?
}
\answer{%
    \begin{align*}
    &\frac 1a + \frac 1b = \frac 1F, \frac 1c + \frac 1d = \frac 1F, a + b = c + d \implies \frac{a + b}{ab} = \frac 1F = \frac{c+d}{cd} \implies ab = cd, \\
    &\implies ab = c(a + b - c) \implies c^2 - ac - bc + ab = 0 \implies c = a \text{ или } c = b \implies c = b \implies d = a.
    \\
    &\Gamma_1 = \frac {H_1}H = \frac ba, \Gamma_2 = \frac {H_2}H = \frac dc = \frac ab \implies \frac {H_1}H \cdot \frac {H_2}H = \frac ba \cdot \frac ab = 1, \\
    &H = \sqrt{H_1 H_2} \approx 13{,}4\,\text{см}.
    \end{align*}
}
\solutionspace{120pt}

\tasknumber{11}%
\task{%
    Какие предметы можно рассмотреть на фотографии, сделанной со спутника,
    если разрешающая способность плёнки $0{,}010\,\text{мм}$? Каким должно быть
    время экспозиции $\tau$ чтобы полностью использовать возможности плёнки?
    Фокусное расстояние объектива используемого фотоаппарата $10\,\text{см}$,
    высота орбиты спутника $100\,\text{км}$.
}
\answer{%
    \begin{align*}
    &H \ll R \implies v = v_{\text{I}} = \sqrt{G R} \approx 7{,}9\,\frac{\text{км}}{\text{с}}.
    \\
    &F \ll H \implies b = F, a = H, \\
    &\Gamma = \frac \delta\ell = \frac ba \implies \ell = \frac{\delta a}b = \frac{\delta H}F \approx \frac{0{,}010\,\text{мм} \cdot 100\,\text{км}}{10\,\text{см}} \approx 10\,\text{м}, \\
    &\implies \tau = \frac \ell v = \frac{\delta H}{F v} = \frac{0{,}010\,\text{мм} \cdot 100\,\text{км}}{10\,\text{см} \cdot 7{,}9\,\frac{\text{км}}{\text{с}}} \approx 1{,}3\,\text{мс}.
    \end{align*}
}


\variantsplitter


\addpersonalvariant{Варвара Егиазарян}

\tasknumber{12}%
\task{%
    При аэрофотосъемках используется фотоаппарат, объектив которого
    имеет фокусиое расстояние $10\,\text{см}$.
    Разрешающая способность плёнки $0{,}015\,\text{мм}$.
    На какой высоте должен лететь самолет, чтобы на фотографии можно
    было различить следы размером $30\,\text{см}$?
    При какой скорости самолета изображение не будет размытым,
    если время экспозиции $1\,\text{мс}$?
}
\answer{%
    \begin{align*}
    &F \ll H \implies b = F, a = H, \\
    &\Gamma = \frac \delta\ell = \frac ba = \frac FH \implies H = \frac{\ell F}\delta = \frac{30\,\text{см} \cdot 10\,\text{см}}{0{,}015\,\text{мм}} \approx 2\,\text{км}, \\
    &\implies v = \frac l\tau = \frac{30\,\text{см}}{1\,\text{мс}} \approx 1080\,\frac{\text{км}}{\text{ч}}.
    \end{align*}
}
\solutionspace{120pt}

\tasknumber{13}%
\task{%
    Две одинаковые собиращие линзы установлены так, что их главные оптические оси совпадают,
    а главный фокус первой находится там же, где главный фокус второй.
    Расстояние от первой линзы до предмета равно $17\,\text{см}$.
    Чему равно расстояние от изображения объекта во второй линзе до второй линзы?
    Определите также увеличение.
    Фокусное расстояние каждой линзы $40\,\text{см}$.
}
\answer{%
    \begin{align*}
    \frac 1a + \frac 1b &= \frac 1F \implies b = \frac{aF}{a - F} \implies 2F - b = \frac{2aF - 2F^2 - aF}{a - F} = \frac{F(a - 2F)}{a - F}.
    \\
    \frac 1{2F - b} + \frac 1c &= \frac 1F \implies c = \frac{F(2F-b)}{(2F - b) - F} = \frac{F \cdot \frac{F(a - 2F)}{a - F}}{\frac{F(a - 2F)}{a - F} - F}  = F \cdot \frac{ \frac{F(a - 2F)}{a - F} }{ \frac{F(a - 2F)}{a - F} - 1} = \\
     &= F \cdot \frac{a - 2F}{a - 2F - a + F} = 2F - a = 63\,\text{см}.
     \\
    \ell &= a + 2F + c = 4F = 160\,\text{см}.
    \\
    &\Gamma = \Gamma_1 \cdot \Gamma_2 = \frac ba \cdot \frac c{2F-b} = \frac F{a - F} \cdot \frac{2F - a}{\frac{F(a - 2F)}{a - F}} = -1.
    \end{align*}
}
\solutionspace{120pt}

\tasknumber{14}%
\task{%
    Собирающая линза с фокусным расстоянием $F_1 > 0$ и рассеивающая линза с фокусным расстоянием $F_2 < 0$
    установлены коаксиально на расстоянии $\ell$.
    Пучок параллельных лучей падает на собирающую линзу.
    Сделайте схематичное построение и определите, в какой точке система из этих линз соберёт пучок.
}
\answer{%
    \begin{align*}
    &\text{Если пучок падает на собирающую линзу:} \\
    \frac 1{\infty} + \frac 1b &= \frac 1{F_1} \implies b = F_1 \implies \ell - b = \ell - F_1 \\
    \frac 1{\ell - b} + \frac 1c &= \frac 1{F_2} \implies c = \frac{F_2(\ell - b)}{\ell - b - F_2} = \frac{F_2(\ell - F_1)}{\ell - F_1 - F_2}.
    \\
    &\text{Если же пучок падает на рассеивающую линзу:} \\
    \frac 1{\infty} + \frac 1b &= \frac 1{F_2} \implies b = F_2 \implies \ell - b = \ell - F_2 \\
    \frac 1{\ell - b} + \frac 1c &= \frac 1{F_1} \implies c = \frac{F_1(\ell - b)}{\ell - b - F_1} = \frac{F_1(\ell - F_2)}{\ell - F_2 - F_1}.
    \end{align*}
}
\solutionspace{120pt}

\tasknumber{15}%
\task{%
    Две собирающих линзы с фокусными расстояниями $30\,\text{см}$ и $45\,\text{см}$ расположены так,
    что их оптические оси совмещены.
    На первую линзу падает пучок параллельных лучей.
    Пройдя через вторую линзу, он остался параллельным.
    Найдите расстояние между линзами и сделайте рисунок.
}
\answer{%
    \begin{align*}
    \frac 1\infty + \frac 1b &= \frac 1{F_1} \implies b = F_1, \\
    \frac 1{\ell - b} + \frac 1{\infty} &= \frac 1{F_2} \implies \ell - b = F_2 \implies \ell = b + F_2 = F_1 + F_2 = 75\,\text{см}.
    \end{align*}
}

\variantsplitter

\addpersonalvariant{Владислав Емелин}

\tasknumber{1}%
\task{%
    Найти оптическую силу собирающей линзы, если действительное изображение предмета,
    помещённого в $55\,\text{см}$ от линзы, получается на расстоянии $40\,\text{см}$ от неё.
}
\answer{%
    $D = \frac 1F = \frac 1a + \frac 1b = \frac 1{55\,\text{см}} + \frac 1{40\,\text{см}} \approx 4{,}32\,\text{дптр}$
}
\solutionspace{80pt}

\tasknumber{2}%
\task{%
    Найти увеличение изображения, если изображение предмета, находящегося
    на расстоянии $25\,\text{см}$ от линзы, получается на расстоянии $12\,\text{см}$ от неё.
}
\answer{%
    $\Gamma = \frac ba = \frac {12\,\text{см}}{25\,\text{см}} \approx 0{,}5$
}
\solutionspace{80pt}

\tasknumber{3}%
\task{%
    Расстояние от предмета до линзы $12\,\text{см}$, а от линзы до мнимого изображения $20\,\text{см}$.
    Чему равно фокусное расстояние линзы?
}
\answer{%
    $\pm \frac 1F = \frac 1a - \frac 1b \implies F = \frac{a b}{\abs{b - a}} \approx 30\,\text{см}$
}
\solutionspace{80pt}

\tasknumber{4}%
\task{%
    Две тонкие собирающие линзы с фокусными расстояниями $12\,\text{см}$ и $30\,\text{см}$ сложены вместе.
    Чему равно фокусное расстояние такой оптической системы?
}
\answer{%
    $\frac 1{f_1} = \frac 1a + \frac 1b; \frac 1{f_2} = - \frac 1b + \frac 1c \implies \frac 1{f_1} + \frac 1{f_2} = \frac 1a + \frac 1c \implies f' = \frac 1{\frac 1{f_1} + \frac 1{f_2}} = \frac{f_1 f_2}{f_1 + f_2} \approx 8{,}6\,\text{см}$
}
\solutionspace{80pt}

\tasknumber{5}%
\task{%
    Линейные размеры прямого изображения предмета, полученного в собирающей линзе,
    в два раза больше линейных размеров предмета.
    Зная, что предмет находится на $40\,\text{см}$ ближе к линзе,
    чем его изображение, найти оптическую силу линзы.
}
\answer{%
    \begin{align*}
    &\text{Если изображение действительное:} \\
    D &= \frac 1F = \frac 1a + \frac 1b, \qquad \Gamma = \frac ba, \qquad b - a = \ell \implies b = \Gamma a \implies \Gamma a - a = \ell \implies  \\
    a &= \frac {\ell}{\Gamma - 1} \implies b = \frac {{\ell} \Gamma}{\Gamma - 1} \implies  \\
    D &= \frac {\Gamma - 1}\ell + \frac {\Gamma - 1}{\ell \Gamma} = \frac 1\ell \cdot \cbr{\Gamma - 1 + \frac {\Gamma - 1}{\Gamma} } =\frac 1\ell \cdot \cbr{\Gamma - \frac 1\Gamma} \approx 3{,}8\,\text{дптр}.
    \\
    &\text{Если изображение мнимое:} \\
    D &= \frac 1F = \frac 1a - \frac 1b, \qquad \Gamma = \frac ba, \qquad b - a = \ell \implies b = \Gamma a \implies \Gamma a - a = \ell \implies  \\
    a &= \frac {\ell}{\Gamma - 1} \implies b = \frac {{\ell} \Gamma}{\Gamma - 1} \implies  \\
    D &= \frac {\Gamma - 1}\ell - \frac {\Gamma - 1}{\ell \Gamma} = \frac 1\ell \cdot \cbr{\Gamma - 1 - \frac {\Gamma - 1}{\Gamma} } =\frac 1\ell \cdot \cbr{\Gamma + \frac 1\Gamma - 2} \approx 1{,}2\,\text{дптр}.
    \\
    &\text{В ответе надо указать оба значения.}
    \end{align*}
}
\solutionspace{120pt}

\tasknumber{6}%
\task{%
    Оптическая сила объектива фотоаппарата равна $6\,\text{дптр}$.
    При фотографировании чертежа с расстояния $0{,}9\,\text{м}$ площадь изображения
    чертежа на фотопластинке оказалась равной $9\,\text{см}^{2}$.
    Какова площадь самого чертежа? Ответ выразите в квадратных сантиметрах.
}
\answer{%
    \begin{align*}
    &\frac 1a + \frac 1b = \frac 1F = D \implies b = \frac{aF}{a - F} \\
    &\frac {S'}S = \Gamma^2 = \sqr{\frac ba} = \sqr{\frac F{a - F}} \implies \\
    &\implies S = S' \cdot \sqr{\frac{a - F}F} = S' \cdot \sqr{\frac aF - 1} = S' \cdot \sqr{aD - 1} \approx 170\,\text{см}^{2}.
    \end{align*}
}


\variantsplitter


\addpersonalvariant{Владислав Емелин}

\tasknumber{7}%
\task{%
    В каком месте на главной оптической оси двояковыпуклой линзы
    нужно поместить точечный источник света,
    чтобы его изображение оказалось в главном фокусе линзы?
}
\answer{%
    $\text{для мнимого - на половине фокусного, для действительного - на бесконечности}$
}
\solutionspace{120pt}

\tasknumber{8}%
\task{%
    Предмет в виде отрезка длиной $\ell$ расположен вдоль оптической оси
    собирающей линзы с фокусным расстоянием $F$.
    Середина отрезка расположена
    на расстоянии $a$ от линзы, которая даёт действительное изображение
    всех точек предмета.
    Определить продольное увеличение предмета.
}
\answer{%
    \begin{align*}
    \frac 1{a + \frac \ell 2} &+ \frac 1b = \frac 1F \implies b = \frac{F\cbr{a + \frac \ell 2}}{a + \frac \ell 2 - F} \\
    \frac 1{a - \frac \ell 2} &+ \frac 1c = \frac 1F \implies c = \frac{F\cbr{a - \frac \ell 2}}{a - \frac \ell 2 - F} \\
    \abs{b - c} &= \abs{\frac{F\cbr{a + \frac \ell 2}}{a + \frac \ell 2 - F} - \frac{F\cbr{a - \frac \ell 2}}{a - \frac \ell 2 - F}}= F\abs{\frac{\cbr{a + \frac \ell 2}\cbr{a - \frac \ell 2 - F} - \cbr{a - \frac \ell 2}\cbr{a + \frac \ell 2 - F}}{ \cbr{a + \frac \ell 2 - F} \cbr{a - \frac \ell 2 - F} }} =  \\
    &= F\abs{\frac{a^2 - \frac {a\ell} 2 - Fa + \frac {a\ell} 2 - \frac {\ell^2} 4 - \frac {F\ell}2 - a^2 - \frac {a\ell}2 + aF + \frac {a\ell}2 + \frac {\ell^2} 4 - \frac {F\ell} 2}{\cbr{a + \frac \ell 2 - F} \cbr{a - \frac \ell 2 - F} }} = \\
    &= F\frac{F\ell}{\sqr{a-F} - \frac {\ell^2}4} = \frac{F^2\ell}{\sqr{a-F} - \frac {\ell^2}4}\implies \Gamma = \frac{\abs{b - c}}\ell = \frac{F^2}{\sqr{a-F} - \frac {\ell^2}4}.
    \end{align*}
}
\solutionspace{120pt}

\tasknumber{9}%
\task{%
    На экране с помощью тонкой линзы получено изображение предмета
    с увеличением $4$.
    Предмет передвинули на $6\,\text{см}$.
    Для того, чтобы получить резкое изображение, пришлось передвинуть экран.
    При этом увеличение оказалось равным $6$.
    На какое расстояние
    пришлось передвинуть экран?
}
\answer{%
    \begin{align*}
    &\frac 1a + \frac 1b = \frac 1F, \Gamma_1 = \frac ba = \frac{F}{a-F} \implies \Gamma_1(a-F) = F \implies a = F \cdot \frac{1 + \Gamma_1}{\Gamma_1} \\
    &\frac 1{a + x} + \frac 1{b + y} = \frac 1F, \Gamma_2 = \frac {b+y}{a+x} = \frac{F}{a+x-F} \implies a + x = F \cdot \frac{1 + \Gamma_2}{\Gamma_2} \\
    &1 + \frac xa = \frac{ \frac{1 + \Gamma_2}{\Gamma_2} }{ \frac{1 + \Gamma_1}{\Gamma_1} } = \frac{\Gamma_1(1 + \Gamma_2)}{\Gamma_2(1 + \Gamma_1)} \\
    &a = \frac x{ \frac{\Gamma_1(1 + \Gamma_2)}{\Gamma_2(1 + \Gamma_1)} - 1} = x \cdot \frac{\Gamma_2(1 + \Gamma_1)}{\Gamma_1 - \Gamma_2} \\
    &y = (a + x)\Gamma_2 - b = (a + x)\Gamma_2 - a\Gamma_1 = a(\Gamma_2 - \Gamma_1) + x\Gamma_2 = -x\Gamma_2(1 + \Gamma_1) + x\Gamma_2 = -x\Gamma_2\Gamma_1 = 144\,\text{см}, \\
    &\text{знаки разные, т.е.
    экран надо было подвинуть в другую сторону чем предмет: $x < 0, y > 0$.}
    \end{align*}
}
\solutionspace{120pt}

\tasknumber{10}%
\task{%
    Тонкая собирающая линза дает изображение предмета на экране при двух положениях линзы между предметом и экраном.
    Высота изображения при первом положении $25\,\text{см}$, во втором — $5\,\text{см}$.
    Расстояние между предметом и экранов постоянно.
    Чему равна высота предмета?
}
\answer{%
    \begin{align*}
    &\frac 1a + \frac 1b = \frac 1F, \frac 1c + \frac 1d = \frac 1F, a + b = c + d \implies \frac{a + b}{ab} = \frac 1F = \frac{c+d}{cd} \implies ab = cd, \\
    &\implies ab = c(a + b - c) \implies c^2 - ac - bc + ab = 0 \implies c = a \text{ или } c = b \implies c = b \implies d = a.
    \\
    &\Gamma_1 = \frac {H_1}H = \frac ba, \Gamma_2 = \frac {H_2}H = \frac dc = \frac ab \implies \frac {H_1}H \cdot \frac {H_2}H = \frac ba \cdot \frac ab = 1, \\
    &H = \sqrt{H_1 H_2} \approx 11{,}2\,\text{см}.
    \end{align*}
}
\solutionspace{120pt}

\tasknumber{11}%
\task{%
    Какие предметы можно рассмотреть на фотографии, сделанной со спутника,
    если разрешающая способность плёнки $0{,}010\,\text{мм}$? Каким должно быть
    время экспозиции $\tau$ чтобы полностью использовать возможности плёнки?
    Фокусное расстояние объектива используемого фотоаппарата $10\,\text{см}$,
    высота орбиты спутника $120\,\text{км}$.
}
\answer{%
    \begin{align*}
    &H \ll R \implies v = v_{\text{I}} = \sqrt{G R} \approx 7{,}9\,\frac{\text{км}}{\text{с}}.
    \\
    &F \ll H \implies b = F, a = H, \\
    &\Gamma = \frac \delta\ell = \frac ba \implies \ell = \frac{\delta a}b = \frac{\delta H}F \approx \frac{0{,}010\,\text{мм} \cdot 120\,\text{км}}{10\,\text{см}} \approx 12\,\text{м}, \\
    &\implies \tau = \frac \ell v = \frac{\delta H}{F v} = \frac{0{,}010\,\text{мм} \cdot 120\,\text{км}}{10\,\text{см} \cdot 7{,}9\,\frac{\text{км}}{\text{с}}} \approx 1{,}5\,\text{мс}.
    \end{align*}
}


\variantsplitter


\addpersonalvariant{Владислав Емелин}

\tasknumber{12}%
\task{%
    При аэрофотосъемках используется фотоаппарат, объектив которого
    имеет фокусиое расстояние $10\,\text{см}$.
    Разрешающая способность плёнки $0{,}015\,\text{мм}$.
    На какой высоте должен лететь самолет, чтобы на фотографии можно
    было различить следы размером $25\,\text{см}$?
    При какой скорости самолета изображение не будет размытым,
    если время экспозиции $1\,\text{мс}$?
}
\answer{%
    \begin{align*}
    &F \ll H \implies b = F, a = H, \\
    &\Gamma = \frac \delta\ell = \frac ba = \frac FH \implies H = \frac{\ell F}\delta = \frac{25\,\text{см} \cdot 10\,\text{см}}{0{,}015\,\text{мм}} \approx 1{,}7\,\text{км}, \\
    &\implies v = \frac l\tau = \frac{25\,\text{см}}{1\,\text{мс}} \approx 900\,\frac{\text{км}}{\text{ч}}.
    \end{align*}
}
\solutionspace{120pt}

\tasknumber{13}%
\task{%
    Две одинаковые собиращие линзы установлены так, что их главные оптические оси совпадают,
    а главный фокус первой находится там же, где главный фокус второй.
    Расстояние от первой линзы до предмета равно $14\,\text{см}$.
    Чему равно расстояние от изображения объекта во второй линзе до второй линзы?
    Определите также увеличение.
    Фокусное расстояние каждой линзы $35\,\text{см}$.
}
\answer{%
    \begin{align*}
    \frac 1a + \frac 1b &= \frac 1F \implies b = \frac{aF}{a - F} \implies 2F - b = \frac{2aF - 2F^2 - aF}{a - F} = \frac{F(a - 2F)}{a - F}.
    \\
    \frac 1{2F - b} + \frac 1c &= \frac 1F \implies c = \frac{F(2F-b)}{(2F - b) - F} = \frac{F \cdot \frac{F(a - 2F)}{a - F}}{\frac{F(a - 2F)}{a - F} - F}  = F \cdot \frac{ \frac{F(a - 2F)}{a - F} }{ \frac{F(a - 2F)}{a - F} - 1} = \\
     &= F \cdot \frac{a - 2F}{a - 2F - a + F} = 2F - a = 56\,\text{см}.
     \\
    \ell &= a + 2F + c = 4F = 140\,\text{см}.
    \\
    &\Gamma = \Gamma_1 \cdot \Gamma_2 = \frac ba \cdot \frac c{2F-b} = \frac F{a - F} \cdot \frac{2F - a}{\frac{F(a - 2F)}{a - F}} = -1.
    \end{align*}
}
\solutionspace{120pt}

\tasknumber{14}%
\task{%
    Собирающая линза с фокусным расстоянием $F_1 > 0$ и рассеивающая линза с фокусным расстоянием $F_2 < 0$
    установлены коаксиально на расстоянии $\ell$.
    Пучок параллельных лучей падает на собирающую линзу.
    Сделайте схематичное построение и определите, в какой точке система из этих линз соберёт пучок.
}
\answer{%
    \begin{align*}
    &\text{Если пучок падает на собирающую линзу:} \\
    \frac 1{\infty} + \frac 1b &= \frac 1{F_1} \implies b = F_1 \implies \ell - b = \ell - F_1 \\
    \frac 1{\ell - b} + \frac 1c &= \frac 1{F_2} \implies c = \frac{F_2(\ell - b)}{\ell - b - F_2} = \frac{F_2(\ell - F_1)}{\ell - F_1 - F_2}.
    \\
    &\text{Если же пучок падает на рассеивающую линзу:} \\
    \frac 1{\infty} + \frac 1b &= \frac 1{F_2} \implies b = F_2 \implies \ell - b = \ell - F_2 \\
    \frac 1{\ell - b} + \frac 1c &= \frac 1{F_1} \implies c = \frac{F_1(\ell - b)}{\ell - b - F_1} = \frac{F_1(\ell - F_2)}{\ell - F_2 - F_1}.
    \end{align*}
}
\solutionspace{120pt}

\tasknumber{15}%
\task{%
    Две собирающих линзы с фокусными расстояниями $40\,\text{см}$ и $45\,\text{см}$ расположены так,
    что их оптические оси совмещены.
    На первую линзу падает пучок параллельных лучей.
    Пройдя через вторую линзу, он остался параллельным.
    Найдите расстояние между линзами и сделайте рисунок.
}
\answer{%
    \begin{align*}
    \frac 1\infty + \frac 1b &= \frac 1{F_1} \implies b = F_1, \\
    \frac 1{\ell - b} + \frac 1{\infty} &= \frac 1{F_2} \implies \ell - b = F_2 \implies \ell = b + F_2 = F_1 + F_2 = 85\,\text{см}.
    \end{align*}
}

\variantsplitter

\addpersonalvariant{Артём Жичин}

\tasknumber{1}%
\task{%
    Найти оптическую силу собирающей линзы, если действительное изображение предмета,
    помещённого в $35\,\text{см}$ от линзы, получается на расстоянии $40\,\text{см}$ от неё.
}
\answer{%
    $D = \frac 1F = \frac 1a + \frac 1b = \frac 1{35\,\text{см}} + \frac 1{40\,\text{см}} \approx 5{,}36\,\text{дптр}$
}
\solutionspace{80pt}

\tasknumber{2}%
\task{%
    Найти увеличение изображения, если изображение предмета, находящегося
    на расстоянии $20\,\text{см}$ от линзы, получается на расстоянии $30\,\text{см}$ от неё.
}
\answer{%
    $\Gamma = \frac ba = \frac {30\,\text{см}}{20\,\text{см}} \approx 1{,}50$
}
\solutionspace{80pt}

\tasknumber{3}%
\task{%
    Расстояние от предмета до линзы $10\,\text{см}$, а от линзы до мнимого изображения $30\,\text{см}$.
    Чему равно фокусное расстояние линзы?
}
\answer{%
    $\pm \frac 1F = \frac 1a - \frac 1b \implies F = \frac{a b}{\abs{b - a}} \approx 15\,\text{см}$
}
\solutionspace{80pt}

\tasknumber{4}%
\task{%
    Две тонкие собирающие линзы с фокусными расстояниями $12\,\text{см}$ и $20\,\text{см}$ сложены вместе.
    Чему равно фокусное расстояние такой оптической системы?
}
\answer{%
    $\frac 1{f_1} = \frac 1a + \frac 1b; \frac 1{f_2} = - \frac 1b + \frac 1c \implies \frac 1{f_1} + \frac 1{f_2} = \frac 1a + \frac 1c \implies f' = \frac 1{\frac 1{f_1} + \frac 1{f_2}} = \frac{f_1 f_2}{f_1 + f_2} \approx 7{,}5\,\text{см}$
}
\solutionspace{80pt}

\tasknumber{5}%
\task{%
    Линейные размеры прямого изображения предмета, полученного в собирающей линзе,
    в четыре раза больше линейных размеров предмета.
    Зная, что предмет находится на $20\,\text{см}$ ближе к линзе,
    чем его изображение, найти оптическую силу линзы.
}
\answer{%
    \begin{align*}
    &\text{Если изображение действительное:} \\
    D &= \frac 1F = \frac 1a + \frac 1b, \qquad \Gamma = \frac ba, \qquad b - a = \ell \implies b = \Gamma a \implies \Gamma a - a = \ell \implies  \\
    a &= \frac {\ell}{\Gamma - 1} \implies b = \frac {{\ell} \Gamma}{\Gamma - 1} \implies  \\
    D &= \frac {\Gamma - 1}\ell + \frac {\Gamma - 1}{\ell \Gamma} = \frac 1\ell \cdot \cbr{\Gamma - 1 + \frac {\Gamma - 1}{\Gamma} } =\frac 1\ell \cdot \cbr{\Gamma - \frac 1\Gamma} \approx 18{,}8\,\text{дптр}.
    \\
    &\text{Если изображение мнимое:} \\
    D &= \frac 1F = \frac 1a - \frac 1b, \qquad \Gamma = \frac ba, \qquad b - a = \ell \implies b = \Gamma a \implies \Gamma a - a = \ell \implies  \\
    a &= \frac {\ell}{\Gamma - 1} \implies b = \frac {{\ell} \Gamma}{\Gamma - 1} \implies  \\
    D &= \frac {\Gamma - 1}\ell - \frac {\Gamma - 1}{\ell \Gamma} = \frac 1\ell \cdot \cbr{\Gamma - 1 - \frac {\Gamma - 1}{\Gamma} } =\frac 1\ell \cdot \cbr{\Gamma + \frac 1\Gamma - 2} \approx 11{,}2\,\text{дптр}.
    \\
    &\text{В ответе надо указать оба значения.}
    \end{align*}
}
\solutionspace{120pt}

\tasknumber{6}%
\task{%
    Оптическая сила объектива фотоаппарата равна $6\,\text{дптр}$.
    При фотографировании чертежа с расстояния $1{,}2\,\text{м}$ площадь изображения
    чертежа на фотопластинке оказалась равной $4\,\text{см}^{2}$.
    Какова площадь самого чертежа? Ответ выразите в квадратных сантиметрах.
}
\answer{%
    \begin{align*}
    &\frac 1a + \frac 1b = \frac 1F = D \implies b = \frac{aF}{a - F} \\
    &\frac {S'}S = \Gamma^2 = \sqr{\frac ba} = \sqr{\frac F{a - F}} \implies \\
    &\implies S = S' \cdot \sqr{\frac{a - F}F} = S' \cdot \sqr{\frac aF - 1} = S' \cdot \sqr{aD - 1} \approx 150\,\text{см}^{2}.
    \end{align*}
}


\variantsplitter


\addpersonalvariant{Артём Жичин}

\tasknumber{7}%
\task{%
    В каком месте на главной оптической оси двояковыгнутой линзы
    нужно поместить точечный источник света,
    чтобы его изображение оказалось в главном фокусе линзы?
}
\answer{%
    $\text{на половине фокусного расстояния}$
}
\solutionspace{120pt}

\tasknumber{8}%
\task{%
    Предмет в виде отрезка длиной $\ell$ расположен вдоль оптической оси
    собирающей линзы с фокусным расстоянием $F$.
    Середина отрезка расположена
    на расстоянии $a$ от линзы, которая даёт действительное изображение
    всех точек предмета.
    Определить продольное увеличение предмета.
}
\answer{%
    \begin{align*}
    \frac 1{a + \frac \ell 2} &+ \frac 1b = \frac 1F \implies b = \frac{F\cbr{a + \frac \ell 2}}{a + \frac \ell 2 - F} \\
    \frac 1{a - \frac \ell 2} &+ \frac 1c = \frac 1F \implies c = \frac{F\cbr{a - \frac \ell 2}}{a - \frac \ell 2 - F} \\
    \abs{b - c} &= \abs{\frac{F\cbr{a + \frac \ell 2}}{a + \frac \ell 2 - F} - \frac{F\cbr{a - \frac \ell 2}}{a - \frac \ell 2 - F}}= F\abs{\frac{\cbr{a + \frac \ell 2}\cbr{a - \frac \ell 2 - F} - \cbr{a - \frac \ell 2}\cbr{a + \frac \ell 2 - F}}{ \cbr{a + \frac \ell 2 - F} \cbr{a - \frac \ell 2 - F} }} =  \\
    &= F\abs{\frac{a^2 - \frac {a\ell} 2 - Fa + \frac {a\ell} 2 - \frac {\ell^2} 4 - \frac {F\ell}2 - a^2 - \frac {a\ell}2 + aF + \frac {a\ell}2 + \frac {\ell^2} 4 - \frac {F\ell} 2}{\cbr{a + \frac \ell 2 - F} \cbr{a - \frac \ell 2 - F} }} = \\
    &= F\frac{F\ell}{\sqr{a-F} - \frac {\ell^2}4} = \frac{F^2\ell}{\sqr{a-F} - \frac {\ell^2}4}\implies \Gamma = \frac{\abs{b - c}}\ell = \frac{F^2}{\sqr{a-F} - \frac {\ell^2}4}.
    \end{align*}
}
\solutionspace{120pt}

\tasknumber{9}%
\task{%
    На экране с помощью тонкой линзы получено изображение предмета
    с увеличением $4$.
    Предмет передвинули на $8\,\text{см}$.
    Для того, чтобы получить резкое изображение, пришлось передвинуть экран.
    При этом увеличение оказалось равным $8$.
    На какое расстояние
    пришлось передвинуть экран?
}
\answer{%
    \begin{align*}
    &\frac 1a + \frac 1b = \frac 1F, \Gamma_1 = \frac ba = \frac{F}{a-F} \implies \Gamma_1(a-F) = F \implies a = F \cdot \frac{1 + \Gamma_1}{\Gamma_1} \\
    &\frac 1{a + x} + \frac 1{b + y} = \frac 1F, \Gamma_2 = \frac {b+y}{a+x} = \frac{F}{a+x-F} \implies a + x = F \cdot \frac{1 + \Gamma_2}{\Gamma_2} \\
    &1 + \frac xa = \frac{ \frac{1 + \Gamma_2}{\Gamma_2} }{ \frac{1 + \Gamma_1}{\Gamma_1} } = \frac{\Gamma_1(1 + \Gamma_2)}{\Gamma_2(1 + \Gamma_1)} \\
    &a = \frac x{ \frac{\Gamma_1(1 + \Gamma_2)}{\Gamma_2(1 + \Gamma_1)} - 1} = x \cdot \frac{\Gamma_2(1 + \Gamma_1)}{\Gamma_1 - \Gamma_2} \\
    &y = (a + x)\Gamma_2 - b = (a + x)\Gamma_2 - a\Gamma_1 = a(\Gamma_2 - \Gamma_1) + x\Gamma_2 = -x\Gamma_2(1 + \Gamma_1) + x\Gamma_2 = -x\Gamma_2\Gamma_1 = 256\,\text{см}, \\
    &\text{знаки разные, т.е.
    экран надо было подвинуть в другую сторону чем предмет: $x < 0, y > 0$.}
    \end{align*}
}
\solutionspace{120pt}

\tasknumber{10}%
\task{%
    Тонкая собирающая линза дает изображение предмета на экране при двух положениях линзы между предметом и экраном.
    Высота изображения при первом положении $30\,\text{см}$, во втором — $9\,\text{см}$.
    Расстояние между предметом и экранов постоянно.
    Чему равна высота предмета?
}
\answer{%
    \begin{align*}
    &\frac 1a + \frac 1b = \frac 1F, \frac 1c + \frac 1d = \frac 1F, a + b = c + d \implies \frac{a + b}{ab} = \frac 1F = \frac{c+d}{cd} \implies ab = cd, \\
    &\implies ab = c(a + b - c) \implies c^2 - ac - bc + ab = 0 \implies c = a \text{ или } c = b \implies c = b \implies d = a.
    \\
    &\Gamma_1 = \frac {H_1}H = \frac ba, \Gamma_2 = \frac {H_2}H = \frac dc = \frac ab \implies \frac {H_1}H \cdot \frac {H_2}H = \frac ba \cdot \frac ab = 1, \\
    &H = \sqrt{H_1 H_2} \approx 16{,}4\,\text{см}.
    \end{align*}
}
\solutionspace{120pt}

\tasknumber{11}%
\task{%
    Какие предметы можно рассмотреть на фотографии, сделанной со спутника,
    если разрешающая способность плёнки $0{,}010\,\text{мм}$? Каким должно быть
    время экспозиции $\tau$ чтобы полностью использовать возможности плёнки?
    Фокусное расстояние объектива используемого фотоаппарата $10\,\text{см}$,
    высота орбиты спутника $80\,\text{км}$.
}
\answer{%
    \begin{align*}
    &H \ll R \implies v = v_{\text{I}} = \sqrt{G R} \approx 7{,}9\,\frac{\text{км}}{\text{с}}.
    \\
    &F \ll H \implies b = F, a = H, \\
    &\Gamma = \frac \delta\ell = \frac ba \implies \ell = \frac{\delta a}b = \frac{\delta H}F \approx \frac{0{,}010\,\text{мм} \cdot 80\,\text{км}}{10\,\text{см}} \approx 8\,\text{м}, \\
    &\implies \tau = \frac \ell v = \frac{\delta H}{F v} = \frac{0{,}010\,\text{мм} \cdot 80\,\text{км}}{10\,\text{см} \cdot 7{,}9\,\frac{\text{км}}{\text{с}}} \approx 1{,}0\,\text{мс}.
    \end{align*}
}


\variantsplitter


\addpersonalvariant{Артём Жичин}

\tasknumber{12}%
\task{%
    При аэрофотосъемках используется фотоаппарат, объектив которого
    имеет фокусиое расстояние $10\,\text{см}$.
    Разрешающая способность плёнки $0{,}010\,\text{мм}$.
    На какой высоте должен лететь самолет, чтобы на фотографии можно
    было различить следы размером $20\,\text{см}$?
    При какой скорости самолета изображение не будет размытым,
    если время экспозиции $1\,\text{мс}$?
}
\answer{%
    \begin{align*}
    &F \ll H \implies b = F, a = H, \\
    &\Gamma = \frac \delta\ell = \frac ba = \frac FH \implies H = \frac{\ell F}\delta = \frac{20\,\text{см} \cdot 10\,\text{см}}{0{,}010\,\text{мм}} \approx 2\,\text{км}, \\
    &\implies v = \frac l\tau = \frac{20\,\text{см}}{1\,\text{мс}} \approx 700\,\frac{\text{км}}{\text{ч}}.
    \end{align*}
}
\solutionspace{120pt}

\tasknumber{13}%
\task{%
    Две одинаковые собиращие линзы установлены так, что их главные оптические оси совпадают,
    а главный фокус первой находится там же, где главный фокус второй.
    Расстояние от первой линзы до предмета равно $15\,\text{см}$.
    Чему равно расстояние от изображения объекта во второй линзе до второй линзы?
    Определите также увеличение.
    Фокусное расстояние каждой линзы $30\,\text{см}$.
}
\answer{%
    \begin{align*}
    \frac 1a + \frac 1b &= \frac 1F \implies b = \frac{aF}{a - F} \implies 2F - b = \frac{2aF - 2F^2 - aF}{a - F} = \frac{F(a - 2F)}{a - F}.
    \\
    \frac 1{2F - b} + \frac 1c &= \frac 1F \implies c = \frac{F(2F-b)}{(2F - b) - F} = \frac{F \cdot \frac{F(a - 2F)}{a - F}}{\frac{F(a - 2F)}{a - F} - F}  = F \cdot \frac{ \frac{F(a - 2F)}{a - F} }{ \frac{F(a - 2F)}{a - F} - 1} = \\
     &= F \cdot \frac{a - 2F}{a - 2F - a + F} = 2F - a = 45\,\text{см}.
     \\
    \ell &= a + 2F + c = 4F = 120\,\text{см}.
    \\
    &\Gamma = \Gamma_1 \cdot \Gamma_2 = \frac ba \cdot \frac c{2F-b} = \frac F{a - F} \cdot \frac{2F - a}{\frac{F(a - 2F)}{a - F}} = -1.
    \end{align*}
}
\solutionspace{120pt}

\tasknumber{14}%
\task{%
    Собирающая линза с фокусным расстоянием $F_1 > 0$ и рассеивающая линза с фокусным расстоянием $F_2 < 0$
    установлены коаксиально на расстоянии $\ell$.
    Пучок параллельных лучей падает на рассеивающую линзу.
    Сделайте схематичное построение и определите, в какой точке система из этих линз соберёт пучок.
}
\answer{%
    \begin{align*}
    &\text{Если пучок падает на собирающую линзу:} \\
    \frac 1{\infty} + \frac 1b &= \frac 1{F_1} \implies b = F_1 \implies \ell - b = \ell - F_1 \\
    \frac 1{\ell - b} + \frac 1c &= \frac 1{F_2} \implies c = \frac{F_2(\ell - b)}{\ell - b - F_2} = \frac{F_2(\ell - F_1)}{\ell - F_1 - F_2}.
    \\
    &\text{Если же пучок падает на рассеивающую линзу:} \\
    \frac 1{\infty} + \frac 1b &= \frac 1{F_2} \implies b = F_2 \implies \ell - b = \ell - F_2 \\
    \frac 1{\ell - b} + \frac 1c &= \frac 1{F_1} \implies c = \frac{F_1(\ell - b)}{\ell - b - F_1} = \frac{F_1(\ell - F_2)}{\ell - F_2 - F_1}.
    \end{align*}
}
\solutionspace{120pt}

\tasknumber{15}%
\task{%
    Две собирающих линзы с фокусными расстояниями $50\,\text{см}$ и $45\,\text{см}$ расположены так,
    что их оптические оси совмещены.
    На первую линзу падает пучок параллельных лучей.
    Пройдя через вторую линзу, он остался параллельным.
    Найдите расстояние между линзами и сделайте рисунок.
}
\answer{%
    \begin{align*}
    \frac 1\infty + \frac 1b &= \frac 1{F_1} \implies b = F_1, \\
    \frac 1{\ell - b} + \frac 1{\infty} &= \frac 1{F_2} \implies \ell - b = F_2 \implies \ell = b + F_2 = F_1 + F_2 = 95\,\text{см}.
    \end{align*}
}

\variantsplitter

\addpersonalvariant{Дарья Кошман}

\tasknumber{1}%
\task{%
    Найти оптическую силу собирающей линзы, если действительное изображение предмета,
    помещённого в $55\,\text{см}$ от линзы, получается на расстоянии $30\,\text{см}$ от неё.
}
\answer{%
    $D = \frac 1F = \frac 1a + \frac 1b = \frac 1{55\,\text{см}} + \frac 1{30\,\text{см}} \approx 5{,}15\,\text{дптр}$
}
\solutionspace{80pt}

\tasknumber{2}%
\task{%
    Найти увеличение изображения, если изображение предмета, находящегося
    на расстоянии $20\,\text{см}$ от линзы, получается на расстоянии $18\,\text{см}$ от неё.
}
\answer{%
    $\Gamma = \frac ba = \frac {18\,\text{см}}{20\,\text{см}} \approx 0{,}9$
}
\solutionspace{80pt}

\tasknumber{3}%
\task{%
    Расстояние от предмета до линзы $10\,\text{см}$, а от линзы до мнимого изображения $25\,\text{см}$.
    Чему равно фокусное расстояние линзы?
}
\answer{%
    $\pm \frac 1F = \frac 1a - \frac 1b \implies F = \frac{a b}{\abs{b - a}} \approx 16{,}7\,\text{см}$
}
\solutionspace{80pt}

\tasknumber{4}%
\task{%
    Две тонкие собирающие линзы с фокусными расстояниями $18\,\text{см}$ и $20\,\text{см}$ сложены вместе.
    Чему равно фокусное расстояние такой оптической системы?
}
\answer{%
    $\frac 1{f_1} = \frac 1a + \frac 1b; \frac 1{f_2} = - \frac 1b + \frac 1c \implies \frac 1{f_1} + \frac 1{f_2} = \frac 1a + \frac 1c \implies f' = \frac 1{\frac 1{f_1} + \frac 1{f_2}} = \frac{f_1 f_2}{f_1 + f_2} \approx 9{,}5\,\text{см}$
}
\solutionspace{80pt}

\tasknumber{5}%
\task{%
    Линейные размеры прямого изображения предмета, полученного в собирающей линзе,
    в три раза больше линейных размеров предмета.
    Зная, что предмет находится на $25\,\text{см}$ ближе к линзе,
    чем его изображение, найти оптическую силу линзы.
}
\answer{%
    \begin{align*}
    &\text{Если изображение действительное:} \\
    D &= \frac 1F = \frac 1a + \frac 1b, \qquad \Gamma = \frac ba, \qquad b - a = \ell \implies b = \Gamma a \implies \Gamma a - a = \ell \implies  \\
    a &= \frac {\ell}{\Gamma - 1} \implies b = \frac {{\ell} \Gamma}{\Gamma - 1} \implies  \\
    D &= \frac {\Gamma - 1}\ell + \frac {\Gamma - 1}{\ell \Gamma} = \frac 1\ell \cdot \cbr{\Gamma - 1 + \frac {\Gamma - 1}{\Gamma} } =\frac 1\ell \cdot \cbr{\Gamma - \frac 1\Gamma} \approx 10{,}7\,\text{дптр}.
    \\
    &\text{Если изображение мнимое:} \\
    D &= \frac 1F = \frac 1a - \frac 1b, \qquad \Gamma = \frac ba, \qquad b - a = \ell \implies b = \Gamma a \implies \Gamma a - a = \ell \implies  \\
    a &= \frac {\ell}{\Gamma - 1} \implies b = \frac {{\ell} \Gamma}{\Gamma - 1} \implies  \\
    D &= \frac {\Gamma - 1}\ell - \frac {\Gamma - 1}{\ell \Gamma} = \frac 1\ell \cdot \cbr{\Gamma - 1 - \frac {\Gamma - 1}{\Gamma} } =\frac 1\ell \cdot \cbr{\Gamma + \frac 1\Gamma - 2} \approx 5{,}3\,\text{дптр}.
    \\
    &\text{В ответе надо указать оба значения.}
    \end{align*}
}
\solutionspace{120pt}

\tasknumber{6}%
\task{%
    Оптическая сила объектива фотоаппарата равна $5\,\text{дптр}$.
    При фотографировании чертежа с расстояния $0{,}8\,\text{м}$ площадь изображения
    чертежа на фотопластинке оказалась равной $4\,\text{см}^{2}$.
    Какова площадь самого чертежа? Ответ выразите в квадратных сантиметрах.
}
\answer{%
    \begin{align*}
    &\frac 1a + \frac 1b = \frac 1F = D \implies b = \frac{aF}{a - F} \\
    &\frac {S'}S = \Gamma^2 = \sqr{\frac ba} = \sqr{\frac F{a - F}} \implies \\
    &\implies S = S' \cdot \sqr{\frac{a - F}F} = S' \cdot \sqr{\frac aF - 1} = S' \cdot \sqr{aD - 1} \approx 36\,\text{см}^{2}.
    \end{align*}
}


\variantsplitter


\addpersonalvariant{Дарья Кошман}

\tasknumber{7}%
\task{%
    В каком месте на главной оптической оси двояковыпуклой линзы
    нужно поместить точечный источник света,
    чтобы его изображение оказалось в главном фокусе линзы?
}
\answer{%
    $\text{для мнимого - на половине фокусного, для действительного - на бесконечности}$
}
\solutionspace{120pt}

\tasknumber{8}%
\task{%
    Предмет в виде отрезка длиной $\ell$ расположен вдоль оптической оси
    собирающей линзы с фокусным расстоянием $F$.
    Середина отрезка расположена
    на расстоянии $a$ от линзы, которая даёт действительное изображение
    всех точек предмета.
    Определить продольное увеличение предмета.
}
\answer{%
    \begin{align*}
    \frac 1{a + \frac \ell 2} &+ \frac 1b = \frac 1F \implies b = \frac{F\cbr{a + \frac \ell 2}}{a + \frac \ell 2 - F} \\
    \frac 1{a - \frac \ell 2} &+ \frac 1c = \frac 1F \implies c = \frac{F\cbr{a - \frac \ell 2}}{a - \frac \ell 2 - F} \\
    \abs{b - c} &= \abs{\frac{F\cbr{a + \frac \ell 2}}{a + \frac \ell 2 - F} - \frac{F\cbr{a - \frac \ell 2}}{a - \frac \ell 2 - F}}= F\abs{\frac{\cbr{a + \frac \ell 2}\cbr{a - \frac \ell 2 - F} - \cbr{a - \frac \ell 2}\cbr{a + \frac \ell 2 - F}}{ \cbr{a + \frac \ell 2 - F} \cbr{a - \frac \ell 2 - F} }} =  \\
    &= F\abs{\frac{a^2 - \frac {a\ell} 2 - Fa + \frac {a\ell} 2 - \frac {\ell^2} 4 - \frac {F\ell}2 - a^2 - \frac {a\ell}2 + aF + \frac {a\ell}2 + \frac {\ell^2} 4 - \frac {F\ell} 2}{\cbr{a + \frac \ell 2 - F} \cbr{a - \frac \ell 2 - F} }} = \\
    &= F\frac{F\ell}{\sqr{a-F} - \frac {\ell^2}4} = \frac{F^2\ell}{\sqr{a-F} - \frac {\ell^2}4}\implies \Gamma = \frac{\abs{b - c}}\ell = \frac{F^2}{\sqr{a-F} - \frac {\ell^2}4}.
    \end{align*}
}
\solutionspace{120pt}

\tasknumber{9}%
\task{%
    На экране с помощью тонкой линзы получено изображение предмета
    с увеличением $4$.
    Предмет передвинули на $2\,\text{см}$.
    Для того, чтобы получить резкое изображение, пришлось передвинуть экран.
    При этом увеличение оказалось равным $8$.
    На какое расстояние
    пришлось передвинуть экран?
}
\answer{%
    \begin{align*}
    &\frac 1a + \frac 1b = \frac 1F, \Gamma_1 = \frac ba = \frac{F}{a-F} \implies \Gamma_1(a-F) = F \implies a = F \cdot \frac{1 + \Gamma_1}{\Gamma_1} \\
    &\frac 1{a + x} + \frac 1{b + y} = \frac 1F, \Gamma_2 = \frac {b+y}{a+x} = \frac{F}{a+x-F} \implies a + x = F \cdot \frac{1 + \Gamma_2}{\Gamma_2} \\
    &1 + \frac xa = \frac{ \frac{1 + \Gamma_2}{\Gamma_2} }{ \frac{1 + \Gamma_1}{\Gamma_1} } = \frac{\Gamma_1(1 + \Gamma_2)}{\Gamma_2(1 + \Gamma_1)} \\
    &a = \frac x{ \frac{\Gamma_1(1 + \Gamma_2)}{\Gamma_2(1 + \Gamma_1)} - 1} = x \cdot \frac{\Gamma_2(1 + \Gamma_1)}{\Gamma_1 - \Gamma_2} \\
    &y = (a + x)\Gamma_2 - b = (a + x)\Gamma_2 - a\Gamma_1 = a(\Gamma_2 - \Gamma_1) + x\Gamma_2 = -x\Gamma_2(1 + \Gamma_1) + x\Gamma_2 = -x\Gamma_2\Gamma_1 = 64\,\text{см}, \\
    &\text{знаки разные, т.е.
    экран надо было подвинуть в другую сторону чем предмет: $x < 0, y > 0$.}
    \end{align*}
}
\solutionspace{120pt}

\tasknumber{10}%
\task{%
    Тонкая собирающая линза дает изображение предмета на экране при двух положениях линзы между предметом и экраном.
    Высота изображения при первом положении $15\,\text{см}$, во втором — $5\,\text{см}$.
    Расстояние между предметом и экранов постоянно.
    Чему равна высота предмета?
}
\answer{%
    \begin{align*}
    &\frac 1a + \frac 1b = \frac 1F, \frac 1c + \frac 1d = \frac 1F, a + b = c + d \implies \frac{a + b}{ab} = \frac 1F = \frac{c+d}{cd} \implies ab = cd, \\
    &\implies ab = c(a + b - c) \implies c^2 - ac - bc + ab = 0 \implies c = a \text{ или } c = b \implies c = b \implies d = a.
    \\
    &\Gamma_1 = \frac {H_1}H = \frac ba, \Gamma_2 = \frac {H_2}H = \frac dc = \frac ab \implies \frac {H_1}H \cdot \frac {H_2}H = \frac ba \cdot \frac ab = 1, \\
    &H = \sqrt{H_1 H_2} \approx 8{,}7\,\text{см}.
    \end{align*}
}
\solutionspace{120pt}

\tasknumber{11}%
\task{%
    Какие предметы можно рассмотреть на фотографии, сделанной со спутника,
    если разрешающая способность плёнки $0{,}02\,\text{мм}$? Каким должно быть
    время экспозиции $\tau$ чтобы полностью использовать возможности плёнки?
    Фокусное расстояние объектива используемого фотоаппарата $10\,\text{см}$,
    высота орбиты спутника $100\,\text{км}$.
}
\answer{%
    \begin{align*}
    &H \ll R \implies v = v_{\text{I}} = \sqrt{G R} \approx 7{,}9\,\frac{\text{км}}{\text{с}}.
    \\
    &F \ll H \implies b = F, a = H, \\
    &\Gamma = \frac \delta\ell = \frac ba \implies \ell = \frac{\delta a}b = \frac{\delta H}F \approx \frac{0{,}02\,\text{мм} \cdot 100\,\text{км}}{10\,\text{см}} \approx 20\,\text{м}, \\
    &\implies \tau = \frac \ell v = \frac{\delta H}{F v} = \frac{0{,}02\,\text{мм} \cdot 100\,\text{км}}{10\,\text{см} \cdot 7{,}9\,\frac{\text{км}}{\text{с}}} \approx 3\,\text{мс}.
    \end{align*}
}


\variantsplitter


\addpersonalvariant{Дарья Кошман}

\tasknumber{12}%
\task{%
    При аэрофотосъемках используется фотоаппарат, объектив которого
    имеет фокусиое расстояние $10\,\text{см}$.
    Разрешающая способность плёнки $0{,}010\,\text{мм}$.
    На какой высоте должен лететь самолет, чтобы на фотографии можно
    было различить следы размером $25\,\text{см}$?
    При какой скорости самолета изображение не будет размытым,
    если время экспозиции $2\,\text{мс}$?
}
\answer{%
    \begin{align*}
    &F \ll H \implies b = F, a = H, \\
    &\Gamma = \frac \delta\ell = \frac ba = \frac FH \implies H = \frac{\ell F}\delta = \frac{25\,\text{см} \cdot 10\,\text{см}}{0{,}010\,\text{мм}} \approx 3\,\text{км}, \\
    &\implies v = \frac l\tau = \frac{25\,\text{см}}{2\,\text{мс}} \approx 450\,\frac{\text{км}}{\text{ч}}.
    \end{align*}
}
\solutionspace{120pt}

\tasknumber{13}%
\task{%
    Две одинаковые собиращие линзы установлены так, что их главные оптические оси совпадают,
    а главный фокус первой находится там же, где главный фокус второй.
    Расстояние от первой линзы до предмета равно $24\,\text{см}$.
    Чему равно расстояние от изображения объекта во второй линзе до второй линзы?
    Определите также увеличение.
    Фокусное расстояние каждой линзы $30\,\text{см}$.
}
\answer{%
    \begin{align*}
    \frac 1a + \frac 1b &= \frac 1F \implies b = \frac{aF}{a - F} \implies 2F - b = \frac{2aF - 2F^2 - aF}{a - F} = \frac{F(a - 2F)}{a - F}.
    \\
    \frac 1{2F - b} + \frac 1c &= \frac 1F \implies c = \frac{F(2F-b)}{(2F - b) - F} = \frac{F \cdot \frac{F(a - 2F)}{a - F}}{\frac{F(a - 2F)}{a - F} - F}  = F \cdot \frac{ \frac{F(a - 2F)}{a - F} }{ \frac{F(a - 2F)}{a - F} - 1} = \\
     &= F \cdot \frac{a - 2F}{a - 2F - a + F} = 2F - a = 36\,\text{см}.
     \\
    \ell &= a + 2F + c = 4F = 120\,\text{см}.
    \\
    &\Gamma = \Gamma_1 \cdot \Gamma_2 = \frac ba \cdot \frac c{2F-b} = \frac F{a - F} \cdot \frac{2F - a}{\frac{F(a - 2F)}{a - F}} = -1.
    \end{align*}
}
\solutionspace{120pt}

\tasknumber{14}%
\task{%
    Собирающая линза с фокусным расстоянием $F_1 > 0$ и рассеивающая линза с фокусным расстоянием $F_2 < 0$
    установлены коаксиально на расстоянии $\ell$.
    Пучок параллельных лучей падает на собирающую линзу.
    Сделайте схематичное построение и определите, в какой точке система из этих линз соберёт пучок.
}
\answer{%
    \begin{align*}
    &\text{Если пучок падает на собирающую линзу:} \\
    \frac 1{\infty} + \frac 1b &= \frac 1{F_1} \implies b = F_1 \implies \ell - b = \ell - F_1 \\
    \frac 1{\ell - b} + \frac 1c &= \frac 1{F_2} \implies c = \frac{F_2(\ell - b)}{\ell - b - F_2} = \frac{F_2(\ell - F_1)}{\ell - F_1 - F_2}.
    \\
    &\text{Если же пучок падает на рассеивающую линзу:} \\
    \frac 1{\infty} + \frac 1b &= \frac 1{F_2} \implies b = F_2 \implies \ell - b = \ell - F_2 \\
    \frac 1{\ell - b} + \frac 1c &= \frac 1{F_1} \implies c = \frac{F_1(\ell - b)}{\ell - b - F_1} = \frac{F_1(\ell - F_2)}{\ell - F_2 - F_1}.
    \end{align*}
}
\solutionspace{120pt}

\tasknumber{15}%
\task{%
    Две собирающих линзы с фокусными расстояниями $20\,\text{см}$ и $45\,\text{см}$ расположены так,
    что их оптические оси совмещены.
    На первую линзу падает пучок параллельных лучей.
    Пройдя через вторую линзу, он остался параллельным.
    Найдите расстояние между линзами и сделайте рисунок.
}
\answer{%
    \begin{align*}
    \frac 1\infty + \frac 1b &= \frac 1{F_1} \implies b = F_1, \\
    \frac 1{\ell - b} + \frac 1{\infty} &= \frac 1{F_2} \implies \ell - b = F_2 \implies \ell = b + F_2 = F_1 + F_2 = 65\,\text{см}.
    \end{align*}
}

\variantsplitter

\addpersonalvariant{Анна Кузьмичёва}

\tasknumber{1}%
\task{%
    Найти оптическую силу собирающей линзы, если действительное изображение предмета,
    помещённого в $35\,\text{см}$ от линзы, получается на расстоянии $40\,\text{см}$ от неё.
}
\answer{%
    $D = \frac 1F = \frac 1a + \frac 1b = \frac 1{35\,\text{см}} + \frac 1{40\,\text{см}} \approx 5{,}36\,\text{дптр}$
}
\solutionspace{80pt}

\tasknumber{2}%
\task{%
    Найти увеличение изображения, если изображение предмета, находящегося
    на расстоянии $20\,\text{см}$ от линзы, получается на расстоянии $18\,\text{см}$ от неё.
}
\answer{%
    $\Gamma = \frac ba = \frac {18\,\text{см}}{20\,\text{см}} \approx 0{,}9$
}
\solutionspace{80pt}

\tasknumber{3}%
\task{%
    Расстояние от предмета до линзы $10\,\text{см}$, а от линзы до мнимого изображения $25\,\text{см}$.
    Чему равно фокусное расстояние линзы?
}
\answer{%
    $\pm \frac 1F = \frac 1a - \frac 1b \implies F = \frac{a b}{\abs{b - a}} \approx 16{,}7\,\text{см}$
}
\solutionspace{80pt}

\tasknumber{4}%
\task{%
    Две тонкие собирающие линзы с фокусными расстояниями $25\,\text{см}$ и $30\,\text{см}$ сложены вместе.
    Чему равно фокусное расстояние такой оптической системы?
}
\answer{%
    $\frac 1{f_1} = \frac 1a + \frac 1b; \frac 1{f_2} = - \frac 1b + \frac 1c \implies \frac 1{f_1} + \frac 1{f_2} = \frac 1a + \frac 1c \implies f' = \frac 1{\frac 1{f_1} + \frac 1{f_2}} = \frac{f_1 f_2}{f_1 + f_2} \approx 13{,}6\,\text{см}$
}
\solutionspace{80pt}

\tasknumber{5}%
\task{%
    Линейные размеры прямого изображения предмета, полученного в собирающей линзе,
    в два раза больше линейных размеров предмета.
    Зная, что предмет находится на $30\,\text{см}$ ближе к линзе,
    чем его изображение, найти оптическую силу линзы.
}
\answer{%
    \begin{align*}
    &\text{Если изображение действительное:} \\
    D &= \frac 1F = \frac 1a + \frac 1b, \qquad \Gamma = \frac ba, \qquad b - a = \ell \implies b = \Gamma a \implies \Gamma a - a = \ell \implies  \\
    a &= \frac {\ell}{\Gamma - 1} \implies b = \frac {{\ell} \Gamma}{\Gamma - 1} \implies  \\
    D &= \frac {\Gamma - 1}\ell + \frac {\Gamma - 1}{\ell \Gamma} = \frac 1\ell \cdot \cbr{\Gamma - 1 + \frac {\Gamma - 1}{\Gamma} } =\frac 1\ell \cdot \cbr{\Gamma - \frac 1\Gamma} \approx 5\,\text{дптр}.
    \\
    &\text{Если изображение мнимое:} \\
    D &= \frac 1F = \frac 1a - \frac 1b, \qquad \Gamma = \frac ba, \qquad b - a = \ell \implies b = \Gamma a \implies \Gamma a - a = \ell \implies  \\
    a &= \frac {\ell}{\Gamma - 1} \implies b = \frac {{\ell} \Gamma}{\Gamma - 1} \implies  \\
    D &= \frac {\Gamma - 1}\ell - \frac {\Gamma - 1}{\ell \Gamma} = \frac 1\ell \cdot \cbr{\Gamma - 1 - \frac {\Gamma - 1}{\Gamma} } =\frac 1\ell \cdot \cbr{\Gamma + \frac 1\Gamma - 2} \approx 1{,}7\,\text{дптр}.
    \\
    &\text{В ответе надо указать оба значения.}
    \end{align*}
}
\solutionspace{120pt}

\tasknumber{6}%
\task{%
    Оптическая сила объектива фотоаппарата равна $5\,\text{дптр}$.
    При фотографировании чертежа с расстояния $1{,}2\,\text{м}$ площадь изображения
    чертежа на фотопластинке оказалась равной $9\,\text{см}^{2}$.
    Какова площадь самого чертежа? Ответ выразите в квадратных сантиметрах.
}
\answer{%
    \begin{align*}
    &\frac 1a + \frac 1b = \frac 1F = D \implies b = \frac{aF}{a - F} \\
    &\frac {S'}S = \Gamma^2 = \sqr{\frac ba} = \sqr{\frac F{a - F}} \implies \\
    &\implies S = S' \cdot \sqr{\frac{a - F}F} = S' \cdot \sqr{\frac aF - 1} = S' \cdot \sqr{aD - 1} \approx 225\,\text{см}^{2}.
    \end{align*}
}


\variantsplitter


\addpersonalvariant{Анна Кузьмичёва}

\tasknumber{7}%
\task{%
    В каком месте на главной оптической оси двояковыпуклой линзы
    нужно поместить точечный источник света,
    чтобы его изображение оказалось в главном фокусе линзы?
}
\answer{%
    $\text{для мнимого - на половине фокусного, для действительного - на бесконечности}$
}
\solutionspace{120pt}

\tasknumber{8}%
\task{%
    Предмет в виде отрезка длиной $\ell$ расположен вдоль оптической оси
    собирающей линзы с фокусным расстоянием $F$.
    Середина отрезка расположена
    на расстоянии $a$ от линзы, которая даёт действительное изображение
    всех точек предмета.
    Определить продольное увеличение предмета.
}
\answer{%
    \begin{align*}
    \frac 1{a + \frac \ell 2} &+ \frac 1b = \frac 1F \implies b = \frac{F\cbr{a + \frac \ell 2}}{a + \frac \ell 2 - F} \\
    \frac 1{a - \frac \ell 2} &+ \frac 1c = \frac 1F \implies c = \frac{F\cbr{a - \frac \ell 2}}{a - \frac \ell 2 - F} \\
    \abs{b - c} &= \abs{\frac{F\cbr{a + \frac \ell 2}}{a + \frac \ell 2 - F} - \frac{F\cbr{a - \frac \ell 2}}{a - \frac \ell 2 - F}}= F\abs{\frac{\cbr{a + \frac \ell 2}\cbr{a - \frac \ell 2 - F} - \cbr{a - \frac \ell 2}\cbr{a + \frac \ell 2 - F}}{ \cbr{a + \frac \ell 2 - F} \cbr{a - \frac \ell 2 - F} }} =  \\
    &= F\abs{\frac{a^2 - \frac {a\ell} 2 - Fa + \frac {a\ell} 2 - \frac {\ell^2} 4 - \frac {F\ell}2 - a^2 - \frac {a\ell}2 + aF + \frac {a\ell}2 + \frac {\ell^2} 4 - \frac {F\ell} 2}{\cbr{a + \frac \ell 2 - F} \cbr{a - \frac \ell 2 - F} }} = \\
    &= F\frac{F\ell}{\sqr{a-F} - \frac {\ell^2}4} = \frac{F^2\ell}{\sqr{a-F} - \frac {\ell^2}4}\implies \Gamma = \frac{\abs{b - c}}\ell = \frac{F^2}{\sqr{a-F} - \frac {\ell^2}4}.
    \end{align*}
}
\solutionspace{120pt}

\tasknumber{9}%
\task{%
    На экране с помощью тонкой линзы получено изображение предмета
    с увеличением $4$.
    Предмет передвинули на $10\,\text{см}$.
    Для того, чтобы получить резкое изображение, пришлось передвинуть экран.
    При этом увеличение оказалось равным $8$.
    На какое расстояние
    пришлось передвинуть экран?
}
\answer{%
    \begin{align*}
    &\frac 1a + \frac 1b = \frac 1F, \Gamma_1 = \frac ba = \frac{F}{a-F} \implies \Gamma_1(a-F) = F \implies a = F \cdot \frac{1 + \Gamma_1}{\Gamma_1} \\
    &\frac 1{a + x} + \frac 1{b + y} = \frac 1F, \Gamma_2 = \frac {b+y}{a+x} = \frac{F}{a+x-F} \implies a + x = F \cdot \frac{1 + \Gamma_2}{\Gamma_2} \\
    &1 + \frac xa = \frac{ \frac{1 + \Gamma_2}{\Gamma_2} }{ \frac{1 + \Gamma_1}{\Gamma_1} } = \frac{\Gamma_1(1 + \Gamma_2)}{\Gamma_2(1 + \Gamma_1)} \\
    &a = \frac x{ \frac{\Gamma_1(1 + \Gamma_2)}{\Gamma_2(1 + \Gamma_1)} - 1} = x \cdot \frac{\Gamma_2(1 + \Gamma_1)}{\Gamma_1 - \Gamma_2} \\
    &y = (a + x)\Gamma_2 - b = (a + x)\Gamma_2 - a\Gamma_1 = a(\Gamma_2 - \Gamma_1) + x\Gamma_2 = -x\Gamma_2(1 + \Gamma_1) + x\Gamma_2 = -x\Gamma_2\Gamma_1 = 320\,\text{см}, \\
    &\text{знаки разные, т.е.
    экран надо было подвинуть в другую сторону чем предмет: $x < 0, y > 0$.}
    \end{align*}
}
\solutionspace{120pt}

\tasknumber{10}%
\task{%
    Тонкая собирающая линза дает изображение предмета на экране при двух положениях линзы между предметом и экраном.
    Высота изображения при первом положении $20\,\text{см}$, во втором — $5\,\text{см}$.
    Расстояние между предметом и экранов постоянно.
    Чему равна высота предмета?
}
\answer{%
    \begin{align*}
    &\frac 1a + \frac 1b = \frac 1F, \frac 1c + \frac 1d = \frac 1F, a + b = c + d \implies \frac{a + b}{ab} = \frac 1F = \frac{c+d}{cd} \implies ab = cd, \\
    &\implies ab = c(a + b - c) \implies c^2 - ac - bc + ab = 0 \implies c = a \text{ или } c = b \implies c = b \implies d = a.
    \\
    &\Gamma_1 = \frac {H_1}H = \frac ba, \Gamma_2 = \frac {H_2}H = \frac dc = \frac ab \implies \frac {H_1}H \cdot \frac {H_2}H = \frac ba \cdot \frac ab = 1, \\
    &H = \sqrt{H_1 H_2} \approx 10\,\text{см}.
    \end{align*}
}
\solutionspace{120pt}

\tasknumber{11}%
\task{%
    Какие предметы можно рассмотреть на фотографии, сделанной со спутника,
    если разрешающая способность плёнки $0{,}02\,\text{мм}$? Каким должно быть
    время экспозиции $\tau$ чтобы полностью использовать возможности плёнки?
    Фокусное расстояние объектива используемого фотоаппарата $20\,\text{см}$,
    высота орбиты спутника $150\,\text{км}$.
}
\answer{%
    \begin{align*}
    &H \ll R \implies v = v_{\text{I}} = \sqrt{G R} \approx 7{,}9\,\frac{\text{км}}{\text{с}}.
    \\
    &F \ll H \implies b = F, a = H, \\
    &\Gamma = \frac \delta\ell = \frac ba \implies \ell = \frac{\delta a}b = \frac{\delta H}F \approx \frac{0{,}02\,\text{мм} \cdot 150\,\text{км}}{20\,\text{см}} \approx 15\,\text{м}, \\
    &\implies \tau = \frac \ell v = \frac{\delta H}{F v} = \frac{0{,}02\,\text{мм} \cdot 150\,\text{км}}{20\,\text{см} \cdot 7{,}9\,\frac{\text{км}}{\text{с}}} \approx 1{,}9\,\text{мс}.
    \end{align*}
}


\variantsplitter


\addpersonalvariant{Анна Кузьмичёва}

\tasknumber{12}%
\task{%
    При аэрофотосъемках используется фотоаппарат, объектив которого
    имеет фокусиое расстояние $15\,\text{см}$.
    Разрешающая способность плёнки $0{,}02\,\text{мм}$.
    На какой высоте должен лететь самолет, чтобы на фотографии можно
    было различить следы размером $25\,\text{см}$?
    При какой скорости самолета изображение не будет размытым,
    если время экспозиции $1\,\text{мс}$?
}
\answer{%
    \begin{align*}
    &F \ll H \implies b = F, a = H, \\
    &\Gamma = \frac \delta\ell = \frac ba = \frac FH \implies H = \frac{\ell F}\delta = \frac{25\,\text{см} \cdot 15\,\text{см}}{0{,}02\,\text{мм}} \approx 1{,}9\,\text{км}, \\
    &\implies v = \frac l\tau = \frac{25\,\text{см}}{1\,\text{мс}} \approx 900\,\frac{\text{км}}{\text{ч}}.
    \end{align*}
}
\solutionspace{120pt}

\tasknumber{13}%
\task{%
    Две одинаковые собиращие линзы установлены так, что их главные оптические оси совпадают,
    а главный фокус первой находится там же, где главный фокус второй.
    Расстояние от первой линзы до предмета равно $29\,\text{см}$.
    Чему равно расстояние от изображения объекта во второй линзе до самого объекта?
    Определите также увеличение.
    Фокусное расстояние каждой линзы $20\,\text{см}$.
}
\answer{%
    \begin{align*}
    \frac 1a + \frac 1b &= \frac 1F \implies b = \frac{aF}{a - F} \implies 2F - b = \frac{2aF - 2F^2 - aF}{a - F} = \frac{F(a - 2F)}{a - F}.
    \\
    \frac 1{2F - b} + \frac 1c &= \frac 1F \implies c = \frac{F(2F-b)}{(2F - b) - F} = \frac{F \cdot \frac{F(a - 2F)}{a - F}}{\frac{F(a - 2F)}{a - F} - F}  = F \cdot \frac{ \frac{F(a - 2F)}{a - F} }{ \frac{F(a - 2F)}{a - F} - 1} = \\
     &= F \cdot \frac{a - 2F}{a - 2F - a + F} = 2F - a = 11\,\text{см}.
     \\
    \ell &= a + 2F + c = 4F = 80\,\text{см}.
    \\
    &\Gamma = \Gamma_1 \cdot \Gamma_2 = \frac ba \cdot \frac c{2F-b} = \frac F{a - F} \cdot \frac{2F - a}{\frac{F(a - 2F)}{a - F}} = -1.
    \end{align*}
}
\solutionspace{120pt}

\tasknumber{14}%
\task{%
    Собирающая линза с фокусным расстоянием $F_1 > 0$ и рассеивающая линза с фокусным расстоянием $F_2 < 0$
    установлены коаксиально на расстоянии $\ell$.
    Пучок параллельных лучей падает на собирающую линзу.
    Сделайте схематичное построение и определите, в какой точке система из этих линз соберёт пучок.
}
\answer{%
    \begin{align*}
    &\text{Если пучок падает на собирающую линзу:} \\
    \frac 1{\infty} + \frac 1b &= \frac 1{F_1} \implies b = F_1 \implies \ell - b = \ell - F_1 \\
    \frac 1{\ell - b} + \frac 1c &= \frac 1{F_2} \implies c = \frac{F_2(\ell - b)}{\ell - b - F_2} = \frac{F_2(\ell - F_1)}{\ell - F_1 - F_2}.
    \\
    &\text{Если же пучок падает на рассеивающую линзу:} \\
    \frac 1{\infty} + \frac 1b &= \frac 1{F_2} \implies b = F_2 \implies \ell - b = \ell - F_2 \\
    \frac 1{\ell - b} + \frac 1c &= \frac 1{F_1} \implies c = \frac{F_1(\ell - b)}{\ell - b - F_1} = \frac{F_1(\ell - F_2)}{\ell - F_2 - F_1}.
    \end{align*}
}
\solutionspace{120pt}

\tasknumber{15}%
\task{%
    Две собирающих линзы с фокусными расстояниями $30\,\text{см}$ и $35\,\text{см}$ расположены так,
    что их оптические оси совмещены.
    На первую линзу падает пучок параллельных лучей.
    Пройдя через вторую линзу, он остался параллельным.
    Найдите расстояние между линзами и сделайте рисунок.
}
\answer{%
    \begin{align*}
    \frac 1\infty + \frac 1b &= \frac 1{F_1} \implies b = F_1, \\
    \frac 1{\ell - b} + \frac 1{\infty} &= \frac 1{F_2} \implies \ell - b = F_2 \implies \ell = b + F_2 = F_1 + F_2 = 65\,\text{см}.
    \end{align*}
}

\variantsplitter

\addpersonalvariant{Алёна Куприянова}

\tasknumber{1}%
\task{%
    Найти оптическую силу собирающей линзы, если действительное изображение предмета,
    помещённого в $55\,\text{см}$ от линзы, получается на расстоянии $30\,\text{см}$ от неё.
}
\answer{%
    $D = \frac 1F = \frac 1a + \frac 1b = \frac 1{55\,\text{см}} + \frac 1{30\,\text{см}} \approx 5{,}15\,\text{дптр}$
}
\solutionspace{80pt}

\tasknumber{2}%
\task{%
    Найти увеличение изображения, если изображение предмета, находящегося
    на расстоянии $20\,\text{см}$ от линзы, получается на расстоянии $30\,\text{см}$ от неё.
}
\answer{%
    $\Gamma = \frac ba = \frac {30\,\text{см}}{20\,\text{см}} \approx 1{,}50$
}
\solutionspace{80pt}

\tasknumber{3}%
\task{%
    Расстояние от предмета до линзы $10\,\text{см}$, а от линзы до мнимого изображения $30\,\text{см}$.
    Чему равно фокусное расстояние линзы?
}
\answer{%
    $\pm \frac 1F = \frac 1a - \frac 1b \implies F = \frac{a b}{\abs{b - a}} \approx 15\,\text{см}$
}
\solutionspace{80pt}

\tasknumber{4}%
\task{%
    Две тонкие собирающие линзы с фокусными расстояниями $18\,\text{см}$ и $20\,\text{см}$ сложены вместе.
    Чему равно фокусное расстояние такой оптической системы?
}
\answer{%
    $\frac 1{f_1} = \frac 1a + \frac 1b; \frac 1{f_2} = - \frac 1b + \frac 1c \implies \frac 1{f_1} + \frac 1{f_2} = \frac 1a + \frac 1c \implies f' = \frac 1{\frac 1{f_1} + \frac 1{f_2}} = \frac{f_1 f_2}{f_1 + f_2} \approx 9{,}5\,\text{см}$
}
\solutionspace{80pt}

\tasknumber{5}%
\task{%
    Линейные размеры прямого изображения предмета, полученного в собирающей линзе,
    в два раза больше линейных размеров предмета.
    Зная, что предмет находится на $30\,\text{см}$ ближе к линзе,
    чем его изображение, найти оптическую силу линзы.
}
\answer{%
    \begin{align*}
    &\text{Если изображение действительное:} \\
    D &= \frac 1F = \frac 1a + \frac 1b, \qquad \Gamma = \frac ba, \qquad b - a = \ell \implies b = \Gamma a \implies \Gamma a - a = \ell \implies  \\
    a &= \frac {\ell}{\Gamma - 1} \implies b = \frac {{\ell} \Gamma}{\Gamma - 1} \implies  \\
    D &= \frac {\Gamma - 1}\ell + \frac {\Gamma - 1}{\ell \Gamma} = \frac 1\ell \cdot \cbr{\Gamma - 1 + \frac {\Gamma - 1}{\Gamma} } =\frac 1\ell \cdot \cbr{\Gamma - \frac 1\Gamma} \approx 5\,\text{дптр}.
    \\
    &\text{Если изображение мнимое:} \\
    D &= \frac 1F = \frac 1a - \frac 1b, \qquad \Gamma = \frac ba, \qquad b - a = \ell \implies b = \Gamma a \implies \Gamma a - a = \ell \implies  \\
    a &= \frac {\ell}{\Gamma - 1} \implies b = \frac {{\ell} \Gamma}{\Gamma - 1} \implies  \\
    D &= \frac {\Gamma - 1}\ell - \frac {\Gamma - 1}{\ell \Gamma} = \frac 1\ell \cdot \cbr{\Gamma - 1 - \frac {\Gamma - 1}{\Gamma} } =\frac 1\ell \cdot \cbr{\Gamma + \frac 1\Gamma - 2} \approx 1{,}7\,\text{дптр}.
    \\
    &\text{В ответе надо указать оба значения.}
    \end{align*}
}
\solutionspace{120pt}

\tasknumber{6}%
\task{%
    Оптическая сила объектива фотоаппарата равна $6\,\text{дптр}$.
    При фотографировании чертежа с расстояния $1{,}1\,\text{м}$ площадь изображения
    чертежа на фотопластинке оказалась равной $4\,\text{см}^{2}$.
    Какова площадь самого чертежа? Ответ выразите в квадратных сантиметрах.
}
\answer{%
    \begin{align*}
    &\frac 1a + \frac 1b = \frac 1F = D \implies b = \frac{aF}{a - F} \\
    &\frac {S'}S = \Gamma^2 = \sqr{\frac ba} = \sqr{\frac F{a - F}} \implies \\
    &\implies S = S' \cdot \sqr{\frac{a - F}F} = S' \cdot \sqr{\frac aF - 1} = S' \cdot \sqr{aD - 1} \approx 130\,\text{см}^{2}.
    \end{align*}
}


\variantsplitter


\addpersonalvariant{Алёна Куприянова}

\tasknumber{7}%
\task{%
    В каком месте на главной оптической оси двояковыпуклой линзы
    нужно поместить точечный источник света,
    чтобы его изображение оказалось в главном фокусе линзы?
}
\answer{%
    $\text{для мнимого - на половине фокусного, для действительного - на бесконечности}$
}
\solutionspace{120pt}

\tasknumber{8}%
\task{%
    Предмет в виде отрезка длиной $\ell$ расположен вдоль оптической оси
    собирающей линзы с фокусным расстоянием $F$.
    Середина отрезка расположена
    на расстоянии $a$ от линзы, которая даёт действительное изображение
    всех точек предмета.
    Определить продольное увеличение предмета.
}
\answer{%
    \begin{align*}
    \frac 1{a + \frac \ell 2} &+ \frac 1b = \frac 1F \implies b = \frac{F\cbr{a + \frac \ell 2}}{a + \frac \ell 2 - F} \\
    \frac 1{a - \frac \ell 2} &+ \frac 1c = \frac 1F \implies c = \frac{F\cbr{a - \frac \ell 2}}{a - \frac \ell 2 - F} \\
    \abs{b - c} &= \abs{\frac{F\cbr{a + \frac \ell 2}}{a + \frac \ell 2 - F} - \frac{F\cbr{a - \frac \ell 2}}{a - \frac \ell 2 - F}}= F\abs{\frac{\cbr{a + \frac \ell 2}\cbr{a - \frac \ell 2 - F} - \cbr{a - \frac \ell 2}\cbr{a + \frac \ell 2 - F}}{ \cbr{a + \frac \ell 2 - F} \cbr{a - \frac \ell 2 - F} }} =  \\
    &= F\abs{\frac{a^2 - \frac {a\ell} 2 - Fa + \frac {a\ell} 2 - \frac {\ell^2} 4 - \frac {F\ell}2 - a^2 - \frac {a\ell}2 + aF + \frac {a\ell}2 + \frac {\ell^2} 4 - \frac {F\ell} 2}{\cbr{a + \frac \ell 2 - F} \cbr{a - \frac \ell 2 - F} }} = \\
    &= F\frac{F\ell}{\sqr{a-F} - \frac {\ell^2}4} = \frac{F^2\ell}{\sqr{a-F} - \frac {\ell^2}4}\implies \Gamma = \frac{\abs{b - c}}\ell = \frac{F^2}{\sqr{a-F} - \frac {\ell^2}4}.
    \end{align*}
}
\solutionspace{120pt}

\tasknumber{9}%
\task{%
    На экране с помощью тонкой линзы получено изображение предмета
    с увеличением $2$.
    Предмет передвинули на $10\,\text{см}$.
    Для того, чтобы получить резкое изображение, пришлось передвинуть экран.
    При этом увеличение оказалось равным $8$.
    На какое расстояние
    пришлось передвинуть экран?
}
\answer{%
    \begin{align*}
    &\frac 1a + \frac 1b = \frac 1F, \Gamma_1 = \frac ba = \frac{F}{a-F} \implies \Gamma_1(a-F) = F \implies a = F \cdot \frac{1 + \Gamma_1}{\Gamma_1} \\
    &\frac 1{a + x} + \frac 1{b + y} = \frac 1F, \Gamma_2 = \frac {b+y}{a+x} = \frac{F}{a+x-F} \implies a + x = F \cdot \frac{1 + \Gamma_2}{\Gamma_2} \\
    &1 + \frac xa = \frac{ \frac{1 + \Gamma_2}{\Gamma_2} }{ \frac{1 + \Gamma_1}{\Gamma_1} } = \frac{\Gamma_1(1 + \Gamma_2)}{\Gamma_2(1 + \Gamma_1)} \\
    &a = \frac x{ \frac{\Gamma_1(1 + \Gamma_2)}{\Gamma_2(1 + \Gamma_1)} - 1} = x \cdot \frac{\Gamma_2(1 + \Gamma_1)}{\Gamma_1 - \Gamma_2} \\
    &y = (a + x)\Gamma_2 - b = (a + x)\Gamma_2 - a\Gamma_1 = a(\Gamma_2 - \Gamma_1) + x\Gamma_2 = -x\Gamma_2(1 + \Gamma_1) + x\Gamma_2 = -x\Gamma_2\Gamma_1 = 160\,\text{см}, \\
    &\text{знаки разные, т.е.
    экран надо было подвинуть в другую сторону чем предмет: $x < 0, y > 0$.}
    \end{align*}
}
\solutionspace{120pt}

\tasknumber{10}%
\task{%
    Тонкая собирающая линза дает изображение предмета на экране при двух положениях линзы между предметом и экраном.
    Высота изображения при первом положении $30\,\text{см}$, во втором — $7\,\text{см}$.
    Расстояние между предметом и экранов постоянно.
    Чему равна высота предмета?
}
\answer{%
    \begin{align*}
    &\frac 1a + \frac 1b = \frac 1F, \frac 1c + \frac 1d = \frac 1F, a + b = c + d \implies \frac{a + b}{ab} = \frac 1F = \frac{c+d}{cd} \implies ab = cd, \\
    &\implies ab = c(a + b - c) \implies c^2 - ac - bc + ab = 0 \implies c = a \text{ или } c = b \implies c = b \implies d = a.
    \\
    &\Gamma_1 = \frac {H_1}H = \frac ba, \Gamma_2 = \frac {H_2}H = \frac dc = \frac ab \implies \frac {H_1}H \cdot \frac {H_2}H = \frac ba \cdot \frac ab = 1, \\
    &H = \sqrt{H_1 H_2} \approx 14{,}5\,\text{см}.
    \end{align*}
}
\solutionspace{120pt}

\tasknumber{11}%
\task{%
    Какие предметы можно рассмотреть на фотографии, сделанной со спутника,
    если разрешающая способность плёнки $0{,}02\,\text{мм}$? Каким должно быть
    время экспозиции $\tau$ чтобы полностью использовать возможности плёнки?
    Фокусное расстояние объектива используемого фотоаппарата $15\,\text{см}$,
    высота орбиты спутника $150\,\text{км}$.
}
\answer{%
    \begin{align*}
    &H \ll R \implies v = v_{\text{I}} = \sqrt{G R} \approx 7{,}9\,\frac{\text{км}}{\text{с}}.
    \\
    &F \ll H \implies b = F, a = H, \\
    &\Gamma = \frac \delta\ell = \frac ba \implies \ell = \frac{\delta a}b = \frac{\delta H}F \approx \frac{0{,}02\,\text{мм} \cdot 150\,\text{км}}{15\,\text{см}} \approx 20\,\text{м}, \\
    &\implies \tau = \frac \ell v = \frac{\delta H}{F v} = \frac{0{,}02\,\text{мм} \cdot 150\,\text{км}}{15\,\text{см} \cdot 7{,}9\,\frac{\text{км}}{\text{с}}} \approx 3\,\text{мс}.
    \end{align*}
}


\variantsplitter


\addpersonalvariant{Алёна Куприянова}

\tasknumber{12}%
\task{%
    При аэрофотосъемках используется фотоаппарат, объектив которого
    имеет фокусиое расстояние $10\,\text{см}$.
    Разрешающая способность плёнки $0{,}02\,\text{мм}$.
    На какой высоте должен лететь самолет, чтобы на фотографии можно
    было различить следы размером $15\,\text{см}$?
    При какой скорости самолета изображение не будет размытым,
    если время экспозиции $1\,\text{мс}$?
}
\answer{%
    \begin{align*}
    &F \ll H \implies b = F, a = H, \\
    &\Gamma = \frac \delta\ell = \frac ba = \frac FH \implies H = \frac{\ell F}\delta = \frac{15\,\text{см} \cdot 10\,\text{см}}{0{,}02\,\text{мм}} \approx 0{,}8\,\text{км}, \\
    &\implies v = \frac l\tau = \frac{15\,\text{см}}{1\,\text{мс}} \approx 540\,\frac{\text{км}}{\text{ч}}.
    \end{align*}
}
\solutionspace{120pt}

\tasknumber{13}%
\task{%
    Две одинаковые собиращие линзы установлены так, что их главные оптические оси совпадают,
    а главный фокус первой находится там же, где главный фокус второй.
    Расстояние от первой линзы до предмета равно $17\,\text{см}$.
    Чему равно расстояние от изображения объекта во второй линзе до второй линзы?
    Определите также увеличение.
    Фокусное расстояние каждой линзы $25\,\text{см}$.
}
\answer{%
    \begin{align*}
    \frac 1a + \frac 1b &= \frac 1F \implies b = \frac{aF}{a - F} \implies 2F - b = \frac{2aF - 2F^2 - aF}{a - F} = \frac{F(a - 2F)}{a - F}.
    \\
    \frac 1{2F - b} + \frac 1c &= \frac 1F \implies c = \frac{F(2F-b)}{(2F - b) - F} = \frac{F \cdot \frac{F(a - 2F)}{a - F}}{\frac{F(a - 2F)}{a - F} - F}  = F \cdot \frac{ \frac{F(a - 2F)}{a - F} }{ \frac{F(a - 2F)}{a - F} - 1} = \\
     &= F \cdot \frac{a - 2F}{a - 2F - a + F} = 2F - a = 33\,\text{см}.
     \\
    \ell &= a + 2F + c = 4F = 100\,\text{см}.
    \\
    &\Gamma = \Gamma_1 \cdot \Gamma_2 = \frac ba \cdot \frac c{2F-b} = \frac F{a - F} \cdot \frac{2F - a}{\frac{F(a - 2F)}{a - F}} = -1.
    \end{align*}
}
\solutionspace{120pt}

\tasknumber{14}%
\task{%
    Собирающая линза с фокусным расстоянием $F_1 > 0$ и рассеивающая линза с фокусным расстоянием $F_2 < 0$
    установлены коаксиально на расстоянии $\ell$.
    Пучок параллельных лучей падает на собирающую линзу.
    Сделайте схематичное построение и определите, в какой точке система из этих линз соберёт пучок.
}
\answer{%
    \begin{align*}
    &\text{Если пучок падает на собирающую линзу:} \\
    \frac 1{\infty} + \frac 1b &= \frac 1{F_1} \implies b = F_1 \implies \ell - b = \ell - F_1 \\
    \frac 1{\ell - b} + \frac 1c &= \frac 1{F_2} \implies c = \frac{F_2(\ell - b)}{\ell - b - F_2} = \frac{F_2(\ell - F_1)}{\ell - F_1 - F_2}.
    \\
    &\text{Если же пучок падает на рассеивающую линзу:} \\
    \frac 1{\infty} + \frac 1b &= \frac 1{F_2} \implies b = F_2 \implies \ell - b = \ell - F_2 \\
    \frac 1{\ell - b} + \frac 1c &= \frac 1{F_1} \implies c = \frac{F_1(\ell - b)}{\ell - b - F_1} = \frac{F_1(\ell - F_2)}{\ell - F_2 - F_1}.
    \end{align*}
}
\solutionspace{120pt}

\tasknumber{15}%
\task{%
    Две собирающих линзы с фокусными расстояниями $30\,\text{см}$ и $35\,\text{см}$ расположены так,
    что их оптические оси совмещены.
    На первую линзу падает пучок параллельных лучей.
    Пройдя через вторую линзу, он остался параллельным.
    Найдите расстояние между линзами и сделайте рисунок.
}
\answer{%
    \begin{align*}
    \frac 1\infty + \frac 1b &= \frac 1{F_1} \implies b = F_1, \\
    \frac 1{\ell - b} + \frac 1{\infty} &= \frac 1{F_2} \implies \ell - b = F_2 \implies \ell = b + F_2 = F_1 + F_2 = 65\,\text{см}.
    \end{align*}
}

\variantsplitter

\addpersonalvariant{Ярослав Лавровский}

\tasknumber{1}%
\task{%
    Найти оптическую силу собирающей линзы, если действительное изображение предмета,
    помещённого в $15\,\text{см}$ от линзы, получается на расстоянии $30\,\text{см}$ от неё.
}
\answer{%
    $D = \frac 1F = \frac 1a + \frac 1b = \frac 1{15\,\text{см}} + \frac 1{30\,\text{см}} \approx 10\,\text{дптр}$
}
\solutionspace{80pt}

\tasknumber{2}%
\task{%
    Найти увеличение изображения, если изображение предмета, находящегося
    на расстоянии $20\,\text{см}$ от линзы, получается на расстоянии $30\,\text{см}$ от неё.
}
\answer{%
    $\Gamma = \frac ba = \frac {30\,\text{см}}{20\,\text{см}} \approx 1{,}50$
}
\solutionspace{80pt}

\tasknumber{3}%
\task{%
    Расстояние от предмета до линзы $10\,\text{см}$, а от линзы до мнимого изображения $30\,\text{см}$.
    Чему равно фокусное расстояние линзы?
}
\answer{%
    $\pm \frac 1F = \frac 1a - \frac 1b \implies F = \frac{a b}{\abs{b - a}} \approx 15\,\text{см}$
}
\solutionspace{80pt}

\tasknumber{4}%
\task{%
    Две тонкие собирающие линзы с фокусными расстояниями $25\,\text{см}$ и $20\,\text{см}$ сложены вместе.
    Чему равно фокусное расстояние такой оптической системы?
}
\answer{%
    $\frac 1{f_1} = \frac 1a + \frac 1b; \frac 1{f_2} = - \frac 1b + \frac 1c \implies \frac 1{f_1} + \frac 1{f_2} = \frac 1a + \frac 1c \implies f' = \frac 1{\frac 1{f_1} + \frac 1{f_2}} = \frac{f_1 f_2}{f_1 + f_2} \approx 11{,}1\,\text{см}$
}
\solutionspace{80pt}

\tasknumber{5}%
\task{%
    Линейные размеры прямого изображения предмета, полученного в собирающей линзе,
    в два раза больше линейных размеров предмета.
    Зная, что предмет находится на $30\,\text{см}$ ближе к линзе,
    чем его изображение, найти оптическую силу линзы.
}
\answer{%
    \begin{align*}
    &\text{Если изображение действительное:} \\
    D &= \frac 1F = \frac 1a + \frac 1b, \qquad \Gamma = \frac ba, \qquad b - a = \ell \implies b = \Gamma a \implies \Gamma a - a = \ell \implies  \\
    a &= \frac {\ell}{\Gamma - 1} \implies b = \frac {{\ell} \Gamma}{\Gamma - 1} \implies  \\
    D &= \frac {\Gamma - 1}\ell + \frac {\Gamma - 1}{\ell \Gamma} = \frac 1\ell \cdot \cbr{\Gamma - 1 + \frac {\Gamma - 1}{\Gamma} } =\frac 1\ell \cdot \cbr{\Gamma - \frac 1\Gamma} \approx 5\,\text{дптр}.
    \\
    &\text{Если изображение мнимое:} \\
    D &= \frac 1F = \frac 1a - \frac 1b, \qquad \Gamma = \frac ba, \qquad b - a = \ell \implies b = \Gamma a \implies \Gamma a - a = \ell \implies  \\
    a &= \frac {\ell}{\Gamma - 1} \implies b = \frac {{\ell} \Gamma}{\Gamma - 1} \implies  \\
    D &= \frac {\Gamma - 1}\ell - \frac {\Gamma - 1}{\ell \Gamma} = \frac 1\ell \cdot \cbr{\Gamma - 1 - \frac {\Gamma - 1}{\Gamma} } =\frac 1\ell \cdot \cbr{\Gamma + \frac 1\Gamma - 2} \approx 1{,}7\,\text{дптр}.
    \\
    &\text{В ответе надо указать оба значения.}
    \end{align*}
}
\solutionspace{120pt}

\tasknumber{6}%
\task{%
    Оптическая сила объектива фотоаппарата равна $3\,\text{дптр}$.
    При фотографировании чертежа с расстояния $1{,}2\,\text{м}$ площадь изображения
    чертежа на фотопластинке оказалась равной $16\,\text{см}^{2}$.
    Какова площадь самого чертежа? Ответ выразите в квадратных сантиметрах.
}
\answer{%
    \begin{align*}
    &\frac 1a + \frac 1b = \frac 1F = D \implies b = \frac{aF}{a - F} \\
    &\frac {S'}S = \Gamma^2 = \sqr{\frac ba} = \sqr{\frac F{a - F}} \implies \\
    &\implies S = S' \cdot \sqr{\frac{a - F}F} = S' \cdot \sqr{\frac aF - 1} = S' \cdot \sqr{aD - 1} \approx 110\,\text{см}^{2}.
    \end{align*}
}


\variantsplitter


\addpersonalvariant{Ярослав Лавровский}

\tasknumber{7}%
\task{%
    В каком месте на главной оптической оси двояковыгнутой линзы
    нужно поместить точечный источник света,
    чтобы его изображение оказалось в главном фокусе линзы?
}
\answer{%
    $\text{на половине фокусного расстояния}$
}
\solutionspace{120pt}

\tasknumber{8}%
\task{%
    Предмет в виде отрезка длиной $\ell$ расположен вдоль оптической оси
    собирающей линзы с фокусным расстоянием $F$.
    Середина отрезка расположена
    на расстоянии $a$ от линзы, которая даёт действительное изображение
    всех точек предмета.
    Определить продольное увеличение предмета.
}
\answer{%
    \begin{align*}
    \frac 1{a + \frac \ell 2} &+ \frac 1b = \frac 1F \implies b = \frac{F\cbr{a + \frac \ell 2}}{a + \frac \ell 2 - F} \\
    \frac 1{a - \frac \ell 2} &+ \frac 1c = \frac 1F \implies c = \frac{F\cbr{a - \frac \ell 2}}{a - \frac \ell 2 - F} \\
    \abs{b - c} &= \abs{\frac{F\cbr{a + \frac \ell 2}}{a + \frac \ell 2 - F} - \frac{F\cbr{a - \frac \ell 2}}{a - \frac \ell 2 - F}}= F\abs{\frac{\cbr{a + \frac \ell 2}\cbr{a - \frac \ell 2 - F} - \cbr{a - \frac \ell 2}\cbr{a + \frac \ell 2 - F}}{ \cbr{a + \frac \ell 2 - F} \cbr{a - \frac \ell 2 - F} }} =  \\
    &= F\abs{\frac{a^2 - \frac {a\ell} 2 - Fa + \frac {a\ell} 2 - \frac {\ell^2} 4 - \frac {F\ell}2 - a^2 - \frac {a\ell}2 + aF + \frac {a\ell}2 + \frac {\ell^2} 4 - \frac {F\ell} 2}{\cbr{a + \frac \ell 2 - F} \cbr{a - \frac \ell 2 - F} }} = \\
    &= F\frac{F\ell}{\sqr{a-F} - \frac {\ell^2}4} = \frac{F^2\ell}{\sqr{a-F} - \frac {\ell^2}4}\implies \Gamma = \frac{\abs{b - c}}\ell = \frac{F^2}{\sqr{a-F} - \frac {\ell^2}4}.
    \end{align*}
}
\solutionspace{120pt}

\tasknumber{9}%
\task{%
    На экране с помощью тонкой линзы получено изображение предмета
    с увеличением $4$.
    Предмет передвинули на $2\,\text{см}$.
    Для того, чтобы получить резкое изображение, пришлось передвинуть экран.
    При этом увеличение оказалось равным $8$.
    На какое расстояние
    пришлось передвинуть экран?
}
\answer{%
    \begin{align*}
    &\frac 1a + \frac 1b = \frac 1F, \Gamma_1 = \frac ba = \frac{F}{a-F} \implies \Gamma_1(a-F) = F \implies a = F \cdot \frac{1 + \Gamma_1}{\Gamma_1} \\
    &\frac 1{a + x} + \frac 1{b + y} = \frac 1F, \Gamma_2 = \frac {b+y}{a+x} = \frac{F}{a+x-F} \implies a + x = F \cdot \frac{1 + \Gamma_2}{\Gamma_2} \\
    &1 + \frac xa = \frac{ \frac{1 + \Gamma_2}{\Gamma_2} }{ \frac{1 + \Gamma_1}{\Gamma_1} } = \frac{\Gamma_1(1 + \Gamma_2)}{\Gamma_2(1 + \Gamma_1)} \\
    &a = \frac x{ \frac{\Gamma_1(1 + \Gamma_2)}{\Gamma_2(1 + \Gamma_1)} - 1} = x \cdot \frac{\Gamma_2(1 + \Gamma_1)}{\Gamma_1 - \Gamma_2} \\
    &y = (a + x)\Gamma_2 - b = (a + x)\Gamma_2 - a\Gamma_1 = a(\Gamma_2 - \Gamma_1) + x\Gamma_2 = -x\Gamma_2(1 + \Gamma_1) + x\Gamma_2 = -x\Gamma_2\Gamma_1 = 64\,\text{см}, \\
    &\text{знаки разные, т.е.
    экран надо было подвинуть в другую сторону чем предмет: $x < 0, y > 0$.}
    \end{align*}
}
\solutionspace{120pt}

\tasknumber{10}%
\task{%
    Тонкая собирающая линза дает изображение предмета на экране при двух положениях линзы между предметом и экраном.
    Высота изображения при первом положении $30\,\text{см}$, во втором — $9\,\text{см}$.
    Расстояние между предметом и экранов постоянно.
    Чему равна высота предмета?
}
\answer{%
    \begin{align*}
    &\frac 1a + \frac 1b = \frac 1F, \frac 1c + \frac 1d = \frac 1F, a + b = c + d \implies \frac{a + b}{ab} = \frac 1F = \frac{c+d}{cd} \implies ab = cd, \\
    &\implies ab = c(a + b - c) \implies c^2 - ac - bc + ab = 0 \implies c = a \text{ или } c = b \implies c = b \implies d = a.
    \\
    &\Gamma_1 = \frac {H_1}H = \frac ba, \Gamma_2 = \frac {H_2}H = \frac dc = \frac ab \implies \frac {H_1}H \cdot \frac {H_2}H = \frac ba \cdot \frac ab = 1, \\
    &H = \sqrt{H_1 H_2} \approx 16{,}4\,\text{см}.
    \end{align*}
}
\solutionspace{120pt}

\tasknumber{11}%
\task{%
    Какие предметы можно рассмотреть на фотографии, сделанной со спутника,
    если разрешающая способность плёнки $0{,}010\,\text{мм}$? Каким должно быть
    время экспозиции $\tau$ чтобы полностью использовать возможности плёнки?
    Фокусное расстояние объектива используемого фотоаппарата $20\,\text{см}$,
    высота орбиты спутника $120\,\text{км}$.
}
\answer{%
    \begin{align*}
    &H \ll R \implies v = v_{\text{I}} = \sqrt{G R} \approx 7{,}9\,\frac{\text{км}}{\text{с}}.
    \\
    &F \ll H \implies b = F, a = H, \\
    &\Gamma = \frac \delta\ell = \frac ba \implies \ell = \frac{\delta a}b = \frac{\delta H}F \approx \frac{0{,}010\,\text{мм} \cdot 120\,\text{км}}{20\,\text{см}} \approx 6\,\text{м}, \\
    &\implies \tau = \frac \ell v = \frac{\delta H}{F v} = \frac{0{,}010\,\text{мм} \cdot 120\,\text{км}}{20\,\text{см} \cdot 7{,}9\,\frac{\text{км}}{\text{с}}} \approx 0{,}8\,\text{мс}.
    \end{align*}
}


\variantsplitter


\addpersonalvariant{Ярослав Лавровский}

\tasknumber{12}%
\task{%
    При аэрофотосъемках используется фотоаппарат, объектив которого
    имеет фокусиое расстояние $15\,\text{см}$.
    Разрешающая способность плёнки $0{,}02\,\text{мм}$.
    На какой высоте должен лететь самолет, чтобы на фотографии можно
    было различить следы размером $30\,\text{см}$?
    При какой скорости самолета изображение не будет размытым,
    если время экспозиции $2\,\text{мс}$?
}
\answer{%
    \begin{align*}
    &F \ll H \implies b = F, a = H, \\
    &\Gamma = \frac \delta\ell = \frac ba = \frac FH \implies H = \frac{\ell F}\delta = \frac{30\,\text{см} \cdot 15\,\text{см}}{0{,}02\,\text{мм}} \approx 2\,\text{км}, \\
    &\implies v = \frac l\tau = \frac{30\,\text{см}}{2\,\text{мс}} \approx 540\,\frac{\text{км}}{\text{ч}}.
    \end{align*}
}
\solutionspace{120pt}

\tasknumber{13}%
\task{%
    Две одинаковые собиращие линзы установлены так, что их главные оптические оси совпадают,
    а главный фокус первой находится там же, где главный фокус второй.
    Расстояние от первой линзы до предмета равно $16\,\text{см}$.
    Чему равно расстояние от изображения объекта во второй линзе до второй линзы?
    Определите также увеличение.
    Фокусное расстояние каждой линзы $30\,\text{см}$.
}
\answer{%
    \begin{align*}
    \frac 1a + \frac 1b &= \frac 1F \implies b = \frac{aF}{a - F} \implies 2F - b = \frac{2aF - 2F^2 - aF}{a - F} = \frac{F(a - 2F)}{a - F}.
    \\
    \frac 1{2F - b} + \frac 1c &= \frac 1F \implies c = \frac{F(2F-b)}{(2F - b) - F} = \frac{F \cdot \frac{F(a - 2F)}{a - F}}{\frac{F(a - 2F)}{a - F} - F}  = F \cdot \frac{ \frac{F(a - 2F)}{a - F} }{ \frac{F(a - 2F)}{a - F} - 1} = \\
     &= F \cdot \frac{a - 2F}{a - 2F - a + F} = 2F - a = 44\,\text{см}.
     \\
    \ell &= a + 2F + c = 4F = 120\,\text{см}.
    \\
    &\Gamma = \Gamma_1 \cdot \Gamma_2 = \frac ba \cdot \frac c{2F-b} = \frac F{a - F} \cdot \frac{2F - a}{\frac{F(a - 2F)}{a - F}} = -1.
    \end{align*}
}
\solutionspace{120pt}

\tasknumber{14}%
\task{%
    Собирающая линза с фокусным расстоянием $F_1 > 0$ и рассеивающая линза с фокусным расстоянием $F_2 < 0$
    установлены коаксиально на расстоянии $\ell$.
    Пучок параллельных лучей падает на рассеивающую линзу.
    Сделайте схематичное построение и определите, в какой точке система из этих линз соберёт пучок.
}
\answer{%
    \begin{align*}
    &\text{Если пучок падает на собирающую линзу:} \\
    \frac 1{\infty} + \frac 1b &= \frac 1{F_1} \implies b = F_1 \implies \ell - b = \ell - F_1 \\
    \frac 1{\ell - b} + \frac 1c &= \frac 1{F_2} \implies c = \frac{F_2(\ell - b)}{\ell - b - F_2} = \frac{F_2(\ell - F_1)}{\ell - F_1 - F_2}.
    \\
    &\text{Если же пучок падает на рассеивающую линзу:} \\
    \frac 1{\infty} + \frac 1b &= \frac 1{F_2} \implies b = F_2 \implies \ell - b = \ell - F_2 \\
    \frac 1{\ell - b} + \frac 1c &= \frac 1{F_1} \implies c = \frac{F_1(\ell - b)}{\ell - b - F_1} = \frac{F_1(\ell - F_2)}{\ell - F_2 - F_1}.
    \end{align*}
}
\solutionspace{120pt}

\tasknumber{15}%
\task{%
    Две собирающих линзы с фокусными расстояниями $20\,\text{см}$ и $35\,\text{см}$ расположены так,
    что их оптические оси совмещены.
    На первую линзу падает пучок параллельных лучей.
    Пройдя через вторую линзу, он остался параллельным.
    Найдите расстояние между линзами и сделайте рисунок.
}
\answer{%
    \begin{align*}
    \frac 1\infty + \frac 1b &= \frac 1{F_1} \implies b = F_1, \\
    \frac 1{\ell - b} + \frac 1{\infty} &= \frac 1{F_2} \implies \ell - b = F_2 \implies \ell = b + F_2 = F_1 + F_2 = 55\,\text{см}.
    \end{align*}
}

\variantsplitter

\addpersonalvariant{Анастасия Ламанова}

\tasknumber{1}%
\task{%
    Найти оптическую силу собирающей линзы, если действительное изображение предмета,
    помещённого в $15\,\text{см}$ от линзы, получается на расстоянии $20\,\text{см}$ от неё.
}
\answer{%
    $D = \frac 1F = \frac 1a + \frac 1b = \frac 1{15\,\text{см}} + \frac 1{20\,\text{см}} \approx 11{,}67\,\text{дптр}$
}
\solutionspace{80pt}

\tasknumber{2}%
\task{%
    Найти увеличение изображения, если изображение предмета, находящегося
    на расстоянии $15\,\text{см}$ от линзы, получается на расстоянии $12\,\text{см}$ от неё.
}
\answer{%
    $\Gamma = \frac ba = \frac {12\,\text{см}}{15\,\text{см}} \approx 0{,}8$
}
\solutionspace{80pt}

\tasknumber{3}%
\task{%
    Расстояние от предмета до линзы $8\,\text{см}$, а от линзы до мнимого изображения $20\,\text{см}$.
    Чему равно фокусное расстояние линзы?
}
\answer{%
    $\pm \frac 1F = \frac 1a - \frac 1b \implies F = \frac{a b}{\abs{b - a}} \approx 13{,}3\,\text{см}$
}
\solutionspace{80pt}

\tasknumber{4}%
\task{%
    Две тонкие собирающие линзы с фокусными расстояниями $18\,\text{см}$ и $20\,\text{см}$ сложены вместе.
    Чему равно фокусное расстояние такой оптической системы?
}
\answer{%
    $\frac 1{f_1} = \frac 1a + \frac 1b; \frac 1{f_2} = - \frac 1b + \frac 1c \implies \frac 1{f_1} + \frac 1{f_2} = \frac 1a + \frac 1c \implies f' = \frac 1{\frac 1{f_1} + \frac 1{f_2}} = \frac{f_1 f_2}{f_1 + f_2} \approx 9{,}5\,\text{см}$
}
\solutionspace{80pt}

\tasknumber{5}%
\task{%
    Линейные размеры прямого изображения предмета, полученного в собирающей линзе,
    в три раза больше линейных размеров предмета.
    Зная, что предмет находится на $35\,\text{см}$ ближе к линзе,
    чем его изображение, найти оптическую силу линзы.
}
\answer{%
    \begin{align*}
    &\text{Если изображение действительное:} \\
    D &= \frac 1F = \frac 1a + \frac 1b, \qquad \Gamma = \frac ba, \qquad b - a = \ell \implies b = \Gamma a \implies \Gamma a - a = \ell \implies  \\
    a &= \frac {\ell}{\Gamma - 1} \implies b = \frac {{\ell} \Gamma}{\Gamma - 1} \implies  \\
    D &= \frac {\Gamma - 1}\ell + \frac {\Gamma - 1}{\ell \Gamma} = \frac 1\ell \cdot \cbr{\Gamma - 1 + \frac {\Gamma - 1}{\Gamma} } =\frac 1\ell \cdot \cbr{\Gamma - \frac 1\Gamma} \approx 7{,}6\,\text{дптр}.
    \\
    &\text{Если изображение мнимое:} \\
    D &= \frac 1F = \frac 1a - \frac 1b, \qquad \Gamma = \frac ba, \qquad b - a = \ell \implies b = \Gamma a \implies \Gamma a - a = \ell \implies  \\
    a &= \frac {\ell}{\Gamma - 1} \implies b = \frac {{\ell} \Gamma}{\Gamma - 1} \implies  \\
    D &= \frac {\Gamma - 1}\ell - \frac {\Gamma - 1}{\ell \Gamma} = \frac 1\ell \cdot \cbr{\Gamma - 1 - \frac {\Gamma - 1}{\Gamma} } =\frac 1\ell \cdot \cbr{\Gamma + \frac 1\Gamma - 2} \approx 3{,}8\,\text{дптр}.
    \\
    &\text{В ответе надо указать оба значения.}
    \end{align*}
}
\solutionspace{120pt}

\tasknumber{6}%
\task{%
    Оптическая сила объектива фотоаппарата равна $3\,\text{дптр}$.
    При фотографировании чертежа с расстояния $0{,}9\,\text{м}$ площадь изображения
    чертежа на фотопластинке оказалась равной $9\,\text{см}^{2}$.
    Какова площадь самого чертежа? Ответ выразите в квадратных сантиметрах.
}
\answer{%
    \begin{align*}
    &\frac 1a + \frac 1b = \frac 1F = D \implies b = \frac{aF}{a - F} \\
    &\frac {S'}S = \Gamma^2 = \sqr{\frac ba} = \sqr{\frac F{a - F}} \implies \\
    &\implies S = S' \cdot \sqr{\frac{a - F}F} = S' \cdot \sqr{\frac aF - 1} = S' \cdot \sqr{aD - 1} \approx 30\,\text{см}^{2}.
    \end{align*}
}


\variantsplitter


\addpersonalvariant{Анастасия Ламанова}

\tasknumber{7}%
\task{%
    В каком месте на главной оптической оси двояковыпуклой линзы
    нужно поместить точечный источник света,
    чтобы его изображение оказалось в главном фокусе линзы?
}
\answer{%
    $\text{для мнимого - на половине фокусного, для действительного - на бесконечности}$
}
\solutionspace{120pt}

\tasknumber{8}%
\task{%
    Предмет в виде отрезка длиной $\ell$ расположен вдоль оптической оси
    собирающей линзы с фокусным расстоянием $F$.
    Середина отрезка расположена
    на расстоянии $a$ от линзы, которая даёт действительное изображение
    всех точек предмета.
    Определить продольное увеличение предмета.
}
\answer{%
    \begin{align*}
    \frac 1{a + \frac \ell 2} &+ \frac 1b = \frac 1F \implies b = \frac{F\cbr{a + \frac \ell 2}}{a + \frac \ell 2 - F} \\
    \frac 1{a - \frac \ell 2} &+ \frac 1c = \frac 1F \implies c = \frac{F\cbr{a - \frac \ell 2}}{a - \frac \ell 2 - F} \\
    \abs{b - c} &= \abs{\frac{F\cbr{a + \frac \ell 2}}{a + \frac \ell 2 - F} - \frac{F\cbr{a - \frac \ell 2}}{a - \frac \ell 2 - F}}= F\abs{\frac{\cbr{a + \frac \ell 2}\cbr{a - \frac \ell 2 - F} - \cbr{a - \frac \ell 2}\cbr{a + \frac \ell 2 - F}}{ \cbr{a + \frac \ell 2 - F} \cbr{a - \frac \ell 2 - F} }} =  \\
    &= F\abs{\frac{a^2 - \frac {a\ell} 2 - Fa + \frac {a\ell} 2 - \frac {\ell^2} 4 - \frac {F\ell}2 - a^2 - \frac {a\ell}2 + aF + \frac {a\ell}2 + \frac {\ell^2} 4 - \frac {F\ell} 2}{\cbr{a + \frac \ell 2 - F} \cbr{a - \frac \ell 2 - F} }} = \\
    &= F\frac{F\ell}{\sqr{a-F} - \frac {\ell^2}4} = \frac{F^2\ell}{\sqr{a-F} - \frac {\ell^2}4}\implies \Gamma = \frac{\abs{b - c}}\ell = \frac{F^2}{\sqr{a-F} - \frac {\ell^2}4}.
    \end{align*}
}
\solutionspace{120pt}

\tasknumber{9}%
\task{%
    На экране с помощью тонкой линзы получено изображение предмета
    с увеличением $4$.
    Предмет передвинули на $2\,\text{см}$.
    Для того, чтобы получить резкое изображение, пришлось передвинуть экран.
    При этом увеличение оказалось равным $6$.
    На какое расстояние
    пришлось передвинуть экран?
}
\answer{%
    \begin{align*}
    &\frac 1a + \frac 1b = \frac 1F, \Gamma_1 = \frac ba = \frac{F}{a-F} \implies \Gamma_1(a-F) = F \implies a = F \cdot \frac{1 + \Gamma_1}{\Gamma_1} \\
    &\frac 1{a + x} + \frac 1{b + y} = \frac 1F, \Gamma_2 = \frac {b+y}{a+x} = \frac{F}{a+x-F} \implies a + x = F \cdot \frac{1 + \Gamma_2}{\Gamma_2} \\
    &1 + \frac xa = \frac{ \frac{1 + \Gamma_2}{\Gamma_2} }{ \frac{1 + \Gamma_1}{\Gamma_1} } = \frac{\Gamma_1(1 + \Gamma_2)}{\Gamma_2(1 + \Gamma_1)} \\
    &a = \frac x{ \frac{\Gamma_1(1 + \Gamma_2)}{\Gamma_2(1 + \Gamma_1)} - 1} = x \cdot \frac{\Gamma_2(1 + \Gamma_1)}{\Gamma_1 - \Gamma_2} \\
    &y = (a + x)\Gamma_2 - b = (a + x)\Gamma_2 - a\Gamma_1 = a(\Gamma_2 - \Gamma_1) + x\Gamma_2 = -x\Gamma_2(1 + \Gamma_1) + x\Gamma_2 = -x\Gamma_2\Gamma_1 = 48\,\text{см}, \\
    &\text{знаки разные, т.е.
    экран надо было подвинуть в другую сторону чем предмет: $x < 0, y > 0$.}
    \end{align*}
}
\solutionspace{120pt}

\tasknumber{10}%
\task{%
    Тонкая собирающая линза дает изображение предмета на экране при двух положениях линзы между предметом и экраном.
    Высота изображения при первом положении $30\,\text{см}$, во втором — $9\,\text{см}$.
    Расстояние между предметом и экранов постоянно.
    Чему равна высота предмета?
}
\answer{%
    \begin{align*}
    &\frac 1a + \frac 1b = \frac 1F, \frac 1c + \frac 1d = \frac 1F, a + b = c + d \implies \frac{a + b}{ab} = \frac 1F = \frac{c+d}{cd} \implies ab = cd, \\
    &\implies ab = c(a + b - c) \implies c^2 - ac - bc + ab = 0 \implies c = a \text{ или } c = b \implies c = b \implies d = a.
    \\
    &\Gamma_1 = \frac {H_1}H = \frac ba, \Gamma_2 = \frac {H_2}H = \frac dc = \frac ab \implies \frac {H_1}H \cdot \frac {H_2}H = \frac ba \cdot \frac ab = 1, \\
    &H = \sqrt{H_1 H_2} \approx 16{,}4\,\text{см}.
    \end{align*}
}
\solutionspace{120pt}

\tasknumber{11}%
\task{%
    Какие предметы можно рассмотреть на фотографии, сделанной со спутника,
    если разрешающая способность плёнки $0{,}02\,\text{мм}$? Каким должно быть
    время экспозиции $\tau$ чтобы полностью использовать возможности плёнки?
    Фокусное расстояние объектива используемого фотоаппарата $15\,\text{см}$,
    высота орбиты спутника $80\,\text{км}$.
}
\answer{%
    \begin{align*}
    &H \ll R \implies v = v_{\text{I}} = \sqrt{G R} \approx 7{,}9\,\frac{\text{км}}{\text{с}}.
    \\
    &F \ll H \implies b = F, a = H, \\
    &\Gamma = \frac \delta\ell = \frac ba \implies \ell = \frac{\delta a}b = \frac{\delta H}F \approx \frac{0{,}02\,\text{мм} \cdot 80\,\text{км}}{15\,\text{см}} \approx 11\,\text{м}, \\
    &\implies \tau = \frac \ell v = \frac{\delta H}{F v} = \frac{0{,}02\,\text{мм} \cdot 80\,\text{км}}{15\,\text{см} \cdot 7{,}9\,\frac{\text{км}}{\text{с}}} \approx 1{,}4\,\text{мс}.
    \end{align*}
}


\variantsplitter


\addpersonalvariant{Анастасия Ламанова}

\tasknumber{12}%
\task{%
    При аэрофотосъемках используется фотоаппарат, объектив которого
    имеет фокусиое расстояние $20\,\text{см}$.
    Разрешающая способность плёнки $0{,}015\,\text{мм}$.
    На какой высоте должен лететь самолет, чтобы на фотографии можно
    было различить следы размером $20\,\text{см}$?
    При какой скорости самолета изображение не будет размытым,
    если время экспозиции $2\,\text{мс}$?
}
\answer{%
    \begin{align*}
    &F \ll H \implies b = F, a = H, \\
    &\Gamma = \frac \delta\ell = \frac ba = \frac FH \implies H = \frac{\ell F}\delta = \frac{20\,\text{см} \cdot 20\,\text{см}}{0{,}015\,\text{мм}} \approx 3\,\text{км}, \\
    &\implies v = \frac l\tau = \frac{20\,\text{см}}{2\,\text{мс}} \approx 360\,\frac{\text{км}}{\text{ч}}.
    \end{align*}
}
\solutionspace{120pt}

\tasknumber{13}%
\task{%
    Две одинаковые собиращие линзы установлены так, что их главные оптические оси совпадают,
    а главный фокус первой находится там же, где главный фокус второй.
    Расстояние от первой линзы до предмета равно $21\,\text{см}$.
    Чему равно расстояние от изображения объекта во второй линзе до самого объекта?
    Определите также увеличение.
    Фокусное расстояние каждой линзы $25\,\text{см}$.
}
\answer{%
    \begin{align*}
    \frac 1a + \frac 1b &= \frac 1F \implies b = \frac{aF}{a - F} \implies 2F - b = \frac{2aF - 2F^2 - aF}{a - F} = \frac{F(a - 2F)}{a - F}.
    \\
    \frac 1{2F - b} + \frac 1c &= \frac 1F \implies c = \frac{F(2F-b)}{(2F - b) - F} = \frac{F \cdot \frac{F(a - 2F)}{a - F}}{\frac{F(a - 2F)}{a - F} - F}  = F \cdot \frac{ \frac{F(a - 2F)}{a - F} }{ \frac{F(a - 2F)}{a - F} - 1} = \\
     &= F \cdot \frac{a - 2F}{a - 2F - a + F} = 2F - a = 29\,\text{см}.
     \\
    \ell &= a + 2F + c = 4F = 100\,\text{см}.
    \\
    &\Gamma = \Gamma_1 \cdot \Gamma_2 = \frac ba \cdot \frac c{2F-b} = \frac F{a - F} \cdot \frac{2F - a}{\frac{F(a - 2F)}{a - F}} = -1.
    \end{align*}
}
\solutionspace{120pt}

\tasknumber{14}%
\task{%
    Собирающая линза с фокусным расстоянием $F_1 > 0$ и рассеивающая линза с фокусным расстоянием $F_2 < 0$
    установлены коаксиально на расстоянии $\ell$.
    Пучок параллельных лучей падает на собирающую линзу.
    Сделайте схематичное построение и определите, в какой точке система из этих линз соберёт пучок.
}
\answer{%
    \begin{align*}
    &\text{Если пучок падает на собирающую линзу:} \\
    \frac 1{\infty} + \frac 1b &= \frac 1{F_1} \implies b = F_1 \implies \ell - b = \ell - F_1 \\
    \frac 1{\ell - b} + \frac 1c &= \frac 1{F_2} \implies c = \frac{F_2(\ell - b)}{\ell - b - F_2} = \frac{F_2(\ell - F_1)}{\ell - F_1 - F_2}.
    \\
    &\text{Если же пучок падает на рассеивающую линзу:} \\
    \frac 1{\infty} + \frac 1b &= \frac 1{F_2} \implies b = F_2 \implies \ell - b = \ell - F_2 \\
    \frac 1{\ell - b} + \frac 1c &= \frac 1{F_1} \implies c = \frac{F_1(\ell - b)}{\ell - b - F_1} = \frac{F_1(\ell - F_2)}{\ell - F_2 - F_1}.
    \end{align*}
}
\solutionspace{120pt}

\tasknumber{15}%
\task{%
    Две собирающих линзы с фокусными расстояниями $30\,\text{см}$ и $45\,\text{см}$ расположены так,
    что их оптические оси совмещены.
    На первую линзу падает пучок параллельных лучей.
    Пройдя через вторую линзу, он остался параллельным.
    Найдите расстояние между линзами и сделайте рисунок.
}
\answer{%
    \begin{align*}
    \frac 1\infty + \frac 1b &= \frac 1{F_1} \implies b = F_1, \\
    \frac 1{\ell - b} + \frac 1{\infty} &= \frac 1{F_2} \implies \ell - b = F_2 \implies \ell = b + F_2 = F_1 + F_2 = 75\,\text{см}.
    \end{align*}
}

\variantsplitter

\addpersonalvariant{Виктория Легонькова}

\tasknumber{1}%
\task{%
    Найти оптическую силу собирающей линзы, если действительное изображение предмета,
    помещённого в $15\,\text{см}$ от линзы, получается на расстоянии $20\,\text{см}$ от неё.
}
\answer{%
    $D = \frac 1F = \frac 1a + \frac 1b = \frac 1{15\,\text{см}} + \frac 1{20\,\text{см}} \approx 11{,}67\,\text{дптр}$
}
\solutionspace{80pt}

\tasknumber{2}%
\task{%
    Найти увеличение изображения, если изображение предмета, находящегося
    на расстоянии $15\,\text{см}$ от линзы, получается на расстоянии $30\,\text{см}$ от неё.
}
\answer{%
    $\Gamma = \frac ba = \frac {30\,\text{см}}{15\,\text{см}} \approx 2$
}
\solutionspace{80pt}

\tasknumber{3}%
\task{%
    Расстояние от предмета до линзы $8\,\text{см}$, а от линзы до мнимого изображения $30\,\text{см}$.
    Чему равно фокусное расстояние линзы?
}
\answer{%
    $\pm \frac 1F = \frac 1a - \frac 1b \implies F = \frac{a b}{\abs{b - a}} \approx 10{,}9\,\text{см}$
}
\solutionspace{80pt}

\tasknumber{4}%
\task{%
    Две тонкие собирающие линзы с фокусными расстояниями $18\,\text{см}$ и $30\,\text{см}$ сложены вместе.
    Чему равно фокусное расстояние такой оптической системы?
}
\answer{%
    $\frac 1{f_1} = \frac 1a + \frac 1b; \frac 1{f_2} = - \frac 1b + \frac 1c \implies \frac 1{f_1} + \frac 1{f_2} = \frac 1a + \frac 1c \implies f' = \frac 1{\frac 1{f_1} + \frac 1{f_2}} = \frac{f_1 f_2}{f_1 + f_2} \approx 11{,}2\,\text{см}$
}
\solutionspace{80pt}

\tasknumber{5}%
\task{%
    Линейные размеры прямого изображения предмета, полученного в собирающей линзе,
    в три раза больше линейных размеров предмета.
    Зная, что предмет находится на $30\,\text{см}$ ближе к линзе,
    чем его изображение, найти оптическую силу линзы.
}
\answer{%
    \begin{align*}
    &\text{Если изображение действительное:} \\
    D &= \frac 1F = \frac 1a + \frac 1b, \qquad \Gamma = \frac ba, \qquad b - a = \ell \implies b = \Gamma a \implies \Gamma a - a = \ell \implies  \\
    a &= \frac {\ell}{\Gamma - 1} \implies b = \frac {{\ell} \Gamma}{\Gamma - 1} \implies  \\
    D &= \frac {\Gamma - 1}\ell + \frac {\Gamma - 1}{\ell \Gamma} = \frac 1\ell \cdot \cbr{\Gamma - 1 + \frac {\Gamma - 1}{\Gamma} } =\frac 1\ell \cdot \cbr{\Gamma - \frac 1\Gamma} \approx 8{,}9\,\text{дптр}.
    \\
    &\text{Если изображение мнимое:} \\
    D &= \frac 1F = \frac 1a - \frac 1b, \qquad \Gamma = \frac ba, \qquad b - a = \ell \implies b = \Gamma a \implies \Gamma a - a = \ell \implies  \\
    a &= \frac {\ell}{\Gamma - 1} \implies b = \frac {{\ell} \Gamma}{\Gamma - 1} \implies  \\
    D &= \frac {\Gamma - 1}\ell - \frac {\Gamma - 1}{\ell \Gamma} = \frac 1\ell \cdot \cbr{\Gamma - 1 - \frac {\Gamma - 1}{\Gamma} } =\frac 1\ell \cdot \cbr{\Gamma + \frac 1\Gamma - 2} \approx 4{,}4\,\text{дптр}.
    \\
    &\text{В ответе надо указать оба значения.}
    \end{align*}
}
\solutionspace{120pt}

\tasknumber{6}%
\task{%
    Оптическая сила объектива фотоаппарата равна $6\,\text{дптр}$.
    При фотографировании чертежа с расстояния $0{,}8\,\text{м}$ площадь изображения
    чертежа на фотопластинке оказалась равной $9\,\text{см}^{2}$.
    Какова площадь самого чертежа? Ответ выразите в квадратных сантиметрах.
}
\answer{%
    \begin{align*}
    &\frac 1a + \frac 1b = \frac 1F = D \implies b = \frac{aF}{a - F} \\
    &\frac {S'}S = \Gamma^2 = \sqr{\frac ba} = \sqr{\frac F{a - F}} \implies \\
    &\implies S = S' \cdot \sqr{\frac{a - F}F} = S' \cdot \sqr{\frac aF - 1} = S' \cdot \sqr{aD - 1} \approx 130\,\text{см}^{2}.
    \end{align*}
}


\variantsplitter


\addpersonalvariant{Виктория Легонькова}

\tasknumber{7}%
\task{%
    В каком месте на главной оптической оси двояковыпуклой линзы
    нужно поместить точечный источник света,
    чтобы его изображение оказалось в главном фокусе линзы?
}
\answer{%
    $\text{для мнимого - на половине фокусного, для действительного - на бесконечности}$
}
\solutionspace{120pt}

\tasknumber{8}%
\task{%
    Предмет в виде отрезка длиной $\ell$ расположен вдоль оптической оси
    собирающей линзы с фокусным расстоянием $F$.
    Середина отрезка расположена
    на расстоянии $a$ от линзы, которая даёт действительное изображение
    всех точек предмета.
    Определить продольное увеличение предмета.
}
\answer{%
    \begin{align*}
    \frac 1{a + \frac \ell 2} &+ \frac 1b = \frac 1F \implies b = \frac{F\cbr{a + \frac \ell 2}}{a + \frac \ell 2 - F} \\
    \frac 1{a - \frac \ell 2} &+ \frac 1c = \frac 1F \implies c = \frac{F\cbr{a - \frac \ell 2}}{a - \frac \ell 2 - F} \\
    \abs{b - c} &= \abs{\frac{F\cbr{a + \frac \ell 2}}{a + \frac \ell 2 - F} - \frac{F\cbr{a - \frac \ell 2}}{a - \frac \ell 2 - F}}= F\abs{\frac{\cbr{a + \frac \ell 2}\cbr{a - \frac \ell 2 - F} - \cbr{a - \frac \ell 2}\cbr{a + \frac \ell 2 - F}}{ \cbr{a + \frac \ell 2 - F} \cbr{a - \frac \ell 2 - F} }} =  \\
    &= F\abs{\frac{a^2 - \frac {a\ell} 2 - Fa + \frac {a\ell} 2 - \frac {\ell^2} 4 - \frac {F\ell}2 - a^2 - \frac {a\ell}2 + aF + \frac {a\ell}2 + \frac {\ell^2} 4 - \frac {F\ell} 2}{\cbr{a + \frac \ell 2 - F} \cbr{a - \frac \ell 2 - F} }} = \\
    &= F\frac{F\ell}{\sqr{a-F} - \frac {\ell^2}4} = \frac{F^2\ell}{\sqr{a-F} - \frac {\ell^2}4}\implies \Gamma = \frac{\abs{b - c}}\ell = \frac{F^2}{\sqr{a-F} - \frac {\ell^2}4}.
    \end{align*}
}
\solutionspace{120pt}

\tasknumber{9}%
\task{%
    На экране с помощью тонкой линзы получено изображение предмета
    с увеличением $4$.
    Предмет передвинули на $10\,\text{см}$.
    Для того, чтобы получить резкое изображение, пришлось передвинуть экран.
    При этом увеличение оказалось равным $8$.
    На какое расстояние
    пришлось передвинуть экран?
}
\answer{%
    \begin{align*}
    &\frac 1a + \frac 1b = \frac 1F, \Gamma_1 = \frac ba = \frac{F}{a-F} \implies \Gamma_1(a-F) = F \implies a = F \cdot \frac{1 + \Gamma_1}{\Gamma_1} \\
    &\frac 1{a + x} + \frac 1{b + y} = \frac 1F, \Gamma_2 = \frac {b+y}{a+x} = \frac{F}{a+x-F} \implies a + x = F \cdot \frac{1 + \Gamma_2}{\Gamma_2} \\
    &1 + \frac xa = \frac{ \frac{1 + \Gamma_2}{\Gamma_2} }{ \frac{1 + \Gamma_1}{\Gamma_1} } = \frac{\Gamma_1(1 + \Gamma_2)}{\Gamma_2(1 + \Gamma_1)} \\
    &a = \frac x{ \frac{\Gamma_1(1 + \Gamma_2)}{\Gamma_2(1 + \Gamma_1)} - 1} = x \cdot \frac{\Gamma_2(1 + \Gamma_1)}{\Gamma_1 - \Gamma_2} \\
    &y = (a + x)\Gamma_2 - b = (a + x)\Gamma_2 - a\Gamma_1 = a(\Gamma_2 - \Gamma_1) + x\Gamma_2 = -x\Gamma_2(1 + \Gamma_1) + x\Gamma_2 = -x\Gamma_2\Gamma_1 = 320\,\text{см}, \\
    &\text{знаки разные, т.е.
    экран надо было подвинуть в другую сторону чем предмет: $x < 0, y > 0$.}
    \end{align*}
}
\solutionspace{120pt}

\tasknumber{10}%
\task{%
    Тонкая собирающая линза дает изображение предмета на экране при двух положениях линзы между предметом и экраном.
    Высота изображения при первом положении $25\,\text{см}$, во втором — $9\,\text{см}$.
    Расстояние между предметом и экранов постоянно.
    Чему равна высота предмета?
}
\answer{%
    \begin{align*}
    &\frac 1a + \frac 1b = \frac 1F, \frac 1c + \frac 1d = \frac 1F, a + b = c + d \implies \frac{a + b}{ab} = \frac 1F = \frac{c+d}{cd} \implies ab = cd, \\
    &\implies ab = c(a + b - c) \implies c^2 - ac - bc + ab = 0 \implies c = a \text{ или } c = b \implies c = b \implies d = a.
    \\
    &\Gamma_1 = \frac {H_1}H = \frac ba, \Gamma_2 = \frac {H_2}H = \frac dc = \frac ab \implies \frac {H_1}H \cdot \frac {H_2}H = \frac ba \cdot \frac ab = 1, \\
    &H = \sqrt{H_1 H_2} \approx 15\,\text{см}.
    \end{align*}
}
\solutionspace{120pt}

\tasknumber{11}%
\task{%
    Какие предметы можно рассмотреть на фотографии, сделанной со спутника,
    если разрешающая способность плёнки $0{,}010\,\text{мм}$? Каким должно быть
    время экспозиции $\tau$ чтобы полностью использовать возможности плёнки?
    Фокусное расстояние объектива используемого фотоаппарата $15\,\text{см}$,
    высота орбиты спутника $120\,\text{км}$.
}
\answer{%
    \begin{align*}
    &H \ll R \implies v = v_{\text{I}} = \sqrt{G R} \approx 7{,}9\,\frac{\text{км}}{\text{с}}.
    \\
    &F \ll H \implies b = F, a = H, \\
    &\Gamma = \frac \delta\ell = \frac ba \implies \ell = \frac{\delta a}b = \frac{\delta H}F \approx \frac{0{,}010\,\text{мм} \cdot 120\,\text{км}}{15\,\text{см}} \approx 8\,\text{м}, \\
    &\implies \tau = \frac \ell v = \frac{\delta H}{F v} = \frac{0{,}010\,\text{мм} \cdot 120\,\text{км}}{15\,\text{см} \cdot 7{,}9\,\frac{\text{км}}{\text{с}}} \approx 1{,}0\,\text{мс}.
    \end{align*}
}


\variantsplitter


\addpersonalvariant{Виктория Легонькова}

\tasknumber{12}%
\task{%
    При аэрофотосъемках используется фотоаппарат, объектив которого
    имеет фокусиое расстояние $10\,\text{см}$.
    Разрешающая способность плёнки $0{,}02\,\text{мм}$.
    На какой высоте должен лететь самолет, чтобы на фотографии можно
    было различить следы размером $30\,\text{см}$?
    При какой скорости самолета изображение не будет размытым,
    если время экспозиции $1\,\text{мс}$?
}
\answer{%
    \begin{align*}
    &F \ll H \implies b = F, a = H, \\
    &\Gamma = \frac \delta\ell = \frac ba = \frac FH \implies H = \frac{\ell F}\delta = \frac{30\,\text{см} \cdot 10\,\text{см}}{0{,}02\,\text{мм}} \approx 1{,}5\,\text{км}, \\
    &\implies v = \frac l\tau = \frac{30\,\text{см}}{1\,\text{мс}} \approx 1080\,\frac{\text{км}}{\text{ч}}.
    \end{align*}
}
\solutionspace{120pt}

\tasknumber{13}%
\task{%
    Две одинаковые собиращие линзы установлены так, что их главные оптические оси совпадают,
    а главный фокус первой находится там же, где главный фокус второй.
    Расстояние от первой линзы до предмета равно $32\,\text{см}$.
    Чему равно расстояние от изображения объекта во второй линзе до второй линзы?
    Определите также увеличение.
    Фокусное расстояние каждой линзы $40\,\text{см}$.
}
\answer{%
    \begin{align*}
    \frac 1a + \frac 1b &= \frac 1F \implies b = \frac{aF}{a - F} \implies 2F - b = \frac{2aF - 2F^2 - aF}{a - F} = \frac{F(a - 2F)}{a - F}.
    \\
    \frac 1{2F - b} + \frac 1c &= \frac 1F \implies c = \frac{F(2F-b)}{(2F - b) - F} = \frac{F \cdot \frac{F(a - 2F)}{a - F}}{\frac{F(a - 2F)}{a - F} - F}  = F \cdot \frac{ \frac{F(a - 2F)}{a - F} }{ \frac{F(a - 2F)}{a - F} - 1} = \\
     &= F \cdot \frac{a - 2F}{a - 2F - a + F} = 2F - a = 48\,\text{см}.
     \\
    \ell &= a + 2F + c = 4F = 160\,\text{см}.
    \\
    &\Gamma = \Gamma_1 \cdot \Gamma_2 = \frac ba \cdot \frac c{2F-b} = \frac F{a - F} \cdot \frac{2F - a}{\frac{F(a - 2F)}{a - F}} = -1.
    \end{align*}
}
\solutionspace{120pt}

\tasknumber{14}%
\task{%
    Собирающая линза с фокусным расстоянием $F_1 > 0$ и рассеивающая линза с фокусным расстоянием $F_2 < 0$
    установлены коаксиально на расстоянии $\ell$.
    Пучок параллельных лучей падает на собирающую линзу.
    Сделайте схематичное построение и определите, в какой точке система из этих линз соберёт пучок.
}
\answer{%
    \begin{align*}
    &\text{Если пучок падает на собирающую линзу:} \\
    \frac 1{\infty} + \frac 1b &= \frac 1{F_1} \implies b = F_1 \implies \ell - b = \ell - F_1 \\
    \frac 1{\ell - b} + \frac 1c &= \frac 1{F_2} \implies c = \frac{F_2(\ell - b)}{\ell - b - F_2} = \frac{F_2(\ell - F_1)}{\ell - F_1 - F_2}.
    \\
    &\text{Если же пучок падает на рассеивающую линзу:} \\
    \frac 1{\infty} + \frac 1b &= \frac 1{F_2} \implies b = F_2 \implies \ell - b = \ell - F_2 \\
    \frac 1{\ell - b} + \frac 1c &= \frac 1{F_1} \implies c = \frac{F_1(\ell - b)}{\ell - b - F_1} = \frac{F_1(\ell - F_2)}{\ell - F_2 - F_1}.
    \end{align*}
}
\solutionspace{120pt}

\tasknumber{15}%
\task{%
    Две собирающих линзы с фокусными расстояниями $40\,\text{см}$ и $45\,\text{см}$ расположены так,
    что их оптические оси совмещены.
    На первую линзу падает пучок параллельных лучей.
    Пройдя через вторую линзу, он остался параллельным.
    Найдите расстояние между линзами и сделайте рисунок.
}
\answer{%
    \begin{align*}
    \frac 1\infty + \frac 1b &= \frac 1{F_1} \implies b = F_1, \\
    \frac 1{\ell - b} + \frac 1{\infty} &= \frac 1{F_2} \implies \ell - b = F_2 \implies \ell = b + F_2 = F_1 + F_2 = 85\,\text{см}.
    \end{align*}
}

\variantsplitter

\addpersonalvariant{Семён Мартынов}

\tasknumber{1}%
\task{%
    Найти оптическую силу собирающей линзы, если действительное изображение предмета,
    помещённого в $35\,\text{см}$ от линзы, получается на расстоянии $40\,\text{см}$ от неё.
}
\answer{%
    $D = \frac 1F = \frac 1a + \frac 1b = \frac 1{35\,\text{см}} + \frac 1{40\,\text{см}} \approx 5{,}36\,\text{дптр}$
}
\solutionspace{80pt}

\tasknumber{2}%
\task{%
    Найти увеличение изображения, если изображение предмета, находящегося
    на расстоянии $15\,\text{см}$ от линзы, получается на расстоянии $30\,\text{см}$ от неё.
}
\answer{%
    $\Gamma = \frac ba = \frac {30\,\text{см}}{15\,\text{см}} \approx 2$
}
\solutionspace{80pt}

\tasknumber{3}%
\task{%
    Расстояние от предмета до линзы $8\,\text{см}$, а от линзы до мнимого изображения $30\,\text{см}$.
    Чему равно фокусное расстояние линзы?
}
\answer{%
    $\pm \frac 1F = \frac 1a - \frac 1b \implies F = \frac{a b}{\abs{b - a}} \approx 10{,}9\,\text{см}$
}
\solutionspace{80pt}

\tasknumber{4}%
\task{%
    Две тонкие собирающие линзы с фокусными расстояниями $12\,\text{см}$ и $20\,\text{см}$ сложены вместе.
    Чему равно фокусное расстояние такой оптической системы?
}
\answer{%
    $\frac 1{f_1} = \frac 1a + \frac 1b; \frac 1{f_2} = - \frac 1b + \frac 1c \implies \frac 1{f_1} + \frac 1{f_2} = \frac 1a + \frac 1c \implies f' = \frac 1{\frac 1{f_1} + \frac 1{f_2}} = \frac{f_1 f_2}{f_1 + f_2} \approx 7{,}5\,\text{см}$
}
\solutionspace{80pt}

\tasknumber{5}%
\task{%
    Линейные размеры прямого изображения предмета, полученного в собирающей линзе,
    в четыре раза больше линейных размеров предмета.
    Зная, что предмет находится на $35\,\text{см}$ ближе к линзе,
    чем его изображение, найти оптическую силу линзы.
}
\answer{%
    \begin{align*}
    &\text{Если изображение действительное:} \\
    D &= \frac 1F = \frac 1a + \frac 1b, \qquad \Gamma = \frac ba, \qquad b - a = \ell \implies b = \Gamma a \implies \Gamma a - a = \ell \implies  \\
    a &= \frac {\ell}{\Gamma - 1} \implies b = \frac {{\ell} \Gamma}{\Gamma - 1} \implies  \\
    D &= \frac {\Gamma - 1}\ell + \frac {\Gamma - 1}{\ell \Gamma} = \frac 1\ell \cdot \cbr{\Gamma - 1 + \frac {\Gamma - 1}{\Gamma} } =\frac 1\ell \cdot \cbr{\Gamma - \frac 1\Gamma} \approx 10{,}7\,\text{дптр}.
    \\
    &\text{Если изображение мнимое:} \\
    D &= \frac 1F = \frac 1a - \frac 1b, \qquad \Gamma = \frac ba, \qquad b - a = \ell \implies b = \Gamma a \implies \Gamma a - a = \ell \implies  \\
    a &= \frac {\ell}{\Gamma - 1} \implies b = \frac {{\ell} \Gamma}{\Gamma - 1} \implies  \\
    D &= \frac {\Gamma - 1}\ell - \frac {\Gamma - 1}{\ell \Gamma} = \frac 1\ell \cdot \cbr{\Gamma - 1 - \frac {\Gamma - 1}{\Gamma} } =\frac 1\ell \cdot \cbr{\Gamma + \frac 1\Gamma - 2} \approx 6{,}4\,\text{дптр}.
    \\
    &\text{В ответе надо указать оба значения.}
    \end{align*}
}
\solutionspace{120pt}

\tasknumber{6}%
\task{%
    Оптическая сила объектива фотоаппарата равна $3\,\text{дптр}$.
    При фотографировании чертежа с расстояния $1{,}1\,\text{м}$ площадь изображения
    чертежа на фотопластинке оказалась равной $9\,\text{см}^{2}$.
    Какова площадь самого чертежа? Ответ выразите в квадратных сантиметрах.
}
\answer{%
    \begin{align*}
    &\frac 1a + \frac 1b = \frac 1F = D \implies b = \frac{aF}{a - F} \\
    &\frac {S'}S = \Gamma^2 = \sqr{\frac ba} = \sqr{\frac F{a - F}} \implies \\
    &\implies S = S' \cdot \sqr{\frac{a - F}F} = S' \cdot \sqr{\frac aF - 1} = S' \cdot \sqr{aD - 1} \approx 50\,\text{см}^{2}.
    \end{align*}
}


\variantsplitter


\addpersonalvariant{Семён Мартынов}

\tasknumber{7}%
\task{%
    В каком месте на главной оптической оси двояковыпуклой линзы
    нужно поместить точечный источник света,
    чтобы его изображение оказалось в главном фокусе линзы?
}
\answer{%
    $\text{для мнимого - на половине фокусного, для действительного - на бесконечности}$
}
\solutionspace{120pt}

\tasknumber{8}%
\task{%
    Предмет в виде отрезка длиной $\ell$ расположен вдоль оптической оси
    собирающей линзы с фокусным расстоянием $F$.
    Середина отрезка расположена
    на расстоянии $a$ от линзы, которая даёт действительное изображение
    всех точек предмета.
    Определить продольное увеличение предмета.
}
\answer{%
    \begin{align*}
    \frac 1{a + \frac \ell 2} &+ \frac 1b = \frac 1F \implies b = \frac{F\cbr{a + \frac \ell 2}}{a + \frac \ell 2 - F} \\
    \frac 1{a - \frac \ell 2} &+ \frac 1c = \frac 1F \implies c = \frac{F\cbr{a - \frac \ell 2}}{a - \frac \ell 2 - F} \\
    \abs{b - c} &= \abs{\frac{F\cbr{a + \frac \ell 2}}{a + \frac \ell 2 - F} - \frac{F\cbr{a - \frac \ell 2}}{a - \frac \ell 2 - F}}= F\abs{\frac{\cbr{a + \frac \ell 2}\cbr{a - \frac \ell 2 - F} - \cbr{a - \frac \ell 2}\cbr{a + \frac \ell 2 - F}}{ \cbr{a + \frac \ell 2 - F} \cbr{a - \frac \ell 2 - F} }} =  \\
    &= F\abs{\frac{a^2 - \frac {a\ell} 2 - Fa + \frac {a\ell} 2 - \frac {\ell^2} 4 - \frac {F\ell}2 - a^2 - \frac {a\ell}2 + aF + \frac {a\ell}2 + \frac {\ell^2} 4 - \frac {F\ell} 2}{\cbr{a + \frac \ell 2 - F} \cbr{a - \frac \ell 2 - F} }} = \\
    &= F\frac{F\ell}{\sqr{a-F} - \frac {\ell^2}4} = \frac{F^2\ell}{\sqr{a-F} - \frac {\ell^2}4}\implies \Gamma = \frac{\abs{b - c}}\ell = \frac{F^2}{\sqr{a-F} - \frac {\ell^2}4}.
    \end{align*}
}
\solutionspace{120pt}

\tasknumber{9}%
\task{%
    На экране с помощью тонкой линзы получено изображение предмета
    с увеличением $2$.
    Предмет передвинули на $8\,\text{см}$.
    Для того, чтобы получить резкое изображение, пришлось передвинуть экран.
    При этом увеличение оказалось равным $8$.
    На какое расстояние
    пришлось передвинуть экран?
}
\answer{%
    \begin{align*}
    &\frac 1a + \frac 1b = \frac 1F, \Gamma_1 = \frac ba = \frac{F}{a-F} \implies \Gamma_1(a-F) = F \implies a = F \cdot \frac{1 + \Gamma_1}{\Gamma_1} \\
    &\frac 1{a + x} + \frac 1{b + y} = \frac 1F, \Gamma_2 = \frac {b+y}{a+x} = \frac{F}{a+x-F} \implies a + x = F \cdot \frac{1 + \Gamma_2}{\Gamma_2} \\
    &1 + \frac xa = \frac{ \frac{1 + \Gamma_2}{\Gamma_2} }{ \frac{1 + \Gamma_1}{\Gamma_1} } = \frac{\Gamma_1(1 + \Gamma_2)}{\Gamma_2(1 + \Gamma_1)} \\
    &a = \frac x{ \frac{\Gamma_1(1 + \Gamma_2)}{\Gamma_2(1 + \Gamma_1)} - 1} = x \cdot \frac{\Gamma_2(1 + \Gamma_1)}{\Gamma_1 - \Gamma_2} \\
    &y = (a + x)\Gamma_2 - b = (a + x)\Gamma_2 - a\Gamma_1 = a(\Gamma_2 - \Gamma_1) + x\Gamma_2 = -x\Gamma_2(1 + \Gamma_1) + x\Gamma_2 = -x\Gamma_2\Gamma_1 = 128\,\text{см}, \\
    &\text{знаки разные, т.е.
    экран надо было подвинуть в другую сторону чем предмет: $x < 0, y > 0$.}
    \end{align*}
}
\solutionspace{120pt}

\tasknumber{10}%
\task{%
    Тонкая собирающая линза дает изображение предмета на экране при двух положениях линзы между предметом и экраном.
    Высота изображения при первом положении $25\,\text{см}$, во втором — $9\,\text{см}$.
    Расстояние между предметом и экранов постоянно.
    Чему равна высота предмета?
}
\answer{%
    \begin{align*}
    &\frac 1a + \frac 1b = \frac 1F, \frac 1c + \frac 1d = \frac 1F, a + b = c + d \implies \frac{a + b}{ab} = \frac 1F = \frac{c+d}{cd} \implies ab = cd, \\
    &\implies ab = c(a + b - c) \implies c^2 - ac - bc + ab = 0 \implies c = a \text{ или } c = b \implies c = b \implies d = a.
    \\
    &\Gamma_1 = \frac {H_1}H = \frac ba, \Gamma_2 = \frac {H_2}H = \frac dc = \frac ab \implies \frac {H_1}H \cdot \frac {H_2}H = \frac ba \cdot \frac ab = 1, \\
    &H = \sqrt{H_1 H_2} \approx 15\,\text{см}.
    \end{align*}
}
\solutionspace{120pt}

\tasknumber{11}%
\task{%
    Какие предметы можно рассмотреть на фотографии, сделанной со спутника,
    если разрешающая способность плёнки $0{,}02\,\text{мм}$? Каким должно быть
    время экспозиции $\tau$ чтобы полностью использовать возможности плёнки?
    Фокусное расстояние объектива используемого фотоаппарата $10\,\text{см}$,
    высота орбиты спутника $120\,\text{км}$.
}
\answer{%
    \begin{align*}
    &H \ll R \implies v = v_{\text{I}} = \sqrt{G R} \approx 7{,}9\,\frac{\text{км}}{\text{с}}.
    \\
    &F \ll H \implies b = F, a = H, \\
    &\Gamma = \frac \delta\ell = \frac ba \implies \ell = \frac{\delta a}b = \frac{\delta H}F \approx \frac{0{,}02\,\text{мм} \cdot 120\,\text{км}}{10\,\text{см}} \approx 24\,\text{м}, \\
    &\implies \tau = \frac \ell v = \frac{\delta H}{F v} = \frac{0{,}02\,\text{мм} \cdot 120\,\text{км}}{10\,\text{см} \cdot 7{,}9\,\frac{\text{км}}{\text{с}}} \approx 3\,\text{мс}.
    \end{align*}
}


\variantsplitter


\addpersonalvariant{Семён Мартынов}

\tasknumber{12}%
\task{%
    При аэрофотосъемках используется фотоаппарат, объектив которого
    имеет фокусиое расстояние $10\,\text{см}$.
    Разрешающая способность плёнки $0{,}015\,\text{мм}$.
    На какой высоте должен лететь самолет, чтобы на фотографии можно
    было различить следы размером $30\,\text{см}$?
    При какой скорости самолета изображение не будет размытым,
    если время экспозиции $2\,\text{мс}$?
}
\answer{%
    \begin{align*}
    &F \ll H \implies b = F, a = H, \\
    &\Gamma = \frac \delta\ell = \frac ba = \frac FH \implies H = \frac{\ell F}\delta = \frac{30\,\text{см} \cdot 10\,\text{см}}{0{,}015\,\text{мм}} \approx 2\,\text{км}, \\
    &\implies v = \frac l\tau = \frac{30\,\text{см}}{2\,\text{мс}} \approx 540\,\frac{\text{км}}{\text{ч}}.
    \end{align*}
}
\solutionspace{120pt}

\tasknumber{13}%
\task{%
    Две одинаковые собиращие линзы установлены так, что их главные оптические оси совпадают,
    а главный фокус первой находится там же, где главный фокус второй.
    Расстояние от первой линзы до предмета равно $16\,\text{см}$.
    Чему равно расстояние от изображения объекта во второй линзе до самого объекта?
    Определите также увеличение.
    Фокусное расстояние каждой линзы $20\,\text{см}$.
}
\answer{%
    \begin{align*}
    \frac 1a + \frac 1b &= \frac 1F \implies b = \frac{aF}{a - F} \implies 2F - b = \frac{2aF - 2F^2 - aF}{a - F} = \frac{F(a - 2F)}{a - F}.
    \\
    \frac 1{2F - b} + \frac 1c &= \frac 1F \implies c = \frac{F(2F-b)}{(2F - b) - F} = \frac{F \cdot \frac{F(a - 2F)}{a - F}}{\frac{F(a - 2F)}{a - F} - F}  = F \cdot \frac{ \frac{F(a - 2F)}{a - F} }{ \frac{F(a - 2F)}{a - F} - 1} = \\
     &= F \cdot \frac{a - 2F}{a - 2F - a + F} = 2F - a = 24\,\text{см}.
     \\
    \ell &= a + 2F + c = 4F = 80\,\text{см}.
    \\
    &\Gamma = \Gamma_1 \cdot \Gamma_2 = \frac ba \cdot \frac c{2F-b} = \frac F{a - F} \cdot \frac{2F - a}{\frac{F(a - 2F)}{a - F}} = -1.
    \end{align*}
}
\solutionspace{120pt}

\tasknumber{14}%
\task{%
    Собирающая линза с фокусным расстоянием $F_1 > 0$ и рассеивающая линза с фокусным расстоянием $F_2 < 0$
    установлены коаксиально на расстоянии $\ell$.
    Пучок параллельных лучей падает на собирающую линзу.
    Сделайте схематичное построение и определите, в какой точке система из этих линз соберёт пучок.
}
\answer{%
    \begin{align*}
    &\text{Если пучок падает на собирающую линзу:} \\
    \frac 1{\infty} + \frac 1b &= \frac 1{F_1} \implies b = F_1 \implies \ell - b = \ell - F_1 \\
    \frac 1{\ell - b} + \frac 1c &= \frac 1{F_2} \implies c = \frac{F_2(\ell - b)}{\ell - b - F_2} = \frac{F_2(\ell - F_1)}{\ell - F_1 - F_2}.
    \\
    &\text{Если же пучок падает на рассеивающую линзу:} \\
    \frac 1{\infty} + \frac 1b &= \frac 1{F_2} \implies b = F_2 \implies \ell - b = \ell - F_2 \\
    \frac 1{\ell - b} + \frac 1c &= \frac 1{F_1} \implies c = \frac{F_1(\ell - b)}{\ell - b - F_1} = \frac{F_1(\ell - F_2)}{\ell - F_2 - F_1}.
    \end{align*}
}
\solutionspace{120pt}

\tasknumber{15}%
\task{%
    Две собирающих линзы с фокусными расстояниями $40\,\text{см}$ и $25\,\text{см}$ расположены так,
    что их оптические оси совмещены.
    На первую линзу падает пучок параллельных лучей.
    Пройдя через вторую линзу, он остался параллельным.
    Найдите расстояние между линзами и сделайте рисунок.
}
\answer{%
    \begin{align*}
    \frac 1\infty + \frac 1b &= \frac 1{F_1} \implies b = F_1, \\
    \frac 1{\ell - b} + \frac 1{\infty} &= \frac 1{F_2} \implies \ell - b = F_2 \implies \ell = b + F_2 = F_1 + F_2 = 65\,\text{см}.
    \end{align*}
}

\variantsplitter

\addpersonalvariant{Варвара Минаева}

\tasknumber{1}%
\task{%
    Найти оптическую силу собирающей линзы, если действительное изображение предмета,
    помещённого в $55\,\text{см}$ от линзы, получается на расстоянии $30\,\text{см}$ от неё.
}
\answer{%
    $D = \frac 1F = \frac 1a + \frac 1b = \frac 1{55\,\text{см}} + \frac 1{30\,\text{см}} \approx 5{,}15\,\text{дптр}$
}
\solutionspace{80pt}

\tasknumber{2}%
\task{%
    Найти увеличение изображения, если изображение предмета, находящегося
    на расстоянии $20\,\text{см}$ от линзы, получается на расстоянии $30\,\text{см}$ от неё.
}
\answer{%
    $\Gamma = \frac ba = \frac {30\,\text{см}}{20\,\text{см}} \approx 1{,}50$
}
\solutionspace{80pt}

\tasknumber{3}%
\task{%
    Расстояние от предмета до линзы $10\,\text{см}$, а от линзы до мнимого изображения $30\,\text{см}$.
    Чему равно фокусное расстояние линзы?
}
\answer{%
    $\pm \frac 1F = \frac 1a - \frac 1b \implies F = \frac{a b}{\abs{b - a}} \approx 15\,\text{см}$
}
\solutionspace{80pt}

\tasknumber{4}%
\task{%
    Две тонкие собирающие линзы с фокусными расстояниями $12\,\text{см}$ и $30\,\text{см}$ сложены вместе.
    Чему равно фокусное расстояние такой оптической системы?
}
\answer{%
    $\frac 1{f_1} = \frac 1a + \frac 1b; \frac 1{f_2} = - \frac 1b + \frac 1c \implies \frac 1{f_1} + \frac 1{f_2} = \frac 1a + \frac 1c \implies f' = \frac 1{\frac 1{f_1} + \frac 1{f_2}} = \frac{f_1 f_2}{f_1 + f_2} \approx 8{,}6\,\text{см}$
}
\solutionspace{80pt}

\tasknumber{5}%
\task{%
    Линейные размеры прямого изображения предмета, полученного в собирающей линзе,
    в четыре раза больше линейных размеров предмета.
    Зная, что предмет находится на $25\,\text{см}$ ближе к линзе,
    чем его изображение, найти оптическую силу линзы.
}
\answer{%
    \begin{align*}
    &\text{Если изображение действительное:} \\
    D &= \frac 1F = \frac 1a + \frac 1b, \qquad \Gamma = \frac ba, \qquad b - a = \ell \implies b = \Gamma a \implies \Gamma a - a = \ell \implies  \\
    a &= \frac {\ell}{\Gamma - 1} \implies b = \frac {{\ell} \Gamma}{\Gamma - 1} \implies  \\
    D &= \frac {\Gamma - 1}\ell + \frac {\Gamma - 1}{\ell \Gamma} = \frac 1\ell \cdot \cbr{\Gamma - 1 + \frac {\Gamma - 1}{\Gamma} } =\frac 1\ell \cdot \cbr{\Gamma - \frac 1\Gamma} \approx 15\,\text{дптр}.
    \\
    &\text{Если изображение мнимое:} \\
    D &= \frac 1F = \frac 1a - \frac 1b, \qquad \Gamma = \frac ba, \qquad b - a = \ell \implies b = \Gamma a \implies \Gamma a - a = \ell \implies  \\
    a &= \frac {\ell}{\Gamma - 1} \implies b = \frac {{\ell} \Gamma}{\Gamma - 1} \implies  \\
    D &= \frac {\Gamma - 1}\ell - \frac {\Gamma - 1}{\ell \Gamma} = \frac 1\ell \cdot \cbr{\Gamma - 1 - \frac {\Gamma - 1}{\Gamma} } =\frac 1\ell \cdot \cbr{\Gamma + \frac 1\Gamma - 2} \approx 9\,\text{дптр}.
    \\
    &\text{В ответе надо указать оба значения.}
    \end{align*}
}
\solutionspace{120pt}

\tasknumber{6}%
\task{%
    Оптическая сила объектива фотоаппарата равна $5\,\text{дптр}$.
    При фотографировании чертежа с расстояния $1{,}2\,\text{м}$ площадь изображения
    чертежа на фотопластинке оказалась равной $4\,\text{см}^{2}$.
    Какова площадь самого чертежа? Ответ выразите в квадратных сантиметрах.
}
\answer{%
    \begin{align*}
    &\frac 1a + \frac 1b = \frac 1F = D \implies b = \frac{aF}{a - F} \\
    &\frac {S'}S = \Gamma^2 = \sqr{\frac ba} = \sqr{\frac F{a - F}} \implies \\
    &\implies S = S' \cdot \sqr{\frac{a - F}F} = S' \cdot \sqr{\frac aF - 1} = S' \cdot \sqr{aD - 1} \approx 100\,\text{см}^{2}.
    \end{align*}
}


\variantsplitter


\addpersonalvariant{Варвара Минаева}

\tasknumber{7}%
\task{%
    В каком месте на главной оптической оси двояковыгнутой линзы
    нужно поместить точечный источник света,
    чтобы его изображение оказалось в главном фокусе линзы?
}
\answer{%
    $\text{на половине фокусного расстояния}$
}
\solutionspace{120pt}

\tasknumber{8}%
\task{%
    Предмет в виде отрезка длиной $\ell$ расположен вдоль оптической оси
    собирающей линзы с фокусным расстоянием $F$.
    Середина отрезка расположена
    на расстоянии $a$ от линзы, которая даёт действительное изображение
    всех точек предмета.
    Определить продольное увеличение предмета.
}
\answer{%
    \begin{align*}
    \frac 1{a + \frac \ell 2} &+ \frac 1b = \frac 1F \implies b = \frac{F\cbr{a + \frac \ell 2}}{a + \frac \ell 2 - F} \\
    \frac 1{a - \frac \ell 2} &+ \frac 1c = \frac 1F \implies c = \frac{F\cbr{a - \frac \ell 2}}{a - \frac \ell 2 - F} \\
    \abs{b - c} &= \abs{\frac{F\cbr{a + \frac \ell 2}}{a + \frac \ell 2 - F} - \frac{F\cbr{a - \frac \ell 2}}{a - \frac \ell 2 - F}}= F\abs{\frac{\cbr{a + \frac \ell 2}\cbr{a - \frac \ell 2 - F} - \cbr{a - \frac \ell 2}\cbr{a + \frac \ell 2 - F}}{ \cbr{a + \frac \ell 2 - F} \cbr{a - \frac \ell 2 - F} }} =  \\
    &= F\abs{\frac{a^2 - \frac {a\ell} 2 - Fa + \frac {a\ell} 2 - \frac {\ell^2} 4 - \frac {F\ell}2 - a^2 - \frac {a\ell}2 + aF + \frac {a\ell}2 + \frac {\ell^2} 4 - \frac {F\ell} 2}{\cbr{a + \frac \ell 2 - F} \cbr{a - \frac \ell 2 - F} }} = \\
    &= F\frac{F\ell}{\sqr{a-F} - \frac {\ell^2}4} = \frac{F^2\ell}{\sqr{a-F} - \frac {\ell^2}4}\implies \Gamma = \frac{\abs{b - c}}\ell = \frac{F^2}{\sqr{a-F} - \frac {\ell^2}4}.
    \end{align*}
}
\solutionspace{120pt}

\tasknumber{9}%
\task{%
    На экране с помощью тонкой линзы получено изображение предмета
    с увеличением $2$.
    Предмет передвинули на $4\,\text{см}$.
    Для того, чтобы получить резкое изображение, пришлось передвинуть экран.
    При этом увеличение оказалось равным $6$.
    На какое расстояние
    пришлось передвинуть экран?
}
\answer{%
    \begin{align*}
    &\frac 1a + \frac 1b = \frac 1F, \Gamma_1 = \frac ba = \frac{F}{a-F} \implies \Gamma_1(a-F) = F \implies a = F \cdot \frac{1 + \Gamma_1}{\Gamma_1} \\
    &\frac 1{a + x} + \frac 1{b + y} = \frac 1F, \Gamma_2 = \frac {b+y}{a+x} = \frac{F}{a+x-F} \implies a + x = F \cdot \frac{1 + \Gamma_2}{\Gamma_2} \\
    &1 + \frac xa = \frac{ \frac{1 + \Gamma_2}{\Gamma_2} }{ \frac{1 + \Gamma_1}{\Gamma_1} } = \frac{\Gamma_1(1 + \Gamma_2)}{\Gamma_2(1 + \Gamma_1)} \\
    &a = \frac x{ \frac{\Gamma_1(1 + \Gamma_2)}{\Gamma_2(1 + \Gamma_1)} - 1} = x \cdot \frac{\Gamma_2(1 + \Gamma_1)}{\Gamma_1 - \Gamma_2} \\
    &y = (a + x)\Gamma_2 - b = (a + x)\Gamma_2 - a\Gamma_1 = a(\Gamma_2 - \Gamma_1) + x\Gamma_2 = -x\Gamma_2(1 + \Gamma_1) + x\Gamma_2 = -x\Gamma_2\Gamma_1 = 48\,\text{см}, \\
    &\text{знаки разные, т.е.
    экран надо было подвинуть в другую сторону чем предмет: $x < 0, y > 0$.}
    \end{align*}
}
\solutionspace{120pt}

\tasknumber{10}%
\task{%
    Тонкая собирающая линза дает изображение предмета на экране при двух положениях линзы между предметом и экраном.
    Высота изображения при первом положении $25\,\text{см}$, во втором — $5\,\text{см}$.
    Расстояние между предметом и экранов постоянно.
    Чему равна высота предмета?
}
\answer{%
    \begin{align*}
    &\frac 1a + \frac 1b = \frac 1F, \frac 1c + \frac 1d = \frac 1F, a + b = c + d \implies \frac{a + b}{ab} = \frac 1F = \frac{c+d}{cd} \implies ab = cd, \\
    &\implies ab = c(a + b - c) \implies c^2 - ac - bc + ab = 0 \implies c = a \text{ или } c = b \implies c = b \implies d = a.
    \\
    &\Gamma_1 = \frac {H_1}H = \frac ba, \Gamma_2 = \frac {H_2}H = \frac dc = \frac ab \implies \frac {H_1}H \cdot \frac {H_2}H = \frac ba \cdot \frac ab = 1, \\
    &H = \sqrt{H_1 H_2} \approx 11{,}2\,\text{см}.
    \end{align*}
}
\solutionspace{120pt}

\tasknumber{11}%
\task{%
    Какие предметы можно рассмотреть на фотографии, сделанной со спутника,
    если разрешающая способность плёнки $0{,}02\,\text{мм}$? Каким должно быть
    время экспозиции $\tau$ чтобы полностью использовать возможности плёнки?
    Фокусное расстояние объектива используемого фотоаппарата $10\,\text{см}$,
    высота орбиты спутника $100\,\text{км}$.
}
\answer{%
    \begin{align*}
    &H \ll R \implies v = v_{\text{I}} = \sqrt{G R} \approx 7{,}9\,\frac{\text{км}}{\text{с}}.
    \\
    &F \ll H \implies b = F, a = H, \\
    &\Gamma = \frac \delta\ell = \frac ba \implies \ell = \frac{\delta a}b = \frac{\delta H}F \approx \frac{0{,}02\,\text{мм} \cdot 100\,\text{км}}{10\,\text{см}} \approx 20\,\text{м}, \\
    &\implies \tau = \frac \ell v = \frac{\delta H}{F v} = \frac{0{,}02\,\text{мм} \cdot 100\,\text{км}}{10\,\text{см} \cdot 7{,}9\,\frac{\text{км}}{\text{с}}} \approx 3\,\text{мс}.
    \end{align*}
}


\variantsplitter


\addpersonalvariant{Варвара Минаева}

\tasknumber{12}%
\task{%
    При аэрофотосъемках используется фотоаппарат, объектив которого
    имеет фокусиое расстояние $15\,\text{см}$.
    Разрешающая способность плёнки $0{,}010\,\text{мм}$.
    На какой высоте должен лететь самолет, чтобы на фотографии можно
    было различить следы размером $25\,\text{см}$?
    При какой скорости самолета изображение не будет размытым,
    если время экспозиции $2\,\text{мс}$?
}
\answer{%
    \begin{align*}
    &F \ll H \implies b = F, a = H, \\
    &\Gamma = \frac \delta\ell = \frac ba = \frac FH \implies H = \frac{\ell F}\delta = \frac{25\,\text{см} \cdot 15\,\text{см}}{0{,}010\,\text{мм}} \approx 4\,\text{км}, \\
    &\implies v = \frac l\tau = \frac{25\,\text{см}}{2\,\text{мс}} \approx 450\,\frac{\text{км}}{\text{ч}}.
    \end{align*}
}
\solutionspace{120pt}

\tasknumber{13}%
\task{%
    Две одинаковые собиращие линзы установлены так, что их главные оптические оси совпадают,
    а главный фокус первой находится там же, где главный фокус второй.
    Расстояние от первой линзы до предмета равно $27\,\text{см}$.
    Чему равно расстояние от изображения объекта во второй линзе до второй линзы?
    Определите также увеличение.
    Фокусное расстояние каждой линзы $25\,\text{см}$.
}
\answer{%
    \begin{align*}
    \frac 1a + \frac 1b &= \frac 1F \implies b = \frac{aF}{a - F} \implies 2F - b = \frac{2aF - 2F^2 - aF}{a - F} = \frac{F(a - 2F)}{a - F}.
    \\
    \frac 1{2F - b} + \frac 1c &= \frac 1F \implies c = \frac{F(2F-b)}{(2F - b) - F} = \frac{F \cdot \frac{F(a - 2F)}{a - F}}{\frac{F(a - 2F)}{a - F} - F}  = F \cdot \frac{ \frac{F(a - 2F)}{a - F} }{ \frac{F(a - 2F)}{a - F} - 1} = \\
     &= F \cdot \frac{a - 2F}{a - 2F - a + F} = 2F - a = 23\,\text{см}.
     \\
    \ell &= a + 2F + c = 4F = 100\,\text{см}.
    \\
    &\Gamma = \Gamma_1 \cdot \Gamma_2 = \frac ba \cdot \frac c{2F-b} = \frac F{a - F} \cdot \frac{2F - a}{\frac{F(a - 2F)}{a - F}} = -1.
    \end{align*}
}
\solutionspace{120pt}

\tasknumber{14}%
\task{%
    Собирающая линза с фокусным расстоянием $F_1 > 0$ и рассеивающая линза с фокусным расстоянием $F_2 < 0$
    установлены коаксиально на расстоянии $\ell$.
    Пучок параллельных лучей падает на рассеивающую линзу.
    Сделайте схематичное построение и определите, в какой точке система из этих линз соберёт пучок.
}
\answer{%
    \begin{align*}
    &\text{Если пучок падает на собирающую линзу:} \\
    \frac 1{\infty} + \frac 1b &= \frac 1{F_1} \implies b = F_1 \implies \ell - b = \ell - F_1 \\
    \frac 1{\ell - b} + \frac 1c &= \frac 1{F_2} \implies c = \frac{F_2(\ell - b)}{\ell - b - F_2} = \frac{F_2(\ell - F_1)}{\ell - F_1 - F_2}.
    \\
    &\text{Если же пучок падает на рассеивающую линзу:} \\
    \frac 1{\infty} + \frac 1b &= \frac 1{F_2} \implies b = F_2 \implies \ell - b = \ell - F_2 \\
    \frac 1{\ell - b} + \frac 1c &= \frac 1{F_1} \implies c = \frac{F_1(\ell - b)}{\ell - b - F_1} = \frac{F_1(\ell - F_2)}{\ell - F_2 - F_1}.
    \end{align*}
}
\solutionspace{120pt}

\tasknumber{15}%
\task{%
    Две собирающих линзы с фокусными расстояниями $40\,\text{см}$ и $25\,\text{см}$ расположены так,
    что их оптические оси совмещены.
    На первую линзу падает пучок параллельных лучей.
    Пройдя через вторую линзу, он остался параллельным.
    Найдите расстояние между линзами и сделайте рисунок.
}
\answer{%
    \begin{align*}
    \frac 1\infty + \frac 1b &= \frac 1{F_1} \implies b = F_1, \\
    \frac 1{\ell - b} + \frac 1{\infty} &= \frac 1{F_2} \implies \ell - b = F_2 \implies \ell = b + F_2 = F_1 + F_2 = 65\,\text{см}.
    \end{align*}
}

\variantsplitter

\addpersonalvariant{Леонид Никитин}

\tasknumber{1}%
\task{%
    Найти оптическую силу собирающей линзы, если действительное изображение предмета,
    помещённого в $35\,\text{см}$ от линзы, получается на расстоянии $20\,\text{см}$ от неё.
}
\answer{%
    $D = \frac 1F = \frac 1a + \frac 1b = \frac 1{35\,\text{см}} + \frac 1{20\,\text{см}} \approx 7{,}86\,\text{дптр}$
}
\solutionspace{80pt}

\tasknumber{2}%
\task{%
    Найти увеличение изображения, если изображение предмета, находящегося
    на расстоянии $25\,\text{см}$ от линзы, получается на расстоянии $18\,\text{см}$ от неё.
}
\answer{%
    $\Gamma = \frac ba = \frac {18\,\text{см}}{25\,\text{см}} \approx 0{,}7$
}
\solutionspace{80pt}

\tasknumber{3}%
\task{%
    Расстояние от предмета до линзы $12\,\text{см}$, а от линзы до мнимого изображения $25\,\text{см}$.
    Чему равно фокусное расстояние линзы?
}
\answer{%
    $\pm \frac 1F = \frac 1a - \frac 1b \implies F = \frac{a b}{\abs{b - a}} \approx 23{,}1\,\text{см}$
}
\solutionspace{80pt}

\tasknumber{4}%
\task{%
    Две тонкие собирающие линзы с фокусными расстояниями $18\,\text{см}$ и $30\,\text{см}$ сложены вместе.
    Чему равно фокусное расстояние такой оптической системы?
}
\answer{%
    $\frac 1{f_1} = \frac 1a + \frac 1b; \frac 1{f_2} = - \frac 1b + \frac 1c \implies \frac 1{f_1} + \frac 1{f_2} = \frac 1a + \frac 1c \implies f' = \frac 1{\frac 1{f_1} + \frac 1{f_2}} = \frac{f_1 f_2}{f_1 + f_2} \approx 11{,}2\,\text{см}$
}
\solutionspace{80pt}

\tasknumber{5}%
\task{%
    Линейные размеры прямого изображения предмета, полученного в собирающей линзе,
    в три раза больше линейных размеров предмета.
    Зная, что предмет находится на $40\,\text{см}$ ближе к линзе,
    чем его изображение, найти оптическую силу линзы.
}
\answer{%
    \begin{align*}
    &\text{Если изображение действительное:} \\
    D &= \frac 1F = \frac 1a + \frac 1b, \qquad \Gamma = \frac ba, \qquad b - a = \ell \implies b = \Gamma a \implies \Gamma a - a = \ell \implies  \\
    a &= \frac {\ell}{\Gamma - 1} \implies b = \frac {{\ell} \Gamma}{\Gamma - 1} \implies  \\
    D &= \frac {\Gamma - 1}\ell + \frac {\Gamma - 1}{\ell \Gamma} = \frac 1\ell \cdot \cbr{\Gamma - 1 + \frac {\Gamma - 1}{\Gamma} } =\frac 1\ell \cdot \cbr{\Gamma - \frac 1\Gamma} \approx 6{,}7\,\text{дптр}.
    \\
    &\text{Если изображение мнимое:} \\
    D &= \frac 1F = \frac 1a - \frac 1b, \qquad \Gamma = \frac ba, \qquad b - a = \ell \implies b = \Gamma a \implies \Gamma a - a = \ell \implies  \\
    a &= \frac {\ell}{\Gamma - 1} \implies b = \frac {{\ell} \Gamma}{\Gamma - 1} \implies  \\
    D &= \frac {\Gamma - 1}\ell - \frac {\Gamma - 1}{\ell \Gamma} = \frac 1\ell \cdot \cbr{\Gamma - 1 - \frac {\Gamma - 1}{\Gamma} } =\frac 1\ell \cdot \cbr{\Gamma + \frac 1\Gamma - 2} \approx 3{,}3\,\text{дптр}.
    \\
    &\text{В ответе надо указать оба значения.}
    \end{align*}
}
\solutionspace{120pt}

\tasknumber{6}%
\task{%
    Оптическая сила объектива фотоаппарата равна $3\,\text{дптр}$.
    При фотографировании чертежа с расстояния $0{,}9\,\text{м}$ площадь изображения
    чертежа на фотопластинке оказалась равной $4\,\text{см}^{2}$.
    Какова площадь самого чертежа? Ответ выразите в квадратных сантиметрах.
}
\answer{%
    \begin{align*}
    &\frac 1a + \frac 1b = \frac 1F = D \implies b = \frac{aF}{a - F} \\
    &\frac {S'}S = \Gamma^2 = \sqr{\frac ba} = \sqr{\frac F{a - F}} \implies \\
    &\implies S = S' \cdot \sqr{\frac{a - F}F} = S' \cdot \sqr{\frac aF - 1} = S' \cdot \sqr{aD - 1} \approx 12\,\text{см}^{2}.
    \end{align*}
}


\variantsplitter


\addpersonalvariant{Леонид Никитин}

\tasknumber{7}%
\task{%
    В каком месте на главной оптической оси двояковыгнутой линзы
    нужно поместить точечный источник света,
    чтобы его изображение оказалось в главном фокусе линзы?
}
\answer{%
    $\text{на половине фокусного расстояния}$
}
\solutionspace{120pt}

\tasknumber{8}%
\task{%
    Предмет в виде отрезка длиной $\ell$ расположен вдоль оптической оси
    собирающей линзы с фокусным расстоянием $F$.
    Середина отрезка расположена
    на расстоянии $a$ от линзы, которая даёт действительное изображение
    всех точек предмета.
    Определить продольное увеличение предмета.
}
\answer{%
    \begin{align*}
    \frac 1{a + \frac \ell 2} &+ \frac 1b = \frac 1F \implies b = \frac{F\cbr{a + \frac \ell 2}}{a + \frac \ell 2 - F} \\
    \frac 1{a - \frac \ell 2} &+ \frac 1c = \frac 1F \implies c = \frac{F\cbr{a - \frac \ell 2}}{a - \frac \ell 2 - F} \\
    \abs{b - c} &= \abs{\frac{F\cbr{a + \frac \ell 2}}{a + \frac \ell 2 - F} - \frac{F\cbr{a - \frac \ell 2}}{a - \frac \ell 2 - F}}= F\abs{\frac{\cbr{a + \frac \ell 2}\cbr{a - \frac \ell 2 - F} - \cbr{a - \frac \ell 2}\cbr{a + \frac \ell 2 - F}}{ \cbr{a + \frac \ell 2 - F} \cbr{a - \frac \ell 2 - F} }} =  \\
    &= F\abs{\frac{a^2 - \frac {a\ell} 2 - Fa + \frac {a\ell} 2 - \frac {\ell^2} 4 - \frac {F\ell}2 - a^2 - \frac {a\ell}2 + aF + \frac {a\ell}2 + \frac {\ell^2} 4 - \frac {F\ell} 2}{\cbr{a + \frac \ell 2 - F} \cbr{a - \frac \ell 2 - F} }} = \\
    &= F\frac{F\ell}{\sqr{a-F} - \frac {\ell^2}4} = \frac{F^2\ell}{\sqr{a-F} - \frac {\ell^2}4}\implies \Gamma = \frac{\abs{b - c}}\ell = \frac{F^2}{\sqr{a-F} - \frac {\ell^2}4}.
    \end{align*}
}
\solutionspace{120pt}

\tasknumber{9}%
\task{%
    На экране с помощью тонкой линзы получено изображение предмета
    с увеличением $4$.
    Предмет передвинули на $6\,\text{см}$.
    Для того, чтобы получить резкое изображение, пришлось передвинуть экран.
    При этом увеличение оказалось равным $8$.
    На какое расстояние
    пришлось передвинуть экран?
}
\answer{%
    \begin{align*}
    &\frac 1a + \frac 1b = \frac 1F, \Gamma_1 = \frac ba = \frac{F}{a-F} \implies \Gamma_1(a-F) = F \implies a = F \cdot \frac{1 + \Gamma_1}{\Gamma_1} \\
    &\frac 1{a + x} + \frac 1{b + y} = \frac 1F, \Gamma_2 = \frac {b+y}{a+x} = \frac{F}{a+x-F} \implies a + x = F \cdot \frac{1 + \Gamma_2}{\Gamma_2} \\
    &1 + \frac xa = \frac{ \frac{1 + \Gamma_2}{\Gamma_2} }{ \frac{1 + \Gamma_1}{\Gamma_1} } = \frac{\Gamma_1(1 + \Gamma_2)}{\Gamma_2(1 + \Gamma_1)} \\
    &a = \frac x{ \frac{\Gamma_1(1 + \Gamma_2)}{\Gamma_2(1 + \Gamma_1)} - 1} = x \cdot \frac{\Gamma_2(1 + \Gamma_1)}{\Gamma_1 - \Gamma_2} \\
    &y = (a + x)\Gamma_2 - b = (a + x)\Gamma_2 - a\Gamma_1 = a(\Gamma_2 - \Gamma_1) + x\Gamma_2 = -x\Gamma_2(1 + \Gamma_1) + x\Gamma_2 = -x\Gamma_2\Gamma_1 = 192\,\text{см}, \\
    &\text{знаки разные, т.е.
    экран надо было подвинуть в другую сторону чем предмет: $x < 0, y > 0$.}
    \end{align*}
}
\solutionspace{120pt}

\tasknumber{10}%
\task{%
    Тонкая собирающая линза дает изображение предмета на экране при двух положениях линзы между предметом и экраном.
    Высота изображения при первом положении $15\,\text{см}$, во втором — $9\,\text{см}$.
    Расстояние между предметом и экранов постоянно.
    Чему равна высота предмета?
}
\answer{%
    \begin{align*}
    &\frac 1a + \frac 1b = \frac 1F, \frac 1c + \frac 1d = \frac 1F, a + b = c + d \implies \frac{a + b}{ab} = \frac 1F = \frac{c+d}{cd} \implies ab = cd, \\
    &\implies ab = c(a + b - c) \implies c^2 - ac - bc + ab = 0 \implies c = a \text{ или } c = b \implies c = b \implies d = a.
    \\
    &\Gamma_1 = \frac {H_1}H = \frac ba, \Gamma_2 = \frac {H_2}H = \frac dc = \frac ab \implies \frac {H_1}H \cdot \frac {H_2}H = \frac ba \cdot \frac ab = 1, \\
    &H = \sqrt{H_1 H_2} \approx 11{,}6\,\text{см}.
    \end{align*}
}
\solutionspace{120pt}

\tasknumber{11}%
\task{%
    Какие предметы можно рассмотреть на фотографии, сделанной со спутника,
    если разрешающая способность плёнки $0{,}010\,\text{мм}$? Каким должно быть
    время экспозиции $\tau$ чтобы полностью использовать возможности плёнки?
    Фокусное расстояние объектива используемого фотоаппарата $15\,\text{см}$,
    высота орбиты спутника $120\,\text{км}$.
}
\answer{%
    \begin{align*}
    &H \ll R \implies v = v_{\text{I}} = \sqrt{G R} \approx 7{,}9\,\frac{\text{км}}{\text{с}}.
    \\
    &F \ll H \implies b = F, a = H, \\
    &\Gamma = \frac \delta\ell = \frac ba \implies \ell = \frac{\delta a}b = \frac{\delta H}F \approx \frac{0{,}010\,\text{мм} \cdot 120\,\text{км}}{15\,\text{см}} \approx 8\,\text{м}, \\
    &\implies \tau = \frac \ell v = \frac{\delta H}{F v} = \frac{0{,}010\,\text{мм} \cdot 120\,\text{км}}{15\,\text{см} \cdot 7{,}9\,\frac{\text{км}}{\text{с}}} \approx 1{,}0\,\text{мс}.
    \end{align*}
}


\variantsplitter


\addpersonalvariant{Леонид Никитин}

\tasknumber{12}%
\task{%
    При аэрофотосъемках используется фотоаппарат, объектив которого
    имеет фокусиое расстояние $10\,\text{см}$.
    Разрешающая способность плёнки $0{,}02\,\text{мм}$.
    На какой высоте должен лететь самолет, чтобы на фотографии можно
    было различить следы размером $25\,\text{см}$?
    При какой скорости самолета изображение не будет размытым,
    если время экспозиции $2\,\text{мс}$?
}
\answer{%
    \begin{align*}
    &F \ll H \implies b = F, a = H, \\
    &\Gamma = \frac \delta\ell = \frac ba = \frac FH \implies H = \frac{\ell F}\delta = \frac{25\,\text{см} \cdot 10\,\text{см}}{0{,}02\,\text{мм}} \approx 1{,}3\,\text{км}, \\
    &\implies v = \frac l\tau = \frac{25\,\text{см}}{2\,\text{мс}} \approx 450\,\frac{\text{км}}{\text{ч}}.
    \end{align*}
}
\solutionspace{120pt}

\tasknumber{13}%
\task{%
    Две одинаковые собиращие линзы установлены так, что их главные оптические оси совпадают,
    а главный фокус первой находится там же, где главный фокус второй.
    Расстояние от первой линзы до предмета равно $32\,\text{см}$.
    Чему равно расстояние от изображения объекта во второй линзе до самого объекта?
    Определите также увеличение.
    Фокусное расстояние каждой линзы $40\,\text{см}$.
}
\answer{%
    \begin{align*}
    \frac 1a + \frac 1b &= \frac 1F \implies b = \frac{aF}{a - F} \implies 2F - b = \frac{2aF - 2F^2 - aF}{a - F} = \frac{F(a - 2F)}{a - F}.
    \\
    \frac 1{2F - b} + \frac 1c &= \frac 1F \implies c = \frac{F(2F-b)}{(2F - b) - F} = \frac{F \cdot \frac{F(a - 2F)}{a - F}}{\frac{F(a - 2F)}{a - F} - F}  = F \cdot \frac{ \frac{F(a - 2F)}{a - F} }{ \frac{F(a - 2F)}{a - F} - 1} = \\
     &= F \cdot \frac{a - 2F}{a - 2F - a + F} = 2F - a = 48\,\text{см}.
     \\
    \ell &= a + 2F + c = 4F = 160\,\text{см}.
    \\
    &\Gamma = \Gamma_1 \cdot \Gamma_2 = \frac ba \cdot \frac c{2F-b} = \frac F{a - F} \cdot \frac{2F - a}{\frac{F(a - 2F)}{a - F}} = -1.
    \end{align*}
}
\solutionspace{120pt}

\tasknumber{14}%
\task{%
    Собирающая линза с фокусным расстоянием $F_1 > 0$ и рассеивающая линза с фокусным расстоянием $F_2 < 0$
    установлены коаксиально на расстоянии $\ell$.
    Пучок параллельных лучей падает на рассеивающую линзу.
    Сделайте схематичное построение и определите, в какой точке система из этих линз соберёт пучок.
}
\answer{%
    \begin{align*}
    &\text{Если пучок падает на собирающую линзу:} \\
    \frac 1{\infty} + \frac 1b &= \frac 1{F_1} \implies b = F_1 \implies \ell - b = \ell - F_1 \\
    \frac 1{\ell - b} + \frac 1c &= \frac 1{F_2} \implies c = \frac{F_2(\ell - b)}{\ell - b - F_2} = \frac{F_2(\ell - F_1)}{\ell - F_1 - F_2}.
    \\
    &\text{Если же пучок падает на рассеивающую линзу:} \\
    \frac 1{\infty} + \frac 1b &= \frac 1{F_2} \implies b = F_2 \implies \ell - b = \ell - F_2 \\
    \frac 1{\ell - b} + \frac 1c &= \frac 1{F_1} \implies c = \frac{F_1(\ell - b)}{\ell - b - F_1} = \frac{F_1(\ell - F_2)}{\ell - F_2 - F_1}.
    \end{align*}
}
\solutionspace{120pt}

\tasknumber{15}%
\task{%
    Две собирающих линзы с фокусными расстояниями $40\,\text{см}$ и $25\,\text{см}$ расположены так,
    что их оптические оси совмещены.
    На первую линзу падает пучок параллельных лучей.
    Пройдя через вторую линзу, он остался параллельным.
    Найдите расстояние между линзами и сделайте рисунок.
}
\answer{%
    \begin{align*}
    \frac 1\infty + \frac 1b &= \frac 1{F_1} \implies b = F_1, \\
    \frac 1{\ell - b} + \frac 1{\infty} &= \frac 1{F_2} \implies \ell - b = F_2 \implies \ell = b + F_2 = F_1 + F_2 = 65\,\text{см}.
    \end{align*}
}

\variantsplitter

\addpersonalvariant{Тимофей Полетаев}

\tasknumber{1}%
\task{%
    Найти оптическую силу собирающей линзы, если действительное изображение предмета,
    помещённого в $15\,\text{см}$ от линзы, получается на расстоянии $20\,\text{см}$ от неё.
}
\answer{%
    $D = \frac 1F = \frac 1a + \frac 1b = \frac 1{15\,\text{см}} + \frac 1{20\,\text{см}} \approx 11{,}67\,\text{дптр}$
}
\solutionspace{80pt}

\tasknumber{2}%
\task{%
    Найти увеличение изображения, если изображение предмета, находящегося
    на расстоянии $15\,\text{см}$ от линзы, получается на расстоянии $12\,\text{см}$ от неё.
}
\answer{%
    $\Gamma = \frac ba = \frac {12\,\text{см}}{15\,\text{см}} \approx 0{,}8$
}
\solutionspace{80pt}

\tasknumber{3}%
\task{%
    Расстояние от предмета до линзы $8\,\text{см}$, а от линзы до мнимого изображения $20\,\text{см}$.
    Чему равно фокусное расстояние линзы?
}
\answer{%
    $\pm \frac 1F = \frac 1a - \frac 1b \implies F = \frac{a b}{\abs{b - a}} \approx 13{,}3\,\text{см}$
}
\solutionspace{80pt}

\tasknumber{4}%
\task{%
    Две тонкие собирающие линзы с фокусными расстояниями $18\,\text{см}$ и $30\,\text{см}$ сложены вместе.
    Чему равно фокусное расстояние такой оптической системы?
}
\answer{%
    $\frac 1{f_1} = \frac 1a + \frac 1b; \frac 1{f_2} = - \frac 1b + \frac 1c \implies \frac 1{f_1} + \frac 1{f_2} = \frac 1a + \frac 1c \implies f' = \frac 1{\frac 1{f_1} + \frac 1{f_2}} = \frac{f_1 f_2}{f_1 + f_2} \approx 11{,}2\,\text{см}$
}
\solutionspace{80pt}

\tasknumber{5}%
\task{%
    Линейные размеры прямого изображения предмета, полученного в собирающей линзе,
    в четыре раза больше линейных размеров предмета.
    Зная, что предмет находится на $40\,\text{см}$ ближе к линзе,
    чем его изображение, найти оптическую силу линзы.
}
\answer{%
    \begin{align*}
    &\text{Если изображение действительное:} \\
    D &= \frac 1F = \frac 1a + \frac 1b, \qquad \Gamma = \frac ba, \qquad b - a = \ell \implies b = \Gamma a \implies \Gamma a - a = \ell \implies  \\
    a &= \frac {\ell}{\Gamma - 1} \implies b = \frac {{\ell} \Gamma}{\Gamma - 1} \implies  \\
    D &= \frac {\Gamma - 1}\ell + \frac {\Gamma - 1}{\ell \Gamma} = \frac 1\ell \cdot \cbr{\Gamma - 1 + \frac {\Gamma - 1}{\Gamma} } =\frac 1\ell \cdot \cbr{\Gamma - \frac 1\Gamma} \approx 9{,}4\,\text{дптр}.
    \\
    &\text{Если изображение мнимое:} \\
    D &= \frac 1F = \frac 1a - \frac 1b, \qquad \Gamma = \frac ba, \qquad b - a = \ell \implies b = \Gamma a \implies \Gamma a - a = \ell \implies  \\
    a &= \frac {\ell}{\Gamma - 1} \implies b = \frac {{\ell} \Gamma}{\Gamma - 1} \implies  \\
    D &= \frac {\Gamma - 1}\ell - \frac {\Gamma - 1}{\ell \Gamma} = \frac 1\ell \cdot \cbr{\Gamma - 1 - \frac {\Gamma - 1}{\Gamma} } =\frac 1\ell \cdot \cbr{\Gamma + \frac 1\Gamma - 2} \approx 5{,}6\,\text{дптр}.
    \\
    &\text{В ответе надо указать оба значения.}
    \end{align*}
}
\solutionspace{120pt}

\tasknumber{6}%
\task{%
    Оптическая сила объектива фотоаппарата равна $5\,\text{дптр}$.
    При фотографировании чертежа с расстояния $1{,}2\,\text{м}$ площадь изображения
    чертежа на фотопластинке оказалась равной $16\,\text{см}^{2}$.
    Какова площадь самого чертежа? Ответ выразите в квадратных сантиметрах.
}
\answer{%
    \begin{align*}
    &\frac 1a + \frac 1b = \frac 1F = D \implies b = \frac{aF}{a - F} \\
    &\frac {S'}S = \Gamma^2 = \sqr{\frac ba} = \sqr{\frac F{a - F}} \implies \\
    &\implies S = S' \cdot \sqr{\frac{a - F}F} = S' \cdot \sqr{\frac aF - 1} = S' \cdot \sqr{aD - 1} \approx 400\,\text{см}^{2}.
    \end{align*}
}


\variantsplitter


\addpersonalvariant{Тимофей Полетаев}

\tasknumber{7}%
\task{%
    В каком месте на главной оптической оси двояковыгнутой линзы
    нужно поместить точечный источник света,
    чтобы его изображение оказалось в главном фокусе линзы?
}
\answer{%
    $\text{на половине фокусного расстояния}$
}
\solutionspace{120pt}

\tasknumber{8}%
\task{%
    Предмет в виде отрезка длиной $\ell$ расположен вдоль оптической оси
    собирающей линзы с фокусным расстоянием $F$.
    Середина отрезка расположена
    на расстоянии $a$ от линзы, которая даёт действительное изображение
    всех точек предмета.
    Определить продольное увеличение предмета.
}
\answer{%
    \begin{align*}
    \frac 1{a + \frac \ell 2} &+ \frac 1b = \frac 1F \implies b = \frac{F\cbr{a + \frac \ell 2}}{a + \frac \ell 2 - F} \\
    \frac 1{a - \frac \ell 2} &+ \frac 1c = \frac 1F \implies c = \frac{F\cbr{a - \frac \ell 2}}{a - \frac \ell 2 - F} \\
    \abs{b - c} &= \abs{\frac{F\cbr{a + \frac \ell 2}}{a + \frac \ell 2 - F} - \frac{F\cbr{a - \frac \ell 2}}{a - \frac \ell 2 - F}}= F\abs{\frac{\cbr{a + \frac \ell 2}\cbr{a - \frac \ell 2 - F} - \cbr{a - \frac \ell 2}\cbr{a + \frac \ell 2 - F}}{ \cbr{a + \frac \ell 2 - F} \cbr{a - \frac \ell 2 - F} }} =  \\
    &= F\abs{\frac{a^2 - \frac {a\ell} 2 - Fa + \frac {a\ell} 2 - \frac {\ell^2} 4 - \frac {F\ell}2 - a^2 - \frac {a\ell}2 + aF + \frac {a\ell}2 + \frac {\ell^2} 4 - \frac {F\ell} 2}{\cbr{a + \frac \ell 2 - F} \cbr{a - \frac \ell 2 - F} }} = \\
    &= F\frac{F\ell}{\sqr{a-F} - \frac {\ell^2}4} = \frac{F^2\ell}{\sqr{a-F} - \frac {\ell^2}4}\implies \Gamma = \frac{\abs{b - c}}\ell = \frac{F^2}{\sqr{a-F} - \frac {\ell^2}4}.
    \end{align*}
}
\solutionspace{120pt}

\tasknumber{9}%
\task{%
    На экране с помощью тонкой линзы получено изображение предмета
    с увеличением $4$.
    Предмет передвинули на $4\,\text{см}$.
    Для того, чтобы получить резкое изображение, пришлось передвинуть экран.
    При этом увеличение оказалось равным $6$.
    На какое расстояние
    пришлось передвинуть экран?
}
\answer{%
    \begin{align*}
    &\frac 1a + \frac 1b = \frac 1F, \Gamma_1 = \frac ba = \frac{F}{a-F} \implies \Gamma_1(a-F) = F \implies a = F \cdot \frac{1 + \Gamma_1}{\Gamma_1} \\
    &\frac 1{a + x} + \frac 1{b + y} = \frac 1F, \Gamma_2 = \frac {b+y}{a+x} = \frac{F}{a+x-F} \implies a + x = F \cdot \frac{1 + \Gamma_2}{\Gamma_2} \\
    &1 + \frac xa = \frac{ \frac{1 + \Gamma_2}{\Gamma_2} }{ \frac{1 + \Gamma_1}{\Gamma_1} } = \frac{\Gamma_1(1 + \Gamma_2)}{\Gamma_2(1 + \Gamma_1)} \\
    &a = \frac x{ \frac{\Gamma_1(1 + \Gamma_2)}{\Gamma_2(1 + \Gamma_1)} - 1} = x \cdot \frac{\Gamma_2(1 + \Gamma_1)}{\Gamma_1 - \Gamma_2} \\
    &y = (a + x)\Gamma_2 - b = (a + x)\Gamma_2 - a\Gamma_1 = a(\Gamma_2 - \Gamma_1) + x\Gamma_2 = -x\Gamma_2(1 + \Gamma_1) + x\Gamma_2 = -x\Gamma_2\Gamma_1 = 96\,\text{см}, \\
    &\text{знаки разные, т.е.
    экран надо было подвинуть в другую сторону чем предмет: $x < 0, y > 0$.}
    \end{align*}
}
\solutionspace{120pt}

\tasknumber{10}%
\task{%
    Тонкая собирающая линза дает изображение предмета на экране при двух положениях линзы между предметом и экраном.
    Высота изображения при первом положении $30\,\text{см}$, во втором — $5\,\text{см}$.
    Расстояние между предметом и экранов постоянно.
    Чему равна высота предмета?
}
\answer{%
    \begin{align*}
    &\frac 1a + \frac 1b = \frac 1F, \frac 1c + \frac 1d = \frac 1F, a + b = c + d \implies \frac{a + b}{ab} = \frac 1F = \frac{c+d}{cd} \implies ab = cd, \\
    &\implies ab = c(a + b - c) \implies c^2 - ac - bc + ab = 0 \implies c = a \text{ или } c = b \implies c = b \implies d = a.
    \\
    &\Gamma_1 = \frac {H_1}H = \frac ba, \Gamma_2 = \frac {H_2}H = \frac dc = \frac ab \implies \frac {H_1}H \cdot \frac {H_2}H = \frac ba \cdot \frac ab = 1, \\
    &H = \sqrt{H_1 H_2} \approx 12{,}2\,\text{см}.
    \end{align*}
}
\solutionspace{120pt}

\tasknumber{11}%
\task{%
    Какие предметы можно рассмотреть на фотографии, сделанной со спутника,
    если разрешающая способность плёнки $0{,}010\,\text{мм}$? Каким должно быть
    время экспозиции $\tau$ чтобы полностью использовать возможности плёнки?
    Фокусное расстояние объектива используемого фотоаппарата $15\,\text{см}$,
    высота орбиты спутника $150\,\text{км}$.
}
\answer{%
    \begin{align*}
    &H \ll R \implies v = v_{\text{I}} = \sqrt{G R} \approx 7{,}9\,\frac{\text{км}}{\text{с}}.
    \\
    &F \ll H \implies b = F, a = H, \\
    &\Gamma = \frac \delta\ell = \frac ba \implies \ell = \frac{\delta a}b = \frac{\delta H}F \approx \frac{0{,}010\,\text{мм} \cdot 150\,\text{км}}{15\,\text{см}} \approx 10\,\text{м}, \\
    &\implies \tau = \frac \ell v = \frac{\delta H}{F v} = \frac{0{,}010\,\text{мм} \cdot 150\,\text{км}}{15\,\text{см} \cdot 7{,}9\,\frac{\text{км}}{\text{с}}} \approx 1{,}3\,\text{мс}.
    \end{align*}
}


\variantsplitter


\addpersonalvariant{Тимофей Полетаев}

\tasknumber{12}%
\task{%
    При аэрофотосъемках используется фотоаппарат, объектив которого
    имеет фокусиое расстояние $15\,\text{см}$.
    Разрешающая способность плёнки $0{,}010\,\text{мм}$.
    На какой высоте должен лететь самолет, чтобы на фотографии можно
    было различить следы размером $30\,\text{см}$?
    При какой скорости самолета изображение не будет размытым,
    если время экспозиции $2\,\text{мс}$?
}
\answer{%
    \begin{align*}
    &F \ll H \implies b = F, a = H, \\
    &\Gamma = \frac \delta\ell = \frac ba = \frac FH \implies H = \frac{\ell F}\delta = \frac{30\,\text{см} \cdot 15\,\text{см}}{0{,}010\,\text{мм}} \approx 5\,\text{км}, \\
    &\implies v = \frac l\tau = \frac{30\,\text{см}}{2\,\text{мс}} \approx 540\,\frac{\text{км}}{\text{ч}}.
    \end{align*}
}
\solutionspace{120pt}

\tasknumber{13}%
\task{%
    Две одинаковые собиращие линзы установлены так, что их главные оптические оси совпадают,
    а главный фокус первой находится там же, где главный фокус второй.
    Расстояние от первой линзы до предмета равно $10\,\text{см}$.
    Чему равно расстояние от изображения объекта во второй линзе до второй линзы?
    Определите также увеличение.
    Фокусное расстояние каждой линзы $20\,\text{см}$.
}
\answer{%
    \begin{align*}
    \frac 1a + \frac 1b &= \frac 1F \implies b = \frac{aF}{a - F} \implies 2F - b = \frac{2aF - 2F^2 - aF}{a - F} = \frac{F(a - 2F)}{a - F}.
    \\
    \frac 1{2F - b} + \frac 1c &= \frac 1F \implies c = \frac{F(2F-b)}{(2F - b) - F} = \frac{F \cdot \frac{F(a - 2F)}{a - F}}{\frac{F(a - 2F)}{a - F} - F}  = F \cdot \frac{ \frac{F(a - 2F)}{a - F} }{ \frac{F(a - 2F)}{a - F} - 1} = \\
     &= F \cdot \frac{a - 2F}{a - 2F - a + F} = 2F - a = 30\,\text{см}.
     \\
    \ell &= a + 2F + c = 4F = 80\,\text{см}.
    \\
    &\Gamma = \Gamma_1 \cdot \Gamma_2 = \frac ba \cdot \frac c{2F-b} = \frac F{a - F} \cdot \frac{2F - a}{\frac{F(a - 2F)}{a - F}} = -1.
    \end{align*}
}
\solutionspace{120pt}

\tasknumber{14}%
\task{%
    Собирающая линза с фокусным расстоянием $F_1 > 0$ и рассеивающая линза с фокусным расстоянием $F_2 < 0$
    установлены коаксиально на расстоянии $\ell$.
    Пучок параллельных лучей падает на рассеивающую линзу.
    Сделайте схематичное построение и определите, в какой точке система из этих линз соберёт пучок.
}
\answer{%
    \begin{align*}
    &\text{Если пучок падает на собирающую линзу:} \\
    \frac 1{\infty} + \frac 1b &= \frac 1{F_1} \implies b = F_1 \implies \ell - b = \ell - F_1 \\
    \frac 1{\ell - b} + \frac 1c &= \frac 1{F_2} \implies c = \frac{F_2(\ell - b)}{\ell - b - F_2} = \frac{F_2(\ell - F_1)}{\ell - F_1 - F_2}.
    \\
    &\text{Если же пучок падает на рассеивающую линзу:} \\
    \frac 1{\infty} + \frac 1b &= \frac 1{F_2} \implies b = F_2 \implies \ell - b = \ell - F_2 \\
    \frac 1{\ell - b} + \frac 1c &= \frac 1{F_1} \implies c = \frac{F_1(\ell - b)}{\ell - b - F_1} = \frac{F_1(\ell - F_2)}{\ell - F_2 - F_1}.
    \end{align*}
}
\solutionspace{120pt}

\tasknumber{15}%
\task{%
    Две собирающих линзы с фокусными расстояниями $40\,\text{см}$ и $25\,\text{см}$ расположены так,
    что их оптические оси совмещены.
    На первую линзу падает пучок параллельных лучей.
    Пройдя через вторую линзу, он остался параллельным.
    Найдите расстояние между линзами и сделайте рисунок.
}
\answer{%
    \begin{align*}
    \frac 1\infty + \frac 1b &= \frac 1{F_1} \implies b = F_1, \\
    \frac 1{\ell - b} + \frac 1{\infty} &= \frac 1{F_2} \implies \ell - b = F_2 \implies \ell = b + F_2 = F_1 + F_2 = 65\,\text{см}.
    \end{align*}
}

\variantsplitter

\addpersonalvariant{Андрей Рожков}

\tasknumber{1}%
\task{%
    Найти оптическую силу собирающей линзы, если действительное изображение предмета,
    помещённого в $15\,\text{см}$ от линзы, получается на расстоянии $20\,\text{см}$ от неё.
}
\answer{%
    $D = \frac 1F = \frac 1a + \frac 1b = \frac 1{15\,\text{см}} + \frac 1{20\,\text{см}} \approx 11{,}67\,\text{дптр}$
}
\solutionspace{80pt}

\tasknumber{2}%
\task{%
    Найти увеличение изображения, если изображение предмета, находящегося
    на расстоянии $15\,\text{см}$ от линзы, получается на расстоянии $30\,\text{см}$ от неё.
}
\answer{%
    $\Gamma = \frac ba = \frac {30\,\text{см}}{15\,\text{см}} \approx 2$
}
\solutionspace{80pt}

\tasknumber{3}%
\task{%
    Расстояние от предмета до линзы $8\,\text{см}$, а от линзы до мнимого изображения $30\,\text{см}$.
    Чему равно фокусное расстояние линзы?
}
\answer{%
    $\pm \frac 1F = \frac 1a - \frac 1b \implies F = \frac{a b}{\abs{b - a}} \approx 10{,}9\,\text{см}$
}
\solutionspace{80pt}

\tasknumber{4}%
\task{%
    Две тонкие собирающие линзы с фокусными расстояниями $12\,\text{см}$ и $30\,\text{см}$ сложены вместе.
    Чему равно фокусное расстояние такой оптической системы?
}
\answer{%
    $\frac 1{f_1} = \frac 1a + \frac 1b; \frac 1{f_2} = - \frac 1b + \frac 1c \implies \frac 1{f_1} + \frac 1{f_2} = \frac 1a + \frac 1c \implies f' = \frac 1{\frac 1{f_1} + \frac 1{f_2}} = \frac{f_1 f_2}{f_1 + f_2} \approx 8{,}6\,\text{см}$
}
\solutionspace{80pt}

\tasknumber{5}%
\task{%
    Линейные размеры прямого изображения предмета, полученного в собирающей линзе,
    в четыре раза больше линейных размеров предмета.
    Зная, что предмет находится на $35\,\text{см}$ ближе к линзе,
    чем его изображение, найти оптическую силу линзы.
}
\answer{%
    \begin{align*}
    &\text{Если изображение действительное:} \\
    D &= \frac 1F = \frac 1a + \frac 1b, \qquad \Gamma = \frac ba, \qquad b - a = \ell \implies b = \Gamma a \implies \Gamma a - a = \ell \implies  \\
    a &= \frac {\ell}{\Gamma - 1} \implies b = \frac {{\ell} \Gamma}{\Gamma - 1} \implies  \\
    D &= \frac {\Gamma - 1}\ell + \frac {\Gamma - 1}{\ell \Gamma} = \frac 1\ell \cdot \cbr{\Gamma - 1 + \frac {\Gamma - 1}{\Gamma} } =\frac 1\ell \cdot \cbr{\Gamma - \frac 1\Gamma} \approx 10{,}7\,\text{дптр}.
    \\
    &\text{Если изображение мнимое:} \\
    D &= \frac 1F = \frac 1a - \frac 1b, \qquad \Gamma = \frac ba, \qquad b - a = \ell \implies b = \Gamma a \implies \Gamma a - a = \ell \implies  \\
    a &= \frac {\ell}{\Gamma - 1} \implies b = \frac {{\ell} \Gamma}{\Gamma - 1} \implies  \\
    D &= \frac {\Gamma - 1}\ell - \frac {\Gamma - 1}{\ell \Gamma} = \frac 1\ell \cdot \cbr{\Gamma - 1 - \frac {\Gamma - 1}{\Gamma} } =\frac 1\ell \cdot \cbr{\Gamma + \frac 1\Gamma - 2} \approx 6{,}4\,\text{дптр}.
    \\
    &\text{В ответе надо указать оба значения.}
    \end{align*}
}
\solutionspace{120pt}

\tasknumber{6}%
\task{%
    Оптическая сила объектива фотоаппарата равна $6\,\text{дптр}$.
    При фотографировании чертежа с расстояния $0{,}8\,\text{м}$ площадь изображения
    чертежа на фотопластинке оказалась равной $9\,\text{см}^{2}$.
    Какова площадь самого чертежа? Ответ выразите в квадратных сантиметрах.
}
\answer{%
    \begin{align*}
    &\frac 1a + \frac 1b = \frac 1F = D \implies b = \frac{aF}{a - F} \\
    &\frac {S'}S = \Gamma^2 = \sqr{\frac ba} = \sqr{\frac F{a - F}} \implies \\
    &\implies S = S' \cdot \sqr{\frac{a - F}F} = S' \cdot \sqr{\frac aF - 1} = S' \cdot \sqr{aD - 1} \approx 130\,\text{см}^{2}.
    \end{align*}
}


\variantsplitter


\addpersonalvariant{Андрей Рожков}

\tasknumber{7}%
\task{%
    В каком месте на главной оптической оси двояковыпуклой линзы
    нужно поместить точечный источник света,
    чтобы его изображение оказалось в главном фокусе линзы?
}
\answer{%
    $\text{для мнимого - на половине фокусного, для действительного - на бесконечности}$
}
\solutionspace{120pt}

\tasknumber{8}%
\task{%
    Предмет в виде отрезка длиной $\ell$ расположен вдоль оптической оси
    собирающей линзы с фокусным расстоянием $F$.
    Середина отрезка расположена
    на расстоянии $a$ от линзы, которая даёт действительное изображение
    всех точек предмета.
    Определить продольное увеличение предмета.
}
\answer{%
    \begin{align*}
    \frac 1{a + \frac \ell 2} &+ \frac 1b = \frac 1F \implies b = \frac{F\cbr{a + \frac \ell 2}}{a + \frac \ell 2 - F} \\
    \frac 1{a - \frac \ell 2} &+ \frac 1c = \frac 1F \implies c = \frac{F\cbr{a - \frac \ell 2}}{a - \frac \ell 2 - F} \\
    \abs{b - c} &= \abs{\frac{F\cbr{a + \frac \ell 2}}{a + \frac \ell 2 - F} - \frac{F\cbr{a - \frac \ell 2}}{a - \frac \ell 2 - F}}= F\abs{\frac{\cbr{a + \frac \ell 2}\cbr{a - \frac \ell 2 - F} - \cbr{a - \frac \ell 2}\cbr{a + \frac \ell 2 - F}}{ \cbr{a + \frac \ell 2 - F} \cbr{a - \frac \ell 2 - F} }} =  \\
    &= F\abs{\frac{a^2 - \frac {a\ell} 2 - Fa + \frac {a\ell} 2 - \frac {\ell^2} 4 - \frac {F\ell}2 - a^2 - \frac {a\ell}2 + aF + \frac {a\ell}2 + \frac {\ell^2} 4 - \frac {F\ell} 2}{\cbr{a + \frac \ell 2 - F} \cbr{a - \frac \ell 2 - F} }} = \\
    &= F\frac{F\ell}{\sqr{a-F} - \frac {\ell^2}4} = \frac{F^2\ell}{\sqr{a-F} - \frac {\ell^2}4}\implies \Gamma = \frac{\abs{b - c}}\ell = \frac{F^2}{\sqr{a-F} - \frac {\ell^2}4}.
    \end{align*}
}
\solutionspace{120pt}

\tasknumber{9}%
\task{%
    На экране с помощью тонкой линзы получено изображение предмета
    с увеличением $2$.
    Предмет передвинули на $4\,\text{см}$.
    Для того, чтобы получить резкое изображение, пришлось передвинуть экран.
    При этом увеличение оказалось равным $8$.
    На какое расстояние
    пришлось передвинуть экран?
}
\answer{%
    \begin{align*}
    &\frac 1a + \frac 1b = \frac 1F, \Gamma_1 = \frac ba = \frac{F}{a-F} \implies \Gamma_1(a-F) = F \implies a = F \cdot \frac{1 + \Gamma_1}{\Gamma_1} \\
    &\frac 1{a + x} + \frac 1{b + y} = \frac 1F, \Gamma_2 = \frac {b+y}{a+x} = \frac{F}{a+x-F} \implies a + x = F \cdot \frac{1 + \Gamma_2}{\Gamma_2} \\
    &1 + \frac xa = \frac{ \frac{1 + \Gamma_2}{\Gamma_2} }{ \frac{1 + \Gamma_1}{\Gamma_1} } = \frac{\Gamma_1(1 + \Gamma_2)}{\Gamma_2(1 + \Gamma_1)} \\
    &a = \frac x{ \frac{\Gamma_1(1 + \Gamma_2)}{\Gamma_2(1 + \Gamma_1)} - 1} = x \cdot \frac{\Gamma_2(1 + \Gamma_1)}{\Gamma_1 - \Gamma_2} \\
    &y = (a + x)\Gamma_2 - b = (a + x)\Gamma_2 - a\Gamma_1 = a(\Gamma_2 - \Gamma_1) + x\Gamma_2 = -x\Gamma_2(1 + \Gamma_1) + x\Gamma_2 = -x\Gamma_2\Gamma_1 = 64\,\text{см}, \\
    &\text{знаки разные, т.е.
    экран надо было подвинуть в другую сторону чем предмет: $x < 0, y > 0$.}
    \end{align*}
}
\solutionspace{120pt}

\tasknumber{10}%
\task{%
    Тонкая собирающая линза дает изображение предмета на экране при двух положениях линзы между предметом и экраном.
    Высота изображения при первом положении $20\,\text{см}$, во втором — $7\,\text{см}$.
    Расстояние между предметом и экранов постоянно.
    Чему равна высота предмета?
}
\answer{%
    \begin{align*}
    &\frac 1a + \frac 1b = \frac 1F, \frac 1c + \frac 1d = \frac 1F, a + b = c + d \implies \frac{a + b}{ab} = \frac 1F = \frac{c+d}{cd} \implies ab = cd, \\
    &\implies ab = c(a + b - c) \implies c^2 - ac - bc + ab = 0 \implies c = a \text{ или } c = b \implies c = b \implies d = a.
    \\
    &\Gamma_1 = \frac {H_1}H = \frac ba, \Gamma_2 = \frac {H_2}H = \frac dc = \frac ab \implies \frac {H_1}H \cdot \frac {H_2}H = \frac ba \cdot \frac ab = 1, \\
    &H = \sqrt{H_1 H_2} \approx 11{,}8\,\text{см}.
    \end{align*}
}
\solutionspace{120pt}

\tasknumber{11}%
\task{%
    Какие предметы можно рассмотреть на фотографии, сделанной со спутника,
    если разрешающая способность плёнки $0{,}010\,\text{мм}$? Каким должно быть
    время экспозиции $\tau$ чтобы полностью использовать возможности плёнки?
    Фокусное расстояние объектива используемого фотоаппарата $10\,\text{см}$,
    высота орбиты спутника $80\,\text{км}$.
}
\answer{%
    \begin{align*}
    &H \ll R \implies v = v_{\text{I}} = \sqrt{G R} \approx 7{,}9\,\frac{\text{км}}{\text{с}}.
    \\
    &F \ll H \implies b = F, a = H, \\
    &\Gamma = \frac \delta\ell = \frac ba \implies \ell = \frac{\delta a}b = \frac{\delta H}F \approx \frac{0{,}010\,\text{мм} \cdot 80\,\text{км}}{10\,\text{см}} \approx 8\,\text{м}, \\
    &\implies \tau = \frac \ell v = \frac{\delta H}{F v} = \frac{0{,}010\,\text{мм} \cdot 80\,\text{км}}{10\,\text{см} \cdot 7{,}9\,\frac{\text{км}}{\text{с}}} \approx 1{,}0\,\text{мс}.
    \end{align*}
}


\variantsplitter


\addpersonalvariant{Андрей Рожков}

\tasknumber{12}%
\task{%
    При аэрофотосъемках используется фотоаппарат, объектив которого
    имеет фокусиое расстояние $15\,\text{см}$.
    Разрешающая способность плёнки $0{,}02\,\text{мм}$.
    На какой высоте должен лететь самолет, чтобы на фотографии можно
    было различить следы размером $15\,\text{см}$?
    При какой скорости самолета изображение не будет размытым,
    если время экспозиции $2\,\text{мс}$?
}
\answer{%
    \begin{align*}
    &F \ll H \implies b = F, a = H, \\
    &\Gamma = \frac \delta\ell = \frac ba = \frac FH \implies H = \frac{\ell F}\delta = \frac{15\,\text{см} \cdot 15\,\text{см}}{0{,}02\,\text{мм}} \approx 1{,}1\,\text{км}, \\
    &\implies v = \frac l\tau = \frac{15\,\text{см}}{2\,\text{мс}} \approx 270\,\frac{\text{км}}{\text{ч}}.
    \end{align*}
}
\solutionspace{120pt}

\tasknumber{13}%
\task{%
    Две одинаковые собиращие линзы установлены так, что их главные оптические оси совпадают,
    а главный фокус первой находится там же, где главный фокус второй.
    Расстояние от первой линзы до предмета равно $21\,\text{см}$.
    Чему равно расстояние от изображения объекта во второй линзе до самого объекта?
    Определите также увеличение.
    Фокусное расстояние каждой линзы $20\,\text{см}$.
}
\answer{%
    \begin{align*}
    \frac 1a + \frac 1b &= \frac 1F \implies b = \frac{aF}{a - F} \implies 2F - b = \frac{2aF - 2F^2 - aF}{a - F} = \frac{F(a - 2F)}{a - F}.
    \\
    \frac 1{2F - b} + \frac 1c &= \frac 1F \implies c = \frac{F(2F-b)}{(2F - b) - F} = \frac{F \cdot \frac{F(a - 2F)}{a - F}}{\frac{F(a - 2F)}{a - F} - F}  = F \cdot \frac{ \frac{F(a - 2F)}{a - F} }{ \frac{F(a - 2F)}{a - F} - 1} = \\
     &= F \cdot \frac{a - 2F}{a - 2F - a + F} = 2F - a = 19\,\text{см}.
     \\
    \ell &= a + 2F + c = 4F = 80\,\text{см}.
    \\
    &\Gamma = \Gamma_1 \cdot \Gamma_2 = \frac ba \cdot \frac c{2F-b} = \frac F{a - F} \cdot \frac{2F - a}{\frac{F(a - 2F)}{a - F}} = -1.
    \end{align*}
}
\solutionspace{120pt}

\tasknumber{14}%
\task{%
    Собирающая линза с фокусным расстоянием $F_1 > 0$ и рассеивающая линза с фокусным расстоянием $F_2 < 0$
    установлены коаксиально на расстоянии $\ell$.
    Пучок параллельных лучей падает на собирающую линзу.
    Сделайте схематичное построение и определите, в какой точке система из этих линз соберёт пучок.
}
\answer{%
    \begin{align*}
    &\text{Если пучок падает на собирающую линзу:} \\
    \frac 1{\infty} + \frac 1b &= \frac 1{F_1} \implies b = F_1 \implies \ell - b = \ell - F_1 \\
    \frac 1{\ell - b} + \frac 1c &= \frac 1{F_2} \implies c = \frac{F_2(\ell - b)}{\ell - b - F_2} = \frac{F_2(\ell - F_1)}{\ell - F_1 - F_2}.
    \\
    &\text{Если же пучок падает на рассеивающую линзу:} \\
    \frac 1{\infty} + \frac 1b &= \frac 1{F_2} \implies b = F_2 \implies \ell - b = \ell - F_2 \\
    \frac 1{\ell - b} + \frac 1c &= \frac 1{F_1} \implies c = \frac{F_1(\ell - b)}{\ell - b - F_1} = \frac{F_1(\ell - F_2)}{\ell - F_2 - F_1}.
    \end{align*}
}
\solutionspace{120pt}

\tasknumber{15}%
\task{%
    Две собирающих линзы с фокусными расстояниями $30\,\text{см}$ и $35\,\text{см}$ расположены так,
    что их оптические оси совмещены.
    На первую линзу падает пучок параллельных лучей.
    Пройдя через вторую линзу, он остался параллельным.
    Найдите расстояние между линзами и сделайте рисунок.
}
\answer{%
    \begin{align*}
    \frac 1\infty + \frac 1b &= \frac 1{F_1} \implies b = F_1, \\
    \frac 1{\ell - b} + \frac 1{\infty} &= \frac 1{F_2} \implies \ell - b = F_2 \implies \ell = b + F_2 = F_1 + F_2 = 65\,\text{см}.
    \end{align*}
}

\variantsplitter

\addpersonalvariant{Рената Таржиманова}

\tasknumber{1}%
\task{%
    Найти оптическую силу собирающей линзы, если действительное изображение предмета,
    помещённого в $35\,\text{см}$ от линзы, получается на расстоянии $20\,\text{см}$ от неё.
}
\answer{%
    $D = \frac 1F = \frac 1a + \frac 1b = \frac 1{35\,\text{см}} + \frac 1{20\,\text{см}} \approx 7{,}86\,\text{дптр}$
}
\solutionspace{80pt}

\tasknumber{2}%
\task{%
    Найти увеличение изображения, если изображение предмета, находящегося
    на расстоянии $25\,\text{см}$ от линзы, получается на расстоянии $12\,\text{см}$ от неё.
}
\answer{%
    $\Gamma = \frac ba = \frac {12\,\text{см}}{25\,\text{см}} \approx 0{,}5$
}
\solutionspace{80pt}

\tasknumber{3}%
\task{%
    Расстояние от предмета до линзы $12\,\text{см}$, а от линзы до мнимого изображения $20\,\text{см}$.
    Чему равно фокусное расстояние линзы?
}
\answer{%
    $\pm \frac 1F = \frac 1a - \frac 1b \implies F = \frac{a b}{\abs{b - a}} \approx 30\,\text{см}$
}
\solutionspace{80pt}

\tasknumber{4}%
\task{%
    Две тонкие собирающие линзы с фокусными расстояниями $12\,\text{см}$ и $20\,\text{см}$ сложены вместе.
    Чему равно фокусное расстояние такой оптической системы?
}
\answer{%
    $\frac 1{f_1} = \frac 1a + \frac 1b; \frac 1{f_2} = - \frac 1b + \frac 1c \implies \frac 1{f_1} + \frac 1{f_2} = \frac 1a + \frac 1c \implies f' = \frac 1{\frac 1{f_1} + \frac 1{f_2}} = \frac{f_1 f_2}{f_1 + f_2} \approx 7{,}5\,\text{см}$
}
\solutionspace{80pt}

\tasknumber{5}%
\task{%
    Линейные размеры прямого изображения предмета, полученного в собирающей линзе,
    в два раза больше линейных размеров предмета.
    Зная, что предмет находится на $30\,\text{см}$ ближе к линзе,
    чем его изображение, найти оптическую силу линзы.
}
\answer{%
    \begin{align*}
    &\text{Если изображение действительное:} \\
    D &= \frac 1F = \frac 1a + \frac 1b, \qquad \Gamma = \frac ba, \qquad b - a = \ell \implies b = \Gamma a \implies \Gamma a - a = \ell \implies  \\
    a &= \frac {\ell}{\Gamma - 1} \implies b = \frac {{\ell} \Gamma}{\Gamma - 1} \implies  \\
    D &= \frac {\Gamma - 1}\ell + \frac {\Gamma - 1}{\ell \Gamma} = \frac 1\ell \cdot \cbr{\Gamma - 1 + \frac {\Gamma - 1}{\Gamma} } =\frac 1\ell \cdot \cbr{\Gamma - \frac 1\Gamma} \approx 5\,\text{дптр}.
    \\
    &\text{Если изображение мнимое:} \\
    D &= \frac 1F = \frac 1a - \frac 1b, \qquad \Gamma = \frac ba, \qquad b - a = \ell \implies b = \Gamma a \implies \Gamma a - a = \ell \implies  \\
    a &= \frac {\ell}{\Gamma - 1} \implies b = \frac {{\ell} \Gamma}{\Gamma - 1} \implies  \\
    D &= \frac {\Gamma - 1}\ell - \frac {\Gamma - 1}{\ell \Gamma} = \frac 1\ell \cdot \cbr{\Gamma - 1 - \frac {\Gamma - 1}{\Gamma} } =\frac 1\ell \cdot \cbr{\Gamma + \frac 1\Gamma - 2} \approx 1{,}7\,\text{дптр}.
    \\
    &\text{В ответе надо указать оба значения.}
    \end{align*}
}
\solutionspace{120pt}

\tasknumber{6}%
\task{%
    Оптическая сила объектива фотоаппарата равна $3\,\text{дптр}$.
    При фотографировании чертежа с расстояния $1{,}1\,\text{м}$ площадь изображения
    чертежа на фотопластинке оказалась равной $9\,\text{см}^{2}$.
    Какова площадь самого чертежа? Ответ выразите в квадратных сантиметрах.
}
\answer{%
    \begin{align*}
    &\frac 1a + \frac 1b = \frac 1F = D \implies b = \frac{aF}{a - F} \\
    &\frac {S'}S = \Gamma^2 = \sqr{\frac ba} = \sqr{\frac F{a - F}} \implies \\
    &\implies S = S' \cdot \sqr{\frac{a - F}F} = S' \cdot \sqr{\frac aF - 1} = S' \cdot \sqr{aD - 1} \approx 50\,\text{см}^{2}.
    \end{align*}
}


\variantsplitter


\addpersonalvariant{Рената Таржиманова}

\tasknumber{7}%
\task{%
    В каком месте на главной оптической оси двояковыпуклой линзы
    нужно поместить точечный источник света,
    чтобы его изображение оказалось в главном фокусе линзы?
}
\answer{%
    $\text{для мнимого - на половине фокусного, для действительного - на бесконечности}$
}
\solutionspace{120pt}

\tasknumber{8}%
\task{%
    Предмет в виде отрезка длиной $\ell$ расположен вдоль оптической оси
    собирающей линзы с фокусным расстоянием $F$.
    Середина отрезка расположена
    на расстоянии $a$ от линзы, которая даёт действительное изображение
    всех точек предмета.
    Определить продольное увеличение предмета.
}
\answer{%
    \begin{align*}
    \frac 1{a + \frac \ell 2} &+ \frac 1b = \frac 1F \implies b = \frac{F\cbr{a + \frac \ell 2}}{a + \frac \ell 2 - F} \\
    \frac 1{a - \frac \ell 2} &+ \frac 1c = \frac 1F \implies c = \frac{F\cbr{a - \frac \ell 2}}{a - \frac \ell 2 - F} \\
    \abs{b - c} &= \abs{\frac{F\cbr{a + \frac \ell 2}}{a + \frac \ell 2 - F} - \frac{F\cbr{a - \frac \ell 2}}{a - \frac \ell 2 - F}}= F\abs{\frac{\cbr{a + \frac \ell 2}\cbr{a - \frac \ell 2 - F} - \cbr{a - \frac \ell 2}\cbr{a + \frac \ell 2 - F}}{ \cbr{a + \frac \ell 2 - F} \cbr{a - \frac \ell 2 - F} }} =  \\
    &= F\abs{\frac{a^2 - \frac {a\ell} 2 - Fa + \frac {a\ell} 2 - \frac {\ell^2} 4 - \frac {F\ell}2 - a^2 - \frac {a\ell}2 + aF + \frac {a\ell}2 + \frac {\ell^2} 4 - \frac {F\ell} 2}{\cbr{a + \frac \ell 2 - F} \cbr{a - \frac \ell 2 - F} }} = \\
    &= F\frac{F\ell}{\sqr{a-F} - \frac {\ell^2}4} = \frac{F^2\ell}{\sqr{a-F} - \frac {\ell^2}4}\implies \Gamma = \frac{\abs{b - c}}\ell = \frac{F^2}{\sqr{a-F} - \frac {\ell^2}4}.
    \end{align*}
}
\solutionspace{120pt}

\tasknumber{9}%
\task{%
    На экране с помощью тонкой линзы получено изображение предмета
    с увеличением $4$.
    Предмет передвинули на $6\,\text{см}$.
    Для того, чтобы получить резкое изображение, пришлось передвинуть экран.
    При этом увеличение оказалось равным $8$.
    На какое расстояние
    пришлось передвинуть экран?
}
\answer{%
    \begin{align*}
    &\frac 1a + \frac 1b = \frac 1F, \Gamma_1 = \frac ba = \frac{F}{a-F} \implies \Gamma_1(a-F) = F \implies a = F \cdot \frac{1 + \Gamma_1}{\Gamma_1} \\
    &\frac 1{a + x} + \frac 1{b + y} = \frac 1F, \Gamma_2 = \frac {b+y}{a+x} = \frac{F}{a+x-F} \implies a + x = F \cdot \frac{1 + \Gamma_2}{\Gamma_2} \\
    &1 + \frac xa = \frac{ \frac{1 + \Gamma_2}{\Gamma_2} }{ \frac{1 + \Gamma_1}{\Gamma_1} } = \frac{\Gamma_1(1 + \Gamma_2)}{\Gamma_2(1 + \Gamma_1)} \\
    &a = \frac x{ \frac{\Gamma_1(1 + \Gamma_2)}{\Gamma_2(1 + \Gamma_1)} - 1} = x \cdot \frac{\Gamma_2(1 + \Gamma_1)}{\Gamma_1 - \Gamma_2} \\
    &y = (a + x)\Gamma_2 - b = (a + x)\Gamma_2 - a\Gamma_1 = a(\Gamma_2 - \Gamma_1) + x\Gamma_2 = -x\Gamma_2(1 + \Gamma_1) + x\Gamma_2 = -x\Gamma_2\Gamma_1 = 192\,\text{см}, \\
    &\text{знаки разные, т.е.
    экран надо было подвинуть в другую сторону чем предмет: $x < 0, y > 0$.}
    \end{align*}
}
\solutionspace{120pt}

\tasknumber{10}%
\task{%
    Тонкая собирающая линза дает изображение предмета на экране при двух положениях линзы между предметом и экраном.
    Высота изображения при первом положении $25\,\text{см}$, во втором — $5\,\text{см}$.
    Расстояние между предметом и экранов постоянно.
    Чему равна высота предмета?
}
\answer{%
    \begin{align*}
    &\frac 1a + \frac 1b = \frac 1F, \frac 1c + \frac 1d = \frac 1F, a + b = c + d \implies \frac{a + b}{ab} = \frac 1F = \frac{c+d}{cd} \implies ab = cd, \\
    &\implies ab = c(a + b - c) \implies c^2 - ac - bc + ab = 0 \implies c = a \text{ или } c = b \implies c = b \implies d = a.
    \\
    &\Gamma_1 = \frac {H_1}H = \frac ba, \Gamma_2 = \frac {H_2}H = \frac dc = \frac ab \implies \frac {H_1}H \cdot \frac {H_2}H = \frac ba \cdot \frac ab = 1, \\
    &H = \sqrt{H_1 H_2} \approx 11{,}2\,\text{см}.
    \end{align*}
}
\solutionspace{120pt}

\tasknumber{11}%
\task{%
    Какие предметы можно рассмотреть на фотографии, сделанной со спутника,
    если разрешающая способность плёнки $0{,}02\,\text{мм}$? Каким должно быть
    время экспозиции $\tau$ чтобы полностью использовать возможности плёнки?
    Фокусное расстояние объектива используемого фотоаппарата $15\,\text{см}$,
    высота орбиты спутника $150\,\text{км}$.
}
\answer{%
    \begin{align*}
    &H \ll R \implies v = v_{\text{I}} = \sqrt{G R} \approx 7{,}9\,\frac{\text{км}}{\text{с}}.
    \\
    &F \ll H \implies b = F, a = H, \\
    &\Gamma = \frac \delta\ell = \frac ba \implies \ell = \frac{\delta a}b = \frac{\delta H}F \approx \frac{0{,}02\,\text{мм} \cdot 150\,\text{км}}{15\,\text{см}} \approx 20\,\text{м}, \\
    &\implies \tau = \frac \ell v = \frac{\delta H}{F v} = \frac{0{,}02\,\text{мм} \cdot 150\,\text{км}}{15\,\text{см} \cdot 7{,}9\,\frac{\text{км}}{\text{с}}} \approx 3\,\text{мс}.
    \end{align*}
}


\variantsplitter


\addpersonalvariant{Рената Таржиманова}

\tasknumber{12}%
\task{%
    При аэрофотосъемках используется фотоаппарат, объектив которого
    имеет фокусиое расстояние $20\,\text{см}$.
    Разрешающая способность плёнки $0{,}010\,\text{мм}$.
    На какой высоте должен лететь самолет, чтобы на фотографии можно
    было различить следы размером $30\,\text{см}$?
    При какой скорости самолета изображение не будет размытым,
    если время экспозиции $2\,\text{мс}$?
}
\answer{%
    \begin{align*}
    &F \ll H \implies b = F, a = H, \\
    &\Gamma = \frac \delta\ell = \frac ba = \frac FH \implies H = \frac{\ell F}\delta = \frac{30\,\text{см} \cdot 20\,\text{см}}{0{,}010\,\text{мм}} \approx 6\,\text{км}, \\
    &\implies v = \frac l\tau = \frac{30\,\text{см}}{2\,\text{мс}} \approx 540\,\frac{\text{км}}{\text{ч}}.
    \end{align*}
}
\solutionspace{120pt}

\tasknumber{13}%
\task{%
    Две одинаковые собиращие линзы установлены так, что их главные оптические оси совпадают,
    а главный фокус первой находится там же, где главный фокус второй.
    Расстояние от первой линзы до предмета равно $17\,\text{см}$.
    Чему равно расстояние от изображения объекта во второй линзе до второй линзы?
    Определите также увеличение.
    Фокусное расстояние каждой линзы $40\,\text{см}$.
}
\answer{%
    \begin{align*}
    \frac 1a + \frac 1b &= \frac 1F \implies b = \frac{aF}{a - F} \implies 2F - b = \frac{2aF - 2F^2 - aF}{a - F} = \frac{F(a - 2F)}{a - F}.
    \\
    \frac 1{2F - b} + \frac 1c &= \frac 1F \implies c = \frac{F(2F-b)}{(2F - b) - F} = \frac{F \cdot \frac{F(a - 2F)}{a - F}}{\frac{F(a - 2F)}{a - F} - F}  = F \cdot \frac{ \frac{F(a - 2F)}{a - F} }{ \frac{F(a - 2F)}{a - F} - 1} = \\
     &= F \cdot \frac{a - 2F}{a - 2F - a + F} = 2F - a = 63\,\text{см}.
     \\
    \ell &= a + 2F + c = 4F = 160\,\text{см}.
    \\
    &\Gamma = \Gamma_1 \cdot \Gamma_2 = \frac ba \cdot \frac c{2F-b} = \frac F{a - F} \cdot \frac{2F - a}{\frac{F(a - 2F)}{a - F}} = -1.
    \end{align*}
}
\solutionspace{120pt}

\tasknumber{14}%
\task{%
    Собирающая линза с фокусным расстоянием $F_1 > 0$ и рассеивающая линза с фокусным расстоянием $F_2 < 0$
    установлены коаксиально на расстоянии $\ell$.
    Пучок параллельных лучей падает на собирающую линзу.
    Сделайте схематичное построение и определите, в какой точке система из этих линз соберёт пучок.
}
\answer{%
    \begin{align*}
    &\text{Если пучок падает на собирающую линзу:} \\
    \frac 1{\infty} + \frac 1b &= \frac 1{F_1} \implies b = F_1 \implies \ell - b = \ell - F_1 \\
    \frac 1{\ell - b} + \frac 1c &= \frac 1{F_2} \implies c = \frac{F_2(\ell - b)}{\ell - b - F_2} = \frac{F_2(\ell - F_1)}{\ell - F_1 - F_2}.
    \\
    &\text{Если же пучок падает на рассеивающую линзу:} \\
    \frac 1{\infty} + \frac 1b &= \frac 1{F_2} \implies b = F_2 \implies \ell - b = \ell - F_2 \\
    \frac 1{\ell - b} + \frac 1c &= \frac 1{F_1} \implies c = \frac{F_1(\ell - b)}{\ell - b - F_1} = \frac{F_1(\ell - F_2)}{\ell - F_2 - F_1}.
    \end{align*}
}
\solutionspace{120pt}

\tasknumber{15}%
\task{%
    Две собирающих линзы с фокусными расстояниями $50\,\text{см}$ и $45\,\text{см}$ расположены так,
    что их оптические оси совмещены.
    На первую линзу падает пучок параллельных лучей.
    Пройдя через вторую линзу, он остался параллельным.
    Найдите расстояние между линзами и сделайте рисунок.
}
\answer{%
    \begin{align*}
    \frac 1\infty + \frac 1b &= \frac 1{F_1} \implies b = F_1, \\
    \frac 1{\ell - b} + \frac 1{\infty} &= \frac 1{F_2} \implies \ell - b = F_2 \implies \ell = b + F_2 = F_1 + F_2 = 95\,\text{см}.
    \end{align*}
}

\variantsplitter

\addpersonalvariant{Андрей Щербаков}

\tasknumber{1}%
\task{%
    Найти оптическую силу собирающей линзы, если действительное изображение предмета,
    помещённого в $55\,\text{см}$ от линзы, получается на расстоянии $30\,\text{см}$ от неё.
}
\answer{%
    $D = \frac 1F = \frac 1a + \frac 1b = \frac 1{55\,\text{см}} + \frac 1{30\,\text{см}} \approx 5{,}15\,\text{дптр}$
}
\solutionspace{80pt}

\tasknumber{2}%
\task{%
    Найти увеличение изображения, если изображение предмета, находящегося
    на расстоянии $20\,\text{см}$ от линзы, получается на расстоянии $30\,\text{см}$ от неё.
}
\answer{%
    $\Gamma = \frac ba = \frac {30\,\text{см}}{20\,\text{см}} \approx 1{,}50$
}
\solutionspace{80pt}

\tasknumber{3}%
\task{%
    Расстояние от предмета до линзы $10\,\text{см}$, а от линзы до мнимого изображения $30\,\text{см}$.
    Чему равно фокусное расстояние линзы?
}
\answer{%
    $\pm \frac 1F = \frac 1a - \frac 1b \implies F = \frac{a b}{\abs{b - a}} \approx 15\,\text{см}$
}
\solutionspace{80pt}

\tasknumber{4}%
\task{%
    Две тонкие собирающие линзы с фокусными расстояниями $18\,\text{см}$ и $20\,\text{см}$ сложены вместе.
    Чему равно фокусное расстояние такой оптической системы?
}
\answer{%
    $\frac 1{f_1} = \frac 1a + \frac 1b; \frac 1{f_2} = - \frac 1b + \frac 1c \implies \frac 1{f_1} + \frac 1{f_2} = \frac 1a + \frac 1c \implies f' = \frac 1{\frac 1{f_1} + \frac 1{f_2}} = \frac{f_1 f_2}{f_1 + f_2} \approx 9{,}5\,\text{см}$
}
\solutionspace{80pt}

\tasknumber{5}%
\task{%
    Линейные размеры прямого изображения предмета, полученного в собирающей линзе,
    в два раза больше линейных размеров предмета.
    Зная, что предмет находится на $20\,\text{см}$ ближе к линзе,
    чем его изображение, найти оптическую силу линзы.
}
\answer{%
    \begin{align*}
    &\text{Если изображение действительное:} \\
    D &= \frac 1F = \frac 1a + \frac 1b, \qquad \Gamma = \frac ba, \qquad b - a = \ell \implies b = \Gamma a \implies \Gamma a - a = \ell \implies  \\
    a &= \frac {\ell}{\Gamma - 1} \implies b = \frac {{\ell} \Gamma}{\Gamma - 1} \implies  \\
    D &= \frac {\Gamma - 1}\ell + \frac {\Gamma - 1}{\ell \Gamma} = \frac 1\ell \cdot \cbr{\Gamma - 1 + \frac {\Gamma - 1}{\Gamma} } =\frac 1\ell \cdot \cbr{\Gamma - \frac 1\Gamma} \approx 7{,}5\,\text{дптр}.
    \\
    &\text{Если изображение мнимое:} \\
    D &= \frac 1F = \frac 1a - \frac 1b, \qquad \Gamma = \frac ba, \qquad b - a = \ell \implies b = \Gamma a \implies \Gamma a - a = \ell \implies  \\
    a &= \frac {\ell}{\Gamma - 1} \implies b = \frac {{\ell} \Gamma}{\Gamma - 1} \implies  \\
    D &= \frac {\Gamma - 1}\ell - \frac {\Gamma - 1}{\ell \Gamma} = \frac 1\ell \cdot \cbr{\Gamma - 1 - \frac {\Gamma - 1}{\Gamma} } =\frac 1\ell \cdot \cbr{\Gamma + \frac 1\Gamma - 2} \approx 2{,}5\,\text{дптр}.
    \\
    &\text{В ответе надо указать оба значения.}
    \end{align*}
}
\solutionspace{120pt}

\tasknumber{6}%
\task{%
    Оптическая сила объектива фотоаппарата равна $5\,\text{дптр}$.
    При фотографировании чертежа с расстояния $0{,}9\,\text{м}$ площадь изображения
    чертежа на фотопластинке оказалась равной $4\,\text{см}^{2}$.
    Какова площадь самого чертежа? Ответ выразите в квадратных сантиметрах.
}
\answer{%
    \begin{align*}
    &\frac 1a + \frac 1b = \frac 1F = D \implies b = \frac{aF}{a - F} \\
    &\frac {S'}S = \Gamma^2 = \sqr{\frac ba} = \sqr{\frac F{a - F}} \implies \\
    &\implies S = S' \cdot \sqr{\frac{a - F}F} = S' \cdot \sqr{\frac aF - 1} = S' \cdot \sqr{aD - 1} \approx 49\,\text{см}^{2}.
    \end{align*}
}


\variantsplitter


\addpersonalvariant{Андрей Щербаков}

\tasknumber{7}%
\task{%
    В каком месте на главной оптической оси двояковыгнутой линзы
    нужно поместить точечный источник света,
    чтобы его изображение оказалось в главном фокусе линзы?
}
\answer{%
    $\text{на половине фокусного расстояния}$
}
\solutionspace{120pt}

\tasknumber{8}%
\task{%
    Предмет в виде отрезка длиной $\ell$ расположен вдоль оптической оси
    собирающей линзы с фокусным расстоянием $F$.
    Середина отрезка расположена
    на расстоянии $a$ от линзы, которая даёт действительное изображение
    всех точек предмета.
    Определить продольное увеличение предмета.
}
\answer{%
    \begin{align*}
    \frac 1{a + \frac \ell 2} &+ \frac 1b = \frac 1F \implies b = \frac{F\cbr{a + \frac \ell 2}}{a + \frac \ell 2 - F} \\
    \frac 1{a - \frac \ell 2} &+ \frac 1c = \frac 1F \implies c = \frac{F\cbr{a - \frac \ell 2}}{a - \frac \ell 2 - F} \\
    \abs{b - c} &= \abs{\frac{F\cbr{a + \frac \ell 2}}{a + \frac \ell 2 - F} - \frac{F\cbr{a - \frac \ell 2}}{a - \frac \ell 2 - F}}= F\abs{\frac{\cbr{a + \frac \ell 2}\cbr{a - \frac \ell 2 - F} - \cbr{a - \frac \ell 2}\cbr{a + \frac \ell 2 - F}}{ \cbr{a + \frac \ell 2 - F} \cbr{a - \frac \ell 2 - F} }} =  \\
    &= F\abs{\frac{a^2 - \frac {a\ell} 2 - Fa + \frac {a\ell} 2 - \frac {\ell^2} 4 - \frac {F\ell}2 - a^2 - \frac {a\ell}2 + aF + \frac {a\ell}2 + \frac {\ell^2} 4 - \frac {F\ell} 2}{\cbr{a + \frac \ell 2 - F} \cbr{a - \frac \ell 2 - F} }} = \\
    &= F\frac{F\ell}{\sqr{a-F} - \frac {\ell^2}4} = \frac{F^2\ell}{\sqr{a-F} - \frac {\ell^2}4}\implies \Gamma = \frac{\abs{b - c}}\ell = \frac{F^2}{\sqr{a-F} - \frac {\ell^2}4}.
    \end{align*}
}
\solutionspace{120pt}

\tasknumber{9}%
\task{%
    На экране с помощью тонкой линзы получено изображение предмета
    с увеличением $4$.
    Предмет передвинули на $2\,\text{см}$.
    Для того, чтобы получить резкое изображение, пришлось передвинуть экран.
    При этом увеличение оказалось равным $8$.
    На какое расстояние
    пришлось передвинуть экран?
}
\answer{%
    \begin{align*}
    &\frac 1a + \frac 1b = \frac 1F, \Gamma_1 = \frac ba = \frac{F}{a-F} \implies \Gamma_1(a-F) = F \implies a = F \cdot \frac{1 + \Gamma_1}{\Gamma_1} \\
    &\frac 1{a + x} + \frac 1{b + y} = \frac 1F, \Gamma_2 = \frac {b+y}{a+x} = \frac{F}{a+x-F} \implies a + x = F \cdot \frac{1 + \Gamma_2}{\Gamma_2} \\
    &1 + \frac xa = \frac{ \frac{1 + \Gamma_2}{\Gamma_2} }{ \frac{1 + \Gamma_1}{\Gamma_1} } = \frac{\Gamma_1(1 + \Gamma_2)}{\Gamma_2(1 + \Gamma_1)} \\
    &a = \frac x{ \frac{\Gamma_1(1 + \Gamma_2)}{\Gamma_2(1 + \Gamma_1)} - 1} = x \cdot \frac{\Gamma_2(1 + \Gamma_1)}{\Gamma_1 - \Gamma_2} \\
    &y = (a + x)\Gamma_2 - b = (a + x)\Gamma_2 - a\Gamma_1 = a(\Gamma_2 - \Gamma_1) + x\Gamma_2 = -x\Gamma_2(1 + \Gamma_1) + x\Gamma_2 = -x\Gamma_2\Gamma_1 = 64\,\text{см}, \\
    &\text{знаки разные, т.е.
    экран надо было подвинуть в другую сторону чем предмет: $x < 0, y > 0$.}
    \end{align*}
}
\solutionspace{120pt}

\tasknumber{10}%
\task{%
    Тонкая собирающая линза дает изображение предмета на экране при двух положениях линзы между предметом и экраном.
    Высота изображения при первом положении $15\,\text{см}$, во втором — $9\,\text{см}$.
    Расстояние между предметом и экранов постоянно.
    Чему равна высота предмета?
}
\answer{%
    \begin{align*}
    &\frac 1a + \frac 1b = \frac 1F, \frac 1c + \frac 1d = \frac 1F, a + b = c + d \implies \frac{a + b}{ab} = \frac 1F = \frac{c+d}{cd} \implies ab = cd, \\
    &\implies ab = c(a + b - c) \implies c^2 - ac - bc + ab = 0 \implies c = a \text{ или } c = b \implies c = b \implies d = a.
    \\
    &\Gamma_1 = \frac {H_1}H = \frac ba, \Gamma_2 = \frac {H_2}H = \frac dc = \frac ab \implies \frac {H_1}H \cdot \frac {H_2}H = \frac ba \cdot \frac ab = 1, \\
    &H = \sqrt{H_1 H_2} \approx 11{,}6\,\text{см}.
    \end{align*}
}
\solutionspace{120pt}

\tasknumber{11}%
\task{%
    Какие предметы можно рассмотреть на фотографии, сделанной со спутника,
    если разрешающая способность плёнки $0{,}010\,\text{мм}$? Каким должно быть
    время экспозиции $\tau$ чтобы полностью использовать возможности плёнки?
    Фокусное расстояние объектива используемого фотоаппарата $20\,\text{см}$,
    высота орбиты спутника $80\,\text{км}$.
}
\answer{%
    \begin{align*}
    &H \ll R \implies v = v_{\text{I}} = \sqrt{G R} \approx 7{,}9\,\frac{\text{км}}{\text{с}}.
    \\
    &F \ll H \implies b = F, a = H, \\
    &\Gamma = \frac \delta\ell = \frac ba \implies \ell = \frac{\delta a}b = \frac{\delta H}F \approx \frac{0{,}010\,\text{мм} \cdot 80\,\text{км}}{20\,\text{см}} \approx 4\,\text{м}, \\
    &\implies \tau = \frac \ell v = \frac{\delta H}{F v} = \frac{0{,}010\,\text{мм} \cdot 80\,\text{км}}{20\,\text{см} \cdot 7{,}9\,\frac{\text{км}}{\text{с}}} \approx 0{,}5\,\text{мс}.
    \end{align*}
}


\variantsplitter


\addpersonalvariant{Андрей Щербаков}

\tasknumber{12}%
\task{%
    При аэрофотосъемках используется фотоаппарат, объектив которого
    имеет фокусиое расстояние $20\,\text{см}$.
    Разрешающая способность плёнки $0{,}010\,\text{мм}$.
    На какой высоте должен лететь самолет, чтобы на фотографии можно
    было различить следы размером $30\,\text{см}$?
    При какой скорости самолета изображение не будет размытым,
    если время экспозиции $2\,\text{мс}$?
}
\answer{%
    \begin{align*}
    &F \ll H \implies b = F, a = H, \\
    &\Gamma = \frac \delta\ell = \frac ba = \frac FH \implies H = \frac{\ell F}\delta = \frac{30\,\text{см} \cdot 20\,\text{см}}{0{,}010\,\text{мм}} \approx 6\,\text{км}, \\
    &\implies v = \frac l\tau = \frac{30\,\text{см}}{2\,\text{мс}} \approx 540\,\frac{\text{км}}{\text{ч}}.
    \end{align*}
}
\solutionspace{120pt}

\tasknumber{13}%
\task{%
    Две одинаковые собиращие линзы установлены так, что их главные оптические оси совпадают,
    а главный фокус первой находится там же, где главный фокус второй.
    Расстояние от первой линзы до предмета равно $25\,\text{см}$.
    Чему равно расстояние от изображения объекта во второй линзе до второй линзы?
    Определите также увеличение.
    Фокусное расстояние каждой линзы $30\,\text{см}$.
}
\answer{%
    \begin{align*}
    \frac 1a + \frac 1b &= \frac 1F \implies b = \frac{aF}{a - F} \implies 2F - b = \frac{2aF - 2F^2 - aF}{a - F} = \frac{F(a - 2F)}{a - F}.
    \\
    \frac 1{2F - b} + \frac 1c &= \frac 1F \implies c = \frac{F(2F-b)}{(2F - b) - F} = \frac{F \cdot \frac{F(a - 2F)}{a - F}}{\frac{F(a - 2F)}{a - F} - F}  = F \cdot \frac{ \frac{F(a - 2F)}{a - F} }{ \frac{F(a - 2F)}{a - F} - 1} = \\
     &= F \cdot \frac{a - 2F}{a - 2F - a + F} = 2F - a = 35\,\text{см}.
     \\
    \ell &= a + 2F + c = 4F = 120\,\text{см}.
    \\
    &\Gamma = \Gamma_1 \cdot \Gamma_2 = \frac ba \cdot \frac c{2F-b} = \frac F{a - F} \cdot \frac{2F - a}{\frac{F(a - 2F)}{a - F}} = -1.
    \end{align*}
}
\solutionspace{120pt}

\tasknumber{14}%
\task{%
    Собирающая линза с фокусным расстоянием $F_1 > 0$ и рассеивающая линза с фокусным расстоянием $F_2 < 0$
    установлены коаксиально на расстоянии $\ell$.
    Пучок параллельных лучей падает на рассеивающую линзу.
    Сделайте схематичное построение и определите, в какой точке система из этих линз соберёт пучок.
}
\answer{%
    \begin{align*}
    &\text{Если пучок падает на собирающую линзу:} \\
    \frac 1{\infty} + \frac 1b &= \frac 1{F_1} \implies b = F_1 \implies \ell - b = \ell - F_1 \\
    \frac 1{\ell - b} + \frac 1c &= \frac 1{F_2} \implies c = \frac{F_2(\ell - b)}{\ell - b - F_2} = \frac{F_2(\ell - F_1)}{\ell - F_1 - F_2}.
    \\
    &\text{Если же пучок падает на рассеивающую линзу:} \\
    \frac 1{\infty} + \frac 1b &= \frac 1{F_2} \implies b = F_2 \implies \ell - b = \ell - F_2 \\
    \frac 1{\ell - b} + \frac 1c &= \frac 1{F_1} \implies c = \frac{F_1(\ell - b)}{\ell - b - F_1} = \frac{F_1(\ell - F_2)}{\ell - F_2 - F_1}.
    \end{align*}
}
\solutionspace{120pt}

\tasknumber{15}%
\task{%
    Две собирающих линзы с фокусными расстояниями $40\,\text{см}$ и $45\,\text{см}$ расположены так,
    что их оптические оси совмещены.
    На первую линзу падает пучок параллельных лучей.
    Пройдя через вторую линзу, он остался параллельным.
    Найдите расстояние между линзами и сделайте рисунок.
}
\answer{%
    \begin{align*}
    \frac 1\infty + \frac 1b &= \frac 1{F_1} \implies b = F_1, \\
    \frac 1{\ell - b} + \frac 1{\infty} &= \frac 1{F_2} \implies \ell - b = F_2 \implies \ell = b + F_2 = F_1 + F_2 = 85\,\text{см}.
    \end{align*}
}

\variantsplitter

\addpersonalvariant{Михаил Ярошевский}

\tasknumber{1}%
\task{%
    Найти оптическую силу собирающей линзы, если действительное изображение предмета,
    помещённого в $35\,\text{см}$ от линзы, получается на расстоянии $20\,\text{см}$ от неё.
}
\answer{%
    $D = \frac 1F = \frac 1a + \frac 1b = \frac 1{35\,\text{см}} + \frac 1{20\,\text{см}} \approx 7{,}86\,\text{дптр}$
}
\solutionspace{80pt}

\tasknumber{2}%
\task{%
    Найти увеличение изображения, если изображение предмета, находящегося
    на расстоянии $25\,\text{см}$ от линзы, получается на расстоянии $30\,\text{см}$ от неё.
}
\answer{%
    $\Gamma = \frac ba = \frac {30\,\text{см}}{25\,\text{см}} \approx 1{,}20$
}
\solutionspace{80pt}

\tasknumber{3}%
\task{%
    Расстояние от предмета до линзы $12\,\text{см}$, а от линзы до мнимого изображения $30\,\text{см}$.
    Чему равно фокусное расстояние линзы?
}
\answer{%
    $\pm \frac 1F = \frac 1a - \frac 1b \implies F = \frac{a b}{\abs{b - a}} \approx 20\,\text{см}$
}
\solutionspace{80pt}

\tasknumber{4}%
\task{%
    Две тонкие собирающие линзы с фокусными расстояниями $18\,\text{см}$ и $20\,\text{см}$ сложены вместе.
    Чему равно фокусное расстояние такой оптической системы?
}
\answer{%
    $\frac 1{f_1} = \frac 1a + \frac 1b; \frac 1{f_2} = - \frac 1b + \frac 1c \implies \frac 1{f_1} + \frac 1{f_2} = \frac 1a + \frac 1c \implies f' = \frac 1{\frac 1{f_1} + \frac 1{f_2}} = \frac{f_1 f_2}{f_1 + f_2} \approx 9{,}5\,\text{см}$
}
\solutionspace{80pt}

\tasknumber{5}%
\task{%
    Линейные размеры прямого изображения предмета, полученного в собирающей линзе,
    в два раза больше линейных размеров предмета.
    Зная, что предмет находится на $35\,\text{см}$ ближе к линзе,
    чем его изображение, найти оптическую силу линзы.
}
\answer{%
    \begin{align*}
    &\text{Если изображение действительное:} \\
    D &= \frac 1F = \frac 1a + \frac 1b, \qquad \Gamma = \frac ba, \qquad b - a = \ell \implies b = \Gamma a \implies \Gamma a - a = \ell \implies  \\
    a &= \frac {\ell}{\Gamma - 1} \implies b = \frac {{\ell} \Gamma}{\Gamma - 1} \implies  \\
    D &= \frac {\Gamma - 1}\ell + \frac {\Gamma - 1}{\ell \Gamma} = \frac 1\ell \cdot \cbr{\Gamma - 1 + \frac {\Gamma - 1}{\Gamma} } =\frac 1\ell \cdot \cbr{\Gamma - \frac 1\Gamma} \approx 4{,}3\,\text{дптр}.
    \\
    &\text{Если изображение мнимое:} \\
    D &= \frac 1F = \frac 1a - \frac 1b, \qquad \Gamma = \frac ba, \qquad b - a = \ell \implies b = \Gamma a \implies \Gamma a - a = \ell \implies  \\
    a &= \frac {\ell}{\Gamma - 1} \implies b = \frac {{\ell} \Gamma}{\Gamma - 1} \implies  \\
    D &= \frac {\Gamma - 1}\ell - \frac {\Gamma - 1}{\ell \Gamma} = \frac 1\ell \cdot \cbr{\Gamma - 1 - \frac {\Gamma - 1}{\Gamma} } =\frac 1\ell \cdot \cbr{\Gamma + \frac 1\Gamma - 2} \approx 1{,}4\,\text{дптр}.
    \\
    &\text{В ответе надо указать оба значения.}
    \end{align*}
}
\solutionspace{120pt}

\tasknumber{6}%
\task{%
    Оптическая сила объектива фотоаппарата равна $3\,\text{дптр}$.
    При фотографировании чертежа с расстояния $1{,}1\,\text{м}$ площадь изображения
    чертежа на фотопластинке оказалась равной $9\,\text{см}^{2}$.
    Какова площадь самого чертежа? Ответ выразите в квадратных сантиметрах.
}
\answer{%
    \begin{align*}
    &\frac 1a + \frac 1b = \frac 1F = D \implies b = \frac{aF}{a - F} \\
    &\frac {S'}S = \Gamma^2 = \sqr{\frac ba} = \sqr{\frac F{a - F}} \implies \\
    &\implies S = S' \cdot \sqr{\frac{a - F}F} = S' \cdot \sqr{\frac aF - 1} = S' \cdot \sqr{aD - 1} \approx 50\,\text{см}^{2}.
    \end{align*}
}


\variantsplitter


\addpersonalvariant{Михаил Ярошевский}

\tasknumber{7}%
\task{%
    В каком месте на главной оптической оси двояковыгнутой линзы
    нужно поместить точечный источник света,
    чтобы его изображение оказалось в главном фокусе линзы?
}
\answer{%
    $\text{на половине фокусного расстояния}$
}
\solutionspace{120pt}

\tasknumber{8}%
\task{%
    Предмет в виде отрезка длиной $\ell$ расположен вдоль оптической оси
    собирающей линзы с фокусным расстоянием $F$.
    Середина отрезка расположена
    на расстоянии $a$ от линзы, которая даёт действительное изображение
    всех точек предмета.
    Определить продольное увеличение предмета.
}
\answer{%
    \begin{align*}
    \frac 1{a + \frac \ell 2} &+ \frac 1b = \frac 1F \implies b = \frac{F\cbr{a + \frac \ell 2}}{a + \frac \ell 2 - F} \\
    \frac 1{a - \frac \ell 2} &+ \frac 1c = \frac 1F \implies c = \frac{F\cbr{a - \frac \ell 2}}{a - \frac \ell 2 - F} \\
    \abs{b - c} &= \abs{\frac{F\cbr{a + \frac \ell 2}}{a + \frac \ell 2 - F} - \frac{F\cbr{a - \frac \ell 2}}{a - \frac \ell 2 - F}}= F\abs{\frac{\cbr{a + \frac \ell 2}\cbr{a - \frac \ell 2 - F} - \cbr{a - \frac \ell 2}\cbr{a + \frac \ell 2 - F}}{ \cbr{a + \frac \ell 2 - F} \cbr{a - \frac \ell 2 - F} }} =  \\
    &= F\abs{\frac{a^2 - \frac {a\ell} 2 - Fa + \frac {a\ell} 2 - \frac {\ell^2} 4 - \frac {F\ell}2 - a^2 - \frac {a\ell}2 + aF + \frac {a\ell}2 + \frac {\ell^2} 4 - \frac {F\ell} 2}{\cbr{a + \frac \ell 2 - F} \cbr{a - \frac \ell 2 - F} }} = \\
    &= F\frac{F\ell}{\sqr{a-F} - \frac {\ell^2}4} = \frac{F^2\ell}{\sqr{a-F} - \frac {\ell^2}4}\implies \Gamma = \frac{\abs{b - c}}\ell = \frac{F^2}{\sqr{a-F} - \frac {\ell^2}4}.
    \end{align*}
}
\solutionspace{120pt}

\tasknumber{9}%
\task{%
    На экране с помощью тонкой линзы получено изображение предмета
    с увеличением $4$.
    Предмет передвинули на $2\,\text{см}$.
    Для того, чтобы получить резкое изображение, пришлось передвинуть экран.
    При этом увеличение оказалось равным $8$.
    На какое расстояние
    пришлось передвинуть экран?
}
\answer{%
    \begin{align*}
    &\frac 1a + \frac 1b = \frac 1F, \Gamma_1 = \frac ba = \frac{F}{a-F} \implies \Gamma_1(a-F) = F \implies a = F \cdot \frac{1 + \Gamma_1}{\Gamma_1} \\
    &\frac 1{a + x} + \frac 1{b + y} = \frac 1F, \Gamma_2 = \frac {b+y}{a+x} = \frac{F}{a+x-F} \implies a + x = F \cdot \frac{1 + \Gamma_2}{\Gamma_2} \\
    &1 + \frac xa = \frac{ \frac{1 + \Gamma_2}{\Gamma_2} }{ \frac{1 + \Gamma_1}{\Gamma_1} } = \frac{\Gamma_1(1 + \Gamma_2)}{\Gamma_2(1 + \Gamma_1)} \\
    &a = \frac x{ \frac{\Gamma_1(1 + \Gamma_2)}{\Gamma_2(1 + \Gamma_1)} - 1} = x \cdot \frac{\Gamma_2(1 + \Gamma_1)}{\Gamma_1 - \Gamma_2} \\
    &y = (a + x)\Gamma_2 - b = (a + x)\Gamma_2 - a\Gamma_1 = a(\Gamma_2 - \Gamma_1) + x\Gamma_2 = -x\Gamma_2(1 + \Gamma_1) + x\Gamma_2 = -x\Gamma_2\Gamma_1 = 64\,\text{см}, \\
    &\text{знаки разные, т.е.
    экран надо было подвинуть в другую сторону чем предмет: $x < 0, y > 0$.}
    \end{align*}
}
\solutionspace{120pt}

\tasknumber{10}%
\task{%
    Тонкая собирающая линза дает изображение предмета на экране при двух положениях линзы между предметом и экраном.
    Высота изображения при первом положении $15\,\text{см}$, во втором — $9\,\text{см}$.
    Расстояние между предметом и экранов постоянно.
    Чему равна высота предмета?
}
\answer{%
    \begin{align*}
    &\frac 1a + \frac 1b = \frac 1F, \frac 1c + \frac 1d = \frac 1F, a + b = c + d \implies \frac{a + b}{ab} = \frac 1F = \frac{c+d}{cd} \implies ab = cd, \\
    &\implies ab = c(a + b - c) \implies c^2 - ac - bc + ab = 0 \implies c = a \text{ или } c = b \implies c = b \implies d = a.
    \\
    &\Gamma_1 = \frac {H_1}H = \frac ba, \Gamma_2 = \frac {H_2}H = \frac dc = \frac ab \implies \frac {H_1}H \cdot \frac {H_2}H = \frac ba \cdot \frac ab = 1, \\
    &H = \sqrt{H_1 H_2} \approx 11{,}6\,\text{см}.
    \end{align*}
}
\solutionspace{120pt}

\tasknumber{11}%
\task{%
    Какие предметы можно рассмотреть на фотографии, сделанной со спутника,
    если разрешающая способность плёнки $0{,}02\,\text{мм}$? Каким должно быть
    время экспозиции $\tau$ чтобы полностью использовать возможности плёнки?
    Фокусное расстояние объектива используемого фотоаппарата $10\,\text{см}$,
    высота орбиты спутника $120\,\text{км}$.
}
\answer{%
    \begin{align*}
    &H \ll R \implies v = v_{\text{I}} = \sqrt{G R} \approx 7{,}9\,\frac{\text{км}}{\text{с}}.
    \\
    &F \ll H \implies b = F, a = H, \\
    &\Gamma = \frac \delta\ell = \frac ba \implies \ell = \frac{\delta a}b = \frac{\delta H}F \approx \frac{0{,}02\,\text{мм} \cdot 120\,\text{км}}{10\,\text{см}} \approx 24\,\text{м}, \\
    &\implies \tau = \frac \ell v = \frac{\delta H}{F v} = \frac{0{,}02\,\text{мм} \cdot 120\,\text{км}}{10\,\text{см} \cdot 7{,}9\,\frac{\text{км}}{\text{с}}} \approx 3\,\text{мс}.
    \end{align*}
}


\variantsplitter


\addpersonalvariant{Михаил Ярошевский}

\tasknumber{12}%
\task{%
    При аэрофотосъемках используется фотоаппарат, объектив которого
    имеет фокусиое расстояние $10\,\text{см}$.
    Разрешающая способность плёнки $0{,}010\,\text{мм}$.
    На какой высоте должен лететь самолет, чтобы на фотографии можно
    было различить следы размером $30\,\text{см}$?
    При какой скорости самолета изображение не будет размытым,
    если время экспозиции $2\,\text{мс}$?
}
\answer{%
    \begin{align*}
    &F \ll H \implies b = F, a = H, \\
    &\Gamma = \frac \delta\ell = \frac ba = \frac FH \implies H = \frac{\ell F}\delta = \frac{30\,\text{см} \cdot 10\,\text{см}}{0{,}010\,\text{мм}} \approx 3\,\text{км}, \\
    &\implies v = \frac l\tau = \frac{30\,\text{см}}{2\,\text{мс}} \approx 540\,\frac{\text{км}}{\text{ч}}.
    \end{align*}
}
\solutionspace{120pt}

\tasknumber{13}%
\task{%
    Две одинаковые собиращие линзы установлены так, что их главные оптические оси совпадают,
    а главный фокус первой находится там же, где главный фокус второй.
    Расстояние от первой линзы до предмета равно $14\,\text{см}$.
    Чему равно расстояние от изображения объекта во второй линзе до второй линзы?
    Определите также увеличение.
    Фокусное расстояние каждой линзы $35\,\text{см}$.
}
\answer{%
    \begin{align*}
    \frac 1a + \frac 1b &= \frac 1F \implies b = \frac{aF}{a - F} \implies 2F - b = \frac{2aF - 2F^2 - aF}{a - F} = \frac{F(a - 2F)}{a - F}.
    \\
    \frac 1{2F - b} + \frac 1c &= \frac 1F \implies c = \frac{F(2F-b)}{(2F - b) - F} = \frac{F \cdot \frac{F(a - 2F)}{a - F}}{\frac{F(a - 2F)}{a - F} - F}  = F \cdot \frac{ \frac{F(a - 2F)}{a - F} }{ \frac{F(a - 2F)}{a - F} - 1} = \\
     &= F \cdot \frac{a - 2F}{a - 2F - a + F} = 2F - a = 56\,\text{см}.
     \\
    \ell &= a + 2F + c = 4F = 140\,\text{см}.
    \\
    &\Gamma = \Gamma_1 \cdot \Gamma_2 = \frac ba \cdot \frac c{2F-b} = \frac F{a - F} \cdot \frac{2F - a}{\frac{F(a - 2F)}{a - F}} = -1.
    \end{align*}
}
\solutionspace{120pt}

\tasknumber{14}%
\task{%
    Собирающая линза с фокусным расстоянием $F_1 > 0$ и рассеивающая линза с фокусным расстоянием $F_2 < 0$
    установлены коаксиально на расстоянии $\ell$.
    Пучок параллельных лучей падает на рассеивающую линзу.
    Сделайте схематичное построение и определите, в какой точке система из этих линз соберёт пучок.
}
\answer{%
    \begin{align*}
    &\text{Если пучок падает на собирающую линзу:} \\
    \frac 1{\infty} + \frac 1b &= \frac 1{F_1} \implies b = F_1 \implies \ell - b = \ell - F_1 \\
    \frac 1{\ell - b} + \frac 1c &= \frac 1{F_2} \implies c = \frac{F_2(\ell - b)}{\ell - b - F_2} = \frac{F_2(\ell - F_1)}{\ell - F_1 - F_2}.
    \\
    &\text{Если же пучок падает на рассеивающую линзу:} \\
    \frac 1{\infty} + \frac 1b &= \frac 1{F_2} \implies b = F_2 \implies \ell - b = \ell - F_2 \\
    \frac 1{\ell - b} + \frac 1c &= \frac 1{F_1} \implies c = \frac{F_1(\ell - b)}{\ell - b - F_1} = \frac{F_1(\ell - F_2)}{\ell - F_2 - F_1}.
    \end{align*}
}
\solutionspace{120pt}

\tasknumber{15}%
\task{%
    Две собирающих линзы с фокусными расстояниями $20\,\text{см}$ и $35\,\text{см}$ расположены так,
    что их оптические оси совмещены.
    На первую линзу падает пучок параллельных лучей.
    Пройдя через вторую линзу, он остался параллельным.
    Найдите расстояние между линзами и сделайте рисунок.
}
\answer{%
    \begin{align*}
    \frac 1\infty + \frac 1b &= \frac 1{F_1} \implies b = F_1, \\
    \frac 1{\ell - b} + \frac 1{\infty} &= \frac 1{F_2} \implies \ell - b = F_2 \implies \ell = b + F_2 = F_1 + F_2 = 55\,\text{см}.
    \end{align*}
}

\variantsplitter

\addpersonalvariant{Алексей Алимпиев}

\tasknumber{1}%
\task{%
    Найти оптическую силу собирающей линзы, если действительное изображение предмета,
    помещённого в $55\,\text{см}$ от линзы, получается на расстоянии $40\,\text{см}$ от неё.
}
\answer{%
    $D = \frac 1F = \frac 1a + \frac 1b = \frac 1{55\,\text{см}} + \frac 1{40\,\text{см}} \approx 4{,}32\,\text{дптр}$
}
\solutionspace{80pt}

\tasknumber{2}%
\task{%
    Найти увеличение изображения, если изображение предмета, находящегося
    на расстоянии $25\,\text{см}$ от линзы, получается на расстоянии $18\,\text{см}$ от неё.
}
\answer{%
    $\Gamma = \frac ba = \frac {18\,\text{см}}{25\,\text{см}} \approx 0{,}7$
}
\solutionspace{80pt}

\tasknumber{3}%
\task{%
    Расстояние от предмета до линзы $12\,\text{см}$, а от линзы до мнимого изображения $25\,\text{см}$.
    Чему равно фокусное расстояние линзы?
}
\answer{%
    $\pm \frac 1F = \frac 1a - \frac 1b \implies F = \frac{a b}{\abs{b - a}} \approx 23{,}1\,\text{см}$
}
\solutionspace{80pt}

\tasknumber{4}%
\task{%
    Две тонкие собирающие линзы с фокусными расстояниями $18\,\text{см}$ и $20\,\text{см}$ сложены вместе.
    Чему равно фокусное расстояние такой оптической системы?
}
\answer{%
    $\frac 1{f_1} = \frac 1a + \frac 1b; \frac 1{f_2} = - \frac 1b + \frac 1c \implies \frac 1{f_1} + \frac 1{f_2} = \frac 1a + \frac 1c \implies f' = \frac 1{\frac 1{f_1} + \frac 1{f_2}} = \frac{f_1 f_2}{f_1 + f_2} \approx 9{,}5\,\text{см}$
}
\solutionspace{80pt}

\tasknumber{5}%
\task{%
    Линейные размеры прямого изображения предмета, полученного в собирающей линзе,
    в три раза больше линейных размеров предмета.
    Зная, что предмет находится на $35\,\text{см}$ ближе к линзе,
    чем его изображение, найти оптическую силу линзы.
}
\answer{%
    \begin{align*}
    &\text{Если изображение действительное:} \\
    D &= \frac 1F = \frac 1a + \frac 1b, \qquad \Gamma = \frac ba, \qquad b - a = \ell \implies b = \Gamma a \implies \Gamma a - a = \ell \implies  \\
    a &= \frac {\ell}{\Gamma - 1} \implies b = \frac {{\ell} \Gamma}{\Gamma - 1} \implies  \\
    D &= \frac {\Gamma - 1}\ell + \frac {\Gamma - 1}{\ell \Gamma} = \frac 1\ell \cdot \cbr{\Gamma - 1 + \frac {\Gamma - 1}{\Gamma} } =\frac 1\ell \cdot \cbr{\Gamma - \frac 1\Gamma} \approx 7{,}6\,\text{дптр}.
    \\
    &\text{Если изображение мнимое:} \\
    D &= \frac 1F = \frac 1a - \frac 1b, \qquad \Gamma = \frac ba, \qquad b - a = \ell \implies b = \Gamma a \implies \Gamma a - a = \ell \implies  \\
    a &= \frac {\ell}{\Gamma - 1} \implies b = \frac {{\ell} \Gamma}{\Gamma - 1} \implies  \\
    D &= \frac {\Gamma - 1}\ell - \frac {\Gamma - 1}{\ell \Gamma} = \frac 1\ell \cdot \cbr{\Gamma - 1 - \frac {\Gamma - 1}{\Gamma} } =\frac 1\ell \cdot \cbr{\Gamma + \frac 1\Gamma - 2} \approx 3{,}8\,\text{дптр}.
    \\
    &\text{В ответе надо указать оба значения.}
    \end{align*}
}
\solutionspace{120pt}

\tasknumber{6}%
\task{%
    Оптическая сила объектива фотоаппарата равна $3\,\text{дптр}$.
    При фотографировании чертежа с расстояния $0{,}8\,\text{м}$ площадь изображения
    чертежа на фотопластинке оказалась равной $16\,\text{см}^{2}$.
    Какова площадь самого чертежа? Ответ выразите в квадратных сантиметрах.
}
\answer{%
    \begin{align*}
    &\frac 1a + \frac 1b = \frac 1F = D \implies b = \frac{aF}{a - F} \\
    &\frac {S'}S = \Gamma^2 = \sqr{\frac ba} = \sqr{\frac F{a - F}} \implies \\
    &\implies S = S' \cdot \sqr{\frac{a - F}F} = S' \cdot \sqr{\frac aF - 1} = S' \cdot \sqr{aD - 1} \approx 30\,\text{см}^{2}.
    \end{align*}
}


\variantsplitter


\addpersonalvariant{Алексей Алимпиев}

\tasknumber{7}%
\task{%
    В каком месте на главной оптической оси двояковыгнутой линзы
    нужно поместить точечный источник света,
    чтобы его изображение оказалось в главном фокусе линзы?
}
\answer{%
    $\text{на половине фокусного расстояния}$
}
\solutionspace{120pt}

\tasknumber{8}%
\task{%
    Предмет в виде отрезка длиной $\ell$ расположен вдоль оптической оси
    собирающей линзы с фокусным расстоянием $F$.
    Середина отрезка расположена
    на расстоянии $a$ от линзы, которая даёт действительное изображение
    всех точек предмета.
    Определить продольное увеличение предмета.
}
\answer{%
    \begin{align*}
    \frac 1{a + \frac \ell 2} &+ \frac 1b = \frac 1F \implies b = \frac{F\cbr{a + \frac \ell 2}}{a + \frac \ell 2 - F} \\
    \frac 1{a - \frac \ell 2} &+ \frac 1c = \frac 1F \implies c = \frac{F\cbr{a - \frac \ell 2}}{a - \frac \ell 2 - F} \\
    \abs{b - c} &= \abs{\frac{F\cbr{a + \frac \ell 2}}{a + \frac \ell 2 - F} - \frac{F\cbr{a - \frac \ell 2}}{a - \frac \ell 2 - F}}= F\abs{\frac{\cbr{a + \frac \ell 2}\cbr{a - \frac \ell 2 - F} - \cbr{a - \frac \ell 2}\cbr{a + \frac \ell 2 - F}}{ \cbr{a + \frac \ell 2 - F} \cbr{a - \frac \ell 2 - F} }} =  \\
    &= F\abs{\frac{a^2 - \frac {a\ell} 2 - Fa + \frac {a\ell} 2 - \frac {\ell^2} 4 - \frac {F\ell}2 - a^2 - \frac {a\ell}2 + aF + \frac {a\ell}2 + \frac {\ell^2} 4 - \frac {F\ell} 2}{\cbr{a + \frac \ell 2 - F} \cbr{a - \frac \ell 2 - F} }} = \\
    &= F\frac{F\ell}{\sqr{a-F} - \frac {\ell^2}4} = \frac{F^2\ell}{\sqr{a-F} - \frac {\ell^2}4}\implies \Gamma = \frac{\abs{b - c}}\ell = \frac{F^2}{\sqr{a-F} - \frac {\ell^2}4}.
    \end{align*}
}
\solutionspace{120pt}

\tasknumber{9}%
\task{%
    На экране с помощью тонкой линзы получено изображение предмета
    с увеличением $2$.
    Предмет передвинули на $4\,\text{см}$.
    Для того, чтобы получить резкое изображение, пришлось передвинуть экран.
    При этом увеличение оказалось равным $6$.
    На какое расстояние
    пришлось передвинуть экран?
}
\answer{%
    \begin{align*}
    &\frac 1a + \frac 1b = \frac 1F, \Gamma_1 = \frac ba = \frac{F}{a-F} \implies \Gamma_1(a-F) = F \implies a = F \cdot \frac{1 + \Gamma_1}{\Gamma_1} \\
    &\frac 1{a + x} + \frac 1{b + y} = \frac 1F, \Gamma_2 = \frac {b+y}{a+x} = \frac{F}{a+x-F} \implies a + x = F \cdot \frac{1 + \Gamma_2}{\Gamma_2} \\
    &1 + \frac xa = \frac{ \frac{1 + \Gamma_2}{\Gamma_2} }{ \frac{1 + \Gamma_1}{\Gamma_1} } = \frac{\Gamma_1(1 + \Gamma_2)}{\Gamma_2(1 + \Gamma_1)} \\
    &a = \frac x{ \frac{\Gamma_1(1 + \Gamma_2)}{\Gamma_2(1 + \Gamma_1)} - 1} = x \cdot \frac{\Gamma_2(1 + \Gamma_1)}{\Gamma_1 - \Gamma_2} \\
    &y = (a + x)\Gamma_2 - b = (a + x)\Gamma_2 - a\Gamma_1 = a(\Gamma_2 - \Gamma_1) + x\Gamma_2 = -x\Gamma_2(1 + \Gamma_1) + x\Gamma_2 = -x\Gamma_2\Gamma_1 = 48\,\text{см}, \\
    &\text{знаки разные, т.е.
    экран надо было подвинуть в другую сторону чем предмет: $x < 0, y > 0$.}
    \end{align*}
}
\solutionspace{120pt}

\tasknumber{10}%
\task{%
    Тонкая собирающая линза дает изображение предмета на экране при двух положениях линзы между предметом и экраном.
    Высота изображения при первом положении $20\,\text{см}$, во втором — $9\,\text{см}$.
    Расстояние между предметом и экранов постоянно.
    Чему равна высота предмета?
}
\answer{%
    \begin{align*}
    &\frac 1a + \frac 1b = \frac 1F, \frac 1c + \frac 1d = \frac 1F, a + b = c + d \implies \frac{a + b}{ab} = \frac 1F = \frac{c+d}{cd} \implies ab = cd, \\
    &\implies ab = c(a + b - c) \implies c^2 - ac - bc + ab = 0 \implies c = a \text{ или } c = b \implies c = b \implies d = a.
    \\
    &\Gamma_1 = \frac {H_1}H = \frac ba, \Gamma_2 = \frac {H_2}H = \frac dc = \frac ab \implies \frac {H_1}H \cdot \frac {H_2}H = \frac ba \cdot \frac ab = 1, \\
    &H = \sqrt{H_1 H_2} \approx 13{,}4\,\text{см}.
    \end{align*}
}
\solutionspace{120pt}

\tasknumber{11}%
\task{%
    Какие предметы можно рассмотреть на фотографии, сделанной со спутника,
    если разрешающая способность плёнки $0{,}010\,\text{мм}$? Каким должно быть
    время экспозиции $\tau$ чтобы полностью использовать возможности плёнки?
    Фокусное расстояние объектива используемого фотоаппарата $20\,\text{см}$,
    высота орбиты спутника $100\,\text{км}$.
}
\answer{%
    \begin{align*}
    &H \ll R \implies v = v_{\text{I}} = \sqrt{G R} \approx 7{,}9\,\frac{\text{км}}{\text{с}}.
    \\
    &F \ll H \implies b = F, a = H, \\
    &\Gamma = \frac \delta\ell = \frac ba \implies \ell = \frac{\delta a}b = \frac{\delta H}F \approx \frac{0{,}010\,\text{мм} \cdot 100\,\text{км}}{20\,\text{см}} \approx 5\,\text{м}, \\
    &\implies \tau = \frac \ell v = \frac{\delta H}{F v} = \frac{0{,}010\,\text{мм} \cdot 100\,\text{км}}{20\,\text{см} \cdot 7{,}9\,\frac{\text{км}}{\text{с}}} \approx 0{,}6\,\text{мс}.
    \end{align*}
}


\variantsplitter


\addpersonalvariant{Алексей Алимпиев}

\tasknumber{12}%
\task{%
    При аэрофотосъемках используется фотоаппарат, объектив которого
    имеет фокусиое расстояние $10\,\text{см}$.
    Разрешающая способность плёнки $0{,}02\,\text{мм}$.
    На какой высоте должен лететь самолет, чтобы на фотографии можно
    было различить следы размером $15\,\text{см}$?
    При какой скорости самолета изображение не будет размытым,
    если время экспозиции $2\,\text{мс}$?
}
\answer{%
    \begin{align*}
    &F \ll H \implies b = F, a = H, \\
    &\Gamma = \frac \delta\ell = \frac ba = \frac FH \implies H = \frac{\ell F}\delta = \frac{15\,\text{см} \cdot 10\,\text{см}}{0{,}02\,\text{мм}} \approx 0{,}8\,\text{км}, \\
    &\implies v = \frac l\tau = \frac{15\,\text{см}}{2\,\text{мс}} \approx 270\,\frac{\text{км}}{\text{ч}}.
    \end{align*}
}
\solutionspace{120pt}

\tasknumber{13}%
\task{%
    Две одинаковые собиращие линзы установлены так, что их главные оптические оси совпадают,
    а главный фокус первой находится там же, где главный фокус второй.
    Расстояние от первой линзы до предмета равно $5\,\text{см}$.
    Чему равно расстояние от изображения объекта во второй линзе до второй линзы?
    Определите также увеличение.
    Фокусное расстояние каждой линзы $20\,\text{см}$.
}
\answer{%
    \begin{align*}
    \frac 1a + \frac 1b &= \frac 1F \implies b = \frac{aF}{a - F} \implies 2F - b = \frac{2aF - 2F^2 - aF}{a - F} = \frac{F(a - 2F)}{a - F}.
    \\
    \frac 1{2F - b} + \frac 1c &= \frac 1F \implies c = \frac{F(2F-b)}{(2F - b) - F} = \frac{F \cdot \frac{F(a - 2F)}{a - F}}{\frac{F(a - 2F)}{a - F} - F}  = F \cdot \frac{ \frac{F(a - 2F)}{a - F} }{ \frac{F(a - 2F)}{a - F} - 1} = \\
     &= F \cdot \frac{a - 2F}{a - 2F - a + F} = 2F - a = 35\,\text{см}.
     \\
    \ell &= a + 2F + c = 4F = 80\,\text{см}.
    \\
    &\Gamma = \Gamma_1 \cdot \Gamma_2 = \frac ba \cdot \frac c{2F-b} = \frac F{a - F} \cdot \frac{2F - a}{\frac{F(a - 2F)}{a - F}} = -1.
    \end{align*}
}
\solutionspace{120pt}

\tasknumber{14}%
\task{%
    Собирающая линза с фокусным расстоянием $F_1 > 0$ и рассеивающая линза с фокусным расстоянием $F_2 < 0$
    установлены коаксиально на расстоянии $\ell$.
    Пучок параллельных лучей падает на рассеивающую линзу.
    Сделайте схематичное построение и определите, в какой точке система из этих линз соберёт пучок.
}
\answer{%
    \begin{align*}
    &\text{Если пучок падает на собирающую линзу:} \\
    \frac 1{\infty} + \frac 1b &= \frac 1{F_1} \implies b = F_1 \implies \ell - b = \ell - F_1 \\
    \frac 1{\ell - b} + \frac 1c &= \frac 1{F_2} \implies c = \frac{F_2(\ell - b)}{\ell - b - F_2} = \frac{F_2(\ell - F_1)}{\ell - F_1 - F_2}.
    \\
    &\text{Если же пучок падает на рассеивающую линзу:} \\
    \frac 1{\infty} + \frac 1b &= \frac 1{F_2} \implies b = F_2 \implies \ell - b = \ell - F_2 \\
    \frac 1{\ell - b} + \frac 1c &= \frac 1{F_1} \implies c = \frac{F_1(\ell - b)}{\ell - b - F_1} = \frac{F_1(\ell - F_2)}{\ell - F_2 - F_1}.
    \end{align*}
}
\solutionspace{120pt}

\tasknumber{15}%
\task{%
    Две собирающих линзы с фокусными расстояниями $20\,\text{см}$ и $45\,\text{см}$ расположены так,
    что их оптические оси совмещены.
    На первую линзу падает пучок параллельных лучей.
    Пройдя через вторую линзу, он остался параллельным.
    Найдите расстояние между линзами и сделайте рисунок.
}
\answer{%
    \begin{align*}
    \frac 1\infty + \frac 1b &= \frac 1{F_1} \implies b = F_1, \\
    \frac 1{\ell - b} + \frac 1{\infty} &= \frac 1{F_2} \implies \ell - b = F_2 \implies \ell = b + F_2 = F_1 + F_2 = 65\,\text{см}.
    \end{align*}
}

\variantsplitter

\addpersonalvariant{Евгений Васин}

\tasknumber{1}%
\task{%
    Найти оптическую силу собирающей линзы, если действительное изображение предмета,
    помещённого в $35\,\text{см}$ от линзы, получается на расстоянии $20\,\text{см}$ от неё.
}
\answer{%
    $D = \frac 1F = \frac 1a + \frac 1b = \frac 1{35\,\text{см}} + \frac 1{20\,\text{см}} \approx 7{,}86\,\text{дптр}$
}
\solutionspace{80pt}

\tasknumber{2}%
\task{%
    Найти увеличение изображения, если изображение предмета, находящегося
    на расстоянии $25\,\text{см}$ от линзы, получается на расстоянии $30\,\text{см}$ от неё.
}
\answer{%
    $\Gamma = \frac ba = \frac {30\,\text{см}}{25\,\text{см}} \approx 1{,}20$
}
\solutionspace{80pt}

\tasknumber{3}%
\task{%
    Расстояние от предмета до линзы $12\,\text{см}$, а от линзы до мнимого изображения $30\,\text{см}$.
    Чему равно фокусное расстояние линзы?
}
\answer{%
    $\pm \frac 1F = \frac 1a - \frac 1b \implies F = \frac{a b}{\abs{b - a}} \approx 20\,\text{см}$
}
\solutionspace{80pt}

\tasknumber{4}%
\task{%
    Две тонкие собирающие линзы с фокусными расстояниями $12\,\text{см}$ и $20\,\text{см}$ сложены вместе.
    Чему равно фокусное расстояние такой оптической системы?
}
\answer{%
    $\frac 1{f_1} = \frac 1a + \frac 1b; \frac 1{f_2} = - \frac 1b + \frac 1c \implies \frac 1{f_1} + \frac 1{f_2} = \frac 1a + \frac 1c \implies f' = \frac 1{\frac 1{f_1} + \frac 1{f_2}} = \frac{f_1 f_2}{f_1 + f_2} \approx 7{,}5\,\text{см}$
}
\solutionspace{80pt}

\tasknumber{5}%
\task{%
    Линейные размеры прямого изображения предмета, полученного в собирающей линзе,
    в три раза больше линейных размеров предмета.
    Зная, что предмет находится на $30\,\text{см}$ ближе к линзе,
    чем его изображение, найти оптическую силу линзы.
}
\answer{%
    \begin{align*}
    &\text{Если изображение действительное:} \\
    D &= \frac 1F = \frac 1a + \frac 1b, \qquad \Gamma = \frac ba, \qquad b - a = \ell \implies b = \Gamma a \implies \Gamma a - a = \ell \implies  \\
    a &= \frac {\ell}{\Gamma - 1} \implies b = \frac {{\ell} \Gamma}{\Gamma - 1} \implies  \\
    D &= \frac {\Gamma - 1}\ell + \frac {\Gamma - 1}{\ell \Gamma} = \frac 1\ell \cdot \cbr{\Gamma - 1 + \frac {\Gamma - 1}{\Gamma} } =\frac 1\ell \cdot \cbr{\Gamma - \frac 1\Gamma} \approx 8{,}9\,\text{дптр}.
    \\
    &\text{Если изображение мнимое:} \\
    D &= \frac 1F = \frac 1a - \frac 1b, \qquad \Gamma = \frac ba, \qquad b - a = \ell \implies b = \Gamma a \implies \Gamma a - a = \ell \implies  \\
    a &= \frac {\ell}{\Gamma - 1} \implies b = \frac {{\ell} \Gamma}{\Gamma - 1} \implies  \\
    D &= \frac {\Gamma - 1}\ell - \frac {\Gamma - 1}{\ell \Gamma} = \frac 1\ell \cdot \cbr{\Gamma - 1 - \frac {\Gamma - 1}{\Gamma} } =\frac 1\ell \cdot \cbr{\Gamma + \frac 1\Gamma - 2} \approx 4{,}4\,\text{дптр}.
    \\
    &\text{В ответе надо указать оба значения.}
    \end{align*}
}
\solutionspace{120pt}

\tasknumber{6}%
\task{%
    Оптическая сила объектива фотоаппарата равна $4\,\text{дптр}$.
    При фотографировании чертежа с расстояния $1{,}2\,\text{м}$ площадь изображения
    чертежа на фотопластинке оказалась равной $16\,\text{см}^{2}$.
    Какова площадь самого чертежа? Ответ выразите в квадратных сантиметрах.
}
\answer{%
    \begin{align*}
    &\frac 1a + \frac 1b = \frac 1F = D \implies b = \frac{aF}{a - F} \\
    &\frac {S'}S = \Gamma^2 = \sqr{\frac ba} = \sqr{\frac F{a - F}} \implies \\
    &\implies S = S' \cdot \sqr{\frac{a - F}F} = S' \cdot \sqr{\frac aF - 1} = S' \cdot \sqr{aD - 1} \approx 200\,\text{см}^{2}.
    \end{align*}
}


\variantsplitter


\addpersonalvariant{Евгений Васин}

\tasknumber{7}%
\task{%
    В каком месте на главной оптической оси двояковыпуклой линзы
    нужно поместить точечный источник света,
    чтобы его изображение оказалось в главном фокусе линзы?
}
\answer{%
    $\text{для мнимого - на половине фокусного, для действительного - на бесконечности}$
}
\solutionspace{120pt}

\tasknumber{8}%
\task{%
    Предмет в виде отрезка длиной $\ell$ расположен вдоль оптической оси
    собирающей линзы с фокусным расстоянием $F$.
    Середина отрезка расположена
    на расстоянии $a$ от линзы, которая даёт действительное изображение
    всех точек предмета.
    Определить продольное увеличение предмета.
}
\answer{%
    \begin{align*}
    \frac 1{a + \frac \ell 2} &+ \frac 1b = \frac 1F \implies b = \frac{F\cbr{a + \frac \ell 2}}{a + \frac \ell 2 - F} \\
    \frac 1{a - \frac \ell 2} &+ \frac 1c = \frac 1F \implies c = \frac{F\cbr{a - \frac \ell 2}}{a - \frac \ell 2 - F} \\
    \abs{b - c} &= \abs{\frac{F\cbr{a + \frac \ell 2}}{a + \frac \ell 2 - F} - \frac{F\cbr{a - \frac \ell 2}}{a - \frac \ell 2 - F}}= F\abs{\frac{\cbr{a + \frac \ell 2}\cbr{a - \frac \ell 2 - F} - \cbr{a - \frac \ell 2}\cbr{a + \frac \ell 2 - F}}{ \cbr{a + \frac \ell 2 - F} \cbr{a - \frac \ell 2 - F} }} =  \\
    &= F\abs{\frac{a^2 - \frac {a\ell} 2 - Fa + \frac {a\ell} 2 - \frac {\ell^2} 4 - \frac {F\ell}2 - a^2 - \frac {a\ell}2 + aF + \frac {a\ell}2 + \frac {\ell^2} 4 - \frac {F\ell} 2}{\cbr{a + \frac \ell 2 - F} \cbr{a - \frac \ell 2 - F} }} = \\
    &= F\frac{F\ell}{\sqr{a-F} - \frac {\ell^2}4} = \frac{F^2\ell}{\sqr{a-F} - \frac {\ell^2}4}\implies \Gamma = \frac{\abs{b - c}}\ell = \frac{F^2}{\sqr{a-F} - \frac {\ell^2}4}.
    \end{align*}
}
\solutionspace{120pt}

\tasknumber{9}%
\task{%
    На экране с помощью тонкой линзы получено изображение предмета
    с увеличением $2$.
    Предмет передвинули на $10\,\text{см}$.
    Для того, чтобы получить резкое изображение, пришлось передвинуть экран.
    При этом увеличение оказалось равным $6$.
    На какое расстояние
    пришлось передвинуть экран?
}
\answer{%
    \begin{align*}
    &\frac 1a + \frac 1b = \frac 1F, \Gamma_1 = \frac ba = \frac{F}{a-F} \implies \Gamma_1(a-F) = F \implies a = F \cdot \frac{1 + \Gamma_1}{\Gamma_1} \\
    &\frac 1{a + x} + \frac 1{b + y} = \frac 1F, \Gamma_2 = \frac {b+y}{a+x} = \frac{F}{a+x-F} \implies a + x = F \cdot \frac{1 + \Gamma_2}{\Gamma_2} \\
    &1 + \frac xa = \frac{ \frac{1 + \Gamma_2}{\Gamma_2} }{ \frac{1 + \Gamma_1}{\Gamma_1} } = \frac{\Gamma_1(1 + \Gamma_2)}{\Gamma_2(1 + \Gamma_1)} \\
    &a = \frac x{ \frac{\Gamma_1(1 + \Gamma_2)}{\Gamma_2(1 + \Gamma_1)} - 1} = x \cdot \frac{\Gamma_2(1 + \Gamma_1)}{\Gamma_1 - \Gamma_2} \\
    &y = (a + x)\Gamma_2 - b = (a + x)\Gamma_2 - a\Gamma_1 = a(\Gamma_2 - \Gamma_1) + x\Gamma_2 = -x\Gamma_2(1 + \Gamma_1) + x\Gamma_2 = -x\Gamma_2\Gamma_1 = 120\,\text{см}, \\
    &\text{знаки разные, т.е.
    экран надо было подвинуть в другую сторону чем предмет: $x < 0, y > 0$.}
    \end{align*}
}
\solutionspace{120pt}

\tasknumber{10}%
\task{%
    Тонкая собирающая линза дает изображение предмета на экране при двух положениях линзы между предметом и экраном.
    Высота изображения при первом положении $20\,\text{см}$, во втором — $9\,\text{см}$.
    Расстояние между предметом и экранов постоянно.
    Чему равна высота предмета?
}
\answer{%
    \begin{align*}
    &\frac 1a + \frac 1b = \frac 1F, \frac 1c + \frac 1d = \frac 1F, a + b = c + d \implies \frac{a + b}{ab} = \frac 1F = \frac{c+d}{cd} \implies ab = cd, \\
    &\implies ab = c(a + b - c) \implies c^2 - ac - bc + ab = 0 \implies c = a \text{ или } c = b \implies c = b \implies d = a.
    \\
    &\Gamma_1 = \frac {H_1}H = \frac ba, \Gamma_2 = \frac {H_2}H = \frac dc = \frac ab \implies \frac {H_1}H \cdot \frac {H_2}H = \frac ba \cdot \frac ab = 1, \\
    &H = \sqrt{H_1 H_2} \approx 13{,}4\,\text{см}.
    \end{align*}
}
\solutionspace{120pt}

\tasknumber{11}%
\task{%
    Какие предметы можно рассмотреть на фотографии, сделанной со спутника,
    если разрешающая способность плёнки $0{,}010\,\text{мм}$? Каким должно быть
    время экспозиции $\tau$ чтобы полностью использовать возможности плёнки?
    Фокусное расстояние объектива используемого фотоаппарата $20\,\text{см}$,
    высота орбиты спутника $80\,\text{км}$.
}
\answer{%
    \begin{align*}
    &H \ll R \implies v = v_{\text{I}} = \sqrt{G R} \approx 7{,}9\,\frac{\text{км}}{\text{с}}.
    \\
    &F \ll H \implies b = F, a = H, \\
    &\Gamma = \frac \delta\ell = \frac ba \implies \ell = \frac{\delta a}b = \frac{\delta H}F \approx \frac{0{,}010\,\text{мм} \cdot 80\,\text{км}}{20\,\text{см}} \approx 4\,\text{м}, \\
    &\implies \tau = \frac \ell v = \frac{\delta H}{F v} = \frac{0{,}010\,\text{мм} \cdot 80\,\text{км}}{20\,\text{см} \cdot 7{,}9\,\frac{\text{км}}{\text{с}}} \approx 0{,}5\,\text{мс}.
    \end{align*}
}


\variantsplitter


\addpersonalvariant{Евгений Васин}

\tasknumber{12}%
\task{%
    При аэрофотосъемках используется фотоаппарат, объектив которого
    имеет фокусиое расстояние $20\,\text{см}$.
    Разрешающая способность плёнки $0{,}015\,\text{мм}$.
    На какой высоте должен лететь самолет, чтобы на фотографии можно
    было различить следы размером $20\,\text{см}$?
    При какой скорости самолета изображение не будет размытым,
    если время экспозиции $1\,\text{мс}$?
}
\answer{%
    \begin{align*}
    &F \ll H \implies b = F, a = H, \\
    &\Gamma = \frac \delta\ell = \frac ba = \frac FH \implies H = \frac{\ell F}\delta = \frac{20\,\text{см} \cdot 20\,\text{см}}{0{,}015\,\text{мм}} \approx 3\,\text{км}, \\
    &\implies v = \frac l\tau = \frac{20\,\text{см}}{1\,\text{мс}} \approx 700\,\frac{\text{км}}{\text{ч}}.
    \end{align*}
}
\solutionspace{120pt}

\tasknumber{13}%
\task{%
    Две одинаковые собиращие линзы установлены так, что их главные оптические оси совпадают,
    а главный фокус первой находится там же, где главный фокус второй.
    Расстояние от первой линзы до предмета равно $33\,\text{см}$.
    Чему равно расстояние от изображения объекта во второй линзе до второй линзы?
    Определите также увеличение.
    Фокусное расстояние каждой линзы $40\,\text{см}$.
}
\answer{%
    \begin{align*}
    \frac 1a + \frac 1b &= \frac 1F \implies b = \frac{aF}{a - F} \implies 2F - b = \frac{2aF - 2F^2 - aF}{a - F} = \frac{F(a - 2F)}{a - F}.
    \\
    \frac 1{2F - b} + \frac 1c &= \frac 1F \implies c = \frac{F(2F-b)}{(2F - b) - F} = \frac{F \cdot \frac{F(a - 2F)}{a - F}}{\frac{F(a - 2F)}{a - F} - F}  = F \cdot \frac{ \frac{F(a - 2F)}{a - F} }{ \frac{F(a - 2F)}{a - F} - 1} = \\
     &= F \cdot \frac{a - 2F}{a - 2F - a + F} = 2F - a = 47\,\text{см}.
     \\
    \ell &= a + 2F + c = 4F = 160\,\text{см}.
    \\
    &\Gamma = \Gamma_1 \cdot \Gamma_2 = \frac ba \cdot \frac c{2F-b} = \frac F{a - F} \cdot \frac{2F - a}{\frac{F(a - 2F)}{a - F}} = -1.
    \end{align*}
}
\solutionspace{120pt}

\tasknumber{14}%
\task{%
    Собирающая линза с фокусным расстоянием $F_1 > 0$ и рассеивающая линза с фокусным расстоянием $F_2 < 0$
    установлены коаксиально на расстоянии $\ell$.
    Пучок параллельных лучей падает на собирающую линзу.
    Сделайте схематичное построение и определите, в какой точке система из этих линз соберёт пучок.
}
\answer{%
    \begin{align*}
    &\text{Если пучок падает на собирающую линзу:} \\
    \frac 1{\infty} + \frac 1b &= \frac 1{F_1} \implies b = F_1 \implies \ell - b = \ell - F_1 \\
    \frac 1{\ell - b} + \frac 1c &= \frac 1{F_2} \implies c = \frac{F_2(\ell - b)}{\ell - b - F_2} = \frac{F_2(\ell - F_1)}{\ell - F_1 - F_2}.
    \\
    &\text{Если же пучок падает на рассеивающую линзу:} \\
    \frac 1{\infty} + \frac 1b &= \frac 1{F_2} \implies b = F_2 \implies \ell - b = \ell - F_2 \\
    \frac 1{\ell - b} + \frac 1c &= \frac 1{F_1} \implies c = \frac{F_1(\ell - b)}{\ell - b - F_1} = \frac{F_1(\ell - F_2)}{\ell - F_2 - F_1}.
    \end{align*}
}
\solutionspace{120pt}

\tasknumber{15}%
\task{%
    Две собирающих линзы с фокусными расстояниями $20\,\text{см}$ и $35\,\text{см}$ расположены так,
    что их оптические оси совмещены.
    На первую линзу падает пучок параллельных лучей.
    Пройдя через вторую линзу, он остался параллельным.
    Найдите расстояние между линзами и сделайте рисунок.
}
\answer{%
    \begin{align*}
    \frac 1\infty + \frac 1b &= \frac 1{F_1} \implies b = F_1, \\
    \frac 1{\ell - b} + \frac 1{\infty} &= \frac 1{F_2} \implies \ell - b = F_2 \implies \ell = b + F_2 = F_1 + F_2 = 55\,\text{см}.
    \end{align*}
}

\variantsplitter

\addpersonalvariant{Вячеслав Волохов}

\tasknumber{1}%
\task{%
    Найти оптическую силу собирающей линзы, если действительное изображение предмета,
    помещённого в $15\,\text{см}$ от линзы, получается на расстоянии $30\,\text{см}$ от неё.
}
\answer{%
    $D = \frac 1F = \frac 1a + \frac 1b = \frac 1{15\,\text{см}} + \frac 1{30\,\text{см}} \approx 10\,\text{дптр}$
}
\solutionspace{80pt}

\tasknumber{2}%
\task{%
    Найти увеличение изображения, если изображение предмета, находящегося
    на расстоянии $20\,\text{см}$ от линзы, получается на расстоянии $30\,\text{см}$ от неё.
}
\answer{%
    $\Gamma = \frac ba = \frac {30\,\text{см}}{20\,\text{см}} \approx 1{,}50$
}
\solutionspace{80pt}

\tasknumber{3}%
\task{%
    Расстояние от предмета до линзы $10\,\text{см}$, а от линзы до мнимого изображения $30\,\text{см}$.
    Чему равно фокусное расстояние линзы?
}
\answer{%
    $\pm \frac 1F = \frac 1a - \frac 1b \implies F = \frac{a b}{\abs{b - a}} \approx 15\,\text{см}$
}
\solutionspace{80pt}

\tasknumber{4}%
\task{%
    Две тонкие собирающие линзы с фокусными расстояниями $12\,\text{см}$ и $20\,\text{см}$ сложены вместе.
    Чему равно фокусное расстояние такой оптической системы?
}
\answer{%
    $\frac 1{f_1} = \frac 1a + \frac 1b; \frac 1{f_2} = - \frac 1b + \frac 1c \implies \frac 1{f_1} + \frac 1{f_2} = \frac 1a + \frac 1c \implies f' = \frac 1{\frac 1{f_1} + \frac 1{f_2}} = \frac{f_1 f_2}{f_1 + f_2} \approx 7{,}5\,\text{см}$
}
\solutionspace{80pt}

\tasknumber{5}%
\task{%
    Линейные размеры прямого изображения предмета, полученного в собирающей линзе,
    в три раза больше линейных размеров предмета.
    Зная, что предмет находится на $25\,\text{см}$ ближе к линзе,
    чем его изображение, найти оптическую силу линзы.
}
\answer{%
    \begin{align*}
    &\text{Если изображение действительное:} \\
    D &= \frac 1F = \frac 1a + \frac 1b, \qquad \Gamma = \frac ba, \qquad b - a = \ell \implies b = \Gamma a \implies \Gamma a - a = \ell \implies  \\
    a &= \frac {\ell}{\Gamma - 1} \implies b = \frac {{\ell} \Gamma}{\Gamma - 1} \implies  \\
    D &= \frac {\Gamma - 1}\ell + \frac {\Gamma - 1}{\ell \Gamma} = \frac 1\ell \cdot \cbr{\Gamma - 1 + \frac {\Gamma - 1}{\Gamma} } =\frac 1\ell \cdot \cbr{\Gamma - \frac 1\Gamma} \approx 10{,}7\,\text{дптр}.
    \\
    &\text{Если изображение мнимое:} \\
    D &= \frac 1F = \frac 1a - \frac 1b, \qquad \Gamma = \frac ba, \qquad b - a = \ell \implies b = \Gamma a \implies \Gamma a - a = \ell \implies  \\
    a &= \frac {\ell}{\Gamma - 1} \implies b = \frac {{\ell} \Gamma}{\Gamma - 1} \implies  \\
    D &= \frac {\Gamma - 1}\ell - \frac {\Gamma - 1}{\ell \Gamma} = \frac 1\ell \cdot \cbr{\Gamma - 1 - \frac {\Gamma - 1}{\Gamma} } =\frac 1\ell \cdot \cbr{\Gamma + \frac 1\Gamma - 2} \approx 5{,}3\,\text{дптр}.
    \\
    &\text{В ответе надо указать оба значения.}
    \end{align*}
}
\solutionspace{120pt}

\tasknumber{6}%
\task{%
    Оптическая сила объектива фотоаппарата равна $5\,\text{дптр}$.
    При фотографировании чертежа с расстояния $1{,}2\,\text{м}$ площадь изображения
    чертежа на фотопластинке оказалась равной $9\,\text{см}^{2}$.
    Какова площадь самого чертежа? Ответ выразите в квадратных сантиметрах.
}
\answer{%
    \begin{align*}
    &\frac 1a + \frac 1b = \frac 1F = D \implies b = \frac{aF}{a - F} \\
    &\frac {S'}S = \Gamma^2 = \sqr{\frac ba} = \sqr{\frac F{a - F}} \implies \\
    &\implies S = S' \cdot \sqr{\frac{a - F}F} = S' \cdot \sqr{\frac aF - 1} = S' \cdot \sqr{aD - 1} \approx 225\,\text{см}^{2}.
    \end{align*}
}


\variantsplitter


\addpersonalvariant{Вячеслав Волохов}

\tasknumber{7}%
\task{%
    В каком месте на главной оптической оси двояковыпуклой линзы
    нужно поместить точечный источник света,
    чтобы его изображение оказалось в главном фокусе линзы?
}
\answer{%
    $\text{для мнимого - на половине фокусного, для действительного - на бесконечности}$
}
\solutionspace{120pt}

\tasknumber{8}%
\task{%
    Предмет в виде отрезка длиной $\ell$ расположен вдоль оптической оси
    собирающей линзы с фокусным расстоянием $F$.
    Середина отрезка расположена
    на расстоянии $a$ от линзы, которая даёт действительное изображение
    всех точек предмета.
    Определить продольное увеличение предмета.
}
\answer{%
    \begin{align*}
    \frac 1{a + \frac \ell 2} &+ \frac 1b = \frac 1F \implies b = \frac{F\cbr{a + \frac \ell 2}}{a + \frac \ell 2 - F} \\
    \frac 1{a - \frac \ell 2} &+ \frac 1c = \frac 1F \implies c = \frac{F\cbr{a - \frac \ell 2}}{a - \frac \ell 2 - F} \\
    \abs{b - c} &= \abs{\frac{F\cbr{a + \frac \ell 2}}{a + \frac \ell 2 - F} - \frac{F\cbr{a - \frac \ell 2}}{a - \frac \ell 2 - F}}= F\abs{\frac{\cbr{a + \frac \ell 2}\cbr{a - \frac \ell 2 - F} - \cbr{a - \frac \ell 2}\cbr{a + \frac \ell 2 - F}}{ \cbr{a + \frac \ell 2 - F} \cbr{a - \frac \ell 2 - F} }} =  \\
    &= F\abs{\frac{a^2 - \frac {a\ell} 2 - Fa + \frac {a\ell} 2 - \frac {\ell^2} 4 - \frac {F\ell}2 - a^2 - \frac {a\ell}2 + aF + \frac {a\ell}2 + \frac {\ell^2} 4 - \frac {F\ell} 2}{\cbr{a + \frac \ell 2 - F} \cbr{a - \frac \ell 2 - F} }} = \\
    &= F\frac{F\ell}{\sqr{a-F} - \frac {\ell^2}4} = \frac{F^2\ell}{\sqr{a-F} - \frac {\ell^2}4}\implies \Gamma = \frac{\abs{b - c}}\ell = \frac{F^2}{\sqr{a-F} - \frac {\ell^2}4}.
    \end{align*}
}
\solutionspace{120pt}

\tasknumber{9}%
\task{%
    На экране с помощью тонкой линзы получено изображение предмета
    с увеличением $4$.
    Предмет передвинули на $4\,\text{см}$.
    Для того, чтобы получить резкое изображение, пришлось передвинуть экран.
    При этом увеличение оказалось равным $8$.
    На какое расстояние
    пришлось передвинуть экран?
}
\answer{%
    \begin{align*}
    &\frac 1a + \frac 1b = \frac 1F, \Gamma_1 = \frac ba = \frac{F}{a-F} \implies \Gamma_1(a-F) = F \implies a = F \cdot \frac{1 + \Gamma_1}{\Gamma_1} \\
    &\frac 1{a + x} + \frac 1{b + y} = \frac 1F, \Gamma_2 = \frac {b+y}{a+x} = \frac{F}{a+x-F} \implies a + x = F \cdot \frac{1 + \Gamma_2}{\Gamma_2} \\
    &1 + \frac xa = \frac{ \frac{1 + \Gamma_2}{\Gamma_2} }{ \frac{1 + \Gamma_1}{\Gamma_1} } = \frac{\Gamma_1(1 + \Gamma_2)}{\Gamma_2(1 + \Gamma_1)} \\
    &a = \frac x{ \frac{\Gamma_1(1 + \Gamma_2)}{\Gamma_2(1 + \Gamma_1)} - 1} = x \cdot \frac{\Gamma_2(1 + \Gamma_1)}{\Gamma_1 - \Gamma_2} \\
    &y = (a + x)\Gamma_2 - b = (a + x)\Gamma_2 - a\Gamma_1 = a(\Gamma_2 - \Gamma_1) + x\Gamma_2 = -x\Gamma_2(1 + \Gamma_1) + x\Gamma_2 = -x\Gamma_2\Gamma_1 = 128\,\text{см}, \\
    &\text{знаки разные, т.е.
    экран надо было подвинуть в другую сторону чем предмет: $x < 0, y > 0$.}
    \end{align*}
}
\solutionspace{120pt}

\tasknumber{10}%
\task{%
    Тонкая собирающая линза дает изображение предмета на экране при двух положениях линзы между предметом и экраном.
    Высота изображения при первом положении $25\,\text{см}$, во втором — $9\,\text{см}$.
    Расстояние между предметом и экранов постоянно.
    Чему равна высота предмета?
}
\answer{%
    \begin{align*}
    &\frac 1a + \frac 1b = \frac 1F, \frac 1c + \frac 1d = \frac 1F, a + b = c + d \implies \frac{a + b}{ab} = \frac 1F = \frac{c+d}{cd} \implies ab = cd, \\
    &\implies ab = c(a + b - c) \implies c^2 - ac - bc + ab = 0 \implies c = a \text{ или } c = b \implies c = b \implies d = a.
    \\
    &\Gamma_1 = \frac {H_1}H = \frac ba, \Gamma_2 = \frac {H_2}H = \frac dc = \frac ab \implies \frac {H_1}H \cdot \frac {H_2}H = \frac ba \cdot \frac ab = 1, \\
    &H = \sqrt{H_1 H_2} \approx 15\,\text{см}.
    \end{align*}
}
\solutionspace{120pt}

\tasknumber{11}%
\task{%
    Какие предметы можно рассмотреть на фотографии, сделанной со спутника,
    если разрешающая способность плёнки $0{,}010\,\text{мм}$? Каким должно быть
    время экспозиции $\tau$ чтобы полностью использовать возможности плёнки?
    Фокусное расстояние объектива используемого фотоаппарата $10\,\text{см}$,
    высота орбиты спутника $120\,\text{км}$.
}
\answer{%
    \begin{align*}
    &H \ll R \implies v = v_{\text{I}} = \sqrt{G R} \approx 7{,}9\,\frac{\text{км}}{\text{с}}.
    \\
    &F \ll H \implies b = F, a = H, \\
    &\Gamma = \frac \delta\ell = \frac ba \implies \ell = \frac{\delta a}b = \frac{\delta H}F \approx \frac{0{,}010\,\text{мм} \cdot 120\,\text{км}}{10\,\text{см}} \approx 12\,\text{м}, \\
    &\implies \tau = \frac \ell v = \frac{\delta H}{F v} = \frac{0{,}010\,\text{мм} \cdot 120\,\text{км}}{10\,\text{см} \cdot 7{,}9\,\frac{\text{км}}{\text{с}}} \approx 1{,}5\,\text{мс}.
    \end{align*}
}


\variantsplitter


\addpersonalvariant{Вячеслав Волохов}

\tasknumber{12}%
\task{%
    При аэрофотосъемках используется фотоаппарат, объектив которого
    имеет фокусиое расстояние $10\,\text{см}$.
    Разрешающая способность плёнки $0{,}010\,\text{мм}$.
    На какой высоте должен лететь самолет, чтобы на фотографии можно
    было различить следы размером $20\,\text{см}$?
    При какой скорости самолета изображение не будет размытым,
    если время экспозиции $2\,\text{мс}$?
}
\answer{%
    \begin{align*}
    &F \ll H \implies b = F, a = H, \\
    &\Gamma = \frac \delta\ell = \frac ba = \frac FH \implies H = \frac{\ell F}\delta = \frac{20\,\text{см} \cdot 10\,\text{см}}{0{,}010\,\text{мм}} \approx 2\,\text{км}, \\
    &\implies v = \frac l\tau = \frac{20\,\text{см}}{2\,\text{мс}} \approx 360\,\frac{\text{км}}{\text{ч}}.
    \end{align*}
}
\solutionspace{120pt}

\tasknumber{13}%
\task{%
    Две одинаковые собиращие линзы установлены так, что их главные оптические оси совпадают,
    а главный фокус первой находится там же, где главный фокус второй.
    Расстояние от первой линзы до предмета равно $5\,\text{см}$.
    Чему равно расстояние от изображения объекта во второй линзе до самого объекта?
    Определите также увеличение.
    Фокусное расстояние каждой линзы $25\,\text{см}$.
}
\answer{%
    \begin{align*}
    \frac 1a + \frac 1b &= \frac 1F \implies b = \frac{aF}{a - F} \implies 2F - b = \frac{2aF - 2F^2 - aF}{a - F} = \frac{F(a - 2F)}{a - F}.
    \\
    \frac 1{2F - b} + \frac 1c &= \frac 1F \implies c = \frac{F(2F-b)}{(2F - b) - F} = \frac{F \cdot \frac{F(a - 2F)}{a - F}}{\frac{F(a - 2F)}{a - F} - F}  = F \cdot \frac{ \frac{F(a - 2F)}{a - F} }{ \frac{F(a - 2F)}{a - F} - 1} = \\
     &= F \cdot \frac{a - 2F}{a - 2F - a + F} = 2F - a = 45\,\text{см}.
     \\
    \ell &= a + 2F + c = 4F = 100\,\text{см}.
    \\
    &\Gamma = \Gamma_1 \cdot \Gamma_2 = \frac ba \cdot \frac c{2F-b} = \frac F{a - F} \cdot \frac{2F - a}{\frac{F(a - 2F)}{a - F}} = -1.
    \end{align*}
}
\solutionspace{120pt}

\tasknumber{14}%
\task{%
    Собирающая линза с фокусным расстоянием $F_1 > 0$ и рассеивающая линза с фокусным расстоянием $F_2 < 0$
    установлены коаксиально на расстоянии $\ell$.
    Пучок параллельных лучей падает на собирающую линзу.
    Сделайте схематичное построение и определите, в какой точке система из этих линз соберёт пучок.
}
\answer{%
    \begin{align*}
    &\text{Если пучок падает на собирающую линзу:} \\
    \frac 1{\infty} + \frac 1b &= \frac 1{F_1} \implies b = F_1 \implies \ell - b = \ell - F_1 \\
    \frac 1{\ell - b} + \frac 1c &= \frac 1{F_2} \implies c = \frac{F_2(\ell - b)}{\ell - b - F_2} = \frac{F_2(\ell - F_1)}{\ell - F_1 - F_2}.
    \\
    &\text{Если же пучок падает на рассеивающую линзу:} \\
    \frac 1{\infty} + \frac 1b &= \frac 1{F_2} \implies b = F_2 \implies \ell - b = \ell - F_2 \\
    \frac 1{\ell - b} + \frac 1c &= \frac 1{F_1} \implies c = \frac{F_1(\ell - b)}{\ell - b - F_1} = \frac{F_1(\ell - F_2)}{\ell - F_2 - F_1}.
    \end{align*}
}
\solutionspace{120pt}

\tasknumber{15}%
\task{%
    Две собирающих линзы с фокусными расстояниями $40\,\text{см}$ и $35\,\text{см}$ расположены так,
    что их оптические оси совмещены.
    На первую линзу падает пучок параллельных лучей.
    Пройдя через вторую линзу, он остался параллельным.
    Найдите расстояние между линзами и сделайте рисунок.
}
\answer{%
    \begin{align*}
    \frac 1\infty + \frac 1b &= \frac 1{F_1} \implies b = F_1, \\
    \frac 1{\ell - b} + \frac 1{\infty} &= \frac 1{F_2} \implies \ell - b = F_2 \implies \ell = b + F_2 = F_1 + F_2 = 75\,\text{см}.
    \end{align*}
}

\variantsplitter

\addpersonalvariant{Герман Говоров}

\tasknumber{1}%
\task{%
    Найти оптическую силу собирающей линзы, если действительное изображение предмета,
    помещённого в $35\,\text{см}$ от линзы, получается на расстоянии $20\,\text{см}$ от неё.
}
\answer{%
    $D = \frac 1F = \frac 1a + \frac 1b = \frac 1{35\,\text{см}} + \frac 1{20\,\text{см}} \approx 7{,}86\,\text{дптр}$
}
\solutionspace{80pt}

\tasknumber{2}%
\task{%
    Найти увеличение изображения, если изображение предмета, находящегося
    на расстоянии $25\,\text{см}$ от линзы, получается на расстоянии $18\,\text{см}$ от неё.
}
\answer{%
    $\Gamma = \frac ba = \frac {18\,\text{см}}{25\,\text{см}} \approx 0{,}7$
}
\solutionspace{80pt}

\tasknumber{3}%
\task{%
    Расстояние от предмета до линзы $12\,\text{см}$, а от линзы до мнимого изображения $25\,\text{см}$.
    Чему равно фокусное расстояние линзы?
}
\answer{%
    $\pm \frac 1F = \frac 1a - \frac 1b \implies F = \frac{a b}{\abs{b - a}} \approx 23{,}1\,\text{см}$
}
\solutionspace{80pt}

\tasknumber{4}%
\task{%
    Две тонкие собирающие линзы с фокусными расстояниями $18\,\text{см}$ и $30\,\text{см}$ сложены вместе.
    Чему равно фокусное расстояние такой оптической системы?
}
\answer{%
    $\frac 1{f_1} = \frac 1a + \frac 1b; \frac 1{f_2} = - \frac 1b + \frac 1c \implies \frac 1{f_1} + \frac 1{f_2} = \frac 1a + \frac 1c \implies f' = \frac 1{\frac 1{f_1} + \frac 1{f_2}} = \frac{f_1 f_2}{f_1 + f_2} \approx 11{,}2\,\text{см}$
}
\solutionspace{80pt}

\tasknumber{5}%
\task{%
    Линейные размеры прямого изображения предмета, полученного в собирающей линзе,
    в три раза больше линейных размеров предмета.
    Зная, что предмет находится на $35\,\text{см}$ ближе к линзе,
    чем его изображение, найти оптическую силу линзы.
}
\answer{%
    \begin{align*}
    &\text{Если изображение действительное:} \\
    D &= \frac 1F = \frac 1a + \frac 1b, \qquad \Gamma = \frac ba, \qquad b - a = \ell \implies b = \Gamma a \implies \Gamma a - a = \ell \implies  \\
    a &= \frac {\ell}{\Gamma - 1} \implies b = \frac {{\ell} \Gamma}{\Gamma - 1} \implies  \\
    D &= \frac {\Gamma - 1}\ell + \frac {\Gamma - 1}{\ell \Gamma} = \frac 1\ell \cdot \cbr{\Gamma - 1 + \frac {\Gamma - 1}{\Gamma} } =\frac 1\ell \cdot \cbr{\Gamma - \frac 1\Gamma} \approx 7{,}6\,\text{дптр}.
    \\
    &\text{Если изображение мнимое:} \\
    D &= \frac 1F = \frac 1a - \frac 1b, \qquad \Gamma = \frac ba, \qquad b - a = \ell \implies b = \Gamma a \implies \Gamma a - a = \ell \implies  \\
    a &= \frac {\ell}{\Gamma - 1} \implies b = \frac {{\ell} \Gamma}{\Gamma - 1} \implies  \\
    D &= \frac {\Gamma - 1}\ell - \frac {\Gamma - 1}{\ell \Gamma} = \frac 1\ell \cdot \cbr{\Gamma - 1 - \frac {\Gamma - 1}{\Gamma} } =\frac 1\ell \cdot \cbr{\Gamma + \frac 1\Gamma - 2} \approx 3{,}8\,\text{дптр}.
    \\
    &\text{В ответе надо указать оба значения.}
    \end{align*}
}
\solutionspace{120pt}

\tasknumber{6}%
\task{%
    Оптическая сила объектива фотоаппарата равна $5\,\text{дптр}$.
    При фотографировании чертежа с расстояния $1{,}2\,\text{м}$ площадь изображения
    чертежа на фотопластинке оказалась равной $4\,\text{см}^{2}$.
    Какова площадь самого чертежа? Ответ выразите в квадратных сантиметрах.
}
\answer{%
    \begin{align*}
    &\frac 1a + \frac 1b = \frac 1F = D \implies b = \frac{aF}{a - F} \\
    &\frac {S'}S = \Gamma^2 = \sqr{\frac ba} = \sqr{\frac F{a - F}} \implies \\
    &\implies S = S' \cdot \sqr{\frac{a - F}F} = S' \cdot \sqr{\frac aF - 1} = S' \cdot \sqr{aD - 1} \approx 100\,\text{см}^{2}.
    \end{align*}
}


\variantsplitter


\addpersonalvariant{Герман Говоров}

\tasknumber{7}%
\task{%
    В каком месте на главной оптической оси двояковыгнутой линзы
    нужно поместить точечный источник света,
    чтобы его изображение оказалось в главном фокусе линзы?
}
\answer{%
    $\text{на половине фокусного расстояния}$
}
\solutionspace{120pt}

\tasknumber{8}%
\task{%
    Предмет в виде отрезка длиной $\ell$ расположен вдоль оптической оси
    собирающей линзы с фокусным расстоянием $F$.
    Середина отрезка расположена
    на расстоянии $a$ от линзы, которая даёт действительное изображение
    всех точек предмета.
    Определить продольное увеличение предмета.
}
\answer{%
    \begin{align*}
    \frac 1{a + \frac \ell 2} &+ \frac 1b = \frac 1F \implies b = \frac{F\cbr{a + \frac \ell 2}}{a + \frac \ell 2 - F} \\
    \frac 1{a - \frac \ell 2} &+ \frac 1c = \frac 1F \implies c = \frac{F\cbr{a - \frac \ell 2}}{a - \frac \ell 2 - F} \\
    \abs{b - c} &= \abs{\frac{F\cbr{a + \frac \ell 2}}{a + \frac \ell 2 - F} - \frac{F\cbr{a - \frac \ell 2}}{a - \frac \ell 2 - F}}= F\abs{\frac{\cbr{a + \frac \ell 2}\cbr{a - \frac \ell 2 - F} - \cbr{a - \frac \ell 2}\cbr{a + \frac \ell 2 - F}}{ \cbr{a + \frac \ell 2 - F} \cbr{a - \frac \ell 2 - F} }} =  \\
    &= F\abs{\frac{a^2 - \frac {a\ell} 2 - Fa + \frac {a\ell} 2 - \frac {\ell^2} 4 - \frac {F\ell}2 - a^2 - \frac {a\ell}2 + aF + \frac {a\ell}2 + \frac {\ell^2} 4 - \frac {F\ell} 2}{\cbr{a + \frac \ell 2 - F} \cbr{a - \frac \ell 2 - F} }} = \\
    &= F\frac{F\ell}{\sqr{a-F} - \frac {\ell^2}4} = \frac{F^2\ell}{\sqr{a-F} - \frac {\ell^2}4}\implies \Gamma = \frac{\abs{b - c}}\ell = \frac{F^2}{\sqr{a-F} - \frac {\ell^2}4}.
    \end{align*}
}
\solutionspace{120pt}

\tasknumber{9}%
\task{%
    На экране с помощью тонкой линзы получено изображение предмета
    с увеличением $4$.
    Предмет передвинули на $10\,\text{см}$.
    Для того, чтобы получить резкое изображение, пришлось передвинуть экран.
    При этом увеличение оказалось равным $8$.
    На какое расстояние
    пришлось передвинуть экран?
}
\answer{%
    \begin{align*}
    &\frac 1a + \frac 1b = \frac 1F, \Gamma_1 = \frac ba = \frac{F}{a-F} \implies \Gamma_1(a-F) = F \implies a = F \cdot \frac{1 + \Gamma_1}{\Gamma_1} \\
    &\frac 1{a + x} + \frac 1{b + y} = \frac 1F, \Gamma_2 = \frac {b+y}{a+x} = \frac{F}{a+x-F} \implies a + x = F \cdot \frac{1 + \Gamma_2}{\Gamma_2} \\
    &1 + \frac xa = \frac{ \frac{1 + \Gamma_2}{\Gamma_2} }{ \frac{1 + \Gamma_1}{\Gamma_1} } = \frac{\Gamma_1(1 + \Gamma_2)}{\Gamma_2(1 + \Gamma_1)} \\
    &a = \frac x{ \frac{\Gamma_1(1 + \Gamma_2)}{\Gamma_2(1 + \Gamma_1)} - 1} = x \cdot \frac{\Gamma_2(1 + \Gamma_1)}{\Gamma_1 - \Gamma_2} \\
    &y = (a + x)\Gamma_2 - b = (a + x)\Gamma_2 - a\Gamma_1 = a(\Gamma_2 - \Gamma_1) + x\Gamma_2 = -x\Gamma_2(1 + \Gamma_1) + x\Gamma_2 = -x\Gamma_2\Gamma_1 = 320\,\text{см}, \\
    &\text{знаки разные, т.е.
    экран надо было подвинуть в другую сторону чем предмет: $x < 0, y > 0$.}
    \end{align*}
}
\solutionspace{120pt}

\tasknumber{10}%
\task{%
    Тонкая собирающая линза дает изображение предмета на экране при двух положениях линзы между предметом и экраном.
    Высота изображения при первом положении $20\,\text{см}$, во втором — $7\,\text{см}$.
    Расстояние между предметом и экранов постоянно.
    Чему равна высота предмета?
}
\answer{%
    \begin{align*}
    &\frac 1a + \frac 1b = \frac 1F, \frac 1c + \frac 1d = \frac 1F, a + b = c + d \implies \frac{a + b}{ab} = \frac 1F = \frac{c+d}{cd} \implies ab = cd, \\
    &\implies ab = c(a + b - c) \implies c^2 - ac - bc + ab = 0 \implies c = a \text{ или } c = b \implies c = b \implies d = a.
    \\
    &\Gamma_1 = \frac {H_1}H = \frac ba, \Gamma_2 = \frac {H_2}H = \frac dc = \frac ab \implies \frac {H_1}H \cdot \frac {H_2}H = \frac ba \cdot \frac ab = 1, \\
    &H = \sqrt{H_1 H_2} \approx 11{,}8\,\text{см}.
    \end{align*}
}
\solutionspace{120pt}

\tasknumber{11}%
\task{%
    Какие предметы можно рассмотреть на фотографии, сделанной со спутника,
    если разрешающая способность плёнки $0{,}010\,\text{мм}$? Каким должно быть
    время экспозиции $\tau$ чтобы полностью использовать возможности плёнки?
    Фокусное расстояние объектива используемого фотоаппарата $10\,\text{см}$,
    высота орбиты спутника $100\,\text{км}$.
}
\answer{%
    \begin{align*}
    &H \ll R \implies v = v_{\text{I}} = \sqrt{G R} \approx 7{,}9\,\frac{\text{км}}{\text{с}}.
    \\
    &F \ll H \implies b = F, a = H, \\
    &\Gamma = \frac \delta\ell = \frac ba \implies \ell = \frac{\delta a}b = \frac{\delta H}F \approx \frac{0{,}010\,\text{мм} \cdot 100\,\text{км}}{10\,\text{см}} \approx 10\,\text{м}, \\
    &\implies \tau = \frac \ell v = \frac{\delta H}{F v} = \frac{0{,}010\,\text{мм} \cdot 100\,\text{км}}{10\,\text{см} \cdot 7{,}9\,\frac{\text{км}}{\text{с}}} \approx 1{,}3\,\text{мс}.
    \end{align*}
}


\variantsplitter


\addpersonalvariant{Герман Говоров}

\tasknumber{12}%
\task{%
    При аэрофотосъемках используется фотоаппарат, объектив которого
    имеет фокусиое расстояние $15\,\text{см}$.
    Разрешающая способность плёнки $0{,}010\,\text{мм}$.
    На какой высоте должен лететь самолет, чтобы на фотографии можно
    было различить следы размером $15\,\text{см}$?
    При какой скорости самолета изображение не будет размытым,
    если время экспозиции $2\,\text{мс}$?
}
\answer{%
    \begin{align*}
    &F \ll H \implies b = F, a = H, \\
    &\Gamma = \frac \delta\ell = \frac ba = \frac FH \implies H = \frac{\ell F}\delta = \frac{15\,\text{см} \cdot 15\,\text{см}}{0{,}010\,\text{мм}} \approx 2\,\text{км}, \\
    &\implies v = \frac l\tau = \frac{15\,\text{см}}{2\,\text{мс}} \approx 270\,\frac{\text{км}}{\text{ч}}.
    \end{align*}
}
\solutionspace{120pt}

\tasknumber{13}%
\task{%
    Две одинаковые собиращие линзы установлены так, что их главные оптические оси совпадают,
    а главный фокус первой находится там же, где главный фокус второй.
    Расстояние от первой линзы до предмета равно $34\,\text{см}$.
    Чему равно расстояние от изображения объекта во второй линзе до самого объекта?
    Определите также увеличение.
    Фокусное расстояние каждой линзы $30\,\text{см}$.
}
\answer{%
    \begin{align*}
    \frac 1a + \frac 1b &= \frac 1F \implies b = \frac{aF}{a - F} \implies 2F - b = \frac{2aF - 2F^2 - aF}{a - F} = \frac{F(a - 2F)}{a - F}.
    \\
    \frac 1{2F - b} + \frac 1c &= \frac 1F \implies c = \frac{F(2F-b)}{(2F - b) - F} = \frac{F \cdot \frac{F(a - 2F)}{a - F}}{\frac{F(a - 2F)}{a - F} - F}  = F \cdot \frac{ \frac{F(a - 2F)}{a - F} }{ \frac{F(a - 2F)}{a - F} - 1} = \\
     &= F \cdot \frac{a - 2F}{a - 2F - a + F} = 2F - a = 26\,\text{см}.
     \\
    \ell &= a + 2F + c = 4F = 120\,\text{см}.
    \\
    &\Gamma = \Gamma_1 \cdot \Gamma_2 = \frac ba \cdot \frac c{2F-b} = \frac F{a - F} \cdot \frac{2F - a}{\frac{F(a - 2F)}{a - F}} = -1.
    \end{align*}
}
\solutionspace{120pt}

\tasknumber{14}%
\task{%
    Собирающая линза с фокусным расстоянием $F_1 > 0$ и рассеивающая линза с фокусным расстоянием $F_2 < 0$
    установлены коаксиально на расстоянии $\ell$.
    Пучок параллельных лучей падает на рассеивающую линзу.
    Сделайте схематичное построение и определите, в какой точке система из этих линз соберёт пучок.
}
\answer{%
    \begin{align*}
    &\text{Если пучок падает на собирающую линзу:} \\
    \frac 1{\infty} + \frac 1b &= \frac 1{F_1} \implies b = F_1 \implies \ell - b = \ell - F_1 \\
    \frac 1{\ell - b} + \frac 1c &= \frac 1{F_2} \implies c = \frac{F_2(\ell - b)}{\ell - b - F_2} = \frac{F_2(\ell - F_1)}{\ell - F_1 - F_2}.
    \\
    &\text{Если же пучок падает на рассеивающую линзу:} \\
    \frac 1{\infty} + \frac 1b &= \frac 1{F_2} \implies b = F_2 \implies \ell - b = \ell - F_2 \\
    \frac 1{\ell - b} + \frac 1c &= \frac 1{F_1} \implies c = \frac{F_1(\ell - b)}{\ell - b - F_1} = \frac{F_1(\ell - F_2)}{\ell - F_2 - F_1}.
    \end{align*}
}
\solutionspace{120pt}

\tasknumber{15}%
\task{%
    Две собирающих линзы с фокусными расстояниями $50\,\text{см}$ и $25\,\text{см}$ расположены так,
    что их оптические оси совмещены.
    На первую линзу падает пучок параллельных лучей.
    Пройдя через вторую линзу, он остался параллельным.
    Найдите расстояние между линзами и сделайте рисунок.
}
\answer{%
    \begin{align*}
    \frac 1\infty + \frac 1b &= \frac 1{F_1} \implies b = F_1, \\
    \frac 1{\ell - b} + \frac 1{\infty} &= \frac 1{F_2} \implies \ell - b = F_2 \implies \ell = b + F_2 = F_1 + F_2 = 75\,\text{см}.
    \end{align*}
}

\variantsplitter

\addpersonalvariant{София Журавлёва}

\tasknumber{1}%
\task{%
    Найти оптическую силу собирающей линзы, если действительное изображение предмета,
    помещённого в $35\,\text{см}$ от линзы, получается на расстоянии $40\,\text{см}$ от неё.
}
\answer{%
    $D = \frac 1F = \frac 1a + \frac 1b = \frac 1{35\,\text{см}} + \frac 1{40\,\text{см}} \approx 5{,}36\,\text{дптр}$
}
\solutionspace{80pt}

\tasknumber{2}%
\task{%
    Найти увеличение изображения, если изображение предмета, находящегося
    на расстоянии $20\,\text{см}$ от линзы, получается на расстоянии $30\,\text{см}$ от неё.
}
\answer{%
    $\Gamma = \frac ba = \frac {30\,\text{см}}{20\,\text{см}} \approx 1{,}50$
}
\solutionspace{80pt}

\tasknumber{3}%
\task{%
    Расстояние от предмета до линзы $10\,\text{см}$, а от линзы до мнимого изображения $30\,\text{см}$.
    Чему равно фокусное расстояние линзы?
}
\answer{%
    $\pm \frac 1F = \frac 1a - \frac 1b \implies F = \frac{a b}{\abs{b - a}} \approx 15\,\text{см}$
}
\solutionspace{80pt}

\tasknumber{4}%
\task{%
    Две тонкие собирающие линзы с фокусными расстояниями $18\,\text{см}$ и $30\,\text{см}$ сложены вместе.
    Чему равно фокусное расстояние такой оптической системы?
}
\answer{%
    $\frac 1{f_1} = \frac 1a + \frac 1b; \frac 1{f_2} = - \frac 1b + \frac 1c \implies \frac 1{f_1} + \frac 1{f_2} = \frac 1a + \frac 1c \implies f' = \frac 1{\frac 1{f_1} + \frac 1{f_2}} = \frac{f_1 f_2}{f_1 + f_2} \approx 11{,}2\,\text{см}$
}
\solutionspace{80pt}

\tasknumber{5}%
\task{%
    Линейные размеры прямого изображения предмета, полученного в собирающей линзе,
    в четыре раза больше линейных размеров предмета.
    Зная, что предмет находится на $20\,\text{см}$ ближе к линзе,
    чем его изображение, найти оптическую силу линзы.
}
\answer{%
    \begin{align*}
    &\text{Если изображение действительное:} \\
    D &= \frac 1F = \frac 1a + \frac 1b, \qquad \Gamma = \frac ba, \qquad b - a = \ell \implies b = \Gamma a \implies \Gamma a - a = \ell \implies  \\
    a &= \frac {\ell}{\Gamma - 1} \implies b = \frac {{\ell} \Gamma}{\Gamma - 1} \implies  \\
    D &= \frac {\Gamma - 1}\ell + \frac {\Gamma - 1}{\ell \Gamma} = \frac 1\ell \cdot \cbr{\Gamma - 1 + \frac {\Gamma - 1}{\Gamma} } =\frac 1\ell \cdot \cbr{\Gamma - \frac 1\Gamma} \approx 18{,}8\,\text{дптр}.
    \\
    &\text{Если изображение мнимое:} \\
    D &= \frac 1F = \frac 1a - \frac 1b, \qquad \Gamma = \frac ba, \qquad b - a = \ell \implies b = \Gamma a \implies \Gamma a - a = \ell \implies  \\
    a &= \frac {\ell}{\Gamma - 1} \implies b = \frac {{\ell} \Gamma}{\Gamma - 1} \implies  \\
    D &= \frac {\Gamma - 1}\ell - \frac {\Gamma - 1}{\ell \Gamma} = \frac 1\ell \cdot \cbr{\Gamma - 1 - \frac {\Gamma - 1}{\Gamma} } =\frac 1\ell \cdot \cbr{\Gamma + \frac 1\Gamma - 2} \approx 11{,}2\,\text{дптр}.
    \\
    &\text{В ответе надо указать оба значения.}
    \end{align*}
}
\solutionspace{120pt}

\tasknumber{6}%
\task{%
    Оптическая сила объектива фотоаппарата равна $3\,\text{дптр}$.
    При фотографировании чертежа с расстояния $1{,}2\,\text{м}$ площадь изображения
    чертежа на фотопластинке оказалась равной $9\,\text{см}^{2}$.
    Какова площадь самого чертежа? Ответ выразите в квадратных сантиметрах.
}
\answer{%
    \begin{align*}
    &\frac 1a + \frac 1b = \frac 1F = D \implies b = \frac{aF}{a - F} \\
    &\frac {S'}S = \Gamma^2 = \sqr{\frac ba} = \sqr{\frac F{a - F}} \implies \\
    &\implies S = S' \cdot \sqr{\frac{a - F}F} = S' \cdot \sqr{\frac aF - 1} = S' \cdot \sqr{aD - 1} \approx 60\,\text{см}^{2}.
    \end{align*}
}


\variantsplitter


\addpersonalvariant{София Журавлёва}

\tasknumber{7}%
\task{%
    В каком месте на главной оптической оси двояковыгнутой линзы
    нужно поместить точечный источник света,
    чтобы его изображение оказалось в главном фокусе линзы?
}
\answer{%
    $\text{на половине фокусного расстояния}$
}
\solutionspace{120pt}

\tasknumber{8}%
\task{%
    Предмет в виде отрезка длиной $\ell$ расположен вдоль оптической оси
    собирающей линзы с фокусным расстоянием $F$.
    Середина отрезка расположена
    на расстоянии $a$ от линзы, которая даёт действительное изображение
    всех точек предмета.
    Определить продольное увеличение предмета.
}
\answer{%
    \begin{align*}
    \frac 1{a + \frac \ell 2} &+ \frac 1b = \frac 1F \implies b = \frac{F\cbr{a + \frac \ell 2}}{a + \frac \ell 2 - F} \\
    \frac 1{a - \frac \ell 2} &+ \frac 1c = \frac 1F \implies c = \frac{F\cbr{a - \frac \ell 2}}{a - \frac \ell 2 - F} \\
    \abs{b - c} &= \abs{\frac{F\cbr{a + \frac \ell 2}}{a + \frac \ell 2 - F} - \frac{F\cbr{a - \frac \ell 2}}{a - \frac \ell 2 - F}}= F\abs{\frac{\cbr{a + \frac \ell 2}\cbr{a - \frac \ell 2 - F} - \cbr{a - \frac \ell 2}\cbr{a + \frac \ell 2 - F}}{ \cbr{a + \frac \ell 2 - F} \cbr{a - \frac \ell 2 - F} }} =  \\
    &= F\abs{\frac{a^2 - \frac {a\ell} 2 - Fa + \frac {a\ell} 2 - \frac {\ell^2} 4 - \frac {F\ell}2 - a^2 - \frac {a\ell}2 + aF + \frac {a\ell}2 + \frac {\ell^2} 4 - \frac {F\ell} 2}{\cbr{a + \frac \ell 2 - F} \cbr{a - \frac \ell 2 - F} }} = \\
    &= F\frac{F\ell}{\sqr{a-F} - \frac {\ell^2}4} = \frac{F^2\ell}{\sqr{a-F} - \frac {\ell^2}4}\implies \Gamma = \frac{\abs{b - c}}\ell = \frac{F^2}{\sqr{a-F} - \frac {\ell^2}4}.
    \end{align*}
}
\solutionspace{120pt}

\tasknumber{9}%
\task{%
    На экране с помощью тонкой линзы получено изображение предмета
    с увеличением $2$.
    Предмет передвинули на $4\,\text{см}$.
    Для того, чтобы получить резкое изображение, пришлось передвинуть экран.
    При этом увеличение оказалось равным $6$.
    На какое расстояние
    пришлось передвинуть экран?
}
\answer{%
    \begin{align*}
    &\frac 1a + \frac 1b = \frac 1F, \Gamma_1 = \frac ba = \frac{F}{a-F} \implies \Gamma_1(a-F) = F \implies a = F \cdot \frac{1 + \Gamma_1}{\Gamma_1} \\
    &\frac 1{a + x} + \frac 1{b + y} = \frac 1F, \Gamma_2 = \frac {b+y}{a+x} = \frac{F}{a+x-F} \implies a + x = F \cdot \frac{1 + \Gamma_2}{\Gamma_2} \\
    &1 + \frac xa = \frac{ \frac{1 + \Gamma_2}{\Gamma_2} }{ \frac{1 + \Gamma_1}{\Gamma_1} } = \frac{\Gamma_1(1 + \Gamma_2)}{\Gamma_2(1 + \Gamma_1)} \\
    &a = \frac x{ \frac{\Gamma_1(1 + \Gamma_2)}{\Gamma_2(1 + \Gamma_1)} - 1} = x \cdot \frac{\Gamma_2(1 + \Gamma_1)}{\Gamma_1 - \Gamma_2} \\
    &y = (a + x)\Gamma_2 - b = (a + x)\Gamma_2 - a\Gamma_1 = a(\Gamma_2 - \Gamma_1) + x\Gamma_2 = -x\Gamma_2(1 + \Gamma_1) + x\Gamma_2 = -x\Gamma_2\Gamma_1 = 48\,\text{см}, \\
    &\text{знаки разные, т.е.
    экран надо было подвинуть в другую сторону чем предмет: $x < 0, y > 0$.}
    \end{align*}
}
\solutionspace{120pt}

\tasknumber{10}%
\task{%
    Тонкая собирающая линза дает изображение предмета на экране при двух положениях линзы между предметом и экраном.
    Высота изображения при первом положении $30\,\text{см}$, во втором — $9\,\text{см}$.
    Расстояние между предметом и экранов постоянно.
    Чему равна высота предмета?
}
\answer{%
    \begin{align*}
    &\frac 1a + \frac 1b = \frac 1F, \frac 1c + \frac 1d = \frac 1F, a + b = c + d \implies \frac{a + b}{ab} = \frac 1F = \frac{c+d}{cd} \implies ab = cd, \\
    &\implies ab = c(a + b - c) \implies c^2 - ac - bc + ab = 0 \implies c = a \text{ или } c = b \implies c = b \implies d = a.
    \\
    &\Gamma_1 = \frac {H_1}H = \frac ba, \Gamma_2 = \frac {H_2}H = \frac dc = \frac ab \implies \frac {H_1}H \cdot \frac {H_2}H = \frac ba \cdot \frac ab = 1, \\
    &H = \sqrt{H_1 H_2} \approx 16{,}4\,\text{см}.
    \end{align*}
}
\solutionspace{120pt}

\tasknumber{11}%
\task{%
    Какие предметы можно рассмотреть на фотографии, сделанной со спутника,
    если разрешающая способность плёнки $0{,}010\,\text{мм}$? Каким должно быть
    время экспозиции $\tau$ чтобы полностью использовать возможности плёнки?
    Фокусное расстояние объектива используемого фотоаппарата $10\,\text{см}$,
    высота орбиты спутника $120\,\text{км}$.
}
\answer{%
    \begin{align*}
    &H \ll R \implies v = v_{\text{I}} = \sqrt{G R} \approx 7{,}9\,\frac{\text{км}}{\text{с}}.
    \\
    &F \ll H \implies b = F, a = H, \\
    &\Gamma = \frac \delta\ell = \frac ba \implies \ell = \frac{\delta a}b = \frac{\delta H}F \approx \frac{0{,}010\,\text{мм} \cdot 120\,\text{км}}{10\,\text{см}} \approx 12\,\text{м}, \\
    &\implies \tau = \frac \ell v = \frac{\delta H}{F v} = \frac{0{,}010\,\text{мм} \cdot 120\,\text{км}}{10\,\text{см} \cdot 7{,}9\,\frac{\text{км}}{\text{с}}} \approx 1{,}5\,\text{мс}.
    \end{align*}
}


\variantsplitter


\addpersonalvariant{София Журавлёва}

\tasknumber{12}%
\task{%
    При аэрофотосъемках используется фотоаппарат, объектив которого
    имеет фокусиое расстояние $20\,\text{см}$.
    Разрешающая способность плёнки $0{,}02\,\text{мм}$.
    На какой высоте должен лететь самолет, чтобы на фотографии можно
    было различить следы размером $25\,\text{см}$?
    При какой скорости самолета изображение не будет размытым,
    если время экспозиции $2\,\text{мс}$?
}
\answer{%
    \begin{align*}
    &F \ll H \implies b = F, a = H, \\
    &\Gamma = \frac \delta\ell = \frac ba = \frac FH \implies H = \frac{\ell F}\delta = \frac{25\,\text{см} \cdot 20\,\text{см}}{0{,}02\,\text{мм}} \approx 3\,\text{км}, \\
    &\implies v = \frac l\tau = \frac{25\,\text{см}}{2\,\text{мс}} \approx 450\,\frac{\text{км}}{\text{ч}}.
    \end{align*}
}
\solutionspace{120pt}

\tasknumber{13}%
\task{%
    Две одинаковые собиращие линзы установлены так, что их главные оптические оси совпадают,
    а главный фокус первой находится там же, где главный фокус второй.
    Расстояние от первой линзы до предмета равно $10\,\text{см}$.
    Чему равно расстояние от изображения объекта во второй линзе до второй линзы?
    Определите также увеличение.
    Фокусное расстояние каждой линзы $25\,\text{см}$.
}
\answer{%
    \begin{align*}
    \frac 1a + \frac 1b &= \frac 1F \implies b = \frac{aF}{a - F} \implies 2F - b = \frac{2aF - 2F^2 - aF}{a - F} = \frac{F(a - 2F)}{a - F}.
    \\
    \frac 1{2F - b} + \frac 1c &= \frac 1F \implies c = \frac{F(2F-b)}{(2F - b) - F} = \frac{F \cdot \frac{F(a - 2F)}{a - F}}{\frac{F(a - 2F)}{a - F} - F}  = F \cdot \frac{ \frac{F(a - 2F)}{a - F} }{ \frac{F(a - 2F)}{a - F} - 1} = \\
     &= F \cdot \frac{a - 2F}{a - 2F - a + F} = 2F - a = 40\,\text{см}.
     \\
    \ell &= a + 2F + c = 4F = 100\,\text{см}.
    \\
    &\Gamma = \Gamma_1 \cdot \Gamma_2 = \frac ba \cdot \frac c{2F-b} = \frac F{a - F} \cdot \frac{2F - a}{\frac{F(a - 2F)}{a - F}} = -1.
    \end{align*}
}
\solutionspace{120pt}

\tasknumber{14}%
\task{%
    Собирающая линза с фокусным расстоянием $F_1 > 0$ и рассеивающая линза с фокусным расстоянием $F_2 < 0$
    установлены коаксиально на расстоянии $\ell$.
    Пучок параллельных лучей падает на рассеивающую линзу.
    Сделайте схематичное построение и определите, в какой точке система из этих линз соберёт пучок.
}
\answer{%
    \begin{align*}
    &\text{Если пучок падает на собирающую линзу:} \\
    \frac 1{\infty} + \frac 1b &= \frac 1{F_1} \implies b = F_1 \implies \ell - b = \ell - F_1 \\
    \frac 1{\ell - b} + \frac 1c &= \frac 1{F_2} \implies c = \frac{F_2(\ell - b)}{\ell - b - F_2} = \frac{F_2(\ell - F_1)}{\ell - F_1 - F_2}.
    \\
    &\text{Если же пучок падает на рассеивающую линзу:} \\
    \frac 1{\infty} + \frac 1b &= \frac 1{F_2} \implies b = F_2 \implies \ell - b = \ell - F_2 \\
    \frac 1{\ell - b} + \frac 1c &= \frac 1{F_1} \implies c = \frac{F_1(\ell - b)}{\ell - b - F_1} = \frac{F_1(\ell - F_2)}{\ell - F_2 - F_1}.
    \end{align*}
}
\solutionspace{120pt}

\tasknumber{15}%
\task{%
    Две собирающих линзы с фокусными расстояниями $50\,\text{см}$ и $45\,\text{см}$ расположены так,
    что их оптические оси совмещены.
    На первую линзу падает пучок параллельных лучей.
    Пройдя через вторую линзу, он остался параллельным.
    Найдите расстояние между линзами и сделайте рисунок.
}
\answer{%
    \begin{align*}
    \frac 1\infty + \frac 1b &= \frac 1{F_1} \implies b = F_1, \\
    \frac 1{\ell - b} + \frac 1{\infty} &= \frac 1{F_2} \implies \ell - b = F_2 \implies \ell = b + F_2 = F_1 + F_2 = 95\,\text{см}.
    \end{align*}
}

\variantsplitter

\addpersonalvariant{Константин Козлов}

\tasknumber{1}%
\task{%
    Найти оптическую силу собирающей линзы, если действительное изображение предмета,
    помещённого в $15\,\text{см}$ от линзы, получается на расстоянии $40\,\text{см}$ от неё.
}
\answer{%
    $D = \frac 1F = \frac 1a + \frac 1b = \frac 1{15\,\text{см}} + \frac 1{40\,\text{см}} \approx 9{,}17\,\text{дптр}$
}
\solutionspace{80pt}

\tasknumber{2}%
\task{%
    Найти увеличение изображения, если изображение предмета, находящегося
    на расстоянии $25\,\text{см}$ от линзы, получается на расстоянии $30\,\text{см}$ от неё.
}
\answer{%
    $\Gamma = \frac ba = \frac {30\,\text{см}}{25\,\text{см}} \approx 1{,}20$
}
\solutionspace{80pt}

\tasknumber{3}%
\task{%
    Расстояние от предмета до линзы $12\,\text{см}$, а от линзы до мнимого изображения $30\,\text{см}$.
    Чему равно фокусное расстояние линзы?
}
\answer{%
    $\pm \frac 1F = \frac 1a - \frac 1b \implies F = \frac{a b}{\abs{b - a}} \approx 20\,\text{см}$
}
\solutionspace{80pt}

\tasknumber{4}%
\task{%
    Две тонкие собирающие линзы с фокусными расстояниями $18\,\text{см}$ и $30\,\text{см}$ сложены вместе.
    Чему равно фокусное расстояние такой оптической системы?
}
\answer{%
    $\frac 1{f_1} = \frac 1a + \frac 1b; \frac 1{f_2} = - \frac 1b + \frac 1c \implies \frac 1{f_1} + \frac 1{f_2} = \frac 1a + \frac 1c \implies f' = \frac 1{\frac 1{f_1} + \frac 1{f_2}} = \frac{f_1 f_2}{f_1 + f_2} \approx 11{,}2\,\text{см}$
}
\solutionspace{80pt}

\tasknumber{5}%
\task{%
    Линейные размеры прямого изображения предмета, полученного в собирающей линзе,
    в четыре раза больше линейных размеров предмета.
    Зная, что предмет находится на $30\,\text{см}$ ближе к линзе,
    чем его изображение, найти оптическую силу линзы.
}
\answer{%
    \begin{align*}
    &\text{Если изображение действительное:} \\
    D &= \frac 1F = \frac 1a + \frac 1b, \qquad \Gamma = \frac ba, \qquad b - a = \ell \implies b = \Gamma a \implies \Gamma a - a = \ell \implies  \\
    a &= \frac {\ell}{\Gamma - 1} \implies b = \frac {{\ell} \Gamma}{\Gamma - 1} \implies  \\
    D &= \frac {\Gamma - 1}\ell + \frac {\Gamma - 1}{\ell \Gamma} = \frac 1\ell \cdot \cbr{\Gamma - 1 + \frac {\Gamma - 1}{\Gamma} } =\frac 1\ell \cdot \cbr{\Gamma - \frac 1\Gamma} \approx 12{,}5\,\text{дптр}.
    \\
    &\text{Если изображение мнимое:} \\
    D &= \frac 1F = \frac 1a - \frac 1b, \qquad \Gamma = \frac ba, \qquad b - a = \ell \implies b = \Gamma a \implies \Gamma a - a = \ell \implies  \\
    a &= \frac {\ell}{\Gamma - 1} \implies b = \frac {{\ell} \Gamma}{\Gamma - 1} \implies  \\
    D &= \frac {\Gamma - 1}\ell - \frac {\Gamma - 1}{\ell \Gamma} = \frac 1\ell \cdot \cbr{\Gamma - 1 - \frac {\Gamma - 1}{\Gamma} } =\frac 1\ell \cdot \cbr{\Gamma + \frac 1\Gamma - 2} \approx 7{,}5\,\text{дптр}.
    \\
    &\text{В ответе надо указать оба значения.}
    \end{align*}
}
\solutionspace{120pt}

\tasknumber{6}%
\task{%
    Оптическая сила объектива фотоаппарата равна $3\,\text{дптр}$.
    При фотографировании чертежа с расстояния $1{,}2\,\text{м}$ площадь изображения
    чертежа на фотопластинке оказалась равной $16\,\text{см}^{2}$.
    Какова площадь самого чертежа? Ответ выразите в квадратных сантиметрах.
}
\answer{%
    \begin{align*}
    &\frac 1a + \frac 1b = \frac 1F = D \implies b = \frac{aF}{a - F} \\
    &\frac {S'}S = \Gamma^2 = \sqr{\frac ba} = \sqr{\frac F{a - F}} \implies \\
    &\implies S = S' \cdot \sqr{\frac{a - F}F} = S' \cdot \sqr{\frac aF - 1} = S' \cdot \sqr{aD - 1} \approx 110\,\text{см}^{2}.
    \end{align*}
}


\variantsplitter


\addpersonalvariant{Константин Козлов}

\tasknumber{7}%
\task{%
    В каком месте на главной оптической оси двояковыпуклой линзы
    нужно поместить точечный источник света,
    чтобы его изображение оказалось в главном фокусе линзы?
}
\answer{%
    $\text{для мнимого - на половине фокусного, для действительного - на бесконечности}$
}
\solutionspace{120pt}

\tasknumber{8}%
\task{%
    Предмет в виде отрезка длиной $\ell$ расположен вдоль оптической оси
    собирающей линзы с фокусным расстоянием $F$.
    Середина отрезка расположена
    на расстоянии $a$ от линзы, которая даёт действительное изображение
    всех точек предмета.
    Определить продольное увеличение предмета.
}
\answer{%
    \begin{align*}
    \frac 1{a + \frac \ell 2} &+ \frac 1b = \frac 1F \implies b = \frac{F\cbr{a + \frac \ell 2}}{a + \frac \ell 2 - F} \\
    \frac 1{a - \frac \ell 2} &+ \frac 1c = \frac 1F \implies c = \frac{F\cbr{a - \frac \ell 2}}{a - \frac \ell 2 - F} \\
    \abs{b - c} &= \abs{\frac{F\cbr{a + \frac \ell 2}}{a + \frac \ell 2 - F} - \frac{F\cbr{a - \frac \ell 2}}{a - \frac \ell 2 - F}}= F\abs{\frac{\cbr{a + \frac \ell 2}\cbr{a - \frac \ell 2 - F} - \cbr{a - \frac \ell 2}\cbr{a + \frac \ell 2 - F}}{ \cbr{a + \frac \ell 2 - F} \cbr{a - \frac \ell 2 - F} }} =  \\
    &= F\abs{\frac{a^2 - \frac {a\ell} 2 - Fa + \frac {a\ell} 2 - \frac {\ell^2} 4 - \frac {F\ell}2 - a^2 - \frac {a\ell}2 + aF + \frac {a\ell}2 + \frac {\ell^2} 4 - \frac {F\ell} 2}{\cbr{a + \frac \ell 2 - F} \cbr{a - \frac \ell 2 - F} }} = \\
    &= F\frac{F\ell}{\sqr{a-F} - \frac {\ell^2}4} = \frac{F^2\ell}{\sqr{a-F} - \frac {\ell^2}4}\implies \Gamma = \frac{\abs{b - c}}\ell = \frac{F^2}{\sqr{a-F} - \frac {\ell^2}4}.
    \end{align*}
}
\solutionspace{120pt}

\tasknumber{9}%
\task{%
    На экране с помощью тонкой линзы получено изображение предмета
    с увеличением $4$.
    Предмет передвинули на $2\,\text{см}$.
    Для того, чтобы получить резкое изображение, пришлось передвинуть экран.
    При этом увеличение оказалось равным $8$.
    На какое расстояние
    пришлось передвинуть экран?
}
\answer{%
    \begin{align*}
    &\frac 1a + \frac 1b = \frac 1F, \Gamma_1 = \frac ba = \frac{F}{a-F} \implies \Gamma_1(a-F) = F \implies a = F \cdot \frac{1 + \Gamma_1}{\Gamma_1} \\
    &\frac 1{a + x} + \frac 1{b + y} = \frac 1F, \Gamma_2 = \frac {b+y}{a+x} = \frac{F}{a+x-F} \implies a + x = F \cdot \frac{1 + \Gamma_2}{\Gamma_2} \\
    &1 + \frac xa = \frac{ \frac{1 + \Gamma_2}{\Gamma_2} }{ \frac{1 + \Gamma_1}{\Gamma_1} } = \frac{\Gamma_1(1 + \Gamma_2)}{\Gamma_2(1 + \Gamma_1)} \\
    &a = \frac x{ \frac{\Gamma_1(1 + \Gamma_2)}{\Gamma_2(1 + \Gamma_1)} - 1} = x \cdot \frac{\Gamma_2(1 + \Gamma_1)}{\Gamma_1 - \Gamma_2} \\
    &y = (a + x)\Gamma_2 - b = (a + x)\Gamma_2 - a\Gamma_1 = a(\Gamma_2 - \Gamma_1) + x\Gamma_2 = -x\Gamma_2(1 + \Gamma_1) + x\Gamma_2 = -x\Gamma_2\Gamma_1 = 64\,\text{см}, \\
    &\text{знаки разные, т.е.
    экран надо было подвинуть в другую сторону чем предмет: $x < 0, y > 0$.}
    \end{align*}
}
\solutionspace{120pt}

\tasknumber{10}%
\task{%
    Тонкая собирающая линза дает изображение предмета на экране при двух положениях линзы между предметом и экраном.
    Высота изображения при первом положении $30\,\text{см}$, во втором — $9\,\text{см}$.
    Расстояние между предметом и экранов постоянно.
    Чему равна высота предмета?
}
\answer{%
    \begin{align*}
    &\frac 1a + \frac 1b = \frac 1F, \frac 1c + \frac 1d = \frac 1F, a + b = c + d \implies \frac{a + b}{ab} = \frac 1F = \frac{c+d}{cd} \implies ab = cd, \\
    &\implies ab = c(a + b - c) \implies c^2 - ac - bc + ab = 0 \implies c = a \text{ или } c = b \implies c = b \implies d = a.
    \\
    &\Gamma_1 = \frac {H_1}H = \frac ba, \Gamma_2 = \frac {H_2}H = \frac dc = \frac ab \implies \frac {H_1}H \cdot \frac {H_2}H = \frac ba \cdot \frac ab = 1, \\
    &H = \sqrt{H_1 H_2} \approx 16{,}4\,\text{см}.
    \end{align*}
}
\solutionspace{120pt}

\tasknumber{11}%
\task{%
    Какие предметы можно рассмотреть на фотографии, сделанной со спутника,
    если разрешающая способность плёнки $0{,}010\,\text{мм}$? Каким должно быть
    время экспозиции $\tau$ чтобы полностью использовать возможности плёнки?
    Фокусное расстояние объектива используемого фотоаппарата $10\,\text{см}$,
    высота орбиты спутника $80\,\text{км}$.
}
\answer{%
    \begin{align*}
    &H \ll R \implies v = v_{\text{I}} = \sqrt{G R} \approx 7{,}9\,\frac{\text{км}}{\text{с}}.
    \\
    &F \ll H \implies b = F, a = H, \\
    &\Gamma = \frac \delta\ell = \frac ba \implies \ell = \frac{\delta a}b = \frac{\delta H}F \approx \frac{0{,}010\,\text{мм} \cdot 80\,\text{км}}{10\,\text{см}} \approx 8\,\text{м}, \\
    &\implies \tau = \frac \ell v = \frac{\delta H}{F v} = \frac{0{,}010\,\text{мм} \cdot 80\,\text{км}}{10\,\text{см} \cdot 7{,}9\,\frac{\text{км}}{\text{с}}} \approx 1{,}0\,\text{мс}.
    \end{align*}
}


\variantsplitter


\addpersonalvariant{Константин Козлов}

\tasknumber{12}%
\task{%
    При аэрофотосъемках используется фотоаппарат, объектив которого
    имеет фокусиое расстояние $15\,\text{см}$.
    Разрешающая способность плёнки $0{,}02\,\text{мм}$.
    На какой высоте должен лететь самолет, чтобы на фотографии можно
    было различить следы размером $15\,\text{см}$?
    При какой скорости самолета изображение не будет размытым,
    если время экспозиции $1\,\text{мс}$?
}
\answer{%
    \begin{align*}
    &F \ll H \implies b = F, a = H, \\
    &\Gamma = \frac \delta\ell = \frac ba = \frac FH \implies H = \frac{\ell F}\delta = \frac{15\,\text{см} \cdot 15\,\text{см}}{0{,}02\,\text{мм}} \approx 1{,}1\,\text{км}, \\
    &\implies v = \frac l\tau = \frac{15\,\text{см}}{1\,\text{мс}} \approx 540\,\frac{\text{км}}{\text{ч}}.
    \end{align*}
}
\solutionspace{120pt}

\tasknumber{13}%
\task{%
    Две одинаковые собиращие линзы установлены так, что их главные оптические оси совпадают,
    а главный фокус первой находится там же, где главный фокус второй.
    Расстояние от первой линзы до предмета равно $27\,\text{см}$.
    Чему равно расстояние от изображения объекта во второй линзе до второй линзы?
    Определите также увеличение.
    Фокусное расстояние каждой линзы $25\,\text{см}$.
}
\answer{%
    \begin{align*}
    \frac 1a + \frac 1b &= \frac 1F \implies b = \frac{aF}{a - F} \implies 2F - b = \frac{2aF - 2F^2 - aF}{a - F} = \frac{F(a - 2F)}{a - F}.
    \\
    \frac 1{2F - b} + \frac 1c &= \frac 1F \implies c = \frac{F(2F-b)}{(2F - b) - F} = \frac{F \cdot \frac{F(a - 2F)}{a - F}}{\frac{F(a - 2F)}{a - F} - F}  = F \cdot \frac{ \frac{F(a - 2F)}{a - F} }{ \frac{F(a - 2F)}{a - F} - 1} = \\
     &= F \cdot \frac{a - 2F}{a - 2F - a + F} = 2F - a = 23\,\text{см}.
     \\
    \ell &= a + 2F + c = 4F = 100\,\text{см}.
    \\
    &\Gamma = \Gamma_1 \cdot \Gamma_2 = \frac ba \cdot \frac c{2F-b} = \frac F{a - F} \cdot \frac{2F - a}{\frac{F(a - 2F)}{a - F}} = -1.
    \end{align*}
}
\solutionspace{120pt}

\tasknumber{14}%
\task{%
    Собирающая линза с фокусным расстоянием $F_1 > 0$ и рассеивающая линза с фокусным расстоянием $F_2 < 0$
    установлены коаксиально на расстоянии $\ell$.
    Пучок параллельных лучей падает на собирающую линзу.
    Сделайте схематичное построение и определите, в какой точке система из этих линз соберёт пучок.
}
\answer{%
    \begin{align*}
    &\text{Если пучок падает на собирающую линзу:} \\
    \frac 1{\infty} + \frac 1b &= \frac 1{F_1} \implies b = F_1 \implies \ell - b = \ell - F_1 \\
    \frac 1{\ell - b} + \frac 1c &= \frac 1{F_2} \implies c = \frac{F_2(\ell - b)}{\ell - b - F_2} = \frac{F_2(\ell - F_1)}{\ell - F_1 - F_2}.
    \\
    &\text{Если же пучок падает на рассеивающую линзу:} \\
    \frac 1{\infty} + \frac 1b &= \frac 1{F_2} \implies b = F_2 \implies \ell - b = \ell - F_2 \\
    \frac 1{\ell - b} + \frac 1c &= \frac 1{F_1} \implies c = \frac{F_1(\ell - b)}{\ell - b - F_1} = \frac{F_1(\ell - F_2)}{\ell - F_2 - F_1}.
    \end{align*}
}
\solutionspace{120pt}

\tasknumber{15}%
\task{%
    Две собирающих линзы с фокусными расстояниями $50\,\text{см}$ и $45\,\text{см}$ расположены так,
    что их оптические оси совмещены.
    На первую линзу падает пучок параллельных лучей.
    Пройдя через вторую линзу, он остался параллельным.
    Найдите расстояние между линзами и сделайте рисунок.
}
\answer{%
    \begin{align*}
    \frac 1\infty + \frac 1b &= \frac 1{F_1} \implies b = F_1, \\
    \frac 1{\ell - b} + \frac 1{\infty} &= \frac 1{F_2} \implies \ell - b = F_2 \implies \ell = b + F_2 = F_1 + F_2 = 95\,\text{см}.
    \end{align*}
}

\variantsplitter

\addpersonalvariant{Наталья Кравченко}

\tasknumber{1}%
\task{%
    Найти оптическую силу собирающей линзы, если действительное изображение предмета,
    помещённого в $15\,\text{см}$ от линзы, получается на расстоянии $30\,\text{см}$ от неё.
}
\answer{%
    $D = \frac 1F = \frac 1a + \frac 1b = \frac 1{15\,\text{см}} + \frac 1{30\,\text{см}} \approx 10\,\text{дптр}$
}
\solutionspace{80pt}

\tasknumber{2}%
\task{%
    Найти увеличение изображения, если изображение предмета, находящегося
    на расстоянии $20\,\text{см}$ от линзы, получается на расстоянии $30\,\text{см}$ от неё.
}
\answer{%
    $\Gamma = \frac ba = \frac {30\,\text{см}}{20\,\text{см}} \approx 1{,}50$
}
\solutionspace{80pt}

\tasknumber{3}%
\task{%
    Расстояние от предмета до линзы $10\,\text{см}$, а от линзы до мнимого изображения $30\,\text{см}$.
    Чему равно фокусное расстояние линзы?
}
\answer{%
    $\pm \frac 1F = \frac 1a - \frac 1b \implies F = \frac{a b}{\abs{b - a}} \approx 15\,\text{см}$
}
\solutionspace{80pt}

\tasknumber{4}%
\task{%
    Две тонкие собирающие линзы с фокусными расстояниями $25\,\text{см}$ и $20\,\text{см}$ сложены вместе.
    Чему равно фокусное расстояние такой оптической системы?
}
\answer{%
    $\frac 1{f_1} = \frac 1a + \frac 1b; \frac 1{f_2} = - \frac 1b + \frac 1c \implies \frac 1{f_1} + \frac 1{f_2} = \frac 1a + \frac 1c \implies f' = \frac 1{\frac 1{f_1} + \frac 1{f_2}} = \frac{f_1 f_2}{f_1 + f_2} \approx 11{,}1\,\text{см}$
}
\solutionspace{80pt}

\tasknumber{5}%
\task{%
    Линейные размеры прямого изображения предмета, полученного в собирающей линзе,
    в три раза больше линейных размеров предмета.
    Зная, что предмет находится на $35\,\text{см}$ ближе к линзе,
    чем его изображение, найти оптическую силу линзы.
}
\answer{%
    \begin{align*}
    &\text{Если изображение действительное:} \\
    D &= \frac 1F = \frac 1a + \frac 1b, \qquad \Gamma = \frac ba, \qquad b - a = \ell \implies b = \Gamma a \implies \Gamma a - a = \ell \implies  \\
    a &= \frac {\ell}{\Gamma - 1} \implies b = \frac {{\ell} \Gamma}{\Gamma - 1} \implies  \\
    D &= \frac {\Gamma - 1}\ell + \frac {\Gamma - 1}{\ell \Gamma} = \frac 1\ell \cdot \cbr{\Gamma - 1 + \frac {\Gamma - 1}{\Gamma} } =\frac 1\ell \cdot \cbr{\Gamma - \frac 1\Gamma} \approx 7{,}6\,\text{дптр}.
    \\
    &\text{Если изображение мнимое:} \\
    D &= \frac 1F = \frac 1a - \frac 1b, \qquad \Gamma = \frac ba, \qquad b - a = \ell \implies b = \Gamma a \implies \Gamma a - a = \ell \implies  \\
    a &= \frac {\ell}{\Gamma - 1} \implies b = \frac {{\ell} \Gamma}{\Gamma - 1} \implies  \\
    D &= \frac {\Gamma - 1}\ell - \frac {\Gamma - 1}{\ell \Gamma} = \frac 1\ell \cdot \cbr{\Gamma - 1 - \frac {\Gamma - 1}{\Gamma} } =\frac 1\ell \cdot \cbr{\Gamma + \frac 1\Gamma - 2} \approx 3{,}8\,\text{дптр}.
    \\
    &\text{В ответе надо указать оба значения.}
    \end{align*}
}
\solutionspace{120pt}

\tasknumber{6}%
\task{%
    Оптическая сила объектива фотоаппарата равна $6\,\text{дптр}$.
    При фотографировании чертежа с расстояния $1{,}1\,\text{м}$ площадь изображения
    чертежа на фотопластинке оказалась равной $16\,\text{см}^{2}$.
    Какова площадь самого чертежа? Ответ выразите в квадратных сантиметрах.
}
\answer{%
    \begin{align*}
    &\frac 1a + \frac 1b = \frac 1F = D \implies b = \frac{aF}{a - F} \\
    &\frac {S'}S = \Gamma^2 = \sqr{\frac ba} = \sqr{\frac F{a - F}} \implies \\
    &\implies S = S' \cdot \sqr{\frac{a - F}F} = S' \cdot \sqr{\frac aF - 1} = S' \cdot \sqr{aD - 1} \approx 500\,\text{см}^{2}.
    \end{align*}
}


\variantsplitter


\addpersonalvariant{Наталья Кравченко}

\tasknumber{7}%
\task{%
    В каком месте на главной оптической оси двояковыпуклой линзы
    нужно поместить точечный источник света,
    чтобы его изображение оказалось в главном фокусе линзы?
}
\answer{%
    $\text{для мнимого - на половине фокусного, для действительного - на бесконечности}$
}
\solutionspace{120pt}

\tasknumber{8}%
\task{%
    Предмет в виде отрезка длиной $\ell$ расположен вдоль оптической оси
    собирающей линзы с фокусным расстоянием $F$.
    Середина отрезка расположена
    на расстоянии $a$ от линзы, которая даёт действительное изображение
    всех точек предмета.
    Определить продольное увеличение предмета.
}
\answer{%
    \begin{align*}
    \frac 1{a + \frac \ell 2} &+ \frac 1b = \frac 1F \implies b = \frac{F\cbr{a + \frac \ell 2}}{a + \frac \ell 2 - F} \\
    \frac 1{a - \frac \ell 2} &+ \frac 1c = \frac 1F \implies c = \frac{F\cbr{a - \frac \ell 2}}{a - \frac \ell 2 - F} \\
    \abs{b - c} &= \abs{\frac{F\cbr{a + \frac \ell 2}}{a + \frac \ell 2 - F} - \frac{F\cbr{a - \frac \ell 2}}{a - \frac \ell 2 - F}}= F\abs{\frac{\cbr{a + \frac \ell 2}\cbr{a - \frac \ell 2 - F} - \cbr{a - \frac \ell 2}\cbr{a + \frac \ell 2 - F}}{ \cbr{a + \frac \ell 2 - F} \cbr{a - \frac \ell 2 - F} }} =  \\
    &= F\abs{\frac{a^2 - \frac {a\ell} 2 - Fa + \frac {a\ell} 2 - \frac {\ell^2} 4 - \frac {F\ell}2 - a^2 - \frac {a\ell}2 + aF + \frac {a\ell}2 + \frac {\ell^2} 4 - \frac {F\ell} 2}{\cbr{a + \frac \ell 2 - F} \cbr{a - \frac \ell 2 - F} }} = \\
    &= F\frac{F\ell}{\sqr{a-F} - \frac {\ell^2}4} = \frac{F^2\ell}{\sqr{a-F} - \frac {\ell^2}4}\implies \Gamma = \frac{\abs{b - c}}\ell = \frac{F^2}{\sqr{a-F} - \frac {\ell^2}4}.
    \end{align*}
}
\solutionspace{120pt}

\tasknumber{9}%
\task{%
    На экране с помощью тонкой линзы получено изображение предмета
    с увеличением $2$.
    Предмет передвинули на $8\,\text{см}$.
    Для того, чтобы получить резкое изображение, пришлось передвинуть экран.
    При этом увеличение оказалось равным $6$.
    На какое расстояние
    пришлось передвинуть экран?
}
\answer{%
    \begin{align*}
    &\frac 1a + \frac 1b = \frac 1F, \Gamma_1 = \frac ba = \frac{F}{a-F} \implies \Gamma_1(a-F) = F \implies a = F \cdot \frac{1 + \Gamma_1}{\Gamma_1} \\
    &\frac 1{a + x} + \frac 1{b + y} = \frac 1F, \Gamma_2 = \frac {b+y}{a+x} = \frac{F}{a+x-F} \implies a + x = F \cdot \frac{1 + \Gamma_2}{\Gamma_2} \\
    &1 + \frac xa = \frac{ \frac{1 + \Gamma_2}{\Gamma_2} }{ \frac{1 + \Gamma_1}{\Gamma_1} } = \frac{\Gamma_1(1 + \Gamma_2)}{\Gamma_2(1 + \Gamma_1)} \\
    &a = \frac x{ \frac{\Gamma_1(1 + \Gamma_2)}{\Gamma_2(1 + \Gamma_1)} - 1} = x \cdot \frac{\Gamma_2(1 + \Gamma_1)}{\Gamma_1 - \Gamma_2} \\
    &y = (a + x)\Gamma_2 - b = (a + x)\Gamma_2 - a\Gamma_1 = a(\Gamma_2 - \Gamma_1) + x\Gamma_2 = -x\Gamma_2(1 + \Gamma_1) + x\Gamma_2 = -x\Gamma_2\Gamma_1 = 96\,\text{см}, \\
    &\text{знаки разные, т.е.
    экран надо было подвинуть в другую сторону чем предмет: $x < 0, y > 0$.}
    \end{align*}
}
\solutionspace{120pt}

\tasknumber{10}%
\task{%
    Тонкая собирающая линза дает изображение предмета на экране при двух положениях линзы между предметом и экраном.
    Высота изображения при первом положении $25\,\text{см}$, во втором — $5\,\text{см}$.
    Расстояние между предметом и экранов постоянно.
    Чему равна высота предмета?
}
\answer{%
    \begin{align*}
    &\frac 1a + \frac 1b = \frac 1F, \frac 1c + \frac 1d = \frac 1F, a + b = c + d \implies \frac{a + b}{ab} = \frac 1F = \frac{c+d}{cd} \implies ab = cd, \\
    &\implies ab = c(a + b - c) \implies c^2 - ac - bc + ab = 0 \implies c = a \text{ или } c = b \implies c = b \implies d = a.
    \\
    &\Gamma_1 = \frac {H_1}H = \frac ba, \Gamma_2 = \frac {H_2}H = \frac dc = \frac ab \implies \frac {H_1}H \cdot \frac {H_2}H = \frac ba \cdot \frac ab = 1, \\
    &H = \sqrt{H_1 H_2} \approx 11{,}2\,\text{см}.
    \end{align*}
}
\solutionspace{120pt}

\tasknumber{11}%
\task{%
    Какие предметы можно рассмотреть на фотографии, сделанной со спутника,
    если разрешающая способность плёнки $0{,}02\,\text{мм}$? Каким должно быть
    время экспозиции $\tau$ чтобы полностью использовать возможности плёнки?
    Фокусное расстояние объектива используемого фотоаппарата $15\,\text{см}$,
    высота орбиты спутника $150\,\text{км}$.
}
\answer{%
    \begin{align*}
    &H \ll R \implies v = v_{\text{I}} = \sqrt{G R} \approx 7{,}9\,\frac{\text{км}}{\text{с}}.
    \\
    &F \ll H \implies b = F, a = H, \\
    &\Gamma = \frac \delta\ell = \frac ba \implies \ell = \frac{\delta a}b = \frac{\delta H}F \approx \frac{0{,}02\,\text{мм} \cdot 150\,\text{км}}{15\,\text{см}} \approx 20\,\text{м}, \\
    &\implies \tau = \frac \ell v = \frac{\delta H}{F v} = \frac{0{,}02\,\text{мм} \cdot 150\,\text{км}}{15\,\text{см} \cdot 7{,}9\,\frac{\text{км}}{\text{с}}} \approx 3\,\text{мс}.
    \end{align*}
}


\variantsplitter


\addpersonalvariant{Наталья Кравченко}

\tasknumber{12}%
\task{%
    При аэрофотосъемках используется фотоаппарат, объектив которого
    имеет фокусиое расстояние $20\,\text{см}$.
    Разрешающая способность плёнки $0{,}010\,\text{мм}$.
    На какой высоте должен лететь самолет, чтобы на фотографии можно
    было различить следы размером $30\,\text{см}$?
    При какой скорости самолета изображение не будет размытым,
    если время экспозиции $1\,\text{мс}$?
}
\answer{%
    \begin{align*}
    &F \ll H \implies b = F, a = H, \\
    &\Gamma = \frac \delta\ell = \frac ba = \frac FH \implies H = \frac{\ell F}\delta = \frac{30\,\text{см} \cdot 20\,\text{см}}{0{,}010\,\text{мм}} \approx 6\,\text{км}, \\
    &\implies v = \frac l\tau = \frac{30\,\text{см}}{1\,\text{мс}} \approx 1080\,\frac{\text{км}}{\text{ч}}.
    \end{align*}
}
\solutionspace{120pt}

\tasknumber{13}%
\task{%
    Две одинаковые собиращие линзы установлены так, что их главные оптические оси совпадают,
    а главный фокус первой находится там же, где главный фокус второй.
    Расстояние от первой линзы до предмета равно $16\,\text{см}$.
    Чему равно расстояние от изображения объекта во второй линзе до самого объекта?
    Определите также увеличение.
    Фокусное расстояние каждой линзы $35\,\text{см}$.
}
\answer{%
    \begin{align*}
    \frac 1a + \frac 1b &= \frac 1F \implies b = \frac{aF}{a - F} \implies 2F - b = \frac{2aF - 2F^2 - aF}{a - F} = \frac{F(a - 2F)}{a - F}.
    \\
    \frac 1{2F - b} + \frac 1c &= \frac 1F \implies c = \frac{F(2F-b)}{(2F - b) - F} = \frac{F \cdot \frac{F(a - 2F)}{a - F}}{\frac{F(a - 2F)}{a - F} - F}  = F \cdot \frac{ \frac{F(a - 2F)}{a - F} }{ \frac{F(a - 2F)}{a - F} - 1} = \\
     &= F \cdot \frac{a - 2F}{a - 2F - a + F} = 2F - a = 54\,\text{см}.
     \\
    \ell &= a + 2F + c = 4F = 140\,\text{см}.
    \\
    &\Gamma = \Gamma_1 \cdot \Gamma_2 = \frac ba \cdot \frac c{2F-b} = \frac F{a - F} \cdot \frac{2F - a}{\frac{F(a - 2F)}{a - F}} = -1.
    \end{align*}
}
\solutionspace{120pt}

\tasknumber{14}%
\task{%
    Собирающая линза с фокусным расстоянием $F_1 > 0$ и рассеивающая линза с фокусным расстоянием $F_2 < 0$
    установлены коаксиально на расстоянии $\ell$.
    Пучок параллельных лучей падает на собирающую линзу.
    Сделайте схематичное построение и определите, в какой точке система из этих линз соберёт пучок.
}
\answer{%
    \begin{align*}
    &\text{Если пучок падает на собирающую линзу:} \\
    \frac 1{\infty} + \frac 1b &= \frac 1{F_1} \implies b = F_1 \implies \ell - b = \ell - F_1 \\
    \frac 1{\ell - b} + \frac 1c &= \frac 1{F_2} \implies c = \frac{F_2(\ell - b)}{\ell - b - F_2} = \frac{F_2(\ell - F_1)}{\ell - F_1 - F_2}.
    \\
    &\text{Если же пучок падает на рассеивающую линзу:} \\
    \frac 1{\infty} + \frac 1b &= \frac 1{F_2} \implies b = F_2 \implies \ell - b = \ell - F_2 \\
    \frac 1{\ell - b} + \frac 1c &= \frac 1{F_1} \implies c = \frac{F_1(\ell - b)}{\ell - b - F_1} = \frac{F_1(\ell - F_2)}{\ell - F_2 - F_1}.
    \end{align*}
}
\solutionspace{120pt}

\tasknumber{15}%
\task{%
    Две собирающих линзы с фокусными расстояниями $30\,\text{см}$ и $35\,\text{см}$ расположены так,
    что их оптические оси совмещены.
    На первую линзу падает пучок параллельных лучей.
    Пройдя через вторую линзу, он остался параллельным.
    Найдите расстояние между линзами и сделайте рисунок.
}
\answer{%
    \begin{align*}
    \frac 1\infty + \frac 1b &= \frac 1{F_1} \implies b = F_1, \\
    \frac 1{\ell - b} + \frac 1{\infty} &= \frac 1{F_2} \implies \ell - b = F_2 \implies \ell = b + F_2 = F_1 + F_2 = 65\,\text{см}.
    \end{align*}
}

\variantsplitter

\addpersonalvariant{Матвей Кузьмин}

\tasknumber{1}%
\task{%
    Найти оптическую силу собирающей линзы, если действительное изображение предмета,
    помещённого в $55\,\text{см}$ от линзы, получается на расстоянии $20\,\text{см}$ от неё.
}
\answer{%
    $D = \frac 1F = \frac 1a + \frac 1b = \frac 1{55\,\text{см}} + \frac 1{20\,\text{см}} \approx 6{,}82\,\text{дптр}$
}
\solutionspace{80pt}

\tasknumber{2}%
\task{%
    Найти увеличение изображения, если изображение предмета, находящегося
    на расстоянии $15\,\text{см}$ от линзы, получается на расстоянии $12\,\text{см}$ от неё.
}
\answer{%
    $\Gamma = \frac ba = \frac {12\,\text{см}}{15\,\text{см}} \approx 0{,}8$
}
\solutionspace{80pt}

\tasknumber{3}%
\task{%
    Расстояние от предмета до линзы $8\,\text{см}$, а от линзы до мнимого изображения $20\,\text{см}$.
    Чему равно фокусное расстояние линзы?
}
\answer{%
    $\pm \frac 1F = \frac 1a - \frac 1b \implies F = \frac{a b}{\abs{b - a}} \approx 13{,}3\,\text{см}$
}
\solutionspace{80pt}

\tasknumber{4}%
\task{%
    Две тонкие собирающие линзы с фокусными расстояниями $25\,\text{см}$ и $30\,\text{см}$ сложены вместе.
    Чему равно фокусное расстояние такой оптической системы?
}
\answer{%
    $\frac 1{f_1} = \frac 1a + \frac 1b; \frac 1{f_2} = - \frac 1b + \frac 1c \implies \frac 1{f_1} + \frac 1{f_2} = \frac 1a + \frac 1c \implies f' = \frac 1{\frac 1{f_1} + \frac 1{f_2}} = \frac{f_1 f_2}{f_1 + f_2} \approx 13{,}6\,\text{см}$
}
\solutionspace{80pt}

\tasknumber{5}%
\task{%
    Линейные размеры прямого изображения предмета, полученного в собирающей линзе,
    в три раза больше линейных размеров предмета.
    Зная, что предмет находится на $25\,\text{см}$ ближе к линзе,
    чем его изображение, найти оптическую силу линзы.
}
\answer{%
    \begin{align*}
    &\text{Если изображение действительное:} \\
    D &= \frac 1F = \frac 1a + \frac 1b, \qquad \Gamma = \frac ba, \qquad b - a = \ell \implies b = \Gamma a \implies \Gamma a - a = \ell \implies  \\
    a &= \frac {\ell}{\Gamma - 1} \implies b = \frac {{\ell} \Gamma}{\Gamma - 1} \implies  \\
    D &= \frac {\Gamma - 1}\ell + \frac {\Gamma - 1}{\ell \Gamma} = \frac 1\ell \cdot \cbr{\Gamma - 1 + \frac {\Gamma - 1}{\Gamma} } =\frac 1\ell \cdot \cbr{\Gamma - \frac 1\Gamma} \approx 10{,}7\,\text{дптр}.
    \\
    &\text{Если изображение мнимое:} \\
    D &= \frac 1F = \frac 1a - \frac 1b, \qquad \Gamma = \frac ba, \qquad b - a = \ell \implies b = \Gamma a \implies \Gamma a - a = \ell \implies  \\
    a &= \frac {\ell}{\Gamma - 1} \implies b = \frac {{\ell} \Gamma}{\Gamma - 1} \implies  \\
    D &= \frac {\Gamma - 1}\ell - \frac {\Gamma - 1}{\ell \Gamma} = \frac 1\ell \cdot \cbr{\Gamma - 1 - \frac {\Gamma - 1}{\Gamma} } =\frac 1\ell \cdot \cbr{\Gamma + \frac 1\Gamma - 2} \approx 5{,}3\,\text{дптр}.
    \\
    &\text{В ответе надо указать оба значения.}
    \end{align*}
}
\solutionspace{120pt}

\tasknumber{6}%
\task{%
    Оптическая сила объектива фотоаппарата равна $6\,\text{дптр}$.
    При фотографировании чертежа с расстояния $0{,}9\,\text{м}$ площадь изображения
    чертежа на фотопластинке оказалась равной $9\,\text{см}^{2}$.
    Какова площадь самого чертежа? Ответ выразите в квадратных сантиметрах.
}
\answer{%
    \begin{align*}
    &\frac 1a + \frac 1b = \frac 1F = D \implies b = \frac{aF}{a - F} \\
    &\frac {S'}S = \Gamma^2 = \sqr{\frac ba} = \sqr{\frac F{a - F}} \implies \\
    &\implies S = S' \cdot \sqr{\frac{a - F}F} = S' \cdot \sqr{\frac aF - 1} = S' \cdot \sqr{aD - 1} \approx 170\,\text{см}^{2}.
    \end{align*}
}


\variantsplitter


\addpersonalvariant{Матвей Кузьмин}

\tasknumber{7}%
\task{%
    В каком месте на главной оптической оси двояковыгнутой линзы
    нужно поместить точечный источник света,
    чтобы его изображение оказалось в главном фокусе линзы?
}
\answer{%
    $\text{на половине фокусного расстояния}$
}
\solutionspace{120pt}

\tasknumber{8}%
\task{%
    Предмет в виде отрезка длиной $\ell$ расположен вдоль оптической оси
    собирающей линзы с фокусным расстоянием $F$.
    Середина отрезка расположена
    на расстоянии $a$ от линзы, которая даёт действительное изображение
    всех точек предмета.
    Определить продольное увеличение предмета.
}
\answer{%
    \begin{align*}
    \frac 1{a + \frac \ell 2} &+ \frac 1b = \frac 1F \implies b = \frac{F\cbr{a + \frac \ell 2}}{a + \frac \ell 2 - F} \\
    \frac 1{a - \frac \ell 2} &+ \frac 1c = \frac 1F \implies c = \frac{F\cbr{a - \frac \ell 2}}{a - \frac \ell 2 - F} \\
    \abs{b - c} &= \abs{\frac{F\cbr{a + \frac \ell 2}}{a + \frac \ell 2 - F} - \frac{F\cbr{a - \frac \ell 2}}{a - \frac \ell 2 - F}}= F\abs{\frac{\cbr{a + \frac \ell 2}\cbr{a - \frac \ell 2 - F} - \cbr{a - \frac \ell 2}\cbr{a + \frac \ell 2 - F}}{ \cbr{a + \frac \ell 2 - F} \cbr{a - \frac \ell 2 - F} }} =  \\
    &= F\abs{\frac{a^2 - \frac {a\ell} 2 - Fa + \frac {a\ell} 2 - \frac {\ell^2} 4 - \frac {F\ell}2 - a^2 - \frac {a\ell}2 + aF + \frac {a\ell}2 + \frac {\ell^2} 4 - \frac {F\ell} 2}{\cbr{a + \frac \ell 2 - F} \cbr{a - \frac \ell 2 - F} }} = \\
    &= F\frac{F\ell}{\sqr{a-F} - \frac {\ell^2}4} = \frac{F^2\ell}{\sqr{a-F} - \frac {\ell^2}4}\implies \Gamma = \frac{\abs{b - c}}\ell = \frac{F^2}{\sqr{a-F} - \frac {\ell^2}4}.
    \end{align*}
}
\solutionspace{120pt}

\tasknumber{9}%
\task{%
    На экране с помощью тонкой линзы получено изображение предмета
    с увеличением $2$.
    Предмет передвинули на $10\,\text{см}$.
    Для того, чтобы получить резкое изображение, пришлось передвинуть экран.
    При этом увеличение оказалось равным $8$.
    На какое расстояние
    пришлось передвинуть экран?
}
\answer{%
    \begin{align*}
    &\frac 1a + \frac 1b = \frac 1F, \Gamma_1 = \frac ba = \frac{F}{a-F} \implies \Gamma_1(a-F) = F \implies a = F \cdot \frac{1 + \Gamma_1}{\Gamma_1} \\
    &\frac 1{a + x} + \frac 1{b + y} = \frac 1F, \Gamma_2 = \frac {b+y}{a+x} = \frac{F}{a+x-F} \implies a + x = F \cdot \frac{1 + \Gamma_2}{\Gamma_2} \\
    &1 + \frac xa = \frac{ \frac{1 + \Gamma_2}{\Gamma_2} }{ \frac{1 + \Gamma_1}{\Gamma_1} } = \frac{\Gamma_1(1 + \Gamma_2)}{\Gamma_2(1 + \Gamma_1)} \\
    &a = \frac x{ \frac{\Gamma_1(1 + \Gamma_2)}{\Gamma_2(1 + \Gamma_1)} - 1} = x \cdot \frac{\Gamma_2(1 + \Gamma_1)}{\Gamma_1 - \Gamma_2} \\
    &y = (a + x)\Gamma_2 - b = (a + x)\Gamma_2 - a\Gamma_1 = a(\Gamma_2 - \Gamma_1) + x\Gamma_2 = -x\Gamma_2(1 + \Gamma_1) + x\Gamma_2 = -x\Gamma_2\Gamma_1 = 160\,\text{см}, \\
    &\text{знаки разные, т.е.
    экран надо было подвинуть в другую сторону чем предмет: $x < 0, y > 0$.}
    \end{align*}
}
\solutionspace{120pt}

\tasknumber{10}%
\task{%
    Тонкая собирающая линза дает изображение предмета на экране при двух положениях линзы между предметом и экраном.
    Высота изображения при первом положении $20\,\text{см}$, во втором — $9\,\text{см}$.
    Расстояние между предметом и экранов постоянно.
    Чему равна высота предмета?
}
\answer{%
    \begin{align*}
    &\frac 1a + \frac 1b = \frac 1F, \frac 1c + \frac 1d = \frac 1F, a + b = c + d \implies \frac{a + b}{ab} = \frac 1F = \frac{c+d}{cd} \implies ab = cd, \\
    &\implies ab = c(a + b - c) \implies c^2 - ac - bc + ab = 0 \implies c = a \text{ или } c = b \implies c = b \implies d = a.
    \\
    &\Gamma_1 = \frac {H_1}H = \frac ba, \Gamma_2 = \frac {H_2}H = \frac dc = \frac ab \implies \frac {H_1}H \cdot \frac {H_2}H = \frac ba \cdot \frac ab = 1, \\
    &H = \sqrt{H_1 H_2} \approx 13{,}4\,\text{см}.
    \end{align*}
}
\solutionspace{120pt}

\tasknumber{11}%
\task{%
    Какие предметы можно рассмотреть на фотографии, сделанной со спутника,
    если разрешающая способность плёнки $0{,}010\,\text{мм}$? Каким должно быть
    время экспозиции $\tau$ чтобы полностью использовать возможности плёнки?
    Фокусное расстояние объектива используемого фотоаппарата $10\,\text{см}$,
    высота орбиты спутника $100\,\text{км}$.
}
\answer{%
    \begin{align*}
    &H \ll R \implies v = v_{\text{I}} = \sqrt{G R} \approx 7{,}9\,\frac{\text{км}}{\text{с}}.
    \\
    &F \ll H \implies b = F, a = H, \\
    &\Gamma = \frac \delta\ell = \frac ba \implies \ell = \frac{\delta a}b = \frac{\delta H}F \approx \frac{0{,}010\,\text{мм} \cdot 100\,\text{км}}{10\,\text{см}} \approx 10\,\text{м}, \\
    &\implies \tau = \frac \ell v = \frac{\delta H}{F v} = \frac{0{,}010\,\text{мм} \cdot 100\,\text{км}}{10\,\text{см} \cdot 7{,}9\,\frac{\text{км}}{\text{с}}} \approx 1{,}3\,\text{мс}.
    \end{align*}
}


\variantsplitter


\addpersonalvariant{Матвей Кузьмин}

\tasknumber{12}%
\task{%
    При аэрофотосъемках используется фотоаппарат, объектив которого
    имеет фокусиое расстояние $10\,\text{см}$.
    Разрешающая способность плёнки $0{,}015\,\text{мм}$.
    На какой высоте должен лететь самолет, чтобы на фотографии можно
    было различить следы размером $25\,\text{см}$?
    При какой скорости самолета изображение не будет размытым,
    если время экспозиции $1\,\text{мс}$?
}
\answer{%
    \begin{align*}
    &F \ll H \implies b = F, a = H, \\
    &\Gamma = \frac \delta\ell = \frac ba = \frac FH \implies H = \frac{\ell F}\delta = \frac{25\,\text{см} \cdot 10\,\text{см}}{0{,}015\,\text{мм}} \approx 1{,}7\,\text{км}, \\
    &\implies v = \frac l\tau = \frac{25\,\text{см}}{1\,\text{мс}} \approx 900\,\frac{\text{км}}{\text{ч}}.
    \end{align*}
}
\solutionspace{120pt}

\tasknumber{13}%
\task{%
    Две одинаковые собиращие линзы установлены так, что их главные оптические оси совпадают,
    а главный фокус первой находится там же, где главный фокус второй.
    Расстояние от первой линзы до предмета равно $18\,\text{см}$.
    Чему равно расстояние от изображения объекта во второй линзе до второй линзы?
    Определите также увеличение.
    Фокусное расстояние каждой линзы $30\,\text{см}$.
}
\answer{%
    \begin{align*}
    \frac 1a + \frac 1b &= \frac 1F \implies b = \frac{aF}{a - F} \implies 2F - b = \frac{2aF - 2F^2 - aF}{a - F} = \frac{F(a - 2F)}{a - F}.
    \\
    \frac 1{2F - b} + \frac 1c &= \frac 1F \implies c = \frac{F(2F-b)}{(2F - b) - F} = \frac{F \cdot \frac{F(a - 2F)}{a - F}}{\frac{F(a - 2F)}{a - F} - F}  = F \cdot \frac{ \frac{F(a - 2F)}{a - F} }{ \frac{F(a - 2F)}{a - F} - 1} = \\
     &= F \cdot \frac{a - 2F}{a - 2F - a + F} = 2F - a = 42\,\text{см}.
     \\
    \ell &= a + 2F + c = 4F = 120\,\text{см}.
    \\
    &\Gamma = \Gamma_1 \cdot \Gamma_2 = \frac ba \cdot \frac c{2F-b} = \frac F{a - F} \cdot \frac{2F - a}{\frac{F(a - 2F)}{a - F}} = -1.
    \end{align*}
}
\solutionspace{120pt}

\tasknumber{14}%
\task{%
    Собирающая линза с фокусным расстоянием $F_1 > 0$ и рассеивающая линза с фокусным расстоянием $F_2 < 0$
    установлены коаксиально на расстоянии $\ell$.
    Пучок параллельных лучей падает на рассеивающую линзу.
    Сделайте схематичное построение и определите, в какой точке система из этих линз соберёт пучок.
}
\answer{%
    \begin{align*}
    &\text{Если пучок падает на собирающую линзу:} \\
    \frac 1{\infty} + \frac 1b &= \frac 1{F_1} \implies b = F_1 \implies \ell - b = \ell - F_1 \\
    \frac 1{\ell - b} + \frac 1c &= \frac 1{F_2} \implies c = \frac{F_2(\ell - b)}{\ell - b - F_2} = \frac{F_2(\ell - F_1)}{\ell - F_1 - F_2}.
    \\
    &\text{Если же пучок падает на рассеивающую линзу:} \\
    \frac 1{\infty} + \frac 1b &= \frac 1{F_2} \implies b = F_2 \implies \ell - b = \ell - F_2 \\
    \frac 1{\ell - b} + \frac 1c &= \frac 1{F_1} \implies c = \frac{F_1(\ell - b)}{\ell - b - F_1} = \frac{F_1(\ell - F_2)}{\ell - F_2 - F_1}.
    \end{align*}
}
\solutionspace{120pt}

\tasknumber{15}%
\task{%
    Две собирающих линзы с фокусными расстояниями $40\,\text{см}$ и $25\,\text{см}$ расположены так,
    что их оптические оси совмещены.
    На первую линзу падает пучок параллельных лучей.
    Пройдя через вторую линзу, он остался параллельным.
    Найдите расстояние между линзами и сделайте рисунок.
}
\answer{%
    \begin{align*}
    \frac 1\infty + \frac 1b &= \frac 1{F_1} \implies b = F_1, \\
    \frac 1{\ell - b} + \frac 1{\infty} &= \frac 1{F_2} \implies \ell - b = F_2 \implies \ell = b + F_2 = F_1 + F_2 = 65\,\text{см}.
    \end{align*}
}

\variantsplitter

\addpersonalvariant{Сергей Малышев}

\tasknumber{1}%
\task{%
    Найти оптическую силу собирающей линзы, если действительное изображение предмета,
    помещённого в $55\,\text{см}$ от линзы, получается на расстоянии $40\,\text{см}$ от неё.
}
\answer{%
    $D = \frac 1F = \frac 1a + \frac 1b = \frac 1{55\,\text{см}} + \frac 1{40\,\text{см}} \approx 4{,}32\,\text{дптр}$
}
\solutionspace{80pt}

\tasknumber{2}%
\task{%
    Найти увеличение изображения, если изображение предмета, находящегося
    на расстоянии $25\,\text{см}$ от линзы, получается на расстоянии $18\,\text{см}$ от неё.
}
\answer{%
    $\Gamma = \frac ba = \frac {18\,\text{см}}{25\,\text{см}} \approx 0{,}7$
}
\solutionspace{80pt}

\tasknumber{3}%
\task{%
    Расстояние от предмета до линзы $12\,\text{см}$, а от линзы до мнимого изображения $25\,\text{см}$.
    Чему равно фокусное расстояние линзы?
}
\answer{%
    $\pm \frac 1F = \frac 1a - \frac 1b \implies F = \frac{a b}{\abs{b - a}} \approx 23{,}1\,\text{см}$
}
\solutionspace{80pt}

\tasknumber{4}%
\task{%
    Две тонкие собирающие линзы с фокусными расстояниями $18\,\text{см}$ и $30\,\text{см}$ сложены вместе.
    Чему равно фокусное расстояние такой оптической системы?
}
\answer{%
    $\frac 1{f_1} = \frac 1a + \frac 1b; \frac 1{f_2} = - \frac 1b + \frac 1c \implies \frac 1{f_1} + \frac 1{f_2} = \frac 1a + \frac 1c \implies f' = \frac 1{\frac 1{f_1} + \frac 1{f_2}} = \frac{f_1 f_2}{f_1 + f_2} \approx 11{,}2\,\text{см}$
}
\solutionspace{80pt}

\tasknumber{5}%
\task{%
    Линейные размеры прямого изображения предмета, полученного в собирающей линзе,
    в три раза больше линейных размеров предмета.
    Зная, что предмет находится на $40\,\text{см}$ ближе к линзе,
    чем его изображение, найти оптическую силу линзы.
}
\answer{%
    \begin{align*}
    &\text{Если изображение действительное:} \\
    D &= \frac 1F = \frac 1a + \frac 1b, \qquad \Gamma = \frac ba, \qquad b - a = \ell \implies b = \Gamma a \implies \Gamma a - a = \ell \implies  \\
    a &= \frac {\ell}{\Gamma - 1} \implies b = \frac {{\ell} \Gamma}{\Gamma - 1} \implies  \\
    D &= \frac {\Gamma - 1}\ell + \frac {\Gamma - 1}{\ell \Gamma} = \frac 1\ell \cdot \cbr{\Gamma - 1 + \frac {\Gamma - 1}{\Gamma} } =\frac 1\ell \cdot \cbr{\Gamma - \frac 1\Gamma} \approx 6{,}7\,\text{дптр}.
    \\
    &\text{Если изображение мнимое:} \\
    D &= \frac 1F = \frac 1a - \frac 1b, \qquad \Gamma = \frac ba, \qquad b - a = \ell \implies b = \Gamma a \implies \Gamma a - a = \ell \implies  \\
    a &= \frac {\ell}{\Gamma - 1} \implies b = \frac {{\ell} \Gamma}{\Gamma - 1} \implies  \\
    D &= \frac {\Gamma - 1}\ell - \frac {\Gamma - 1}{\ell \Gamma} = \frac 1\ell \cdot \cbr{\Gamma - 1 - \frac {\Gamma - 1}{\Gamma} } =\frac 1\ell \cdot \cbr{\Gamma + \frac 1\Gamma - 2} \approx 3{,}3\,\text{дптр}.
    \\
    &\text{В ответе надо указать оба значения.}
    \end{align*}
}
\solutionspace{120pt}

\tasknumber{6}%
\task{%
    Оптическая сила объектива фотоаппарата равна $5\,\text{дптр}$.
    При фотографировании чертежа с расстояния $0{,}9\,\text{м}$ площадь изображения
    чертежа на фотопластинке оказалась равной $16\,\text{см}^{2}$.
    Какова площадь самого чертежа? Ответ выразите в квадратных сантиметрах.
}
\answer{%
    \begin{align*}
    &\frac 1a + \frac 1b = \frac 1F = D \implies b = \frac{aF}{a - F} \\
    &\frac {S'}S = \Gamma^2 = \sqr{\frac ba} = \sqr{\frac F{a - F}} \implies \\
    &\implies S = S' \cdot \sqr{\frac{a - F}F} = S' \cdot \sqr{\frac aF - 1} = S' \cdot \sqr{aD - 1} \approx 196\,\text{см}^{2}.
    \end{align*}
}


\variantsplitter


\addpersonalvariant{Сергей Малышев}

\tasknumber{7}%
\task{%
    В каком месте на главной оптической оси двояковыгнутой линзы
    нужно поместить точечный источник света,
    чтобы его изображение оказалось в главном фокусе линзы?
}
\answer{%
    $\text{на половине фокусного расстояния}$
}
\solutionspace{120pt}

\tasknumber{8}%
\task{%
    Предмет в виде отрезка длиной $\ell$ расположен вдоль оптической оси
    собирающей линзы с фокусным расстоянием $F$.
    Середина отрезка расположена
    на расстоянии $a$ от линзы, которая даёт действительное изображение
    всех точек предмета.
    Определить продольное увеличение предмета.
}
\answer{%
    \begin{align*}
    \frac 1{a + \frac \ell 2} &+ \frac 1b = \frac 1F \implies b = \frac{F\cbr{a + \frac \ell 2}}{a + \frac \ell 2 - F} \\
    \frac 1{a - \frac \ell 2} &+ \frac 1c = \frac 1F \implies c = \frac{F\cbr{a - \frac \ell 2}}{a - \frac \ell 2 - F} \\
    \abs{b - c} &= \abs{\frac{F\cbr{a + \frac \ell 2}}{a + \frac \ell 2 - F} - \frac{F\cbr{a - \frac \ell 2}}{a - \frac \ell 2 - F}}= F\abs{\frac{\cbr{a + \frac \ell 2}\cbr{a - \frac \ell 2 - F} - \cbr{a - \frac \ell 2}\cbr{a + \frac \ell 2 - F}}{ \cbr{a + \frac \ell 2 - F} \cbr{a - \frac \ell 2 - F} }} =  \\
    &= F\abs{\frac{a^2 - \frac {a\ell} 2 - Fa + \frac {a\ell} 2 - \frac {\ell^2} 4 - \frac {F\ell}2 - a^2 - \frac {a\ell}2 + aF + \frac {a\ell}2 + \frac {\ell^2} 4 - \frac {F\ell} 2}{\cbr{a + \frac \ell 2 - F} \cbr{a - \frac \ell 2 - F} }} = \\
    &= F\frac{F\ell}{\sqr{a-F} - \frac {\ell^2}4} = \frac{F^2\ell}{\sqr{a-F} - \frac {\ell^2}4}\implies \Gamma = \frac{\abs{b - c}}\ell = \frac{F^2}{\sqr{a-F} - \frac {\ell^2}4}.
    \end{align*}
}
\solutionspace{120pt}

\tasknumber{9}%
\task{%
    На экране с помощью тонкой линзы получено изображение предмета
    с увеличением $4$.
    Предмет передвинули на $4\,\text{см}$.
    Для того, чтобы получить резкое изображение, пришлось передвинуть экран.
    При этом увеличение оказалось равным $6$.
    На какое расстояние
    пришлось передвинуть экран?
}
\answer{%
    \begin{align*}
    &\frac 1a + \frac 1b = \frac 1F, \Gamma_1 = \frac ba = \frac{F}{a-F} \implies \Gamma_1(a-F) = F \implies a = F \cdot \frac{1 + \Gamma_1}{\Gamma_1} \\
    &\frac 1{a + x} + \frac 1{b + y} = \frac 1F, \Gamma_2 = \frac {b+y}{a+x} = \frac{F}{a+x-F} \implies a + x = F \cdot \frac{1 + \Gamma_2}{\Gamma_2} \\
    &1 + \frac xa = \frac{ \frac{1 + \Gamma_2}{\Gamma_2} }{ \frac{1 + \Gamma_1}{\Gamma_1} } = \frac{\Gamma_1(1 + \Gamma_2)}{\Gamma_2(1 + \Gamma_1)} \\
    &a = \frac x{ \frac{\Gamma_1(1 + \Gamma_2)}{\Gamma_2(1 + \Gamma_1)} - 1} = x \cdot \frac{\Gamma_2(1 + \Gamma_1)}{\Gamma_1 - \Gamma_2} \\
    &y = (a + x)\Gamma_2 - b = (a + x)\Gamma_2 - a\Gamma_1 = a(\Gamma_2 - \Gamma_1) + x\Gamma_2 = -x\Gamma_2(1 + \Gamma_1) + x\Gamma_2 = -x\Gamma_2\Gamma_1 = 96\,\text{см}, \\
    &\text{знаки разные, т.е.
    экран надо было подвинуть в другую сторону чем предмет: $x < 0, y > 0$.}
    \end{align*}
}
\solutionspace{120pt}

\tasknumber{10}%
\task{%
    Тонкая собирающая линза дает изображение предмета на экране при двух положениях линзы между предметом и экраном.
    Высота изображения при первом положении $25\,\text{см}$, во втором — $9\,\text{см}$.
    Расстояние между предметом и экранов постоянно.
    Чему равна высота предмета?
}
\answer{%
    \begin{align*}
    &\frac 1a + \frac 1b = \frac 1F, \frac 1c + \frac 1d = \frac 1F, a + b = c + d \implies \frac{a + b}{ab} = \frac 1F = \frac{c+d}{cd} \implies ab = cd, \\
    &\implies ab = c(a + b - c) \implies c^2 - ac - bc + ab = 0 \implies c = a \text{ или } c = b \implies c = b \implies d = a.
    \\
    &\Gamma_1 = \frac {H_1}H = \frac ba, \Gamma_2 = \frac {H_2}H = \frac dc = \frac ab \implies \frac {H_1}H \cdot \frac {H_2}H = \frac ba \cdot \frac ab = 1, \\
    &H = \sqrt{H_1 H_2} \approx 15\,\text{см}.
    \end{align*}
}
\solutionspace{120pt}

\tasknumber{11}%
\task{%
    Какие предметы можно рассмотреть на фотографии, сделанной со спутника,
    если разрешающая способность плёнки $0{,}010\,\text{мм}$? Каким должно быть
    время экспозиции $\tau$ чтобы полностью использовать возможности плёнки?
    Фокусное расстояние объектива используемого фотоаппарата $10\,\text{см}$,
    высота орбиты спутника $150\,\text{км}$.
}
\answer{%
    \begin{align*}
    &H \ll R \implies v = v_{\text{I}} = \sqrt{G R} \approx 7{,}9\,\frac{\text{км}}{\text{с}}.
    \\
    &F \ll H \implies b = F, a = H, \\
    &\Gamma = \frac \delta\ell = \frac ba \implies \ell = \frac{\delta a}b = \frac{\delta H}F \approx \frac{0{,}010\,\text{мм} \cdot 150\,\text{км}}{10\,\text{см}} \approx 15\,\text{м}, \\
    &\implies \tau = \frac \ell v = \frac{\delta H}{F v} = \frac{0{,}010\,\text{мм} \cdot 150\,\text{км}}{10\,\text{см} \cdot 7{,}9\,\frac{\text{км}}{\text{с}}} \approx 1{,}9\,\text{мс}.
    \end{align*}
}


\variantsplitter


\addpersonalvariant{Сергей Малышев}

\tasknumber{12}%
\task{%
    При аэрофотосъемках используется фотоаппарат, объектив которого
    имеет фокусиое расстояние $20\,\text{см}$.
    Разрешающая способность плёнки $0{,}02\,\text{мм}$.
    На какой высоте должен лететь самолет, чтобы на фотографии можно
    было различить следы размером $25\,\text{см}$?
    При какой скорости самолета изображение не будет размытым,
    если время экспозиции $2\,\text{мс}$?
}
\answer{%
    \begin{align*}
    &F \ll H \implies b = F, a = H, \\
    &\Gamma = \frac \delta\ell = \frac ba = \frac FH \implies H = \frac{\ell F}\delta = \frac{25\,\text{см} \cdot 20\,\text{см}}{0{,}02\,\text{мм}} \approx 3\,\text{км}, \\
    &\implies v = \frac l\tau = \frac{25\,\text{см}}{2\,\text{мс}} \approx 450\,\frac{\text{км}}{\text{ч}}.
    \end{align*}
}
\solutionspace{120pt}

\tasknumber{13}%
\task{%
    Две одинаковые собиращие линзы установлены так, что их главные оптические оси совпадают,
    а главный фокус первой находится там же, где главный фокус второй.
    Расстояние от первой линзы до предмета равно $27\,\text{см}$.
    Чему равно расстояние от изображения объекта во второй линзе до второй линзы?
    Определите также увеличение.
    Фокусное расстояние каждой линзы $35\,\text{см}$.
}
\answer{%
    \begin{align*}
    \frac 1a + \frac 1b &= \frac 1F \implies b = \frac{aF}{a - F} \implies 2F - b = \frac{2aF - 2F^2 - aF}{a - F} = \frac{F(a - 2F)}{a - F}.
    \\
    \frac 1{2F - b} + \frac 1c &= \frac 1F \implies c = \frac{F(2F-b)}{(2F - b) - F} = \frac{F \cdot \frac{F(a - 2F)}{a - F}}{\frac{F(a - 2F)}{a - F} - F}  = F \cdot \frac{ \frac{F(a - 2F)}{a - F} }{ \frac{F(a - 2F)}{a - F} - 1} = \\
     &= F \cdot \frac{a - 2F}{a - 2F - a + F} = 2F - a = 43\,\text{см}.
     \\
    \ell &= a + 2F + c = 4F = 140\,\text{см}.
    \\
    &\Gamma = \Gamma_1 \cdot \Gamma_2 = \frac ba \cdot \frac c{2F-b} = \frac F{a - F} \cdot \frac{2F - a}{\frac{F(a - 2F)}{a - F}} = -1.
    \end{align*}
}
\solutionspace{120pt}

\tasknumber{14}%
\task{%
    Собирающая линза с фокусным расстоянием $F_1 > 0$ и рассеивающая линза с фокусным расстоянием $F_2 < 0$
    установлены коаксиально на расстоянии $\ell$.
    Пучок параллельных лучей падает на рассеивающую линзу.
    Сделайте схематичное построение и определите, в какой точке система из этих линз соберёт пучок.
}
\answer{%
    \begin{align*}
    &\text{Если пучок падает на собирающую линзу:} \\
    \frac 1{\infty} + \frac 1b &= \frac 1{F_1} \implies b = F_1 \implies \ell - b = \ell - F_1 \\
    \frac 1{\ell - b} + \frac 1c &= \frac 1{F_2} \implies c = \frac{F_2(\ell - b)}{\ell - b - F_2} = \frac{F_2(\ell - F_1)}{\ell - F_1 - F_2}.
    \\
    &\text{Если же пучок падает на рассеивающую линзу:} \\
    \frac 1{\infty} + \frac 1b &= \frac 1{F_2} \implies b = F_2 \implies \ell - b = \ell - F_2 \\
    \frac 1{\ell - b} + \frac 1c &= \frac 1{F_1} \implies c = \frac{F_1(\ell - b)}{\ell - b - F_1} = \frac{F_1(\ell - F_2)}{\ell - F_2 - F_1}.
    \end{align*}
}
\solutionspace{120pt}

\tasknumber{15}%
\task{%
    Две собирающих линзы с фокусными расстояниями $50\,\text{см}$ и $45\,\text{см}$ расположены так,
    что их оптические оси совмещены.
    На первую линзу падает пучок параллельных лучей.
    Пройдя через вторую линзу, он остался параллельным.
    Найдите расстояние между линзами и сделайте рисунок.
}
\answer{%
    \begin{align*}
    \frac 1\infty + \frac 1b &= \frac 1{F_1} \implies b = F_1, \\
    \frac 1{\ell - b} + \frac 1{\infty} &= \frac 1{F_2} \implies \ell - b = F_2 \implies \ell = b + F_2 = F_1 + F_2 = 95\,\text{см}.
    \end{align*}
}

\variantsplitter

\addpersonalvariant{Алина Полканова}

\tasknumber{1}%
\task{%
    Найти оптическую силу собирающей линзы, если действительное изображение предмета,
    помещённого в $35\,\text{см}$ от линзы, получается на расстоянии $20\,\text{см}$ от неё.
}
\answer{%
    $D = \frac 1F = \frac 1a + \frac 1b = \frac 1{35\,\text{см}} + \frac 1{20\,\text{см}} \approx 7{,}86\,\text{дптр}$
}
\solutionspace{80pt}

\tasknumber{2}%
\task{%
    Найти увеличение изображения, если изображение предмета, находящегося
    на расстоянии $25\,\text{см}$ от линзы, получается на расстоянии $12\,\text{см}$ от неё.
}
\answer{%
    $\Gamma = \frac ba = \frac {12\,\text{см}}{25\,\text{см}} \approx 0{,}5$
}
\solutionspace{80pt}

\tasknumber{3}%
\task{%
    Расстояние от предмета до линзы $12\,\text{см}$, а от линзы до мнимого изображения $20\,\text{см}$.
    Чему равно фокусное расстояние линзы?
}
\answer{%
    $\pm \frac 1F = \frac 1a - \frac 1b \implies F = \frac{a b}{\abs{b - a}} \approx 30\,\text{см}$
}
\solutionspace{80pt}

\tasknumber{4}%
\task{%
    Две тонкие собирающие линзы с фокусными расстояниями $25\,\text{см}$ и $30\,\text{см}$ сложены вместе.
    Чему равно фокусное расстояние такой оптической системы?
}
\answer{%
    $\frac 1{f_1} = \frac 1a + \frac 1b; \frac 1{f_2} = - \frac 1b + \frac 1c \implies \frac 1{f_1} + \frac 1{f_2} = \frac 1a + \frac 1c \implies f' = \frac 1{\frac 1{f_1} + \frac 1{f_2}} = \frac{f_1 f_2}{f_1 + f_2} \approx 13{,}6\,\text{см}$
}
\solutionspace{80pt}

\tasknumber{5}%
\task{%
    Линейные размеры прямого изображения предмета, полученного в собирающей линзе,
    в два раза больше линейных размеров предмета.
    Зная, что предмет находится на $40\,\text{см}$ ближе к линзе,
    чем его изображение, найти оптическую силу линзы.
}
\answer{%
    \begin{align*}
    &\text{Если изображение действительное:} \\
    D &= \frac 1F = \frac 1a + \frac 1b, \qquad \Gamma = \frac ba, \qquad b - a = \ell \implies b = \Gamma a \implies \Gamma a - a = \ell \implies  \\
    a &= \frac {\ell}{\Gamma - 1} \implies b = \frac {{\ell} \Gamma}{\Gamma - 1} \implies  \\
    D &= \frac {\Gamma - 1}\ell + \frac {\Gamma - 1}{\ell \Gamma} = \frac 1\ell \cdot \cbr{\Gamma - 1 + \frac {\Gamma - 1}{\Gamma} } =\frac 1\ell \cdot \cbr{\Gamma - \frac 1\Gamma} \approx 3{,}8\,\text{дптр}.
    \\
    &\text{Если изображение мнимое:} \\
    D &= \frac 1F = \frac 1a - \frac 1b, \qquad \Gamma = \frac ba, \qquad b - a = \ell \implies b = \Gamma a \implies \Gamma a - a = \ell \implies  \\
    a &= \frac {\ell}{\Gamma - 1} \implies b = \frac {{\ell} \Gamma}{\Gamma - 1} \implies  \\
    D &= \frac {\Gamma - 1}\ell - \frac {\Gamma - 1}{\ell \Gamma} = \frac 1\ell \cdot \cbr{\Gamma - 1 - \frac {\Gamma - 1}{\Gamma} } =\frac 1\ell \cdot \cbr{\Gamma + \frac 1\Gamma - 2} \approx 1{,}2\,\text{дптр}.
    \\
    &\text{В ответе надо указать оба значения.}
    \end{align*}
}
\solutionspace{120pt}

\tasknumber{6}%
\task{%
    Оптическая сила объектива фотоаппарата равна $3\,\text{дптр}$.
    При фотографировании чертежа с расстояния $0{,}9\,\text{м}$ площадь изображения
    чертежа на фотопластинке оказалась равной $4\,\text{см}^{2}$.
    Какова площадь самого чертежа? Ответ выразите в квадратных сантиметрах.
}
\answer{%
    \begin{align*}
    &\frac 1a + \frac 1b = \frac 1F = D \implies b = \frac{aF}{a - F} \\
    &\frac {S'}S = \Gamma^2 = \sqr{\frac ba} = \sqr{\frac F{a - F}} \implies \\
    &\implies S = S' \cdot \sqr{\frac{a - F}F} = S' \cdot \sqr{\frac aF - 1} = S' \cdot \sqr{aD - 1} \approx 12\,\text{см}^{2}.
    \end{align*}
}


\variantsplitter


\addpersonalvariant{Алина Полканова}

\tasknumber{7}%
\task{%
    В каком месте на главной оптической оси двояковыгнутой линзы
    нужно поместить точечный источник света,
    чтобы его изображение оказалось в главном фокусе линзы?
}
\answer{%
    $\text{на половине фокусного расстояния}$
}
\solutionspace{120pt}

\tasknumber{8}%
\task{%
    Предмет в виде отрезка длиной $\ell$ расположен вдоль оптической оси
    собирающей линзы с фокусным расстоянием $F$.
    Середина отрезка расположена
    на расстоянии $a$ от линзы, которая даёт действительное изображение
    всех точек предмета.
    Определить продольное увеличение предмета.
}
\answer{%
    \begin{align*}
    \frac 1{a + \frac \ell 2} &+ \frac 1b = \frac 1F \implies b = \frac{F\cbr{a + \frac \ell 2}}{a + \frac \ell 2 - F} \\
    \frac 1{a - \frac \ell 2} &+ \frac 1c = \frac 1F \implies c = \frac{F\cbr{a - \frac \ell 2}}{a - \frac \ell 2 - F} \\
    \abs{b - c} &= \abs{\frac{F\cbr{a + \frac \ell 2}}{a + \frac \ell 2 - F} - \frac{F\cbr{a - \frac \ell 2}}{a - \frac \ell 2 - F}}= F\abs{\frac{\cbr{a + \frac \ell 2}\cbr{a - \frac \ell 2 - F} - \cbr{a - \frac \ell 2}\cbr{a + \frac \ell 2 - F}}{ \cbr{a + \frac \ell 2 - F} \cbr{a - \frac \ell 2 - F} }} =  \\
    &= F\abs{\frac{a^2 - \frac {a\ell} 2 - Fa + \frac {a\ell} 2 - \frac {\ell^2} 4 - \frac {F\ell}2 - a^2 - \frac {a\ell}2 + aF + \frac {a\ell}2 + \frac {\ell^2} 4 - \frac {F\ell} 2}{\cbr{a + \frac \ell 2 - F} \cbr{a - \frac \ell 2 - F} }} = \\
    &= F\frac{F\ell}{\sqr{a-F} - \frac {\ell^2}4} = \frac{F^2\ell}{\sqr{a-F} - \frac {\ell^2}4}\implies \Gamma = \frac{\abs{b - c}}\ell = \frac{F^2}{\sqr{a-F} - \frac {\ell^2}4}.
    \end{align*}
}
\solutionspace{120pt}

\tasknumber{9}%
\task{%
    На экране с помощью тонкой линзы получено изображение предмета
    с увеличением $4$.
    Предмет передвинули на $10\,\text{см}$.
    Для того, чтобы получить резкое изображение, пришлось передвинуть экран.
    При этом увеличение оказалось равным $8$.
    На какое расстояние
    пришлось передвинуть экран?
}
\answer{%
    \begin{align*}
    &\frac 1a + \frac 1b = \frac 1F, \Gamma_1 = \frac ba = \frac{F}{a-F} \implies \Gamma_1(a-F) = F \implies a = F \cdot \frac{1 + \Gamma_1}{\Gamma_1} \\
    &\frac 1{a + x} + \frac 1{b + y} = \frac 1F, \Gamma_2 = \frac {b+y}{a+x} = \frac{F}{a+x-F} \implies a + x = F \cdot \frac{1 + \Gamma_2}{\Gamma_2} \\
    &1 + \frac xa = \frac{ \frac{1 + \Gamma_2}{\Gamma_2} }{ \frac{1 + \Gamma_1}{\Gamma_1} } = \frac{\Gamma_1(1 + \Gamma_2)}{\Gamma_2(1 + \Gamma_1)} \\
    &a = \frac x{ \frac{\Gamma_1(1 + \Gamma_2)}{\Gamma_2(1 + \Gamma_1)} - 1} = x \cdot \frac{\Gamma_2(1 + \Gamma_1)}{\Gamma_1 - \Gamma_2} \\
    &y = (a + x)\Gamma_2 - b = (a + x)\Gamma_2 - a\Gamma_1 = a(\Gamma_2 - \Gamma_1) + x\Gamma_2 = -x\Gamma_2(1 + \Gamma_1) + x\Gamma_2 = -x\Gamma_2\Gamma_1 = 320\,\text{см}, \\
    &\text{знаки разные, т.е.
    экран надо было подвинуть в другую сторону чем предмет: $x < 0, y > 0$.}
    \end{align*}
}
\solutionspace{120pt}

\tasknumber{10}%
\task{%
    Тонкая собирающая линза дает изображение предмета на экране при двух положениях линзы между предметом и экраном.
    Высота изображения при первом положении $30\,\text{см}$, во втором — $7\,\text{см}$.
    Расстояние между предметом и экранов постоянно.
    Чему равна высота предмета?
}
\answer{%
    \begin{align*}
    &\frac 1a + \frac 1b = \frac 1F, \frac 1c + \frac 1d = \frac 1F, a + b = c + d \implies \frac{a + b}{ab} = \frac 1F = \frac{c+d}{cd} \implies ab = cd, \\
    &\implies ab = c(a + b - c) \implies c^2 - ac - bc + ab = 0 \implies c = a \text{ или } c = b \implies c = b \implies d = a.
    \\
    &\Gamma_1 = \frac {H_1}H = \frac ba, \Gamma_2 = \frac {H_2}H = \frac dc = \frac ab \implies \frac {H_1}H \cdot \frac {H_2}H = \frac ba \cdot \frac ab = 1, \\
    &H = \sqrt{H_1 H_2} \approx 14{,}5\,\text{см}.
    \end{align*}
}
\solutionspace{120pt}

\tasknumber{11}%
\task{%
    Какие предметы можно рассмотреть на фотографии, сделанной со спутника,
    если разрешающая способность плёнки $0{,}010\,\text{мм}$? Каким должно быть
    время экспозиции $\tau$ чтобы полностью использовать возможности плёнки?
    Фокусное расстояние объектива используемого фотоаппарата $10\,\text{см}$,
    высота орбиты спутника $100\,\text{км}$.
}
\answer{%
    \begin{align*}
    &H \ll R \implies v = v_{\text{I}} = \sqrt{G R} \approx 7{,}9\,\frac{\text{км}}{\text{с}}.
    \\
    &F \ll H \implies b = F, a = H, \\
    &\Gamma = \frac \delta\ell = \frac ba \implies \ell = \frac{\delta a}b = \frac{\delta H}F \approx \frac{0{,}010\,\text{мм} \cdot 100\,\text{км}}{10\,\text{см}} \approx 10\,\text{м}, \\
    &\implies \tau = \frac \ell v = \frac{\delta H}{F v} = \frac{0{,}010\,\text{мм} \cdot 100\,\text{км}}{10\,\text{см} \cdot 7{,}9\,\frac{\text{км}}{\text{с}}} \approx 1{,}3\,\text{мс}.
    \end{align*}
}


\variantsplitter


\addpersonalvariant{Алина Полканова}

\tasknumber{12}%
\task{%
    При аэрофотосъемках используется фотоаппарат, объектив которого
    имеет фокусиое расстояние $15\,\text{см}$.
    Разрешающая способность плёнки $0{,}010\,\text{мм}$.
    На какой высоте должен лететь самолет, чтобы на фотографии можно
    было различить следы размером $20\,\text{см}$?
    При какой скорости самолета изображение не будет размытым,
    если время экспозиции $2\,\text{мс}$?
}
\answer{%
    \begin{align*}
    &F \ll H \implies b = F, a = H, \\
    &\Gamma = \frac \delta\ell = \frac ba = \frac FH \implies H = \frac{\ell F}\delta = \frac{20\,\text{см} \cdot 15\,\text{см}}{0{,}010\,\text{мм}} \approx 3\,\text{км}, \\
    &\implies v = \frac l\tau = \frac{20\,\text{см}}{2\,\text{мс}} \approx 360\,\frac{\text{км}}{\text{ч}}.
    \end{align*}
}
\solutionspace{120pt}

\tasknumber{13}%
\task{%
    Две одинаковые собиращие линзы установлены так, что их главные оптические оси совпадают,
    а главный фокус первой находится там же, где главный фокус второй.
    Расстояние от первой линзы до предмета равно $14\,\text{см}$.
    Чему равно расстояние от изображения объекта во второй линзе до второй линзы?
    Определите также увеличение.
    Фокусное расстояние каждой линзы $25\,\text{см}$.
}
\answer{%
    \begin{align*}
    \frac 1a + \frac 1b &= \frac 1F \implies b = \frac{aF}{a - F} \implies 2F - b = \frac{2aF - 2F^2 - aF}{a - F} = \frac{F(a - 2F)}{a - F}.
    \\
    \frac 1{2F - b} + \frac 1c &= \frac 1F \implies c = \frac{F(2F-b)}{(2F - b) - F} = \frac{F \cdot \frac{F(a - 2F)}{a - F}}{\frac{F(a - 2F)}{a - F} - F}  = F \cdot \frac{ \frac{F(a - 2F)}{a - F} }{ \frac{F(a - 2F)}{a - F} - 1} = \\
     &= F \cdot \frac{a - 2F}{a - 2F - a + F} = 2F - a = 36\,\text{см}.
     \\
    \ell &= a + 2F + c = 4F = 100\,\text{см}.
    \\
    &\Gamma = \Gamma_1 \cdot \Gamma_2 = \frac ba \cdot \frac c{2F-b} = \frac F{a - F} \cdot \frac{2F - a}{\frac{F(a - 2F)}{a - F}} = -1.
    \end{align*}
}
\solutionspace{120pt}

\tasknumber{14}%
\task{%
    Собирающая линза с фокусным расстоянием $F_1 > 0$ и рассеивающая линза с фокусным расстоянием $F_2 < 0$
    установлены коаксиально на расстоянии $\ell$.
    Пучок параллельных лучей падает на рассеивающую линзу.
    Сделайте схематичное построение и определите, в какой точке система из этих линз соберёт пучок.
}
\answer{%
    \begin{align*}
    &\text{Если пучок падает на собирающую линзу:} \\
    \frac 1{\infty} + \frac 1b &= \frac 1{F_1} \implies b = F_1 \implies \ell - b = \ell - F_1 \\
    \frac 1{\ell - b} + \frac 1c &= \frac 1{F_2} \implies c = \frac{F_2(\ell - b)}{\ell - b - F_2} = \frac{F_2(\ell - F_1)}{\ell - F_1 - F_2}.
    \\
    &\text{Если же пучок падает на рассеивающую линзу:} \\
    \frac 1{\infty} + \frac 1b &= \frac 1{F_2} \implies b = F_2 \implies \ell - b = \ell - F_2 \\
    \frac 1{\ell - b} + \frac 1c &= \frac 1{F_1} \implies c = \frac{F_1(\ell - b)}{\ell - b - F_1} = \frac{F_1(\ell - F_2)}{\ell - F_2 - F_1}.
    \end{align*}
}
\solutionspace{120pt}

\tasknumber{15}%
\task{%
    Две собирающих линзы с фокусными расстояниями $30\,\text{см}$ и $25\,\text{см}$ расположены так,
    что их оптические оси совмещены.
    На первую линзу падает пучок параллельных лучей.
    Пройдя через вторую линзу, он остался параллельным.
    Найдите расстояние между линзами и сделайте рисунок.
}
\answer{%
    \begin{align*}
    \frac 1\infty + \frac 1b &= \frac 1{F_1} \implies b = F_1, \\
    \frac 1{\ell - b} + \frac 1{\infty} &= \frac 1{F_2} \implies \ell - b = F_2 \implies \ell = b + F_2 = F_1 + F_2 = 55\,\text{см}.
    \end{align*}
}

\variantsplitter

\addpersonalvariant{Сергей Пономарёв}

\tasknumber{1}%
\task{%
    Найти оптическую силу собирающей линзы, если действительное изображение предмета,
    помещённого в $35\,\text{см}$ от линзы, получается на расстоянии $20\,\text{см}$ от неё.
}
\answer{%
    $D = \frac 1F = \frac 1a + \frac 1b = \frac 1{35\,\text{см}} + \frac 1{20\,\text{см}} \approx 7{,}86\,\text{дптр}$
}
\solutionspace{80pt}

\tasknumber{2}%
\task{%
    Найти увеличение изображения, если изображение предмета, находящегося
    на расстоянии $20\,\text{см}$ от линзы, получается на расстоянии $18\,\text{см}$ от неё.
}
\answer{%
    $\Gamma = \frac ba = \frac {18\,\text{см}}{20\,\text{см}} \approx 0{,}9$
}
\solutionspace{80pt}

\tasknumber{3}%
\task{%
    Расстояние от предмета до линзы $10\,\text{см}$, а от линзы до мнимого изображения $25\,\text{см}$.
    Чему равно фокусное расстояние линзы?
}
\answer{%
    $\pm \frac 1F = \frac 1a - \frac 1b \implies F = \frac{a b}{\abs{b - a}} \approx 16{,}7\,\text{см}$
}
\solutionspace{80pt}

\tasknumber{4}%
\task{%
    Две тонкие собирающие линзы с фокусными расстояниями $18\,\text{см}$ и $30\,\text{см}$ сложены вместе.
    Чему равно фокусное расстояние такой оптической системы?
}
\answer{%
    $\frac 1{f_1} = \frac 1a + \frac 1b; \frac 1{f_2} = - \frac 1b + \frac 1c \implies \frac 1{f_1} + \frac 1{f_2} = \frac 1a + \frac 1c \implies f' = \frac 1{\frac 1{f_1} + \frac 1{f_2}} = \frac{f_1 f_2}{f_1 + f_2} \approx 11{,}2\,\text{см}$
}
\solutionspace{80pt}

\tasknumber{5}%
\task{%
    Линейные размеры прямого изображения предмета, полученного в собирающей линзе,
    в три раза больше линейных размеров предмета.
    Зная, что предмет находится на $25\,\text{см}$ ближе к линзе,
    чем его изображение, найти оптическую силу линзы.
}
\answer{%
    \begin{align*}
    &\text{Если изображение действительное:} \\
    D &= \frac 1F = \frac 1a + \frac 1b, \qquad \Gamma = \frac ba, \qquad b - a = \ell \implies b = \Gamma a \implies \Gamma a - a = \ell \implies  \\
    a &= \frac {\ell}{\Gamma - 1} \implies b = \frac {{\ell} \Gamma}{\Gamma - 1} \implies  \\
    D &= \frac {\Gamma - 1}\ell + \frac {\Gamma - 1}{\ell \Gamma} = \frac 1\ell \cdot \cbr{\Gamma - 1 + \frac {\Gamma - 1}{\Gamma} } =\frac 1\ell \cdot \cbr{\Gamma - \frac 1\Gamma} \approx 10{,}7\,\text{дптр}.
    \\
    &\text{Если изображение мнимое:} \\
    D &= \frac 1F = \frac 1a - \frac 1b, \qquad \Gamma = \frac ba, \qquad b - a = \ell \implies b = \Gamma a \implies \Gamma a - a = \ell \implies  \\
    a &= \frac {\ell}{\Gamma - 1} \implies b = \frac {{\ell} \Gamma}{\Gamma - 1} \implies  \\
    D &= \frac {\Gamma - 1}\ell - \frac {\Gamma - 1}{\ell \Gamma} = \frac 1\ell \cdot \cbr{\Gamma - 1 - \frac {\Gamma - 1}{\Gamma} } =\frac 1\ell \cdot \cbr{\Gamma + \frac 1\Gamma - 2} \approx 5{,}3\,\text{дптр}.
    \\
    &\text{В ответе надо указать оба значения.}
    \end{align*}
}
\solutionspace{120pt}

\tasknumber{6}%
\task{%
    Оптическая сила объектива фотоаппарата равна $4\,\text{дптр}$.
    При фотографировании чертежа с расстояния $0{,}8\,\text{м}$ площадь изображения
    чертежа на фотопластинке оказалась равной $9\,\text{см}^{2}$.
    Какова площадь самого чертежа? Ответ выразите в квадратных сантиметрах.
}
\answer{%
    \begin{align*}
    &\frac 1a + \frac 1b = \frac 1F = D \implies b = \frac{aF}{a - F} \\
    &\frac {S'}S = \Gamma^2 = \sqr{\frac ba} = \sqr{\frac F{a - F}} \implies \\
    &\implies S = S' \cdot \sqr{\frac{a - F}F} = S' \cdot \sqr{\frac aF - 1} = S' \cdot \sqr{aD - 1} \approx 40\,\text{см}^{2}.
    \end{align*}
}


\variantsplitter


\addpersonalvariant{Сергей Пономарёв}

\tasknumber{7}%
\task{%
    В каком месте на главной оптической оси двояковыпуклой линзы
    нужно поместить точечный источник света,
    чтобы его изображение оказалось в главном фокусе линзы?
}
\answer{%
    $\text{для мнимого - на половине фокусного, для действительного - на бесконечности}$
}
\solutionspace{120pt}

\tasknumber{8}%
\task{%
    Предмет в виде отрезка длиной $\ell$ расположен вдоль оптической оси
    собирающей линзы с фокусным расстоянием $F$.
    Середина отрезка расположена
    на расстоянии $a$ от линзы, которая даёт действительное изображение
    всех точек предмета.
    Определить продольное увеличение предмета.
}
\answer{%
    \begin{align*}
    \frac 1{a + \frac \ell 2} &+ \frac 1b = \frac 1F \implies b = \frac{F\cbr{a + \frac \ell 2}}{a + \frac \ell 2 - F} \\
    \frac 1{a - \frac \ell 2} &+ \frac 1c = \frac 1F \implies c = \frac{F\cbr{a - \frac \ell 2}}{a - \frac \ell 2 - F} \\
    \abs{b - c} &= \abs{\frac{F\cbr{a + \frac \ell 2}}{a + \frac \ell 2 - F} - \frac{F\cbr{a - \frac \ell 2}}{a - \frac \ell 2 - F}}= F\abs{\frac{\cbr{a + \frac \ell 2}\cbr{a - \frac \ell 2 - F} - \cbr{a - \frac \ell 2}\cbr{a + \frac \ell 2 - F}}{ \cbr{a + \frac \ell 2 - F} \cbr{a - \frac \ell 2 - F} }} =  \\
    &= F\abs{\frac{a^2 - \frac {a\ell} 2 - Fa + \frac {a\ell} 2 - \frac {\ell^2} 4 - \frac {F\ell}2 - a^2 - \frac {a\ell}2 + aF + \frac {a\ell}2 + \frac {\ell^2} 4 - \frac {F\ell} 2}{\cbr{a + \frac \ell 2 - F} \cbr{a - \frac \ell 2 - F} }} = \\
    &= F\frac{F\ell}{\sqr{a-F} - \frac {\ell^2}4} = \frac{F^2\ell}{\sqr{a-F} - \frac {\ell^2}4}\implies \Gamma = \frac{\abs{b - c}}\ell = \frac{F^2}{\sqr{a-F} - \frac {\ell^2}4}.
    \end{align*}
}
\solutionspace{120pt}

\tasknumber{9}%
\task{%
    На экране с помощью тонкой линзы получено изображение предмета
    с увеличением $4$.
    Предмет передвинули на $4\,\text{см}$.
    Для того, чтобы получить резкое изображение, пришлось передвинуть экран.
    При этом увеличение оказалось равным $8$.
    На какое расстояние
    пришлось передвинуть экран?
}
\answer{%
    \begin{align*}
    &\frac 1a + \frac 1b = \frac 1F, \Gamma_1 = \frac ba = \frac{F}{a-F} \implies \Gamma_1(a-F) = F \implies a = F \cdot \frac{1 + \Gamma_1}{\Gamma_1} \\
    &\frac 1{a + x} + \frac 1{b + y} = \frac 1F, \Gamma_2 = \frac {b+y}{a+x} = \frac{F}{a+x-F} \implies a + x = F \cdot \frac{1 + \Gamma_2}{\Gamma_2} \\
    &1 + \frac xa = \frac{ \frac{1 + \Gamma_2}{\Gamma_2} }{ \frac{1 + \Gamma_1}{\Gamma_1} } = \frac{\Gamma_1(1 + \Gamma_2)}{\Gamma_2(1 + \Gamma_1)} \\
    &a = \frac x{ \frac{\Gamma_1(1 + \Gamma_2)}{\Gamma_2(1 + \Gamma_1)} - 1} = x \cdot \frac{\Gamma_2(1 + \Gamma_1)}{\Gamma_1 - \Gamma_2} \\
    &y = (a + x)\Gamma_2 - b = (a + x)\Gamma_2 - a\Gamma_1 = a(\Gamma_2 - \Gamma_1) + x\Gamma_2 = -x\Gamma_2(1 + \Gamma_1) + x\Gamma_2 = -x\Gamma_2\Gamma_1 = 128\,\text{см}, \\
    &\text{знаки разные, т.е.
    экран надо было подвинуть в другую сторону чем предмет: $x < 0, y > 0$.}
    \end{align*}
}
\solutionspace{120pt}

\tasknumber{10}%
\task{%
    Тонкая собирающая линза дает изображение предмета на экране при двух положениях линзы между предметом и экраном.
    Высота изображения при первом положении $30\,\text{см}$, во втором — $9\,\text{см}$.
    Расстояние между предметом и экранов постоянно.
    Чему равна высота предмета?
}
\answer{%
    \begin{align*}
    &\frac 1a + \frac 1b = \frac 1F, \frac 1c + \frac 1d = \frac 1F, a + b = c + d \implies \frac{a + b}{ab} = \frac 1F = \frac{c+d}{cd} \implies ab = cd, \\
    &\implies ab = c(a + b - c) \implies c^2 - ac - bc + ab = 0 \implies c = a \text{ или } c = b \implies c = b \implies d = a.
    \\
    &\Gamma_1 = \frac {H_1}H = \frac ba, \Gamma_2 = \frac {H_2}H = \frac dc = \frac ab \implies \frac {H_1}H \cdot \frac {H_2}H = \frac ba \cdot \frac ab = 1, \\
    &H = \sqrt{H_1 H_2} \approx 16{,}4\,\text{см}.
    \end{align*}
}
\solutionspace{120pt}

\tasknumber{11}%
\task{%
    Какие предметы можно рассмотреть на фотографии, сделанной со спутника,
    если разрешающая способность плёнки $0{,}010\,\text{мм}$? Каким должно быть
    время экспозиции $\tau$ чтобы полностью использовать возможности плёнки?
    Фокусное расстояние объектива используемого фотоаппарата $20\,\text{см}$,
    высота орбиты спутника $80\,\text{км}$.
}
\answer{%
    \begin{align*}
    &H \ll R \implies v = v_{\text{I}} = \sqrt{G R} \approx 7{,}9\,\frac{\text{км}}{\text{с}}.
    \\
    &F \ll H \implies b = F, a = H, \\
    &\Gamma = \frac \delta\ell = \frac ba \implies \ell = \frac{\delta a}b = \frac{\delta H}F \approx \frac{0{,}010\,\text{мм} \cdot 80\,\text{км}}{20\,\text{см}} \approx 4\,\text{м}, \\
    &\implies \tau = \frac \ell v = \frac{\delta H}{F v} = \frac{0{,}010\,\text{мм} \cdot 80\,\text{км}}{20\,\text{см} \cdot 7{,}9\,\frac{\text{км}}{\text{с}}} \approx 0{,}5\,\text{мс}.
    \end{align*}
}


\variantsplitter


\addpersonalvariant{Сергей Пономарёв}

\tasknumber{12}%
\task{%
    При аэрофотосъемках используется фотоаппарат, объектив которого
    имеет фокусиое расстояние $15\,\text{см}$.
    Разрешающая способность плёнки $0{,}02\,\text{мм}$.
    На какой высоте должен лететь самолет, чтобы на фотографии можно
    было различить следы размером $15\,\text{см}$?
    При какой скорости самолета изображение не будет размытым,
    если время экспозиции $2\,\text{мс}$?
}
\answer{%
    \begin{align*}
    &F \ll H \implies b = F, a = H, \\
    &\Gamma = \frac \delta\ell = \frac ba = \frac FH \implies H = \frac{\ell F}\delta = \frac{15\,\text{см} \cdot 15\,\text{см}}{0{,}02\,\text{мм}} \approx 1{,}1\,\text{км}, \\
    &\implies v = \frac l\tau = \frac{15\,\text{см}}{2\,\text{мс}} \approx 270\,\frac{\text{км}}{\text{ч}}.
    \end{align*}
}
\solutionspace{120pt}

\tasknumber{13}%
\task{%
    Две одинаковые собиращие линзы установлены так, что их главные оптические оси совпадают,
    а главный фокус первой находится там же, где главный фокус второй.
    Расстояние от первой линзы до предмета равно $13\,\text{см}$.
    Чему равно расстояние от изображения объекта во второй линзе до самого объекта?
    Определите также увеличение.
    Фокусное расстояние каждой линзы $35\,\text{см}$.
}
\answer{%
    \begin{align*}
    \frac 1a + \frac 1b &= \frac 1F \implies b = \frac{aF}{a - F} \implies 2F - b = \frac{2aF - 2F^2 - aF}{a - F} = \frac{F(a - 2F)}{a - F}.
    \\
    \frac 1{2F - b} + \frac 1c &= \frac 1F \implies c = \frac{F(2F-b)}{(2F - b) - F} = \frac{F \cdot \frac{F(a - 2F)}{a - F}}{\frac{F(a - 2F)}{a - F} - F}  = F \cdot \frac{ \frac{F(a - 2F)}{a - F} }{ \frac{F(a - 2F)}{a - F} - 1} = \\
     &= F \cdot \frac{a - 2F}{a - 2F - a + F} = 2F - a = 57\,\text{см}.
     \\
    \ell &= a + 2F + c = 4F = 140\,\text{см}.
    \\
    &\Gamma = \Gamma_1 \cdot \Gamma_2 = \frac ba \cdot \frac c{2F-b} = \frac F{a - F} \cdot \frac{2F - a}{\frac{F(a - 2F)}{a - F}} = -1.
    \end{align*}
}
\solutionspace{120pt}

\tasknumber{14}%
\task{%
    Собирающая линза с фокусным расстоянием $F_1 > 0$ и рассеивающая линза с фокусным расстоянием $F_2 < 0$
    установлены коаксиально на расстоянии $\ell$.
    Пучок параллельных лучей падает на собирающую линзу.
    Сделайте схематичное построение и определите, в какой точке система из этих линз соберёт пучок.
}
\answer{%
    \begin{align*}
    &\text{Если пучок падает на собирающую линзу:} \\
    \frac 1{\infty} + \frac 1b &= \frac 1{F_1} \implies b = F_1 \implies \ell - b = \ell - F_1 \\
    \frac 1{\ell - b} + \frac 1c &= \frac 1{F_2} \implies c = \frac{F_2(\ell - b)}{\ell - b - F_2} = \frac{F_2(\ell - F_1)}{\ell - F_1 - F_2}.
    \\
    &\text{Если же пучок падает на рассеивающую линзу:} \\
    \frac 1{\infty} + \frac 1b &= \frac 1{F_2} \implies b = F_2 \implies \ell - b = \ell - F_2 \\
    \frac 1{\ell - b} + \frac 1c &= \frac 1{F_1} \implies c = \frac{F_1(\ell - b)}{\ell - b - F_1} = \frac{F_1(\ell - F_2)}{\ell - F_2 - F_1}.
    \end{align*}
}
\solutionspace{120pt}

\tasknumber{15}%
\task{%
    Две собирающих линзы с фокусными расстояниями $40\,\text{см}$ и $35\,\text{см}$ расположены так,
    что их оптические оси совмещены.
    На первую линзу падает пучок параллельных лучей.
    Пройдя через вторую линзу, он остался параллельным.
    Найдите расстояние между линзами и сделайте рисунок.
}
\answer{%
    \begin{align*}
    \frac 1\infty + \frac 1b &= \frac 1{F_1} \implies b = F_1, \\
    \frac 1{\ell - b} + \frac 1{\infty} &= \frac 1{F_2} \implies \ell - b = F_2 \implies \ell = b + F_2 = F_1 + F_2 = 75\,\text{см}.
    \end{align*}
}

\variantsplitter

\addpersonalvariant{Егор Свистушкин}

\tasknumber{1}%
\task{%
    Найти оптическую силу собирающей линзы, если действительное изображение предмета,
    помещённого в $55\,\text{см}$ от линзы, получается на расстоянии $30\,\text{см}$ от неё.
}
\answer{%
    $D = \frac 1F = \frac 1a + \frac 1b = \frac 1{55\,\text{см}} + \frac 1{30\,\text{см}} \approx 5{,}15\,\text{дптр}$
}
\solutionspace{80pt}

\tasknumber{2}%
\task{%
    Найти увеличение изображения, если изображение предмета, находящегося
    на расстоянии $20\,\text{см}$ от линзы, получается на расстоянии $18\,\text{см}$ от неё.
}
\answer{%
    $\Gamma = \frac ba = \frac {18\,\text{см}}{20\,\text{см}} \approx 0{,}9$
}
\solutionspace{80pt}

\tasknumber{3}%
\task{%
    Расстояние от предмета до линзы $10\,\text{см}$, а от линзы до мнимого изображения $25\,\text{см}$.
    Чему равно фокусное расстояние линзы?
}
\answer{%
    $\pm \frac 1F = \frac 1a - \frac 1b \implies F = \frac{a b}{\abs{b - a}} \approx 16{,}7\,\text{см}$
}
\solutionspace{80pt}

\tasknumber{4}%
\task{%
    Две тонкие собирающие линзы с фокусными расстояниями $18\,\text{см}$ и $30\,\text{см}$ сложены вместе.
    Чему равно фокусное расстояние такой оптической системы?
}
\answer{%
    $\frac 1{f_1} = \frac 1a + \frac 1b; \frac 1{f_2} = - \frac 1b + \frac 1c \implies \frac 1{f_1} + \frac 1{f_2} = \frac 1a + \frac 1c \implies f' = \frac 1{\frac 1{f_1} + \frac 1{f_2}} = \frac{f_1 f_2}{f_1 + f_2} \approx 11{,}2\,\text{см}$
}
\solutionspace{80pt}

\tasknumber{5}%
\task{%
    Линейные размеры прямого изображения предмета, полученного в собирающей линзе,
    в три раза больше линейных размеров предмета.
    Зная, что предмет находится на $40\,\text{см}$ ближе к линзе,
    чем его изображение, найти оптическую силу линзы.
}
\answer{%
    \begin{align*}
    &\text{Если изображение действительное:} \\
    D &= \frac 1F = \frac 1a + \frac 1b, \qquad \Gamma = \frac ba, \qquad b - a = \ell \implies b = \Gamma a \implies \Gamma a - a = \ell \implies  \\
    a &= \frac {\ell}{\Gamma - 1} \implies b = \frac {{\ell} \Gamma}{\Gamma - 1} \implies  \\
    D &= \frac {\Gamma - 1}\ell + \frac {\Gamma - 1}{\ell \Gamma} = \frac 1\ell \cdot \cbr{\Gamma - 1 + \frac {\Gamma - 1}{\Gamma} } =\frac 1\ell \cdot \cbr{\Gamma - \frac 1\Gamma} \approx 6{,}7\,\text{дптр}.
    \\
    &\text{Если изображение мнимое:} \\
    D &= \frac 1F = \frac 1a - \frac 1b, \qquad \Gamma = \frac ba, \qquad b - a = \ell \implies b = \Gamma a \implies \Gamma a - a = \ell \implies  \\
    a &= \frac {\ell}{\Gamma - 1} \implies b = \frac {{\ell} \Gamma}{\Gamma - 1} \implies  \\
    D &= \frac {\Gamma - 1}\ell - \frac {\Gamma - 1}{\ell \Gamma} = \frac 1\ell \cdot \cbr{\Gamma - 1 - \frac {\Gamma - 1}{\Gamma} } =\frac 1\ell \cdot \cbr{\Gamma + \frac 1\Gamma - 2} \approx 3{,}3\,\text{дптр}.
    \\
    &\text{В ответе надо указать оба значения.}
    \end{align*}
}
\solutionspace{120pt}

\tasknumber{6}%
\task{%
    Оптическая сила объектива фотоаппарата равна $5\,\text{дптр}$.
    При фотографировании чертежа с расстояния $0{,}8\,\text{м}$ площадь изображения
    чертежа на фотопластинке оказалась равной $16\,\text{см}^{2}$.
    Какова площадь самого чертежа? Ответ выразите в квадратных сантиметрах.
}
\answer{%
    \begin{align*}
    &\frac 1a + \frac 1b = \frac 1F = D \implies b = \frac{aF}{a - F} \\
    &\frac {S'}S = \Gamma^2 = \sqr{\frac ba} = \sqr{\frac F{a - F}} \implies \\
    &\implies S = S' \cdot \sqr{\frac{a - F}F} = S' \cdot \sqr{\frac aF - 1} = S' \cdot \sqr{aD - 1} \approx 144\,\text{см}^{2}.
    \end{align*}
}


\variantsplitter


\addpersonalvariant{Егор Свистушкин}

\tasknumber{7}%
\task{%
    В каком месте на главной оптической оси двояковыгнутой линзы
    нужно поместить точечный источник света,
    чтобы его изображение оказалось в главном фокусе линзы?
}
\answer{%
    $\text{на половине фокусного расстояния}$
}
\solutionspace{120pt}

\tasknumber{8}%
\task{%
    Предмет в виде отрезка длиной $\ell$ расположен вдоль оптической оси
    собирающей линзы с фокусным расстоянием $F$.
    Середина отрезка расположена
    на расстоянии $a$ от линзы, которая даёт действительное изображение
    всех точек предмета.
    Определить продольное увеличение предмета.
}
\answer{%
    \begin{align*}
    \frac 1{a + \frac \ell 2} &+ \frac 1b = \frac 1F \implies b = \frac{F\cbr{a + \frac \ell 2}}{a + \frac \ell 2 - F} \\
    \frac 1{a - \frac \ell 2} &+ \frac 1c = \frac 1F \implies c = \frac{F\cbr{a - \frac \ell 2}}{a - \frac \ell 2 - F} \\
    \abs{b - c} &= \abs{\frac{F\cbr{a + \frac \ell 2}}{a + \frac \ell 2 - F} - \frac{F\cbr{a - \frac \ell 2}}{a - \frac \ell 2 - F}}= F\abs{\frac{\cbr{a + \frac \ell 2}\cbr{a - \frac \ell 2 - F} - \cbr{a - \frac \ell 2}\cbr{a + \frac \ell 2 - F}}{ \cbr{a + \frac \ell 2 - F} \cbr{a - \frac \ell 2 - F} }} =  \\
    &= F\abs{\frac{a^2 - \frac {a\ell} 2 - Fa + \frac {a\ell} 2 - \frac {\ell^2} 4 - \frac {F\ell}2 - a^2 - \frac {a\ell}2 + aF + \frac {a\ell}2 + \frac {\ell^2} 4 - \frac {F\ell} 2}{\cbr{a + \frac \ell 2 - F} \cbr{a - \frac \ell 2 - F} }} = \\
    &= F\frac{F\ell}{\sqr{a-F} - \frac {\ell^2}4} = \frac{F^2\ell}{\sqr{a-F} - \frac {\ell^2}4}\implies \Gamma = \frac{\abs{b - c}}\ell = \frac{F^2}{\sqr{a-F} - \frac {\ell^2}4}.
    \end{align*}
}
\solutionspace{120pt}

\tasknumber{9}%
\task{%
    На экране с помощью тонкой линзы получено изображение предмета
    с увеличением $2$.
    Предмет передвинули на $4\,\text{см}$.
    Для того, чтобы получить резкое изображение, пришлось передвинуть экран.
    При этом увеличение оказалось равным $6$.
    На какое расстояние
    пришлось передвинуть экран?
}
\answer{%
    \begin{align*}
    &\frac 1a + \frac 1b = \frac 1F, \Gamma_1 = \frac ba = \frac{F}{a-F} \implies \Gamma_1(a-F) = F \implies a = F \cdot \frac{1 + \Gamma_1}{\Gamma_1} \\
    &\frac 1{a + x} + \frac 1{b + y} = \frac 1F, \Gamma_2 = \frac {b+y}{a+x} = \frac{F}{a+x-F} \implies a + x = F \cdot \frac{1 + \Gamma_2}{\Gamma_2} \\
    &1 + \frac xa = \frac{ \frac{1 + \Gamma_2}{\Gamma_2} }{ \frac{1 + \Gamma_1}{\Gamma_1} } = \frac{\Gamma_1(1 + \Gamma_2)}{\Gamma_2(1 + \Gamma_1)} \\
    &a = \frac x{ \frac{\Gamma_1(1 + \Gamma_2)}{\Gamma_2(1 + \Gamma_1)} - 1} = x \cdot \frac{\Gamma_2(1 + \Gamma_1)}{\Gamma_1 - \Gamma_2} \\
    &y = (a + x)\Gamma_2 - b = (a + x)\Gamma_2 - a\Gamma_1 = a(\Gamma_2 - \Gamma_1) + x\Gamma_2 = -x\Gamma_2(1 + \Gamma_1) + x\Gamma_2 = -x\Gamma_2\Gamma_1 = 48\,\text{см}, \\
    &\text{знаки разные, т.е.
    экран надо было подвинуть в другую сторону чем предмет: $x < 0, y > 0$.}
    \end{align*}
}
\solutionspace{120pt}

\tasknumber{10}%
\task{%
    Тонкая собирающая линза дает изображение предмета на экране при двух положениях линзы между предметом и экраном.
    Высота изображения при первом положении $15\,\text{см}$, во втором — $5\,\text{см}$.
    Расстояние между предметом и экранов постоянно.
    Чему равна высота предмета?
}
\answer{%
    \begin{align*}
    &\frac 1a + \frac 1b = \frac 1F, \frac 1c + \frac 1d = \frac 1F, a + b = c + d \implies \frac{a + b}{ab} = \frac 1F = \frac{c+d}{cd} \implies ab = cd, \\
    &\implies ab = c(a + b - c) \implies c^2 - ac - bc + ab = 0 \implies c = a \text{ или } c = b \implies c = b \implies d = a.
    \\
    &\Gamma_1 = \frac {H_1}H = \frac ba, \Gamma_2 = \frac {H_2}H = \frac dc = \frac ab \implies \frac {H_1}H \cdot \frac {H_2}H = \frac ba \cdot \frac ab = 1, \\
    &H = \sqrt{H_1 H_2} \approx 8{,}7\,\text{см}.
    \end{align*}
}
\solutionspace{120pt}

\tasknumber{11}%
\task{%
    Какие предметы можно рассмотреть на фотографии, сделанной со спутника,
    если разрешающая способность плёнки $0{,}010\,\text{мм}$? Каким должно быть
    время экспозиции $\tau$ чтобы полностью использовать возможности плёнки?
    Фокусное расстояние объектива используемого фотоаппарата $10\,\text{см}$,
    высота орбиты спутника $150\,\text{км}$.
}
\answer{%
    \begin{align*}
    &H \ll R \implies v = v_{\text{I}} = \sqrt{G R} \approx 7{,}9\,\frac{\text{км}}{\text{с}}.
    \\
    &F \ll H \implies b = F, a = H, \\
    &\Gamma = \frac \delta\ell = \frac ba \implies \ell = \frac{\delta a}b = \frac{\delta H}F \approx \frac{0{,}010\,\text{мм} \cdot 150\,\text{км}}{10\,\text{см}} \approx 15\,\text{м}, \\
    &\implies \tau = \frac \ell v = \frac{\delta H}{F v} = \frac{0{,}010\,\text{мм} \cdot 150\,\text{км}}{10\,\text{см} \cdot 7{,}9\,\frac{\text{км}}{\text{с}}} \approx 1{,}9\,\text{мс}.
    \end{align*}
}


\variantsplitter


\addpersonalvariant{Егор Свистушкин}

\tasknumber{12}%
\task{%
    При аэрофотосъемках используется фотоаппарат, объектив которого
    имеет фокусиое расстояние $10\,\text{см}$.
    Разрешающая способность плёнки $0{,}02\,\text{мм}$.
    На какой высоте должен лететь самолет, чтобы на фотографии можно
    было различить следы размером $30\,\text{см}$?
    При какой скорости самолета изображение не будет размытым,
    если время экспозиции $1\,\text{мс}$?
}
\answer{%
    \begin{align*}
    &F \ll H \implies b = F, a = H, \\
    &\Gamma = \frac \delta\ell = \frac ba = \frac FH \implies H = \frac{\ell F}\delta = \frac{30\,\text{см} \cdot 10\,\text{см}}{0{,}02\,\text{мм}} \approx 1{,}5\,\text{км}, \\
    &\implies v = \frac l\tau = \frac{30\,\text{см}}{1\,\text{мс}} \approx 1080\,\frac{\text{км}}{\text{ч}}.
    \end{align*}
}
\solutionspace{120pt}

\tasknumber{13}%
\task{%
    Две одинаковые собиращие линзы установлены так, что их главные оптические оси совпадают,
    а главный фокус первой находится там же, где главный фокус второй.
    Расстояние от первой линзы до предмета равно $31\,\text{см}$.
    Чему равно расстояние от изображения объекта во второй линзе до второй линзы?
    Определите также увеличение.
    Фокусное расстояние каждой линзы $25\,\text{см}$.
}
\answer{%
    \begin{align*}
    \frac 1a + \frac 1b &= \frac 1F \implies b = \frac{aF}{a - F} \implies 2F - b = \frac{2aF - 2F^2 - aF}{a - F} = \frac{F(a - 2F)}{a - F}.
    \\
    \frac 1{2F - b} + \frac 1c &= \frac 1F \implies c = \frac{F(2F-b)}{(2F - b) - F} = \frac{F \cdot \frac{F(a - 2F)}{a - F}}{\frac{F(a - 2F)}{a - F} - F}  = F \cdot \frac{ \frac{F(a - 2F)}{a - F} }{ \frac{F(a - 2F)}{a - F} - 1} = \\
     &= F \cdot \frac{a - 2F}{a - 2F - a + F} = 2F - a = 19\,\text{см}.
     \\
    \ell &= a + 2F + c = 4F = 100\,\text{см}.
    \\
    &\Gamma = \Gamma_1 \cdot \Gamma_2 = \frac ba \cdot \frac c{2F-b} = \frac F{a - F} \cdot \frac{2F - a}{\frac{F(a - 2F)}{a - F}} = -1.
    \end{align*}
}
\solutionspace{120pt}

\tasknumber{14}%
\task{%
    Собирающая линза с фокусным расстоянием $F_1 > 0$ и рассеивающая линза с фокусным расстоянием $F_2 < 0$
    установлены коаксиально на расстоянии $\ell$.
    Пучок параллельных лучей падает на рассеивающую линзу.
    Сделайте схематичное построение и определите, в какой точке система из этих линз соберёт пучок.
}
\answer{%
    \begin{align*}
    &\text{Если пучок падает на собирающую линзу:} \\
    \frac 1{\infty} + \frac 1b &= \frac 1{F_1} \implies b = F_1 \implies \ell - b = \ell - F_1 \\
    \frac 1{\ell - b} + \frac 1c &= \frac 1{F_2} \implies c = \frac{F_2(\ell - b)}{\ell - b - F_2} = \frac{F_2(\ell - F_1)}{\ell - F_1 - F_2}.
    \\
    &\text{Если же пучок падает на рассеивающую линзу:} \\
    \frac 1{\infty} + \frac 1b &= \frac 1{F_2} \implies b = F_2 \implies \ell - b = \ell - F_2 \\
    \frac 1{\ell - b} + \frac 1c &= \frac 1{F_1} \implies c = \frac{F_1(\ell - b)}{\ell - b - F_1} = \frac{F_1(\ell - F_2)}{\ell - F_2 - F_1}.
    \end{align*}
}
\solutionspace{120pt}

\tasknumber{15}%
\task{%
    Две собирающих линзы с фокусными расстояниями $40\,\text{см}$ и $25\,\text{см}$ расположены так,
    что их оптические оси совмещены.
    На первую линзу падает пучок параллельных лучей.
    Пройдя через вторую линзу, он остался параллельным.
    Найдите расстояние между линзами и сделайте рисунок.
}
\answer{%
    \begin{align*}
    \frac 1\infty + \frac 1b &= \frac 1{F_1} \implies b = F_1, \\
    \frac 1{\ell - b} + \frac 1{\infty} &= \frac 1{F_2} \implies \ell - b = F_2 \implies \ell = b + F_2 = F_1 + F_2 = 65\,\text{см}.
    \end{align*}
}

\variantsplitter

\addpersonalvariant{Дмитрий Соколов}

\tasknumber{1}%
\task{%
    Найти оптическую силу собирающей линзы, если действительное изображение предмета,
    помещённого в $55\,\text{см}$ от линзы, получается на расстоянии $40\,\text{см}$ от неё.
}
\answer{%
    $D = \frac 1F = \frac 1a + \frac 1b = \frac 1{55\,\text{см}} + \frac 1{40\,\text{см}} \approx 4{,}32\,\text{дптр}$
}
\solutionspace{80pt}

\tasknumber{2}%
\task{%
    Найти увеличение изображения, если изображение предмета, находящегося
    на расстоянии $25\,\text{см}$ от линзы, получается на расстоянии $18\,\text{см}$ от неё.
}
\answer{%
    $\Gamma = \frac ba = \frac {18\,\text{см}}{25\,\text{см}} \approx 0{,}7$
}
\solutionspace{80pt}

\tasknumber{3}%
\task{%
    Расстояние от предмета до линзы $12\,\text{см}$, а от линзы до мнимого изображения $25\,\text{см}$.
    Чему равно фокусное расстояние линзы?
}
\answer{%
    $\pm \frac 1F = \frac 1a - \frac 1b \implies F = \frac{a b}{\abs{b - a}} \approx 23{,}1\,\text{см}$
}
\solutionspace{80pt}

\tasknumber{4}%
\task{%
    Две тонкие собирающие линзы с фокусными расстояниями $12\,\text{см}$ и $30\,\text{см}$ сложены вместе.
    Чему равно фокусное расстояние такой оптической системы?
}
\answer{%
    $\frac 1{f_1} = \frac 1a + \frac 1b; \frac 1{f_2} = - \frac 1b + \frac 1c \implies \frac 1{f_1} + \frac 1{f_2} = \frac 1a + \frac 1c \implies f' = \frac 1{\frac 1{f_1} + \frac 1{f_2}} = \frac{f_1 f_2}{f_1 + f_2} \approx 8{,}6\,\text{см}$
}
\solutionspace{80pt}

\tasknumber{5}%
\task{%
    Линейные размеры прямого изображения предмета, полученного в собирающей линзе,
    в четыре раза больше линейных размеров предмета.
    Зная, что предмет находится на $25\,\text{см}$ ближе к линзе,
    чем его изображение, найти оптическую силу линзы.
}
\answer{%
    \begin{align*}
    &\text{Если изображение действительное:} \\
    D &= \frac 1F = \frac 1a + \frac 1b, \qquad \Gamma = \frac ba, \qquad b - a = \ell \implies b = \Gamma a \implies \Gamma a - a = \ell \implies  \\
    a &= \frac {\ell}{\Gamma - 1} \implies b = \frac {{\ell} \Gamma}{\Gamma - 1} \implies  \\
    D &= \frac {\Gamma - 1}\ell + \frac {\Gamma - 1}{\ell \Gamma} = \frac 1\ell \cdot \cbr{\Gamma - 1 + \frac {\Gamma - 1}{\Gamma} } =\frac 1\ell \cdot \cbr{\Gamma - \frac 1\Gamma} \approx 15\,\text{дптр}.
    \\
    &\text{Если изображение мнимое:} \\
    D &= \frac 1F = \frac 1a - \frac 1b, \qquad \Gamma = \frac ba, \qquad b - a = \ell \implies b = \Gamma a \implies \Gamma a - a = \ell \implies  \\
    a &= \frac {\ell}{\Gamma - 1} \implies b = \frac {{\ell} \Gamma}{\Gamma - 1} \implies  \\
    D &= \frac {\Gamma - 1}\ell - \frac {\Gamma - 1}{\ell \Gamma} = \frac 1\ell \cdot \cbr{\Gamma - 1 - \frac {\Gamma - 1}{\Gamma} } =\frac 1\ell \cdot \cbr{\Gamma + \frac 1\Gamma - 2} \approx 9\,\text{дптр}.
    \\
    &\text{В ответе надо указать оба значения.}
    \end{align*}
}
\solutionspace{120pt}

\tasknumber{6}%
\task{%
    Оптическая сила объектива фотоаппарата равна $3\,\text{дптр}$.
    При фотографировании чертежа с расстояния $1{,}1\,\text{м}$ площадь изображения
    чертежа на фотопластинке оказалась равной $9\,\text{см}^{2}$.
    Какова площадь самого чертежа? Ответ выразите в квадратных сантиметрах.
}
\answer{%
    \begin{align*}
    &\frac 1a + \frac 1b = \frac 1F = D \implies b = \frac{aF}{a - F} \\
    &\frac {S'}S = \Gamma^2 = \sqr{\frac ba} = \sqr{\frac F{a - F}} \implies \\
    &\implies S = S' \cdot \sqr{\frac{a - F}F} = S' \cdot \sqr{\frac aF - 1} = S' \cdot \sqr{aD - 1} \approx 50\,\text{см}^{2}.
    \end{align*}
}


\variantsplitter


\addpersonalvariant{Дмитрий Соколов}

\tasknumber{7}%
\task{%
    В каком месте на главной оптической оси двояковыпуклой линзы
    нужно поместить точечный источник света,
    чтобы его изображение оказалось в главном фокусе линзы?
}
\answer{%
    $\text{для мнимого - на половине фокусного, для действительного - на бесконечности}$
}
\solutionspace{120pt}

\tasknumber{8}%
\task{%
    Предмет в виде отрезка длиной $\ell$ расположен вдоль оптической оси
    собирающей линзы с фокусным расстоянием $F$.
    Середина отрезка расположена
    на расстоянии $a$ от линзы, которая даёт действительное изображение
    всех точек предмета.
    Определить продольное увеличение предмета.
}
\answer{%
    \begin{align*}
    \frac 1{a + \frac \ell 2} &+ \frac 1b = \frac 1F \implies b = \frac{F\cbr{a + \frac \ell 2}}{a + \frac \ell 2 - F} \\
    \frac 1{a - \frac \ell 2} &+ \frac 1c = \frac 1F \implies c = \frac{F\cbr{a - \frac \ell 2}}{a - \frac \ell 2 - F} \\
    \abs{b - c} &= \abs{\frac{F\cbr{a + \frac \ell 2}}{a + \frac \ell 2 - F} - \frac{F\cbr{a - \frac \ell 2}}{a - \frac \ell 2 - F}}= F\abs{\frac{\cbr{a + \frac \ell 2}\cbr{a - \frac \ell 2 - F} - \cbr{a - \frac \ell 2}\cbr{a + \frac \ell 2 - F}}{ \cbr{a + \frac \ell 2 - F} \cbr{a - \frac \ell 2 - F} }} =  \\
    &= F\abs{\frac{a^2 - \frac {a\ell} 2 - Fa + \frac {a\ell} 2 - \frac {\ell^2} 4 - \frac {F\ell}2 - a^2 - \frac {a\ell}2 + aF + \frac {a\ell}2 + \frac {\ell^2} 4 - \frac {F\ell} 2}{\cbr{a + \frac \ell 2 - F} \cbr{a - \frac \ell 2 - F} }} = \\
    &= F\frac{F\ell}{\sqr{a-F} - \frac {\ell^2}4} = \frac{F^2\ell}{\sqr{a-F} - \frac {\ell^2}4}\implies \Gamma = \frac{\abs{b - c}}\ell = \frac{F^2}{\sqr{a-F} - \frac {\ell^2}4}.
    \end{align*}
}
\solutionspace{120pt}

\tasknumber{9}%
\task{%
    На экране с помощью тонкой линзы получено изображение предмета
    с увеличением $2$.
    Предмет передвинули на $2\,\text{см}$.
    Для того, чтобы получить резкое изображение, пришлось передвинуть экран.
    При этом увеличение оказалось равным $6$.
    На какое расстояние
    пришлось передвинуть экран?
}
\answer{%
    \begin{align*}
    &\frac 1a + \frac 1b = \frac 1F, \Gamma_1 = \frac ba = \frac{F}{a-F} \implies \Gamma_1(a-F) = F \implies a = F \cdot \frac{1 + \Gamma_1}{\Gamma_1} \\
    &\frac 1{a + x} + \frac 1{b + y} = \frac 1F, \Gamma_2 = \frac {b+y}{a+x} = \frac{F}{a+x-F} \implies a + x = F \cdot \frac{1 + \Gamma_2}{\Gamma_2} \\
    &1 + \frac xa = \frac{ \frac{1 + \Gamma_2}{\Gamma_2} }{ \frac{1 + \Gamma_1}{\Gamma_1} } = \frac{\Gamma_1(1 + \Gamma_2)}{\Gamma_2(1 + \Gamma_1)} \\
    &a = \frac x{ \frac{\Gamma_1(1 + \Gamma_2)}{\Gamma_2(1 + \Gamma_1)} - 1} = x \cdot \frac{\Gamma_2(1 + \Gamma_1)}{\Gamma_1 - \Gamma_2} \\
    &y = (a + x)\Gamma_2 - b = (a + x)\Gamma_2 - a\Gamma_1 = a(\Gamma_2 - \Gamma_1) + x\Gamma_2 = -x\Gamma_2(1 + \Gamma_1) + x\Gamma_2 = -x\Gamma_2\Gamma_1 = 24\,\text{см}, \\
    &\text{знаки разные, т.е.
    экран надо было подвинуть в другую сторону чем предмет: $x < 0, y > 0$.}
    \end{align*}
}
\solutionspace{120pt}

\tasknumber{10}%
\task{%
    Тонкая собирающая линза дает изображение предмета на экране при двух положениях линзы между предметом и экраном.
    Высота изображения при первом положении $25\,\text{см}$, во втором — $9\,\text{см}$.
    Расстояние между предметом и экранов постоянно.
    Чему равна высота предмета?
}
\answer{%
    \begin{align*}
    &\frac 1a + \frac 1b = \frac 1F, \frac 1c + \frac 1d = \frac 1F, a + b = c + d \implies \frac{a + b}{ab} = \frac 1F = \frac{c+d}{cd} \implies ab = cd, \\
    &\implies ab = c(a + b - c) \implies c^2 - ac - bc + ab = 0 \implies c = a \text{ или } c = b \implies c = b \implies d = a.
    \\
    &\Gamma_1 = \frac {H_1}H = \frac ba, \Gamma_2 = \frac {H_2}H = \frac dc = \frac ab \implies \frac {H_1}H \cdot \frac {H_2}H = \frac ba \cdot \frac ab = 1, \\
    &H = \sqrt{H_1 H_2} \approx 15\,\text{см}.
    \end{align*}
}
\solutionspace{120pt}

\tasknumber{11}%
\task{%
    Какие предметы можно рассмотреть на фотографии, сделанной со спутника,
    если разрешающая способность плёнки $0{,}010\,\text{мм}$? Каким должно быть
    время экспозиции $\tau$ чтобы полностью использовать возможности плёнки?
    Фокусное расстояние объектива используемого фотоаппарата $20\,\text{см}$,
    высота орбиты спутника $80\,\text{км}$.
}
\answer{%
    \begin{align*}
    &H \ll R \implies v = v_{\text{I}} = \sqrt{G R} \approx 7{,}9\,\frac{\text{км}}{\text{с}}.
    \\
    &F \ll H \implies b = F, a = H, \\
    &\Gamma = \frac \delta\ell = \frac ba \implies \ell = \frac{\delta a}b = \frac{\delta H}F \approx \frac{0{,}010\,\text{мм} \cdot 80\,\text{км}}{20\,\text{см}} \approx 4\,\text{м}, \\
    &\implies \tau = \frac \ell v = \frac{\delta H}{F v} = \frac{0{,}010\,\text{мм} \cdot 80\,\text{км}}{20\,\text{см} \cdot 7{,}9\,\frac{\text{км}}{\text{с}}} \approx 0{,}5\,\text{мс}.
    \end{align*}
}


\variantsplitter


\addpersonalvariant{Дмитрий Соколов}

\tasknumber{12}%
\task{%
    При аэрофотосъемках используется фотоаппарат, объектив которого
    имеет фокусиое расстояние $20\,\text{см}$.
    Разрешающая способность плёнки $0{,}010\,\text{мм}$.
    На какой высоте должен лететь самолет, чтобы на фотографии можно
    было различить следы размером $30\,\text{см}$?
    При какой скорости самолета изображение не будет размытым,
    если время экспозиции $1\,\text{мс}$?
}
\answer{%
    \begin{align*}
    &F \ll H \implies b = F, a = H, \\
    &\Gamma = \frac \delta\ell = \frac ba = \frac FH \implies H = \frac{\ell F}\delta = \frac{30\,\text{см} \cdot 20\,\text{см}}{0{,}010\,\text{мм}} \approx 6\,\text{км}, \\
    &\implies v = \frac l\tau = \frac{30\,\text{см}}{1\,\text{мс}} \approx 1080\,\frac{\text{км}}{\text{ч}}.
    \end{align*}
}
\solutionspace{120pt}

\tasknumber{13}%
\task{%
    Две одинаковые собиращие линзы установлены так, что их главные оптические оси совпадают,
    а главный фокус первой находится там же, где главный фокус второй.
    Расстояние от первой линзы до предмета равно $15\,\text{см}$.
    Чему равно расстояние от изображения объекта во второй линзе до второй линзы?
    Определите также увеличение.
    Фокусное расстояние каждой линзы $30\,\text{см}$.
}
\answer{%
    \begin{align*}
    \frac 1a + \frac 1b &= \frac 1F \implies b = \frac{aF}{a - F} \implies 2F - b = \frac{2aF - 2F^2 - aF}{a - F} = \frac{F(a - 2F)}{a - F}.
    \\
    \frac 1{2F - b} + \frac 1c &= \frac 1F \implies c = \frac{F(2F-b)}{(2F - b) - F} = \frac{F \cdot \frac{F(a - 2F)}{a - F}}{\frac{F(a - 2F)}{a - F} - F}  = F \cdot \frac{ \frac{F(a - 2F)}{a - F} }{ \frac{F(a - 2F)}{a - F} - 1} = \\
     &= F \cdot \frac{a - 2F}{a - 2F - a + F} = 2F - a = 45\,\text{см}.
     \\
    \ell &= a + 2F + c = 4F = 120\,\text{см}.
    \\
    &\Gamma = \Gamma_1 \cdot \Gamma_2 = \frac ba \cdot \frac c{2F-b} = \frac F{a - F} \cdot \frac{2F - a}{\frac{F(a - 2F)}{a - F}} = -1.
    \end{align*}
}
\solutionspace{120pt}

\tasknumber{14}%
\task{%
    Собирающая линза с фокусным расстоянием $F_1 > 0$ и рассеивающая линза с фокусным расстоянием $F_2 < 0$
    установлены коаксиально на расстоянии $\ell$.
    Пучок параллельных лучей падает на собирающую линзу.
    Сделайте схематичное построение и определите, в какой точке система из этих линз соберёт пучок.
}
\answer{%
    \begin{align*}
    &\text{Если пучок падает на собирающую линзу:} \\
    \frac 1{\infty} + \frac 1b &= \frac 1{F_1} \implies b = F_1 \implies \ell - b = \ell - F_1 \\
    \frac 1{\ell - b} + \frac 1c &= \frac 1{F_2} \implies c = \frac{F_2(\ell - b)}{\ell - b - F_2} = \frac{F_2(\ell - F_1)}{\ell - F_1 - F_2}.
    \\
    &\text{Если же пучок падает на рассеивающую линзу:} \\
    \frac 1{\infty} + \frac 1b &= \frac 1{F_2} \implies b = F_2 \implies \ell - b = \ell - F_2 \\
    \frac 1{\ell - b} + \frac 1c &= \frac 1{F_1} \implies c = \frac{F_1(\ell - b)}{\ell - b - F_1} = \frac{F_1(\ell - F_2)}{\ell - F_2 - F_1}.
    \end{align*}
}
\solutionspace{120pt}

\tasknumber{15}%
\task{%
    Две собирающих линзы с фокусными расстояниями $20\,\text{см}$ и $35\,\text{см}$ расположены так,
    что их оптические оси совмещены.
    На первую линзу падает пучок параллельных лучей.
    Пройдя через вторую линзу, он остался параллельным.
    Найдите расстояние между линзами и сделайте рисунок.
}
\answer{%
    \begin{align*}
    \frac 1\infty + \frac 1b &= \frac 1{F_1} \implies b = F_1, \\
    \frac 1{\ell - b} + \frac 1{\infty} &= \frac 1{F_2} \implies \ell - b = F_2 \implies \ell = b + F_2 = F_1 + F_2 = 55\,\text{см}.
    \end{align*}
}

\variantsplitter

\addpersonalvariant{Арсений Трофимов}

\tasknumber{1}%
\task{%
    Найти оптическую силу собирающей линзы, если действительное изображение предмета,
    помещённого в $35\,\text{см}$ от линзы, получается на расстоянии $20\,\text{см}$ от неё.
}
\answer{%
    $D = \frac 1F = \frac 1a + \frac 1b = \frac 1{35\,\text{см}} + \frac 1{20\,\text{см}} \approx 7{,}86\,\text{дптр}$
}
\solutionspace{80pt}

\tasknumber{2}%
\task{%
    Найти увеличение изображения, если изображение предмета, находящегося
    на расстоянии $20\,\text{см}$ от линзы, получается на расстоянии $18\,\text{см}$ от неё.
}
\answer{%
    $\Gamma = \frac ba = \frac {18\,\text{см}}{20\,\text{см}} \approx 0{,}9$
}
\solutionspace{80pt}

\tasknumber{3}%
\task{%
    Расстояние от предмета до линзы $10\,\text{см}$, а от линзы до мнимого изображения $25\,\text{см}$.
    Чему равно фокусное расстояние линзы?
}
\answer{%
    $\pm \frac 1F = \frac 1a - \frac 1b \implies F = \frac{a b}{\abs{b - a}} \approx 16{,}7\,\text{см}$
}
\solutionspace{80pt}

\tasknumber{4}%
\task{%
    Две тонкие собирающие линзы с фокусными расстояниями $25\,\text{см}$ и $20\,\text{см}$ сложены вместе.
    Чему равно фокусное расстояние такой оптической системы?
}
\answer{%
    $\frac 1{f_1} = \frac 1a + \frac 1b; \frac 1{f_2} = - \frac 1b + \frac 1c \implies \frac 1{f_1} + \frac 1{f_2} = \frac 1a + \frac 1c \implies f' = \frac 1{\frac 1{f_1} + \frac 1{f_2}} = \frac{f_1 f_2}{f_1 + f_2} \approx 11{,}1\,\text{см}$
}
\solutionspace{80pt}

\tasknumber{5}%
\task{%
    Линейные размеры прямого изображения предмета, полученного в собирающей линзе,
    в два раза больше линейных размеров предмета.
    Зная, что предмет находится на $30\,\text{см}$ ближе к линзе,
    чем его изображение, найти оптическую силу линзы.
}
\answer{%
    \begin{align*}
    &\text{Если изображение действительное:} \\
    D &= \frac 1F = \frac 1a + \frac 1b, \qquad \Gamma = \frac ba, \qquad b - a = \ell \implies b = \Gamma a \implies \Gamma a - a = \ell \implies  \\
    a &= \frac {\ell}{\Gamma - 1} \implies b = \frac {{\ell} \Gamma}{\Gamma - 1} \implies  \\
    D &= \frac {\Gamma - 1}\ell + \frac {\Gamma - 1}{\ell \Gamma} = \frac 1\ell \cdot \cbr{\Gamma - 1 + \frac {\Gamma - 1}{\Gamma} } =\frac 1\ell \cdot \cbr{\Gamma - \frac 1\Gamma} \approx 5\,\text{дптр}.
    \\
    &\text{Если изображение мнимое:} \\
    D &= \frac 1F = \frac 1a - \frac 1b, \qquad \Gamma = \frac ba, \qquad b - a = \ell \implies b = \Gamma a \implies \Gamma a - a = \ell \implies  \\
    a &= \frac {\ell}{\Gamma - 1} \implies b = \frac {{\ell} \Gamma}{\Gamma - 1} \implies  \\
    D &= \frac {\Gamma - 1}\ell - \frac {\Gamma - 1}{\ell \Gamma} = \frac 1\ell \cdot \cbr{\Gamma - 1 - \frac {\Gamma - 1}{\Gamma} } =\frac 1\ell \cdot \cbr{\Gamma + \frac 1\Gamma - 2} \approx 1{,}7\,\text{дптр}.
    \\
    &\text{В ответе надо указать оба значения.}
    \end{align*}
}
\solutionspace{120pt}

\tasknumber{6}%
\task{%
    Оптическая сила объектива фотоаппарата равна $5\,\text{дптр}$.
    При фотографировании чертежа с расстояния $1{,}1\,\text{м}$ площадь изображения
    чертежа на фотопластинке оказалась равной $9\,\text{см}^{2}$.
    Какова площадь самого чертежа? Ответ выразите в квадратных сантиметрах.
}
\answer{%
    \begin{align*}
    &\frac 1a + \frac 1b = \frac 1F = D \implies b = \frac{aF}{a - F} \\
    &\frac {S'}S = \Gamma^2 = \sqr{\frac ba} = \sqr{\frac F{a - F}} \implies \\
    &\implies S = S' \cdot \sqr{\frac{a - F}F} = S' \cdot \sqr{\frac aF - 1} = S' \cdot \sqr{aD - 1} \approx 180\,\text{см}^{2}.
    \end{align*}
}


\variantsplitter


\addpersonalvariant{Арсений Трофимов}

\tasknumber{7}%
\task{%
    В каком месте на главной оптической оси двояковыгнутой линзы
    нужно поместить точечный источник света,
    чтобы его изображение оказалось в главном фокусе линзы?
}
\answer{%
    $\text{на половине фокусного расстояния}$
}
\solutionspace{120pt}

\tasknumber{8}%
\task{%
    Предмет в виде отрезка длиной $\ell$ расположен вдоль оптической оси
    собирающей линзы с фокусным расстоянием $F$.
    Середина отрезка расположена
    на расстоянии $a$ от линзы, которая даёт действительное изображение
    всех точек предмета.
    Определить продольное увеличение предмета.
}
\answer{%
    \begin{align*}
    \frac 1{a + \frac \ell 2} &+ \frac 1b = \frac 1F \implies b = \frac{F\cbr{a + \frac \ell 2}}{a + \frac \ell 2 - F} \\
    \frac 1{a - \frac \ell 2} &+ \frac 1c = \frac 1F \implies c = \frac{F\cbr{a - \frac \ell 2}}{a - \frac \ell 2 - F} \\
    \abs{b - c} &= \abs{\frac{F\cbr{a + \frac \ell 2}}{a + \frac \ell 2 - F} - \frac{F\cbr{a - \frac \ell 2}}{a - \frac \ell 2 - F}}= F\abs{\frac{\cbr{a + \frac \ell 2}\cbr{a - \frac \ell 2 - F} - \cbr{a - \frac \ell 2}\cbr{a + \frac \ell 2 - F}}{ \cbr{a + \frac \ell 2 - F} \cbr{a - \frac \ell 2 - F} }} =  \\
    &= F\abs{\frac{a^2 - \frac {a\ell} 2 - Fa + \frac {a\ell} 2 - \frac {\ell^2} 4 - \frac {F\ell}2 - a^2 - \frac {a\ell}2 + aF + \frac {a\ell}2 + \frac {\ell^2} 4 - \frac {F\ell} 2}{\cbr{a + \frac \ell 2 - F} \cbr{a - \frac \ell 2 - F} }} = \\
    &= F\frac{F\ell}{\sqr{a-F} - \frac {\ell^2}4} = \frac{F^2\ell}{\sqr{a-F} - \frac {\ell^2}4}\implies \Gamma = \frac{\abs{b - c}}\ell = \frac{F^2}{\sqr{a-F} - \frac {\ell^2}4}.
    \end{align*}
}
\solutionspace{120pt}

\tasknumber{9}%
\task{%
    На экране с помощью тонкой линзы получено изображение предмета
    с увеличением $2$.
    Предмет передвинули на $10\,\text{см}$.
    Для того, чтобы получить резкое изображение, пришлось передвинуть экран.
    При этом увеличение оказалось равным $8$.
    На какое расстояние
    пришлось передвинуть экран?
}
\answer{%
    \begin{align*}
    &\frac 1a + \frac 1b = \frac 1F, \Gamma_1 = \frac ba = \frac{F}{a-F} \implies \Gamma_1(a-F) = F \implies a = F \cdot \frac{1 + \Gamma_1}{\Gamma_1} \\
    &\frac 1{a + x} + \frac 1{b + y} = \frac 1F, \Gamma_2 = \frac {b+y}{a+x} = \frac{F}{a+x-F} \implies a + x = F \cdot \frac{1 + \Gamma_2}{\Gamma_2} \\
    &1 + \frac xa = \frac{ \frac{1 + \Gamma_2}{\Gamma_2} }{ \frac{1 + \Gamma_1}{\Gamma_1} } = \frac{\Gamma_1(1 + \Gamma_2)}{\Gamma_2(1 + \Gamma_1)} \\
    &a = \frac x{ \frac{\Gamma_1(1 + \Gamma_2)}{\Gamma_2(1 + \Gamma_1)} - 1} = x \cdot \frac{\Gamma_2(1 + \Gamma_1)}{\Gamma_1 - \Gamma_2} \\
    &y = (a + x)\Gamma_2 - b = (a + x)\Gamma_2 - a\Gamma_1 = a(\Gamma_2 - \Gamma_1) + x\Gamma_2 = -x\Gamma_2(1 + \Gamma_1) + x\Gamma_2 = -x\Gamma_2\Gamma_1 = 160\,\text{см}, \\
    &\text{знаки разные, т.е.
    экран надо было подвинуть в другую сторону чем предмет: $x < 0, y > 0$.}
    \end{align*}
}
\solutionspace{120pt}

\tasknumber{10}%
\task{%
    Тонкая собирающая линза дает изображение предмета на экране при двух положениях линзы между предметом и экраном.
    Высота изображения при первом положении $20\,\text{см}$, во втором — $9\,\text{см}$.
    Расстояние между предметом и экранов постоянно.
    Чему равна высота предмета?
}
\answer{%
    \begin{align*}
    &\frac 1a + \frac 1b = \frac 1F, \frac 1c + \frac 1d = \frac 1F, a + b = c + d \implies \frac{a + b}{ab} = \frac 1F = \frac{c+d}{cd} \implies ab = cd, \\
    &\implies ab = c(a + b - c) \implies c^2 - ac - bc + ab = 0 \implies c = a \text{ или } c = b \implies c = b \implies d = a.
    \\
    &\Gamma_1 = \frac {H_1}H = \frac ba, \Gamma_2 = \frac {H_2}H = \frac dc = \frac ab \implies \frac {H_1}H \cdot \frac {H_2}H = \frac ba \cdot \frac ab = 1, \\
    &H = \sqrt{H_1 H_2} \approx 13{,}4\,\text{см}.
    \end{align*}
}
\solutionspace{120pt}

\tasknumber{11}%
\task{%
    Какие предметы можно рассмотреть на фотографии, сделанной со спутника,
    если разрешающая способность плёнки $0{,}02\,\text{мм}$? Каким должно быть
    время экспозиции $\tau$ чтобы полностью использовать возможности плёнки?
    Фокусное расстояние объектива используемого фотоаппарата $15\,\text{см}$,
    высота орбиты спутника $120\,\text{км}$.
}
\answer{%
    \begin{align*}
    &H \ll R \implies v = v_{\text{I}} = \sqrt{G R} \approx 7{,}9\,\frac{\text{км}}{\text{с}}.
    \\
    &F \ll H \implies b = F, a = H, \\
    &\Gamma = \frac \delta\ell = \frac ba \implies \ell = \frac{\delta a}b = \frac{\delta H}F \approx \frac{0{,}02\,\text{мм} \cdot 120\,\text{км}}{15\,\text{см}} \approx 16\,\text{м}, \\
    &\implies \tau = \frac \ell v = \frac{\delta H}{F v} = \frac{0{,}02\,\text{мм} \cdot 120\,\text{км}}{15\,\text{см} \cdot 7{,}9\,\frac{\text{км}}{\text{с}}} \approx 2\,\text{мс}.
    \end{align*}
}


\variantsplitter


\addpersonalvariant{Арсений Трофимов}

\tasknumber{12}%
\task{%
    При аэрофотосъемках используется фотоаппарат, объектив которого
    имеет фокусиое расстояние $20\,\text{см}$.
    Разрешающая способность плёнки $0{,}02\,\text{мм}$.
    На какой высоте должен лететь самолет, чтобы на фотографии можно
    было различить следы размером $15\,\text{см}$?
    При какой скорости самолета изображение не будет размытым,
    если время экспозиции $2\,\text{мс}$?
}
\answer{%
    \begin{align*}
    &F \ll H \implies b = F, a = H, \\
    &\Gamma = \frac \delta\ell = \frac ba = \frac FH \implies H = \frac{\ell F}\delta = \frac{15\,\text{см} \cdot 20\,\text{см}}{0{,}02\,\text{мм}} \approx 1{,}5\,\text{км}, \\
    &\implies v = \frac l\tau = \frac{15\,\text{см}}{2\,\text{мс}} \approx 270\,\frac{\text{км}}{\text{ч}}.
    \end{align*}
}
\solutionspace{120pt}

\tasknumber{13}%
\task{%
    Две одинаковые собиращие линзы установлены так, что их главные оптические оси совпадают,
    а главный фокус первой находится там же, где главный фокус второй.
    Расстояние от первой линзы до предмета равно $10\,\text{см}$.
    Чему равно расстояние от изображения объекта во второй линзе до самого объекта?
    Определите также увеличение.
    Фокусное расстояние каждой линзы $30\,\text{см}$.
}
\answer{%
    \begin{align*}
    \frac 1a + \frac 1b &= \frac 1F \implies b = \frac{aF}{a - F} \implies 2F - b = \frac{2aF - 2F^2 - aF}{a - F} = \frac{F(a - 2F)}{a - F}.
    \\
    \frac 1{2F - b} + \frac 1c &= \frac 1F \implies c = \frac{F(2F-b)}{(2F - b) - F} = \frac{F \cdot \frac{F(a - 2F)}{a - F}}{\frac{F(a - 2F)}{a - F} - F}  = F \cdot \frac{ \frac{F(a - 2F)}{a - F} }{ \frac{F(a - 2F)}{a - F} - 1} = \\
     &= F \cdot \frac{a - 2F}{a - 2F - a + F} = 2F - a = 50\,\text{см}.
     \\
    \ell &= a + 2F + c = 4F = 120\,\text{см}.
    \\
    &\Gamma = \Gamma_1 \cdot \Gamma_2 = \frac ba \cdot \frac c{2F-b} = \frac F{a - F} \cdot \frac{2F - a}{\frac{F(a - 2F)}{a - F}} = -1.
    \end{align*}
}
\solutionspace{120pt}

\tasknumber{14}%
\task{%
    Собирающая линза с фокусным расстоянием $F_1 > 0$ и рассеивающая линза с фокусным расстоянием $F_2 < 0$
    установлены коаксиально на расстоянии $\ell$.
    Пучок параллельных лучей падает на рассеивающую линзу.
    Сделайте схематичное построение и определите, в какой точке система из этих линз соберёт пучок.
}
\answer{%
    \begin{align*}
    &\text{Если пучок падает на собирающую линзу:} \\
    \frac 1{\infty} + \frac 1b &= \frac 1{F_1} \implies b = F_1 \implies \ell - b = \ell - F_1 \\
    \frac 1{\ell - b} + \frac 1c &= \frac 1{F_2} \implies c = \frac{F_2(\ell - b)}{\ell - b - F_2} = \frac{F_2(\ell - F_1)}{\ell - F_1 - F_2}.
    \\
    &\text{Если же пучок падает на рассеивающую линзу:} \\
    \frac 1{\infty} + \frac 1b &= \frac 1{F_2} \implies b = F_2 \implies \ell - b = \ell - F_2 \\
    \frac 1{\ell - b} + \frac 1c &= \frac 1{F_1} \implies c = \frac{F_1(\ell - b)}{\ell - b - F_1} = \frac{F_1(\ell - F_2)}{\ell - F_2 - F_1}.
    \end{align*}
}
\solutionspace{120pt}

\tasknumber{15}%
\task{%
    Две собирающих линзы с фокусными расстояниями $40\,\text{см}$ и $45\,\text{см}$ расположены так,
    что их оптические оси совмещены.
    На первую линзу падает пучок параллельных лучей.
    Пройдя через вторую линзу, он остался параллельным.
    Найдите расстояние между линзами и сделайте рисунок.
}
\answer{%
    \begin{align*}
    \frac 1\infty + \frac 1b &= \frac 1{F_1} \implies b = F_1, \\
    \frac 1{\ell - b} + \frac 1{\infty} &= \frac 1{F_2} \implies \ell - b = F_2 \implies \ell = b + F_2 = F_1 + F_2 = 85\,\text{см}.
    \end{align*}
}
% autogenerated
