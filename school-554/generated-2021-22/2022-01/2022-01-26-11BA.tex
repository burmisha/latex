\setdate{26~января~2022}
\setclass{11«БА»}

\addpersonalvariant{Михаил Бурмистров}

\tasknumber{1}%
\task{%
    Найти оптическую силу собирающей линзы, если действительное изображение предмета,
    помещённого в $15\,\text{см}$ от линзы, получается на расстоянии $30\,\text{см}$ от неё.
}
\answer{%
    $D = \frac 1F = \frac 1a + \frac 1b = \frac 1{15\,\text{см}} + \frac 1{30\,\text{см}} \approx 10\,\text{дптр}$
}
\solutionspace{180pt}

\tasknumber{2}%
\task{%
    Найти увеличение изображения, если изображение предмета, находящегося
    на расстоянии $20\,\text{см}$ от линзы, получается на расстоянии $30\,\text{см}$ от неё.
}
\answer{%
    $\Gamma = \frac ba = \frac {30\,\text{см}}{20\,\text{см}} \approx 1{,}50$
}
\solutionspace{180pt}

\tasknumber{3}%
\task{%
    Расстояние от предмета до линзы $10\,\text{см}$, а от линзы до мнимого изображения $30\,\text{см}$.
    Чему равно фокусное расстояние линзы?
}
\answer{%
    $\pm \frac 1F = \frac 1a - \frac 1b \implies F = \frac{a b}{\abs{b - a}} \approx 15\,\text{см}$
}
\solutionspace{180pt}

\tasknumber{4}%
\task{%
    Две тонкие собирающие линзы с фокусными расстояниями $25\,\text{см}$ и $30\,\text{см}$ сложены вместе.
    Чему равно фокусное расстояние такой оптической системы?
}
\answer{%
    $\frac 1{f_1} = \frac 1a + \frac 1b; \frac 1{f_2} = - \frac 1b + \frac 1c \implies \frac 1{f_1} + \frac 1{f_2} = \frac 1a + \frac 1c \implies f' = \frac 1{\frac 1{f_1} + \frac 1{f_2}} = \frac{f_1 f_2}{f_1 + f_2} \approx 13{,}6\,\text{см}$
}
\solutionspace{180pt}

\tasknumber{5}%
\task{%
    Линейные размеры прямого изображения предмета, полученного в собирающей линзе,
    в два раза больше линейных размеров предмета.
    Зная, что предмет находится на $40\,\text{см}$ ближе к линзе,
    чем его изображение, найти оптическую силу линзы.
}
\answer{%
    \begin{align*}
    D &= \frac 1F = \frac 1a + \frac 1b, \qquad \Gamma = \frac ba, \qquad b - a = \ell \implies b = \Gamma a \implies \Gamma a - a = \ell \implies  \\
    a &= \frac {\ell}{\Gamma - 1} \implies b = \frac {{\ell} \Gamma}{\Gamma - 1} \implies  \\
    D &= \frac {\Gamma - 1}\ell + \frac {\Gamma - 1}{\ell \Gamma} = \frac 1\ell \cdot \cbr{\Gamma - 1 + \frac {\Gamma - 1}{\Gamma} } =\frac 1\ell \cdot \cbr{\Gamma - \frac 1\Gamma} \approx 3{,}8\,\text{дптр}.
    \end{align*}
}
\solutionspace{180pt}

\tasknumber{6}%
\task{%
    Оптическая сила объектива фотоаппарата равна $4\,\text{дптр}$.
    При фотографировании чертежа с расстояния $1{,}1\,\text{м}$ площадь изображения
    чертежа на фотопластинке оказалась равной $9\,\text{см}^{2}$.
    Какова площадь самого чертежа? Ответ выразите в квадратных сантиметрах.
}
\solutionspace{180pt}

\tasknumber{7}%
\task{%
    В каком месте на главной оптической оси двояковыгнутой линзы
    нужно поместить точечный источник света,
    чтобы его изображение оказалось в главном фокусе линзы?
}
\answer{%
    $\text{на половине фокусного расстояния}$
}
\solutionspace{120pt}

\tasknumber{8}%
\task{%
    Предмет высотой $h = 30\,\text{см}$ находится на расстоянии $d = 1\,\text{м}$
    от вертикально расположенной рассеивающей линзы с фокусным расстоянием $F = 25\,\text{см}$.
    Где находится изображение предмета? Определите тип изображения и его высоту.
}
\solutionspace{120pt}

\tasknumber{9}%
\task{%
    На каком расстоянии от двояковыпуклой линзы с оптической силой $D = 2{,}5\,\text{дптр}$
    надо поместить предмет, чтобы его изображение получилось на расстоянии $2\,\text{м}$ от линзы?
}
\solutionspace{120pt}

\tasknumber{10}%
\task{%
    Предмет в виде отрезка длиной $\ell$ расположен вдоль оптической оси
    собирающей линзы с фокусным расстоянием $F$.
    Середина отрезка расположена
    на расстоянии $a$ от линзы, которая даёт действительное изображение
    всех точек предмета.
    Определить продольное увеличение предмета.
}
\answer{%
    \begin{align*}
    \frac 1{a + \frac \ell 2} &+ \frac 1b = \frac 1F \implies b = \frac{F\cbr{a + \frac \ell 2}}{a + \frac \ell 2 - F} \\
    \frac 1{a - \frac \ell 2} &+ \frac 1c = \frac 1F \implies c = \frac{F\cbr{a - \frac \ell 2}}{a - \frac \ell 2 - F} \\
    \abs{b - c} &= \abs{\frac{F\cbr{a + \frac \ell 2}}{a + \frac \ell 2 - F} - \frac{F\cbr{a - \frac \ell 2}}{a - \frac \ell 2 - F}}= F\abs{\frac{\cbr{a + \frac \ell 2}\cbr{a - \frac \ell 2 - F} - \cbr{a - \frac \ell 2}\cbr{a + \frac \ell 2 - F}}{ \cbr{a + \frac \ell 2 - F} \cbr{a - \frac \ell 2 - F} }} =  \\
    &= F\abs{\frac{a^2 - \frac {a\ell} 2 - Fa + \frac {a\ell} 2 - \frac {\ell^2} 4 - \frac {F\ell}2 - a^2 - \frac {a\ell}2 + aF + \frac {a\ell}2 + \frac {\ell^2} 4 - \frac {F\ell} 2}{\cbr{a + \frac \ell 2 - F} \cbr{a - \frac \ell 2 - F} }} = \\
    &= F\frac{F\ell}{\sqr{a-F} - \frac {\ell^2}4} = \frac{F^2\ell}{\sqr{a-F} - \frac {\ell^2}4}\implies \Gamma = \frac{\abs{b - c}}\ell = \frac{F^2}{\sqr{a-F} - \frac {\ell^2}4}.
    \end{align*}
}
\solutionspace{120pt}

\tasknumber{11}%
\task{%
    На экране с помощью тонкой линзы получено изображение предмета
    с увеличением $4$.
    Предмет передвинули на $6\,\text{см}$.
    Для того, чтобы получить резкое изображение, пришлось передвинуть экран.
    При этом увеличение оказалось равным $6$.
    На какое расстояние
    пришлось передвинуть экран?
}
\solutionspace{120pt}

\tasknumber{12}%
\task{%
    Тонкая собирающая линза дает изображение предмета на экране высотой $H_1$,
    и $H_2$, при двух положениях линзы между предметом и экраном.
    Расстояние между ними неизменно.
    Чему равна высота предмета $h$?
}
\answer{%
    $h = \sqrt{H_1 H_2}$
}
\solutionspace{120pt}

\tasknumber{13}%
\task{%
    Какие предметы можно рассмотреть на фотографии, сделанной со спутника,
    если разрешающая способность пленки $0{,}010\,\text{мм}$? Каким должно быть
    время экспозиции $\tau$ чтобы полностью использовать возможности пленки?
    Фокусное расстояние объектива используемого фотоаппарата $10\,\text{cм}$,
    высота орбиты спутника $150\,\text{км}$.
}
\solutionspace{120pt}

\tasknumber{14}%
\task{%
    При аэрофотосъемках используется фотоаппарат, объектив которого
    имеет фокусиое расстояние $12\,\text{cм}$.
    Разрешающая способность пленки $0{,}010\,\text{мм}$.
    На какой высоте должен лететь самолет, чтобы на фотографии можно
    было различить листья деревьев размером $4\,\text{cм}$?
    При какой скорости самолета изображение не будет размытым,
    если время зкспозиции $2\,\text{мс}$?
}

\variantsplitter

\addpersonalvariant{Ирина Ан}

\tasknumber{1}%
\task{%
    Найти оптическую силу собирающей линзы, если действительное изображение предмета,
    помещённого в $15\,\text{см}$ от линзы, получается на расстоянии $30\,\text{см}$ от неё.
}
\answer{%
    $D = \frac 1F = \frac 1a + \frac 1b = \frac 1{15\,\text{см}} + \frac 1{30\,\text{см}} \approx 10\,\text{дптр}$
}
\solutionspace{180pt}

\tasknumber{2}%
\task{%
    Найти увеличение изображения, если изображение предмета, находящегося
    на расстоянии $20\,\text{см}$ от линзы, получается на расстоянии $18\,\text{см}$ от неё.
}
\answer{%
    $\Gamma = \frac ba = \frac {18\,\text{см}}{20\,\text{см}} \approx 0{,}9$
}
\solutionspace{180pt}

\tasknumber{3}%
\task{%
    Расстояние от предмета до линзы $10\,\text{см}$, а от линзы до мнимого изображения $25\,\text{см}$.
    Чему равно фокусное расстояние линзы?
}
\answer{%
    $\pm \frac 1F = \frac 1a - \frac 1b \implies F = \frac{a b}{\abs{b - a}} \approx 16{,}7\,\text{см}$
}
\solutionspace{180pt}

\tasknumber{4}%
\task{%
    Две тонкие собирающие линзы с фокусными расстояниями $25\,\text{см}$ и $30\,\text{см}$ сложены вместе.
    Чему равно фокусное расстояние такой оптической системы?
}
\answer{%
    $\frac 1{f_1} = \frac 1a + \frac 1b; \frac 1{f_2} = - \frac 1b + \frac 1c \implies \frac 1{f_1} + \frac 1{f_2} = \frac 1a + \frac 1c \implies f' = \frac 1{\frac 1{f_1} + \frac 1{f_2}} = \frac{f_1 f_2}{f_1 + f_2} \approx 13{,}6\,\text{см}$
}
\solutionspace{180pt}

\tasknumber{5}%
\task{%
    Линейные размеры прямого изображения предмета, полученного в собирающей линзе,
    в четыре раза больше линейных размеров предмета.
    Зная, что предмет находится на $20\,\text{см}$ ближе к линзе,
    чем его изображение, найти оптическую силу линзы.
}
\answer{%
    \begin{align*}
    D &= \frac 1F = \frac 1a + \frac 1b, \qquad \Gamma = \frac ba, \qquad b - a = \ell \implies b = \Gamma a \implies \Gamma a - a = \ell \implies  \\
    a &= \frac {\ell}{\Gamma - 1} \implies b = \frac {{\ell} \Gamma}{\Gamma - 1} \implies  \\
    D &= \frac {\Gamma - 1}\ell + \frac {\Gamma - 1}{\ell \Gamma} = \frac 1\ell \cdot \cbr{\Gamma - 1 + \frac {\Gamma - 1}{\Gamma} } =\frac 1\ell \cdot \cbr{\Gamma - \frac 1\Gamma} \approx 18{,}8\,\text{дптр}.
    \end{align*}
}
\solutionspace{180pt}

\tasknumber{6}%
\task{%
    Оптическая сила объектива фотоаппарата равна $4\,\text{дптр}$.
    При фотографировании чертежа с расстояния $0{,}8\,\text{м}$ площадь изображения
    чертежа на фотопластинке оказалась равной $4\,\text{см}^{2}$.
    Какова площадь самого чертежа? Ответ выразите в квадратных сантиметрах.
}
\solutionspace{180pt}

\tasknumber{7}%
\task{%
    В каком месте на главной оптической оси двояковыгнутой линзы
    нужно поместить точечный источник света,
    чтобы его изображение оказалось в главном фокусе линзы?
}
\answer{%
    $\text{на половине фокусного расстояния}$
}
\solutionspace{120pt}

\tasknumber{8}%
\task{%
    Предмет высотой $h = 40\,\text{см}$ находится на расстоянии $d = 1{,}2\,\text{м}$
    от вертикально расположенной рассеивающей линзы с фокусным расстоянием $F = -15\,\text{см}$.
    Где находится изображение предмета? Определите тип изображения и его высоту.
}
\solutionspace{120pt}

\tasknumber{9}%
\task{%
    На каком расстоянии от двояковыпуклой линзы с оптической силой $D = 1{,}5\,\text{дптр}$
    надо поместить предмет, чтобы его изображение получилось на расстоянии $2\,\text{м}$ от линзы?
}
\solutionspace{120pt}

\tasknumber{10}%
\task{%
    Предмет в виде отрезка длиной $\ell$ расположен вдоль оптической оси
    собирающей линзы с фокусным расстоянием $F$.
    Середина отрезка расположена
    на расстоянии $a$ от линзы, которая даёт действительное изображение
    всех точек предмета.
    Определить продольное увеличение предмета.
}
\answer{%
    \begin{align*}
    \frac 1{a + \frac \ell 2} &+ \frac 1b = \frac 1F \implies b = \frac{F\cbr{a + \frac \ell 2}}{a + \frac \ell 2 - F} \\
    \frac 1{a - \frac \ell 2} &+ \frac 1c = \frac 1F \implies c = \frac{F\cbr{a - \frac \ell 2}}{a - \frac \ell 2 - F} \\
    \abs{b - c} &= \abs{\frac{F\cbr{a + \frac \ell 2}}{a + \frac \ell 2 - F} - \frac{F\cbr{a - \frac \ell 2}}{a - \frac \ell 2 - F}}= F\abs{\frac{\cbr{a + \frac \ell 2}\cbr{a - \frac \ell 2 - F} - \cbr{a - \frac \ell 2}\cbr{a + \frac \ell 2 - F}}{ \cbr{a + \frac \ell 2 - F} \cbr{a - \frac \ell 2 - F} }} =  \\
    &= F\abs{\frac{a^2 - \frac {a\ell} 2 - Fa + \frac {a\ell} 2 - \frac {\ell^2} 4 - \frac {F\ell}2 - a^2 - \frac {a\ell}2 + aF + \frac {a\ell}2 + \frac {\ell^2} 4 - \frac {F\ell} 2}{\cbr{a + \frac \ell 2 - F} \cbr{a - \frac \ell 2 - F} }} = \\
    &= F\frac{F\ell}{\sqr{a-F} - \frac {\ell^2}4} = \frac{F^2\ell}{\sqr{a-F} - \frac {\ell^2}4}\implies \Gamma = \frac{\abs{b - c}}\ell = \frac{F^2}{\sqr{a-F} - \frac {\ell^2}4}.
    \end{align*}
}
\solutionspace{120pt}

\tasknumber{11}%
\task{%
    На экране с помощью тонкой линзы получено изображение предмета
    с увеличением $2$.
    Предмет передвинули на $4\,\text{см}$.
    Для того, чтобы получить резкое изображение, пришлось передвинуть экран.
    При этом увеличение оказалось равным $8$.
    На какое расстояние
    пришлось передвинуть экран?
}
\solutionspace{120pt}

\tasknumber{12}%
\task{%
    Тонкая собирающая линза дает изображение предмета на экране высотой $H_1$,
    и $H_2$, при двух положениях линзы между предметом и экраном.
    Расстояние между ними неизменно.
    Чему равна высота предмета $h$?
}
\answer{%
    $h = \sqrt{H_1 H_2}$
}
\solutionspace{120pt}

\tasknumber{13}%
\task{%
    Какие предметы можно рассмотреть на фотографии, сделанной со спутника,
    если разрешающая способность пленки $0{,}010\,\text{мм}$? Каким должно быть
    время экспозиции $\tau$ чтобы полностью использовать возможности пленки?
    Фокусное расстояние объектива используемого фотоаппарата $20\,\text{cм}$,
    высота орбиты спутника $100\,\text{км}$.
}
\solutionspace{120pt}

\tasknumber{14}%
\task{%
    При аэрофотосъемках используется фотоаппарат, объектив которого
    имеет фокусиое расстояние $8\,\text{cм}$.
    Разрешающая способность пленки $0{,}02\,\text{мм}$.
    На какой высоте должен лететь самолет, чтобы на фотографии можно
    было различить листья деревьев размером $4\,\text{cм}$?
    При какой скорости самолета изображение не будет размытым,
    если время зкспозиции $1\,\text{мс}$?
}

\variantsplitter

\addpersonalvariant{Софья Андрианова}

\tasknumber{1}%
\task{%
    Найти оптическую силу собирающей линзы, если действительное изображение предмета,
    помещённого в $15\,\text{см}$ от линзы, получается на расстоянии $20\,\text{см}$ от неё.
}
\answer{%
    $D = \frac 1F = \frac 1a + \frac 1b = \frac 1{15\,\text{см}} + \frac 1{20\,\text{см}} \approx 11{,}67\,\text{дптр}$
}
\solutionspace{180pt}

\tasknumber{2}%
\task{%
    Найти увеличение изображения, если изображение предмета, находящегося
    на расстоянии $15\,\text{см}$ от линзы, получается на расстоянии $18\,\text{см}$ от неё.
}
\answer{%
    $\Gamma = \frac ba = \frac {18\,\text{см}}{15\,\text{см}} \approx 1{,}2$
}
\solutionspace{180pt}

\tasknumber{3}%
\task{%
    Расстояние от предмета до линзы $8\,\text{см}$, а от линзы до мнимого изображения $25\,\text{см}$.
    Чему равно фокусное расстояние линзы?
}
\answer{%
    $\pm \frac 1F = \frac 1a - \frac 1b \implies F = \frac{a b}{\abs{b - a}} \approx 11{,}8\,\text{см}$
}
\solutionspace{180pt}

\tasknumber{4}%
\task{%
    Две тонкие собирающие линзы с фокусными расстояниями $25\,\text{см}$ и $20\,\text{см}$ сложены вместе.
    Чему равно фокусное расстояние такой оптической системы?
}
\answer{%
    $\frac 1{f_1} = \frac 1a + \frac 1b; \frac 1{f_2} = - \frac 1b + \frac 1c \implies \frac 1{f_1} + \frac 1{f_2} = \frac 1a + \frac 1c \implies f' = \frac 1{\frac 1{f_1} + \frac 1{f_2}} = \frac{f_1 f_2}{f_1 + f_2} \approx 11{,}1\,\text{см}$
}
\solutionspace{180pt}

\tasknumber{5}%
\task{%
    Линейные размеры прямого изображения предмета, полученного в собирающей линзе,
    в три раза больше линейных размеров предмета.
    Зная, что предмет находится на $30\,\text{см}$ ближе к линзе,
    чем его изображение, найти оптическую силу линзы.
}
\answer{%
    \begin{align*}
    D &= \frac 1F = \frac 1a + \frac 1b, \qquad \Gamma = \frac ba, \qquad b - a = \ell \implies b = \Gamma a \implies \Gamma a - a = \ell \implies  \\
    a &= \frac {\ell}{\Gamma - 1} \implies b = \frac {{\ell} \Gamma}{\Gamma - 1} \implies  \\
    D &= \frac {\Gamma - 1}\ell + \frac {\Gamma - 1}{\ell \Gamma} = \frac 1\ell \cdot \cbr{\Gamma - 1 + \frac {\Gamma - 1}{\Gamma} } =\frac 1\ell \cdot \cbr{\Gamma - \frac 1\Gamma} \approx 8{,}9\,\text{дптр}.
    \end{align*}
}
\solutionspace{180pt}

\tasknumber{6}%
\task{%
    Оптическая сила объектива фотоаппарата равна $3\,\text{дптр}$.
    При фотографировании чертежа с расстояния $1{,}1\,\text{м}$ площадь изображения
    чертежа на фотопластинке оказалась равной $4\,\text{см}^{2}$.
    Какова площадь самого чертежа? Ответ выразите в квадратных сантиметрах.
}
\solutionspace{180pt}

\tasknumber{7}%
\task{%
    В каком месте на главной оптической оси двояковыпуклой линзы
    нужно поместить точечный источник света,
    чтобы его изображение оказалось в главном фокусе линзы?
}
\answer{%
    $\text{для мнимого - на половине фокусного, для действительного - на бесконечности}$
}
\solutionspace{120pt}

\tasknumber{8}%
\task{%
    Предмет высотой $h = 30\,\text{см}$ находится на расстоянии $d = 1{,}2\,\text{м}$
    от вертикально расположенной рассеивающей линзы с фокусным расстоянием $F = -15\,\text{см}$.
    Где находится изображение предмета? Определите тип изображения и его высоту.
}
\solutionspace{120pt}

\tasknumber{9}%
\task{%
    На каком расстоянии от двояковыпуклой линзы с оптической силой $D = 2\,\text{дптр}$
    надо поместить предмет, чтобы его изображение получилось на расстоянии $1{,}5\,\text{м}$ от линзы?
}
\solutionspace{120pt}

\tasknumber{10}%
\task{%
    Предмет в виде отрезка длиной $\ell$ расположен вдоль оптической оси
    собирающей линзы с фокусным расстоянием $F$.
    Середина отрезка расположена
    на расстоянии $a$ от линзы, которая даёт действительное изображение
    всех точек предмета.
    Определить продольное увеличение предмета.
}
\answer{%
    \begin{align*}
    \frac 1{a + \frac \ell 2} &+ \frac 1b = \frac 1F \implies b = \frac{F\cbr{a + \frac \ell 2}}{a + \frac \ell 2 - F} \\
    \frac 1{a - \frac \ell 2} &+ \frac 1c = \frac 1F \implies c = \frac{F\cbr{a - \frac \ell 2}}{a - \frac \ell 2 - F} \\
    \abs{b - c} &= \abs{\frac{F\cbr{a + \frac \ell 2}}{a + \frac \ell 2 - F} - \frac{F\cbr{a - \frac \ell 2}}{a - \frac \ell 2 - F}}= F\abs{\frac{\cbr{a + \frac \ell 2}\cbr{a - \frac \ell 2 - F} - \cbr{a - \frac \ell 2}\cbr{a + \frac \ell 2 - F}}{ \cbr{a + \frac \ell 2 - F} \cbr{a - \frac \ell 2 - F} }} =  \\
    &= F\abs{\frac{a^2 - \frac {a\ell} 2 - Fa + \frac {a\ell} 2 - \frac {\ell^2} 4 - \frac {F\ell}2 - a^2 - \frac {a\ell}2 + aF + \frac {a\ell}2 + \frac {\ell^2} 4 - \frac {F\ell} 2}{\cbr{a + \frac \ell 2 - F} \cbr{a - \frac \ell 2 - F} }} = \\
    &= F\frac{F\ell}{\sqr{a-F} - \frac {\ell^2}4} = \frac{F^2\ell}{\sqr{a-F} - \frac {\ell^2}4}\implies \Gamma = \frac{\abs{b - c}}\ell = \frac{F^2}{\sqr{a-F} - \frac {\ell^2}4}.
    \end{align*}
}
\solutionspace{120pt}

\tasknumber{11}%
\task{%
    На экране с помощью тонкой линзы получено изображение предмета
    с увеличением $4$.
    Предмет передвинули на $6\,\text{см}$.
    Для того, чтобы получить резкое изображение, пришлось передвинуть экран.
    При этом увеличение оказалось равным $6$.
    На какое расстояние
    пришлось передвинуть экран?
}
\solutionspace{120pt}

\tasknumber{12}%
\task{%
    Тонкая собирающая линза дает изображение предмета на экране высотой $H_1$,
    и $H_2$, при двух положениях линзы между предметом и экраном.
    Расстояние между ними неизменно.
    Чему равна высота предмета $h$?
}
\answer{%
    $h = \sqrt{H_1 H_2}$
}
\solutionspace{120pt}

\tasknumber{13}%
\task{%
    Какие предметы можно рассмотреть на фотографии, сделанной со спутника,
    если разрешающая способность пленки $0{,}02\,\text{мм}$? Каким должно быть
    время экспозиции $\tau$ чтобы полностью использовать возможности пленки?
    Фокусное расстояние объектива используемого фотоаппарата $10\,\text{cм}$,
    высота орбиты спутника $80\,\text{км}$.
}
\solutionspace{120pt}

\tasknumber{14}%
\task{%
    При аэрофотосъемках используется фотоаппарат, объектив которого
    имеет фокусиое расстояние $8\,\text{cм}$.
    Разрешающая способность пленки $0{,}02\,\text{мм}$.
    На какой высоте должен лететь самолет, чтобы на фотографии можно
    было различить листья деревьев размером $5\,\text{cм}$?
    При какой скорости самолета изображение не будет размытым,
    если время зкспозиции $1\,\text{мс}$?
}

\variantsplitter

\addpersonalvariant{Владимир Артемчук}

\tasknumber{1}%
\task{%
    Найти оптическую силу собирающей линзы, если действительное изображение предмета,
    помещённого в $15\,\text{см}$ от линзы, получается на расстоянии $40\,\text{см}$ от неё.
}
\answer{%
    $D = \frac 1F = \frac 1a + \frac 1b = \frac 1{15\,\text{см}} + \frac 1{40\,\text{см}} \approx 9{,}17\,\text{дптр}$
}
\solutionspace{180pt}

\tasknumber{2}%
\task{%
    Найти увеличение изображения, если изображение предмета, находящегося
    на расстоянии $25\,\text{см}$ от линзы, получается на расстоянии $30\,\text{см}$ от неё.
}
\answer{%
    $\Gamma = \frac ba = \frac {30\,\text{см}}{25\,\text{см}} \approx 1{,}20$
}
\solutionspace{180pt}

\tasknumber{3}%
\task{%
    Расстояние от предмета до линзы $12\,\text{см}$, а от линзы до мнимого изображения $30\,\text{см}$.
    Чему равно фокусное расстояние линзы?
}
\answer{%
    $\pm \frac 1F = \frac 1a - \frac 1b \implies F = \frac{a b}{\abs{b - a}} \approx 20\,\text{см}$
}
\solutionspace{180pt}

\tasknumber{4}%
\task{%
    Две тонкие собирающие линзы с фокусными расстояниями $12\,\text{см}$ и $30\,\text{см}$ сложены вместе.
    Чему равно фокусное расстояние такой оптической системы?
}
\answer{%
    $\frac 1{f_1} = \frac 1a + \frac 1b; \frac 1{f_2} = - \frac 1b + \frac 1c \implies \frac 1{f_1} + \frac 1{f_2} = \frac 1a + \frac 1c \implies f' = \frac 1{\frac 1{f_1} + \frac 1{f_2}} = \frac{f_1 f_2}{f_1 + f_2} \approx 8{,}6\,\text{см}$
}
\solutionspace{180pt}

\tasknumber{5}%
\task{%
    Линейные размеры прямого изображения предмета, полученного в собирающей линзе,
    в четыре раза больше линейных размеров предмета.
    Зная, что предмет находится на $20\,\text{см}$ ближе к линзе,
    чем его изображение, найти оптическую силу линзы.
}
\answer{%
    \begin{align*}
    D &= \frac 1F = \frac 1a + \frac 1b, \qquad \Gamma = \frac ba, \qquad b - a = \ell \implies b = \Gamma a \implies \Gamma a - a = \ell \implies  \\
    a &= \frac {\ell}{\Gamma - 1} \implies b = \frac {{\ell} \Gamma}{\Gamma - 1} \implies  \\
    D &= \frac {\Gamma - 1}\ell + \frac {\Gamma - 1}{\ell \Gamma} = \frac 1\ell \cdot \cbr{\Gamma - 1 + \frac {\Gamma - 1}{\Gamma} } =\frac 1\ell \cdot \cbr{\Gamma - \frac 1\Gamma} \approx 18{,}8\,\text{дптр}.
    \end{align*}
}
\solutionspace{180pt}

\tasknumber{6}%
\task{%
    Оптическая сила объектива фотоаппарата равна $4\,\text{дптр}$.
    При фотографировании чертежа с расстояния $1{,}2\,\text{м}$ площадь изображения
    чертежа на фотопластинке оказалась равной $4\,\text{см}^{2}$.
    Какова площадь самого чертежа? Ответ выразите в квадратных сантиметрах.
}
\solutionspace{180pt}

\tasknumber{7}%
\task{%
    В каком месте на главной оптической оси двояковыпуклой линзы
    нужно поместить точечный источник света,
    чтобы его изображение оказалось в главном фокусе линзы?
}
\answer{%
    $\text{для мнимого - на половине фокусного, для действительного - на бесконечности}$
}
\solutionspace{120pt}

\tasknumber{8}%
\task{%
    Предмет высотой $h = 40\,\text{см}$ находится на расстоянии $d = 1{,}2\,\text{м}$
    от вертикально расположенной рассеивающей линзы с фокусным расстоянием $F = 20\,\text{см}$.
    Где находится изображение предмета? Определите тип изображения и его высоту.
}
\solutionspace{120pt}

\tasknumber{9}%
\task{%
    На каком расстоянии от двояковыпуклой линзы с оптической силой $D = 1{,}5\,\text{дптр}$
    надо поместить предмет, чтобы его изображение получилось на расстоянии $2\,\text{м}$ от линзы?
}
\solutionspace{120pt}

\tasknumber{10}%
\task{%
    Предмет в виде отрезка длиной $\ell$ расположен вдоль оптической оси
    собирающей линзы с фокусным расстоянием $F$.
    Середина отрезка расположена
    на расстоянии $a$ от линзы, которая даёт действительное изображение
    всех точек предмета.
    Определить продольное увеличение предмета.
}
\answer{%
    \begin{align*}
    \frac 1{a + \frac \ell 2} &+ \frac 1b = \frac 1F \implies b = \frac{F\cbr{a + \frac \ell 2}}{a + \frac \ell 2 - F} \\
    \frac 1{a - \frac \ell 2} &+ \frac 1c = \frac 1F \implies c = \frac{F\cbr{a - \frac \ell 2}}{a - \frac \ell 2 - F} \\
    \abs{b - c} &= \abs{\frac{F\cbr{a + \frac \ell 2}}{a + \frac \ell 2 - F} - \frac{F\cbr{a - \frac \ell 2}}{a - \frac \ell 2 - F}}= F\abs{\frac{\cbr{a + \frac \ell 2}\cbr{a - \frac \ell 2 - F} - \cbr{a - \frac \ell 2}\cbr{a + \frac \ell 2 - F}}{ \cbr{a + \frac \ell 2 - F} \cbr{a - \frac \ell 2 - F} }} =  \\
    &= F\abs{\frac{a^2 - \frac {a\ell} 2 - Fa + \frac {a\ell} 2 - \frac {\ell^2} 4 - \frac {F\ell}2 - a^2 - \frac {a\ell}2 + aF + \frac {a\ell}2 + \frac {\ell^2} 4 - \frac {F\ell} 2}{\cbr{a + \frac \ell 2 - F} \cbr{a - \frac \ell 2 - F} }} = \\
    &= F\frac{F\ell}{\sqr{a-F} - \frac {\ell^2}4} = \frac{F^2\ell}{\sqr{a-F} - \frac {\ell^2}4}\implies \Gamma = \frac{\abs{b - c}}\ell = \frac{F^2}{\sqr{a-F} - \frac {\ell^2}4}.
    \end{align*}
}
\solutionspace{120pt}

\tasknumber{11}%
\task{%
    На экране с помощью тонкой линзы получено изображение предмета
    с увеличением $2$.
    Предмет передвинули на $8\,\text{см}$.
    Для того, чтобы получить резкое изображение, пришлось передвинуть экран.
    При этом увеличение оказалось равным $8$.
    На какое расстояние
    пришлось передвинуть экран?
}
\solutionspace{120pt}

\tasknumber{12}%
\task{%
    Тонкая собирающая линза дает изображение предмета на экране высотой $H_1$,
    и $H_2$, при двух положениях линзы между предметом и экраном.
    Расстояние между ними неизменно.
    Чему равна высота предмета $h$?
}
\answer{%
    $h = \sqrt{H_1 H_2}$
}
\solutionspace{120pt}

\tasknumber{13}%
\task{%
    Какие предметы можно рассмотреть на фотографии, сделанной со спутника,
    если разрешающая способность пленки $0{,}010\,\text{мм}$? Каким должно быть
    время экспозиции $\tau$ чтобы полностью использовать возможности пленки?
    Фокусное расстояние объектива используемого фотоаппарата $15\,\text{cм}$,
    высота орбиты спутника $100\,\text{км}$.
}
\solutionspace{120pt}

\tasknumber{14}%
\task{%
    При аэрофотосъемках используется фотоаппарат, объектив которого
    имеет фокусиое расстояние $10\,\text{cм}$.
    Разрешающая способность пленки $0{,}010\,\text{мм}$.
    На какой высоте должен лететь самолет, чтобы на фотографии можно
    было различить листья деревьев размером $4\,\text{cм}$?
    При какой скорости самолета изображение не будет размытым,
    если время зкспозиции $2\,\text{мс}$?
}

\variantsplitter

\addpersonalvariant{Софья Белянкина}

\tasknumber{1}%
\task{%
    Найти оптическую силу собирающей линзы, если действительное изображение предмета,
    помещённого в $35\,\text{см}$ от линзы, получается на расстоянии $40\,\text{см}$ от неё.
}
\answer{%
    $D = \frac 1F = \frac 1a + \frac 1b = \frac 1{35\,\text{см}} + \frac 1{40\,\text{см}} \approx 5{,}36\,\text{дптр}$
}
\solutionspace{180pt}

\tasknumber{2}%
\task{%
    Найти увеличение изображения, если изображение предмета, находящегося
    на расстоянии $20\,\text{см}$ от линзы, получается на расстоянии $30\,\text{см}$ от неё.
}
\answer{%
    $\Gamma = \frac ba = \frac {30\,\text{см}}{20\,\text{см}} \approx 1{,}50$
}
\solutionspace{180pt}

\tasknumber{3}%
\task{%
    Расстояние от предмета до линзы $10\,\text{см}$, а от линзы до мнимого изображения $30\,\text{см}$.
    Чему равно фокусное расстояние линзы?
}
\answer{%
    $\pm \frac 1F = \frac 1a - \frac 1b \implies F = \frac{a b}{\abs{b - a}} \approx 15\,\text{см}$
}
\solutionspace{180pt}

\tasknumber{4}%
\task{%
    Две тонкие собирающие линзы с фокусными расстояниями $12\,\text{см}$ и $30\,\text{см}$ сложены вместе.
    Чему равно фокусное расстояние такой оптической системы?
}
\answer{%
    $\frac 1{f_1} = \frac 1a + \frac 1b; \frac 1{f_2} = - \frac 1b + \frac 1c \implies \frac 1{f_1} + \frac 1{f_2} = \frac 1a + \frac 1c \implies f' = \frac 1{\frac 1{f_1} + \frac 1{f_2}} = \frac{f_1 f_2}{f_1 + f_2} \approx 8{,}6\,\text{см}$
}
\solutionspace{180pt}

\tasknumber{5}%
\task{%
    Линейные размеры прямого изображения предмета, полученного в собирающей линзе,
    в четыре раза больше линейных размеров предмета.
    Зная, что предмет находится на $30\,\text{см}$ ближе к линзе,
    чем его изображение, найти оптическую силу линзы.
}
\answer{%
    \begin{align*}
    D &= \frac 1F = \frac 1a + \frac 1b, \qquad \Gamma = \frac ba, \qquad b - a = \ell \implies b = \Gamma a \implies \Gamma a - a = \ell \implies  \\
    a &= \frac {\ell}{\Gamma - 1} \implies b = \frac {{\ell} \Gamma}{\Gamma - 1} \implies  \\
    D &= \frac {\Gamma - 1}\ell + \frac {\Gamma - 1}{\ell \Gamma} = \frac 1\ell \cdot \cbr{\Gamma - 1 + \frac {\Gamma - 1}{\Gamma} } =\frac 1\ell \cdot \cbr{\Gamma - \frac 1\Gamma} \approx 12{,}5\,\text{дптр}.
    \end{align*}
}
\solutionspace{180pt}

\tasknumber{6}%
\task{%
    Оптическая сила объектива фотоаппарата равна $6\,\text{дптр}$.
    При фотографировании чертежа с расстояния $1{,}2\,\text{м}$ площадь изображения
    чертежа на фотопластинке оказалась равной $4\,\text{см}^{2}$.
    Какова площадь самого чертежа? Ответ выразите в квадратных сантиметрах.
}
\solutionspace{180pt}

\tasknumber{7}%
\task{%
    В каком месте на главной оптической оси двояковыпуклой линзы
    нужно поместить точечный источник света,
    чтобы его изображение оказалось в главном фокусе линзы?
}
\answer{%
    $\text{для мнимого - на половине фокусного, для действительного - на бесконечности}$
}
\solutionspace{120pt}

\tasknumber{8}%
\task{%
    Предмет высотой $h = 40\,\text{см}$ находится на расстоянии $d = 1\,\text{м}$
    от вертикально расположенной рассеивающей линзы с фокусным расстоянием $F = 20\,\text{см}$.
    Где находится изображение предмета? Определите тип изображения и его высоту.
}
\solutionspace{120pt}

\tasknumber{9}%
\task{%
    На каком расстоянии от двояковыпуклой линзы с оптической силой $D = 2{,}5\,\text{дптр}$
    надо поместить предмет, чтобы его изображение получилось на расстоянии $2{,}5\,\text{м}$ от линзы?
}
\solutionspace{120pt}

\tasknumber{10}%
\task{%
    Предмет в виде отрезка длиной $\ell$ расположен вдоль оптической оси
    собирающей линзы с фокусным расстоянием $F$.
    Середина отрезка расположена
    на расстоянии $a$ от линзы, которая даёт действительное изображение
    всех точек предмета.
    Определить продольное увеличение предмета.
}
\answer{%
    \begin{align*}
    \frac 1{a + \frac \ell 2} &+ \frac 1b = \frac 1F \implies b = \frac{F\cbr{a + \frac \ell 2}}{a + \frac \ell 2 - F} \\
    \frac 1{a - \frac \ell 2} &+ \frac 1c = \frac 1F \implies c = \frac{F\cbr{a - \frac \ell 2}}{a - \frac \ell 2 - F} \\
    \abs{b - c} &= \abs{\frac{F\cbr{a + \frac \ell 2}}{a + \frac \ell 2 - F} - \frac{F\cbr{a - \frac \ell 2}}{a - \frac \ell 2 - F}}= F\abs{\frac{\cbr{a + \frac \ell 2}\cbr{a - \frac \ell 2 - F} - \cbr{a - \frac \ell 2}\cbr{a + \frac \ell 2 - F}}{ \cbr{a + \frac \ell 2 - F} \cbr{a - \frac \ell 2 - F} }} =  \\
    &= F\abs{\frac{a^2 - \frac {a\ell} 2 - Fa + \frac {a\ell} 2 - \frac {\ell^2} 4 - \frac {F\ell}2 - a^2 - \frac {a\ell}2 + aF + \frac {a\ell}2 + \frac {\ell^2} 4 - \frac {F\ell} 2}{\cbr{a + \frac \ell 2 - F} \cbr{a - \frac \ell 2 - F} }} = \\
    &= F\frac{F\ell}{\sqr{a-F} - \frac {\ell^2}4} = \frac{F^2\ell}{\sqr{a-F} - \frac {\ell^2}4}\implies \Gamma = \frac{\abs{b - c}}\ell = \frac{F^2}{\sqr{a-F} - \frac {\ell^2}4}.
    \end{align*}
}
\solutionspace{120pt}

\tasknumber{11}%
\task{%
    На экране с помощью тонкой линзы получено изображение предмета
    с увеличением $4$.
    Предмет передвинули на $6\,\text{см}$.
    Для того, чтобы получить резкое изображение, пришлось передвинуть экран.
    При этом увеличение оказалось равным $6$.
    На какое расстояние
    пришлось передвинуть экран?
}
\solutionspace{120pt}

\tasknumber{12}%
\task{%
    Тонкая собирающая линза дает изображение предмета на экране высотой $H_1$,
    и $H_2$, при двух положениях линзы между предметом и экраном.
    Расстояние между ними неизменно.
    Чему равна высота предмета $h$?
}
\answer{%
    $h = \sqrt{H_1 H_2}$
}
\solutionspace{120pt}

\tasknumber{13}%
\task{%
    Какие предметы можно рассмотреть на фотографии, сделанной со спутника,
    если разрешающая способность пленки $0{,}010\,\text{мм}$? Каким должно быть
    время экспозиции $\tau$ чтобы полностью использовать возможности пленки?
    Фокусное расстояние объектива используемого фотоаппарата $20\,\text{cм}$,
    высота орбиты спутника $80\,\text{км}$.
}
\solutionspace{120pt}

\tasknumber{14}%
\task{%
    При аэрофотосъемках используется фотоаппарат, объектив которого
    имеет фокусиое расстояние $12\,\text{cм}$.
    Разрешающая способность пленки $0{,}015\,\text{мм}$.
    На какой высоте должен лететь самолет, чтобы на фотографии можно
    было различить листья деревьев размером $4\,\text{cм}$?
    При какой скорости самолета изображение не будет размытым,
    если время зкспозиции $1\,\text{мс}$?
}

\variantsplitter

\addpersonalvariant{Варвара Егиазарян}

\tasknumber{1}%
\task{%
    Найти оптическую силу собирающей линзы, если действительное изображение предмета,
    помещённого в $35\,\text{см}$ от линзы, получается на расстоянии $40\,\text{см}$ от неё.
}
\answer{%
    $D = \frac 1F = \frac 1a + \frac 1b = \frac 1{35\,\text{см}} + \frac 1{40\,\text{см}} \approx 5{,}36\,\text{дптр}$
}
\solutionspace{180pt}

\tasknumber{2}%
\task{%
    Найти увеличение изображения, если изображение предмета, находящегося
    на расстоянии $15\,\text{см}$ от линзы, получается на расстоянии $12\,\text{см}$ от неё.
}
\answer{%
    $\Gamma = \frac ba = \frac {12\,\text{см}}{15\,\text{см}} \approx 0{,}8$
}
\solutionspace{180pt}

\tasknumber{3}%
\task{%
    Расстояние от предмета до линзы $8\,\text{см}$, а от линзы до мнимого изображения $20\,\text{см}$.
    Чему равно фокусное расстояние линзы?
}
\answer{%
    $\pm \frac 1F = \frac 1a - \frac 1b \implies F = \frac{a b}{\abs{b - a}} \approx 13{,}3\,\text{см}$
}
\solutionspace{180pt}

\tasknumber{4}%
\task{%
    Две тонкие собирающие линзы с фокусными расстояниями $18\,\text{см}$ и $30\,\text{см}$ сложены вместе.
    Чему равно фокусное расстояние такой оптической системы?
}
\answer{%
    $\frac 1{f_1} = \frac 1a + \frac 1b; \frac 1{f_2} = - \frac 1b + \frac 1c \implies \frac 1{f_1} + \frac 1{f_2} = \frac 1a + \frac 1c \implies f' = \frac 1{\frac 1{f_1} + \frac 1{f_2}} = \frac{f_1 f_2}{f_1 + f_2} \approx 11{,}2\,\text{см}$
}
\solutionspace{180pt}

\tasknumber{5}%
\task{%
    Линейные размеры прямого изображения предмета, полученного в собирающей линзе,
    в два раза больше линейных размеров предмета.
    Зная, что предмет находится на $30\,\text{см}$ ближе к линзе,
    чем его изображение, найти оптическую силу линзы.
}
\answer{%
    \begin{align*}
    D &= \frac 1F = \frac 1a + \frac 1b, \qquad \Gamma = \frac ba, \qquad b - a = \ell \implies b = \Gamma a \implies \Gamma a - a = \ell \implies  \\
    a &= \frac {\ell}{\Gamma - 1} \implies b = \frac {{\ell} \Gamma}{\Gamma - 1} \implies  \\
    D &= \frac {\Gamma - 1}\ell + \frac {\Gamma - 1}{\ell \Gamma} = \frac 1\ell \cdot \cbr{\Gamma - 1 + \frac {\Gamma - 1}{\Gamma} } =\frac 1\ell \cdot \cbr{\Gamma - \frac 1\Gamma} \approx 5\,\text{дптр}.
    \end{align*}
}
\solutionspace{180pt}

\tasknumber{6}%
\task{%
    Оптическая сила объектива фотоаппарата равна $5\,\text{дптр}$.
    При фотографировании чертежа с расстояния $0{,}8\,\text{м}$ площадь изображения
    чертежа на фотопластинке оказалась равной $4\,\text{см}^{2}$.
    Какова площадь самого чертежа? Ответ выразите в квадратных сантиметрах.
}
\solutionspace{180pt}

\tasknumber{7}%
\task{%
    В каком месте на главной оптической оси двояковыпуклой линзы
    нужно поместить точечный источник света,
    чтобы его изображение оказалось в главном фокусе линзы?
}
\answer{%
    $\text{для мнимого - на половине фокусного, для действительного - на бесконечности}$
}
\solutionspace{120pt}

\tasknumber{8}%
\task{%
    Предмет высотой $h = 30\,\text{см}$ находится на расстоянии $d = 0{,}8\,\text{м}$
    от вертикально расположенной рассеивающей линзы с фокусным расстоянием $F = 25\,\text{см}$.
    Где находится изображение предмета? Определите тип изображения и его высоту.
}
\solutionspace{120pt}

\tasknumber{9}%
\task{%
    На каком расстоянии от двояковыпуклой линзы с оптической силой $D = 2\,\text{дптр}$
    надо поместить предмет, чтобы его изображение получилось на расстоянии $1{,}5\,\text{м}$ от линзы?
}
\solutionspace{120pt}

\tasknumber{10}%
\task{%
    Предмет в виде отрезка длиной $\ell$ расположен вдоль оптической оси
    собирающей линзы с фокусным расстоянием $F$.
    Середина отрезка расположена
    на расстоянии $a$ от линзы, которая даёт действительное изображение
    всех точек предмета.
    Определить продольное увеличение предмета.
}
\answer{%
    \begin{align*}
    \frac 1{a + \frac \ell 2} &+ \frac 1b = \frac 1F \implies b = \frac{F\cbr{a + \frac \ell 2}}{a + \frac \ell 2 - F} \\
    \frac 1{a - \frac \ell 2} &+ \frac 1c = \frac 1F \implies c = \frac{F\cbr{a - \frac \ell 2}}{a - \frac \ell 2 - F} \\
    \abs{b - c} &= \abs{\frac{F\cbr{a + \frac \ell 2}}{a + \frac \ell 2 - F} - \frac{F\cbr{a - \frac \ell 2}}{a - \frac \ell 2 - F}}= F\abs{\frac{\cbr{a + \frac \ell 2}\cbr{a - \frac \ell 2 - F} - \cbr{a - \frac \ell 2}\cbr{a + \frac \ell 2 - F}}{ \cbr{a + \frac \ell 2 - F} \cbr{a - \frac \ell 2 - F} }} =  \\
    &= F\abs{\frac{a^2 - \frac {a\ell} 2 - Fa + \frac {a\ell} 2 - \frac {\ell^2} 4 - \frac {F\ell}2 - a^2 - \frac {a\ell}2 + aF + \frac {a\ell}2 + \frac {\ell^2} 4 - \frac {F\ell} 2}{\cbr{a + \frac \ell 2 - F} \cbr{a - \frac \ell 2 - F} }} = \\
    &= F\frac{F\ell}{\sqr{a-F} - \frac {\ell^2}4} = \frac{F^2\ell}{\sqr{a-F} - \frac {\ell^2}4}\implies \Gamma = \frac{\abs{b - c}}\ell = \frac{F^2}{\sqr{a-F} - \frac {\ell^2}4}.
    \end{align*}
}
\solutionspace{120pt}

\tasknumber{11}%
\task{%
    На экране с помощью тонкой линзы получено изображение предмета
    с увеличением $2$.
    Предмет передвинули на $6\,\text{см}$.
    Для того, чтобы получить резкое изображение, пришлось передвинуть экран.
    При этом увеличение оказалось равным $8$.
    На какое расстояние
    пришлось передвинуть экран?
}
\solutionspace{120pt}

\tasknumber{12}%
\task{%
    Тонкая собирающая линза дает изображение предмета на экране высотой $H_1$,
    и $H_2$, при двух положениях линзы между предметом и экраном.
    Расстояние между ними неизменно.
    Чему равна высота предмета $h$?
}
\answer{%
    $h = \sqrt{H_1 H_2}$
}
\solutionspace{120pt}

\tasknumber{13}%
\task{%
    Какие предметы можно рассмотреть на фотографии, сделанной со спутника,
    если разрешающая способность пленки $0{,}010\,\text{мм}$? Каким должно быть
    время экспозиции $\tau$ чтобы полностью использовать возможности пленки?
    Фокусное расстояние объектива используемого фотоаппарата $10\,\text{cм}$,
    высота орбиты спутника $100\,\text{км}$.
}
\solutionspace{120pt}

\tasknumber{14}%
\task{%
    При аэрофотосъемках используется фотоаппарат, объектив которого
    имеет фокусиое расстояние $8\,\text{cм}$.
    Разрешающая способность пленки $0{,}010\,\text{мм}$.
    На какой высоте должен лететь самолет, чтобы на фотографии можно
    было различить листья деревьев размером $6\,\text{cм}$?
    При какой скорости самолета изображение не будет размытым,
    если время зкспозиции $2\,\text{мс}$?
}

\variantsplitter

\addpersonalvariant{Владислав Емелин}

\tasknumber{1}%
\task{%
    Найти оптическую силу собирающей линзы, если действительное изображение предмета,
    помещённого в $55\,\text{см}$ от линзы, получается на расстоянии $40\,\text{см}$ от неё.
}
\answer{%
    $D = \frac 1F = \frac 1a + \frac 1b = \frac 1{55\,\text{см}} + \frac 1{40\,\text{см}} \approx 4{,}32\,\text{дптр}$
}
\solutionspace{180pt}

\tasknumber{2}%
\task{%
    Найти увеличение изображения, если изображение предмета, находящегося
    на расстоянии $25\,\text{см}$ от линзы, получается на расстоянии $12\,\text{см}$ от неё.
}
\answer{%
    $\Gamma = \frac ba = \frac {12\,\text{см}}{25\,\text{см}} \approx 0{,}5$
}
\solutionspace{180pt}

\tasknumber{3}%
\task{%
    Расстояние от предмета до линзы $12\,\text{см}$, а от линзы до мнимого изображения $20\,\text{см}$.
    Чему равно фокусное расстояние линзы?
}
\answer{%
    $\pm \frac 1F = \frac 1a - \frac 1b \implies F = \frac{a b}{\abs{b - a}} \approx 30\,\text{см}$
}
\solutionspace{180pt}

\tasknumber{4}%
\task{%
    Две тонкие собирающие линзы с фокусными расстояниями $12\,\text{см}$ и $30\,\text{см}$ сложены вместе.
    Чему равно фокусное расстояние такой оптической системы?
}
\answer{%
    $\frac 1{f_1} = \frac 1a + \frac 1b; \frac 1{f_2} = - \frac 1b + \frac 1c \implies \frac 1{f_1} + \frac 1{f_2} = \frac 1a + \frac 1c \implies f' = \frac 1{\frac 1{f_1} + \frac 1{f_2}} = \frac{f_1 f_2}{f_1 + f_2} \approx 8{,}6\,\text{см}$
}
\solutionspace{180pt}

\tasknumber{5}%
\task{%
    Линейные размеры прямого изображения предмета, полученного в собирающей линзе,
    в два раза больше линейных размеров предмета.
    Зная, что предмет находится на $40\,\text{см}$ ближе к линзе,
    чем его изображение, найти оптическую силу линзы.
}
\answer{%
    \begin{align*}
    D &= \frac 1F = \frac 1a + \frac 1b, \qquad \Gamma = \frac ba, \qquad b - a = \ell \implies b = \Gamma a \implies \Gamma a - a = \ell \implies  \\
    a &= \frac {\ell}{\Gamma - 1} \implies b = \frac {{\ell} \Gamma}{\Gamma - 1} \implies  \\
    D &= \frac {\Gamma - 1}\ell + \frac {\Gamma - 1}{\ell \Gamma} = \frac 1\ell \cdot \cbr{\Gamma - 1 + \frac {\Gamma - 1}{\Gamma} } =\frac 1\ell \cdot \cbr{\Gamma - \frac 1\Gamma} \approx 3{,}8\,\text{дптр}.
    \end{align*}
}
\solutionspace{180pt}

\tasknumber{6}%
\task{%
    Оптическая сила объектива фотоаппарата равна $6\,\text{дптр}$.
    При фотографировании чертежа с расстояния $1{,}1\,\text{м}$ площадь изображения
    чертежа на фотопластинке оказалась равной $9\,\text{см}^{2}$.
    Какова площадь самого чертежа? Ответ выразите в квадратных сантиметрах.
}
\solutionspace{180pt}

\tasknumber{7}%
\task{%
    В каком месте на главной оптической оси двояковыпуклой линзы
    нужно поместить точечный источник света,
    чтобы его изображение оказалось в главном фокусе линзы?
}
\answer{%
    $\text{для мнимого - на половине фокусного, для действительного - на бесконечности}$
}
\solutionspace{120pt}

\tasknumber{8}%
\task{%
    Предмет высотой $h = 50\,\text{см}$ находится на расстоянии $d = 1\,\text{м}$
    от вертикально расположенной рассеивающей линзы с фокусным расстоянием $F = 25\,\text{см}$.
    Где находится изображение предмета? Определите тип изображения и его высоту.
}
\solutionspace{120pt}

\tasknumber{9}%
\task{%
    На каком расстоянии от двояковыпуклой линзы с оптической силой $D = 2{,}5\,\text{дптр}$
    надо поместить предмет, чтобы его изображение получилось на расстоянии $1{,}5\,\text{м}$ от линзы?
}
\solutionspace{120pt}

\tasknumber{10}%
\task{%
    Предмет в виде отрезка длиной $\ell$ расположен вдоль оптической оси
    собирающей линзы с фокусным расстоянием $F$.
    Середина отрезка расположена
    на расстоянии $a$ от линзы, которая даёт действительное изображение
    всех точек предмета.
    Определить продольное увеличение предмета.
}
\answer{%
    \begin{align*}
    \frac 1{a + \frac \ell 2} &+ \frac 1b = \frac 1F \implies b = \frac{F\cbr{a + \frac \ell 2}}{a + \frac \ell 2 - F} \\
    \frac 1{a - \frac \ell 2} &+ \frac 1c = \frac 1F \implies c = \frac{F\cbr{a - \frac \ell 2}}{a - \frac \ell 2 - F} \\
    \abs{b - c} &= \abs{\frac{F\cbr{a + \frac \ell 2}}{a + \frac \ell 2 - F} - \frac{F\cbr{a - \frac \ell 2}}{a - \frac \ell 2 - F}}= F\abs{\frac{\cbr{a + \frac \ell 2}\cbr{a - \frac \ell 2 - F} - \cbr{a - \frac \ell 2}\cbr{a + \frac \ell 2 - F}}{ \cbr{a + \frac \ell 2 - F} \cbr{a - \frac \ell 2 - F} }} =  \\
    &= F\abs{\frac{a^2 - \frac {a\ell} 2 - Fa + \frac {a\ell} 2 - \frac {\ell^2} 4 - \frac {F\ell}2 - a^2 - \frac {a\ell}2 + aF + \frac {a\ell}2 + \frac {\ell^2} 4 - \frac {F\ell} 2}{\cbr{a + \frac \ell 2 - F} \cbr{a - \frac \ell 2 - F} }} = \\
    &= F\frac{F\ell}{\sqr{a-F} - \frac {\ell^2}4} = \frac{F^2\ell}{\sqr{a-F} - \frac {\ell^2}4}\implies \Gamma = \frac{\abs{b - c}}\ell = \frac{F^2}{\sqr{a-F} - \frac {\ell^2}4}.
    \end{align*}
}
\solutionspace{120pt}

\tasknumber{11}%
\task{%
    На экране с помощью тонкой линзы получено изображение предмета
    с увеличением $4$.
    Предмет передвинули на $8\,\text{см}$.
    Для того, чтобы получить резкое изображение, пришлось передвинуть экран.
    При этом увеличение оказалось равным $8$.
    На какое расстояние
    пришлось передвинуть экран?
}
\solutionspace{120pt}

\tasknumber{12}%
\task{%
    Тонкая собирающая линза дает изображение предмета на экране высотой $H_1$,
    и $H_2$, при двух положениях линзы между предметом и экраном.
    Расстояние между ними неизменно.
    Чему равна высота предмета $h$?
}
\answer{%
    $h = \sqrt{H_1 H_2}$
}
\solutionspace{120pt}

\tasknumber{13}%
\task{%
    Какие предметы можно рассмотреть на фотографии, сделанной со спутника,
    если разрешающая способность пленки $0{,}010\,\text{мм}$? Каким должно быть
    время экспозиции $\tau$ чтобы полностью использовать возможности пленки?
    Фокусное расстояние объектива используемого фотоаппарата $10\,\text{cм}$,
    высота орбиты спутника $120\,\text{км}$.
}
\solutionspace{120pt}

\tasknumber{14}%
\task{%
    При аэрофотосъемках используется фотоаппарат, объектив которого
    имеет фокусиое расстояние $8\,\text{cм}$.
    Разрешающая способность пленки $0{,}02\,\text{мм}$.
    На какой высоте должен лететь самолет, чтобы на фотографии можно
    было различить листья деревьев размером $5\,\text{cм}$?
    При какой скорости самолета изображение не будет размытым,
    если время зкспозиции $2\,\text{мс}$?
}

\variantsplitter

\addpersonalvariant{Артём Жичин}

\tasknumber{1}%
\task{%
    Найти оптическую силу собирающей линзы, если действительное изображение предмета,
    помещённого в $35\,\text{см}$ от линзы, получается на расстоянии $40\,\text{см}$ от неё.
}
\answer{%
    $D = \frac 1F = \frac 1a + \frac 1b = \frac 1{35\,\text{см}} + \frac 1{40\,\text{см}} \approx 5{,}36\,\text{дптр}$
}
\solutionspace{180pt}

\tasknumber{2}%
\task{%
    Найти увеличение изображения, если изображение предмета, находящегося
    на расстоянии $20\,\text{см}$ от линзы, получается на расстоянии $30\,\text{см}$ от неё.
}
\answer{%
    $\Gamma = \frac ba = \frac {30\,\text{см}}{20\,\text{см}} \approx 1{,}50$
}
\solutionspace{180pt}

\tasknumber{3}%
\task{%
    Расстояние от предмета до линзы $10\,\text{см}$, а от линзы до мнимого изображения $30\,\text{см}$.
    Чему равно фокусное расстояние линзы?
}
\answer{%
    $\pm \frac 1F = \frac 1a - \frac 1b \implies F = \frac{a b}{\abs{b - a}} \approx 15\,\text{см}$
}
\solutionspace{180pt}

\tasknumber{4}%
\task{%
    Две тонкие собирающие линзы с фокусными расстояниями $12\,\text{см}$ и $20\,\text{см}$ сложены вместе.
    Чему равно фокусное расстояние такой оптической системы?
}
\answer{%
    $\frac 1{f_1} = \frac 1a + \frac 1b; \frac 1{f_2} = - \frac 1b + \frac 1c \implies \frac 1{f_1} + \frac 1{f_2} = \frac 1a + \frac 1c \implies f' = \frac 1{\frac 1{f_1} + \frac 1{f_2}} = \frac{f_1 f_2}{f_1 + f_2} \approx 7{,}5\,\text{см}$
}
\solutionspace{180pt}

\tasknumber{5}%
\task{%
    Линейные размеры прямого изображения предмета, полученного в собирающей линзе,
    в четыре раза больше линейных размеров предмета.
    Зная, что предмет находится на $20\,\text{см}$ ближе к линзе,
    чем его изображение, найти оптическую силу линзы.
}
\answer{%
    \begin{align*}
    D &= \frac 1F = \frac 1a + \frac 1b, \qquad \Gamma = \frac ba, \qquad b - a = \ell \implies b = \Gamma a \implies \Gamma a - a = \ell \implies  \\
    a &= \frac {\ell}{\Gamma - 1} \implies b = \frac {{\ell} \Gamma}{\Gamma - 1} \implies  \\
    D &= \frac {\Gamma - 1}\ell + \frac {\Gamma - 1}{\ell \Gamma} = \frac 1\ell \cdot \cbr{\Gamma - 1 + \frac {\Gamma - 1}{\Gamma} } =\frac 1\ell \cdot \cbr{\Gamma - \frac 1\Gamma} \approx 18{,}8\,\text{дптр}.
    \end{align*}
}
\solutionspace{180pt}

\tasknumber{6}%
\task{%
    Оптическая сила объектива фотоаппарата равна $6\,\text{дптр}$.
    При фотографировании чертежа с расстояния $0{,}8\,\text{м}$ площадь изображения
    чертежа на фотопластинке оказалась равной $9\,\text{см}^{2}$.
    Какова площадь самого чертежа? Ответ выразите в квадратных сантиметрах.
}
\solutionspace{180pt}

\tasknumber{7}%
\task{%
    В каком месте на главной оптической оси двояковыгнутой линзы
    нужно поместить точечный источник света,
    чтобы его изображение оказалось в главном фокусе линзы?
}
\answer{%
    $\text{на половине фокусного расстояния}$
}
\solutionspace{120pt}

\tasknumber{8}%
\task{%
    Предмет высотой $h = 50\,\text{см}$ находится на расстоянии $d = 0{,}8\,\text{м}$
    от вертикально расположенной рассеивающей линзы с фокусным расстоянием $F = 20\,\text{см}$.
    Где находится изображение предмета? Определите тип изображения и его высоту.
}
\solutionspace{120pt}

\tasknumber{9}%
\task{%
    На каком расстоянии от двояковыпуклой линзы с оптической силой $D = 2{,}5\,\text{дптр}$
    надо поместить предмет, чтобы его изображение получилось на расстоянии $2{,}5\,\text{м}$ от линзы?
}
\solutionspace{120pt}

\tasknumber{10}%
\task{%
    Предмет в виде отрезка длиной $\ell$ расположен вдоль оптической оси
    собирающей линзы с фокусным расстоянием $F$.
    Середина отрезка расположена
    на расстоянии $a$ от линзы, которая даёт действительное изображение
    всех точек предмета.
    Определить продольное увеличение предмета.
}
\answer{%
    \begin{align*}
    \frac 1{a + \frac \ell 2} &+ \frac 1b = \frac 1F \implies b = \frac{F\cbr{a + \frac \ell 2}}{a + \frac \ell 2 - F} \\
    \frac 1{a - \frac \ell 2} &+ \frac 1c = \frac 1F \implies c = \frac{F\cbr{a - \frac \ell 2}}{a - \frac \ell 2 - F} \\
    \abs{b - c} &= \abs{\frac{F\cbr{a + \frac \ell 2}}{a + \frac \ell 2 - F} - \frac{F\cbr{a - \frac \ell 2}}{a - \frac \ell 2 - F}}= F\abs{\frac{\cbr{a + \frac \ell 2}\cbr{a - \frac \ell 2 - F} - \cbr{a - \frac \ell 2}\cbr{a + \frac \ell 2 - F}}{ \cbr{a + \frac \ell 2 - F} \cbr{a - \frac \ell 2 - F} }} =  \\
    &= F\abs{\frac{a^2 - \frac {a\ell} 2 - Fa + \frac {a\ell} 2 - \frac {\ell^2} 4 - \frac {F\ell}2 - a^2 - \frac {a\ell}2 + aF + \frac {a\ell}2 + \frac {\ell^2} 4 - \frac {F\ell} 2}{\cbr{a + \frac \ell 2 - F} \cbr{a - \frac \ell 2 - F} }} = \\
    &= F\frac{F\ell}{\sqr{a-F} - \frac {\ell^2}4} = \frac{F^2\ell}{\sqr{a-F} - \frac {\ell^2}4}\implies \Gamma = \frac{\abs{b - c}}\ell = \frac{F^2}{\sqr{a-F} - \frac {\ell^2}4}.
    \end{align*}
}
\solutionspace{120pt}

\tasknumber{11}%
\task{%
    На экране с помощью тонкой линзы получено изображение предмета
    с увеличением $4$.
    Предмет передвинули на $4\,\text{см}$.
    Для того, чтобы получить резкое изображение, пришлось передвинуть экран.
    При этом увеличение оказалось равным $8$.
    На какое расстояние
    пришлось передвинуть экран?
}
\solutionspace{120pt}

\tasknumber{12}%
\task{%
    Тонкая собирающая линза дает изображение предмета на экране высотой $H_1$,
    и $H_2$, при двух положениях линзы между предметом и экраном.
    Расстояние между ними неизменно.
    Чему равна высота предмета $h$?
}
\answer{%
    $h = \sqrt{H_1 H_2}$
}
\solutionspace{120pt}

\tasknumber{13}%
\task{%
    Какие предметы можно рассмотреть на фотографии, сделанной со спутника,
    если разрешающая способность пленки $0{,}010\,\text{мм}$? Каким должно быть
    время экспозиции $\tau$ чтобы полностью использовать возможности пленки?
    Фокусное расстояние объектива используемого фотоаппарата $10\,\text{cм}$,
    высота орбиты спутника $80\,\text{км}$.
}
\solutionspace{120pt}

\tasknumber{14}%
\task{%
    При аэрофотосъемках используется фотоаппарат, объектив которого
    имеет фокусиое расстояние $8\,\text{cм}$.
    Разрешающая способность пленки $0{,}02\,\text{мм}$.
    На какой высоте должен лететь самолет, чтобы на фотографии можно
    было различить листья деревьев размером $5\,\text{cм}$?
    При какой скорости самолета изображение не будет размытым,
    если время зкспозиции $1\,\text{мс}$?
}

\variantsplitter

\addpersonalvariant{Дарья Кошман}

\tasknumber{1}%
\task{%
    Найти оптическую силу собирающей линзы, если действительное изображение предмета,
    помещённого в $55\,\text{см}$ от линзы, получается на расстоянии $30\,\text{см}$ от неё.
}
\answer{%
    $D = \frac 1F = \frac 1a + \frac 1b = \frac 1{55\,\text{см}} + \frac 1{30\,\text{см}} \approx 5{,}15\,\text{дптр}$
}
\solutionspace{180pt}

\tasknumber{2}%
\task{%
    Найти увеличение изображения, если изображение предмета, находящегося
    на расстоянии $20\,\text{см}$ от линзы, получается на расстоянии $18\,\text{см}$ от неё.
}
\answer{%
    $\Gamma = \frac ba = \frac {18\,\text{см}}{20\,\text{см}} \approx 0{,}9$
}
\solutionspace{180pt}

\tasknumber{3}%
\task{%
    Расстояние от предмета до линзы $10\,\text{см}$, а от линзы до мнимого изображения $25\,\text{см}$.
    Чему равно фокусное расстояние линзы?
}
\answer{%
    $\pm \frac 1F = \frac 1a - \frac 1b \implies F = \frac{a b}{\abs{b - a}} \approx 16{,}7\,\text{см}$
}
\solutionspace{180pt}

\tasknumber{4}%
\task{%
    Две тонкие собирающие линзы с фокусными расстояниями $18\,\text{см}$ и $20\,\text{см}$ сложены вместе.
    Чему равно фокусное расстояние такой оптической системы?
}
\answer{%
    $\frac 1{f_1} = \frac 1a + \frac 1b; \frac 1{f_2} = - \frac 1b + \frac 1c \implies \frac 1{f_1} + \frac 1{f_2} = \frac 1a + \frac 1c \implies f' = \frac 1{\frac 1{f_1} + \frac 1{f_2}} = \frac{f_1 f_2}{f_1 + f_2} \approx 9{,}5\,\text{см}$
}
\solutionspace{180pt}

\tasknumber{5}%
\task{%
    Линейные размеры прямого изображения предмета, полученного в собирающей линзе,
    в три раза больше линейных размеров предмета.
    Зная, что предмет находится на $25\,\text{см}$ ближе к линзе,
    чем его изображение, найти оптическую силу линзы.
}
\answer{%
    \begin{align*}
    D &= \frac 1F = \frac 1a + \frac 1b, \qquad \Gamma = \frac ba, \qquad b - a = \ell \implies b = \Gamma a \implies \Gamma a - a = \ell \implies  \\
    a &= \frac {\ell}{\Gamma - 1} \implies b = \frac {{\ell} \Gamma}{\Gamma - 1} \implies  \\
    D &= \frac {\Gamma - 1}\ell + \frac {\Gamma - 1}{\ell \Gamma} = \frac 1\ell \cdot \cbr{\Gamma - 1 + \frac {\Gamma - 1}{\Gamma} } =\frac 1\ell \cdot \cbr{\Gamma - \frac 1\Gamma} \approx 10{,}7\,\text{дптр}.
    \end{align*}
}
\solutionspace{180pt}

\tasknumber{6}%
\task{%
    Оптическая сила объектива фотоаппарата равна $5\,\text{дптр}$.
    При фотографировании чертежа с расстояния $1{,}1\,\text{м}$ площадь изображения
    чертежа на фотопластинке оказалась равной $9\,\text{см}^{2}$.
    Какова площадь самого чертежа? Ответ выразите в квадратных сантиметрах.
}
\solutionspace{180pt}

\tasknumber{7}%
\task{%
    В каком месте на главной оптической оси двояковыпуклой линзы
    нужно поместить точечный источник света,
    чтобы его изображение оказалось в главном фокусе линзы?
}
\answer{%
    $\text{для мнимого - на половине фокусного, для действительного - на бесконечности}$
}
\solutionspace{120pt}

\tasknumber{8}%
\task{%
    Предмет высотой $h = 30\,\text{см}$ находится на расстоянии $d = 0{,}8\,\text{м}$
    от вертикально расположенной рассеивающей линзы с фокусным расстоянием $F = 25\,\text{см}$.
    Где находится изображение предмета? Определите тип изображения и его высоту.
}
\solutionspace{120pt}

\tasknumber{9}%
\task{%
    На каком расстоянии от двояковыпуклой линзы с оптической силой $D = 1{,}5\,\text{дптр}$
    надо поместить предмет, чтобы его изображение получилось на расстоянии $1{,}5\,\text{м}$ от линзы?
}
\solutionspace{120pt}

\tasknumber{10}%
\task{%
    Предмет в виде отрезка длиной $\ell$ расположен вдоль оптической оси
    собирающей линзы с фокусным расстоянием $F$.
    Середина отрезка расположена
    на расстоянии $a$ от линзы, которая даёт действительное изображение
    всех точек предмета.
    Определить продольное увеличение предмета.
}
\answer{%
    \begin{align*}
    \frac 1{a + \frac \ell 2} &+ \frac 1b = \frac 1F \implies b = \frac{F\cbr{a + \frac \ell 2}}{a + \frac \ell 2 - F} \\
    \frac 1{a - \frac \ell 2} &+ \frac 1c = \frac 1F \implies c = \frac{F\cbr{a - \frac \ell 2}}{a - \frac \ell 2 - F} \\
    \abs{b - c} &= \abs{\frac{F\cbr{a + \frac \ell 2}}{a + \frac \ell 2 - F} - \frac{F\cbr{a - \frac \ell 2}}{a - \frac \ell 2 - F}}= F\abs{\frac{\cbr{a + \frac \ell 2}\cbr{a - \frac \ell 2 - F} - \cbr{a - \frac \ell 2}\cbr{a + \frac \ell 2 - F}}{ \cbr{a + \frac \ell 2 - F} \cbr{a - \frac \ell 2 - F} }} =  \\
    &= F\abs{\frac{a^2 - \frac {a\ell} 2 - Fa + \frac {a\ell} 2 - \frac {\ell^2} 4 - \frac {F\ell}2 - a^2 - \frac {a\ell}2 + aF + \frac {a\ell}2 + \frac {\ell^2} 4 - \frac {F\ell} 2}{\cbr{a + \frac \ell 2 - F} \cbr{a - \frac \ell 2 - F} }} = \\
    &= F\frac{F\ell}{\sqr{a-F} - \frac {\ell^2}4} = \frac{F^2\ell}{\sqr{a-F} - \frac {\ell^2}4}\implies \Gamma = \frac{\abs{b - c}}\ell = \frac{F^2}{\sqr{a-F} - \frac {\ell^2}4}.
    \end{align*}
}
\solutionspace{120pt}

\tasknumber{11}%
\task{%
    На экране с помощью тонкой линзы получено изображение предмета
    с увеличением $4$.
    Предмет передвинули на $4\,\text{см}$.
    Для того, чтобы получить резкое изображение, пришлось передвинуть экран.
    При этом увеличение оказалось равным $8$.
    На какое расстояние
    пришлось передвинуть экран?
}
\solutionspace{120pt}

\tasknumber{12}%
\task{%
    Тонкая собирающая линза дает изображение предмета на экране высотой $H_1$,
    и $H_2$, при двух положениях линзы между предметом и экраном.
    Расстояние между ними неизменно.
    Чему равна высота предмета $h$?
}
\answer{%
    $h = \sqrt{H_1 H_2}$
}
\solutionspace{120pt}

\tasknumber{13}%
\task{%
    Какие предметы можно рассмотреть на фотографии, сделанной со спутника,
    если разрешающая способность пленки $0{,}02\,\text{мм}$? Каким должно быть
    время экспозиции $\tau$ чтобы полностью использовать возможности пленки?
    Фокусное расстояние объектива используемого фотоаппарата $10\,\text{cм}$,
    высота орбиты спутника $100\,\text{км}$.
}
\solutionspace{120pt}

\tasknumber{14}%
\task{%
    При аэрофотосъемках используется фотоаппарат, объектив которого
    имеет фокусиое расстояние $8\,\text{cм}$.
    Разрешающая способность пленки $0{,}015\,\text{мм}$.
    На какой высоте должен лететь самолет, чтобы на фотографии можно
    было различить листья деревьев размером $4\,\text{cм}$?
    При какой скорости самолета изображение не будет размытым,
    если время зкспозиции $1\,\text{мс}$?
}

\variantsplitter

\addpersonalvariant{Анна Кузьмичёва}

\tasknumber{1}%
\task{%
    Найти оптическую силу собирающей линзы, если действительное изображение предмета,
    помещённого в $35\,\text{см}$ от линзы, получается на расстоянии $40\,\text{см}$ от неё.
}
\answer{%
    $D = \frac 1F = \frac 1a + \frac 1b = \frac 1{35\,\text{см}} + \frac 1{40\,\text{см}} \approx 5{,}36\,\text{дптр}$
}
\solutionspace{180pt}

\tasknumber{2}%
\task{%
    Найти увеличение изображения, если изображение предмета, находящегося
    на расстоянии $20\,\text{см}$ от линзы, получается на расстоянии $18\,\text{см}$ от неё.
}
\answer{%
    $\Gamma = \frac ba = \frac {18\,\text{см}}{20\,\text{см}} \approx 0{,}9$
}
\solutionspace{180pt}

\tasknumber{3}%
\task{%
    Расстояние от предмета до линзы $10\,\text{см}$, а от линзы до мнимого изображения $25\,\text{см}$.
    Чему равно фокусное расстояние линзы?
}
\answer{%
    $\pm \frac 1F = \frac 1a - \frac 1b \implies F = \frac{a b}{\abs{b - a}} \approx 16{,}7\,\text{см}$
}
\solutionspace{180pt}

\tasknumber{4}%
\task{%
    Две тонкие собирающие линзы с фокусными расстояниями $25\,\text{см}$ и $30\,\text{см}$ сложены вместе.
    Чему равно фокусное расстояние такой оптической системы?
}
\answer{%
    $\frac 1{f_1} = \frac 1a + \frac 1b; \frac 1{f_2} = - \frac 1b + \frac 1c \implies \frac 1{f_1} + \frac 1{f_2} = \frac 1a + \frac 1c \implies f' = \frac 1{\frac 1{f_1} + \frac 1{f_2}} = \frac{f_1 f_2}{f_1 + f_2} \approx 13{,}6\,\text{см}$
}
\solutionspace{180pt}

\tasknumber{5}%
\task{%
    Линейные размеры прямого изображения предмета, полученного в собирающей линзе,
    в два раза больше линейных размеров предмета.
    Зная, что предмет находится на $30\,\text{см}$ ближе к линзе,
    чем его изображение, найти оптическую силу линзы.
}
\answer{%
    \begin{align*}
    D &= \frac 1F = \frac 1a + \frac 1b, \qquad \Gamma = \frac ba, \qquad b - a = \ell \implies b = \Gamma a \implies \Gamma a - a = \ell \implies  \\
    a &= \frac {\ell}{\Gamma - 1} \implies b = \frac {{\ell} \Gamma}{\Gamma - 1} \implies  \\
    D &= \frac {\Gamma - 1}\ell + \frac {\Gamma - 1}{\ell \Gamma} = \frac 1\ell \cdot \cbr{\Gamma - 1 + \frac {\Gamma - 1}{\Gamma} } =\frac 1\ell \cdot \cbr{\Gamma - \frac 1\Gamma} \approx 5\,\text{дптр}.
    \end{align*}
}
\solutionspace{180pt}

\tasknumber{6}%
\task{%
    Оптическая сила объектива фотоаппарата равна $3\,\text{дптр}$.
    При фотографировании чертежа с расстояния $1{,}2\,\text{м}$ площадь изображения
    чертежа на фотопластинке оказалась равной $16\,\text{см}^{2}$.
    Какова площадь самого чертежа? Ответ выразите в квадратных сантиметрах.
}
\solutionspace{180pt}

\tasknumber{7}%
\task{%
    В каком месте на главной оптической оси двояковыпуклой линзы
    нужно поместить точечный источник света,
    чтобы его изображение оказалось в главном фокусе линзы?
}
\answer{%
    $\text{для мнимого - на половине фокусного, для действительного - на бесконечности}$
}
\solutionspace{120pt}

\tasknumber{8}%
\task{%
    Предмет высотой $h = 40\,\text{см}$ находится на расстоянии $d = 1\,\text{м}$
    от вертикально расположенной рассеивающей линзы с фокусным расстоянием $F = 20\,\text{см}$.
    Где находится изображение предмета? Определите тип изображения и его высоту.
}
\solutionspace{120pt}

\tasknumber{9}%
\task{%
    На каком расстоянии от двояковыпуклой линзы с оптической силой $D = 2\,\text{дптр}$
    надо поместить предмет, чтобы его изображение получилось на расстоянии $2\,\text{м}$ от линзы?
}
\solutionspace{120pt}

\tasknumber{10}%
\task{%
    Предмет в виде отрезка длиной $\ell$ расположен вдоль оптической оси
    собирающей линзы с фокусным расстоянием $F$.
    Середина отрезка расположена
    на расстоянии $a$ от линзы, которая даёт действительное изображение
    всех точек предмета.
    Определить продольное увеличение предмета.
}
\answer{%
    \begin{align*}
    \frac 1{a + \frac \ell 2} &+ \frac 1b = \frac 1F \implies b = \frac{F\cbr{a + \frac \ell 2}}{a + \frac \ell 2 - F} \\
    \frac 1{a - \frac \ell 2} &+ \frac 1c = \frac 1F \implies c = \frac{F\cbr{a - \frac \ell 2}}{a - \frac \ell 2 - F} \\
    \abs{b - c} &= \abs{\frac{F\cbr{a + \frac \ell 2}}{a + \frac \ell 2 - F} - \frac{F\cbr{a - \frac \ell 2}}{a - \frac \ell 2 - F}}= F\abs{\frac{\cbr{a + \frac \ell 2}\cbr{a - \frac \ell 2 - F} - \cbr{a - \frac \ell 2}\cbr{a + \frac \ell 2 - F}}{ \cbr{a + \frac \ell 2 - F} \cbr{a - \frac \ell 2 - F} }} =  \\
    &= F\abs{\frac{a^2 - \frac {a\ell} 2 - Fa + \frac {a\ell} 2 - \frac {\ell^2} 4 - \frac {F\ell}2 - a^2 - \frac {a\ell}2 + aF + \frac {a\ell}2 + \frac {\ell^2} 4 - \frac {F\ell} 2}{\cbr{a + \frac \ell 2 - F} \cbr{a - \frac \ell 2 - F} }} = \\
    &= F\frac{F\ell}{\sqr{a-F} - \frac {\ell^2}4} = \frac{F^2\ell}{\sqr{a-F} - \frac {\ell^2}4}\implies \Gamma = \frac{\abs{b - c}}\ell = \frac{F^2}{\sqr{a-F} - \frac {\ell^2}4}.
    \end{align*}
}
\solutionspace{120pt}

\tasknumber{11}%
\task{%
    На экране с помощью тонкой линзы получено изображение предмета
    с увеличением $2$.
    Предмет передвинули на $8\,\text{см}$.
    Для того, чтобы получить резкое изображение, пришлось передвинуть экран.
    При этом увеличение оказалось равным $8$.
    На какое расстояние
    пришлось передвинуть экран?
}
\solutionspace{120pt}

\tasknumber{12}%
\task{%
    Тонкая собирающая линза дает изображение предмета на экране высотой $H_1$,
    и $H_2$, при двух положениях линзы между предметом и экраном.
    Расстояние между ними неизменно.
    Чему равна высота предмета $h$?
}
\answer{%
    $h = \sqrt{H_1 H_2}$
}
\solutionspace{120pt}

\tasknumber{13}%
\task{%
    Какие предметы можно рассмотреть на фотографии, сделанной со спутника,
    если разрешающая способность пленки $0{,}02\,\text{мм}$? Каким должно быть
    время экспозиции $\tau$ чтобы полностью использовать возможности пленки?
    Фокусное расстояние объектива используемого фотоаппарата $20\,\text{cм}$,
    высота орбиты спутника $150\,\text{км}$.
}
\solutionspace{120pt}

\tasknumber{14}%
\task{%
    При аэрофотосъемках используется фотоаппарат, объектив которого
    имеет фокусиое расстояние $10\,\text{cм}$.
    Разрешающая способность пленки $0{,}015\,\text{мм}$.
    На какой высоте должен лететь самолет, чтобы на фотографии можно
    было различить листья деревьев размером $4\,\text{cм}$?
    При какой скорости самолета изображение не будет размытым,
    если время зкспозиции $1\,\text{мс}$?
}

\variantsplitter

\addpersonalvariant{Алёна Куприянова}

\tasknumber{1}%
\task{%
    Найти оптическую силу собирающей линзы, если действительное изображение предмета,
    помещённого в $55\,\text{см}$ от линзы, получается на расстоянии $30\,\text{см}$ от неё.
}
\answer{%
    $D = \frac 1F = \frac 1a + \frac 1b = \frac 1{55\,\text{см}} + \frac 1{30\,\text{см}} \approx 5{,}15\,\text{дптр}$
}
\solutionspace{180pt}

\tasknumber{2}%
\task{%
    Найти увеличение изображения, если изображение предмета, находящегося
    на расстоянии $20\,\text{см}$ от линзы, получается на расстоянии $30\,\text{см}$ от неё.
}
\answer{%
    $\Gamma = \frac ba = \frac {30\,\text{см}}{20\,\text{см}} \approx 1{,}50$
}
\solutionspace{180pt}

\tasknumber{3}%
\task{%
    Расстояние от предмета до линзы $10\,\text{см}$, а от линзы до мнимого изображения $30\,\text{см}$.
    Чему равно фокусное расстояние линзы?
}
\answer{%
    $\pm \frac 1F = \frac 1a - \frac 1b \implies F = \frac{a b}{\abs{b - a}} \approx 15\,\text{см}$
}
\solutionspace{180pt}

\tasknumber{4}%
\task{%
    Две тонкие собирающие линзы с фокусными расстояниями $18\,\text{см}$ и $20\,\text{см}$ сложены вместе.
    Чему равно фокусное расстояние такой оптической системы?
}
\answer{%
    $\frac 1{f_1} = \frac 1a + \frac 1b; \frac 1{f_2} = - \frac 1b + \frac 1c \implies \frac 1{f_1} + \frac 1{f_2} = \frac 1a + \frac 1c \implies f' = \frac 1{\frac 1{f_1} + \frac 1{f_2}} = \frac{f_1 f_2}{f_1 + f_2} \approx 9{,}5\,\text{см}$
}
\solutionspace{180pt}

\tasknumber{5}%
\task{%
    Линейные размеры прямого изображения предмета, полученного в собирающей линзе,
    в два раза больше линейных размеров предмета.
    Зная, что предмет находится на $30\,\text{см}$ ближе к линзе,
    чем его изображение, найти оптическую силу линзы.
}
\answer{%
    \begin{align*}
    D &= \frac 1F = \frac 1a + \frac 1b, \qquad \Gamma = \frac ba, \qquad b - a = \ell \implies b = \Gamma a \implies \Gamma a - a = \ell \implies  \\
    a &= \frac {\ell}{\Gamma - 1} \implies b = \frac {{\ell} \Gamma}{\Gamma - 1} \implies  \\
    D &= \frac {\Gamma - 1}\ell + \frac {\Gamma - 1}{\ell \Gamma} = \frac 1\ell \cdot \cbr{\Gamma - 1 + \frac {\Gamma - 1}{\Gamma} } =\frac 1\ell \cdot \cbr{\Gamma - \frac 1\Gamma} \approx 5\,\text{дптр}.
    \end{align*}
}
\solutionspace{180pt}

\tasknumber{6}%
\task{%
    Оптическая сила объектива фотоаппарата равна $5\,\text{дптр}$.
    При фотографировании чертежа с расстояния $0{,}8\,\text{м}$ площадь изображения
    чертежа на фотопластинке оказалась равной $9\,\text{см}^{2}$.
    Какова площадь самого чертежа? Ответ выразите в квадратных сантиметрах.
}
\solutionspace{180pt}

\tasknumber{7}%
\task{%
    В каком месте на главной оптической оси двояковыпуклой линзы
    нужно поместить точечный источник света,
    чтобы его изображение оказалось в главном фокусе линзы?
}
\answer{%
    $\text{для мнимого - на половине фокусного, для действительного - на бесконечности}$
}
\solutionspace{120pt}

\tasknumber{8}%
\task{%
    Предмет высотой $h = 30\,\text{см}$ находится на расстоянии $d = 0{,}8\,\text{м}$
    от вертикально расположенной рассеивающей линзы с фокусным расстоянием $F = 20\,\text{см}$.
    Где находится изображение предмета? Определите тип изображения и его высоту.
}
\solutionspace{120pt}

\tasknumber{9}%
\task{%
    На каком расстоянии от двояковыпуклой линзы с оптической силой $D = 2{,}5\,\text{дптр}$
    надо поместить предмет, чтобы его изображение получилось на расстоянии $2\,\text{м}$ от линзы?
}
\solutionspace{120pt}

\tasknumber{10}%
\task{%
    Предмет в виде отрезка длиной $\ell$ расположен вдоль оптической оси
    собирающей линзы с фокусным расстоянием $F$.
    Середина отрезка расположена
    на расстоянии $a$ от линзы, которая даёт действительное изображение
    всех точек предмета.
    Определить продольное увеличение предмета.
}
\answer{%
    \begin{align*}
    \frac 1{a + \frac \ell 2} &+ \frac 1b = \frac 1F \implies b = \frac{F\cbr{a + \frac \ell 2}}{a + \frac \ell 2 - F} \\
    \frac 1{a - \frac \ell 2} &+ \frac 1c = \frac 1F \implies c = \frac{F\cbr{a - \frac \ell 2}}{a - \frac \ell 2 - F} \\
    \abs{b - c} &= \abs{\frac{F\cbr{a + \frac \ell 2}}{a + \frac \ell 2 - F} - \frac{F\cbr{a - \frac \ell 2}}{a - \frac \ell 2 - F}}= F\abs{\frac{\cbr{a + \frac \ell 2}\cbr{a - \frac \ell 2 - F} - \cbr{a - \frac \ell 2}\cbr{a + \frac \ell 2 - F}}{ \cbr{a + \frac \ell 2 - F} \cbr{a - \frac \ell 2 - F} }} =  \\
    &= F\abs{\frac{a^2 - \frac {a\ell} 2 - Fa + \frac {a\ell} 2 - \frac {\ell^2} 4 - \frac {F\ell}2 - a^2 - \frac {a\ell}2 + aF + \frac {a\ell}2 + \frac {\ell^2} 4 - \frac {F\ell} 2}{\cbr{a + \frac \ell 2 - F} \cbr{a - \frac \ell 2 - F} }} = \\
    &= F\frac{F\ell}{\sqr{a-F} - \frac {\ell^2}4} = \frac{F^2\ell}{\sqr{a-F} - \frac {\ell^2}4}\implies \Gamma = \frac{\abs{b - c}}\ell = \frac{F^2}{\sqr{a-F} - \frac {\ell^2}4}.
    \end{align*}
}
\solutionspace{120pt}

\tasknumber{11}%
\task{%
    На экране с помощью тонкой линзы получено изображение предмета
    с увеличением $4$.
    Предмет передвинули на $6\,\text{см}$.
    Для того, чтобы получить резкое изображение, пришлось передвинуть экран.
    При этом увеличение оказалось равным $8$.
    На какое расстояние
    пришлось передвинуть экран?
}
\solutionspace{120pt}

\tasknumber{12}%
\task{%
    Тонкая собирающая линза дает изображение предмета на экране высотой $H_1$,
    и $H_2$, при двух положениях линзы между предметом и экраном.
    Расстояние между ними неизменно.
    Чему равна высота предмета $h$?
}
\answer{%
    $h = \sqrt{H_1 H_2}$
}
\solutionspace{120pt}

\tasknumber{13}%
\task{%
    Какие предметы можно рассмотреть на фотографии, сделанной со спутника,
    если разрешающая способность пленки $0{,}02\,\text{мм}$? Каким должно быть
    время экспозиции $\tau$ чтобы полностью использовать возможности пленки?
    Фокусное расстояние объектива используемого фотоаппарата $15\,\text{cм}$,
    высота орбиты спутника $150\,\text{км}$.
}
\solutionspace{120pt}

\tasknumber{14}%
\task{%
    При аэрофотосъемках используется фотоаппарат, объектив которого
    имеет фокусиое расстояние $8\,\text{cм}$.
    Разрешающая способность пленки $0{,}010\,\text{мм}$.
    На какой высоте должен лететь самолет, чтобы на фотографии можно
    было различить листья деревьев размером $5\,\text{cм}$?
    При какой скорости самолета изображение не будет размытым,
    если время зкспозиции $2\,\text{мс}$?
}

\variantsplitter

\addpersonalvariant{Ярослав Лавровский}

\tasknumber{1}%
\task{%
    Найти оптическую силу собирающей линзы, если действительное изображение предмета,
    помещённого в $15\,\text{см}$ от линзы, получается на расстоянии $30\,\text{см}$ от неё.
}
\answer{%
    $D = \frac 1F = \frac 1a + \frac 1b = \frac 1{15\,\text{см}} + \frac 1{30\,\text{см}} \approx 10\,\text{дптр}$
}
\solutionspace{180pt}

\tasknumber{2}%
\task{%
    Найти увеличение изображения, если изображение предмета, находящегося
    на расстоянии $20\,\text{см}$ от линзы, получается на расстоянии $30\,\text{см}$ от неё.
}
\answer{%
    $\Gamma = \frac ba = \frac {30\,\text{см}}{20\,\text{см}} \approx 1{,}50$
}
\solutionspace{180pt}

\tasknumber{3}%
\task{%
    Расстояние от предмета до линзы $10\,\text{см}$, а от линзы до мнимого изображения $30\,\text{см}$.
    Чему равно фокусное расстояние линзы?
}
\answer{%
    $\pm \frac 1F = \frac 1a - \frac 1b \implies F = \frac{a b}{\abs{b - a}} \approx 15\,\text{см}$
}
\solutionspace{180pt}

\tasknumber{4}%
\task{%
    Две тонкие собирающие линзы с фокусными расстояниями $25\,\text{см}$ и $20\,\text{см}$ сложены вместе.
    Чему равно фокусное расстояние такой оптической системы?
}
\answer{%
    $\frac 1{f_1} = \frac 1a + \frac 1b; \frac 1{f_2} = - \frac 1b + \frac 1c \implies \frac 1{f_1} + \frac 1{f_2} = \frac 1a + \frac 1c \implies f' = \frac 1{\frac 1{f_1} + \frac 1{f_2}} = \frac{f_1 f_2}{f_1 + f_2} \approx 11{,}1\,\text{см}$
}
\solutionspace{180pt}

\tasknumber{5}%
\task{%
    Линейные размеры прямого изображения предмета, полученного в собирающей линзе,
    в два раза больше линейных размеров предмета.
    Зная, что предмет находится на $30\,\text{см}$ ближе к линзе,
    чем его изображение, найти оптическую силу линзы.
}
\answer{%
    \begin{align*}
    D &= \frac 1F = \frac 1a + \frac 1b, \qquad \Gamma = \frac ba, \qquad b - a = \ell \implies b = \Gamma a \implies \Gamma a - a = \ell \implies  \\
    a &= \frac {\ell}{\Gamma - 1} \implies b = \frac {{\ell} \Gamma}{\Gamma - 1} \implies  \\
    D &= \frac {\Gamma - 1}\ell + \frac {\Gamma - 1}{\ell \Gamma} = \frac 1\ell \cdot \cbr{\Gamma - 1 + \frac {\Gamma - 1}{\Gamma} } =\frac 1\ell \cdot \cbr{\Gamma - \frac 1\Gamma} \approx 5\,\text{дптр}.
    \end{align*}
}
\solutionspace{180pt}

\tasknumber{6}%
\task{%
    Оптическая сила объектива фотоаппарата равна $4\,\text{дптр}$.
    При фотографировании чертежа с расстояния $1{,}1\,\text{м}$ площадь изображения
    чертежа на фотопластинке оказалась равной $4\,\text{см}^{2}$.
    Какова площадь самого чертежа? Ответ выразите в квадратных сантиметрах.
}
\solutionspace{180pt}

\tasknumber{7}%
\task{%
    В каком месте на главной оптической оси двояковыгнутой линзы
    нужно поместить точечный источник света,
    чтобы его изображение оказалось в главном фокусе линзы?
}
\answer{%
    $\text{на половине фокусного расстояния}$
}
\solutionspace{120pt}

\tasknumber{8}%
\task{%
    Предмет высотой $h = 40\,\text{см}$ находится на расстоянии $d = 0{,}8\,\text{м}$
    от вертикально расположенной рассеивающей линзы с фокусным расстоянием $F = 25\,\text{см}$.
    Где находится изображение предмета? Определите тип изображения и его высоту.
}
\solutionspace{120pt}

\tasknumber{9}%
\task{%
    На каком расстоянии от двояковыпуклой линзы с оптической силой $D = 1{,}5\,\text{дптр}$
    надо поместить предмет, чтобы его изображение получилось на расстоянии $2\,\text{м}$ от линзы?
}
\solutionspace{120pt}

\tasknumber{10}%
\task{%
    Предмет в виде отрезка длиной $\ell$ расположен вдоль оптической оси
    собирающей линзы с фокусным расстоянием $F$.
    Середина отрезка расположена
    на расстоянии $a$ от линзы, которая даёт действительное изображение
    всех точек предмета.
    Определить продольное увеличение предмета.
}
\answer{%
    \begin{align*}
    \frac 1{a + \frac \ell 2} &+ \frac 1b = \frac 1F \implies b = \frac{F\cbr{a + \frac \ell 2}}{a + \frac \ell 2 - F} \\
    \frac 1{a - \frac \ell 2} &+ \frac 1c = \frac 1F \implies c = \frac{F\cbr{a - \frac \ell 2}}{a - \frac \ell 2 - F} \\
    \abs{b - c} &= \abs{\frac{F\cbr{a + \frac \ell 2}}{a + \frac \ell 2 - F} - \frac{F\cbr{a - \frac \ell 2}}{a - \frac \ell 2 - F}}= F\abs{\frac{\cbr{a + \frac \ell 2}\cbr{a - \frac \ell 2 - F} - \cbr{a - \frac \ell 2}\cbr{a + \frac \ell 2 - F}}{ \cbr{a + \frac \ell 2 - F} \cbr{a - \frac \ell 2 - F} }} =  \\
    &= F\abs{\frac{a^2 - \frac {a\ell} 2 - Fa + \frac {a\ell} 2 - \frac {\ell^2} 4 - \frac {F\ell}2 - a^2 - \frac {a\ell}2 + aF + \frac {a\ell}2 + \frac {\ell^2} 4 - \frac {F\ell} 2}{\cbr{a + \frac \ell 2 - F} \cbr{a - \frac \ell 2 - F} }} = \\
    &= F\frac{F\ell}{\sqr{a-F} - \frac {\ell^2}4} = \frac{F^2\ell}{\sqr{a-F} - \frac {\ell^2}4}\implies \Gamma = \frac{\abs{b - c}}\ell = \frac{F^2}{\sqr{a-F} - \frac {\ell^2}4}.
    \end{align*}
}
\solutionspace{120pt}

\tasknumber{11}%
\task{%
    На экране с помощью тонкой линзы получено изображение предмета
    с увеличением $4$.
    Предмет передвинули на $6\,\text{см}$.
    Для того, чтобы получить резкое изображение, пришлось передвинуть экран.
    При этом увеличение оказалось равным $6$.
    На какое расстояние
    пришлось передвинуть экран?
}
\solutionspace{120pt}

\tasknumber{12}%
\task{%
    Тонкая собирающая линза дает изображение предмета на экране высотой $H_1$,
    и $H_2$, при двух положениях линзы между предметом и экраном.
    Расстояние между ними неизменно.
    Чему равна высота предмета $h$?
}
\answer{%
    $h = \sqrt{H_1 H_2}$
}
\solutionspace{120pt}

\tasknumber{13}%
\task{%
    Какие предметы можно рассмотреть на фотографии, сделанной со спутника,
    если разрешающая способность пленки $0{,}010\,\text{мм}$? Каким должно быть
    время экспозиции $\tau$ чтобы полностью использовать возможности пленки?
    Фокусное расстояние объектива используемого фотоаппарата $20\,\text{cм}$,
    высота орбиты спутника $120\,\text{км}$.
}
\solutionspace{120pt}

\tasknumber{14}%
\task{%
    При аэрофотосъемках используется фотоаппарат, объектив которого
    имеет фокусиое расстояние $10\,\text{cм}$.
    Разрешающая способность пленки $0{,}010\,\text{мм}$.
    На какой высоте должен лететь самолет, чтобы на фотографии можно
    было различить листья деревьев размером $6\,\text{cм}$?
    При какой скорости самолета изображение не будет размытым,
    если время зкспозиции $2\,\text{мс}$?
}

\variantsplitter

\addpersonalvariant{Анастасия Ламанова}

\tasknumber{1}%
\task{%
    Найти оптическую силу собирающей линзы, если действительное изображение предмета,
    помещённого в $15\,\text{см}$ от линзы, получается на расстоянии $20\,\text{см}$ от неё.
}
\answer{%
    $D = \frac 1F = \frac 1a + \frac 1b = \frac 1{15\,\text{см}} + \frac 1{20\,\text{см}} \approx 11{,}67\,\text{дптр}$
}
\solutionspace{180pt}

\tasknumber{2}%
\task{%
    Найти увеличение изображения, если изображение предмета, находящегося
    на расстоянии $15\,\text{см}$ от линзы, получается на расстоянии $12\,\text{см}$ от неё.
}
\answer{%
    $\Gamma = \frac ba = \frac {12\,\text{см}}{15\,\text{см}} \approx 0{,}8$
}
\solutionspace{180pt}

\tasknumber{3}%
\task{%
    Расстояние от предмета до линзы $8\,\text{см}$, а от линзы до мнимого изображения $20\,\text{см}$.
    Чему равно фокусное расстояние линзы?
}
\answer{%
    $\pm \frac 1F = \frac 1a - \frac 1b \implies F = \frac{a b}{\abs{b - a}} \approx 13{,}3\,\text{см}$
}
\solutionspace{180pt}

\tasknumber{4}%
\task{%
    Две тонкие собирающие линзы с фокусными расстояниями $18\,\text{см}$ и $20\,\text{см}$ сложены вместе.
    Чему равно фокусное расстояние такой оптической системы?
}
\answer{%
    $\frac 1{f_1} = \frac 1a + \frac 1b; \frac 1{f_2} = - \frac 1b + \frac 1c \implies \frac 1{f_1} + \frac 1{f_2} = \frac 1a + \frac 1c \implies f' = \frac 1{\frac 1{f_1} + \frac 1{f_2}} = \frac{f_1 f_2}{f_1 + f_2} \approx 9{,}5\,\text{см}$
}
\solutionspace{180pt}

\tasknumber{5}%
\task{%
    Линейные размеры прямого изображения предмета, полученного в собирающей линзе,
    в три раза больше линейных размеров предмета.
    Зная, что предмет находится на $35\,\text{см}$ ближе к линзе,
    чем его изображение, найти оптическую силу линзы.
}
\answer{%
    \begin{align*}
    D &= \frac 1F = \frac 1a + \frac 1b, \qquad \Gamma = \frac ba, \qquad b - a = \ell \implies b = \Gamma a \implies \Gamma a - a = \ell \implies  \\
    a &= \frac {\ell}{\Gamma - 1} \implies b = \frac {{\ell} \Gamma}{\Gamma - 1} \implies  \\
    D &= \frac {\Gamma - 1}\ell + \frac {\Gamma - 1}{\ell \Gamma} = \frac 1\ell \cdot \cbr{\Gamma - 1 + \frac {\Gamma - 1}{\Gamma} } =\frac 1\ell \cdot \cbr{\Gamma - \frac 1\Gamma} \approx 7{,}6\,\text{дптр}.
    \end{align*}
}
\solutionspace{180pt}

\tasknumber{6}%
\task{%
    Оптическая сила объектива фотоаппарата равна $4\,\text{дптр}$.
    При фотографировании чертежа с расстояния $1{,}1\,\text{м}$ площадь изображения
    чертежа на фотопластинке оказалась равной $9\,\text{см}^{2}$.
    Какова площадь самого чертежа? Ответ выразите в квадратных сантиметрах.
}
\solutionspace{180pt}

\tasknumber{7}%
\task{%
    В каком месте на главной оптической оси двояковыпуклой линзы
    нужно поместить точечный источник света,
    чтобы его изображение оказалось в главном фокусе линзы?
}
\answer{%
    $\text{для мнимого - на половине фокусного, для действительного - на бесконечности}$
}
\solutionspace{120pt}

\tasknumber{8}%
\task{%
    Предмет высотой $h = 50\,\text{см}$ находится на расстоянии $d = 0{,}8\,\text{м}$
    от вертикально расположенной рассеивающей линзы с фокусным расстоянием $F = 20\,\text{см}$.
    Где находится изображение предмета? Определите тип изображения и его высоту.
}
\solutionspace{120pt}

\tasknumber{9}%
\task{%
    На каком расстоянии от двояковыпуклой линзы с оптической силой $D = 2{,}5\,\text{дптр}$
    надо поместить предмет, чтобы его изображение получилось на расстоянии $1{,}5\,\text{м}$ от линзы?
}
\solutionspace{120pt}

\tasknumber{10}%
\task{%
    Предмет в виде отрезка длиной $\ell$ расположен вдоль оптической оси
    собирающей линзы с фокусным расстоянием $F$.
    Середина отрезка расположена
    на расстоянии $a$ от линзы, которая даёт действительное изображение
    всех точек предмета.
    Определить продольное увеличение предмета.
}
\answer{%
    \begin{align*}
    \frac 1{a + \frac \ell 2} &+ \frac 1b = \frac 1F \implies b = \frac{F\cbr{a + \frac \ell 2}}{a + \frac \ell 2 - F} \\
    \frac 1{a - \frac \ell 2} &+ \frac 1c = \frac 1F \implies c = \frac{F\cbr{a - \frac \ell 2}}{a - \frac \ell 2 - F} \\
    \abs{b - c} &= \abs{\frac{F\cbr{a + \frac \ell 2}}{a + \frac \ell 2 - F} - \frac{F\cbr{a - \frac \ell 2}}{a - \frac \ell 2 - F}}= F\abs{\frac{\cbr{a + \frac \ell 2}\cbr{a - \frac \ell 2 - F} - \cbr{a - \frac \ell 2}\cbr{a + \frac \ell 2 - F}}{ \cbr{a + \frac \ell 2 - F} \cbr{a - \frac \ell 2 - F} }} =  \\
    &= F\abs{\frac{a^2 - \frac {a\ell} 2 - Fa + \frac {a\ell} 2 - \frac {\ell^2} 4 - \frac {F\ell}2 - a^2 - \frac {a\ell}2 + aF + \frac {a\ell}2 + \frac {\ell^2} 4 - \frac {F\ell} 2}{\cbr{a + \frac \ell 2 - F} \cbr{a - \frac \ell 2 - F} }} = \\
    &= F\frac{F\ell}{\sqr{a-F} - \frac {\ell^2}4} = \frac{F^2\ell}{\sqr{a-F} - \frac {\ell^2}4}\implies \Gamma = \frac{\abs{b - c}}\ell = \frac{F^2}{\sqr{a-F} - \frac {\ell^2}4}.
    \end{align*}
}
\solutionspace{120pt}

\tasknumber{11}%
\task{%
    На экране с помощью тонкой линзы получено изображение предмета
    с увеличением $2$.
    Предмет передвинули на $2\,\text{см}$.
    Для того, чтобы получить резкое изображение, пришлось передвинуть экран.
    При этом увеличение оказалось равным $8$.
    На какое расстояние
    пришлось передвинуть экран?
}
\solutionspace{120pt}

\tasknumber{12}%
\task{%
    Тонкая собирающая линза дает изображение предмета на экране высотой $H_1$,
    и $H_2$, при двух положениях линзы между предметом и экраном.
    Расстояние между ними неизменно.
    Чему равна высота предмета $h$?
}
\answer{%
    $h = \sqrt{H_1 H_2}$
}
\solutionspace{120pt}

\tasknumber{13}%
\task{%
    Какие предметы можно рассмотреть на фотографии, сделанной со спутника,
    если разрешающая способность пленки $0{,}02\,\text{мм}$? Каким должно быть
    время экспозиции $\tau$ чтобы полностью использовать возможности пленки?
    Фокусное расстояние объектива используемого фотоаппарата $15\,\text{cм}$,
    высота орбиты спутника $80\,\text{км}$.
}
\solutionspace{120pt}

\tasknumber{14}%
\task{%
    При аэрофотосъемках используется фотоаппарат, объектив которого
    имеет фокусиое расстояние $12\,\text{cм}$.
    Разрешающая способность пленки $0{,}015\,\text{мм}$.
    На какой высоте должен лететь самолет, чтобы на фотографии можно
    было различить листья деревьев размером $5\,\text{cм}$?
    При какой скорости самолета изображение не будет размытым,
    если время зкспозиции $1\,\text{мс}$?
}

\variantsplitter

\addpersonalvariant{Виктория Легонькова}

\tasknumber{1}%
\task{%
    Найти оптическую силу собирающей линзы, если действительное изображение предмета,
    помещённого в $15\,\text{см}$ от линзы, получается на расстоянии $20\,\text{см}$ от неё.
}
\answer{%
    $D = \frac 1F = \frac 1a + \frac 1b = \frac 1{15\,\text{см}} + \frac 1{20\,\text{см}} \approx 11{,}67\,\text{дптр}$
}
\solutionspace{180pt}

\tasknumber{2}%
\task{%
    Найти увеличение изображения, если изображение предмета, находящегося
    на расстоянии $15\,\text{см}$ от линзы, получается на расстоянии $30\,\text{см}$ от неё.
}
\answer{%
    $\Gamma = \frac ba = \frac {30\,\text{см}}{15\,\text{см}} \approx 2$
}
\solutionspace{180pt}

\tasknumber{3}%
\task{%
    Расстояние от предмета до линзы $8\,\text{см}$, а от линзы до мнимого изображения $30\,\text{см}$.
    Чему равно фокусное расстояние линзы?
}
\answer{%
    $\pm \frac 1F = \frac 1a - \frac 1b \implies F = \frac{a b}{\abs{b - a}} \approx 10{,}9\,\text{см}$
}
\solutionspace{180pt}

\tasknumber{4}%
\task{%
    Две тонкие собирающие линзы с фокусными расстояниями $18\,\text{см}$ и $30\,\text{см}$ сложены вместе.
    Чему равно фокусное расстояние такой оптической системы?
}
\answer{%
    $\frac 1{f_1} = \frac 1a + \frac 1b; \frac 1{f_2} = - \frac 1b + \frac 1c \implies \frac 1{f_1} + \frac 1{f_2} = \frac 1a + \frac 1c \implies f' = \frac 1{\frac 1{f_1} + \frac 1{f_2}} = \frac{f_1 f_2}{f_1 + f_2} \approx 11{,}2\,\text{см}$
}
\solutionspace{180pt}

\tasknumber{5}%
\task{%
    Линейные размеры прямого изображения предмета, полученного в собирающей линзе,
    в три раза больше линейных размеров предмета.
    Зная, что предмет находится на $30\,\text{см}$ ближе к линзе,
    чем его изображение, найти оптическую силу линзы.
}
\answer{%
    \begin{align*}
    D &= \frac 1F = \frac 1a + \frac 1b, \qquad \Gamma = \frac ba, \qquad b - a = \ell \implies b = \Gamma a \implies \Gamma a - a = \ell \implies  \\
    a &= \frac {\ell}{\Gamma - 1} \implies b = \frac {{\ell} \Gamma}{\Gamma - 1} \implies  \\
    D &= \frac {\Gamma - 1}\ell + \frac {\Gamma - 1}{\ell \Gamma} = \frac 1\ell \cdot \cbr{\Gamma - 1 + \frac {\Gamma - 1}{\Gamma} } =\frac 1\ell \cdot \cbr{\Gamma - \frac 1\Gamma} \approx 8{,}9\,\text{дптр}.
    \end{align*}
}
\solutionspace{180pt}

\tasknumber{6}%
\task{%
    Оптическая сила объектива фотоаппарата равна $5\,\text{дптр}$.
    При фотографировании чертежа с расстояния $0{,}8\,\text{м}$ площадь изображения
    чертежа на фотопластинке оказалась равной $9\,\text{см}^{2}$.
    Какова площадь самого чертежа? Ответ выразите в квадратных сантиметрах.
}
\solutionspace{180pt}

\tasknumber{7}%
\task{%
    В каком месте на главной оптической оси двояковыпуклой линзы
    нужно поместить точечный источник света,
    чтобы его изображение оказалось в главном фокусе линзы?
}
\answer{%
    $\text{для мнимого - на половине фокусного, для действительного - на бесконечности}$
}
\solutionspace{120pt}

\tasknumber{8}%
\task{%
    Предмет высотой $h = 30\,\text{см}$ находится на расстоянии $d = 1\,\text{м}$
    от вертикально расположенной рассеивающей линзы с фокусным расстоянием $F = -15\,\text{см}$.
    Где находится изображение предмета? Определите тип изображения и его высоту.
}
\solutionspace{120pt}

\tasknumber{9}%
\task{%
    На каком расстоянии от двояковыпуклой линзы с оптической силой $D = 1{,}5\,\text{дптр}$
    надо поместить предмет, чтобы его изображение получилось на расстоянии $1{,}5\,\text{м}$ от линзы?
}
\solutionspace{120pt}

\tasknumber{10}%
\task{%
    Предмет в виде отрезка длиной $\ell$ расположен вдоль оптической оси
    собирающей линзы с фокусным расстоянием $F$.
    Середина отрезка расположена
    на расстоянии $a$ от линзы, которая даёт действительное изображение
    всех точек предмета.
    Определить продольное увеличение предмета.
}
\answer{%
    \begin{align*}
    \frac 1{a + \frac \ell 2} &+ \frac 1b = \frac 1F \implies b = \frac{F\cbr{a + \frac \ell 2}}{a + \frac \ell 2 - F} \\
    \frac 1{a - \frac \ell 2} &+ \frac 1c = \frac 1F \implies c = \frac{F\cbr{a - \frac \ell 2}}{a - \frac \ell 2 - F} \\
    \abs{b - c} &= \abs{\frac{F\cbr{a + \frac \ell 2}}{a + \frac \ell 2 - F} - \frac{F\cbr{a - \frac \ell 2}}{a - \frac \ell 2 - F}}= F\abs{\frac{\cbr{a + \frac \ell 2}\cbr{a - \frac \ell 2 - F} - \cbr{a - \frac \ell 2}\cbr{a + \frac \ell 2 - F}}{ \cbr{a + \frac \ell 2 - F} \cbr{a - \frac \ell 2 - F} }} =  \\
    &= F\abs{\frac{a^2 - \frac {a\ell} 2 - Fa + \frac {a\ell} 2 - \frac {\ell^2} 4 - \frac {F\ell}2 - a^2 - \frac {a\ell}2 + aF + \frac {a\ell}2 + \frac {\ell^2} 4 - \frac {F\ell} 2}{\cbr{a + \frac \ell 2 - F} \cbr{a - \frac \ell 2 - F} }} = \\
    &= F\frac{F\ell}{\sqr{a-F} - \frac {\ell^2}4} = \frac{F^2\ell}{\sqr{a-F} - \frac {\ell^2}4}\implies \Gamma = \frac{\abs{b - c}}\ell = \frac{F^2}{\sqr{a-F} - \frac {\ell^2}4}.
    \end{align*}
}
\solutionspace{120pt}

\tasknumber{11}%
\task{%
    На экране с помощью тонкой линзы получено изображение предмета
    с увеличением $2$.
    Предмет передвинули на $8\,\text{см}$.
    Для того, чтобы получить резкое изображение, пришлось передвинуть экран.
    При этом увеличение оказалось равным $6$.
    На какое расстояние
    пришлось передвинуть экран?
}
\solutionspace{120pt}

\tasknumber{12}%
\task{%
    Тонкая собирающая линза дает изображение предмета на экране высотой $H_1$,
    и $H_2$, при двух положениях линзы между предметом и экраном.
    Расстояние между ними неизменно.
    Чему равна высота предмета $h$?
}
\answer{%
    $h = \sqrt{H_1 H_2}$
}
\solutionspace{120pt}

\tasknumber{13}%
\task{%
    Какие предметы можно рассмотреть на фотографии, сделанной со спутника,
    если разрешающая способность пленки $0{,}010\,\text{мм}$? Каким должно быть
    время экспозиции $\tau$ чтобы полностью использовать возможности пленки?
    Фокусное расстояние объектива используемого фотоаппарата $15\,\text{cм}$,
    высота орбиты спутника $120\,\text{км}$.
}
\solutionspace{120pt}

\tasknumber{14}%
\task{%
    При аэрофотосъемках используется фотоаппарат, объектив которого
    имеет фокусиое расстояние $8\,\text{cм}$.
    Разрешающая способность пленки $0{,}02\,\text{мм}$.
    На какой высоте должен лететь самолет, чтобы на фотографии можно
    было различить листья деревьев размером $5\,\text{cм}$?
    При какой скорости самолета изображение не будет размытым,
    если время зкспозиции $1\,\text{мс}$?
}

\variantsplitter

\addpersonalvariant{Семён Мартынов}

\tasknumber{1}%
\task{%
    Найти оптическую силу собирающей линзы, если действительное изображение предмета,
    помещённого в $35\,\text{см}$ от линзы, получается на расстоянии $40\,\text{см}$ от неё.
}
\answer{%
    $D = \frac 1F = \frac 1a + \frac 1b = \frac 1{35\,\text{см}} + \frac 1{40\,\text{см}} \approx 5{,}36\,\text{дптр}$
}
\solutionspace{180pt}

\tasknumber{2}%
\task{%
    Найти увеличение изображения, если изображение предмета, находящегося
    на расстоянии $15\,\text{см}$ от линзы, получается на расстоянии $30\,\text{см}$ от неё.
}
\answer{%
    $\Gamma = \frac ba = \frac {30\,\text{см}}{15\,\text{см}} \approx 2$
}
\solutionspace{180pt}

\tasknumber{3}%
\task{%
    Расстояние от предмета до линзы $8\,\text{см}$, а от линзы до мнимого изображения $30\,\text{см}$.
    Чему равно фокусное расстояние линзы?
}
\answer{%
    $\pm \frac 1F = \frac 1a - \frac 1b \implies F = \frac{a b}{\abs{b - a}} \approx 10{,}9\,\text{см}$
}
\solutionspace{180pt}

\tasknumber{4}%
\task{%
    Две тонкие собирающие линзы с фокусными расстояниями $12\,\text{см}$ и $20\,\text{см}$ сложены вместе.
    Чему равно фокусное расстояние такой оптической системы?
}
\answer{%
    $\frac 1{f_1} = \frac 1a + \frac 1b; \frac 1{f_2} = - \frac 1b + \frac 1c \implies \frac 1{f_1} + \frac 1{f_2} = \frac 1a + \frac 1c \implies f' = \frac 1{\frac 1{f_1} + \frac 1{f_2}} = \frac{f_1 f_2}{f_1 + f_2} \approx 7{,}5\,\text{см}$
}
\solutionspace{180pt}

\tasknumber{5}%
\task{%
    Линейные размеры прямого изображения предмета, полученного в собирающей линзе,
    в четыре раза больше линейных размеров предмета.
    Зная, что предмет находится на $35\,\text{см}$ ближе к линзе,
    чем его изображение, найти оптическую силу линзы.
}
\answer{%
    \begin{align*}
    D &= \frac 1F = \frac 1a + \frac 1b, \qquad \Gamma = \frac ba, \qquad b - a = \ell \implies b = \Gamma a \implies \Gamma a - a = \ell \implies  \\
    a &= \frac {\ell}{\Gamma - 1} \implies b = \frac {{\ell} \Gamma}{\Gamma - 1} \implies  \\
    D &= \frac {\Gamma - 1}\ell + \frac {\Gamma - 1}{\ell \Gamma} = \frac 1\ell \cdot \cbr{\Gamma - 1 + \frac {\Gamma - 1}{\Gamma} } =\frac 1\ell \cdot \cbr{\Gamma - \frac 1\Gamma} \approx 10{,}7\,\text{дптр}.
    \end{align*}
}
\solutionspace{180pt}

\tasknumber{6}%
\task{%
    Оптическая сила объектива фотоаппарата равна $4\,\text{дптр}$.
    При фотографировании чертежа с расстояния $0{,}9\,\text{м}$ площадь изображения
    чертежа на фотопластинке оказалась равной $16\,\text{см}^{2}$.
    Какова площадь самого чертежа? Ответ выразите в квадратных сантиметрах.
}
\solutionspace{180pt}

\tasknumber{7}%
\task{%
    В каком месте на главной оптической оси двояковыпуклой линзы
    нужно поместить точечный источник света,
    чтобы его изображение оказалось в главном фокусе линзы?
}
\answer{%
    $\text{для мнимого - на половине фокусного, для действительного - на бесконечности}$
}
\solutionspace{120pt}

\tasknumber{8}%
\task{%
    Предмет высотой $h = 50\,\text{см}$ находится на расстоянии $d = 1{,}2\,\text{м}$
    от вертикально расположенной рассеивающей линзы с фокусным расстоянием $F = 20\,\text{см}$.
    Где находится изображение предмета? Определите тип изображения и его высоту.
}
\solutionspace{120pt}

\tasknumber{9}%
\task{%
    На каком расстоянии от двояковыпуклой линзы с оптической силой $D = 1{,}5\,\text{дптр}$
    надо поместить предмет, чтобы его изображение получилось на расстоянии $2\,\text{м}$ от линзы?
}
\solutionspace{120pt}

\tasknumber{10}%
\task{%
    Предмет в виде отрезка длиной $\ell$ расположен вдоль оптической оси
    собирающей линзы с фокусным расстоянием $F$.
    Середина отрезка расположена
    на расстоянии $a$ от линзы, которая даёт действительное изображение
    всех точек предмета.
    Определить продольное увеличение предмета.
}
\answer{%
    \begin{align*}
    \frac 1{a + \frac \ell 2} &+ \frac 1b = \frac 1F \implies b = \frac{F\cbr{a + \frac \ell 2}}{a + \frac \ell 2 - F} \\
    \frac 1{a - \frac \ell 2} &+ \frac 1c = \frac 1F \implies c = \frac{F\cbr{a - \frac \ell 2}}{a - \frac \ell 2 - F} \\
    \abs{b - c} &= \abs{\frac{F\cbr{a + \frac \ell 2}}{a + \frac \ell 2 - F} - \frac{F\cbr{a - \frac \ell 2}}{a - \frac \ell 2 - F}}= F\abs{\frac{\cbr{a + \frac \ell 2}\cbr{a - \frac \ell 2 - F} - \cbr{a - \frac \ell 2}\cbr{a + \frac \ell 2 - F}}{ \cbr{a + \frac \ell 2 - F} \cbr{a - \frac \ell 2 - F} }} =  \\
    &= F\abs{\frac{a^2 - \frac {a\ell} 2 - Fa + \frac {a\ell} 2 - \frac {\ell^2} 4 - \frac {F\ell}2 - a^2 - \frac {a\ell}2 + aF + \frac {a\ell}2 + \frac {\ell^2} 4 - \frac {F\ell} 2}{\cbr{a + \frac \ell 2 - F} \cbr{a - \frac \ell 2 - F} }} = \\
    &= F\frac{F\ell}{\sqr{a-F} - \frac {\ell^2}4} = \frac{F^2\ell}{\sqr{a-F} - \frac {\ell^2}4}\implies \Gamma = \frac{\abs{b - c}}\ell = \frac{F^2}{\sqr{a-F} - \frac {\ell^2}4}.
    \end{align*}
}
\solutionspace{120pt}

\tasknumber{11}%
\task{%
    На экране с помощью тонкой линзы получено изображение предмета
    с увеличением $4$.
    Предмет передвинули на $8\,\text{см}$.
    Для того, чтобы получить резкое изображение, пришлось передвинуть экран.
    При этом увеличение оказалось равным $6$.
    На какое расстояние
    пришлось передвинуть экран?
}
\solutionspace{120pt}

\tasknumber{12}%
\task{%
    Тонкая собирающая линза дает изображение предмета на экране высотой $H_1$,
    и $H_2$, при двух положениях линзы между предметом и экраном.
    Расстояние между ними неизменно.
    Чему равна высота предмета $h$?
}
\answer{%
    $h = \sqrt{H_1 H_2}$
}
\solutionspace{120pt}

\tasknumber{13}%
\task{%
    Какие предметы можно рассмотреть на фотографии, сделанной со спутника,
    если разрешающая способность пленки $0{,}02\,\text{мм}$? Каким должно быть
    время экспозиции $\tau$ чтобы полностью использовать возможности пленки?
    Фокусное расстояние объектива используемого фотоаппарата $10\,\text{cм}$,
    высота орбиты спутника $120\,\text{км}$.
}
\solutionspace{120pt}

\tasknumber{14}%
\task{%
    При аэрофотосъемках используется фотоаппарат, объектив которого
    имеет фокусиое расстояние $8\,\text{cм}$.
    Разрешающая способность пленки $0{,}015\,\text{мм}$.
    На какой высоте должен лететь самолет, чтобы на фотографии можно
    было различить листья деревьев размером $4\,\text{cм}$?
    При какой скорости самолета изображение не будет размытым,
    если время зкспозиции $2\,\text{мс}$?
}

\variantsplitter

\addpersonalvariant{Варвара Минаева}

\tasknumber{1}%
\task{%
    Найти оптическую силу собирающей линзы, если действительное изображение предмета,
    помещённого в $55\,\text{см}$ от линзы, получается на расстоянии $30\,\text{см}$ от неё.
}
\answer{%
    $D = \frac 1F = \frac 1a + \frac 1b = \frac 1{55\,\text{см}} + \frac 1{30\,\text{см}} \approx 5{,}15\,\text{дптр}$
}
\solutionspace{180pt}

\tasknumber{2}%
\task{%
    Найти увеличение изображения, если изображение предмета, находящегося
    на расстоянии $20\,\text{см}$ от линзы, получается на расстоянии $30\,\text{см}$ от неё.
}
\answer{%
    $\Gamma = \frac ba = \frac {30\,\text{см}}{20\,\text{см}} \approx 1{,}50$
}
\solutionspace{180pt}

\tasknumber{3}%
\task{%
    Расстояние от предмета до линзы $10\,\text{см}$, а от линзы до мнимого изображения $30\,\text{см}$.
    Чему равно фокусное расстояние линзы?
}
\answer{%
    $\pm \frac 1F = \frac 1a - \frac 1b \implies F = \frac{a b}{\abs{b - a}} \approx 15\,\text{см}$
}
\solutionspace{180pt}

\tasknumber{4}%
\task{%
    Две тонкие собирающие линзы с фокусными расстояниями $12\,\text{см}$ и $30\,\text{см}$ сложены вместе.
    Чему равно фокусное расстояние такой оптической системы?
}
\answer{%
    $\frac 1{f_1} = \frac 1a + \frac 1b; \frac 1{f_2} = - \frac 1b + \frac 1c \implies \frac 1{f_1} + \frac 1{f_2} = \frac 1a + \frac 1c \implies f' = \frac 1{\frac 1{f_1} + \frac 1{f_2}} = \frac{f_1 f_2}{f_1 + f_2} \approx 8{,}6\,\text{см}$
}
\solutionspace{180pt}

\tasknumber{5}%
\task{%
    Линейные размеры прямого изображения предмета, полученного в собирающей линзе,
    в четыре раза больше линейных размеров предмета.
    Зная, что предмет находится на $25\,\text{см}$ ближе к линзе,
    чем его изображение, найти оптическую силу линзы.
}
\answer{%
    \begin{align*}
    D &= \frac 1F = \frac 1a + \frac 1b, \qquad \Gamma = \frac ba, \qquad b - a = \ell \implies b = \Gamma a \implies \Gamma a - a = \ell \implies  \\
    a &= \frac {\ell}{\Gamma - 1} \implies b = \frac {{\ell} \Gamma}{\Gamma - 1} \implies  \\
    D &= \frac {\Gamma - 1}\ell + \frac {\Gamma - 1}{\ell \Gamma} = \frac 1\ell \cdot \cbr{\Gamma - 1 + \frac {\Gamma - 1}{\Gamma} } =\frac 1\ell \cdot \cbr{\Gamma - \frac 1\Gamma} \approx 15\,\text{дптр}.
    \end{align*}
}
\solutionspace{180pt}

\tasknumber{6}%
\task{%
    Оптическая сила объектива фотоаппарата равна $4\,\text{дптр}$.
    При фотографировании чертежа с расстояния $0{,}8\,\text{м}$ площадь изображения
    чертежа на фотопластинке оказалась равной $16\,\text{см}^{2}$.
    Какова площадь самого чертежа? Ответ выразите в квадратных сантиметрах.
}
\solutionspace{180pt}

\tasknumber{7}%
\task{%
    В каком месте на главной оптической оси двояковыгнутой линзы
    нужно поместить точечный источник света,
    чтобы его изображение оказалось в главном фокусе линзы?
}
\answer{%
    $\text{на половине фокусного расстояния}$
}
\solutionspace{120pt}

\tasknumber{8}%
\task{%
    Предмет высотой $h = 50\,\text{см}$ находится на расстоянии $d = 1{,}2\,\text{м}$
    от вертикально расположенной рассеивающей линзы с фокусным расстоянием $F = -15\,\text{см}$.
    Где находится изображение предмета? Определите тип изображения и его высоту.
}
\solutionspace{120pt}

\tasknumber{9}%
\task{%
    На каком расстоянии от двояковыпуклой линзы с оптической силой $D = 2\,\text{дптр}$
    надо поместить предмет, чтобы его изображение получилось на расстоянии $1{,}5\,\text{м}$ от линзы?
}
\solutionspace{120pt}

\tasknumber{10}%
\task{%
    Предмет в виде отрезка длиной $\ell$ расположен вдоль оптической оси
    собирающей линзы с фокусным расстоянием $F$.
    Середина отрезка расположена
    на расстоянии $a$ от линзы, которая даёт действительное изображение
    всех точек предмета.
    Определить продольное увеличение предмета.
}
\answer{%
    \begin{align*}
    \frac 1{a + \frac \ell 2} &+ \frac 1b = \frac 1F \implies b = \frac{F\cbr{a + \frac \ell 2}}{a + \frac \ell 2 - F} \\
    \frac 1{a - \frac \ell 2} &+ \frac 1c = \frac 1F \implies c = \frac{F\cbr{a - \frac \ell 2}}{a - \frac \ell 2 - F} \\
    \abs{b - c} &= \abs{\frac{F\cbr{a + \frac \ell 2}}{a + \frac \ell 2 - F} - \frac{F\cbr{a - \frac \ell 2}}{a - \frac \ell 2 - F}}= F\abs{\frac{\cbr{a + \frac \ell 2}\cbr{a - \frac \ell 2 - F} - \cbr{a - \frac \ell 2}\cbr{a + \frac \ell 2 - F}}{ \cbr{a + \frac \ell 2 - F} \cbr{a - \frac \ell 2 - F} }} =  \\
    &= F\abs{\frac{a^2 - \frac {a\ell} 2 - Fa + \frac {a\ell} 2 - \frac {\ell^2} 4 - \frac {F\ell}2 - a^2 - \frac {a\ell}2 + aF + \frac {a\ell}2 + \frac {\ell^2} 4 - \frac {F\ell} 2}{\cbr{a + \frac \ell 2 - F} \cbr{a - \frac \ell 2 - F} }} = \\
    &= F\frac{F\ell}{\sqr{a-F} - \frac {\ell^2}4} = \frac{F^2\ell}{\sqr{a-F} - \frac {\ell^2}4}\implies \Gamma = \frac{\abs{b - c}}\ell = \frac{F^2}{\sqr{a-F} - \frac {\ell^2}4}.
    \end{align*}
}
\solutionspace{120pt}

\tasknumber{11}%
\task{%
    На экране с помощью тонкой линзы получено изображение предмета
    с увеличением $4$.
    Предмет передвинули на $10\,\text{см}$.
    Для того, чтобы получить резкое изображение, пришлось передвинуть экран.
    При этом увеличение оказалось равным $6$.
    На какое расстояние
    пришлось передвинуть экран?
}
\solutionspace{120pt}

\tasknumber{12}%
\task{%
    Тонкая собирающая линза дает изображение предмета на экране высотой $H_1$,
    и $H_2$, при двух положениях линзы между предметом и экраном.
    Расстояние между ними неизменно.
    Чему равна высота предмета $h$?
}
\answer{%
    $h = \sqrt{H_1 H_2}$
}
\solutionspace{120pt}

\tasknumber{13}%
\task{%
    Какие предметы можно рассмотреть на фотографии, сделанной со спутника,
    если разрешающая способность пленки $0{,}02\,\text{мм}$? Каким должно быть
    время экспозиции $\tau$ чтобы полностью использовать возможности пленки?
    Фокусное расстояние объектива используемого фотоаппарата $10\,\text{cм}$,
    высота орбиты спутника $100\,\text{км}$.
}
\solutionspace{120pt}

\tasknumber{14}%
\task{%
    При аэрофотосъемках используется фотоаппарат, объектив которого
    имеет фокусиое расстояние $10\,\text{cм}$.
    Разрешающая способность пленки $0{,}015\,\text{мм}$.
    На какой высоте должен лететь самолет, чтобы на фотографии можно
    было различить листья деревьев размером $4\,\text{cм}$?
    При какой скорости самолета изображение не будет размытым,
    если время зкспозиции $1\,\text{мс}$?
}

\variantsplitter

\addpersonalvariant{Леонид Никитин}

\tasknumber{1}%
\task{%
    Найти оптическую силу собирающей линзы, если действительное изображение предмета,
    помещённого в $35\,\text{см}$ от линзы, получается на расстоянии $20\,\text{см}$ от неё.
}
\answer{%
    $D = \frac 1F = \frac 1a + \frac 1b = \frac 1{35\,\text{см}} + \frac 1{20\,\text{см}} \approx 7{,}86\,\text{дптр}$
}
\solutionspace{180pt}

\tasknumber{2}%
\task{%
    Найти увеличение изображения, если изображение предмета, находящегося
    на расстоянии $25\,\text{см}$ от линзы, получается на расстоянии $18\,\text{см}$ от неё.
}
\answer{%
    $\Gamma = \frac ba = \frac {18\,\text{см}}{25\,\text{см}} \approx 0{,}7$
}
\solutionspace{180pt}

\tasknumber{3}%
\task{%
    Расстояние от предмета до линзы $12\,\text{см}$, а от линзы до мнимого изображения $25\,\text{см}$.
    Чему равно фокусное расстояние линзы?
}
\answer{%
    $\pm \frac 1F = \frac 1a - \frac 1b \implies F = \frac{a b}{\abs{b - a}} \approx 23{,}1\,\text{см}$
}
\solutionspace{180pt}

\tasknumber{4}%
\task{%
    Две тонкие собирающие линзы с фокусными расстояниями $18\,\text{см}$ и $30\,\text{см}$ сложены вместе.
    Чему равно фокусное расстояние такой оптической системы?
}
\answer{%
    $\frac 1{f_1} = \frac 1a + \frac 1b; \frac 1{f_2} = - \frac 1b + \frac 1c \implies \frac 1{f_1} + \frac 1{f_2} = \frac 1a + \frac 1c \implies f' = \frac 1{\frac 1{f_1} + \frac 1{f_2}} = \frac{f_1 f_2}{f_1 + f_2} \approx 11{,}2\,\text{см}$
}
\solutionspace{180pt}

\tasknumber{5}%
\task{%
    Линейные размеры прямого изображения предмета, полученного в собирающей линзе,
    в три раза больше линейных размеров предмета.
    Зная, что предмет находится на $40\,\text{см}$ ближе к линзе,
    чем его изображение, найти оптическую силу линзы.
}
\answer{%
    \begin{align*}
    D &= \frac 1F = \frac 1a + \frac 1b, \qquad \Gamma = \frac ba, \qquad b - a = \ell \implies b = \Gamma a \implies \Gamma a - a = \ell \implies  \\
    a &= \frac {\ell}{\Gamma - 1} \implies b = \frac {{\ell} \Gamma}{\Gamma - 1} \implies  \\
    D &= \frac {\Gamma - 1}\ell + \frac {\Gamma - 1}{\ell \Gamma} = \frac 1\ell \cdot \cbr{\Gamma - 1 + \frac {\Gamma - 1}{\Gamma} } =\frac 1\ell \cdot \cbr{\Gamma - \frac 1\Gamma} \approx 6{,}7\,\text{дптр}.
    \end{align*}
}
\solutionspace{180pt}

\tasknumber{6}%
\task{%
    Оптическая сила объектива фотоаппарата равна $5\,\text{дптр}$.
    При фотографировании чертежа с расстояния $0{,}8\,\text{м}$ площадь изображения
    чертежа на фотопластинке оказалась равной $9\,\text{см}^{2}$.
    Какова площадь самого чертежа? Ответ выразите в квадратных сантиметрах.
}
\solutionspace{180pt}

\tasknumber{7}%
\task{%
    В каком месте на главной оптической оси двояковыгнутой линзы
    нужно поместить точечный источник света,
    чтобы его изображение оказалось в главном фокусе линзы?
}
\answer{%
    $\text{на половине фокусного расстояния}$
}
\solutionspace{120pt}

\tasknumber{8}%
\task{%
    Предмет высотой $h = 40\,\text{см}$ находится на расстоянии $d = 1{,}2\,\text{м}$
    от вертикально расположенной рассеивающей линзы с фокусным расстоянием $F = 20\,\text{см}$.
    Где находится изображение предмета? Определите тип изображения и его высоту.
}
\solutionspace{120pt}

\tasknumber{9}%
\task{%
    На каком расстоянии от двояковыпуклой линзы с оптической силой $D = 1{,}5\,\text{дптр}$
    надо поместить предмет, чтобы его изображение получилось на расстоянии $1{,}5\,\text{м}$ от линзы?
}
\solutionspace{120pt}

\tasknumber{10}%
\task{%
    Предмет в виде отрезка длиной $\ell$ расположен вдоль оптической оси
    собирающей линзы с фокусным расстоянием $F$.
    Середина отрезка расположена
    на расстоянии $a$ от линзы, которая даёт действительное изображение
    всех точек предмета.
    Определить продольное увеличение предмета.
}
\answer{%
    \begin{align*}
    \frac 1{a + \frac \ell 2} &+ \frac 1b = \frac 1F \implies b = \frac{F\cbr{a + \frac \ell 2}}{a + \frac \ell 2 - F} \\
    \frac 1{a - \frac \ell 2} &+ \frac 1c = \frac 1F \implies c = \frac{F\cbr{a - \frac \ell 2}}{a - \frac \ell 2 - F} \\
    \abs{b - c} &= \abs{\frac{F\cbr{a + \frac \ell 2}}{a + \frac \ell 2 - F} - \frac{F\cbr{a - \frac \ell 2}}{a - \frac \ell 2 - F}}= F\abs{\frac{\cbr{a + \frac \ell 2}\cbr{a - \frac \ell 2 - F} - \cbr{a - \frac \ell 2}\cbr{a + \frac \ell 2 - F}}{ \cbr{a + \frac \ell 2 - F} \cbr{a - \frac \ell 2 - F} }} =  \\
    &= F\abs{\frac{a^2 - \frac {a\ell} 2 - Fa + \frac {a\ell} 2 - \frac {\ell^2} 4 - \frac {F\ell}2 - a^2 - \frac {a\ell}2 + aF + \frac {a\ell}2 + \frac {\ell^2} 4 - \frac {F\ell} 2}{\cbr{a + \frac \ell 2 - F} \cbr{a - \frac \ell 2 - F} }} = \\
    &= F\frac{F\ell}{\sqr{a-F} - \frac {\ell^2}4} = \frac{F^2\ell}{\sqr{a-F} - \frac {\ell^2}4}\implies \Gamma = \frac{\abs{b - c}}\ell = \frac{F^2}{\sqr{a-F} - \frac {\ell^2}4}.
    \end{align*}
}
\solutionspace{120pt}

\tasknumber{11}%
\task{%
    На экране с помощью тонкой линзы получено изображение предмета
    с увеличением $2$.
    Предмет передвинули на $4\,\text{см}$.
    Для того, чтобы получить резкое изображение, пришлось передвинуть экран.
    При этом увеличение оказалось равным $6$.
    На какое расстояние
    пришлось передвинуть экран?
}
\solutionspace{120pt}

\tasknumber{12}%
\task{%
    Тонкая собирающая линза дает изображение предмета на экране высотой $H_1$,
    и $H_2$, при двух положениях линзы между предметом и экраном.
    Расстояние между ними неизменно.
    Чему равна высота предмета $h$?
}
\answer{%
    $h = \sqrt{H_1 H_2}$
}
\solutionspace{120pt}

\tasknumber{13}%
\task{%
    Какие предметы можно рассмотреть на фотографии, сделанной со спутника,
    если разрешающая способность пленки $0{,}010\,\text{мм}$? Каким должно быть
    время экспозиции $\tau$ чтобы полностью использовать возможности пленки?
    Фокусное расстояние объектива используемого фотоаппарата $15\,\text{cм}$,
    высота орбиты спутника $120\,\text{км}$.
}
\solutionspace{120pt}

\tasknumber{14}%
\task{%
    При аэрофотосъемках используется фотоаппарат, объектив которого
    имеет фокусиое расстояние $8\,\text{cм}$.
    Разрешающая способность пленки $0{,}010\,\text{мм}$.
    На какой высоте должен лететь самолет, чтобы на фотографии можно
    было различить листья деревьев размером $5\,\text{cм}$?
    При какой скорости самолета изображение не будет размытым,
    если время зкспозиции $2\,\text{мс}$?
}

\variantsplitter

\addpersonalvariant{Тимофей Полетаев}

\tasknumber{1}%
\task{%
    Найти оптическую силу собирающей линзы, если действительное изображение предмета,
    помещённого в $15\,\text{см}$ от линзы, получается на расстоянии $20\,\text{см}$ от неё.
}
\answer{%
    $D = \frac 1F = \frac 1a + \frac 1b = \frac 1{15\,\text{см}} + \frac 1{20\,\text{см}} \approx 11{,}67\,\text{дптр}$
}
\solutionspace{180pt}

\tasknumber{2}%
\task{%
    Найти увеличение изображения, если изображение предмета, находящегося
    на расстоянии $15\,\text{см}$ от линзы, получается на расстоянии $12\,\text{см}$ от неё.
}
\answer{%
    $\Gamma = \frac ba = \frac {12\,\text{см}}{15\,\text{см}} \approx 0{,}8$
}
\solutionspace{180pt}

\tasknumber{3}%
\task{%
    Расстояние от предмета до линзы $8\,\text{см}$, а от линзы до мнимого изображения $20\,\text{см}$.
    Чему равно фокусное расстояние линзы?
}
\answer{%
    $\pm \frac 1F = \frac 1a - \frac 1b \implies F = \frac{a b}{\abs{b - a}} \approx 13{,}3\,\text{см}$
}
\solutionspace{180pt}

\tasknumber{4}%
\task{%
    Две тонкие собирающие линзы с фокусными расстояниями $18\,\text{см}$ и $30\,\text{см}$ сложены вместе.
    Чему равно фокусное расстояние такой оптической системы?
}
\answer{%
    $\frac 1{f_1} = \frac 1a + \frac 1b; \frac 1{f_2} = - \frac 1b + \frac 1c \implies \frac 1{f_1} + \frac 1{f_2} = \frac 1a + \frac 1c \implies f' = \frac 1{\frac 1{f_1} + \frac 1{f_2}} = \frac{f_1 f_2}{f_1 + f_2} \approx 11{,}2\,\text{см}$
}
\solutionspace{180pt}

\tasknumber{5}%
\task{%
    Линейные размеры прямого изображения предмета, полученного в собирающей линзе,
    в четыре раза больше линейных размеров предмета.
    Зная, что предмет находится на $40\,\text{см}$ ближе к линзе,
    чем его изображение, найти оптическую силу линзы.
}
\answer{%
    \begin{align*}
    D &= \frac 1F = \frac 1a + \frac 1b, \qquad \Gamma = \frac ba, \qquad b - a = \ell \implies b = \Gamma a \implies \Gamma a - a = \ell \implies  \\
    a &= \frac {\ell}{\Gamma - 1} \implies b = \frac {{\ell} \Gamma}{\Gamma - 1} \implies  \\
    D &= \frac {\Gamma - 1}\ell + \frac {\Gamma - 1}{\ell \Gamma} = \frac 1\ell \cdot \cbr{\Gamma - 1 + \frac {\Gamma - 1}{\Gamma} } =\frac 1\ell \cdot \cbr{\Gamma - \frac 1\Gamma} \approx 9{,}4\,\text{дптр}.
    \end{align*}
}
\solutionspace{180pt}

\tasknumber{6}%
\task{%
    Оптическая сила объектива фотоаппарата равна $3\,\text{дптр}$.
    При фотографировании чертежа с расстояния $1{,}1\,\text{м}$ площадь изображения
    чертежа на фотопластинке оказалась равной $16\,\text{см}^{2}$.
    Какова площадь самого чертежа? Ответ выразите в квадратных сантиметрах.
}
\solutionspace{180pt}

\tasknumber{7}%
\task{%
    В каком месте на главной оптической оси двояковыгнутой линзы
    нужно поместить точечный источник света,
    чтобы его изображение оказалось в главном фокусе линзы?
}
\answer{%
    $\text{на половине фокусного расстояния}$
}
\solutionspace{120pt}

\tasknumber{8}%
\task{%
    Предмет высотой $h = 40\,\text{см}$ находится на расстоянии $d = 1\,\text{м}$
    от вертикально расположенной рассеивающей линзы с фокусным расстоянием $F = 25\,\text{см}$.
    Где находится изображение предмета? Определите тип изображения и его высоту.
}
\solutionspace{120pt}

\tasknumber{9}%
\task{%
    На каком расстоянии от двояковыпуклой линзы с оптической силой $D = 1{,}5\,\text{дптр}$
    надо поместить предмет, чтобы его изображение получилось на расстоянии $1{,}5\,\text{м}$ от линзы?
}
\solutionspace{120pt}

\tasknumber{10}%
\task{%
    Предмет в виде отрезка длиной $\ell$ расположен вдоль оптической оси
    собирающей линзы с фокусным расстоянием $F$.
    Середина отрезка расположена
    на расстоянии $a$ от линзы, которая даёт действительное изображение
    всех точек предмета.
    Определить продольное увеличение предмета.
}
\answer{%
    \begin{align*}
    \frac 1{a + \frac \ell 2} &+ \frac 1b = \frac 1F \implies b = \frac{F\cbr{a + \frac \ell 2}}{a + \frac \ell 2 - F} \\
    \frac 1{a - \frac \ell 2} &+ \frac 1c = \frac 1F \implies c = \frac{F\cbr{a - \frac \ell 2}}{a - \frac \ell 2 - F} \\
    \abs{b - c} &= \abs{\frac{F\cbr{a + \frac \ell 2}}{a + \frac \ell 2 - F} - \frac{F\cbr{a - \frac \ell 2}}{a - \frac \ell 2 - F}}= F\abs{\frac{\cbr{a + \frac \ell 2}\cbr{a - \frac \ell 2 - F} - \cbr{a - \frac \ell 2}\cbr{a + \frac \ell 2 - F}}{ \cbr{a + \frac \ell 2 - F} \cbr{a - \frac \ell 2 - F} }} =  \\
    &= F\abs{\frac{a^2 - \frac {a\ell} 2 - Fa + \frac {a\ell} 2 - \frac {\ell^2} 4 - \frac {F\ell}2 - a^2 - \frac {a\ell}2 + aF + \frac {a\ell}2 + \frac {\ell^2} 4 - \frac {F\ell} 2}{\cbr{a + \frac \ell 2 - F} \cbr{a - \frac \ell 2 - F} }} = \\
    &= F\frac{F\ell}{\sqr{a-F} - \frac {\ell^2}4} = \frac{F^2\ell}{\sqr{a-F} - \frac {\ell^2}4}\implies \Gamma = \frac{\abs{b - c}}\ell = \frac{F^2}{\sqr{a-F} - \frac {\ell^2}4}.
    \end{align*}
}
\solutionspace{120pt}

\tasknumber{11}%
\task{%
    На экране с помощью тонкой линзы получено изображение предмета
    с увеличением $2$.
    Предмет передвинули на $4\,\text{см}$.
    Для того, чтобы получить резкое изображение, пришлось передвинуть экран.
    При этом увеличение оказалось равным $6$.
    На какое расстояние
    пришлось передвинуть экран?
}
\solutionspace{120pt}

\tasknumber{12}%
\task{%
    Тонкая собирающая линза дает изображение предмета на экране высотой $H_1$,
    и $H_2$, при двух положениях линзы между предметом и экраном.
    Расстояние между ними неизменно.
    Чему равна высота предмета $h$?
}
\answer{%
    $h = \sqrt{H_1 H_2}$
}
\solutionspace{120pt}

\tasknumber{13}%
\task{%
    Какие предметы можно рассмотреть на фотографии, сделанной со спутника,
    если разрешающая способность пленки $0{,}010\,\text{мм}$? Каким должно быть
    время экспозиции $\tau$ чтобы полностью использовать возможности пленки?
    Фокусное расстояние объектива используемого фотоаппарата $15\,\text{cм}$,
    высота орбиты спутника $150\,\text{км}$.
}
\solutionspace{120pt}

\tasknumber{14}%
\task{%
    При аэрофотосъемках используется фотоаппарат, объектив которого
    имеет фокусиое расстояние $10\,\text{cм}$.
    Разрешающая способность пленки $0{,}015\,\text{мм}$.
    На какой высоте должен лететь самолет, чтобы на фотографии можно
    было различить листья деревьев размером $5\,\text{cм}$?
    При какой скорости самолета изображение не будет размытым,
    если время зкспозиции $1\,\text{мс}$?
}

\variantsplitter

\addpersonalvariant{Андрей Рожков}

\tasknumber{1}%
\task{%
    Найти оптическую силу собирающей линзы, если действительное изображение предмета,
    помещённого в $15\,\text{см}$ от линзы, получается на расстоянии $20\,\text{см}$ от неё.
}
\answer{%
    $D = \frac 1F = \frac 1a + \frac 1b = \frac 1{15\,\text{см}} + \frac 1{20\,\text{см}} \approx 11{,}67\,\text{дптр}$
}
\solutionspace{180pt}

\tasknumber{2}%
\task{%
    Найти увеличение изображения, если изображение предмета, находящегося
    на расстоянии $15\,\text{см}$ от линзы, получается на расстоянии $30\,\text{см}$ от неё.
}
\answer{%
    $\Gamma = \frac ba = \frac {30\,\text{см}}{15\,\text{см}} \approx 2$
}
\solutionspace{180pt}

\tasknumber{3}%
\task{%
    Расстояние от предмета до линзы $8\,\text{см}$, а от линзы до мнимого изображения $30\,\text{см}$.
    Чему равно фокусное расстояние линзы?
}
\answer{%
    $\pm \frac 1F = \frac 1a - \frac 1b \implies F = \frac{a b}{\abs{b - a}} \approx 10{,}9\,\text{см}$
}
\solutionspace{180pt}

\tasknumber{4}%
\task{%
    Две тонкие собирающие линзы с фокусными расстояниями $12\,\text{см}$ и $30\,\text{см}$ сложены вместе.
    Чему равно фокусное расстояние такой оптической системы?
}
\answer{%
    $\frac 1{f_1} = \frac 1a + \frac 1b; \frac 1{f_2} = - \frac 1b + \frac 1c \implies \frac 1{f_1} + \frac 1{f_2} = \frac 1a + \frac 1c \implies f' = \frac 1{\frac 1{f_1} + \frac 1{f_2}} = \frac{f_1 f_2}{f_1 + f_2} \approx 8{,}6\,\text{см}$
}
\solutionspace{180pt}

\tasknumber{5}%
\task{%
    Линейные размеры прямого изображения предмета, полученного в собирающей линзе,
    в четыре раза больше линейных размеров предмета.
    Зная, что предмет находится на $35\,\text{см}$ ближе к линзе,
    чем его изображение, найти оптическую силу линзы.
}
\answer{%
    \begin{align*}
    D &= \frac 1F = \frac 1a + \frac 1b, \qquad \Gamma = \frac ba, \qquad b - a = \ell \implies b = \Gamma a \implies \Gamma a - a = \ell \implies  \\
    a &= \frac {\ell}{\Gamma - 1} \implies b = \frac {{\ell} \Gamma}{\Gamma - 1} \implies  \\
    D &= \frac {\Gamma - 1}\ell + \frac {\Gamma - 1}{\ell \Gamma} = \frac 1\ell \cdot \cbr{\Gamma - 1 + \frac {\Gamma - 1}{\Gamma} } =\frac 1\ell \cdot \cbr{\Gamma - \frac 1\Gamma} \approx 10{,}7\,\text{дптр}.
    \end{align*}
}
\solutionspace{180pt}

\tasknumber{6}%
\task{%
    Оптическая сила объектива фотоаппарата равна $3\,\text{дптр}$.
    При фотографировании чертежа с расстояния $1{,}1\,\text{м}$ площадь изображения
    чертежа на фотопластинке оказалась равной $9\,\text{см}^{2}$.
    Какова площадь самого чертежа? Ответ выразите в квадратных сантиметрах.
}
\solutionspace{180pt}

\tasknumber{7}%
\task{%
    В каком месте на главной оптической оси двояковыпуклой линзы
    нужно поместить точечный источник света,
    чтобы его изображение оказалось в главном фокусе линзы?
}
\answer{%
    $\text{для мнимого - на половине фокусного, для действительного - на бесконечности}$
}
\solutionspace{120pt}

\tasknumber{8}%
\task{%
    Предмет высотой $h = 50\,\text{см}$ находится на расстоянии $d = 0{,}8\,\text{м}$
    от вертикально расположенной рассеивающей линзы с фокусным расстоянием $F = -15\,\text{см}$.
    Где находится изображение предмета? Определите тип изображения и его высоту.
}
\solutionspace{120pt}

\tasknumber{9}%
\task{%
    На каком расстоянии от двояковыпуклой линзы с оптической силой $D = 1{,}5\,\text{дптр}$
    надо поместить предмет, чтобы его изображение получилось на расстоянии $2{,}5\,\text{м}$ от линзы?
}
\solutionspace{120pt}

\tasknumber{10}%
\task{%
    Предмет в виде отрезка длиной $\ell$ расположен вдоль оптической оси
    собирающей линзы с фокусным расстоянием $F$.
    Середина отрезка расположена
    на расстоянии $a$ от линзы, которая даёт действительное изображение
    всех точек предмета.
    Определить продольное увеличение предмета.
}
\answer{%
    \begin{align*}
    \frac 1{a + \frac \ell 2} &+ \frac 1b = \frac 1F \implies b = \frac{F\cbr{a + \frac \ell 2}}{a + \frac \ell 2 - F} \\
    \frac 1{a - \frac \ell 2} &+ \frac 1c = \frac 1F \implies c = \frac{F\cbr{a - \frac \ell 2}}{a - \frac \ell 2 - F} \\
    \abs{b - c} &= \abs{\frac{F\cbr{a + \frac \ell 2}}{a + \frac \ell 2 - F} - \frac{F\cbr{a - \frac \ell 2}}{a - \frac \ell 2 - F}}= F\abs{\frac{\cbr{a + \frac \ell 2}\cbr{a - \frac \ell 2 - F} - \cbr{a - \frac \ell 2}\cbr{a + \frac \ell 2 - F}}{ \cbr{a + \frac \ell 2 - F} \cbr{a - \frac \ell 2 - F} }} =  \\
    &= F\abs{\frac{a^2 - \frac {a\ell} 2 - Fa + \frac {a\ell} 2 - \frac {\ell^2} 4 - \frac {F\ell}2 - a^2 - \frac {a\ell}2 + aF + \frac {a\ell}2 + \frac {\ell^2} 4 - \frac {F\ell} 2}{\cbr{a + \frac \ell 2 - F} \cbr{a - \frac \ell 2 - F} }} = \\
    &= F\frac{F\ell}{\sqr{a-F} - \frac {\ell^2}4} = \frac{F^2\ell}{\sqr{a-F} - \frac {\ell^2}4}\implies \Gamma = \frac{\abs{b - c}}\ell = \frac{F^2}{\sqr{a-F} - \frac {\ell^2}4}.
    \end{align*}
}
\solutionspace{120pt}

\tasknumber{11}%
\task{%
    На экране с помощью тонкой линзы получено изображение предмета
    с увеличением $4$.
    Предмет передвинули на $8\,\text{см}$.
    Для того, чтобы получить резкое изображение, пришлось передвинуть экран.
    При этом увеличение оказалось равным $6$.
    На какое расстояние
    пришлось передвинуть экран?
}
\solutionspace{120pt}

\tasknumber{12}%
\task{%
    Тонкая собирающая линза дает изображение предмета на экране высотой $H_1$,
    и $H_2$, при двух положениях линзы между предметом и экраном.
    Расстояние между ними неизменно.
    Чему равна высота предмета $h$?
}
\answer{%
    $h = \sqrt{H_1 H_2}$
}
\solutionspace{120pt}

\tasknumber{13}%
\task{%
    Какие предметы можно рассмотреть на фотографии, сделанной со спутника,
    если разрешающая способность пленки $0{,}010\,\text{мм}$? Каким должно быть
    время экспозиции $\tau$ чтобы полностью использовать возможности пленки?
    Фокусное расстояние объектива используемого фотоаппарата $10\,\text{cм}$,
    высота орбиты спутника $80\,\text{км}$.
}
\solutionspace{120pt}

\tasknumber{14}%
\task{%
    При аэрофотосъемках используется фотоаппарат, объектив которого
    имеет фокусиое расстояние $10\,\text{cм}$.
    Разрешающая способность пленки $0{,}02\,\text{мм}$.
    На какой высоте должен лететь самолет, чтобы на фотографии можно
    было различить листья деревьев размером $6\,\text{cм}$?
    При какой скорости самолета изображение не будет размытым,
    если время зкспозиции $1\,\text{мс}$?
}

\variantsplitter

\addpersonalvariant{Рената Таржиманова}

\tasknumber{1}%
\task{%
    Найти оптическую силу собирающей линзы, если действительное изображение предмета,
    помещённого в $35\,\text{см}$ от линзы, получается на расстоянии $20\,\text{см}$ от неё.
}
\answer{%
    $D = \frac 1F = \frac 1a + \frac 1b = \frac 1{35\,\text{см}} + \frac 1{20\,\text{см}} \approx 7{,}86\,\text{дптр}$
}
\solutionspace{180pt}

\tasknumber{2}%
\task{%
    Найти увеличение изображения, если изображение предмета, находящегося
    на расстоянии $25\,\text{см}$ от линзы, получается на расстоянии $12\,\text{см}$ от неё.
}
\answer{%
    $\Gamma = \frac ba = \frac {12\,\text{см}}{25\,\text{см}} \approx 0{,}5$
}
\solutionspace{180pt}

\tasknumber{3}%
\task{%
    Расстояние от предмета до линзы $12\,\text{см}$, а от линзы до мнимого изображения $20\,\text{см}$.
    Чему равно фокусное расстояние линзы?
}
\answer{%
    $\pm \frac 1F = \frac 1a - \frac 1b \implies F = \frac{a b}{\abs{b - a}} \approx 30\,\text{см}$
}
\solutionspace{180pt}

\tasknumber{4}%
\task{%
    Две тонкие собирающие линзы с фокусными расстояниями $12\,\text{см}$ и $20\,\text{см}$ сложены вместе.
    Чему равно фокусное расстояние такой оптической системы?
}
\answer{%
    $\frac 1{f_1} = \frac 1a + \frac 1b; \frac 1{f_2} = - \frac 1b + \frac 1c \implies \frac 1{f_1} + \frac 1{f_2} = \frac 1a + \frac 1c \implies f' = \frac 1{\frac 1{f_1} + \frac 1{f_2}} = \frac{f_1 f_2}{f_1 + f_2} \approx 7{,}5\,\text{см}$
}
\solutionspace{180pt}

\tasknumber{5}%
\task{%
    Линейные размеры прямого изображения предмета, полученного в собирающей линзе,
    в два раза больше линейных размеров предмета.
    Зная, что предмет находится на $30\,\text{см}$ ближе к линзе,
    чем его изображение, найти оптическую силу линзы.
}
\answer{%
    \begin{align*}
    D &= \frac 1F = \frac 1a + \frac 1b, \qquad \Gamma = \frac ba, \qquad b - a = \ell \implies b = \Gamma a \implies \Gamma a - a = \ell \implies  \\
    a &= \frac {\ell}{\Gamma - 1} \implies b = \frac {{\ell} \Gamma}{\Gamma - 1} \implies  \\
    D &= \frac {\Gamma - 1}\ell + \frac {\Gamma - 1}{\ell \Gamma} = \frac 1\ell \cdot \cbr{\Gamma - 1 + \frac {\Gamma - 1}{\Gamma} } =\frac 1\ell \cdot \cbr{\Gamma - \frac 1\Gamma} \approx 5\,\text{дптр}.
    \end{align*}
}
\solutionspace{180pt}

\tasknumber{6}%
\task{%
    Оптическая сила объектива фотоаппарата равна $3\,\text{дптр}$.
    При фотографировании чертежа с расстояния $0{,}9\,\text{м}$ площадь изображения
    чертежа на фотопластинке оказалась равной $9\,\text{см}^{2}$.
    Какова площадь самого чертежа? Ответ выразите в квадратных сантиметрах.
}
\solutionspace{180pt}

\tasknumber{7}%
\task{%
    В каком месте на главной оптической оси двояковыпуклой линзы
    нужно поместить точечный источник света,
    чтобы его изображение оказалось в главном фокусе линзы?
}
\answer{%
    $\text{для мнимого - на половине фокусного, для действительного - на бесконечности}$
}
\solutionspace{120pt}

\tasknumber{8}%
\task{%
    Предмет высотой $h = 40\,\text{см}$ находится на расстоянии $d = 1\,\text{м}$
    от вертикально расположенной рассеивающей линзы с фокусным расстоянием $F = 20\,\text{см}$.
    Где находится изображение предмета? Определите тип изображения и его высоту.
}
\solutionspace{120pt}

\tasknumber{9}%
\task{%
    На каком расстоянии от двояковыпуклой линзы с оптической силой $D = 2{,}5\,\text{дптр}$
    надо поместить предмет, чтобы его изображение получилось на расстоянии $1{,}5\,\text{м}$ от линзы?
}
\solutionspace{120pt}

\tasknumber{10}%
\task{%
    Предмет в виде отрезка длиной $\ell$ расположен вдоль оптической оси
    собирающей линзы с фокусным расстоянием $F$.
    Середина отрезка расположена
    на расстоянии $a$ от линзы, которая даёт действительное изображение
    всех точек предмета.
    Определить продольное увеличение предмета.
}
\answer{%
    \begin{align*}
    \frac 1{a + \frac \ell 2} &+ \frac 1b = \frac 1F \implies b = \frac{F\cbr{a + \frac \ell 2}}{a + \frac \ell 2 - F} \\
    \frac 1{a - \frac \ell 2} &+ \frac 1c = \frac 1F \implies c = \frac{F\cbr{a - \frac \ell 2}}{a - \frac \ell 2 - F} \\
    \abs{b - c} &= \abs{\frac{F\cbr{a + \frac \ell 2}}{a + \frac \ell 2 - F} - \frac{F\cbr{a - \frac \ell 2}}{a - \frac \ell 2 - F}}= F\abs{\frac{\cbr{a + \frac \ell 2}\cbr{a - \frac \ell 2 - F} - \cbr{a - \frac \ell 2}\cbr{a + \frac \ell 2 - F}}{ \cbr{a + \frac \ell 2 - F} \cbr{a - \frac \ell 2 - F} }} =  \\
    &= F\abs{\frac{a^2 - \frac {a\ell} 2 - Fa + \frac {a\ell} 2 - \frac {\ell^2} 4 - \frac {F\ell}2 - a^2 - \frac {a\ell}2 + aF + \frac {a\ell}2 + \frac {\ell^2} 4 - \frac {F\ell} 2}{\cbr{a + \frac \ell 2 - F} \cbr{a - \frac \ell 2 - F} }} = \\
    &= F\frac{F\ell}{\sqr{a-F} - \frac {\ell^2}4} = \frac{F^2\ell}{\sqr{a-F} - \frac {\ell^2}4}\implies \Gamma = \frac{\abs{b - c}}\ell = \frac{F^2}{\sqr{a-F} - \frac {\ell^2}4}.
    \end{align*}
}
\solutionspace{120pt}

\tasknumber{11}%
\task{%
    На экране с помощью тонкой линзы получено изображение предмета
    с увеличением $4$.
    Предмет передвинули на $8\,\text{см}$.
    Для того, чтобы получить резкое изображение, пришлось передвинуть экран.
    При этом увеличение оказалось равным $8$.
    На какое расстояние
    пришлось передвинуть экран?
}
\solutionspace{120pt}

\tasknumber{12}%
\task{%
    Тонкая собирающая линза дает изображение предмета на экране высотой $H_1$,
    и $H_2$, при двух положениях линзы между предметом и экраном.
    Расстояние между ними неизменно.
    Чему равна высота предмета $h$?
}
\answer{%
    $h = \sqrt{H_1 H_2}$
}
\solutionspace{120pt}

\tasknumber{13}%
\task{%
    Какие предметы можно рассмотреть на фотографии, сделанной со спутника,
    если разрешающая способность пленки $0{,}02\,\text{мм}$? Каким должно быть
    время экспозиции $\tau$ чтобы полностью использовать возможности пленки?
    Фокусное расстояние объектива используемого фотоаппарата $15\,\text{cм}$,
    высота орбиты спутника $150\,\text{км}$.
}
\solutionspace{120pt}

\tasknumber{14}%
\task{%
    При аэрофотосъемках используется фотоаппарат, объектив которого
    имеет фокусиое расстояние $12\,\text{cм}$.
    Разрешающая способность пленки $0{,}015\,\text{мм}$.
    На какой высоте должен лететь самолет, чтобы на фотографии можно
    было различить листья деревьев размером $5\,\text{cм}$?
    При какой скорости самолета изображение не будет размытым,
    если время зкспозиции $1\,\text{мс}$?
}

\variantsplitter

\addpersonalvariant{Андрей Щербаков}

\tasknumber{1}%
\task{%
    Найти оптическую силу собирающей линзы, если действительное изображение предмета,
    помещённого в $55\,\text{см}$ от линзы, получается на расстоянии $30\,\text{см}$ от неё.
}
\answer{%
    $D = \frac 1F = \frac 1a + \frac 1b = \frac 1{55\,\text{см}} + \frac 1{30\,\text{см}} \approx 5{,}15\,\text{дптр}$
}
\solutionspace{180pt}

\tasknumber{2}%
\task{%
    Найти увеличение изображения, если изображение предмета, находящегося
    на расстоянии $20\,\text{см}$ от линзы, получается на расстоянии $30\,\text{см}$ от неё.
}
\answer{%
    $\Gamma = \frac ba = \frac {30\,\text{см}}{20\,\text{см}} \approx 1{,}50$
}
\solutionspace{180pt}

\tasknumber{3}%
\task{%
    Расстояние от предмета до линзы $10\,\text{см}$, а от линзы до мнимого изображения $30\,\text{см}$.
    Чему равно фокусное расстояние линзы?
}
\answer{%
    $\pm \frac 1F = \frac 1a - \frac 1b \implies F = \frac{a b}{\abs{b - a}} \approx 15\,\text{см}$
}
\solutionspace{180pt}

\tasknumber{4}%
\task{%
    Две тонкие собирающие линзы с фокусными расстояниями $18\,\text{см}$ и $20\,\text{см}$ сложены вместе.
    Чему равно фокусное расстояние такой оптической системы?
}
\answer{%
    $\frac 1{f_1} = \frac 1a + \frac 1b; \frac 1{f_2} = - \frac 1b + \frac 1c \implies \frac 1{f_1} + \frac 1{f_2} = \frac 1a + \frac 1c \implies f' = \frac 1{\frac 1{f_1} + \frac 1{f_2}} = \frac{f_1 f_2}{f_1 + f_2} \approx 9{,}5\,\text{см}$
}
\solutionspace{180pt}

\tasknumber{5}%
\task{%
    Линейные размеры прямого изображения предмета, полученного в собирающей линзе,
    в два раза больше линейных размеров предмета.
    Зная, что предмет находится на $20\,\text{см}$ ближе к линзе,
    чем его изображение, найти оптическую силу линзы.
}
\answer{%
    \begin{align*}
    D &= \frac 1F = \frac 1a + \frac 1b, \qquad \Gamma = \frac ba, \qquad b - a = \ell \implies b = \Gamma a \implies \Gamma a - a = \ell \implies  \\
    a &= \frac {\ell}{\Gamma - 1} \implies b = \frac {{\ell} \Gamma}{\Gamma - 1} \implies  \\
    D &= \frac {\Gamma - 1}\ell + \frac {\Gamma - 1}{\ell \Gamma} = \frac 1\ell \cdot \cbr{\Gamma - 1 + \frac {\Gamma - 1}{\Gamma} } =\frac 1\ell \cdot \cbr{\Gamma - \frac 1\Gamma} \approx 7{,}5\,\text{дптр}.
    \end{align*}
}
\solutionspace{180pt}

\tasknumber{6}%
\task{%
    Оптическая сила объектива фотоаппарата равна $3\,\text{дптр}$.
    При фотографировании чертежа с расстояния $0{,}9\,\text{м}$ площадь изображения
    чертежа на фотопластинке оказалась равной $9\,\text{см}^{2}$.
    Какова площадь самого чертежа? Ответ выразите в квадратных сантиметрах.
}
\solutionspace{180pt}

\tasknumber{7}%
\task{%
    В каком месте на главной оптической оси двояковыгнутой линзы
    нужно поместить точечный источник света,
    чтобы его изображение оказалось в главном фокусе линзы?
}
\answer{%
    $\text{на половине фокусного расстояния}$
}
\solutionspace{120pt}

\tasknumber{8}%
\task{%
    Предмет высотой $h = 50\,\text{см}$ находится на расстоянии $d = 1\,\text{м}$
    от вертикально расположенной рассеивающей линзы с фокусным расстоянием $F = 25\,\text{см}$.
    Где находится изображение предмета? Определите тип изображения и его высоту.
}
\solutionspace{120pt}

\tasknumber{9}%
\task{%
    На каком расстоянии от двояковыпуклой линзы с оптической силой $D = 1{,}5\,\text{дптр}$
    надо поместить предмет, чтобы его изображение получилось на расстоянии $2\,\text{м}$ от линзы?
}
\solutionspace{120pt}

\tasknumber{10}%
\task{%
    Предмет в виде отрезка длиной $\ell$ расположен вдоль оптической оси
    собирающей линзы с фокусным расстоянием $F$.
    Середина отрезка расположена
    на расстоянии $a$ от линзы, которая даёт действительное изображение
    всех точек предмета.
    Определить продольное увеличение предмета.
}
\answer{%
    \begin{align*}
    \frac 1{a + \frac \ell 2} &+ \frac 1b = \frac 1F \implies b = \frac{F\cbr{a + \frac \ell 2}}{a + \frac \ell 2 - F} \\
    \frac 1{a - \frac \ell 2} &+ \frac 1c = \frac 1F \implies c = \frac{F\cbr{a - \frac \ell 2}}{a - \frac \ell 2 - F} \\
    \abs{b - c} &= \abs{\frac{F\cbr{a + \frac \ell 2}}{a + \frac \ell 2 - F} - \frac{F\cbr{a - \frac \ell 2}}{a - \frac \ell 2 - F}}= F\abs{\frac{\cbr{a + \frac \ell 2}\cbr{a - \frac \ell 2 - F} - \cbr{a - \frac \ell 2}\cbr{a + \frac \ell 2 - F}}{ \cbr{a + \frac \ell 2 - F} \cbr{a - \frac \ell 2 - F} }} =  \\
    &= F\abs{\frac{a^2 - \frac {a\ell} 2 - Fa + \frac {a\ell} 2 - \frac {\ell^2} 4 - \frac {F\ell}2 - a^2 - \frac {a\ell}2 + aF + \frac {a\ell}2 + \frac {\ell^2} 4 - \frac {F\ell} 2}{\cbr{a + \frac \ell 2 - F} \cbr{a - \frac \ell 2 - F} }} = \\
    &= F\frac{F\ell}{\sqr{a-F} - \frac {\ell^2}4} = \frac{F^2\ell}{\sqr{a-F} - \frac {\ell^2}4}\implies \Gamma = \frac{\abs{b - c}}\ell = \frac{F^2}{\sqr{a-F} - \frac {\ell^2}4}.
    \end{align*}
}
\solutionspace{120pt}

\tasknumber{11}%
\task{%
    На экране с помощью тонкой линзы получено изображение предмета
    с увеличением $2$.
    Предмет передвинули на $10\,\text{см}$.
    Для того, чтобы получить резкое изображение, пришлось передвинуть экран.
    При этом увеличение оказалось равным $8$.
    На какое расстояние
    пришлось передвинуть экран?
}
\solutionspace{120pt}

\tasknumber{12}%
\task{%
    Тонкая собирающая линза дает изображение предмета на экране высотой $H_1$,
    и $H_2$, при двух положениях линзы между предметом и экраном.
    Расстояние между ними неизменно.
    Чему равна высота предмета $h$?
}
\answer{%
    $h = \sqrt{H_1 H_2}$
}
\solutionspace{120pt}

\tasknumber{13}%
\task{%
    Какие предметы можно рассмотреть на фотографии, сделанной со спутника,
    если разрешающая способность пленки $0{,}010\,\text{мм}$? Каким должно быть
    время экспозиции $\tau$ чтобы полностью использовать возможности пленки?
    Фокусное расстояние объектива используемого фотоаппарата $20\,\text{cм}$,
    высота орбиты спутника $80\,\text{км}$.
}
\solutionspace{120pt}

\tasknumber{14}%
\task{%
    При аэрофотосъемках используется фотоаппарат, объектив которого
    имеет фокусиое расстояние $12\,\text{cм}$.
    Разрешающая способность пленки $0{,}015\,\text{мм}$.
    На какой высоте должен лететь самолет, чтобы на фотографии можно
    было различить листья деревьев размером $5\,\text{cм}$?
    При какой скорости самолета изображение не будет размытым,
    если время зкспозиции $1\,\text{мс}$?
}

\variantsplitter

\addpersonalvariant{Михаил Ярошевский}

\tasknumber{1}%
\task{%
    Найти оптическую силу собирающей линзы, если действительное изображение предмета,
    помещённого в $35\,\text{см}$ от линзы, получается на расстоянии $20\,\text{см}$ от неё.
}
\answer{%
    $D = \frac 1F = \frac 1a + \frac 1b = \frac 1{35\,\text{см}} + \frac 1{20\,\text{см}} \approx 7{,}86\,\text{дптр}$
}
\solutionspace{180pt}

\tasknumber{2}%
\task{%
    Найти увеличение изображения, если изображение предмета, находящегося
    на расстоянии $25\,\text{см}$ от линзы, получается на расстоянии $30\,\text{см}$ от неё.
}
\answer{%
    $\Gamma = \frac ba = \frac {30\,\text{см}}{25\,\text{см}} \approx 1{,}20$
}
\solutionspace{180pt}

\tasknumber{3}%
\task{%
    Расстояние от предмета до линзы $12\,\text{см}$, а от линзы до мнимого изображения $30\,\text{см}$.
    Чему равно фокусное расстояние линзы?
}
\answer{%
    $\pm \frac 1F = \frac 1a - \frac 1b \implies F = \frac{a b}{\abs{b - a}} \approx 20\,\text{см}$
}
\solutionspace{180pt}

\tasknumber{4}%
\task{%
    Две тонкие собирающие линзы с фокусными расстояниями $18\,\text{см}$ и $20\,\text{см}$ сложены вместе.
    Чему равно фокусное расстояние такой оптической системы?
}
\answer{%
    $\frac 1{f_1} = \frac 1a + \frac 1b; \frac 1{f_2} = - \frac 1b + \frac 1c \implies \frac 1{f_1} + \frac 1{f_2} = \frac 1a + \frac 1c \implies f' = \frac 1{\frac 1{f_1} + \frac 1{f_2}} = \frac{f_1 f_2}{f_1 + f_2} \approx 9{,}5\,\text{см}$
}
\solutionspace{180pt}

\tasknumber{5}%
\task{%
    Линейные размеры прямого изображения предмета, полученного в собирающей линзе,
    в два раза больше линейных размеров предмета.
    Зная, что предмет находится на $35\,\text{см}$ ближе к линзе,
    чем его изображение, найти оптическую силу линзы.
}
\answer{%
    \begin{align*}
    D &= \frac 1F = \frac 1a + \frac 1b, \qquad \Gamma = \frac ba, \qquad b - a = \ell \implies b = \Gamma a \implies \Gamma a - a = \ell \implies  \\
    a &= \frac {\ell}{\Gamma - 1} \implies b = \frac {{\ell} \Gamma}{\Gamma - 1} \implies  \\
    D &= \frac {\Gamma - 1}\ell + \frac {\Gamma - 1}{\ell \Gamma} = \frac 1\ell \cdot \cbr{\Gamma - 1 + \frac {\Gamma - 1}{\Gamma} } =\frac 1\ell \cdot \cbr{\Gamma - \frac 1\Gamma} \approx 4{,}3\,\text{дптр}.
    \end{align*}
}
\solutionspace{180pt}

\tasknumber{6}%
\task{%
    Оптическая сила объектива фотоаппарата равна $6\,\text{дптр}$.
    При фотографировании чертежа с расстояния $0{,}8\,\text{м}$ площадь изображения
    чертежа на фотопластинке оказалась равной $9\,\text{см}^{2}$.
    Какова площадь самого чертежа? Ответ выразите в квадратных сантиметрах.
}
\solutionspace{180pt}

\tasknumber{7}%
\task{%
    В каком месте на главной оптической оси двояковыгнутой линзы
    нужно поместить точечный источник света,
    чтобы его изображение оказалось в главном фокусе линзы?
}
\answer{%
    $\text{на половине фокусного расстояния}$
}
\solutionspace{120pt}

\tasknumber{8}%
\task{%
    Предмет высотой $h = 50\,\text{см}$ находится на расстоянии $d = 0{,}8\,\text{м}$
    от вертикально расположенной рассеивающей линзы с фокусным расстоянием $F = 25\,\text{см}$.
    Где находится изображение предмета? Определите тип изображения и его высоту.
}
\solutionspace{120pt}

\tasknumber{9}%
\task{%
    На каком расстоянии от двояковыпуклой линзы с оптической силой $D = 1{,}5\,\text{дптр}$
    надо поместить предмет, чтобы его изображение получилось на расстоянии $1{,}5\,\text{м}$ от линзы?
}
\solutionspace{120pt}

\tasknumber{10}%
\task{%
    Предмет в виде отрезка длиной $\ell$ расположен вдоль оптической оси
    собирающей линзы с фокусным расстоянием $F$.
    Середина отрезка расположена
    на расстоянии $a$ от линзы, которая даёт действительное изображение
    всех точек предмета.
    Определить продольное увеличение предмета.
}
\answer{%
    \begin{align*}
    \frac 1{a + \frac \ell 2} &+ \frac 1b = \frac 1F \implies b = \frac{F\cbr{a + \frac \ell 2}}{a + \frac \ell 2 - F} \\
    \frac 1{a - \frac \ell 2} &+ \frac 1c = \frac 1F \implies c = \frac{F\cbr{a - \frac \ell 2}}{a - \frac \ell 2 - F} \\
    \abs{b - c} &= \abs{\frac{F\cbr{a + \frac \ell 2}}{a + \frac \ell 2 - F} - \frac{F\cbr{a - \frac \ell 2}}{a - \frac \ell 2 - F}}= F\abs{\frac{\cbr{a + \frac \ell 2}\cbr{a - \frac \ell 2 - F} - \cbr{a - \frac \ell 2}\cbr{a + \frac \ell 2 - F}}{ \cbr{a + \frac \ell 2 - F} \cbr{a - \frac \ell 2 - F} }} =  \\
    &= F\abs{\frac{a^2 - \frac {a\ell} 2 - Fa + \frac {a\ell} 2 - \frac {\ell^2} 4 - \frac {F\ell}2 - a^2 - \frac {a\ell}2 + aF + \frac {a\ell}2 + \frac {\ell^2} 4 - \frac {F\ell} 2}{\cbr{a + \frac \ell 2 - F} \cbr{a - \frac \ell 2 - F} }} = \\
    &= F\frac{F\ell}{\sqr{a-F} - \frac {\ell^2}4} = \frac{F^2\ell}{\sqr{a-F} - \frac {\ell^2}4}\implies \Gamma = \frac{\abs{b - c}}\ell = \frac{F^2}{\sqr{a-F} - \frac {\ell^2}4}.
    \end{align*}
}
\solutionspace{120pt}

\tasknumber{11}%
\task{%
    На экране с помощью тонкой линзы получено изображение предмета
    с увеличением $2$.
    Предмет передвинули на $4\,\text{см}$.
    Для того, чтобы получить резкое изображение, пришлось передвинуть экран.
    При этом увеличение оказалось равным $8$.
    На какое расстояние
    пришлось передвинуть экран?
}
\solutionspace{120pt}

\tasknumber{12}%
\task{%
    Тонкая собирающая линза дает изображение предмета на экране высотой $H_1$,
    и $H_2$, при двух положениях линзы между предметом и экраном.
    Расстояние между ними неизменно.
    Чему равна высота предмета $h$?
}
\answer{%
    $h = \sqrt{H_1 H_2}$
}
\solutionspace{120pt}

\tasknumber{13}%
\task{%
    Какие предметы можно рассмотреть на фотографии, сделанной со спутника,
    если разрешающая способность пленки $0{,}02\,\text{мм}$? Каким должно быть
    время экспозиции $\tau$ чтобы полностью использовать возможности пленки?
    Фокусное расстояние объектива используемого фотоаппарата $10\,\text{cм}$,
    высота орбиты спутника $120\,\text{км}$.
}
\solutionspace{120pt}

\tasknumber{14}%
\task{%
    При аэрофотосъемках используется фотоаппарат, объектив которого
    имеет фокусиое расстояние $8\,\text{cм}$.
    Разрешающая способность пленки $0{,}015\,\text{мм}$.
    На какой высоте должен лететь самолет, чтобы на фотографии можно
    было различить листья деревьев размером $5\,\text{cм}$?
    При какой скорости самолета изображение не будет размытым,
    если время зкспозиции $1\,\text{мс}$?
}

\variantsplitter

\addpersonalvariant{Алексей Алимпиев}

\tasknumber{1}%
\task{%
    Найти оптическую силу собирающей линзы, если действительное изображение предмета,
    помещённого в $55\,\text{см}$ от линзы, получается на расстоянии $40\,\text{см}$ от неё.
}
\answer{%
    $D = \frac 1F = \frac 1a + \frac 1b = \frac 1{55\,\text{см}} + \frac 1{40\,\text{см}} \approx 4{,}32\,\text{дптр}$
}
\solutionspace{180pt}

\tasknumber{2}%
\task{%
    Найти увеличение изображения, если изображение предмета, находящегося
    на расстоянии $25\,\text{см}$ от линзы, получается на расстоянии $18\,\text{см}$ от неё.
}
\answer{%
    $\Gamma = \frac ba = \frac {18\,\text{см}}{25\,\text{см}} \approx 0{,}7$
}
\solutionspace{180pt}

\tasknumber{3}%
\task{%
    Расстояние от предмета до линзы $12\,\text{см}$, а от линзы до мнимого изображения $25\,\text{см}$.
    Чему равно фокусное расстояние линзы?
}
\answer{%
    $\pm \frac 1F = \frac 1a - \frac 1b \implies F = \frac{a b}{\abs{b - a}} \approx 23{,}1\,\text{см}$
}
\solutionspace{180pt}

\tasknumber{4}%
\task{%
    Две тонкие собирающие линзы с фокусными расстояниями $18\,\text{см}$ и $20\,\text{см}$ сложены вместе.
    Чему равно фокусное расстояние такой оптической системы?
}
\answer{%
    $\frac 1{f_1} = \frac 1a + \frac 1b; \frac 1{f_2} = - \frac 1b + \frac 1c \implies \frac 1{f_1} + \frac 1{f_2} = \frac 1a + \frac 1c \implies f' = \frac 1{\frac 1{f_1} + \frac 1{f_2}} = \frac{f_1 f_2}{f_1 + f_2} \approx 9{,}5\,\text{см}$
}
\solutionspace{180pt}

\tasknumber{5}%
\task{%
    Линейные размеры прямого изображения предмета, полученного в собирающей линзе,
    в три раза больше линейных размеров предмета.
    Зная, что предмет находится на $35\,\text{см}$ ближе к линзе,
    чем его изображение, найти оптическую силу линзы.
}
\answer{%
    \begin{align*}
    D &= \frac 1F = \frac 1a + \frac 1b, \qquad \Gamma = \frac ba, \qquad b - a = \ell \implies b = \Gamma a \implies \Gamma a - a = \ell \implies  \\
    a &= \frac {\ell}{\Gamma - 1} \implies b = \frac {{\ell} \Gamma}{\Gamma - 1} \implies  \\
    D &= \frac {\Gamma - 1}\ell + \frac {\Gamma - 1}{\ell \Gamma} = \frac 1\ell \cdot \cbr{\Gamma - 1 + \frac {\Gamma - 1}{\Gamma} } =\frac 1\ell \cdot \cbr{\Gamma - \frac 1\Gamma} \approx 7{,}6\,\text{дптр}.
    \end{align*}
}
\solutionspace{180pt}

\tasknumber{6}%
\task{%
    Оптическая сила объектива фотоаппарата равна $6\,\text{дптр}$.
    При фотографировании чертежа с расстояния $1{,}2\,\text{м}$ площадь изображения
    чертежа на фотопластинке оказалась равной $9\,\text{см}^{2}$.
    Какова площадь самого чертежа? Ответ выразите в квадратных сантиметрах.
}
\solutionspace{180pt}

\tasknumber{7}%
\task{%
    В каком месте на главной оптической оси двояковыгнутой линзы
    нужно поместить точечный источник света,
    чтобы его изображение оказалось в главном фокусе линзы?
}
\answer{%
    $\text{на половине фокусного расстояния}$
}
\solutionspace{120pt}

\tasknumber{8}%
\task{%
    Предмет высотой $h = 40\,\text{см}$ находится на расстоянии $d = 0{,}8\,\text{м}$
    от вертикально расположенной рассеивающей линзы с фокусным расстоянием $F = -15\,\text{см}$.
    Где находится изображение предмета? Определите тип изображения и его высоту.
}
\solutionspace{120pt}

\tasknumber{9}%
\task{%
    На каком расстоянии от двояковыпуклой линзы с оптической силой $D = 1{,}5\,\text{дптр}$
    надо поместить предмет, чтобы его изображение получилось на расстоянии $2\,\text{м}$ от линзы?
}
\solutionspace{120pt}

\tasknumber{10}%
\task{%
    Предмет в виде отрезка длиной $\ell$ расположен вдоль оптической оси
    собирающей линзы с фокусным расстоянием $F$.
    Середина отрезка расположена
    на расстоянии $a$ от линзы, которая даёт действительное изображение
    всех точек предмета.
    Определить продольное увеличение предмета.
}
\answer{%
    \begin{align*}
    \frac 1{a + \frac \ell 2} &+ \frac 1b = \frac 1F \implies b = \frac{F\cbr{a + \frac \ell 2}}{a + \frac \ell 2 - F} \\
    \frac 1{a - \frac \ell 2} &+ \frac 1c = \frac 1F \implies c = \frac{F\cbr{a - \frac \ell 2}}{a - \frac \ell 2 - F} \\
    \abs{b - c} &= \abs{\frac{F\cbr{a + \frac \ell 2}}{a + \frac \ell 2 - F} - \frac{F\cbr{a - \frac \ell 2}}{a - \frac \ell 2 - F}}= F\abs{\frac{\cbr{a + \frac \ell 2}\cbr{a - \frac \ell 2 - F} - \cbr{a - \frac \ell 2}\cbr{a + \frac \ell 2 - F}}{ \cbr{a + \frac \ell 2 - F} \cbr{a - \frac \ell 2 - F} }} =  \\
    &= F\abs{\frac{a^2 - \frac {a\ell} 2 - Fa + \frac {a\ell} 2 - \frac {\ell^2} 4 - \frac {F\ell}2 - a^2 - \frac {a\ell}2 + aF + \frac {a\ell}2 + \frac {\ell^2} 4 - \frac {F\ell} 2}{\cbr{a + \frac \ell 2 - F} \cbr{a - \frac \ell 2 - F} }} = \\
    &= F\frac{F\ell}{\sqr{a-F} - \frac {\ell^2}4} = \frac{F^2\ell}{\sqr{a-F} - \frac {\ell^2}4}\implies \Gamma = \frac{\abs{b - c}}\ell = \frac{F^2}{\sqr{a-F} - \frac {\ell^2}4}.
    \end{align*}
}
\solutionspace{120pt}

\tasknumber{11}%
\task{%
    На экране с помощью тонкой линзы получено изображение предмета
    с увеличением $4$.
    Предмет передвинули на $6\,\text{см}$.
    Для того, чтобы получить резкое изображение, пришлось передвинуть экран.
    При этом увеличение оказалось равным $8$.
    На какое расстояние
    пришлось передвинуть экран?
}
\solutionspace{120pt}

\tasknumber{12}%
\task{%
    Тонкая собирающая линза дает изображение предмета на экране высотой $H_1$,
    и $H_2$, при двух положениях линзы между предметом и экраном.
    Расстояние между ними неизменно.
    Чему равна высота предмета $h$?
}
\answer{%
    $h = \sqrt{H_1 H_2}$
}
\solutionspace{120pt}

\tasknumber{13}%
\task{%
    Какие предметы можно рассмотреть на фотографии, сделанной со спутника,
    если разрешающая способность пленки $0{,}010\,\text{мм}$? Каким должно быть
    время экспозиции $\tau$ чтобы полностью использовать возможности пленки?
    Фокусное расстояние объектива используемого фотоаппарата $20\,\text{cм}$,
    высота орбиты спутника $100\,\text{км}$.
}
\solutionspace{120pt}

\tasknumber{14}%
\task{%
    При аэрофотосъемках используется фотоаппарат, объектив которого
    имеет фокусиое расстояние $8\,\text{cм}$.
    Разрешающая способность пленки $0{,}02\,\text{мм}$.
    На какой высоте должен лететь самолет, чтобы на фотографии можно
    было различить листья деревьев размером $6\,\text{cм}$?
    При какой скорости самолета изображение не будет размытым,
    если время зкспозиции $1\,\text{мс}$?
}

\variantsplitter

\addpersonalvariant{Евгений Васин}

\tasknumber{1}%
\task{%
    Найти оптическую силу собирающей линзы, если действительное изображение предмета,
    помещённого в $35\,\text{см}$ от линзы, получается на расстоянии $20\,\text{см}$ от неё.
}
\answer{%
    $D = \frac 1F = \frac 1a + \frac 1b = \frac 1{35\,\text{см}} + \frac 1{20\,\text{см}} \approx 7{,}86\,\text{дптр}$
}
\solutionspace{180pt}

\tasknumber{2}%
\task{%
    Найти увеличение изображения, если изображение предмета, находящегося
    на расстоянии $25\,\text{см}$ от линзы, получается на расстоянии $30\,\text{см}$ от неё.
}
\answer{%
    $\Gamma = \frac ba = \frac {30\,\text{см}}{25\,\text{см}} \approx 1{,}20$
}
\solutionspace{180pt}

\tasknumber{3}%
\task{%
    Расстояние от предмета до линзы $12\,\text{см}$, а от линзы до мнимого изображения $30\,\text{см}$.
    Чему равно фокусное расстояние линзы?
}
\answer{%
    $\pm \frac 1F = \frac 1a - \frac 1b \implies F = \frac{a b}{\abs{b - a}} \approx 20\,\text{см}$
}
\solutionspace{180pt}

\tasknumber{4}%
\task{%
    Две тонкие собирающие линзы с фокусными расстояниями $12\,\text{см}$ и $20\,\text{см}$ сложены вместе.
    Чему равно фокусное расстояние такой оптической системы?
}
\answer{%
    $\frac 1{f_1} = \frac 1a + \frac 1b; \frac 1{f_2} = - \frac 1b + \frac 1c \implies \frac 1{f_1} + \frac 1{f_2} = \frac 1a + \frac 1c \implies f' = \frac 1{\frac 1{f_1} + \frac 1{f_2}} = \frac{f_1 f_2}{f_1 + f_2} \approx 7{,}5\,\text{см}$
}
\solutionspace{180pt}

\tasknumber{5}%
\task{%
    Линейные размеры прямого изображения предмета, полученного в собирающей линзе,
    в три раза больше линейных размеров предмета.
    Зная, что предмет находится на $30\,\text{см}$ ближе к линзе,
    чем его изображение, найти оптическую силу линзы.
}
\answer{%
    \begin{align*}
    D &= \frac 1F = \frac 1a + \frac 1b, \qquad \Gamma = \frac ba, \qquad b - a = \ell \implies b = \Gamma a \implies \Gamma a - a = \ell \implies  \\
    a &= \frac {\ell}{\Gamma - 1} \implies b = \frac {{\ell} \Gamma}{\Gamma - 1} \implies  \\
    D &= \frac {\Gamma - 1}\ell + \frac {\Gamma - 1}{\ell \Gamma} = \frac 1\ell \cdot \cbr{\Gamma - 1 + \frac {\Gamma - 1}{\Gamma} } =\frac 1\ell \cdot \cbr{\Gamma - \frac 1\Gamma} \approx 8{,}9\,\text{дптр}.
    \end{align*}
}
\solutionspace{180pt}

\tasknumber{6}%
\task{%
    Оптическая сила объектива фотоаппарата равна $5\,\text{дптр}$.
    При фотографировании чертежа с расстояния $0{,}8\,\text{м}$ площадь изображения
    чертежа на фотопластинке оказалась равной $9\,\text{см}^{2}$.
    Какова площадь самого чертежа? Ответ выразите в квадратных сантиметрах.
}
\solutionspace{180pt}

\tasknumber{7}%
\task{%
    В каком месте на главной оптической оси двояковыпуклой линзы
    нужно поместить точечный источник света,
    чтобы его изображение оказалось в главном фокусе линзы?
}
\answer{%
    $\text{для мнимого - на половине фокусного, для действительного - на бесконечности}$
}
\solutionspace{120pt}

\tasknumber{8}%
\task{%
    Предмет высотой $h = 30\,\text{см}$ находится на расстоянии $d = 1\,\text{м}$
    от вертикально расположенной рассеивающей линзы с фокусным расстоянием $F = -15\,\text{см}$.
    Где находится изображение предмета? Определите тип изображения и его высоту.
}
\solutionspace{120pt}

\tasknumber{9}%
\task{%
    На каком расстоянии от двояковыпуклой линзы с оптической силой $D = 1{,}5\,\text{дптр}$
    надо поместить предмет, чтобы его изображение получилось на расстоянии $1{,}5\,\text{м}$ от линзы?
}
\solutionspace{120pt}

\tasknumber{10}%
\task{%
    Предмет в виде отрезка длиной $\ell$ расположен вдоль оптической оси
    собирающей линзы с фокусным расстоянием $F$.
    Середина отрезка расположена
    на расстоянии $a$ от линзы, которая даёт действительное изображение
    всех точек предмета.
    Определить продольное увеличение предмета.
}
\answer{%
    \begin{align*}
    \frac 1{a + \frac \ell 2} &+ \frac 1b = \frac 1F \implies b = \frac{F\cbr{a + \frac \ell 2}}{a + \frac \ell 2 - F} \\
    \frac 1{a - \frac \ell 2} &+ \frac 1c = \frac 1F \implies c = \frac{F\cbr{a - \frac \ell 2}}{a - \frac \ell 2 - F} \\
    \abs{b - c} &= \abs{\frac{F\cbr{a + \frac \ell 2}}{a + \frac \ell 2 - F} - \frac{F\cbr{a - \frac \ell 2}}{a - \frac \ell 2 - F}}= F\abs{\frac{\cbr{a + \frac \ell 2}\cbr{a - \frac \ell 2 - F} - \cbr{a - \frac \ell 2}\cbr{a + \frac \ell 2 - F}}{ \cbr{a + \frac \ell 2 - F} \cbr{a - \frac \ell 2 - F} }} =  \\
    &= F\abs{\frac{a^2 - \frac {a\ell} 2 - Fa + \frac {a\ell} 2 - \frac {\ell^2} 4 - \frac {F\ell}2 - a^2 - \frac {a\ell}2 + aF + \frac {a\ell}2 + \frac {\ell^2} 4 - \frac {F\ell} 2}{\cbr{a + \frac \ell 2 - F} \cbr{a - \frac \ell 2 - F} }} = \\
    &= F\frac{F\ell}{\sqr{a-F} - \frac {\ell^2}4} = \frac{F^2\ell}{\sqr{a-F} - \frac {\ell^2}4}\implies \Gamma = \frac{\abs{b - c}}\ell = \frac{F^2}{\sqr{a-F} - \frac {\ell^2}4}.
    \end{align*}
}
\solutionspace{120pt}

\tasknumber{11}%
\task{%
    На экране с помощью тонкой линзы получено изображение предмета
    с увеличением $4$.
    Предмет передвинули на $4\,\text{см}$.
    Для того, чтобы получить резкое изображение, пришлось передвинуть экран.
    При этом увеличение оказалось равным $6$.
    На какое расстояние
    пришлось передвинуть экран?
}
\solutionspace{120pt}

\tasknumber{12}%
\task{%
    Тонкая собирающая линза дает изображение предмета на экране высотой $H_1$,
    и $H_2$, при двух положениях линзы между предметом и экраном.
    Расстояние между ними неизменно.
    Чему равна высота предмета $h$?
}
\answer{%
    $h = \sqrt{H_1 H_2}$
}
\solutionspace{120pt}

\tasknumber{13}%
\task{%
    Какие предметы можно рассмотреть на фотографии, сделанной со спутника,
    если разрешающая способность пленки $0{,}010\,\text{мм}$? Каким должно быть
    время экспозиции $\tau$ чтобы полностью использовать возможности пленки?
    Фокусное расстояние объектива используемого фотоаппарата $20\,\text{cм}$,
    высота орбиты спутника $80\,\text{км}$.
}
\solutionspace{120pt}

\tasknumber{14}%
\task{%
    При аэрофотосъемках используется фотоаппарат, объектив которого
    имеет фокусиое расстояние $12\,\text{cм}$.
    Разрешающая способность пленки $0{,}015\,\text{мм}$.
    На какой высоте должен лететь самолет, чтобы на фотографии можно
    было различить листья деревьев размером $4\,\text{cм}$?
    При какой скорости самолета изображение не будет размытым,
    если время зкспозиции $2\,\text{мс}$?
}

\variantsplitter

\addpersonalvariant{Вячеслав Волохов}

\tasknumber{1}%
\task{%
    Найти оптическую силу собирающей линзы, если действительное изображение предмета,
    помещённого в $15\,\text{см}$ от линзы, получается на расстоянии $30\,\text{см}$ от неё.
}
\answer{%
    $D = \frac 1F = \frac 1a + \frac 1b = \frac 1{15\,\text{см}} + \frac 1{30\,\text{см}} \approx 10\,\text{дптр}$
}
\solutionspace{180pt}

\tasknumber{2}%
\task{%
    Найти увеличение изображения, если изображение предмета, находящегося
    на расстоянии $20\,\text{см}$ от линзы, получается на расстоянии $30\,\text{см}$ от неё.
}
\answer{%
    $\Gamma = \frac ba = \frac {30\,\text{см}}{20\,\text{см}} \approx 1{,}50$
}
\solutionspace{180pt}

\tasknumber{3}%
\task{%
    Расстояние от предмета до линзы $10\,\text{см}$, а от линзы до мнимого изображения $30\,\text{см}$.
    Чему равно фокусное расстояние линзы?
}
\answer{%
    $\pm \frac 1F = \frac 1a - \frac 1b \implies F = \frac{a b}{\abs{b - a}} \approx 15\,\text{см}$
}
\solutionspace{180pt}

\tasknumber{4}%
\task{%
    Две тонкие собирающие линзы с фокусными расстояниями $12\,\text{см}$ и $20\,\text{см}$ сложены вместе.
    Чему равно фокусное расстояние такой оптической системы?
}
\answer{%
    $\frac 1{f_1} = \frac 1a + \frac 1b; \frac 1{f_2} = - \frac 1b + \frac 1c \implies \frac 1{f_1} + \frac 1{f_2} = \frac 1a + \frac 1c \implies f' = \frac 1{\frac 1{f_1} + \frac 1{f_2}} = \frac{f_1 f_2}{f_1 + f_2} \approx 7{,}5\,\text{см}$
}
\solutionspace{180pt}

\tasknumber{5}%
\task{%
    Линейные размеры прямого изображения предмета, полученного в собирающей линзе,
    в три раза больше линейных размеров предмета.
    Зная, что предмет находится на $25\,\text{см}$ ближе к линзе,
    чем его изображение, найти оптическую силу линзы.
}
\answer{%
    \begin{align*}
    D &= \frac 1F = \frac 1a + \frac 1b, \qquad \Gamma = \frac ba, \qquad b - a = \ell \implies b = \Gamma a \implies \Gamma a - a = \ell \implies  \\
    a &= \frac {\ell}{\Gamma - 1} \implies b = \frac {{\ell} \Gamma}{\Gamma - 1} \implies  \\
    D &= \frac {\Gamma - 1}\ell + \frac {\Gamma - 1}{\ell \Gamma} = \frac 1\ell \cdot \cbr{\Gamma - 1 + \frac {\Gamma - 1}{\Gamma} } =\frac 1\ell \cdot \cbr{\Gamma - \frac 1\Gamma} \approx 10{,}7\,\text{дптр}.
    \end{align*}
}
\solutionspace{180pt}

\tasknumber{6}%
\task{%
    Оптическая сила объектива фотоаппарата равна $4\,\text{дптр}$.
    При фотографировании чертежа с расстояния $1{,}1\,\text{м}$ площадь изображения
    чертежа на фотопластинке оказалась равной $16\,\text{см}^{2}$.
    Какова площадь самого чертежа? Ответ выразите в квадратных сантиметрах.
}
\solutionspace{180pt}

\tasknumber{7}%
\task{%
    В каком месте на главной оптической оси двояковыпуклой линзы
    нужно поместить точечный источник света,
    чтобы его изображение оказалось в главном фокусе линзы?
}
\answer{%
    $\text{для мнимого - на половине фокусного, для действительного - на бесконечности}$
}
\solutionspace{120pt}

\tasknumber{8}%
\task{%
    Предмет высотой $h = 30\,\text{см}$ находится на расстоянии $d = 1{,}2\,\text{м}$
    от вертикально расположенной рассеивающей линзы с фокусным расстоянием $F = -15\,\text{см}$.
    Где находится изображение предмета? Определите тип изображения и его высоту.
}
\solutionspace{120pt}

\tasknumber{9}%
\task{%
    На каком расстоянии от двояковыпуклой линзы с оптической силой $D = 2{,}5\,\text{дптр}$
    надо поместить предмет, чтобы его изображение получилось на расстоянии $2\,\text{м}$ от линзы?
}
\solutionspace{120pt}

\tasknumber{10}%
\task{%
    Предмет в виде отрезка длиной $\ell$ расположен вдоль оптической оси
    собирающей линзы с фокусным расстоянием $F$.
    Середина отрезка расположена
    на расстоянии $a$ от линзы, которая даёт действительное изображение
    всех точек предмета.
    Определить продольное увеличение предмета.
}
\answer{%
    \begin{align*}
    \frac 1{a + \frac \ell 2} &+ \frac 1b = \frac 1F \implies b = \frac{F\cbr{a + \frac \ell 2}}{a + \frac \ell 2 - F} \\
    \frac 1{a - \frac \ell 2} &+ \frac 1c = \frac 1F \implies c = \frac{F\cbr{a - \frac \ell 2}}{a - \frac \ell 2 - F} \\
    \abs{b - c} &= \abs{\frac{F\cbr{a + \frac \ell 2}}{a + \frac \ell 2 - F} - \frac{F\cbr{a - \frac \ell 2}}{a - \frac \ell 2 - F}}= F\abs{\frac{\cbr{a + \frac \ell 2}\cbr{a - \frac \ell 2 - F} - \cbr{a - \frac \ell 2}\cbr{a + \frac \ell 2 - F}}{ \cbr{a + \frac \ell 2 - F} \cbr{a - \frac \ell 2 - F} }} =  \\
    &= F\abs{\frac{a^2 - \frac {a\ell} 2 - Fa + \frac {a\ell} 2 - \frac {\ell^2} 4 - \frac {F\ell}2 - a^2 - \frac {a\ell}2 + aF + \frac {a\ell}2 + \frac {\ell^2} 4 - \frac {F\ell} 2}{\cbr{a + \frac \ell 2 - F} \cbr{a - \frac \ell 2 - F} }} = \\
    &= F\frac{F\ell}{\sqr{a-F} - \frac {\ell^2}4} = \frac{F^2\ell}{\sqr{a-F} - \frac {\ell^2}4}\implies \Gamma = \frac{\abs{b - c}}\ell = \frac{F^2}{\sqr{a-F} - \frac {\ell^2}4}.
    \end{align*}
}
\solutionspace{120pt}

\tasknumber{11}%
\task{%
    На экране с помощью тонкой линзы получено изображение предмета
    с увеличением $4$.
    Предмет передвинули на $6\,\text{см}$.
    Для того, чтобы получить резкое изображение, пришлось передвинуть экран.
    При этом увеличение оказалось равным $8$.
    На какое расстояние
    пришлось передвинуть экран?
}
\solutionspace{120pt}

\tasknumber{12}%
\task{%
    Тонкая собирающая линза дает изображение предмета на экране высотой $H_1$,
    и $H_2$, при двух положениях линзы между предметом и экраном.
    Расстояние между ними неизменно.
    Чему равна высота предмета $h$?
}
\answer{%
    $h = \sqrt{H_1 H_2}$
}
\solutionspace{120pt}

\tasknumber{13}%
\task{%
    Какие предметы можно рассмотреть на фотографии, сделанной со спутника,
    если разрешающая способность пленки $0{,}010\,\text{мм}$? Каким должно быть
    время экспозиции $\tau$ чтобы полностью использовать возможности пленки?
    Фокусное расстояние объектива используемого фотоаппарата $10\,\text{cм}$,
    высота орбиты спутника $120\,\text{км}$.
}
\solutionspace{120pt}

\tasknumber{14}%
\task{%
    При аэрофотосъемках используется фотоаппарат, объектив которого
    имеет фокусиое расстояние $8\,\text{cм}$.
    Разрешающая способность пленки $0{,}015\,\text{мм}$.
    На какой высоте должен лететь самолет, чтобы на фотографии можно
    было различить листья деревьев размером $6\,\text{cм}$?
    При какой скорости самолета изображение не будет размытым,
    если время зкспозиции $2\,\text{мс}$?
}

\variantsplitter

\addpersonalvariant{Герман Говоров}

\tasknumber{1}%
\task{%
    Найти оптическую силу собирающей линзы, если действительное изображение предмета,
    помещённого в $35\,\text{см}$ от линзы, получается на расстоянии $20\,\text{см}$ от неё.
}
\answer{%
    $D = \frac 1F = \frac 1a + \frac 1b = \frac 1{35\,\text{см}} + \frac 1{20\,\text{см}} \approx 7{,}86\,\text{дптр}$
}
\solutionspace{180pt}

\tasknumber{2}%
\task{%
    Найти увеличение изображения, если изображение предмета, находящегося
    на расстоянии $25\,\text{см}$ от линзы, получается на расстоянии $18\,\text{см}$ от неё.
}
\answer{%
    $\Gamma = \frac ba = \frac {18\,\text{см}}{25\,\text{см}} \approx 0{,}7$
}
\solutionspace{180pt}

\tasknumber{3}%
\task{%
    Расстояние от предмета до линзы $12\,\text{см}$, а от линзы до мнимого изображения $25\,\text{см}$.
    Чему равно фокусное расстояние линзы?
}
\answer{%
    $\pm \frac 1F = \frac 1a - \frac 1b \implies F = \frac{a b}{\abs{b - a}} \approx 23{,}1\,\text{см}$
}
\solutionspace{180pt}

\tasknumber{4}%
\task{%
    Две тонкие собирающие линзы с фокусными расстояниями $18\,\text{см}$ и $30\,\text{см}$ сложены вместе.
    Чему равно фокусное расстояние такой оптической системы?
}
\answer{%
    $\frac 1{f_1} = \frac 1a + \frac 1b; \frac 1{f_2} = - \frac 1b + \frac 1c \implies \frac 1{f_1} + \frac 1{f_2} = \frac 1a + \frac 1c \implies f' = \frac 1{\frac 1{f_1} + \frac 1{f_2}} = \frac{f_1 f_2}{f_1 + f_2} \approx 11{,}2\,\text{см}$
}
\solutionspace{180pt}

\tasknumber{5}%
\task{%
    Линейные размеры прямого изображения предмета, полученного в собирающей линзе,
    в три раза больше линейных размеров предмета.
    Зная, что предмет находится на $35\,\text{см}$ ближе к линзе,
    чем его изображение, найти оптическую силу линзы.
}
\answer{%
    \begin{align*}
    D &= \frac 1F = \frac 1a + \frac 1b, \qquad \Gamma = \frac ba, \qquad b - a = \ell \implies b = \Gamma a \implies \Gamma a - a = \ell \implies  \\
    a &= \frac {\ell}{\Gamma - 1} \implies b = \frac {{\ell} \Gamma}{\Gamma - 1} \implies  \\
    D &= \frac {\Gamma - 1}\ell + \frac {\Gamma - 1}{\ell \Gamma} = \frac 1\ell \cdot \cbr{\Gamma - 1 + \frac {\Gamma - 1}{\Gamma} } =\frac 1\ell \cdot \cbr{\Gamma - \frac 1\Gamma} \approx 7{,}6\,\text{дптр}.
    \end{align*}
}
\solutionspace{180pt}

\tasknumber{6}%
\task{%
    Оптическая сила объектива фотоаппарата равна $4\,\text{дптр}$.
    При фотографировании чертежа с расстояния $1{,}2\,\text{м}$ площадь изображения
    чертежа на фотопластинке оказалась равной $4\,\text{см}^{2}$.
    Какова площадь самого чертежа? Ответ выразите в квадратных сантиметрах.
}
\solutionspace{180pt}

\tasknumber{7}%
\task{%
    В каком месте на главной оптической оси двояковыгнутой линзы
    нужно поместить точечный источник света,
    чтобы его изображение оказалось в главном фокусе линзы?
}
\answer{%
    $\text{на половине фокусного расстояния}$
}
\solutionspace{120pt}

\tasknumber{8}%
\task{%
    Предмет высотой $h = 40\,\text{см}$ находится на расстоянии $d = 0{,}8\,\text{м}$
    от вертикально расположенной рассеивающей линзы с фокусным расстоянием $F = -15\,\text{см}$.
    Где находится изображение предмета? Определите тип изображения и его высоту.
}
\solutionspace{120pt}

\tasknumber{9}%
\task{%
    На каком расстоянии от двояковыпуклой линзы с оптической силой $D = 2{,}5\,\text{дптр}$
    надо поместить предмет, чтобы его изображение получилось на расстоянии $2\,\text{м}$ от линзы?
}
\solutionspace{120pt}

\tasknumber{10}%
\task{%
    Предмет в виде отрезка длиной $\ell$ расположен вдоль оптической оси
    собирающей линзы с фокусным расстоянием $F$.
    Середина отрезка расположена
    на расстоянии $a$ от линзы, которая даёт действительное изображение
    всех точек предмета.
    Определить продольное увеличение предмета.
}
\answer{%
    \begin{align*}
    \frac 1{a + \frac \ell 2} &+ \frac 1b = \frac 1F \implies b = \frac{F\cbr{a + \frac \ell 2}}{a + \frac \ell 2 - F} \\
    \frac 1{a - \frac \ell 2} &+ \frac 1c = \frac 1F \implies c = \frac{F\cbr{a - \frac \ell 2}}{a - \frac \ell 2 - F} \\
    \abs{b - c} &= \abs{\frac{F\cbr{a + \frac \ell 2}}{a + \frac \ell 2 - F} - \frac{F\cbr{a - \frac \ell 2}}{a - \frac \ell 2 - F}}= F\abs{\frac{\cbr{a + \frac \ell 2}\cbr{a - \frac \ell 2 - F} - \cbr{a - \frac \ell 2}\cbr{a + \frac \ell 2 - F}}{ \cbr{a + \frac \ell 2 - F} \cbr{a - \frac \ell 2 - F} }} =  \\
    &= F\abs{\frac{a^2 - \frac {a\ell} 2 - Fa + \frac {a\ell} 2 - \frac {\ell^2} 4 - \frac {F\ell}2 - a^2 - \frac {a\ell}2 + aF + \frac {a\ell}2 + \frac {\ell^2} 4 - \frac {F\ell} 2}{\cbr{a + \frac \ell 2 - F} \cbr{a - \frac \ell 2 - F} }} = \\
    &= F\frac{F\ell}{\sqr{a-F} - \frac {\ell^2}4} = \frac{F^2\ell}{\sqr{a-F} - \frac {\ell^2}4}\implies \Gamma = \frac{\abs{b - c}}\ell = \frac{F^2}{\sqr{a-F} - \frac {\ell^2}4}.
    \end{align*}
}
\solutionspace{120pt}

\tasknumber{11}%
\task{%
    На экране с помощью тонкой линзы получено изображение предмета
    с увеличением $2$.
    Предмет передвинули на $10\,\text{см}$.
    Для того, чтобы получить резкое изображение, пришлось передвинуть экран.
    При этом увеличение оказалось равным $8$.
    На какое расстояние
    пришлось передвинуть экран?
}
\solutionspace{120pt}

\tasknumber{12}%
\task{%
    Тонкая собирающая линза дает изображение предмета на экране высотой $H_1$,
    и $H_2$, при двух положениях линзы между предметом и экраном.
    Расстояние между ними неизменно.
    Чему равна высота предмета $h$?
}
\answer{%
    $h = \sqrt{H_1 H_2}$
}
\solutionspace{120pt}

\tasknumber{13}%
\task{%
    Какие предметы можно рассмотреть на фотографии, сделанной со спутника,
    если разрешающая способность пленки $0{,}010\,\text{мм}$? Каким должно быть
    время экспозиции $\tau$ чтобы полностью использовать возможности пленки?
    Фокусное расстояние объектива используемого фотоаппарата $10\,\text{cм}$,
    высота орбиты спутника $100\,\text{км}$.
}
\solutionspace{120pt}

\tasknumber{14}%
\task{%
    При аэрофотосъемках используется фотоаппарат, объектив которого
    имеет фокусиое расстояние $10\,\text{cм}$.
    Разрешающая способность пленки $0{,}010\,\text{мм}$.
    На какой высоте должен лететь самолет, чтобы на фотографии можно
    было различить листья деревьев размером $5\,\text{cм}$?
    При какой скорости самолета изображение не будет размытым,
    если время зкспозиции $2\,\text{мс}$?
}

\variantsplitter

\addpersonalvariant{София Журавлёва}

\tasknumber{1}%
\task{%
    Найти оптическую силу собирающей линзы, если действительное изображение предмета,
    помещённого в $35\,\text{см}$ от линзы, получается на расстоянии $40\,\text{см}$ от неё.
}
\answer{%
    $D = \frac 1F = \frac 1a + \frac 1b = \frac 1{35\,\text{см}} + \frac 1{40\,\text{см}} \approx 5{,}36\,\text{дптр}$
}
\solutionspace{180pt}

\tasknumber{2}%
\task{%
    Найти увеличение изображения, если изображение предмета, находящегося
    на расстоянии $20\,\text{см}$ от линзы, получается на расстоянии $30\,\text{см}$ от неё.
}
\answer{%
    $\Gamma = \frac ba = \frac {30\,\text{см}}{20\,\text{см}} \approx 1{,}50$
}
\solutionspace{180pt}

\tasknumber{3}%
\task{%
    Расстояние от предмета до линзы $10\,\text{см}$, а от линзы до мнимого изображения $30\,\text{см}$.
    Чему равно фокусное расстояние линзы?
}
\answer{%
    $\pm \frac 1F = \frac 1a - \frac 1b \implies F = \frac{a b}{\abs{b - a}} \approx 15\,\text{см}$
}
\solutionspace{180pt}

\tasknumber{4}%
\task{%
    Две тонкие собирающие линзы с фокусными расстояниями $18\,\text{см}$ и $30\,\text{см}$ сложены вместе.
    Чему равно фокусное расстояние такой оптической системы?
}
\answer{%
    $\frac 1{f_1} = \frac 1a + \frac 1b; \frac 1{f_2} = - \frac 1b + \frac 1c \implies \frac 1{f_1} + \frac 1{f_2} = \frac 1a + \frac 1c \implies f' = \frac 1{\frac 1{f_1} + \frac 1{f_2}} = \frac{f_1 f_2}{f_1 + f_2} \approx 11{,}2\,\text{см}$
}
\solutionspace{180pt}

\tasknumber{5}%
\task{%
    Линейные размеры прямого изображения предмета, полученного в собирающей линзе,
    в четыре раза больше линейных размеров предмета.
    Зная, что предмет находится на $20\,\text{см}$ ближе к линзе,
    чем его изображение, найти оптическую силу линзы.
}
\answer{%
    \begin{align*}
    D &= \frac 1F = \frac 1a + \frac 1b, \qquad \Gamma = \frac ba, \qquad b - a = \ell \implies b = \Gamma a \implies \Gamma a - a = \ell \implies  \\
    a &= \frac {\ell}{\Gamma - 1} \implies b = \frac {{\ell} \Gamma}{\Gamma - 1} \implies  \\
    D &= \frac {\Gamma - 1}\ell + \frac {\Gamma - 1}{\ell \Gamma} = \frac 1\ell \cdot \cbr{\Gamma - 1 + \frac {\Gamma - 1}{\Gamma} } =\frac 1\ell \cdot \cbr{\Gamma - \frac 1\Gamma} \approx 18{,}8\,\text{дптр}.
    \end{align*}
}
\solutionspace{180pt}

\tasknumber{6}%
\task{%
    Оптическая сила объектива фотоаппарата равна $5\,\text{дптр}$.
    При фотографировании чертежа с расстояния $1{,}1\,\text{м}$ площадь изображения
    чертежа на фотопластинке оказалась равной $9\,\text{см}^{2}$.
    Какова площадь самого чертежа? Ответ выразите в квадратных сантиметрах.
}
\solutionspace{180pt}

\tasknumber{7}%
\task{%
    В каком месте на главной оптической оси двояковыгнутой линзы
    нужно поместить точечный источник света,
    чтобы его изображение оказалось в главном фокусе линзы?
}
\answer{%
    $\text{на половине фокусного расстояния}$
}
\solutionspace{120pt}

\tasknumber{8}%
\task{%
    Предмет высотой $h = 30\,\text{см}$ находится на расстоянии $d = 1{,}2\,\text{м}$
    от вертикально расположенной рассеивающей линзы с фокусным расстоянием $F = 25\,\text{см}$.
    Где находится изображение предмета? Определите тип изображения и его высоту.
}
\solutionspace{120pt}

\tasknumber{9}%
\task{%
    На каком расстоянии от двояковыпуклой линзы с оптической силой $D = 2{,}5\,\text{дптр}$
    надо поместить предмет, чтобы его изображение получилось на расстоянии $2{,}5\,\text{м}$ от линзы?
}
\solutionspace{120pt}

\tasknumber{10}%
\task{%
    Предмет в виде отрезка длиной $\ell$ расположен вдоль оптической оси
    собирающей линзы с фокусным расстоянием $F$.
    Середина отрезка расположена
    на расстоянии $a$ от линзы, которая даёт действительное изображение
    всех точек предмета.
    Определить продольное увеличение предмета.
}
\answer{%
    \begin{align*}
    \frac 1{a + \frac \ell 2} &+ \frac 1b = \frac 1F \implies b = \frac{F\cbr{a + \frac \ell 2}}{a + \frac \ell 2 - F} \\
    \frac 1{a - \frac \ell 2} &+ \frac 1c = \frac 1F \implies c = \frac{F\cbr{a - \frac \ell 2}}{a - \frac \ell 2 - F} \\
    \abs{b - c} &= \abs{\frac{F\cbr{a + \frac \ell 2}}{a + \frac \ell 2 - F} - \frac{F\cbr{a - \frac \ell 2}}{a - \frac \ell 2 - F}}= F\abs{\frac{\cbr{a + \frac \ell 2}\cbr{a - \frac \ell 2 - F} - \cbr{a - \frac \ell 2}\cbr{a + \frac \ell 2 - F}}{ \cbr{a + \frac \ell 2 - F} \cbr{a - \frac \ell 2 - F} }} =  \\
    &= F\abs{\frac{a^2 - \frac {a\ell} 2 - Fa + \frac {a\ell} 2 - \frac {\ell^2} 4 - \frac {F\ell}2 - a^2 - \frac {a\ell}2 + aF + \frac {a\ell}2 + \frac {\ell^2} 4 - \frac {F\ell} 2}{\cbr{a + \frac \ell 2 - F} \cbr{a - \frac \ell 2 - F} }} = \\
    &= F\frac{F\ell}{\sqr{a-F} - \frac {\ell^2}4} = \frac{F^2\ell}{\sqr{a-F} - \frac {\ell^2}4}\implies \Gamma = \frac{\abs{b - c}}\ell = \frac{F^2}{\sqr{a-F} - \frac {\ell^2}4}.
    \end{align*}
}
\solutionspace{120pt}

\tasknumber{11}%
\task{%
    На экране с помощью тонкой линзы получено изображение предмета
    с увеличением $4$.
    Предмет передвинули на $4\,\text{см}$.
    Для того, чтобы получить резкое изображение, пришлось передвинуть экран.
    При этом увеличение оказалось равным $8$.
    На какое расстояние
    пришлось передвинуть экран?
}
\solutionspace{120pt}

\tasknumber{12}%
\task{%
    Тонкая собирающая линза дает изображение предмета на экране высотой $H_1$,
    и $H_2$, при двух положениях линзы между предметом и экраном.
    Расстояние между ними неизменно.
    Чему равна высота предмета $h$?
}
\answer{%
    $h = \sqrt{H_1 H_2}$
}
\solutionspace{120pt}

\tasknumber{13}%
\task{%
    Какие предметы можно рассмотреть на фотографии, сделанной со спутника,
    если разрешающая способность пленки $0{,}010\,\text{мм}$? Каким должно быть
    время экспозиции $\tau$ чтобы полностью использовать возможности пленки?
    Фокусное расстояние объектива используемого фотоаппарата $10\,\text{cм}$,
    высота орбиты спутника $120\,\text{км}$.
}
\solutionspace{120pt}

\tasknumber{14}%
\task{%
    При аэрофотосъемках используется фотоаппарат, объектив которого
    имеет фокусиое расстояние $12\,\text{cм}$.
    Разрешающая способность пленки $0{,}015\,\text{мм}$.
    На какой высоте должен лететь самолет, чтобы на фотографии можно
    было различить листья деревьев размером $5\,\text{cм}$?
    При какой скорости самолета изображение не будет размытым,
    если время зкспозиции $2\,\text{мс}$?
}

\variantsplitter

\addpersonalvariant{Константин Козлов}

\tasknumber{1}%
\task{%
    Найти оптическую силу собирающей линзы, если действительное изображение предмета,
    помещённого в $15\,\text{см}$ от линзы, получается на расстоянии $40\,\text{см}$ от неё.
}
\answer{%
    $D = \frac 1F = \frac 1a + \frac 1b = \frac 1{15\,\text{см}} + \frac 1{40\,\text{см}} \approx 9{,}17\,\text{дптр}$
}
\solutionspace{180pt}

\tasknumber{2}%
\task{%
    Найти увеличение изображения, если изображение предмета, находящегося
    на расстоянии $25\,\text{см}$ от линзы, получается на расстоянии $30\,\text{см}$ от неё.
}
\answer{%
    $\Gamma = \frac ba = \frac {30\,\text{см}}{25\,\text{см}} \approx 1{,}20$
}
\solutionspace{180pt}

\tasknumber{3}%
\task{%
    Расстояние от предмета до линзы $12\,\text{см}$, а от линзы до мнимого изображения $30\,\text{см}$.
    Чему равно фокусное расстояние линзы?
}
\answer{%
    $\pm \frac 1F = \frac 1a - \frac 1b \implies F = \frac{a b}{\abs{b - a}} \approx 20\,\text{см}$
}
\solutionspace{180pt}

\tasknumber{4}%
\task{%
    Две тонкие собирающие линзы с фокусными расстояниями $18\,\text{см}$ и $30\,\text{см}$ сложены вместе.
    Чему равно фокусное расстояние такой оптической системы?
}
\answer{%
    $\frac 1{f_1} = \frac 1a + \frac 1b; \frac 1{f_2} = - \frac 1b + \frac 1c \implies \frac 1{f_1} + \frac 1{f_2} = \frac 1a + \frac 1c \implies f' = \frac 1{\frac 1{f_1} + \frac 1{f_2}} = \frac{f_1 f_2}{f_1 + f_2} \approx 11{,}2\,\text{см}$
}
\solutionspace{180pt}

\tasknumber{5}%
\task{%
    Линейные размеры прямого изображения предмета, полученного в собирающей линзе,
    в четыре раза больше линейных размеров предмета.
    Зная, что предмет находится на $30\,\text{см}$ ближе к линзе,
    чем его изображение, найти оптическую силу линзы.
}
\answer{%
    \begin{align*}
    D &= \frac 1F = \frac 1a + \frac 1b, \qquad \Gamma = \frac ba, \qquad b - a = \ell \implies b = \Gamma a \implies \Gamma a - a = \ell \implies  \\
    a &= \frac {\ell}{\Gamma - 1} \implies b = \frac {{\ell} \Gamma}{\Gamma - 1} \implies  \\
    D &= \frac {\Gamma - 1}\ell + \frac {\Gamma - 1}{\ell \Gamma} = \frac 1\ell \cdot \cbr{\Gamma - 1 + \frac {\Gamma - 1}{\Gamma} } =\frac 1\ell \cdot \cbr{\Gamma - \frac 1\Gamma} \approx 12{,}5\,\text{дптр}.
    \end{align*}
}
\solutionspace{180pt}

\tasknumber{6}%
\task{%
    Оптическая сила объектива фотоаппарата равна $3\,\text{дптр}$.
    При фотографировании чертежа с расстояния $1{,}1\,\text{м}$ площадь изображения
    чертежа на фотопластинке оказалась равной $9\,\text{см}^{2}$.
    Какова площадь самого чертежа? Ответ выразите в квадратных сантиметрах.
}
\solutionspace{180pt}

\tasknumber{7}%
\task{%
    В каком месте на главной оптической оси двояковыпуклой линзы
    нужно поместить точечный источник света,
    чтобы его изображение оказалось в главном фокусе линзы?
}
\answer{%
    $\text{для мнимого - на половине фокусного, для действительного - на бесконечности}$
}
\solutionspace{120pt}

\tasknumber{8}%
\task{%
    Предмет высотой $h = 40\,\text{см}$ находится на расстоянии $d = 1{,}2\,\text{м}$
    от вертикально расположенной рассеивающей линзы с фокусным расстоянием $F = 25\,\text{см}$.
    Где находится изображение предмета? Определите тип изображения и его высоту.
}
\solutionspace{120pt}

\tasknumber{9}%
\task{%
    На каком расстоянии от двояковыпуклой линзы с оптической силой $D = 2\,\text{дптр}$
    надо поместить предмет, чтобы его изображение получилось на расстоянии $1{,}5\,\text{м}$ от линзы?
}
\solutionspace{120pt}

\tasknumber{10}%
\task{%
    Предмет в виде отрезка длиной $\ell$ расположен вдоль оптической оси
    собирающей линзы с фокусным расстоянием $F$.
    Середина отрезка расположена
    на расстоянии $a$ от линзы, которая даёт действительное изображение
    всех точек предмета.
    Определить продольное увеличение предмета.
}
\answer{%
    \begin{align*}
    \frac 1{a + \frac \ell 2} &+ \frac 1b = \frac 1F \implies b = \frac{F\cbr{a + \frac \ell 2}}{a + \frac \ell 2 - F} \\
    \frac 1{a - \frac \ell 2} &+ \frac 1c = \frac 1F \implies c = \frac{F\cbr{a - \frac \ell 2}}{a - \frac \ell 2 - F} \\
    \abs{b - c} &= \abs{\frac{F\cbr{a + \frac \ell 2}}{a + \frac \ell 2 - F} - \frac{F\cbr{a - \frac \ell 2}}{a - \frac \ell 2 - F}}= F\abs{\frac{\cbr{a + \frac \ell 2}\cbr{a - \frac \ell 2 - F} - \cbr{a - \frac \ell 2}\cbr{a + \frac \ell 2 - F}}{ \cbr{a + \frac \ell 2 - F} \cbr{a - \frac \ell 2 - F} }} =  \\
    &= F\abs{\frac{a^2 - \frac {a\ell} 2 - Fa + \frac {a\ell} 2 - \frac {\ell^2} 4 - \frac {F\ell}2 - a^2 - \frac {a\ell}2 + aF + \frac {a\ell}2 + \frac {\ell^2} 4 - \frac {F\ell} 2}{\cbr{a + \frac \ell 2 - F} \cbr{a - \frac \ell 2 - F} }} = \\
    &= F\frac{F\ell}{\sqr{a-F} - \frac {\ell^2}4} = \frac{F^2\ell}{\sqr{a-F} - \frac {\ell^2}4}\implies \Gamma = \frac{\abs{b - c}}\ell = \frac{F^2}{\sqr{a-F} - \frac {\ell^2}4}.
    \end{align*}
}
\solutionspace{120pt}

\tasknumber{11}%
\task{%
    На экране с помощью тонкой линзы получено изображение предмета
    с увеличением $2$.
    Предмет передвинули на $10\,\text{см}$.
    Для того, чтобы получить резкое изображение, пришлось передвинуть экран.
    При этом увеличение оказалось равным $8$.
    На какое расстояние
    пришлось передвинуть экран?
}
\solutionspace{120pt}

\tasknumber{12}%
\task{%
    Тонкая собирающая линза дает изображение предмета на экране высотой $H_1$,
    и $H_2$, при двух положениях линзы между предметом и экраном.
    Расстояние между ними неизменно.
    Чему равна высота предмета $h$?
}
\answer{%
    $h = \sqrt{H_1 H_2}$
}
\solutionspace{120pt}

\tasknumber{13}%
\task{%
    Какие предметы можно рассмотреть на фотографии, сделанной со спутника,
    если разрешающая способность пленки $0{,}010\,\text{мм}$? Каким должно быть
    время экспозиции $\tau$ чтобы полностью использовать возможности пленки?
    Фокусное расстояние объектива используемого фотоаппарата $10\,\text{cм}$,
    высота орбиты спутника $80\,\text{км}$.
}
\solutionspace{120pt}

\tasknumber{14}%
\task{%
    При аэрофотосъемках используется фотоаппарат, объектив которого
    имеет фокусиое расстояние $10\,\text{cм}$.
    Разрешающая способность пленки $0{,}02\,\text{мм}$.
    На какой высоте должен лететь самолет, чтобы на фотографии можно
    было различить листья деревьев размером $5\,\text{cм}$?
    При какой скорости самолета изображение не будет размытым,
    если время зкспозиции $2\,\text{мс}$?
}

\variantsplitter

\addpersonalvariant{Наталья Кравченко}

\tasknumber{1}%
\task{%
    Найти оптическую силу собирающей линзы, если действительное изображение предмета,
    помещённого в $15\,\text{см}$ от линзы, получается на расстоянии $30\,\text{см}$ от неё.
}
\answer{%
    $D = \frac 1F = \frac 1a + \frac 1b = \frac 1{15\,\text{см}} + \frac 1{30\,\text{см}} \approx 10\,\text{дптр}$
}
\solutionspace{180pt}

\tasknumber{2}%
\task{%
    Найти увеличение изображения, если изображение предмета, находящегося
    на расстоянии $20\,\text{см}$ от линзы, получается на расстоянии $30\,\text{см}$ от неё.
}
\answer{%
    $\Gamma = \frac ba = \frac {30\,\text{см}}{20\,\text{см}} \approx 1{,}50$
}
\solutionspace{180pt}

\tasknumber{3}%
\task{%
    Расстояние от предмета до линзы $10\,\text{см}$, а от линзы до мнимого изображения $30\,\text{см}$.
    Чему равно фокусное расстояние линзы?
}
\answer{%
    $\pm \frac 1F = \frac 1a - \frac 1b \implies F = \frac{a b}{\abs{b - a}} \approx 15\,\text{см}$
}
\solutionspace{180pt}

\tasknumber{4}%
\task{%
    Две тонкие собирающие линзы с фокусными расстояниями $25\,\text{см}$ и $20\,\text{см}$ сложены вместе.
    Чему равно фокусное расстояние такой оптической системы?
}
\answer{%
    $\frac 1{f_1} = \frac 1a + \frac 1b; \frac 1{f_2} = - \frac 1b + \frac 1c \implies \frac 1{f_1} + \frac 1{f_2} = \frac 1a + \frac 1c \implies f' = \frac 1{\frac 1{f_1} + \frac 1{f_2}} = \frac{f_1 f_2}{f_1 + f_2} \approx 11{,}1\,\text{см}$
}
\solutionspace{180pt}

\tasknumber{5}%
\task{%
    Линейные размеры прямого изображения предмета, полученного в собирающей линзе,
    в три раза больше линейных размеров предмета.
    Зная, что предмет находится на $35\,\text{см}$ ближе к линзе,
    чем его изображение, найти оптическую силу линзы.
}
\answer{%
    \begin{align*}
    D &= \frac 1F = \frac 1a + \frac 1b, \qquad \Gamma = \frac ba, \qquad b - a = \ell \implies b = \Gamma a \implies \Gamma a - a = \ell \implies  \\
    a &= \frac {\ell}{\Gamma - 1} \implies b = \frac {{\ell} \Gamma}{\Gamma - 1} \implies  \\
    D &= \frac {\Gamma - 1}\ell + \frac {\Gamma - 1}{\ell \Gamma} = \frac 1\ell \cdot \cbr{\Gamma - 1 + \frac {\Gamma - 1}{\Gamma} } =\frac 1\ell \cdot \cbr{\Gamma - \frac 1\Gamma} \approx 7{,}6\,\text{дптр}.
    \end{align*}
}
\solutionspace{180pt}

\tasknumber{6}%
\task{%
    Оптическая сила объектива фотоаппарата равна $6\,\text{дптр}$.
    При фотографировании чертежа с расстояния $1{,}2\,\text{м}$ площадь изображения
    чертежа на фотопластинке оказалась равной $4\,\text{см}^{2}$.
    Какова площадь самого чертежа? Ответ выразите в квадратных сантиметрах.
}
\solutionspace{180pt}

\tasknumber{7}%
\task{%
    В каком месте на главной оптической оси двояковыпуклой линзы
    нужно поместить точечный источник света,
    чтобы его изображение оказалось в главном фокусе линзы?
}
\answer{%
    $\text{для мнимого - на половине фокусного, для действительного - на бесконечности}$
}
\solutionspace{120pt}

\tasknumber{8}%
\task{%
    Предмет высотой $h = 50\,\text{см}$ находится на расстоянии $d = 0{,}8\,\text{м}$
    от вертикально расположенной рассеивающей линзы с фокусным расстоянием $F = -15\,\text{см}$.
    Где находится изображение предмета? Определите тип изображения и его высоту.
}
\solutionspace{120pt}

\tasknumber{9}%
\task{%
    На каком расстоянии от двояковыпуклой линзы с оптической силой $D = 2\,\text{дптр}$
    надо поместить предмет, чтобы его изображение получилось на расстоянии $1{,}5\,\text{м}$ от линзы?
}
\solutionspace{120pt}

\tasknumber{10}%
\task{%
    Предмет в виде отрезка длиной $\ell$ расположен вдоль оптической оси
    собирающей линзы с фокусным расстоянием $F$.
    Середина отрезка расположена
    на расстоянии $a$ от линзы, которая даёт действительное изображение
    всех точек предмета.
    Определить продольное увеличение предмета.
}
\answer{%
    \begin{align*}
    \frac 1{a + \frac \ell 2} &+ \frac 1b = \frac 1F \implies b = \frac{F\cbr{a + \frac \ell 2}}{a + \frac \ell 2 - F} \\
    \frac 1{a - \frac \ell 2} &+ \frac 1c = \frac 1F \implies c = \frac{F\cbr{a - \frac \ell 2}}{a - \frac \ell 2 - F} \\
    \abs{b - c} &= \abs{\frac{F\cbr{a + \frac \ell 2}}{a + \frac \ell 2 - F} - \frac{F\cbr{a - \frac \ell 2}}{a - \frac \ell 2 - F}}= F\abs{\frac{\cbr{a + \frac \ell 2}\cbr{a - \frac \ell 2 - F} - \cbr{a - \frac \ell 2}\cbr{a + \frac \ell 2 - F}}{ \cbr{a + \frac \ell 2 - F} \cbr{a - \frac \ell 2 - F} }} =  \\
    &= F\abs{\frac{a^2 - \frac {a\ell} 2 - Fa + \frac {a\ell} 2 - \frac {\ell^2} 4 - \frac {F\ell}2 - a^2 - \frac {a\ell}2 + aF + \frac {a\ell}2 + \frac {\ell^2} 4 - \frac {F\ell} 2}{\cbr{a + \frac \ell 2 - F} \cbr{a - \frac \ell 2 - F} }} = \\
    &= F\frac{F\ell}{\sqr{a-F} - \frac {\ell^2}4} = \frac{F^2\ell}{\sqr{a-F} - \frac {\ell^2}4}\implies \Gamma = \frac{\abs{b - c}}\ell = \frac{F^2}{\sqr{a-F} - \frac {\ell^2}4}.
    \end{align*}
}
\solutionspace{120pt}

\tasknumber{11}%
\task{%
    На экране с помощью тонкой линзы получено изображение предмета
    с увеличением $4$.
    Предмет передвинули на $6\,\text{см}$.
    Для того, чтобы получить резкое изображение, пришлось передвинуть экран.
    При этом увеличение оказалось равным $8$.
    На какое расстояние
    пришлось передвинуть экран?
}
\solutionspace{120pt}

\tasknumber{12}%
\task{%
    Тонкая собирающая линза дает изображение предмета на экране высотой $H_1$,
    и $H_2$, при двух положениях линзы между предметом и экраном.
    Расстояние между ними неизменно.
    Чему равна высота предмета $h$?
}
\answer{%
    $h = \sqrt{H_1 H_2}$
}
\solutionspace{120pt}

\tasknumber{13}%
\task{%
    Какие предметы можно рассмотреть на фотографии, сделанной со спутника,
    если разрешающая способность пленки $0{,}02\,\text{мм}$? Каким должно быть
    время экспозиции $\tau$ чтобы полностью использовать возможности пленки?
    Фокусное расстояние объектива используемого фотоаппарата $15\,\text{cм}$,
    высота орбиты спутника $150\,\text{км}$.
}
\solutionspace{120pt}

\tasknumber{14}%
\task{%
    При аэрофотосъемках используется фотоаппарат, объектив которого
    имеет фокусиое расстояние $12\,\text{cм}$.
    Разрешающая способность пленки $0{,}010\,\text{мм}$.
    На какой высоте должен лететь самолет, чтобы на фотографии можно
    было различить листья деревьев размером $4\,\text{cм}$?
    При какой скорости самолета изображение не будет размытым,
    если время зкспозиции $2\,\text{мс}$?
}

\variantsplitter

\addpersonalvariant{Матвей Кузьмин}

\tasknumber{1}%
\task{%
    Найти оптическую силу собирающей линзы, если действительное изображение предмета,
    помещённого в $55\,\text{см}$ от линзы, получается на расстоянии $20\,\text{см}$ от неё.
}
\answer{%
    $D = \frac 1F = \frac 1a + \frac 1b = \frac 1{55\,\text{см}} + \frac 1{20\,\text{см}} \approx 6{,}82\,\text{дптр}$
}
\solutionspace{180pt}

\tasknumber{2}%
\task{%
    Найти увеличение изображения, если изображение предмета, находящегося
    на расстоянии $15\,\text{см}$ от линзы, получается на расстоянии $12\,\text{см}$ от неё.
}
\answer{%
    $\Gamma = \frac ba = \frac {12\,\text{см}}{15\,\text{см}} \approx 0{,}8$
}
\solutionspace{180pt}

\tasknumber{3}%
\task{%
    Расстояние от предмета до линзы $8\,\text{см}$, а от линзы до мнимого изображения $20\,\text{см}$.
    Чему равно фокусное расстояние линзы?
}
\answer{%
    $\pm \frac 1F = \frac 1a - \frac 1b \implies F = \frac{a b}{\abs{b - a}} \approx 13{,}3\,\text{см}$
}
\solutionspace{180pt}

\tasknumber{4}%
\task{%
    Две тонкие собирающие линзы с фокусными расстояниями $25\,\text{см}$ и $30\,\text{см}$ сложены вместе.
    Чему равно фокусное расстояние такой оптической системы?
}
\answer{%
    $\frac 1{f_1} = \frac 1a + \frac 1b; \frac 1{f_2} = - \frac 1b + \frac 1c \implies \frac 1{f_1} + \frac 1{f_2} = \frac 1a + \frac 1c \implies f' = \frac 1{\frac 1{f_1} + \frac 1{f_2}} = \frac{f_1 f_2}{f_1 + f_2} \approx 13{,}6\,\text{см}$
}
\solutionspace{180pt}

\tasknumber{5}%
\task{%
    Линейные размеры прямого изображения предмета, полученного в собирающей линзе,
    в три раза больше линейных размеров предмета.
    Зная, что предмет находится на $25\,\text{см}$ ближе к линзе,
    чем его изображение, найти оптическую силу линзы.
}
\answer{%
    \begin{align*}
    D &= \frac 1F = \frac 1a + \frac 1b, \qquad \Gamma = \frac ba, \qquad b - a = \ell \implies b = \Gamma a \implies \Gamma a - a = \ell \implies  \\
    a &= \frac {\ell}{\Gamma - 1} \implies b = \frac {{\ell} \Gamma}{\Gamma - 1} \implies  \\
    D &= \frac {\Gamma - 1}\ell + \frac {\Gamma - 1}{\ell \Gamma} = \frac 1\ell \cdot \cbr{\Gamma - 1 + \frac {\Gamma - 1}{\Gamma} } =\frac 1\ell \cdot \cbr{\Gamma - \frac 1\Gamma} \approx 10{,}7\,\text{дптр}.
    \end{align*}
}
\solutionspace{180pt}

\tasknumber{6}%
\task{%
    Оптическая сила объектива фотоаппарата равна $5\,\text{дптр}$.
    При фотографировании чертежа с расстояния $0{,}8\,\text{м}$ площадь изображения
    чертежа на фотопластинке оказалась равной $16\,\text{см}^{2}$.
    Какова площадь самого чертежа? Ответ выразите в квадратных сантиметрах.
}
\solutionspace{180pt}

\tasknumber{7}%
\task{%
    В каком месте на главной оптической оси двояковыгнутой линзы
    нужно поместить точечный источник света,
    чтобы его изображение оказалось в главном фокусе линзы?
}
\answer{%
    $\text{на половине фокусного расстояния}$
}
\solutionspace{120pt}

\tasknumber{8}%
\task{%
    Предмет высотой $h = 50\,\text{см}$ находится на расстоянии $d = 0{,}8\,\text{м}$
    от вертикально расположенной рассеивающей линзы с фокусным расстоянием $F = 20\,\text{см}$.
    Где находится изображение предмета? Определите тип изображения и его высоту.
}
\solutionspace{120pt}

\tasknumber{9}%
\task{%
    На каком расстоянии от двояковыпуклой линзы с оптической силой $D = 2\,\text{дптр}$
    надо поместить предмет, чтобы его изображение получилось на расстоянии $2{,}5\,\text{м}$ от линзы?
}
\solutionspace{120pt}

\tasknumber{10}%
\task{%
    Предмет в виде отрезка длиной $\ell$ расположен вдоль оптической оси
    собирающей линзы с фокусным расстоянием $F$.
    Середина отрезка расположена
    на расстоянии $a$ от линзы, которая даёт действительное изображение
    всех точек предмета.
    Определить продольное увеличение предмета.
}
\answer{%
    \begin{align*}
    \frac 1{a + \frac \ell 2} &+ \frac 1b = \frac 1F \implies b = \frac{F\cbr{a + \frac \ell 2}}{a + \frac \ell 2 - F} \\
    \frac 1{a - \frac \ell 2} &+ \frac 1c = \frac 1F \implies c = \frac{F\cbr{a - \frac \ell 2}}{a - \frac \ell 2 - F} \\
    \abs{b - c} &= \abs{\frac{F\cbr{a + \frac \ell 2}}{a + \frac \ell 2 - F} - \frac{F\cbr{a - \frac \ell 2}}{a - \frac \ell 2 - F}}= F\abs{\frac{\cbr{a + \frac \ell 2}\cbr{a - \frac \ell 2 - F} - \cbr{a - \frac \ell 2}\cbr{a + \frac \ell 2 - F}}{ \cbr{a + \frac \ell 2 - F} \cbr{a - \frac \ell 2 - F} }} =  \\
    &= F\abs{\frac{a^2 - \frac {a\ell} 2 - Fa + \frac {a\ell} 2 - \frac {\ell^2} 4 - \frac {F\ell}2 - a^2 - \frac {a\ell}2 + aF + \frac {a\ell}2 + \frac {\ell^2} 4 - \frac {F\ell} 2}{\cbr{a + \frac \ell 2 - F} \cbr{a - \frac \ell 2 - F} }} = \\
    &= F\frac{F\ell}{\sqr{a-F} - \frac {\ell^2}4} = \frac{F^2\ell}{\sqr{a-F} - \frac {\ell^2}4}\implies \Gamma = \frac{\abs{b - c}}\ell = \frac{F^2}{\sqr{a-F} - \frac {\ell^2}4}.
    \end{align*}
}
\solutionspace{120pt}

\tasknumber{11}%
\task{%
    На экране с помощью тонкой линзы получено изображение предмета
    с увеличением $4$.
    Предмет передвинули на $4\,\text{см}$.
    Для того, чтобы получить резкое изображение, пришлось передвинуть экран.
    При этом увеличение оказалось равным $6$.
    На какое расстояние
    пришлось передвинуть экран?
}
\solutionspace{120pt}

\tasknumber{12}%
\task{%
    Тонкая собирающая линза дает изображение предмета на экране высотой $H_1$,
    и $H_2$, при двух положениях линзы между предметом и экраном.
    Расстояние между ними неизменно.
    Чему равна высота предмета $h$?
}
\answer{%
    $h = \sqrt{H_1 H_2}$
}
\solutionspace{120pt}

\tasknumber{13}%
\task{%
    Какие предметы можно рассмотреть на фотографии, сделанной со спутника,
    если разрешающая способность пленки $0{,}010\,\text{мм}$? Каким должно быть
    время экспозиции $\tau$ чтобы полностью использовать возможности пленки?
    Фокусное расстояние объектива используемого фотоаппарата $10\,\text{cм}$,
    высота орбиты спутника $100\,\text{км}$.
}
\solutionspace{120pt}

\tasknumber{14}%
\task{%
    При аэрофотосъемках используется фотоаппарат, объектив которого
    имеет фокусиое расстояние $8\,\text{cм}$.
    Разрешающая способность пленки $0{,}010\,\text{мм}$.
    На какой высоте должен лететь самолет, чтобы на фотографии можно
    было различить листья деревьев размером $5\,\text{cм}$?
    При какой скорости самолета изображение не будет размытым,
    если время зкспозиции $2\,\text{мс}$?
}

\variantsplitter

\addpersonalvariant{Сергей Малышев}

\tasknumber{1}%
\task{%
    Найти оптическую силу собирающей линзы, если действительное изображение предмета,
    помещённого в $55\,\text{см}$ от линзы, получается на расстоянии $40\,\text{см}$ от неё.
}
\answer{%
    $D = \frac 1F = \frac 1a + \frac 1b = \frac 1{55\,\text{см}} + \frac 1{40\,\text{см}} \approx 4{,}32\,\text{дптр}$
}
\solutionspace{180pt}

\tasknumber{2}%
\task{%
    Найти увеличение изображения, если изображение предмета, находящегося
    на расстоянии $25\,\text{см}$ от линзы, получается на расстоянии $18\,\text{см}$ от неё.
}
\answer{%
    $\Gamma = \frac ba = \frac {18\,\text{см}}{25\,\text{см}} \approx 0{,}7$
}
\solutionspace{180pt}

\tasknumber{3}%
\task{%
    Расстояние от предмета до линзы $12\,\text{см}$, а от линзы до мнимого изображения $25\,\text{см}$.
    Чему равно фокусное расстояние линзы?
}
\answer{%
    $\pm \frac 1F = \frac 1a - \frac 1b \implies F = \frac{a b}{\abs{b - a}} \approx 23{,}1\,\text{см}$
}
\solutionspace{180pt}

\tasknumber{4}%
\task{%
    Две тонкие собирающие линзы с фокусными расстояниями $18\,\text{см}$ и $30\,\text{см}$ сложены вместе.
    Чему равно фокусное расстояние такой оптической системы?
}
\answer{%
    $\frac 1{f_1} = \frac 1a + \frac 1b; \frac 1{f_2} = - \frac 1b + \frac 1c \implies \frac 1{f_1} + \frac 1{f_2} = \frac 1a + \frac 1c \implies f' = \frac 1{\frac 1{f_1} + \frac 1{f_2}} = \frac{f_1 f_2}{f_1 + f_2} \approx 11{,}2\,\text{см}$
}
\solutionspace{180pt}

\tasknumber{5}%
\task{%
    Линейные размеры прямого изображения предмета, полученного в собирающей линзе,
    в три раза больше линейных размеров предмета.
    Зная, что предмет находится на $40\,\text{см}$ ближе к линзе,
    чем его изображение, найти оптическую силу линзы.
}
\answer{%
    \begin{align*}
    D &= \frac 1F = \frac 1a + \frac 1b, \qquad \Gamma = \frac ba, \qquad b - a = \ell \implies b = \Gamma a \implies \Gamma a - a = \ell \implies  \\
    a &= \frac {\ell}{\Gamma - 1} \implies b = \frac {{\ell} \Gamma}{\Gamma - 1} \implies  \\
    D &= \frac {\Gamma - 1}\ell + \frac {\Gamma - 1}{\ell \Gamma} = \frac 1\ell \cdot \cbr{\Gamma - 1 + \frac {\Gamma - 1}{\Gamma} } =\frac 1\ell \cdot \cbr{\Gamma - \frac 1\Gamma} \approx 6{,}7\,\text{дптр}.
    \end{align*}
}
\solutionspace{180pt}

\tasknumber{6}%
\task{%
    Оптическая сила объектива фотоаппарата равна $4\,\text{дптр}$.
    При фотографировании чертежа с расстояния $0{,}8\,\text{м}$ площадь изображения
    чертежа на фотопластинке оказалась равной $4\,\text{см}^{2}$.
    Какова площадь самого чертежа? Ответ выразите в квадратных сантиметрах.
}
\solutionspace{180pt}

\tasknumber{7}%
\task{%
    В каком месте на главной оптической оси двояковыгнутой линзы
    нужно поместить точечный источник света,
    чтобы его изображение оказалось в главном фокусе линзы?
}
\answer{%
    $\text{на половине фокусного расстояния}$
}
\solutionspace{120pt}

\tasknumber{8}%
\task{%
    Предмет высотой $h = 40\,\text{см}$ находится на расстоянии $d = 0{,}8\,\text{м}$
    от вертикально расположенной рассеивающей линзы с фокусным расстоянием $F = 20\,\text{см}$.
    Где находится изображение предмета? Определите тип изображения и его высоту.
}
\solutionspace{120pt}

\tasknumber{9}%
\task{%
    На каком расстоянии от двояковыпуклой линзы с оптической силой $D = 2{,}5\,\text{дптр}$
    надо поместить предмет, чтобы его изображение получилось на расстоянии $2\,\text{м}$ от линзы?
}
\solutionspace{120pt}

\tasknumber{10}%
\task{%
    Предмет в виде отрезка длиной $\ell$ расположен вдоль оптической оси
    собирающей линзы с фокусным расстоянием $F$.
    Середина отрезка расположена
    на расстоянии $a$ от линзы, которая даёт действительное изображение
    всех точек предмета.
    Определить продольное увеличение предмета.
}
\answer{%
    \begin{align*}
    \frac 1{a + \frac \ell 2} &+ \frac 1b = \frac 1F \implies b = \frac{F\cbr{a + \frac \ell 2}}{a + \frac \ell 2 - F} \\
    \frac 1{a - \frac \ell 2} &+ \frac 1c = \frac 1F \implies c = \frac{F\cbr{a - \frac \ell 2}}{a - \frac \ell 2 - F} \\
    \abs{b - c} &= \abs{\frac{F\cbr{a + \frac \ell 2}}{a + \frac \ell 2 - F} - \frac{F\cbr{a - \frac \ell 2}}{a - \frac \ell 2 - F}}= F\abs{\frac{\cbr{a + \frac \ell 2}\cbr{a - \frac \ell 2 - F} - \cbr{a - \frac \ell 2}\cbr{a + \frac \ell 2 - F}}{ \cbr{a + \frac \ell 2 - F} \cbr{a - \frac \ell 2 - F} }} =  \\
    &= F\abs{\frac{a^2 - \frac {a\ell} 2 - Fa + \frac {a\ell} 2 - \frac {\ell^2} 4 - \frac {F\ell}2 - a^2 - \frac {a\ell}2 + aF + \frac {a\ell}2 + \frac {\ell^2} 4 - \frac {F\ell} 2}{\cbr{a + \frac \ell 2 - F} \cbr{a - \frac \ell 2 - F} }} = \\
    &= F\frac{F\ell}{\sqr{a-F} - \frac {\ell^2}4} = \frac{F^2\ell}{\sqr{a-F} - \frac {\ell^2}4}\implies \Gamma = \frac{\abs{b - c}}\ell = \frac{F^2}{\sqr{a-F} - \frac {\ell^2}4}.
    \end{align*}
}
\solutionspace{120pt}

\tasknumber{11}%
\task{%
    На экране с помощью тонкой линзы получено изображение предмета
    с увеличением $4$.
    Предмет передвинули на $6\,\text{см}$.
    Для того, чтобы получить резкое изображение, пришлось передвинуть экран.
    При этом увеличение оказалось равным $8$.
    На какое расстояние
    пришлось передвинуть экран?
}
\solutionspace{120pt}

\tasknumber{12}%
\task{%
    Тонкая собирающая линза дает изображение предмета на экране высотой $H_1$,
    и $H_2$, при двух положениях линзы между предметом и экраном.
    Расстояние между ними неизменно.
    Чему равна высота предмета $h$?
}
\answer{%
    $h = \sqrt{H_1 H_2}$
}
\solutionspace{120pt}

\tasknumber{13}%
\task{%
    Какие предметы можно рассмотреть на фотографии, сделанной со спутника,
    если разрешающая способность пленки $0{,}010\,\text{мм}$? Каким должно быть
    время экспозиции $\tau$ чтобы полностью использовать возможности пленки?
    Фокусное расстояние объектива используемого фотоаппарата $10\,\text{cм}$,
    высота орбиты спутника $150\,\text{км}$.
}
\solutionspace{120pt}

\tasknumber{14}%
\task{%
    При аэрофотосъемках используется фотоаппарат, объектив которого
    имеет фокусиое расстояние $12\,\text{cм}$.
    Разрешающая способность пленки $0{,}015\,\text{мм}$.
    На какой высоте должен лететь самолет, чтобы на фотографии можно
    было различить листья деревьев размером $5\,\text{cм}$?
    При какой скорости самолета изображение не будет размытым,
    если время зкспозиции $2\,\text{мс}$?
}

\variantsplitter

\addpersonalvariant{Алина Полканова}

\tasknumber{1}%
\task{%
    Найти оптическую силу собирающей линзы, если действительное изображение предмета,
    помещённого в $35\,\text{см}$ от линзы, получается на расстоянии $20\,\text{см}$ от неё.
}
\answer{%
    $D = \frac 1F = \frac 1a + \frac 1b = \frac 1{35\,\text{см}} + \frac 1{20\,\text{см}} \approx 7{,}86\,\text{дптр}$
}
\solutionspace{180pt}

\tasknumber{2}%
\task{%
    Найти увеличение изображения, если изображение предмета, находящегося
    на расстоянии $25\,\text{см}$ от линзы, получается на расстоянии $12\,\text{см}$ от неё.
}
\answer{%
    $\Gamma = \frac ba = \frac {12\,\text{см}}{25\,\text{см}} \approx 0{,}5$
}
\solutionspace{180pt}

\tasknumber{3}%
\task{%
    Расстояние от предмета до линзы $12\,\text{см}$, а от линзы до мнимого изображения $20\,\text{см}$.
    Чему равно фокусное расстояние линзы?
}
\answer{%
    $\pm \frac 1F = \frac 1a - \frac 1b \implies F = \frac{a b}{\abs{b - a}} \approx 30\,\text{см}$
}
\solutionspace{180pt}

\tasknumber{4}%
\task{%
    Две тонкие собирающие линзы с фокусными расстояниями $25\,\text{см}$ и $30\,\text{см}$ сложены вместе.
    Чему равно фокусное расстояние такой оптической системы?
}
\answer{%
    $\frac 1{f_1} = \frac 1a + \frac 1b; \frac 1{f_2} = - \frac 1b + \frac 1c \implies \frac 1{f_1} + \frac 1{f_2} = \frac 1a + \frac 1c \implies f' = \frac 1{\frac 1{f_1} + \frac 1{f_2}} = \frac{f_1 f_2}{f_1 + f_2} \approx 13{,}6\,\text{см}$
}
\solutionspace{180pt}

\tasknumber{5}%
\task{%
    Линейные размеры прямого изображения предмета, полученного в собирающей линзе,
    в два раза больше линейных размеров предмета.
    Зная, что предмет находится на $40\,\text{см}$ ближе к линзе,
    чем его изображение, найти оптическую силу линзы.
}
\answer{%
    \begin{align*}
    D &= \frac 1F = \frac 1a + \frac 1b, \qquad \Gamma = \frac ba, \qquad b - a = \ell \implies b = \Gamma a \implies \Gamma a - a = \ell \implies  \\
    a &= \frac {\ell}{\Gamma - 1} \implies b = \frac {{\ell} \Gamma}{\Gamma - 1} \implies  \\
    D &= \frac {\Gamma - 1}\ell + \frac {\Gamma - 1}{\ell \Gamma} = \frac 1\ell \cdot \cbr{\Gamma - 1 + \frac {\Gamma - 1}{\Gamma} } =\frac 1\ell \cdot \cbr{\Gamma - \frac 1\Gamma} \approx 3{,}8\,\text{дптр}.
    \end{align*}
}
\solutionspace{180pt}

\tasknumber{6}%
\task{%
    Оптическая сила объектива фотоаппарата равна $5\,\text{дптр}$.
    При фотографировании чертежа с расстояния $0{,}8\,\text{м}$ площадь изображения
    чертежа на фотопластинке оказалась равной $9\,\text{см}^{2}$.
    Какова площадь самого чертежа? Ответ выразите в квадратных сантиметрах.
}
\solutionspace{180pt}

\tasknumber{7}%
\task{%
    В каком месте на главной оптической оси двояковыгнутой линзы
    нужно поместить точечный источник света,
    чтобы его изображение оказалось в главном фокусе линзы?
}
\answer{%
    $\text{на половине фокусного расстояния}$
}
\solutionspace{120pt}

\tasknumber{8}%
\task{%
    Предмет высотой $h = 50\,\text{см}$ находится на расстоянии $d = 0{,}8\,\text{м}$
    от вертикально расположенной рассеивающей линзы с фокусным расстоянием $F = 20\,\text{см}$.
    Где находится изображение предмета? Определите тип изображения и его высоту.
}
\solutionspace{120pt}

\tasknumber{9}%
\task{%
    На каком расстоянии от двояковыпуклой линзы с оптической силой $D = 2{,}5\,\text{дптр}$
    надо поместить предмет, чтобы его изображение получилось на расстоянии $1{,}5\,\text{м}$ от линзы?
}
\solutionspace{120pt}

\tasknumber{10}%
\task{%
    Предмет в виде отрезка длиной $\ell$ расположен вдоль оптической оси
    собирающей линзы с фокусным расстоянием $F$.
    Середина отрезка расположена
    на расстоянии $a$ от линзы, которая даёт действительное изображение
    всех точек предмета.
    Определить продольное увеличение предмета.
}
\answer{%
    \begin{align*}
    \frac 1{a + \frac \ell 2} &+ \frac 1b = \frac 1F \implies b = \frac{F\cbr{a + \frac \ell 2}}{a + \frac \ell 2 - F} \\
    \frac 1{a - \frac \ell 2} &+ \frac 1c = \frac 1F \implies c = \frac{F\cbr{a - \frac \ell 2}}{a - \frac \ell 2 - F} \\
    \abs{b - c} &= \abs{\frac{F\cbr{a + \frac \ell 2}}{a + \frac \ell 2 - F} - \frac{F\cbr{a - \frac \ell 2}}{a - \frac \ell 2 - F}}= F\abs{\frac{\cbr{a + \frac \ell 2}\cbr{a - \frac \ell 2 - F} - \cbr{a - \frac \ell 2}\cbr{a + \frac \ell 2 - F}}{ \cbr{a + \frac \ell 2 - F} \cbr{a - \frac \ell 2 - F} }} =  \\
    &= F\abs{\frac{a^2 - \frac {a\ell} 2 - Fa + \frac {a\ell} 2 - \frac {\ell^2} 4 - \frac {F\ell}2 - a^2 - \frac {a\ell}2 + aF + \frac {a\ell}2 + \frac {\ell^2} 4 - \frac {F\ell} 2}{\cbr{a + \frac \ell 2 - F} \cbr{a - \frac \ell 2 - F} }} = \\
    &= F\frac{F\ell}{\sqr{a-F} - \frac {\ell^2}4} = \frac{F^2\ell}{\sqr{a-F} - \frac {\ell^2}4}\implies \Gamma = \frac{\abs{b - c}}\ell = \frac{F^2}{\sqr{a-F} - \frac {\ell^2}4}.
    \end{align*}
}
\solutionspace{120pt}

\tasknumber{11}%
\task{%
    На экране с помощью тонкой линзы получено изображение предмета
    с увеличением $2$.
    Предмет передвинули на $2\,\text{см}$.
    Для того, чтобы получить резкое изображение, пришлось передвинуть экран.
    При этом увеличение оказалось равным $8$.
    На какое расстояние
    пришлось передвинуть экран?
}
\solutionspace{120pt}

\tasknumber{12}%
\task{%
    Тонкая собирающая линза дает изображение предмета на экране высотой $H_1$,
    и $H_2$, при двух положениях линзы между предметом и экраном.
    Расстояние между ними неизменно.
    Чему равна высота предмета $h$?
}
\answer{%
    $h = \sqrt{H_1 H_2}$
}
\solutionspace{120pt}

\tasknumber{13}%
\task{%
    Какие предметы можно рассмотреть на фотографии, сделанной со спутника,
    если разрешающая способность пленки $0{,}010\,\text{мм}$? Каким должно быть
    время экспозиции $\tau$ чтобы полностью использовать возможности пленки?
    Фокусное расстояние объектива используемого фотоаппарата $10\,\text{cм}$,
    высота орбиты спутника $100\,\text{км}$.
}
\solutionspace{120pt}

\tasknumber{14}%
\task{%
    При аэрофотосъемках используется фотоаппарат, объектив которого
    имеет фокусиое расстояние $10\,\text{cм}$.
    Разрешающая способность пленки $0{,}015\,\text{мм}$.
    На какой высоте должен лететь самолет, чтобы на фотографии можно
    было различить листья деревьев размером $6\,\text{cм}$?
    При какой скорости самолета изображение не будет размытым,
    если время зкспозиции $2\,\text{мс}$?
}

\variantsplitter

\addpersonalvariant{Сергей Пономарёв}

\tasknumber{1}%
\task{%
    Найти оптическую силу собирающей линзы, если действительное изображение предмета,
    помещённого в $35\,\text{см}$ от линзы, получается на расстоянии $20\,\text{см}$ от неё.
}
\answer{%
    $D = \frac 1F = \frac 1a + \frac 1b = \frac 1{35\,\text{см}} + \frac 1{20\,\text{см}} \approx 7{,}86\,\text{дптр}$
}
\solutionspace{180pt}

\tasknumber{2}%
\task{%
    Найти увеличение изображения, если изображение предмета, находящегося
    на расстоянии $20\,\text{см}$ от линзы, получается на расстоянии $18\,\text{см}$ от неё.
}
\answer{%
    $\Gamma = \frac ba = \frac {18\,\text{см}}{20\,\text{см}} \approx 0{,}9$
}
\solutionspace{180pt}

\tasknumber{3}%
\task{%
    Расстояние от предмета до линзы $10\,\text{см}$, а от линзы до мнимого изображения $25\,\text{см}$.
    Чему равно фокусное расстояние линзы?
}
\answer{%
    $\pm \frac 1F = \frac 1a - \frac 1b \implies F = \frac{a b}{\abs{b - a}} \approx 16{,}7\,\text{см}$
}
\solutionspace{180pt}

\tasknumber{4}%
\task{%
    Две тонкие собирающие линзы с фокусными расстояниями $18\,\text{см}$ и $30\,\text{см}$ сложены вместе.
    Чему равно фокусное расстояние такой оптической системы?
}
\answer{%
    $\frac 1{f_1} = \frac 1a + \frac 1b; \frac 1{f_2} = - \frac 1b + \frac 1c \implies \frac 1{f_1} + \frac 1{f_2} = \frac 1a + \frac 1c \implies f' = \frac 1{\frac 1{f_1} + \frac 1{f_2}} = \frac{f_1 f_2}{f_1 + f_2} \approx 11{,}2\,\text{см}$
}
\solutionspace{180pt}

\tasknumber{5}%
\task{%
    Линейные размеры прямого изображения предмета, полученного в собирающей линзе,
    в три раза больше линейных размеров предмета.
    Зная, что предмет находится на $25\,\text{см}$ ближе к линзе,
    чем его изображение, найти оптическую силу линзы.
}
\answer{%
    \begin{align*}
    D &= \frac 1F = \frac 1a + \frac 1b, \qquad \Gamma = \frac ba, \qquad b - a = \ell \implies b = \Gamma a \implies \Gamma a - a = \ell \implies  \\
    a &= \frac {\ell}{\Gamma - 1} \implies b = \frac {{\ell} \Gamma}{\Gamma - 1} \implies  \\
    D &= \frac {\Gamma - 1}\ell + \frac {\Gamma - 1}{\ell \Gamma} = \frac 1\ell \cdot \cbr{\Gamma - 1 + \frac {\Gamma - 1}{\Gamma} } =\frac 1\ell \cdot \cbr{\Gamma - \frac 1\Gamma} \approx 10{,}7\,\text{дптр}.
    \end{align*}
}
\solutionspace{180pt}

\tasknumber{6}%
\task{%
    Оптическая сила объектива фотоаппарата равна $5\,\text{дптр}$.
    При фотографировании чертежа с расстояния $0{,}8\,\text{м}$ площадь изображения
    чертежа на фотопластинке оказалась равной $16\,\text{см}^{2}$.
    Какова площадь самого чертежа? Ответ выразите в квадратных сантиметрах.
}
\solutionspace{180pt}

\tasknumber{7}%
\task{%
    В каком месте на главной оптической оси двояковыпуклой линзы
    нужно поместить точечный источник света,
    чтобы его изображение оказалось в главном фокусе линзы?
}
\answer{%
    $\text{для мнимого - на половине фокусного, для действительного - на бесконечности}$
}
\solutionspace{120pt}

\tasknumber{8}%
\task{%
    Предмет высотой $h = 50\,\text{см}$ находится на расстоянии $d = 1{,}2\,\text{м}$
    от вертикально расположенной рассеивающей линзы с фокусным расстоянием $F = 25\,\text{см}$.
    Где находится изображение предмета? Определите тип изображения и его высоту.
}
\solutionspace{120pt}

\tasknumber{9}%
\task{%
    На каком расстоянии от двояковыпуклой линзы с оптической силой $D = 2\,\text{дптр}$
    надо поместить предмет, чтобы его изображение получилось на расстоянии $2{,}5\,\text{м}$ от линзы?
}
\solutionspace{120pt}

\tasknumber{10}%
\task{%
    Предмет в виде отрезка длиной $\ell$ расположен вдоль оптической оси
    собирающей линзы с фокусным расстоянием $F$.
    Середина отрезка расположена
    на расстоянии $a$ от линзы, которая даёт действительное изображение
    всех точек предмета.
    Определить продольное увеличение предмета.
}
\answer{%
    \begin{align*}
    \frac 1{a + \frac \ell 2} &+ \frac 1b = \frac 1F \implies b = \frac{F\cbr{a + \frac \ell 2}}{a + \frac \ell 2 - F} \\
    \frac 1{a - \frac \ell 2} &+ \frac 1c = \frac 1F \implies c = \frac{F\cbr{a - \frac \ell 2}}{a - \frac \ell 2 - F} \\
    \abs{b - c} &= \abs{\frac{F\cbr{a + \frac \ell 2}}{a + \frac \ell 2 - F} - \frac{F\cbr{a - \frac \ell 2}}{a - \frac \ell 2 - F}}= F\abs{\frac{\cbr{a + \frac \ell 2}\cbr{a - \frac \ell 2 - F} - \cbr{a - \frac \ell 2}\cbr{a + \frac \ell 2 - F}}{ \cbr{a + \frac \ell 2 - F} \cbr{a - \frac \ell 2 - F} }} =  \\
    &= F\abs{\frac{a^2 - \frac {a\ell} 2 - Fa + \frac {a\ell} 2 - \frac {\ell^2} 4 - \frac {F\ell}2 - a^2 - \frac {a\ell}2 + aF + \frac {a\ell}2 + \frac {\ell^2} 4 - \frac {F\ell} 2}{\cbr{a + \frac \ell 2 - F} \cbr{a - \frac \ell 2 - F} }} = \\
    &= F\frac{F\ell}{\sqr{a-F} - \frac {\ell^2}4} = \frac{F^2\ell}{\sqr{a-F} - \frac {\ell^2}4}\implies \Gamma = \frac{\abs{b - c}}\ell = \frac{F^2}{\sqr{a-F} - \frac {\ell^2}4}.
    \end{align*}
}
\solutionspace{120pt}

\tasknumber{11}%
\task{%
    На экране с помощью тонкой линзы получено изображение предмета
    с увеличением $4$.
    Предмет передвинули на $2\,\text{см}$.
    Для того, чтобы получить резкое изображение, пришлось передвинуть экран.
    При этом увеличение оказалось равным $8$.
    На какое расстояние
    пришлось передвинуть экран?
}
\solutionspace{120pt}

\tasknumber{12}%
\task{%
    Тонкая собирающая линза дает изображение предмета на экране высотой $H_1$,
    и $H_2$, при двух положениях линзы между предметом и экраном.
    Расстояние между ними неизменно.
    Чему равна высота предмета $h$?
}
\answer{%
    $h = \sqrt{H_1 H_2}$
}
\solutionspace{120pt}

\tasknumber{13}%
\task{%
    Какие предметы можно рассмотреть на фотографии, сделанной со спутника,
    если разрешающая способность пленки $0{,}010\,\text{мм}$? Каким должно быть
    время экспозиции $\tau$ чтобы полностью использовать возможности пленки?
    Фокусное расстояние объектива используемого фотоаппарата $20\,\text{cм}$,
    высота орбиты спутника $80\,\text{км}$.
}
\solutionspace{120pt}

\tasknumber{14}%
\task{%
    При аэрофотосъемках используется фотоаппарат, объектив которого
    имеет фокусиое расстояние $10\,\text{cм}$.
    Разрешающая способность пленки $0{,}02\,\text{мм}$.
    На какой высоте должен лететь самолет, чтобы на фотографии можно
    было различить листья деревьев размером $6\,\text{cм}$?
    При какой скорости самолета изображение не будет размытым,
    если время зкспозиции $1\,\text{мс}$?
}

\variantsplitter

\addpersonalvariant{Егор Свистушкин}

\tasknumber{1}%
\task{%
    Найти оптическую силу собирающей линзы, если действительное изображение предмета,
    помещённого в $55\,\text{см}$ от линзы, получается на расстоянии $30\,\text{см}$ от неё.
}
\answer{%
    $D = \frac 1F = \frac 1a + \frac 1b = \frac 1{55\,\text{см}} + \frac 1{30\,\text{см}} \approx 5{,}15\,\text{дптр}$
}
\solutionspace{180pt}

\tasknumber{2}%
\task{%
    Найти увеличение изображения, если изображение предмета, находящегося
    на расстоянии $20\,\text{см}$ от линзы, получается на расстоянии $18\,\text{см}$ от неё.
}
\answer{%
    $\Gamma = \frac ba = \frac {18\,\text{см}}{20\,\text{см}} \approx 0{,}9$
}
\solutionspace{180pt}

\tasknumber{3}%
\task{%
    Расстояние от предмета до линзы $10\,\text{см}$, а от линзы до мнимого изображения $25\,\text{см}$.
    Чему равно фокусное расстояние линзы?
}
\answer{%
    $\pm \frac 1F = \frac 1a - \frac 1b \implies F = \frac{a b}{\abs{b - a}} \approx 16{,}7\,\text{см}$
}
\solutionspace{180pt}

\tasknumber{4}%
\task{%
    Две тонкие собирающие линзы с фокусными расстояниями $18\,\text{см}$ и $30\,\text{см}$ сложены вместе.
    Чему равно фокусное расстояние такой оптической системы?
}
\answer{%
    $\frac 1{f_1} = \frac 1a + \frac 1b; \frac 1{f_2} = - \frac 1b + \frac 1c \implies \frac 1{f_1} + \frac 1{f_2} = \frac 1a + \frac 1c \implies f' = \frac 1{\frac 1{f_1} + \frac 1{f_2}} = \frac{f_1 f_2}{f_1 + f_2} \approx 11{,}2\,\text{см}$
}
\solutionspace{180pt}

\tasknumber{5}%
\task{%
    Линейные размеры прямого изображения предмета, полученного в собирающей линзе,
    в три раза больше линейных размеров предмета.
    Зная, что предмет находится на $40\,\text{см}$ ближе к линзе,
    чем его изображение, найти оптическую силу линзы.
}
\answer{%
    \begin{align*}
    D &= \frac 1F = \frac 1a + \frac 1b, \qquad \Gamma = \frac ba, \qquad b - a = \ell \implies b = \Gamma a \implies \Gamma a - a = \ell \implies  \\
    a &= \frac {\ell}{\Gamma - 1} \implies b = \frac {{\ell} \Gamma}{\Gamma - 1} \implies  \\
    D &= \frac {\Gamma - 1}\ell + \frac {\Gamma - 1}{\ell \Gamma} = \frac 1\ell \cdot \cbr{\Gamma - 1 + \frac {\Gamma - 1}{\Gamma} } =\frac 1\ell \cdot \cbr{\Gamma - \frac 1\Gamma} \approx 6{,}7\,\text{дптр}.
    \end{align*}
}
\solutionspace{180pt}

\tasknumber{6}%
\task{%
    Оптическая сила объектива фотоаппарата равна $3\,\text{дптр}$.
    При фотографировании чертежа с расстояния $1{,}1\,\text{м}$ площадь изображения
    чертежа на фотопластинке оказалась равной $16\,\text{см}^{2}$.
    Какова площадь самого чертежа? Ответ выразите в квадратных сантиметрах.
}
\solutionspace{180pt}

\tasknumber{7}%
\task{%
    В каком месте на главной оптической оси двояковыгнутой линзы
    нужно поместить точечный источник света,
    чтобы его изображение оказалось в главном фокусе линзы?
}
\answer{%
    $\text{на половине фокусного расстояния}$
}
\solutionspace{120pt}

\tasknumber{8}%
\task{%
    Предмет высотой $h = 50\,\text{см}$ находится на расстоянии $d = 1{,}2\,\text{м}$
    от вертикально расположенной рассеивающей линзы с фокусным расстоянием $F = 20\,\text{см}$.
    Где находится изображение предмета? Определите тип изображения и его высоту.
}
\solutionspace{120pt}

\tasknumber{9}%
\task{%
    На каком расстоянии от двояковыпуклой линзы с оптической силой $D = 2{,}5\,\text{дптр}$
    надо поместить предмет, чтобы его изображение получилось на расстоянии $1{,}5\,\text{м}$ от линзы?
}
\solutionspace{120pt}

\tasknumber{10}%
\task{%
    Предмет в виде отрезка длиной $\ell$ расположен вдоль оптической оси
    собирающей линзы с фокусным расстоянием $F$.
    Середина отрезка расположена
    на расстоянии $a$ от линзы, которая даёт действительное изображение
    всех точек предмета.
    Определить продольное увеличение предмета.
}
\answer{%
    \begin{align*}
    \frac 1{a + \frac \ell 2} &+ \frac 1b = \frac 1F \implies b = \frac{F\cbr{a + \frac \ell 2}}{a + \frac \ell 2 - F} \\
    \frac 1{a - \frac \ell 2} &+ \frac 1c = \frac 1F \implies c = \frac{F\cbr{a - \frac \ell 2}}{a - \frac \ell 2 - F} \\
    \abs{b - c} &= \abs{\frac{F\cbr{a + \frac \ell 2}}{a + \frac \ell 2 - F} - \frac{F\cbr{a - \frac \ell 2}}{a - \frac \ell 2 - F}}= F\abs{\frac{\cbr{a + \frac \ell 2}\cbr{a - \frac \ell 2 - F} - \cbr{a - \frac \ell 2}\cbr{a + \frac \ell 2 - F}}{ \cbr{a + \frac \ell 2 - F} \cbr{a - \frac \ell 2 - F} }} =  \\
    &= F\abs{\frac{a^2 - \frac {a\ell} 2 - Fa + \frac {a\ell} 2 - \frac {\ell^2} 4 - \frac {F\ell}2 - a^2 - \frac {a\ell}2 + aF + \frac {a\ell}2 + \frac {\ell^2} 4 - \frac {F\ell} 2}{\cbr{a + \frac \ell 2 - F} \cbr{a - \frac \ell 2 - F} }} = \\
    &= F\frac{F\ell}{\sqr{a-F} - \frac {\ell^2}4} = \frac{F^2\ell}{\sqr{a-F} - \frac {\ell^2}4}\implies \Gamma = \frac{\abs{b - c}}\ell = \frac{F^2}{\sqr{a-F} - \frac {\ell^2}4}.
    \end{align*}
}
\solutionspace{120pt}

\tasknumber{11}%
\task{%
    На экране с помощью тонкой линзы получено изображение предмета
    с увеличением $4$.
    Предмет передвинули на $4\,\text{см}$.
    Для того, чтобы получить резкое изображение, пришлось передвинуть экран.
    При этом увеличение оказалось равным $8$.
    На какое расстояние
    пришлось передвинуть экран?
}
\solutionspace{120pt}

\tasknumber{12}%
\task{%
    Тонкая собирающая линза дает изображение предмета на экране высотой $H_1$,
    и $H_2$, при двух положениях линзы между предметом и экраном.
    Расстояние между ними неизменно.
    Чему равна высота предмета $h$?
}
\answer{%
    $h = \sqrt{H_1 H_2}$
}
\solutionspace{120pt}

\tasknumber{13}%
\task{%
    Какие предметы можно рассмотреть на фотографии, сделанной со спутника,
    если разрешающая способность пленки $0{,}010\,\text{мм}$? Каким должно быть
    время экспозиции $\tau$ чтобы полностью использовать возможности пленки?
    Фокусное расстояние объектива используемого фотоаппарата $10\,\text{cм}$,
    высота орбиты спутника $150\,\text{км}$.
}
\solutionspace{120pt}

\tasknumber{14}%
\task{%
    При аэрофотосъемках используется фотоаппарат, объектив которого
    имеет фокусиое расстояние $8\,\text{cм}$.
    Разрешающая способность пленки $0{,}010\,\text{мм}$.
    На какой высоте должен лететь самолет, чтобы на фотографии можно
    было различить листья деревьев размером $5\,\text{cм}$?
    При какой скорости самолета изображение не будет размытым,
    если время зкспозиции $1\,\text{мс}$?
}

\variantsplitter

\addpersonalvariant{Дмитрий Соколов}

\tasknumber{1}%
\task{%
    Найти оптическую силу собирающей линзы, если действительное изображение предмета,
    помещённого в $55\,\text{см}$ от линзы, получается на расстоянии $40\,\text{см}$ от неё.
}
\answer{%
    $D = \frac 1F = \frac 1a + \frac 1b = \frac 1{55\,\text{см}} + \frac 1{40\,\text{см}} \approx 4{,}32\,\text{дптр}$
}
\solutionspace{180pt}

\tasknumber{2}%
\task{%
    Найти увеличение изображения, если изображение предмета, находящегося
    на расстоянии $25\,\text{см}$ от линзы, получается на расстоянии $18\,\text{см}$ от неё.
}
\answer{%
    $\Gamma = \frac ba = \frac {18\,\text{см}}{25\,\text{см}} \approx 0{,}7$
}
\solutionspace{180pt}

\tasknumber{3}%
\task{%
    Расстояние от предмета до линзы $12\,\text{см}$, а от линзы до мнимого изображения $25\,\text{см}$.
    Чему равно фокусное расстояние линзы?
}
\answer{%
    $\pm \frac 1F = \frac 1a - \frac 1b \implies F = \frac{a b}{\abs{b - a}} \approx 23{,}1\,\text{см}$
}
\solutionspace{180pt}

\tasknumber{4}%
\task{%
    Две тонкие собирающие линзы с фокусными расстояниями $12\,\text{см}$ и $30\,\text{см}$ сложены вместе.
    Чему равно фокусное расстояние такой оптической системы?
}
\answer{%
    $\frac 1{f_1} = \frac 1a + \frac 1b; \frac 1{f_2} = - \frac 1b + \frac 1c \implies \frac 1{f_1} + \frac 1{f_2} = \frac 1a + \frac 1c \implies f' = \frac 1{\frac 1{f_1} + \frac 1{f_2}} = \frac{f_1 f_2}{f_1 + f_2} \approx 8{,}6\,\text{см}$
}
\solutionspace{180pt}

\tasknumber{5}%
\task{%
    Линейные размеры прямого изображения предмета, полученного в собирающей линзе,
    в четыре раза больше линейных размеров предмета.
    Зная, что предмет находится на $25\,\text{см}$ ближе к линзе,
    чем его изображение, найти оптическую силу линзы.
}
\answer{%
    \begin{align*}
    D &= \frac 1F = \frac 1a + \frac 1b, \qquad \Gamma = \frac ba, \qquad b - a = \ell \implies b = \Gamma a \implies \Gamma a - a = \ell \implies  \\
    a &= \frac {\ell}{\Gamma - 1} \implies b = \frac {{\ell} \Gamma}{\Gamma - 1} \implies  \\
    D &= \frac {\Gamma - 1}\ell + \frac {\Gamma - 1}{\ell \Gamma} = \frac 1\ell \cdot \cbr{\Gamma - 1 + \frac {\Gamma - 1}{\Gamma} } =\frac 1\ell \cdot \cbr{\Gamma - \frac 1\Gamma} \approx 15\,\text{дптр}.
    \end{align*}
}
\solutionspace{180pt}

\tasknumber{6}%
\task{%
    Оптическая сила объектива фотоаппарата равна $5\,\text{дптр}$.
    При фотографировании чертежа с расстояния $0{,}8\,\text{м}$ площадь изображения
    чертежа на фотопластинке оказалась равной $4\,\text{см}^{2}$.
    Какова площадь самого чертежа? Ответ выразите в квадратных сантиметрах.
}
\solutionspace{180pt}

\tasknumber{7}%
\task{%
    В каком месте на главной оптической оси двояковыпуклой линзы
    нужно поместить точечный источник света,
    чтобы его изображение оказалось в главном фокусе линзы?
}
\answer{%
    $\text{для мнимого - на половине фокусного, для действительного - на бесконечности}$
}
\solutionspace{120pt}

\tasknumber{8}%
\task{%
    Предмет высотой $h = 40\,\text{см}$ находится на расстоянии $d = 0{,}8\,\text{м}$
    от вертикально расположенной рассеивающей линзы с фокусным расстоянием $F = 20\,\text{см}$.
    Где находится изображение предмета? Определите тип изображения и его высоту.
}
\solutionspace{120pt}

\tasknumber{9}%
\task{%
    На каком расстоянии от двояковыпуклой линзы с оптической силой $D = 2{,}5\,\text{дптр}$
    надо поместить предмет, чтобы его изображение получилось на расстоянии $1{,}5\,\text{м}$ от линзы?
}
\solutionspace{120pt}

\tasknumber{10}%
\task{%
    Предмет в виде отрезка длиной $\ell$ расположен вдоль оптической оси
    собирающей линзы с фокусным расстоянием $F$.
    Середина отрезка расположена
    на расстоянии $a$ от линзы, которая даёт действительное изображение
    всех точек предмета.
    Определить продольное увеличение предмета.
}
\answer{%
    \begin{align*}
    \frac 1{a + \frac \ell 2} &+ \frac 1b = \frac 1F \implies b = \frac{F\cbr{a + \frac \ell 2}}{a + \frac \ell 2 - F} \\
    \frac 1{a - \frac \ell 2} &+ \frac 1c = \frac 1F \implies c = \frac{F\cbr{a - \frac \ell 2}}{a - \frac \ell 2 - F} \\
    \abs{b - c} &= \abs{\frac{F\cbr{a + \frac \ell 2}}{a + \frac \ell 2 - F} - \frac{F\cbr{a - \frac \ell 2}}{a - \frac \ell 2 - F}}= F\abs{\frac{\cbr{a + \frac \ell 2}\cbr{a - \frac \ell 2 - F} - \cbr{a - \frac \ell 2}\cbr{a + \frac \ell 2 - F}}{ \cbr{a + \frac \ell 2 - F} \cbr{a - \frac \ell 2 - F} }} =  \\
    &= F\abs{\frac{a^2 - \frac {a\ell} 2 - Fa + \frac {a\ell} 2 - \frac {\ell^2} 4 - \frac {F\ell}2 - a^2 - \frac {a\ell}2 + aF + \frac {a\ell}2 + \frac {\ell^2} 4 - \frac {F\ell} 2}{\cbr{a + \frac \ell 2 - F} \cbr{a - \frac \ell 2 - F} }} = \\
    &= F\frac{F\ell}{\sqr{a-F} - \frac {\ell^2}4} = \frac{F^2\ell}{\sqr{a-F} - \frac {\ell^2}4}\implies \Gamma = \frac{\abs{b - c}}\ell = \frac{F^2}{\sqr{a-F} - \frac {\ell^2}4}.
    \end{align*}
}
\solutionspace{120pt}

\tasknumber{11}%
\task{%
    На экране с помощью тонкой линзы получено изображение предмета
    с увеличением $2$.
    Предмет передвинули на $4\,\text{см}$.
    Для того, чтобы получить резкое изображение, пришлось передвинуть экран.
    При этом увеличение оказалось равным $6$.
    На какое расстояние
    пришлось передвинуть экран?
}
\solutionspace{120pt}

\tasknumber{12}%
\task{%
    Тонкая собирающая линза дает изображение предмета на экране высотой $H_1$,
    и $H_2$, при двух положениях линзы между предметом и экраном.
    Расстояние между ними неизменно.
    Чему равна высота предмета $h$?
}
\answer{%
    $h = \sqrt{H_1 H_2}$
}
\solutionspace{120pt}

\tasknumber{13}%
\task{%
    Какие предметы можно рассмотреть на фотографии, сделанной со спутника,
    если разрешающая способность пленки $0{,}010\,\text{мм}$? Каким должно быть
    время экспозиции $\tau$ чтобы полностью использовать возможности пленки?
    Фокусное расстояние объектива используемого фотоаппарата $20\,\text{cм}$,
    высота орбиты спутника $80\,\text{км}$.
}
\solutionspace{120pt}

\tasknumber{14}%
\task{%
    При аэрофотосъемках используется фотоаппарат, объектив которого
    имеет фокусиое расстояние $12\,\text{cм}$.
    Разрешающая способность пленки $0{,}015\,\text{мм}$.
    На какой высоте должен лететь самолет, чтобы на фотографии можно
    было различить листья деревьев размером $4\,\text{cм}$?
    При какой скорости самолета изображение не будет размытым,
    если время зкспозиции $2\,\text{мс}$?
}

\variantsplitter

\addpersonalvariant{Арсений Трофимов}

\tasknumber{1}%
\task{%
    Найти оптическую силу собирающей линзы, если действительное изображение предмета,
    помещённого в $35\,\text{см}$ от линзы, получается на расстоянии $20\,\text{см}$ от неё.
}
\answer{%
    $D = \frac 1F = \frac 1a + \frac 1b = \frac 1{35\,\text{см}} + \frac 1{20\,\text{см}} \approx 7{,}86\,\text{дптр}$
}
\solutionspace{180pt}

\tasknumber{2}%
\task{%
    Найти увеличение изображения, если изображение предмета, находящегося
    на расстоянии $20\,\text{см}$ от линзы, получается на расстоянии $18\,\text{см}$ от неё.
}
\answer{%
    $\Gamma = \frac ba = \frac {18\,\text{см}}{20\,\text{см}} \approx 0{,}9$
}
\solutionspace{180pt}

\tasknumber{3}%
\task{%
    Расстояние от предмета до линзы $10\,\text{см}$, а от линзы до мнимого изображения $25\,\text{см}$.
    Чему равно фокусное расстояние линзы?
}
\answer{%
    $\pm \frac 1F = \frac 1a - \frac 1b \implies F = \frac{a b}{\abs{b - a}} \approx 16{,}7\,\text{см}$
}
\solutionspace{180pt}

\tasknumber{4}%
\task{%
    Две тонкие собирающие линзы с фокусными расстояниями $25\,\text{см}$ и $20\,\text{см}$ сложены вместе.
    Чему равно фокусное расстояние такой оптической системы?
}
\answer{%
    $\frac 1{f_1} = \frac 1a + \frac 1b; \frac 1{f_2} = - \frac 1b + \frac 1c \implies \frac 1{f_1} + \frac 1{f_2} = \frac 1a + \frac 1c \implies f' = \frac 1{\frac 1{f_1} + \frac 1{f_2}} = \frac{f_1 f_2}{f_1 + f_2} \approx 11{,}1\,\text{см}$
}
\solutionspace{180pt}

\tasknumber{5}%
\task{%
    Линейные размеры прямого изображения предмета, полученного в собирающей линзе,
    в два раза больше линейных размеров предмета.
    Зная, что предмет находится на $30\,\text{см}$ ближе к линзе,
    чем его изображение, найти оптическую силу линзы.
}
\answer{%
    \begin{align*}
    D &= \frac 1F = \frac 1a + \frac 1b, \qquad \Gamma = \frac ba, \qquad b - a = \ell \implies b = \Gamma a \implies \Gamma a - a = \ell \implies  \\
    a &= \frac {\ell}{\Gamma - 1} \implies b = \frac {{\ell} \Gamma}{\Gamma - 1} \implies  \\
    D &= \frac {\Gamma - 1}\ell + \frac {\Gamma - 1}{\ell \Gamma} = \frac 1\ell \cdot \cbr{\Gamma - 1 + \frac {\Gamma - 1}{\Gamma} } =\frac 1\ell \cdot \cbr{\Gamma - \frac 1\Gamma} \approx 5\,\text{дптр}.
    \end{align*}
}
\solutionspace{180pt}

\tasknumber{6}%
\task{%
    Оптическая сила объектива фотоаппарата равна $4\,\text{дптр}$.
    При фотографировании чертежа с расстояния $1{,}1\,\text{м}$ площадь изображения
    чертежа на фотопластинке оказалась равной $9\,\text{см}^{2}$.
    Какова площадь самого чертежа? Ответ выразите в квадратных сантиметрах.
}
\solutionspace{180pt}

\tasknumber{7}%
\task{%
    В каком месте на главной оптической оси двояковыгнутой линзы
    нужно поместить точечный источник света,
    чтобы его изображение оказалось в главном фокусе линзы?
}
\answer{%
    $\text{на половине фокусного расстояния}$
}
\solutionspace{120pt}

\tasknumber{8}%
\task{%
    Предмет высотой $h = 30\,\text{см}$ находится на расстоянии $d = 1\,\text{м}$
    от вертикально расположенной рассеивающей линзы с фокусным расстоянием $F = -15\,\text{см}$.
    Где находится изображение предмета? Определите тип изображения и его высоту.
}
\solutionspace{120pt}

\tasknumber{9}%
\task{%
    На каком расстоянии от двояковыпуклой линзы с оптической силой $D = 1{,}5\,\text{дптр}$
    надо поместить предмет, чтобы его изображение получилось на расстоянии $2{,}5\,\text{м}$ от линзы?
}
\solutionspace{120pt}

\tasknumber{10}%
\task{%
    Предмет в виде отрезка длиной $\ell$ расположен вдоль оптической оси
    собирающей линзы с фокусным расстоянием $F$.
    Середина отрезка расположена
    на расстоянии $a$ от линзы, которая даёт действительное изображение
    всех точек предмета.
    Определить продольное увеличение предмета.
}
\answer{%
    \begin{align*}
    \frac 1{a + \frac \ell 2} &+ \frac 1b = \frac 1F \implies b = \frac{F\cbr{a + \frac \ell 2}}{a + \frac \ell 2 - F} \\
    \frac 1{a - \frac \ell 2} &+ \frac 1c = \frac 1F \implies c = \frac{F\cbr{a - \frac \ell 2}}{a - \frac \ell 2 - F} \\
    \abs{b - c} &= \abs{\frac{F\cbr{a + \frac \ell 2}}{a + \frac \ell 2 - F} - \frac{F\cbr{a - \frac \ell 2}}{a - \frac \ell 2 - F}}= F\abs{\frac{\cbr{a + \frac \ell 2}\cbr{a - \frac \ell 2 - F} - \cbr{a - \frac \ell 2}\cbr{a + \frac \ell 2 - F}}{ \cbr{a + \frac \ell 2 - F} \cbr{a - \frac \ell 2 - F} }} =  \\
    &= F\abs{\frac{a^2 - \frac {a\ell} 2 - Fa + \frac {a\ell} 2 - \frac {\ell^2} 4 - \frac {F\ell}2 - a^2 - \frac {a\ell}2 + aF + \frac {a\ell}2 + \frac {\ell^2} 4 - \frac {F\ell} 2}{\cbr{a + \frac \ell 2 - F} \cbr{a - \frac \ell 2 - F} }} = \\
    &= F\frac{F\ell}{\sqr{a-F} - \frac {\ell^2}4} = \frac{F^2\ell}{\sqr{a-F} - \frac {\ell^2}4}\implies \Gamma = \frac{\abs{b - c}}\ell = \frac{F^2}{\sqr{a-F} - \frac {\ell^2}4}.
    \end{align*}
}
\solutionspace{120pt}

\tasknumber{11}%
\task{%
    На экране с помощью тонкой линзы получено изображение предмета
    с увеличением $2$.
    Предмет передвинули на $2\,\text{см}$.
    Для того, чтобы получить резкое изображение, пришлось передвинуть экран.
    При этом увеличение оказалось равным $8$.
    На какое расстояние
    пришлось передвинуть экран?
}
\solutionspace{120pt}

\tasknumber{12}%
\task{%
    Тонкая собирающая линза дает изображение предмета на экране высотой $H_1$,
    и $H_2$, при двух положениях линзы между предметом и экраном.
    Расстояние между ними неизменно.
    Чему равна высота предмета $h$?
}
\answer{%
    $h = \sqrt{H_1 H_2}$
}
\solutionspace{120pt}

\tasknumber{13}%
\task{%
    Какие предметы можно рассмотреть на фотографии, сделанной со спутника,
    если разрешающая способность пленки $0{,}02\,\text{мм}$? Каким должно быть
    время экспозиции $\tau$ чтобы полностью использовать возможности пленки?
    Фокусное расстояние объектива используемого фотоаппарата $15\,\text{cм}$,
    высота орбиты спутника $120\,\text{км}$.
}
\solutionspace{120pt}

\tasknumber{14}%
\task{%
    При аэрофотосъемках используется фотоаппарат, объектив которого
    имеет фокусиое расстояние $12\,\text{cм}$.
    Разрешающая способность пленки $0{,}02\,\text{мм}$.
    На какой высоте должен лететь самолет, чтобы на фотографии можно
    было различить листья деревьев размером $6\,\text{cм}$?
    При какой скорости самолета изображение не будет размытым,
    если время зкспозиции $1\,\text{мс}$?
}
% autogenerated
