\setdate{14~сентября~2021}
\setclass{11«БА»}

\addpersonalvariant{Михаил Бурмистров}

\tasknumber{1}%
\task{%
    Рядом с каждой единицей измерения укажите физическую величину, которая в ней измеряется, и один из вариантов обозначений этой физической величины.
    \begin{enumerate}
        \item Кл,
        \item м,
        \item радиан.
    \end{enumerate}
}
\solutionspace{40pt}

\tasknumber{2}%
\task{%
    Запишите формулой закон для вычисления модуля силы, действующей
    на проводник, по которому течёт электрический ток, в магнитном поле, и выразите из него индукцию магнитного поля.
}
\solutionspace{40pt}

\tasknumber{3}%
\task{%
    Запишите формулой закон Ампера и укажите
    для каждой величины её название и единицы измерения в системе СИ.
}
\solutionspace{80pt}

\tasknumber{4}%
\task{%
    Положительно заряженная частица движется со скоростью $v$ в магнитном поле перпендикулярно линиям его индукции.
    Индукция магнитного поля равна $B$, масса частицы $m$, её заряд — $q$.
    Выведите из базовых физических законов формулы для радиуса траектории частицы и её периода обращения.
}
\answer{%
    $
        F = ma, F = qvB, a = v^2 / R \implies R = \frac{mv}{qB}.
        \quad T = \frac{2\pi R}{v} = \frac{2\pi m}{qB}.
        \quad \omega = \frac vR = \frac{qB}{m}.
        \quad \nu = \frac 1T = \frac{qB}{2\pi m}.
    $
}
\solutionspace{120pt}

\tasknumber{5}%
\task{%
    Протон со скоростью $3 \cdot 10^{4}\,\frac{\text{км}}{\text{c}}$ влетает в магнитное поле индукцией $20\,\text{мТл}$ перпендикулярно линиям его индукции.
    Определите радиус траектории частицы, для вычислений воспользуйтесь табличными значениями.
}
\answer{%
    $
        F = ma, F = evB, a = v^2 / R \implies R = \frac{mv}{eB} = \frac{1{,}672 \cdot 10^{-27}\,\text{кг} \cdot 3 \cdot 10^{4}\,\frac{\text{км}}{\text{c}}}{1{,}6 \cdot 10^{-19}\,\text{Кл} \cdot 20\,\text{мТл}} \approx 15{,}675000\,\text{м}.
    $
}
\solutionspace{120pt}

\tasknumber{6}%
\task{%
    Заряженная частица находится в постоянном магнитном поле, а её мгновенная скорость параллельна линиям его индукции.
    По какой траектории движется частица? (прямая, гипербола, парабола, окружность, спираль, ...)
    Действием всех других полей пренебречь.
}
\answer{%
    $\text{Прямая}$
}

\variantsplitter

\addpersonalvariant{Ирина Ан}

\tasknumber{1}%
\task{%
    Рядом с каждой единицей измерения укажите физическую величину, которая в ней измеряется, и один из вариантов обозначений этой физической величины.
    \begin{enumerate}
        \item Кл,
        \item м,
        \item радиан.
    \end{enumerate}
}
\solutionspace{40pt}

\tasknumber{2}%
\task{%
    Запишите формулой закон для вычисления модуля силы, действующей
    на заряженную частицу, движущуюся в магнитном поле, и выразите из него индукцию магнитного поля.
}
\solutionspace{40pt}

\tasknumber{3}%
\task{%
    Запишите формулой закон Ампера и укажите
    для каждой величины её название и единицы измерения в системе СИ.
}
\solutionspace{80pt}

\tasknumber{4}%
\task{%
    Положительно заряженная частица движется со скоростью $v$ в магнитном поле перпендикулярно линиям его индукции.
    Индукция магнитного поля равна $B$, масса частицы $m$, её заряд — $q$.
    Выведите из базовых физических законов формулы для радиуса траектории частицы и её периода обращения.
}
\answer{%
    $
        F = ma, F = qvB, a = v^2 / R \implies R = \frac{mv}{qB}.
        \quad T = \frac{2\pi R}{v} = \frac{2\pi m}{qB}.
        \quad \omega = \frac vR = \frac{qB}{m}.
        \quad \nu = \frac 1T = \frac{qB}{2\pi m}.
    $
}
\solutionspace{120pt}

\tasknumber{5}%
\task{%
    Электрон со скоростью $5 \cdot 10^{3}\,\frac{\text{км}}{\text{c}}$ влетает в магнитное поле индукцией $200\,\text{мТл}$ перпендикулярно линиям его индукции.
    Определите радиус траектории частицы, для вычислений воспользуйтесь табличными значениями.
}
\answer{%
    $
        F = ma, F = evB, a = v^2 / R \implies R = \frac{mv}{eB} = \frac{9{,}1 \cdot 10^{-31}\,\text{кг} \cdot 5 \cdot 10^{3}\,\frac{\text{км}}{\text{c}}}{1{,}6 \cdot 10^{-19}\,\text{Кл} \cdot 200\,\text{мТл}} \approx 0{,}000142\,\text{м}.
    $
}
\solutionspace{120pt}

\tasknumber{6}%
\task{%
    Заряженная частица находится в постоянном электростатическом поле, а её мгновенная скорость параллельна линиям его напряжённости.
    По какой траектории движется частица? (прямая, гипербола, парабола, окружность, спираль, ...)
    Действием всех других полей пренебречь.
}
\answer{%
    $\text{Прямая}$
}

\variantsplitter

\addpersonalvariant{Софья Андрианова}

\tasknumber{1}%
\task{%
    Рядом с каждой единицей измерения укажите физическую величину, которая в ней измеряется, и один из вариантов обозначений этой физической величины.
    \begin{enumerate}
        \item Тл,
        \item м/с,
        \item А.
    \end{enumerate}
}
\solutionspace{40pt}

\tasknumber{2}%
\task{%
    Запишите формулой закон для вычисления модуля силы, действующей
    на заряженную частицу, движущуюся в магнитном поле, и выразите из него индукцию магнитного поля.
}
\solutionspace{40pt}

\tasknumber{3}%
\task{%
    Запишите формулой закон Лоренца и укажите
    для каждой величины её название и единицы измерения в системе СИ.
}
\solutionspace{80pt}

\tasknumber{4}%
\task{%
    Положительно заряженная частица движется со скоростью $v$ в магнитном поле перпендикулярно линиям его индукции.
    Индукция магнитного поля равна $B$, масса частицы $m$, её заряд — $q$.
    Выведите из базовых физических законов формулы для радиуса траектории частицы и её периода обращения.
}
\answer{%
    $
        F = ma, F = qvB, a = v^2 / R \implies R = \frac{mv}{qB}.
        \quad T = \frac{2\pi R}{v} = \frac{2\pi m}{qB}.
        \quad \omega = \frac vR = \frac{qB}{m}.
        \quad \nu = \frac 1T = \frac{qB}{2\pi m}.
    $
}
\solutionspace{120pt}

\tasknumber{5}%
\task{%
    Электрон со скоростью $5 \cdot 10^{3}\,\frac{\text{км}}{\text{c}}$ влетает в магнитное поле индукцией $40\,\text{мТл}$ перпендикулярно линиям его индукции.
    Определите радиус траектории частицы, для вычислений воспользуйтесь табличными значениями.
}
\answer{%
    $
        F = ma, F = evB, a = v^2 / R \implies R = \frac{mv}{eB} = \frac{9{,}1 \cdot 10^{-31}\,\text{кг} \cdot 5 \cdot 10^{3}\,\frac{\text{км}}{\text{c}}}{1{,}6 \cdot 10^{-19}\,\text{Кл} \cdot 40\,\text{мТл}} \approx 0{,}000711\,\text{м}.
    $
}
\solutionspace{120pt}

\tasknumber{6}%
\task{%
    Незаряженная частица находится в постоянном электростатическом поле, а её мгновенная скорость параллельна линиям его напряжённости.
    По какой траектории движется частица? (прямая, гипербола, парабола, окружность, спираль, ...)
    Действием всех других полей пренебречь.
}
\answer{%
    $\text{Прямая}$
}

\variantsplitter

\addpersonalvariant{Владимир Артемчук}

\tasknumber{1}%
\task{%
    Рядом с каждой единицей измерения укажите физическую величину, которая в ней измеряется, и один из вариантов обозначений этой физической величины.
    \begin{enumerate}
        \item Тл,
        \item м/с,
        \item А.
    \end{enumerate}
}
\solutionspace{40pt}

\tasknumber{2}%
\task{%
    Запишите формулой закон для вычисления модуля силы, действующей
    на заряженную частицу, движущуюся в магнитном поле, и выразите из него значение угла.
}
\solutionspace{40pt}

\tasknumber{3}%
\task{%
    Запишите формулой закон Лоренца и укажите
    для каждой величины её название и единицы измерения в системе СИ.
}
\solutionspace{80pt}

\tasknumber{4}%
\task{%
    Положительно заряженная частица движется со скоростью $v$ в магнитном поле перпендикулярно линиям его индукции.
    Индукция магнитного поля равна $B$, масса частицы $m$, её заряд — $q$.
    Выведите из базовых физических законов формулы для радиуса траектории частицы и её угловой скорости.
}
\answer{%
    $
        F = ma, F = qvB, a = v^2 / R \implies R = \frac{mv}{qB}.
        \quad T = \frac{2\pi R}{v} = \frac{2\pi m}{qB}.
        \quad \omega = \frac vR = \frac{qB}{m}.
        \quad \nu = \frac 1T = \frac{qB}{2\pi m}.
    $
}
\solutionspace{120pt}

\tasknumber{5}%
\task{%
    Протон со скоростью $3 \cdot 10^{4}\,\frac{\text{км}}{\text{c}}$ влетает в магнитное поле индукцией $50\,\text{мТл}$ перпендикулярно линиям его индукции.
    Определите радиус траектории частицы, для вычислений воспользуйтесь табличными значениями.
}
\answer{%
    $
        F = ma, F = evB, a = v^2 / R \implies R = \frac{mv}{eB} = \frac{1{,}672 \cdot 10^{-27}\,\text{кг} \cdot 3 \cdot 10^{4}\,\frac{\text{км}}{\text{c}}}{1{,}6 \cdot 10^{-19}\,\text{Кл} \cdot 50\,\text{мТл}} \approx 6{,}270000\,\text{м}.
    $
}
\solutionspace{120pt}

\tasknumber{6}%
\task{%
    Незаряженная частица находится в постоянном магнитном поле, а её мгновенная скорость параллельна линиям его индукции.
    По какой траектории движется частица? (прямая, гипербола, парабола, окружность, спираль, ...)
    Действием всех других полей пренебречь.
}
\answer{%
    $\text{Прямая}$
}

\variantsplitter

\addpersonalvariant{Софья Белянкина}

\tasknumber{1}%
\task{%
    Рядом с каждой единицей измерения укажите физическую величину, которая в ней измеряется, и один из вариантов обозначений этой физической величины.
    \begin{enumerate}
        \item Тл,
        \item м,
        \item радиан.
    \end{enumerate}
}
\solutionspace{40pt}

\tasknumber{2}%
\task{%
    Запишите формулой закон для вычисления модуля силы, действующей
    на заряженную частицу, движущуюся в магнитном поле, и выразите из него значение угла.
}
\solutionspace{40pt}

\tasknumber{3}%
\task{%
    Запишите формулой закон Лоренца и укажите
    для каждой величины её название и единицы измерения в системе СИ.
}
\solutionspace{80pt}

\tasknumber{4}%
\task{%
    Положительно заряженная частица движется со скоростью $v$ в магнитном поле перпендикулярно линиям его индукции.
    Индукция магнитного поля равна $B$, масса частицы $m$, её заряд — $q$.
    Выведите из базовых физических законов формулы для радиуса траектории частицы и её периода обращения.
}
\answer{%
    $
        F = ma, F = qvB, a = v^2 / R \implies R = \frac{mv}{qB}.
        \quad T = \frac{2\pi R}{v} = \frac{2\pi m}{qB}.
        \quad \omega = \frac vR = \frac{qB}{m}.
        \quad \nu = \frac 1T = \frac{qB}{2\pi m}.
    $
}
\solutionspace{120pt}

\tasknumber{5}%
\task{%
    Электрон со скоростью $2 \cdot 10^{3}\,\frac{\text{км}}{\text{c}}$ влетает в магнитное поле индукцией $500\,\text{мТл}$ перпендикулярно линиям его индукции.
    Определите радиус траектории частицы, для вычислений воспользуйтесь табличными значениями.
}
\answer{%
    $
        F = ma, F = evB, a = v^2 / R \implies R = \frac{mv}{eB} = \frac{9{,}1 \cdot 10^{-31}\,\text{кг} \cdot 2 \cdot 10^{3}\,\frac{\text{км}}{\text{c}}}{1{,}6 \cdot 10^{-19}\,\text{Кл} \cdot 500\,\text{мТл}} \approx 0{,}000023\,\text{м}.
    $
}
\solutionspace{120pt}

\tasknumber{6}%
\task{%
    Незаряженная частица находится в постоянном магнитном поле, а её мгновенная скорость перпендикулярна линиям его индукции.
    По какой траектории движется частица? (прямая, гипербола, парабола, окружность, спираль, ...)
    Действием всех других полей пренебречь.
}
\answer{%
    $\text{Прямая}$
}

\variantsplitter

\addpersonalvariant{Варвара Егиазарян}

\tasknumber{1}%
\task{%
    Рядом с каждой единицей измерения укажите физическую величину, которая в ней измеряется, и один из вариантов обозначений этой физической величины.
    \begin{enumerate}
        \item Кл,
        \item м/с,
        \item А.
    \end{enumerate}
}
\solutionspace{40pt}

\tasknumber{2}%
\task{%
    Запишите формулой закон для вычисления модуля силы, действующей
    на заряженную частицу, движущуюся в магнитном поле, и выразите из него индукцию магнитного поля.
}
\solutionspace{40pt}

\tasknumber{3}%
\task{%
    Запишите формулой закон Лоренца и укажите
    для каждой величины её название и единицы измерения в системе СИ.
}
\solutionspace{80pt}

\tasknumber{4}%
\task{%
    Положительно заряженная частица движется со скоростью $v$ в магнитном поле перпендикулярно линиям его индукции.
    Индукция магнитного поля равна $B$, масса частицы $m$, её заряд — $q$.
    Выведите из базовых физических законов формулы для радиуса траектории частицы и её угловой скорости.
}
\answer{%
    $
        F = ma, F = qvB, a = v^2 / R \implies R = \frac{mv}{qB}.
        \quad T = \frac{2\pi R}{v} = \frac{2\pi m}{qB}.
        \quad \omega = \frac vR = \frac{qB}{m}.
        \quad \nu = \frac 1T = \frac{qB}{2\pi m}.
    $
}
\solutionspace{120pt}

\tasknumber{5}%
\task{%
    Протон со скоростью $2 \cdot 10^{4}\,\frac{\text{км}}{\text{c}}$ влетает в магнитное поле индукцией $500\,\text{мТл}$ перпендикулярно линиям его индукции.
    Определите радиус траектории частицы, для вычислений воспользуйтесь табличными значениями.
}
\answer{%
    $
        F = ma, F = evB, a = v^2 / R \implies R = \frac{mv}{eB} = \frac{1{,}672 \cdot 10^{-27}\,\text{кг} \cdot 2 \cdot 10^{4}\,\frac{\text{км}}{\text{c}}}{1{,}6 \cdot 10^{-19}\,\text{Кл} \cdot 500\,\text{мТл}} \approx 0{,}418000\,\text{м}.
    $
}
\solutionspace{120pt}

\tasknumber{6}%
\task{%
    Заряженная частица находится в постоянном электростатическом поле, а её мгновенная скорость перпендикулярна линиям его напряжённости.
    По какой траектории движется частица? (прямая, гипербола, парабола, окружность, спираль, ...)
    Действием всех других полей пренебречь.
}
\answer{%
    $\text{Парабола}$
}

\variantsplitter

\addpersonalvariant{Владислав Емелин}

\tasknumber{1}%
\task{%
    Рядом с каждой единицей измерения укажите физическую величину, которая в ней измеряется, и один из вариантов обозначений этой физической величины.
    \begin{enumerate}
        \item Кл,
        \item м/с,
        \item А.
    \end{enumerate}
}
\solutionspace{40pt}

\tasknumber{2}%
\task{%
    Запишите формулой закон для вычисления модуля силы, действующей
    на проводник, по которому течёт электрический ток, в магнитном поле, и выразите из него значение угла.
}
\solutionspace{40pt}

\tasknumber{3}%
\task{%
    Запишите формулой закон Ампера и укажите
    для каждой величины её название и единицы измерения в системе СИ.
}
\solutionspace{80pt}

\tasknumber{4}%
\task{%
    Положительно заряженная частица движется со скоростью $v$ в магнитном поле перпендикулярно линиям его индукции.
    Индукция магнитного поля равна $B$, масса частицы $m$, её заряд — $q$.
    Выведите из базовых физических законов формулы для радиуса траектории частицы и её периода обращения.
}
\answer{%
    $
        F = ma, F = qvB, a = v^2 / R \implies R = \frac{mv}{qB}.
        \quad T = \frac{2\pi R}{v} = \frac{2\pi m}{qB}.
        \quad \omega = \frac vR = \frac{qB}{m}.
        \quad \nu = \frac 1T = \frac{qB}{2\pi m}.
    $
}
\solutionspace{120pt}

\tasknumber{5}%
\task{%
    Электрон со скоростью $5 \cdot 10^{3}\,\frac{\text{км}}{\text{c}}$ влетает в магнитное поле индукцией $40\,\text{мТл}$ перпендикулярно линиям его индукции.
    Определите радиус траектории частицы, для вычислений воспользуйтесь табличными значениями.
}
\answer{%
    $
        F = ma, F = evB, a = v^2 / R \implies R = \frac{mv}{eB} = \frac{9{,}1 \cdot 10^{-31}\,\text{кг} \cdot 5 \cdot 10^{3}\,\frac{\text{км}}{\text{c}}}{1{,}6 \cdot 10^{-19}\,\text{Кл} \cdot 40\,\text{мТл}} \approx 0{,}000711\,\text{м}.
    $
}
\solutionspace{120pt}

\tasknumber{6}%
\task{%
    Незаряженная частица находится в постоянном магнитном поле, а её мгновенная скорость перпендикулярна линиям его индукции.
    По какой траектории движется частица? (прямая, гипербола, парабола, окружность, спираль, ...)
    Действием всех других полей пренебречь.
}
\answer{%
    $\text{Прямая}$
}

\variantsplitter

\addpersonalvariant{Артём Жичин}

\tasknumber{1}%
\task{%
    Рядом с каждой единицей измерения укажите физическую величину, которая в ней измеряется, и один из вариантов обозначений этой физической величины.
    \begin{enumerate}
        \item Кл,
        \item м/с,
        \item А.
    \end{enumerate}
}
\solutionspace{40pt}

\tasknumber{2}%
\task{%
    Запишите формулой закон для вычисления модуля силы, действующей
    на проводник, по которому течёт электрический ток, в магнитном поле, и выразите из него значение угла.
}
\solutionspace{40pt}

\tasknumber{3}%
\task{%
    Запишите формулой закон Лоренца и укажите
    для каждой величины её название и единицы измерения в системе СИ.
}
\solutionspace{80pt}

\tasknumber{4}%
\task{%
    Положительно заряженная частица движется со скоростью $v$ в магнитном поле перпендикулярно линиям его индукции.
    Индукция магнитного поля равна $B$, масса частицы $m$, её заряд — $q$.
    Выведите из базовых физических законов формулы для радиуса траектории частицы и её угловой скорости.
}
\answer{%
    $
        F = ma, F = qvB, a = v^2 / R \implies R = \frac{mv}{qB}.
        \quad T = \frac{2\pi R}{v} = \frac{2\pi m}{qB}.
        \quad \omega = \frac vR = \frac{qB}{m}.
        \quad \nu = \frac 1T = \frac{qB}{2\pi m}.
    $
}
\solutionspace{120pt}

\tasknumber{5}%
\task{%
    Электрон со скоростью $3 \cdot 10^{4}\,\frac{\text{км}}{\text{c}}$ влетает в магнитное поле индукцией $20\,\text{мТл}$ перпендикулярно линиям его индукции.
    Определите радиус траектории частицы, для вычислений воспользуйтесь табличными значениями.
}
\answer{%
    $
        F = ma, F = evB, a = v^2 / R \implies R = \frac{mv}{eB} = \frac{9{,}1 \cdot 10^{-31}\,\text{кг} \cdot 3 \cdot 10^{4}\,\frac{\text{км}}{\text{c}}}{1{,}6 \cdot 10^{-19}\,\text{Кл} \cdot 20\,\text{мТл}} \approx 0{,}008531\,\text{м}.
    $
}
\solutionspace{120pt}

\tasknumber{6}%
\task{%
    Незаряженная частица находится в постоянном электростатическом поле, а её мгновенная скорость параллельна линиям его напряжённости.
    По какой траектории движется частица? (прямая, гипербола, парабола, окружность, спираль, ...)
    Действием всех других полей пренебречь.
}
\answer{%
    $\text{Прямая}$
}

\variantsplitter

\addpersonalvariant{Дарья Кошман}

\tasknumber{1}%
\task{%
    Рядом с каждой единицей измерения укажите физическую величину, которая в ней измеряется, и один из вариантов обозначений этой физической величины.
    \begin{enumerate}
        \item Тл,
        \item м/с,
        \item А.
    \end{enumerate}
}
\solutionspace{40pt}

\tasknumber{2}%
\task{%
    Запишите формулой закон для вычисления модуля силы, действующей
    на заряженную частицу, движущуюся в магнитном поле, и выразите из него значение угла.
}
\solutionspace{40pt}

\tasknumber{3}%
\task{%
    Запишите формулой закон Лоренца и укажите
    для каждой величины её название и единицы измерения в системе СИ.
}
\solutionspace{80pt}

\tasknumber{4}%
\task{%
    Положительно заряженная частица движется со скоростью $v$ в магнитном поле перпендикулярно линиям его индукции.
    Индукция магнитного поля равна $B$, масса частицы $m$, её заряд — $q$.
    Выведите из базовых физических законов формулы для радиуса траектории частицы и её периода обращения.
}
\answer{%
    $
        F = ma, F = qvB, a = v^2 / R \implies R = \frac{mv}{qB}.
        \quad T = \frac{2\pi R}{v} = \frac{2\pi m}{qB}.
        \quad \omega = \frac vR = \frac{qB}{m}.
        \quad \nu = \frac 1T = \frac{qB}{2\pi m}.
    $
}
\solutionspace{120pt}

\tasknumber{5}%
\task{%
    Протон со скоростью $2 \cdot 10^{3}\,\frac{\text{км}}{\text{c}}$ влетает в магнитное поле индукцией $500\,\text{мТл}$ перпендикулярно линиям его индукции.
    Определите радиус траектории частицы, для вычислений воспользуйтесь табличными значениями.
}
\answer{%
    $
        F = ma, F = evB, a = v^2 / R \implies R = \frac{mv}{eB} = \frac{1{,}672 \cdot 10^{-27}\,\text{кг} \cdot 2 \cdot 10^{3}\,\frac{\text{км}}{\text{c}}}{1{,}6 \cdot 10^{-19}\,\text{Кл} \cdot 500\,\text{мТл}} \approx 0{,}041800\,\text{м}.
    $
}
\solutionspace{120pt}

\tasknumber{6}%
\task{%
    Заряженная частица находится в постоянном электростатическом поле, а её мгновенная скорость параллельна линиям его напряжённости.
    По какой траектории движется частица? (прямая, гипербола, парабола, окружность, спираль, ...)
    Действием всех других полей пренебречь.
}
\answer{%
    $\text{Прямая}$
}

\variantsplitter

\addpersonalvariant{Анна Кузьмичёва}

\tasknumber{1}%
\task{%
    Рядом с каждой единицей измерения укажите физическую величину, которая в ней измеряется, и один из вариантов обозначений этой физической величины.
    \begin{enumerate}
        \item Кл,
        \item м/с,
        \item А.
    \end{enumerate}
}
\solutionspace{40pt}

\tasknumber{2}%
\task{%
    Запишите формулой закон для вычисления модуля силы, действующей
    на заряженную частицу, движущуюся в магнитном поле, и выразите из него значение угла.
}
\solutionspace{40pt}

\tasknumber{3}%
\task{%
    Запишите формулой закон Лоренца и укажите
    для каждой величины её название и единицы измерения в системе СИ.
}
\solutionspace{80pt}

\tasknumber{4}%
\task{%
    Положительно заряженная частица движется со скоростью $v$ в магнитном поле перпендикулярно линиям его индукции.
    Индукция магнитного поля равна $B$, масса частицы $m$, её заряд — $q$.
    Выведите из базовых физических законов формулы для радиуса траектории частицы и её угловой скорости.
}
\answer{%
    $
        F = ma, F = qvB, a = v^2 / R \implies R = \frac{mv}{qB}.
        \quad T = \frac{2\pi R}{v} = \frac{2\pi m}{qB}.
        \quad \omega = \frac vR = \frac{qB}{m}.
        \quad \nu = \frac 1T = \frac{qB}{2\pi m}.
    $
}
\solutionspace{120pt}

\tasknumber{5}%
\task{%
    Протон со скоростью $3 \cdot 10^{3}\,\frac{\text{км}}{\text{c}}$ влетает в магнитное поле индукцией $40\,\text{мТл}$ перпендикулярно линиям его индукции.
    Определите радиус траектории частицы, для вычислений воспользуйтесь табличными значениями.
}
\answer{%
    $
        F = ma, F = evB, a = v^2 / R \implies R = \frac{mv}{eB} = \frac{1{,}672 \cdot 10^{-27}\,\text{кг} \cdot 3 \cdot 10^{3}\,\frac{\text{км}}{\text{c}}}{1{,}6 \cdot 10^{-19}\,\text{Кл} \cdot 40\,\text{мТл}} \approx 0{,}783750\,\text{м}.
    $
}
\solutionspace{120pt}

\tasknumber{6}%
\task{%
    Заряженная частица находится в постоянном электростатическом поле, а её мгновенная скорость параллельна линиям его напряжённости.
    По какой траектории движется частица? (прямая, гипербола, парабола, окружность, спираль, ...)
    Действием всех других полей пренебречь.
}
\answer{%
    $\text{Прямая}$
}

\variantsplitter

\addpersonalvariant{Алёна Куприянова}

\tasknumber{1}%
\task{%
    Рядом с каждой единицей измерения укажите физическую величину, которая в ней измеряется, и один из вариантов обозначений этой физической величины.
    \begin{enumerate}
        \item Кл,
        \item м/с,
        \item А.
    \end{enumerate}
}
\solutionspace{40pt}

\tasknumber{2}%
\task{%
    Запишите формулой закон для вычисления модуля силы, действующей
    на проводник, по которому течёт электрический ток, в магнитном поле, и выразите из него индукцию магнитного поля.
}
\solutionspace{40pt}

\tasknumber{3}%
\task{%
    Запишите формулой закон Лоренца и укажите
    для каждой величины её название и единицы измерения в системе СИ.
}
\solutionspace{80pt}

\tasknumber{4}%
\task{%
    Положительно заряженная частица движется со скоростью $v$ в магнитном поле перпендикулярно линиям его индукции.
    Индукция магнитного поля равна $B$, масса частицы $m$, её заряд — $q$.
    Выведите из базовых физических законов формулы для радиуса траектории частицы и её частоты обращения.
}
\answer{%
    $
        F = ma, F = qvB, a = v^2 / R \implies R = \frac{mv}{qB}.
        \quad T = \frac{2\pi R}{v} = \frac{2\pi m}{qB}.
        \quad \omega = \frac vR = \frac{qB}{m}.
        \quad \nu = \frac 1T = \frac{qB}{2\pi m}.
    $
}
\solutionspace{120pt}

\tasknumber{5}%
\task{%
    Протон со скоростью $2 \cdot 10^{3}\,\frac{\text{км}}{\text{c}}$ влетает в магнитное поле индукцией $200\,\text{мТл}$ перпендикулярно линиям его индукции.
    Определите радиус траектории частицы, для вычислений воспользуйтесь табличными значениями.
}
\answer{%
    $
        F = ma, F = evB, a = v^2 / R \implies R = \frac{mv}{eB} = \frac{1{,}672 \cdot 10^{-27}\,\text{кг} \cdot 2 \cdot 10^{3}\,\frac{\text{км}}{\text{c}}}{1{,}6 \cdot 10^{-19}\,\text{Кл} \cdot 200\,\text{мТл}} \approx 0{,}104500\,\text{м}.
    $
}
\solutionspace{120pt}

\tasknumber{6}%
\task{%
    Незаряженная частица находится в постоянном электростатическом поле, а её мгновенная скорость перпендикулярна линиям его напряжённости.
    По какой траектории движется частица? (прямая, гипербола, парабола, окружность, спираль, ...)
    Действием всех других полей пренебречь.
}
\answer{%
    $\text{Прямая}$
}

\variantsplitter

\addpersonalvariant{Ярослав Лавровский}

\tasknumber{1}%
\task{%
    Рядом с каждой единицей измерения укажите физическую величину, которая в ней измеряется, и один из вариантов обозначений этой физической величины.
    \begin{enumerate}
        \item Кл,
        \item м,
        \item радиан.
    \end{enumerate}
}
\solutionspace{40pt}

\tasknumber{2}%
\task{%
    Запишите формулой закон для вычисления модуля силы, действующей
    на проводник, по которому течёт электрический ток, в магнитном поле, и выразите из него значение угла.
}
\solutionspace{40pt}

\tasknumber{3}%
\task{%
    Запишите формулой закон Лоренца и укажите
    для каждой величины её название и единицы измерения в системе СИ.
}
\solutionspace{80pt}

\tasknumber{4}%
\task{%
    Положительно заряженная частица движется со скоростью $v$ в магнитном поле перпендикулярно линиям его индукции.
    Индукция магнитного поля равна $B$, масса частицы $m$, её заряд — $q$.
    Выведите из базовых физических законов формулы для радиуса траектории частицы и её периода обращения.
}
\answer{%
    $
        F = ma, F = qvB, a = v^2 / R \implies R = \frac{mv}{qB}.
        \quad T = \frac{2\pi R}{v} = \frac{2\pi m}{qB}.
        \quad \omega = \frac vR = \frac{qB}{m}.
        \quad \nu = \frac 1T = \frac{qB}{2\pi m}.
    $
}
\solutionspace{120pt}

\tasknumber{5}%
\task{%
    Электрон со скоростью $5 \cdot 10^{4}\,\frac{\text{км}}{\text{c}}$ влетает в магнитное поле индукцией $50\,\text{мТл}$ перпендикулярно линиям его индукции.
    Определите радиус траектории частицы, для вычислений воспользуйтесь табличными значениями.
}
\answer{%
    $
        F = ma, F = evB, a = v^2 / R \implies R = \frac{mv}{eB} = \frac{9{,}1 \cdot 10^{-31}\,\text{кг} \cdot 5 \cdot 10^{4}\,\frac{\text{км}}{\text{c}}}{1{,}6 \cdot 10^{-19}\,\text{Кл} \cdot 50\,\text{мТл}} \approx 0{,}005687\,\text{м}.
    $
}
\solutionspace{120pt}

\tasknumber{6}%
\task{%
    Незаряженная частица находится в постоянном электростатическом поле, а её мгновенная скорость перпендикулярна линиям его напряжённости.
    По какой траектории движется частица? (прямая, гипербола, парабола, окружность, спираль, ...)
    Действием всех других полей пренебречь.
}
\answer{%
    $\text{Прямая}$
}

\variantsplitter

\addpersonalvariant{Анастасия Ламанова}

\tasknumber{1}%
\task{%
    Рядом с каждой единицей измерения укажите физическую величину, которая в ней измеряется, и один из вариантов обозначений этой физической величины.
    \begin{enumerate}
        \item Кл,
        \item м/с,
        \item А.
    \end{enumerate}
}
\solutionspace{40pt}

\tasknumber{2}%
\task{%
    Запишите формулой закон для вычисления модуля силы, действующей
    на заряженную частицу, движущуюся в магнитном поле, и выразите из него индукцию магнитного поля.
}
\solutionspace{40pt}

\tasknumber{3}%
\task{%
    Запишите формулой закон Лоренца и укажите
    для каждой величины её название и единицы измерения в системе СИ.
}
\solutionspace{80pt}

\tasknumber{4}%
\task{%
    Положительно заряженная частица движется со скоростью $v$ в магнитном поле перпендикулярно линиям его индукции.
    Индукция магнитного поля равна $B$, масса частицы $m$, её заряд — $q$.
    Выведите из базовых физических законов формулы для радиуса траектории частицы и её периода обращения.
}
\answer{%
    $
        F = ma, F = qvB, a = v^2 / R \implies R = \frac{mv}{qB}.
        \quad T = \frac{2\pi R}{v} = \frac{2\pi m}{qB}.
        \quad \omega = \frac vR = \frac{qB}{m}.
        \quad \nu = \frac 1T = \frac{qB}{2\pi m}.
    $
}
\solutionspace{120pt}

\tasknumber{5}%
\task{%
    Протон со скоростью $2 \cdot 10^{4}\,\frac{\text{км}}{\text{c}}$ влетает в магнитное поле индукцией $50\,\text{мТл}$ перпендикулярно линиям его индукции.
    Определите радиус траектории частицы, для вычислений воспользуйтесь табличными значениями.
}
\answer{%
    $
        F = ma, F = evB, a = v^2 / R \implies R = \frac{mv}{eB} = \frac{1{,}672 \cdot 10^{-27}\,\text{кг} \cdot 2 \cdot 10^{4}\,\frac{\text{км}}{\text{c}}}{1{,}6 \cdot 10^{-19}\,\text{Кл} \cdot 50\,\text{мТл}} \approx 4{,}180000\,\text{м}.
    $
}
\solutionspace{120pt}

\tasknumber{6}%
\task{%
    Незаряженная частица находится в постоянном магнитном поле, а её мгновенная скорость перпендикулярна линиям его индукции.
    По какой траектории движется частица? (прямая, гипербола, парабола, окружность, спираль, ...)
    Действием всех других полей пренебречь.
}
\answer{%
    $\text{Прямая}$
}

\variantsplitter

\addpersonalvariant{Виктория Легонькова}

\tasknumber{1}%
\task{%
    Рядом с каждой единицей измерения укажите физическую величину, которая в ней измеряется, и один из вариантов обозначений этой физической величины.
    \begin{enumerate}
        \item Кл,
        \item м/с,
        \item радиан.
    \end{enumerate}
}
\solutionspace{40pt}

\tasknumber{2}%
\task{%
    Запишите формулой закон для вычисления модуля силы, действующей
    на проводник, по которому течёт электрический ток, в магнитном поле, и выразите из него значение угла.
}
\solutionspace{40pt}

\tasknumber{3}%
\task{%
    Запишите формулой закон Ампера и укажите
    для каждой величины её название и единицы измерения в системе СИ.
}
\solutionspace{80pt}

\tasknumber{4}%
\task{%
    Положительно заряженная частица движется со скоростью $v$ в магнитном поле перпендикулярно линиям его индукции.
    Индукция магнитного поля равна $B$, масса частицы $m$, её заряд — $q$.
    Выведите из базовых физических законов формулы для радиуса траектории частицы и её частоты обращения.
}
\answer{%
    $
        F = ma, F = qvB, a = v^2 / R \implies R = \frac{mv}{qB}.
        \quad T = \frac{2\pi R}{v} = \frac{2\pi m}{qB}.
        \quad \omega = \frac vR = \frac{qB}{m}.
        \quad \nu = \frac 1T = \frac{qB}{2\pi m}.
    $
}
\solutionspace{120pt}

\tasknumber{5}%
\task{%
    Электрон со скоростью $5 \cdot 10^{4}\,\frac{\text{км}}{\text{c}}$ влетает в магнитное поле индукцией $500\,\text{мТл}$ перпендикулярно линиям его индукции.
    Определите радиус траектории частицы, для вычислений воспользуйтесь табличными значениями.
}
\answer{%
    $
        F = ma, F = evB, a = v^2 / R \implies R = \frac{mv}{eB} = \frac{9{,}1 \cdot 10^{-31}\,\text{кг} \cdot 5 \cdot 10^{4}\,\frac{\text{км}}{\text{c}}}{1{,}6 \cdot 10^{-19}\,\text{Кл} \cdot 500\,\text{мТл}} \approx 0{,}000569\,\text{м}.
    $
}
\solutionspace{120pt}

\tasknumber{6}%
\task{%
    Незаряженная частица находится в постоянном электростатическом поле, а её мгновенная скорость перпендикулярна линиям его напряжённости.
    По какой траектории движется частица? (прямая, гипербола, парабола, окружность, спираль, ...)
    Действием всех других полей пренебречь.
}
\answer{%
    $\text{Прямая}$
}

\variantsplitter

\addpersonalvariant{Семён Мартынов}

\tasknumber{1}%
\task{%
    Рядом с каждой единицей измерения укажите физическую величину, которая в ней измеряется, и один из вариантов обозначений этой физической величины.
    \begin{enumerate}
        \item Тл,
        \item м,
        \item радиан.
    \end{enumerate}
}
\solutionspace{40pt}

\tasknumber{2}%
\task{%
    Запишите формулой закон для вычисления модуля силы, действующей
    на заряженную частицу, движущуюся в магнитном поле, и выразите из него значение угла.
}
\solutionspace{40pt}

\tasknumber{3}%
\task{%
    Запишите формулой закон Лоренца и укажите
    для каждой величины её название и единицы измерения в системе СИ.
}
\solutionspace{80pt}

\tasknumber{4}%
\task{%
    Положительно заряженная частица движется со скоростью $v$ в магнитном поле перпендикулярно линиям его индукции.
    Индукция магнитного поля равна $B$, масса частицы $m$, её заряд — $q$.
    Выведите из базовых физических законов формулы для радиуса траектории частицы и её периода обращения.
}
\answer{%
    $
        F = ma, F = qvB, a = v^2 / R \implies R = \frac{mv}{qB}.
        \quad T = \frac{2\pi R}{v} = \frac{2\pi m}{qB}.
        \quad \omega = \frac vR = \frac{qB}{m}.
        \quad \nu = \frac 1T = \frac{qB}{2\pi m}.
    $
}
\solutionspace{120pt}

\tasknumber{5}%
\task{%
    Протон со скоростью $5 \cdot 10^{3}\,\frac{\text{км}}{\text{c}}$ влетает в магнитное поле индукцией $300\,\text{мТл}$ перпендикулярно линиям его индукции.
    Определите радиус траектории частицы, для вычислений воспользуйтесь табличными значениями.
}
\answer{%
    $
        F = ma, F = evB, a = v^2 / R \implies R = \frac{mv}{eB} = \frac{1{,}672 \cdot 10^{-27}\,\text{кг} \cdot 5 \cdot 10^{3}\,\frac{\text{км}}{\text{c}}}{1{,}6 \cdot 10^{-19}\,\text{Кл} \cdot 300\,\text{мТл}} \approx 0{,}174167\,\text{м}.
    $
}
\solutionspace{120pt}

\tasknumber{6}%
\task{%
    Незаряженная частица находится в постоянном магнитном поле, а её мгновенная скорость параллельна линиям его индукции.
    По какой траектории движется частица? (прямая, гипербола, парабола, окружность, спираль, ...)
    Действием всех других полей пренебречь.
}
\answer{%
    $\text{Прямая}$
}

\variantsplitter

\addpersonalvariant{Варвара Минаева}

\tasknumber{1}%
\task{%
    Рядом с каждой единицей измерения укажите физическую величину, которая в ней измеряется, и один из вариантов обозначений этой физической величины.
    \begin{enumerate}
        \item Кл,
        \item м/с,
        \item радиан.
    \end{enumerate}
}
\solutionspace{40pt}

\tasknumber{2}%
\task{%
    Запишите формулой закон для вычисления модуля силы, действующей
    на проводник, по которому течёт электрический ток, в магнитном поле, и выразите из него значение угла.
}
\solutionspace{40pt}

\tasknumber{3}%
\task{%
    Запишите формулой закон Лоренца и укажите
    для каждой величины её название и единицы измерения в системе СИ.
}
\solutionspace{80pt}

\tasknumber{4}%
\task{%
    Положительно заряженная частица движется со скоростью $v$ в магнитном поле перпендикулярно линиям его индукции.
    Индукция магнитного поля равна $B$, масса частицы $m$, её заряд — $q$.
    Выведите из базовых физических законов формулы для радиуса траектории частицы и её периода обращения.
}
\answer{%
    $
        F = ma, F = qvB, a = v^2 / R \implies R = \frac{mv}{qB}.
        \quad T = \frac{2\pi R}{v} = \frac{2\pi m}{qB}.
        \quad \omega = \frac vR = \frac{qB}{m}.
        \quad \nu = \frac 1T = \frac{qB}{2\pi m}.
    $
}
\solutionspace{120pt}

\tasknumber{5}%
\task{%
    Протон со скоростью $5 \cdot 10^{4}\,\frac{\text{км}}{\text{c}}$ влетает в магнитное поле индукцией $300\,\text{мТл}$ перпендикулярно линиям его индукции.
    Определите радиус траектории частицы, для вычислений воспользуйтесь табличными значениями.
}
\answer{%
    $
        F = ma, F = evB, a = v^2 / R \implies R = \frac{mv}{eB} = \frac{1{,}672 \cdot 10^{-27}\,\text{кг} \cdot 5 \cdot 10^{4}\,\frac{\text{км}}{\text{c}}}{1{,}6 \cdot 10^{-19}\,\text{Кл} \cdot 300\,\text{мТл}} \approx 1{,}741667\,\text{м}.
    $
}
\solutionspace{120pt}

\tasknumber{6}%
\task{%
    Незаряженная частица находится в постоянном электростатическом поле, а её мгновенная скорость перпендикулярна линиям его напряжённости.
    По какой траектории движется частица? (прямая, гипербола, парабола, окружность, спираль, ...)
    Действием всех других полей пренебречь.
}
\answer{%
    $\text{Прямая}$
}

\variantsplitter

\addpersonalvariant{Леонид Никитин}

\tasknumber{1}%
\task{%
    Рядом с каждой единицей измерения укажите физическую величину, которая в ней измеряется, и один из вариантов обозначений этой физической величины.
    \begin{enumerate}
        \item Кл,
        \item м,
        \item А.
    \end{enumerate}
}
\solutionspace{40pt}

\tasknumber{2}%
\task{%
    Запишите формулой закон для вычисления модуля силы, действующей
    на проводник, по которому течёт электрический ток, в магнитном поле, и выразите из него значение угла.
}
\solutionspace{40pt}

\tasknumber{3}%
\task{%
    Запишите формулой закон Лоренца и укажите
    для каждой величины её название и единицы измерения в системе СИ.
}
\solutionspace{80pt}

\tasknumber{4}%
\task{%
    Положительно заряженная частица движется со скоростью $v$ в магнитном поле перпендикулярно линиям его индукции.
    Индукция магнитного поля равна $B$, масса частицы $m$, её заряд — $q$.
    Выведите из базовых физических законов формулы для радиуса траектории частицы и её угловой скорости.
}
\answer{%
    $
        F = ma, F = qvB, a = v^2 / R \implies R = \frac{mv}{qB}.
        \quad T = \frac{2\pi R}{v} = \frac{2\pi m}{qB}.
        \quad \omega = \frac vR = \frac{qB}{m}.
        \quad \nu = \frac 1T = \frac{qB}{2\pi m}.
    $
}
\solutionspace{120pt}

\tasknumber{5}%
\task{%
    Протон со скоростью $3 \cdot 10^{3}\,\frac{\text{км}}{\text{c}}$ влетает в магнитное поле индукцией $500\,\text{мТл}$ перпендикулярно линиям его индукции.
    Определите радиус траектории частицы, для вычислений воспользуйтесь табличными значениями.
}
\answer{%
    $
        F = ma, F = evB, a = v^2 / R \implies R = \frac{mv}{eB} = \frac{1{,}672 \cdot 10^{-27}\,\text{кг} \cdot 3 \cdot 10^{3}\,\frac{\text{км}}{\text{c}}}{1{,}6 \cdot 10^{-19}\,\text{Кл} \cdot 500\,\text{мТл}} \approx 0{,}062700\,\text{м}.
    $
}
\solutionspace{120pt}

\tasknumber{6}%
\task{%
    Заряженная частица находится в постоянном электростатическом поле, а её мгновенная скорость перпендикулярна линиям его напряжённости.
    По какой траектории движется частица? (прямая, гипербола, парабола, окружность, спираль, ...)
    Действием всех других полей пренебречь.
}
\answer{%
    $\text{Парабола}$
}

\variantsplitter

\addpersonalvariant{Тимофей Полетаев}

\tasknumber{1}%
\task{%
    Рядом с каждой единицей измерения укажите физическую величину, которая в ней измеряется, и один из вариантов обозначений этой физической величины.
    \begin{enumerate}
        \item Кл,
        \item м/с,
        \item радиан.
    \end{enumerate}
}
\solutionspace{40pt}

\tasknumber{2}%
\task{%
    Запишите формулой закон для вычисления модуля силы, действующей
    на заряженную частицу, движущуюся в магнитном поле, и выразите из него значение угла.
}
\solutionspace{40pt}

\tasknumber{3}%
\task{%
    Запишите формулой закон Лоренца и укажите
    для каждой величины её название и единицы измерения в системе СИ.
}
\solutionspace{80pt}

\tasknumber{4}%
\task{%
    Положительно заряженная частица движется со скоростью $v$ в магнитном поле перпендикулярно линиям его индукции.
    Индукция магнитного поля равна $B$, масса частицы $m$, её заряд — $q$.
    Выведите из базовых физических законов формулы для радиуса траектории частицы и её периода обращения.
}
\answer{%
    $
        F = ma, F = qvB, a = v^2 / R \implies R = \frac{mv}{qB}.
        \quad T = \frac{2\pi R}{v} = \frac{2\pi m}{qB}.
        \quad \omega = \frac vR = \frac{qB}{m}.
        \quad \nu = \frac 1T = \frac{qB}{2\pi m}.
    $
}
\solutionspace{120pt}

\tasknumber{5}%
\task{%
    Электрон со скоростью $5 \cdot 10^{4}\,\frac{\text{км}}{\text{c}}$ влетает в магнитное поле индукцией $300\,\text{мТл}$ перпендикулярно линиям его индукции.
    Определите радиус траектории частицы, для вычислений воспользуйтесь табличными значениями.
}
\answer{%
    $
        F = ma, F = evB, a = v^2 / R \implies R = \frac{mv}{eB} = \frac{9{,}1 \cdot 10^{-31}\,\text{кг} \cdot 5 \cdot 10^{4}\,\frac{\text{км}}{\text{c}}}{1{,}6 \cdot 10^{-19}\,\text{Кл} \cdot 300\,\text{мТл}} \approx 0{,}000948\,\text{м}.
    $
}
\solutionspace{120pt}

\tasknumber{6}%
\task{%
    Заряженная частица находится в постоянном магнитном поле, а её мгновенная скорость параллельна линиям его индукции.
    По какой траектории движется частица? (прямая, гипербола, парабола, окружность, спираль, ...)
    Действием всех других полей пренебречь.
}
\answer{%
    $\text{Прямая}$
}

\variantsplitter

\addpersonalvariant{Андрей Рожков}

\tasknumber{1}%
\task{%
    Рядом с каждой единицей измерения укажите физическую величину, которая в ней измеряется, и один из вариантов обозначений этой физической величины.
    \begin{enumerate}
        \item Кл,
        \item м/с,
        \item радиан.
    \end{enumerate}
}
\solutionspace{40pt}

\tasknumber{2}%
\task{%
    Запишите формулой закон для вычисления модуля силы, действующей
    на проводник, по которому течёт электрический ток, в магнитном поле, и выразите из него значение угла.
}
\solutionspace{40pt}

\tasknumber{3}%
\task{%
    Запишите формулой закон Ампера и укажите
    для каждой величины её название и единицы измерения в системе СИ.
}
\solutionspace{80pt}

\tasknumber{4}%
\task{%
    Положительно заряженная частица движется со скоростью $v$ в магнитном поле перпендикулярно линиям его индукции.
    Индукция магнитного поля равна $B$, масса частицы $m$, её заряд — $q$.
    Выведите из базовых физических законов формулы для радиуса траектории частицы и её угловой скорости.
}
\answer{%
    $
        F = ma, F = qvB, a = v^2 / R \implies R = \frac{mv}{qB}.
        \quad T = \frac{2\pi R}{v} = \frac{2\pi m}{qB}.
        \quad \omega = \frac vR = \frac{qB}{m}.
        \quad \nu = \frac 1T = \frac{qB}{2\pi m}.
    $
}
\solutionspace{120pt}

\tasknumber{5}%
\task{%
    Электрон со скоростью $2 \cdot 10^{4}\,\frac{\text{км}}{\text{c}}$ влетает в магнитное поле индукцией $200\,\text{мТл}$ перпендикулярно линиям его индукции.
    Определите радиус траектории частицы, для вычислений воспользуйтесь табличными значениями.
}
\answer{%
    $
        F = ma, F = evB, a = v^2 / R \implies R = \frac{mv}{eB} = \frac{9{,}1 \cdot 10^{-31}\,\text{кг} \cdot 2 \cdot 10^{4}\,\frac{\text{км}}{\text{c}}}{1{,}6 \cdot 10^{-19}\,\text{Кл} \cdot 200\,\text{мТл}} \approx 0{,}000569\,\text{м}.
    $
}
\solutionspace{120pt}

\tasknumber{6}%
\task{%
    Заряженная частица находится в постоянном магнитном поле, а её мгновенная скорость параллельна линиям его индукции.
    По какой траектории движется частица? (прямая, гипербола, парабола, окружность, спираль, ...)
    Действием всех других полей пренебречь.
}
\answer{%
    $\text{Прямая}$
}

\variantsplitter

\addpersonalvariant{Рената Таржиманова}

\tasknumber{1}%
\task{%
    Рядом с каждой единицей измерения укажите физическую величину, которая в ней измеряется, и один из вариантов обозначений этой физической величины.
    \begin{enumerate}
        \item Кл,
        \item м/с,
        \item А.
    \end{enumerate}
}
\solutionspace{40pt}

\tasknumber{2}%
\task{%
    Запишите формулой закон для вычисления модуля силы, действующей
    на заряженную частицу, движущуюся в магнитном поле, и выразите из него индукцию магнитного поля.
}
\solutionspace{40pt}

\tasknumber{3}%
\task{%
    Запишите формулой закон Лоренца и укажите
    для каждой величины её название и единицы измерения в системе СИ.
}
\solutionspace{80pt}

\tasknumber{4}%
\task{%
    Положительно заряженная частица движется со скоростью $v$ в магнитном поле перпендикулярно линиям его индукции.
    Индукция магнитного поля равна $B$, масса частицы $m$, её заряд — $q$.
    Выведите из базовых физических законов формулы для радиуса траектории частицы и её угловой скорости.
}
\answer{%
    $
        F = ma, F = qvB, a = v^2 / R \implies R = \frac{mv}{qB}.
        \quad T = \frac{2\pi R}{v} = \frac{2\pi m}{qB}.
        \quad \omega = \frac vR = \frac{qB}{m}.
        \quad \nu = \frac 1T = \frac{qB}{2\pi m}.
    $
}
\solutionspace{120pt}

\tasknumber{5}%
\task{%
    Электрон со скоростью $3 \cdot 10^{3}\,\frac{\text{км}}{\text{c}}$ влетает в магнитное поле индукцией $300\,\text{мТл}$ перпендикулярно линиям его индукции.
    Определите радиус траектории частицы, для вычислений воспользуйтесь табличными значениями.
}
\answer{%
    $
        F = ma, F = evB, a = v^2 / R \implies R = \frac{mv}{eB} = \frac{9{,}1 \cdot 10^{-31}\,\text{кг} \cdot 3 \cdot 10^{3}\,\frac{\text{км}}{\text{c}}}{1{,}6 \cdot 10^{-19}\,\text{Кл} \cdot 300\,\text{мТл}} \approx 0{,}000057\,\text{м}.
    $
}
\solutionspace{120pt}

\tasknumber{6}%
\task{%
    Заряженная частица находится в постоянном магнитном поле, а её мгновенная скорость перпендикулярна линиям его индукции.
    По какой траектории движется частица? (прямая, гипербола, парабола, окружность, спираль, ...)
    Действием всех других полей пренебречь.
}
\answer{%
    $\text{Окружность}$
}

\variantsplitter

\addpersonalvariant{Андрей Щербаков}

\tasknumber{1}%
\task{%
    Рядом с каждой единицей измерения укажите физическую величину, которая в ней измеряется, и один из вариантов обозначений этой физической величины.
    \begin{enumerate}
        \item Кл,
        \item м,
        \item радиан.
    \end{enumerate}
}
\solutionspace{40pt}

\tasknumber{2}%
\task{%
    Запишите формулой закон для вычисления модуля силы, действующей
    на проводник, по которому течёт электрический ток, в магнитном поле, и выразите из него значение угла.
}
\solutionspace{40pt}

\tasknumber{3}%
\task{%
    Запишите формулой закон Лоренца и укажите
    для каждой величины её название и единицы измерения в системе СИ.
}
\solutionspace{80pt}

\tasknumber{4}%
\task{%
    Положительно заряженная частица движется со скоростью $v$ в магнитном поле перпендикулярно линиям его индукции.
    Индукция магнитного поля равна $B$, масса частицы $m$, её заряд — $q$.
    Выведите из базовых физических законов формулы для радиуса траектории частицы и её периода обращения.
}
\answer{%
    $
        F = ma, F = qvB, a = v^2 / R \implies R = \frac{mv}{qB}.
        \quad T = \frac{2\pi R}{v} = \frac{2\pi m}{qB}.
        \quad \omega = \frac vR = \frac{qB}{m}.
        \quad \nu = \frac 1T = \frac{qB}{2\pi m}.
    $
}
\solutionspace{120pt}

\tasknumber{5}%
\task{%
    Протон со скоростью $2 \cdot 10^{4}\,\frac{\text{км}}{\text{c}}$ влетает в магнитное поле индукцией $300\,\text{мТл}$ перпендикулярно линиям его индукции.
    Определите радиус траектории частицы, для вычислений воспользуйтесь табличными значениями.
}
\answer{%
    $
        F = ma, F = evB, a = v^2 / R \implies R = \frac{mv}{eB} = \frac{1{,}672 \cdot 10^{-27}\,\text{кг} \cdot 2 \cdot 10^{4}\,\frac{\text{км}}{\text{c}}}{1{,}6 \cdot 10^{-19}\,\text{Кл} \cdot 300\,\text{мТл}} \approx 0{,}696667\,\text{м}.
    $
}
\solutionspace{120pt}

\tasknumber{6}%
\task{%
    Заряженная частица находится в постоянном электростатическом поле, а её мгновенная скорость перпендикулярна линиям его напряжённости.
    По какой траектории движется частица? (прямая, гипербола, парабола, окружность, спираль, ...)
    Действием всех других полей пренебречь.
}
\answer{%
    $\text{Парабола}$
}

\variantsplitter

\addpersonalvariant{Михаил Ярошевский}

\tasknumber{1}%
\task{%
    Рядом с каждой единицей измерения укажите физическую величину, которая в ней измеряется, и один из вариантов обозначений этой физической величины.
    \begin{enumerate}
        \item Тл,
        \item м,
        \item радиан.
    \end{enumerate}
}
\solutionspace{40pt}

\tasknumber{2}%
\task{%
    Запишите формулой закон для вычисления модуля силы, действующей
    на проводник, по которому течёт электрический ток, в магнитном поле, и выразите из него индукцию магнитного поля.
}
\solutionspace{40pt}

\tasknumber{3}%
\task{%
    Запишите формулой закон Лоренца и укажите
    для каждой величины её название и единицы измерения в системе СИ.
}
\solutionspace{80pt}

\tasknumber{4}%
\task{%
    Положительно заряженная частица движется со скоростью $v$ в магнитном поле перпендикулярно линиям его индукции.
    Индукция магнитного поля равна $B$, масса частицы $m$, её заряд — $q$.
    Выведите из базовых физических законов формулы для радиуса траектории частицы и её периода обращения.
}
\answer{%
    $
        F = ma, F = qvB, a = v^2 / R \implies R = \frac{mv}{qB}.
        \quad T = \frac{2\pi R}{v} = \frac{2\pi m}{qB}.
        \quad \omega = \frac vR = \frac{qB}{m}.
        \quad \nu = \frac 1T = \frac{qB}{2\pi m}.
    $
}
\solutionspace{120pt}

\tasknumber{5}%
\task{%
    Протон со скоростью $5 \cdot 10^{4}\,\frac{\text{км}}{\text{c}}$ влетает в магнитное поле индукцией $50\,\text{мТл}$ перпендикулярно линиям его индукции.
    Определите радиус траектории частицы, для вычислений воспользуйтесь табличными значениями.
}
\answer{%
    $
        F = ma, F = evB, a = v^2 / R \implies R = \frac{mv}{eB} = \frac{1{,}672 \cdot 10^{-27}\,\text{кг} \cdot 5 \cdot 10^{4}\,\frac{\text{км}}{\text{c}}}{1{,}6 \cdot 10^{-19}\,\text{Кл} \cdot 50\,\text{мТл}} \approx 10{,}450000\,\text{м}.
    $
}
\solutionspace{120pt}

\tasknumber{6}%
\task{%
    Незаряженная частица находится в постоянном электростатическом поле, а её мгновенная скорость перпендикулярна линиям его напряжённости.
    По какой траектории движется частица? (прямая, гипербола, парабола, окружность, спираль, ...)
    Действием всех других полей пренебречь.
}
\answer{%
    $\text{Прямая}$
}

\variantsplitter

\addpersonalvariant{Алексей Алимпиев}

\tasknumber{1}%
\task{%
    Рядом с каждой единицей измерения укажите физическую величину, которая в ней измеряется, и один из вариантов обозначений этой физической величины.
    \begin{enumerate}
        \item Тл,
        \item м,
        \item А.
    \end{enumerate}
}
\solutionspace{40pt}

\tasknumber{2}%
\task{%
    Запишите формулой закон для вычисления модуля силы, действующей
    на заряженную частицу, движущуюся в магнитном поле, и выразите из него индукцию магнитного поля.
}
\solutionspace{40pt}

\tasknumber{3}%
\task{%
    Запишите формулой закон Лоренца и укажите
    для каждой величины её название и единицы измерения в системе СИ.
}
\solutionspace{80pt}

\tasknumber{4}%
\task{%
    Положительно заряженная частица движется со скоростью $v$ в магнитном поле перпендикулярно линиям его индукции.
    Индукция магнитного поля равна $B$, масса частицы $m$, её заряд — $q$.
    Выведите из базовых физических законов формулы для радиуса траектории частицы и её угловой скорости.
}
\answer{%
    $
        F = ma, F = qvB, a = v^2 / R \implies R = \frac{mv}{qB}.
        \quad T = \frac{2\pi R}{v} = \frac{2\pi m}{qB}.
        \quad \omega = \frac vR = \frac{qB}{m}.
        \quad \nu = \frac 1T = \frac{qB}{2\pi m}.
    $
}
\solutionspace{120pt}

\tasknumber{5}%
\task{%
    Электрон со скоростью $5 \cdot 10^{4}\,\frac{\text{км}}{\text{c}}$ влетает в магнитное поле индукцией $500\,\text{мТл}$ перпендикулярно линиям его индукции.
    Определите радиус траектории частицы, для вычислений воспользуйтесь табличными значениями.
}
\answer{%
    $
        F = ma, F = evB, a = v^2 / R \implies R = \frac{mv}{eB} = \frac{9{,}1 \cdot 10^{-31}\,\text{кг} \cdot 5 \cdot 10^{4}\,\frac{\text{км}}{\text{c}}}{1{,}6 \cdot 10^{-19}\,\text{Кл} \cdot 500\,\text{мТл}} \approx 0{,}000569\,\text{м}.
    $
}
\solutionspace{120pt}

\tasknumber{6}%
\task{%
    Незаряженная частица находится в постоянном магнитном поле, а её мгновенная скорость параллельна линиям его индукции.
    По какой траектории движется частица? (прямая, гипербола, парабола, окружность, спираль, ...)
    Действием всех других полей пренебречь.
}
\answer{%
    $\text{Прямая}$
}

\variantsplitter

\addpersonalvariant{Евгений Васин}

\tasknumber{1}%
\task{%
    Рядом с каждой единицей измерения укажите физическую величину, которая в ней измеряется, и один из вариантов обозначений этой физической величины.
    \begin{enumerate}
        \item Кл,
        \item м/с,
        \item А.
    \end{enumerate}
}
\solutionspace{40pt}

\tasknumber{2}%
\task{%
    Запишите формулой закон для вычисления модуля силы, действующей
    на проводник, по которому течёт электрический ток, в магнитном поле, и выразите из него значение угла.
}
\solutionspace{40pt}

\tasknumber{3}%
\task{%
    Запишите формулой закон Лоренца и укажите
    для каждой величины её название и единицы измерения в системе СИ.
}
\solutionspace{80pt}

\tasknumber{4}%
\task{%
    Положительно заряженная частица движется со скоростью $v$ в магнитном поле перпендикулярно линиям его индукции.
    Индукция магнитного поля равна $B$, масса частицы $m$, её заряд — $q$.
    Выведите из базовых физических законов формулы для радиуса траектории частицы и её частоты обращения.
}
\answer{%
    $
        F = ma, F = qvB, a = v^2 / R \implies R = \frac{mv}{qB}.
        \quad T = \frac{2\pi R}{v} = \frac{2\pi m}{qB}.
        \quad \omega = \frac vR = \frac{qB}{m}.
        \quad \nu = \frac 1T = \frac{qB}{2\pi m}.
    $
}
\solutionspace{120pt}

\tasknumber{5}%
\task{%
    Электрон со скоростью $2 \cdot 10^{4}\,\frac{\text{км}}{\text{c}}$ влетает в магнитное поле индукцией $50\,\text{мТл}$ перпендикулярно линиям его индукции.
    Определите радиус траектории частицы, для вычислений воспользуйтесь табличными значениями.
}
\answer{%
    $
        F = ma, F = evB, a = v^2 / R \implies R = \frac{mv}{eB} = \frac{9{,}1 \cdot 10^{-31}\,\text{кг} \cdot 2 \cdot 10^{4}\,\frac{\text{км}}{\text{c}}}{1{,}6 \cdot 10^{-19}\,\text{Кл} \cdot 50\,\text{мТл}} \approx 0{,}002275\,\text{м}.
    $
}
\solutionspace{120pt}

\tasknumber{6}%
\task{%
    Незаряженная частица находится в постоянном электростатическом поле, а её мгновенная скорость перпендикулярна линиям его напряжённости.
    По какой траектории движется частица? (прямая, гипербола, парабола, окружность, спираль, ...)
    Действием всех других полей пренебречь.
}
\answer{%
    $\text{Прямая}$
}

\variantsplitter

\addpersonalvariant{Вячеслав Волохов}

\tasknumber{1}%
\task{%
    Рядом с каждой единицей измерения укажите физическую величину, которая в ней измеряется, и один из вариантов обозначений этой физической величины.
    \begin{enumerate}
        \item Тл,
        \item м/с,
        \item радиан.
    \end{enumerate}
}
\solutionspace{40pt}

\tasknumber{2}%
\task{%
    Запишите формулой закон для вычисления модуля силы, действующей
    на заряженную частицу, движущуюся в магнитном поле, и выразите из него значение угла.
}
\solutionspace{40pt}

\tasknumber{3}%
\task{%
    Запишите формулой закон Лоренца и укажите
    для каждой величины её название и единицы измерения в системе СИ.
}
\solutionspace{80pt}

\tasknumber{4}%
\task{%
    Положительно заряженная частица движется со скоростью $v$ в магнитном поле перпендикулярно линиям его индукции.
    Индукция магнитного поля равна $B$, масса частицы $m$, её заряд — $q$.
    Выведите из базовых физических законов формулы для радиуса траектории частицы и её частоты обращения.
}
\answer{%
    $
        F = ma, F = qvB, a = v^2 / R \implies R = \frac{mv}{qB}.
        \quad T = \frac{2\pi R}{v} = \frac{2\pi m}{qB}.
        \quad \omega = \frac vR = \frac{qB}{m}.
        \quad \nu = \frac 1T = \frac{qB}{2\pi m}.
    $
}
\solutionspace{120pt}

\tasknumber{5}%
\task{%
    Протон со скоростью $5 \cdot 10^{4}\,\frac{\text{км}}{\text{c}}$ влетает в магнитное поле индукцией $200\,\text{мТл}$ перпендикулярно линиям его индукции.
    Определите радиус траектории частицы, для вычислений воспользуйтесь табличными значениями.
}
\answer{%
    $
        F = ma, F = evB, a = v^2 / R \implies R = \frac{mv}{eB} = \frac{1{,}672 \cdot 10^{-27}\,\text{кг} \cdot 5 \cdot 10^{4}\,\frac{\text{км}}{\text{c}}}{1{,}6 \cdot 10^{-19}\,\text{Кл} \cdot 200\,\text{мТл}} \approx 2{,}612500\,\text{м}.
    $
}
\solutionspace{120pt}

\tasknumber{6}%
\task{%
    Незаряженная частица находится в постоянном электростатическом поле, а её мгновенная скорость перпендикулярна линиям его напряжённости.
    По какой траектории движется частица? (прямая, гипербола, парабола, окружность, спираль, ...)
    Действием всех других полей пренебречь.
}
\answer{%
    $\text{Прямая}$
}

\variantsplitter

\addpersonalvariant{Герман Говоров}

\tasknumber{1}%
\task{%
    Рядом с каждой единицей измерения укажите физическую величину, которая в ней измеряется, и один из вариантов обозначений этой физической величины.
    \begin{enumerate}
        \item Кл,
        \item м/с,
        \item радиан.
    \end{enumerate}
}
\solutionspace{40pt}

\tasknumber{2}%
\task{%
    Запишите формулой закон для вычисления модуля силы, действующей
    на заряженную частицу, движущуюся в магнитном поле, и выразите из него значение угла.
}
\solutionspace{40pt}

\tasknumber{3}%
\task{%
    Запишите формулой закон Лоренца и укажите
    для каждой величины её название и единицы измерения в системе СИ.
}
\solutionspace{80pt}

\tasknumber{4}%
\task{%
    Положительно заряженная частица движется со скоростью $v$ в магнитном поле перпендикулярно линиям его индукции.
    Индукция магнитного поля равна $B$, масса частицы $m$, её заряд — $q$.
    Выведите из базовых физических законов формулы для радиуса траектории частицы и её частоты обращения.
}
\answer{%
    $
        F = ma, F = qvB, a = v^2 / R \implies R = \frac{mv}{qB}.
        \quad T = \frac{2\pi R}{v} = \frac{2\pi m}{qB}.
        \quad \omega = \frac vR = \frac{qB}{m}.
        \quad \nu = \frac 1T = \frac{qB}{2\pi m}.
    $
}
\solutionspace{120pt}

\tasknumber{5}%
\task{%
    Протон со скоростью $5 \cdot 10^{3}\,\frac{\text{км}}{\text{c}}$ влетает в магнитное поле индукцией $300\,\text{мТл}$ перпендикулярно линиям его индукции.
    Определите радиус траектории частицы, для вычислений воспользуйтесь табличными значениями.
}
\answer{%
    $
        F = ma, F = evB, a = v^2 / R \implies R = \frac{mv}{eB} = \frac{1{,}672 \cdot 10^{-27}\,\text{кг} \cdot 5 \cdot 10^{3}\,\frac{\text{км}}{\text{c}}}{1{,}6 \cdot 10^{-19}\,\text{Кл} \cdot 300\,\text{мТл}} \approx 0{,}174167\,\text{м}.
    $
}
\solutionspace{120pt}

\tasknumber{6}%
\task{%
    Незаряженная частица находится в постоянном электростатическом поле, а её мгновенная скорость параллельна линиям его напряжённости.
    По какой траектории движется частица? (прямая, гипербола, парабола, окружность, спираль, ...)
    Действием всех других полей пренебречь.
}
\answer{%
    $\text{Прямая}$
}

\variantsplitter

\addpersonalvariant{София Журавлёва}

\tasknumber{1}%
\task{%
    Рядом с каждой единицей измерения укажите физическую величину, которая в ней измеряется, и один из вариантов обозначений этой физической величины.
    \begin{enumerate}
        \item Кл,
        \item м/с,
        \item А.
    \end{enumerate}
}
\solutionspace{40pt}

\tasknumber{2}%
\task{%
    Запишите формулой закон для вычисления модуля силы, действующей
    на заряженную частицу, движущуюся в магнитном поле, и выразите из него значение угла.
}
\solutionspace{40pt}

\tasknumber{3}%
\task{%
    Запишите формулой закон Ампера и укажите
    для каждой величины её название и единицы измерения в системе СИ.
}
\solutionspace{80pt}

\tasknumber{4}%
\task{%
    Положительно заряженная частица движется со скоростью $v$ в магнитном поле перпендикулярно линиям его индукции.
    Индукция магнитного поля равна $B$, масса частицы $m$, её заряд — $q$.
    Выведите из базовых физических законов формулы для радиуса траектории частицы и её частоты обращения.
}
\answer{%
    $
        F = ma, F = qvB, a = v^2 / R \implies R = \frac{mv}{qB}.
        \quad T = \frac{2\pi R}{v} = \frac{2\pi m}{qB}.
        \quad \omega = \frac vR = \frac{qB}{m}.
        \quad \nu = \frac 1T = \frac{qB}{2\pi m}.
    $
}
\solutionspace{120pt}

\tasknumber{5}%
\task{%
    Протон со скоростью $5 \cdot 10^{4}\,\frac{\text{км}}{\text{c}}$ влетает в магнитное поле индукцией $500\,\text{мТл}$ перпендикулярно линиям его индукции.
    Определите радиус траектории частицы, для вычислений воспользуйтесь табличными значениями.
}
\answer{%
    $
        F = ma, F = evB, a = v^2 / R \implies R = \frac{mv}{eB} = \frac{1{,}672 \cdot 10^{-27}\,\text{кг} \cdot 5 \cdot 10^{4}\,\frac{\text{км}}{\text{c}}}{1{,}6 \cdot 10^{-19}\,\text{Кл} \cdot 500\,\text{мТл}} \approx 1{,}045000\,\text{м}.
    $
}
\solutionspace{120pt}

\tasknumber{6}%
\task{%
    Заряженная частица находится в постоянном электростатическом поле, а её мгновенная скорость параллельна линиям его напряжённости.
    По какой траектории движется частица? (прямая, гипербола, парабола, окружность, спираль, ...)
    Действием всех других полей пренебречь.
}
\answer{%
    $\text{Прямая}$
}

\variantsplitter

\addpersonalvariant{Константин Козлов}

\tasknumber{1}%
\task{%
    Рядом с каждой единицей измерения укажите физическую величину, которая в ней измеряется, и один из вариантов обозначений этой физической величины.
    \begin{enumerate}
        \item Кл,
        \item м,
        \item А.
    \end{enumerate}
}
\solutionspace{40pt}

\tasknumber{2}%
\task{%
    Запишите формулой закон для вычисления модуля силы, действующей
    на проводник, по которому течёт электрический ток, в магнитном поле, и выразите из него значение угла.
}
\solutionspace{40pt}

\tasknumber{3}%
\task{%
    Запишите формулой закон Ампера и укажите
    для каждой величины её название и единицы измерения в системе СИ.
}
\solutionspace{80pt}

\tasknumber{4}%
\task{%
    Положительно заряженная частица движется со скоростью $v$ в магнитном поле перпендикулярно линиям его индукции.
    Индукция магнитного поля равна $B$, масса частицы $m$, её заряд — $q$.
    Выведите из базовых физических законов формулы для радиуса траектории частицы и её частоты обращения.
}
\answer{%
    $
        F = ma, F = qvB, a = v^2 / R \implies R = \frac{mv}{qB}.
        \quad T = \frac{2\pi R}{v} = \frac{2\pi m}{qB}.
        \quad \omega = \frac vR = \frac{qB}{m}.
        \quad \nu = \frac 1T = \frac{qB}{2\pi m}.
    $
}
\solutionspace{120pt}

\tasknumber{5}%
\task{%
    Электрон со скоростью $5 \cdot 10^{3}\,\frac{\text{км}}{\text{c}}$ влетает в магнитное поле индукцией $40\,\text{мТл}$ перпендикулярно линиям его индукции.
    Определите радиус траектории частицы, для вычислений воспользуйтесь табличными значениями.
}
\answer{%
    $
        F = ma, F = evB, a = v^2 / R \implies R = \frac{mv}{eB} = \frac{9{,}1 \cdot 10^{-31}\,\text{кг} \cdot 5 \cdot 10^{3}\,\frac{\text{км}}{\text{c}}}{1{,}6 \cdot 10^{-19}\,\text{Кл} \cdot 40\,\text{мТл}} \approx 0{,}000711\,\text{м}.
    $
}
\solutionspace{120pt}

\tasknumber{6}%
\task{%
    Незаряженная частица находится в постоянном магнитном поле, а её мгновенная скорость параллельна линиям его индукции.
    По какой траектории движется частица? (прямая, гипербола, парабола, окружность, спираль, ...)
    Действием всех других полей пренебречь.
}
\answer{%
    $\text{Прямая}$
}

\variantsplitter

\addpersonalvariant{Наталья Кравченко}

\tasknumber{1}%
\task{%
    Рядом с каждой единицей измерения укажите физическую величину, которая в ней измеряется, и один из вариантов обозначений этой физической величины.
    \begin{enumerate}
        \item Кл,
        \item м/с,
        \item радиан.
    \end{enumerate}
}
\solutionspace{40pt}

\tasknumber{2}%
\task{%
    Запишите формулой закон для вычисления модуля силы, действующей
    на заряженную частицу, движущуюся в магнитном поле, и выразите из него значение угла.
}
\solutionspace{40pt}

\tasknumber{3}%
\task{%
    Запишите формулой закон Ампера и укажите
    для каждой величины её название и единицы измерения в системе СИ.
}
\solutionspace{80pt}

\tasknumber{4}%
\task{%
    Положительно заряженная частица движется со скоростью $v$ в магнитном поле перпендикулярно линиям его индукции.
    Индукция магнитного поля равна $B$, масса частицы $m$, её заряд — $q$.
    Выведите из базовых физических законов формулы для радиуса траектории частицы и её частоты обращения.
}
\answer{%
    $
        F = ma, F = qvB, a = v^2 / R \implies R = \frac{mv}{qB}.
        \quad T = \frac{2\pi R}{v} = \frac{2\pi m}{qB}.
        \quad \omega = \frac vR = \frac{qB}{m}.
        \quad \nu = \frac 1T = \frac{qB}{2\pi m}.
    $
}
\solutionspace{120pt}

\tasknumber{5}%
\task{%
    Протон со скоростью $3 \cdot 10^{4}\,\frac{\text{км}}{\text{c}}$ влетает в магнитное поле индукцией $500\,\text{мТл}$ перпендикулярно линиям его индукции.
    Определите радиус траектории частицы, для вычислений воспользуйтесь табличными значениями.
}
\answer{%
    $
        F = ma, F = evB, a = v^2 / R \implies R = \frac{mv}{eB} = \frac{1{,}672 \cdot 10^{-27}\,\text{кг} \cdot 3 \cdot 10^{4}\,\frac{\text{км}}{\text{c}}}{1{,}6 \cdot 10^{-19}\,\text{Кл} \cdot 500\,\text{мТл}} \approx 0{,}627000\,\text{м}.
    $
}
\solutionspace{120pt}

\tasknumber{6}%
\task{%
    Незаряженная частица находится в постоянном магнитном поле, а её мгновенная скорость перпендикулярна линиям его индукции.
    По какой траектории движется частица? (прямая, гипербола, парабола, окружность, спираль, ...)
    Действием всех других полей пренебречь.
}
\answer{%
    $\text{Прямая}$
}

\variantsplitter

\addpersonalvariant{Сергей Малышев}

\tasknumber{1}%
\task{%
    Рядом с каждой единицей измерения укажите физическую величину, которая в ней измеряется, и один из вариантов обозначений этой физической величины.
    \begin{enumerate}
        \item Кл,
        \item м,
        \item А.
    \end{enumerate}
}
\solutionspace{40pt}

\tasknumber{2}%
\task{%
    Запишите формулой закон для вычисления модуля силы, действующей
    на заряженную частицу, движущуюся в магнитном поле, и выразите из него индукцию магнитного поля.
}
\solutionspace{40pt}

\tasknumber{3}%
\task{%
    Запишите формулой закон Ампера и укажите
    для каждой величины её название и единицы измерения в системе СИ.
}
\solutionspace{80pt}

\tasknumber{4}%
\task{%
    Положительно заряженная частица движется со скоростью $v$ в магнитном поле перпендикулярно линиям его индукции.
    Индукция магнитного поля равна $B$, масса частицы $m$, её заряд — $q$.
    Выведите из базовых физических законов формулы для радиуса траектории частицы и её частоты обращения.
}
\answer{%
    $
        F = ma, F = qvB, a = v^2 / R \implies R = \frac{mv}{qB}.
        \quad T = \frac{2\pi R}{v} = \frac{2\pi m}{qB}.
        \quad \omega = \frac vR = \frac{qB}{m}.
        \quad \nu = \frac 1T = \frac{qB}{2\pi m}.
    $
}
\solutionspace{120pt}

\tasknumber{5}%
\task{%
    Электрон со скоростью $2 \cdot 10^{3}\,\frac{\text{км}}{\text{c}}$ влетает в магнитное поле индукцией $500\,\text{мТл}$ перпендикулярно линиям его индукции.
    Определите радиус траектории частицы, для вычислений воспользуйтесь табличными значениями.
}
\answer{%
    $
        F = ma, F = evB, a = v^2 / R \implies R = \frac{mv}{eB} = \frac{9{,}1 \cdot 10^{-31}\,\text{кг} \cdot 2 \cdot 10^{3}\,\frac{\text{км}}{\text{c}}}{1{,}6 \cdot 10^{-19}\,\text{Кл} \cdot 500\,\text{мТл}} \approx 0{,}000023\,\text{м}.
    $
}
\solutionspace{120pt}

\tasknumber{6}%
\task{%
    Незаряженная частица находится в постоянном магнитном поле, а её мгновенная скорость перпендикулярна линиям его индукции.
    По какой траектории движется частица? (прямая, гипербола, парабола, окружность, спираль, ...)
    Действием всех других полей пренебречь.
}
\answer{%
    $\text{Прямая}$
}

\variantsplitter

\addpersonalvariant{Алина Полканова}

\tasknumber{1}%
\task{%
    Рядом с каждой единицей измерения укажите физическую величину, которая в ней измеряется, и один из вариантов обозначений этой физической величины.
    \begin{enumerate}
        \item Кл,
        \item м,
        \item радиан.
    \end{enumerate}
}
\solutionspace{40pt}

\tasknumber{2}%
\task{%
    Запишите формулой закон для вычисления модуля силы, действующей
    на проводник, по которому течёт электрический ток, в магнитном поле, и выразите из него значение угла.
}
\solutionspace{40pt}

\tasknumber{3}%
\task{%
    Запишите формулой закон Лоренца и укажите
    для каждой величины её название и единицы измерения в системе СИ.
}
\solutionspace{80pt}

\tasknumber{4}%
\task{%
    Положительно заряженная частица движется со скоростью $v$ в магнитном поле перпендикулярно линиям его индукции.
    Индукция магнитного поля равна $B$, масса частицы $m$, её заряд — $q$.
    Выведите из базовых физических законов формулы для радиуса траектории частицы и её периода обращения.
}
\answer{%
    $
        F = ma, F = qvB, a = v^2 / R \implies R = \frac{mv}{qB}.
        \quad T = \frac{2\pi R}{v} = \frac{2\pi m}{qB}.
        \quad \omega = \frac vR = \frac{qB}{m}.
        \quad \nu = \frac 1T = \frac{qB}{2\pi m}.
    $
}
\solutionspace{120pt}

\tasknumber{5}%
\task{%
    Протон со скоростью $3 \cdot 10^{4}\,\frac{\text{км}}{\text{c}}$ влетает в магнитное поле индукцией $40\,\text{мТл}$ перпендикулярно линиям его индукции.
    Определите радиус траектории частицы, для вычислений воспользуйтесь табличными значениями.
}
\answer{%
    $
        F = ma, F = evB, a = v^2 / R \implies R = \frac{mv}{eB} = \frac{1{,}672 \cdot 10^{-27}\,\text{кг} \cdot 3 \cdot 10^{4}\,\frac{\text{км}}{\text{c}}}{1{,}6 \cdot 10^{-19}\,\text{Кл} \cdot 40\,\text{мТл}} \approx 7{,}837500\,\text{м}.
    $
}
\solutionspace{120pt}

\tasknumber{6}%
\task{%
    Заряженная частица находится в постоянном магнитном поле, а её мгновенная скорость параллельна линиям его индукции.
    По какой траектории движется частица? (прямая, гипербола, парабола, окружность, спираль, ...)
    Действием всех других полей пренебречь.
}
\answer{%
    $\text{Прямая}$
}

\variantsplitter

\addpersonalvariant{Сергей Пономарёв}

\tasknumber{1}%
\task{%
    Рядом с каждой единицей измерения укажите физическую величину, которая в ней измеряется, и один из вариантов обозначений этой физической величины.
    \begin{enumerate}
        \item Кл,
        \item м/с,
        \item радиан.
    \end{enumerate}
}
\solutionspace{40pt}

\tasknumber{2}%
\task{%
    Запишите формулой закон для вычисления модуля силы, действующей
    на заряженную частицу, движущуюся в магнитном поле, и выразите из него значение угла.
}
\solutionspace{40pt}

\tasknumber{3}%
\task{%
    Запишите формулой закон Лоренца и укажите
    для каждой величины её название и единицы измерения в системе СИ.
}
\solutionspace{80pt}

\tasknumber{4}%
\task{%
    Положительно заряженная частица движется со скоростью $v$ в магнитном поле перпендикулярно линиям его индукции.
    Индукция магнитного поля равна $B$, масса частицы $m$, её заряд — $q$.
    Выведите из базовых физических законов формулы для радиуса траектории частицы и её частоты обращения.
}
\answer{%
    $
        F = ma, F = qvB, a = v^2 / R \implies R = \frac{mv}{qB}.
        \quad T = \frac{2\pi R}{v} = \frac{2\pi m}{qB}.
        \quad \omega = \frac vR = \frac{qB}{m}.
        \quad \nu = \frac 1T = \frac{qB}{2\pi m}.
    $
}
\solutionspace{120pt}

\tasknumber{5}%
\task{%
    Электрон со скоростью $3 \cdot 10^{4}\,\frac{\text{км}}{\text{c}}$ влетает в магнитное поле индукцией $500\,\text{мТл}$ перпендикулярно линиям его индукции.
    Определите радиус траектории частицы, для вычислений воспользуйтесь табличными значениями.
}
\answer{%
    $
        F = ma, F = evB, a = v^2 / R \implies R = \frac{mv}{eB} = \frac{9{,}1 \cdot 10^{-31}\,\text{кг} \cdot 3 \cdot 10^{4}\,\frac{\text{км}}{\text{c}}}{1{,}6 \cdot 10^{-19}\,\text{Кл} \cdot 500\,\text{мТл}} \approx 0{,}000341\,\text{м}.
    $
}
\solutionspace{120pt}

\tasknumber{6}%
\task{%
    Заряженная частица находится в постоянном магнитном поле, а её мгновенная скорость перпендикулярна линиям его индукции.
    По какой траектории движется частица? (прямая, гипербола, парабола, окружность, спираль, ...)
    Действием всех других полей пренебречь.
}
\answer{%
    $\text{Окружность}$
}

\variantsplitter

\addpersonalvariant{Егор Свистушкин}

\tasknumber{1}%
\task{%
    Рядом с каждой единицей измерения укажите физическую величину, которая в ней измеряется, и один из вариантов обозначений этой физической величины.
    \begin{enumerate}
        \item Кл,
        \item м/с,
        \item радиан.
    \end{enumerate}
}
\solutionspace{40pt}

\tasknumber{2}%
\task{%
    Запишите формулой закон для вычисления модуля силы, действующей
    на заряженную частицу, движущуюся в магнитном поле, и выразите из него значение угла.
}
\solutionspace{40pt}

\tasknumber{3}%
\task{%
    Запишите формулой закон Лоренца и укажите
    для каждой величины её название и единицы измерения в системе СИ.
}
\solutionspace{80pt}

\tasknumber{4}%
\task{%
    Положительно заряженная частица движется со скоростью $v$ в магнитном поле перпендикулярно линиям его индукции.
    Индукция магнитного поля равна $B$, масса частицы $m$, её заряд — $q$.
    Выведите из базовых физических законов формулы для радиуса траектории частицы и её периода обращения.
}
\answer{%
    $
        F = ma, F = qvB, a = v^2 / R \implies R = \frac{mv}{qB}.
        \quad T = \frac{2\pi R}{v} = \frac{2\pi m}{qB}.
        \quad \omega = \frac vR = \frac{qB}{m}.
        \quad \nu = \frac 1T = \frac{qB}{2\pi m}.
    $
}
\solutionspace{120pt}

\tasknumber{5}%
\task{%
    Протон со скоростью $5 \cdot 10^{4}\,\frac{\text{км}}{\text{c}}$ влетает в магнитное поле индукцией $200\,\text{мТл}$ перпендикулярно линиям его индукции.
    Определите радиус траектории частицы, для вычислений воспользуйтесь табличными значениями.
}
\answer{%
    $
        F = ma, F = evB, a = v^2 / R \implies R = \frac{mv}{eB} = \frac{1{,}672 \cdot 10^{-27}\,\text{кг} \cdot 5 \cdot 10^{4}\,\frac{\text{км}}{\text{c}}}{1{,}6 \cdot 10^{-19}\,\text{Кл} \cdot 200\,\text{мТл}} \approx 2{,}612500\,\text{м}.
    $
}
\solutionspace{120pt}

\tasknumber{6}%
\task{%
    Незаряженная частица находится в постоянном магнитном поле, а её мгновенная скорость параллельна линиям его индукции.
    По какой траектории движется частица? (прямая, гипербола, парабола, окружность, спираль, ...)
    Действием всех других полей пренебречь.
}
\answer{%
    $\text{Прямая}$
}

\variantsplitter

\addpersonalvariant{Дмитрий Соколов}

\tasknumber{1}%
\task{%
    Рядом с каждой единицей измерения укажите физическую величину, которая в ней измеряется, и один из вариантов обозначений этой физической величины.
    \begin{enumerate}
        \item Кл,
        \item м/с,
        \item А.
    \end{enumerate}
}
\solutionspace{40pt}

\tasknumber{2}%
\task{%
    Запишите формулой закон для вычисления модуля силы, действующей
    на заряженную частицу, движущуюся в магнитном поле, и выразите из него индукцию магнитного поля.
}
\solutionspace{40pt}

\tasknumber{3}%
\task{%
    Запишите формулой закон Лоренца и укажите
    для каждой величины её название и единицы измерения в системе СИ.
}
\solutionspace{80pt}

\tasknumber{4}%
\task{%
    Положительно заряженная частица движется со скоростью $v$ в магнитном поле перпендикулярно линиям его индукции.
    Индукция магнитного поля равна $B$, масса частицы $m$, её заряд — $q$.
    Выведите из базовых физических законов формулы для радиуса траектории частицы и её периода обращения.
}
\answer{%
    $
        F = ma, F = qvB, a = v^2 / R \implies R = \frac{mv}{qB}.
        \quad T = \frac{2\pi R}{v} = \frac{2\pi m}{qB}.
        \quad \omega = \frac vR = \frac{qB}{m}.
        \quad \nu = \frac 1T = \frac{qB}{2\pi m}.
    $
}
\solutionspace{120pt}

\tasknumber{5}%
\task{%
    Протон со скоростью $5 \cdot 10^{4}\,\frac{\text{км}}{\text{c}}$ влетает в магнитное поле индукцией $40\,\text{мТл}$ перпендикулярно линиям его индукции.
    Определите радиус траектории частицы, для вычислений воспользуйтесь табличными значениями.
}
\answer{%
    $
        F = ma, F = evB, a = v^2 / R \implies R = \frac{mv}{eB} = \frac{1{,}672 \cdot 10^{-27}\,\text{кг} \cdot 5 \cdot 10^{4}\,\frac{\text{км}}{\text{c}}}{1{,}6 \cdot 10^{-19}\,\text{Кл} \cdot 40\,\text{мТл}} \approx 13{,}062500\,\text{м}.
    $
}
\solutionspace{120pt}

\tasknumber{6}%
\task{%
    Незаряженная частица находится в постоянном магнитном поле, а её мгновенная скорость параллельна линиям его индукции.
    По какой траектории движется частица? (прямая, гипербола, парабола, окружность, спираль, ...)
    Действием всех других полей пренебречь.
}
\answer{%
    $\text{Прямая}$
}

\variantsplitter

\addpersonalvariant{Арсений Трофимов}

\tasknumber{1}%
\task{%
    Рядом с каждой единицей измерения укажите физическую величину, которая в ней измеряется, и один из вариантов обозначений этой физической величины.
    \begin{enumerate}
        \item Тл,
        \item м/с,
        \item А.
    \end{enumerate}
}
\solutionspace{40pt}

\tasknumber{2}%
\task{%
    Запишите формулой закон для вычисления модуля силы, действующей
    на проводник, по которому течёт электрический ток, в магнитном поле, и выразите из него значение угла.
}
\solutionspace{40pt}

\tasknumber{3}%
\task{%
    Запишите формулой закон Лоренца и укажите
    для каждой величины её название и единицы измерения в системе СИ.
}
\solutionspace{80pt}

\tasknumber{4}%
\task{%
    Положительно заряженная частица движется со скоростью $v$ в магнитном поле перпендикулярно линиям его индукции.
    Индукция магнитного поля равна $B$, масса частицы $m$, её заряд — $q$.
    Выведите из базовых физических законов формулы для радиуса траектории частицы и её частоты обращения.
}
\answer{%
    $
        F = ma, F = qvB, a = v^2 / R \implies R = \frac{mv}{qB}.
        \quad T = \frac{2\pi R}{v} = \frac{2\pi m}{qB}.
        \quad \omega = \frac vR = \frac{qB}{m}.
        \quad \nu = \frac 1T = \frac{qB}{2\pi m}.
    $
}
\solutionspace{120pt}

\tasknumber{5}%
\task{%
    Протон со скоростью $2 \cdot 10^{4}\,\frac{\text{км}}{\text{c}}$ влетает в магнитное поле индукцией $200\,\text{мТл}$ перпендикулярно линиям его индукции.
    Определите радиус траектории частицы, для вычислений воспользуйтесь табличными значениями.
}
\answer{%
    $
        F = ma, F = evB, a = v^2 / R \implies R = \frac{mv}{eB} = \frac{1{,}672 \cdot 10^{-27}\,\text{кг} \cdot 2 \cdot 10^{4}\,\frac{\text{км}}{\text{c}}}{1{,}6 \cdot 10^{-19}\,\text{Кл} \cdot 200\,\text{мТл}} \approx 1{,}045000\,\text{м}.
    $
}
\solutionspace{120pt}

\tasknumber{6}%
\task{%
    Заряженная частица находится в постоянном магнитном поле, а её мгновенная скорость перпендикулярна линиям его индукции.
    По какой траектории движется частица? (прямая, гипербола, парабола, окружность, спираль, ...)
    Действием всех других полей пренебречь.
}
\answer{%
    $\text{Окружность}$
}
% autogenerated
