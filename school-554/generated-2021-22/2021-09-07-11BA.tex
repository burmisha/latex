\setdate{7~сентября~2021}
\setclass{11«БА»}

\addpersonalvariant{Михаил Бурмистров}

\tasknumber{1}%
\task{%
    Укажите, верны ли утверждения («да» или «нет» слева от каждого утверждения):
    \begin{itemize}
        \item  Если распилить постоянный магнит на 2, то мы получим 2 магнита:
                один только с южным полюсом, а второй — только с северным.
        \item  Полосовой магнит можно распилить 3 разрезами на 4 магнита поменьше.
        \item  Между линиями индукции магнитного поля величина этого поля пренебрежимо мала.
        \item  Линии магнитного поля всегда замкнуты.
        \item  Линии магнитного поля могут пересекаться в полюсах магнитов.
        \item  Линии магнитного поля заканчиваются у северного полюса и начинаются у южного.
        \item  Чем гуще линии — тем слабее магнитное поле.
        \item  Северный географический полюс Земли в точности совпадает с северным магнитным полюсом Земли.
        \item  Если в компасе установить сильный магнит, то его не удастся отклонить магнитным полем неподалёку.
                Так не делают лишь потому, что компас станет слишком неудобным в бытовом использовании.
        \item  Внутри магнита есть магнитное поле, поэтому для честности мы обязаны рисовать поле как снаружи, так и внутри него.
    \end{itemize}
}
\answer{%
    $\text{нет, да, нет, да, нет, нет, нет, нет, нет, да}$
}

\tasknumber{2}%
\task{%
    Изобразите линии индукции магнитного поля вокруг постоянного магнита.

    \begin{tikzpicture}[x=1cm,y=1cm,thick]
        \draw (0, 0) rectangle (3, 0.6);
        \node [right] (right) at (0, 0.3) {N};
        \node [left] (left) at (3, 0.3) {S};
        \node [right] (right) at (0, 2) {};
        \node [right] (right) at (0, -2) {};
        \node [right] (right) at (-2, 0) {};
        \node [right] (right) at (5, 0) {};
    \end{tikzpicture}
}

\tasknumber{3}%
\task{%
    Опишите взаимодействие параллельных прямых токов.
    Нужны рисунки и необходимый минимум пояснений и терминов, трактат не нужен.
}
\solutionspace{100pt}

\tasknumber{4}%
\task{%
    Для постоянного магнита, изображённого на рис.
    1б)
    изобразите линии индукции магнитного поля
    и укажите, как соориентируются магнитные стрелки в точках B и D.
}
\solutionspace{80pt}

\tasknumber{5}%
\task{%
    Магнитная стрелка вблизи длинного прямолинейного проводника
    повёрнута в точке $B$ северным полюсом вверх (см.
    рис.
    2в).
    Сделайте рисунок, укажите направление протекания электрического тока,
    изобразите линии индукции магнитного поля.
}

\variantsplitter

\addpersonalvariant{Ирина Ан}

\tasknumber{1}%
\task{%
    Укажите, верны ли утверждения («да» или «нет» слева от каждого утверждения):
    \begin{itemize}
        \item  Если распилить постоянный магнит на 2, то мы получим 2 магнита:
                один только с южным полюсом, а второй — только с северным.
        \item  Полосовой магнит можно распилить 2 разрезами на 3 магнита поменьше.
        \item  Между линиями индукции магнитного поля величина этого поля пренебрежимо мала.
        \item  Линии магнитного поля всегда замкнуты.
        \item  Линии магнитного поля могут пересекаться в полюсах магнитов.
        \item  Линии магнитного поля начинаются у северного полюса и заканчиваются у южного.
        \item  Чем гуще линии — тем слабее магнитное поле.
        \item  Северный географический полюс Земли в точности совпадает с южным магнитным полюсом Земли.
        \item  Если в компасе установить сильный магнит, то его не удастся отклонить магнитным полем неподалёку.
                Так не делают лишь потому, что компас станет слишком неудобным в бытовом использовании.
        \item  Внутри магнита есть магнитное поле, поэтому для честности мы обязаны рисовать поле как снаружи, так и внутри него.
    \end{itemize}
}
\answer{%
    $\text{нет, да, нет, да, нет, нет, нет, нет, нет, да}$
}

\tasknumber{2}%
\task{%
    Изобразите линии индукции магнитного поля вокруг постоянного магнита.

    \begin{tikzpicture}[x=1cm,y=1cm,thick]
        \draw (0, 0) rectangle (3, 0.6);
        \node [right] (right) at (0, 0.3) {N};
        \node [left] (left) at (3, 0.3) {S};
        \node [right] (right) at (0, 2) {};
        \node [right] (right) at (0, -2) {};
        \node [right] (right) at (-2, 0) {};
        \node [right] (right) at (5, 0) {};
    \end{tikzpicture}
}

\tasknumber{3}%
\task{%
    Опишите опыт Эрстеда.
    Нужны рисунки и необходимый минимум пояснений и терминов, трактат не нужен.
}
\solutionspace{100pt}

\tasknumber{4}%
\task{%
    Для постоянного магнита, изображённого на рис.
    1а)
    изобразите линии индукции магнитного поля
    и укажите, как соориентируются магнитные стрелки в точках A и D.
}
\solutionspace{80pt}

\tasknumber{5}%
\task{%
    Магнитная стрелка вблизи длинного прямолинейного проводника
    повёрнута в точке $A$ северным полюсом вверх (см.
    рис.
    2в).
    Сделайте рисунок, укажите направление протекания электрического тока,
    изобразите линии индукции магнитного поля.
}

\variantsplitter

\addpersonalvariant{Софья Андрианова}

\tasknumber{1}%
\task{%
    Укажите, верны ли утверждения («да» или «нет» слева от каждого утверждения):
    \begin{itemize}
        \item  Если распилить постоянный магнит на 2, то мы получим 2 магнита:
                один только с южным полюсом, а второй — только с северным.
        \item  Полосовой магнит можно распилить 2 разрезами на 3 магнита поменьше.
        \item  Между линиями индукции магнитного поля величина этого поля пренебрежимо мала.
        \item  Линии магнитного поля всегда замкнуты.
        \item  Линии магнитного поля могут пересекаться в полюсах магнитов.
        \item  Линии магнитного поля заканчиваются у северного полюса и начинаются у южного.
        \item  Чем гуще линии — тем сильнее магнитное поле.
        \item  Северный географический полюс Земли в точности совпадает с северным магнитным полюсом Земли.
        \item  Если в компасе установить сильный магнит, то его не удастся отклонить магнитным полем неподалёку.
                Так не делают лишь потому, что компас станет слишком неудобным в бытовом использовании.
        \item  Внутри магнита есть магнитное поле, поэтому для честности мы обязаны рисовать поле как снаружи, так и внутри него.
    \end{itemize}
}
\answer{%
    $\text{нет, да, нет, да, нет, нет, да, нет, нет, да}$
}

\tasknumber{2}%
\task{%
    Изобразите линии индукции магнитного поля вокруг постоянного магнита.

    \begin{tikzpicture}[x=1cm,y=1cm,thick]
        \draw (0, 0) rectangle (3, 0.6);
        \node [right] (right) at (0, 0.3) {N};
        \node [left] (left) at (3, 0.3) {S};
        \node [right] (right) at (0, 2) {};
        \node [right] (right) at (0, -2) {};
        \node [right] (right) at (-2, 0) {};
        \node [right] (right) at (5, 0) {};
    \end{tikzpicture}
}

\tasknumber{3}%
\task{%
    Опишите взаимодействие полосовых магнитов.
    Нужны рисунки и необходимый минимум пояснений и терминов, трактат не нужен.
}
\solutionspace{100pt}

\tasknumber{4}%
\task{%
    Для постоянного магнита, изображённого на рис.
    1а)
    изобразите линии индукции магнитного поля
    и укажите, как соориентируются магнитные стрелки в точках A и D.
}
\solutionspace{80pt}

\tasknumber{5}%
\task{%
    Магнитная стрелка вблизи длинного прямолинейного проводника
    повёрнута в точке $B$ северным полюсом направо (см.
    рис.
    2б).
    Сделайте рисунок, укажите направление протекания электрического тока,
    изобразите линии индукции магнитного поля.
}

\variantsplitter

\addpersonalvariant{Владимир Артемчук}

\tasknumber{1}%
\task{%
    Укажите, верны ли утверждения («да» или «нет» слева от каждого утверждения):
    \begin{itemize}
        \item  Если распилить постоянный магнит на 2, то мы получим 2 магнита:
                один только с южным полюсом, а второй — только с северным.
        \item  Полосовой магнит можно распилить 2 разрезами на 3 магнита поменьше.
        \item  Между линиями индукции магнитного поля величина этого поля пренебрежимо мала.
        \item  Линии магнитного поля всегда замкнуты.
        \item  Линии магнитного поля могут пересекаться в полюсах магнитов.
        \item  Линии магнитного поля начинаются у северного полюса и заканчиваются у южного.
        \item  Чем гуще линии — тем слабее магнитное поле.
        \item  Северный географический полюс Земли в точности совпадает с северным магнитным полюсом Земли.
        \item  Если в компасе установить сильный магнит, то его не удастся отклонить магнитным полем неподалёку.
                Так не делают лишь потому, что компас станет слишком неудобным в бытовом использовании.
        \item  Внутри магнита есть магнитное поле, поэтому для честности мы обязаны рисовать поле как снаружи, так и внутри него.
    \end{itemize}
}
\answer{%
    $\text{нет, да, нет, да, нет, нет, нет, нет, нет, да}$
}

\tasknumber{2}%
\task{%
    Изобразите линии индукции магнитного поля вокруг постоянного магнита.

    \begin{tikzpicture}[x=1cm,y=1cm,thick]
        \draw (0, 0) rectangle (3, 0.6);
        \node [right] (right) at (0, 0.3) {S};
        \node [left] (left) at (3, 0.3) {N};
        \node [right] (right) at (0, 2) {};
        \node [right] (right) at (0, -2) {};
        \node [right] (right) at (-2, 0) {};
        \node [right] (right) at (5, 0) {};
    \end{tikzpicture}
}

\tasknumber{3}%
\task{%
    Опишите взаимодействие параллельных прямых токов.
    Нужны рисунки и необходимый минимум пояснений и терминов, трактат не нужен.
}
\solutionspace{100pt}

\tasknumber{4}%
\task{%
    Для постоянного магнита, изображённого на рис.
    1а)
    изобразите линии индукции магнитного поля
    и укажите, как соориентируются магнитные стрелки в точках B и C.
}
\solutionspace{80pt}

\tasknumber{5}%
\task{%
    Магнитная стрелка вблизи длинного прямолинейного проводника
    повёрнута в точке $A$ северным полюсом вверх (см.
    рис.
    2в).
    Сделайте рисунок, укажите направление протекания электрического тока,
    изобразите линии индукции магнитного поля.
}

\variantsplitter

\addpersonalvariant{Софья Белянкина}

\tasknumber{1}%
\task{%
    Укажите, верны ли утверждения («да» или «нет» слева от каждого утверждения):
    \begin{itemize}
        \item  Если распилить постоянный магнит на 2, то мы получим 2 магнита:
                один только с южным полюсом, а второй — только с северным.
        \item  Полосовой магнит можно распилить 2 разрезами на 3 магнита поменьше.
        \item  Между линиями индукции магнитного поля величина этого поля пренебрежимо мала.
        \item  Линии магнитного поля всегда замкнуты.
        \item  Линии магнитного поля могут пересекаться в полюсах магнитов.
        \item  Линии магнитного поля начинаются у северного полюса и заканчиваются у южного.
        \item  Чем гуще линии — тем сильнее магнитное поле.
        \item  Северный географический полюс Земли в точности совпадает с северным магнитным полюсом Земли.
        \item  Если в компасе установить сильный магнит, то его не удастся отклонить магнитным полем неподалёку.
                Так не делают лишь потому, что компас станет слишком неудобным в бытовом использовании.
        \item  Внутри магнита есть магнитное поле, поэтому для честности мы обязаны рисовать поле как снаружи, так и внутри него.
    \end{itemize}
}
\answer{%
    $\text{нет, да, нет, да, нет, нет, да, нет, нет, да}$
}

\tasknumber{2}%
\task{%
    Изобразите линии индукции магнитного поля вокруг постоянного магнита.

    \begin{tikzpicture}[x=1cm,y=1cm,thick]
        \draw (0, 0) rectangle (3, 0.6);
        \node [right] (right) at (0, 0.3) {N};
        \node [left] (left) at (3, 0.3) {S};
        \node [right] (right) at (0, 2) {};
        \node [right] (right) at (0, -2) {};
        \node [right] (right) at (-2, 0) {};
        \node [right] (right) at (5, 0) {};
    \end{tikzpicture}
}

\tasknumber{3}%
\task{%
    Опишите опыт Эрстеда.
    Нужны рисунки и необходимый минимум пояснений и терминов, трактат не нужен.
}
\solutionspace{100pt}

\tasknumber{4}%
\task{%
    Для постоянного магнита, изображённого на рис.
    1г)
    изобразите линии индукции магнитного поля
    и укажите, как соориентируются магнитные стрелки в точках A и D.
}
\solutionspace{80pt}

\tasknumber{5}%
\task{%
    Магнитная стрелка вблизи длинного прямолинейного проводника
    повёрнута в точке $B$ северным полюсом вниз (см.
    рис.
    2в).
    Сделайте рисунок, укажите направление протекания электрического тока,
    изобразите линии индукции магнитного поля.
}

\variantsplitter

\addpersonalvariant{Варвара Егиазарян}

\tasknumber{1}%
\task{%
    Укажите, верны ли утверждения («да» или «нет» слева от каждого утверждения):
    \begin{itemize}
        \item  Если распилить постоянный магнит на 2, то мы получим 2 магнита:
                один только с южным полюсом, а второй — только с северным.
        \item  Полосовой магнит можно распилить 2 разрезами на 3 магнита поменьше.
        \item  Между линиями индукции магнитного поля величина этого поля пренебрежимо мала.
        \item  Линии магнитного поля всегда замкнуты.
        \item  Линии магнитного поля могут пересекаться в полюсах магнитов.
        \item  Линии магнитного поля заканчиваются у северного полюса и начинаются у южного.
        \item  Чем гуще линии — тем слабее магнитное поле.
        \item  Северный географический полюс Земли в точности совпадает с северным магнитным полюсом Земли.
        \item  Если в компасе установить сильный магнит, то его не удастся отклонить магнитным полем неподалёку.
                Так не делают лишь потому, что компас станет слишком неудобным в бытовом использовании.
        \item  Внутри магнита есть магнитное поле, поэтому для честности мы обязаны рисовать поле как снаружи, так и внутри него.
    \end{itemize}
}
\answer{%
    $\text{нет, да, нет, да, нет, нет, нет, нет, нет, да}$
}

\tasknumber{2}%
\task{%
    Изобразите линии индукции магнитного поля вокруг постоянного магнита.

    \begin{tikzpicture}[x=1cm,y=1cm,thick]
        \draw (0, 0) rectangle (3, 0.6);
        \node [right] (right) at (0, 0.3) {N};
        \node [left] (left) at (3, 0.3) {S};
        \node [right] (right) at (0, 2) {};
        \node [right] (right) at (0, -2) {};
        \node [right] (right) at (-2, 0) {};
        \node [right] (right) at (5, 0) {};
    \end{tikzpicture}
}

\tasknumber{3}%
\task{%
    Опишите опыт Эрстеда.
    Нужны рисунки и необходимый минимум пояснений и терминов, трактат не нужен.
}
\solutionspace{100pt}

\tasknumber{4}%
\task{%
    Для постоянного магнита, изображённого на рис.
    1б)
    изобразите линии индукции магнитного поля
    и укажите, как соориентируются магнитные стрелки в точках A и D.
}
\solutionspace{80pt}

\tasknumber{5}%
\task{%
    Магнитная стрелка вблизи длинного прямолинейного проводника
    повёрнута в точке $A$ северным полюсом налево (см.
    рис.
    2а).
    Сделайте рисунок, укажите направление протекания электрического тока,
    изобразите линии индукции магнитного поля.
}

\variantsplitter

\addpersonalvariant{Владислав Емелин}

\tasknumber{1}%
\task{%
    Укажите, верны ли утверждения («да» или «нет» слева от каждого утверждения):
    \begin{itemize}
        \item  Если распилить постоянный магнит на 2, то мы получим 2 магнита:
                один только с южным полюсом, а второй — только с северным.
        \item  Полосовой магнит можно распилить 2 разрезами на 3 магнита поменьше.
        \item  Между линиями индукции магнитного поля величина этого поля пренебрежимо мала.
        \item  Линии магнитного поля всегда замкнуты.
        \item  Линии магнитного поля могут пересекаться в полюсах магнитов.
        \item  Линии магнитного поля заканчиваются у северного полюса и начинаются у южного.
        \item  Чем гуще линии — тем слабее магнитное поле.
        \item  Северный географический полюс Земли в точности совпадает с северным магнитным полюсом Земли.
        \item  Если в компасе установить сильный магнит, то его не удастся отклонить магнитным полем неподалёку.
                Так не делают лишь потому, что компас станет слишком неудобным в бытовом использовании.
        \item  Внутри магнита есть магнитное поле, поэтому для честности мы обязаны рисовать поле как снаружи, так и внутри него.
    \end{itemize}
}
\answer{%
    $\text{нет, да, нет, да, нет, нет, нет, нет, нет, да}$
}

\tasknumber{2}%
\task{%
    Изобразите линии индукции магнитного поля вокруг постоянного магнита.

    \begin{tikzpicture}[x=1cm,y=1cm,thick]
        \draw (0, 0) rectangle (3, 0.6);
        \node [right] (right) at (0, 0.3) {S};
        \node [left] (left) at (3, 0.3) {N};
        \node [right] (right) at (0, 2) {};
        \node [right] (right) at (0, -2) {};
        \node [right] (right) at (-2, 0) {};
        \node [right] (right) at (5, 0) {};
    \end{tikzpicture}
}

\tasknumber{3}%
\task{%
    Опишите взаимодействие параллельных прямых токов.
    Нужны рисунки и необходимый минимум пояснений и терминов, трактат не нужен.
}
\solutionspace{100pt}

\tasknumber{4}%
\task{%
    Для постоянного магнита, изображённого на рис.
    1б)
    изобразите линии индукции магнитного поля
    и укажите, как соориентируются магнитные стрелки в точках A и B.
}
\solutionspace{80pt}

\tasknumber{5}%
\task{%
    Магнитная стрелка вблизи длинного прямолинейного проводника
    повёрнута в точке $A$ северным полюсом направо (см.
    рис.
    2б).
    Сделайте рисунок, укажите направление протекания электрического тока,
    изобразите линии индукции магнитного поля.
}

\variantsplitter

\addpersonalvariant{Артём Жичин}

\tasknumber{1}%
\task{%
    Укажите, верны ли утверждения («да» или «нет» слева от каждого утверждения):
    \begin{itemize}
        \item  Если распилить постоянный магнит на 2, то мы получим 2 магнита:
                один только с южным полюсом, а второй — только с северным.
        \item  Полосовой магнит можно распилить 3 разрезами на 4 магнита поменьше.
        \item  Между линиями индукции магнитного поля величина этого поля пренебрежимо мала.
        \item  Линии магнитного поля всегда замкнуты.
        \item  Линии магнитного поля могут пересекаться в полюсах магнитов.
        \item  Линии магнитного поля начинаются у северного полюса и заканчиваются у южного.
        \item  Чем гуще линии — тем сильнее магнитное поле.
        \item  Северный географический полюс Земли в точности совпадает с южным магнитным полюсом Земли.
        \item  Если в компасе установить сильный магнит, то его не удастся отклонить магнитным полем неподалёку.
                Так не делают лишь потому, что компас станет слишком неудобным в бытовом использовании.
        \item  Внутри магнита есть магнитное поле, поэтому для честности мы обязаны рисовать поле как снаружи, так и внутри него.
    \end{itemize}
}
\answer{%
    $\text{нет, да, нет, да, нет, нет, да, нет, нет, да}$
}

\tasknumber{2}%
\task{%
    Изобразите линии индукции магнитного поля вокруг постоянного магнита.

    \begin{tikzpicture}[x=1cm,y=1cm,thick]
        \draw (0, 0) rectangle (3, 0.6);
        \node [right] (right) at (0, 0.3) {S};
        \node [left] (left) at (3, 0.3) {N};
        \node [right] (right) at (0, 2) {};
        \node [right] (right) at (0, -2) {};
        \node [right] (right) at (-2, 0) {};
        \node [right] (right) at (5, 0) {};
    \end{tikzpicture}
}

\tasknumber{3}%
\task{%
    Опишите опыт Эрстеда.
    Нужны рисунки и необходимый минимум пояснений и терминов, трактат не нужен.
}
\solutionspace{100pt}

\tasknumber{4}%
\task{%
    Для постоянного магнита, изображённого на рис.
    1б)
    изобразите линии индукции магнитного поля
    и укажите, как соориентируются магнитные стрелки в точках A и C.
}
\solutionspace{80pt}

\tasknumber{5}%
\task{%
    Магнитная стрелка вблизи длинного прямолинейного проводника
    повёрнута в точке $B$ северным полюсом вверх (см.
    рис.
    2г).
    Сделайте рисунок, укажите направление протекания электрического тока,
    изобразите линии индукции магнитного поля.
}

\variantsplitter

\addpersonalvariant{Дарья Кошман}

\tasknumber{1}%
\task{%
    Укажите, верны ли утверждения («да» или «нет» слева от каждого утверждения):
    \begin{itemize}
        \item  Если распилить постоянный магнит на 2, то мы получим 2 магнита:
                один только с южным полюсом, а второй — только с северным.
        \item  Полосовой магнит можно распилить 2 разрезами на 3 магнита поменьше.
        \item  Между линиями индукции магнитного поля величина этого поля пренебрежимо мала.
        \item  Линии магнитного поля всегда замкнуты.
        \item  Линии магнитного поля могут пересекаться в полюсах магнитов.
        \item  Линии магнитного поля начинаются у северного полюса и заканчиваются у южного.
        \item  Чем гуще линии — тем сильнее магнитное поле.
        \item  Северный географический полюс Земли в точности совпадает с северным магнитным полюсом Земли.
        \item  Если в компасе установить сильный магнит, то его не удастся отклонить магнитным полем неподалёку.
                Так не делают лишь потому, что компас станет слишком неудобным в бытовом использовании.
        \item  Внутри магнита есть магнитное поле, поэтому для честности мы обязаны рисовать поле как снаружи, так и внутри него.
    \end{itemize}
}
\answer{%
    $\text{нет, да, нет, да, нет, нет, да, нет, нет, да}$
}

\tasknumber{2}%
\task{%
    Изобразите линии индукции магнитного поля вокруг постоянного магнита.

    \begin{tikzpicture}[x=1cm,y=1cm,thick]
        \draw (0, 0) rectangle (3, 0.6);
        \node [right] (right) at (0, 0.3) {S};
        \node [left] (left) at (3, 0.3) {N};
        \node [right] (right) at (0, 2) {};
        \node [right] (right) at (0, -2) {};
        \node [right] (right) at (-2, 0) {};
        \node [right] (right) at (5, 0) {};
    \end{tikzpicture}
}

\tasknumber{3}%
\task{%
    Опишите взаимодействие полосовых магнитов.
    Нужны рисунки и необходимый минимум пояснений и терминов, трактат не нужен.
}
\solutionspace{100pt}

\tasknumber{4}%
\task{%
    Для постоянного магнита, изображённого на рис.
    1а)
    изобразите линии индукции магнитного поля
    и укажите, как соориентируются магнитные стрелки в точках B и D.
}
\solutionspace{80pt}

\tasknumber{5}%
\task{%
    Магнитная стрелка вблизи длинного прямолинейного проводника
    повёрнута в точке $A$ северным полюсом направо (см.
    рис.
    2б).
    Сделайте рисунок, укажите направление протекания электрического тока,
    изобразите линии индукции магнитного поля.
}

\variantsplitter

\addpersonalvariant{Анна Кузьмичёва}

\tasknumber{1}%
\task{%
    Укажите, верны ли утверждения («да» или «нет» слева от каждого утверждения):
    \begin{itemize}
        \item  Если распилить постоянный магнит на 2, то мы получим 2 магнита:
                один только с южным полюсом, а второй — только с северным.
        \item  Полосовой магнит можно распилить 2 разрезами на 3 магнита поменьше.
        \item  Между линиями индукции магнитного поля величина этого поля пренебрежимо мала.
        \item  Линии магнитного поля всегда замкнуты.
        \item  Линии магнитного поля могут пересекаться в полюсах магнитов.
        \item  Линии магнитного поля начинаются у северного полюса и заканчиваются у южного.
        \item  Чем гуще линии — тем слабее магнитное поле.
        \item  Северный географический полюс Земли в точности совпадает с северным магнитным полюсом Земли.
        \item  Если в компасе установить сильный магнит, то его не удастся отклонить магнитным полем неподалёку.
                Так не делают лишь потому, что компас станет слишком неудобным в бытовом использовании.
        \item  Внутри магнита есть магнитное поле, поэтому для честности мы обязаны рисовать поле как снаружи, так и внутри него.
    \end{itemize}
}
\answer{%
    $\text{нет, да, нет, да, нет, нет, нет, нет, нет, да}$
}

\tasknumber{2}%
\task{%
    Изобразите линии индукции магнитного поля вокруг постоянного магнита.

    \begin{tikzpicture}[x=1cm,y=1cm,thick]
        \draw (0, 0) rectangle (3, 0.6);
        \node [right] (right) at (0, 0.3) {S};
        \node [left] (left) at (3, 0.3) {N};
        \node [right] (right) at (0, 2) {};
        \node [right] (right) at (0, -2) {};
        \node [right] (right) at (-2, 0) {};
        \node [right] (right) at (5, 0) {};
    \end{tikzpicture}
}

\tasknumber{3}%
\task{%
    Опишите опыт Эрстеда.
    Нужны рисунки и необходимый минимум пояснений и терминов, трактат не нужен.
}
\solutionspace{100pt}

\tasknumber{4}%
\task{%
    Для постоянного магнита, изображённого на рис.
    1а)
    изобразите линии индукции магнитного поля
    и укажите, как соориентируются магнитные стрелки в точках C и D.
}
\solutionspace{80pt}

\tasknumber{5}%
\task{%
    Магнитная стрелка вблизи длинного прямолинейного проводника
    повёрнута в точке $A$ северным полюсом вниз (см.
    рис.
    2г).
    Сделайте рисунок, укажите направление протекания электрического тока,
    изобразите линии индукции магнитного поля.
}

\variantsplitter

\addpersonalvariant{Алёна Куприянова}

\tasknumber{1}%
\task{%
    Укажите, верны ли утверждения («да» или «нет» слева от каждого утверждения):
    \begin{itemize}
        \item  Если распилить постоянный магнит на 2, то мы получим 2 магнита:
                один только с южным полюсом, а второй — только с северным.
        \item  Полосовой магнит можно распилить 3 разрезами на 4 магнита поменьше.
        \item  Между линиями индукции магнитного поля величина этого поля пренебрежимо мала.
        \item  Линии магнитного поля всегда замкнуты.
        \item  Линии магнитного поля могут пересекаться в полюсах магнитов.
        \item  Линии магнитного поля заканчиваются у северного полюса и начинаются у южного.
        \item  Чем гуще линии — тем сильнее магнитное поле.
        \item  Северный географический полюс Земли в точности совпадает с южным магнитным полюсом Земли.
        \item  Если в компасе установить сильный магнит, то его не удастся отклонить магнитным полем неподалёку.
                Так не делают лишь потому, что компас станет слишком неудобным в бытовом использовании.
        \item  Внутри магнита есть магнитное поле, поэтому для честности мы обязаны рисовать поле как снаружи, так и внутри него.
    \end{itemize}
}
\answer{%
    $\text{нет, да, нет, да, нет, нет, да, нет, нет, да}$
}

\tasknumber{2}%
\task{%
    Изобразите линии индукции магнитного поля вокруг постоянного магнита.

    \begin{tikzpicture}[x=1cm,y=1cm,thick]
        \draw (0, 0) rectangle (3, 0.6);
        \node [right] (right) at (0, 0.3) {S};
        \node [left] (left) at (3, 0.3) {N};
        \node [right] (right) at (0, 2) {};
        \node [right] (right) at (0, -2) {};
        \node [right] (right) at (-2, 0) {};
        \node [right] (right) at (5, 0) {};
    \end{tikzpicture}
}

\tasknumber{3}%
\task{%
    Опишите опыт Эрстеда.
    Нужны рисунки и необходимый минимум пояснений и терминов, трактат не нужен.
}
\solutionspace{100pt}

\tasknumber{4}%
\task{%
    Для постоянного магнита, изображённого на рис.
    1б)
    изобразите линии индукции магнитного поля
    и укажите, как соориентируются магнитные стрелки в точках C и D.
}
\solutionspace{80pt}

\tasknumber{5}%
\task{%
    Магнитная стрелка вблизи длинного прямолинейного проводника
    повёрнута в точке $A$ северным полюсом направо (см.
    рис.
    2б).
    Сделайте рисунок, укажите направление протекания электрического тока,
    изобразите линии индукции магнитного поля.
}

\variantsplitter

\addpersonalvariant{Ярослав Лавровский}

\tasknumber{1}%
\task{%
    Укажите, верны ли утверждения («да» или «нет» слева от каждого утверждения):
    \begin{itemize}
        \item  Если распилить постоянный магнит на 2, то мы получим 2 магнита:
                один только с южным полюсом, а второй — только с северным.
        \item  Полосовой магнит можно распилить 2 разрезами на 3 магнита поменьше.
        \item  Между линиями индукции магнитного поля величина этого поля пренебрежимо мала.
        \item  Линии магнитного поля всегда замкнуты.
        \item  Линии магнитного поля могут пересекаться в полюсах магнитов.
        \item  Линии магнитного поля начинаются у северного полюса и заканчиваются у южного.
        \item  Чем гуще линии — тем слабее магнитное поле.
        \item  Северный географический полюс Земли в точности совпадает с южным магнитным полюсом Земли.
        \item  Если в компасе установить сильный магнит, то его не удастся отклонить магнитным полем неподалёку.
                Так не делают лишь потому, что компас станет слишком неудобным в бытовом использовании.
        \item  Внутри магнита есть магнитное поле, поэтому для честности мы обязаны рисовать поле как снаружи, так и внутри него.
    \end{itemize}
}
\answer{%
    $\text{нет, да, нет, да, нет, нет, нет, нет, нет, да}$
}

\tasknumber{2}%
\task{%
    Изобразите линии индукции магнитного поля вокруг постоянного магнита.

    \begin{tikzpicture}[x=1cm,y=1cm,thick]
        \draw (0, 0) rectangle (3, 0.6);
        \node [right] (right) at (0, 0.3) {S};
        \node [left] (left) at (3, 0.3) {N};
        \node [right] (right) at (0, 2) {};
        \node [right] (right) at (0, -2) {};
        \node [right] (right) at (-2, 0) {};
        \node [right] (right) at (5, 0) {};
    \end{tikzpicture}
}

\tasknumber{3}%
\task{%
    Опишите опыт Эрстеда.
    Нужны рисунки и необходимый минимум пояснений и терминов, трактат не нужен.
}
\solutionspace{100pt}

\tasknumber{4}%
\task{%
    Для постоянного магнита, изображённого на рис.
    1а)
    изобразите линии индукции магнитного поля
    и укажите, как соориентируются магнитные стрелки в точках A и C.
}
\solutionspace{80pt}

\tasknumber{5}%
\task{%
    Магнитная стрелка вблизи длинного прямолинейного проводника
    повёрнута в точке $B$ северным полюсом налево (см.
    рис.
    2б).
    Сделайте рисунок, укажите направление протекания электрического тока,
    изобразите линии индукции магнитного поля.
}

\variantsplitter

\addpersonalvariant{Анастасия Ламанова}

\tasknumber{1}%
\task{%
    Укажите, верны ли утверждения («да» или «нет» слева от каждого утверждения):
    \begin{itemize}
        \item  Если распилить постоянный магнит на 2, то мы получим 2 магнита:
                один только с южным полюсом, а второй — только с северным.
        \item  Полосовой магнит можно распилить 2 разрезами на 3 магнита поменьше.
        \item  Между линиями индукции магнитного поля величина этого поля пренебрежимо мала.
        \item  Линии магнитного поля всегда замкнуты.
        \item  Линии магнитного поля могут пересекаться в полюсах магнитов.
        \item  Линии магнитного поля заканчиваются у северного полюса и начинаются у южного.
        \item  Чем гуще линии — тем слабее магнитное поле.
        \item  Северный географический полюс Земли в точности совпадает с северным магнитным полюсом Земли.
        \item  Если в компасе установить сильный магнит, то его не удастся отклонить магнитным полем неподалёку.
                Так не делают лишь потому, что компас станет слишком неудобным в бытовом использовании.
        \item  Внутри магнита есть магнитное поле, поэтому для честности мы обязаны рисовать поле как снаружи, так и внутри него.
    \end{itemize}
}
\answer{%
    $\text{нет, да, нет, да, нет, нет, нет, нет, нет, да}$
}

\tasknumber{2}%
\task{%
    Изобразите линии индукции магнитного поля вокруг постоянного магнита.

    \begin{tikzpicture}[x=1cm,y=1cm,thick]
        \draw (0, 0) rectangle (3, 0.6);
        \node [right] (right) at (0, 0.3) {S};
        \node [left] (left) at (3, 0.3) {N};
        \node [right] (right) at (0, 2) {};
        \node [right] (right) at (0, -2) {};
        \node [right] (right) at (-2, 0) {};
        \node [right] (right) at (5, 0) {};
    \end{tikzpicture}
}

\tasknumber{3}%
\task{%
    Опишите взаимодействие полосовых магнитов.
    Нужны рисунки и необходимый минимум пояснений и терминов, трактат не нужен.
}
\solutionspace{100pt}

\tasknumber{4}%
\task{%
    Для постоянного магнита, изображённого на рис.
    1в)
    изобразите линии индукции магнитного поля
    и укажите, как соориентируются магнитные стрелки в точках B и D.
}
\solutionspace{80pt}

\tasknumber{5}%
\task{%
    Магнитная стрелка вблизи длинного прямолинейного проводника
    повёрнута в точке $A$ северным полюсом налево (см.
    рис.
    2б).
    Сделайте рисунок, укажите направление протекания электрического тока,
    изобразите линии индукции магнитного поля.
}

\variantsplitter

\addpersonalvariant{Виктория Легонькова}

\tasknumber{1}%
\task{%
    Укажите, верны ли утверждения («да» или «нет» слева от каждого утверждения):
    \begin{itemize}
        \item  Если распилить постоянный магнит на 2, то мы получим 2 магнита:
                один только с южным полюсом, а второй — только с северным.
        \item  Полосовой магнит можно распилить 3 разрезами на 4 магнита поменьше.
        \item  Между линиями индукции магнитного поля величина этого поля пренебрежимо мала.
        \item  Линии магнитного поля всегда замкнуты.
        \item  Линии магнитного поля могут пересекаться в полюсах магнитов.
        \item  Линии магнитного поля начинаются у северного полюса и заканчиваются у южного.
        \item  Чем гуще линии — тем сильнее магнитное поле.
        \item  Северный географический полюс Земли в точности совпадает с северным магнитным полюсом Земли.
        \item  Если в компасе установить сильный магнит, то его не удастся отклонить магнитным полем неподалёку.
                Так не делают лишь потому, что компас станет слишком неудобным в бытовом использовании.
        \item  Внутри магнита есть магнитное поле, поэтому для честности мы обязаны рисовать поле как снаружи, так и внутри него.
    \end{itemize}
}
\answer{%
    $\text{нет, да, нет, да, нет, нет, да, нет, нет, да}$
}

\tasknumber{2}%
\task{%
    Изобразите линии индукции магнитного поля вокруг постоянного магнита.

    \begin{tikzpicture}[x=1cm,y=1cm,thick]
        \draw (0, 0) rectangle (3, 0.6);
        \node [right] (right) at (0, 0.3) {N};
        \node [left] (left) at (3, 0.3) {S};
        \node [right] (right) at (0, 2) {};
        \node [right] (right) at (0, -2) {};
        \node [right] (right) at (-2, 0) {};
        \node [right] (right) at (5, 0) {};
    \end{tikzpicture}
}

\tasknumber{3}%
\task{%
    Опишите взаимодействие полосовых магнитов.
    Нужны рисунки и необходимый минимум пояснений и терминов, трактат не нужен.
}
\solutionspace{100pt}

\tasknumber{4}%
\task{%
    Для постоянного магнита, изображённого на рис.
    1а)
    изобразите линии индукции магнитного поля
    и укажите, как соориентируются магнитные стрелки в точках A и C.
}
\solutionspace{80pt}

\tasknumber{5}%
\task{%
    Магнитная стрелка вблизи длинного прямолинейного проводника
    повёрнута в точке $A$ северным полюсом вниз (см.
    рис.
    2г).
    Сделайте рисунок, укажите направление протекания электрического тока,
    изобразите линии индукции магнитного поля.
}

\variantsplitter

\addpersonalvariant{Семён Мартынов}

\tasknumber{1}%
\task{%
    Укажите, верны ли утверждения («да» или «нет» слева от каждого утверждения):
    \begin{itemize}
        \item  Если распилить постоянный магнит на 2, то мы получим 2 магнита:
                один только с южным полюсом, а второй — только с северным.
        \item  Полосовой магнит можно распилить 2 разрезами на 3 магнита поменьше.
        \item  Между линиями индукции магнитного поля величина этого поля пренебрежимо мала.
        \item  Линии магнитного поля всегда замкнуты.
        \item  Линии магнитного поля могут пересекаться в полюсах магнитов.
        \item  Линии магнитного поля начинаются у северного полюса и заканчиваются у южного.
        \item  Чем гуще линии — тем сильнее магнитное поле.
        \item  Северный географический полюс Земли в точности совпадает с южным магнитным полюсом Земли.
        \item  Если в компасе установить сильный магнит, то его не удастся отклонить магнитным полем неподалёку.
                Так не делают лишь потому, что компас станет слишком неудобным в бытовом использовании.
        \item  Внутри магнита есть магнитное поле, поэтому для честности мы обязаны рисовать поле как снаружи, так и внутри него.
    \end{itemize}
}
\answer{%
    $\text{нет, да, нет, да, нет, нет, да, нет, нет, да}$
}

\tasknumber{2}%
\task{%
    Изобразите линии индукции магнитного поля вокруг постоянного магнита.

    \begin{tikzpicture}[x=1cm,y=1cm,thick]
        \draw (0, 0) rectangle (3, 0.6);
        \node [right] (right) at (0, 0.3) {S};
        \node [left] (left) at (3, 0.3) {N};
        \node [right] (right) at (0, 2) {};
        \node [right] (right) at (0, -2) {};
        \node [right] (right) at (-2, 0) {};
        \node [right] (right) at (5, 0) {};
    \end{tikzpicture}
}

\tasknumber{3}%
\task{%
    Опишите взаимодействие параллельных прямых токов.
    Нужны рисунки и необходимый минимум пояснений и терминов, трактат не нужен.
}
\solutionspace{100pt}

\tasknumber{4}%
\task{%
    Для постоянного магнита, изображённого на рис.
    1б)
    изобразите линии индукции магнитного поля
    и укажите, как соориентируются магнитные стрелки в точках A и B.
}
\solutionspace{80pt}

\tasknumber{5}%
\task{%
    Магнитная стрелка вблизи длинного прямолинейного проводника
    повёрнута в точке $A$ северным полюсом направо (см.
    рис.
    2б).
    Сделайте рисунок, укажите направление протекания электрического тока,
    изобразите линии индукции магнитного поля.
}

\variantsplitter

\addpersonalvariant{Варвара Минаева}

\tasknumber{1}%
\task{%
    Укажите, верны ли утверждения («да» или «нет» слева от каждого утверждения):
    \begin{itemize}
        \item  Если распилить постоянный магнит на 2, то мы получим 2 магнита:
                один только с южным полюсом, а второй — только с северным.
        \item  Полосовой магнит можно распилить 3 разрезами на 4 магнита поменьше.
        \item  Между линиями индукции магнитного поля величина этого поля пренебрежимо мала.
        \item  Линии магнитного поля всегда замкнуты.
        \item  Линии магнитного поля могут пересекаться в полюсах магнитов.
        \item  Линии магнитного поля начинаются у северного полюса и заканчиваются у южного.
        \item  Чем гуще линии — тем слабее магнитное поле.
        \item  Северный географический полюс Земли в точности совпадает с северным магнитным полюсом Земли.
        \item  Если в компасе установить сильный магнит, то его не удастся отклонить магнитным полем неподалёку.
                Так не делают лишь потому, что компас станет слишком неудобным в бытовом использовании.
        \item  Внутри магнита есть магнитное поле, поэтому для честности мы обязаны рисовать поле как снаружи, так и внутри него.
    \end{itemize}
}
\answer{%
    $\text{нет, да, нет, да, нет, нет, нет, нет, нет, да}$
}

\tasknumber{2}%
\task{%
    Изобразите линии индукции магнитного поля вокруг постоянного магнита.

    \begin{tikzpicture}[x=1cm,y=1cm,thick]
        \draw (0, 0) rectangle (3, 0.6);
        \node [right] (right) at (0, 0.3) {N};
        \node [left] (left) at (3, 0.3) {S};
        \node [right] (right) at (0, 2) {};
        \node [right] (right) at (0, -2) {};
        \node [right] (right) at (-2, 0) {};
        \node [right] (right) at (5, 0) {};
    \end{tikzpicture}
}

\tasknumber{3}%
\task{%
    Опишите опыт Эрстеда.
    Нужны рисунки и необходимый минимум пояснений и терминов, трактат не нужен.
}
\solutionspace{100pt}

\tasknumber{4}%
\task{%
    Для постоянного магнита, изображённого на рис.
    1а)
    изобразите линии индукции магнитного поля
    и укажите, как соориентируются магнитные стрелки в точках B и C.
}
\solutionspace{80pt}

\tasknumber{5}%
\task{%
    Магнитная стрелка вблизи длинного прямолинейного проводника
    повёрнута в точке $B$ северным полюсом вниз (см.
    рис.
    2в).
    Сделайте рисунок, укажите направление протекания электрического тока,
    изобразите линии индукции магнитного поля.
}

\variantsplitter

\addpersonalvariant{Леонид Никитин}

\tasknumber{1}%
\task{%
    Укажите, верны ли утверждения («да» или «нет» слева от каждого утверждения):
    \begin{itemize}
        \item  Если распилить постоянный магнит на 2, то мы получим 2 магнита:
                один только с южным полюсом, а второй — только с северным.
        \item  Полосовой магнит можно распилить 2 разрезами на 3 магнита поменьше.
        \item  Между линиями индукции магнитного поля величина этого поля пренебрежимо мала.
        \item  Линии магнитного поля всегда замкнуты.
        \item  Линии магнитного поля могут пересекаться в полюсах магнитов.
        \item  Линии магнитного поля заканчиваются у северного полюса и начинаются у южного.
        \item  Чем гуще линии — тем слабее магнитное поле.
        \item  Северный географический полюс Земли в точности совпадает с южным магнитным полюсом Земли.
        \item  Если в компасе установить сильный магнит, то его не удастся отклонить магнитным полем неподалёку.
                Так не делают лишь потому, что компас станет слишком неудобным в бытовом использовании.
        \item  Внутри магнита есть магнитное поле, поэтому для честности мы обязаны рисовать поле как снаружи, так и внутри него.
    \end{itemize}
}
\answer{%
    $\text{нет, да, нет, да, нет, нет, нет, нет, нет, да}$
}

\tasknumber{2}%
\task{%
    Изобразите линии индукции магнитного поля вокруг постоянного магнита.

    \begin{tikzpicture}[x=1cm,y=1cm,thick]
        \draw (0, 0) rectangle (3, 0.6);
        \node [right] (right) at (0, 0.3) {N};
        \node [left] (left) at (3, 0.3) {S};
        \node [right] (right) at (0, 2) {};
        \node [right] (right) at (0, -2) {};
        \node [right] (right) at (-2, 0) {};
        \node [right] (right) at (5, 0) {};
    \end{tikzpicture}
}

\tasknumber{3}%
\task{%
    Опишите взаимодействие полосовых магнитов.
    Нужны рисунки и необходимый минимум пояснений и терминов, трактат не нужен.
}
\solutionspace{100pt}

\tasknumber{4}%
\task{%
    Для постоянного магнита, изображённого на рис.
    1г)
    изобразите линии индукции магнитного поля
    и укажите, как соориентируются магнитные стрелки в точках C и D.
}
\solutionspace{80pt}

\tasknumber{5}%
\task{%
    Магнитная стрелка вблизи длинного прямолинейного проводника
    повёрнута в точке $B$ северным полюсом вверх (см.
    рис.
    2в).
    Сделайте рисунок, укажите направление протекания электрического тока,
    изобразите линии индукции магнитного поля.
}

\variantsplitter

\addpersonalvariant{Тимофей Полетаев}

\tasknumber{1}%
\task{%
    Укажите, верны ли утверждения («да» или «нет» слева от каждого утверждения):
    \begin{itemize}
        \item  Если распилить постоянный магнит на 2, то мы получим 2 магнита:
                один только с южным полюсом, а второй — только с северным.
        \item  Полосовой магнит можно распилить 2 разрезами на 3 магнита поменьше.
        \item  Между линиями индукции магнитного поля величина этого поля пренебрежимо мала.
        \item  Линии магнитного поля всегда замкнуты.
        \item  Линии магнитного поля могут пересекаться в полюсах магнитов.
        \item  Линии магнитного поля начинаются у северного полюса и заканчиваются у южного.
        \item  Чем гуще линии — тем сильнее магнитное поле.
        \item  Северный географический полюс Земли в точности совпадает с северным магнитным полюсом Земли.
        \item  Если в компасе установить сильный магнит, то его не удастся отклонить магнитным полем неподалёку.
                Так не делают лишь потому, что компас станет слишком неудобным в бытовом использовании.
        \item  Внутри магнита есть магнитное поле, поэтому для честности мы обязаны рисовать поле как снаружи, так и внутри него.
    \end{itemize}
}
\answer{%
    $\text{нет, да, нет, да, нет, нет, да, нет, нет, да}$
}

\tasknumber{2}%
\task{%
    Изобразите линии индукции магнитного поля вокруг постоянного магнита.

    \begin{tikzpicture}[x=1cm,y=1cm,thick]
        \draw (0, 0) rectangle (3, 0.6);
        \node [right] (right) at (0, 0.3) {S};
        \node [left] (left) at (3, 0.3) {N};
        \node [right] (right) at (0, 2) {};
        \node [right] (right) at (0, -2) {};
        \node [right] (right) at (-2, 0) {};
        \node [right] (right) at (5, 0) {};
    \end{tikzpicture}
}

\tasknumber{3}%
\task{%
    Опишите опыт Эрстеда.
    Нужны рисунки и необходимый минимум пояснений и терминов, трактат не нужен.
}
\solutionspace{100pt}

\tasknumber{4}%
\task{%
    Для постоянного магнита, изображённого на рис.
    1г)
    изобразите линии индукции магнитного поля
    и укажите, как соориентируются магнитные стрелки в точках A и B.
}
\solutionspace{80pt}

\tasknumber{5}%
\task{%
    Магнитная стрелка вблизи длинного прямолинейного проводника
    повёрнута в точке $A$ северным полюсом вверх (см.
    рис.
    2в).
    Сделайте рисунок, укажите направление протекания электрического тока,
    изобразите линии индукции магнитного поля.
}

\variantsplitter

\addpersonalvariant{Андрей Рожков}

\tasknumber{1}%
\task{%
    Укажите, верны ли утверждения («да» или «нет» слева от каждого утверждения):
    \begin{itemize}
        \item  Если распилить постоянный магнит на 2, то мы получим 2 магнита:
                один только с южным полюсом, а второй — только с северным.
        \item  Полосовой магнит можно распилить 3 разрезами на 4 магнита поменьше.
        \item  Между линиями индукции магнитного поля величина этого поля пренебрежимо мала.
        \item  Линии магнитного поля всегда замкнуты.
        \item  Линии магнитного поля могут пересекаться в полюсах магнитов.
        \item  Линии магнитного поля начинаются у северного полюса и заканчиваются у южного.
        \item  Чем гуще линии — тем сильнее магнитное поле.
        \item  Северный географический полюс Земли в точности совпадает с южным магнитным полюсом Земли.
        \item  Если в компасе установить сильный магнит, то его не удастся отклонить магнитным полем неподалёку.
                Так не делают лишь потому, что компас станет слишком неудобным в бытовом использовании.
        \item  Внутри магнита есть магнитное поле, поэтому для честности мы обязаны рисовать поле как снаружи, так и внутри него.
    \end{itemize}
}
\answer{%
    $\text{нет, да, нет, да, нет, нет, да, нет, нет, да}$
}

\tasknumber{2}%
\task{%
    Изобразите линии индукции магнитного поля вокруг постоянного магнита.

    \begin{tikzpicture}[x=1cm,y=1cm,thick]
        \draw (0, 0) rectangle (3, 0.6);
        \node [right] (right) at (0, 0.3) {S};
        \node [left] (left) at (3, 0.3) {N};
        \node [right] (right) at (0, 2) {};
        \node [right] (right) at (0, -2) {};
        \node [right] (right) at (-2, 0) {};
        \node [right] (right) at (5, 0) {};
    \end{tikzpicture}
}

\tasknumber{3}%
\task{%
    Опишите взаимодействие полосовых магнитов.
    Нужны рисунки и необходимый минимум пояснений и терминов, трактат не нужен.
}
\solutionspace{100pt}

\tasknumber{4}%
\task{%
    Для постоянного магнита, изображённого на рис.
    1г)
    изобразите линии индукции магнитного поля
    и укажите, как соориентируются магнитные стрелки в точках A и D.
}
\solutionspace{80pt}

\tasknumber{5}%
\task{%
    Магнитная стрелка вблизи длинного прямолинейного проводника
    повёрнута в точке $B$ северным полюсом налево (см.
    рис.
    2б).
    Сделайте рисунок, укажите направление протекания электрического тока,
    изобразите линии индукции магнитного поля.
}

\variantsplitter

\addpersonalvariant{Рената Таржиманова}

\tasknumber{1}%
\task{%
    Укажите, верны ли утверждения («да» или «нет» слева от каждого утверждения):
    \begin{itemize}
        \item  Если распилить постоянный магнит на 2, то мы получим 2 магнита:
                один только с южным полюсом, а второй — только с северным.
        \item  Полосовой магнит можно распилить 3 разрезами на 4 магнита поменьше.
        \item  Между линиями индукции магнитного поля величина этого поля пренебрежимо мала.
        \item  Линии магнитного поля всегда замкнуты.
        \item  Линии магнитного поля могут пересекаться в полюсах магнитов.
        \item  Линии магнитного поля заканчиваются у северного полюса и начинаются у южного.
        \item  Чем гуще линии — тем сильнее магнитное поле.
        \item  Северный географический полюс Земли в точности совпадает с северным магнитным полюсом Земли.
        \item  Если в компасе установить сильный магнит, то его не удастся отклонить магнитным полем неподалёку.
                Так не делают лишь потому, что компас станет слишком неудобным в бытовом использовании.
        \item  Внутри магнита есть магнитное поле, поэтому для честности мы обязаны рисовать поле как снаружи, так и внутри него.
    \end{itemize}
}
\answer{%
    $\text{нет, да, нет, да, нет, нет, да, нет, нет, да}$
}

\tasknumber{2}%
\task{%
    Изобразите линии индукции магнитного поля вокруг постоянного магнита.

    \begin{tikzpicture}[x=1cm,y=1cm,thick]
        \draw (0, 0) rectangle (3, 0.6);
        \node [right] (right) at (0, 0.3) {S};
        \node [left] (left) at (3, 0.3) {N};
        \node [right] (right) at (0, 2) {};
        \node [right] (right) at (0, -2) {};
        \node [right] (right) at (-2, 0) {};
        \node [right] (right) at (5, 0) {};
    \end{tikzpicture}
}

\tasknumber{3}%
\task{%
    Опишите взаимодействие полосовых магнитов.
    Нужны рисунки и необходимый минимум пояснений и терминов, трактат не нужен.
}
\solutionspace{100pt}

\tasknumber{4}%
\task{%
    Для постоянного магнита, изображённого на рис.
    1г)
    изобразите линии индукции магнитного поля
    и укажите, как соориентируются магнитные стрелки в точках B и D.
}
\solutionspace{80pt}

\tasknumber{5}%
\task{%
    Магнитная стрелка вблизи длинного прямолинейного проводника
    повёрнута в точке $B$ северным полюсом направо (см.
    рис.
    2б).
    Сделайте рисунок, укажите направление протекания электрического тока,
    изобразите линии индукции магнитного поля.
}

\variantsplitter

\addpersonalvariant{Андрей Щербаков}

\tasknumber{1}%
\task{%
    Укажите, верны ли утверждения («да» или «нет» слева от каждого утверждения):
    \begin{itemize}
        \item  Если распилить постоянный магнит на 2, то мы получим 2 магнита:
                один только с южным полюсом, а второй — только с северным.
        \item  Полосовой магнит можно распилить 2 разрезами на 3 магнита поменьше.
        \item  Между линиями индукции магнитного поля величина этого поля пренебрежимо мала.
        \item  Линии магнитного поля всегда замкнуты.
        \item  Линии магнитного поля могут пересекаться в полюсах магнитов.
        \item  Линии магнитного поля заканчиваются у северного полюса и начинаются у южного.
        \item  Чем гуще линии — тем слабее магнитное поле.
        \item  Северный географический полюс Земли в точности совпадает с южным магнитным полюсом Земли.
        \item  Если в компасе установить сильный магнит, то его не удастся отклонить магнитным полем неподалёку.
                Так не делают лишь потому, что компас станет слишком неудобным в бытовом использовании.
        \item  Внутри магнита есть магнитное поле, поэтому для честности мы обязаны рисовать поле как снаружи, так и внутри него.
    \end{itemize}
}
\answer{%
    $\text{нет, да, нет, да, нет, нет, нет, нет, нет, да}$
}

\tasknumber{2}%
\task{%
    Изобразите линии индукции магнитного поля вокруг постоянного магнита.

    \begin{tikzpicture}[x=1cm,y=1cm,thick]
        \draw (0, 0) rectangle (3, 0.6);
        \node [right] (right) at (0, 0.3) {S};
        \node [left] (left) at (3, 0.3) {N};
        \node [right] (right) at (0, 2) {};
        \node [right] (right) at (0, -2) {};
        \node [right] (right) at (-2, 0) {};
        \node [right] (right) at (5, 0) {};
    \end{tikzpicture}
}

\tasknumber{3}%
\task{%
    Опишите опыт Эрстеда.
    Нужны рисунки и необходимый минимум пояснений и терминов, трактат не нужен.
}
\solutionspace{100pt}

\tasknumber{4}%
\task{%
    Для постоянного магнита, изображённого на рис.
    1б)
    изобразите линии индукции магнитного поля
    и укажите, как соориентируются магнитные стрелки в точках C и D.
}
\solutionspace{80pt}

\tasknumber{5}%
\task{%
    Магнитная стрелка вблизи длинного прямолинейного проводника
    повёрнута в точке $B$ северным полюсом вверх (см.
    рис.
    2г).
    Сделайте рисунок, укажите направление протекания электрического тока,
    изобразите линии индукции магнитного поля.
}

\variantsplitter

\addpersonalvariant{Михаил Ярошевский}

\tasknumber{1}%
\task{%
    Укажите, верны ли утверждения («да» или «нет» слева от каждого утверждения):
    \begin{itemize}
        \item  Если распилить постоянный магнит на 2, то мы получим 2 магнита:
                один только с южным полюсом, а второй — только с северным.
        \item  Полосовой магнит можно распилить 3 разрезами на 4 магнита поменьше.
        \item  Между линиями индукции магнитного поля величина этого поля пренебрежимо мала.
        \item  Линии магнитного поля всегда замкнуты.
        \item  Линии магнитного поля могут пересекаться в полюсах магнитов.
        \item  Линии магнитного поля заканчиваются у северного полюса и начинаются у южного.
        \item  Чем гуще линии — тем сильнее магнитное поле.
        \item  Северный географический полюс Земли в точности совпадает с северным магнитным полюсом Земли.
        \item  Если в компасе установить сильный магнит, то его не удастся отклонить магнитным полем неподалёку.
                Так не делают лишь потому, что компас станет слишком неудобным в бытовом использовании.
        \item  Внутри магнита есть магнитное поле, поэтому для честности мы обязаны рисовать поле как снаружи, так и внутри него.
    \end{itemize}
}
\answer{%
    $\text{нет, да, нет, да, нет, нет, да, нет, нет, да}$
}

\tasknumber{2}%
\task{%
    Изобразите линии индукции магнитного поля вокруг постоянного магнита.

    \begin{tikzpicture}[x=1cm,y=1cm,thick]
        \draw (0, 0) rectangle (3, 0.6);
        \node [right] (right) at (0, 0.3) {S};
        \node [left] (left) at (3, 0.3) {N};
        \node [right] (right) at (0, 2) {};
        \node [right] (right) at (0, -2) {};
        \node [right] (right) at (-2, 0) {};
        \node [right] (right) at (5, 0) {};
    \end{tikzpicture}
}

\tasknumber{3}%
\task{%
    Опишите взаимодействие полосовых магнитов.
    Нужны рисунки и необходимый минимум пояснений и терминов, трактат не нужен.
}
\solutionspace{100pt}

\tasknumber{4}%
\task{%
    Для постоянного магнита, изображённого на рис.
    1в)
    изобразите линии индукции магнитного поля
    и укажите, как соориентируются магнитные стрелки в точках A и B.
}
\solutionspace{80pt}

\tasknumber{5}%
\task{%
    Магнитная стрелка вблизи длинного прямолинейного проводника
    повёрнута в точке $B$ северным полюсом налево (см.
    рис.
    2б).
    Сделайте рисунок, укажите направление протекания электрического тока,
    изобразите линии индукции магнитного поля.
}

\variantsplitter

\addpersonalvariant{Алексей Алимпиев}

\tasknumber{1}%
\task{%
    Укажите, верны ли утверждения («да» или «нет» слева от каждого утверждения):
    \begin{itemize}
        \item  Если распилить постоянный магнит на 2, то мы получим 2 магнита:
                один только с южным полюсом, а второй — только с северным.
        \item  Полосовой магнит можно распилить 2 разрезами на 3 магнита поменьше.
        \item  Между линиями индукции магнитного поля величина этого поля пренебрежимо мала.
        \item  Линии магнитного поля всегда замкнуты.
        \item  Линии магнитного поля могут пересекаться в полюсах магнитов.
        \item  Линии магнитного поля начинаются у северного полюса и заканчиваются у южного.
        \item  Чем гуще линии — тем слабее магнитное поле.
        \item  Северный географический полюс Земли в точности совпадает с южным магнитным полюсом Земли.
        \item  Если в компасе установить сильный магнит, то его не удастся отклонить магнитным полем неподалёку.
                Так не делают лишь потому, что компас станет слишком неудобным в бытовом использовании.
        \item  Внутри магнита есть магнитное поле, поэтому для честности мы обязаны рисовать поле как снаружи, так и внутри него.
    \end{itemize}
}
\answer{%
    $\text{нет, да, нет, да, нет, нет, нет, нет, нет, да}$
}

\tasknumber{2}%
\task{%
    Изобразите линии индукции магнитного поля вокруг постоянного магнита.

    \begin{tikzpicture}[x=1cm,y=1cm,thick]
        \draw (0, 0) rectangle (3, 0.6);
        \node [right] (right) at (0, 0.3) {N};
        \node [left] (left) at (3, 0.3) {S};
        \node [right] (right) at (0, 2) {};
        \node [right] (right) at (0, -2) {};
        \node [right] (right) at (-2, 0) {};
        \node [right] (right) at (5, 0) {};
    \end{tikzpicture}
}

\tasknumber{3}%
\task{%
    Опишите опыт Эрстеда.
    Нужны рисунки и необходимый минимум пояснений и терминов, трактат не нужен.
}
\solutionspace{100pt}

\tasknumber{4}%
\task{%
    Для постоянного магнита, изображённого на рис.
    1г)
    изобразите линии индукции магнитного поля
    и укажите, как соориентируются магнитные стрелки в точках C и D.
}
\solutionspace{80pt}

\tasknumber{5}%
\task{%
    Магнитная стрелка вблизи длинного прямолинейного проводника
    повёрнута в точке $A$ северным полюсом вниз (см.
    рис.
    2в).
    Сделайте рисунок, укажите направление протекания электрического тока,
    изобразите линии индукции магнитного поля.
}

\variantsplitter

\addpersonalvariant{Евгений Васин}

\tasknumber{1}%
\task{%
    Укажите, верны ли утверждения («да» или «нет» слева от каждого утверждения):
    \begin{itemize}
        \item  Если распилить постоянный магнит на 2, то мы получим 2 магнита:
                один только с южным полюсом, а второй — только с северным.
        \item  Полосовой магнит можно распилить 2 разрезами на 3 магнита поменьше.
        \item  Между линиями индукции магнитного поля величина этого поля пренебрежимо мала.
        \item  Линии магнитного поля всегда замкнуты.
        \item  Линии магнитного поля могут пересекаться в полюсах магнитов.
        \item  Линии магнитного поля заканчиваются у северного полюса и начинаются у южного.
        \item  Чем гуще линии — тем сильнее магнитное поле.
        \item  Северный географический полюс Земли в точности совпадает с северным магнитным полюсом Земли.
        \item  Если в компасе установить сильный магнит, то его не удастся отклонить магнитным полем неподалёку.
                Так не делают лишь потому, что компас станет слишком неудобным в бытовом использовании.
        \item  Внутри магнита есть магнитное поле, поэтому для честности мы обязаны рисовать поле как снаружи, так и внутри него.
    \end{itemize}
}
\answer{%
    $\text{нет, да, нет, да, нет, нет, да, нет, нет, да}$
}

\tasknumber{2}%
\task{%
    Изобразите линии индукции магнитного поля вокруг постоянного магнита.

    \begin{tikzpicture}[x=1cm,y=1cm,thick]
        \draw (0, 0) rectangle (3, 0.6);
        \node [right] (right) at (0, 0.3) {N};
        \node [left] (left) at (3, 0.3) {S};
        \node [right] (right) at (0, 2) {};
        \node [right] (right) at (0, -2) {};
        \node [right] (right) at (-2, 0) {};
        \node [right] (right) at (5, 0) {};
    \end{tikzpicture}
}

\tasknumber{3}%
\task{%
    Опишите опыт Эрстеда.
    Нужны рисунки и необходимый минимум пояснений и терминов, трактат не нужен.
}
\solutionspace{100pt}

\tasknumber{4}%
\task{%
    Для постоянного магнита, изображённого на рис.
    1б)
    изобразите линии индукции магнитного поля
    и укажите, как соориентируются магнитные стрелки в точках A и D.
}
\solutionspace{80pt}

\tasknumber{5}%
\task{%
    Магнитная стрелка вблизи длинного прямолинейного проводника
    повёрнута в точке $A$ северным полюсом вниз (см.
    рис.
    2в).
    Сделайте рисунок, укажите направление протекания электрического тока,
    изобразите линии индукции магнитного поля.
}

\variantsplitter

\addpersonalvariant{Вячеслав Волохов}

\tasknumber{1}%
\task{%
    Укажите, верны ли утверждения («да» или «нет» слева от каждого утверждения):
    \begin{itemize}
        \item  Если распилить постоянный магнит на 2, то мы получим 2 магнита:
                один только с южным полюсом, а второй — только с северным.
        \item  Полосовой магнит можно распилить 2 разрезами на 3 магнита поменьше.
        \item  Между линиями индукции магнитного поля величина этого поля пренебрежимо мала.
        \item  Линии магнитного поля всегда замкнуты.
        \item  Линии магнитного поля могут пересекаться в полюсах магнитов.
        \item  Линии магнитного поля заканчиваются у северного полюса и начинаются у южного.
        \item  Чем гуще линии — тем слабее магнитное поле.
        \item  Северный географический полюс Земли в точности совпадает с южным магнитным полюсом Земли.
        \item  Если в компасе установить сильный магнит, то его не удастся отклонить магнитным полем неподалёку.
                Так не делают лишь потому, что компас станет слишком неудобным в бытовом использовании.
        \item  Внутри магнита есть магнитное поле, поэтому для честности мы обязаны рисовать поле как снаружи, так и внутри него.
    \end{itemize}
}
\answer{%
    $\text{нет, да, нет, да, нет, нет, нет, нет, нет, да}$
}

\tasknumber{2}%
\task{%
    Изобразите линии индукции магнитного поля вокруг постоянного магнита.

    \begin{tikzpicture}[x=1cm,y=1cm,thick]
        \draw (0, 0) rectangle (3, 0.6);
        \node [right] (right) at (0, 0.3) {N};
        \node [left] (left) at (3, 0.3) {S};
        \node [right] (right) at (0, 2) {};
        \node [right] (right) at (0, -2) {};
        \node [right] (right) at (-2, 0) {};
        \node [right] (right) at (5, 0) {};
    \end{tikzpicture}
}

\tasknumber{3}%
\task{%
    Опишите взаимодействие полосовых магнитов.
    Нужны рисунки и необходимый минимум пояснений и терминов, трактат не нужен.
}
\solutionspace{100pt}

\tasknumber{4}%
\task{%
    Для постоянного магнита, изображённого на рис.
    1б)
    изобразите линии индукции магнитного поля
    и укажите, как соориентируются магнитные стрелки в точках B и D.
}
\solutionspace{80pt}

\tasknumber{5}%
\task{%
    Магнитная стрелка вблизи длинного прямолинейного проводника
    повёрнута в точке $B$ северным полюсом налево (см.
    рис.
    2б).
    Сделайте рисунок, укажите направление протекания электрического тока,
    изобразите линии индукции магнитного поля.
}

\variantsplitter

\addpersonalvariant{Герман Говоров}

\tasknumber{1}%
\task{%
    Укажите, верны ли утверждения («да» или «нет» слева от каждого утверждения):
    \begin{itemize}
        \item  Если распилить постоянный магнит на 2, то мы получим 2 магнита:
                один только с южным полюсом, а второй — только с северным.
        \item  Полосовой магнит можно распилить 2 разрезами на 3 магнита поменьше.
        \item  Между линиями индукции магнитного поля величина этого поля пренебрежимо мала.
        \item  Линии магнитного поля всегда замкнуты.
        \item  Линии магнитного поля могут пересекаться в полюсах магнитов.
        \item  Линии магнитного поля начинаются у северного полюса и заканчиваются у южного.
        \item  Чем гуще линии — тем сильнее магнитное поле.
        \item  Северный географический полюс Земли в точности совпадает с южным магнитным полюсом Земли.
        \item  Если в компасе установить сильный магнит, то его не удастся отклонить магнитным полем неподалёку.
                Так не делают лишь потому, что компас станет слишком неудобным в бытовом использовании.
        \item  Внутри магнита есть магнитное поле, поэтому для честности мы обязаны рисовать поле как снаружи, так и внутри него.
    \end{itemize}
}
\answer{%
    $\text{нет, да, нет, да, нет, нет, да, нет, нет, да}$
}

\tasknumber{2}%
\task{%
    Изобразите линии индукции магнитного поля вокруг постоянного магнита.

    \begin{tikzpicture}[x=1cm,y=1cm,thick]
        \draw (0, 0) rectangle (3, 0.6);
        \node [right] (right) at (0, 0.3) {N};
        \node [left] (left) at (3, 0.3) {S};
        \node [right] (right) at (0, 2) {};
        \node [right] (right) at (0, -2) {};
        \node [right] (right) at (-2, 0) {};
        \node [right] (right) at (5, 0) {};
    \end{tikzpicture}
}

\tasknumber{3}%
\task{%
    Опишите опыт Эрстеда.
    Нужны рисунки и необходимый минимум пояснений и терминов, трактат не нужен.
}
\solutionspace{100pt}

\tasknumber{4}%
\task{%
    Для постоянного магнита, изображённого на рис.
    1б)
    изобразите линии индукции магнитного поля
    и укажите, как соориентируются магнитные стрелки в точках A и D.
}
\solutionspace{80pt}

\tasknumber{5}%
\task{%
    Магнитная стрелка вблизи длинного прямолинейного проводника
    повёрнута в точке $B$ северным полюсом направо (см.
    рис.
    2б).
    Сделайте рисунок, укажите направление протекания электрического тока,
    изобразите линии индукции магнитного поля.
}

\variantsplitter

\addpersonalvariant{София Журавлёва}

\tasknumber{1}%
\task{%
    Укажите, верны ли утверждения («да» или «нет» слева от каждого утверждения):
    \begin{itemize}
        \item  Если распилить постоянный магнит на 2, то мы получим 2 магнита:
                один только с южным полюсом, а второй — только с северным.
        \item  Полосовой магнит можно распилить 3 разрезами на 4 магнита поменьше.
        \item  Между линиями индукции магнитного поля величина этого поля пренебрежимо мала.
        \item  Линии магнитного поля всегда замкнуты.
        \item  Линии магнитного поля могут пересекаться в полюсах магнитов.
        \item  Линии магнитного поля заканчиваются у северного полюса и начинаются у южного.
        \item  Чем гуще линии — тем слабее магнитное поле.
        \item  Северный географический полюс Земли в точности совпадает с северным магнитным полюсом Земли.
        \item  Если в компасе установить сильный магнит, то его не удастся отклонить магнитным полем неподалёку.
                Так не делают лишь потому, что компас станет слишком неудобным в бытовом использовании.
        \item  Внутри магнита есть магнитное поле, поэтому для честности мы обязаны рисовать поле как снаружи, так и внутри него.
    \end{itemize}
}
\answer{%
    $\text{нет, да, нет, да, нет, нет, нет, нет, нет, да}$
}

\tasknumber{2}%
\task{%
    Изобразите линии индукции магнитного поля вокруг постоянного магнита.

    \begin{tikzpicture}[x=1cm,y=1cm,thick]
        \draw (0, 0) rectangle (3, 0.6);
        \node [right] (right) at (0, 0.3) {S};
        \node [left] (left) at (3, 0.3) {N};
        \node [right] (right) at (0, 2) {};
        \node [right] (right) at (0, -2) {};
        \node [right] (right) at (-2, 0) {};
        \node [right] (right) at (5, 0) {};
    \end{tikzpicture}
}

\tasknumber{3}%
\task{%
    Опишите опыт Эрстеда.
    Нужны рисунки и необходимый минимум пояснений и терминов, трактат не нужен.
}
\solutionspace{100pt}

\tasknumber{4}%
\task{%
    Для постоянного магнита, изображённого на рис.
    1а)
    изобразите линии индукции магнитного поля
    и укажите, как соориентируются магнитные стрелки в точках B и D.
}
\solutionspace{80pt}

\tasknumber{5}%
\task{%
    Магнитная стрелка вблизи длинного прямолинейного проводника
    повёрнута в точке $B$ северным полюсом вниз (см.
    рис.
    2г).
    Сделайте рисунок, укажите направление протекания электрического тока,
    изобразите линии индукции магнитного поля.
}

\variantsplitter

\addpersonalvariant{Константин Козлов}

\tasknumber{1}%
\task{%
    Укажите, верны ли утверждения («да» или «нет» слева от каждого утверждения):
    \begin{itemize}
        \item  Если распилить постоянный магнит на 2, то мы получим 2 магнита:
                один только с южным полюсом, а второй — только с северным.
        \item  Полосовой магнит можно распилить 3 разрезами на 4 магнита поменьше.
        \item  Между линиями индукции магнитного поля величина этого поля пренебрежимо мала.
        \item  Линии магнитного поля всегда замкнуты.
        \item  Линии магнитного поля могут пересекаться в полюсах магнитов.
        \item  Линии магнитного поля начинаются у северного полюса и заканчиваются у южного.
        \item  Чем гуще линии — тем слабее магнитное поле.
        \item  Северный географический полюс Земли в точности совпадает с южным магнитным полюсом Земли.
        \item  Если в компасе установить сильный магнит, то его не удастся отклонить магнитным полем неподалёку.
                Так не делают лишь потому, что компас станет слишком неудобным в бытовом использовании.
        \item  Внутри магнита есть магнитное поле, поэтому для честности мы обязаны рисовать поле как снаружи, так и внутри него.
    \end{itemize}
}
\answer{%
    $\text{нет, да, нет, да, нет, нет, нет, нет, нет, да}$
}

\tasknumber{2}%
\task{%
    Изобразите линии индукции магнитного поля вокруг постоянного магнита.

    \begin{tikzpicture}[x=1cm,y=1cm,thick]
        \draw (0, 0) rectangle (3, 0.6);
        \node [right] (right) at (0, 0.3) {N};
        \node [left] (left) at (3, 0.3) {S};
        \node [right] (right) at (0, 2) {};
        \node [right] (right) at (0, -2) {};
        \node [right] (right) at (-2, 0) {};
        \node [right] (right) at (5, 0) {};
    \end{tikzpicture}
}

\tasknumber{3}%
\task{%
    Опишите взаимодействие полосовых магнитов.
    Нужны рисунки и необходимый минимум пояснений и терминов, трактат не нужен.
}
\solutionspace{100pt}

\tasknumber{4}%
\task{%
    Для постоянного магнита, изображённого на рис.
    1б)
    изобразите линии индукции магнитного поля
    и укажите, как соориентируются магнитные стрелки в точках B и D.
}
\solutionspace{80pt}

\tasknumber{5}%
\task{%
    Магнитная стрелка вблизи длинного прямолинейного проводника
    повёрнута в точке $A$ северным полюсом налево (см.
    рис.
    2а).
    Сделайте рисунок, укажите направление протекания электрического тока,
    изобразите линии индукции магнитного поля.
}

\variantsplitter

\addpersonalvariant{Наталья Кравченко}

\tasknumber{1}%
\task{%
    Укажите, верны ли утверждения («да» или «нет» слева от каждого утверждения):
    \begin{itemize}
        \item  Если распилить постоянный магнит на 2, то мы получим 2 магнита:
                один только с южным полюсом, а второй — только с северным.
        \item  Полосовой магнит можно распилить 2 разрезами на 3 магнита поменьше.
        \item  Между линиями индукции магнитного поля величина этого поля пренебрежимо мала.
        \item  Линии магнитного поля всегда замкнуты.
        \item  Линии магнитного поля могут пересекаться в полюсах магнитов.
        \item  Линии магнитного поля начинаются у северного полюса и заканчиваются у южного.
        \item  Чем гуще линии — тем сильнее магнитное поле.
        \item  Северный географический полюс Земли в точности совпадает с северным магнитным полюсом Земли.
        \item  Если в компасе установить сильный магнит, то его не удастся отклонить магнитным полем неподалёку.
                Так не делают лишь потому, что компас станет слишком неудобным в бытовом использовании.
        \item  Внутри магнита есть магнитное поле, поэтому для честности мы обязаны рисовать поле как снаружи, так и внутри него.
    \end{itemize}
}
\answer{%
    $\text{нет, да, нет, да, нет, нет, да, нет, нет, да}$
}

\tasknumber{2}%
\task{%
    Изобразите линии индукции магнитного поля вокруг постоянного магнита.

    \begin{tikzpicture}[x=1cm,y=1cm,thick]
        \draw (0, 0) rectangle (3, 0.6);
        \node [right] (right) at (0, 0.3) {S};
        \node [left] (left) at (3, 0.3) {N};
        \node [right] (right) at (0, 2) {};
        \node [right] (right) at (0, -2) {};
        \node [right] (right) at (-2, 0) {};
        \node [right] (right) at (5, 0) {};
    \end{tikzpicture}
}

\tasknumber{3}%
\task{%
    Опишите взаимодействие параллельных прямых токов.
    Нужны рисунки и необходимый минимум пояснений и терминов, трактат не нужен.
}
\solutionspace{100pt}

\tasknumber{4}%
\task{%
    Для постоянного магнита, изображённого на рис.
    1а)
    изобразите линии индукции магнитного поля
    и укажите, как соориентируются магнитные стрелки в точках C и D.
}
\solutionspace{80pt}

\tasknumber{5}%
\task{%
    Магнитная стрелка вблизи длинного прямолинейного проводника
    повёрнута в точке $B$ северным полюсом направо (см.
    рис.
    2б).
    Сделайте рисунок, укажите направление протекания электрического тока,
    изобразите линии индукции магнитного поля.
}

\variantsplitter

\addpersonalvariant{Сергей Малышев}

\tasknumber{1}%
\task{%
    Укажите, верны ли утверждения («да» или «нет» слева от каждого утверждения):
    \begin{itemize}
        \item  Если распилить постоянный магнит на 2, то мы получим 2 магнита:
                один только с южным полюсом, а второй — только с северным.
        \item  Полосовой магнит можно распилить 2 разрезами на 3 магнита поменьше.
        \item  Между линиями индукции магнитного поля величина этого поля пренебрежимо мала.
        \item  Линии магнитного поля всегда замкнуты.
        \item  Линии магнитного поля могут пересекаться в полюсах магнитов.
        \item  Линии магнитного поля заканчиваются у северного полюса и начинаются у южного.
        \item  Чем гуще линии — тем сильнее магнитное поле.
        \item  Северный географический полюс Земли в точности совпадает с северным магнитным полюсом Земли.
        \item  Если в компасе установить сильный магнит, то его не удастся отклонить магнитным полем неподалёку.
                Так не делают лишь потому, что компас станет слишком неудобным в бытовом использовании.
        \item  Внутри магнита есть магнитное поле, поэтому для честности мы обязаны рисовать поле как снаружи, так и внутри него.
    \end{itemize}
}
\answer{%
    $\text{нет, да, нет, да, нет, нет, да, нет, нет, да}$
}

\tasknumber{2}%
\task{%
    Изобразите линии индукции магнитного поля вокруг постоянного магнита.

    \begin{tikzpicture}[x=1cm,y=1cm,thick]
        \draw (0, 0) rectangle (3, 0.6);
        \node [right] (right) at (0, 0.3) {S};
        \node [left] (left) at (3, 0.3) {N};
        \node [right] (right) at (0, 2) {};
        \node [right] (right) at (0, -2) {};
        \node [right] (right) at (-2, 0) {};
        \node [right] (right) at (5, 0) {};
    \end{tikzpicture}
}

\tasknumber{3}%
\task{%
    Опишите опыт Эрстеда.
    Нужны рисунки и необходимый минимум пояснений и терминов, трактат не нужен.
}
\solutionspace{100pt}

\tasknumber{4}%
\task{%
    Для постоянного магнита, изображённого на рис.
    1в)
    изобразите линии индукции магнитного поля
    и укажите, как соориентируются магнитные стрелки в точках C и D.
}
\solutionspace{80pt}

\tasknumber{5}%
\task{%
    Магнитная стрелка вблизи длинного прямолинейного проводника
    повёрнута в точке $A$ северным полюсом направо (см.
    рис.
    2б).
    Сделайте рисунок, укажите направление протекания электрического тока,
    изобразите линии индукции магнитного поля.
}

\variantsplitter

\addpersonalvariant{Алина Полканова}

\tasknumber{1}%
\task{%
    Укажите, верны ли утверждения («да» или «нет» слева от каждого утверждения):
    \begin{itemize}
        \item  Если распилить постоянный магнит на 2, то мы получим 2 магнита:
                один только с южным полюсом, а второй — только с северным.
        \item  Полосовой магнит можно распилить 3 разрезами на 4 магнита поменьше.
        \item  Между линиями индукции магнитного поля величина этого поля пренебрежимо мала.
        \item  Линии магнитного поля всегда замкнуты.
        \item  Линии магнитного поля могут пересекаться в полюсах магнитов.
        \item  Линии магнитного поля заканчиваются у северного полюса и начинаются у южного.
        \item  Чем гуще линии — тем сильнее магнитное поле.
        \item  Северный географический полюс Земли в точности совпадает с южным магнитным полюсом Земли.
        \item  Если в компасе установить сильный магнит, то его не удастся отклонить магнитным полем неподалёку.
                Так не делают лишь потому, что компас станет слишком неудобным в бытовом использовании.
        \item  Внутри магнита есть магнитное поле, поэтому для честности мы обязаны рисовать поле как снаружи, так и внутри него.
    \end{itemize}
}
\answer{%
    $\text{нет, да, нет, да, нет, нет, да, нет, нет, да}$
}

\tasknumber{2}%
\task{%
    Изобразите линии индукции магнитного поля вокруг постоянного магнита.

    \begin{tikzpicture}[x=1cm,y=1cm,thick]
        \draw (0, 0) rectangle (3, 0.6);
        \node [right] (right) at (0, 0.3) {N};
        \node [left] (left) at (3, 0.3) {S};
        \node [right] (right) at (0, 2) {};
        \node [right] (right) at (0, -2) {};
        \node [right] (right) at (-2, 0) {};
        \node [right] (right) at (5, 0) {};
    \end{tikzpicture}
}

\tasknumber{3}%
\task{%
    Опишите опыт Эрстеда.
    Нужны рисунки и необходимый минимум пояснений и терминов, трактат не нужен.
}
\solutionspace{100pt}

\tasknumber{4}%
\task{%
    Для постоянного магнита, изображённого на рис.
    1б)
    изобразите линии индукции магнитного поля
    и укажите, как соориентируются магнитные стрелки в точках B и C.
}
\solutionspace{80pt}

\tasknumber{5}%
\task{%
    Магнитная стрелка вблизи длинного прямолинейного проводника
    повёрнута в точке $A$ северным полюсом направо (см.
    рис.
    2б).
    Сделайте рисунок, укажите направление протекания электрического тока,
    изобразите линии индукции магнитного поля.
}

\variantsplitter

\addpersonalvariant{Сергей Пономарёв}

\tasknumber{1}%
\task{%
    Укажите, верны ли утверждения («да» или «нет» слева от каждого утверждения):
    \begin{itemize}
        \item  Если распилить постоянный магнит на 2, то мы получим 2 магнита:
                один только с южным полюсом, а второй — только с северным.
        \item  Полосовой магнит можно распилить 2 разрезами на 3 магнита поменьше.
        \item  Между линиями индукции магнитного поля величина этого поля пренебрежимо мала.
        \item  Линии магнитного поля всегда замкнуты.
        \item  Линии магнитного поля могут пересекаться в полюсах магнитов.
        \item  Линии магнитного поля начинаются у северного полюса и заканчиваются у южного.
        \item  Чем гуще линии — тем слабее магнитное поле.
        \item  Северный географический полюс Земли в точности совпадает с северным магнитным полюсом Земли.
        \item  Если в компасе установить сильный магнит, то его не удастся отклонить магнитным полем неподалёку.
                Так не делают лишь потому, что компас станет слишком неудобным в бытовом использовании.
        \item  Внутри магнита есть магнитное поле, поэтому для честности мы обязаны рисовать поле как снаружи, так и внутри него.
    \end{itemize}
}
\answer{%
    $\text{нет, да, нет, да, нет, нет, нет, нет, нет, да}$
}

\tasknumber{2}%
\task{%
    Изобразите линии индукции магнитного поля вокруг постоянного магнита.

    \begin{tikzpicture}[x=1cm,y=1cm,thick]
        \draw (0, 0) rectangle (3, 0.6);
        \node [right] (right) at (0, 0.3) {N};
        \node [left] (left) at (3, 0.3) {S};
        \node [right] (right) at (0, 2) {};
        \node [right] (right) at (0, -2) {};
        \node [right] (right) at (-2, 0) {};
        \node [right] (right) at (5, 0) {};
    \end{tikzpicture}
}

\tasknumber{3}%
\task{%
    Опишите опыт Эрстеда.
    Нужны рисунки и необходимый минимум пояснений и терминов, трактат не нужен.
}
\solutionspace{100pt}

\tasknumber{4}%
\task{%
    Для постоянного магнита, изображённого на рис.
    1б)
    изобразите линии индукции магнитного поля
    и укажите, как соориентируются магнитные стрелки в точках A и B.
}
\solutionspace{80pt}

\tasknumber{5}%
\task{%
    Магнитная стрелка вблизи длинного прямолинейного проводника
    повёрнута в точке $A$ северным полюсом вверх (см.
    рис.
    2г).
    Сделайте рисунок, укажите направление протекания электрического тока,
    изобразите линии индукции магнитного поля.
}

\variantsplitter

\addpersonalvariant{Егор Свистушкин}

\tasknumber{1}%
\task{%
    Укажите, верны ли утверждения («да» или «нет» слева от каждого утверждения):
    \begin{itemize}
        \item  Если распилить постоянный магнит на 2, то мы получим 2 магнита:
                один только с южным полюсом, а второй — только с северным.
        \item  Полосовой магнит можно распилить 2 разрезами на 3 магнита поменьше.
        \item  Между линиями индукции магнитного поля величина этого поля пренебрежимо мала.
        \item  Линии магнитного поля всегда замкнуты.
        \item  Линии магнитного поля могут пересекаться в полюсах магнитов.
        \item  Линии магнитного поля заканчиваются у северного полюса и начинаются у южного.
        \item  Чем гуще линии — тем сильнее магнитное поле.
        \item  Северный географический полюс Земли в точности совпадает с южным магнитным полюсом Земли.
        \item  Если в компасе установить сильный магнит, то его не удастся отклонить магнитным полем неподалёку.
                Так не делают лишь потому, что компас станет слишком неудобным в бытовом использовании.
        \item  Внутри магнита есть магнитное поле, поэтому для честности мы обязаны рисовать поле как снаружи, так и внутри него.
    \end{itemize}
}
\answer{%
    $\text{нет, да, нет, да, нет, нет, да, нет, нет, да}$
}

\tasknumber{2}%
\task{%
    Изобразите линии индукции магнитного поля вокруг постоянного магнита.

    \begin{tikzpicture}[x=1cm,y=1cm,thick]
        \draw (0, 0) rectangle (3, 0.6);
        \node [right] (right) at (0, 0.3) {N};
        \node [left] (left) at (3, 0.3) {S};
        \node [right] (right) at (0, 2) {};
        \node [right] (right) at (0, -2) {};
        \node [right] (right) at (-2, 0) {};
        \node [right] (right) at (5, 0) {};
    \end{tikzpicture}
}

\tasknumber{3}%
\task{%
    Опишите взаимодействие параллельных прямых токов.
    Нужны рисунки и необходимый минимум пояснений и терминов, трактат не нужен.
}
\solutionspace{100pt}

\tasknumber{4}%
\task{%
    Для постоянного магнита, изображённого на рис.
    1а)
    изобразите линии индукции магнитного поля
    и укажите, как соориентируются магнитные стрелки в точках B и D.
}
\solutionspace{80pt}

\tasknumber{5}%
\task{%
    Магнитная стрелка вблизи длинного прямолинейного проводника
    повёрнута в точке $A$ северным полюсом направо (см.
    рис.
    2б).
    Сделайте рисунок, укажите направление протекания электрического тока,
    изобразите линии индукции магнитного поля.
}

\variantsplitter

\addpersonalvariant{Дмитрий Соколов}

\tasknumber{1}%
\task{%
    Укажите, верны ли утверждения («да» или «нет» слева от каждого утверждения):
    \begin{itemize}
        \item  Если распилить постоянный магнит на 2, то мы получим 2 магнита:
                один только с южным полюсом, а второй — только с северным.
        \item  Полосовой магнит можно распилить 3 разрезами на 4 магнита поменьше.
        \item  Между линиями индукции магнитного поля величина этого поля пренебрежимо мала.
        \item  Линии магнитного поля всегда замкнуты.
        \item  Линии магнитного поля могут пересекаться в полюсах магнитов.
        \item  Линии магнитного поля начинаются у северного полюса и заканчиваются у южного.
        \item  Чем гуще линии — тем сильнее магнитное поле.
        \item  Северный географический полюс Земли в точности совпадает с северным магнитным полюсом Земли.
        \item  Если в компасе установить сильный магнит, то его не удастся отклонить магнитным полем неподалёку.
                Так не делают лишь потому, что компас станет слишком неудобным в бытовом использовании.
        \item  Внутри магнита есть магнитное поле, поэтому для честности мы обязаны рисовать поле как снаружи, так и внутри него.
    \end{itemize}
}
\answer{%
    $\text{нет, да, нет, да, нет, нет, да, нет, нет, да}$
}

\tasknumber{2}%
\task{%
    Изобразите линии индукции магнитного поля вокруг постоянного магнита.

    \begin{tikzpicture}[x=1cm,y=1cm,thick]
        \draw (0, 0) rectangle (3, 0.6);
        \node [right] (right) at (0, 0.3) {N};
        \node [left] (left) at (3, 0.3) {S};
        \node [right] (right) at (0, 2) {};
        \node [right] (right) at (0, -2) {};
        \node [right] (right) at (-2, 0) {};
        \node [right] (right) at (5, 0) {};
    \end{tikzpicture}
}

\tasknumber{3}%
\task{%
    Опишите опыт Эрстеда.
    Нужны рисунки и необходимый минимум пояснений и терминов, трактат не нужен.
}
\solutionspace{100pt}

\tasknumber{4}%
\task{%
    Для постоянного магнита, изображённого на рис.
    1а)
    изобразите линии индукции магнитного поля
    и укажите, как соориентируются магнитные стрелки в точках B и D.
}
\solutionspace{80pt}

\tasknumber{5}%
\task{%
    Магнитная стрелка вблизи длинного прямолинейного проводника
    повёрнута в точке $A$ северным полюсом вверх (см.
    рис.
    2в).
    Сделайте рисунок, укажите направление протекания электрического тока,
    изобразите линии индукции магнитного поля.
}

\variantsplitter

\addpersonalvariant{Арсений Трофимов}

\tasknumber{1}%
\task{%
    Укажите, верны ли утверждения («да» или «нет» слева от каждого утверждения):
    \begin{itemize}
        \item  Если распилить постоянный магнит на 2, то мы получим 2 магнита:
                один только с южным полюсом, а второй — только с северным.
        \item  Полосовой магнит можно распилить 3 разрезами на 4 магнита поменьше.
        \item  Между линиями индукции магнитного поля величина этого поля пренебрежимо мала.
        \item  Линии магнитного поля всегда замкнуты.
        \item  Линии магнитного поля могут пересекаться в полюсах магнитов.
        \item  Линии магнитного поля начинаются у северного полюса и заканчиваются у южного.
        \item  Чем гуще линии — тем слабее магнитное поле.
        \item  Северный географический полюс Земли в точности совпадает с северным магнитным полюсом Земли.
        \item  Если в компасе установить сильный магнит, то его не удастся отклонить магнитным полем неподалёку.
                Так не делают лишь потому, что компас станет слишком неудобным в бытовом использовании.
        \item  Внутри магнита есть магнитное поле, поэтому для честности мы обязаны рисовать поле как снаружи, так и внутри него.
    \end{itemize}
}
\answer{%
    $\text{нет, да, нет, да, нет, нет, нет, нет, нет, да}$
}

\tasknumber{2}%
\task{%
    Изобразите линии индукции магнитного поля вокруг постоянного магнита.

    \begin{tikzpicture}[x=1cm,y=1cm,thick]
        \draw (0, 0) rectangle (3, 0.6);
        \node [right] (right) at (0, 0.3) {S};
        \node [left] (left) at (3, 0.3) {N};
        \node [right] (right) at (0, 2) {};
        \node [right] (right) at (0, -2) {};
        \node [right] (right) at (-2, 0) {};
        \node [right] (right) at (5, 0) {};
    \end{tikzpicture}
}

\tasknumber{3}%
\task{%
    Опишите взаимодействие параллельных прямых токов.
    Нужны рисунки и необходимый минимум пояснений и терминов, трактат не нужен.
}
\solutionspace{100pt}

\tasknumber{4}%
\task{%
    Для постоянного магнита, изображённого на рис.
    1г)
    изобразите линии индукции магнитного поля
    и укажите, как соориентируются магнитные стрелки в точках A и B.
}
\solutionspace{80pt}

\tasknumber{5}%
\task{%
    Магнитная стрелка вблизи длинного прямолинейного проводника
    повёрнута в точке $B$ северным полюсом вверх (см.
    рис.
    2в).
    Сделайте рисунок, укажите направление протекания электрического тока,
    изобразите линии индукции магнитного поля.
}
% autogenerated
