\setdate{22~сентября~2021}
\setclass{11«БА»}

\addpersonalvariant{Михаил Бурмистров}

\tasknumber{1}%
\task{%
    Установите каждой букве в соответствие ровно одну цифру и запишите ответ (только цифры, без других символов).

    А) площадь контура, Б) сопротивление контура.

    1) $R$, 2) $U$, 3) $S$, 4) $\Phi$.
}
\answer{%
    $31$
}
\solutionspace{20pt}

\tasknumber{2}%
\task{%
    Установите каждой букве в соответствие ровно одну цифру и запишите ответ (только цифры, без других символов).

    А) ЭДС индукции, Б) индукция магнитного поля, В) магнитный поток.

    1) $\text{м}^2$, 2) Тл, 3) Вб, 4) В, 5) Кл.
}
\answer{%
    $423$
}
\solutionspace{20pt}

\tasknumber{3}%
\task{%
    Однородное магнитное поле пронизывает плоский контур площадью $800\,\text{см}^{2}$.
    Индукция магнитного поля равна $300\,\text{мТл}$.
    Чему равен магнитный поток через контур, если его плоскость
    расположена под углом $90\degrees$ к вектору магнитной индукции?
    Ответ выразите в милливеберах и округлите до целого, единицы измерения писать не нужно.
}
\answer{%
    $\alpha = 0\degrees, \Phi_B = BS\cos\alpha = 24{,}00\,\text{мВб} \to 24$
}
\solutionspace{100pt}

\tasknumber{4}%
\task{%
    Определите притягивается (А), не взаимодействует (Б) или отталкивается (В) металлическое кольцо к магниту,
    если выдвигать южным полюсом (см.
    рис).
}
\answer{%
    \text{А}
}

\tasknumber{5}%
\task{%
    Определите притягивается (А), не взаимодействует (Б) или отталкивается (В) кольцо из диэлектрика к магниту,
    если вдвигать магнит в кольцо северным полюсом (см.
    рис).
}
\answer{%
    \text{Б}
}

\tasknumber{6}%
\task{%
    Магнитный поток, пронизывающий замкнутый контур, равномерно изменяется от $95\,\text{мВб}$ до $20\,\text{мВб}$ за $1{,}1\,\text{c}$.
    Чему равна ЭДС в контуре? Ответ выразите в милливольтах и округлите до целого, единицы измерения писать не нужно.
}
\answer{%
    $\ele = 68{,}182\,\text{мВ} \to 68$
}
\solutionspace{60pt}

\tasknumber{7}%
\task{%
    Определите магнитный поток через контур,
    находящийся в однородном магнитном поле индукцией $700\,\text{мТл}$.
    Контур имеет форму прямоугольного треугольника с катетами $50\,\text{см}$ и $80\,\text{см}$.
    Угол между нормалью к плоскости контура и вектором индукции магнитного поля
    составляет $50\degrees$.
    Ответ выразите в милливеберах и округлите до целого, единицы измерения писать не нужно.
}
\answer{%
    $\alpha=50\degrees, \Phi_B = BS\cos\alpha = 89{,}99\,\text{мВб} \to 90$
}
\solutionspace{100pt}

\tasknumber{8}%
\task{%
    Какой средний индукционный ток возник в плоском контуре площадью $250\,\text{см}^{2}$
    и сопротивлением $2{,}5\,\text{Ом}$, если сперва он располагался перпендикулярно линиям индукции магнитного поля,
    а затем его за $0{,}6\,\text{мc}$ повернули, и теперь угол между плоскостью контура и индукцией магнитного поля
    равен $50\degrees$.
    Магнитное поле однородно, его индукция равна $50\,\text{мТл}$.
    Ответ выразите в микроамперах и округлите до целого, единицы измерения писать не нужно.
}
\answer{%
    \begin{align*}
    \alpha_1 &= 0\degrees, \alpha_2 = 40\degrees, \\
    \ele &= \abs{\frac{\Delta \Phi}{\Delta t}}, \\
    \eli &= \frac \ele R = \frac{\Delta \Phi}{R\Delta t}= \frac{ \abs{ B S \cos\alpha_2 - B S \cos\alpha_1 } }{R\Delta t}= \frac{ B S \abs{ \cos\alpha_2 -  \cos\alpha_1 } }{R\Delta t} =  \\
    &= \frac{ 50\,\text{мТл} \cdot 250\,\text{см}^{2} \abs{ \cos40\degrees -  \cos0\degrees } }{2{,}5\,\text{Ом} \cdot 0{,}6\,\text{мc}}\approx 194{,}963\,\text{мкА} \to 195
    \end{align*}
}

\variantsplitter

\addpersonalvariant{Ирина Ан}

\tasknumber{1}%
\task{%
    Установите каждой букве в соответствие ровно одну цифру и запишите ответ (только цифры, без других символов).

    А) площадь контура, Б) индукционый ток.

    1) $S$, 2) $\Phi$, 3) $l$, 4) $\eli$.
}
\answer{%
    $14$
}
\solutionspace{20pt}

\tasknumber{2}%
\task{%
    Установите каждой букве в соответствие ровно одну цифру и запишите ответ (только цифры, без других символов).

    А) индукционый ток, Б) ЭДС индукции, В) индукция магнитного поля.

    1) Гц, 2) А, 3) $\text{м}^2$, 4) Тл, 5) В.
}
\answer{%
    $254$
}
\solutionspace{20pt}

\tasknumber{3}%
\task{%
    Однородное магнитное поле пронизывает плоский контур площадью $600\,\text{см}^{2}$.
    Индукция магнитного поля равна $300\,\text{мТл}$.
    Чему равен магнитный поток через контур, если его плоскость
    расположена под углом $90\degrees$ к вектору магнитной индукции?
    Ответ выразите в милливеберах и округлите до целого, единицы измерения писать не нужно.
}
\answer{%
    $\alpha = 0\degrees, \Phi_B = BS\cos\alpha = 18{,}00\,\text{мВб} \to 18$
}
\solutionspace{100pt}

\tasknumber{4}%
\task{%
    Определите притягивается (А), не взаимодействует (Б) или отталкивается (В) металлическое кольцо к магниту,
    если вдвигать северным полюсом (см.
    рис).
}
\answer{%
    \text{В}
}

\tasknumber{5}%
\task{%
    Определите притягивается (А), не взаимодействует (Б) или отталкивается (В) кольцо из диэлектрика к магниту,
    если вдвигать магнит в кольцо северным полюсом (см.
    рис).
}
\answer{%
    \text{Б}
}

\tasknumber{6}%
\task{%
    Магнитный поток, пронизывающий замкнутый контур, равномерно изменяется от $65\,\text{мВб}$ до $20\,\text{мВб}$ за $1{,}5\,\text{c}$.
    Чему равна ЭДС в контуре? Ответ выразите в милливольтах и округлите до целого, единицы измерения писать не нужно.
}
\answer{%
    $\ele = 30{,}000\,\text{мВ} \to 30$
}
\solutionspace{60pt}

\tasknumber{7}%
\task{%
    Определите магнитный поток через контур,
    находящийся в однородном магнитном поле индукцией $700\,\text{мТл}$.
    Контур имеет форму прямоугольника со сторонами $40\,\text{см}$ и $80\,\text{см}$.
    Угол между плоскостью контура и вектором индукции магнитного поля
    составляет $50\degrees$.
    Ответ выразите в милливеберах и округлите до целого, единицы измерения писать не нужно.
}
\answer{%
    $\alpha=40\degrees, \Phi_B = BS\cos\alpha = 171{,}59\,\text{мВб} \to 172$
}
\solutionspace{100pt}

\tasknumber{8}%
\task{%
    Какой средний индукционный ток возник в плоском контуре площадью $150\,\text{см}^{2}$
    и сопротивлением $2{,}5\,\text{Ом}$, если сперва он располагался параллельной линиям индукции магнитного поля,
    а затем его за $0{,}6\,\text{мc}$ повернули, и теперь угол между плоскостью контура и индукцией магнитного поля
    равен $20\degrees$.
    Магнитное поле однородно, его индукция равна $50\,\text{мТл}$.
    Ответ выразите в микроамперах и округлите до целого, единицы измерения писать не нужно.
}
\answer{%
    \begin{align*}
    \alpha_1 &= 90\degrees, \alpha_2 = 70\degrees, \\
    \ele &= \abs{\frac{\Delta \Phi}{\Delta t}}, \\
    \eli &= \frac \ele R = \frac{\Delta \Phi}{R\Delta t}= \frac{ \abs{ B S \cos\alpha_2 - B S \cos\alpha_1 } }{R\Delta t}= \frac{ B S \abs{ \cos\alpha_2 -  \cos\alpha_1 } }{R\Delta t} =  \\
    &= \frac{ 50\,\text{мТл} \cdot 150\,\text{см}^{2} \abs{ \cos70\degrees -  \cos90\degrees } }{2{,}5\,\text{Ом} \cdot 0{,}6\,\text{мc}}\approx 171{,}010\,\text{мкА} \to 171
    \end{align*}
}

\variantsplitter

\addpersonalvariant{Софья Андрианова}

\tasknumber{1}%
\task{%
    Установите каждой букве в соответствие ровно одну цифру и запишите ответ (только цифры, без других символов).

    А) магнитный поток, Б) сопротивление контура.

    1) $\Phi$, 2) $R$, 3) $\vec n$, 4) $D$.
}
\answer{%
    $12$
}
\solutionspace{20pt}

\tasknumber{2}%
\task{%
    Установите каждой букве в соответствие ровно одну цифру и запишите ответ (только цифры, без других символов).

    А) площадь контура, Б) ЭДС индукции, В) индукционый ток.

    1) Гн, 2) $\text{м}^2$, 3) В, 4) Ом, 5) А.
}
\answer{%
    $235$
}
\solutionspace{20pt}

\tasknumber{3}%
\task{%
    Однородное магнитное поле пронизывает плоский контур площадью $600\,\text{см}^{2}$.
    Индукция магнитного поля равна $500\,\text{мТл}$.
    Чему равен магнитный поток через контур, если его плоскость
    расположена под углом $60\degrees$ к вектору магнитной индукции?
    Ответ выразите в милливеберах и округлите до целого, единицы измерения писать не нужно.
}
\answer{%
    $\alpha = 30\degrees, \Phi_B = BS\cos\alpha = 25{,}98\,\text{мВб} \to 26$
}
\solutionspace{100pt}

\tasknumber{4}%
\task{%
    Определите притягивается (А), не взаимодействует (Б) или отталкивается (В) металлическое кольцо к магниту,
    если выдвигать южным полюсом (см.
    рис).
}
\answer{%
    \text{А}
}

\tasknumber{5}%
\task{%
    Определите притягивается (А), не взаимодействует (Б) или отталкивается (В) кольцо из диэлектрика к магниту,
    если вдвигать магнит в кольцо южным полюсом (см.
    рис).
}
\answer{%
    \text{Б}
}

\tasknumber{6}%
\task{%
    Магнитный поток, пронизывающий замкнутый контур, равномерно изменяется от $95\,\text{мВб}$ до $80\,\text{мВб}$ за $1{,}1\,\text{c}$.
    Чему равна ЭДС в контуре? Ответ выразите в милливольтах и округлите до целого, единицы измерения писать не нужно.
}
\answer{%
    $\ele = 13{,}636\,\text{мВ} \to 14$
}
\solutionspace{60pt}

\tasknumber{7}%
\task{%
    Определите магнитный поток через контур,
    находящийся в однородном магнитном поле индукцией $500\,\text{мТл}$.
    Контур имеет форму прямоугольного треугольника с катетами $60\,\text{см}$ и $75\,\text{см}$.
    Угол между плоскостью контура и вектором индукции магнитного поля
    составляет $10\degrees$.
    Ответ выразите в милливеберах и округлите до целого, единицы измерения писать не нужно.
}
\answer{%
    $\alpha=80\degrees, \Phi_B = BS\cos\alpha = 19{,}54\,\text{мВб} \to 20$
}
\solutionspace{100pt}

\tasknumber{8}%
\task{%
    Какой средний индукционный ток возник в плоском контуре площадью $150\,\text{см}^{2}$
    и сопротивлением $3\,\text{Ом}$, если сперва он располагался перпендикулярно линиям индукции магнитного поля,
    а затем его за $0{,}8\,\text{мc}$ повернули, и теперь угол между плоскостью контура и индукцией магнитного поля
    равен $20\degrees$.
    Магнитное поле однородно, его индукция равна $70\,\text{мТл}$.
    Ответ выразите в микроамперах и округлите до целого, единицы измерения писать не нужно.
}
\answer{%
    \begin{align*}
    \alpha_1 &= 0\degrees, \alpha_2 = 70\degrees, \\
    \ele &= \abs{\frac{\Delta \Phi}{\Delta t}}, \\
    \eli &= \frac \ele R = \frac{\Delta \Phi}{R\Delta t}= \frac{ \abs{ B S \cos\alpha_2 - B S \cos\alpha_1 } }{R\Delta t}= \frac{ B S \abs{ \cos\alpha_2 -  \cos\alpha_1 } }{R\Delta t} =  \\
    &= \frac{ 70\,\text{мТл} \cdot 150\,\text{см}^{2} \abs{ \cos70\degrees -  \cos0\degrees } }{3\,\text{Ом} \cdot 0{,}8\,\text{мc}}\approx 287{,}866\,\text{мкА} \to 288
    \end{align*}
}

\variantsplitter

\addpersonalvariant{Владимир Артемчук}

\tasknumber{1}%
\task{%
    Установите каждой букве в соответствие ровно одну цифру и запишите ответ (только цифры, без других символов).

    А) индукция магнитного поля, Б) магнитный поток.

    1) $B$, 2) $\Phi$, 3) $S$, 4) $\vec n$.
}
\answer{%
    $12$
}
\solutionspace{20pt}

\tasknumber{2}%
\task{%
    Установите каждой букве в соответствие ровно одну цифру и запишите ответ (только цифры, без других символов).

    А) ЭДС индукции, Б) площадь контура, В) индукция магнитного поля.

    1) А, 2) В, 3) $\text{м}^2$, 4) Тл, 5) Ом.
}
\answer{%
    $234$
}
\solutionspace{20pt}

\tasknumber{3}%
\task{%
    Однородное магнитное поле пронизывает плоский контур площадью $800\,\text{см}^{2}$.
    Индукция магнитного поля равна $500\,\text{мТл}$.
    Чему равен магнитный поток через контур, если его плоскость
    расположена под углом $30\degrees$ к вектору магнитной индукции?
    Ответ выразите в милливеберах и округлите до целого, единицы измерения писать не нужно.
}
\answer{%
    $\alpha = 60\degrees, \Phi_B = BS\cos\alpha = 20{,}00\,\text{мВб} \to 20$
}
\solutionspace{100pt}

\tasknumber{4}%
\task{%
    Определите притягивается (А), не взаимодействует (Б) или отталкивается (В) металлическое кольцо к магниту,
    если выдвигать северным полюсом (см.
    рис).
}
\answer{%
    \text{А}
}

\tasknumber{5}%
\task{%
    Определите притягивается (А), не взаимодействует (Б) или отталкивается (В) кольцо из диэлектрика к магниту,
    если выдвигать магнит из кольца северным полюсом (см.
    рис).
}
\answer{%
    \text{Б}
}

\tasknumber{6}%
\task{%
    Магнитный поток, пронизывающий замкнутый контур, равномерно изменяется от $95\,\text{мВб}$ до $50\,\text{мВб}$ за $1{,}1\,\text{c}$.
    Чему равна ЭДС в контуре? Ответ выразите в милливольтах и округлите до целого, единицы измерения писать не нужно.
}
\answer{%
    $\ele = 40{,}909\,\text{мВ} \to 41$
}
\solutionspace{60pt}

\tasknumber{7}%
\task{%
    Определите магнитный поток через контур,
    находящийся в однородном магнитном поле индукцией $300\,\text{мТл}$.
    Контур имеет форму прямоугольника со сторонами $40\,\text{см}$ и $75\,\text{см}$.
    Угол между плоскостью контура и вектором индукции магнитного поля
    составляет $70\degrees$.
    Ответ выразите в милливеберах и округлите до целого, единицы измерения писать не нужно.
}
\answer{%
    $\alpha=20\degrees, \Phi_B = BS\cos\alpha = 84{,}57\,\text{мВб} \to 85$
}
\solutionspace{100pt}

\tasknumber{8}%
\task{%
    Какой средний индукционный ток возник в плоском контуре площадью $250\,\text{см}^{2}$
    и сопротивлением $2\,\text{Ом}$, если сперва он располагался перпендикулярно линиям индукции магнитного поля,
    а затем его за $0{,}8\,\text{мc}$ повернули, и теперь угол между нормалью к плоскости контура и индукцией магнитного поля
    равен $50\degrees$.
    Магнитное поле однородно, его индукция равна $80\,\text{мТл}$.
    Ответ выразите в микроамперах и округлите до целого, единицы измерения писать не нужно.
}
\answer{%
    \begin{align*}
    \alpha_1 &= 0\degrees, \alpha_2 = 50\degrees, \\
    \ele &= \abs{\frac{\Delta \Phi}{\Delta t}}, \\
    \eli &= \frac \ele R = \frac{\Delta \Phi}{R\Delta t}= \frac{ \abs{ B S \cos\alpha_2 - B S \cos\alpha_1 } }{R\Delta t}= \frac{ B S \abs{ \cos\alpha_2 -  \cos\alpha_1 } }{R\Delta t} =  \\
    &= \frac{ 80\,\text{мТл} \cdot 250\,\text{см}^{2} \abs{ \cos50\degrees -  \cos0\degrees } }{2\,\text{Ом} \cdot 0{,}8\,\text{мc}}\approx 446{,}515\,\text{мкА} \to 447
    \end{align*}
}

\variantsplitter

\addpersonalvariant{Софья Белянкина}

\tasknumber{1}%
\task{%
    Установите каждой букве в соответствие ровно одну цифру и запишите ответ (только цифры, без других символов).

    А) индукция магнитного поля, Б) вектор нормали к поверхности.

    1) $\ele$, 2) $B$, 3) $\vec n$, 4) $S$.
}
\answer{%
    $23$
}
\solutionspace{20pt}

\tasknumber{2}%
\task{%
    Установите каждой букве в соответствие ровно одну цифру и запишите ответ (только цифры, без других символов).

    А) магнитный поток, Б) площадь контура, В) ЭДС индукции.

    1) А, 2) В, 3) Вб, 4) $\text{м}^2$, 5) Ом.
}
\answer{%
    $342$
}
\solutionspace{20pt}

\tasknumber{3}%
\task{%
    Однородное магнитное поле пронизывает плоский контур площадью $400\,\text{см}^{2}$.
    Индукция магнитного поля равна $700\,\text{мТл}$.
    Чему равен магнитный поток через контур, если его плоскость
    расположена под углом $30\degrees$ к вектору магнитной индукции?
    Ответ выразите в милливеберах и округлите до целого, единицы измерения писать не нужно.
}
\answer{%
    $\alpha = 60\degrees, \Phi_B = BS\cos\alpha = 14{,}00\,\text{мВб} \to 14$
}
\solutionspace{100pt}

\tasknumber{4}%
\task{%
    Определите притягивается (А), не взаимодействует (Б) или отталкивается (В) металлическое кольцо к магниту,
    если выдвигать северным полюсом (см.
    рис).
}
\answer{%
    \text{А}
}

\tasknumber{5}%
\task{%
    Определите притягивается (А), не взаимодействует (Б) или отталкивается (В) кольцо из диэлектрика к магниту,
    если вдвигать магнит в кольцо северным полюсом (см.
    рис).
}
\answer{%
    \text{Б}
}

\tasknumber{6}%
\task{%
    Магнитный поток, пронизывающий замкнутый контур, равномерно изменяется от $95\,\text{мВб}$ до $50\,\text{мВб}$ за $1{,}5\,\text{c}$.
    Чему равна ЭДС в контуре? Ответ выразите в милливольтах и округлите до целого, единицы измерения писать не нужно.
}
\answer{%
    $\ele = 30{,}000\,\text{мВ} \to 30$
}
\solutionspace{60pt}

\tasknumber{7}%
\task{%
    Определите магнитный поток через контур,
    находящийся в однородном магнитном поле индукцией $700\,\text{мТл}$.
    Контур имеет форму прямоугольника со сторонами $40\,\text{см}$ и $45\,\text{см}$.
    Угол между нормалью к плоскости контура и вектором индукции магнитного поля
    составляет $20\degrees$.
    Ответ выразите в милливеберах и округлите до целого, единицы измерения писать не нужно.
}
\answer{%
    $\alpha=20\degrees, \Phi_B = BS\cos\alpha = 118{,}40\,\text{мВб} \to 118$
}
\solutionspace{100pt}

\tasknumber{8}%
\task{%
    Какой средний индукционный ток возник в плоском контуре площадью $150\,\text{см}^{2}$
    и сопротивлением $2{,}5\,\text{Ом}$, если сперва он располагался параллельной линиям индукции магнитного поля,
    а затем его за $0{,}5\,\text{мc}$ повернули, и теперь угол между плоскостью контура и индукцией магнитного поля
    равен $50\degrees$.
    Магнитное поле однородно, его индукция равна $50\,\text{мТл}$.
    Ответ выразите в микроамперах и округлите до целого, единицы измерения писать не нужно.
}
\answer{%
    \begin{align*}
    \alpha_1 &= 90\degrees, \alpha_2 = 40\degrees, \\
    \ele &= \abs{\frac{\Delta \Phi}{\Delta t}}, \\
    \eli &= \frac \ele R = \frac{\Delta \Phi}{R\Delta t}= \frac{ \abs{ B S \cos\alpha_2 - B S \cos\alpha_1 } }{R\Delta t}= \frac{ B S \abs{ \cos\alpha_2 -  \cos\alpha_1 } }{R\Delta t} =  \\
    &= \frac{ 50\,\text{мТл} \cdot 150\,\text{см}^{2} \abs{ \cos40\degrees -  \cos90\degrees } }{2{,}5\,\text{Ом} \cdot 0{,}5\,\text{мc}}\approx 459{,}627\,\text{мкА} \to 460
    \end{align*}
}

\variantsplitter

\addpersonalvariant{Варвара Егиазарян}

\tasknumber{1}%
\task{%
    Установите каждой букве в соответствие ровно одну цифру и запишите ответ (только цифры, без других символов).

    А) индукция магнитного поля, Б) магнитный поток.

    1) $\ele$, 2) $v$, 3) $\Phi$, 4) $B$.
}
\answer{%
    $43$
}
\solutionspace{20pt}

\tasknumber{2}%
\task{%
    Установите каждой букве в соответствие ровно одну цифру и запишите ответ (только цифры, без других символов).

    А) индукция магнитного поля, Б) ЭДС индукции, В) магнитный поток.

    1) $\text{м}^2$, 2) Кл, 3) Вб, 4) В, 5) Тл.
}
\answer{%
    $543$
}
\solutionspace{20pt}

\tasknumber{3}%
\task{%
    Однородное магнитное поле пронизывает плоский контур площадью $600\,\text{см}^{2}$.
    Индукция магнитного поля равна $300\,\text{мТл}$.
    Чему равен магнитный поток через контур, если его плоскость
    расположена под углом $0\degrees$ к вектору магнитной индукции?
    Ответ выразите в милливеберах и округлите до целого, единицы измерения писать не нужно.
}
\answer{%
    $\alpha = 90\degrees, \Phi_B = BS\cos\alpha = 0\,\text{мВб} \to 0$
}
\solutionspace{100pt}

\tasknumber{4}%
\task{%
    Определите притягивается (А), не взаимодействует (Б) или отталкивается (В) металлическое кольцо к магниту,
    если выдвигать южным полюсом (см.
    рис).
}
\answer{%
    \text{А}
}

\tasknumber{5}%
\task{%
    Определите притягивается (А), не взаимодействует (Б) или отталкивается (В) кольцо из диэлектрика к магниту,
    если выдвигать магнит из кольца южным полюсом (см.
    рис).
}
\answer{%
    \text{Б}
}

\tasknumber{6}%
\task{%
    Магнитный поток, пронизывающий замкнутый контур, равномерно изменяется от $95\,\text{мВб}$ до $80\,\text{мВб}$ за $1{,}3\,\text{c}$.
    Чему равна ЭДС в контуре? Ответ выразите в милливольтах и округлите до целого, единицы измерения писать не нужно.
}
\answer{%
    $\ele = 11{,}538\,\text{мВ} \to 12$
}
\solutionspace{60pt}

\tasknumber{7}%
\task{%
    Определите магнитный поток через контур,
    находящийся в однородном магнитном поле индукцией $300\,\text{мТл}$.
    Контур имеет форму прямоугольного треугольника с катетами $40\,\text{см}$ и $80\,\text{см}$.
    Угол между нормалью к плоскости контура и вектором индукции магнитного поля
    составляет $40\degrees$.
    Ответ выразите в милливеберах и округлите до целого, единицы измерения писать не нужно.
}
\answer{%
    $\alpha=40\degrees, \Phi_B = BS\cos\alpha = 36{,}77\,\text{мВб} \to 37$
}
\solutionspace{100pt}

\tasknumber{8}%
\task{%
    Какой средний индукционный ток возник в плоском контуре площадью $150\,\text{см}^{2}$
    и сопротивлением $2\,\text{Ом}$, если сперва он располагался перпендикулярно линиям индукции магнитного поля,
    а затем его за $0{,}2\,\text{мc}$ повернули, и теперь угол между нормалью к плоскости контура и индукцией магнитного поля
    равен $20\degrees$.
    Магнитное поле однородно, его индукция равна $50\,\text{мТл}$.
    Ответ выразите в микроамперах и округлите до целого, единицы измерения писать не нужно.
}
\answer{%
    \begin{align*}
    \alpha_1 &= 0\degrees, \alpha_2 = 20\degrees, \\
    \ele &= \abs{\frac{\Delta \Phi}{\Delta t}}, \\
    \eli &= \frac \ele R = \frac{\Delta \Phi}{R\Delta t}= \frac{ \abs{ B S \cos\alpha_2 - B S \cos\alpha_1 } }{R\Delta t}= \frac{ B S \abs{ \cos\alpha_2 -  \cos\alpha_1 } }{R\Delta t} =  \\
    &= \frac{ 50\,\text{мТл} \cdot 150\,\text{см}^{2} \abs{ \cos20\degrees -  \cos0\degrees } }{2\,\text{Ом} \cdot 0{,}2\,\text{мc}}\approx 113{,}076\,\text{мкА} \to 113
    \end{align*}
}

\variantsplitter

\addpersonalvariant{Владислав Емелин}

\tasknumber{1}%
\task{%
    Установите каждой букве в соответствие ровно одну цифру и запишите ответ (только цифры, без других символов).

    А) магнитный поток, Б) ЭДС индукции.

    1) $\Phi$, 2) $\ele$, 3) $U$, 4) $B$.
}
\answer{%
    $12$
}
\solutionspace{20pt}

\tasknumber{2}%
\task{%
    Установите каждой букве в соответствие ровно одну цифру и запишите ответ (только цифры, без других символов).

    А) индукционый ток, Б) площадь контура, В) индукция магнитного поля.

    1) Гц, 2) А, 3) $\text{м}^2$, 4) Вб, 5) Тл.
}
\answer{%
    $235$
}
\solutionspace{20pt}

\tasknumber{3}%
\task{%
    Однородное магнитное поле пронизывает плоский контур площадью $400\,\text{см}^{2}$.
    Индукция магнитного поля равна $500\,\text{мТл}$.
    Чему равен магнитный поток через контур, если его плоскость
    расположена под углом $60\degrees$ к вектору магнитной индукции?
    Ответ выразите в милливеберах и округлите до целого, единицы измерения писать не нужно.
}
\answer{%
    $\alpha = 30\degrees, \Phi_B = BS\cos\alpha = 17{,}32\,\text{мВб} \to 17$
}
\solutionspace{100pt}

\tasknumber{4}%
\task{%
    Определите притягивается (А), не взаимодействует (Б) или отталкивается (В) металлическое кольцо к магниту,
    если вдвигать северным полюсом (см.
    рис).
}
\answer{%
    \text{В}
}

\tasknumber{5}%
\task{%
    Определите притягивается (А), не взаимодействует (Б) или отталкивается (В) кольцо из диэлектрика к магниту,
    если выдвигать магнит из кольца северным полюсом (см.
    рис).
}
\answer{%
    \text{Б}
}

\tasknumber{6}%
\task{%
    Магнитный поток, пронизывающий замкнутый контур, равномерно изменяется от $95\,\text{мВб}$ до $20\,\text{мВб}$ за $1{,}5\,\text{c}$.
    Чему равна ЭДС в контуре? Ответ выразите в милливольтах и округлите до целого, единицы измерения писать не нужно.
}
\answer{%
    $\ele = 50{,}000\,\text{мВ} \to 50$
}
\solutionspace{60pt}

\tasknumber{7}%
\task{%
    Определите магнитный поток через контур,
    находящийся в однородном магнитном поле индукцией $300\,\text{мТл}$.
    Контур имеет форму прямоугольника со сторонами $40\,\text{см}$ и $80\,\text{см}$.
    Угол между нормалью к плоскости контура и вектором индукции магнитного поля
    составляет $20\degrees$.
    Ответ выразите в милливеберах и округлите до целого, единицы измерения писать не нужно.
}
\answer{%
    $\alpha=20\degrees, \Phi_B = BS\cos\alpha = 90{,}21\,\text{мВб} \to 90$
}
\solutionspace{100pt}

\tasknumber{8}%
\task{%
    Какой средний индукционный ток возник в плоском контуре площадью $120\,\text{см}^{2}$
    и сопротивлением $2{,}5\,\text{Ом}$, если сперва он располагался параллельной линиям индукции магнитного поля,
    а затем его за $0{,}8\,\text{мc}$ повернули, и теперь угол между плоскостью контура и индукцией магнитного поля
    равен $70\degrees$.
    Магнитное поле однородно, его индукция равна $70\,\text{мТл}$.
    Ответ выразите в микроамперах и округлите до целого, единицы измерения писать не нужно.
}
\answer{%
    \begin{align*}
    \alpha_1 &= 90\degrees, \alpha_2 = 20\degrees, \\
    \ele &= \abs{\frac{\Delta \Phi}{\Delta t}}, \\
    \eli &= \frac \ele R = \frac{\Delta \Phi}{R\Delta t}= \frac{ \abs{ B S \cos\alpha_2 - B S \cos\alpha_1 } }{R\Delta t}= \frac{ B S \abs{ \cos\alpha_2 -  \cos\alpha_1 } }{R\Delta t} =  \\
    &= \frac{ 70\,\text{мТл} \cdot 120\,\text{см}^{2} \abs{ \cos20\degrees -  \cos90\degrees } }{2{,}5\,\text{Ом} \cdot 0{,}8\,\text{мc}}\approx 394{,}671\,\text{мкА} \to 395
    \end{align*}
}

\variantsplitter

\addpersonalvariant{Артём Жичин}

\tasknumber{1}%
\task{%
    Установите каждой букве в соответствие ровно одну цифру и запишите ответ (только цифры, без других символов).

    А) сопротивление контура, Б) вектор нормали к поверхности.

    1) $D$, 2) $R$, 3) $\eli$, 4) $\vec n$.
}
\answer{%
    $24$
}
\solutionspace{20pt}

\tasknumber{2}%
\task{%
    Установите каждой букве в соответствие ровно одну цифру и запишите ответ (только цифры, без других символов).

    А) площадь контура, Б) магнитный поток, В) индукция магнитного поля.

    1) Кл, 2) А, 3) $\text{м}^2$, 4) Тл, 5) Вб.
}
\answer{%
    $354$
}
\solutionspace{20pt}

\tasknumber{3}%
\task{%
    Однородное магнитное поле пронизывает плоский контур площадью $400\,\text{см}^{2}$.
    Индукция магнитного поля равна $300\,\text{мТл}$.
    Чему равен магнитный поток через контур, если его плоскость
    расположена под углом $0\degrees$ к вектору магнитной индукции?
    Ответ выразите в милливеберах и округлите до целого, единицы измерения писать не нужно.
}
\answer{%
    $\alpha = 90\degrees, \Phi_B = BS\cos\alpha = 0\,\text{мВб} \to 0$
}
\solutionspace{100pt}

\tasknumber{4}%
\task{%
    Определите притягивается (А), не взаимодействует (Б) или отталкивается (В) металлическое кольцо к магниту,
    если вдвигать северным полюсом (см.
    рис).
}
\answer{%
    \text{В}
}

\tasknumber{5}%
\task{%
    Определите притягивается (А), не взаимодействует (Б) или отталкивается (В) кольцо из диэлектрика к магниту,
    если вдвигать магнит в кольцо северным полюсом (см.
    рис).
}
\answer{%
    \text{Б}
}

\tasknumber{6}%
\task{%
    Магнитный поток, пронизывающий замкнутый контур, равномерно изменяется от $65\,\text{мВб}$ до $50\,\text{мВб}$ за $1{,}1\,\text{c}$.
    Чему равна ЭДС в контуре? Ответ выразите в милливольтах и округлите до целого, единицы измерения писать не нужно.
}
\answer{%
    $\ele = 13{,}636\,\text{мВ} \to 14$
}
\solutionspace{60pt}

\tasknumber{7}%
\task{%
    Определите магнитный поток через контур,
    находящийся в однородном магнитном поле индукцией $300\,\text{мТл}$.
    Контур имеет форму прямоугольника со сторонами $40\,\text{см}$ и $75\,\text{см}$.
    Угол между плоскостью контура и вектором индукции магнитного поля
    составляет $80\degrees$.
    Ответ выразите в милливеберах и округлите до целого, единицы измерения писать не нужно.
}
\answer{%
    $\alpha=10\degrees, \Phi_B = BS\cos\alpha = 88{,}63\,\text{мВб} \to 89$
}
\solutionspace{100pt}

\tasknumber{8}%
\task{%
    Какой средний индукционный ток возник в плоском контуре площадью $250\,\text{см}^{2}$
    и сопротивлением $2\,\text{Ом}$, если сперва он располагался параллельной линиям индукции магнитного поля,
    а затем его за $0{,}8\,\text{мc}$ повернули, и теперь угол между нормалью к плоскости контура и индукцией магнитного поля
    равен $40\degrees$.
    Магнитное поле однородно, его индукция равна $80\,\text{мТл}$.
    Ответ выразите в микроамперах и округлите до целого, единицы измерения писать не нужно.
}
\answer{%
    \begin{align*}
    \alpha_1 &= 90\degrees, \alpha_2 = 40\degrees, \\
    \ele &= \abs{\frac{\Delta \Phi}{\Delta t}}, \\
    \eli &= \frac \ele R = \frac{\Delta \Phi}{R\Delta t}= \frac{ \abs{ B S \cos\alpha_2 - B S \cos\alpha_1 } }{R\Delta t}= \frac{ B S \abs{ \cos\alpha_2 -  \cos\alpha_1 } }{R\Delta t} =  \\
    &= \frac{ 80\,\text{мТл} \cdot 250\,\text{см}^{2} \abs{ \cos40\degrees -  \cos90\degrees } }{2\,\text{Ом} \cdot 0{,}8\,\text{мc}}\approx 957{,}556\,\text{мкА} \to 958
    \end{align*}
}

\variantsplitter

\addpersonalvariant{Дарья Кошман}

\tasknumber{1}%
\task{%
    Установите каждой букве в соответствие ровно одну цифру и запишите ответ (только цифры, без других символов).

    А) индукция магнитного поля, Б) вектор нормали к поверхности.

    1) $R$, 2) $B$, 3) $\vec n$, 4) $D$.
}
\answer{%
    $23$
}
\solutionspace{20pt}

\tasknumber{2}%
\task{%
    Установите каждой букве в соответствие ровно одну цифру и запишите ответ (только цифры, без других символов).

    А) индукция магнитного поля, Б) магнитный поток, В) индукционый ток.

    1) А, 2) В, 3) Тл, 4) Вб, 5) Ом.
}
\answer{%
    $341$
}
\solutionspace{20pt}

\tasknumber{3}%
\task{%
    Однородное магнитное поле пронизывает плоский контур площадью $200\,\text{см}^{2}$.
    Индукция магнитного поля равна $300\,\text{мТл}$.
    Чему равен магнитный поток через контур, если его плоскость
    расположена под углом $90\degrees$ к вектору магнитной индукции?
    Ответ выразите в милливеберах и округлите до целого, единицы измерения писать не нужно.
}
\answer{%
    $\alpha = 0\degrees, \Phi_B = BS\cos\alpha = 6{,}00\,\text{мВб} \to 6$
}
\solutionspace{100pt}

\tasknumber{4}%
\task{%
    Определите притягивается (А), не взаимодействует (Б) или отталкивается (В) металлическое кольцо к магниту,
    если вдвигать южным полюсом (см.
    рис).
}
\answer{%
    \text{В}
}

\tasknumber{5}%
\task{%
    Определите притягивается (А), не взаимодействует (Б) или отталкивается (В) кольцо из диэлектрика к магниту,
    если выдвигать магнит из кольца северным полюсом (см.
    рис).
}
\answer{%
    \text{Б}
}

\tasknumber{6}%
\task{%
    Магнитный поток, пронизывающий замкнутый контур, равномерно изменяется от $35\,\text{мВб}$ до $110\,\text{мВб}$ за $1{,}1\,\text{c}$.
    Чему равна ЭДС в контуре? Ответ выразите в милливольтах и округлите до целого, единицы измерения писать не нужно.
}
\answer{%
    $\ele = 68{,}182\,\text{мВ} \to 68$
}
\solutionspace{60pt}

\tasknumber{7}%
\task{%
    Определите магнитный поток через контур,
    находящийся в однородном магнитном поле индукцией $700\,\text{мТл}$.
    Контур имеет форму прямоугольного треугольника с катетами $40\,\text{см}$ и $45\,\text{см}$.
    Угол между плоскостью контура и вектором индукции магнитного поля
    составляет $40\degrees$.
    Ответ выразите в милливеберах и округлите до целого, единицы измерения писать не нужно.
}
\answer{%
    $\alpha=50\degrees, \Phi_B = BS\cos\alpha = 40{,}50\,\text{мВб} \to 40$
}
\solutionspace{100pt}

\tasknumber{8}%
\task{%
    Какой средний индукционный ток возник в плоском контуре площадью $250\,\text{см}^{2}$
    и сопротивлением $3\,\text{Ом}$, если сперва он располагался параллельной линиям индукции магнитного поля,
    а затем его за $0{,}6\,\text{мc}$ повернули, и теперь угол между плоскостью контура и индукцией магнитного поля
    равен $20\degrees$.
    Магнитное поле однородно, его индукция равна $50\,\text{мТл}$.
    Ответ выразите в микроамперах и округлите до целого, единицы измерения писать не нужно.
}
\answer{%
    \begin{align*}
    \alpha_1 &= 90\degrees, \alpha_2 = 70\degrees, \\
    \ele &= \abs{\frac{\Delta \Phi}{\Delta t}}, \\
    \eli &= \frac \ele R = \frac{\Delta \Phi}{R\Delta t}= \frac{ \abs{ B S \cos\alpha_2 - B S \cos\alpha_1 } }{R\Delta t}= \frac{ B S \abs{ \cos\alpha_2 -  \cos\alpha_1 } }{R\Delta t} =  \\
    &= \frac{ 50\,\text{мТл} \cdot 250\,\text{см}^{2} \abs{ \cos70\degrees -  \cos90\degrees } }{3\,\text{Ом} \cdot 0{,}6\,\text{мc}}\approx 237{,}514\,\text{мкА} \to 238
    \end{align*}
}

\variantsplitter

\addpersonalvariant{Анна Кузьмичёва}

\tasknumber{1}%
\task{%
    Установите каждой букве в соответствие ровно одну цифру и запишите ответ (только цифры, без других символов).

    А) ЭДС индукции, Б) индукция магнитного поля.

    1) $\ele$, 2) $R$, 3) $\Phi$, 4) $B$.
}
\answer{%
    $14$
}
\solutionspace{20pt}

\tasknumber{2}%
\task{%
    Установите каждой букве в соответствие ровно одну цифру и запишите ответ (только цифры, без других символов).

    А) ЭДС индукции, Б) магнитный поток, В) площадь контура.

    1) Вб, 2) В, 3) Тл, 4) Гн, 5) $\text{м}^2$.
}
\answer{%
    $215$
}
\solutionspace{20pt}

\tasknumber{3}%
\task{%
    Однородное магнитное поле пронизывает плоский контур площадью $200\,\text{см}^{2}$.
    Индукция магнитного поля равна $700\,\text{мТл}$.
    Чему равен магнитный поток через контур, если его плоскость
    расположена под углом $60\degrees$ к вектору магнитной индукции?
    Ответ выразите в милливеберах и округлите до целого, единицы измерения писать не нужно.
}
\answer{%
    $\alpha = 30\degrees, \Phi_B = BS\cos\alpha = 12{,}12\,\text{мВб} \to 12$
}
\solutionspace{100pt}

\tasknumber{4}%
\task{%
    Определите притягивается (А), не взаимодействует (Б) или отталкивается (В) металлическое кольцо к магниту,
    если выдвигать южным полюсом (см.
    рис).
}
\answer{%
    \text{А}
}

\tasknumber{5}%
\task{%
    Определите притягивается (А), не взаимодействует (Б) или отталкивается (В) кольцо из диэлектрика к магниту,
    если выдвигать магнит из кольца южным полюсом (см.
    рис).
}
\answer{%
    \text{Б}
}

\tasknumber{6}%
\task{%
    Магнитный поток, пронизывающий замкнутый контур, равномерно изменяется от $35\,\text{мВб}$ до $50\,\text{мВб}$ за $1{,}1\,\text{c}$.
    Чему равна ЭДС в контуре? Ответ выразите в милливольтах и округлите до целого, единицы измерения писать не нужно.
}
\answer{%
    $\ele = 13{,}636\,\text{мВ} \to 14$
}
\solutionspace{60pt}

\tasknumber{7}%
\task{%
    Определите магнитный поток через контур,
    находящийся в однородном магнитном поле индукцией $300\,\text{мТл}$.
    Контур имеет форму прямоугольника со сторонами $40\,\text{см}$ и $80\,\text{см}$.
    Угол между нормалью к плоскости контура и вектором индукции магнитного поля
    составляет $80\degrees$.
    Ответ выразите в милливеберах и округлите до целого, единицы измерения писать не нужно.
}
\answer{%
    $\alpha=80\degrees, \Phi_B = BS\cos\alpha = 16{,}67\,\text{мВб} \to 17$
}
\solutionspace{100pt}

\tasknumber{8}%
\task{%
    Какой средний индукционный ток возник в плоском контуре площадью $150\,\text{см}^{2}$
    и сопротивлением $1{,}5\,\text{Ом}$, если сперва он располагался параллельной линиям индукции магнитного поля,
    а затем его за $0{,}8\,\text{мc}$ повернули, и теперь угол между плоскостью контура и индукцией магнитного поля
    равен $50\degrees$.
    Магнитное поле однородно, его индукция равна $50\,\text{мТл}$.
    Ответ выразите в микроамперах и округлите до целого, единицы измерения писать не нужно.
}
\answer{%
    \begin{align*}
    \alpha_1 &= 90\degrees, \alpha_2 = 40\degrees, \\
    \ele &= \abs{\frac{\Delta \Phi}{\Delta t}}, \\
    \eli &= \frac \ele R = \frac{\Delta \Phi}{R\Delta t}= \frac{ \abs{ B S \cos\alpha_2 - B S \cos\alpha_1 } }{R\Delta t}= \frac{ B S \abs{ \cos\alpha_2 -  \cos\alpha_1 } }{R\Delta t} =  \\
    &= \frac{ 50\,\text{мТл} \cdot 150\,\text{см}^{2} \abs{ \cos40\degrees -  \cos90\degrees } }{1{,}5\,\text{Ом} \cdot 0{,}8\,\text{мc}}\approx 478{,}778\,\text{мкА} \to 479
    \end{align*}
}

\variantsplitter

\addpersonalvariant{Алёна Куприянова}

\tasknumber{1}%
\task{%
    Установите каждой букве в соответствие ровно одну цифру и запишите ответ (только цифры, без других символов).

    А) магнитный поток, Б) индукционый ток.

    1) $\eli$, 2) $R$, 3) $S$, 4) $\Phi$.
}
\answer{%
    $41$
}
\solutionspace{20pt}

\tasknumber{2}%
\task{%
    Установите каждой букве в соответствие ровно одну цифру и запишите ответ (только цифры, без других символов).

    А) ЭДС индукции, Б) магнитный поток, В) индукционый ток.

    1) А, 2) Вб, 3) Тл, 4) Гн, 5) В.
}
\answer{%
    $521$
}
\solutionspace{20pt}

\tasknumber{3}%
\task{%
    Однородное магнитное поле пронизывает плоский контур площадью $600\,\text{см}^{2}$.
    Индукция магнитного поля равна $300\,\text{мТл}$.
    Чему равен магнитный поток через контур, если его плоскость
    расположена под углом $30\degrees$ к вектору магнитной индукции?
    Ответ выразите в милливеберах и округлите до целого, единицы измерения писать не нужно.
}
\answer{%
    $\alpha = 60\degrees, \Phi_B = BS\cos\alpha = 9{,}00\,\text{мВб} \to 9$
}
\solutionspace{100pt}

\tasknumber{4}%
\task{%
    Определите притягивается (А), не взаимодействует (Б) или отталкивается (В) металлическое кольцо к магниту,
    если выдвигать южным полюсом (см.
    рис).
}
\answer{%
    \text{А}
}

\tasknumber{5}%
\task{%
    Определите притягивается (А), не взаимодействует (Б) или отталкивается (В) кольцо из диэлектрика к магниту,
    если вдвигать магнит в кольцо южным полюсом (см.
    рис).
}
\answer{%
    \text{Б}
}

\tasknumber{6}%
\task{%
    Магнитный поток, пронизывающий замкнутый контур, равномерно изменяется от $65\,\text{мВб}$ до $110\,\text{мВб}$ за $1{,}5\,\text{c}$.
    Чему равна ЭДС в контуре? Ответ выразите в милливольтах и округлите до целого, единицы измерения писать не нужно.
}
\answer{%
    $\ele = 30{,}000\,\text{мВ} \to 30$
}
\solutionspace{60pt}

\tasknumber{7}%
\task{%
    Определите магнитный поток через контур,
    находящийся в однородном магнитном поле индукцией $700\,\text{мТл}$.
    Контур имеет форму прямоугольного треугольника с катетами $40\,\text{см}$ и $45\,\text{см}$.
    Угол между нормалью к плоскости контура и вектором индукции магнитного поля
    составляет $10\degrees$.
    Ответ выразите в милливеберах и округлите до целого, единицы измерения писать не нужно.
}
\answer{%
    $\alpha=10\degrees, \Phi_B = BS\cos\alpha = 62{,}04\,\text{мВб} \to 62$
}
\solutionspace{100pt}

\tasknumber{8}%
\task{%
    Какой средний индукционный ток возник в плоском контуре площадью $120\,\text{см}^{2}$
    и сопротивлением $2\,\text{Ом}$, если сперва он располагался перпендикулярно линиям индукции магнитного поля,
    а затем его за $0{,}8\,\text{мc}$ повернули, и теперь угол между плоскостью контура и индукцией магнитного поля
    равен $10\degrees$.
    Магнитное поле однородно, его индукция равна $50\,\text{мТл}$.
    Ответ выразите в микроамперах и округлите до целого, единицы измерения писать не нужно.
}
\answer{%
    \begin{align*}
    \alpha_1 &= 0\degrees, \alpha_2 = 80\degrees, \\
    \ele &= \abs{\frac{\Delta \Phi}{\Delta t}}, \\
    \eli &= \frac \ele R = \frac{\Delta \Phi}{R\Delta t}= \frac{ \abs{ B S \cos\alpha_2 - B S \cos\alpha_1 } }{R\Delta t}= \frac{ B S \abs{ \cos\alpha_2 -  \cos\alpha_1 } }{R\Delta t} =  \\
    &= \frac{ 50\,\text{мТл} \cdot 120\,\text{см}^{2} \abs{ \cos80\degrees -  \cos0\degrees } }{2\,\text{Ом} \cdot 0{,}8\,\text{мc}}\approx 309{,}882\,\text{мкА} \to 310
    \end{align*}
}

\variantsplitter

\addpersonalvariant{Ярослав Лавровский}

\tasknumber{1}%
\task{%
    Установите каждой букве в соответствие ровно одну цифру и запишите ответ (только цифры, без других символов).

    А) сопротивление контура, Б) индукция магнитного поля.

    1) $l$, 2) $R$, 3) $B$, 4) $U$.
}
\answer{%
    $23$
}
\solutionspace{20pt}

\tasknumber{2}%
\task{%
    Установите каждой букве в соответствие ровно одну цифру и запишите ответ (только цифры, без других символов).

    А) индукция магнитного поля, Б) площадь контура, В) магнитный поток.

    1) В, 2) А, 3) Тл, 4) $\text{м}^2$, 5) Вб.
}
\answer{%
    $345$
}
\solutionspace{20pt}

\tasknumber{3}%
\task{%
    Однородное магнитное поле пронизывает плоский контур площадью $600\,\text{см}^{2}$.
    Индукция магнитного поля равна $700\,\text{мТл}$.
    Чему равен магнитный поток через контур, если его плоскость
    расположена под углом $0\degrees$ к вектору магнитной индукции?
    Ответ выразите в милливеберах и округлите до целого, единицы измерения писать не нужно.
}
\answer{%
    $\alpha = 90\degrees, \Phi_B = BS\cos\alpha = 0\,\text{мВб} \to 0$
}
\solutionspace{100pt}

\tasknumber{4}%
\task{%
    Определите притягивается (А), не взаимодействует (Б) или отталкивается (В) металлическое кольцо к магниту,
    если выдвигать южным полюсом (см.
    рис).
}
\answer{%
    \text{А}
}

\tasknumber{5}%
\task{%
    Определите притягивается (А), не взаимодействует (Б) или отталкивается (В) кольцо из диэлектрика к магниту,
    если выдвигать магнит из кольца южным полюсом (см.
    рис).
}
\answer{%
    \text{Б}
}

\tasknumber{6}%
\task{%
    Магнитный поток, пронизывающий замкнутый контур, равномерно изменяется от $95\,\text{мВб}$ до $80\,\text{мВб}$ за $1{,}5\,\text{c}$.
    Чему равна ЭДС в контуре? Ответ выразите в милливольтах и округлите до целого, единицы измерения писать не нужно.
}
\answer{%
    $\ele = 10{,}000\,\text{мВ} \to 10$
}
\solutionspace{60pt}

\tasknumber{7}%
\task{%
    Определите магнитный поток через контур,
    находящийся в однородном магнитном поле индукцией $300\,\text{мТл}$.
    Контур имеет форму прямоугольного треугольника с катетами $40\,\text{см}$ и $75\,\text{см}$.
    Угол между нормалью к плоскости контура и вектором индукции магнитного поля
    составляет $50\degrees$.
    Ответ выразите в милливеберах и округлите до целого, единицы измерения писать не нужно.
}
\answer{%
    $\alpha=50\degrees, \Phi_B = BS\cos\alpha = 28{,}93\,\text{мВб} \to 29$
}
\solutionspace{100pt}

\tasknumber{8}%
\task{%
    Какой средний индукционный ток возник в плоском контуре площадью $150\,\text{см}^{2}$
    и сопротивлением $2\,\text{Ом}$, если сперва он располагался перпендикулярно линиям индукции магнитного поля,
    а затем его за $0{,}6\,\text{мc}$ повернули, и теперь угол между плоскостью контура и индукцией магнитного поля
    равен $20\degrees$.
    Магнитное поле однородно, его индукция равна $50\,\text{мТл}$.
    Ответ выразите в микроамперах и округлите до целого, единицы измерения писать не нужно.
}
\answer{%
    \begin{align*}
    \alpha_1 &= 0\degrees, \alpha_2 = 70\degrees, \\
    \ele &= \abs{\frac{\Delta \Phi}{\Delta t}}, \\
    \eli &= \frac \ele R = \frac{\Delta \Phi}{R\Delta t}= \frac{ \abs{ B S \cos\alpha_2 - B S \cos\alpha_1 } }{R\Delta t}= \frac{ B S \abs{ \cos\alpha_2 -  \cos\alpha_1 } }{R\Delta t} =  \\
    &= \frac{ 50\,\text{мТл} \cdot 150\,\text{см}^{2} \abs{ \cos70\degrees -  \cos0\degrees } }{2\,\text{Ом} \cdot 0{,}6\,\text{мc}}\approx 411{,}237\,\text{мкА} \to 411
    \end{align*}
}

\variantsplitter

\addpersonalvariant{Анастасия Ламанова}

\tasknumber{1}%
\task{%
    Установите каждой букве в соответствие ровно одну цифру и запишите ответ (только цифры, без других символов).

    А) индукция магнитного поля, Б) магнитный поток.

    1) $U$, 2) $\Phi$, 3) $B$, 4) $\vec n$.
}
\answer{%
    $32$
}
\solutionspace{20pt}

\tasknumber{2}%
\task{%
    Установите каждой букве в соответствие ровно одну цифру и запишите ответ (только цифры, без других символов).

    А) индукция магнитного поля, Б) индукционый ток, В) площадь контура.

    1) Кл, 2) $\text{м}^2$, 3) А, 4) Тл, 5) Гн.
}
\answer{%
    $432$
}
\solutionspace{20pt}

\tasknumber{3}%
\task{%
    Однородное магнитное поле пронизывает плоский контур площадью $600\,\text{см}^{2}$.
    Индукция магнитного поля равна $700\,\text{мТл}$.
    Чему равен магнитный поток через контур, если его плоскость
    расположена под углом $60\degrees$ к вектору магнитной индукции?
    Ответ выразите в милливеберах и округлите до целого, единицы измерения писать не нужно.
}
\answer{%
    $\alpha = 30\degrees, \Phi_B = BS\cos\alpha = 36{,}37\,\text{мВб} \to 36$
}
\solutionspace{100pt}

\tasknumber{4}%
\task{%
    Определите притягивается (А), не взаимодействует (Б) или отталкивается (В) металлическое кольцо к магниту,
    если выдвигать южным полюсом (см.
    рис).
}
\answer{%
    \text{А}
}

\tasknumber{5}%
\task{%
    Определите притягивается (А), не взаимодействует (Б) или отталкивается (В) кольцо из диэлектрика к магниту,
    если выдвигать магнит из кольца южным полюсом (см.
    рис).
}
\answer{%
    \text{Б}
}

\tasknumber{6}%
\task{%
    Магнитный поток, пронизывающий замкнутый контур, равномерно изменяется от $35\,\text{мВб}$ до $50\,\text{мВб}$ за $1{,}1\,\text{c}$.
    Чему равна ЭДС в контуре? Ответ выразите в милливольтах и округлите до целого, единицы измерения писать не нужно.
}
\answer{%
    $\ele = 13{,}636\,\text{мВ} \to 14$
}
\solutionspace{60pt}

\tasknumber{7}%
\task{%
    Определите магнитный поток через контур,
    находящийся в однородном магнитном поле индукцией $300\,\text{мТл}$.
    Контур имеет форму прямоугольного треугольника с катетами $50\,\text{см}$ и $75\,\text{см}$.
    Угол между плоскостью контура и вектором индукции магнитного поля
    составляет $80\degrees$.
    Ответ выразите в милливеберах и округлите до целого, единицы измерения писать не нужно.
}
\answer{%
    $\alpha=10\degrees, \Phi_B = BS\cos\alpha = 55{,}40\,\text{мВб} \to 55$
}
\solutionspace{100pt}

\tasknumber{8}%
\task{%
    Какой средний индукционный ток возник в плоском контуре площадью $250\,\text{см}^{2}$
    и сопротивлением $2{,}5\,\text{Ом}$, если сперва он располагался параллельной линиям индукции магнитного поля,
    а затем его за $0{,}6\,\text{мc}$ повернули, и теперь угол между плоскостью контура и индукцией магнитного поля
    равен $20\degrees$.
    Магнитное поле однородно, его индукция равна $80\,\text{мТл}$.
    Ответ выразите в микроамперах и округлите до целого, единицы измерения писать не нужно.
}
\answer{%
    \begin{align*}
    \alpha_1 &= 90\degrees, \alpha_2 = 70\degrees, \\
    \ele &= \abs{\frac{\Delta \Phi}{\Delta t}}, \\
    \eli &= \frac \ele R = \frac{\Delta \Phi}{R\Delta t}= \frac{ \abs{ B S \cos\alpha_2 - B S \cos\alpha_1 } }{R\Delta t}= \frac{ B S \abs{ \cos\alpha_2 -  \cos\alpha_1 } }{R\Delta t} =  \\
    &= \frac{ 80\,\text{мТл} \cdot 250\,\text{см}^{2} \abs{ \cos70\degrees -  \cos90\degrees } }{2{,}5\,\text{Ом} \cdot 0{,}6\,\text{мc}}\approx 456{,}027\,\text{мкА} \to 456
    \end{align*}
}

\variantsplitter

\addpersonalvariant{Виктория Легонькова}

\tasknumber{1}%
\task{%
    Установите каждой букве в соответствие ровно одну цифру и запишите ответ (только цифры, без других символов).

    А) сопротивление контура, Б) магнитный поток.

    1) $D$, 2) $R$, 3) $\eli$, 4) $\Phi$.
}
\answer{%
    $24$
}
\solutionspace{20pt}

\tasknumber{2}%
\task{%
    Установите каждой букве в соответствие ровно одну цифру и запишите ответ (только цифры, без других символов).

    А) индукция магнитного поля, Б) индукционый ток, В) магнитный поток.

    1) В, 2) Гн, 3) Тл, 4) Вб, 5) А.
}
\answer{%
    $354$
}
\solutionspace{20pt}

\tasknumber{3}%
\task{%
    Однородное магнитное поле пронизывает плоский контур площадью $200\,\text{см}^{2}$.
    Индукция магнитного поля равна $300\,\text{мТл}$.
    Чему равен магнитный поток через контур, если его плоскость
    расположена под углом $60\degrees$ к вектору магнитной индукции?
    Ответ выразите в милливеберах и округлите до целого, единицы измерения писать не нужно.
}
\answer{%
    $\alpha = 30\degrees, \Phi_B = BS\cos\alpha = 5{,}20\,\text{мВб} \to 5$
}
\solutionspace{100pt}

\tasknumber{4}%
\task{%
    Определите притягивается (А), не взаимодействует (Б) или отталкивается (В) металлическое кольцо к магниту,
    если выдвигать южным полюсом (см.
    рис).
}
\answer{%
    \text{А}
}

\tasknumber{5}%
\task{%
    Определите притягивается (А), не взаимодействует (Б) или отталкивается (В) кольцо из диэлектрика к магниту,
    если вдвигать магнит в кольцо южным полюсом (см.
    рис).
}
\answer{%
    \text{Б}
}

\tasknumber{6}%
\task{%
    Магнитный поток, пронизывающий замкнутый контур, равномерно изменяется от $95\,\text{мВб}$ до $110\,\text{мВб}$ за $1{,}5\,\text{c}$.
    Чему равна ЭДС в контуре? Ответ выразите в милливольтах и округлите до целого, единицы измерения писать не нужно.
}
\answer{%
    $\ele = 10{,}000\,\text{мВ} \to 10$
}
\solutionspace{60pt}

\tasknumber{7}%
\task{%
    Определите магнитный поток через контур,
    находящийся в однородном магнитном поле индукцией $700\,\text{мТл}$.
    Контур имеет форму прямоугольника со сторонами $60\,\text{см}$ и $75\,\text{см}$.
    Угол между нормалью к плоскости контура и вектором индукции магнитного поля
    составляет $40\degrees$.
    Ответ выразите в милливеберах и округлите до целого, единицы измерения писать не нужно.
}
\answer{%
    $\alpha=40\degrees, \Phi_B = BS\cos\alpha = 241{,}30\,\text{мВб} \to 241$
}
\solutionspace{100pt}

\tasknumber{8}%
\task{%
    Какой средний индукционный ток возник в плоском контуре площадью $120\,\text{см}^{2}$
    и сопротивлением $3\,\text{Ом}$, если сперва он располагался параллельной линиям индукции магнитного поля,
    а затем его за $0{,}8\,\text{мc}$ повернули, и теперь угол между нормалью к плоскости контура и индукцией магнитного поля
    равен $70\degrees$.
    Магнитное поле однородно, его индукция равна $50\,\text{мТл}$.
    Ответ выразите в микроамперах и округлите до целого, единицы измерения писать не нужно.
}
\answer{%
    \begin{align*}
    \alpha_1 &= 90\degrees, \alpha_2 = 70\degrees, \\
    \ele &= \abs{\frac{\Delta \Phi}{\Delta t}}, \\
    \eli &= \frac \ele R = \frac{\Delta \Phi}{R\Delta t}= \frac{ \abs{ B S \cos\alpha_2 - B S \cos\alpha_1 } }{R\Delta t}= \frac{ B S \abs{ \cos\alpha_2 -  \cos\alpha_1 } }{R\Delta t} =  \\
    &= \frac{ 50\,\text{мТл} \cdot 120\,\text{см}^{2} \abs{ \cos70\degrees -  \cos90\degrees } }{3\,\text{Ом} \cdot 0{,}8\,\text{мc}}\approx 85{,}505\,\text{мкА} \to 86
    \end{align*}
}

\variantsplitter

\addpersonalvariant{Семён Мартынов}

\tasknumber{1}%
\task{%
    Установите каждой букве в соответствие ровно одну цифру и запишите ответ (только цифры, без других символов).

    А) сопротивление контура, Б) индукция магнитного поля.

    1) $D$, 2) $B$, 3) $R$, 4) $\eli$.
}
\answer{%
    $32$
}
\solutionspace{20pt}

\tasknumber{2}%
\task{%
    Установите каждой букве в соответствие ровно одну цифру и запишите ответ (только цифры, без других символов).

    А) индукционый ток, Б) ЭДС индукции, В) площадь контура.

    1) Ом, 2) Кл, 3) В, 4) А, 5) $\text{м}^2$.
}
\answer{%
    $435$
}
\solutionspace{20pt}

\tasknumber{3}%
\task{%
    Однородное магнитное поле пронизывает плоский контур площадью $400\,\text{см}^{2}$.
    Индукция магнитного поля равна $300\,\text{мТл}$.
    Чему равен магнитный поток через контур, если его плоскость
    расположена под углом $90\degrees$ к вектору магнитной индукции?
    Ответ выразите в милливеберах и округлите до целого, единицы измерения писать не нужно.
}
\answer{%
    $\alpha = 0\degrees, \Phi_B = BS\cos\alpha = 12{,}00\,\text{мВб} \to 12$
}
\solutionspace{100pt}

\tasknumber{4}%
\task{%
    Определите притягивается (А), не взаимодействует (Б) или отталкивается (В) металлическое кольцо к магниту,
    если выдвигать южным полюсом (см.
    рис).
}
\answer{%
    \text{А}
}

\tasknumber{5}%
\task{%
    Определите притягивается (А), не взаимодействует (Б) или отталкивается (В) кольцо из диэлектрика к магниту,
    если выдвигать магнит из кольца северным полюсом (см.
    рис).
}
\answer{%
    \text{Б}
}

\tasknumber{6}%
\task{%
    Магнитный поток, пронизывающий замкнутый контур, равномерно изменяется от $65\,\text{мВб}$ до $110\,\text{мВб}$ за $1{,}1\,\text{c}$.
    Чему равна ЭДС в контуре? Ответ выразите в милливольтах и округлите до целого, единицы измерения писать не нужно.
}
\answer{%
    $\ele = 40{,}909\,\text{мВ} \to 41$
}
\solutionspace{60pt}

\tasknumber{7}%
\task{%
    Определите магнитный поток через контур,
    находящийся в однородном магнитном поле индукцией $500\,\text{мТл}$.
    Контур имеет форму прямоугольника со сторонами $40\,\text{см}$ и $75\,\text{см}$.
    Угол между плоскостью контура и вектором индукции магнитного поля
    составляет $40\degrees$.
    Ответ выразите в милливеберах и округлите до целого, единицы измерения писать не нужно.
}
\answer{%
    $\alpha=50\degrees, \Phi_B = BS\cos\alpha = 96{,}42\,\text{мВб} \to 96$
}
\solutionspace{100pt}

\tasknumber{8}%
\task{%
    Какой средний индукционный ток возник в плоском контуре площадью $250\,\text{см}^{2}$
    и сопротивлением $2{,}5\,\text{Ом}$, если сперва он располагался параллельной линиям индукции магнитного поля,
    а затем его за $0{,}5\,\text{мc}$ повернули, и теперь угол между плоскостью контура и индукцией магнитного поля
    равен $10\degrees$.
    Магнитное поле однородно, его индукция равна $70\,\text{мТл}$.
    Ответ выразите в микроамперах и округлите до целого, единицы измерения писать не нужно.
}
\answer{%
    \begin{align*}
    \alpha_1 &= 90\degrees, \alpha_2 = 80\degrees, \\
    \ele &= \abs{\frac{\Delta \Phi}{\Delta t}}, \\
    \eli &= \frac \ele R = \frac{\Delta \Phi}{R\Delta t}= \frac{ \abs{ B S \cos\alpha_2 - B S \cos\alpha_1 } }{R\Delta t}= \frac{ B S \abs{ \cos\alpha_2 -  \cos\alpha_1 } }{R\Delta t} =  \\
    &= \frac{ 70\,\text{мТл} \cdot 250\,\text{см}^{2} \abs{ \cos80\degrees -  \cos90\degrees } }{2{,}5\,\text{Ом} \cdot 0{,}5\,\text{мc}}\approx 243{,}107\,\text{мкА} \to 243
    \end{align*}
}

\variantsplitter

\addpersonalvariant{Варвара Минаева}

\tasknumber{1}%
\task{%
    Установите каждой букве в соответствие ровно одну цифру и запишите ответ (только цифры, без других символов).

    А) индукционый ток, Б) площадь контура.

    1) $S$, 2) $D$, 3) $\eli$, 4) $\ele$.
}
\answer{%
    $31$
}
\solutionspace{20pt}

\tasknumber{2}%
\task{%
    Установите каждой букве в соответствие ровно одну цифру и запишите ответ (только цифры, без других символов).

    А) площадь контура, Б) ЭДС индукции, В) индукция магнитного поля.

    1) $\text{м}^2$, 2) Тл, 3) Кл, 4) В, 5) Ом.
}
\answer{%
    $142$
}
\solutionspace{20pt}

\tasknumber{3}%
\task{%
    Однородное магнитное поле пронизывает плоский контур площадью $400\,\text{см}^{2}$.
    Индукция магнитного поля равна $300\,\text{мТл}$.
    Чему равен магнитный поток через контур, если его плоскость
    расположена под углом $60\degrees$ к вектору магнитной индукции?
    Ответ выразите в милливеберах и округлите до целого, единицы измерения писать не нужно.
}
\answer{%
    $\alpha = 30\degrees, \Phi_B = BS\cos\alpha = 10{,}39\,\text{мВб} \to 10$
}
\solutionspace{100pt}

\tasknumber{4}%
\task{%
    Определите притягивается (А), не взаимодействует (Б) или отталкивается (В) металлическое кольцо к магниту,
    если выдвигать южным полюсом (см.
    рис).
}
\answer{%
    \text{А}
}

\tasknumber{5}%
\task{%
    Определите притягивается (А), не взаимодействует (Б) или отталкивается (В) кольцо из диэлектрика к магниту,
    если выдвигать магнит из кольца северным полюсом (см.
    рис).
}
\answer{%
    \text{Б}
}

\tasknumber{6}%
\task{%
    Магнитный поток, пронизывающий замкнутый контур, равномерно изменяется от $35\,\text{мВб}$ до $50\,\text{мВб}$ за $1{,}3\,\text{c}$.
    Чему равна ЭДС в контуре? Ответ выразите в милливольтах и округлите до целого, единицы измерения писать не нужно.
}
\answer{%
    $\ele = 11{,}538\,\text{мВ} \to 12$
}
\solutionspace{60pt}

\tasknumber{7}%
\task{%
    Определите магнитный поток через контур,
    находящийся в однородном магнитном поле индукцией $500\,\text{мТл}$.
    Контур имеет форму прямоугольника со сторонами $40\,\text{см}$ и $45\,\text{см}$.
    Угол между плоскостью контура и вектором индукции магнитного поля
    составляет $10\degrees$.
    Ответ выразите в милливеберах и округлите до целого, единицы измерения писать не нужно.
}
\answer{%
    $\alpha=80\degrees, \Phi_B = BS\cos\alpha = 15{,}63\,\text{мВб} \to 16$
}
\solutionspace{100pt}

\tasknumber{8}%
\task{%
    Какой средний индукционный ток возник в плоском контуре площадью $250\,\text{см}^{2}$
    и сопротивлением $3\,\text{Ом}$, если сперва он располагался перпендикулярно линиям индукции магнитного поля,
    а затем его за $0{,}8\,\text{мc}$ повернули, и теперь угол между плоскостью контура и индукцией магнитного поля
    равен $10\degrees$.
    Магнитное поле однородно, его индукция равна $70\,\text{мТл}$.
    Ответ выразите в микроамперах и округлите до целого, единицы измерения писать не нужно.
}
\answer{%
    \begin{align*}
    \alpha_1 &= 0\degrees, \alpha_2 = 80\degrees, \\
    \ele &= \abs{\frac{\Delta \Phi}{\Delta t}}, \\
    \eli &= \frac \ele R = \frac{\Delta \Phi}{R\Delta t}= \frac{ \abs{ B S \cos\alpha_2 - B S \cos\alpha_1 } }{R\Delta t}= \frac{ B S \abs{ \cos\alpha_2 -  \cos\alpha_1 } }{R\Delta t} =  \\
    &= \frac{ 70\,\text{мТл} \cdot 250\,\text{см}^{2} \abs{ \cos80\degrees -  \cos0\degrees } }{3\,\text{Ом} \cdot 0{,}8\,\text{мc}}\approx 602{,}548\,\text{мкА} \to 603
    \end{align*}
}

\variantsplitter

\addpersonalvariant{Леонид Никитин}

\tasknumber{1}%
\task{%
    Установите каждой букве в соответствие ровно одну цифру и запишите ответ (только цифры, без других символов).

    А) вектор нормали к поверхности, Б) индукционый ток.

    1) $\vec n$, 2) $v$, 3) $R$, 4) $\eli$.
}
\answer{%
    $14$
}
\solutionspace{20pt}

\tasknumber{2}%
\task{%
    Установите каждой букве в соответствие ровно одну цифру и запишите ответ (только цифры, без других символов).

    А) ЭДС индукции, Б) магнитный поток, В) индукция магнитного поля.

    1) В, 2) Вб, 3) Кл, 4) Ом, 5) Тл.
}
\answer{%
    $125$
}
\solutionspace{20pt}

\tasknumber{3}%
\task{%
    Однородное магнитное поле пронизывает плоский контур площадью $400\,\text{см}^{2}$.
    Индукция магнитного поля равна $700\,\text{мТл}$.
    Чему равен магнитный поток через контур, если его плоскость
    расположена под углом $60\degrees$ к вектору магнитной индукции?
    Ответ выразите в милливеберах и округлите до целого, единицы измерения писать не нужно.
}
\answer{%
    $\alpha = 30\degrees, \Phi_B = BS\cos\alpha = 24{,}25\,\text{мВб} \to 24$
}
\solutionspace{100pt}

\tasknumber{4}%
\task{%
    Определите притягивается (А), не взаимодействует (Б) или отталкивается (В) металлическое кольцо к магниту,
    если вдвигать северным полюсом (см.
    рис).
}
\answer{%
    \text{В}
}

\tasknumber{5}%
\task{%
    Определите притягивается (А), не взаимодействует (Б) или отталкивается (В) кольцо из диэлектрика к магниту,
    если вдвигать магнит в кольцо северным полюсом (см.
    рис).
}
\answer{%
    \text{Б}
}

\tasknumber{6}%
\task{%
    Магнитный поток, пронизывающий замкнутый контур, равномерно изменяется от $65\,\text{мВб}$ до $20\,\text{мВб}$ за $1{,}3\,\text{c}$.
    Чему равна ЭДС в контуре? Ответ выразите в милливольтах и округлите до целого, единицы измерения писать не нужно.
}
\answer{%
    $\ele = 34{,}615\,\text{мВ} \to 35$
}
\solutionspace{60pt}

\tasknumber{7}%
\task{%
    Определите магнитный поток через контур,
    находящийся в однородном магнитном поле индукцией $700\,\text{мТл}$.
    Контур имеет форму прямоугольного треугольника с катетами $60\,\text{см}$ и $75\,\text{см}$.
    Угол между нормалью к плоскости контура и вектором индукции магнитного поля
    составляет $50\degrees$.
    Ответ выразите в милливеберах и округлите до целого, единицы измерения писать не нужно.
}
\answer{%
    $\alpha=50\degrees, \Phi_B = BS\cos\alpha = 101{,}24\,\text{мВб} \to 101$
}
\solutionspace{100pt}

\tasknumber{8}%
\task{%
    Какой средний индукционный ток возник в плоском контуре площадью $120\,\text{см}^{2}$
    и сопротивлением $2\,\text{Ом}$, если сперва он располагался параллельной линиям индукции магнитного поля,
    а затем его за $0{,}6\,\text{мc}$ повернули, и теперь угол между плоскостью контура и индукцией магнитного поля
    равен $40\degrees$.
    Магнитное поле однородно, его индукция равна $50\,\text{мТл}$.
    Ответ выразите в микроамперах и округлите до целого, единицы измерения писать не нужно.
}
\answer{%
    \begin{align*}
    \alpha_1 &= 90\degrees, \alpha_2 = 50\degrees, \\
    \ele &= \abs{\frac{\Delta \Phi}{\Delta t}}, \\
    \eli &= \frac \ele R = \frac{\Delta \Phi}{R\Delta t}= \frac{ \abs{ B S \cos\alpha_2 - B S \cos\alpha_1 } }{R\Delta t}= \frac{ B S \abs{ \cos\alpha_2 -  \cos\alpha_1 } }{R\Delta t} =  \\
    &= \frac{ 50\,\text{мТл} \cdot 120\,\text{см}^{2} \abs{ \cos50\degrees -  \cos90\degrees } }{2\,\text{Ом} \cdot 0{,}6\,\text{мc}}\approx 321{,}394\,\text{мкА} \to 321
    \end{align*}
}

\variantsplitter

\addpersonalvariant{Тимофей Полетаев}

\tasknumber{1}%
\task{%
    Установите каждой букве в соответствие ровно одну цифру и запишите ответ (только цифры, без других символов).

    А) индукция магнитного поля, Б) магнитный поток.

    1) $B$, 2) $\eli$, 3) $\Phi$, 4) $S$.
}
\answer{%
    $13$
}
\solutionspace{20pt}

\tasknumber{2}%
\task{%
    Установите каждой букве в соответствие ровно одну цифру и запишите ответ (только цифры, без других символов).

    А) индукционый ток, Б) площадь контура, В) индукция магнитного поля.

    1) Гц, 2) А, 3) Тл, 4) $\text{м}^2$, 5) В.
}
\answer{%
    $243$
}
\solutionspace{20pt}

\tasknumber{3}%
\task{%
    Однородное магнитное поле пронизывает плоский контур площадью $400\,\text{см}^{2}$.
    Индукция магнитного поля равна $300\,\text{мТл}$.
    Чему равен магнитный поток через контур, если его плоскость
    расположена под углом $0\degrees$ к вектору магнитной индукции?
    Ответ выразите в милливеберах и округлите до целого, единицы измерения писать не нужно.
}
\answer{%
    $\alpha = 90\degrees, \Phi_B = BS\cos\alpha = 0\,\text{мВб} \to 0$
}
\solutionspace{100pt}

\tasknumber{4}%
\task{%
    Определите притягивается (А), не взаимодействует (Б) или отталкивается (В) металлическое кольцо к магниту,
    если выдвигать южным полюсом (см.
    рис).
}
\answer{%
    \text{А}
}

\tasknumber{5}%
\task{%
    Определите притягивается (А), не взаимодействует (Б) или отталкивается (В) кольцо из диэлектрика к магниту,
    если выдвигать магнит из кольца северным полюсом (см.
    рис).
}
\answer{%
    \text{Б}
}

\tasknumber{6}%
\task{%
    Магнитный поток, пронизывающий замкнутый контур, равномерно изменяется от $95\,\text{мВб}$ до $110\,\text{мВб}$ за $1{,}3\,\text{c}$.
    Чему равна ЭДС в контуре? Ответ выразите в милливольтах и округлите до целого, единицы измерения писать не нужно.
}
\answer{%
    $\ele = 11{,}538\,\text{мВ} \to 12$
}
\solutionspace{60pt}

\tasknumber{7}%
\task{%
    Определите магнитный поток через контур,
    находящийся в однородном магнитном поле индукцией $300\,\text{мТл}$.
    Контур имеет форму прямоугольника со сторонами $40\,\text{см}$ и $80\,\text{см}$.
    Угол между плоскостью контура и вектором индукции магнитного поля
    составляет $20\degrees$.
    Ответ выразите в милливеберах и округлите до целого, единицы измерения писать не нужно.
}
\answer{%
    $\alpha=70\degrees, \Phi_B = BS\cos\alpha = 32{,}83\,\text{мВб} \to 33$
}
\solutionspace{100pt}

\tasknumber{8}%
\task{%
    Какой средний индукционный ток возник в плоском контуре площадью $150\,\text{см}^{2}$
    и сопротивлением $2\,\text{Ом}$, если сперва он располагался перпендикулярно линиям индукции магнитного поля,
    а затем его за $0{,}5\,\text{мc}$ повернули, и теперь угол между нормалью к плоскости контура и индукцией магнитного поля
    равен $10\degrees$.
    Магнитное поле однородно, его индукция равна $50\,\text{мТл}$.
    Ответ выразите в микроамперах и округлите до целого, единицы измерения писать не нужно.
}
\answer{%
    \begin{align*}
    \alpha_1 &= 0\degrees, \alpha_2 = 10\degrees, \\
    \ele &= \abs{\frac{\Delta \Phi}{\Delta t}}, \\
    \eli &= \frac \ele R = \frac{\Delta \Phi}{R\Delta t}= \frac{ \abs{ B S \cos\alpha_2 - B S \cos\alpha_1 } }{R\Delta t}= \frac{ B S \abs{ \cos\alpha_2 -  \cos\alpha_1 } }{R\Delta t} =  \\
    &= \frac{ 50\,\text{мТл} \cdot 150\,\text{см}^{2} \abs{ \cos10\degrees -  \cos0\degrees } }{2\,\text{Ом} \cdot 0{,}5\,\text{мc}}\approx 11{,}394\,\text{мкА} \to 11
    \end{align*}
}

\variantsplitter

\addpersonalvariant{Андрей Рожков}

\tasknumber{1}%
\task{%
    Установите каждой букве в соответствие ровно одну цифру и запишите ответ (только цифры, без других символов).

    А) ЭДС индукции, Б) индукция магнитного поля.

    1) $\ele$, 2) $B$, 3) $\eli$, 4) $R$.
}
\answer{%
    $12$
}
\solutionspace{20pt}

\tasknumber{2}%
\task{%
    Установите каждой букве в соответствие ровно одну цифру и запишите ответ (только цифры, без других символов).

    А) ЭДС индукции, Б) индукционый ток, В) магнитный поток.

    1) А, 2) В, 3) Вб, 4) Тл, 5) Кл.
}
\answer{%
    $213$
}
\solutionspace{20pt}

\tasknumber{3}%
\task{%
    Однородное магнитное поле пронизывает плоский контур площадью $600\,\text{см}^{2}$.
    Индукция магнитного поля равна $300\,\text{мТл}$.
    Чему равен магнитный поток через контур, если его плоскость
    расположена под углом $90\degrees$ к вектору магнитной индукции?
    Ответ выразите в милливеберах и округлите до целого, единицы измерения писать не нужно.
}
\answer{%
    $\alpha = 0\degrees, \Phi_B = BS\cos\alpha = 18{,}00\,\text{мВб} \to 18$
}
\solutionspace{100pt}

\tasknumber{4}%
\task{%
    Определите притягивается (А), не взаимодействует (Б) или отталкивается (В) металлическое кольцо к магниту,
    если вдвигать южным полюсом (см.
    рис).
}
\answer{%
    \text{В}
}

\tasknumber{5}%
\task{%
    Определите притягивается (А), не взаимодействует (Б) или отталкивается (В) кольцо из диэлектрика к магниту,
    если выдвигать магнит из кольца северным полюсом (см.
    рис).
}
\answer{%
    \text{Б}
}

\tasknumber{6}%
\task{%
    Магнитный поток, пронизывающий замкнутый контур, равномерно изменяется от $95\,\text{мВб}$ до $20\,\text{мВб}$ за $1{,}3\,\text{c}$.
    Чему равна ЭДС в контуре? Ответ выразите в милливольтах и округлите до целого, единицы измерения писать не нужно.
}
\answer{%
    $\ele = 57{,}692\,\text{мВ} \to 58$
}
\solutionspace{60pt}

\tasknumber{7}%
\task{%
    Определите магнитный поток через контур,
    находящийся в однородном магнитном поле индукцией $500\,\text{мТл}$.
    Контур имеет форму прямоугольника со сторонами $60\,\text{см}$ и $75\,\text{см}$.
    Угол между плоскостью контура и вектором индукции магнитного поля
    составляет $40\degrees$.
    Ответ выразите в милливеберах и округлите до целого, единицы измерения писать не нужно.
}
\answer{%
    $\alpha=50\degrees, \Phi_B = BS\cos\alpha = 144{,}63\,\text{мВб} \to 145$
}
\solutionspace{100pt}

\tasknumber{8}%
\task{%
    Какой средний индукционный ток возник в плоском контуре площадью $250\,\text{см}^{2}$
    и сопротивлением $3\,\text{Ом}$, если сперва он располагался перпендикулярно линиям индукции магнитного поля,
    а затем его за $0{,}8\,\text{мc}$ повернули, и теперь угол между нормалью к плоскости контура и индукцией магнитного поля
    равен $80\degrees$.
    Магнитное поле однородно, его индукция равна $50\,\text{мТл}$.
    Ответ выразите в микроамперах и округлите до целого, единицы измерения писать не нужно.
}
\answer{%
    \begin{align*}
    \alpha_1 &= 0\degrees, \alpha_2 = 80\degrees, \\
    \ele &= \abs{\frac{\Delta \Phi}{\Delta t}}, \\
    \eli &= \frac \ele R = \frac{\Delta \Phi}{R\Delta t}= \frac{ \abs{ B S \cos\alpha_2 - B S \cos\alpha_1 } }{R\Delta t}= \frac{ B S \abs{ \cos\alpha_2 -  \cos\alpha_1 } }{R\Delta t} =  \\
    &= \frac{ 50\,\text{мТл} \cdot 250\,\text{см}^{2} \abs{ \cos80\degrees -  \cos0\degrees } }{3\,\text{Ом} \cdot 0{,}8\,\text{мc}}\approx 430{,}392\,\text{мкА} \to 430
    \end{align*}
}

\variantsplitter

\addpersonalvariant{Рената Таржиманова}

\tasknumber{1}%
\task{%
    Установите каждой букве в соответствие ровно одну цифру и запишите ответ (только цифры, без других символов).

    А) вектор нормали к поверхности, Б) сопротивление контура.

    1) $R$, 2) $D$, 3) $\vec n$, 4) $S$.
}
\answer{%
    $31$
}
\solutionspace{20pt}

\tasknumber{2}%
\task{%
    Установите каждой букве в соответствие ровно одну цифру и запишите ответ (только цифры, без других символов).

    А) индукционый ток, Б) индукция магнитного поля, В) площадь контура.

    1) Гц, 2) Тл, 3) Вб, 4) А, 5) $\text{м}^2$.
}
\answer{%
    $425$
}
\solutionspace{20pt}

\tasknumber{3}%
\task{%
    Однородное магнитное поле пронизывает плоский контур площадью $400\,\text{см}^{2}$.
    Индукция магнитного поля равна $500\,\text{мТл}$.
    Чему равен магнитный поток через контур, если его плоскость
    расположена под углом $90\degrees$ к вектору магнитной индукции?
    Ответ выразите в милливеберах и округлите до целого, единицы измерения писать не нужно.
}
\answer{%
    $\alpha = 0\degrees, \Phi_B = BS\cos\alpha = 20{,}00\,\text{мВб} \to 20$
}
\solutionspace{100pt}

\tasknumber{4}%
\task{%
    Определите притягивается (А), не взаимодействует (Б) или отталкивается (В) металлическое кольцо к магниту,
    если вдвигать северным полюсом (см.
    рис).
}
\answer{%
    \text{В}
}

\tasknumber{5}%
\task{%
    Определите притягивается (А), не взаимодействует (Б) или отталкивается (В) кольцо из диэлектрика к магниту,
    если выдвигать магнит из кольца южным полюсом (см.
    рис).
}
\answer{%
    \text{Б}
}

\tasknumber{6}%
\task{%
    Магнитный поток, пронизывающий замкнутый контур, равномерно изменяется от $95\,\text{мВб}$ до $20\,\text{мВб}$ за $1{,}1\,\text{c}$.
    Чему равна ЭДС в контуре? Ответ выразите в милливольтах и округлите до целого, единицы измерения писать не нужно.
}
\answer{%
    $\ele = 68{,}182\,\text{мВ} \to 68$
}
\solutionspace{60pt}

\tasknumber{7}%
\task{%
    Определите магнитный поток через контур,
    находящийся в однородном магнитном поле индукцией $700\,\text{мТл}$.
    Контур имеет форму прямоугольного треугольника с катетами $60\,\text{см}$ и $75\,\text{см}$.
    Угол между плоскостью контура и вектором индукции магнитного поля
    составляет $20\degrees$.
    Ответ выразите в милливеберах и округлите до целого, единицы измерения писать не нужно.
}
\answer{%
    $\alpha=70\degrees, \Phi_B = BS\cos\alpha = 53{,}87\,\text{мВб} \to 54$
}
\solutionspace{100pt}

\tasknumber{8}%
\task{%
    Какой средний индукционный ток возник в плоском контуре площадью $250\,\text{см}^{2}$
    и сопротивлением $2\,\text{Ом}$, если сперва он располагался параллельной линиям индукции магнитного поля,
    а затем его за $0{,}2\,\text{мc}$ повернули, и теперь угол между нормалью к плоскости контура и индукцией магнитного поля
    равен $40\degrees$.
    Магнитное поле однородно, его индукция равна $70\,\text{мТл}$.
    Ответ выразите в микроамперах и округлите до целого, единицы измерения писать не нужно.
}
\answer{%
    \begin{align*}
    \alpha_1 &= 90\degrees, \alpha_2 = 40\degrees, \\
    \ele &= \abs{\frac{\Delta \Phi}{\Delta t}}, \\
    \eli &= \frac \ele R = \frac{\Delta \Phi}{R\Delta t}= \frac{ \abs{ B S \cos\alpha_2 - B S \cos\alpha_1 } }{R\Delta t}= \frac{ B S \abs{ \cos\alpha_2 -  \cos\alpha_1 } }{R\Delta t} =  \\
    &= \frac{ 70\,\text{мТл} \cdot 250\,\text{см}^{2} \abs{ \cos40\degrees -  \cos90\degrees } }{2\,\text{Ом} \cdot 0{,}2\,\text{мc}}\approx 3351{,}444\,\text{мкА} \to 3351
    \end{align*}
}

\variantsplitter

\addpersonalvariant{Андрей Щербаков}

\tasknumber{1}%
\task{%
    Установите каждой букве в соответствие ровно одну цифру и запишите ответ (только цифры, без других символов).

    А) магнитный поток, Б) ЭДС индукции.

    1) $\ele$, 2) $\eli$, 3) $D$, 4) $\Phi$.
}
\answer{%
    $41$
}
\solutionspace{20pt}

\tasknumber{2}%
\task{%
    Установите каждой букве в соответствие ровно одну цифру и запишите ответ (только цифры, без других символов).

    А) индукция магнитного поля, Б) площадь контура, В) ЭДС индукции.

    1) В, 2) $\text{м}^2$, 3) Вб, 4) Ом, 5) Тл.
}
\answer{%
    $521$
}
\solutionspace{20pt}

\tasknumber{3}%
\task{%
    Однородное магнитное поле пронизывает плоский контур площадью $800\,\text{см}^{2}$.
    Индукция магнитного поля равна $700\,\text{мТл}$.
    Чему равен магнитный поток через контур, если его плоскость
    расположена под углом $90\degrees$ к вектору магнитной индукции?
    Ответ выразите в милливеберах и округлите до целого, единицы измерения писать не нужно.
}
\answer{%
    $\alpha = 0\degrees, \Phi_B = BS\cos\alpha = 56{,}00\,\text{мВб} \to 56$
}
\solutionspace{100pt}

\tasknumber{4}%
\task{%
    Определите притягивается (А), не взаимодействует (Б) или отталкивается (В) металлическое кольцо к магниту,
    если вдвигать северным полюсом (см.
    рис).
}
\answer{%
    \text{В}
}

\tasknumber{5}%
\task{%
    Определите притягивается (А), не взаимодействует (Б) или отталкивается (В) кольцо из диэлектрика к магниту,
    если выдвигать магнит из кольца южным полюсом (см.
    рис).
}
\answer{%
    \text{Б}
}

\tasknumber{6}%
\task{%
    Магнитный поток, пронизывающий замкнутый контур, равномерно изменяется от $35\,\text{мВб}$ до $20\,\text{мВб}$ за $1{,}3\,\text{c}$.
    Чему равна ЭДС в контуре? Ответ выразите в милливольтах и округлите до целого, единицы измерения писать не нужно.
}
\answer{%
    $\ele = 11{,}538\,\text{мВ} \to 12$
}
\solutionspace{60pt}

\tasknumber{7}%
\task{%
    Определите магнитный поток через контур,
    находящийся в однородном магнитном поле индукцией $500\,\text{мТл}$.
    Контур имеет форму прямоугольного треугольника с катетами $60\,\text{см}$ и $45\,\text{см}$.
    Угол между плоскостью контура и вектором индукции магнитного поля
    составляет $10\degrees$.
    Ответ выразите в милливеберах и округлите до целого, единицы измерения писать не нужно.
}
\answer{%
    $\alpha=80\degrees, \Phi_B = BS\cos\alpha = 11{,}72\,\text{мВб} \to 12$
}
\solutionspace{100pt}

\tasknumber{8}%
\task{%
    Какой средний индукционный ток возник в плоском контуре площадью $250\,\text{см}^{2}$
    и сопротивлением $2\,\text{Ом}$, если сперва он располагался перпендикулярно линиям индукции магнитного поля,
    а затем его за $0{,}2\,\text{мc}$ повернули, и теперь угол между плоскостью контура и индукцией магнитного поля
    равен $40\degrees$.
    Магнитное поле однородно, его индукция равна $80\,\text{мТл}$.
    Ответ выразите в микроамперах и округлите до целого, единицы измерения писать не нужно.
}
\answer{%
    \begin{align*}
    \alpha_1 &= 0\degrees, \alpha_2 = 50\degrees, \\
    \ele &= \abs{\frac{\Delta \Phi}{\Delta t}}, \\
    \eli &= \frac \ele R = \frac{\Delta \Phi}{R\Delta t}= \frac{ \abs{ B S \cos\alpha_2 - B S \cos\alpha_1 } }{R\Delta t}= \frac{ B S \abs{ \cos\alpha_2 -  \cos\alpha_1 } }{R\Delta t} =  \\
    &= \frac{ 80\,\text{мТл} \cdot 250\,\text{см}^{2} \abs{ \cos50\degrees -  \cos0\degrees } }{2\,\text{Ом} \cdot 0{,}2\,\text{мc}}\approx 1786{,}062\,\text{мкА} \to 1786
    \end{align*}
}

\variantsplitter

\addpersonalvariant{Михаил Ярошевский}

\tasknumber{1}%
\task{%
    Установите каждой букве в соответствие ровно одну цифру и запишите ответ (только цифры, без других символов).

    А) сопротивление контура, Б) ЭДС индукции.

    1) $R$, 2) $\ele$, 3) $\Phi$, 4) $B$.
}
\answer{%
    $12$
}
\solutionspace{20pt}

\tasknumber{2}%
\task{%
    Установите каждой букве в соответствие ровно одну цифру и запишите ответ (только цифры, без других символов).

    А) ЭДС индукции, Б) площадь контура, В) индукционый ток.

    1) Тл, 2) В, 3) $\text{м}^2$, 4) Кл, 5) А.
}
\answer{%
    $235$
}
\solutionspace{20pt}

\tasknumber{3}%
\task{%
    Однородное магнитное поле пронизывает плоский контур площадью $200\,\text{см}^{2}$.
    Индукция магнитного поля равна $500\,\text{мТл}$.
    Чему равен магнитный поток через контур, если его плоскость
    расположена под углом $30\degrees$ к вектору магнитной индукции?
    Ответ выразите в милливеберах и округлите до целого, единицы измерения писать не нужно.
}
\answer{%
    $\alpha = 60\degrees, \Phi_B = BS\cos\alpha = 5{,}00\,\text{мВб} \to 5$
}
\solutionspace{100pt}

\tasknumber{4}%
\task{%
    Определите притягивается (А), не взаимодействует (Б) или отталкивается (В) металлическое кольцо к магниту,
    если вдвигать северным полюсом (см.
    рис).
}
\answer{%
    \text{В}
}

\tasknumber{5}%
\task{%
    Определите притягивается (А), не взаимодействует (Б) или отталкивается (В) кольцо из диэлектрика к магниту,
    если выдвигать магнит из кольца южным полюсом (см.
    рис).
}
\answer{%
    \text{Б}
}

\tasknumber{6}%
\task{%
    Магнитный поток, пронизывающий замкнутый контур, равномерно изменяется от $65\,\text{мВб}$ до $110\,\text{мВб}$ за $1{,}3\,\text{c}$.
    Чему равна ЭДС в контуре? Ответ выразите в милливольтах и округлите до целого, единицы измерения писать не нужно.
}
\answer{%
    $\ele = 34{,}615\,\text{мВ} \to 35$
}
\solutionspace{60pt}

\tasknumber{7}%
\task{%
    Определите магнитный поток через контур,
    находящийся в однородном магнитном поле индукцией $500\,\text{мТл}$.
    Контур имеет форму прямоугольного треугольника с катетами $40\,\text{см}$ и $80\,\text{см}$.
    Угол между плоскостью контура и вектором индукции магнитного поля
    составляет $70\degrees$.
    Ответ выразите в милливеберах и округлите до целого, единицы измерения писать не нужно.
}
\answer{%
    $\alpha=20\degrees, \Phi_B = BS\cos\alpha = 75{,}18\,\text{мВб} \to 75$
}
\solutionspace{100pt}

\tasknumber{8}%
\task{%
    Какой средний индукционный ток возник в плоском контуре площадью $250\,\text{см}^{2}$
    и сопротивлением $2\,\text{Ом}$, если сперва он располагался параллельной линиям индукции магнитного поля,
    а затем его за $0{,}2\,\text{мc}$ повернули, и теперь угол между нормалью к плоскости контура и индукцией магнитного поля
    равен $70\degrees$.
    Магнитное поле однородно, его индукция равна $80\,\text{мТл}$.
    Ответ выразите в микроамперах и округлите до целого, единицы измерения писать не нужно.
}
\answer{%
    \begin{align*}
    \alpha_1 &= 90\degrees, \alpha_2 = 70\degrees, \\
    \ele &= \abs{\frac{\Delta \Phi}{\Delta t}}, \\
    \eli &= \frac \ele R = \frac{\Delta \Phi}{R\Delta t}= \frac{ \abs{ B S \cos\alpha_2 - B S \cos\alpha_1 } }{R\Delta t}= \frac{ B S \abs{ \cos\alpha_2 -  \cos\alpha_1 } }{R\Delta t} =  \\
    &= \frac{ 80\,\text{мТл} \cdot 250\,\text{см}^{2} \abs{ \cos70\degrees -  \cos90\degrees } }{2\,\text{Ом} \cdot 0{,}2\,\text{мc}}\approx 1710{,}101\,\text{мкА} \to 1710
    \end{align*}
}

\variantsplitter

\addpersonalvariant{Алексей Алимпиев}

\tasknumber{1}%
\task{%
    Установите каждой букве в соответствие ровно одну цифру и запишите ответ (только цифры, без других символов).

    А) сопротивление контура, Б) вектор нормали к поверхности.

    1) $\ele$, 2) $\vec n$, 3) $R$, 4) $\eli$.
}
\answer{%
    $32$
}
\solutionspace{20pt}

\tasknumber{2}%
\task{%
    Установите каждой букве в соответствие ровно одну цифру и запишите ответ (только цифры, без других символов).

    А) ЭДС индукции, Б) магнитный поток, В) индукция магнитного поля.

    1) Ом, 2) Тл, 3) Вб, 4) В, 5) Гц.
}
\answer{%
    $432$
}
\solutionspace{20pt}

\tasknumber{3}%
\task{%
    Однородное магнитное поле пронизывает плоский контур площадью $600\,\text{см}^{2}$.
    Индукция магнитного поля равна $700\,\text{мТл}$.
    Чему равен магнитный поток через контур, если его плоскость
    расположена под углом $60\degrees$ к вектору магнитной индукции?
    Ответ выразите в милливеберах и округлите до целого, единицы измерения писать не нужно.
}
\answer{%
    $\alpha = 30\degrees, \Phi_B = BS\cos\alpha = 36{,}37\,\text{мВб} \to 36$
}
\solutionspace{100pt}

\tasknumber{4}%
\task{%
    Определите притягивается (А), не взаимодействует (Б) или отталкивается (В) металлическое кольцо к магниту,
    если выдвигать северным полюсом (см.
    рис).
}
\answer{%
    \text{А}
}

\tasknumber{5}%
\task{%
    Определите притягивается (А), не взаимодействует (Б) или отталкивается (В) кольцо из диэлектрика к магниту,
    если выдвигать магнит из кольца южным полюсом (см.
    рис).
}
\answer{%
    \text{Б}
}

\tasknumber{6}%
\task{%
    Магнитный поток, пронизывающий замкнутый контур, равномерно изменяется от $35\,\text{мВб}$ до $20\,\text{мВб}$ за $1{,}3\,\text{c}$.
    Чему равна ЭДС в контуре? Ответ выразите в милливольтах и округлите до целого, единицы измерения писать не нужно.
}
\answer{%
    $\ele = 11{,}538\,\text{мВ} \to 12$
}
\solutionspace{60pt}

\tasknumber{7}%
\task{%
    Определите магнитный поток через контур,
    находящийся в однородном магнитном поле индукцией $700\,\text{мТл}$.
    Контур имеет форму прямоугольного треугольника с катетами $60\,\text{см}$ и $45\,\text{см}$.
    Угол между нормалью к плоскости контура и вектором индукции магнитного поля
    составляет $80\degrees$.
    Ответ выразите в милливеберах и округлите до целого, единицы измерения писать не нужно.
}
\answer{%
    $\alpha=80\degrees, \Phi_B = BS\cos\alpha = 16{,}41\,\text{мВб} \to 16$
}
\solutionspace{100pt}

\tasknumber{8}%
\task{%
    Какой средний индукционный ток возник в плоском контуре площадью $250\,\text{см}^{2}$
    и сопротивлением $1{,}5\,\text{Ом}$, если сперва он располагался перпендикулярно линиям индукции магнитного поля,
    а затем его за $0{,}5\,\text{мc}$ повернули, и теперь угол между плоскостью контура и индукцией магнитного поля
    равен $40\degrees$.
    Магнитное поле однородно, его индукция равна $80\,\text{мТл}$.
    Ответ выразите в микроамперах и округлите до целого, единицы измерения писать не нужно.
}
\answer{%
    \begin{align*}
    \alpha_1 &= 0\degrees, \alpha_2 = 50\degrees, \\
    \ele &= \abs{\frac{\Delta \Phi}{\Delta t}}, \\
    \eli &= \frac \ele R = \frac{\Delta \Phi}{R\Delta t}= \frac{ \abs{ B S \cos\alpha_2 - B S \cos\alpha_1 } }{R\Delta t}= \frac{ B S \abs{ \cos\alpha_2 -  \cos\alpha_1 } }{R\Delta t} =  \\
    &= \frac{ 80\,\text{мТл} \cdot 250\,\text{см}^{2} \abs{ \cos50\degrees -  \cos0\degrees } }{1{,}5\,\text{Ом} \cdot 0{,}5\,\text{мc}}\approx 952{,}566\,\text{мкА} \to 953
    \end{align*}
}

\variantsplitter

\addpersonalvariant{Евгений Васин}

\tasknumber{1}%
\task{%
    Установите каждой букве в соответствие ровно одну цифру и запишите ответ (только цифры, без других символов).

    А) площадь контура, Б) индукционый ток.

    1) $S$, 2) $B$, 3) $U$, 4) $\eli$.
}
\answer{%
    $14$
}
\solutionspace{20pt}

\tasknumber{2}%
\task{%
    Установите каждой букве в соответствие ровно одну цифру и запишите ответ (только цифры, без других символов).

    А) магнитный поток, Б) индукция магнитного поля, В) индукционый ток.

    1) В, 2) Вб, 3) Гц, 4) А, 5) Тл.
}
\answer{%
    $254$
}
\solutionspace{20pt}

\tasknumber{3}%
\task{%
    Однородное магнитное поле пронизывает плоский контур площадью $800\,\text{см}^{2}$.
    Индукция магнитного поля равна $500\,\text{мТл}$.
    Чему равен магнитный поток через контур, если его плоскость
    расположена под углом $30\degrees$ к вектору магнитной индукции?
    Ответ выразите в милливеберах и округлите до целого, единицы измерения писать не нужно.
}
\answer{%
    $\alpha = 60\degrees, \Phi_B = BS\cos\alpha = 20{,}00\,\text{мВб} \to 20$
}
\solutionspace{100pt}

\tasknumber{4}%
\task{%
    Определите притягивается (А), не взаимодействует (Б) или отталкивается (В) металлическое кольцо к магниту,
    если выдвигать северным полюсом (см.
    рис).
}
\answer{%
    \text{А}
}

\tasknumber{5}%
\task{%
    Определите притягивается (А), не взаимодействует (Б) или отталкивается (В) кольцо из диэлектрика к магниту,
    если вдвигать магнит в кольцо северным полюсом (см.
    рис).
}
\answer{%
    \text{Б}
}

\tasknumber{6}%
\task{%
    Магнитный поток, пронизывающий замкнутый контур, равномерно изменяется от $65\,\text{мВб}$ до $80\,\text{мВб}$ за $1{,}1\,\text{c}$.
    Чему равна ЭДС в контуре? Ответ выразите в милливольтах и округлите до целого, единицы измерения писать не нужно.
}
\answer{%
    $\ele = 13{,}636\,\text{мВ} \to 14$
}
\solutionspace{60pt}

\tasknumber{7}%
\task{%
    Определите магнитный поток через контур,
    находящийся в однородном магнитном поле индукцией $300\,\text{мТл}$.
    Контур имеет форму прямоугольника со сторонами $50\,\text{см}$ и $80\,\text{см}$.
    Угол между плоскостью контура и вектором индукции магнитного поля
    составляет $20\degrees$.
    Ответ выразите в милливеберах и округлите до целого, единицы измерения писать не нужно.
}
\answer{%
    $\alpha=70\degrees, \Phi_B = BS\cos\alpha = 41{,}04\,\text{мВб} \to 41$
}
\solutionspace{100pt}

\tasknumber{8}%
\task{%
    Какой средний индукционный ток возник в плоском контуре площадью $150\,\text{см}^{2}$
    и сопротивлением $3\,\text{Ом}$, если сперва он располагался параллельной линиям индукции магнитного поля,
    а затем его за $0{,}6\,\text{мc}$ повернули, и теперь угол между нормалью к плоскости контура и индукцией магнитного поля
    равен $40\degrees$.
    Магнитное поле однородно, его индукция равна $50\,\text{мТл}$.
    Ответ выразите в микроамперах и округлите до целого, единицы измерения писать не нужно.
}
\answer{%
    \begin{align*}
    \alpha_1 &= 90\degrees, \alpha_2 = 40\degrees, \\
    \ele &= \abs{\frac{\Delta \Phi}{\Delta t}}, \\
    \eli &= \frac \ele R = \frac{\Delta \Phi}{R\Delta t}= \frac{ \abs{ B S \cos\alpha_2 - B S \cos\alpha_1 } }{R\Delta t}= \frac{ B S \abs{ \cos\alpha_2 -  \cos\alpha_1 } }{R\Delta t} =  \\
    &= \frac{ 50\,\text{мТл} \cdot 150\,\text{см}^{2} \abs{ \cos40\degrees -  \cos90\degrees } }{3\,\text{Ом} \cdot 0{,}6\,\text{мc}}\approx 319{,}185\,\text{мкА} \to 319
    \end{align*}
}

\variantsplitter

\addpersonalvariant{Вячеслав Волохов}

\tasknumber{1}%
\task{%
    Установите каждой букве в соответствие ровно одну цифру и запишите ответ (только цифры, без других символов).

    А) индукция магнитного поля, Б) сопротивление контура.

    1) $R$, 2) $B$, 3) $S$, 4) $\ele$.
}
\answer{%
    $21$
}
\solutionspace{20pt}

\tasknumber{2}%
\task{%
    Установите каждой букве в соответствие ровно одну цифру и запишите ответ (только цифры, без других символов).

    А) площадь контура, Б) ЭДС индукции, В) индукционый ток.

    1) Вб, 2) В, 3) $\text{м}^2$, 4) Тл, 5) А.
}
\answer{%
    $325$
}
\solutionspace{20pt}

\tasknumber{3}%
\task{%
    Однородное магнитное поле пронизывает плоский контур площадью $200\,\text{см}^{2}$.
    Индукция магнитного поля равна $300\,\text{мТл}$.
    Чему равен магнитный поток через контур, если его плоскость
    расположена под углом $60\degrees$ к вектору магнитной индукции?
    Ответ выразите в милливеберах и округлите до целого, единицы измерения писать не нужно.
}
\answer{%
    $\alpha = 30\degrees, \Phi_B = BS\cos\alpha = 5{,}20\,\text{мВб} \to 5$
}
\solutionspace{100pt}

\tasknumber{4}%
\task{%
    Определите притягивается (А), не взаимодействует (Б) или отталкивается (В) металлическое кольцо к магниту,
    если вдвигать южным полюсом (см.
    рис).
}
\answer{%
    \text{В}
}

\tasknumber{5}%
\task{%
    Определите притягивается (А), не взаимодействует (Б) или отталкивается (В) кольцо из диэлектрика к магниту,
    если вдвигать магнит в кольцо северным полюсом (см.
    рис).
}
\answer{%
    \text{Б}
}

\tasknumber{6}%
\task{%
    Магнитный поток, пронизывающий замкнутый контур, равномерно изменяется от $35\,\text{мВб}$ до $20\,\text{мВб}$ за $1{,}3\,\text{c}$.
    Чему равна ЭДС в контуре? Ответ выразите в милливольтах и округлите до целого, единицы измерения писать не нужно.
}
\answer{%
    $\ele = 11{,}538\,\text{мВ} \to 12$
}
\solutionspace{60pt}

\tasknumber{7}%
\task{%
    Определите магнитный поток через контур,
    находящийся в однородном магнитном поле индукцией $300\,\text{мТл}$.
    Контур имеет форму прямоугольника со сторонами $50\,\text{см}$ и $80\,\text{см}$.
    Угол между плоскостью контура и вектором индукции магнитного поля
    составляет $50\degrees$.
    Ответ выразите в милливеберах и округлите до целого, единицы измерения писать не нужно.
}
\answer{%
    $\alpha=40\degrees, \Phi_B = BS\cos\alpha = 91{,}93\,\text{мВб} \to 92$
}
\solutionspace{100pt}

\tasknumber{8}%
\task{%
    Какой средний индукционный ток возник в плоском контуре площадью $250\,\text{см}^{2}$
    и сопротивлением $3\,\text{Ом}$, если сперва он располагался перпендикулярно линиям индукции магнитного поля,
    а затем его за $0{,}2\,\text{мc}$ повернули, и теперь угол между нормалью к плоскости контура и индукцией магнитного поля
    равен $20\degrees$.
    Магнитное поле однородно, его индукция равна $70\,\text{мТл}$.
    Ответ выразите в микроамперах и округлите до целого, единицы измерения писать не нужно.
}
\answer{%
    \begin{align*}
    \alpha_1 &= 0\degrees, \alpha_2 = 20\degrees, \\
    \ele &= \abs{\frac{\Delta \Phi}{\Delta t}}, \\
    \eli &= \frac \ele R = \frac{\Delta \Phi}{R\Delta t}= \frac{ \abs{ B S \cos\alpha_2 - B S \cos\alpha_1 } }{R\Delta t}= \frac{ B S \abs{ \cos\alpha_2 -  \cos\alpha_1 } }{R\Delta t} =  \\
    &= \frac{ 70\,\text{мТл} \cdot 250\,\text{см}^{2} \abs{ \cos20\degrees -  \cos0\degrees } }{3\,\text{Ом} \cdot 0{,}2\,\text{мc}}\approx 175{,}897\,\text{мкА} \to 176
    \end{align*}
}

\variantsplitter

\addpersonalvariant{Герман Говоров}

\tasknumber{1}%
\task{%
    Установите каждой букве в соответствие ровно одну цифру и запишите ответ (только цифры, без других символов).

    А) сопротивление контура, Б) индукционый ток.

    1) $U$, 2) $\eli$, 3) $R$, 4) $B$.
}
\answer{%
    $32$
}
\solutionspace{20pt}

\tasknumber{2}%
\task{%
    Установите каждой букве в соответствие ровно одну цифру и запишите ответ (только цифры, без других символов).

    А) ЭДС индукции, Б) площадь контура, В) индукция магнитного поля.

    1) Тл, 2) Ом, 3) $\text{м}^2$, 4) В, 5) Вб.
}
\answer{%
    $431$
}
\solutionspace{20pt}

\tasknumber{3}%
\task{%
    Однородное магнитное поле пронизывает плоский контур площадью $200\,\text{см}^{2}$.
    Индукция магнитного поля равна $300\,\text{мТл}$.
    Чему равен магнитный поток через контур, если его плоскость
    расположена под углом $30\degrees$ к вектору магнитной индукции?
    Ответ выразите в милливеберах и округлите до целого, единицы измерения писать не нужно.
}
\answer{%
    $\alpha = 60\degrees, \Phi_B = BS\cos\alpha = 3{,}00\,\text{мВб} \to 3$
}
\solutionspace{100pt}

\tasknumber{4}%
\task{%
    Определите притягивается (А), не взаимодействует (Б) или отталкивается (В) металлическое кольцо к магниту,
    если выдвигать южным полюсом (см.
    рис).
}
\answer{%
    \text{А}
}

\tasknumber{5}%
\task{%
    Определите притягивается (А), не взаимодействует (Б) или отталкивается (В) кольцо из диэлектрика к магниту,
    если вдвигать магнит в кольцо южным полюсом (см.
    рис).
}
\answer{%
    \text{Б}
}

\tasknumber{6}%
\task{%
    Магнитный поток, пронизывающий замкнутый контур, равномерно изменяется от $35\,\text{мВб}$ до $50\,\text{мВб}$ за $1{,}3\,\text{c}$.
    Чему равна ЭДС в контуре? Ответ выразите в милливольтах и округлите до целого, единицы измерения писать не нужно.
}
\answer{%
    $\ele = 11{,}538\,\text{мВ} \to 12$
}
\solutionspace{60pt}

\tasknumber{7}%
\task{%
    Определите магнитный поток через контур,
    находящийся в однородном магнитном поле индукцией $300\,\text{мТл}$.
    Контур имеет форму прямоугольника со сторонами $60\,\text{см}$ и $45\,\text{см}$.
    Угол между плоскостью контура и вектором индукции магнитного поля
    составляет $40\degrees$.
    Ответ выразите в милливеберах и округлите до целого, единицы измерения писать не нужно.
}
\answer{%
    $\alpha=50\degrees, \Phi_B = BS\cos\alpha = 52{,}07\,\text{мВб} \to 52$
}
\solutionspace{100pt}

\tasknumber{8}%
\task{%
    Какой средний индукционный ток возник в плоском контуре площадью $250\,\text{см}^{2}$
    и сопротивлением $2{,}5\,\text{Ом}$, если сперва он располагался перпендикулярно линиям индукции магнитного поля,
    а затем его за $0{,}2\,\text{мc}$ повернули, и теперь угол между нормалью к плоскости контура и индукцией магнитного поля
    равен $50\degrees$.
    Магнитное поле однородно, его индукция равна $70\,\text{мТл}$.
    Ответ выразите в микроамперах и округлите до целого, единицы измерения писать не нужно.
}
\answer{%
    \begin{align*}
    \alpha_1 &= 0\degrees, \alpha_2 = 50\degrees, \\
    \ele &= \abs{\frac{\Delta \Phi}{\Delta t}}, \\
    \eli &= \frac \ele R = \frac{\Delta \Phi}{R\Delta t}= \frac{ \abs{ B S \cos\alpha_2 - B S \cos\alpha_1 } }{R\Delta t}= \frac{ B S \abs{ \cos\alpha_2 -  \cos\alpha_1 } }{R\Delta t} =  \\
    &= \frac{ 70\,\text{мТл} \cdot 250\,\text{см}^{2} \abs{ \cos50\degrees -  \cos0\degrees } }{2{,}5\,\text{Ом} \cdot 0{,}2\,\text{мc}}\approx 1250{,}243\,\text{мкА} \to 1250
    \end{align*}
}

\variantsplitter

\addpersonalvariant{София Журавлёва}

\tasknumber{1}%
\task{%
    Установите каждой букве в соответствие ровно одну цифру и запишите ответ (только цифры, без других символов).

    А) ЭДС индукции, Б) магнитный поток.

    1) $S$, 2) $B$, 3) $\ele$, 4) $\Phi$.
}
\answer{%
    $34$
}
\solutionspace{20pt}

\tasknumber{2}%
\task{%
    Установите каждой букве в соответствие ровно одну цифру и запишите ответ (только цифры, без других символов).

    А) площадь контура, Б) ЭДС индукции, В) индукция магнитного поля.

    1) Тл, 2) Ом, 3) Вб, 4) $\text{м}^2$, 5) В.
}
\answer{%
    $451$
}
\solutionspace{20pt}

\tasknumber{3}%
\task{%
    Однородное магнитное поле пронизывает плоский контур площадью $400\,\text{см}^{2}$.
    Индукция магнитного поля равна $500\,\text{мТл}$.
    Чему равен магнитный поток через контур, если его плоскость
    расположена под углом $0\degrees$ к вектору магнитной индукции?
    Ответ выразите в милливеберах и округлите до целого, единицы измерения писать не нужно.
}
\answer{%
    $\alpha = 90\degrees, \Phi_B = BS\cos\alpha = 0\,\text{мВб} \to 0$
}
\solutionspace{100pt}

\tasknumber{4}%
\task{%
    Определите притягивается (А), не взаимодействует (Б) или отталкивается (В) металлическое кольцо к магниту,
    если выдвигать северным полюсом (см.
    рис).
}
\answer{%
    \text{А}
}

\tasknumber{5}%
\task{%
    Определите притягивается (А), не взаимодействует (Б) или отталкивается (В) кольцо из диэлектрика к магниту,
    если вдвигать магнит в кольцо южным полюсом (см.
    рис).
}
\answer{%
    \text{Б}
}

\tasknumber{6}%
\task{%
    Магнитный поток, пронизывающий замкнутый контур, равномерно изменяется от $65\,\text{мВб}$ до $110\,\text{мВб}$ за $1{,}3\,\text{c}$.
    Чему равна ЭДС в контуре? Ответ выразите в милливольтах и округлите до целого, единицы измерения писать не нужно.
}
\answer{%
    $\ele = 34{,}615\,\text{мВ} \to 35$
}
\solutionspace{60pt}

\tasknumber{7}%
\task{%
    Определите магнитный поток через контур,
    находящийся в однородном магнитном поле индукцией $500\,\text{мТл}$.
    Контур имеет форму прямоугольника со сторонами $60\,\text{см}$ и $45\,\text{см}$.
    Угол между нормалью к плоскости контура и вектором индукции магнитного поля
    составляет $70\degrees$.
    Ответ выразите в милливеберах и округлите до целого, единицы измерения писать не нужно.
}
\answer{%
    $\alpha=70\degrees, \Phi_B = BS\cos\alpha = 46{,}17\,\text{мВб} \to 46$
}
\solutionspace{100pt}

\tasknumber{8}%
\task{%
    Какой средний индукционный ток возник в плоском контуре площадью $120\,\text{см}^{2}$
    и сопротивлением $3\,\text{Ом}$, если сперва он располагался параллельной линиям индукции магнитного поля,
    а затем его за $0{,}2\,\text{мc}$ повернули, и теперь угол между нормалью к плоскости контура и индукцией магнитного поля
    равен $20\degrees$.
    Магнитное поле однородно, его индукция равна $50\,\text{мТл}$.
    Ответ выразите в микроамперах и округлите до целого, единицы измерения писать не нужно.
}
\answer{%
    \begin{align*}
    \alpha_1 &= 90\degrees, \alpha_2 = 20\degrees, \\
    \ele &= \abs{\frac{\Delta \Phi}{\Delta t}}, \\
    \eli &= \frac \ele R = \frac{\Delta \Phi}{R\Delta t}= \frac{ \abs{ B S \cos\alpha_2 - B S \cos\alpha_1 } }{R\Delta t}= \frac{ B S \abs{ \cos\alpha_2 -  \cos\alpha_1 } }{R\Delta t} =  \\
    &= \frac{ 50\,\text{мТл} \cdot 120\,\text{см}^{2} \abs{ \cos20\degrees -  \cos90\degrees } }{3\,\text{Ом} \cdot 0{,}2\,\text{мc}}\approx 939{,}693\,\text{мкА} \to 940
    \end{align*}
}

\variantsplitter

\addpersonalvariant{Константин Козлов}

\tasknumber{1}%
\task{%
    Установите каждой букве в соответствие ровно одну цифру и запишите ответ (только цифры, без других символов).

    А) индукция магнитного поля, Б) ЭДС индукции.

    1) $B$, 2) $l$, 3) $S$, 4) $\ele$.
}
\answer{%
    $14$
}
\solutionspace{20pt}

\tasknumber{2}%
\task{%
    Установите каждой букве в соответствие ровно одну цифру и запишите ответ (только цифры, без других символов).

    А) ЭДС индукции, Б) индукционый ток, В) площадь контура.

    1) Гн, 2) В, 3) Вб, 4) $\text{м}^2$, 5) А.
}
\answer{%
    $254$
}
\solutionspace{20pt}

\tasknumber{3}%
\task{%
    Однородное магнитное поле пронизывает плоский контур площадью $800\,\text{см}^{2}$.
    Индукция магнитного поля равна $300\,\text{мТл}$.
    Чему равен магнитный поток через контур, если его плоскость
    расположена под углом $0\degrees$ к вектору магнитной индукции?
    Ответ выразите в милливеберах и округлите до целого, единицы измерения писать не нужно.
}
\answer{%
    $\alpha = 90\degrees, \Phi_B = BS\cos\alpha = 0\,\text{мВб} \to 0$
}
\solutionspace{100pt}

\tasknumber{4}%
\task{%
    Определите притягивается (А), не взаимодействует (Б) или отталкивается (В) металлическое кольцо к магниту,
    если выдвигать южным полюсом (см.
    рис).
}
\answer{%
    \text{А}
}

\tasknumber{5}%
\task{%
    Определите притягивается (А), не взаимодействует (Б) или отталкивается (В) кольцо из диэлектрика к магниту,
    если вдвигать магнит в кольцо северным полюсом (см.
    рис).
}
\answer{%
    \text{Б}
}

\tasknumber{6}%
\task{%
    Магнитный поток, пронизывающий замкнутый контур, равномерно изменяется от $65\,\text{мВб}$ до $80\,\text{мВб}$ за $1{,}1\,\text{c}$.
    Чему равна ЭДС в контуре? Ответ выразите в милливольтах и округлите до целого, единицы измерения писать не нужно.
}
\answer{%
    $\ele = 13{,}636\,\text{мВ} \to 14$
}
\solutionspace{60pt}

\tasknumber{7}%
\task{%
    Определите магнитный поток через контур,
    находящийся в однородном магнитном поле индукцией $500\,\text{мТл}$.
    Контур имеет форму прямоугольника со сторонами $60\,\text{см}$ и $45\,\text{см}$.
    Угол между нормалью к плоскости контура и вектором индукции магнитного поля
    составляет $80\degrees$.
    Ответ выразите в милливеберах и округлите до целого, единицы измерения писать не нужно.
}
\answer{%
    $\alpha=80\degrees, \Phi_B = BS\cos\alpha = 23{,}44\,\text{мВб} \to 23$
}
\solutionspace{100pt}

\tasknumber{8}%
\task{%
    Какой средний индукционный ток возник в плоском контуре площадью $120\,\text{см}^{2}$
    и сопротивлением $2\,\text{Ом}$, если сперва он располагался перпендикулярно линиям индукции магнитного поля,
    а затем его за $0{,}5\,\text{мc}$ повернули, и теперь угол между плоскостью контура и индукцией магнитного поля
    равен $10\degrees$.
    Магнитное поле однородно, его индукция равна $50\,\text{мТл}$.
    Ответ выразите в микроамперах и округлите до целого, единицы измерения писать не нужно.
}
\answer{%
    \begin{align*}
    \alpha_1 &= 0\degrees, \alpha_2 = 80\degrees, \\
    \ele &= \abs{\frac{\Delta \Phi}{\Delta t}}, \\
    \eli &= \frac \ele R = \frac{\Delta \Phi}{R\Delta t}= \frac{ \abs{ B S \cos\alpha_2 - B S \cos\alpha_1 } }{R\Delta t}= \frac{ B S \abs{ \cos\alpha_2 -  \cos\alpha_1 } }{R\Delta t} =  \\
    &= \frac{ 50\,\text{мТл} \cdot 120\,\text{см}^{2} \abs{ \cos80\degrees -  \cos0\degrees } }{2\,\text{Ом} \cdot 0{,}5\,\text{мc}}\approx 495{,}811\,\text{мкА} \to 496
    \end{align*}
}

\variantsplitter

\addpersonalvariant{Наталья Кравченко}

\tasknumber{1}%
\task{%
    Установите каждой букве в соответствие ровно одну цифру и запишите ответ (только цифры, без других символов).

    А) сопротивление контура, Б) площадь контура.

    1) $\Phi$, 2) $S$, 3) $R$, 4) $\vec n$.
}
\answer{%
    $32$
}
\solutionspace{20pt}

\tasknumber{2}%
\task{%
    Установите каждой букве в соответствие ровно одну цифру и запишите ответ (только цифры, без других символов).

    А) площадь контура, Б) индукционый ток, В) ЭДС индукции.

    1) $\text{м}^2$, 2) Гн, 3) В, 4) А, 5) Кл.
}
\answer{%
    $143$
}
\solutionspace{20pt}

\tasknumber{3}%
\task{%
    Однородное магнитное поле пронизывает плоский контур площадью $800\,\text{см}^{2}$.
    Индукция магнитного поля равна $300\,\text{мТл}$.
    Чему равен магнитный поток через контур, если его плоскость
    расположена под углом $90\degrees$ к вектору магнитной индукции?
    Ответ выразите в милливеберах и округлите до целого, единицы измерения писать не нужно.
}
\answer{%
    $\alpha = 0\degrees, \Phi_B = BS\cos\alpha = 24{,}00\,\text{мВб} \to 24$
}
\solutionspace{100pt}

\tasknumber{4}%
\task{%
    Определите притягивается (А), не взаимодействует (Б) или отталкивается (В) металлическое кольцо к магниту,
    если выдвигать южным полюсом (см.
    рис).
}
\answer{%
    \text{А}
}

\tasknumber{5}%
\task{%
    Определите притягивается (А), не взаимодействует (Б) или отталкивается (В) кольцо из диэлектрика к магниту,
    если выдвигать магнит из кольца южным полюсом (см.
    рис).
}
\answer{%
    \text{Б}
}

\tasknumber{6}%
\task{%
    Магнитный поток, пронизывающий замкнутый контур, равномерно изменяется от $35\,\text{мВб}$ до $50\,\text{мВб}$ за $1{,}1\,\text{c}$.
    Чему равна ЭДС в контуре? Ответ выразите в милливольтах и округлите до целого, единицы измерения писать не нужно.
}
\answer{%
    $\ele = 13{,}636\,\text{мВ} \to 14$
}
\solutionspace{60pt}

\tasknumber{7}%
\task{%
    Определите магнитный поток через контур,
    находящийся в однородном магнитном поле индукцией $300\,\text{мТл}$.
    Контур имеет форму прямоугольного треугольника с катетами $50\,\text{см}$ и $80\,\text{см}$.
    Угол между плоскостью контура и вектором индукции магнитного поля
    составляет $10\degrees$.
    Ответ выразите в милливеберах и округлите до целого, единицы измерения писать не нужно.
}
\answer{%
    $\alpha=80\degrees, \Phi_B = BS\cos\alpha = 10{,}42\,\text{мВб} \to 10$
}
\solutionspace{100pt}

\tasknumber{8}%
\task{%
    Какой средний индукционный ток возник в плоском контуре площадью $150\,\text{см}^{2}$
    и сопротивлением $2{,}5\,\text{Ом}$, если сперва он располагался параллельной линиям индукции магнитного поля,
    а затем его за $0{,}2\,\text{мc}$ повернули, и теперь угол между нормалью к плоскости контура и индукцией магнитного поля
    равен $50\degrees$.
    Магнитное поле однородно, его индукция равна $70\,\text{мТл}$.
    Ответ выразите в микроамперах и округлите до целого, единицы измерения писать не нужно.
}
\answer{%
    \begin{align*}
    \alpha_1 &= 90\degrees, \alpha_2 = 50\degrees, \\
    \ele &= \abs{\frac{\Delta \Phi}{\Delta t}}, \\
    \eli &= \frac \ele R = \frac{\Delta \Phi}{R\Delta t}= \frac{ \abs{ B S \cos\alpha_2 - B S \cos\alpha_1 } }{R\Delta t}= \frac{ B S \abs{ \cos\alpha_2 -  \cos\alpha_1 } }{R\Delta t} =  \\
    &= \frac{ 70\,\text{мТл} \cdot 150\,\text{см}^{2} \abs{ \cos50\degrees -  \cos90\degrees } }{2{,}5\,\text{Ом} \cdot 0{,}2\,\text{мc}}\approx 1349{,}854\,\text{мкА} \to 1350
    \end{align*}
}

\variantsplitter

\addpersonalvariant{Сергей Малышев}

\tasknumber{1}%
\task{%
    Установите каждой букве в соответствие ровно одну цифру и запишите ответ (только цифры, без других символов).

    А) магнитный поток, Б) индукционый ток.

    1) $\eli$, 2) $v$, 3) $\Phi$, 4) $R$.
}
\answer{%
    $31$
}
\solutionspace{20pt}

\tasknumber{2}%
\task{%
    Установите каждой букве в соответствие ровно одну цифру и запишите ответ (только цифры, без других символов).

    А) площадь контура, Б) индукционый ток, В) магнитный поток.

    1) Вб, 2) А, 3) Гн, 4) $\text{м}^2$, 5) Ом.
}
\answer{%
    $421$
}
\solutionspace{20pt}

\tasknumber{3}%
\task{%
    Однородное магнитное поле пронизывает плоский контур площадью $200\,\text{см}^{2}$.
    Индукция магнитного поля равна $500\,\text{мТл}$.
    Чему равен магнитный поток через контур, если его плоскость
    расположена под углом $0\degrees$ к вектору магнитной индукции?
    Ответ выразите в милливеберах и округлите до целого, единицы измерения писать не нужно.
}
\answer{%
    $\alpha = 90\degrees, \Phi_B = BS\cos\alpha = 0\,\text{мВб} \to 0$
}
\solutionspace{100pt}

\tasknumber{4}%
\task{%
    Определите притягивается (А), не взаимодействует (Б) или отталкивается (В) металлическое кольцо к магниту,
    если вдвигать южным полюсом (см.
    рис).
}
\answer{%
    \text{В}
}

\tasknumber{5}%
\task{%
    Определите притягивается (А), не взаимодействует (Б) или отталкивается (В) кольцо из диэлектрика к магниту,
    если вдвигать магнит в кольцо северным полюсом (см.
    рис).
}
\answer{%
    \text{Б}
}

\tasknumber{6}%
\task{%
    Магнитный поток, пронизывающий замкнутый контур, равномерно изменяется от $65\,\text{мВб}$ до $50\,\text{мВб}$ за $1{,}1\,\text{c}$.
    Чему равна ЭДС в контуре? Ответ выразите в милливольтах и округлите до целого, единицы измерения писать не нужно.
}
\answer{%
    $\ele = 13{,}636\,\text{мВ} \to 14$
}
\solutionspace{60pt}

\tasknumber{7}%
\task{%
    Определите магнитный поток через контур,
    находящийся в однородном магнитном поле индукцией $700\,\text{мТл}$.
    Контур имеет форму прямоугольного треугольника с катетами $50\,\text{см}$ и $80\,\text{см}$.
    Угол между плоскостью контура и вектором индукции магнитного поля
    составляет $40\degrees$.
    Ответ выразите в милливеберах и округлите до целого, единицы измерения писать не нужно.
}
\answer{%
    $\alpha=50\degrees, \Phi_B = BS\cos\alpha = 89{,}99\,\text{мВб} \to 90$
}
\solutionspace{100pt}

\tasknumber{8}%
\task{%
    Какой средний индукционный ток возник в плоском контуре площадью $150\,\text{см}^{2}$
    и сопротивлением $2{,}5\,\text{Ом}$, если сперва он располагался перпендикулярно линиям индукции магнитного поля,
    а затем его за $0{,}6\,\text{мc}$ повернули, и теперь угол между плоскостью контура и индукцией магнитного поля
    равен $70\degrees$.
    Магнитное поле однородно, его индукция равна $80\,\text{мТл}$.
    Ответ выразите в микроамперах и округлите до целого, единицы измерения писать не нужно.
}
\answer{%
    \begin{align*}
    \alpha_1 &= 0\degrees, \alpha_2 = 20\degrees, \\
    \ele &= \abs{\frac{\Delta \Phi}{\Delta t}}, \\
    \eli &= \frac \ele R = \frac{\Delta \Phi}{R\Delta t}= \frac{ \abs{ B S \cos\alpha_2 - B S \cos\alpha_1 } }{R\Delta t}= \frac{ B S \abs{ \cos\alpha_2 -  \cos\alpha_1 } }{R\Delta t} =  \\
    &= \frac{ 80\,\text{мТл} \cdot 150\,\text{см}^{2} \abs{ \cos20\degrees -  \cos0\degrees } }{2{,}5\,\text{Ом} \cdot 0{,}6\,\text{мc}}\approx 48{,}246\,\text{мкА} \to 48
    \end{align*}
}

\variantsplitter

\addpersonalvariant{Алина Полканова}

\tasknumber{1}%
\task{%
    Установите каждой букве в соответствие ровно одну цифру и запишите ответ (только цифры, без других символов).

    А) ЭДС индукции, Б) вектор нормали к поверхности.

    1) $\ele$, 2) $D$, 3) $\vec n$, 4) $\Phi$.
}
\answer{%
    $13$
}
\solutionspace{20pt}

\tasknumber{2}%
\task{%
    Установите каждой букве в соответствие ровно одну цифру и запишите ответ (только цифры, без других символов).

    А) индукция магнитного поля, Б) ЭДС индукции, В) магнитный поток.

    1) Тл, 2) В, 3) Гн, 4) Вб, 5) $\text{м}^2$.
}
\answer{%
    $124$
}
\solutionspace{20pt}

\tasknumber{3}%
\task{%
    Однородное магнитное поле пронизывает плоский контур площадью $600\,\text{см}^{2}$.
    Индукция магнитного поля равна $500\,\text{мТл}$.
    Чему равен магнитный поток через контур, если его плоскость
    расположена под углом $0\degrees$ к вектору магнитной индукции?
    Ответ выразите в милливеберах и округлите до целого, единицы измерения писать не нужно.
}
\answer{%
    $\alpha = 90\degrees, \Phi_B = BS\cos\alpha = 0\,\text{мВб} \to 0$
}
\solutionspace{100pt}

\tasknumber{4}%
\task{%
    Определите притягивается (А), не взаимодействует (Б) или отталкивается (В) металлическое кольцо к магниту,
    если выдвигать северным полюсом (см.
    рис).
}
\answer{%
    \text{А}
}

\tasknumber{5}%
\task{%
    Определите притягивается (А), не взаимодействует (Б) или отталкивается (В) кольцо из диэлектрика к магниту,
    если вдвигать магнит в кольцо южным полюсом (см.
    рис).
}
\answer{%
    \text{Б}
}

\tasknumber{6}%
\task{%
    Магнитный поток, пронизывающий замкнутый контур, равномерно изменяется от $35\,\text{мВб}$ до $80\,\text{мВб}$ за $1{,}3\,\text{c}$.
    Чему равна ЭДС в контуре? Ответ выразите в милливольтах и округлите до целого, единицы измерения писать не нужно.
}
\answer{%
    $\ele = 34{,}615\,\text{мВ} \to 35$
}
\solutionspace{60pt}

\tasknumber{7}%
\task{%
    Определите магнитный поток через контур,
    находящийся в однородном магнитном поле индукцией $300\,\text{мТл}$.
    Контур имеет форму прямоугольного треугольника с катетами $50\,\text{см}$ и $80\,\text{см}$.
    Угол между плоскостью контура и вектором индукции магнитного поля
    составляет $80\degrees$.
    Ответ выразите в милливеберах и округлите до целого, единицы измерения писать не нужно.
}
\answer{%
    $\alpha=10\degrees, \Phi_B = BS\cos\alpha = 59{,}09\,\text{мВб} \to 59$
}
\solutionspace{100pt}

\tasknumber{8}%
\task{%
    Какой средний индукционный ток возник в плоском контуре площадью $250\,\text{см}^{2}$
    и сопротивлением $1{,}5\,\text{Ом}$, если сперва он располагался перпендикулярно линиям индукции магнитного поля,
    а затем его за $0{,}2\,\text{мc}$ повернули, и теперь угол между плоскостью контура и индукцией магнитного поля
    равен $70\degrees$.
    Магнитное поле однородно, его индукция равна $50\,\text{мТл}$.
    Ответ выразите в микроамперах и округлите до целого, единицы измерения писать не нужно.
}
\answer{%
    \begin{align*}
    \alpha_1 &= 0\degrees, \alpha_2 = 20\degrees, \\
    \ele &= \abs{\frac{\Delta \Phi}{\Delta t}}, \\
    \eli &= \frac \ele R = \frac{\Delta \Phi}{R\Delta t}= \frac{ \abs{ B S \cos\alpha_2 - B S \cos\alpha_1 } }{R\Delta t}= \frac{ B S \abs{ \cos\alpha_2 -  \cos\alpha_1 } }{R\Delta t} =  \\
    &= \frac{ 50\,\text{мТл} \cdot 250\,\text{см}^{2} \abs{ \cos20\degrees -  \cos0\degrees } }{1{,}5\,\text{Ом} \cdot 0{,}2\,\text{мc}}\approx 251{,}281\,\text{мкА} \to 251
    \end{align*}
}

\variantsplitter

\addpersonalvariant{Сергей Пономарёв}

\tasknumber{1}%
\task{%
    Установите каждой букве в соответствие ровно одну цифру и запишите ответ (только цифры, без других символов).

    А) индукционый ток, Б) индукция магнитного поля.

    1) $\Phi$, 2) $\eli$, 3) $B$, 4) $R$.
}
\answer{%
    $23$
}
\solutionspace{20pt}

\tasknumber{2}%
\task{%
    Установите каждой букве в соответствие ровно одну цифру и запишите ответ (только цифры, без других символов).

    А) площадь контура, Б) индукция магнитного поля, В) индукционый ток.

    1) Тл, 2) Ом, 3) $\text{м}^2$, 4) А, 5) Гн.
}
\answer{%
    $314$
}
\solutionspace{20pt}

\tasknumber{3}%
\task{%
    Однородное магнитное поле пронизывает плоский контур площадью $800\,\text{см}^{2}$.
    Индукция магнитного поля равна $300\,\text{мТл}$.
    Чему равен магнитный поток через контур, если его плоскость
    расположена под углом $90\degrees$ к вектору магнитной индукции?
    Ответ выразите в милливеберах и округлите до целого, единицы измерения писать не нужно.
}
\answer{%
    $\alpha = 0\degrees, \Phi_B = BS\cos\alpha = 24{,}00\,\text{мВб} \to 24$
}
\solutionspace{100pt}

\tasknumber{4}%
\task{%
    Определите притягивается (А), не взаимодействует (Б) или отталкивается (В) металлическое кольцо к магниту,
    если вдвигать южным полюсом (см.
    рис).
}
\answer{%
    \text{В}
}

\tasknumber{5}%
\task{%
    Определите притягивается (А), не взаимодействует (Б) или отталкивается (В) кольцо из диэлектрика к магниту,
    если вдвигать магнит в кольцо северным полюсом (см.
    рис).
}
\answer{%
    \text{Б}
}

\tasknumber{6}%
\task{%
    Магнитный поток, пронизывающий замкнутый контур, равномерно изменяется от $65\,\text{мВб}$ до $110\,\text{мВб}$ за $1{,}1\,\text{c}$.
    Чему равна ЭДС в контуре? Ответ выразите в милливольтах и округлите до целого, единицы измерения писать не нужно.
}
\answer{%
    $\ele = 40{,}909\,\text{мВ} \to 41$
}
\solutionspace{60pt}

\tasknumber{7}%
\task{%
    Определите магнитный поток через контур,
    находящийся в однородном магнитном поле индукцией $300\,\text{мТл}$.
    Контур имеет форму прямоугольника со сторонами $50\,\text{см}$ и $75\,\text{см}$.
    Угол между нормалью к плоскости контура и вектором индукции магнитного поля
    составляет $80\degrees$.
    Ответ выразите в милливеберах и округлите до целого, единицы измерения писать не нужно.
}
\answer{%
    $\alpha=80\degrees, \Phi_B = BS\cos\alpha = 19{,}54\,\text{мВб} \to 20$
}
\solutionspace{100pt}

\tasknumber{8}%
\task{%
    Какой средний индукционный ток возник в плоском контуре площадью $250\,\text{см}^{2}$
    и сопротивлением $2\,\text{Ом}$, если сперва он располагался параллельной линиям индукции магнитного поля,
    а затем его за $0{,}2\,\text{мc}$ повернули, и теперь угол между плоскостью контура и индукцией магнитного поля
    равен $70\degrees$.
    Магнитное поле однородно, его индукция равна $80\,\text{мТл}$.
    Ответ выразите в микроамперах и округлите до целого, единицы измерения писать не нужно.
}
\answer{%
    \begin{align*}
    \alpha_1 &= 90\degrees, \alpha_2 = 20\degrees, \\
    \ele &= \abs{\frac{\Delta \Phi}{\Delta t}}, \\
    \eli &= \frac \ele R = \frac{\Delta \Phi}{R\Delta t}= \frac{ \abs{ B S \cos\alpha_2 - B S \cos\alpha_1 } }{R\Delta t}= \frac{ B S \abs{ \cos\alpha_2 -  \cos\alpha_1 } }{R\Delta t} =  \\
    &= \frac{ 80\,\text{мТл} \cdot 250\,\text{см}^{2} \abs{ \cos20\degrees -  \cos90\degrees } }{2\,\text{Ом} \cdot 0{,}2\,\text{мc}}\approx 4698{,}463\,\text{мкА} \to 4698
    \end{align*}
}

\variantsplitter

\addpersonalvariant{Егор Свистушкин}

\tasknumber{1}%
\task{%
    Установите каждой букве в соответствие ровно одну цифру и запишите ответ (только цифры, без других символов).

    А) сопротивление контура, Б) площадь контура.

    1) $l$, 2) $S$, 3) $R$, 4) $\ele$.
}
\answer{%
    $32$
}
\solutionspace{20pt}

\tasknumber{2}%
\task{%
    Установите каждой букве в соответствие ровно одну цифру и запишите ответ (только цифры, без других символов).

    А) площадь контура, Б) индукционый ток, В) индукция магнитного поля.

    1) Ом, 2) В, 3) А, 4) $\text{м}^2$, 5) Тл.
}
\answer{%
    $435$
}
\solutionspace{20pt}

\tasknumber{3}%
\task{%
    Однородное магнитное поле пронизывает плоский контур площадью $600\,\text{см}^{2}$.
    Индукция магнитного поля равна $700\,\text{мТл}$.
    Чему равен магнитный поток через контур, если его плоскость
    расположена под углом $90\degrees$ к вектору магнитной индукции?
    Ответ выразите в милливеберах и округлите до целого, единицы измерения писать не нужно.
}
\answer{%
    $\alpha = 0\degrees, \Phi_B = BS\cos\alpha = 42{,}00\,\text{мВб} \to 42$
}
\solutionspace{100pt}

\tasknumber{4}%
\task{%
    Определите притягивается (А), не взаимодействует (Б) или отталкивается (В) металлическое кольцо к магниту,
    если выдвигать южным полюсом (см.
    рис).
}
\answer{%
    \text{А}
}

\tasknumber{5}%
\task{%
    Определите притягивается (А), не взаимодействует (Б) или отталкивается (В) кольцо из диэлектрика к магниту,
    если вдвигать магнит в кольцо южным полюсом (см.
    рис).
}
\answer{%
    \text{Б}
}

\tasknumber{6}%
\task{%
    Магнитный поток, пронизывающий замкнутый контур, равномерно изменяется от $65\,\text{мВб}$ до $80\,\text{мВб}$ за $1{,}5\,\text{c}$.
    Чему равна ЭДС в контуре? Ответ выразите в милливольтах и округлите до целого, единицы измерения писать не нужно.
}
\answer{%
    $\ele = 10{,}000\,\text{мВ} \to 10$
}
\solutionspace{60pt}

\tasknumber{7}%
\task{%
    Определите магнитный поток через контур,
    находящийся в однородном магнитном поле индукцией $300\,\text{мТл}$.
    Контур имеет форму прямоугольного треугольника с катетами $60\,\text{см}$ и $80\,\text{см}$.
    Угол между плоскостью контура и вектором индукции магнитного поля
    составляет $70\degrees$.
    Ответ выразите в милливеберах и округлите до целого, единицы измерения писать не нужно.
}
\answer{%
    $\alpha=20\degrees, \Phi_B = BS\cos\alpha = 67{,}66\,\text{мВб} \to 68$
}
\solutionspace{100pt}

\tasknumber{8}%
\task{%
    Какой средний индукционный ток возник в плоском контуре площадью $150\,\text{см}^{2}$
    и сопротивлением $2\,\text{Ом}$, если сперва он располагался перпендикулярно линиям индукции магнитного поля,
    а затем его за $0{,}8\,\text{мc}$ повернули, и теперь угол между нормалью к плоскости контура и индукцией магнитного поля
    равен $70\degrees$.
    Магнитное поле однородно, его индукция равна $50\,\text{мТл}$.
    Ответ выразите в микроамперах и округлите до целого, единицы измерения писать не нужно.
}
\answer{%
    \begin{align*}
    \alpha_1 &= 0\degrees, \alpha_2 = 70\degrees, \\
    \ele &= \abs{\frac{\Delta \Phi}{\Delta t}}, \\
    \eli &= \frac \ele R = \frac{\Delta \Phi}{R\Delta t}= \frac{ \abs{ B S \cos\alpha_2 - B S \cos\alpha_1 } }{R\Delta t}= \frac{ B S \abs{ \cos\alpha_2 -  \cos\alpha_1 } }{R\Delta t} =  \\
    &= \frac{ 50\,\text{мТл} \cdot 150\,\text{см}^{2} \abs{ \cos70\degrees -  \cos0\degrees } }{2\,\text{Ом} \cdot 0{,}8\,\text{мc}}\approx 308{,}428\,\text{мкА} \to 308
    \end{align*}
}

\variantsplitter

\addpersonalvariant{Дмитрий Соколов}

\tasknumber{1}%
\task{%
    Установите каждой букве в соответствие ровно одну цифру и запишите ответ (только цифры, без других символов).

    А) сопротивление контура, Б) индукционый ток.

    1) $\eli$, 2) $R$, 3) $U$, 4) $l$.
}
\answer{%
    $21$
}
\solutionspace{20pt}

\tasknumber{2}%
\task{%
    Установите каждой букве в соответствие ровно одну цифру и запишите ответ (только цифры, без других символов).

    А) ЭДС индукции, Б) индукционый ток, В) площадь контура.

    1) Ом, 2) А, 3) В, 4) $\text{м}^2$, 5) Гн.
}
\answer{%
    $324$
}
\solutionspace{20pt}

\tasknumber{3}%
\task{%
    Однородное магнитное поле пронизывает плоский контур площадью $600\,\text{см}^{2}$.
    Индукция магнитного поля равна $700\,\text{мТл}$.
    Чему равен магнитный поток через контур, если его плоскость
    расположена под углом $60\degrees$ к вектору магнитной индукции?
    Ответ выразите в милливеберах и округлите до целого, единицы измерения писать не нужно.
}
\answer{%
    $\alpha = 30\degrees, \Phi_B = BS\cos\alpha = 36{,}37\,\text{мВб} \to 36$
}
\solutionspace{100pt}

\tasknumber{4}%
\task{%
    Определите притягивается (А), не взаимодействует (Б) или отталкивается (В) металлическое кольцо к магниту,
    если вдвигать северным полюсом (см.
    рис).
}
\answer{%
    \text{В}
}

\tasknumber{5}%
\task{%
    Определите притягивается (А), не взаимодействует (Б) или отталкивается (В) кольцо из диэлектрика к магниту,
    если выдвигать магнит из кольца северным полюсом (см.
    рис).
}
\answer{%
    \text{Б}
}

\tasknumber{6}%
\task{%
    Магнитный поток, пронизывающий замкнутый контур, равномерно изменяется от $35\,\text{мВб}$ до $80\,\text{мВб}$ за $1{,}1\,\text{c}$.
    Чему равна ЭДС в контуре? Ответ выразите в милливольтах и округлите до целого, единицы измерения писать не нужно.
}
\answer{%
    $\ele = 40{,}909\,\text{мВ} \to 41$
}
\solutionspace{60pt}

\tasknumber{7}%
\task{%
    Определите магнитный поток через контур,
    находящийся в однородном магнитном поле индукцией $500\,\text{мТл}$.
    Контур имеет форму прямоугольного треугольника с катетами $50\,\text{см}$ и $80\,\text{см}$.
    Угол между плоскостью контура и вектором индукции магнитного поля
    составляет $50\degrees$.
    Ответ выразите в милливеберах и округлите до целого, единицы измерения писать не нужно.
}
\answer{%
    $\alpha=40\degrees, \Phi_B = BS\cos\alpha = 76{,}60\,\text{мВб} \to 77$
}
\solutionspace{100pt}

\tasknumber{8}%
\task{%
    Какой средний индукционный ток возник в плоском контуре площадью $150\,\text{см}^{2}$
    и сопротивлением $3\,\text{Ом}$, если сперва он располагался перпендикулярно линиям индукции магнитного поля,
    а затем его за $0{,}6\,\text{мc}$ повернули, и теперь угол между нормалью к плоскости контура и индукцией магнитного поля
    равен $50\degrees$.
    Магнитное поле однородно, его индукция равна $80\,\text{мТл}$.
    Ответ выразите в микроамперах и округлите до целого, единицы измерения писать не нужно.
}
\answer{%
    \begin{align*}
    \alpha_1 &= 0\degrees, \alpha_2 = 50\degrees, \\
    \ele &= \abs{\frac{\Delta \Phi}{\Delta t}}, \\
    \eli &= \frac \ele R = \frac{\Delta \Phi}{R\Delta t}= \frac{ \abs{ B S \cos\alpha_2 - B S \cos\alpha_1 } }{R\Delta t}= \frac{ B S \abs{ \cos\alpha_2 -  \cos\alpha_1 } }{R\Delta t} =  \\
    &= \frac{ 80\,\text{мТл} \cdot 150\,\text{см}^{2} \abs{ \cos50\degrees -  \cos0\degrees } }{3\,\text{Ом} \cdot 0{,}6\,\text{мc}}\approx 238{,}142\,\text{мкА} \to 238
    \end{align*}
}

\variantsplitter

\addpersonalvariant{Арсений Трофимов}

\tasknumber{1}%
\task{%
    Установите каждой букве в соответствие ровно одну цифру и запишите ответ (только цифры, без других символов).

    А) вектор нормали к поверхности, Б) индукция магнитного поля.

    1) $R$, 2) $\eli$, 3) $B$, 4) $\vec n$.
}
\answer{%
    $43$
}
\solutionspace{20pt}

\tasknumber{2}%
\task{%
    Установите каждой букве в соответствие ровно одну цифру и запишите ответ (только цифры, без других символов).

    А) магнитный поток, Б) ЭДС индукции, В) площадь контура.

    1) Вб, 2) Ом, 3) Кл, 4) $\text{м}^2$, 5) В.
}
\answer{%
    $154$
}
\solutionspace{20pt}

\tasknumber{3}%
\task{%
    Однородное магнитное поле пронизывает плоский контур площадью $200\,\text{см}^{2}$.
    Индукция магнитного поля равна $300\,\text{мТл}$.
    Чему равен магнитный поток через контур, если его плоскость
    расположена под углом $60\degrees$ к вектору магнитной индукции?
    Ответ выразите в милливеберах и округлите до целого, единицы измерения писать не нужно.
}
\answer{%
    $\alpha = 30\degrees, \Phi_B = BS\cos\alpha = 5{,}20\,\text{мВб} \to 5$
}
\solutionspace{100pt}

\tasknumber{4}%
\task{%
    Определите притягивается (А), не взаимодействует (Б) или отталкивается (В) металлическое кольцо к магниту,
    если вдвигать северным полюсом (см.
    рис).
}
\answer{%
    \text{В}
}

\tasknumber{5}%
\task{%
    Определите притягивается (А), не взаимодействует (Б) или отталкивается (В) кольцо из диэлектрика к магниту,
    если вдвигать магнит в кольцо южным полюсом (см.
    рис).
}
\answer{%
    \text{Б}
}

\tasknumber{6}%
\task{%
    Магнитный поток, пронизывающий замкнутый контур, равномерно изменяется от $65\,\text{мВб}$ до $20\,\text{мВб}$ за $1{,}3\,\text{c}$.
    Чему равна ЭДС в контуре? Ответ выразите в милливольтах и округлите до целого, единицы измерения писать не нужно.
}
\answer{%
    $\ele = 34{,}615\,\text{мВ} \to 35$
}
\solutionspace{60pt}

\tasknumber{7}%
\task{%
    Определите магнитный поток через контур,
    находящийся в однородном магнитном поле индукцией $500\,\text{мТл}$.
    Контур имеет форму прямоугольного треугольника с катетами $40\,\text{см}$ и $80\,\text{см}$.
    Угол между плоскостью контура и вектором индукции магнитного поля
    составляет $20\degrees$.
    Ответ выразите в милливеберах и округлите до целого, единицы измерения писать не нужно.
}
\answer{%
    $\alpha=70\degrees, \Phi_B = BS\cos\alpha = 27{,}36\,\text{мВб} \to 27$
}
\solutionspace{100pt}

\tasknumber{8}%
\task{%
    Какой средний индукционный ток возник в плоском контуре площадью $250\,\text{см}^{2}$
    и сопротивлением $2{,}5\,\text{Ом}$, если сперва он располагался перпендикулярно линиям индукции магнитного поля,
    а затем его за $0{,}2\,\text{мc}$ повернули, и теперь угол между плоскостью контура и индукцией магнитного поля
    равен $20\degrees$.
    Магнитное поле однородно, его индукция равна $50\,\text{мТл}$.
    Ответ выразите в микроамперах и округлите до целого, единицы измерения писать не нужно.
}
\answer{%
    \begin{align*}
    \alpha_1 &= 0\degrees, \alpha_2 = 70\degrees, \\
    \ele &= \abs{\frac{\Delta \Phi}{\Delta t}}, \\
    \eli &= \frac \ele R = \frac{\Delta \Phi}{R\Delta t}= \frac{ \abs{ B S \cos\alpha_2 - B S \cos\alpha_1 } }{R\Delta t}= \frac{ B S \abs{ \cos\alpha_2 -  \cos\alpha_1 } }{R\Delta t} =  \\
    &= \frac{ 50\,\text{мТл} \cdot 250\,\text{см}^{2} \abs{ \cos70\degrees -  \cos0\degrees } }{2{,}5\,\text{Ом} \cdot 0{,}2\,\text{мc}}\approx 1644{,}950\,\text{мкА} \to 1645
    \end{align*}
}
% autogenerated
