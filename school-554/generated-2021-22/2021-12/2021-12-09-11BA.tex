\setdate{9~декабря~2021}
\setclass{11«БА»}

\addpersonalvariant{Михаил Бурмистров}

\tasknumber{1}%
\task{%
    Сформулируйте:
    \begin{itemize}
        \item принцип Гюйгенса-Френеля,
        \item закон отражения (в двух частях).
    \end{itemize}
}
\solutionspace{60pt}

\tasknumber{2}%
\task{%
    На дне водоема глубиной $3\,\text{м}$ лежит зеркало.
    Луч света, пройдя через воду, отражается от зеркала и выходит из воды.
    Найти расстояние между точкой входа луча в воду и точкой выхода луча из воды,
    если показатель преломления воды $1{,}33$, а угол падения луча $25\degrees$.
}
\answer{%
    \begin{align*}
    \ctg \beta &= \frac{h}{d:L:s} \implies d = \frac{h}{\ctg \beta} \\
    \frac 1{\sin^2 \beta} &= \ctg^2 \beta + 1 \implies \ctg \beta = \sqrt{\frac 1{\sin^2 \beta} - 1} \\
    \sin\alpha &= n\sin \beta \implies \sin \beta = \frac{\sin\alpha}{n} \\
    d &= \frac{h}{\sqrt{\frac 1{\sin^2 \beta} - 1}} = \frac{h}{\sqrt{\sqr{\frac{n}{\sin\alpha}} - 1}} \\
    2d &= \frac{2{h}}{\sqrt{\sqr{\frac{n}{\sin\alpha}} - 1}} \approx 201{,}1\,\text{см}
    \end{align*}
}
\solutionspace{120pt}

\tasknumber{3}%
\task{%
    Луч света падает на вертикально расположенную стеклянную пластинку толщиной $1{,}6\,\text{см}$.
    Пройдя через пластину, он выходит из неё в точке, смещённой по вертикали от точки падения на расстояние $4\,\text{мм}$.
    Показатель преломления стекла $1{,}5$.
    Найти синус угла падения.
}
\answer{%
    \begin{align*}
    \ctg \beta &= \frac{h}{d} \implies \\
    \implies \frac 1{\sin^2 \beta} &= \ctg^2 \beta + 1 = \sqr{\frac{h}{d}} + 1 \implies \\
    \implies \sin\alpha &= n\sin \beta = n\sqrt {\frac 1{\sqr{\frac{h}{d}} + 1}} \approx 0{,}36
    \end{align*}
}
\solutionspace{120pt}

\tasknumber{4}%
\task{%
    Определите диаметр полутени на экране диска размером $D = 3\,\text{см}$ от протяжённого источника, также обладающего формой диска размером $d = 3\,\text{см}$ (см.
    рис.
    на доске, вид сбоку).
    Расстояние от источника до диска равно $l = 12\,\text{см}$, а расстояние от диска до экрана — $L = 30\,\text{см}$.
}
\answer{%
    $\cfrac{\frac d2 + \frac D2}l = \cfrac{\frac d2 + r}{l + L} \implies r = \cfrac{Dl + dL + DL}{2l} = \cfrac D2 + \cfrac{ L }{ l } \cdot \cfrac{d+D}2 \approx 9\,\text{см} \implies 2r \approx 18\,\text{см}$
}
\solutionspace{80pt}

\tasknumber{5}%
\task{%
    Высота солнца над горизонтом составляет $ 35 \degrees$.
    Посреди ровного бетонного поля стоит одинокий цилиндрический столб высотой $3\,\text{м}$ и радиусом $10\,\text{см}$.
    Определите площадь полной тени от этого столба на бетоне.
    Угловой размер Солнца принять равным $30'$ и считать малым углом.
}
\answer{%
    $\text{трапеция} \implies S = 0{,}95\,\text{м}^{2}$
}

\variantsplitter

\addpersonalvariant{Ирина Ан}

\tasknumber{1}%
\task{%
    Сформулируйте:
    \begin{itemize}
        \item принцип Гюйгенса-Френеля,
        \item закон отражения (в двух частях).
    \end{itemize}
}
\solutionspace{60pt}

\tasknumber{2}%
\task{%
    На дне водоема глубиной $3\,\text{м}$ лежит зеркало.
    Луч света, пройдя через воду, отражается от зеркала и выходит из воды.
    Найти расстояние между точкой входа луча в воду и точкой выхода луча из воды,
    если показатель преломления воды $1{,}33$, а угол падения луча $25\degrees$.
}
\answer{%
    \begin{align*}
    \ctg \beta &= \frac{h}{d:L:s} \implies d = \frac{h}{\ctg \beta} \\
    \frac 1{\sin^2 \beta} &= \ctg^2 \beta + 1 \implies \ctg \beta = \sqrt{\frac 1{\sin^2 \beta} - 1} \\
    \sin\alpha &= n\sin \beta \implies \sin \beta = \frac{\sin\alpha}{n} \\
    d &= \frac{h}{\sqrt{\frac 1{\sin^2 \beta} - 1}} = \frac{h}{\sqrt{\sqr{\frac{n}{\sin\alpha}} - 1}} \\
    2d &= \frac{2{h}}{\sqrt{\sqr{\frac{n}{\sin\alpha}} - 1}} \approx 201{,}1\,\text{см}
    \end{align*}
}
\solutionspace{120pt}

\tasknumber{3}%
\task{%
    Луч света падает на горизонтально расположенную стеклянную пластинку толщиной $1{,}6\,\text{см}$.
    Пройдя через пластину, он выходит из неё в точке, смещённой по горизонтали от точки падения на расстояние $5\,\text{мм}$.
    Показатель преломления стекла $1{,}4$.
    Найти синус угла падения.
}
\answer{%
    \begin{align*}
    \ctg \beta &= \frac{h}{d} \implies \\
    \implies \frac 1{\sin^2 \beta} &= \ctg^2 \beta + 1 = \sqr{\frac{h}{d}} + 1 \implies \\
    \implies \sin\alpha &= n\sin \beta = n\sqrt {\frac 1{\sqr{\frac{h}{d}} + 1}} \approx 0{,}42
    \end{align*}
}
\solutionspace{120pt}

\tasknumber{4}%
\task{%
    Определите радиус полутени на экране диска размером $D = 3\,\text{см}$ от протяжённого источника, также обладающего формой диска размером $d = 2\,\text{см}$ (см.
    рис.
    на доске, вид сбоку).
    Расстояние от источника до диска равно $l = 12\,\text{см}$, а расстояние от диска до экрана — $L = 30\,\text{см}$.
}
\answer{%
    $\cfrac{\frac d2 + \frac D2}l = \cfrac{\frac d2 + r}{l + L} \implies r = \cfrac{Dl + dL + DL}{2l} = \cfrac D2 + \cfrac{ L }{ l } \cdot \cfrac{d+D}2 \approx 7{,}8\,\text{см} \implies 2r \approx 15{,}5\,\text{см}$
}
\solutionspace{80pt}

\tasknumber{5}%
\task{%
    Высота солнца над горизонтом составляет $ 40 \degrees$.
    Посреди ровного бетонного поля стоит одинокий цилиндрический столб высотой $5\,\text{м}$ и радиусом $4\,\text{см}$.
    Определите площадь полной тени от этого столба на бетоне.
    Угловой размер Солнца принять равным $30'$ и считать малым углом.
}
\answer{%
    $\text{трапеция} \implies S = 0{,}68\,\text{м}^{2}$
}

\variantsplitter

\addpersonalvariant{Софья Андрианова}

\tasknumber{1}%
\task{%
    Сформулируйте:
    \begin{itemize}
        \item принцип Гюйгенса-Френеля,
        \item закон отражения (в двух частях).
    \end{itemize}
}
\solutionspace{60pt}

\tasknumber{2}%
\task{%
    На дне водоема глубиной $3\,\text{м}$ лежит зеркало.
    Луч света, пройдя через воду, отражается от зеркала и выходит из воды.
    Найти расстояние между точкой входа луча в воду и точкой выхода луча из воды,
    если показатель преломления воды $1{,}33$, а угол падения луча $35\degrees$.
}
\answer{%
    \begin{align*}
    \ctg \beta &= \frac{h}{d:L:s} \implies d = \frac{h}{\ctg \beta} \\
    \frac 1{\sin^2 \beta} &= \ctg^2 \beta + 1 \implies \ctg \beta = \sqrt{\frac 1{\sin^2 \beta} - 1} \\
    \sin\alpha &= n\sin \beta \implies \sin \beta = \frac{\sin\alpha}{n} \\
    d &= \frac{h}{\sqrt{\frac 1{\sin^2 \beta} - 1}} = \frac{h}{\sqrt{\sqr{\frac{n}{\sin\alpha}} - 1}} \\
    2d &= \frac{2{h}}{\sqrt{\sqr{\frac{n}{\sin\alpha}} - 1}} \approx 286{,}8\,\text{см}
    \end{align*}
}
\solutionspace{120pt}

\tasknumber{3}%
\task{%
    Луч света падает на вертикально расположенную стеклянную пластинку толщиной $1{,}5\,\text{см}$.
    Пройдя через пластину, он выходит из неё в точке, смещённой по вертикали от точки падения на расстояние $6\,\text{мм}$.
    Показатель преломления стекла $1{,}6$.
    Найти синус угла падения.
}
\answer{%
    \begin{align*}
    \ctg \beta &= \frac{h}{d} \implies \\
    \implies \frac 1{\sin^2 \beta} &= \ctg^2 \beta + 1 = \sqr{\frac{h}{d}} + 1 \implies \\
    \implies \sin\alpha &= n\sin \beta = n\sqrt {\frac 1{\sqr{\frac{h}{d}} + 1}} \approx 0{,}59
    \end{align*}
}
\solutionspace{120pt}

\tasknumber{4}%
\task{%
    Определите диаметр полутени на экране диска размером $D = 3\,\text{см}$ от протяжённого источника, также обладающего формой диска размером $d = 2\,\text{см}$ (см.
    рис.
    на доске, вид сбоку).
    Расстояние от источника до диска равно $l = 18\,\text{см}$, а расстояние от диска до экрана — $L = 20\,\text{см}$.
}
\answer{%
    $\cfrac{\frac d2 + \frac D2}l = \cfrac{\frac d2 + r}{l + L} \implies r = \cfrac{Dl + dL + DL}{2l} = \cfrac D2 + \cfrac{ L }{ l } \cdot \cfrac{d+D}2 \approx 4{,}3\,\text{см} \implies 2r \approx 8{,}6\,\text{см}$
}
\solutionspace{80pt}

\tasknumber{5}%
\task{%
    Высота солнца над горизонтом составляет $ 40 \degrees$.
    Посреди ровного бетонного поля стоит одинокий цилиндрический столб высотой $5\,\text{м}$ и радиусом $4\,\text{см}$.
    Определите площадь полной тени от этого столба на бетоне.
    Угловой размер Солнца принять равным $30'$ и считать малым углом.
}
\answer{%
    $\text{трапеция} \implies S = 0{,}68\,\text{м}^{2}$
}

\variantsplitter

\addpersonalvariant{Владимир Артемчук}

\tasknumber{1}%
\task{%
    Сформулируйте:
    \begin{itemize}
        \item принцип Гюйгенса-Френеля,
        \item закон преломления (в двух частях).
    \end{itemize}
}
\solutionspace{60pt}

\tasknumber{2}%
\task{%
    На дне водоема глубиной $3\,\text{м}$ лежит зеркало.
    Луч света, пройдя через воду, отражается от зеркала и выходит из воды.
    Найти расстояние между точкой входа луча в воду и точкой выхода луча из воды,
    если показатель преломления воды $1{,}33$, а угол падения луча $35\degrees$.
}
\answer{%
    \begin{align*}
    \ctg \beta &= \frac{h}{d:L:s} \implies d = \frac{h}{\ctg \beta} \\
    \frac 1{\sin^2 \beta} &= \ctg^2 \beta + 1 \implies \ctg \beta = \sqrt{\frac 1{\sin^2 \beta} - 1} \\
    \sin\alpha &= n\sin \beta \implies \sin \beta = \frac{\sin\alpha}{n} \\
    d &= \frac{h}{\sqrt{\frac 1{\sin^2 \beta} - 1}} = \frac{h}{\sqrt{\sqr{\frac{n}{\sin\alpha}} - 1}} \\
    2d &= \frac{2{h}}{\sqrt{\sqr{\frac{n}{\sin\alpha}} - 1}} \approx 286{,}8\,\text{см}
    \end{align*}
}
\solutionspace{120pt}

\tasknumber{3}%
\task{%
    Луч света падает на горизонтально расположенную стеклянную пластинку толщиной $1{,}4\,\text{см}$.
    Пройдя через пластину, он выходит из неё в точке, смещённой по горизонтали от точки падения на расстояние $6\,\text{мм}$.
    Показатель преломления стекла $1{,}5$.
    Найти синус угла падения.
}
\answer{%
    \begin{align*}
    \ctg \beta &= \frac{h}{d} \implies \\
    \implies \frac 1{\sin^2 \beta} &= \ctg^2 \beta + 1 = \sqr{\frac{h}{d}} + 1 \implies \\
    \implies \sin\alpha &= n\sin \beta = n\sqrt {\frac 1{\sqr{\frac{h}{d}} + 1}} \approx 0{,}59
    \end{align*}
}
\solutionspace{120pt}

\tasknumber{4}%
\task{%
    Определите радиус полутени на экране диска размером $D = 3\,\text{см}$ от протяжённого источника, также обладающего формой диска размером $d = 4\,\text{см}$ (см.
    рис.
    на доске, вид сбоку).
    Расстояние от источника до диска равно $l = 15\,\text{см}$, а расстояние от диска до экрана — $L = 10\,\text{см}$.
}
\answer{%
    $\cfrac{\frac d2 + \frac D2}l = \cfrac{\frac d2 + r}{l + L} \implies r = \cfrac{Dl + dL + DL}{2l} = \cfrac D2 + \cfrac{ L }{ l } \cdot \cfrac{d+D}2 \approx 3{,}8\,\text{см} \implies 2r \approx 7{,}7\,\text{см}$
}
\solutionspace{80pt}

\tasknumber{5}%
\task{%
    Высота солнца над горизонтом составляет $ 50 \degrees$.
    Посреди ровного бетонного поля стоит одинокий цилиндрический столб высотой $5\,\text{м}$ и радиусом $8\,\text{см}$.
    Определите площадь полной тени от этого столба на бетоне.
    Угловой размер Солнца принять равным $30'$ и считать малым углом.
}
\answer{%
    $\text{трапеция} \implies S = 0{,}79\,\text{м}^{2}$
}

\variantsplitter

\addpersonalvariant{Софья Белянкина}

\tasknumber{1}%
\task{%
    Сформулируйте:
    \begin{itemize}
        \item принцип Гюйгенса-Френеля,
        \item закон отражения (в двух частях).
    \end{itemize}
}
\solutionspace{60pt}

\tasknumber{2}%
\task{%
    На дне водоема глубиной $2\,\text{м}$ лежит зеркало.
    Луч света, пройдя через воду, отражается от зеркала и выходит из воды.
    Найти расстояние между точкой входа луча в воду и точкой выхода луча из воды,
    если показатель преломления воды $1{,}33$, а угол падения луча $30\degrees$.
}
\answer{%
    \begin{align*}
    \ctg \beta &= \frac{h}{d:L:s} \implies d = \frac{h}{\ctg \beta} \\
    \frac 1{\sin^2 \beta} &= \ctg^2 \beta + 1 \implies \ctg \beta = \sqrt{\frac 1{\sin^2 \beta} - 1} \\
    \sin\alpha &= n\sin \beta \implies \sin \beta = \frac{\sin\alpha}{n} \\
    d &= \frac{h}{\sqrt{\frac 1{\sin^2 \beta} - 1}} = \frac{h}{\sqrt{\sqr{\frac{n}{\sin\alpha}} - 1}} \\
    2d &= \frac{2{h}}{\sqrt{\sqr{\frac{n}{\sin\alpha}} - 1}} \approx 162{,}3\,\text{см}
    \end{align*}
}
\solutionspace{120pt}

\tasknumber{3}%
\task{%
    Луч света падает на горизонтально расположенную стеклянную пластинку толщиной $1{,}5\,\text{см}$.
    Пройдя через пластину, он выходит из неё в точке, смещённой по горизонтали от точки падения на расстояние $6\,\text{мм}$.
    Показатель преломления стекла $1{,}5$.
    Найти синус угла падения.
}
\answer{%
    \begin{align*}
    \ctg \beta &= \frac{h}{d} \implies \\
    \implies \frac 1{\sin^2 \beta} &= \ctg^2 \beta + 1 = \sqr{\frac{h}{d}} + 1 \implies \\
    \implies \sin\alpha &= n\sin \beta = n\sqrt {\frac 1{\sqr{\frac{h}{d}} + 1}} \approx 0{,}56
    \end{align*}
}
\solutionspace{120pt}

\tasknumber{4}%
\task{%
    Определите диаметр полутени на экране диска размером $D = 2\,\text{см}$ от протяжённого источника, также обладающего формой диска размером $d = 3\,\text{см}$ (см.
    рис.
    на доске, вид сбоку).
    Расстояние от источника до диска равно $l = 12\,\text{см}$, а расстояние от диска до экрана — $L = 20\,\text{см}$.
}
\answer{%
    $\cfrac{\frac d2 + \frac D2}l = \cfrac{\frac d2 + r}{l + L} \implies r = \cfrac{Dl + dL + DL}{2l} = \cfrac D2 + \cfrac{ L }{ l } \cdot \cfrac{d+D}2 \approx 5{,}2\,\text{см} \implies 2r \approx 10{,}3\,\text{см}$
}
\solutionspace{80pt}

\tasknumber{5}%
\task{%
    Высота солнца над горизонтом составляет $ 35 \degrees$.
    Посреди ровного бетонного поля стоит одинокий цилиндрический столб высотой $3\,\text{м}$ и радиусом $8\,\text{см}$.
    Определите площадь полной тени от этого столба на бетоне.
    Угловой размер Солнца принять равным $30'$ и считать малым углом.
}
\answer{%
    $\text{трапеция} \implies S = 0{,}78\,\text{м}^{2}$
}

\variantsplitter

\addpersonalvariant{Варвара Егиазарян}

\tasknumber{1}%
\task{%
    Сформулируйте:
    \begin{itemize}
        \item принцип Гюйгенса-Френеля,
        \item закон отражения (в двух частях).
    \end{itemize}
}
\solutionspace{60pt}

\tasknumber{2}%
\task{%
    На дне водоема глубиной $2\,\text{м}$ лежит зеркало.
    Луч света, пройдя через воду, отражается от зеркала и выходит из воды.
    Найти расстояние между точкой входа луча в воду и точкой выхода луча из воды,
    если показатель преломления воды $1{,}33$, а угол падения луча $35\degrees$.
}
\answer{%
    \begin{align*}
    \ctg \beta &= \frac{h}{d:L:s} \implies d = \frac{h}{\ctg \beta} \\
    \frac 1{\sin^2 \beta} &= \ctg^2 \beta + 1 \implies \ctg \beta = \sqrt{\frac 1{\sin^2 \beta} - 1} \\
    \sin\alpha &= n\sin \beta \implies \sin \beta = \frac{\sin\alpha}{n} \\
    d &= \frac{h}{\sqrt{\frac 1{\sin^2 \beta} - 1}} = \frac{h}{\sqrt{\sqr{\frac{n}{\sin\alpha}} - 1}} \\
    2d &= \frac{2{h}}{\sqrt{\sqr{\frac{n}{\sin\alpha}} - 1}} \approx 191{,}2\,\text{см}
    \end{align*}
}
\solutionspace{120pt}

\tasknumber{3}%
\task{%
    Луч света падает на горизонтально расположенную стеклянную пластинку толщиной $1{,}4\,\text{см}$.
    Пройдя через пластину, он выходит из неё в точке, смещённой по горизонтали от точки падения на расстояние $5\,\text{мм}$.
    Показатель преломления стекла $1{,}5$.
    Найти синус угла падения.
}
\answer{%
    \begin{align*}
    \ctg \beta &= \frac{h}{d} \implies \\
    \implies \frac 1{\sin^2 \beta} &= \ctg^2 \beta + 1 = \sqr{\frac{h}{d}} + 1 \implies \\
    \implies \sin\alpha &= n\sin \beta = n\sqrt {\frac 1{\sqr{\frac{h}{d}} + 1}} \approx 0{,}50
    \end{align*}
}
\solutionspace{120pt}

\tasknumber{4}%
\task{%
    Определите диаметр полутени на экране диска размером $D = 3{,}5\,\text{см}$ от протяжённого источника, также обладающего формой диска размером $d = 3\,\text{см}$ (см.
    рис.
    на доске, вид сбоку).
    Расстояние от источника до диска равно $l = 15\,\text{см}$, а расстояние от диска до экрана — $L = 20\,\text{см}$.
}
\answer{%
    $\cfrac{\frac d2 + \frac D2}l = \cfrac{\frac d2 + r}{l + L} \implies r = \cfrac{Dl + dL + DL}{2l} = \cfrac D2 + \cfrac{ L }{ l } \cdot \cfrac{d+D}2 \approx 6{,}1\,\text{см} \implies 2r \approx 12{,}2\,\text{см}$
}
\solutionspace{80pt}

\tasknumber{5}%
\task{%
    Высота солнца над горизонтом составляет $ 50 \degrees$.
    Посреди ровного бетонного поля стоит одинокий цилиндрический столб высотой $3\,\text{м}$ и радиусом $10\,\text{см}$.
    Определите площадь полной тени от этого столба на бетоне.
    Угловой размер Солнца принять равным $30'$ и считать малым углом.
}
\answer{%
    $\text{трапеция} \implies S = 0{,}55\,\text{м}^{2}$
}

\variantsplitter

\addpersonalvariant{Владислав Емелин}

\tasknumber{1}%
\task{%
    Сформулируйте:
    \begin{itemize}
        \item принцип Гюйгенса-Френеля,
        \item закон преломления (в двух частях).
    \end{itemize}
}
\solutionspace{60pt}

\tasknumber{2}%
\task{%
    На дне водоема глубиной $3\,\text{м}$ лежит зеркало.
    Луч света, пройдя через воду, отражается от зеркала и выходит из воды.
    Найти расстояние между точкой входа луча в воду и точкой выхода луча из воды,
    если показатель преломления воды $1{,}33$, а угол падения луча $35\degrees$.
}
\answer{%
    \begin{align*}
    \ctg \beta &= \frac{h}{d:L:s} \implies d = \frac{h}{\ctg \beta} \\
    \frac 1{\sin^2 \beta} &= \ctg^2 \beta + 1 \implies \ctg \beta = \sqrt{\frac 1{\sin^2 \beta} - 1} \\
    \sin\alpha &= n\sin \beta \implies \sin \beta = \frac{\sin\alpha}{n} \\
    d &= \frac{h}{\sqrt{\frac 1{\sin^2 \beta} - 1}} = \frac{h}{\sqrt{\sqr{\frac{n}{\sin\alpha}} - 1}} \\
    2d &= \frac{2{h}}{\sqrt{\sqr{\frac{n}{\sin\alpha}} - 1}} \approx 286{,}8\,\text{см}
    \end{align*}
}
\solutionspace{120pt}

\tasknumber{3}%
\task{%
    Луч света падает на горизонтально расположенную стеклянную пластинку толщиной $1{,}2\,\text{см}$.
    Пройдя через пластину, он выходит из неё в точке, смещённой по горизонтали от точки падения на расстояние $5\,\text{мм}$.
    Показатель преломления стекла $1{,}6$.
    Найти синус угла падения.
}
\answer{%
    \begin{align*}
    \ctg \beta &= \frac{h}{d} \implies \\
    \implies \frac 1{\sin^2 \beta} &= \ctg^2 \beta + 1 = \sqr{\frac{h}{d}} + 1 \implies \\
    \implies \sin\alpha &= n\sin \beta = n\sqrt {\frac 1{\sqr{\frac{h}{d}} + 1}} \approx 0{,}62
    \end{align*}
}
\solutionspace{120pt}

\tasknumber{4}%
\task{%
    Определите диаметр полутени на экране диска размером $D = 2{,}5\,\text{см}$ от протяжённого источника, также обладающего формой диска размером $d = 4\,\text{см}$ (см.
    рис.
    на доске, вид сбоку).
    Расстояние от источника до диска равно $l = 15\,\text{см}$, а расстояние от диска до экрана — $L = 10\,\text{см}$.
}
\answer{%
    $\cfrac{\frac d2 + \frac D2}l = \cfrac{\frac d2 + r}{l + L} \implies r = \cfrac{Dl + dL + DL}{2l} = \cfrac D2 + \cfrac{ L }{ l } \cdot \cfrac{d+D}2 \approx 3{,}4\,\text{см} \implies 2r \approx 6{,}8\,\text{см}$
}
\solutionspace{80pt}

\tasknumber{5}%
\task{%
    Высота солнца над горизонтом составляет $ 50 \degrees$.
    Посреди ровного бетонного поля стоит одинокий цилиндрический столб высотой $5\,\text{м}$ и радиусом $2\,\text{см}$.
    Определите площадь полной тени от этого столба на бетоне.
    Угловой размер Солнца принять равным $30'$ и считать малым углом.
}
\answer{%
    $\text{треугольник} \implies S = 0{,}06\,\text{м}^{2}$
}

\variantsplitter

\addpersonalvariant{Артём Жичин}

\tasknumber{1}%
\task{%
    Сформулируйте:
    \begin{itemize}
        \item принцип Гюйгенса-Френеля,
        \item закон преломления (в двух частях).
    \end{itemize}
}
\solutionspace{60pt}

\tasknumber{2}%
\task{%
    На дне водоема глубиной $4\,\text{м}$ лежит зеркало.
    Луч света, пройдя через воду, отражается от зеркала и выходит из воды.
    Найти расстояние между точкой входа луча в воду и точкой выхода луча из воды,
    если показатель преломления воды $1{,}33$, а угол падения луча $30\degrees$.
}
\answer{%
    \begin{align*}
    \ctg \beta &= \frac{h}{d:L:s} \implies d = \frac{h}{\ctg \beta} \\
    \frac 1{\sin^2 \beta} &= \ctg^2 \beta + 1 \implies \ctg \beta = \sqrt{\frac 1{\sin^2 \beta} - 1} \\
    \sin\alpha &= n\sin \beta \implies \sin \beta = \frac{\sin\alpha}{n} \\
    d &= \frac{h}{\sqrt{\frac 1{\sin^2 \beta} - 1}} = \frac{h}{\sqrt{\sqr{\frac{n}{\sin\alpha}} - 1}} \\
    2d &= \frac{2{h}}{\sqrt{\sqr{\frac{n}{\sin\alpha}} - 1}} \approx 324{,}6\,\text{см}
    \end{align*}
}
\solutionspace{120pt}

\tasknumber{3}%
\task{%
    Луч света падает на вертикально расположенную стеклянную пластинку толщиной $1{,}6\,\text{см}$.
    Пройдя через пластину, он выходит из неё в точке, смещённой по вертикали от точки падения на расстояние $5\,\text{мм}$.
    Показатель преломления стекла $1{,}4$.
    Найти синус угла падения.
}
\answer{%
    \begin{align*}
    \ctg \beta &= \frac{h}{d} \implies \\
    \implies \frac 1{\sin^2 \beta} &= \ctg^2 \beta + 1 = \sqr{\frac{h}{d}} + 1 \implies \\
    \implies \sin\alpha &= n\sin \beta = n\sqrt {\frac 1{\sqr{\frac{h}{d}} + 1}} \approx 0{,}42
    \end{align*}
}
\solutionspace{120pt}

\tasknumber{4}%
\task{%
    Определите радиус полутени на экране диска размером $D = 3{,}5\,\text{см}$ от протяжённого источника, также обладающего формой диска размером $d = 3\,\text{см}$ (см.
    рис.
    на доске, вид сбоку).
    Расстояние от источника до диска равно $l = 12\,\text{см}$, а расстояние от диска до экрана — $L = 30\,\text{см}$.
}
\answer{%
    $\cfrac{\frac d2 + \frac D2}l = \cfrac{\frac d2 + r}{l + L} \implies r = \cfrac{Dl + dL + DL}{2l} = \cfrac D2 + \cfrac{ L }{ l } \cdot \cfrac{d+D}2 \approx 9{,}9\,\text{см} \implies 2r \approx 19{,}8\,\text{см}$
}
\solutionspace{80pt}

\tasknumber{5}%
\task{%
    Высота солнца над горизонтом составляет $ 35 \degrees$.
    Посреди ровного бетонного поля стоит одинокий цилиндрический столб высотой $3\,\text{м}$ и радиусом $10\,\text{см}$.
    Определите площадь полной тени от этого столба на бетоне.
    Угловой размер Солнца принять равным $30'$ и считать малым углом.
}
\answer{%
    $\text{трапеция} \implies S = 0{,}95\,\text{м}^{2}$
}

\variantsplitter

\addpersonalvariant{Дарья Кошман}

\tasknumber{1}%
\task{%
    Сформулируйте:
    \begin{itemize}
        \item принцип Гюйгенса-Френеля,
        \item закон отражения (в двух частях).
    \end{itemize}
}
\solutionspace{60pt}

\tasknumber{2}%
\task{%
    На дне водоема глубиной $2\,\text{м}$ лежит зеркало.
    Луч света, пройдя через воду, отражается от зеркала и выходит из воды.
    Найти расстояние между точкой входа луча в воду и точкой выхода луча из воды,
    если показатель преломления воды $1{,}33$, а угол падения луча $25\degrees$.
}
\answer{%
    \begin{align*}
    \ctg \beta &= \frac{h}{d:L:s} \implies d = \frac{h}{\ctg \beta} \\
    \frac 1{\sin^2 \beta} &= \ctg^2 \beta + 1 \implies \ctg \beta = \sqrt{\frac 1{\sin^2 \beta} - 1} \\
    \sin\alpha &= n\sin \beta \implies \sin \beta = \frac{\sin\alpha}{n} \\
    d &= \frac{h}{\sqrt{\frac 1{\sin^2 \beta} - 1}} = \frac{h}{\sqrt{\sqr{\frac{n}{\sin\alpha}} - 1}} \\
    2d &= \frac{2{h}}{\sqrt{\sqr{\frac{n}{\sin\alpha}} - 1}} \approx 134{,}1\,\text{см}
    \end{align*}
}
\solutionspace{120pt}

\tasknumber{3}%
\task{%
    Луч света падает на горизонтально расположенную стеклянную пластинку толщиной $1{,}4\,\text{см}$.
    Пройдя через пластину, он выходит из неё в точке, смещённой по горизонтали от точки падения на расстояние $6\,\text{мм}$.
    Показатель преломления стекла $1{,}4$.
    Найти синус угла падения.
}
\answer{%
    \begin{align*}
    \ctg \beta &= \frac{h}{d} \implies \\
    \implies \frac 1{\sin^2 \beta} &= \ctg^2 \beta + 1 = \sqr{\frac{h}{d}} + 1 \implies \\
    \implies \sin\alpha &= n\sin \beta = n\sqrt {\frac 1{\sqr{\frac{h}{d}} + 1}} \approx 0{,}55
    \end{align*}
}
\solutionspace{120pt}

\tasknumber{4}%
\task{%
    Определите радиус полутени на экране диска размером $D = 2\,\text{см}$ от протяжённого источника, также обладающего формой диска размером $d = 3\,\text{см}$ (см.
    рис.
    на доске, вид сбоку).
    Расстояние от источника до диска равно $l = 18\,\text{см}$, а расстояние от диска до экрана — $L = 20\,\text{см}$.
}
\answer{%
    $\cfrac{\frac d2 + \frac D2}l = \cfrac{\frac d2 + r}{l + L} \implies r = \cfrac{Dl + dL + DL}{2l} = \cfrac D2 + \cfrac{ L }{ l } \cdot \cfrac{d+D}2 \approx 3{,}8\,\text{см} \implies 2r \approx 7{,}6\,\text{см}$
}
\solutionspace{80pt}

\tasknumber{5}%
\task{%
    Высота солнца над горизонтом составляет $ 35 \degrees$.
    Посреди ровного бетонного поля стоит одинокий цилиндрический столб высотой $5\,\text{м}$ и радиусом $2\,\text{см}$.
    Определите площадь полной тени от этого столба на бетоне.
    Угловой размер Солнца принять равным $30'$ и считать малым углом.
}
\answer{%
    $\text{треугольник} \implies S = 0{,}08\,\text{м}^{2}$
}

\variantsplitter

\addpersonalvariant{Анна Кузьмичёва}

\tasknumber{1}%
\task{%
    Сформулируйте:
    \begin{itemize}
        \item принцип Гюйгенса-Френеля,
        \item закон отражения (в двух частях).
    \end{itemize}
}
\solutionspace{60pt}

\tasknumber{2}%
\task{%
    На дне водоема глубиной $3\,\text{м}$ лежит зеркало.
    Луч света, пройдя через воду, отражается от зеркала и выходит из воды.
    Найти расстояние между точкой входа луча в воду и точкой выхода луча из воды,
    если показатель преломления воды $1{,}33$, а угол падения луча $30\degrees$.
}
\answer{%
    \begin{align*}
    \ctg \beta &= \frac{h}{d:L:s} \implies d = \frac{h}{\ctg \beta} \\
    \frac 1{\sin^2 \beta} &= \ctg^2 \beta + 1 \implies \ctg \beta = \sqrt{\frac 1{\sin^2 \beta} - 1} \\
    \sin\alpha &= n\sin \beta \implies \sin \beta = \frac{\sin\alpha}{n} \\
    d &= \frac{h}{\sqrt{\frac 1{\sin^2 \beta} - 1}} = \frac{h}{\sqrt{\sqr{\frac{n}{\sin\alpha}} - 1}} \\
    2d &= \frac{2{h}}{\sqrt{\sqr{\frac{n}{\sin\alpha}} - 1}} \approx 243{,}4\,\text{см}
    \end{align*}
}
\solutionspace{120pt}

\tasknumber{3}%
\task{%
    Луч света падает на вертикально расположенную стеклянную пластинку толщиной $1{,}3\,\text{см}$.
    Пройдя через пластину, он выходит из неё в точке, смещённой по вертикали от точки падения на расстояние $4\,\text{мм}$.
    Показатель преломления стекла $1{,}4$.
    Найти синус угла падения.
}
\answer{%
    \begin{align*}
    \ctg \beta &= \frac{h}{d} \implies \\
    \implies \frac 1{\sin^2 \beta} &= \ctg^2 \beta + 1 = \sqr{\frac{h}{d}} + 1 \implies \\
    \implies \sin\alpha &= n\sin \beta = n\sqrt {\frac 1{\sqr{\frac{h}{d}} + 1}} \approx 0{,}41
    \end{align*}
}
\solutionspace{120pt}

\tasknumber{4}%
\task{%
    Определите диаметр полутени на экране диска размером $D = 4\,\text{см}$ от протяжённого источника, также обладающего формой диска размером $d = 3\,\text{см}$ (см.
    рис.
    на доске, вид сбоку).
    Расстояние от источника до диска равно $l = 18\,\text{см}$, а расстояние от диска до экрана — $L = 10\,\text{см}$.
}
\answer{%
    $\cfrac{\frac d2 + \frac D2}l = \cfrac{\frac d2 + r}{l + L} \implies r = \cfrac{Dl + dL + DL}{2l} = \cfrac D2 + \cfrac{ L }{ l } \cdot \cfrac{d+D}2 \approx 3{,}9\,\text{см} \implies 2r \approx 7{,}9\,\text{см}$
}
\solutionspace{80pt}

\tasknumber{5}%
\task{%
    Высота солнца над горизонтом составляет $ 55 \degrees$.
    Посреди ровного бетонного поля стоит одинокий цилиндрический столб высотой $3\,\text{м}$ и радиусом $10\,\text{см}$.
    Определите площадь полной тени от этого столба на бетоне.
    Угловой размер Солнца принять равным $30'$ и считать малым углом.
}
\answer{%
    $\text{трапеция} \implies S = 0{,}45\,\text{м}^{2}$
}

\variantsplitter

\addpersonalvariant{Алёна Куприянова}

\tasknumber{1}%
\task{%
    Сформулируйте:
    \begin{itemize}
        \item принцип Гюйгенса-Френеля,
        \item закон отражения (в двух частях).
    \end{itemize}
}
\solutionspace{60pt}

\tasknumber{2}%
\task{%
    На дне водоема глубиной $2\,\text{м}$ лежит зеркало.
    Луч света, пройдя через воду, отражается от зеркала и выходит из воды.
    Найти расстояние между точкой входа луча в воду и точкой выхода луча из воды,
    если показатель преломления воды $1{,}33$, а угол падения луча $35\degrees$.
}
\answer{%
    \begin{align*}
    \ctg \beta &= \frac{h}{d:L:s} \implies d = \frac{h}{\ctg \beta} \\
    \frac 1{\sin^2 \beta} &= \ctg^2 \beta + 1 \implies \ctg \beta = \sqrt{\frac 1{\sin^2 \beta} - 1} \\
    \sin\alpha &= n\sin \beta \implies \sin \beta = \frac{\sin\alpha}{n} \\
    d &= \frac{h}{\sqrt{\frac 1{\sin^2 \beta} - 1}} = \frac{h}{\sqrt{\sqr{\frac{n}{\sin\alpha}} - 1}} \\
    2d &= \frac{2{h}}{\sqrt{\sqr{\frac{n}{\sin\alpha}} - 1}} \approx 191{,}2\,\text{см}
    \end{align*}
}
\solutionspace{120pt}

\tasknumber{3}%
\task{%
    Луч света падает на вертикально расположенную стеклянную пластинку толщиной $1{,}6\,\text{см}$.
    Пройдя через пластину, он выходит из неё в точке, смещённой по вертикали от точки падения на расстояние $5\,\text{мм}$.
    Показатель преломления стекла $1{,}5$.
    Найти синус угла падения.
}
\answer{%
    \begin{align*}
    \ctg \beta &= \frac{h}{d} \implies \\
    \implies \frac 1{\sin^2 \beta} &= \ctg^2 \beta + 1 = \sqr{\frac{h}{d}} + 1 \implies \\
    \implies \sin\alpha &= n\sin \beta = n\sqrt {\frac 1{\sqr{\frac{h}{d}} + 1}} \approx 0{,}45
    \end{align*}
}
\solutionspace{120pt}

\tasknumber{4}%
\task{%
    Определите диаметр полутени на экране диска размером $D = 2\,\text{см}$ от протяжённого источника, также обладающего формой диска размером $d = 4\,\text{см}$ (см.
    рис.
    на доске, вид сбоку).
    Расстояние от источника до диска равно $l = 15\,\text{см}$, а расстояние от диска до экрана — $L = 30\,\text{см}$.
}
\answer{%
    $\cfrac{\frac d2 + \frac D2}l = \cfrac{\frac d2 + r}{l + L} \implies r = \cfrac{Dl + dL + DL}{2l} = \cfrac D2 + \cfrac{ L }{ l } \cdot \cfrac{d+D}2 \approx 7\,\text{см} \implies 2r \approx 14\,\text{см}$
}
\solutionspace{80pt}

\tasknumber{5}%
\task{%
    Высота солнца над горизонтом составляет $ 35 \degrees$.
    Посреди ровного бетонного поля стоит одинокий цилиндрический столб высотой $7\,\text{м}$ и радиусом $4\,\text{см}$.
    Определите площадь полной тени от этого столба на бетоне.
    Угловой размер Солнца принять равным $30'$ и считать малым углом.
}
\answer{%
    $\text{треугольник} \implies S = 0{,}30\,\text{м}^{2}$
}

\variantsplitter

\addpersonalvariant{Ярослав Лавровский}

\tasknumber{1}%
\task{%
    Сформулируйте:
    \begin{itemize}
        \item принцип Гюйгенса-Френеля,
        \item закон отражения (в двух частях).
    \end{itemize}
}
\solutionspace{60pt}

\tasknumber{2}%
\task{%
    На дне водоема глубиной $2\,\text{м}$ лежит зеркало.
    Луч света, пройдя через воду, отражается от зеркала и выходит из воды.
    Найти расстояние между точкой входа луча в воду и точкой выхода луча из воды,
    если показатель преломления воды $1{,}33$, а угол падения луча $30\degrees$.
}
\answer{%
    \begin{align*}
    \ctg \beta &= \frac{h}{d:L:s} \implies d = \frac{h}{\ctg \beta} \\
    \frac 1{\sin^2 \beta} &= \ctg^2 \beta + 1 \implies \ctg \beta = \sqrt{\frac 1{\sin^2 \beta} - 1} \\
    \sin\alpha &= n\sin \beta \implies \sin \beta = \frac{\sin\alpha}{n} \\
    d &= \frac{h}{\sqrt{\frac 1{\sin^2 \beta} - 1}} = \frac{h}{\sqrt{\sqr{\frac{n}{\sin\alpha}} - 1}} \\
    2d &= \frac{2{h}}{\sqrt{\sqr{\frac{n}{\sin\alpha}} - 1}} \approx 162{,}3\,\text{см}
    \end{align*}
}
\solutionspace{120pt}

\tasknumber{3}%
\task{%
    Луч света падает на вертикально расположенную стеклянную пластинку толщиной $1{,}4\,\text{см}$.
    Пройдя через пластину, он выходит из неё в точке, смещённой по вертикали от точки падения на расстояние $5\,\text{мм}$.
    Показатель преломления стекла $1{,}6$.
    Найти синус угла падения.
}
\answer{%
    \begin{align*}
    \ctg \beta &= \frac{h}{d} \implies \\
    \implies \frac 1{\sin^2 \beta} &= \ctg^2 \beta + 1 = \sqr{\frac{h}{d}} + 1 \implies \\
    \implies \sin\alpha &= n\sin \beta = n\sqrt {\frac 1{\sqr{\frac{h}{d}} + 1}} \approx 0{,}54
    \end{align*}
}
\solutionspace{120pt}

\tasknumber{4}%
\task{%
    Определите диаметр полутени на экране диска размером $D = 2{,}5\,\text{см}$ от протяжённого источника, также обладающего формой диска размером $d = 2\,\text{см}$ (см.
    рис.
    на доске, вид сбоку).
    Расстояние от источника до диска равно $l = 15\,\text{см}$, а расстояние от диска до экрана — $L = 10\,\text{см}$.
}
\answer{%
    $\cfrac{\frac d2 + \frac D2}l = \cfrac{\frac d2 + r}{l + L} \implies r = \cfrac{Dl + dL + DL}{2l} = \cfrac D2 + \cfrac{ L }{ l } \cdot \cfrac{d+D}2 \approx 2{,}8\,\text{см} \implies 2r \approx 5{,}5\,\text{см}$
}
\solutionspace{80pt}

\tasknumber{5}%
\task{%
    Высота солнца над горизонтом составляет $ 40 \degrees$.
    Посреди ровного бетонного поля стоит одинокий цилиндрический столб высотой $7\,\text{м}$ и радиусом $2\,\text{см}$.
    Определите площадь полной тени от этого столба на бетоне.
    Угловой размер Солнца принять равным $30'$ и считать малым углом.
}
\answer{%
    $\text{треугольник} \implies S = 0{,}07\,\text{м}^{2}$
}

\variantsplitter

\addpersonalvariant{Анастасия Ламанова}

\tasknumber{1}%
\task{%
    Сформулируйте:
    \begin{itemize}
        \item принцип Гюйгенса-Френеля,
        \item закон отражения (в двух частях).
    \end{itemize}
}
\solutionspace{60pt}

\tasknumber{2}%
\task{%
    На дне водоема глубиной $4\,\text{м}$ лежит зеркало.
    Луч света, пройдя через воду, отражается от зеркала и выходит из воды.
    Найти расстояние между точкой входа луча в воду и точкой выхода луча из воды,
    если показатель преломления воды $1{,}33$, а угол падения луча $25\degrees$.
}
\answer{%
    \begin{align*}
    \ctg \beta &= \frac{h}{d:L:s} \implies d = \frac{h}{\ctg \beta} \\
    \frac 1{\sin^2 \beta} &= \ctg^2 \beta + 1 \implies \ctg \beta = \sqrt{\frac 1{\sin^2 \beta} - 1} \\
    \sin\alpha &= n\sin \beta \implies \sin \beta = \frac{\sin\alpha}{n} \\
    d &= \frac{h}{\sqrt{\frac 1{\sin^2 \beta} - 1}} = \frac{h}{\sqrt{\sqr{\frac{n}{\sin\alpha}} - 1}} \\
    2d &= \frac{2{h}}{\sqrt{\sqr{\frac{n}{\sin\alpha}} - 1}} \approx 268{,}1\,\text{см}
    \end{align*}
}
\solutionspace{120pt}

\tasknumber{3}%
\task{%
    Луч света падает на горизонтально расположенную стеклянную пластинку толщиной $1{,}6\,\text{см}$.
    Пройдя через пластину, он выходит из неё в точке, смещённой по горизонтали от точки падения на расстояние $6\,\text{мм}$.
    Показатель преломления стекла $1{,}6$.
    Найти синус угла падения.
}
\answer{%
    \begin{align*}
    \ctg \beta &= \frac{h}{d} \implies \\
    \implies \frac 1{\sin^2 \beta} &= \ctg^2 \beta + 1 = \sqr{\frac{h}{d}} + 1 \implies \\
    \implies \sin\alpha &= n\sin \beta = n\sqrt {\frac 1{\sqr{\frac{h}{d}} + 1}} \approx 0{,}56
    \end{align*}
}
\solutionspace{120pt}

\tasknumber{4}%
\task{%
    Определите диаметр полутени на экране диска размером $D = 2{,}5\,\text{см}$ от протяжённого источника, также обладающего формой диска размером $d = 3\,\text{см}$ (см.
    рис.
    на доске, вид сбоку).
    Расстояние от источника до диска равно $l = 18\,\text{см}$, а расстояние от диска до экрана — $L = 30\,\text{см}$.
}
\answer{%
    $\cfrac{\frac d2 + \frac D2}l = \cfrac{\frac d2 + r}{l + L} \implies r = \cfrac{Dl + dL + DL}{2l} = \cfrac D2 + \cfrac{ L }{ l } \cdot \cfrac{d+D}2 \approx 5{,}8\,\text{см} \implies 2r \approx 11{,}7\,\text{см}$
}
\solutionspace{80pt}

\tasknumber{5}%
\task{%
    Высота солнца над горизонтом составляет $ 40 \degrees$.
    Посреди ровного бетонного поля стоит одинокий цилиндрический столб высотой $5\,\text{м}$ и радиусом $10\,\text{см}$.
    Определите площадь полной тени от этого столба на бетоне.
    Угловой размер Солнца принять равным $30'$ и считать малым углом.
}
\answer{%
    $\text{трапеция} \implies S = 1{,}39\,\text{м}^{2}$
}

\variantsplitter

\addpersonalvariant{Виктория Легонькова}

\tasknumber{1}%
\task{%
    Сформулируйте:
    \begin{itemize}
        \item принцип Гюйгенса-Френеля,
        \item закон отражения (в двух частях).
    \end{itemize}
}
\solutionspace{60pt}

\tasknumber{2}%
\task{%
    На дне водоема глубиной $4\,\text{м}$ лежит зеркало.
    Луч света, пройдя через воду, отражается от зеркала и выходит из воды.
    Найти расстояние между точкой входа луча в воду и точкой выхода луча из воды,
    если показатель преломления воды $1{,}33$, а угол падения луча $25\degrees$.
}
\answer{%
    \begin{align*}
    \ctg \beta &= \frac{h}{d:L:s} \implies d = \frac{h}{\ctg \beta} \\
    \frac 1{\sin^2 \beta} &= \ctg^2 \beta + 1 \implies \ctg \beta = \sqrt{\frac 1{\sin^2 \beta} - 1} \\
    \sin\alpha &= n\sin \beta \implies \sin \beta = \frac{\sin\alpha}{n} \\
    d &= \frac{h}{\sqrt{\frac 1{\sin^2 \beta} - 1}} = \frac{h}{\sqrt{\sqr{\frac{n}{\sin\alpha}} - 1}} \\
    2d &= \frac{2{h}}{\sqrt{\sqr{\frac{n}{\sin\alpha}} - 1}} \approx 268{,}1\,\text{см}
    \end{align*}
}
\solutionspace{120pt}

\tasknumber{3}%
\task{%
    Луч света падает на вертикально расположенную стеклянную пластинку толщиной $1{,}3\,\text{см}$.
    Пройдя через пластину, он выходит из неё в точке, смещённой по вертикали от точки падения на расстояние $6\,\text{мм}$.
    Показатель преломления стекла $1{,}4$.
    Найти синус угла падения.
}
\answer{%
    \begin{align*}
    \ctg \beta &= \frac{h}{d} \implies \\
    \implies \frac 1{\sin^2 \beta} &= \ctg^2 \beta + 1 = \sqr{\frac{h}{d}} + 1 \implies \\
    \implies \sin\alpha &= n\sin \beta = n\sqrt {\frac 1{\sqr{\frac{h}{d}} + 1}} \approx 0{,}59
    \end{align*}
}
\solutionspace{120pt}

\tasknumber{4}%
\task{%
    Определите радиус полутени на экране диска размером $D = 2\,\text{см}$ от протяжённого источника, также обладающего формой диска размером $d = 4\,\text{см}$ (см.
    рис.
    на доске, вид сбоку).
    Расстояние от источника до диска равно $l = 18\,\text{см}$, а расстояние от диска до экрана — $L = 10\,\text{см}$.
}
\answer{%
    $\cfrac{\frac d2 + \frac D2}l = \cfrac{\frac d2 + r}{l + L} \implies r = \cfrac{Dl + dL + DL}{2l} = \cfrac D2 + \cfrac{ L }{ l } \cdot \cfrac{d+D}2 \approx 2{,}7\,\text{см} \implies 2r \approx 5{,}3\,\text{см}$
}
\solutionspace{80pt}

\tasknumber{5}%
\task{%
    Высота солнца над горизонтом составляет $ 35 \degrees$.
    Посреди ровного бетонного поля стоит одинокий цилиндрический столб высотой $7\,\text{м}$ и радиусом $8\,\text{см}$.
    Определите площадь полной тени от этого столба на бетоне.
    Угловой размер Солнца принять равным $30'$ и считать малым углом.
}
\answer{%
    $\text{трапеция} \implies S = 2{,}13\,\text{м}^{2}$
}

\variantsplitter

\addpersonalvariant{Семён Мартынов}

\tasknumber{1}%
\task{%
    Сформулируйте:
    \begin{itemize}
        \item принцип Гюйгенса-Френеля,
        \item закон преломления (в двух частях).
    \end{itemize}
}
\solutionspace{60pt}

\tasknumber{2}%
\task{%
    На дне водоема глубиной $3\,\text{м}$ лежит зеркало.
    Луч света, пройдя через воду, отражается от зеркала и выходит из воды.
    Найти расстояние между точкой входа луча в воду и точкой выхода луча из воды,
    если показатель преломления воды $1{,}33$, а угол падения луча $35\degrees$.
}
\answer{%
    \begin{align*}
    \ctg \beta &= \frac{h}{d:L:s} \implies d = \frac{h}{\ctg \beta} \\
    \frac 1{\sin^2 \beta} &= \ctg^2 \beta + 1 \implies \ctg \beta = \sqrt{\frac 1{\sin^2 \beta} - 1} \\
    \sin\alpha &= n\sin \beta \implies \sin \beta = \frac{\sin\alpha}{n} \\
    d &= \frac{h}{\sqrt{\frac 1{\sin^2 \beta} - 1}} = \frac{h}{\sqrt{\sqr{\frac{n}{\sin\alpha}} - 1}} \\
    2d &= \frac{2{h}}{\sqrt{\sqr{\frac{n}{\sin\alpha}} - 1}} \approx 286{,}8\,\text{см}
    \end{align*}
}
\solutionspace{120pt}

\tasknumber{3}%
\task{%
    Луч света падает на вертикально расположенную стеклянную пластинку толщиной $1{,}5\,\text{см}$.
    Пройдя через пластину, он выходит из неё в точке, смещённой по вертикали от точки падения на расстояние $4\,\text{мм}$.
    Показатель преломления стекла $1{,}5$.
    Найти синус угла падения.
}
\answer{%
    \begin{align*}
    \ctg \beta &= \frac{h}{d} \implies \\
    \implies \frac 1{\sin^2 \beta} &= \ctg^2 \beta + 1 = \sqr{\frac{h}{d}} + 1 \implies \\
    \implies \sin\alpha &= n\sin \beta = n\sqrt {\frac 1{\sqr{\frac{h}{d}} + 1}} \approx 0{,}39
    \end{align*}
}
\solutionspace{120pt}

\tasknumber{4}%
\task{%
    Определите радиус полутени на экране диска размером $D = 3\,\text{см}$ от протяжённого источника, также обладающего формой диска размером $d = 2\,\text{см}$ (см.
    рис.
    на доске, вид сбоку).
    Расстояние от источника до диска равно $l = 18\,\text{см}$, а расстояние от диска до экрана — $L = 10\,\text{см}$.
}
\answer{%
    $\cfrac{\frac d2 + \frac D2}l = \cfrac{\frac d2 + r}{l + L} \implies r = \cfrac{Dl + dL + DL}{2l} = \cfrac D2 + \cfrac{ L }{ l } \cdot \cfrac{d+D}2 \approx 2{,}9\,\text{см} \implies 2r \approx 5{,}8\,\text{см}$
}
\solutionspace{80pt}

\tasknumber{5}%
\task{%
    Высота солнца над горизонтом составляет $ 55 \degrees$.
    Посреди ровного бетонного поля стоит одинокий цилиндрический столб высотой $5\,\text{м}$ и радиусом $4\,\text{см}$.
    Определите площадь полной тени от этого столба на бетоне.
    Угловой размер Солнца принять равным $30'$ и считать малым углом.
}
\answer{%
    $\text{трапеция} \implies S = 0{,}37\,\text{м}^{2}$
}

\variantsplitter

\addpersonalvariant{Варвара Минаева}

\tasknumber{1}%
\task{%
    Сформулируйте:
    \begin{itemize}
        \item принцип Гюйгенса-Френеля,
        \item закон преломления (в двух частях).
    \end{itemize}
}
\solutionspace{60pt}

\tasknumber{2}%
\task{%
    На дне водоема глубиной $2\,\text{м}$ лежит зеркало.
    Луч света, пройдя через воду, отражается от зеркала и выходит из воды.
    Найти расстояние между точкой входа луча в воду и точкой выхода луча из воды,
    если показатель преломления воды $1{,}33$, а угол падения луча $25\degrees$.
}
\answer{%
    \begin{align*}
    \ctg \beta &= \frac{h}{d:L:s} \implies d = \frac{h}{\ctg \beta} \\
    \frac 1{\sin^2 \beta} &= \ctg^2 \beta + 1 \implies \ctg \beta = \sqrt{\frac 1{\sin^2 \beta} - 1} \\
    \sin\alpha &= n\sin \beta \implies \sin \beta = \frac{\sin\alpha}{n} \\
    d &= \frac{h}{\sqrt{\frac 1{\sin^2 \beta} - 1}} = \frac{h}{\sqrt{\sqr{\frac{n}{\sin\alpha}} - 1}} \\
    2d &= \frac{2{h}}{\sqrt{\sqr{\frac{n}{\sin\alpha}} - 1}} \approx 134{,}1\,\text{см}
    \end{align*}
}
\solutionspace{120pt}

\tasknumber{3}%
\task{%
    Луч света падает на вертикально расположенную стеклянную пластинку толщиной $1{,}2\,\text{см}$.
    Пройдя через пластину, он выходит из неё в точке, смещённой по вертикали от точки падения на расстояние $6\,\text{мм}$.
    Показатель преломления стекла $1{,}6$.
    Найти синус угла падения.
}
\answer{%
    \begin{align*}
    \ctg \beta &= \frac{h}{d} \implies \\
    \implies \frac 1{\sin^2 \beta} &= \ctg^2 \beta + 1 = \sqr{\frac{h}{d}} + 1 \implies \\
    \implies \sin\alpha &= n\sin \beta = n\sqrt {\frac 1{\sqr{\frac{h}{d}} + 1}} \approx 0{,}72
    \end{align*}
}
\solutionspace{120pt}

\tasknumber{4}%
\task{%
    Определите диаметр полутени на экране диска размером $D = 3{,}5\,\text{см}$ от протяжённого источника, также обладающего формой диска размером $d = 2\,\text{см}$ (см.
    рис.
    на доске, вид сбоку).
    Расстояние от источника до диска равно $l = 18\,\text{см}$, а расстояние от диска до экрана — $L = 30\,\text{см}$.
}
\answer{%
    $\cfrac{\frac d2 + \frac D2}l = \cfrac{\frac d2 + r}{l + L} \implies r = \cfrac{Dl + dL + DL}{2l} = \cfrac D2 + \cfrac{ L }{ l } \cdot \cfrac{d+D}2 \approx 6{,}3\,\text{см} \implies 2r \approx 12{,}7\,\text{см}$
}
\solutionspace{80pt}

\tasknumber{5}%
\task{%
    Высота солнца над горизонтом составляет $ 40 \degrees$.
    Посреди ровного бетонного поля стоит одинокий цилиндрический столб высотой $7\,\text{м}$ и радиусом $8\,\text{см}$.
    Определите площадь полной тени от этого столба на бетоне.
    Угловой размер Солнца принять равным $30'$ и считать малым углом.
}
\answer{%
    $\text{трапеция} \implies S = 1{,}73\,\text{м}^{2}$
}

\variantsplitter

\addpersonalvariant{Леонид Никитин}

\tasknumber{1}%
\task{%
    Сформулируйте:
    \begin{itemize}
        \item принцип Гюйгенса-Френеля,
        \item закон отражения (в двух частях).
    \end{itemize}
}
\solutionspace{60pt}

\tasknumber{2}%
\task{%
    На дне водоема глубиной $4\,\text{м}$ лежит зеркало.
    Луч света, пройдя через воду, отражается от зеркала и выходит из воды.
    Найти расстояние между точкой входа луча в воду и точкой выхода луча из воды,
    если показатель преломления воды $1{,}33$, а угол падения луча $25\degrees$.
}
\answer{%
    \begin{align*}
    \ctg \beta &= \frac{h}{d:L:s} \implies d = \frac{h}{\ctg \beta} \\
    \frac 1{\sin^2 \beta} &= \ctg^2 \beta + 1 \implies \ctg \beta = \sqrt{\frac 1{\sin^2 \beta} - 1} \\
    \sin\alpha &= n\sin \beta \implies \sin \beta = \frac{\sin\alpha}{n} \\
    d &= \frac{h}{\sqrt{\frac 1{\sin^2 \beta} - 1}} = \frac{h}{\sqrt{\sqr{\frac{n}{\sin\alpha}} - 1}} \\
    2d &= \frac{2{h}}{\sqrt{\sqr{\frac{n}{\sin\alpha}} - 1}} \approx 268{,}1\,\text{см}
    \end{align*}
}
\solutionspace{120pt}

\tasknumber{3}%
\task{%
    Луч света падает на вертикально расположенную стеклянную пластинку толщиной $1{,}2\,\text{см}$.
    Пройдя через пластину, он выходит из неё в точке, смещённой по вертикали от точки падения на расстояние $6\,\text{мм}$.
    Показатель преломления стекла $1{,}4$.
    Найти синус угла падения.
}
\answer{%
    \begin{align*}
    \ctg \beta &= \frac{h}{d} \implies \\
    \implies \frac 1{\sin^2 \beta} &= \ctg^2 \beta + 1 = \sqr{\frac{h}{d}} + 1 \implies \\
    \implies \sin\alpha &= n\sin \beta = n\sqrt {\frac 1{\sqr{\frac{h}{d}} + 1}} \approx 0{,}63
    \end{align*}
}
\solutionspace{120pt}

\tasknumber{4}%
\task{%
    Определите радиус полутени на экране диска размером $D = 3\,\text{см}$ от протяжённого источника, также обладающего формой диска размером $d = 4\,\text{см}$ (см.
    рис.
    на доске, вид сбоку).
    Расстояние от источника до диска равно $l = 12\,\text{см}$, а расстояние от диска до экрана — $L = 10\,\text{см}$.
}
\answer{%
    $\cfrac{\frac d2 + \frac D2}l = \cfrac{\frac d2 + r}{l + L} \implies r = \cfrac{Dl + dL + DL}{2l} = \cfrac D2 + \cfrac{ L }{ l } \cdot \cfrac{d+D}2 \approx 4{,}4\,\text{см} \implies 2r \approx 8{,}8\,\text{см}$
}
\solutionspace{80pt}

\tasknumber{5}%
\task{%
    Высота солнца над горизонтом составляет $ 50 \degrees$.
    Посреди ровного бетонного поля стоит одинокий цилиндрический столб высотой $5\,\text{м}$ и радиусом $10\,\text{см}$.
    Определите площадь полной тени от этого столба на бетоне.
    Угловой размер Солнца принять равным $30'$ и считать малым углом.
}
\answer{%
    $\text{трапеция} \implies S = 0{,}96\,\text{м}^{2}$
}

\variantsplitter

\addpersonalvariant{Тимофей Полетаев}

\tasknumber{1}%
\task{%
    Сформулируйте:
    \begin{itemize}
        \item принцип Гюйгенса-Френеля,
        \item закон преломления (в двух частях).
    \end{itemize}
}
\solutionspace{60pt}

\tasknumber{2}%
\task{%
    На дне водоема глубиной $3\,\text{м}$ лежит зеркало.
    Луч света, пройдя через воду, отражается от зеркала и выходит из воды.
    Найти расстояние между точкой входа луча в воду и точкой выхода луча из воды,
    если показатель преломления воды $1{,}33$, а угол падения луча $25\degrees$.
}
\answer{%
    \begin{align*}
    \ctg \beta &= \frac{h}{d:L:s} \implies d = \frac{h}{\ctg \beta} \\
    \frac 1{\sin^2 \beta} &= \ctg^2 \beta + 1 \implies \ctg \beta = \sqrt{\frac 1{\sin^2 \beta} - 1} \\
    \sin\alpha &= n\sin \beta \implies \sin \beta = \frac{\sin\alpha}{n} \\
    d &= \frac{h}{\sqrt{\frac 1{\sin^2 \beta} - 1}} = \frac{h}{\sqrt{\sqr{\frac{n}{\sin\alpha}} - 1}} \\
    2d &= \frac{2{h}}{\sqrt{\sqr{\frac{n}{\sin\alpha}} - 1}} \approx 201{,}1\,\text{см}
    \end{align*}
}
\solutionspace{120pt}

\tasknumber{3}%
\task{%
    Луч света падает на вертикально расположенную стеклянную пластинку толщиной $1{,}3\,\text{см}$.
    Пройдя через пластину, он выходит из неё в точке, смещённой по вертикали от точки падения на расстояние $6\,\text{мм}$.
    Показатель преломления стекла $1{,}5$.
    Найти синус угла падения.
}
\answer{%
    \begin{align*}
    \ctg \beta &= \frac{h}{d} \implies \\
    \implies \frac 1{\sin^2 \beta} &= \ctg^2 \beta + 1 = \sqr{\frac{h}{d}} + 1 \implies \\
    \implies \sin\alpha &= n\sin \beta = n\sqrt {\frac 1{\sqr{\frac{h}{d}} + 1}} \approx 0{,}63
    \end{align*}
}
\solutionspace{120pt}

\tasknumber{4}%
\task{%
    Определите радиус полутени на экране диска размером $D = 3\,\text{см}$ от протяжённого источника, также обладающего формой диска размером $d = 4\,\text{см}$ (см.
    рис.
    на доске, вид сбоку).
    Расстояние от источника до диска равно $l = 18\,\text{см}$, а расстояние от диска до экрана — $L = 30\,\text{см}$.
}
\answer{%
    $\cfrac{\frac d2 + \frac D2}l = \cfrac{\frac d2 + r}{l + L} \implies r = \cfrac{Dl + dL + DL}{2l} = \cfrac D2 + \cfrac{ L }{ l } \cdot \cfrac{d+D}2 \approx 7{,}3\,\text{см} \implies 2r \approx 14{,}7\,\text{см}$
}
\solutionspace{80pt}

\tasknumber{5}%
\task{%
    Высота солнца над горизонтом составляет $ 40 \degrees$.
    Посреди ровного бетонного поля стоит одинокий цилиндрический столб высотой $3\,\text{м}$ и радиусом $4\,\text{см}$.
    Определите площадь полной тени от этого столба на бетоне.
    Угловой размер Солнца принять равным $30'$ и считать малым углом.
}
\answer{%
    $\text{трапеция} \implies S = 0{,}36\,\text{м}^{2}$
}

\variantsplitter

\addpersonalvariant{Андрей Рожков}

\tasknumber{1}%
\task{%
    Сформулируйте:
    \begin{itemize}
        \item принцип Гюйгенса-Френеля,
        \item закон отражения (в двух частях).
    \end{itemize}
}
\solutionspace{60pt}

\tasknumber{2}%
\task{%
    На дне водоема глубиной $3\,\text{м}$ лежит зеркало.
    Луч света, пройдя через воду, отражается от зеркала и выходит из воды.
    Найти расстояние между точкой входа луча в воду и точкой выхода луча из воды,
    если показатель преломления воды $1{,}33$, а угол падения луча $30\degrees$.
}
\answer{%
    \begin{align*}
    \ctg \beta &= \frac{h}{d:L:s} \implies d = \frac{h}{\ctg \beta} \\
    \frac 1{\sin^2 \beta} &= \ctg^2 \beta + 1 \implies \ctg \beta = \sqrt{\frac 1{\sin^2 \beta} - 1} \\
    \sin\alpha &= n\sin \beta \implies \sin \beta = \frac{\sin\alpha}{n} \\
    d &= \frac{h}{\sqrt{\frac 1{\sin^2 \beta} - 1}} = \frac{h}{\sqrt{\sqr{\frac{n}{\sin\alpha}} - 1}} \\
    2d &= \frac{2{h}}{\sqrt{\sqr{\frac{n}{\sin\alpha}} - 1}} \approx 243{,}4\,\text{см}
    \end{align*}
}
\solutionspace{120pt}

\tasknumber{3}%
\task{%
    Луч света падает на горизонтально расположенную стеклянную пластинку толщиной $1{,}5\,\text{см}$.
    Пройдя через пластину, он выходит из неё в точке, смещённой по горизонтали от точки падения на расстояние $6\,\text{мм}$.
    Показатель преломления стекла $1{,}6$.
    Найти синус угла падения.
}
\answer{%
    \begin{align*}
    \ctg \beta &= \frac{h}{d} \implies \\
    \implies \frac 1{\sin^2 \beta} &= \ctg^2 \beta + 1 = \sqr{\frac{h}{d}} + 1 \implies \\
    \implies \sin\alpha &= n\sin \beta = n\sqrt {\frac 1{\sqr{\frac{h}{d}} + 1}} \approx 0{,}59
    \end{align*}
}
\solutionspace{120pt}

\tasknumber{4}%
\task{%
    Определите диаметр полутени на экране диска размером $D = 3{,}5\,\text{см}$ от протяжённого источника, также обладающего формой диска размером $d = 2\,\text{см}$ (см.
    рис.
    на доске, вид сбоку).
    Расстояние от источника до диска равно $l = 15\,\text{см}$, а расстояние от диска до экрана — $L = 10\,\text{см}$.
}
\answer{%
    $\cfrac{\frac d2 + \frac D2}l = \cfrac{\frac d2 + r}{l + L} \implies r = \cfrac{Dl + dL + DL}{2l} = \cfrac D2 + \cfrac{ L }{ l } \cdot \cfrac{d+D}2 \approx 3{,}6\,\text{см} \implies 2r \approx 7{,}2\,\text{см}$
}
\solutionspace{80pt}

\tasknumber{5}%
\task{%
    Высота солнца над горизонтом составляет $ 40 \degrees$.
    Посреди ровного бетонного поля стоит одинокий цилиндрический столб высотой $5\,\text{м}$ и радиусом $10\,\text{см}$.
    Определите площадь полной тени от этого столба на бетоне.
    Угловой размер Солнца принять равным $30'$ и считать малым углом.
}
\answer{%
    $\text{трапеция} \implies S = 1{,}39\,\text{м}^{2}$
}

\variantsplitter

\addpersonalvariant{Рената Таржиманова}

\tasknumber{1}%
\task{%
    Сформулируйте:
    \begin{itemize}
        \item принцип Гюйгенса-Френеля,
        \item закон отражения (в двух частях).
    \end{itemize}
}
\solutionspace{60pt}

\tasknumber{2}%
\task{%
    На дне водоема глубиной $2\,\text{м}$ лежит зеркало.
    Луч света, пройдя через воду, отражается от зеркала и выходит из воды.
    Найти расстояние между точкой входа луча в воду и точкой выхода луча из воды,
    если показатель преломления воды $1{,}33$, а угол падения луча $35\degrees$.
}
\answer{%
    \begin{align*}
    \ctg \beta &= \frac{h}{d:L:s} \implies d = \frac{h}{\ctg \beta} \\
    \frac 1{\sin^2 \beta} &= \ctg^2 \beta + 1 \implies \ctg \beta = \sqrt{\frac 1{\sin^2 \beta} - 1} \\
    \sin\alpha &= n\sin \beta \implies \sin \beta = \frac{\sin\alpha}{n} \\
    d &= \frac{h}{\sqrt{\frac 1{\sin^2 \beta} - 1}} = \frac{h}{\sqrt{\sqr{\frac{n}{\sin\alpha}} - 1}} \\
    2d &= \frac{2{h}}{\sqrt{\sqr{\frac{n}{\sin\alpha}} - 1}} \approx 191{,}2\,\text{см}
    \end{align*}
}
\solutionspace{120pt}

\tasknumber{3}%
\task{%
    Луч света падает на горизонтально расположенную стеклянную пластинку толщиной $1{,}3\,\text{см}$.
    Пройдя через пластину, он выходит из неё в точке, смещённой по горизонтали от точки падения на расстояние $5\,\text{мм}$.
    Показатель преломления стекла $1{,}5$.
    Найти синус угла падения.
}
\answer{%
    \begin{align*}
    \ctg \beta &= \frac{h}{d} \implies \\
    \implies \frac 1{\sin^2 \beta} &= \ctg^2 \beta + 1 = \sqr{\frac{h}{d}} + 1 \implies \\
    \implies \sin\alpha &= n\sin \beta = n\sqrt {\frac 1{\sqr{\frac{h}{d}} + 1}} \approx 0{,}54
    \end{align*}
}
\solutionspace{120pt}

\tasknumber{4}%
\task{%
    Определите диаметр полутени на экране диска размером $D = 2\,\text{см}$ от протяжённого источника, также обладающего формой диска размером $d = 3\,\text{см}$ (см.
    рис.
    на доске, вид сбоку).
    Расстояние от источника до диска равно $l = 15\,\text{см}$, а расстояние от диска до экрана — $L = 10\,\text{см}$.
}
\answer{%
    $\cfrac{\frac d2 + \frac D2}l = \cfrac{\frac d2 + r}{l + L} \implies r = \cfrac{Dl + dL + DL}{2l} = \cfrac D2 + \cfrac{ L }{ l } \cdot \cfrac{d+D}2 \approx 2{,}7\,\text{см} \implies 2r \approx 5{,}3\,\text{см}$
}
\solutionspace{80pt}

\tasknumber{5}%
\task{%
    Высота солнца над горизонтом составляет $ 55 \degrees$.
    Посреди ровного бетонного поля стоит одинокий цилиндрический столб высотой $5\,\text{м}$ и радиусом $10\,\text{см}$.
    Определите площадь полной тени от этого столба на бетоне.
    Угловой размер Солнца принять равным $30'$ и считать малым углом.
}
\answer{%
    $\text{трапеция} \implies S = 0{,}79\,\text{м}^{2}$
}

\variantsplitter

\addpersonalvariant{Андрей Щербаков}

\tasknumber{1}%
\task{%
    Сформулируйте:
    \begin{itemize}
        \item принцип Гюйгенса-Френеля,
        \item закон отражения (в двух частях).
    \end{itemize}
}
\solutionspace{60pt}

\tasknumber{2}%
\task{%
    На дне водоема глубиной $2\,\text{м}$ лежит зеркало.
    Луч света, пройдя через воду, отражается от зеркала и выходит из воды.
    Найти расстояние между точкой входа луча в воду и точкой выхода луча из воды,
    если показатель преломления воды $1{,}33$, а угол падения луча $35\degrees$.
}
\answer{%
    \begin{align*}
    \ctg \beta &= \frac{h}{d:L:s} \implies d = \frac{h}{\ctg \beta} \\
    \frac 1{\sin^2 \beta} &= \ctg^2 \beta + 1 \implies \ctg \beta = \sqrt{\frac 1{\sin^2 \beta} - 1} \\
    \sin\alpha &= n\sin \beta \implies \sin \beta = \frac{\sin\alpha}{n} \\
    d &= \frac{h}{\sqrt{\frac 1{\sin^2 \beta} - 1}} = \frac{h}{\sqrt{\sqr{\frac{n}{\sin\alpha}} - 1}} \\
    2d &= \frac{2{h}}{\sqrt{\sqr{\frac{n}{\sin\alpha}} - 1}} \approx 191{,}2\,\text{см}
    \end{align*}
}
\solutionspace{120pt}

\tasknumber{3}%
\task{%
    Луч света падает на вертикально расположенную стеклянную пластинку толщиной $1{,}5\,\text{см}$.
    Пройдя через пластину, он выходит из неё в точке, смещённой по вертикали от точки падения на расстояние $5\,\text{мм}$.
    Показатель преломления стекла $1{,}6$.
    Найти синус угла падения.
}
\answer{%
    \begin{align*}
    \ctg \beta &= \frac{h}{d} \implies \\
    \implies \frac 1{\sin^2 \beta} &= \ctg^2 \beta + 1 = \sqr{\frac{h}{d}} + 1 \implies \\
    \implies \sin\alpha &= n\sin \beta = n\sqrt {\frac 1{\sqr{\frac{h}{d}} + 1}} \approx 0{,}51
    \end{align*}
}
\solutionspace{120pt}

\tasknumber{4}%
\task{%
    Определите радиус полутени на экране диска размером $D = 2{,}5\,\text{см}$ от протяжённого источника, также обладающего формой диска размером $d = 4\,\text{см}$ (см.
    рис.
    на доске, вид сбоку).
    Расстояние от источника до диска равно $l = 18\,\text{см}$, а расстояние от диска до экрана — $L = 20\,\text{см}$.
}
\answer{%
    $\cfrac{\frac d2 + \frac D2}l = \cfrac{\frac d2 + r}{l + L} \implies r = \cfrac{Dl + dL + DL}{2l} = \cfrac D2 + \cfrac{ L }{ l } \cdot \cfrac{d+D}2 \approx 4{,}9\,\text{см} \implies 2r \approx 9{,}7\,\text{см}$
}
\solutionspace{80pt}

\tasknumber{5}%
\task{%
    Высота солнца над горизонтом составляет $ 55 \degrees$.
    Посреди ровного бетонного поля стоит одинокий цилиндрический столб высотой $3\,\text{м}$ и радиусом $2\,\text{см}$.
    Определите площадь полной тени от этого столба на бетоне.
    Угловой размер Солнца принять равным $30'$ и считать малым углом.
}
\answer{%
    $\text{трапеция} \implies S = 0{,}12\,\text{м}^{2}$
}

\variantsplitter

\addpersonalvariant{Михаил Ярошевский}

\tasknumber{1}%
\task{%
    Сформулируйте:
    \begin{itemize}
        \item принцип Гюйгенса-Френеля,
        \item закон отражения (в двух частях).
    \end{itemize}
}
\solutionspace{60pt}

\tasknumber{2}%
\task{%
    На дне водоема глубиной $4\,\text{м}$ лежит зеркало.
    Луч света, пройдя через воду, отражается от зеркала и выходит из воды.
    Найти расстояние между точкой входа луча в воду и точкой выхода луча из воды,
    если показатель преломления воды $1{,}33$, а угол падения луча $35\degrees$.
}
\answer{%
    \begin{align*}
    \ctg \beta &= \frac{h}{d:L:s} \implies d = \frac{h}{\ctg \beta} \\
    \frac 1{\sin^2 \beta} &= \ctg^2 \beta + 1 \implies \ctg \beta = \sqrt{\frac 1{\sin^2 \beta} - 1} \\
    \sin\alpha &= n\sin \beta \implies \sin \beta = \frac{\sin\alpha}{n} \\
    d &= \frac{h}{\sqrt{\frac 1{\sin^2 \beta} - 1}} = \frac{h}{\sqrt{\sqr{\frac{n}{\sin\alpha}} - 1}} \\
    2d &= \frac{2{h}}{\sqrt{\sqr{\frac{n}{\sin\alpha}} - 1}} \approx 382{,}4\,\text{см}
    \end{align*}
}
\solutionspace{120pt}

\tasknumber{3}%
\task{%
    Луч света падает на горизонтально расположенную стеклянную пластинку толщиной $1{,}5\,\text{см}$.
    Пройдя через пластину, он выходит из неё в точке, смещённой по горизонтали от точки падения на расстояние $6\,\text{мм}$.
    Показатель преломления стекла $1{,}6$.
    Найти синус угла падения.
}
\answer{%
    \begin{align*}
    \ctg \beta &= \frac{h}{d} \implies \\
    \implies \frac 1{\sin^2 \beta} &= \ctg^2 \beta + 1 = \sqr{\frac{h}{d}} + 1 \implies \\
    \implies \sin\alpha &= n\sin \beta = n\sqrt {\frac 1{\sqr{\frac{h}{d}} + 1}} \approx 0{,}59
    \end{align*}
}
\solutionspace{120pt}

\tasknumber{4}%
\task{%
    Определите диаметр полутени на экране диска размером $D = 3\,\text{см}$ от протяжённого источника, также обладающего формой диска размером $d = 2\,\text{см}$ (см.
    рис.
    на доске, вид сбоку).
    Расстояние от источника до диска равно $l = 15\,\text{см}$, а расстояние от диска до экрана — $L = 30\,\text{см}$.
}
\answer{%
    $\cfrac{\frac d2 + \frac D2}l = \cfrac{\frac d2 + r}{l + L} \implies r = \cfrac{Dl + dL + DL}{2l} = \cfrac D2 + \cfrac{ L }{ l } \cdot \cfrac{d+D}2 \approx 6{,}5\,\text{см} \implies 2r \approx 13\,\text{см}$
}
\solutionspace{80pt}

\tasknumber{5}%
\task{%
    Высота солнца над горизонтом составляет $ 50 \degrees$.
    Посреди ровного бетонного поля стоит одинокий цилиндрический столб высотой $5\,\text{м}$ и радиусом $2\,\text{см}$.
    Определите площадь полной тени от этого столба на бетоне.
    Угловой размер Солнца принять равным $30'$ и считать малым углом.
}
\answer{%
    $\text{треугольник} \implies S = 0{,}06\,\text{м}^{2}$
}

\variantsplitter

\addpersonalvariant{Алексей Алимпиев}

\tasknumber{1}%
\task{%
    Сформулируйте:
    \begin{itemize}
        \item принцип Гюйгенса-Френеля,
        \item закон преломления (в двух частях).
    \end{itemize}
}
\solutionspace{60pt}

\tasknumber{2}%
\task{%
    На дне водоема глубиной $2\,\text{м}$ лежит зеркало.
    Луч света, пройдя через воду, отражается от зеркала и выходит из воды.
    Найти расстояние между точкой входа луча в воду и точкой выхода луча из воды,
    если показатель преломления воды $1{,}33$, а угол падения луча $25\degrees$.
}
\answer{%
    \begin{align*}
    \ctg \beta &= \frac{h}{d:L:s} \implies d = \frac{h}{\ctg \beta} \\
    \frac 1{\sin^2 \beta} &= \ctg^2 \beta + 1 \implies \ctg \beta = \sqrt{\frac 1{\sin^2 \beta} - 1} \\
    \sin\alpha &= n\sin \beta \implies \sin \beta = \frac{\sin\alpha}{n} \\
    d &= \frac{h}{\sqrt{\frac 1{\sin^2 \beta} - 1}} = \frac{h}{\sqrt{\sqr{\frac{n}{\sin\alpha}} - 1}} \\
    2d &= \frac{2{h}}{\sqrt{\sqr{\frac{n}{\sin\alpha}} - 1}} \approx 134{,}1\,\text{см}
    \end{align*}
}
\solutionspace{120pt}

\tasknumber{3}%
\task{%
    Луч света падает на горизонтально расположенную стеклянную пластинку толщиной $1{,}2\,\text{см}$.
    Пройдя через пластину, он выходит из неё в точке, смещённой по горизонтали от точки падения на расстояние $4\,\text{мм}$.
    Показатель преломления стекла $1{,}5$.
    Найти синус угла падения.
}
\answer{%
    \begin{align*}
    \ctg \beta &= \frac{h}{d} \implies \\
    \implies \frac 1{\sin^2 \beta} &= \ctg^2 \beta + 1 = \sqr{\frac{h}{d}} + 1 \implies \\
    \implies \sin\alpha &= n\sin \beta = n\sqrt {\frac 1{\sqr{\frac{h}{d}} + 1}} \approx 0{,}47
    \end{align*}
}
\solutionspace{120pt}

\tasknumber{4}%
\task{%
    Определите радиус полутени на экране диска размером $D = 2{,}5\,\text{см}$ от протяжённого источника, также обладающего формой диска размером $d = 3\,\text{см}$ (см.
    рис.
    на доске, вид сбоку).
    Расстояние от источника до диска равно $l = 18\,\text{см}$, а расстояние от диска до экрана — $L = 10\,\text{см}$.
}
\answer{%
    $\cfrac{\frac d2 + \frac D2}l = \cfrac{\frac d2 + r}{l + L} \implies r = \cfrac{Dl + dL + DL}{2l} = \cfrac D2 + \cfrac{ L }{ l } \cdot \cfrac{d+D}2 \approx 2{,}8\,\text{см} \implies 2r \approx 5{,}6\,\text{см}$
}
\solutionspace{80pt}

\tasknumber{5}%
\task{%
    Высота солнца над горизонтом составляет $ 50 \degrees$.
    Посреди ровного бетонного поля стоит одинокий цилиндрический столб высотой $7\,\text{м}$ и радиусом $8\,\text{см}$.
    Определите площадь полной тени от этого столба на бетоне.
    Угловой размер Солнца принять равным $30'$ и считать малым углом.
}
\answer{%
    $\text{трапеция} \implies S = 1{,}17\,\text{м}^{2}$
}

\variantsplitter

\addpersonalvariant{Евгений Васин}

\tasknumber{1}%
\task{%
    Сформулируйте:
    \begin{itemize}
        \item принцип Гюйгенса-Френеля,
        \item закон отражения (в двух частях).
    \end{itemize}
}
\solutionspace{60pt}

\tasknumber{2}%
\task{%
    На дне водоема глубиной $2\,\text{м}$ лежит зеркало.
    Луч света, пройдя через воду, отражается от зеркала и выходит из воды.
    Найти расстояние между точкой входа луча в воду и точкой выхода луча из воды,
    если показатель преломления воды $1{,}33$, а угол падения луча $25\degrees$.
}
\answer{%
    \begin{align*}
    \ctg \beta &= \frac{h}{d:L:s} \implies d = \frac{h}{\ctg \beta} \\
    \frac 1{\sin^2 \beta} &= \ctg^2 \beta + 1 \implies \ctg \beta = \sqrt{\frac 1{\sin^2 \beta} - 1} \\
    \sin\alpha &= n\sin \beta \implies \sin \beta = \frac{\sin\alpha}{n} \\
    d &= \frac{h}{\sqrt{\frac 1{\sin^2 \beta} - 1}} = \frac{h}{\sqrt{\sqr{\frac{n}{\sin\alpha}} - 1}} \\
    2d &= \frac{2{h}}{\sqrt{\sqr{\frac{n}{\sin\alpha}} - 1}} \approx 134{,}1\,\text{см}
    \end{align*}
}
\solutionspace{120pt}

\tasknumber{3}%
\task{%
    Луч света падает на вертикально расположенную стеклянную пластинку толщиной $1{,}4\,\text{см}$.
    Пройдя через пластину, он выходит из неё в точке, смещённой по вертикали от точки падения на расстояние $6\,\text{мм}$.
    Показатель преломления стекла $1{,}6$.
    Найти синус угла падения.
}
\answer{%
    \begin{align*}
    \ctg \beta &= \frac{h}{d} \implies \\
    \implies \frac 1{\sin^2 \beta} &= \ctg^2 \beta + 1 = \sqr{\frac{h}{d}} + 1 \implies \\
    \implies \sin\alpha &= n\sin \beta = n\sqrt {\frac 1{\sqr{\frac{h}{d}} + 1}} \approx 0{,}63
    \end{align*}
}
\solutionspace{120pt}

\tasknumber{4}%
\task{%
    Определите диаметр полутени на экране диска размером $D = 3{,}5\,\text{см}$ от протяжённого источника, также обладающего формой диска размером $d = 2\,\text{см}$ (см.
    рис.
    на доске, вид сбоку).
    Расстояние от источника до диска равно $l = 15\,\text{см}$, а расстояние от диска до экрана — $L = 10\,\text{см}$.
}
\answer{%
    $\cfrac{\frac d2 + \frac D2}l = \cfrac{\frac d2 + r}{l + L} \implies r = \cfrac{Dl + dL + DL}{2l} = \cfrac D2 + \cfrac{ L }{ l } \cdot \cfrac{d+D}2 \approx 3{,}6\,\text{см} \implies 2r \approx 7{,}2\,\text{см}$
}
\solutionspace{80pt}

\tasknumber{5}%
\task{%
    Высота солнца над горизонтом составляет $ 35 \degrees$.
    Посреди ровного бетонного поля стоит одинокий цилиндрический столб высотой $3\,\text{м}$ и радиусом $8\,\text{см}$.
    Определите площадь полной тени от этого столба на бетоне.
    Угловой размер Солнца принять равным $30'$ и считать малым углом.
}
\answer{%
    $\text{трапеция} \implies S = 0{,}78\,\text{м}^{2}$
}

\variantsplitter

\addpersonalvariant{Вячеслав Волохов}

\tasknumber{1}%
\task{%
    Сформулируйте:
    \begin{itemize}
        \item принцип Гюйгенса-Френеля,
        \item закон отражения (в двух частях).
    \end{itemize}
}
\solutionspace{60pt}

\tasknumber{2}%
\task{%
    На дне водоема глубиной $3\,\text{м}$ лежит зеркало.
    Луч света, пройдя через воду, отражается от зеркала и выходит из воды.
    Найти расстояние между точкой входа луча в воду и точкой выхода луча из воды,
    если показатель преломления воды $1{,}33$, а угол падения луча $30\degrees$.
}
\answer{%
    \begin{align*}
    \ctg \beta &= \frac{h}{d:L:s} \implies d = \frac{h}{\ctg \beta} \\
    \frac 1{\sin^2 \beta} &= \ctg^2 \beta + 1 \implies \ctg \beta = \sqrt{\frac 1{\sin^2 \beta} - 1} \\
    \sin\alpha &= n\sin \beta \implies \sin \beta = \frac{\sin\alpha}{n} \\
    d &= \frac{h}{\sqrt{\frac 1{\sin^2 \beta} - 1}} = \frac{h}{\sqrt{\sqr{\frac{n}{\sin\alpha}} - 1}} \\
    2d &= \frac{2{h}}{\sqrt{\sqr{\frac{n}{\sin\alpha}} - 1}} \approx 243{,}4\,\text{см}
    \end{align*}
}
\solutionspace{120pt}

\tasknumber{3}%
\task{%
    Луч света падает на вертикально расположенную стеклянную пластинку толщиной $1{,}2\,\text{см}$.
    Пройдя через пластину, он выходит из неё в точке, смещённой по вертикали от точки падения на расстояние $5\,\text{мм}$.
    Показатель преломления стекла $1{,}4$.
    Найти синус угла падения.
}
\answer{%
    \begin{align*}
    \ctg \beta &= \frac{h}{d} \implies \\
    \implies \frac 1{\sin^2 \beta} &= \ctg^2 \beta + 1 = \sqr{\frac{h}{d}} + 1 \implies \\
    \implies \sin\alpha &= n\sin \beta = n\sqrt {\frac 1{\sqr{\frac{h}{d}} + 1}} \approx 0{,}54
    \end{align*}
}
\solutionspace{120pt}

\tasknumber{4}%
\task{%
    Определите диаметр полутени на экране диска размером $D = 3\,\text{см}$ от протяжённого источника, также обладающего формой диска размером $d = 4\,\text{см}$ (см.
    рис.
    на доске, вид сбоку).
    Расстояние от источника до диска равно $l = 15\,\text{см}$, а расстояние от диска до экрана — $L = 10\,\text{см}$.
}
\answer{%
    $\cfrac{\frac d2 + \frac D2}l = \cfrac{\frac d2 + r}{l + L} \implies r = \cfrac{Dl + dL + DL}{2l} = \cfrac D2 + \cfrac{ L }{ l } \cdot \cfrac{d+D}2 \approx 3{,}8\,\text{см} \implies 2r \approx 7{,}7\,\text{см}$
}
\solutionspace{80pt}

\tasknumber{5}%
\task{%
    Высота солнца над горизонтом составляет $ 35 \degrees$.
    Посреди ровного бетонного поля стоит одинокий цилиндрический столб высотой $3\,\text{м}$ и радиусом $2\,\text{см}$.
    Определите площадь полной тени от этого столба на бетоне.
    Угловой размер Солнца принять равным $30'$ и считать малым углом.
}
\answer{%
    $\text{треугольник} \implies S = 0{,}08\,\text{м}^{2}$
}

\variantsplitter

\addpersonalvariant{Герман Говоров}

\tasknumber{1}%
\task{%
    Сформулируйте:
    \begin{itemize}
        \item принцип Гюйгенса-Френеля,
        \item закон преломления (в двух частях).
    \end{itemize}
}
\solutionspace{60pt}

\tasknumber{2}%
\task{%
    На дне водоема глубиной $4\,\text{м}$ лежит зеркало.
    Луч света, пройдя через воду, отражается от зеркала и выходит из воды.
    Найти расстояние между точкой входа луча в воду и точкой выхода луча из воды,
    если показатель преломления воды $1{,}33$, а угол падения луча $35\degrees$.
}
\answer{%
    \begin{align*}
    \ctg \beta &= \frac{h}{d:L:s} \implies d = \frac{h}{\ctg \beta} \\
    \frac 1{\sin^2 \beta} &= \ctg^2 \beta + 1 \implies \ctg \beta = \sqrt{\frac 1{\sin^2 \beta} - 1} \\
    \sin\alpha &= n\sin \beta \implies \sin \beta = \frac{\sin\alpha}{n} \\
    d &= \frac{h}{\sqrt{\frac 1{\sin^2 \beta} - 1}} = \frac{h}{\sqrt{\sqr{\frac{n}{\sin\alpha}} - 1}} \\
    2d &= \frac{2{h}}{\sqrt{\sqr{\frac{n}{\sin\alpha}} - 1}} \approx 382{,}4\,\text{см}
    \end{align*}
}
\solutionspace{120pt}

\tasknumber{3}%
\task{%
    Луч света падает на горизонтально расположенную стеклянную пластинку толщиной $1{,}3\,\text{см}$.
    Пройдя через пластину, он выходит из неё в точке, смещённой по горизонтали от точки падения на расстояние $5\,\text{мм}$.
    Показатель преломления стекла $1{,}5$.
    Найти синус угла падения.
}
\answer{%
    \begin{align*}
    \ctg \beta &= \frac{h}{d} \implies \\
    \implies \frac 1{\sin^2 \beta} &= \ctg^2 \beta + 1 = \sqr{\frac{h}{d}} + 1 \implies \\
    \implies \sin\alpha &= n\sin \beta = n\sqrt {\frac 1{\sqr{\frac{h}{d}} + 1}} \approx 0{,}54
    \end{align*}
}
\solutionspace{120pt}

\tasknumber{4}%
\task{%
    Определите диаметр полутени на экране диска размером $D = 4\,\text{см}$ от протяжённого источника, также обладающего формой диска размером $d = 4\,\text{см}$ (см.
    рис.
    на доске, вид сбоку).
    Расстояние от источника до диска равно $l = 12\,\text{см}$, а расстояние от диска до экрана — $L = 10\,\text{см}$.
}
\answer{%
    $\cfrac{\frac d2 + \frac D2}l = \cfrac{\frac d2 + r}{l + L} \implies r = \cfrac{Dl + dL + DL}{2l} = \cfrac D2 + \cfrac{ L }{ l } \cdot \cfrac{d+D}2 \approx 5{,}3\,\text{см} \implies 2r \approx 10{,}7\,\text{см}$
}
\solutionspace{80pt}

\tasknumber{5}%
\task{%
    Высота солнца над горизонтом составляет $ 50 \degrees$.
    Посреди ровного бетонного поля стоит одинокий цилиндрический столб высотой $3\,\text{м}$ и радиусом $10\,\text{см}$.
    Определите площадь полной тени от этого столба на бетоне.
    Угловой размер Солнца принять равным $30'$ и считать малым углом.
}
\answer{%
    $\text{трапеция} \implies S = 0{,}55\,\text{м}^{2}$
}

\variantsplitter

\addpersonalvariant{София Журавлёва}

\tasknumber{1}%
\task{%
    Сформулируйте:
    \begin{itemize}
        \item принцип Гюйгенса-Френеля,
        \item закон отражения (в двух частях).
    \end{itemize}
}
\solutionspace{60pt}

\tasknumber{2}%
\task{%
    На дне водоема глубиной $4\,\text{м}$ лежит зеркало.
    Луч света, пройдя через воду, отражается от зеркала и выходит из воды.
    Найти расстояние между точкой входа луча в воду и точкой выхода луча из воды,
    если показатель преломления воды $1{,}33$, а угол падения луча $35\degrees$.
}
\answer{%
    \begin{align*}
    \ctg \beta &= \frac{h}{d:L:s} \implies d = \frac{h}{\ctg \beta} \\
    \frac 1{\sin^2 \beta} &= \ctg^2 \beta + 1 \implies \ctg \beta = \sqrt{\frac 1{\sin^2 \beta} - 1} \\
    \sin\alpha &= n\sin \beta \implies \sin \beta = \frac{\sin\alpha}{n} \\
    d &= \frac{h}{\sqrt{\frac 1{\sin^2 \beta} - 1}} = \frac{h}{\sqrt{\sqr{\frac{n}{\sin\alpha}} - 1}} \\
    2d &= \frac{2{h}}{\sqrt{\sqr{\frac{n}{\sin\alpha}} - 1}} \approx 382{,}4\,\text{см}
    \end{align*}
}
\solutionspace{120pt}

\tasknumber{3}%
\task{%
    Луч света падает на горизонтально расположенную стеклянную пластинку толщиной $1{,}3\,\text{см}$.
    Пройдя через пластину, он выходит из неё в точке, смещённой по горизонтали от точки падения на расстояние $4\,\text{мм}$.
    Показатель преломления стекла $1{,}6$.
    Найти синус угла падения.
}
\answer{%
    \begin{align*}
    \ctg \beta &= \frac{h}{d} \implies \\
    \implies \frac 1{\sin^2 \beta} &= \ctg^2 \beta + 1 = \sqr{\frac{h}{d}} + 1 \implies \\
    \implies \sin\alpha &= n\sin \beta = n\sqrt {\frac 1{\sqr{\frac{h}{d}} + 1}} \approx 0{,}47
    \end{align*}
}
\solutionspace{120pt}

\tasknumber{4}%
\task{%
    Определите диаметр полутени на экране диска размером $D = 3\,\text{см}$ от протяжённого источника, также обладающего формой диска размером $d = 2\,\text{см}$ (см.
    рис.
    на доске, вид сбоку).
    Расстояние от источника до диска равно $l = 18\,\text{см}$, а расстояние от диска до экрана — $L = 10\,\text{см}$.
}
\answer{%
    $\cfrac{\frac d2 + \frac D2}l = \cfrac{\frac d2 + r}{l + L} \implies r = \cfrac{Dl + dL + DL}{2l} = \cfrac D2 + \cfrac{ L }{ l } \cdot \cfrac{d+D}2 \approx 2{,}9\,\text{см} \implies 2r \approx 5{,}8\,\text{см}$
}
\solutionspace{80pt}

\tasknumber{5}%
\task{%
    Высота солнца над горизонтом составляет $ 35 \degrees$.
    Посреди ровного бетонного поля стоит одинокий цилиндрический столб высотой $3\,\text{м}$ и радиусом $8\,\text{см}$.
    Определите площадь полной тени от этого столба на бетоне.
    Угловой размер Солнца принять равным $30'$ и считать малым углом.
}
\answer{%
    $\text{трапеция} \implies S = 0{,}78\,\text{м}^{2}$
}

\variantsplitter

\addpersonalvariant{Константин Козлов}

\tasknumber{1}%
\task{%
    Сформулируйте:
    \begin{itemize}
        \item принцип Гюйгенса-Френеля,
        \item закон преломления (в двух частях).
    \end{itemize}
}
\solutionspace{60pt}

\tasknumber{2}%
\task{%
    На дне водоема глубиной $3\,\text{м}$ лежит зеркало.
    Луч света, пройдя через воду, отражается от зеркала и выходит из воды.
    Найти расстояние между точкой входа луча в воду и точкой выхода луча из воды,
    если показатель преломления воды $1{,}33$, а угол падения луча $25\degrees$.
}
\answer{%
    \begin{align*}
    \ctg \beta &= \frac{h}{d:L:s} \implies d = \frac{h}{\ctg \beta} \\
    \frac 1{\sin^2 \beta} &= \ctg^2 \beta + 1 \implies \ctg \beta = \sqrt{\frac 1{\sin^2 \beta} - 1} \\
    \sin\alpha &= n\sin \beta \implies \sin \beta = \frac{\sin\alpha}{n} \\
    d &= \frac{h}{\sqrt{\frac 1{\sin^2 \beta} - 1}} = \frac{h}{\sqrt{\sqr{\frac{n}{\sin\alpha}} - 1}} \\
    2d &= \frac{2{h}}{\sqrt{\sqr{\frac{n}{\sin\alpha}} - 1}} \approx 201{,}1\,\text{см}
    \end{align*}
}
\solutionspace{120pt}

\tasknumber{3}%
\task{%
    Луч света падает на вертикально расположенную стеклянную пластинку толщиной $1{,}5\,\text{см}$.
    Пройдя через пластину, он выходит из неё в точке, смещённой по вертикали от точки падения на расстояние $5\,\text{мм}$.
    Показатель преломления стекла $1{,}6$.
    Найти синус угла падения.
}
\answer{%
    \begin{align*}
    \ctg \beta &= \frac{h}{d} \implies \\
    \implies \frac 1{\sin^2 \beta} &= \ctg^2 \beta + 1 = \sqr{\frac{h}{d}} + 1 \implies \\
    \implies \sin\alpha &= n\sin \beta = n\sqrt {\frac 1{\sqr{\frac{h}{d}} + 1}} \approx 0{,}51
    \end{align*}
}
\solutionspace{120pt}

\tasknumber{4}%
\task{%
    Определите диаметр полутени на экране диска размером $D = 4\,\text{см}$ от протяжённого источника, также обладающего формой диска размером $d = 2\,\text{см}$ (см.
    рис.
    на доске, вид сбоку).
    Расстояние от источника до диска равно $l = 18\,\text{см}$, а расстояние от диска до экрана — $L = 20\,\text{см}$.
}
\answer{%
    $\cfrac{\frac d2 + \frac D2}l = \cfrac{\frac d2 + r}{l + L} \implies r = \cfrac{Dl + dL + DL}{2l} = \cfrac D2 + \cfrac{ L }{ l } \cdot \cfrac{d+D}2 \approx 5{,}3\,\text{см} \implies 2r \approx 10{,}7\,\text{см}$
}
\solutionspace{80pt}

\tasknumber{5}%
\task{%
    Высота солнца над горизонтом составляет $ 35 \degrees$.
    Посреди ровного бетонного поля стоит одинокий цилиндрический столб высотой $3\,\text{м}$ и радиусом $8\,\text{см}$.
    Определите площадь полной тени от этого столба на бетоне.
    Угловой размер Солнца принять равным $30'$ и считать малым углом.
}
\answer{%
    $\text{трапеция} \implies S = 0{,}78\,\text{м}^{2}$
}

\variantsplitter

\addpersonalvariant{Наталья Кравченко}

\tasknumber{1}%
\task{%
    Сформулируйте:
    \begin{itemize}
        \item принцип Гюйгенса-Френеля,
        \item закон отражения (в двух частях).
    \end{itemize}
}
\solutionspace{60pt}

\tasknumber{2}%
\task{%
    На дне водоема глубиной $4\,\text{м}$ лежит зеркало.
    Луч света, пройдя через воду, отражается от зеркала и выходит из воды.
    Найти расстояние между точкой входа луча в воду и точкой выхода луча из воды,
    если показатель преломления воды $1{,}33$, а угол падения луча $35\degrees$.
}
\answer{%
    \begin{align*}
    \ctg \beta &= \frac{h}{d:L:s} \implies d = \frac{h}{\ctg \beta} \\
    \frac 1{\sin^2 \beta} &= \ctg^2 \beta + 1 \implies \ctg \beta = \sqrt{\frac 1{\sin^2 \beta} - 1} \\
    \sin\alpha &= n\sin \beta \implies \sin \beta = \frac{\sin\alpha}{n} \\
    d &= \frac{h}{\sqrt{\frac 1{\sin^2 \beta} - 1}} = \frac{h}{\sqrt{\sqr{\frac{n}{\sin\alpha}} - 1}} \\
    2d &= \frac{2{h}}{\sqrt{\sqr{\frac{n}{\sin\alpha}} - 1}} \approx 382{,}4\,\text{см}
    \end{align*}
}
\solutionspace{120pt}

\tasknumber{3}%
\task{%
    Луч света падает на вертикально расположенную стеклянную пластинку толщиной $1{,}4\,\text{см}$.
    Пройдя через пластину, он выходит из неё в точке, смещённой по вертикали от точки падения на расстояние $6\,\text{мм}$.
    Показатель преломления стекла $1{,}5$.
    Найти синус угла падения.
}
\answer{%
    \begin{align*}
    \ctg \beta &= \frac{h}{d} \implies \\
    \implies \frac 1{\sin^2 \beta} &= \ctg^2 \beta + 1 = \sqr{\frac{h}{d}} + 1 \implies \\
    \implies \sin\alpha &= n\sin \beta = n\sqrt {\frac 1{\sqr{\frac{h}{d}} + 1}} \approx 0{,}59
    \end{align*}
}
\solutionspace{120pt}

\tasknumber{4}%
\task{%
    Определите диаметр полутени на экране диска размером $D = 2\,\text{см}$ от протяжённого источника, также обладающего формой диска размером $d = 2\,\text{см}$ (см.
    рис.
    на доске, вид сбоку).
    Расстояние от источника до диска равно $l = 18\,\text{см}$, а расстояние от диска до экрана — $L = 20\,\text{см}$.
}
\answer{%
    $\cfrac{\frac d2 + \frac D2}l = \cfrac{\frac d2 + r}{l + L} \implies r = \cfrac{Dl + dL + DL}{2l} = \cfrac D2 + \cfrac{ L }{ l } \cdot \cfrac{d+D}2 \approx 3{,}2\,\text{см} \implies 2r \approx 6{,}4\,\text{см}$
}
\solutionspace{80pt}

\tasknumber{5}%
\task{%
    Высота солнца над горизонтом составляет $ 50 \degrees$.
    Посреди ровного бетонного поля стоит одинокий цилиндрический столб высотой $5\,\text{м}$ и радиусом $2\,\text{см}$.
    Определите площадь полной тени от этого столба на бетоне.
    Угловой размер Солнца принять равным $30'$ и считать малым углом.
}
\answer{%
    $\text{треугольник} \implies S = 0{,}06\,\text{м}^{2}$
}

\variantsplitter

\addpersonalvariant{Матвей Кузьмин}

\tasknumber{1}%
\task{%
    Сформулируйте:
    \begin{itemize}
        \item принцип Гюйгенса-Френеля,
        \item закон отражения (в двух частях).
    \end{itemize}
}
\solutionspace{60pt}

\tasknumber{2}%
\task{%
    На дне водоема глубиной $4\,\text{м}$ лежит зеркало.
    Луч света, пройдя через воду, отражается от зеркала и выходит из воды.
    Найти расстояние между точкой входа луча в воду и точкой выхода луча из воды,
    если показатель преломления воды $1{,}33$, а угол падения луча $25\degrees$.
}
\answer{%
    \begin{align*}
    \ctg \beta &= \frac{h}{d:L:s} \implies d = \frac{h}{\ctg \beta} \\
    \frac 1{\sin^2 \beta} &= \ctg^2 \beta + 1 \implies \ctg \beta = \sqrt{\frac 1{\sin^2 \beta} - 1} \\
    \sin\alpha &= n\sin \beta \implies \sin \beta = \frac{\sin\alpha}{n} \\
    d &= \frac{h}{\sqrt{\frac 1{\sin^2 \beta} - 1}} = \frac{h}{\sqrt{\sqr{\frac{n}{\sin\alpha}} - 1}} \\
    2d &= \frac{2{h}}{\sqrt{\sqr{\frac{n}{\sin\alpha}} - 1}} \approx 268{,}1\,\text{см}
    \end{align*}
}
\solutionspace{120pt}

\tasknumber{3}%
\task{%
    Луч света падает на вертикально расположенную стеклянную пластинку толщиной $1{,}2\,\text{см}$.
    Пройдя через пластину, он выходит из неё в точке, смещённой по вертикали от точки падения на расстояние $4\,\text{мм}$.
    Показатель преломления стекла $1{,}4$.
    Найти синус угла падения.
}
\answer{%
    \begin{align*}
    \ctg \beta &= \frac{h}{d} \implies \\
    \implies \frac 1{\sin^2 \beta} &= \ctg^2 \beta + 1 = \sqr{\frac{h}{d}} + 1 \implies \\
    \implies \sin\alpha &= n\sin \beta = n\sqrt {\frac 1{\sqr{\frac{h}{d}} + 1}} \approx 0{,}44
    \end{align*}
}
\solutionspace{120pt}

\tasknumber{4}%
\task{%
    Определите радиус полутени на экране диска размером $D = 2\,\text{см}$ от протяжённого источника, также обладающего формой диска размером $d = 2\,\text{см}$ (см.
    рис.
    на доске, вид сбоку).
    Расстояние от источника до диска равно $l = 18\,\text{см}$, а расстояние от диска до экрана — $L = 20\,\text{см}$.
}
\answer{%
    $\cfrac{\frac d2 + \frac D2}l = \cfrac{\frac d2 + r}{l + L} \implies r = \cfrac{Dl + dL + DL}{2l} = \cfrac D2 + \cfrac{ L }{ l } \cdot \cfrac{d+D}2 \approx 3{,}2\,\text{см} \implies 2r \approx 6{,}4\,\text{см}$
}
\solutionspace{80pt}

\tasknumber{5}%
\task{%
    Высота солнца над горизонтом составляет $ 55 \degrees$.
    Посреди ровного бетонного поля стоит одинокий цилиндрический столб высотой $7\,\text{м}$ и радиусом $8\,\text{см}$.
    Определите площадь полной тени от этого столба на бетоне.
    Угловой размер Солнца принять равным $30'$ и считать малым углом.
}
\answer{%
    $\text{трапеция} \implies S = 0{,}97\,\text{м}^{2}$
}

\variantsplitter

\addpersonalvariant{Сергей Малышев}

\tasknumber{1}%
\task{%
    Сформулируйте:
    \begin{itemize}
        \item принцип Гюйгенса-Френеля,
        \item закон отражения (в двух частях).
    \end{itemize}
}
\solutionspace{60pt}

\tasknumber{2}%
\task{%
    На дне водоема глубиной $2\,\text{м}$ лежит зеркало.
    Луч света, пройдя через воду, отражается от зеркала и выходит из воды.
    Найти расстояние между точкой входа луча в воду и точкой выхода луча из воды,
    если показатель преломления воды $1{,}33$, а угол падения луча $30\degrees$.
}
\answer{%
    \begin{align*}
    \ctg \beta &= \frac{h}{d:L:s} \implies d = \frac{h}{\ctg \beta} \\
    \frac 1{\sin^2 \beta} &= \ctg^2 \beta + 1 \implies \ctg \beta = \sqrt{\frac 1{\sin^2 \beta} - 1} \\
    \sin\alpha &= n\sin \beta \implies \sin \beta = \frac{\sin\alpha}{n} \\
    d &= \frac{h}{\sqrt{\frac 1{\sin^2 \beta} - 1}} = \frac{h}{\sqrt{\sqr{\frac{n}{\sin\alpha}} - 1}} \\
    2d &= \frac{2{h}}{\sqrt{\sqr{\frac{n}{\sin\alpha}} - 1}} \approx 162{,}3\,\text{см}
    \end{align*}
}
\solutionspace{120pt}

\tasknumber{3}%
\task{%
    Луч света падает на вертикально расположенную стеклянную пластинку толщиной $1{,}5\,\text{см}$.
    Пройдя через пластину, он выходит из неё в точке, смещённой по вертикали от точки падения на расстояние $5\,\text{мм}$.
    Показатель преломления стекла $1{,}4$.
    Найти синус угла падения.
}
\answer{%
    \begin{align*}
    \ctg \beta &= \frac{h}{d} \implies \\
    \implies \frac 1{\sin^2 \beta} &= \ctg^2 \beta + 1 = \sqr{\frac{h}{d}} + 1 \implies \\
    \implies \sin\alpha &= n\sin \beta = n\sqrt {\frac 1{\sqr{\frac{h}{d}} + 1}} \approx 0{,}44
    \end{align*}
}
\solutionspace{120pt}

\tasknumber{4}%
\task{%
    Определите диаметр полутени на экране диска размером $D = 3\,\text{см}$ от протяжённого источника, также обладающего формой диска размером $d = 3\,\text{см}$ (см.
    рис.
    на доске, вид сбоку).
    Расстояние от источника до диска равно $l = 15\,\text{см}$, а расстояние от диска до экрана — $L = 30\,\text{см}$.
}
\answer{%
    $\cfrac{\frac d2 + \frac D2}l = \cfrac{\frac d2 + r}{l + L} \implies r = \cfrac{Dl + dL + DL}{2l} = \cfrac D2 + \cfrac{ L }{ l } \cdot \cfrac{d+D}2 \approx 7{,}5\,\text{см} \implies 2r \approx 15\,\text{см}$
}
\solutionspace{80pt}

\tasknumber{5}%
\task{%
    Высота солнца над горизонтом составляет $ 40 \degrees$.
    Посреди ровного бетонного поля стоит одинокий цилиндрический столб высотой $3\,\text{м}$ и радиусом $2\,\text{см}$.
    Определите площадь полной тени от этого столба на бетоне.
    Угловой размер Солнца принять равным $30'$ и считать малым углом.
}
\answer{%
    $\text{треугольник} \implies S = 0{,}07\,\text{м}^{2}$
}

\variantsplitter

\addpersonalvariant{Алина Полканова}

\tasknumber{1}%
\task{%
    Сформулируйте:
    \begin{itemize}
        \item принцип Гюйгенса-Френеля,
        \item закон преломления (в двух частях).
    \end{itemize}
}
\solutionspace{60pt}

\tasknumber{2}%
\task{%
    На дне водоема глубиной $3\,\text{м}$ лежит зеркало.
    Луч света, пройдя через воду, отражается от зеркала и выходит из воды.
    Найти расстояние между точкой входа луча в воду и точкой выхода луча из воды,
    если показатель преломления воды $1{,}33$, а угол падения луча $35\degrees$.
}
\answer{%
    \begin{align*}
    \ctg \beta &= \frac{h}{d:L:s} \implies d = \frac{h}{\ctg \beta} \\
    \frac 1{\sin^2 \beta} &= \ctg^2 \beta + 1 \implies \ctg \beta = \sqrt{\frac 1{\sin^2 \beta} - 1} \\
    \sin\alpha &= n\sin \beta \implies \sin \beta = \frac{\sin\alpha}{n} \\
    d &= \frac{h}{\sqrt{\frac 1{\sin^2 \beta} - 1}} = \frac{h}{\sqrt{\sqr{\frac{n}{\sin\alpha}} - 1}} \\
    2d &= \frac{2{h}}{\sqrt{\sqr{\frac{n}{\sin\alpha}} - 1}} \approx 286{,}8\,\text{см}
    \end{align*}
}
\solutionspace{120pt}

\tasknumber{3}%
\task{%
    Луч света падает на горизонтально расположенную стеклянную пластинку толщиной $1{,}4\,\text{см}$.
    Пройдя через пластину, он выходит из неё в точке, смещённой по горизонтали от точки падения на расстояние $5\,\text{мм}$.
    Показатель преломления стекла $1{,}4$.
    Найти синус угла падения.
}
\answer{%
    \begin{align*}
    \ctg \beta &= \frac{h}{d} \implies \\
    \implies \frac 1{\sin^2 \beta} &= \ctg^2 \beta + 1 = \sqr{\frac{h}{d}} + 1 \implies \\
    \implies \sin\alpha &= n\sin \beta = n\sqrt {\frac 1{\sqr{\frac{h}{d}} + 1}} \approx 0{,}47
    \end{align*}
}
\solutionspace{120pt}

\tasknumber{4}%
\task{%
    Определите радиус полутени на экране диска размером $D = 4\,\text{см}$ от протяжённого источника, также обладающего формой диска размером $d = 2\,\text{см}$ (см.
    рис.
    на доске, вид сбоку).
    Расстояние от источника до диска равно $l = 12\,\text{см}$, а расстояние от диска до экрана — $L = 10\,\text{см}$.
}
\answer{%
    $\cfrac{\frac d2 + \frac D2}l = \cfrac{\frac d2 + r}{l + L} \implies r = \cfrac{Dl + dL + DL}{2l} = \cfrac D2 + \cfrac{ L }{ l } \cdot \cfrac{d+D}2 \approx 4{,}5\,\text{см} \implies 2r \approx 9\,\text{см}$
}
\solutionspace{80pt}

\tasknumber{5}%
\task{%
    Высота солнца над горизонтом составляет $ 35 \degrees$.
    Посреди ровного бетонного поля стоит одинокий цилиндрический столб высотой $5\,\text{м}$ и радиусом $4\,\text{см}$.
    Определите площадь полной тени от этого столба на бетоне.
    Угловой размер Солнца принять равным $30'$ и считать малым углом.
}
\answer{%
    $\text{трапеция} \implies S = 0{,}84\,\text{м}^{2}$
}

\variantsplitter

\addpersonalvariant{Сергей Пономарёв}

\tasknumber{1}%
\task{%
    Сформулируйте:
    \begin{itemize}
        \item принцип Гюйгенса-Френеля,
        \item закон преломления (в двух частях).
    \end{itemize}
}
\solutionspace{60pt}

\tasknumber{2}%
\task{%
    На дне водоема глубиной $2\,\text{м}$ лежит зеркало.
    Луч света, пройдя через воду, отражается от зеркала и выходит из воды.
    Найти расстояние между точкой входа луча в воду и точкой выхода луча из воды,
    если показатель преломления воды $1{,}33$, а угол падения луча $25\degrees$.
}
\answer{%
    \begin{align*}
    \ctg \beta &= \frac{h}{d:L:s} \implies d = \frac{h}{\ctg \beta} \\
    \frac 1{\sin^2 \beta} &= \ctg^2 \beta + 1 \implies \ctg \beta = \sqrt{\frac 1{\sin^2 \beta} - 1} \\
    \sin\alpha &= n\sin \beta \implies \sin \beta = \frac{\sin\alpha}{n} \\
    d &= \frac{h}{\sqrt{\frac 1{\sin^2 \beta} - 1}} = \frac{h}{\sqrt{\sqr{\frac{n}{\sin\alpha}} - 1}} \\
    2d &= \frac{2{h}}{\sqrt{\sqr{\frac{n}{\sin\alpha}} - 1}} \approx 134{,}1\,\text{см}
    \end{align*}
}
\solutionspace{120pt}

\tasknumber{3}%
\task{%
    Луч света падает на вертикально расположенную стеклянную пластинку толщиной $1{,}2\,\text{см}$.
    Пройдя через пластину, он выходит из неё в точке, смещённой по вертикали от точки падения на расстояние $6\,\text{мм}$.
    Показатель преломления стекла $1{,}5$.
    Найти синус угла падения.
}
\answer{%
    \begin{align*}
    \ctg \beta &= \frac{h}{d} \implies \\
    \implies \frac 1{\sin^2 \beta} &= \ctg^2 \beta + 1 = \sqr{\frac{h}{d}} + 1 \implies \\
    \implies \sin\alpha &= n\sin \beta = n\sqrt {\frac 1{\sqr{\frac{h}{d}} + 1}} \approx 0{,}67
    \end{align*}
}
\solutionspace{120pt}

\tasknumber{4}%
\task{%
    Определите диаметр полутени на экране диска размером $D = 4\,\text{см}$ от протяжённого источника, также обладающего формой диска размером $d = 2\,\text{см}$ (см.
    рис.
    на доске, вид сбоку).
    Расстояние от источника до диска равно $l = 18\,\text{см}$, а расстояние от диска до экрана — $L = 30\,\text{см}$.
}
\answer{%
    $\cfrac{\frac d2 + \frac D2}l = \cfrac{\frac d2 + r}{l + L} \implies r = \cfrac{Dl + dL + DL}{2l} = \cfrac D2 + \cfrac{ L }{ l } \cdot \cfrac{d+D}2 \approx 7\,\text{см} \implies 2r \approx 14\,\text{см}$
}
\solutionspace{80pt}

\tasknumber{5}%
\task{%
    Высота солнца над горизонтом составляет $ 35 \degrees$.
    Посреди ровного бетонного поля стоит одинокий цилиндрический столб высотой $3\,\text{м}$ и радиусом $4\,\text{см}$.
    Определите площадь полной тени от этого столба на бетоне.
    Угловой размер Солнца принять равным $30'$ и считать малым углом.
}
\answer{%
    $\text{трапеция} \implies S = 0{,}44\,\text{м}^{2}$
}

\variantsplitter

\addpersonalvariant{Егор Свистушкин}

\tasknumber{1}%
\task{%
    Сформулируйте:
    \begin{itemize}
        \item принцип Гюйгенса-Френеля,
        \item закон преломления (в двух частях).
    \end{itemize}
}
\solutionspace{60pt}

\tasknumber{2}%
\task{%
    На дне водоема глубиной $3\,\text{м}$ лежит зеркало.
    Луч света, пройдя через воду, отражается от зеркала и выходит из воды.
    Найти расстояние между точкой входа луча в воду и точкой выхода луча из воды,
    если показатель преломления воды $1{,}33$, а угол падения луча $25\degrees$.
}
\answer{%
    \begin{align*}
    \ctg \beta &= \frac{h}{d:L:s} \implies d = \frac{h}{\ctg \beta} \\
    \frac 1{\sin^2 \beta} &= \ctg^2 \beta + 1 \implies \ctg \beta = \sqrt{\frac 1{\sin^2 \beta} - 1} \\
    \sin\alpha &= n\sin \beta \implies \sin \beta = \frac{\sin\alpha}{n} \\
    d &= \frac{h}{\sqrt{\frac 1{\sin^2 \beta} - 1}} = \frac{h}{\sqrt{\sqr{\frac{n}{\sin\alpha}} - 1}} \\
    2d &= \frac{2{h}}{\sqrt{\sqr{\frac{n}{\sin\alpha}} - 1}} \approx 201{,}1\,\text{см}
    \end{align*}
}
\solutionspace{120pt}

\tasknumber{3}%
\task{%
    Луч света падает на вертикально расположенную стеклянную пластинку толщиной $1{,}6\,\text{см}$.
    Пройдя через пластину, он выходит из неё в точке, смещённой по вертикали от точки падения на расстояние $5\,\text{мм}$.
    Показатель преломления стекла $1{,}6$.
    Найти синус угла падения.
}
\answer{%
    \begin{align*}
    \ctg \beta &= \frac{h}{d} \implies \\
    \implies \frac 1{\sin^2 \beta} &= \ctg^2 \beta + 1 = \sqr{\frac{h}{d}} + 1 \implies \\
    \implies \sin\alpha &= n\sin \beta = n\sqrt {\frac 1{\sqr{\frac{h}{d}} + 1}} \approx 0{,}48
    \end{align*}
}
\solutionspace{120pt}

\tasknumber{4}%
\task{%
    Определите диаметр полутени на экране диска размером $D = 2{,}5\,\text{см}$ от протяжённого источника, также обладающего формой диска размером $d = 4\,\text{см}$ (см.
    рис.
    на доске, вид сбоку).
    Расстояние от источника до диска равно $l = 15\,\text{см}$, а расстояние от диска до экрана — $L = 30\,\text{см}$.
}
\answer{%
    $\cfrac{\frac d2 + \frac D2}l = \cfrac{\frac d2 + r}{l + L} \implies r = \cfrac{Dl + dL + DL}{2l} = \cfrac D2 + \cfrac{ L }{ l } \cdot \cfrac{d+D}2 \approx 7{,}8\,\text{см} \implies 2r \approx 15{,}5\,\text{см}$
}
\solutionspace{80pt}

\tasknumber{5}%
\task{%
    Высота солнца над горизонтом составляет $ 40 \degrees$.
    Посреди ровного бетонного поля стоит одинокий цилиндрический столб высотой $3\,\text{м}$ и радиусом $4\,\text{см}$.
    Определите площадь полной тени от этого столба на бетоне.
    Угловой размер Солнца принять равным $30'$ и считать малым углом.
}
\answer{%
    $\text{трапеция} \implies S = 0{,}36\,\text{м}^{2}$
}

\variantsplitter

\addpersonalvariant{Дмитрий Соколов}

\tasknumber{1}%
\task{%
    Сформулируйте:
    \begin{itemize}
        \item принцип Гюйгенса-Френеля,
        \item закон преломления (в двух частях).
    \end{itemize}
}
\solutionspace{60pt}

\tasknumber{2}%
\task{%
    На дне водоема глубиной $4\,\text{м}$ лежит зеркало.
    Луч света, пройдя через воду, отражается от зеркала и выходит из воды.
    Найти расстояние между точкой входа луча в воду и точкой выхода луча из воды,
    если показатель преломления воды $1{,}33$, а угол падения луча $25\degrees$.
}
\answer{%
    \begin{align*}
    \ctg \beta &= \frac{h}{d:L:s} \implies d = \frac{h}{\ctg \beta} \\
    \frac 1{\sin^2 \beta} &= \ctg^2 \beta + 1 \implies \ctg \beta = \sqrt{\frac 1{\sin^2 \beta} - 1} \\
    \sin\alpha &= n\sin \beta \implies \sin \beta = \frac{\sin\alpha}{n} \\
    d &= \frac{h}{\sqrt{\frac 1{\sin^2 \beta} - 1}} = \frac{h}{\sqrt{\sqr{\frac{n}{\sin\alpha}} - 1}} \\
    2d &= \frac{2{h}}{\sqrt{\sqr{\frac{n}{\sin\alpha}} - 1}} \approx 268{,}1\,\text{см}
    \end{align*}
}
\solutionspace{120pt}

\tasknumber{3}%
\task{%
    Луч света падает на вертикально расположенную стеклянную пластинку толщиной $1{,}4\,\text{см}$.
    Пройдя через пластину, он выходит из неё в точке, смещённой по вертикали от точки падения на расстояние $6\,\text{мм}$.
    Показатель преломления стекла $1{,}4$.
    Найти синус угла падения.
}
\answer{%
    \begin{align*}
    \ctg \beta &= \frac{h}{d} \implies \\
    \implies \frac 1{\sin^2 \beta} &= \ctg^2 \beta + 1 = \sqr{\frac{h}{d}} + 1 \implies \\
    \implies \sin\alpha &= n\sin \beta = n\sqrt {\frac 1{\sqr{\frac{h}{d}} + 1}} \approx 0{,}55
    \end{align*}
}
\solutionspace{120pt}

\tasknumber{4}%
\task{%
    Определите диаметр полутени на экране диска размером $D = 3{,}5\,\text{см}$ от протяжённого источника, также обладающего формой диска размером $d = 4\,\text{см}$ (см.
    рис.
    на доске, вид сбоку).
    Расстояние от источника до диска равно $l = 18\,\text{см}$, а расстояние от диска до экрана — $L = 20\,\text{см}$.
}
\answer{%
    $\cfrac{\frac d2 + \frac D2}l = \cfrac{\frac d2 + r}{l + L} \implies r = \cfrac{Dl + dL + DL}{2l} = \cfrac D2 + \cfrac{ L }{ l } \cdot \cfrac{d+D}2 \approx 5{,}9\,\text{см} \implies 2r \approx 11{,}8\,\text{см}$
}
\solutionspace{80pt}

\tasknumber{5}%
\task{%
    Высота солнца над горизонтом составляет $ 55 \degrees$.
    Посреди ровного бетонного поля стоит одинокий цилиндрический столб высотой $5\,\text{м}$ и радиусом $10\,\text{см}$.
    Определите площадь полной тени от этого столба на бетоне.
    Угловой размер Солнца принять равным $30'$ и считать малым углом.
}
\answer{%
    $\text{трапеция} \implies S = 0{,}79\,\text{м}^{2}$
}

\variantsplitter

\addpersonalvariant{Арсений Трофимов}

\tasknumber{1}%
\task{%
    Сформулируйте:
    \begin{itemize}
        \item принцип Гюйгенса-Френеля,
        \item закон отражения (в двух частях).
    \end{itemize}
}
\solutionspace{60pt}

\tasknumber{2}%
\task{%
    На дне водоема глубиной $3\,\text{м}$ лежит зеркало.
    Луч света, пройдя через воду, отражается от зеркала и выходит из воды.
    Найти расстояние между точкой входа луча в воду и точкой выхода луча из воды,
    если показатель преломления воды $1{,}33$, а угол падения луча $30\degrees$.
}
\answer{%
    \begin{align*}
    \ctg \beta &= \frac{h}{d:L:s} \implies d = \frac{h}{\ctg \beta} \\
    \frac 1{\sin^2 \beta} &= \ctg^2 \beta + 1 \implies \ctg \beta = \sqrt{\frac 1{\sin^2 \beta} - 1} \\
    \sin\alpha &= n\sin \beta \implies \sin \beta = \frac{\sin\alpha}{n} \\
    d &= \frac{h}{\sqrt{\frac 1{\sin^2 \beta} - 1}} = \frac{h}{\sqrt{\sqr{\frac{n}{\sin\alpha}} - 1}} \\
    2d &= \frac{2{h}}{\sqrt{\sqr{\frac{n}{\sin\alpha}} - 1}} \approx 243{,}4\,\text{см}
    \end{align*}
}
\solutionspace{120pt}

\tasknumber{3}%
\task{%
    Луч света падает на вертикально расположенную стеклянную пластинку толщиной $1{,}2\,\text{см}$.
    Пройдя через пластину, он выходит из неё в точке, смещённой по вертикали от точки падения на расстояние $5\,\text{мм}$.
    Показатель преломления стекла $1{,}6$.
    Найти синус угла падения.
}
\answer{%
    \begin{align*}
    \ctg \beta &= \frac{h}{d} \implies \\
    \implies \frac 1{\sin^2 \beta} &= \ctg^2 \beta + 1 = \sqr{\frac{h}{d}} + 1 \implies \\
    \implies \sin\alpha &= n\sin \beta = n\sqrt {\frac 1{\sqr{\frac{h}{d}} + 1}} \approx 0{,}62
    \end{align*}
}
\solutionspace{120pt}

\tasknumber{4}%
\task{%
    Определите радиус полутени на экране диска размером $D = 2\,\text{см}$ от протяжённого источника, также обладающего формой диска размером $d = 3\,\text{см}$ (см.
    рис.
    на доске, вид сбоку).
    Расстояние от источника до диска равно $l = 18\,\text{см}$, а расстояние от диска до экрана — $L = 30\,\text{см}$.
}
\answer{%
    $\cfrac{\frac d2 + \frac D2}l = \cfrac{\frac d2 + r}{l + L} \implies r = \cfrac{Dl + dL + DL}{2l} = \cfrac D2 + \cfrac{ L }{ l } \cdot \cfrac{d+D}2 \approx 5{,}2\,\text{см} \implies 2r \approx 10{,}3\,\text{см}$
}
\solutionspace{80pt}

\tasknumber{5}%
\task{%
    Высота солнца над горизонтом составляет $ 55 \degrees$.
    Посреди ровного бетонного поля стоит одинокий цилиндрический столб высотой $7\,\text{м}$ и радиусом $2\,\text{см}$.
    Определите площадь полной тени от этого столба на бетоне.
    Угловой размер Солнца принять равным $30'$ и считать малым углом.
}
\answer{%
    $\text{треугольник} \implies S = 0{,}05\,\text{м}^{2}$
}
% autogenerated
