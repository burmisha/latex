\setdate{8~декабря~2021}
\setclass{11«Б»}

\addpersonalvariant{Михаил Бурмистров}

\tasknumber{1}%
\task{%
    Напротив физических величин укажите их обозначения и единицы измерения в СИ, а в пункте «г)» запишите физический закон или формулу:
    \begin{enumerate}
        \item скорость света в вакууме,
        \item частоты волны,
        \item период колебаний напряжённости электрического поля в электромагнитной волне,
        \item относительный показатель преломления среды.
    \end{enumerate}
}
\solutionspace{20pt}

\tasknumber{2}%
\task{%
    Получите из базовых физических законов:
    \begin{enumerate}
        \item период колебаний через длину волны и скорость её распространения,
        \item энергию фотона через длину электромагнитной волны,
        \item скорость света в вакууме через скорость света в среде и её абсолютный показатель преломления.
    \end{enumerate}
}
\solutionspace{60pt}

\tasknumber{3}%
\task{%
    Определить абсолютный показатель преломления прозрачной среды,
    в которой распространяется свет с длиной волны $0{,}650\,\text{мкм}$ и частотой $360\,\text{ТГц}$.
    Скорость света в вакууме $3 \cdot 10^{8}\,\frac{\text{м}}{\text{с}}$.
}
\answer{%
    $
        n = \frac{c}{v}
        = \frac{c}{\frac \lambda T}
        = \frac{c}{\lambda \nu}
        = \frac{3 \cdot 10^{8}\,\frac{\text{м}}{\text{с}}}{0{,}650\,\text{мкм} \cdot {360\,\text{ТГц}}}
        \approx1{,}30
    $
}
\solutionspace{80pt}

\tasknumber{4}%
\task{%
    В некоторую точку пространства приходят две когерентные световые волны
    с разностью хода $2{,}450\,\text{мкм}$.
    Определите, что наблюдается в этой точке.
    Длина волны равна $700\,\text{нм}$.
}
\answer{%
    $\text{точка минимума}$
}
\solutionspace{80pt}

\tasknumber{5}%
\task{%
    Разность фаз двух интерферирующих световых волн равна $6\pi$,
    а разность хода между ними равна $10{,}5 \cdot 10^{-7}\,\text{м}$.
    Определить длину и частоту волны.
}
\answer{%
    \begin{align*}
    \Delta \varphi &= k\Delta l = \frac{2 \pi}{\lambda} \Delta l = 6\pi \implies \lambda = \frac13\Delta l \approx 350\,\text{нм}, \\
    &\nu = \frac 1T = \frac c\lambda = 3 \frac c{\Delta l} = 3 \cdot \frac{3 \cdot 10^{8}\,\frac{\text{м}}{\text{с}}}{10{,}5 \cdot 10^{-7}\,\text{м}} \approx 857\,\text{ТГц}.
    \end{align*}
}
\solutionspace{100pt}

\tasknumber{6}%
\task{%
    Установка для наблюдения интерференции состоит
    из двух когерентных источников света и экрана.
    Расстояние между источниками $l = 2{,}4\,\text{мм}$,
    а от каждого источника до экрана — $L = 3\,\text{м}$.
    Сделайте рисунок и укажите положение нулевого максимума освещенности,
    а также определите расстояние между четвёртым максимумом и нулевым максимумом.
    Длина волны падающего света составляет $\lambda = 550\,\text{нм}$.
}
\answer{%
    \begin{align*}
    l_1^2 &= L^2 + \sqr{x - \frac \ell 2} \\
    l_2^2 &= L^2 + \sqr{x + \frac \ell 2} \\
    l_2^2 - l_1^2 &= 2x\ell \implies (l_2 - l_1)(l_2 + l_1) = 2x\ell \implies n\lambda \cdot 2L \approx 2x_n\ell \implies x_n = \frac{\lambda L}{\ell} n, n\in \mathbb{N} \\
    x &= \frac{\lambda L}{\ell} \cdot 4 = \frac{550\,\text{нм} \cdot 3\,\text{м}}{2{,}4\,\text{мм}} \cdot 4 \approx 2{,}8\,\text{мм}
    \end{align*}
}

\variantsplitter

\addpersonalvariant{Снежана Авдошина}

\tasknumber{1}%
\task{%
    Напротив физических величин укажите их обозначения и единицы измерения в СИ, а в пункте «г)» запишите физический закон или формулу:
    \begin{enumerate}
        \item скорость света в вакууме,
        \item частоты волны,
        \item период колебаний напряжённости электрического поля в электромагнитной волне,
        \item относительный показатель преломления среды.
    \end{enumerate}
}
\solutionspace{20pt}

\tasknumber{2}%
\task{%
    Получите из базовых физических законов:
    \begin{enumerate}
        \item частоту колебаний через длину волны и скорость её распространения,
        \item энергию фотона через длину электромагнитной волны,
        \item скорость света в вакууме через скорость света в среде и её абсолютный показатель преломления.
    \end{enumerate}
}
\solutionspace{60pt}

\tasknumber{3}%
\task{%
    Определить абсолютный показатель преломления прозрачной среды,
    в которой распространяется свет с длиной волны $0{,}650\,\text{мкм}$ и частотой $270\,\text{ТГц}$.
    Скорость света в вакууме $3 \cdot 10^{8}\,\frac{\text{м}}{\text{с}}$.
}
\answer{%
    $
        n = \frac{c}{v}
        = \frac{c}{\frac \lambda T}
        = \frac{c}{\lambda \nu}
        = \frac{3 \cdot 10^{8}\,\frac{\text{м}}{\text{с}}}{0{,}650\,\text{мкм} \cdot {270\,\text{ТГц}}}
        \approx1{,}70
    $
}
\solutionspace{80pt}

\tasknumber{4}%
\task{%
    В некоторую точку пространства приходят две когерентные световые волны
    с разностью хода $2{,}400\,\text{мкм}$.
    Определите, что наблюдается в этой точке.
    Длина волны равна $600\,\text{нм}$.
}
\answer{%
    $\text{точка максимума}$
}
\solutionspace{80pt}

\tasknumber{5}%
\task{%
    Разность фаз двух интерферирующих световых волн равна $3\pi$,
    а разность хода между ними равна $9{,}5 \cdot 10^{-7}\,\text{м}$.
    Определить длину и частоту волны.
}
\answer{%
    \begin{align*}
    \Delta \varphi &= k\Delta l = \frac{2 \pi}{\lambda} \Delta l = 3\pi \implies \lambda = \frac23\Delta l \approx 633\,\text{нм}, \\
    &\nu = \frac 1T = \frac c\lambda = \frac32 \frac c{\Delta l} = \frac32 \cdot \frac{3 \cdot 10^{8}\,\frac{\text{м}}{\text{с}}}{9{,}5 \cdot 10^{-7}\,\text{м}} \approx 474\,\text{ТГц}.
    \end{align*}
}
\solutionspace{100pt}

\tasknumber{6}%
\task{%
    Установка для наблюдения интерференции состоит
    из двух когерентных источников света и экрана.
    Расстояние между источниками $l = 1{,}2\,\text{мм}$,
    а от каждого источника до экрана — $L = 4\,\text{м}$.
    Сделайте рисунок и укажите положение нулевого максимума освещенности,
    а также определите расстояние между четвёртым минимумом и нулевым максимумом.
    Длина волны падающего света составляет $\lambda = 400\,\text{нм}$.
}
\answer{%
    \begin{align*}
    l_1^2 &= L^2 + \sqr{x - \frac \ell 2} \\
    l_2^2 &= L^2 + \sqr{x + \frac \ell 2} \\
    l_2^2 - l_1^2 &= 2x\ell \implies (l_2 - l_1)(l_2 + l_1) = 2x\ell \implies n\lambda \cdot 2L \approx 2x_n\ell \implies x_n = \frac{\lambda L}{\ell} n, n\in \mathbb{N} \\
    x &= \frac{\lambda L}{\ell} \cdot \frac72 = \frac{400\,\text{нм} \cdot 4\,\text{м}}{1{,}2\,\text{мм}} \cdot \frac72 \approx 4{,}7\,\text{мм}
    \end{align*}
}

\variantsplitter

\addpersonalvariant{Марьяна Аристова}

\tasknumber{1}%
\task{%
    Напротив физических величин укажите их обозначения и единицы измерения в СИ, а в пункте «г)» запишите физический закон или формулу:
    \begin{enumerate}
        \item скорость света в вакууме,
        \item частоты волны,
        \item период колебаний напряжённости электрического поля в электромагнитной волне,
        \item абсолютный показатель преломления среды.
    \end{enumerate}
}
\solutionspace{20pt}

\tasknumber{2}%
\task{%
    Получите из базовых физических законов:
    \begin{enumerate}
        \item период колебаний через длину волны и скорость её распространения,
        \item энергию фотона через период колебаний в электромагнитной волне,
        \item скорость света в среде через её абсолютный показатель преломления и скорость света в вакууме.
    \end{enumerate}
}
\solutionspace{60pt}

\tasknumber{3}%
\task{%
    Определить абсолютный показатель преломления прозрачной среды,
    в которой распространяется свет с длиной волны $0{,}650\,\text{мкм}$ и частотой $270\,\text{ТГц}$.
    Скорость света в вакууме $3 \cdot 10^{8}\,\frac{\text{м}}{\text{с}}$.
}
\answer{%
    $
        n = \frac{c}{v}
        = \frac{c}{\frac \lambda T}
        = \frac{c}{\lambda \nu}
        = \frac{3 \cdot 10^{8}\,\frac{\text{м}}{\text{с}}}{0{,}650\,\text{мкм} \cdot {270\,\text{ТГц}}}
        \approx1{,}70
    $
}
\solutionspace{80pt}

\tasknumber{4}%
\task{%
    В некоторую точку пространства приходят две когерентные световые волны
    с разностью хода $1{,}8000\,\text{мкм}$.
    Определите, что наблюдается в этой точке.
    Длина волны равна $600\,\text{нм}$.
}
\answer{%
    $\text{точка максимума}$
}
\solutionspace{80pt}

\tasknumber{5}%
\task{%
    Разность фаз двух интерферирующих световых волн равна $5\pi$,
    а разность хода между ними равна $9{,}5 \cdot 10^{-7}\,\text{м}$.
    Определить длину и частоту волны.
}
\answer{%
    \begin{align*}
    \Delta \varphi &= k\Delta l = \frac{2 \pi}{\lambda} \Delta l = 5\pi \implies \lambda = \frac25\Delta l \approx 380\,\text{нм}, \\
    &\nu = \frac 1T = \frac c\lambda = \frac52 \frac c{\Delta l} = \frac52 \cdot \frac{3 \cdot 10^{8}\,\frac{\text{м}}{\text{с}}}{9{,}5 \cdot 10^{-7}\,\text{м}} \approx 789\,\text{ТГц}.
    \end{align*}
}
\solutionspace{100pt}

\tasknumber{6}%
\task{%
    Установка для наблюдения интерференции состоит
    из двух когерентных источников света и экрана.
    Расстояние между источниками $l = 0{,}8\,\text{мм}$,
    а от каждого источника до экрана — $L = 4\,\text{м}$.
    Сделайте рисунок и укажите положение нулевого максимума освещенности,
    а также определите расстояние между вторым максимумом и нулевым максимумом.
    Длина волны падающего света составляет $\lambda = 400\,\text{нм}$.
}
\answer{%
    \begin{align*}
    l_1^2 &= L^2 + \sqr{x - \frac \ell 2} \\
    l_2^2 &= L^2 + \sqr{x + \frac \ell 2} \\
    l_2^2 - l_1^2 &= 2x\ell \implies (l_2 - l_1)(l_2 + l_1) = 2x\ell \implies n\lambda \cdot 2L \approx 2x_n\ell \implies x_n = \frac{\lambda L}{\ell} n, n\in \mathbb{N} \\
    x &= \frac{\lambda L}{\ell} \cdot 2 = \frac{400\,\text{нм} \cdot 4\,\text{м}}{0{,}8\,\text{мм}} \cdot 2 \approx 4\,\text{мм}
    \end{align*}
}

\variantsplitter

\addpersonalvariant{Никита Иванов}

\tasknumber{1}%
\task{%
    Напротив физических величин укажите их обозначения и единицы измерения в СИ, а в пункте «г)» запишите физический закон или формулу:
    \begin{enumerate}
        \item скорость света в среде,
        \item длина волны,
        \item период колебаний напряжённости электрического поля в электромагнитной волне,
        \item абсолютный показатель преломления среды.
    \end{enumerate}
}
\solutionspace{20pt}

\tasknumber{2}%
\task{%
    Получите из базовых физических законов:
    \begin{enumerate}
        \item частоту колебаний через длину волны и скорость её распространения,
        \item энергию фотона через длину электромагнитной волны,
        \item скорость света в вакууме через скорость света в среде и её абсолютный показатель преломления.
    \end{enumerate}
}
\solutionspace{60pt}

\tasknumber{3}%
\task{%
    Определить абсолютный показатель преломления прозрачной среды,
    в которой распространяется свет с длиной волны $0{,}650\,\text{мкм}$ и частотой $330\,\text{ТГц}$.
    Скорость света в вакууме $3 \cdot 10^{8}\,\frac{\text{м}}{\text{с}}$.
}
\answer{%
    $
        n = \frac{c}{v}
        = \frac{c}{\frac \lambda T}
        = \frac{c}{\lambda \nu}
        = \frac{3 \cdot 10^{8}\,\frac{\text{м}}{\text{с}}}{0{,}650\,\text{мкм} \cdot {330\,\text{ТГц}}}
        \approx1{,}40
    $
}
\solutionspace{80pt}

\tasknumber{4}%
\task{%
    В некоторую точку пространства приходят две когерентные световые волны
    с разностью хода $1{,}6000\,\text{мкм}$.
    Определите, что наблюдается в этой точке.
    Длина волны равна $400\,\text{нм}$.
}
\answer{%
    $\text{точка максимума}$
}
\solutionspace{80pt}

\tasknumber{5}%
\task{%
    Разность фаз двух интерферирующих световых волн равна $8\pi$,
    а разность хода между ними равна $12{,}5 \cdot 10^{-7}\,\text{м}$.
    Определить длину и частоту волны.
}
\answer{%
    \begin{align*}
    \Delta \varphi &= k\Delta l = \frac{2 \pi}{\lambda} \Delta l = 8\pi \implies \lambda = \frac14\Delta l \approx 313\,\text{нм}, \\
    &\nu = \frac 1T = \frac c\lambda = 4 \frac c{\Delta l} = 4 \cdot \frac{3 \cdot 10^{8}\,\frac{\text{м}}{\text{с}}}{12{,}5 \cdot 10^{-7}\,\text{м}} \approx 960\,\text{ТГц}.
    \end{align*}
}
\solutionspace{100pt}

\tasknumber{6}%
\task{%
    Установка для наблюдения интерференции состоит
    из двух когерентных источников света и экрана.
    Расстояние между источниками $l = 1{,}5\,\text{мм}$,
    а от каждого источника до экрана — $L = 4\,\text{м}$.
    Сделайте рисунок и укажите положение нулевого максимума освещенности,
    а также определите расстояние между четвёртым максимумом и нулевым максимумом.
    Длина волны падающего света составляет $\lambda = 600\,\text{нм}$.
}
\answer{%
    \begin{align*}
    l_1^2 &= L^2 + \sqr{x - \frac \ell 2} \\
    l_2^2 &= L^2 + \sqr{x + \frac \ell 2} \\
    l_2^2 - l_1^2 &= 2x\ell \implies (l_2 - l_1)(l_2 + l_1) = 2x\ell \implies n\lambda \cdot 2L \approx 2x_n\ell \implies x_n = \frac{\lambda L}{\ell} n, n\in \mathbb{N} \\
    x &= \frac{\lambda L}{\ell} \cdot 4 = \frac{600\,\text{нм} \cdot 4\,\text{м}}{1{,}5\,\text{мм}} \cdot 4 \approx 6{,}4\,\text{мм}
    \end{align*}
}

\variantsplitter

\addpersonalvariant{Анастасия Князева}

\tasknumber{1}%
\task{%
    Напротив физических величин укажите их обозначения и единицы измерения в СИ, а в пункте «г)» запишите физический закон или формулу:
    \begin{enumerate}
        \item скорость света в вакууме,
        \item частоты волны,
        \item период колебаний индукции магнитного поля в электромагнитной волне,
        \item абсолютный показатель преломления среды.
    \end{enumerate}
}
\solutionspace{20pt}

\tasknumber{2}%
\task{%
    Получите из базовых физических законов:
    \begin{enumerate}
        \item частоту колебаний через длину волны и скорость её распространения,
        \item энергию фотона через длину электромагнитной волны,
        \item скорость света в вакууме через скорость света в среде и её абсолютный показатель преломления.
    \end{enumerate}
}
\solutionspace{60pt}

\tasknumber{3}%
\task{%
    Определить абсолютный показатель преломления прозрачной среды,
    в которой распространяется свет с длиной волны $0{,}550\,\text{мкм}$ и частотой $390\,\text{ТГц}$.
    Скорость света в вакууме $3 \cdot 10^{8}\,\frac{\text{м}}{\text{с}}$.
}
\answer{%
    $
        n = \frac{c}{v}
        = \frac{c}{\frac \lambda T}
        = \frac{c}{\lambda \nu}
        = \frac{3 \cdot 10^{8}\,\frac{\text{м}}{\text{с}}}{0{,}550\,\text{мкм} \cdot {390\,\text{ТГц}}}
        \approx1{,}40
    $
}
\solutionspace{80pt}

\tasknumber{4}%
\task{%
    В некоторую точку пространства приходят две когерентные световые волны
    с разностью хода $2{,}450\,\text{мкм}$.
    Определите, что наблюдается в этой точке.
    Длина волны равна $700\,\text{нм}$.
}
\answer{%
    $\text{точка минимума}$
}
\solutionspace{80pt}

\tasknumber{5}%
\task{%
    Разность фаз двух интерферирующих световых волн равна $5\pi$,
    а разность хода между ними равна $15{,}5 \cdot 10^{-7}\,\text{м}$.
    Определить длину и частоту волны.
}
\answer{%
    \begin{align*}
    \Delta \varphi &= k\Delta l = \frac{2 \pi}{\lambda} \Delta l = 5\pi \implies \lambda = \frac25\Delta l \approx 620\,\text{нм}, \\
    &\nu = \frac 1T = \frac c\lambda = \frac52 \frac c{\Delta l} = \frac52 \cdot \frac{3 \cdot 10^{8}\,\frac{\text{м}}{\text{с}}}{15{,}5 \cdot 10^{-7}\,\text{м}} \approx 484\,\text{ТГц}.
    \end{align*}
}
\solutionspace{100pt}

\tasknumber{6}%
\task{%
    Установка для наблюдения интерференции состоит
    из двух когерентных источников света и экрана.
    Расстояние между источниками $l = 1{,}2\,\text{мм}$,
    а от каждого источника до экрана — $L = 2\,\text{м}$.
    Сделайте рисунок и укажите положение нулевого максимума освещенности,
    а также определите расстояние между вторым максимумом и нулевым максимумом.
    Длина волны падающего света составляет $\lambda = 400\,\text{нм}$.
}
\answer{%
    \begin{align*}
    l_1^2 &= L^2 + \sqr{x - \frac \ell 2} \\
    l_2^2 &= L^2 + \sqr{x + \frac \ell 2} \\
    l_2^2 - l_1^2 &= 2x\ell \implies (l_2 - l_1)(l_2 + l_1) = 2x\ell \implies n\lambda \cdot 2L \approx 2x_n\ell \implies x_n = \frac{\lambda L}{\ell} n, n\in \mathbb{N} \\
    x &= \frac{\lambda L}{\ell} \cdot 2 = \frac{400\,\text{нм} \cdot 2\,\text{м}}{1{,}2\,\text{мм}} \cdot 2 \approx 1{,}33\,\text{мм}
    \end{align*}
}

\variantsplitter

\addpersonalvariant{Елизавета Кутумова}

\tasknumber{1}%
\task{%
    Напротив физических величин укажите их обозначения и единицы измерения в СИ, а в пункте «г)» запишите физический закон или формулу:
    \begin{enumerate}
        \item скорость света в среде,
        \item частоты волны,
        \item период колебаний индукции магнитного поля в электромагнитной волне,
        \item относительный показатель преломления среды.
    \end{enumerate}
}
\solutionspace{20pt}

\tasknumber{2}%
\task{%
    Получите из базовых физических законов:
    \begin{enumerate}
        \item период колебаний через длину волны и скорость её распространения,
        \item энергию фотона через период колебаний в электромагнитной волне,
        \item скорость света в вакууме через скорость света в среде и её абсолютный показатель преломления.
    \end{enumerate}
}
\solutionspace{60pt}

\tasknumber{3}%
\task{%
    Определить абсолютный показатель преломления прозрачной среды,
    в которой распространяется свет с длиной волны $0{,}450\,\text{мкм}$ и частотой $510\,\text{ТГц}$.
    Скорость света в вакууме $3 \cdot 10^{8}\,\frac{\text{м}}{\text{с}}$.
}
\answer{%
    $
        n = \frac{c}{v}
        = \frac{c}{\frac \lambda T}
        = \frac{c}{\lambda \nu}
        = \frac{3 \cdot 10^{8}\,\frac{\text{м}}{\text{с}}}{0{,}450\,\text{мкм} \cdot {510\,\text{ТГц}}}
        \approx1{,}30
    $
}
\solutionspace{80pt}

\tasknumber{4}%
\task{%
    В некоторую точку пространства приходят две когерентные световые волны
    с разностью хода $3{,}150\,\text{мкм}$.
    Определите, что наблюдается в этой точке.
    Длина волны равна $700\,\text{нм}$.
}
\answer{%
    $\text{точка минимума}$
}
\solutionspace{80pt}

\tasknumber{5}%
\task{%
    Разность фаз двух интерферирующих световых волн равна $4\pi$,
    а разность хода между ними равна $12{,}5 \cdot 10^{-7}\,\text{м}$.
    Определить длину и частоту волны.
}
\answer{%
    \begin{align*}
    \Delta \varphi &= k\Delta l = \frac{2 \pi}{\lambda} \Delta l = 4\pi \implies \lambda = \frac12\Delta l \approx 625\,\text{нм}, \\
    &\nu = \frac 1T = \frac c\lambda = 2 \frac c{\Delta l} = 2 \cdot \frac{3 \cdot 10^{8}\,\frac{\text{м}}{\text{с}}}{12{,}5 \cdot 10^{-7}\,\text{м}} \approx 480\,\text{ТГц}.
    \end{align*}
}
\solutionspace{100pt}

\tasknumber{6}%
\task{%
    Установка для наблюдения интерференции состоит
    из двух когерентных источников света и экрана.
    Расстояние между источниками $l = 2{,}4\,\text{мм}$,
    а от каждого источника до экрана — $L = 4\,\text{м}$.
    Сделайте рисунок и укажите положение нулевого максимума освещенности,
    а также определите расстояние между вторым максимумом и нулевым максимумом.
    Длина волны падающего света составляет $\lambda = 450\,\text{нм}$.
}
\answer{%
    \begin{align*}
    l_1^2 &= L^2 + \sqr{x - \frac \ell 2} \\
    l_2^2 &= L^2 + \sqr{x + \frac \ell 2} \\
    l_2^2 - l_1^2 &= 2x\ell \implies (l_2 - l_1)(l_2 + l_1) = 2x\ell \implies n\lambda \cdot 2L \approx 2x_n\ell \implies x_n = \frac{\lambda L}{\ell} n, n\in \mathbb{N} \\
    x &= \frac{\lambda L}{\ell} \cdot 2 = \frac{450\,\text{нм} \cdot 4\,\text{м}}{2{,}4\,\text{мм}} \cdot 2 \approx 1{,}50\,\text{мм}
    \end{align*}
}

\variantsplitter

\addpersonalvariant{Роксана Мехтиева}

\tasknumber{1}%
\task{%
    Напротив физических величин укажите их обозначения и единицы измерения в СИ, а в пункте «г)» запишите физический закон или формулу:
    \begin{enumerate}
        \item скорость света в вакууме,
        \item длина волны,
        \item период колебаний напряжённости электрического поля в электромагнитной волне,
        \item относительный показатель преломления среды.
    \end{enumerate}
}
\solutionspace{20pt}

\tasknumber{2}%
\task{%
    Получите из базовых физических законов:
    \begin{enumerate}
        \item частоту колебаний через длину волны и скорость её распространения,
        \item энергию фотона через длину электромагнитной волны,
        \item скорость света в вакууме через скорость света в среде и её абсолютный показатель преломления.
    \end{enumerate}
}
\solutionspace{60pt}

\tasknumber{3}%
\task{%
    Определить абсолютный показатель преломления прозрачной среды,
    в которой распространяется свет с длиной волны $0{,}550\,\text{мкм}$ и частотой $360\,\text{ТГц}$.
    Скорость света в вакууме $3 \cdot 10^{8}\,\frac{\text{м}}{\text{с}}$.
}
\answer{%
    $
        n = \frac{c}{v}
        = \frac{c}{\frac \lambda T}
        = \frac{c}{\lambda \nu}
        = \frac{3 \cdot 10^{8}\,\frac{\text{м}}{\text{с}}}{0{,}550\,\text{мкм} \cdot {360\,\text{ТГц}}}
        \approx1{,}50
    $
}
\solutionspace{80pt}

\tasknumber{4}%
\task{%
    В некоторую точку пространства приходят две когерентные световые волны
    с разностью хода $2{,}200\,\text{мкм}$.
    Определите, что наблюдается в этой точке.
    Длина волны равна $400\,\text{нм}$.
}
\answer{%
    $\text{точка минимума}$
}
\solutionspace{80pt}

\tasknumber{5}%
\task{%
    Разность фаз двух интерферирующих световых волн равна $5\pi$,
    а разность хода между ними равна $10{,}5 \cdot 10^{-7}\,\text{м}$.
    Определить длину и частоту волны.
}
\answer{%
    \begin{align*}
    \Delta \varphi &= k\Delta l = \frac{2 \pi}{\lambda} \Delta l = 5\pi \implies \lambda = \frac25\Delta l \approx 420\,\text{нм}, \\
    &\nu = \frac 1T = \frac c\lambda = \frac52 \frac c{\Delta l} = \frac52 \cdot \frac{3 \cdot 10^{8}\,\frac{\text{м}}{\text{с}}}{10{,}5 \cdot 10^{-7}\,\text{м}} \approx 714\,\text{ТГц}.
    \end{align*}
}
\solutionspace{100pt}

\tasknumber{6}%
\task{%
    Установка для наблюдения интерференции состоит
    из двух когерентных источников света и экрана.
    Расстояние между источниками $l = 1{,}5\,\text{мм}$,
    а от каждого источника до экрана — $L = 4\,\text{м}$.
    Сделайте рисунок и укажите положение нулевого максимума освещенности,
    а также определите расстояние между четвёртым максимумом и нулевым максимумом.
    Длина волны падающего света составляет $\lambda = 550\,\text{нм}$.
}
\answer{%
    \begin{align*}
    l_1^2 &= L^2 + \sqr{x - \frac \ell 2} \\
    l_2^2 &= L^2 + \sqr{x + \frac \ell 2} \\
    l_2^2 - l_1^2 &= 2x\ell \implies (l_2 - l_1)(l_2 + l_1) = 2x\ell \implies n\lambda \cdot 2L \approx 2x_n\ell \implies x_n = \frac{\lambda L}{\ell} n, n\in \mathbb{N} \\
    x &= \frac{\lambda L}{\ell} \cdot 4 = \frac{550\,\text{нм} \cdot 4\,\text{м}}{1{,}5\,\text{мм}} \cdot 4 \approx 5{,}9\,\text{мм}
    \end{align*}
}

\variantsplitter

\addpersonalvariant{Дилноза Нодиршоева}

\tasknumber{1}%
\task{%
    Напротив физических величин укажите их обозначения и единицы измерения в СИ, а в пункте «г)» запишите физический закон или формулу:
    \begin{enumerate}
        \item скорость света в среде,
        \item длина волны,
        \item период колебаний индукции магнитного поля в электромагнитной волне,
        \item относительный показатель преломления среды.
    \end{enumerate}
}
\solutionspace{20pt}

\tasknumber{2}%
\task{%
    Получите из базовых физических законов:
    \begin{enumerate}
        \item частоту колебаний через длину волны и скорость её распространения,
        \item энергию фотона через период колебаний в электромагнитной волне,
        \item скорость света в вакууме через скорость света в среде и её абсолютный показатель преломления.
    \end{enumerate}
}
\solutionspace{60pt}

\tasknumber{3}%
\task{%
    Определить абсолютный показатель преломления прозрачной среды,
    в которой распространяется свет с длиной волны $0{,}650\,\text{мкм}$ и частотой $310\,\text{ТГц}$.
    Скорость света в вакууме $3 \cdot 10^{8}\,\frac{\text{м}}{\text{с}}$.
}
\answer{%
    $
        n = \frac{c}{v}
        = \frac{c}{\frac \lambda T}
        = \frac{c}{\lambda \nu}
        = \frac{3 \cdot 10^{8}\,\frac{\text{м}}{\text{с}}}{0{,}650\,\text{мкм} \cdot {310\,\text{ТГц}}}
        \approx1{,}50
    $
}
\solutionspace{80pt}

\tasknumber{4}%
\task{%
    В некоторую точку пространства приходят две когерентные световые волны
    с разностью хода $3\,\text{мкм}$.
    Определите, что наблюдается в этой точке.
    Длина волны равна $500\,\text{нм}$.
}
\answer{%
    $\text{точка максимума}$
}
\solutionspace{80pt}

\tasknumber{5}%
\task{%
    Разность фаз двух интерферирующих световых волн равна $3\pi$,
    а разность хода между ними равна $12{,}5 \cdot 10^{-7}\,\text{м}$.
    Определить длину и частоту волны.
}
\answer{%
    \begin{align*}
    \Delta \varphi &= k\Delta l = \frac{2 \pi}{\lambda} \Delta l = 3\pi \implies \lambda = \frac23\Delta l \approx 833\,\text{нм}, \\
    &\nu = \frac 1T = \frac c\lambda = \frac32 \frac c{\Delta l} = \frac32 \cdot \frac{3 \cdot 10^{8}\,\frac{\text{м}}{\text{с}}}{12{,}5 \cdot 10^{-7}\,\text{м}} \approx 360\,\text{ТГц}.
    \end{align*}
}
\solutionspace{100pt}

\tasknumber{6}%
\task{%
    Установка для наблюдения интерференции состоит
    из двух когерентных источников света и экрана.
    Расстояние между источниками $l = 0{,}8\,\text{мм}$,
    а от каждого источника до экрана — $L = 2\,\text{м}$.
    Сделайте рисунок и укажите положение нулевого максимума освещенности,
    а также определите расстояние между четвёртым максимумом и нулевым максимумом.
    Длина волны падающего света составляет $\lambda = 500\,\text{нм}$.
}
\answer{%
    \begin{align*}
    l_1^2 &= L^2 + \sqr{x - \frac \ell 2} \\
    l_2^2 &= L^2 + \sqr{x + \frac \ell 2} \\
    l_2^2 - l_1^2 &= 2x\ell \implies (l_2 - l_1)(l_2 + l_1) = 2x\ell \implies n\lambda \cdot 2L \approx 2x_n\ell \implies x_n = \frac{\lambda L}{\ell} n, n\in \mathbb{N} \\
    x &= \frac{\lambda L}{\ell} \cdot 4 = \frac{500\,\text{нм} \cdot 2\,\text{м}}{0{,}8\,\text{мм}} \cdot 4 \approx 5\,\text{мм}
    \end{align*}
}

\variantsplitter

\addpersonalvariant{Жаклин Пантелеева}

\tasknumber{1}%
\task{%
    Напротив физических величин укажите их обозначения и единицы измерения в СИ, а в пункте «г)» запишите физический закон или формулу:
    \begin{enumerate}
        \item скорость света в вакууме,
        \item частоты волны,
        \item период колебаний индукции магнитного поля в электромагнитной волне,
        \item абсолютный показатель преломления среды.
    \end{enumerate}
}
\solutionspace{20pt}

\tasknumber{2}%
\task{%
    Получите из базовых физических законов:
    \begin{enumerate}
        \item частоту колебаний через длину волны и скорость её распространения,
        \item энергию фотона через длину электромагнитной волны,
        \item скорость света в вакууме через скорость света в среде и её абсолютный показатель преломления.
    \end{enumerate}
}
\solutionspace{60pt}

\tasknumber{3}%
\task{%
    Определить абсолютный показатель преломления прозрачной среды,
    в которой распространяется свет с длиной волны $0{,}600\,\text{мкм}$ и частотой $290\,\text{ТГц}$.
    Скорость света в вакууме $3 \cdot 10^{8}\,\frac{\text{м}}{\text{с}}$.
}
\answer{%
    $
        n = \frac{c}{v}
        = \frac{c}{\frac \lambda T}
        = \frac{c}{\lambda \nu}
        = \frac{3 \cdot 10^{8}\,\frac{\text{м}}{\text{с}}}{0{,}600\,\text{мкм} \cdot {290\,\text{ТГц}}}
        \approx1{,}70
    $
}
\solutionspace{80pt}

\tasknumber{4}%
\task{%
    В некоторую точку пространства приходят две когерентные световые волны
    с разностью хода $1{,}4000\,\text{мкм}$.
    Определите, что наблюдается в этой точке.
    Длина волны равна $400\,\text{нм}$.
}
\answer{%
    $\text{точка минимума}$
}
\solutionspace{80pt}

\tasknumber{5}%
\task{%
    Разность фаз двух интерферирующих световых волн равна $7\pi$,
    а разность хода между ними равна $10{,}5 \cdot 10^{-7}\,\text{м}$.
    Определить длину и частоту волны.
}
\answer{%
    \begin{align*}
    \Delta \varphi &= k\Delta l = \frac{2 \pi}{\lambda} \Delta l = 7\pi \implies \lambda = \frac27\Delta l \approx 300\,\text{нм}, \\
    &\nu = \frac 1T = \frac c\lambda = \frac72 \frac c{\Delta l} = \frac72 \cdot \frac{3 \cdot 10^{8}\,\frac{\text{м}}{\text{с}}}{10{,}5 \cdot 10^{-7}\,\text{м}} \approx 1000\,\text{ТГц}.
    \end{align*}
}
\solutionspace{100pt}

\tasknumber{6}%
\task{%
    Установка для наблюдения интерференции состоит
    из двух когерентных источников света и экрана.
    Расстояние между источниками $l = 2{,}4\,\text{мм}$,
    а от каждого источника до экрана — $L = 4\,\text{м}$.
    Сделайте рисунок и укажите положение нулевого максимума освещенности,
    а также определите расстояние между третьим минимумом и нулевым максимумом.
    Длина волны падающего света составляет $\lambda = 450\,\text{нм}$.
}
\answer{%
    \begin{align*}
    l_1^2 &= L^2 + \sqr{x - \frac \ell 2} \\
    l_2^2 &= L^2 + \sqr{x + \frac \ell 2} \\
    l_2^2 - l_1^2 &= 2x\ell \implies (l_2 - l_1)(l_2 + l_1) = 2x\ell \implies n\lambda \cdot 2L \approx 2x_n\ell \implies x_n = \frac{\lambda L}{\ell} n, n\in \mathbb{N} \\
    x &= \frac{\lambda L}{\ell} \cdot \frac52 = \frac{450\,\text{нм} \cdot 4\,\text{м}}{2{,}4\,\text{мм}} \cdot \frac52 \approx 1{,}88\,\text{мм}
    \end{align*}
}

\variantsplitter

\addpersonalvariant{Артём Переверзев}

\tasknumber{1}%
\task{%
    Напротив физических величин укажите их обозначения и единицы измерения в СИ, а в пункте «г)» запишите физический закон или формулу:
    \begin{enumerate}
        \item скорость света в вакууме,
        \item длина волны,
        \item период колебаний напряжённости электрического поля в электромагнитной волне,
        \item абсолютный показатель преломления среды.
    \end{enumerate}
}
\solutionspace{20pt}

\tasknumber{2}%
\task{%
    Получите из базовых физических законов:
    \begin{enumerate}
        \item частоту колебаний через длину волны и скорость её распространения,
        \item энергию фотона через период колебаний в электромагнитной волне,
        \item скорость света в вакууме через скорость света в среде и её абсолютный показатель преломления.
    \end{enumerate}
}
\solutionspace{60pt}

\tasknumber{3}%
\task{%
    Определить абсолютный показатель преломления прозрачной среды,
    в которой распространяется свет с длиной волны $0{,}600\,\text{мкм}$ и частотой $290\,\text{ТГц}$.
    Скорость света в вакууме $3 \cdot 10^{8}\,\frac{\text{м}}{\text{с}}$.
}
\answer{%
    $
        n = \frac{c}{v}
        = \frac{c}{\frac \lambda T}
        = \frac{c}{\lambda \nu}
        = \frac{3 \cdot 10^{8}\,\frac{\text{м}}{\text{с}}}{0{,}600\,\text{мкм} \cdot {290\,\text{ТГц}}}
        \approx1{,}70
    $
}
\solutionspace{80pt}

\tasknumber{4}%
\task{%
    В некоторую точку пространства приходят две когерентные световые волны
    с разностью хода $1{,}4000\,\text{мкм}$.
    Определите, что наблюдается в этой точке.
    Длина волны равна $400\,\text{нм}$.
}
\answer{%
    $\text{точка минимума}$
}
\solutionspace{80pt}

\tasknumber{5}%
\task{%
    Разность фаз двух интерферирующих световых волн равна $6\pi$,
    а разность хода между ними равна $7{,}5 \cdot 10^{-7}\,\text{м}$.
    Определить длину и частоту волны.
}
\answer{%
    \begin{align*}
    \Delta \varphi &= k\Delta l = \frac{2 \pi}{\lambda} \Delta l = 6\pi \implies \lambda = \frac13\Delta l \approx 250\,\text{нм}, \\
    &\nu = \frac 1T = \frac c\lambda = 3 \frac c{\Delta l} = 3 \cdot \frac{3 \cdot 10^{8}\,\frac{\text{м}}{\text{с}}}{7{,}5 \cdot 10^{-7}\,\text{м}} \approx 1200\,\text{ТГц}.
    \end{align*}
}
\solutionspace{100pt}

\tasknumber{6}%
\task{%
    Установка для наблюдения интерференции состоит
    из двух когерентных источников света и экрана.
    Расстояние между источниками $l = 1{,}5\,\text{мм}$,
    а от каждого источника до экрана — $L = 4\,\text{м}$.
    Сделайте рисунок и укажите положение нулевого максимума освещенности,
    а также определите расстояние между вторым максимумом и нулевым максимумом.
    Длина волны падающего света составляет $\lambda = 500\,\text{нм}$.
}
\answer{%
    \begin{align*}
    l_1^2 &= L^2 + \sqr{x - \frac \ell 2} \\
    l_2^2 &= L^2 + \sqr{x + \frac \ell 2} \\
    l_2^2 - l_1^2 &= 2x\ell \implies (l_2 - l_1)(l_2 + l_1) = 2x\ell \implies n\lambda \cdot 2L \approx 2x_n\ell \implies x_n = \frac{\lambda L}{\ell} n, n\in \mathbb{N} \\
    x &= \frac{\lambda L}{\ell} \cdot 2 = \frac{500\,\text{нм} \cdot 4\,\text{м}}{1{,}5\,\text{мм}} \cdot 2 \approx 2{,}7\,\text{мм}
    \end{align*}
}

\variantsplitter

\addpersonalvariant{Варвара Пранова}

\tasknumber{1}%
\task{%
    Напротив физических величин укажите их обозначения и единицы измерения в СИ, а в пункте «г)» запишите физический закон или формулу:
    \begin{enumerate}
        \item скорость света в вакууме,
        \item длина волны,
        \item период колебаний индукции магнитного поля в электромагнитной волне,
        \item относительный показатель преломления среды.
    \end{enumerate}
}
\solutionspace{20pt}

\tasknumber{2}%
\task{%
    Получите из базовых физических законов:
    \begin{enumerate}
        \item частоту колебаний через длину волны и скорость её распространения,
        \item энергию фотона через период колебаний в электромагнитной волне,
        \item скорость света в среде через её абсолютный показатель преломления и скорость света в вакууме.
    \end{enumerate}
}
\solutionspace{60pt}

\tasknumber{3}%
\task{%
    Определить абсолютный показатель преломления прозрачной среды,
    в которой распространяется свет с длиной волны $0{,}650\,\text{мкм}$ и частотой $360\,\text{ТГц}$.
    Скорость света в вакууме $3 \cdot 10^{8}\,\frac{\text{м}}{\text{с}}$.
}
\answer{%
    $
        n = \frac{c}{v}
        = \frac{c}{\frac \lambda T}
        = \frac{c}{\lambda \nu}
        = \frac{3 \cdot 10^{8}\,\frac{\text{м}}{\text{с}}}{0{,}650\,\text{мкм} \cdot {360\,\text{ТГц}}}
        \approx1{,}30
    $
}
\solutionspace{80pt}

\tasknumber{4}%
\task{%
    В некоторую точку пространства приходят две когерентные световые волны
    с разностью хода $1{,}8000\,\text{мкм}$.
    Определите, что наблюдается в этой точке.
    Длина волны равна $600\,\text{нм}$.
}
\answer{%
    $\text{точка максимума}$
}
\solutionspace{80pt}

\tasknumber{5}%
\task{%
    Разность фаз двух интерферирующих световых волн равна $4\pi$,
    а разность хода между ними равна $12{,}5 \cdot 10^{-7}\,\text{м}$.
    Определить длину и частоту волны.
}
\answer{%
    \begin{align*}
    \Delta \varphi &= k\Delta l = \frac{2 \pi}{\lambda} \Delta l = 4\pi \implies \lambda = \frac12\Delta l \approx 625\,\text{нм}, \\
    &\nu = \frac 1T = \frac c\lambda = 2 \frac c{\Delta l} = 2 \cdot \frac{3 \cdot 10^{8}\,\frac{\text{м}}{\text{с}}}{12{,}5 \cdot 10^{-7}\,\text{м}} \approx 480\,\text{ТГц}.
    \end{align*}
}
\solutionspace{100pt}

\tasknumber{6}%
\task{%
    Установка для наблюдения интерференции состоит
    из двух когерентных источников света и экрана.
    Расстояние между источниками $l = 1{,}2\,\text{мм}$,
    а от каждого источника до экрана — $L = 3\,\text{м}$.
    Сделайте рисунок и укажите положение нулевого максимума освещенности,
    а также определите расстояние между четвёртым минимумом и нулевым максимумом.
    Длина волны падающего света составляет $\lambda = 400\,\text{нм}$.
}
\answer{%
    \begin{align*}
    l_1^2 &= L^2 + \sqr{x - \frac \ell 2} \\
    l_2^2 &= L^2 + \sqr{x + \frac \ell 2} \\
    l_2^2 - l_1^2 &= 2x\ell \implies (l_2 - l_1)(l_2 + l_1) = 2x\ell \implies n\lambda \cdot 2L \approx 2x_n\ell \implies x_n = \frac{\lambda L}{\ell} n, n\in \mathbb{N} \\
    x &= \frac{\lambda L}{\ell} \cdot \frac72 = \frac{400\,\text{нм} \cdot 3\,\text{м}}{1{,}2\,\text{мм}} \cdot \frac72 \approx 3{,}5\,\text{мм}
    \end{align*}
}

\variantsplitter

\addpersonalvariant{Марьям Салимова}

\tasknumber{1}%
\task{%
    Напротив физических величин укажите их обозначения и единицы измерения в СИ, а в пункте «г)» запишите физический закон или формулу:
    \begin{enumerate}
        \item скорость света в среде,
        \item длина волны,
        \item период колебаний индукции магнитного поля в электромагнитной волне,
        \item абсолютный показатель преломления среды.
    \end{enumerate}
}
\solutionspace{20pt}

\tasknumber{2}%
\task{%
    Получите из базовых физических законов:
    \begin{enumerate}
        \item частоту колебаний через длину волны и скорость её распространения,
        \item энергию фотона через длину электромагнитной волны,
        \item скорость света в вакууме через скорость света в среде и её абсолютный показатель преломления.
    \end{enumerate}
}
\solutionspace{60pt}

\tasknumber{3}%
\task{%
    Определить абсолютный показатель преломления прозрачной среды,
    в которой распространяется свет с длиной волны $0{,}600\,\text{мкм}$ и частотой $330\,\text{ТГц}$.
    Скорость света в вакууме $3 \cdot 10^{8}\,\frac{\text{м}}{\text{с}}$.
}
\answer{%
    $
        n = \frac{c}{v}
        = \frac{c}{\frac \lambda T}
        = \frac{c}{\lambda \nu}
        = \frac{3 \cdot 10^{8}\,\frac{\text{м}}{\text{с}}}{0{,}600\,\text{мкм} \cdot {330\,\text{ТГц}}}
        \approx1{,}50
    $
}
\solutionspace{80pt}

\tasknumber{4}%
\task{%
    В некоторую точку пространства приходят две когерентные световые волны
    с разностью хода $3\,\text{мкм}$.
    Определите, что наблюдается в этой точке.
    Длина волны равна $600\,\text{нм}$.
}
\answer{%
    $\text{точка максимума}$
}
\solutionspace{80pt}

\tasknumber{5}%
\task{%
    Разность фаз двух интерферирующих световых волн равна $4\pi$,
    а разность хода между ними равна $12{,}5 \cdot 10^{-7}\,\text{м}$.
    Определить длину и частоту волны.
}
\answer{%
    \begin{align*}
    \Delta \varphi &= k\Delta l = \frac{2 \pi}{\lambda} \Delta l = 4\pi \implies \lambda = \frac12\Delta l \approx 625\,\text{нм}, \\
    &\nu = \frac 1T = \frac c\lambda = 2 \frac c{\Delta l} = 2 \cdot \frac{3 \cdot 10^{8}\,\frac{\text{м}}{\text{с}}}{12{,}5 \cdot 10^{-7}\,\text{м}} \approx 480\,\text{ТГц}.
    \end{align*}
}
\solutionspace{100pt}

\tasknumber{6}%
\task{%
    Установка для наблюдения интерференции состоит
    из двух когерентных источников света и экрана.
    Расстояние между источниками $l = 0{,}8\,\text{мм}$,
    а от каждого источника до экрана — $L = 2\,\text{м}$.
    Сделайте рисунок и укажите положение нулевого максимума освещенности,
    а также определите расстояние между четвёртым минимумом и нулевым максимумом.
    Длина волны падающего света составляет $\lambda = 450\,\text{нм}$.
}
\answer{%
    \begin{align*}
    l_1^2 &= L^2 + \sqr{x - \frac \ell 2} \\
    l_2^2 &= L^2 + \sqr{x + \frac \ell 2} \\
    l_2^2 - l_1^2 &= 2x\ell \implies (l_2 - l_1)(l_2 + l_1) = 2x\ell \implies n\lambda \cdot 2L \approx 2x_n\ell \implies x_n = \frac{\lambda L}{\ell} n, n\in \mathbb{N} \\
    x &= \frac{\lambda L}{\ell} \cdot \frac72 = \frac{450\,\text{нм} \cdot 2\,\text{м}}{0{,}8\,\text{мм}} \cdot \frac72 \approx 3{,}9\,\text{мм}
    \end{align*}
}

\variantsplitter

\addpersonalvariant{Юлия Шевченко}

\tasknumber{1}%
\task{%
    Напротив физических величин укажите их обозначения и единицы измерения в СИ, а в пункте «г)» запишите физический закон или формулу:
    \begin{enumerate}
        \item скорость света в среде,
        \item частоты волны,
        \item период колебаний индукции магнитного поля в электромагнитной волне,
        \item абсолютный показатель преломления среды.
    \end{enumerate}
}
\solutionspace{20pt}

\tasknumber{2}%
\task{%
    Получите из базовых физических законов:
    \begin{enumerate}
        \item период колебаний через длину волны и скорость её распространения,
        \item энергию фотона через длину электромагнитной волны,
        \item скорость света в вакууме через скорость света в среде и её абсолютный показатель преломления.
    \end{enumerate}
}
\solutionspace{60pt}

\tasknumber{3}%
\task{%
    Определить абсолютный показатель преломления прозрачной среды,
    в которой распространяется свет с длиной волны $0{,}500\,\text{мкм}$ и частотой $375\,\text{ТГц}$.
    Скорость света в вакууме $3 \cdot 10^{8}\,\frac{\text{м}}{\text{с}}$.
}
\answer{%
    $
        n = \frac{c}{v}
        = \frac{c}{\frac \lambda T}
        = \frac{c}{\lambda \nu}
        = \frac{3 \cdot 10^{8}\,\frac{\text{м}}{\text{с}}}{0{,}500\,\text{мкм} \cdot {375\,\text{ТГц}}}
        \approx1{,}60
    $
}
\solutionspace{80pt}

\tasknumber{4}%
\task{%
    В некоторую точку пространства приходят две когерентные световые волны
    с разностью хода $3\,\text{мкм}$.
    Определите, что наблюдается в этой точке.
    Длина волны равна $600\,\text{нм}$.
}
\answer{%
    $\text{точка максимума}$
}
\solutionspace{80pt}

\tasknumber{5}%
\task{%
    Разность фаз двух интерферирующих световых волн равна $7\pi$,
    а разность хода между ними равна $15{,}5 \cdot 10^{-7}\,\text{м}$.
    Определить длину и частоту волны.
}
\answer{%
    \begin{align*}
    \Delta \varphi &= k\Delta l = \frac{2 \pi}{\lambda} \Delta l = 7\pi \implies \lambda = \frac27\Delta l \approx 443\,\text{нм}, \\
    &\nu = \frac 1T = \frac c\lambda = \frac72 \frac c{\Delta l} = \frac72 \cdot \frac{3 \cdot 10^{8}\,\frac{\text{м}}{\text{с}}}{15{,}5 \cdot 10^{-7}\,\text{м}} \approx 677\,\text{ТГц}.
    \end{align*}
}
\solutionspace{100pt}

\tasknumber{6}%
\task{%
    Установка для наблюдения интерференции состоит
    из двух когерентных источников света и экрана.
    Расстояние между источниками $l = 1{,}5\,\text{мм}$,
    а от каждого источника до экрана — $L = 2\,\text{м}$.
    Сделайте рисунок и укажите положение нулевого максимума освещенности,
    а также определите расстояние между третьим минимумом и нулевым максимумом.
    Длина волны падающего света составляет $\lambda = 400\,\text{нм}$.
}
\answer{%
    \begin{align*}
    l_1^2 &= L^2 + \sqr{x - \frac \ell 2} \\
    l_2^2 &= L^2 + \sqr{x + \frac \ell 2} \\
    l_2^2 - l_1^2 &= 2x\ell \implies (l_2 - l_1)(l_2 + l_1) = 2x\ell \implies n\lambda \cdot 2L \approx 2x_n\ell \implies x_n = \frac{\lambda L}{\ell} n, n\in \mathbb{N} \\
    x &= \frac{\lambda L}{\ell} \cdot \frac52 = \frac{400\,\text{нм} \cdot 2\,\text{м}}{1{,}5\,\text{мм}} \cdot \frac52 \approx 1{,}33\,\text{мм}
    \end{align*}
}
% autogenerated
