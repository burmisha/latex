\setdate{15~декабря~2021}
\setclass{11«БА»}

\addpersonalvariant{Михаил Бурмистров}

\tasknumber{1}%
\task{%
    Параллельный пучок света распространяется горизонтально.
    Под каким углом (в градусах) к горизонту следует расположить плоское зеркало,
    чтобы отраженный пучок распространялся вертикально?
}
\answer{%
    $2\,\frac{\text{см}}{\text{с}}$
}
\solutionspace{80pt}

\tasknumber{2}%
\task{%
    Под каким углом (в градусах) к горизонту следует расположить плоское зеркало,
    чтобы осветить дно вертикального колодца отраженными от зеркала солнечными лучами,
    падающими под углом $64\degrees$ к горизонту?
}
\answer{%
    $\alpha = 64\degrees, (90\degrees - \beta + \alpha) + (90\degrees - \beta) = 90\degrees \implies \beta = 45\degrees + \frac \alpha 2 = 77\degrees$
}
\solutionspace{80pt}

\tasknumber{3}%
\task{%
    Тимур стоит перед плоским зеркалом, укрепленным на вертикальной стене.
    Какова должна быть минимальная высота зеркала, чтобы Тимур мог видеть себя в полный рост?
    Рост Тимур $190\,\text{см}$.
    Определите также расстояние от пола до нижнего края зеркала,
    приняв высоту головы равной $34\,\text{см}$, и считая, что глаза находятся посередине (по высоте) головы.
}
\answer{%
    $95\,\text{см}, 86{,}500\,\text{см}$
}
\solutionspace{80pt}

\tasknumber{4}%
\task{%
    Во сколько раз увеличится расстояние между предметом и его изображением
    в плоском зеркале, если зеркало переместить в то место, где было изображение? Предмет остаётся неподвижным.
}
\answer{%
    $2$
}
\solutionspace{80pt}

\tasknumber{5}%
\task{%
    Плоское зеркало движется по направлению к точечному источнику света со скоростью $15\,\frac{\text{см}}{\text{с}}$.
    Определите скорость движения изображения относительно источника света.
    Направление скорости зеркала перпендикулярно плоскости зеркала.
}
\answer{%
    $30\,\frac{\text{см}}{\text{с}}$
}
\solutionspace{80pt}

\tasknumber{6}%
\task{%
    Сколько изображений получится от предмета в двух плоских зеркалах,
    поставленных под углом $90\degrees$ друг к другу?
}
\answer{%
    $3$
}
\solutionspace{80pt}

\tasknumber{7}%
\task{%
    Два плоских зеркала располагаются под углом друг к другу
    и между ними помещается точечный источник света.
    Расстояние от этого источника до одного зеркала $4\,\text{см}$, до другого $6\,\text{см}$.
    Расстояние между первыми изображениями в зеркалах $18{,}54\,\text{см}$.
    Найдите угол (в градусах) между зеркалами.
}
\answer{%
    $\cos \alpha = \frac{c^2 - \sqr{2a} - \sqr{2b}}{2 \cdot 2a \cdot 2b} \approx 0{,}707 \implies \alpha = 45{,}0\degrees$
}

\variantsplitter

\addpersonalvariant{Ирина Ан}

\tasknumber{1}%
\task{%
    Параллельный пучок света распространяется горизонтально.
    Под каким углом (в градусах) к горизонту следует расположить плоское зеркало,
    чтобы отраженный пучок распространялся вертикально?
}
\answer{%
    $4\,\frac{\text{см}}{\text{с}}$
}
\solutionspace{80pt}

\tasknumber{2}%
\task{%
    Под каким углом (в градусах) к горизонту следует расположить плоское зеркало,
    чтобы осветить дно вертикального колодца отраженными от зеркала солнечными лучами,
    падающими под углом $38\degrees$ к горизонту?
}
\answer{%
    $\alpha = 38\degrees, (90\degrees - \beta + \alpha) + (90\degrees - \beta) = 90\degrees \implies \beta = 45\degrees + \frac \alpha 2 = 64\degrees$
}
\solutionspace{80pt}

\tasknumber{3}%
\task{%
    Аня стоит перед плоским зеркалом, укрепленным на вертикальной стене.
    Какова должна быть минимальная высота зеркала, чтобы Аня мог видеть себя в полный рост?
    Рост Ани $156\,\text{см}$.
    Определите также расстояние от пола до нижнего края зеркала,
    приняв высоту головы равной $30\,\text{см}$, и считая, что глаза находятся посередине (по высоте) головы.
}
\answer{%
    $78\,\text{см}, 70{,}500\,\text{см}$
}
\solutionspace{80pt}

\tasknumber{4}%
\task{%
    Во сколько раз увеличится расстояние между предметом и его изображением
    в плоском зеркале, если зеркало переместить в то место, где было изображение? Предмет остаётся неподвижным.
}
\answer{%
    $2$
}
\solutionspace{80pt}

\tasknumber{5}%
\task{%
    Плоское зеркало движется по направлению к точечному источнику света со скоростью $10\,\frac{\text{см}}{\text{с}}$.
    Определите скорость движения изображения относительно зеркала.
    Направление скорости зеркала перпендикулярно плоскости зеркала.
}
\answer{%
    $10\,\frac{\text{см}}{\text{с}}$
}
\solutionspace{80pt}

\tasknumber{6}%
\task{%
    Сколько изображений получится от предмета в двух плоских зеркалах,
    поставленных под углом $90\degrees$ друг к другу?
}
\answer{%
    $3$
}
\solutionspace{80pt}

\tasknumber{7}%
\task{%
    Два плоских зеркала располагаются под углом друг к другу
    и между ними помещается точечный источник света.
    Расстояние от этого источника до одного зеркала $3\,\text{см}$, до другого $9\,\text{см}$.
    Расстояние между первыми изображениями в зеркалах $21{,}63\,\text{см}$.
    Найдите угол (в градусах) между зеркалами.
}
\answer{%
    $\cos \alpha = \frac{c^2 - \sqr{2a} - \sqr{2b}}{2 \cdot 2a \cdot 2b} \approx 0{,}499 \implies \alpha = 60{,}0\degrees$
}

\variantsplitter

\addpersonalvariant{Софья Андрианова}

\tasknumber{1}%
\task{%
    Параллельный пучок света распространяется горизонтально.
    Под каким углом (в градусах) к горизонту следует расположить плоское зеркало,
    чтобы отраженный пучок распространялся вертикально?
}
\answer{%
    $3\,\frac{\text{см}}{\text{с}}$
}
\solutionspace{80pt}

\tasknumber{2}%
\task{%
    Под каким углом (в градусах) к горизонту следует расположить плоское зеркало,
    чтобы осветить дно вертикального колодца отраженными от зеркала солнечными лучами,
    падающими под углом $52\degrees$ к горизонту?
}
\answer{%
    $\alpha = 52\degrees, (90\degrees - \beta + \alpha) + (90\degrees - \beta) = 90\degrees \implies \beta = 45\degrees + \frac \alpha 2 = 71\degrees$
}
\solutionspace{80pt}

\tasknumber{3}%
\task{%
    Лиана стоит перед плоским зеркалом, укрепленным на вертикальной стене.
    Какова должна быть минимальная высота зеркала, чтобы Лиана мог видеть себя в полный рост?
    Рост Лианы $188\,\text{см}$.
    Определите также расстояние от пола до нижнего края зеркала,
    приняв высоту головы равной $30\,\text{см}$, и считая, что глаза находятся посередине (по высоте) головы.
}
\answer{%
    $94\,\text{см}, 86{,}500\,\text{см}$
}
\solutionspace{80pt}

\tasknumber{4}%
\task{%
    Во сколько раз увеличится расстояние между предметом и его изображением
    в плоском зеркале, если зеркало переместить в то место, где было изображение? Предмет остаётся неподвижным.
}
\answer{%
    $2$
}
\solutionspace{80pt}

\tasknumber{5}%
\task{%
    Плоское зеркало движется по направлению к точечному источнику света со скоростью $12\,\frac{\text{см}}{\text{с}}$.
    Определите скорость движения изображения относительно источника света.
    Направление скорости зеркала перпендикулярно плоскости зеркала.
}
\answer{%
    $24\,\frac{\text{см}}{\text{с}}$
}
\solutionspace{80pt}

\tasknumber{6}%
\task{%
    Сколько изображений получится от предмета в двух плоских зеркалах,
    поставленных под углом $45\degrees$ друг к другу?
}
\answer{%
    $7$
}
\solutionspace{80pt}

\tasknumber{7}%
\task{%
    Два плоских зеркала располагаются под углом друг к другу
    и между ними помещается точечный источник света.
    Расстояние от этого источника до одного зеркала $5\,\text{см}$, до другого $7\,\text{см}$.
    Расстояние между первыми изображениями в зеркалах $23{,}21\,\text{см}$.
    Найдите угол (в градусах) между зеркалами.
}
\answer{%
    $\cos \alpha = \frac{c^2 - \sqr{2a} - \sqr{2b}}{2 \cdot 2a \cdot 2b} \approx 0{,}867 \implies \alpha = 29{,}9\degrees$
}

\variantsplitter

\addpersonalvariant{Владимир Артемчук}

\tasknumber{1}%
\task{%
    Параллельный пучок света распространяется горизонтально.
    Под каким углом (в градусах) к горизонту следует расположить плоское зеркало,
    чтобы отраженный пучок распространялся вертикально?
}
\answer{%
    $2\,\frac{\text{см}}{\text{с}}$
}
\solutionspace{80pt}

\tasknumber{2}%
\task{%
    Под каким углом (в градусах) к горизонту следует расположить плоское зеркало,
    чтобы осветить дно вертикального колодца отраженными от зеркала солнечными лучами,
    падающими под углом $32\degrees$ к горизонту?
}
\answer{%
    $\alpha = 32\degrees, (90\degrees - \beta + \alpha) + (90\degrees - \beta) = 90\degrees \implies \beta = 45\degrees + \frac \alpha 2 = 61\degrees$
}
\solutionspace{80pt}

\tasknumber{3}%
\task{%
    Рипсиме стоит перед плоским зеркалом, укрепленным на вертикальной стене.
    Какова должна быть минимальная высота зеркала, чтобы Рипсиме мог видеть себя в полный рост?
    Рост Рипсиме $176\,\text{см}$.
    Определите также расстояние от пола до нижнего края зеркала,
    приняв высоту головы равной $32\,\text{см}$, и считая, что глаза находятся посередине (по высоте) головы.
}
\answer{%
    $88\,\text{см}, 80\,\text{см}$
}
\solutionspace{80pt}

\tasknumber{4}%
\task{%
    Во сколько раз увеличится расстояние между предметом и его изображением
    в плоском зеркале, если зеркало переместить в то место, где было изображение? Предмет остаётся неподвижным.
}
\answer{%
    $2$
}
\solutionspace{80pt}

\tasknumber{5}%
\task{%
    Плоское зеркало движется по направлению к точечному источнику света со скоростью $18\,\frac{\text{см}}{\text{с}}$.
    Определите скорость движения изображения относительно источника света.
    Направление скорости зеркала перпендикулярно плоскости зеркала.
}
\answer{%
    $36\,\frac{\text{см}}{\text{с}}$
}
\solutionspace{80pt}

\tasknumber{6}%
\task{%
    Сколько изображений получится от предмета в двух плоских зеркалах,
    поставленных под углом $90\degrees$ друг к другу?
}
\answer{%
    $3$
}
\solutionspace{80pt}

\tasknumber{7}%
\task{%
    Два плоских зеркала располагаются под углом друг к другу
    и между ними помещается точечный источник света.
    Расстояние от этого источника до одного зеркала $4\,\text{см}$, до другого $6\,\text{см}$.
    Расстояние между первыми изображениями в зеркалах $17{,}44\,\text{см}$.
    Найдите угол (в градусах) между зеркалами.
}
\answer{%
    $\cos \alpha = \frac{c^2 - \sqr{2a} - \sqr{2b}}{2 \cdot 2a \cdot 2b} \approx 0{,}501 \implies \alpha = 59{,}9\degrees$
}

\variantsplitter

\addpersonalvariant{Софья Белянкина}

\tasknumber{1}%
\task{%
    Параллельный пучок света распространяется горизонтально.
    Под каким углом (в градусах) к горизонту следует расположить плоское зеркало,
    чтобы отраженный пучок распространялся вертикально?
}
\answer{%
    $2\,\frac{\text{см}}{\text{с}}$
}
\solutionspace{80pt}

\tasknumber{2}%
\task{%
    Под каким углом (в градусах) к горизонту следует расположить плоское зеркало,
    чтобы осветить дно вертикального колодца отраженными от зеркала солнечными лучами,
    падающими под углом $24\degrees$ к горизонту?
}
\answer{%
    $\alpha = 24\degrees, (90\degrees - \beta + \alpha) + (90\degrees - \beta) = 90\degrees \implies \beta = 45\degrees + \frac \alpha 2 = 57\degrees$
}
\solutionspace{80pt}

\tasknumber{3}%
\task{%
    Лиана стоит перед плоским зеркалом, укрепленным на вертикальной стене.
    Какова должна быть минимальная высота зеркала, чтобы Лиана мог видеть себя в полный рост?
    Рост Лианы $158\,\text{см}$.
    Определите также расстояние от пола до нижнего края зеркала,
    приняв высоту головы равной $30\,\text{см}$, и считая, что глаза находятся посередине (по высоте) головы.
}
\answer{%
    $79\,\text{см}, 71{,}500\,\text{см}$
}
\solutionspace{80pt}

\tasknumber{4}%
\task{%
    Во сколько раз увеличится расстояние между предметом и его изображением
    в плоском зеркале, если зеркало переместить в то место, где было изображение? Предмет остаётся неподвижным.
}
\answer{%
    $2$
}
\solutionspace{80pt}

\tasknumber{5}%
\task{%
    Плоское зеркало движется по направлению к точечному источнику света со скоростью $18\,\frac{\text{см}}{\text{с}}$.
    Определите скорость движения изображения относительно зеркала.
    Направление скорости зеркала перпендикулярно плоскости зеркала.
}
\answer{%
    $18\,\frac{\text{см}}{\text{с}}$
}
\solutionspace{80pt}

\tasknumber{6}%
\task{%
    Сколько изображений получится от предмета в двух плоских зеркалах,
    поставленных под углом $60\degrees$ друг к другу?
}
\answer{%
    $5$
}
\solutionspace{80pt}

\tasknumber{7}%
\task{%
    Два плоских зеркала располагаются под углом друг к другу
    и между ними помещается точечный источник света.
    Расстояние от этого источника до одного зеркала $4\,\text{см}$, до другого $6\,\text{см}$.
    Расстояние между первыми изображениями в зеркалах $19{,}35\,\text{см}$.
    Найдите угол (в градусах) между зеркалами.
}
\answer{%
    $\cos \alpha = \frac{c^2 - \sqr{2a} - \sqr{2b}}{2 \cdot 2a \cdot 2b} \approx 0{,}867 \implies \alpha = 29{,}9\degrees$
}

\variantsplitter

\addpersonalvariant{Варвара Егиазарян}

\tasknumber{1}%
\task{%
    Параллельный пучок света распространяется горизонтально.
    Под каким углом (в градусах) к горизонту следует расположить плоское зеркало,
    чтобы отраженный пучок распространялся вертикально?
}
\answer{%
    $2\,\frac{\text{см}}{\text{с}}$
}
\solutionspace{80pt}

\tasknumber{2}%
\task{%
    Под каким углом (в градусах) к горизонту следует расположить плоское зеркало,
    чтобы осветить дно вертикального колодца отраженными от зеркала солнечными лучами,
    падающими под углом $44\degrees$ к горизонту?
}
\answer{%
    $\alpha = 44\degrees, (90\degrees - \beta + \alpha) + (90\degrees - \beta) = 90\degrees \implies \beta = 45\degrees + \frac \alpha 2 = 67\degrees$
}
\solutionspace{80pt}

\tasknumber{3}%
\task{%
    Аня стоит перед плоским зеркалом, укрепленным на вертикальной стене.
    Какова должна быть минимальная высота зеркала, чтобы Аня мог видеть себя в полный рост?
    Рост Ани $190\,\text{см}$.
    Определите также расстояние от пола до нижнего края зеркала,
    приняв высоту головы равной $30\,\text{см}$, и считая, что глаза находятся посередине (по высоте) головы.
}
\answer{%
    $95\,\text{см}, 87{,}500\,\text{см}$
}
\solutionspace{80pt}

\tasknumber{4}%
\task{%
    Во сколько раз увеличится расстояние между предметом и его изображением
    в плоском зеркале, если зеркало переместить в то место, где было изображение? Предмет остаётся неподвижным.
}
\answer{%
    $2$
}
\solutionspace{80pt}

\tasknumber{5}%
\task{%
    Плоское зеркало движется по направлению к точечному источнику света со скоростью $12\,\frac{\text{см}}{\text{с}}$.
    Определите скорость движения изображения относительно источника света.
    Направление скорости зеркала перпендикулярно плоскости зеркала.
}
\answer{%
    $24\,\frac{\text{см}}{\text{с}}$
}
\solutionspace{80pt}

\tasknumber{6}%
\task{%
    Сколько изображений получится от предмета в двух плоских зеркалах,
    поставленных под углом $90\degrees$ друг к другу?
}
\answer{%
    $3$
}
\solutionspace{80pt}

\tasknumber{7}%
\task{%
    Два плоских зеркала располагаются под углом друг к другу
    и между ними помещается точечный источник света.
    Расстояние от этого источника до одного зеркала $5\,\text{см}$, до другого $8\,\text{см}$.
    Расстояние между первыми изображениями в зеркалах $22{,}72\,\text{см}$.
    Найдите угол (в градусах) между зеркалами.
}
\answer{%
    $\cos \alpha = \frac{c^2 - \sqr{2a} - \sqr{2b}}{2 \cdot 2a \cdot 2b} \approx 0{,}501 \implies \alpha = 60{,}0\degrees$
}

\variantsplitter

\addpersonalvariant{Владислав Емелин}

\tasknumber{1}%
\task{%
    Параллельный пучок света распространяется горизонтально.
    Под каким углом (в градусах) к горизонту следует расположить плоское зеркало,
    чтобы отраженный пучок распространялся вертикально?
}
\answer{%
    $3\,\frac{\text{см}}{\text{с}}$
}
\solutionspace{80pt}

\tasknumber{2}%
\task{%
    Под каким углом (в градусах) к горизонту следует расположить плоское зеркало,
    чтобы осветить дно вертикального колодца отраженными от зеркала солнечными лучами,
    падающими под углом $54\degrees$ к вертикали?
}
\answer{%
    $\alpha = 36\degrees, (90\degrees - \beta + \alpha) + (90\degrees - \beta) = 90\degrees \implies \beta = 45\degrees + \frac \alpha 2 = 63\degrees$
}
\solutionspace{80pt}

\tasknumber{3}%
\task{%
    Малика стоит перед плоским зеркалом, укрепленным на вертикальной стене.
    Какова должна быть минимальная высота зеркала, чтобы Малика мог видеть себя в полный рост?
    Рост Малики $184\,\text{см}$.
    Определите также расстояние от пола до нижнего края зеркала,
    приняв высоту головы равной $34\,\text{см}$, и считая, что глаза находятся посередине (по высоте) головы.
}
\answer{%
    $92\,\text{см}, 83{,}500\,\text{см}$
}
\solutionspace{80pt}

\tasknumber{4}%
\task{%
    Во сколько раз увеличится расстояние между предметом и его изображением
    в плоском зеркале, если зеркало переместить в то место, где было изображение? Предмет остаётся неподвижным.
}
\answer{%
    $2$
}
\solutionspace{80pt}

\tasknumber{5}%
\task{%
    Плоское зеркало движется по направлению к точечному источнику света со скоростью $20\,\frac{\text{см}}{\text{с}}$.
    Определите скорость движения изображения относительно источника света.
    Направление скорости зеркала перпендикулярно плоскости зеркала.
}
\answer{%
    $40\,\frac{\text{см}}{\text{с}}$
}
\solutionspace{80pt}

\tasknumber{6}%
\task{%
    Сколько изображений получится от предмета в двух плоских зеркалах,
    поставленных под углом $60\degrees$ друг к другу?
}
\answer{%
    $5$
}
\solutionspace{80pt}

\tasknumber{7}%
\task{%
    Два плоских зеркала располагаются под углом друг к другу
    и между ними помещается точечный источник света.
    Расстояние от этого источника до одного зеркала $3\,\text{см}$, до другого $7\,\text{см}$.
    Расстояние между первыми изображениями в зеркалах $18{,}73\,\text{см}$.
    Найдите угол (в градусах) между зеркалами.
}
\answer{%
    $\cos \alpha = \frac{c^2 - \sqr{2a} - \sqr{2b}}{2 \cdot 2a \cdot 2b} \approx 0{,}707 \implies \alpha = 45{,}0\degrees$
}

\variantsplitter

\addpersonalvariant{Артём Жичин}

\tasknumber{1}%
\task{%
    Параллельный пучок света распространяется горизонтально.
    Под каким углом (в градусах) к горизонту следует расположить плоское зеркало,
    чтобы отраженный пучок распространялся вертикально?
}
\answer{%
    $2\,\frac{\text{см}}{\text{с}}$
}
\solutionspace{80pt}

\tasknumber{2}%
\task{%
    Под каким углом (в градусах) к горизонту следует расположить плоское зеркало,
    чтобы осветить дно вертикального колодца отраженными от зеркала солнечными лучами,
    падающими под углом $54\degrees$ к вертикали?
}
\answer{%
    $\alpha = 36\degrees, (90\degrees - \beta + \alpha) + (90\degrees - \beta) = 90\degrees \implies \beta = 45\degrees + \frac \alpha 2 = 63\degrees$
}
\solutionspace{80pt}

\tasknumber{3}%
\task{%
    Ульяна стоит перед плоским зеркалом, укрепленным на вертикальной стене.
    Какова должна быть минимальная высота зеркала, чтобы Ульяна мог видеть себя в полный рост?
    Рост Ульяны $184\,\text{см}$.
    Определите также расстояние от пола до нижнего края зеркала,
    приняв высоту головы равной $30\,\text{см}$, и считая, что глаза находятся посередине (по высоте) головы.
}
\answer{%
    $92\,\text{см}, 84{,}500\,\text{см}$
}
\solutionspace{80pt}

\tasknumber{4}%
\task{%
    Во сколько раз увеличится расстояние между предметом и его изображением
    в плоском зеркале, если зеркало переместить в то место, где было изображение? Предмет остаётся неподвижным.
}
\answer{%
    $2$
}
\solutionspace{80pt}

\tasknumber{5}%
\task{%
    Плоское зеркало движется по направлению к точечному источнику света со скоростью $20\,\frac{\text{см}}{\text{с}}$.
    Определите скорость движения изображения относительно зеркала.
    Направление скорости зеркала перпендикулярно плоскости зеркала.
}
\answer{%
    $20\,\frac{\text{см}}{\text{с}}$
}
\solutionspace{80pt}

\tasknumber{6}%
\task{%
    Сколько изображений получится от предмета в двух плоских зеркалах,
    поставленных под углом $60\degrees$ друг к другу?
}
\answer{%
    $5$
}
\solutionspace{80pt}

\tasknumber{7}%
\task{%
    Два плоских зеркала располагаются под углом друг к другу
    и между ними помещается точечный источник света.
    Расстояние от этого источника до одного зеркала $4\,\text{см}$, до другого $6\,\text{см}$.
    Расстояние между первыми изображениями в зеркалах $17{,}44\,\text{см}$.
    Найдите угол (в градусах) между зеркалами.
}
\answer{%
    $\cos \alpha = \frac{c^2 - \sqr{2a} - \sqr{2b}}{2 \cdot 2a \cdot 2b} \approx 0{,}501 \implies \alpha = 59{,}9\degrees$
}

\variantsplitter

\addpersonalvariant{Дарья Кошман}

\tasknumber{1}%
\task{%
    Параллельный пучок света распространяется горизонтально.
    Под каким углом (в градусах) к горизонту следует расположить плоское зеркало,
    чтобы отраженный пучок распространялся вертикально?
}
\answer{%
    $3\,\frac{\text{см}}{\text{с}}$
}
\solutionspace{80pt}

\tasknumber{2}%
\task{%
    Под каким углом (в градусах) к горизонту следует расположить плоское зеркало,
    чтобы осветить дно вертикального колодца отраженными от зеркала солнечными лучами,
    падающими под углом $50\degrees$ к горизонту?
}
\answer{%
    $\alpha = 50\degrees, (90\degrees - \beta + \alpha) + (90\degrees - \beta) = 90\degrees \implies \beta = 45\degrees + \frac \alpha 2 = 70\degrees$
}
\solutionspace{80pt}

\tasknumber{3}%
\task{%
    Полина стоит перед плоским зеркалом, укрепленным на вертикальной стене.
    Какова должна быть минимальная высота зеркала, чтобы Полина мог видеть себя в полный рост?
    Рост Полины $160\,\text{см}$.
    Определите также расстояние от пола до нижнего края зеркала,
    приняв высоту головы равной $32\,\text{см}$, и считая, что глаза находятся посередине (по высоте) головы.
}
\answer{%
    $80\,\text{см}, 72\,\text{см}$
}
\solutionspace{80pt}

\tasknumber{4}%
\task{%
    Во сколько раз увеличится расстояние между предметом и его изображением
    в плоском зеркале, если зеркало переместить в то место, где было изображение? Предмет остаётся неподвижным.
}
\answer{%
    $2$
}
\solutionspace{80pt}

\tasknumber{5}%
\task{%
    Плоское зеркало движется по направлению к точечному источнику света со скоростью $10\,\frac{\text{см}}{\text{с}}$.
    Определите скорость движения изображения относительно источника света.
    Направление скорости зеркала перпендикулярно плоскости зеркала.
}
\answer{%
    $20\,\frac{\text{см}}{\text{с}}$
}
\solutionspace{80pt}

\tasknumber{6}%
\task{%
    Сколько изображений получится от предмета в двух плоских зеркалах,
    поставленных под углом $90\degrees$ друг к другу?
}
\answer{%
    $3$
}
\solutionspace{80pt}

\tasknumber{7}%
\task{%
    Два плоских зеркала располагаются под углом друг к другу
    и между ними помещается точечный источник света.
    Расстояние от этого источника до одного зеркала $3\,\text{см}$, до другого $8\,\text{см}$.
    Расстояние между первыми изображениями в зеркалах $21{,}41\,\text{см}$.
    Найдите угол (в градусах) между зеркалами.
}
\answer{%
    $\cos \alpha = \frac{c^2 - \sqr{2a} - \sqr{2b}}{2 \cdot 2a \cdot 2b} \approx 0{,}867 \implies \alpha = 29{,}9\degrees$
}

\variantsplitter

\addpersonalvariant{Анна Кузьмичёва}

\tasknumber{1}%
\task{%
    Параллельный пучок света распространяется горизонтально.
    Под каким углом (в градусах) к горизонту следует расположить плоское зеркало,
    чтобы отраженный пучок распространялся вертикально?
}
\answer{%
    $3\,\frac{\text{см}}{\text{с}}$
}
\solutionspace{80pt}

\tasknumber{2}%
\task{%
    Под каким углом (в градусах) к горизонту следует расположить плоское зеркало,
    чтобы осветить дно вертикального колодца отраженными от зеркала солнечными лучами,
    падающими под углом $38\degrees$ к горизонту?
}
\answer{%
    $\alpha = 38\degrees, (90\degrees - \beta + \alpha) + (90\degrees - \beta) = 90\degrees \implies \beta = 45\degrees + \frac \alpha 2 = 64\degrees$
}
\solutionspace{80pt}

\tasknumber{3}%
\task{%
    Нелли стоит перед плоским зеркалом, укрепленным на вертикальной стене.
    Какова должна быть минимальная высота зеркала, чтобы Нелли мог видеть себя в полный рост?
    Рост Нелли $190\,\text{см}$.
    Определите также расстояние от пола до нижнего края зеркала,
    приняв высоту головы равной $34\,\text{см}$, и считая, что глаза находятся посередине (по высоте) головы.
}
\answer{%
    $95\,\text{см}, 86{,}500\,\text{см}$
}
\solutionspace{80pt}

\tasknumber{4}%
\task{%
    Во сколько раз увеличится расстояние между предметом и его изображением
    в плоском зеркале, если зеркало переместить в то место, где было изображение? Предмет остаётся неподвижным.
}
\answer{%
    $2$
}
\solutionspace{80pt}

\tasknumber{5}%
\task{%
    Плоское зеркало движется по направлению к точечному источнику света со скоростью $12\,\frac{\text{см}}{\text{с}}$.
    Определите скорость движения изображения относительно зеркала.
    Направление скорости зеркала перпендикулярно плоскости зеркала.
}
\answer{%
    $12\,\frac{\text{см}}{\text{с}}$
}
\solutionspace{80pt}

\tasknumber{6}%
\task{%
    Сколько изображений получится от предмета в двух плоских зеркалах,
    поставленных под углом $60\degrees$ друг к другу?
}
\answer{%
    $5$
}
\solutionspace{80pt}

\tasknumber{7}%
\task{%
    Два плоских зеркала располагаются под углом друг к другу
    и между ними помещается точечный источник света.
    Расстояние от этого источника до одного зеркала $5\,\text{см}$, до другого $8\,\text{см}$.
    Расстояние между первыми изображениями в зеркалах $24{,}13\,\text{см}$.
    Найдите угол (в градусах) между зеркалами.
}
\answer{%
    $\cos \alpha = \frac{c^2 - \sqr{2a} - \sqr{2b}}{2 \cdot 2a \cdot 2b} \approx 0{,}707 \implies \alpha = 45{,}0\degrees$
}

\variantsplitter

\addpersonalvariant{Алёна Куприянова}

\tasknumber{1}%
\task{%
    Параллельный пучок света распространяется горизонтально.
    Под каким углом (в градусах) к горизонту следует расположить плоское зеркало,
    чтобы отраженный пучок распространялся вертикально?
}
\answer{%
    $2\,\frac{\text{см}}{\text{с}}$
}
\solutionspace{80pt}

\tasknumber{2}%
\task{%
    Под каким углом (в градусах) к горизонту следует расположить плоское зеркало,
    чтобы осветить дно вертикального колодца отраженными от зеркала солнечными лучами,
    падающими под углом $40\degrees$ к вертикали?
}
\answer{%
    $\alpha = 50\degrees, (90\degrees - \beta + \alpha) + (90\degrees - \beta) = 90\degrees \implies \beta = 45\degrees + \frac \alpha 2 = 70\degrees$
}
\solutionspace{80pt}

\tasknumber{3}%
\task{%
    Варя стоит перед плоским зеркалом, укрепленным на вертикальной стене.
    Какова должна быть минимальная высота зеркала, чтобы Варя мог видеть себя в полный рост?
    Рост Вари $178\,\text{см}$.
    Определите также расстояние от пола до нижнего края зеркала,
    приняв высоту головы равной $34\,\text{см}$, и считая, что глаза находятся посередине (по высоте) головы.
}
\answer{%
    $89\,\text{см}, 80{,}500\,\text{см}$
}
\solutionspace{80pt}

\tasknumber{4}%
\task{%
    Во сколько раз увеличится расстояние между предметом и его изображением
    в плоском зеркале, если зеркало переместить в то место, где было изображение? Предмет остаётся неподвижным.
}
\answer{%
    $2$
}
\solutionspace{80pt}

\tasknumber{5}%
\task{%
    Плоское зеркало движется по направлению к точечному источнику света со скоростью $12\,\frac{\text{см}}{\text{с}}$.
    Определите скорость движения изображения относительно источника света.
    Направление скорости зеркала перпендикулярно плоскости зеркала.
}
\answer{%
    $24\,\frac{\text{см}}{\text{с}}$
}
\solutionspace{80pt}

\tasknumber{6}%
\task{%
    Сколько изображений получится от предмета в двух плоских зеркалах,
    поставленных под углом $30\degrees$ друг к другу?
}
\answer{%
    $11$
}
\solutionspace{80pt}

\tasknumber{7}%
\task{%
    Два плоских зеркала располагаются под углом друг к другу
    и между ними помещается точечный источник света.
    Расстояние от этого источника до одного зеркала $4\,\text{см}$, до другого $8\,\text{см}$.
    Расстояние между первыми изображениями в зеркалах $23{,}27\,\text{см}$.
    Найдите угол (в градусах) между зеркалами.
}
\answer{%
    $\cos \alpha = \frac{c^2 - \sqr{2a} - \sqr{2b}}{2 \cdot 2a \cdot 2b} \approx 0{,}865 \implies \alpha = 30{,}1\degrees$
}

\variantsplitter

\addpersonalvariant{Ярослав Лавровский}

\tasknumber{1}%
\task{%
    Параллельный пучок света распространяется горизонтально.
    Под каким углом (в градусах) к горизонту следует расположить плоское зеркало,
    чтобы отраженный пучок распространялся вертикально?
}
\answer{%
    $4\,\frac{\text{см}}{\text{с}}$
}
\solutionspace{80pt}

\tasknumber{2}%
\task{%
    Под каким углом (в градусах) к горизонту следует расположить плоское зеркало,
    чтобы осветить дно вертикального колодца отраженными от зеркала солнечными лучами,
    падающими под углом $26\degrees$ к горизонту?
}
\answer{%
    $\alpha = 26\degrees, (90\degrees - \beta + \alpha) + (90\degrees - \beta) = 90\degrees \implies \beta = 45\degrees + \frac \alpha 2 = 58\degrees$
}
\solutionspace{80pt}

\tasknumber{3}%
\task{%
    Борис стоит перед плоским зеркалом, укрепленным на вертикальной стене.
    Какова должна быть минимальная высота зеркала, чтобы Борис мог видеть себя в полный рост?
    Рост Бориса $176\,\text{см}$.
    Определите также расстояние от пола до нижнего края зеркала,
    приняв высоту головы равной $32\,\text{см}$, и считая, что глаза находятся посередине (по высоте) головы.
}
\answer{%
    $88\,\text{см}, 80\,\text{см}$
}
\solutionspace{80pt}

\tasknumber{4}%
\task{%
    Во сколько раз увеличится расстояние между предметом и его изображением
    в плоском зеркале, если зеркало переместить в то место, где было изображение? Предмет остаётся неподвижным.
}
\answer{%
    $2$
}
\solutionspace{80pt}

\tasknumber{5}%
\task{%
    Плоское зеркало движется по направлению к точечному источнику света со скоростью $18\,\frac{\text{см}}{\text{с}}$.
    Определите скорость движения изображения относительно зеркала.
    Направление скорости зеркала перпендикулярно плоскости зеркала.
}
\answer{%
    $18\,\frac{\text{см}}{\text{с}}$
}
\solutionspace{80pt}

\tasknumber{6}%
\task{%
    Сколько изображений получится от предмета в двух плоских зеркалах,
    поставленных под углом $90\degrees$ друг к другу?
}
\answer{%
    $3$
}
\solutionspace{80pt}

\tasknumber{7}%
\task{%
    Два плоских зеркала располагаются под углом друг к другу
    и между ними помещается точечный источник света.
    Расстояние от этого источника до одного зеркала $4\,\text{см}$, до другого $7\,\text{см}$.
    Расстояние между первыми изображениями в зеркалах $21{,}31\,\text{см}$.
    Найдите угол (в градусах) между зеркалами.
}
\answer{%
    $\cos \alpha = \frac{c^2 - \sqr{2a} - \sqr{2b}}{2 \cdot 2a \cdot 2b} \approx 0{,}867 \implies \alpha = 29{,}9\degrees$
}

\variantsplitter

\addpersonalvariant{Анастасия Ламанова}

\tasknumber{1}%
\task{%
    Параллельный пучок света распространяется горизонтально.
    Под каким углом (в градусах) к горизонту следует расположить плоское зеркало,
    чтобы отраженный пучок распространялся вертикально?
}
\answer{%
    $4\,\frac{\text{см}}{\text{с}}$
}
\solutionspace{80pt}

\tasknumber{2}%
\task{%
    Под каким углом (в градусах) к горизонту следует расположить плоское зеркало,
    чтобы осветить дно вертикального колодца отраженными от зеркала солнечными лучами,
    падающими под углом $56\degrees$ к горизонту?
}
\answer{%
    $\alpha = 56\degrees, (90\degrees - \beta + \alpha) + (90\degrees - \beta) = 90\degrees \implies \beta = 45\degrees + \frac \alpha 2 = 73\degrees$
}
\solutionspace{80pt}

\tasknumber{3}%
\task{%
    Малика стоит перед плоским зеркалом, укрепленным на вертикальной стене.
    Какова должна быть минимальная высота зеркала, чтобы Малика мог видеть себя в полный рост?
    Рост Малики $172\,\text{см}$.
    Определите также расстояние от пола до нижнего края зеркала,
    приняв высоту головы равной $32\,\text{см}$, и считая, что глаза находятся посередине (по высоте) головы.
}
\answer{%
    $86\,\text{см}, 78\,\text{см}$
}
\solutionspace{80pt}

\tasknumber{4}%
\task{%
    Во сколько раз увеличится расстояние между предметом и его изображением
    в плоском зеркале, если зеркало переместить в то место, где было изображение? Предмет остаётся неподвижным.
}
\answer{%
    $2$
}
\solutionspace{80pt}

\tasknumber{5}%
\task{%
    Плоское зеркало движется по направлению к точечному источнику света со скоростью $15\,\frac{\text{см}}{\text{с}}$.
    Определите скорость движения изображения относительно зеркала.
    Направление скорости зеркала перпендикулярно плоскости зеркала.
}
\answer{%
    $15\,\frac{\text{см}}{\text{с}}$
}
\solutionspace{80pt}

\tasknumber{6}%
\task{%
    Сколько изображений получится от предмета в двух плоских зеркалах,
    поставленных под углом $30\degrees$ друг к другу?
}
\answer{%
    $11$
}
\solutionspace{80pt}

\tasknumber{7}%
\task{%
    Два плоских зеркала располагаются под углом друг к другу
    и между ними помещается точечный источник света.
    Расстояние от этого источника до одного зеркала $3\,\text{см}$, до другого $8\,\text{см}$.
    Расстояние между первыми изображениями в зеркалах $20{,}68\,\text{см}$.
    Найдите угол (в градусах) между зеркалами.
}
\answer{%
    $\cos \alpha = \frac{c^2 - \sqr{2a} - \sqr{2b}}{2 \cdot 2a \cdot 2b} \approx 0{,}707 \implies \alpha = 45{,}0\degrees$
}

\variantsplitter

\addpersonalvariant{Виктория Легонькова}

\tasknumber{1}%
\task{%
    Параллельный пучок света распространяется горизонтально.
    Под каким углом (в градусах) к горизонту следует расположить плоское зеркало,
    чтобы отраженный пучок распространялся вертикально?
}
\answer{%
    $4\,\frac{\text{см}}{\text{с}}$
}
\solutionspace{80pt}

\tasknumber{2}%
\task{%
    Под каким углом (в градусах) к горизонту следует расположить плоское зеркало,
    чтобы осветить дно вертикального колодца отраженными от зеркала солнечными лучами,
    падающими под углом $54\degrees$ к горизонту?
}
\answer{%
    $\alpha = 54\degrees, (90\degrees - \beta + \alpha) + (90\degrees - \beta) = 90\degrees \implies \beta = 45\degrees + \frac \alpha 2 = 72\degrees$
}
\solutionspace{80pt}

\tasknumber{3}%
\task{%
    Ульяна стоит перед плоским зеркалом, укрепленным на вертикальной стене.
    Какова должна быть минимальная высота зеркала, чтобы Ульяна мог видеть себя в полный рост?
    Рост Ульяны $160\,\text{см}$.
    Определите также расстояние от пола до нижнего края зеркала,
    приняв высоту головы равной $32\,\text{см}$, и считая, что глаза находятся посередине (по высоте) головы.
}
\answer{%
    $80\,\text{см}, 72\,\text{см}$
}
\solutionspace{80pt}

\tasknumber{4}%
\task{%
    Во сколько раз увеличится расстояние между предметом и его изображением
    в плоском зеркале, если зеркало переместить в то место, где было изображение? Предмет остаётся неподвижным.
}
\answer{%
    $2$
}
\solutionspace{80pt}

\tasknumber{5}%
\task{%
    Плоское зеркало движется по направлению к точечному источнику света со скоростью $12\,\frac{\text{см}}{\text{с}}$.
    Определите скорость движения изображения относительно зеркала.
    Направление скорости зеркала перпендикулярно плоскости зеркала.
}
\answer{%
    $12\,\frac{\text{см}}{\text{с}}$
}
\solutionspace{80pt}

\tasknumber{6}%
\task{%
    Сколько изображений получится от предмета в двух плоских зеркалах,
    поставленных под углом $90\degrees$ друг к другу?
}
\answer{%
    $3$
}
\solutionspace{80pt}

\tasknumber{7}%
\task{%
    Два плоских зеркала располагаются под углом друг к другу
    и между ними помещается точечный источник света.
    Расстояние от этого источника до одного зеркала $5\,\text{см}$, до другого $8\,\text{см}$.
    Расстояние между первыми изображениями в зеркалах $22{,}72\,\text{см}$.
    Найдите угол (в градусах) между зеркалами.
}
\answer{%
    $\cos \alpha = \frac{c^2 - \sqr{2a} - \sqr{2b}}{2 \cdot 2a \cdot 2b} \approx 0{,}501 \implies \alpha = 60{,}0\degrees$
}

\variantsplitter

\addpersonalvariant{Семён Мартынов}

\tasknumber{1}%
\task{%
    Параллельный пучок света распространяется горизонтально.
    Под каким углом (в градусах) к горизонту следует расположить плоское зеркало,
    чтобы отраженный пучок распространялся вертикально?
}
\answer{%
    $3\,\frac{\text{см}}{\text{с}}$
}
\solutionspace{80pt}

\tasknumber{2}%
\task{%
    Под каким углом (в градусах) к горизонту следует расположить плоское зеркало,
    чтобы осветить дно вертикального колодца отраженными от зеркала солнечными лучами,
    падающими под углом $62\degrees$ к вертикали?
}
\answer{%
    $\alpha = 28\degrees, (90\degrees - \beta + \alpha) + (90\degrees - \beta) = 90\degrees \implies \beta = 45\degrees + \frac \alpha 2 = 59\degrees$
}
\solutionspace{80pt}

\tasknumber{3}%
\task{%
    Лиана стоит перед плоским зеркалом, укрепленным на вертикальной стене.
    Какова должна быть минимальная высота зеркала, чтобы Лиана мог видеть себя в полный рост?
    Рост Лианы $166\,\text{см}$.
    Определите также расстояние от пола до нижнего края зеркала,
    приняв высоту головы равной $30\,\text{см}$, и считая, что глаза находятся посередине (по высоте) головы.
}
\answer{%
    $83\,\text{см}, 75{,}500\,\text{см}$
}
\solutionspace{80pt}

\tasknumber{4}%
\task{%
    Во сколько раз увеличится расстояние между предметом и его изображением
    в плоском зеркале, если зеркало переместить в то место, где было изображение? Предмет остаётся неподвижным.
}
\answer{%
    $2$
}
\solutionspace{80pt}

\tasknumber{5}%
\task{%
    Плоское зеркало движется по направлению к точечному источнику света со скоростью $18\,\frac{\text{см}}{\text{с}}$.
    Определите скорость движения изображения относительно зеркала.
    Направление скорости зеркала перпендикулярно плоскости зеркала.
}
\answer{%
    $18\,\frac{\text{см}}{\text{с}}$
}
\solutionspace{80pt}

\tasknumber{6}%
\task{%
    Сколько изображений получится от предмета в двух плоских зеркалах,
    поставленных под углом $90\degrees$ друг к другу?
}
\answer{%
    $3$
}
\solutionspace{80pt}

\tasknumber{7}%
\task{%
    Два плоских зеркала располагаются под углом друг к другу
    и между ними помещается точечный источник света.
    Расстояние от этого источника до одного зеркала $4\,\text{см}$, до другого $8\,\text{см}$.
    Расстояние между первыми изображениями в зеркалах $23{,}27\,\text{см}$.
    Найдите угол (в градусах) между зеркалами.
}
\answer{%
    $\cos \alpha = \frac{c^2 - \sqr{2a} - \sqr{2b}}{2 \cdot 2a \cdot 2b} \approx 0{,}865 \implies \alpha = 30{,}1\degrees$
}

\variantsplitter

\addpersonalvariant{Варвара Минаева}

\tasknumber{1}%
\task{%
    Параллельный пучок света распространяется горизонтально.
    Под каким углом (в градусах) к горизонту следует расположить плоское зеркало,
    чтобы отраженный пучок распространялся вертикально?
}
\answer{%
    $4\,\frac{\text{см}}{\text{с}}$
}
\solutionspace{80pt}

\tasknumber{2}%
\task{%
    Под каким углом (в градусах) к горизонту следует расположить плоское зеркало,
    чтобы осветить дно вертикального колодца отраженными от зеркала солнечными лучами,
    падающими под углом $34\degrees$ к вертикали?
}
\answer{%
    $\alpha = 56\degrees, (90\degrees - \beta + \alpha) + (90\degrees - \beta) = 90\degrees \implies \beta = 45\degrees + \frac \alpha 2 = 73\degrees$
}
\solutionspace{80pt}

\tasknumber{3}%
\task{%
    Камиля стоит перед плоским зеркалом, укрепленным на вертикальной стене.
    Какова должна быть минимальная высота зеркала, чтобы Камиля мог видеть себя в полный рост?
    Рост Камили $168\,\text{см}$.
    Определите также расстояние от пола до нижнего края зеркала,
    приняв высоту головы равной $32\,\text{см}$, и считая, что глаза находятся посередине (по высоте) головы.
}
\answer{%
    $84\,\text{см}, 76\,\text{см}$
}
\solutionspace{80pt}

\tasknumber{4}%
\task{%
    Во сколько раз увеличится расстояние между предметом и его изображением
    в плоском зеркале, если зеркало переместить в то место, где было изображение? Предмет остаётся неподвижным.
}
\answer{%
    $2$
}
\solutionspace{80pt}

\tasknumber{5}%
\task{%
    Плоское зеркало движется по направлению к точечному источнику света со скоростью $18\,\frac{\text{см}}{\text{с}}$.
    Определите скорость движения изображения относительно источника света.
    Направление скорости зеркала перпендикулярно плоскости зеркала.
}
\answer{%
    $36\,\frac{\text{см}}{\text{с}}$
}
\solutionspace{80pt}

\tasknumber{6}%
\task{%
    Сколько изображений получится от предмета в двух плоских зеркалах,
    поставленных под углом $90\degrees$ друг к другу?
}
\answer{%
    $3$
}
\solutionspace{80pt}

\tasknumber{7}%
\task{%
    Два плоских зеркала располагаются под углом друг к другу
    и между ними помещается точечный источник света.
    Расстояние от этого источника до одного зеркала $4\,\text{см}$, до другого $9\,\text{см}$.
    Расстояние между первыми изображениями в зеркалах $23{,}07\,\text{см}$.
    Найдите угол (в градусах) между зеркалами.
}
\answer{%
    $\cos \alpha = \frac{c^2 - \sqr{2a} - \sqr{2b}}{2 \cdot 2a \cdot 2b} \approx 0{,}501 \implies \alpha = 59{,}9\degrees$
}

\variantsplitter

\addpersonalvariant{Леонид Никитин}

\tasknumber{1}%
\task{%
    Параллельный пучок света распространяется горизонтально.
    Под каким углом (в градусах) к горизонту следует расположить плоское зеркало,
    чтобы отраженный пучок распространялся вертикально?
}
\answer{%
    $4\,\frac{\text{см}}{\text{с}}$
}
\solutionspace{80pt}

\tasknumber{2}%
\task{%
    Под каким углом (в градусах) к горизонту следует расположить плоское зеркало,
    чтобы осветить дно вертикального колодца отраженными от зеркала солнечными лучами,
    падающими под углом $22\degrees$ к горизонту?
}
\answer{%
    $\alpha = 22\degrees, (90\degrees - \beta + \alpha) + (90\degrees - \beta) = 90\degrees \implies \beta = 45\degrees + \frac \alpha 2 = 56\degrees$
}
\solutionspace{80pt}

\tasknumber{3}%
\task{%
    Нелли стоит перед плоским зеркалом, укрепленным на вертикальной стене.
    Какова должна быть минимальная высота зеркала, чтобы Нелли мог видеть себя в полный рост?
    Рост Нелли $166\,\text{см}$.
    Определите также расстояние от пола до нижнего края зеркала,
    приняв высоту головы равной $34\,\text{см}$, и считая, что глаза находятся посередине (по высоте) головы.
}
\answer{%
    $83\,\text{см}, 74{,}500\,\text{см}$
}
\solutionspace{80pt}

\tasknumber{4}%
\task{%
    Во сколько раз увеличится расстояние между предметом и его изображением
    в плоском зеркале, если зеркало переместить в то место, где было изображение? Предмет остаётся неподвижным.
}
\answer{%
    $2$
}
\solutionspace{80pt}

\tasknumber{5}%
\task{%
    Плоское зеркало движется по направлению к точечному источнику света со скоростью $18\,\frac{\text{см}}{\text{с}}$.
    Определите скорость движения изображения относительно источника света.
    Направление скорости зеркала перпендикулярно плоскости зеркала.
}
\answer{%
    $36\,\frac{\text{см}}{\text{с}}$
}
\solutionspace{80pt}

\tasknumber{6}%
\task{%
    Сколько изображений получится от предмета в двух плоских зеркалах,
    поставленных под углом $30\degrees$ друг к другу?
}
\answer{%
    $11$
}
\solutionspace{80pt}

\tasknumber{7}%
\task{%
    Два плоских зеркала располагаются под углом друг к другу
    и между ними помещается точечный источник света.
    Расстояние от этого источника до одного зеркала $4\,\text{см}$, до другого $6\,\text{см}$.
    Расстояние между первыми изображениями в зеркалах $17{,}44\,\text{см}$.
    Найдите угол (в градусах) между зеркалами.
}
\answer{%
    $\cos \alpha = \frac{c^2 - \sqr{2a} - \sqr{2b}}{2 \cdot 2a \cdot 2b} \approx 0{,}501 \implies \alpha = 59{,}9\degrees$
}

\variantsplitter

\addpersonalvariant{Тимофей Полетаев}

\tasknumber{1}%
\task{%
    Параллельный пучок света распространяется горизонтально.
    Под каким углом (в градусах) к горизонту следует расположить плоское зеркало,
    чтобы отраженный пучок распространялся вертикально?
}
\answer{%
    $2\,\frac{\text{см}}{\text{с}}$
}
\solutionspace{80pt}

\tasknumber{2}%
\task{%
    Под каким углом (в градусах) к горизонту следует расположить плоское зеркало,
    чтобы осветить дно вертикального колодца отраженными от зеркала солнечными лучами,
    падающими под углом $26\degrees$ к горизонту?
}
\answer{%
    $\alpha = 26\degrees, (90\degrees - \beta + \alpha) + (90\degrees - \beta) = 90\degrees \implies \beta = 45\degrees + \frac \alpha 2 = 58\degrees$
}
\solutionspace{80pt}

\tasknumber{3}%
\task{%
    Женя стоит перед плоским зеркалом, укрепленным на вертикальной стене.
    Какова должна быть минимальная высота зеркала, чтобы Женя мог видеть себя в полный рост?
    Рост Жени $166\,\text{см}$.
    Определите также расстояние от пола до нижнего края зеркала,
    приняв высоту головы равной $30\,\text{см}$, и считая, что глаза находятся посередине (по высоте) головы.
}
\answer{%
    $83\,\text{см}, 75{,}500\,\text{см}$
}
\solutionspace{80pt}

\tasknumber{4}%
\task{%
    Во сколько раз увеличится расстояние между предметом и его изображением
    в плоском зеркале, если зеркало переместить в то место, где было изображение? Предмет остаётся неподвижным.
}
\answer{%
    $2$
}
\solutionspace{80pt}

\tasknumber{5}%
\task{%
    Плоское зеркало движется по направлению к точечному источнику света со скоростью $12\,\frac{\text{см}}{\text{с}}$.
    Определите скорость движения изображения относительно источника света.
    Направление скорости зеркала перпендикулярно плоскости зеркала.
}
\answer{%
    $24\,\frac{\text{см}}{\text{с}}$
}
\solutionspace{80pt}

\tasknumber{6}%
\task{%
    Сколько изображений получится от предмета в двух плоских зеркалах,
    поставленных под углом $60\degrees$ друг к другу?
}
\answer{%
    $5$
}
\solutionspace{80pt}

\tasknumber{7}%
\task{%
    Два плоских зеркала располагаются под углом друг к другу
    и между ними помещается точечный источник света.
    Расстояние от этого источника до одного зеркала $5\,\text{см}$, до другого $8\,\text{см}$.
    Расстояние между первыми изображениями в зеркалах $24{,}13\,\text{см}$.
    Найдите угол (в градусах) между зеркалами.
}
\answer{%
    $\cos \alpha = \frac{c^2 - \sqr{2a} - \sqr{2b}}{2 \cdot 2a \cdot 2b} \approx 0{,}707 \implies \alpha = 45{,}0\degrees$
}

\variantsplitter

\addpersonalvariant{Андрей Рожков}

\tasknumber{1}%
\task{%
    Параллельный пучок света распространяется горизонтально.
    Под каким углом (в градусах) к горизонту следует расположить плоское зеркало,
    чтобы отраженный пучок распространялся вертикально?
}
\answer{%
    $2\,\frac{\text{см}}{\text{с}}$
}
\solutionspace{80pt}

\tasknumber{2}%
\task{%
    Под каким углом (в градусах) к горизонту следует расположить плоское зеркало,
    чтобы осветить дно вертикального колодца отраженными от зеркала солнечными лучами,
    падающими под углом $38\degrees$ к вертикали?
}
\answer{%
    $\alpha = 52\degrees, (90\degrees - \beta + \alpha) + (90\degrees - \beta) = 90\degrees \implies \beta = 45\degrees + \frac \alpha 2 = 71\degrees$
}
\solutionspace{80pt}

\tasknumber{3}%
\task{%
    Денис стоит перед плоским зеркалом, укрепленным на вертикальной стене.
    Какова должна быть минимальная высота зеркала, чтобы Денис мог видеть себя в полный рост?
    Рост Дениса $180\,\text{см}$.
    Определите также расстояние от пола до нижнего края зеркала,
    приняв высоту головы равной $32\,\text{см}$, и считая, что глаза находятся посередине (по высоте) головы.
}
\answer{%
    $90\,\text{см}, 82\,\text{см}$
}
\solutionspace{80pt}

\tasknumber{4}%
\task{%
    Во сколько раз увеличится расстояние между предметом и его изображением
    в плоском зеркале, если зеркало переместить в то место, где было изображение? Предмет остаётся неподвижным.
}
\answer{%
    $2$
}
\solutionspace{80pt}

\tasknumber{5}%
\task{%
    Плоское зеркало движется по направлению к точечному источнику света со скоростью $20\,\frac{\text{см}}{\text{с}}$.
    Определите скорость движения изображения относительно зеркала.
    Направление скорости зеркала перпендикулярно плоскости зеркала.
}
\answer{%
    $20\,\frac{\text{см}}{\text{с}}$
}
\solutionspace{80pt}

\tasknumber{6}%
\task{%
    Сколько изображений получится от предмета в двух плоских зеркалах,
    поставленных под углом $45\degrees$ друг к другу?
}
\answer{%
    $7$
}
\solutionspace{80pt}

\tasknumber{7}%
\task{%
    Два плоских зеркала располагаются под углом друг к другу
    и между ними помещается точечный источник света.
    Расстояние от этого источника до одного зеркала $3\,\text{см}$, до другого $6\,\text{см}$.
    Расстояние между первыми изображениями в зеркалах $15{,}87\,\text{см}$.
    Найдите угол (в градусах) между зеркалами.
}
\answer{%
    $\cos \alpha = \frac{c^2 - \sqr{2a} - \sqr{2b}}{2 \cdot 2a \cdot 2b} \approx 0{,}499 \implies \alpha = 60{,}1\degrees$
}

\variantsplitter

\addpersonalvariant{Рената Таржиманова}

\tasknumber{1}%
\task{%
    Параллельный пучок света распространяется горизонтально.
    Под каким углом (в градусах) к горизонту следует расположить плоское зеркало,
    чтобы отраженный пучок распространялся вертикально?
}
\answer{%
    $4\,\frac{\text{см}}{\text{с}}$
}
\solutionspace{80pt}

\tasknumber{2}%
\task{%
    Под каким углом (в градусах) к горизонту следует расположить плоское зеркало,
    чтобы осветить дно вертикального колодца отраженными от зеркала солнечными лучами,
    падающими под углом $36\degrees$ к вертикали?
}
\answer{%
    $\alpha = 54\degrees, (90\degrees - \beta + \alpha) + (90\degrees - \beta) = 90\degrees \implies \beta = 45\degrees + \frac \alpha 2 = 72\degrees$
}
\solutionspace{80pt}

\tasknumber{3}%
\task{%
    Малика стоит перед плоским зеркалом, укрепленным на вертикальной стене.
    Какова должна быть минимальная высота зеркала, чтобы Малика мог видеть себя в полный рост?
    Рост Малики $170\,\text{см}$.
    Определите также расстояние от пола до нижнего края зеркала,
    приняв высоту головы равной $32\,\text{см}$, и считая, что глаза находятся посередине (по высоте) головы.
}
\answer{%
    $85\,\text{см}, 77\,\text{см}$
}
\solutionspace{80pt}

\tasknumber{4}%
\task{%
    Во сколько раз увеличится расстояние между предметом и его изображением
    в плоском зеркале, если зеркало переместить в то место, где было изображение? Предмет остаётся неподвижным.
}
\answer{%
    $2$
}
\solutionspace{80pt}

\tasknumber{5}%
\task{%
    Плоское зеркало движется по направлению к точечному источнику света со скоростью $10\,\frac{\text{см}}{\text{с}}$.
    Определите скорость движения изображения относительно зеркала.
    Направление скорости зеркала перпендикулярно плоскости зеркала.
}
\answer{%
    $10\,\frac{\text{см}}{\text{с}}$
}
\solutionspace{80pt}

\tasknumber{6}%
\task{%
    Сколько изображений получится от предмета в двух плоских зеркалах,
    поставленных под углом $90\degrees$ друг к другу?
}
\answer{%
    $3$
}
\solutionspace{80pt}

\tasknumber{7}%
\task{%
    Два плоских зеркала располагаются под углом друг к другу
    и между ними помещается точечный источник света.
    Расстояние от этого источника до одного зеркала $4\,\text{см}$, до другого $8\,\text{см}$.
    Расстояние между первыми изображениями в зеркалах $22{,}38\,\text{см}$.
    Найдите угол (в градусах) между зеркалами.
}
\answer{%
    $\cos \alpha = \frac{c^2 - \sqr{2a} - \sqr{2b}}{2 \cdot 2a \cdot 2b} \approx 0{,}707 \implies \alpha = 45{,}0\degrees$
}

\variantsplitter

\addpersonalvariant{Андрей Щербаков}

\tasknumber{1}%
\task{%
    Параллельный пучок света распространяется горизонтально.
    Под каким углом (в градусах) к горизонту следует расположить плоское зеркало,
    чтобы отраженный пучок распространялся вертикально?
}
\answer{%
    $4\,\frac{\text{см}}{\text{с}}$
}
\solutionspace{80pt}

\tasknumber{2}%
\task{%
    Под каким углом (в градусах) к горизонту следует расположить плоское зеркало,
    чтобы осветить дно вертикального колодца отраженными от зеркала солнечными лучами,
    падающими под углом $38\degrees$ к вертикали?
}
\answer{%
    $\alpha = 52\degrees, (90\degrees - \beta + \alpha) + (90\degrees - \beta) = 90\degrees \implies \beta = 45\degrees + \frac \alpha 2 = 71\degrees$
}
\solutionspace{80pt}

\tasknumber{3}%
\task{%
    Залина стоит перед плоским зеркалом, укрепленным на вертикальной стене.
    Какова должна быть минимальная высота зеркала, чтобы Залина мог видеть себя в полный рост?
    Рост Залина $190\,\text{см}$.
    Определите также расстояние от пола до нижнего края зеркала,
    приняв высоту головы равной $32\,\text{см}$, и считая, что глаза находятся посередине (по высоте) головы.
}
\answer{%
    $95\,\text{см}, 87\,\text{см}$
}
\solutionspace{80pt}

\tasknumber{4}%
\task{%
    Во сколько раз увеличится расстояние между предметом и его изображением
    в плоском зеркале, если зеркало переместить в то место, где было изображение? Предмет остаётся неподвижным.
}
\answer{%
    $2$
}
\solutionspace{80pt}

\tasknumber{5}%
\task{%
    Плоское зеркало движется по направлению к точечному источнику света со скоростью $15\,\frac{\text{см}}{\text{с}}$.
    Определите скорость движения изображения относительно зеркала.
    Направление скорости зеркала перпендикулярно плоскости зеркала.
}
\answer{%
    $15\,\frac{\text{см}}{\text{с}}$
}
\solutionspace{80pt}

\tasknumber{6}%
\task{%
    Сколько изображений получится от предмета в двух плоских зеркалах,
    поставленных под углом $30\degrees$ друг к другу?
}
\answer{%
    $11$
}
\solutionspace{80pt}

\tasknumber{7}%
\task{%
    Два плоских зеркала располагаются под углом друг к другу
    и между ними помещается точечный источник света.
    Расстояние от этого источника до одного зеркала $4\,\text{см}$, до другого $9\,\text{см}$.
    Расстояние между первыми изображениями в зеркалах $25{,}25\,\text{см}$.
    Найдите угол (в градусах) между зеркалами.
}
\answer{%
    $\cos \alpha = \frac{c^2 - \sqr{2a} - \sqr{2b}}{2 \cdot 2a \cdot 2b} \approx 0{,}867 \implies \alpha = 29{,}9\degrees$
}

\variantsplitter

\addpersonalvariant{Михаил Ярошевский}

\tasknumber{1}%
\task{%
    Параллельный пучок света распространяется горизонтально.
    Под каким углом (в градусах) к горизонту следует расположить плоское зеркало,
    чтобы отраженный пучок распространялся вертикально?
}
\answer{%
    $4\,\frac{\text{см}}{\text{с}}$
}
\solutionspace{80pt}

\tasknumber{2}%
\task{%
    Под каким углом (в градусах) к горизонту следует расположить плоское зеркало,
    чтобы осветить дно вертикального колодца отраженными от зеркала солнечными лучами,
    падающими под углом $46\degrees$ к вертикали?
}
\answer{%
    $\alpha = 44\degrees, (90\degrees - \beta + \alpha) + (90\degrees - \beta) = 90\degrees \implies \beta = 45\degrees + \frac \alpha 2 = 67\degrees$
}
\solutionspace{80pt}

\tasknumber{3}%
\task{%
    Савелий стоит перед плоским зеркалом, укрепленным на вертикальной стене.
    Какова должна быть минимальная высота зеркала, чтобы Савелий мог видеть себя в полный рост?
    Рост Савелия $156\,\text{см}$.
    Определите также расстояние от пола до нижнего края зеркала,
    приняв высоту головы равной $34\,\text{см}$, и считая, что глаза находятся посередине (по высоте) головы.
}
\answer{%
    $78\,\text{см}, 69{,}500\,\text{см}$
}
\solutionspace{80pt}

\tasknumber{4}%
\task{%
    Во сколько раз увеличится расстояние между предметом и его изображением
    в плоском зеркале, если зеркало переместить в то место, где было изображение? Предмет остаётся неподвижным.
}
\answer{%
    $2$
}
\solutionspace{80pt}

\tasknumber{5}%
\task{%
    Плоское зеркало движется по направлению к точечному источнику света со скоростью $12\,\frac{\text{см}}{\text{с}}$.
    Определите скорость движения изображения относительно зеркала.
    Направление скорости зеркала перпендикулярно плоскости зеркала.
}
\answer{%
    $12\,\frac{\text{см}}{\text{с}}$
}
\solutionspace{80pt}

\tasknumber{6}%
\task{%
    Сколько изображений получится от предмета в двух плоских зеркалах,
    поставленных под углом $45\degrees$ друг к другу?
}
\answer{%
    $7$
}
\solutionspace{80pt}

\tasknumber{7}%
\task{%
    Два плоских зеркала располагаются под углом друг к другу
    и между ними помещается точечный источник света.
    Расстояние от этого источника до одного зеркала $4\,\text{см}$, до другого $8\,\text{см}$.
    Расстояние между первыми изображениями в зеркалах $23{,}27\,\text{см}$.
    Найдите угол (в градусах) между зеркалами.
}
\answer{%
    $\cos \alpha = \frac{c^2 - \sqr{2a} - \sqr{2b}}{2 \cdot 2a \cdot 2b} \approx 0{,}865 \implies \alpha = 30{,}1\degrees$
}

\variantsplitter

\addpersonalvariant{Алексей Алимпиев}

\tasknumber{1}%
\task{%
    Параллельный пучок света распространяется горизонтально.
    Под каким углом (в градусах) к горизонту следует расположить плоское зеркало,
    чтобы отраженный пучок распространялся вертикально?
}
\answer{%
    $4\,\frac{\text{см}}{\text{с}}$
}
\solutionspace{80pt}

\tasknumber{2}%
\task{%
    Под каким углом (в градусах) к горизонту следует расположить плоское зеркало,
    чтобы осветить дно вертикального колодца отраженными от зеркала солнечными лучами,
    падающими под углом $60\degrees$ к вертикали?
}
\answer{%
    $\alpha = 30\degrees, (90\degrees - \beta + \alpha) + (90\degrees - \beta) = 90\degrees \implies \beta = 45\degrees + \frac \alpha 2 = 60\degrees$
}
\solutionspace{80pt}

\tasknumber{3}%
\task{%
    Камиля стоит перед плоским зеркалом, укрепленным на вертикальной стене.
    Какова должна быть минимальная высота зеркала, чтобы Камиля мог видеть себя в полный рост?
    Рост Камили $178\,\text{см}$.
    Определите также расстояние от пола до нижнего края зеркала,
    приняв высоту головы равной $34\,\text{см}$, и считая, что глаза находятся посередине (по высоте) головы.
}
\answer{%
    $89\,\text{см}, 80{,}500\,\text{см}$
}
\solutionspace{80pt}

\tasknumber{4}%
\task{%
    Во сколько раз увеличится расстояние между предметом и его изображением
    в плоском зеркале, если зеркало переместить в то место, где было изображение? Предмет остаётся неподвижным.
}
\answer{%
    $2$
}
\solutionspace{80pt}

\tasknumber{5}%
\task{%
    Плоское зеркало движется по направлению к точечному источнику света со скоростью $10\,\frac{\text{см}}{\text{с}}$.
    Определите скорость движения изображения относительно зеркала.
    Направление скорости зеркала перпендикулярно плоскости зеркала.
}
\answer{%
    $10\,\frac{\text{см}}{\text{с}}$
}
\solutionspace{80pt}

\tasknumber{6}%
\task{%
    Сколько изображений получится от предмета в двух плоских зеркалах,
    поставленных под углом $90\degrees$ друг к другу?
}
\answer{%
    $3$
}
\solutionspace{80pt}

\tasknumber{7}%
\task{%
    Два плоских зеркала располагаются под углом друг к другу
    и между ними помещается точечный источник света.
    Расстояние от этого источника до одного зеркала $5\,\text{см}$, до другого $6\,\text{см}$.
    Расстояние между первыми изображениями в зеркалах $19{,}08\,\text{см}$.
    Найдите угол (в градусах) между зеркалами.
}
\answer{%
    $\cos \alpha = \frac{c^2 - \sqr{2a} - \sqr{2b}}{2 \cdot 2a \cdot 2b} \approx 0{,}500 \implies \alpha = 60{,}0\degrees$
}

\variantsplitter

\addpersonalvariant{Евгений Васин}

\tasknumber{1}%
\task{%
    Параллельный пучок света распространяется горизонтально.
    Под каким углом (в градусах) к горизонту следует расположить плоское зеркало,
    чтобы отраженный пучок распространялся вертикально?
}
\answer{%
    $4\,\frac{\text{см}}{\text{с}}$
}
\solutionspace{80pt}

\tasknumber{2}%
\task{%
    Под каким углом (в градусах) к горизонту следует расположить плоское зеркало,
    чтобы осветить дно вертикального колодца отраженными от зеркала солнечными лучами,
    падающими под углом $38\degrees$ к вертикали?
}
\answer{%
    $\alpha = 52\degrees, (90\degrees - \beta + \alpha) + (90\degrees - \beta) = 90\degrees \implies \beta = 45\degrees + \frac \alpha 2 = 71\degrees$
}
\solutionspace{80pt}

\tasknumber{3}%
\task{%
    Лиана стоит перед плоским зеркалом, укрепленным на вертикальной стене.
    Какова должна быть минимальная высота зеркала, чтобы Лиана мог видеть себя в полный рост?
    Рост Лианы $168\,\text{см}$.
    Определите также расстояние от пола до нижнего края зеркала,
    приняв высоту головы равной $34\,\text{см}$, и считая, что глаза находятся посередине (по высоте) головы.
}
\answer{%
    $84\,\text{см}, 75{,}500\,\text{см}$
}
\solutionspace{80pt}

\tasknumber{4}%
\task{%
    Во сколько раз увеличится расстояние между предметом и его изображением
    в плоском зеркале, если зеркало переместить в то место, где было изображение? Предмет остаётся неподвижным.
}
\answer{%
    $2$
}
\solutionspace{80pt}

\tasknumber{5}%
\task{%
    Плоское зеркало движется по направлению к точечному источнику света со скоростью $15\,\frac{\text{см}}{\text{с}}$.
    Определите скорость движения изображения относительно источника света.
    Направление скорости зеркала перпендикулярно плоскости зеркала.
}
\answer{%
    $30\,\frac{\text{см}}{\text{с}}$
}
\solutionspace{80pt}

\tasknumber{6}%
\task{%
    Сколько изображений получится от предмета в двух плоских зеркалах,
    поставленных под углом $90\degrees$ друг к другу?
}
\answer{%
    $3$
}
\solutionspace{80pt}

\tasknumber{7}%
\task{%
    Два плоских зеркала располагаются под углом друг к другу
    и между ними помещается точечный источник света.
    Расстояние от этого источника до одного зеркала $5\,\text{см}$, до другого $8\,\text{см}$.
    Расстояние между первыми изображениями в зеркалах $22{,}72\,\text{см}$.
    Найдите угол (в градусах) между зеркалами.
}
\answer{%
    $\cos \alpha = \frac{c^2 - \sqr{2a} - \sqr{2b}}{2 \cdot 2a \cdot 2b} \approx 0{,}501 \implies \alpha = 60{,}0\degrees$
}

\variantsplitter

\addpersonalvariant{Вячеслав Волохов}

\tasknumber{1}%
\task{%
    Параллельный пучок света распространяется горизонтально.
    Под каким углом (в градусах) к горизонту следует расположить плоское зеркало,
    чтобы отраженный пучок распространялся вертикально?
}
\answer{%
    $3\,\frac{\text{см}}{\text{с}}$
}
\solutionspace{80pt}

\tasknumber{2}%
\task{%
    Под каким углом (в градусах) к горизонту следует расположить плоское зеркало,
    чтобы осветить дно вертикального колодца отраженными от зеркала солнечными лучами,
    падающими под углом $28\degrees$ к горизонту?
}
\answer{%
    $\alpha = 28\degrees, (90\degrees - \beta + \alpha) + (90\degrees - \beta) = 90\degrees \implies \beta = 45\degrees + \frac \alpha 2 = 59\degrees$
}
\solutionspace{80pt}

\tasknumber{3}%
\task{%
    Лиана стоит перед плоским зеркалом, укрепленным на вертикальной стене.
    Какова должна быть минимальная высота зеркала, чтобы Лиана мог видеть себя в полный рост?
    Рост Лианы $188\,\text{см}$.
    Определите также расстояние от пола до нижнего края зеркала,
    приняв высоту головы равной $30\,\text{см}$, и считая, что глаза находятся посередине (по высоте) головы.
}
\answer{%
    $94\,\text{см}, 86{,}500\,\text{см}$
}
\solutionspace{80pt}

\tasknumber{4}%
\task{%
    Во сколько раз увеличится расстояние между предметом и его изображением
    в плоском зеркале, если зеркало переместить в то место, где было изображение? Предмет остаётся неподвижным.
}
\answer{%
    $2$
}
\solutionspace{80pt}

\tasknumber{5}%
\task{%
    Плоское зеркало движется по направлению к точечному источнику света со скоростью $15\,\frac{\text{см}}{\text{с}}$.
    Определите скорость движения изображения относительно зеркала.
    Направление скорости зеркала перпендикулярно плоскости зеркала.
}
\answer{%
    $15\,\frac{\text{см}}{\text{с}}$
}
\solutionspace{80pt}

\tasknumber{6}%
\task{%
    Сколько изображений получится от предмета в двух плоских зеркалах,
    поставленных под углом $30\degrees$ друг к другу?
}
\answer{%
    $11$
}
\solutionspace{80pt}

\tasknumber{7}%
\task{%
    Два плоских зеркала располагаются под углом друг к другу
    и между ними помещается точечный источник света.
    Расстояние от этого источника до одного зеркала $4\,\text{см}$, до другого $6\,\text{см}$.
    Расстояние между первыми изображениями в зеркалах $18{,}54\,\text{см}$.
    Найдите угол (в градусах) между зеркалами.
}
\answer{%
    $\cos \alpha = \frac{c^2 - \sqr{2a} - \sqr{2b}}{2 \cdot 2a \cdot 2b} \approx 0{,}707 \implies \alpha = 45{,}0\degrees$
}

\variantsplitter

\addpersonalvariant{Герман Говоров}

\tasknumber{1}%
\task{%
    Параллельный пучок света распространяется горизонтально.
    Под каким углом (в градусах) к горизонту следует расположить плоское зеркало,
    чтобы отраженный пучок распространялся вертикально?
}
\answer{%
    $3\,\frac{\text{см}}{\text{с}}$
}
\solutionspace{80pt}

\tasknumber{2}%
\task{%
    Под каким углом (в градусах) к горизонту следует расположить плоское зеркало,
    чтобы осветить дно вертикального колодца отраженными от зеркала солнечными лучами,
    падающими под углом $34\degrees$ к вертикали?
}
\answer{%
    $\alpha = 56\degrees, (90\degrees - \beta + \alpha) + (90\degrees - \beta) = 90\degrees \implies \beta = 45\degrees + \frac \alpha 2 = 73\degrees$
}
\solutionspace{80pt}

\tasknumber{3}%
\task{%
    Залина стоит перед плоским зеркалом, укрепленным на вертикальной стене.
    Какова должна быть минимальная высота зеркала, чтобы Залина мог видеть себя в полный рост?
    Рост Залина $164\,\text{см}$.
    Определите также расстояние от пола до нижнего края зеркала,
    приняв высоту головы равной $30\,\text{см}$, и считая, что глаза находятся посередине (по высоте) головы.
}
\answer{%
    $82\,\text{см}, 74{,}500\,\text{см}$
}
\solutionspace{80pt}

\tasknumber{4}%
\task{%
    Во сколько раз увеличится расстояние между предметом и его изображением
    в плоском зеркале, если зеркало переместить в то место, где было изображение? Предмет остаётся неподвижным.
}
\answer{%
    $2$
}
\solutionspace{80pt}

\tasknumber{5}%
\task{%
    Плоское зеркало движется по направлению к точечному источнику света со скоростью $18\,\frac{\text{см}}{\text{с}}$.
    Определите скорость движения изображения относительно зеркала.
    Направление скорости зеркала перпендикулярно плоскости зеркала.
}
\answer{%
    $18\,\frac{\text{см}}{\text{с}}$
}
\solutionspace{80pt}

\tasknumber{6}%
\task{%
    Сколько изображений получится от предмета в двух плоских зеркалах,
    поставленных под углом $60\degrees$ друг к другу?
}
\answer{%
    $5$
}
\solutionspace{80pt}

\tasknumber{7}%
\task{%
    Два плоских зеркала располагаются под углом друг к другу
    и между ними помещается точечный источник света.
    Расстояние от этого источника до одного зеркала $5\,\text{см}$, до другого $8\,\text{см}$.
    Расстояние между первыми изображениями в зеркалах $22{,}72\,\text{см}$.
    Найдите угол (в градусах) между зеркалами.
}
\answer{%
    $\cos \alpha = \frac{c^2 - \sqr{2a} - \sqr{2b}}{2 \cdot 2a \cdot 2b} \approx 0{,}501 \implies \alpha = 60{,}0\degrees$
}

\variantsplitter

\addpersonalvariant{София Журавлёва}

\tasknumber{1}%
\task{%
    Параллельный пучок света распространяется горизонтально.
    Под каким углом (в градусах) к горизонту следует расположить плоское зеркало,
    чтобы отраженный пучок распространялся вертикально?
}
\answer{%
    $2\,\frac{\text{см}}{\text{с}}$
}
\solutionspace{80pt}

\tasknumber{2}%
\task{%
    Под каким углом (в градусах) к горизонту следует расположить плоское зеркало,
    чтобы осветить дно вертикального колодца отраженными от зеркала солнечными лучами,
    падающими под углом $68\degrees$ к горизонту?
}
\answer{%
    $\alpha = 68\degrees, (90\degrees - \beta + \alpha) + (90\degrees - \beta) = 90\degrees \implies \beta = 45\degrees + \frac \alpha 2 = 79\degrees$
}
\solutionspace{80pt}

\tasknumber{3}%
\task{%
    Ираклий стоит перед плоским зеркалом, укрепленным на вертикальной стене.
    Какова должна быть минимальная высота зеркала, чтобы Ираклий мог видеть себя в полный рост?
    Рост Ираклия $168\,\text{см}$.
    Определите также расстояние от пола до нижнего края зеркала,
    приняв высоту головы равной $32\,\text{см}$, и считая, что глаза находятся посередине (по высоте) головы.
}
\answer{%
    $84\,\text{см}, 76\,\text{см}$
}
\solutionspace{80pt}

\tasknumber{4}%
\task{%
    Во сколько раз увеличится расстояние между предметом и его изображением
    в плоском зеркале, если зеркало переместить в то место, где было изображение? Предмет остаётся неподвижным.
}
\answer{%
    $2$
}
\solutionspace{80pt}

\tasknumber{5}%
\task{%
    Плоское зеркало движется по направлению к точечному источнику света со скоростью $18\,\frac{\text{см}}{\text{с}}$.
    Определите скорость движения изображения относительно источника света.
    Направление скорости зеркала перпендикулярно плоскости зеркала.
}
\answer{%
    $36\,\frac{\text{см}}{\text{с}}$
}
\solutionspace{80pt}

\tasknumber{6}%
\task{%
    Сколько изображений получится от предмета в двух плоских зеркалах,
    поставленных под углом $90\degrees$ друг к другу?
}
\answer{%
    $3$
}
\solutionspace{80pt}

\tasknumber{7}%
\task{%
    Два плоских зеркала располагаются под углом друг к другу
    и между ними помещается точечный источник света.
    Расстояние от этого источника до одного зеркала $3\,\text{см}$, до другого $7\,\text{см}$.
    Расстояние между первыми изображениями в зеркалах $17{,}78\,\text{см}$.
    Найдите угол (в градусах) между зеркалами.
}
\answer{%
    $\cos \alpha = \frac{c^2 - \sqr{2a} - \sqr{2b}}{2 \cdot 2a \cdot 2b} \approx 0{,}501 \implies \alpha = 59{,}9\degrees$
}

\variantsplitter

\addpersonalvariant{Константин Козлов}

\tasknumber{1}%
\task{%
    Параллельный пучок света распространяется горизонтально.
    Под каким углом (в градусах) к горизонту следует расположить плоское зеркало,
    чтобы отраженный пучок распространялся вертикально?
}
\answer{%
    $2\,\frac{\text{см}}{\text{с}}$
}
\solutionspace{80pt}

\tasknumber{2}%
\task{%
    Под каким углом (в градусах) к горизонту следует расположить плоское зеркало,
    чтобы осветить дно вертикального колодца отраженными от зеркала солнечными лучами,
    падающими под углом $42\degrees$ к горизонту?
}
\answer{%
    $\alpha = 42\degrees, (90\degrees - \beta + \alpha) + (90\degrees - \beta) = 90\degrees \implies \beta = 45\degrees + \frac \alpha 2 = 66\degrees$
}
\solutionspace{80pt}

\tasknumber{3}%
\task{%
    Ульяна стоит перед плоским зеркалом, укрепленным на вертикальной стене.
    Какова должна быть минимальная высота зеркала, чтобы Ульяна мог видеть себя в полный рост?
    Рост Ульяны $160\,\text{см}$.
    Определите также расстояние от пола до нижнего края зеркала,
    приняв высоту головы равной $34\,\text{см}$, и считая, что глаза находятся посередине (по высоте) головы.
}
\answer{%
    $80\,\text{см}, 71{,}500\,\text{см}$
}
\solutionspace{80pt}

\tasknumber{4}%
\task{%
    Во сколько раз увеличится расстояние между предметом и его изображением
    в плоском зеркале, если зеркало переместить в то место, где было изображение? Предмет остаётся неподвижным.
}
\answer{%
    $2$
}
\solutionspace{80pt}

\tasknumber{5}%
\task{%
    Плоское зеркало движется по направлению к точечному источнику света со скоростью $12\,\frac{\text{см}}{\text{с}}$.
    Определите скорость движения изображения относительно зеркала.
    Направление скорости зеркала перпендикулярно плоскости зеркала.
}
\answer{%
    $12\,\frac{\text{см}}{\text{с}}$
}
\solutionspace{80pt}

\tasknumber{6}%
\task{%
    Сколько изображений получится от предмета в двух плоских зеркалах,
    поставленных под углом $30\degrees$ друг к другу?
}
\answer{%
    $11$
}
\solutionspace{80pt}

\tasknumber{7}%
\task{%
    Два плоских зеркала располагаются под углом друг к другу
    и между ними помещается точечный источник света.
    Расстояние от этого источника до одного зеркала $3\,\text{см}$, до другого $7\,\text{см}$.
    Расстояние между первыми изображениями в зеркалах $19{,}43\,\text{см}$.
    Найдите угол (в градусах) между зеркалами.
}
\answer{%
    $\cos \alpha = \frac{c^2 - \sqr{2a} - \sqr{2b}}{2 \cdot 2a \cdot 2b} \approx 0{,}866 \implies \alpha = 30{,}0\degrees$
}

\variantsplitter

\addpersonalvariant{Наталья Кравченко}

\tasknumber{1}%
\task{%
    Параллельный пучок света распространяется горизонтально.
    Под каким углом (в градусах) к горизонту следует расположить плоское зеркало,
    чтобы отраженный пучок распространялся вертикально?
}
\answer{%
    $2\,\frac{\text{см}}{\text{с}}$
}
\solutionspace{80pt}

\tasknumber{2}%
\task{%
    Под каким углом (в градусах) к горизонту следует расположить плоское зеркало,
    чтобы осветить дно вертикального колодца отраженными от зеркала солнечными лучами,
    падающими под углом $64\degrees$ к горизонту?
}
\answer{%
    $\alpha = 64\degrees, (90\degrees - \beta + \alpha) + (90\degrees - \beta) = 90\degrees \implies \beta = 45\degrees + \frac \alpha 2 = 77\degrees$
}
\solutionspace{80pt}

\tasknumber{3}%
\task{%
    Лиана стоит перед плоским зеркалом, укрепленным на вертикальной стене.
    Какова должна быть минимальная высота зеркала, чтобы Лиана мог видеть себя в полный рост?
    Рост Лианы $164\,\text{см}$.
    Определите также расстояние от пола до нижнего края зеркала,
    приняв высоту головы равной $32\,\text{см}$, и считая, что глаза находятся посередине (по высоте) головы.
}
\answer{%
    $82\,\text{см}, 74\,\text{см}$
}
\solutionspace{80pt}

\tasknumber{4}%
\task{%
    Во сколько раз увеличится расстояние между предметом и его изображением
    в плоском зеркале, если зеркало переместить в то место, где было изображение? Предмет остаётся неподвижным.
}
\answer{%
    $2$
}
\solutionspace{80pt}

\tasknumber{5}%
\task{%
    Плоское зеркало движется по направлению к точечному источнику света со скоростью $12\,\frac{\text{см}}{\text{с}}$.
    Определите скорость движения изображения относительно зеркала.
    Направление скорости зеркала перпендикулярно плоскости зеркала.
}
\answer{%
    $12\,\frac{\text{см}}{\text{с}}$
}
\solutionspace{80pt}

\tasknumber{6}%
\task{%
    Сколько изображений получится от предмета в двух плоских зеркалах,
    поставленных под углом $60\degrees$ друг к другу?
}
\answer{%
    $5$
}
\solutionspace{80pt}

\tasknumber{7}%
\task{%
    Два плоских зеркала располагаются под углом друг к другу
    и между ними помещается точечный источник света.
    Расстояние от этого источника до одного зеркала $5\,\text{см}$, до другого $9\,\text{см}$.
    Расстояние между первыми изображениями в зеркалах $24{,}58\,\text{см}$.
    Найдите угол (в градусах) между зеркалами.
}
\answer{%
    $\cos \alpha = \frac{c^2 - \sqr{2a} - \sqr{2b}}{2 \cdot 2a \cdot 2b} \approx 0{,}500 \implies \alpha = 60{,}0\degrees$
}

\variantsplitter

\addpersonalvariant{Матвей Кузьмин}

\tasknumber{1}%
\task{%
    Параллельный пучок света распространяется горизонтально.
    Под каким углом (в градусах) к горизонту следует расположить плоское зеркало,
    чтобы отраженный пучок распространялся вертикально?
}
\answer{%
    $2\,\frac{\text{см}}{\text{с}}$
}
\solutionspace{80pt}

\tasknumber{2}%
\task{%
    Под каким углом (в градусах) к горизонту следует расположить плоское зеркало,
    чтобы осветить дно вертикального колодца отраженными от зеркала солнечными лучами,
    падающими под углом $54\degrees$ к горизонту?
}
\answer{%
    $\alpha = 54\degrees, (90\degrees - \beta + \alpha) + (90\degrees - \beta) = 90\degrees \implies \beta = 45\degrees + \frac \alpha 2 = 72\degrees$
}
\solutionspace{80pt}

\tasknumber{3}%
\task{%
    Тимур стоит перед плоским зеркалом, укрепленным на вертикальной стене.
    Какова должна быть минимальная высота зеркала, чтобы Тимур мог видеть себя в полный рост?
    Рост Тимур $178\,\text{см}$.
    Определите также расстояние от пола до нижнего края зеркала,
    приняв высоту головы равной $34\,\text{см}$, и считая, что глаза находятся посередине (по высоте) головы.
}
\answer{%
    $89\,\text{см}, 80{,}500\,\text{см}$
}
\solutionspace{80pt}

\tasknumber{4}%
\task{%
    Во сколько раз увеличится расстояние между предметом и его изображением
    в плоском зеркале, если зеркало переместить в то место, где было изображение? Предмет остаётся неподвижным.
}
\answer{%
    $2$
}
\solutionspace{80pt}

\tasknumber{5}%
\task{%
    Плоское зеркало движется по направлению к точечному источнику света со скоростью $12\,\frac{\text{см}}{\text{с}}$.
    Определите скорость движения изображения относительно источника света.
    Направление скорости зеркала перпендикулярно плоскости зеркала.
}
\answer{%
    $24\,\frac{\text{см}}{\text{с}}$
}
\solutionspace{80pt}

\tasknumber{6}%
\task{%
    Сколько изображений получится от предмета в двух плоских зеркалах,
    поставленных под углом $45\degrees$ друг к другу?
}
\answer{%
    $7$
}
\solutionspace{80pt}

\tasknumber{7}%
\task{%
    Два плоских зеркала располагаются под углом друг к другу
    и между ними помещается точечный источник света.
    Расстояние от этого источника до одного зеркала $4\,\text{см}$, до другого $6\,\text{см}$.
    Расстояние между первыми изображениями в зеркалах $19{,}35\,\text{см}$.
    Найдите угол (в градусах) между зеркалами.
}
\answer{%
    $\cos \alpha = \frac{c^2 - \sqr{2a} - \sqr{2b}}{2 \cdot 2a \cdot 2b} \approx 0{,}867 \implies \alpha = 29{,}9\degrees$
}

\variantsplitter

\addpersonalvariant{Сергей Малышев}

\tasknumber{1}%
\task{%
    Параллельный пучок света распространяется горизонтально.
    Под каким углом (в градусах) к горизонту следует расположить плоское зеркало,
    чтобы отраженный пучок распространялся вертикально?
}
\answer{%
    $4\,\frac{\text{см}}{\text{с}}$
}
\solutionspace{80pt}

\tasknumber{2}%
\task{%
    Под каким углом (в градусах) к горизонту следует расположить плоское зеркало,
    чтобы осветить дно вертикального колодца отраженными от зеркала солнечными лучами,
    падающими под углом $66\degrees$ к вертикали?
}
\answer{%
    $\alpha = 24\degrees, (90\degrees - \beta + \alpha) + (90\degrees - \beta) = 90\degrees \implies \beta = 45\degrees + \frac \alpha 2 = 57\degrees$
}
\solutionspace{80pt}

\tasknumber{3}%
\task{%
    Жанна стоит перед плоским зеркалом, укрепленным на вертикальной стене.
    Какова должна быть минимальная высота зеркала, чтобы Жанна мог видеть себя в полный рост?
    Рост Жанны $176\,\text{см}$.
    Определите также расстояние от пола до нижнего края зеркала,
    приняв высоту головы равной $34\,\text{см}$, и считая, что глаза находятся посередине (по высоте) головы.
}
\answer{%
    $88\,\text{см}, 79{,}500\,\text{см}$
}
\solutionspace{80pt}

\tasknumber{4}%
\task{%
    Во сколько раз увеличится расстояние между предметом и его изображением
    в плоском зеркале, если зеркало переместить в то место, где было изображение? Предмет остаётся неподвижным.
}
\answer{%
    $2$
}
\solutionspace{80pt}

\tasknumber{5}%
\task{%
    Плоское зеркало движется по направлению к точечному источнику света со скоростью $20\,\frac{\text{см}}{\text{с}}$.
    Определите скорость движения изображения относительно зеркала.
    Направление скорости зеркала перпендикулярно плоскости зеркала.
}
\answer{%
    $20\,\frac{\text{см}}{\text{с}}$
}
\solutionspace{80pt}

\tasknumber{6}%
\task{%
    Сколько изображений получится от предмета в двух плоских зеркалах,
    поставленных под углом $45\degrees$ друг к другу?
}
\answer{%
    $7$
}
\solutionspace{80pt}

\tasknumber{7}%
\task{%
    Два плоских зеркала располагаются под углом друг к другу
    и между ними помещается точечный источник света.
    Расстояние от этого источника до одного зеркала $4\,\text{см}$, до другого $8\,\text{см}$.
    Расстояние между первыми изображениями в зеркалах $23{,}27\,\text{см}$.
    Найдите угол (в градусах) между зеркалами.
}
\answer{%
    $\cos \alpha = \frac{c^2 - \sqr{2a} - \sqr{2b}}{2 \cdot 2a \cdot 2b} \approx 0{,}865 \implies \alpha = 30{,}1\degrees$
}

\variantsplitter

\addpersonalvariant{Алина Полканова}

\tasknumber{1}%
\task{%
    Параллельный пучок света распространяется горизонтально.
    Под каким углом (в градусах) к горизонту следует расположить плоское зеркало,
    чтобы отраженный пучок распространялся вертикально?
}
\answer{%
    $2\,\frac{\text{см}}{\text{с}}$
}
\solutionspace{80pt}

\tasknumber{2}%
\task{%
    Под каким углом (в градусах) к горизонту следует расположить плоское зеркало,
    чтобы осветить дно вертикального колодца отраженными от зеркала солнечными лучами,
    падающими под углом $50\degrees$ к вертикали?
}
\answer{%
    $\alpha = 40\degrees, (90\degrees - \beta + \alpha) + (90\degrees - \beta) = 90\degrees \implies \beta = 45\degrees + \frac \alpha 2 = 65\degrees$
}
\solutionspace{80pt}

\tasknumber{3}%
\task{%
    Залина стоит перед плоским зеркалом, укрепленным на вертикальной стене.
    Какова должна быть минимальная высота зеркала, чтобы Залина мог видеть себя в полный рост?
    Рост Залина $182\,\text{см}$.
    Определите также расстояние от пола до нижнего края зеркала,
    приняв высоту головы равной $34\,\text{см}$, и считая, что глаза находятся посередине (по высоте) головы.
}
\answer{%
    $91\,\text{см}, 82{,}500\,\text{см}$
}
\solutionspace{80pt}

\tasknumber{4}%
\task{%
    Во сколько раз увеличится расстояние между предметом и его изображением
    в плоском зеркале, если зеркало переместить в то место, где было изображение? Предмет остаётся неподвижным.
}
\answer{%
    $2$
}
\solutionspace{80pt}

\tasknumber{5}%
\task{%
    Плоское зеркало движется по направлению к точечному источнику света со скоростью $20\,\frac{\text{см}}{\text{с}}$.
    Определите скорость движения изображения относительно зеркала.
    Направление скорости зеркала перпендикулярно плоскости зеркала.
}
\answer{%
    $20\,\frac{\text{см}}{\text{с}}$
}
\solutionspace{80pt}

\tasknumber{6}%
\task{%
    Сколько изображений получится от предмета в двух плоских зеркалах,
    поставленных под углом $30\degrees$ друг к другу?
}
\answer{%
    $11$
}
\solutionspace{80pt}

\tasknumber{7}%
\task{%
    Два плоских зеркала располагаются под углом друг к другу
    и между ними помещается точечный источник света.
    Расстояние от этого источника до одного зеркала $5\,\text{см}$, до другого $6\,\text{см}$.
    Расстояние между первыми изображениями в зеркалах $21{,}26\,\text{см}$.
    Найдите угол (в градусах) между зеркалами.
}
\answer{%
    $\cos \alpha = \frac{c^2 - \sqr{2a} - \sqr{2b}}{2 \cdot 2a \cdot 2b} \approx 0{,}867 \implies \alpha = 29{,}9\degrees$
}

\variantsplitter

\addpersonalvariant{Сергей Пономарёв}

\tasknumber{1}%
\task{%
    Параллельный пучок света распространяется горизонтально.
    Под каким углом (в градусах) к горизонту следует расположить плоское зеркало,
    чтобы отраженный пучок распространялся вертикально?
}
\answer{%
    $4\,\frac{\text{см}}{\text{с}}$
}
\solutionspace{80pt}

\tasknumber{2}%
\task{%
    Под каким углом (в градусах) к горизонту следует расположить плоское зеркало,
    чтобы осветить дно вертикального колодца отраженными от зеркала солнечными лучами,
    падающими под углом $28\degrees$ к вертикали?
}
\answer{%
    $\alpha = 62\degrees, (90\degrees - \beta + \alpha) + (90\degrees - \beta) = 90\degrees \implies \beta = 45\degrees + \frac \alpha 2 = 76\degrees$
}
\solutionspace{80pt}

\tasknumber{3}%
\task{%
    Женя стоит перед плоским зеркалом, укрепленным на вертикальной стене.
    Какова должна быть минимальная высота зеркала, чтобы Женя мог видеть себя в полный рост?
    Рост Жени $188\,\text{см}$.
    Определите также расстояние от пола до нижнего края зеркала,
    приняв высоту головы равной $34\,\text{см}$, и считая, что глаза находятся посередине (по высоте) головы.
}
\answer{%
    $94\,\text{см}, 85{,}500\,\text{см}$
}
\solutionspace{80pt}

\tasknumber{4}%
\task{%
    Во сколько раз увеличится расстояние между предметом и его изображением
    в плоском зеркале, если зеркало переместить в то место, где было изображение? Предмет остаётся неподвижным.
}
\answer{%
    $2$
}
\solutionspace{80pt}

\tasknumber{5}%
\task{%
    Плоское зеркало движется по направлению к точечному источнику света со скоростью $15\,\frac{\text{см}}{\text{с}}$.
    Определите скорость движения изображения относительно источника света.
    Направление скорости зеркала перпендикулярно плоскости зеркала.
}
\answer{%
    $30\,\frac{\text{см}}{\text{с}}$
}
\solutionspace{80pt}

\tasknumber{6}%
\task{%
    Сколько изображений получится от предмета в двух плоских зеркалах,
    поставленных под углом $60\degrees$ друг к другу?
}
\answer{%
    $5$
}
\solutionspace{80pt}

\tasknumber{7}%
\task{%
    Два плоских зеркала располагаются под углом друг к другу
    и между ними помещается точечный источник света.
    Расстояние от этого источника до одного зеркала $5\,\text{см}$, до другого $9\,\text{см}$.
    Расстояние между первыми изображениями в зеркалах $26{,}05\,\text{см}$.
    Найдите угол (в градусах) между зеркалами.
}
\answer{%
    $\cos \alpha = \frac{c^2 - \sqr{2a} - \sqr{2b}}{2 \cdot 2a \cdot 2b} \approx 0{,}707 \implies \alpha = 45{,}0\degrees$
}

\variantsplitter

\addpersonalvariant{Егор Свистушкин}

\tasknumber{1}%
\task{%
    Параллельный пучок света распространяется горизонтально.
    Под каким углом (в градусах) к горизонту следует расположить плоское зеркало,
    чтобы отраженный пучок распространялся вертикально?
}
\answer{%
    $3\,\frac{\text{см}}{\text{с}}$
}
\solutionspace{80pt}

\tasknumber{2}%
\task{%
    Под каким углом (в градусах) к горизонту следует расположить плоское зеркало,
    чтобы осветить дно вертикального колодца отраженными от зеркала солнечными лучами,
    падающими под углом $52\degrees$ к вертикали?
}
\answer{%
    $\alpha = 38\degrees, (90\degrees - \beta + \alpha) + (90\degrees - \beta) = 90\degrees \implies \beta = 45\degrees + \frac \alpha 2 = 64\degrees$
}
\solutionspace{80pt}

\tasknumber{3}%
\task{%
    Осип стоит перед плоским зеркалом, укрепленным на вертикальной стене.
    Какова должна быть минимальная высота зеркала, чтобы Осип мог видеть себя в полный рост?
    Рост Осипа $176\,\text{см}$.
    Определите также расстояние от пола до нижнего края зеркала,
    приняв высоту головы равной $34\,\text{см}$, и считая, что глаза находятся посередине (по высоте) головы.
}
\answer{%
    $88\,\text{см}, 79{,}500\,\text{см}$
}
\solutionspace{80pt}

\tasknumber{4}%
\task{%
    Во сколько раз увеличится расстояние между предметом и его изображением
    в плоском зеркале, если зеркало переместить в то место, где было изображение? Предмет остаётся неподвижным.
}
\answer{%
    $2$
}
\solutionspace{80pt}

\tasknumber{5}%
\task{%
    Плоское зеркало движется по направлению к точечному источнику света со скоростью $15\,\frac{\text{см}}{\text{с}}$.
    Определите скорость движения изображения относительно зеркала.
    Направление скорости зеркала перпендикулярно плоскости зеркала.
}
\answer{%
    $15\,\frac{\text{см}}{\text{с}}$
}
\solutionspace{80pt}

\tasknumber{6}%
\task{%
    Сколько изображений получится от предмета в двух плоских зеркалах,
    поставленных под углом $45\degrees$ друг к другу?
}
\answer{%
    $7$
}
\solutionspace{80pt}

\tasknumber{7}%
\task{%
    Два плоских зеркала располагаются под углом друг к другу
    и между ними помещается точечный источник света.
    Расстояние от этого источника до одного зеркала $5\,\text{см}$, до другого $8\,\text{см}$.
    Расстояние между первыми изображениями в зеркалах $25{,}16\,\text{см}$.
    Найдите угол (в градусах) между зеркалами.
}
\answer{%
    $\cos \alpha = \frac{c^2 - \sqr{2a} - \sqr{2b}}{2 \cdot 2a \cdot 2b} \approx 0{,}866 \implies \alpha = 30{,}0\degrees$
}

\variantsplitter

\addpersonalvariant{Дмитрий Соколов}

\tasknumber{1}%
\task{%
    Параллельный пучок света распространяется горизонтально.
    Под каким углом (в градусах) к горизонту следует расположить плоское зеркало,
    чтобы отраженный пучок распространялся вертикально?
}
\answer{%
    $4\,\frac{\text{см}}{\text{с}}$
}
\solutionspace{80pt}

\tasknumber{2}%
\task{%
    Под каким углом (в градусах) к горизонту следует расположить плоское зеркало,
    чтобы осветить дно вертикального колодца отраженными от зеркала солнечными лучами,
    падающими под углом $26\degrees$ к горизонту?
}
\answer{%
    $\alpha = 26\degrees, (90\degrees - \beta + \alpha) + (90\degrees - \beta) = 90\degrees \implies \beta = 45\degrees + \frac \alpha 2 = 58\degrees$
}
\solutionspace{80pt}

\tasknumber{3}%
\task{%
    Рипсиме стоит перед плоским зеркалом, укрепленным на вертикальной стене.
    Какова должна быть минимальная высота зеркала, чтобы Рипсиме мог видеть себя в полный рост?
    Рост Рипсиме $158\,\text{см}$.
    Определите также расстояние от пола до нижнего края зеркала,
    приняв высоту головы равной $34\,\text{см}$, и считая, что глаза находятся посередине (по высоте) головы.
}
\answer{%
    $79\,\text{см}, 70{,}500\,\text{см}$
}
\solutionspace{80pt}

\tasknumber{4}%
\task{%
    Во сколько раз увеличится расстояние между предметом и его изображением
    в плоском зеркале, если зеркало переместить в то место, где было изображение? Предмет остаётся неподвижным.
}
\answer{%
    $2$
}
\solutionspace{80pt}

\tasknumber{5}%
\task{%
    Плоское зеркало движется по направлению к точечному источнику света со скоростью $20\,\frac{\text{см}}{\text{с}}$.
    Определите скорость движения изображения относительно зеркала.
    Направление скорости зеркала перпендикулярно плоскости зеркала.
}
\answer{%
    $20\,\frac{\text{см}}{\text{с}}$
}
\solutionspace{80pt}

\tasknumber{6}%
\task{%
    Сколько изображений получится от предмета в двух плоских зеркалах,
    поставленных под углом $45\degrees$ друг к другу?
}
\answer{%
    $7$
}
\solutionspace{80pt}

\tasknumber{7}%
\task{%
    Два плоских зеркала располагаются под углом друг к другу
    и между ними помещается точечный источник света.
    Расстояние от этого источника до одного зеркала $4\,\text{см}$, до другого $8\,\text{см}$.
    Расстояние между первыми изображениями в зеркалах $22{,}38\,\text{см}$.
    Найдите угол (в градусах) между зеркалами.
}
\answer{%
    $\cos \alpha = \frac{c^2 - \sqr{2a} - \sqr{2b}}{2 \cdot 2a \cdot 2b} \approx 0{,}707 \implies \alpha = 45{,}0\degrees$
}

\variantsplitter

\addpersonalvariant{Арсений Трофимов}

\tasknumber{1}%
\task{%
    Параллельный пучок света распространяется горизонтально.
    Под каким углом (в градусах) к горизонту следует расположить плоское зеркало,
    чтобы отраженный пучок распространялся вертикально?
}
\answer{%
    $4\,\frac{\text{см}}{\text{с}}$
}
\solutionspace{80pt}

\tasknumber{2}%
\task{%
    Под каким углом (в градусах) к горизонту следует расположить плоское зеркало,
    чтобы осветить дно вертикального колодца отраженными от зеркала солнечными лучами,
    падающими под углом $34\degrees$ к горизонту?
}
\answer{%
    $\alpha = 34\degrees, (90\degrees - \beta + \alpha) + (90\degrees - \beta) = 90\degrees \implies \beta = 45\degrees + \frac \alpha 2 = 62\degrees$
}
\solutionspace{80pt}

\tasknumber{3}%
\task{%
    Осип стоит перед плоским зеркалом, укрепленным на вертикальной стене.
    Какова должна быть минимальная высота зеркала, чтобы Осип мог видеть себя в полный рост?
    Рост Осипа $180\,\text{см}$.
    Определите также расстояние от пола до нижнего края зеркала,
    приняв высоту головы равной $32\,\text{см}$, и считая, что глаза находятся посередине (по высоте) головы.
}
\answer{%
    $90\,\text{см}, 82\,\text{см}$
}
\solutionspace{80pt}

\tasknumber{4}%
\task{%
    Во сколько раз увеличится расстояние между предметом и его изображением
    в плоском зеркале, если зеркало переместить в то место, где было изображение? Предмет остаётся неподвижным.
}
\answer{%
    $2$
}
\solutionspace{80pt}

\tasknumber{5}%
\task{%
    Плоское зеркало движется по направлению к точечному источнику света со скоростью $20\,\frac{\text{см}}{\text{с}}$.
    Определите скорость движения изображения относительно источника света.
    Направление скорости зеркала перпендикулярно плоскости зеркала.
}
\answer{%
    $40\,\frac{\text{см}}{\text{с}}$
}
\solutionspace{80pt}

\tasknumber{6}%
\task{%
    Сколько изображений получится от предмета в двух плоских зеркалах,
    поставленных под углом $90\degrees$ друг к другу?
}
\answer{%
    $3$
}
\solutionspace{80pt}

\tasknumber{7}%
\task{%
    Два плоских зеркала располагаются под углом друг к другу
    и между ними помещается точечный источник света.
    Расстояние от этого источника до одного зеркала $5\,\text{см}$, до другого $7\,\text{см}$.
    Расстояние между первыми изображениями в зеркалах $20{,}88\,\text{см}$.
    Найдите угол (в градусах) между зеркалами.
}
\answer{%
    $\cos \alpha = \frac{c^2 - \sqr{2a} - \sqr{2b}}{2 \cdot 2a \cdot 2b} \approx 0{,}500 \implies \alpha = 60{,}0\degrees$
}
% autogenerated
