\setdate{16~декабря~2021}
\setclass{11«БА»}

\addpersonalvariant{Михаил Бурмистров}

\tasknumber{1}%
\task{%
    Для закона преломления:
    \begin{itemize}
        \item сделайте рисунок,
        \item отметьте все необходимые углы и подпишите их названия,
        \item запишите этот закон формулой.
    \end{itemize}
}
\solutionspace{80pt}

\tasknumber{2}%
\task{%
    На плоскопараллельную стеклянную пластинку под углом $50\degrees$
    падают два параллельных луча света, расстояние между которыми $6\,\text{см}$.
    Определите расстояние между точками, в которых эти лучи выходят из пластинки.
}
\answer{%
    $\ell = \frac{d}{\cos \alpha} \approx 93\,\text{мм}$
}
\solutionspace{80pt}

\tasknumber{3}%
\task{%
    Солнце составляет с горизонтом угол, синус которого  0{,}8 .
    Шест высотой $150\,\text{см}$ вбит в дно водоёма глубиной $80\,\text{см}$.
    Найдите длину тени от этого шеста на дне водоёма, если показатель преломления воды 1{,}33.
}
\answer{%
    $
        n \sin \beta = 1 \cdot \cos \alpha \implies \beta \approx 26{,}8\degrees,
        L = (H - h)\ctg \alpha + h \tg \beta \approx 52{,}5\,\text{см} + 40{,}4\,\text{см} \approx 92{,}9\,\text{см}.
    $
}
\solutionspace{80pt}

\tasknumber{4}%
\task{%
    Луч света падает на плоское зеркало под углом, синус которого  0{,}65 .
    На сколько миллиметров сместится отражённый луч,
    если на зеркало положить прозрачную пластину толщиной $18\,\text{мм}$ с показателем преломления $1{,}6$?
}
\answer{%
    $1 \cdot \sin \alpha = n \cdot \sin \beta \implies \beta \approx 24{,}0\degrees, L = \cbr{2 d \tg \alpha - 2 d \tg \beta} \cdot \cos \alpha \approx 11{,}2\,\text{мм}$
}
\solutionspace{80pt}

\tasknumber{5}%
\task{%
    В некотором прозрачном веществе свет распространяется со скоростью,
    втрое меньшей скорости света в вакууме.
    Чему будет равен предельный угол
    внутреннего отражения для поверхности раздела этого вещества с воздухом?
}
\answer{%
    $n_1 = \frac c v = 3, n_1 \sin \varphi_\text{п.в.о.} = n_2 \sin \frac \pi 2 \implies \varphi_\text{п.в.о.} \approx 19{,}5\degrees.$
}
\solutionspace{80pt}

\tasknumber{6}%
\task{%
    Луч света падает на вертикально расположенную стеклянную пластинку толщиной $1{,}3\,\text{см}$.
    Пройдя через пластину, он выходит из неё в точке, смещённой по вертикали от точки падения на расстояние $4\,\text{мм}$.
    Показатель преломления стекла $1{,}4$.
    Найти синус угла падения.
}
\answer{%
    \begin{align*}
    \ctg \beta &= \frac{h}{d} \implies \\
    \implies \frac 1{\sin^2 \beta} &= \ctg^2 \beta + 1 = \sqr{\frac{h}{d}} + 1 \implies \\
    \implies \sin\alpha &= n\sin \beta = n\sqrt {\frac 1{\sqr{\frac{h}{d}} + 1}} \approx 0{,}41
    \end{align*}
}

\variantsplitter

\addpersonalvariant{Ирина Ан}

\tasknumber{1}%
\task{%
    Для закона преломления:
    \begin{itemize}
        \item сделайте рисунок,
        \item отметьте все необходимые углы и подпишите их названия,
        \item запишите этот закон формулой.
    \end{itemize}
}
\solutionspace{80pt}

\tasknumber{2}%
\task{%
    На плоскопараллельную стеклянную пластинку под углом $50\degrees$
    падают два параллельных луча света, расстояние между которыми $6\,\text{см}$.
    Определите расстояние между точками, в которых эти лучи выходят из пластинки.
}
\answer{%
    $\ell = \frac{d}{\cos \alpha} \approx 93\,\text{мм}$
}
\solutionspace{80pt}

\tasknumber{3}%
\task{%
    Солнце составляет с горизонтом угол, синус которого  0{,}6 .
    Шест высотой $170\,\text{см}$ вбит в дно водоёма глубиной $70\,\text{см}$.
    Найдите длину тени от этого шеста на дне водоёма, если показатель преломления воды 1{,}33.
}
\answer{%
    $
        n \sin \beta = 1 \cdot \cos \alpha \implies \beta \approx 37{,}0\degrees,
        L = (H - h)\ctg \alpha + h \tg \beta \approx 133{,}3\,\text{см} + 52{,}7\,\text{см} \approx 186\,\text{см}.
    $
}
\solutionspace{80pt}

\tasknumber{4}%
\task{%
    Луч света падает на плоское зеркало под углом, синус которого  0{,}75 .
    На сколько миллиметров сместится отражённый луч,
    если на зеркало положить прозрачную пластину толщиной $17\,\text{мм}$ с показателем преломления $1{,}35$?
}
\answer{%
    $1 \cdot \sin \alpha = n \cdot \sin \beta \implies \beta \approx 33{,}7\degrees, L = \cbr{2 d \tg \alpha - 2 d \tg \beta} \cdot \cos \alpha \approx 10{,}5\,\text{мм}$
}
\solutionspace{80pt}

\tasknumber{5}%
\task{%
    В некотором прозрачном веществе свет распространяется со скоростью,
    вчетверо меньшей скорости света в вакууме.
    Чему будет равен предельный угол
    внутреннего отражения для поверхности раздела этого вещества с водой?
}
\answer{%
    $n_1 = \frac c v = 4, n_1 \sin \varphi_\text{п.в.о.} = n_2 \sin \frac \pi 2 \implies \varphi_\text{п.в.о.} \approx 19{,}4\degrees.$
}
\solutionspace{80pt}

\tasknumber{6}%
\task{%
    Луч света падает на горизонтально расположенную стеклянную пластинку толщиной $1{,}6\,\text{см}$.
    Пройдя через пластину, он выходит из неё в точке, смещённой по горизонтали от точки падения на расстояние $4\,\text{мм}$.
    Показатель преломления стекла $1{,}5$.
    Найти синус угла падения.
}
\answer{%
    \begin{align*}
    \ctg \beta &= \frac{h}{d} \implies \\
    \implies \frac 1{\sin^2 \beta} &= \ctg^2 \beta + 1 = \sqr{\frac{h}{d}} + 1 \implies \\
    \implies \sin\alpha &= n\sin \beta = n\sqrt {\frac 1{\sqr{\frac{h}{d}} + 1}} \approx 0{,}36
    \end{align*}
}

\variantsplitter

\addpersonalvariant{Софья Андрианова}

\tasknumber{1}%
\task{%
    Для закона отражения:
    \begin{itemize}
        \item сделайте рисунок,
        \item отметьте все необходимые углы и подпишите их названия,
        \item запишите этот закон формулой.
    \end{itemize}
}
\solutionspace{80pt}

\tasknumber{2}%
\task{%
    На плоскопараллельную стеклянную пластинку под углом $40\degrees$
    падают два параллельных луча света, расстояние между которыми $4\,\text{см}$.
    Определите расстояние между точками, в которых эти лучи выходят из пластинки.
}
\answer{%
    $\ell = \frac{d}{\cos \alpha} \approx 52\,\text{мм}$
}
\solutionspace{80pt}

\tasknumber{3}%
\task{%
    Солнце составляет с горизонтом угол, синус которого  0{,}8 .
    Шест высотой $140\,\text{см}$ вбит в дно водоёма глубиной $80\,\text{см}$.
    Найдите длину тени от этого шеста на дне водоёма, если показатель преломления воды 1{,}33.
}
\answer{%
    $
        n \sin \beta = 1 \cdot \cos \alpha \implies \beta \approx 26{,}8\degrees,
        L = (H - h)\ctg \alpha + h \tg \beta \approx 45\,\text{см} + 40{,}4\,\text{см} \approx 85{,}4\,\text{см}.
    $
}
\solutionspace{80pt}

\tasknumber{4}%
\task{%
    Луч света падает на плоское зеркало под углом, синус которого  0{,}65 .
    На сколько миллиметров сместится отражённый луч,
    если на зеркало положить прозрачную пластину толщиной $15\,\text{мм}$ с показателем преломления $1{,}6$?
}
\answer{%
    $1 \cdot \sin \alpha = n \cdot \sin \beta \implies \beta \approx 24{,}0\degrees, L = \cbr{2 d \tg \alpha - 2 d \tg \beta} \cdot \cos \alpha \approx 9{,}4\,\text{мм}$
}
\solutionspace{80pt}

\tasknumber{5}%
\task{%
    В некотором прозрачном веществе свет распространяется со скоростью,
    втрое меньшей скорости света в вакууме.
    Чему будет равен предельный угол
    внутреннего отражения для поверхности раздела этого вещества с водой?
}
\answer{%
    $n_1 = \frac c v = 3, n_1 \sin \varphi_\text{п.в.о.} = n_2 \sin \frac \pi 2 \implies \varphi_\text{п.в.о.} \approx 26{,}3\degrees.$
}
\solutionspace{80pt}

\tasknumber{6}%
\task{%
    Луч света падает на горизонтально расположенную стеклянную пластинку толщиной $1{,}6\,\text{см}$.
    Пройдя через пластину, он выходит из неё в точке, смещённой по горизонтали от точки падения на расстояние $6\,\text{мм}$.
    Показатель преломления стекла $1{,}6$.
    Найти синус угла падения.
}
\answer{%
    \begin{align*}
    \ctg \beta &= \frac{h}{d} \implies \\
    \implies \frac 1{\sin^2 \beta} &= \ctg^2 \beta + 1 = \sqr{\frac{h}{d}} + 1 \implies \\
    \implies \sin\alpha &= n\sin \beta = n\sqrt {\frac 1{\sqr{\frac{h}{d}} + 1}} \approx 0{,}56
    \end{align*}
}

\variantsplitter

\addpersonalvariant{Владимир Артемчук}

\tasknumber{1}%
\task{%
    Для закона преломления:
    \begin{itemize}
        \item сделайте рисунок,
        \item отметьте все необходимые углы и подпишите их названия,
        \item запишите этот закон формулой.
    \end{itemize}
}
\solutionspace{80pt}

\tasknumber{2}%
\task{%
    На плоскопараллельную стеклянную пластинку под углом $40\degrees$
    падают два параллельных луча света, расстояние между которыми $6\,\text{см}$.
    Определите расстояние между точками, в которых эти лучи выходят из пластинки.
}
\answer{%
    $\ell = \frac{d}{\cos \alpha} \approx 78\,\text{мм}$
}
\solutionspace{80pt}

\tasknumber{3}%
\task{%
    Солнце составляет с горизонтом угол, синус которого  0{,}5 .
    Шест высотой $160\,\text{см}$ вбит в дно водоёма глубиной $80\,\text{см}$.
    Найдите длину тени от этого шеста на дне водоёма, если показатель преломления воды 1{,}33.
}
\answer{%
    $
        n \sin \beta = 1 \cdot \cos \alpha \implies \beta \approx 40{,}6\degrees,
        L = (H - h)\ctg \alpha + h \tg \beta \approx 138{,}6\,\text{см} + 68{,}6\,\text{см} \approx 207{,}2\,\text{см}.
    $
}
\solutionspace{80pt}

\tasknumber{4}%
\task{%
    Луч света падает на плоское зеркало под углом, синус которого  0{,}65 .
    На сколько миллиметров сместится отражённый луч,
    если на зеркало положить прозрачную пластину толщиной $12\,\text{мм}$ с показателем преломления $1{,}3$?
}
\answer{%
    $1 \cdot \sin \alpha = n \cdot \sin \beta \implies \beta \approx 30{,}0\degrees, L = \cbr{2 d \tg \alpha - 2 d \tg \beta} \cdot \cos \alpha \approx 5{,}1\,\text{мм}$
}
\solutionspace{80pt}

\tasknumber{5}%
\task{%
    В некотором прозрачном веществе свет распространяется со скоростью,
    вдвое меньшей скорости света в вакууме.
    Чему будет равен предельный угол
    внутреннего отражения для поверхности раздела этого вещества с водой?
}
\answer{%
    $n_1 = \frac c v = 2, n_1 \sin \varphi_\text{п.в.о.} = n_2 \sin \frac \pi 2 \implies \varphi_\text{п.в.о.} \approx 41{,}7\degrees.$
}
\solutionspace{80pt}

\tasknumber{6}%
\task{%
    Луч света падает на горизонтально расположенную стеклянную пластинку толщиной $1{,}4\,\text{см}$.
    Пройдя через пластину, он выходит из неё в точке, смещённой по горизонтали от точки падения на расстояние $6\,\text{мм}$.
    Показатель преломления стекла $1{,}5$.
    Найти синус угла падения.
}
\answer{%
    \begin{align*}
    \ctg \beta &= \frac{h}{d} \implies \\
    \implies \frac 1{\sin^2 \beta} &= \ctg^2 \beta + 1 = \sqr{\frac{h}{d}} + 1 \implies \\
    \implies \sin\alpha &= n\sin \beta = n\sqrt {\frac 1{\sqr{\frac{h}{d}} + 1}} \approx 0{,}59
    \end{align*}
}

\variantsplitter

\addpersonalvariant{Софья Белянкина}

\tasknumber{1}%
\task{%
    Для закона преломления:
    \begin{itemize}
        \item сделайте рисунок,
        \item отметьте все необходимые углы и подпишите их названия,
        \item запишите этот закон формулой.
    \end{itemize}
}
\solutionspace{80pt}

\tasknumber{2}%
\task{%
    На плоскопараллельную стеклянную пластинку под углом $50\degrees$
    падают два параллельных луча света, расстояние между которыми $5\,\text{см}$.
    Определите расстояние между точками, в которых эти лучи выходят из пластинки.
}
\answer{%
    $\ell = \frac{d}{\cos \alpha} \approx 78\,\text{мм}$
}
\solutionspace{80pt}

\tasknumber{3}%
\task{%
    Солнце составляет с горизонтом угол, синус которого  0{,}8 .
    Шест высотой $170\,\text{см}$ вбит в дно водоёма глубиной $90\,\text{см}$.
    Найдите длину тени от этого шеста на дне водоёма, если показатель преломления воды 1{,}33.
}
\answer{%
    $
        n \sin \beta = 1 \cdot \cos \alpha \implies \beta \approx 26{,}8\degrees,
        L = (H - h)\ctg \alpha + h \tg \beta \approx 60\,\text{см} + 45{,}5\,\text{см} \approx 105{,}5\,\text{см}.
    $
}
\solutionspace{80pt}

\tasknumber{4}%
\task{%
    Луч света падает на плоское зеркало под углом, синус которого  0{,}75 .
    На сколько миллиметров сместится отражённый луч,
    если на зеркало положить прозрачную пластину толщиной $13\,\text{мм}$ с показателем преломления $1{,}45$?
}
\answer{%
    $1 \cdot \sin \alpha = n \cdot \sin \beta \implies \beta \approx 31{,}1\degrees, L = \cbr{2 d \tg \alpha - 2 d \tg \beta} \cdot \cos \alpha \approx 9{,}1\,\text{мм}$
}
\solutionspace{80pt}

\tasknumber{5}%
\task{%
    В некотором прозрачном веществе свет распространяется со скоростью,
    вдвое меньшей скорости света в вакууме.
    Чему будет равен предельный угол
    внутреннего отражения для поверхности раздела этого вещества с воздухом?
}
\answer{%
    $n_1 = \frac c v = 2, n_1 \sin \varphi_\text{п.в.о.} = n_2 \sin \frac \pi 2 \implies \varphi_\text{п.в.о.} \approx 30{,}0\degrees.$
}
\solutionspace{80pt}

\tasknumber{6}%
\task{%
    Луч света падает на вертикально расположенную стеклянную пластинку толщиной $1{,}2\,\text{см}$.
    Пройдя через пластину, он выходит из неё в точке, смещённой по вертикали от точки падения на расстояние $5\,\text{мм}$.
    Показатель преломления стекла $1{,}6$.
    Найти синус угла падения.
}
\answer{%
    \begin{align*}
    \ctg \beta &= \frac{h}{d} \implies \\
    \implies \frac 1{\sin^2 \beta} &= \ctg^2 \beta + 1 = \sqr{\frac{h}{d}} + 1 \implies \\
    \implies \sin\alpha &= n\sin \beta = n\sqrt {\frac 1{\sqr{\frac{h}{d}} + 1}} \approx 0{,}62
    \end{align*}
}

\variantsplitter

\addpersonalvariant{Варвара Егиазарян}

\tasknumber{1}%
\task{%
    Для закона отражения:
    \begin{itemize}
        \item сделайте рисунок,
        \item отметьте все необходимые углы и подпишите их названия,
        \item запишите этот закон формулой.
    \end{itemize}
}
\solutionspace{80pt}

\tasknumber{2}%
\task{%
    На плоскопараллельную стеклянную пластинку под углом $50\degrees$
    падают два параллельных луча света, расстояние между которыми $4\,\text{см}$.
    Определите расстояние между точками, в которых эти лучи выходят из пластинки.
}
\answer{%
    $\ell = \frac{d}{\cos \alpha} \approx 62\,\text{мм}$
}
\solutionspace{80pt}

\tasknumber{3}%
\task{%
    Солнце составляет с горизонтом угол, синус которого  0{,}6 .
    Шест высотой $120\,\text{см}$ вбит в дно водоёма глубиной $70\,\text{см}$.
    Найдите длину тени от этого шеста на дне водоёма, если показатель преломления воды 1{,}33.
}
\answer{%
    $
        n \sin \beta = 1 \cdot \cos \alpha \implies \beta \approx 37{,}0\degrees,
        L = (H - h)\ctg \alpha + h \tg \beta \approx 66{,}7\,\text{см} + 52{,}7\,\text{см} \approx 119{,}4\,\text{см}.
    $
}
\solutionspace{80pt}

\tasknumber{4}%
\task{%
    Луч света падает на плоское зеркало под углом, синус которого  0{,}65 .
    На сколько миллиметров сместится отражённый луч,
    если на зеркало положить прозрачную пластину толщиной $15\,\text{мм}$ с показателем преломления $1{,}3$?
}
\answer{%
    $1 \cdot \sin \alpha = n \cdot \sin \beta \implies \beta \approx 30{,}0\degrees, L = \cbr{2 d \tg \alpha - 2 d \tg \beta} \cdot \cos \alpha \approx 6{,}3\,\text{мм}$
}
\solutionspace{80pt}

\tasknumber{5}%
\task{%
    В некотором прозрачном веществе свет распространяется со скоростью,
    втрое меньшей скорости света в вакууме.
    Чему будет равен предельный угол
    внутреннего отражения для поверхности раздела этого вещества с воздухом?
}
\answer{%
    $n_1 = \frac c v = 3, n_1 \sin \varphi_\text{п.в.о.} = n_2 \sin \frac \pi 2 \implies \varphi_\text{п.в.о.} \approx 19{,}5\degrees.$
}
\solutionspace{80pt}

\tasknumber{6}%
\task{%
    Луч света падает на вертикально расположенную стеклянную пластинку толщиной $1{,}3\,\text{см}$.
    Пройдя через пластину, он выходит из неё в точке, смещённой по вертикали от точки падения на расстояние $5\,\text{мм}$.
    Показатель преломления стекла $1{,}6$.
    Найти синус угла падения.
}
\answer{%
    \begin{align*}
    \ctg \beta &= \frac{h}{d} \implies \\
    \implies \frac 1{\sin^2 \beta} &= \ctg^2 \beta + 1 = \sqr{\frac{h}{d}} + 1 \implies \\
    \implies \sin\alpha &= n\sin \beta = n\sqrt {\frac 1{\sqr{\frac{h}{d}} + 1}} \approx 0{,}57
    \end{align*}
}

\variantsplitter

\addpersonalvariant{Владислав Емелин}

\tasknumber{1}%
\task{%
    Для закона преломления:
    \begin{itemize}
        \item сделайте рисунок,
        \item отметьте все необходимые углы и подпишите их названия,
        \item запишите этот закон формулой.
    \end{itemize}
}
\solutionspace{80pt}

\tasknumber{2}%
\task{%
    На плоскопараллельную стеклянную пластинку под углом $50\degrees$
    падают два параллельных луча света, расстояние между которыми $4\,\text{см}$.
    Определите расстояние между точками, в которых эти лучи выходят из пластинки.
}
\answer{%
    $\ell = \frac{d}{\cos \alpha} \approx 62\,\text{мм}$
}
\solutionspace{80pt}

\tasknumber{3}%
\task{%
    Солнце составляет с горизонтом угол, синус которого  0{,}7 .
    Шест высотой $120\,\text{см}$ вбит в дно водоёма глубиной $70\,\text{см}$.
    Найдите длину тени от этого шеста на дне водоёма, если показатель преломления воды 1{,}33.
}
\answer{%
    $
        n \sin \beta = 1 \cdot \cos \alpha \implies \beta \approx 32{,}5\degrees,
        L = (H - h)\ctg \alpha + h \tg \beta \approx 51\,\text{см} + 44{,}6\,\text{см} \approx 95{,}6\,\text{см}.
    $
}
\solutionspace{80pt}

\tasknumber{4}%
\task{%
    Луч света падает на плоское зеркало под углом, синус которого  0{,}65 .
    На сколько миллиметров сместится отражённый луч,
    если на зеркало положить прозрачную пластину толщиной $12\,\text{мм}$ с показателем преломления $1{,}3$?
}
\answer{%
    $1 \cdot \sin \alpha = n \cdot \sin \beta \implies \beta \approx 30{,}0\degrees, L = \cbr{2 d \tg \alpha - 2 d \tg \beta} \cdot \cos \alpha \approx 5{,}1\,\text{мм}$
}
\solutionspace{80pt}

\tasknumber{5}%
\task{%
    В некотором прозрачном веществе свет распространяется со скоростью,
    вчетверо меньшей скорости света в вакууме.
    Чему будет равен предельный угол
    внутреннего отражения для поверхности раздела этого вещества с водой?
}
\answer{%
    $n_1 = \frac c v = 4, n_1 \sin \varphi_\text{п.в.о.} = n_2 \sin \frac \pi 2 \implies \varphi_\text{п.в.о.} \approx 19{,}4\degrees.$
}
\solutionspace{80pt}

\tasknumber{6}%
\task{%
    Луч света падает на вертикально расположенную стеклянную пластинку толщиной $1{,}2\,\text{см}$.
    Пройдя через пластину, он выходит из неё в точке, смещённой по вертикали от точки падения на расстояние $6\,\text{мм}$.
    Показатель преломления стекла $1{,}6$.
    Найти синус угла падения.
}
\answer{%
    \begin{align*}
    \ctg \beta &= \frac{h}{d} \implies \\
    \implies \frac 1{\sin^2 \beta} &= \ctg^2 \beta + 1 = \sqr{\frac{h}{d}} + 1 \implies \\
    \implies \sin\alpha &= n\sin \beta = n\sqrt {\frac 1{\sqr{\frac{h}{d}} + 1}} \approx 0{,}72
    \end{align*}
}

\variantsplitter

\addpersonalvariant{Артём Жичин}

\tasknumber{1}%
\task{%
    Для закона преломления:
    \begin{itemize}
        \item сделайте рисунок,
        \item отметьте все необходимые углы и подпишите их названия,
        \item запишите этот закон формулой.
    \end{itemize}
}
\solutionspace{80pt}

\tasknumber{2}%
\task{%
    На плоскопараллельную стеклянную пластинку под углом $50\degrees$
    падают два параллельных луча света, расстояние между которыми $8\,\text{см}$.
    Определите расстояние между точками, в которых эти лучи выходят из пластинки.
}
\answer{%
    $\ell = \frac{d}{\cos \alpha} \approx 124\,\text{мм}$
}
\solutionspace{80pt}

\tasknumber{3}%
\task{%
    Солнце составляет с горизонтом угол, синус которого  0{,}7 .
    Шест высотой $180\,\text{см}$ вбит в дно водоёма глубиной $90\,\text{см}$.
    Найдите длину тени от этого шеста на дне водоёма, если показатель преломления воды 1{,}33.
}
\answer{%
    $
        n \sin \beta = 1 \cdot \cos \alpha \implies \beta \approx 32{,}5\degrees,
        L = (H - h)\ctg \alpha + h \tg \beta \approx 91{,}8\,\text{см} + 57{,}3\,\text{см} \approx 149{,}1\,\text{см}.
    $
}
\solutionspace{80pt}

\tasknumber{4}%
\task{%
    Луч света падает на плоское зеркало под углом, синус которого  0{,}75 .
    На сколько миллиметров сместится отражённый луч,
    если на зеркало положить прозрачную пластину толщиной $11\,\text{мм}$ с показателем преломления $1{,}45$?
}
\answer{%
    $1 \cdot \sin \alpha = n \cdot \sin \beta \implies \beta \approx 31{,}1\degrees, L = \cbr{2 d \tg \alpha - 2 d \tg \beta} \cdot \cos \alpha \approx 7{,}7\,\text{мм}$
}
\solutionspace{80pt}

\tasknumber{5}%
\task{%
    В некотором прозрачном веществе свет распространяется со скоростью,
    вдвое меньшей скорости света в вакууме.
    Чему будет равен предельный угол
    внутреннего отражения для поверхности раздела этого вещества с водой?
}
\answer{%
    $n_1 = \frac c v = 2, n_1 \sin \varphi_\text{п.в.о.} = n_2 \sin \frac \pi 2 \implies \varphi_\text{п.в.о.} \approx 41{,}7\degrees.$
}
\solutionspace{80pt}

\tasknumber{6}%
\task{%
    Луч света падает на горизонтально расположенную стеклянную пластинку толщиной $1{,}6\,\text{см}$.
    Пройдя через пластину, он выходит из неё в точке, смещённой по горизонтали от точки падения на расстояние $6\,\text{мм}$.
    Показатель преломления стекла $1{,}5$.
    Найти синус угла падения.
}
\answer{%
    \begin{align*}
    \ctg \beta &= \frac{h}{d} \implies \\
    \implies \frac 1{\sin^2 \beta} &= \ctg^2 \beta + 1 = \sqr{\frac{h}{d}} + 1 \implies \\
    \implies \sin\alpha &= n\sin \beta = n\sqrt {\frac 1{\sqr{\frac{h}{d}} + 1}} \approx 0{,}53
    \end{align*}
}

\variantsplitter

\addpersonalvariant{Дарья Кошман}

\tasknumber{1}%
\task{%
    Для закона отражения:
    \begin{itemize}
        \item сделайте рисунок,
        \item отметьте все необходимые углы и подпишите их названия,
        \item запишите этот закон формулой.
    \end{itemize}
}
\solutionspace{80pt}

\tasknumber{2}%
\task{%
    На плоскопараллельную стеклянную пластинку под углом $55\degrees$
    падают два параллельных луча света, расстояние между которыми $4\,\text{см}$.
    Определите расстояние между точками, в которых эти лучи выходят из пластинки.
}
\answer{%
    $\ell = \frac{d}{\cos \alpha} \approx 70\,\text{мм}$
}
\solutionspace{80pt}

\tasknumber{3}%
\task{%
    Солнце составляет с горизонтом угол, синус которого  0{,}5 .
    Шест высотой $150\,\text{см}$ вбит в дно водоёма глубиной $70\,\text{см}$.
    Найдите длину тени от этого шеста на дне водоёма, если показатель преломления воды 1{,}33.
}
\answer{%
    $
        n \sin \beta = 1 \cdot \cos \alpha \implies \beta \approx 40{,}6\degrees,
        L = (H - h)\ctg \alpha + h \tg \beta \approx 138{,}6\,\text{см} + 60{,}1\,\text{см} \approx 198{,}6\,\text{см}.
    $
}
\solutionspace{80pt}

\tasknumber{4}%
\task{%
    Луч света падает на плоское зеркало под углом, синус которого  0{,}85 .
    На сколько миллиметров сместится отражённый луч,
    если на зеркало положить прозрачную пластину толщиной $18\,\text{мм}$ с показателем преломления $1{,}55$?
}
\answer{%
    $1 \cdot \sin \alpha = n \cdot \sin \beta \implies \beta \approx 33{,}3\degrees, L = \cbr{2 d \tg \alpha - 2 d \tg \beta} \cdot \cos \alpha \approx 18{,}2\,\text{мм}$
}
\solutionspace{80pt}

\tasknumber{5}%
\task{%
    В некотором прозрачном веществе свет распространяется со скоростью,
    вчетверо меньшей скорости света в вакууме.
    Чему будет равен предельный угол
    внутреннего отражения для поверхности раздела этого вещества с воздухом?
}
\answer{%
    $n_1 = \frac c v = 4, n_1 \sin \varphi_\text{п.в.о.} = n_2 \sin \frac \pi 2 \implies \varphi_\text{п.в.о.} \approx 14{,}5\degrees.$
}
\solutionspace{80pt}

\tasknumber{6}%
\task{%
    Луч света падает на вертикально расположенную стеклянную пластинку толщиной $1{,}2\,\text{см}$.
    Пройдя через пластину, он выходит из неё в точке, смещённой по вертикали от точки падения на расстояние $4\,\text{мм}$.
    Показатель преломления стекла $1{,}6$.
    Найти синус угла падения.
}
\answer{%
    \begin{align*}
    \ctg \beta &= \frac{h}{d} \implies \\
    \implies \frac 1{\sin^2 \beta} &= \ctg^2 \beta + 1 = \sqr{\frac{h}{d}} + 1 \implies \\
    \implies \sin\alpha &= n\sin \beta = n\sqrt {\frac 1{\sqr{\frac{h}{d}} + 1}} \approx 0{,}51
    \end{align*}
}

\variantsplitter

\addpersonalvariant{Анна Кузьмичёва}

\tasknumber{1}%
\task{%
    Для закона отражения:
    \begin{itemize}
        \item сделайте рисунок,
        \item отметьте все необходимые углы и подпишите их названия,
        \item запишите этот закон формулой.
    \end{itemize}
}
\solutionspace{80pt}

\tasknumber{2}%
\task{%
    На плоскопараллельную стеклянную пластинку под углом $50\degrees$
    падают два параллельных луча света, расстояние между которыми $3\,\text{см}$.
    Определите расстояние между точками, в которых эти лучи выходят из пластинки.
}
\answer{%
    $\ell = \frac{d}{\cos \alpha} \approx 47\,\text{мм}$
}
\solutionspace{80pt}

\tasknumber{3}%
\task{%
    Солнце составляет с горизонтом угол, синус которого  0{,}8 .
    Шест высотой $170\,\text{см}$ вбит в дно водоёма глубиной $90\,\text{см}$.
    Найдите длину тени от этого шеста на дне водоёма, если показатель преломления воды 1{,}33.
}
\answer{%
    $
        n \sin \beta = 1 \cdot \cos \alpha \implies \beta \approx 26{,}8\degrees,
        L = (H - h)\ctg \alpha + h \tg \beta \approx 60\,\text{см} + 45{,}5\,\text{см} \approx 105{,}5\,\text{см}.
    $
}
\solutionspace{80pt}

\tasknumber{4}%
\task{%
    Луч света падает на плоское зеркало под углом, синус которого  0{,}75 .
    На сколько миллиметров сместится отражённый луч,
    если на зеркало положить прозрачную пластину толщиной $13\,\text{мм}$ с показателем преломления $1{,}55$?
}
\answer{%
    $1 \cdot \sin \alpha = n \cdot \sin \beta \implies \beta \approx 28{,}9\degrees, L = \cbr{2 d \tg \alpha - 2 d \tg \beta} \cdot \cos \alpha \approx 10{,}0\,\text{мм}$
}
\solutionspace{80pt}

\tasknumber{5}%
\task{%
    В некотором прозрачном веществе свет распространяется со скоростью,
    втрое меньшей скорости света в вакууме.
    Чему будет равен предельный угол
    внутреннего отражения для поверхности раздела этого вещества с воздухом?
}
\answer{%
    $n_1 = \frac c v = 3, n_1 \sin \varphi_\text{п.в.о.} = n_2 \sin \frac \pi 2 \implies \varphi_\text{п.в.о.} \approx 19{,}5\degrees.$
}
\solutionspace{80pt}

\tasknumber{6}%
\task{%
    Луч света падает на вертикально расположенную стеклянную пластинку толщиной $1{,}5\,\text{см}$.
    Пройдя через пластину, он выходит из неё в точке, смещённой по вертикали от точки падения на расстояние $5\,\text{мм}$.
    Показатель преломления стекла $1{,}5$.
    Найти синус угла падения.
}
\answer{%
    \begin{align*}
    \ctg \beta &= \frac{h}{d} \implies \\
    \implies \frac 1{\sin^2 \beta} &= \ctg^2 \beta + 1 = \sqr{\frac{h}{d}} + 1 \implies \\
    \implies \sin\alpha &= n\sin \beta = n\sqrt {\frac 1{\sqr{\frac{h}{d}} + 1}} \approx 0{,}47
    \end{align*}
}

\variantsplitter

\addpersonalvariant{Алёна Куприянова}

\tasknumber{1}%
\task{%
    Для закона отражения:
    \begin{itemize}
        \item сделайте рисунок,
        \item отметьте все необходимые углы и подпишите их названия,
        \item запишите этот закон формулой.
    \end{itemize}
}
\solutionspace{80pt}

\tasknumber{2}%
\task{%
    На плоскопараллельную стеклянную пластинку под углом $55\degrees$
    падают два параллельных луча света, расстояние между которыми $4\,\text{см}$.
    Определите расстояние между точками, в которых эти лучи выходят из пластинки.
}
\answer{%
    $\ell = \frac{d}{\cos \alpha} \approx 70\,\text{мм}$
}
\solutionspace{80pt}

\tasknumber{3}%
\task{%
    Солнце составляет с горизонтом угол, синус которого  0{,}7 .
    Шест высотой $180\,\text{см}$ вбит в дно водоёма глубиной $80\,\text{см}$.
    Найдите длину тени от этого шеста на дне водоёма, если показатель преломления воды 1{,}33.
}
\answer{%
    $
        n \sin \beta = 1 \cdot \cos \alpha \implies \beta \approx 32{,}5\degrees,
        L = (H - h)\ctg \alpha + h \tg \beta \approx 102\,\text{см} + 50{,}9\,\text{см} \approx 152{,}9\,\text{см}.
    $
}
\solutionspace{80pt}

\tasknumber{4}%
\task{%
    Луч света падает на плоское зеркало под углом, синус которого  0{,}85 .
    На сколько миллиметров сместится отражённый луч,
    если на зеркало положить прозрачную пластину толщиной $11\,\text{мм}$ с показателем преломления $1{,}3$?
}
\answer{%
    $1 \cdot \sin \alpha = n \cdot \sin \beta \implies \beta \approx 40{,}8\degrees, L = \cbr{2 d \tg \alpha - 2 d \tg \beta} \cdot \cos \alpha \approx 8{,}7\,\text{мм}$
}
\solutionspace{80pt}

\tasknumber{5}%
\task{%
    В некотором прозрачном веществе свет распространяется со скоростью,
    втрое меньшей скорости света в вакууме.
    Чему будет равен предельный угол
    внутреннего отражения для поверхности раздела этого вещества с воздухом?
}
\answer{%
    $n_1 = \frac c v = 3, n_1 \sin \varphi_\text{п.в.о.} = n_2 \sin \frac \pi 2 \implies \varphi_\text{п.в.о.} \approx 19{,}5\degrees.$
}
\solutionspace{80pt}

\tasknumber{6}%
\task{%
    Луч света падает на горизонтально расположенную стеклянную пластинку толщиной $1{,}4\,\text{см}$.
    Пройдя через пластину, он выходит из неё в точке, смещённой по горизонтали от точки падения на расстояние $5\,\text{мм}$.
    Показатель преломления стекла $1{,}5$.
    Найти синус угла падения.
}
\answer{%
    \begin{align*}
    \ctg \beta &= \frac{h}{d} \implies \\
    \implies \frac 1{\sin^2 \beta} &= \ctg^2 \beta + 1 = \sqr{\frac{h}{d}} + 1 \implies \\
    \implies \sin\alpha &= n\sin \beta = n\sqrt {\frac 1{\sqr{\frac{h}{d}} + 1}} \approx 0{,}50
    \end{align*}
}

\variantsplitter

\addpersonalvariant{Ярослав Лавровский}

\tasknumber{1}%
\task{%
    Для закона преломления:
    \begin{itemize}
        \item сделайте рисунок,
        \item отметьте все необходимые углы и подпишите их названия,
        \item запишите этот закон формулой.
    \end{itemize}
}
\solutionspace{80pt}

\tasknumber{2}%
\task{%
    На плоскопараллельную стеклянную пластинку под углом $55\degrees$
    падают два параллельных луча света, расстояние между которыми $3\,\text{см}$.
    Определите расстояние между точками, в которых эти лучи выходят из пластинки.
}
\answer{%
    $\ell = \frac{d}{\cos \alpha} \approx 52\,\text{мм}$
}
\solutionspace{80pt}

\tasknumber{3}%
\task{%
    Солнце составляет с горизонтом угол, синус которого  0{,}7 .
    Шест высотой $170\,\text{см}$ вбит в дно водоёма глубиной $80\,\text{см}$.
    Найдите длину тени от этого шеста на дне водоёма, если показатель преломления воды 1{,}33.
}
\answer{%
    $
        n \sin \beta = 1 \cdot \cos \alpha \implies \beta \approx 32{,}5\degrees,
        L = (H - h)\ctg \alpha + h \tg \beta \approx 91{,}8\,\text{см} + 50{,}9\,\text{см} \approx 142{,}7\,\text{см}.
    $
}
\solutionspace{80pt}

\tasknumber{4}%
\task{%
    Луч света падает на плоское зеркало под углом, синус которого  0{,}85 .
    На сколько миллиметров сместится отражённый луч,
    если на зеркало положить прозрачную пластину толщиной $11\,\text{мм}$ с показателем преломления $1{,}3$?
}
\answer{%
    $1 \cdot \sin \alpha = n \cdot \sin \beta \implies \beta \approx 40{,}8\degrees, L = \cbr{2 d \tg \alpha - 2 d \tg \beta} \cdot \cos \alpha \approx 8{,}7\,\text{мм}$
}
\solutionspace{80pt}

\tasknumber{5}%
\task{%
    В некотором прозрачном веществе свет распространяется со скоростью,
    втрое меньшей скорости света в вакууме.
    Чему будет равен предельный угол
    внутреннего отражения для поверхности раздела этого вещества с водой?
}
\answer{%
    $n_1 = \frac c v = 3, n_1 \sin \varphi_\text{п.в.о.} = n_2 \sin \frac \pi 2 \implies \varphi_\text{п.в.о.} \approx 26{,}3\degrees.$
}
\solutionspace{80pt}

\tasknumber{6}%
\task{%
    Луч света падает на горизонтально расположенную стеклянную пластинку толщиной $1{,}3\,\text{см}$.
    Пройдя через пластину, он выходит из неё в точке, смещённой по горизонтали от точки падения на расстояние $6\,\text{мм}$.
    Показатель преломления стекла $1{,}5$.
    Найти синус угла падения.
}
\answer{%
    \begin{align*}
    \ctg \beta &= \frac{h}{d} \implies \\
    \implies \frac 1{\sin^2 \beta} &= \ctg^2 \beta + 1 = \sqr{\frac{h}{d}} + 1 \implies \\
    \implies \sin\alpha &= n\sin \beta = n\sqrt {\frac 1{\sqr{\frac{h}{d}} + 1}} \approx 0{,}63
    \end{align*}
}

\variantsplitter

\addpersonalvariant{Анастасия Ламанова}

\tasknumber{1}%
\task{%
    Для закона отражения:
    \begin{itemize}
        \item сделайте рисунок,
        \item отметьте все необходимые углы и подпишите их названия,
        \item запишите этот закон формулой.
    \end{itemize}
}
\solutionspace{80pt}

\tasknumber{2}%
\task{%
    На плоскопараллельную стеклянную пластинку под углом $50\degrees$
    падают два параллельных луча света, расстояние между которыми $7\,\text{см}$.
    Определите расстояние между точками, в которых эти лучи выходят из пластинки.
}
\answer{%
    $\ell = \frac{d}{\cos \alpha} \approx 109\,\text{мм}$
}
\solutionspace{80pt}

\tasknumber{3}%
\task{%
    Солнце составляет с горизонтом угол, синус которого  0{,}7 .
    Шест высотой $120\,\text{см}$ вбит в дно водоёма глубиной $70\,\text{см}$.
    Найдите длину тени от этого шеста на дне водоёма, если показатель преломления воды 1{,}33.
}
\answer{%
    $
        n \sin \beta = 1 \cdot \cos \alpha \implies \beta \approx 32{,}5\degrees,
        L = (H - h)\ctg \alpha + h \tg \beta \approx 51\,\text{см} + 44{,}6\,\text{см} \approx 95{,}6\,\text{см}.
    $
}
\solutionspace{80pt}

\tasknumber{4}%
\task{%
    Луч света падает на плоское зеркало под углом, синус которого  0{,}65 .
    На сколько миллиметров сместится отражённый луч,
    если на зеркало положить прозрачную пластину толщиной $15\,\text{мм}$ с показателем преломления $1{,}6$?
}
\answer{%
    $1 \cdot \sin \alpha = n \cdot \sin \beta \implies \beta \approx 24{,}0\degrees, L = \cbr{2 d \tg \alpha - 2 d \tg \beta} \cdot \cos \alpha \approx 9{,}4\,\text{мм}$
}
\solutionspace{80pt}

\tasknumber{5}%
\task{%
    В некотором прозрачном веществе свет распространяется со скоростью,
    вчетверо меньшей скорости света в вакууме.
    Чему будет равен предельный угол
    внутреннего отражения для поверхности раздела этого вещества с воздухом?
}
\answer{%
    $n_1 = \frac c v = 4, n_1 \sin \varphi_\text{п.в.о.} = n_2 \sin \frac \pi 2 \implies \varphi_\text{п.в.о.} \approx 14{,}5\degrees.$
}
\solutionspace{80pt}

\tasknumber{6}%
\task{%
    Луч света падает на горизонтально расположенную стеклянную пластинку толщиной $1{,}5\,\text{см}$.
    Пройдя через пластину, он выходит из неё в точке, смещённой по горизонтали от точки падения на расстояние $6\,\text{мм}$.
    Показатель преломления стекла $1{,}4$.
    Найти синус угла падения.
}
\answer{%
    \begin{align*}
    \ctg \beta &= \frac{h}{d} \implies \\
    \implies \frac 1{\sin^2 \beta} &= \ctg^2 \beta + 1 = \sqr{\frac{h}{d}} + 1 \implies \\
    \implies \sin\alpha &= n\sin \beta = n\sqrt {\frac 1{\sqr{\frac{h}{d}} + 1}} \approx 0{,}52
    \end{align*}
}

\variantsplitter

\addpersonalvariant{Виктория Легонькова}

\tasknumber{1}%
\task{%
    Для закона отражения:
    \begin{itemize}
        \item сделайте рисунок,
        \item отметьте все необходимые углы и подпишите их названия,
        \item запишите этот закон формулой.
    \end{itemize}
}
\solutionspace{80pt}

\tasknumber{2}%
\task{%
    На плоскопараллельную стеклянную пластинку под углом $40\degrees$
    падают два параллельных луча света, расстояние между которыми $6\,\text{см}$.
    Определите расстояние между точками, в которых эти лучи выходят из пластинки.
}
\answer{%
    $\ell = \frac{d}{\cos \alpha} \approx 78\,\text{мм}$
}
\solutionspace{80pt}

\tasknumber{3}%
\task{%
    Солнце составляет с горизонтом угол, синус которого  0{,}7 .
    Шест высотой $140\,\text{см}$ вбит в дно водоёма глубиной $70\,\text{см}$.
    Найдите длину тени от этого шеста на дне водоёма, если показатель преломления воды 1{,}33.
}
\answer{%
    $
        n \sin \beta = 1 \cdot \cos \alpha \implies \beta \approx 32{,}5\degrees,
        L = (H - h)\ctg \alpha + h \tg \beta \approx 71{,}4\,\text{см} + 44{,}6\,\text{см} \approx 116\,\text{см}.
    $
}
\solutionspace{80pt}

\tasknumber{4}%
\task{%
    Луч света падает на плоское зеркало под углом, синус которого  0{,}85 .
    На сколько миллиметров сместится отражённый луч,
    если на зеркало положить прозрачную пластину толщиной $12\,\text{мм}$ с показателем преломления $1{,}55$?
}
\answer{%
    $1 \cdot \sin \alpha = n \cdot \sin \beta \implies \beta \approx 33{,}3\degrees, L = \cbr{2 d \tg \alpha - 2 d \tg \beta} \cdot \cos \alpha \approx 12{,}1\,\text{мм}$
}
\solutionspace{80pt}

\tasknumber{5}%
\task{%
    В некотором прозрачном веществе свет распространяется со скоростью,
    вчетверо меньшей скорости света в вакууме.
    Чему будет равен предельный угол
    внутреннего отражения для поверхности раздела этого вещества с водой?
}
\answer{%
    $n_1 = \frac c v = 4, n_1 \sin \varphi_\text{п.в.о.} = n_2 \sin \frac \pi 2 \implies \varphi_\text{п.в.о.} \approx 19{,}4\degrees.$
}
\solutionspace{80pt}

\tasknumber{6}%
\task{%
    Луч света падает на горизонтально расположенную стеклянную пластинку толщиной $1{,}6\,\text{см}$.
    Пройдя через пластину, он выходит из неё в точке, смещённой по горизонтали от точки падения на расстояние $6\,\text{мм}$.
    Показатель преломления стекла $1{,}4$.
    Найти синус угла падения.
}
\answer{%
    \begin{align*}
    \ctg \beta &= \frac{h}{d} \implies \\
    \implies \frac 1{\sin^2 \beta} &= \ctg^2 \beta + 1 = \sqr{\frac{h}{d}} + 1 \implies \\
    \implies \sin\alpha &= n\sin \beta = n\sqrt {\frac 1{\sqr{\frac{h}{d}} + 1}} \approx 0{,}49
    \end{align*}
}

\variantsplitter

\addpersonalvariant{Семён Мартынов}

\tasknumber{1}%
\task{%
    Для закона отражения:
    \begin{itemize}
        \item сделайте рисунок,
        \item отметьте все необходимые углы и подпишите их названия,
        \item запишите этот закон формулой.
    \end{itemize}
}
\solutionspace{80pt}

\tasknumber{2}%
\task{%
    На плоскопараллельную стеклянную пластинку под углом $50\degrees$
    падают два параллельных луча света, расстояние между которыми $5\,\text{см}$.
    Определите расстояние между точками, в которых эти лучи выходят из пластинки.
}
\answer{%
    $\ell = \frac{d}{\cos \alpha} \approx 78\,\text{мм}$
}
\solutionspace{80pt}

\tasknumber{3}%
\task{%
    Солнце составляет с горизонтом угол, синус которого  0{,}6 .
    Шест высотой $150\,\text{см}$ вбит в дно водоёма глубиной $90\,\text{см}$.
    Найдите длину тени от этого шеста на дне водоёма, если показатель преломления воды 1{,}33.
}
\answer{%
    $
        n \sin \beta = 1 \cdot \cos \alpha \implies \beta \approx 37{,}0\degrees,
        L = (H - h)\ctg \alpha + h \tg \beta \approx 80\,\text{см} + 67{,}8\,\text{см} \approx 147{,}8\,\text{см}.
    $
}
\solutionspace{80pt}

\tasknumber{4}%
\task{%
    Луч света падает на плоское зеркало под углом, синус которого  0{,}65 .
    На сколько миллиметров сместится отражённый луч,
    если на зеркало положить прозрачную пластину толщиной $14\,\text{мм}$ с показателем преломления $1{,}4$?
}
\answer{%
    $1 \cdot \sin \alpha = n \cdot \sin \beta \implies \beta \approx 27{,}7\degrees, L = \cbr{2 d \tg \alpha - 2 d \tg \beta} \cdot \cos \alpha \approx 7{,}0\,\text{мм}$
}
\solutionspace{80pt}

\tasknumber{5}%
\task{%
    В некотором прозрачном веществе свет распространяется со скоростью,
    вдвое меньшей скорости света в вакууме.
    Чему будет равен предельный угол
    внутреннего отражения для поверхности раздела этого вещества с водой?
}
\answer{%
    $n_1 = \frac c v = 2, n_1 \sin \varphi_\text{п.в.о.} = n_2 \sin \frac \pi 2 \implies \varphi_\text{п.в.о.} \approx 41{,}7\degrees.$
}
\solutionspace{80pt}

\tasknumber{6}%
\task{%
    Луч света падает на вертикально расположенную стеклянную пластинку толщиной $1{,}5\,\text{см}$.
    Пройдя через пластину, он выходит из неё в точке, смещённой по вертикали от точки падения на расстояние $5\,\text{мм}$.
    Показатель преломления стекла $1{,}4$.
    Найти синус угла падения.
}
\answer{%
    \begin{align*}
    \ctg \beta &= \frac{h}{d} \implies \\
    \implies \frac 1{\sin^2 \beta} &= \ctg^2 \beta + 1 = \sqr{\frac{h}{d}} + 1 \implies \\
    \implies \sin\alpha &= n\sin \beta = n\sqrt {\frac 1{\sqr{\frac{h}{d}} + 1}} \approx 0{,}44
    \end{align*}
}

\variantsplitter

\addpersonalvariant{Варвара Минаева}

\tasknumber{1}%
\task{%
    Для закона отражения:
    \begin{itemize}
        \item сделайте рисунок,
        \item отметьте все необходимые углы и подпишите их названия,
        \item запишите этот закон формулой.
    \end{itemize}
}
\solutionspace{80pt}

\tasknumber{2}%
\task{%
    На плоскопараллельную стеклянную пластинку под углом $35\degrees$
    падают два параллельных луча света, расстояние между которыми $7\,\text{см}$.
    Определите расстояние между точками, в которых эти лучи выходят из пластинки.
}
\answer{%
    $\ell = \frac{d}{\cos \alpha} \approx 85\,\text{мм}$
}
\solutionspace{80pt}

\tasknumber{3}%
\task{%
    Солнце составляет с горизонтом угол, синус которого  0{,}7 .
    Шест высотой $140\,\text{см}$ вбит в дно водоёма глубиной $80\,\text{см}$.
    Найдите длину тени от этого шеста на дне водоёма, если показатель преломления воды 1{,}33.
}
\answer{%
    $
        n \sin \beta = 1 \cdot \cos \alpha \implies \beta \approx 32{,}5\degrees,
        L = (H - h)\ctg \alpha + h \tg \beta \approx 61{,}2\,\text{см} + 50{,}9\,\text{см} \approx 112{,}1\,\text{см}.
    $
}
\solutionspace{80pt}

\tasknumber{4}%
\task{%
    Луч света падает на плоское зеркало под углом, синус которого  0{,}75 .
    На сколько миллиметров сместится отражённый луч,
    если на зеркало положить прозрачную пластину толщиной $18\,\text{мм}$ с показателем преломления $1{,}5$?
}
\answer{%
    $1 \cdot \sin \alpha = n \cdot \sin \beta \implies \beta \approx 30{,}0\degrees, L = \cbr{2 d \tg \alpha - 2 d \tg \beta} \cdot \cos \alpha \approx 13{,}3\,\text{мм}$
}
\solutionspace{80pt}

\tasknumber{5}%
\task{%
    В некотором прозрачном веществе свет распространяется со скоростью,
    вчетверо меньшей скорости света в вакууме.
    Чему будет равен предельный угол
    внутреннего отражения для поверхности раздела этого вещества с водой?
}
\answer{%
    $n_1 = \frac c v = 4, n_1 \sin \varphi_\text{п.в.о.} = n_2 \sin \frac \pi 2 \implies \varphi_\text{п.в.о.} \approx 19{,}4\degrees.$
}
\solutionspace{80pt}

\tasknumber{6}%
\task{%
    Луч света падает на горизонтально расположенную стеклянную пластинку толщиной $1{,}6\,\text{см}$.
    Пройдя через пластину, он выходит из неё в точке, смещённой по горизонтали от точки падения на расстояние $4\,\text{мм}$.
    Показатель преломления стекла $1{,}5$.
    Найти синус угла падения.
}
\answer{%
    \begin{align*}
    \ctg \beta &= \frac{h}{d} \implies \\
    \implies \frac 1{\sin^2 \beta} &= \ctg^2 \beta + 1 = \sqr{\frac{h}{d}} + 1 \implies \\
    \implies \sin\alpha &= n\sin \beta = n\sqrt {\frac 1{\sqr{\frac{h}{d}} + 1}} \approx 0{,}36
    \end{align*}
}

\variantsplitter

\addpersonalvariant{Леонид Никитин}

\tasknumber{1}%
\task{%
    Для закона отражения:
    \begin{itemize}
        \item сделайте рисунок,
        \item отметьте все необходимые углы и подпишите их названия,
        \item запишите этот закон формулой.
    \end{itemize}
}
\solutionspace{80pt}

\tasknumber{2}%
\task{%
    На плоскопараллельную стеклянную пластинку под углом $35\degrees$
    падают два параллельных луча света, расстояние между которыми $4\,\text{см}$.
    Определите расстояние между точками, в которых эти лучи выходят из пластинки.
}
\answer{%
    $\ell = \frac{d}{\cos \alpha} \approx 49\,\text{мм}$
}
\solutionspace{80pt}

\tasknumber{3}%
\task{%
    Солнце составляет с горизонтом угол, синус которого  0{,}6 .
    Шест высотой $150\,\text{см}$ вбит в дно водоёма глубиной $90\,\text{см}$.
    Найдите длину тени от этого шеста на дне водоёма, если показатель преломления воды 1{,}33.
}
\answer{%
    $
        n \sin \beta = 1 \cdot \cos \alpha \implies \beta \approx 37{,}0\degrees,
        L = (H - h)\ctg \alpha + h \tg \beta \approx 80\,\text{см} + 67{,}8\,\text{см} \approx 147{,}8\,\text{см}.
    $
}
\solutionspace{80pt}

\tasknumber{4}%
\task{%
    Луч света падает на плоское зеркало под углом, синус которого  0{,}75 .
    На сколько миллиметров сместится отражённый луч,
    если на зеркало положить прозрачную пластину толщиной $18\,\text{мм}$ с показателем преломления $1{,}3$?
}
\answer{%
    $1 \cdot \sin \alpha = n \cdot \sin \beta \implies \beta \approx 35{,}2\degrees, L = \cbr{2 d \tg \alpha - 2 d \tg \beta} \cdot \cos \alpha \approx 10{,}2\,\text{мм}$
}
\solutionspace{80pt}

\tasknumber{5}%
\task{%
    В некотором прозрачном веществе свет распространяется со скоростью,
    вчетверо меньшей скорости света в вакууме.
    Чему будет равен предельный угол
    внутреннего отражения для поверхности раздела этого вещества с водой?
}
\answer{%
    $n_1 = \frac c v = 4, n_1 \sin \varphi_\text{п.в.о.} = n_2 \sin \frac \pi 2 \implies \varphi_\text{п.в.о.} \approx 19{,}4\degrees.$
}
\solutionspace{80pt}

\tasknumber{6}%
\task{%
    Луч света падает на горизонтально расположенную стеклянную пластинку толщиной $1{,}4\,\text{см}$.
    Пройдя через пластину, он выходит из неё в точке, смещённой по горизонтали от точки падения на расстояние $6\,\text{мм}$.
    Показатель преломления стекла $1{,}5$.
    Найти синус угла падения.
}
\answer{%
    \begin{align*}
    \ctg \beta &= \frac{h}{d} \implies \\
    \implies \frac 1{\sin^2 \beta} &= \ctg^2 \beta + 1 = \sqr{\frac{h}{d}} + 1 \implies \\
    \implies \sin\alpha &= n\sin \beta = n\sqrt {\frac 1{\sqr{\frac{h}{d}} + 1}} \approx 0{,}59
    \end{align*}
}

\variantsplitter

\addpersonalvariant{Тимофей Полетаев}

\tasknumber{1}%
\task{%
    Для закона отражения:
    \begin{itemize}
        \item сделайте рисунок,
        \item отметьте все необходимые углы и подпишите их названия,
        \item запишите этот закон формулой.
    \end{itemize}
}
\solutionspace{80pt}

\tasknumber{2}%
\task{%
    На плоскопараллельную стеклянную пластинку под углом $50\degrees$
    падают два параллельных луча света, расстояние между которыми $4\,\text{см}$.
    Определите расстояние между точками, в которых эти лучи выходят из пластинки.
}
\answer{%
    $\ell = \frac{d}{\cos \alpha} \approx 62\,\text{мм}$
}
\solutionspace{80pt}

\tasknumber{3}%
\task{%
    Солнце составляет с горизонтом угол, синус которого  0{,}8 .
    Шест высотой $150\,\text{см}$ вбит в дно водоёма глубиной $90\,\text{см}$.
    Найдите длину тени от этого шеста на дне водоёма, если показатель преломления воды 1{,}33.
}
\answer{%
    $
        n \sin \beta = 1 \cdot \cos \alpha \implies \beta \approx 26{,}8\degrees,
        L = (H - h)\ctg \alpha + h \tg \beta \approx 45\,\text{см} + 45{,}5\,\text{см} \approx 90{,}5\,\text{см}.
    $
}
\solutionspace{80pt}

\tasknumber{4}%
\task{%
    Луч света падает на плоское зеркало под углом, синус которого  0{,}65 .
    На сколько миллиметров сместится отражённый луч,
    если на зеркало положить прозрачную пластину толщиной $19\,\text{мм}$ с показателем преломления $1{,}3$?
}
\answer{%
    $1 \cdot \sin \alpha = n \cdot \sin \beta \implies \beta \approx 30{,}0\degrees, L = \cbr{2 d \tg \alpha - 2 d \tg \beta} \cdot \cos \alpha \approx 8{,}0\,\text{мм}$
}
\solutionspace{80pt}

\tasknumber{5}%
\task{%
    В некотором прозрачном веществе свет распространяется со скоростью,
    вдвое меньшей скорости света в вакууме.
    Чему будет равен предельный угол
    внутреннего отражения для поверхности раздела этого вещества с воздухом?
}
\answer{%
    $n_1 = \frac c v = 2, n_1 \sin \varphi_\text{п.в.о.} = n_2 \sin \frac \pi 2 \implies \varphi_\text{п.в.о.} \approx 30{,}0\degrees.$
}
\solutionspace{80pt}

\tasknumber{6}%
\task{%
    Луч света падает на вертикально расположенную стеклянную пластинку толщиной $1{,}3\,\text{см}$.
    Пройдя через пластину, он выходит из неё в точке, смещённой по вертикали от точки падения на расстояние $6\,\text{мм}$.
    Показатель преломления стекла $1{,}6$.
    Найти синус угла падения.
}
\answer{%
    \begin{align*}
    \ctg \beta &= \frac{h}{d} \implies \\
    \implies \frac 1{\sin^2 \beta} &= \ctg^2 \beta + 1 = \sqr{\frac{h}{d}} + 1 \implies \\
    \implies \sin\alpha &= n\sin \beta = n\sqrt {\frac 1{\sqr{\frac{h}{d}} + 1}} \approx 0{,}67
    \end{align*}
}

\variantsplitter

\addpersonalvariant{Андрей Рожков}

\tasknumber{1}%
\task{%
    Для закона преломления:
    \begin{itemize}
        \item сделайте рисунок,
        \item отметьте все необходимые углы и подпишите их названия,
        \item запишите этот закон формулой.
    \end{itemize}
}
\solutionspace{80pt}

\tasknumber{2}%
\task{%
    На плоскопараллельную стеклянную пластинку под углом $40\degrees$
    падают два параллельных луча света, расстояние между которыми $4\,\text{см}$.
    Определите расстояние между точками, в которых эти лучи выходят из пластинки.
}
\answer{%
    $\ell = \frac{d}{\cos \alpha} \approx 52\,\text{мм}$
}
\solutionspace{80pt}

\tasknumber{3}%
\task{%
    Солнце составляет с горизонтом угол, синус которого  0{,}5 .
    Шест высотой $170\,\text{см}$ вбит в дно водоёма глубиной $90\,\text{см}$.
    Найдите длину тени от этого шеста на дне водоёма, если показатель преломления воды 1{,}33.
}
\answer{%
    $
        n \sin \beta = 1 \cdot \cos \alpha \implies \beta \approx 40{,}6\degrees,
        L = (H - h)\ctg \alpha + h \tg \beta \approx 138{,}6\,\text{см} + 77{,}2\,\text{см} \approx 215{,}8\,\text{см}.
    $
}
\solutionspace{80pt}

\tasknumber{4}%
\task{%
    Луч света падает на плоское зеркало под углом, синус которого  0{,}75 .
    На сколько миллиметров сместится отражённый луч,
    если на зеркало положить прозрачную пластину толщиной $13\,\text{мм}$ с показателем преломления $1{,}3$?
}
\answer{%
    $1 \cdot \sin \alpha = n \cdot \sin \beta \implies \beta \approx 35{,}2\degrees, L = \cbr{2 d \tg \alpha - 2 d \tg \beta} \cdot \cos \alpha \approx 7{,}4\,\text{мм}$
}
\solutionspace{80pt}

\tasknumber{5}%
\task{%
    В некотором прозрачном веществе свет распространяется со скоростью,
    втрое меньшей скорости света в вакууме.
    Чему будет равен предельный угол
    внутреннего отражения для поверхности раздела этого вещества с водой?
}
\answer{%
    $n_1 = \frac c v = 3, n_1 \sin \varphi_\text{п.в.о.} = n_2 \sin \frac \pi 2 \implies \varphi_\text{п.в.о.} \approx 26{,}3\degrees.$
}
\solutionspace{80pt}

\tasknumber{6}%
\task{%
    Луч света падает на вертикально расположенную стеклянную пластинку толщиной $1{,}4\,\text{см}$.
    Пройдя через пластину, он выходит из неё в точке, смещённой по вертикали от точки падения на расстояние $4\,\text{мм}$.
    Показатель преломления стекла $1{,}6$.
    Найти синус угла падения.
}
\answer{%
    \begin{align*}
    \ctg \beta &= \frac{h}{d} \implies \\
    \implies \frac 1{\sin^2 \beta} &= \ctg^2 \beta + 1 = \sqr{\frac{h}{d}} + 1 \implies \\
    \implies \sin\alpha &= n\sin \beta = n\sqrt {\frac 1{\sqr{\frac{h}{d}} + 1}} \approx 0{,}44
    \end{align*}
}

\variantsplitter

\addpersonalvariant{Рената Таржиманова}

\tasknumber{1}%
\task{%
    Для закона преломления:
    \begin{itemize}
        \item сделайте рисунок,
        \item отметьте все необходимые углы и подпишите их названия,
        \item запишите этот закон формулой.
    \end{itemize}
}
\solutionspace{80pt}

\tasknumber{2}%
\task{%
    На плоскопараллельную стеклянную пластинку под углом $55\degrees$
    падают два параллельных луча света, расстояние между которыми $3\,\text{см}$.
    Определите расстояние между точками, в которых эти лучи выходят из пластинки.
}
\answer{%
    $\ell = \frac{d}{\cos \alpha} \approx 52\,\text{мм}$
}
\solutionspace{80pt}

\tasknumber{3}%
\task{%
    Солнце составляет с горизонтом угол, синус которого  0{,}6 .
    Шест высотой $150\,\text{см}$ вбит в дно водоёма глубиной $80\,\text{см}$.
    Найдите длину тени от этого шеста на дне водоёма, если показатель преломления воды 1{,}33.
}
\answer{%
    $
        n \sin \beta = 1 \cdot \cos \alpha \implies \beta \approx 37{,}0\degrees,
        L = (H - h)\ctg \alpha + h \tg \beta \approx 93{,}3\,\text{см} + 60{,}2\,\text{см} \approx 153{,}6\,\text{см}.
    $
}
\solutionspace{80pt}

\tasknumber{4}%
\task{%
    Луч света падает на плоское зеркало под углом, синус которого  0{,}85 .
    На сколько миллиметров сместится отражённый луч,
    если на зеркало положить прозрачную пластину толщиной $15\,\text{мм}$ с показателем преломления $1{,}6$?
}
\answer{%
    $1 \cdot \sin \alpha = n \cdot \sin \beta \implies \beta \approx 32{,}1\degrees, L = \cbr{2 d \tg \alpha - 2 d \tg \beta} \cdot \cos \alpha \approx 15{,}6\,\text{мм}$
}
\solutionspace{80pt}

\tasknumber{5}%
\task{%
    В некотором прозрачном веществе свет распространяется со скоростью,
    вчетверо меньшей скорости света в вакууме.
    Чему будет равен предельный угол
    внутреннего отражения для поверхности раздела этого вещества с водой?
}
\answer{%
    $n_1 = \frac c v = 4, n_1 \sin \varphi_\text{п.в.о.} = n_2 \sin \frac \pi 2 \implies \varphi_\text{п.в.о.} \approx 19{,}4\degrees.$
}
\solutionspace{80pt}

\tasknumber{6}%
\task{%
    Луч света падает на вертикально расположенную стеклянную пластинку толщиной $1{,}3\,\text{см}$.
    Пройдя через пластину, он выходит из неё в точке, смещённой по вертикали от точки падения на расстояние $5\,\text{мм}$.
    Показатель преломления стекла $1{,}5$.
    Найти синус угла падения.
}
\answer{%
    \begin{align*}
    \ctg \beta &= \frac{h}{d} \implies \\
    \implies \frac 1{\sin^2 \beta} &= \ctg^2 \beta + 1 = \sqr{\frac{h}{d}} + 1 \implies \\
    \implies \sin\alpha &= n\sin \beta = n\sqrt {\frac 1{\sqr{\frac{h}{d}} + 1}} \approx 0{,}54
    \end{align*}
}

\variantsplitter

\addpersonalvariant{Андрей Щербаков}

\tasknumber{1}%
\task{%
    Для закона преломления:
    \begin{itemize}
        \item сделайте рисунок,
        \item отметьте все необходимые углы и подпишите их названия,
        \item запишите этот закон формулой.
    \end{itemize}
}
\solutionspace{80pt}

\tasknumber{2}%
\task{%
    На плоскопараллельную стеклянную пластинку под углом $35\degrees$
    падают два параллельных луча света, расстояние между которыми $5\,\text{см}$.
    Определите расстояние между точками, в которых эти лучи выходят из пластинки.
}
\answer{%
    $\ell = \frac{d}{\cos \alpha} \approx 61\,\text{мм}$
}
\solutionspace{80pt}

\tasknumber{3}%
\task{%
    Солнце составляет с горизонтом угол, синус которого  0{,}7 .
    Шест высотой $150\,\text{см}$ вбит в дно водоёма глубиной $90\,\text{см}$.
    Найдите длину тени от этого шеста на дне водоёма, если показатель преломления воды 1{,}33.
}
\answer{%
    $
        n \sin \beta = 1 \cdot \cos \alpha \implies \beta \approx 32{,}5\degrees,
        L = (H - h)\ctg \alpha + h \tg \beta \approx 61{,}2\,\text{см} + 57{,}3\,\text{см} \approx 118{,}5\,\text{см}.
    $
}
\solutionspace{80pt}

\tasknumber{4}%
\task{%
    Луч света падает на плоское зеркало под углом, синус которого  0{,}65 .
    На сколько миллиметров сместится отражённый луч,
    если на зеркало положить прозрачную пластину толщиной $17\,\text{мм}$ с показателем преломления $1{,}45$?
}
\answer{%
    $1 \cdot \sin \alpha = n \cdot \sin \beta \implies \beta \approx 26{,}6\degrees, L = \cbr{2 d \tg \alpha - 2 d \tg \beta} \cdot \cos \alpha \approx 9{,}1\,\text{мм}$
}
\solutionspace{80pt}

\tasknumber{5}%
\task{%
    В некотором прозрачном веществе свет распространяется со скоростью,
    вдвое меньшей скорости света в вакууме.
    Чему будет равен предельный угол
    внутреннего отражения для поверхности раздела этого вещества с водой?
}
\answer{%
    $n_1 = \frac c v = 2, n_1 \sin \varphi_\text{п.в.о.} = n_2 \sin \frac \pi 2 \implies \varphi_\text{п.в.о.} \approx 41{,}7\degrees.$
}
\solutionspace{80pt}

\tasknumber{6}%
\task{%
    Луч света падает на вертикально расположенную стеклянную пластинку толщиной $1{,}2\,\text{см}$.
    Пройдя через пластину, он выходит из неё в точке, смещённой по вертикали от точки падения на расстояние $4\,\text{мм}$.
    Показатель преломления стекла $1{,}4$.
    Найти синус угла падения.
}
\answer{%
    \begin{align*}
    \ctg \beta &= \frac{h}{d} \implies \\
    \implies \frac 1{\sin^2 \beta} &= \ctg^2 \beta + 1 = \sqr{\frac{h}{d}} + 1 \implies \\
    \implies \sin\alpha &= n\sin \beta = n\sqrt {\frac 1{\sqr{\frac{h}{d}} + 1}} \approx 0{,}44
    \end{align*}
}

\variantsplitter

\addpersonalvariant{Михаил Ярошевский}

\tasknumber{1}%
\task{%
    Для закона отражения:
    \begin{itemize}
        \item сделайте рисунок,
        \item отметьте все необходимые углы и подпишите их названия,
        \item запишите этот закон формулой.
    \end{itemize}
}
\solutionspace{80pt}

\tasknumber{2}%
\task{%
    На плоскопараллельную стеклянную пластинку под углом $40\degrees$
    падают два параллельных луча света, расстояние между которыми $4\,\text{см}$.
    Определите расстояние между точками, в которых эти лучи выходят из пластинки.
}
\answer{%
    $\ell = \frac{d}{\cos \alpha} \approx 52\,\text{мм}$
}
\solutionspace{80pt}

\tasknumber{3}%
\task{%
    Солнце составляет с горизонтом угол, синус которого  0{,}7 .
    Шест высотой $160\,\text{см}$ вбит в дно водоёма глубиной $80\,\text{см}$.
    Найдите длину тени от этого шеста на дне водоёма, если показатель преломления воды 1{,}33.
}
\answer{%
    $
        n \sin \beta = 1 \cdot \cos \alpha \implies \beta \approx 32{,}5\degrees,
        L = (H - h)\ctg \alpha + h \tg \beta \approx 81{,}6\,\text{см} + 50{,}9\,\text{см} \approx 132{,}5\,\text{см}.
    $
}
\solutionspace{80pt}

\tasknumber{4}%
\task{%
    Луч света падает на плоское зеркало под углом, синус которого  0{,}75 .
    На сколько миллиметров сместится отражённый луч,
    если на зеркало положить прозрачную пластину толщиной $17\,\text{мм}$ с показателем преломления $1{,}4$?
}
\answer{%
    $1 \cdot \sin \alpha = n \cdot \sin \beta \implies \beta \approx 32{,}4\degrees, L = \cbr{2 d \tg \alpha - 2 d \tg \beta} \cdot \cos \alpha \approx 11{,}2\,\text{мм}$
}
\solutionspace{80pt}

\tasknumber{5}%
\task{%
    В некотором прозрачном веществе свет распространяется со скоростью,
    вдвое меньшей скорости света в вакууме.
    Чему будет равен предельный угол
    внутреннего отражения для поверхности раздела этого вещества с водой?
}
\answer{%
    $n_1 = \frac c v = 2, n_1 \sin \varphi_\text{п.в.о.} = n_2 \sin \frac \pi 2 \implies \varphi_\text{п.в.о.} \approx 41{,}7\degrees.$
}
\solutionspace{80pt}

\tasknumber{6}%
\task{%
    Луч света падает на горизонтально расположенную стеклянную пластинку толщиной $1{,}2\,\text{см}$.
    Пройдя через пластину, он выходит из неё в точке, смещённой по горизонтали от точки падения на расстояние $5\,\text{мм}$.
    Показатель преломления стекла $1{,}6$.
    Найти синус угла падения.
}
\answer{%
    \begin{align*}
    \ctg \beta &= \frac{h}{d} \implies \\
    \implies \frac 1{\sin^2 \beta} &= \ctg^2 \beta + 1 = \sqr{\frac{h}{d}} + 1 \implies \\
    \implies \sin\alpha &= n\sin \beta = n\sqrt {\frac 1{\sqr{\frac{h}{d}} + 1}} \approx 0{,}62
    \end{align*}
}

\variantsplitter

\addpersonalvariant{Алексей Алимпиев}

\tasknumber{1}%
\task{%
    Для закона преломления:
    \begin{itemize}
        \item сделайте рисунок,
        \item отметьте все необходимые углы и подпишите их названия,
        \item запишите этот закон формулой.
    \end{itemize}
}
\solutionspace{80pt}

\tasknumber{2}%
\task{%
    На плоскопараллельную стеклянную пластинку под углом $50\degrees$
    падают два параллельных луча света, расстояние между которыми $8\,\text{см}$.
    Определите расстояние между точками, в которых эти лучи выходят из пластинки.
}
\answer{%
    $\ell = \frac{d}{\cos \alpha} \approx 124\,\text{мм}$
}
\solutionspace{80pt}

\tasknumber{3}%
\task{%
    Солнце составляет с горизонтом угол, синус которого  0{,}5 .
    Шест высотой $130\,\text{см}$ вбит в дно водоёма глубиной $80\,\text{см}$.
    Найдите длину тени от этого шеста на дне водоёма, если показатель преломления воды 1{,}33.
}
\answer{%
    $
        n \sin \beta = 1 \cdot \cos \alpha \implies \beta \approx 40{,}6\degrees,
        L = (H - h)\ctg \alpha + h \tg \beta \approx 86{,}6\,\text{см} + 68{,}6\,\text{см} \approx 155{,}2\,\text{см}.
    $
}
\solutionspace{80pt}

\tasknumber{4}%
\task{%
    Луч света падает на плоское зеркало под углом, синус которого  0{,}85 .
    На сколько миллиметров сместится отражённый луч,
    если на зеркало положить прозрачную пластину толщиной $18\,\text{мм}$ с показателем преломления $1{,}55$?
}
\answer{%
    $1 \cdot \sin \alpha = n \cdot \sin \beta \implies \beta \approx 33{,}3\degrees, L = \cbr{2 d \tg \alpha - 2 d \tg \beta} \cdot \cos \alpha \approx 18{,}2\,\text{мм}$
}
\solutionspace{80pt}

\tasknumber{5}%
\task{%
    В некотором прозрачном веществе свет распространяется со скоростью,
    втрое меньшей скорости света в вакууме.
    Чему будет равен предельный угол
    внутреннего отражения для поверхности раздела этого вещества с воздухом?
}
\answer{%
    $n_1 = \frac c v = 3, n_1 \sin \varphi_\text{п.в.о.} = n_2 \sin \frac \pi 2 \implies \varphi_\text{п.в.о.} \approx 19{,}5\degrees.$
}
\solutionspace{80pt}

\tasknumber{6}%
\task{%
    Луч света падает на горизонтально расположенную стеклянную пластинку толщиной $1{,}2\,\text{см}$.
    Пройдя через пластину, он выходит из неё в точке, смещённой по горизонтали от точки падения на расстояние $5\,\text{мм}$.
    Показатель преломления стекла $1{,}6$.
    Найти синус угла падения.
}
\answer{%
    \begin{align*}
    \ctg \beta &= \frac{h}{d} \implies \\
    \implies \frac 1{\sin^2 \beta} &= \ctg^2 \beta + 1 = \sqr{\frac{h}{d}} + 1 \implies \\
    \implies \sin\alpha &= n\sin \beta = n\sqrt {\frac 1{\sqr{\frac{h}{d}} + 1}} \approx 0{,}62
    \end{align*}
}

\variantsplitter

\addpersonalvariant{Евгений Васин}

\tasknumber{1}%
\task{%
    Для закона отражения:
    \begin{itemize}
        \item сделайте рисунок,
        \item отметьте все необходимые углы и подпишите их названия,
        \item запишите этот закон формулой.
    \end{itemize}
}
\solutionspace{80pt}

\tasknumber{2}%
\task{%
    На плоскопараллельную стеклянную пластинку под углом $40\degrees$
    падают два параллельных луча света, расстояние между которыми $5\,\text{см}$.
    Определите расстояние между точками, в которых эти лучи выходят из пластинки.
}
\answer{%
    $\ell = \frac{d}{\cos \alpha} \approx 65\,\text{мм}$
}
\solutionspace{80pt}

\tasknumber{3}%
\task{%
    Солнце составляет с горизонтом угол, синус которого  0{,}7 .
    Шест высотой $180\,\text{см}$ вбит в дно водоёма глубиной $90\,\text{см}$.
    Найдите длину тени от этого шеста на дне водоёма, если показатель преломления воды 1{,}33.
}
\answer{%
    $
        n \sin \beta = 1 \cdot \cos \alpha \implies \beta \approx 32{,}5\degrees,
        L = (H - h)\ctg \alpha + h \tg \beta \approx 91{,}8\,\text{см} + 57{,}3\,\text{см} \approx 149{,}1\,\text{см}.
    $
}
\solutionspace{80pt}

\tasknumber{4}%
\task{%
    Луч света падает на плоское зеркало под углом, синус которого  0{,}85 .
    На сколько миллиметров сместится отражённый луч,
    если на зеркало положить прозрачную пластину толщиной $13\,\text{мм}$ с показателем преломления $1{,}55$?
}
\answer{%
    $1 \cdot \sin \alpha = n \cdot \sin \beta \implies \beta \approx 33{,}3\degrees, L = \cbr{2 d \tg \alpha - 2 d \tg \beta} \cdot \cos \alpha \approx 13{,}1\,\text{мм}$
}
\solutionspace{80pt}

\tasknumber{5}%
\task{%
    В некотором прозрачном веществе свет распространяется со скоростью,
    втрое меньшей скорости света в вакууме.
    Чему будет равен предельный угол
    внутреннего отражения для поверхности раздела этого вещества с водой?
}
\answer{%
    $n_1 = \frac c v = 3, n_1 \sin \varphi_\text{п.в.о.} = n_2 \sin \frac \pi 2 \implies \varphi_\text{п.в.о.} \approx 26{,}3\degrees.$
}
\solutionspace{80pt}

\tasknumber{6}%
\task{%
    Луч света падает на вертикально расположенную стеклянную пластинку толщиной $1{,}2\,\text{см}$.
    Пройдя через пластину, он выходит из неё в точке, смещённой по вертикали от точки падения на расстояние $6\,\text{мм}$.
    Показатель преломления стекла $1{,}6$.
    Найти синус угла падения.
}
\answer{%
    \begin{align*}
    \ctg \beta &= \frac{h}{d} \implies \\
    \implies \frac 1{\sin^2 \beta} &= \ctg^2 \beta + 1 = \sqr{\frac{h}{d}} + 1 \implies \\
    \implies \sin\alpha &= n\sin \beta = n\sqrt {\frac 1{\sqr{\frac{h}{d}} + 1}} \approx 0{,}72
    \end{align*}
}

\variantsplitter

\addpersonalvariant{Вячеслав Волохов}

\tasknumber{1}%
\task{%
    Для закона преломления:
    \begin{itemize}
        \item сделайте рисунок,
        \item отметьте все необходимые углы и подпишите их названия,
        \item запишите этот закон формулой.
    \end{itemize}
}
\solutionspace{80pt}

\tasknumber{2}%
\task{%
    На плоскопараллельную стеклянную пластинку под углом $35\degrees$
    падают два параллельных луча света, расстояние между которыми $8\,\text{см}$.
    Определите расстояние между точками, в которых эти лучи выходят из пластинки.
}
\answer{%
    $\ell = \frac{d}{\cos \alpha} \approx 98\,\text{мм}$
}
\solutionspace{80pt}

\tasknumber{3}%
\task{%
    Солнце составляет с горизонтом угол, синус которого  0{,}6 .
    Шест высотой $160\,\text{см}$ вбит в дно водоёма глубиной $80\,\text{см}$.
    Найдите длину тени от этого шеста на дне водоёма, если показатель преломления воды 1{,}33.
}
\answer{%
    $
        n \sin \beta = 1 \cdot \cos \alpha \implies \beta \approx 37{,}0\degrees,
        L = (H - h)\ctg \alpha + h \tg \beta \approx 106{,}7\,\text{см} + 60{,}2\,\text{см} \approx 166{,}9\,\text{см}.
    $
}
\solutionspace{80pt}

\tasknumber{4}%
\task{%
    Луч света падает на плоское зеркало под углом, синус которого  0{,}75 .
    На сколько миллиметров сместится отражённый луч,
    если на зеркало положить прозрачную пластину толщиной $19\,\text{мм}$ с показателем преломления $1{,}35$?
}
\answer{%
    $1 \cdot \sin \alpha = n \cdot \sin \beta \implies \beta \approx 33{,}7\degrees, L = \cbr{2 d \tg \alpha - 2 d \tg \beta} \cdot \cos \alpha \approx 11{,}7\,\text{мм}$
}
\solutionspace{80pt}

\tasknumber{5}%
\task{%
    В некотором прозрачном веществе свет распространяется со скоростью,
    вчетверо меньшей скорости света в вакууме.
    Чему будет равен предельный угол
    внутреннего отражения для поверхности раздела этого вещества с воздухом?
}
\answer{%
    $n_1 = \frac c v = 4, n_1 \sin \varphi_\text{п.в.о.} = n_2 \sin \frac \pi 2 \implies \varphi_\text{п.в.о.} \approx 14{,}5\degrees.$
}
\solutionspace{80pt}

\tasknumber{6}%
\task{%
    Луч света падает на вертикально расположенную стеклянную пластинку толщиной $1{,}6\,\text{см}$.
    Пройдя через пластину, он выходит из неё в точке, смещённой по вертикали от точки падения на расстояние $6\,\text{мм}$.
    Показатель преломления стекла $1{,}6$.
    Найти синус угла падения.
}
\answer{%
    \begin{align*}
    \ctg \beta &= \frac{h}{d} \implies \\
    \implies \frac 1{\sin^2 \beta} &= \ctg^2 \beta + 1 = \sqr{\frac{h}{d}} + 1 \implies \\
    \implies \sin\alpha &= n\sin \beta = n\sqrt {\frac 1{\sqr{\frac{h}{d}} + 1}} \approx 0{,}56
    \end{align*}
}

\variantsplitter

\addpersonalvariant{Герман Говоров}

\tasknumber{1}%
\task{%
    Для закона отражения:
    \begin{itemize}
        \item сделайте рисунок,
        \item отметьте все необходимые углы и подпишите их названия,
        \item запишите этот закон формулой.
    \end{itemize}
}
\solutionspace{80pt}

\tasknumber{2}%
\task{%
    На плоскопараллельную стеклянную пластинку под углом $50\degrees$
    падают два параллельных луча света, расстояние между которыми $4\,\text{см}$.
    Определите расстояние между точками, в которых эти лучи выходят из пластинки.
}
\answer{%
    $\ell = \frac{d}{\cos \alpha} \approx 62\,\text{мм}$
}
\solutionspace{80pt}

\tasknumber{3}%
\task{%
    Солнце составляет с горизонтом угол, синус которого  0{,}7 .
    Шест высотой $160\,\text{см}$ вбит в дно водоёма глубиной $80\,\text{см}$.
    Найдите длину тени от этого шеста на дне водоёма, если показатель преломления воды 1{,}33.
}
\answer{%
    $
        n \sin \beta = 1 \cdot \cos \alpha \implies \beta \approx 32{,}5\degrees,
        L = (H - h)\ctg \alpha + h \tg \beta \approx 81{,}6\,\text{см} + 50{,}9\,\text{см} \approx 132{,}5\,\text{см}.
    $
}
\solutionspace{80pt}

\tasknumber{4}%
\task{%
    Луч света падает на плоское зеркало под углом, синус которого  0{,}85 .
    На сколько миллиметров сместится отражённый луч,
    если на зеркало положить прозрачную пластину толщиной $17\,\text{мм}$ с показателем преломления $1{,}55$?
}
\answer{%
    $1 \cdot \sin \alpha = n \cdot \sin \beta \implies \beta \approx 33{,}3\degrees, L = \cbr{2 d \tg \alpha - 2 d \tg \beta} \cdot \cos \alpha \approx 17{,}2\,\text{мм}$
}
\solutionspace{80pt}

\tasknumber{5}%
\task{%
    В некотором прозрачном веществе свет распространяется со скоростью,
    вдвое меньшей скорости света в вакууме.
    Чему будет равен предельный угол
    внутреннего отражения для поверхности раздела этого вещества с воздухом?
}
\answer{%
    $n_1 = \frac c v = 2, n_1 \sin \varphi_\text{п.в.о.} = n_2 \sin \frac \pi 2 \implies \varphi_\text{п.в.о.} \approx 30{,}0\degrees.$
}
\solutionspace{80pt}

\tasknumber{6}%
\task{%
    Луч света падает на горизонтально расположенную стеклянную пластинку толщиной $1{,}3\,\text{см}$.
    Пройдя через пластину, он выходит из неё в точке, смещённой по горизонтали от точки падения на расстояние $5\,\text{мм}$.
    Показатель преломления стекла $1{,}4$.
    Найти синус угла падения.
}
\answer{%
    \begin{align*}
    \ctg \beta &= \frac{h}{d} \implies \\
    \implies \frac 1{\sin^2 \beta} &= \ctg^2 \beta + 1 = \sqr{\frac{h}{d}} + 1 \implies \\
    \implies \sin\alpha &= n\sin \beta = n\sqrt {\frac 1{\sqr{\frac{h}{d}} + 1}} \approx 0{,}50
    \end{align*}
}

\variantsplitter

\addpersonalvariant{София Журавлёва}

\tasknumber{1}%
\task{%
    Для закона отражения:
    \begin{itemize}
        \item сделайте рисунок,
        \item отметьте все необходимые углы и подпишите их названия,
        \item запишите этот закон формулой.
    \end{itemize}
}
\solutionspace{80pt}

\tasknumber{2}%
\task{%
    На плоскопараллельную стеклянную пластинку под углом $40\degrees$
    падают два параллельных луча света, расстояние между которыми $8\,\text{см}$.
    Определите расстояние между точками, в которых эти лучи выходят из пластинки.
}
\answer{%
    $\ell = \frac{d}{\cos \alpha} \approx 104\,\text{мм}$
}
\solutionspace{80pt}

\tasknumber{3}%
\task{%
    Солнце составляет с горизонтом угол, синус которого  0{,}6 .
    Шест высотой $130\,\text{см}$ вбит в дно водоёма глубиной $90\,\text{см}$.
    Найдите длину тени от этого шеста на дне водоёма, если показатель преломления воды 1{,}33.
}
\answer{%
    $
        n \sin \beta = 1 \cdot \cos \alpha \implies \beta \approx 37{,}0\degrees,
        L = (H - h)\ctg \alpha + h \tg \beta \approx 53{,}3\,\text{см} + 67{,}8\,\text{см} \approx 121{,}1\,\text{см}.
    $
}
\solutionspace{80pt}

\tasknumber{4}%
\task{%
    Луч света падает на плоское зеркало под углом, синус которого  0{,}65 .
    На сколько миллиметров сместится отражённый луч,
    если на зеркало положить прозрачную пластину толщиной $14\,\text{мм}$ с показателем преломления $1{,}3$?
}
\answer{%
    $1 \cdot \sin \alpha = n \cdot \sin \beta \implies \beta \approx 30{,}0\degrees, L = \cbr{2 d \tg \alpha - 2 d \tg \beta} \cdot \cos \alpha \approx 5{,}9\,\text{мм}$
}
\solutionspace{80pt}

\tasknumber{5}%
\task{%
    В некотором прозрачном веществе свет распространяется со скоростью,
    втрое меньшей скорости света в вакууме.
    Чему будет равен предельный угол
    внутреннего отражения для поверхности раздела этого вещества с воздухом?
}
\answer{%
    $n_1 = \frac c v = 3, n_1 \sin \varphi_\text{п.в.о.} = n_2 \sin \frac \pi 2 \implies \varphi_\text{п.в.о.} \approx 19{,}5\degrees.$
}
\solutionspace{80pt}

\tasknumber{6}%
\task{%
    Луч света падает на горизонтально расположенную стеклянную пластинку толщиной $1{,}4\,\text{см}$.
    Пройдя через пластину, он выходит из неё в точке, смещённой по горизонтали от точки падения на расстояние $6\,\text{мм}$.
    Показатель преломления стекла $1{,}6$.
    Найти синус угла падения.
}
\answer{%
    \begin{align*}
    \ctg \beta &= \frac{h}{d} \implies \\
    \implies \frac 1{\sin^2 \beta} &= \ctg^2 \beta + 1 = \sqr{\frac{h}{d}} + 1 \implies \\
    \implies \sin\alpha &= n\sin \beta = n\sqrt {\frac 1{\sqr{\frac{h}{d}} + 1}} \approx 0{,}63
    \end{align*}
}

\variantsplitter

\addpersonalvariant{Константин Козлов}

\tasknumber{1}%
\task{%
    Для закона отражения:
    \begin{itemize}
        \item сделайте рисунок,
        \item отметьте все необходимые углы и подпишите их названия,
        \item запишите этот закон формулой.
    \end{itemize}
}
\solutionspace{80pt}

\tasknumber{2}%
\task{%
    На плоскопараллельную стеклянную пластинку под углом $40\degrees$
    падают два параллельных луча света, расстояние между которыми $4\,\text{см}$.
    Определите расстояние между точками, в которых эти лучи выходят из пластинки.
}
\answer{%
    $\ell = \frac{d}{\cos \alpha} \approx 52\,\text{мм}$
}
\solutionspace{80pt}

\tasknumber{3}%
\task{%
    Солнце составляет с горизонтом угол, синус которого  0{,}5 .
    Шест высотой $150\,\text{см}$ вбит в дно водоёма глубиной $80\,\text{см}$.
    Найдите длину тени от этого шеста на дне водоёма, если показатель преломления воды 1{,}33.
}
\answer{%
    $
        n \sin \beta = 1 \cdot \cos \alpha \implies \beta \approx 40{,}6\degrees,
        L = (H - h)\ctg \alpha + h \tg \beta \approx 121{,}2\,\text{см} + 68{,}6\,\text{см} \approx 189{,}9\,\text{см}.
    $
}
\solutionspace{80pt}

\tasknumber{4}%
\task{%
    Луч света падает на плоское зеркало под углом, синус которого  0{,}85 .
    На сколько миллиметров сместится отражённый луч,
    если на зеркало положить прозрачную пластину толщиной $15\,\text{мм}$ с показателем преломления $1{,}45$?
}
\answer{%
    $1 \cdot \sin \alpha = n \cdot \sin \beta \implies \beta \approx 35{,}9\degrees, L = \cbr{2 d \tg \alpha - 2 d \tg \beta} \cdot \cos \alpha \approx 14{,}1\,\text{мм}$
}
\solutionspace{80pt}

\tasknumber{5}%
\task{%
    В некотором прозрачном веществе свет распространяется со скоростью,
    вчетверо меньшей скорости света в вакууме.
    Чему будет равен предельный угол
    внутреннего отражения для поверхности раздела этого вещества с воздухом?
}
\answer{%
    $n_1 = \frac c v = 4, n_1 \sin \varphi_\text{п.в.о.} = n_2 \sin \frac \pi 2 \implies \varphi_\text{п.в.о.} \approx 14{,}5\degrees.$
}
\solutionspace{80pt}

\tasknumber{6}%
\task{%
    Луч света падает на вертикально расположенную стеклянную пластинку толщиной $1{,}6\,\text{см}$.
    Пройдя через пластину, он выходит из неё в точке, смещённой по вертикали от точки падения на расстояние $4\,\text{мм}$.
    Показатель преломления стекла $1{,}5$.
    Найти синус угла падения.
}
\answer{%
    \begin{align*}
    \ctg \beta &= \frac{h}{d} \implies \\
    \implies \frac 1{\sin^2 \beta} &= \ctg^2 \beta + 1 = \sqr{\frac{h}{d}} + 1 \implies \\
    \implies \sin\alpha &= n\sin \beta = n\sqrt {\frac 1{\sqr{\frac{h}{d}} + 1}} \approx 0{,}36
    \end{align*}
}

\variantsplitter

\addpersonalvariant{Наталья Кравченко}

\tasknumber{1}%
\task{%
    Для закона отражения:
    \begin{itemize}
        \item сделайте рисунок,
        \item отметьте все необходимые углы и подпишите их названия,
        \item запишите этот закон формулой.
    \end{itemize}
}
\solutionspace{80pt}

\tasknumber{2}%
\task{%
    На плоскопараллельную стеклянную пластинку под углом $55\degrees$
    падают два параллельных луча света, расстояние между которыми $7\,\text{см}$.
    Определите расстояние между точками, в которых эти лучи выходят из пластинки.
}
\answer{%
    $\ell = \frac{d}{\cos \alpha} \approx 122\,\text{мм}$
}
\solutionspace{80pt}

\tasknumber{3}%
\task{%
    Солнце составляет с горизонтом угол, синус которого  0{,}5 .
    Шест высотой $160\,\text{см}$ вбит в дно водоёма глубиной $80\,\text{см}$.
    Найдите длину тени от этого шеста на дне водоёма, если показатель преломления воды 1{,}33.
}
\answer{%
    $
        n \sin \beta = 1 \cdot \cos \alpha \implies \beta \approx 40{,}6\degrees,
        L = (H - h)\ctg \alpha + h \tg \beta \approx 138{,}6\,\text{см} + 68{,}6\,\text{см} \approx 207{,}2\,\text{см}.
    $
}
\solutionspace{80pt}

\tasknumber{4}%
\task{%
    Луч света падает на плоское зеркало под углом, синус которого  0{,}85 .
    На сколько миллиметров сместится отражённый луч,
    если на зеркало положить прозрачную пластину толщиной $19\,\text{мм}$ с показателем преломления $1{,}3$?
}
\answer{%
    $1 \cdot \sin \alpha = n \cdot \sin \beta \implies \beta \approx 40{,}8\degrees, L = \cbr{2 d \tg \alpha - 2 d \tg \beta} \cdot \cos \alpha \approx 15{,}0\,\text{мм}$
}
\solutionspace{80pt}

\tasknumber{5}%
\task{%
    В некотором прозрачном веществе свет распространяется со скоростью,
    вдвое меньшей скорости света в вакууме.
    Чему будет равен предельный угол
    внутреннего отражения для поверхности раздела этого вещества с водой?
}
\answer{%
    $n_1 = \frac c v = 2, n_1 \sin \varphi_\text{п.в.о.} = n_2 \sin \frac \pi 2 \implies \varphi_\text{п.в.о.} \approx 41{,}7\degrees.$
}
\solutionspace{80pt}

\tasknumber{6}%
\task{%
    Луч света падает на вертикально расположенную стеклянную пластинку толщиной $1{,}3\,\text{см}$.
    Пройдя через пластину, он выходит из неё в точке, смещённой по вертикали от точки падения на расстояние $5\,\text{мм}$.
    Показатель преломления стекла $1{,}4$.
    Найти синус угла падения.
}
\answer{%
    \begin{align*}
    \ctg \beta &= \frac{h}{d} \implies \\
    \implies \frac 1{\sin^2 \beta} &= \ctg^2 \beta + 1 = \sqr{\frac{h}{d}} + 1 \implies \\
    \implies \sin\alpha &= n\sin \beta = n\sqrt {\frac 1{\sqr{\frac{h}{d}} + 1}} \approx 0{,}50
    \end{align*}
}

\variantsplitter

\addpersonalvariant{Матвей Кузьмин}

\tasknumber{1}%
\task{%
    Для закона отражения:
    \begin{itemize}
        \item сделайте рисунок,
        \item отметьте все необходимые углы и подпишите их названия,
        \item запишите этот закон формулой.
    \end{itemize}
}
\solutionspace{80pt}

\tasknumber{2}%
\task{%
    На плоскопараллельную стеклянную пластинку под углом $50\degrees$
    падают два параллельных луча света, расстояние между которыми $5\,\text{см}$.
    Определите расстояние между точками, в которых эти лучи выходят из пластинки.
}
\answer{%
    $\ell = \frac{d}{\cos \alpha} \approx 78\,\text{мм}$
}
\solutionspace{80pt}

\tasknumber{3}%
\task{%
    Солнце составляет с горизонтом угол, синус которого  0{,}6 .
    Шест высотой $160\,\text{см}$ вбит в дно водоёма глубиной $80\,\text{см}$.
    Найдите длину тени от этого шеста на дне водоёма, если показатель преломления воды 1{,}33.
}
\answer{%
    $
        n \sin \beta = 1 \cdot \cos \alpha \implies \beta \approx 37{,}0\degrees,
        L = (H - h)\ctg \alpha + h \tg \beta \approx 106{,}7\,\text{см} + 60{,}2\,\text{см} \approx 166{,}9\,\text{см}.
    $
}
\solutionspace{80pt}

\tasknumber{4}%
\task{%
    Луч света падает на плоское зеркало под углом, синус которого  0{,}85 .
    На сколько миллиметров сместится отражённый луч,
    если на зеркало положить прозрачную пластину толщиной $11\,\text{мм}$ с показателем преломления $1{,}35$?
}
\answer{%
    $1 \cdot \sin \alpha = n \cdot \sin \beta \implies \beta \approx 39{,}0\degrees, L = \cbr{2 d \tg \alpha - 2 d \tg \beta} \cdot \cos \alpha \approx 9{,}3\,\text{мм}$
}
\solutionspace{80pt}

\tasknumber{5}%
\task{%
    В некотором прозрачном веществе свет распространяется со скоростью,
    втрое меньшей скорости света в вакууме.
    Чему будет равен предельный угол
    внутреннего отражения для поверхности раздела этого вещества с воздухом?
}
\answer{%
    $n_1 = \frac c v = 3, n_1 \sin \varphi_\text{п.в.о.} = n_2 \sin \frac \pi 2 \implies \varphi_\text{п.в.о.} \approx 19{,}5\degrees.$
}
\solutionspace{80pt}

\tasknumber{6}%
\task{%
    Луч света падает на вертикально расположенную стеклянную пластинку толщиной $1{,}2\,\text{см}$.
    Пройдя через пластину, он выходит из неё в точке, смещённой по вертикали от точки падения на расстояние $6\,\text{мм}$.
    Показатель преломления стекла $1{,}4$.
    Найти синус угла падения.
}
\answer{%
    \begin{align*}
    \ctg \beta &= \frac{h}{d} \implies \\
    \implies \frac 1{\sin^2 \beta} &= \ctg^2 \beta + 1 = \sqr{\frac{h}{d}} + 1 \implies \\
    \implies \sin\alpha &= n\sin \beta = n\sqrt {\frac 1{\sqr{\frac{h}{d}} + 1}} \approx 0{,}63
    \end{align*}
}

\variantsplitter

\addpersonalvariant{Сергей Малышев}

\tasknumber{1}%
\task{%
    Для закона отражения:
    \begin{itemize}
        \item сделайте рисунок,
        \item отметьте все необходимые углы и подпишите их названия,
        \item запишите этот закон формулой.
    \end{itemize}
}
\solutionspace{80pt}

\tasknumber{2}%
\task{%
    На плоскопараллельную стеклянную пластинку под углом $40\degrees$
    падают два параллельных луча света, расстояние между которыми $8\,\text{см}$.
    Определите расстояние между точками, в которых эти лучи выходят из пластинки.
}
\answer{%
    $\ell = \frac{d}{\cos \alpha} \approx 104\,\text{мм}$
}
\solutionspace{80pt}

\tasknumber{3}%
\task{%
    Солнце составляет с горизонтом угол, синус которого  0{,}8 .
    Шест высотой $170\,\text{см}$ вбит в дно водоёма глубиной $80\,\text{см}$.
    Найдите длину тени от этого шеста на дне водоёма, если показатель преломления воды 1{,}33.
}
\answer{%
    $
        n \sin \beta = 1 \cdot \cos \alpha \implies \beta \approx 26{,}8\degrees,
        L = (H - h)\ctg \alpha + h \tg \beta \approx 67{,}5\,\text{см} + 40{,}4\,\text{см} \approx 107{,}9\,\text{см}.
    $
}
\solutionspace{80pt}

\tasknumber{4}%
\task{%
    Луч света падает на плоское зеркало под углом, синус которого  0{,}75 .
    На сколько миллиметров сместится отражённый луч,
    если на зеркало положить прозрачную пластину толщиной $11\,\text{мм}$ с показателем преломления $1{,}3$?
}
\answer{%
    $1 \cdot \sin \alpha = n \cdot \sin \beta \implies \beta \approx 35{,}2\degrees, L = \cbr{2 d \tg \alpha - 2 d \tg \beta} \cdot \cos \alpha \approx 6{,}2\,\text{мм}$
}
\solutionspace{80pt}

\tasknumber{5}%
\task{%
    В некотором прозрачном веществе свет распространяется со скоростью,
    вдвое меньшей скорости света в вакууме.
    Чему будет равен предельный угол
    внутреннего отражения для поверхности раздела этого вещества с водой?
}
\answer{%
    $n_1 = \frac c v = 2, n_1 \sin \varphi_\text{п.в.о.} = n_2 \sin \frac \pi 2 \implies \varphi_\text{п.в.о.} \approx 41{,}7\degrees.$
}
\solutionspace{80pt}

\tasknumber{6}%
\task{%
    Луч света падает на вертикально расположенную стеклянную пластинку толщиной $1{,}6\,\text{см}$.
    Пройдя через пластину, он выходит из неё в точке, смещённой по вертикали от точки падения на расстояние $4\,\text{мм}$.
    Показатель преломления стекла $1{,}5$.
    Найти синус угла падения.
}
\answer{%
    \begin{align*}
    \ctg \beta &= \frac{h}{d} \implies \\
    \implies \frac 1{\sin^2 \beta} &= \ctg^2 \beta + 1 = \sqr{\frac{h}{d}} + 1 \implies \\
    \implies \sin\alpha &= n\sin \beta = n\sqrt {\frac 1{\sqr{\frac{h}{d}} + 1}} \approx 0{,}36
    \end{align*}
}

\variantsplitter

\addpersonalvariant{Алина Полканова}

\tasknumber{1}%
\task{%
    Для закона преломления:
    \begin{itemize}
        \item сделайте рисунок,
        \item отметьте все необходимые углы и подпишите их названия,
        \item запишите этот закон формулой.
    \end{itemize}
}
\solutionspace{80pt}

\tasknumber{2}%
\task{%
    На плоскопараллельную стеклянную пластинку под углом $40\degrees$
    падают два параллельных луча света, расстояние между которыми $6\,\text{см}$.
    Определите расстояние между точками, в которых эти лучи выходят из пластинки.
}
\answer{%
    $\ell = \frac{d}{\cos \alpha} \approx 78\,\text{мм}$
}
\solutionspace{80pt}

\tasknumber{3}%
\task{%
    Солнце составляет с горизонтом угол, синус которого  0{,}5 .
    Шест высотой $160\,\text{см}$ вбит в дно водоёма глубиной $90\,\text{см}$.
    Найдите длину тени от этого шеста на дне водоёма, если показатель преломления воды 1{,}33.
}
\answer{%
    $
        n \sin \beta = 1 \cdot \cos \alpha \implies \beta \approx 40{,}6\degrees,
        L = (H - h)\ctg \alpha + h \tg \beta \approx 121{,}2\,\text{см} + 77{,}2\,\text{см} \approx 198{,}5\,\text{см}.
    $
}
\solutionspace{80pt}

\tasknumber{4}%
\task{%
    Луч света падает на плоское зеркало под углом, синус которого  0{,}75 .
    На сколько миллиметров сместится отражённый луч,
    если на зеркало положить прозрачную пластину толщиной $16\,\text{мм}$ с показателем преломления $1{,}6$?
}
\answer{%
    $1 \cdot \sin \alpha = n \cdot \sin \beta \implies \beta \approx 28{,}0\degrees, L = \cbr{2 d \tg \alpha - 2 d \tg \beta} \cdot \cos \alpha \approx 12{,}8\,\text{мм}$
}
\solutionspace{80pt}

\tasknumber{5}%
\task{%
    В некотором прозрачном веществе свет распространяется со скоростью,
    вчетверо меньшей скорости света в вакууме.
    Чему будет равен предельный угол
    внутреннего отражения для поверхности раздела этого вещества с воздухом?
}
\answer{%
    $n_1 = \frac c v = 4, n_1 \sin \varphi_\text{п.в.о.} = n_2 \sin \frac \pi 2 \implies \varphi_\text{п.в.о.} \approx 14{,}5\degrees.$
}
\solutionspace{80pt}

\tasknumber{6}%
\task{%
    Луч света падает на горизонтально расположенную стеклянную пластинку толщиной $1{,}3\,\text{см}$.
    Пройдя через пластину, он выходит из неё в точке, смещённой по горизонтали от точки падения на расстояние $4\,\text{мм}$.
    Показатель преломления стекла $1{,}5$.
    Найти синус угла падения.
}
\answer{%
    \begin{align*}
    \ctg \beta &= \frac{h}{d} \implies \\
    \implies \frac 1{\sin^2 \beta} &= \ctg^2 \beta + 1 = \sqr{\frac{h}{d}} + 1 \implies \\
    \implies \sin\alpha &= n\sin \beta = n\sqrt {\frac 1{\sqr{\frac{h}{d}} + 1}} \approx 0{,}44
    \end{align*}
}

\variantsplitter

\addpersonalvariant{Сергей Пономарёв}

\tasknumber{1}%
\task{%
    Для закона отражения:
    \begin{itemize}
        \item сделайте рисунок,
        \item отметьте все необходимые углы и подпишите их названия,
        \item запишите этот закон формулой.
    \end{itemize}
}
\solutionspace{80pt}

\tasknumber{2}%
\task{%
    На плоскопараллельную стеклянную пластинку под углом $40\degrees$
    падают два параллельных луча света, расстояние между которыми $3\,\text{см}$.
    Определите расстояние между точками, в которых эти лучи выходят из пластинки.
}
\answer{%
    $\ell = \frac{d}{\cos \alpha} \approx 39\,\text{мм}$
}
\solutionspace{80pt}

\tasknumber{3}%
\task{%
    Солнце составляет с горизонтом угол, синус которого  0{,}5 .
    Шест высотой $130\,\text{см}$ вбит в дно водоёма глубиной $80\,\text{см}$.
    Найдите длину тени от этого шеста на дне водоёма, если показатель преломления воды 1{,}33.
}
\answer{%
    $
        n \sin \beta = 1 \cdot \cos \alpha \implies \beta \approx 40{,}6\degrees,
        L = (H - h)\ctg \alpha + h \tg \beta \approx 86{,}6\,\text{см} + 68{,}6\,\text{см} \approx 155{,}2\,\text{см}.
    $
}
\solutionspace{80pt}

\tasknumber{4}%
\task{%
    Луч света падает на плоское зеркало под углом, синус которого  0{,}65 .
    На сколько миллиметров сместится отражённый луч,
    если на зеркало положить прозрачную пластину толщиной $13\,\text{мм}$ с показателем преломления $1{,}4$?
}
\answer{%
    $1 \cdot \sin \alpha = n \cdot \sin \beta \implies \beta \approx 27{,}7\degrees, L = \cbr{2 d \tg \alpha - 2 d \tg \beta} \cdot \cos \alpha \approx 6{,}5\,\text{мм}$
}
\solutionspace{80pt}

\tasknumber{5}%
\task{%
    В некотором прозрачном веществе свет распространяется со скоростью,
    втрое меньшей скорости света в вакууме.
    Чему будет равен предельный угол
    внутреннего отражения для поверхности раздела этого вещества с воздухом?
}
\answer{%
    $n_1 = \frac c v = 3, n_1 \sin \varphi_\text{п.в.о.} = n_2 \sin \frac \pi 2 \implies \varphi_\text{п.в.о.} \approx 19{,}5\degrees.$
}
\solutionspace{80pt}

\tasknumber{6}%
\task{%
    Луч света падает на горизонтально расположенную стеклянную пластинку толщиной $1{,}2\,\text{см}$.
    Пройдя через пластину, он выходит из неё в точке, смещённой по горизонтали от точки падения на расстояние $4\,\text{мм}$.
    Показатель преломления стекла $1{,}5$.
    Найти синус угла падения.
}
\answer{%
    \begin{align*}
    \ctg \beta &= \frac{h}{d} \implies \\
    \implies \frac 1{\sin^2 \beta} &= \ctg^2 \beta + 1 = \sqr{\frac{h}{d}} + 1 \implies \\
    \implies \sin\alpha &= n\sin \beta = n\sqrt {\frac 1{\sqr{\frac{h}{d}} + 1}} \approx 0{,}47
    \end{align*}
}

\variantsplitter

\addpersonalvariant{Егор Свистушкин}

\tasknumber{1}%
\task{%
    Для закона преломления:
    \begin{itemize}
        \item сделайте рисунок,
        \item отметьте все необходимые углы и подпишите их названия,
        \item запишите этот закон формулой.
    \end{itemize}
}
\solutionspace{80pt}

\tasknumber{2}%
\task{%
    На плоскопараллельную стеклянную пластинку под углом $35\degrees$
    падают два параллельных луча света, расстояние между которыми $8\,\text{см}$.
    Определите расстояние между точками, в которых эти лучи выходят из пластинки.
}
\answer{%
    $\ell = \frac{d}{\cos \alpha} \approx 98\,\text{мм}$
}
\solutionspace{80pt}

\tasknumber{3}%
\task{%
    Солнце составляет с горизонтом угол, синус которого  0{,}8 .
    Шест высотой $170\,\text{см}$ вбит в дно водоёма глубиной $80\,\text{см}$.
    Найдите длину тени от этого шеста на дне водоёма, если показатель преломления воды 1{,}33.
}
\answer{%
    $
        n \sin \beta = 1 \cdot \cos \alpha \implies \beta \approx 26{,}8\degrees,
        L = (H - h)\ctg \alpha + h \tg \beta \approx 67{,}5\,\text{см} + 40{,}4\,\text{см} \approx 107{,}9\,\text{см}.
    $
}
\solutionspace{80pt}

\tasknumber{4}%
\task{%
    Луч света падает на плоское зеркало под углом, синус которого  0{,}75 .
    На сколько миллиметров сместится отражённый луч,
    если на зеркало положить прозрачную пластину толщиной $19\,\text{мм}$ с показателем преломления $1{,}35$?
}
\answer{%
    $1 \cdot \sin \alpha = n \cdot \sin \beta \implies \beta \approx 33{,}7\degrees, L = \cbr{2 d \tg \alpha - 2 d \tg \beta} \cdot \cos \alpha \approx 11{,}7\,\text{мм}$
}
\solutionspace{80pt}

\tasknumber{5}%
\task{%
    В некотором прозрачном веществе свет распространяется со скоростью,
    вчетверо меньшей скорости света в вакууме.
    Чему будет равен предельный угол
    внутреннего отражения для поверхности раздела этого вещества с водой?
}
\answer{%
    $n_1 = \frac c v = 4, n_1 \sin \varphi_\text{п.в.о.} = n_2 \sin \frac \pi 2 \implies \varphi_\text{п.в.о.} \approx 19{,}4\degrees.$
}
\solutionspace{80pt}

\tasknumber{6}%
\task{%
    Луч света падает на вертикально расположенную стеклянную пластинку толщиной $1{,}5\,\text{см}$.
    Пройдя через пластину, он выходит из неё в точке, смещённой по вертикали от точки падения на расстояние $5\,\text{мм}$.
    Показатель преломления стекла $1{,}5$.
    Найти синус угла падения.
}
\answer{%
    \begin{align*}
    \ctg \beta &= \frac{h}{d} \implies \\
    \implies \frac 1{\sin^2 \beta} &= \ctg^2 \beta + 1 = \sqr{\frac{h}{d}} + 1 \implies \\
    \implies \sin\alpha &= n\sin \beta = n\sqrt {\frac 1{\sqr{\frac{h}{d}} + 1}} \approx 0{,}47
    \end{align*}
}

\variantsplitter

\addpersonalvariant{Дмитрий Соколов}

\tasknumber{1}%
\task{%
    Для закона отражения:
    \begin{itemize}
        \item сделайте рисунок,
        \item отметьте все необходимые углы и подпишите их названия,
        \item запишите этот закон формулой.
    \end{itemize}
}
\solutionspace{80pt}

\tasknumber{2}%
\task{%
    На плоскопараллельную стеклянную пластинку под углом $40\degrees$
    падают два параллельных луча света, расстояние между которыми $8\,\text{см}$.
    Определите расстояние между точками, в которых эти лучи выходят из пластинки.
}
\answer{%
    $\ell = \frac{d}{\cos \alpha} \approx 104\,\text{мм}$
}
\solutionspace{80pt}

\tasknumber{3}%
\task{%
    Солнце составляет с горизонтом угол, синус которого  0{,}8 .
    Шест высотой $120\,\text{см}$ вбит в дно водоёма глубиной $70\,\text{см}$.
    Найдите длину тени от этого шеста на дне водоёма, если показатель преломления воды 1{,}33.
}
\answer{%
    $
        n \sin \beta = 1 \cdot \cos \alpha \implies \beta \approx 26{,}8\degrees,
        L = (H - h)\ctg \alpha + h \tg \beta \approx 37{,}5\,\text{см} + 35{,}4\,\text{см} \approx 72{,}9\,\text{см}.
    $
}
\solutionspace{80pt}

\tasknumber{4}%
\task{%
    Луч света падает на плоское зеркало под углом, синус которого  0{,}75 .
    На сколько миллиметров сместится отражённый луч,
    если на зеркало положить прозрачную пластину толщиной $17\,\text{мм}$ с показателем преломления $1{,}45$?
}
\answer{%
    $1 \cdot \sin \alpha = n \cdot \sin \beta \implies \beta \approx 31{,}1\degrees, L = \cbr{2 d \tg \alpha - 2 d \tg \beta} \cdot \cos \alpha \approx 11{,}9\,\text{мм}$
}
\solutionspace{80pt}

\tasknumber{5}%
\task{%
    В некотором прозрачном веществе свет распространяется со скоростью,
    вчетверо меньшей скорости света в вакууме.
    Чему будет равен предельный угол
    внутреннего отражения для поверхности раздела этого вещества с воздухом?
}
\answer{%
    $n_1 = \frac c v = 4, n_1 \sin \varphi_\text{п.в.о.} = n_2 \sin \frac \pi 2 \implies \varphi_\text{п.в.о.} \approx 14{,}5\degrees.$
}
\solutionspace{80pt}

\tasknumber{6}%
\task{%
    Луч света падает на горизонтально расположенную стеклянную пластинку толщиной $1{,}5\,\text{см}$.
    Пройдя через пластину, он выходит из неё в точке, смещённой по горизонтали от точки падения на расстояние $4\,\text{мм}$.
    Показатель преломления стекла $1{,}4$.
    Найти синус угла падения.
}
\answer{%
    \begin{align*}
    \ctg \beta &= \frac{h}{d} \implies \\
    \implies \frac 1{\sin^2 \beta} &= \ctg^2 \beta + 1 = \sqr{\frac{h}{d}} + 1 \implies \\
    \implies \sin\alpha &= n\sin \beta = n\sqrt {\frac 1{\sqr{\frac{h}{d}} + 1}} \approx 0{,}36
    \end{align*}
}

\variantsplitter

\addpersonalvariant{Арсений Трофимов}

\tasknumber{1}%
\task{%
    Для закона отражения:
    \begin{itemize}
        \item сделайте рисунок,
        \item отметьте все необходимые углы и подпишите их названия,
        \item запишите этот закон формулой.
    \end{itemize}
}
\solutionspace{80pt}

\tasknumber{2}%
\task{%
    На плоскопараллельную стеклянную пластинку под углом $50\degrees$
    падают два параллельных луча света, расстояние между которыми $6\,\text{см}$.
    Определите расстояние между точками, в которых эти лучи выходят из пластинки.
}
\answer{%
    $\ell = \frac{d}{\cos \alpha} \approx 93\,\text{мм}$
}
\solutionspace{80pt}

\tasknumber{3}%
\task{%
    Солнце составляет с горизонтом угол, синус которого  0{,}6 .
    Шест высотой $140\,\text{см}$ вбит в дно водоёма глубиной $70\,\text{см}$.
    Найдите длину тени от этого шеста на дне водоёма, если показатель преломления воды 1{,}33.
}
\answer{%
    $
        n \sin \beta = 1 \cdot \cos \alpha \implies \beta \approx 37{,}0\degrees,
        L = (H - h)\ctg \alpha + h \tg \beta \approx 93{,}3\,\text{см} + 52{,}7\,\text{см} \approx 146\,\text{см}.
    $
}
\solutionspace{80pt}

\tasknumber{4}%
\task{%
    Луч света падает на плоское зеркало под углом, синус которого  0{,}65 .
    На сколько миллиметров сместится отражённый луч,
    если на зеркало положить прозрачную пластину толщиной $17\,\text{мм}$ с показателем преломления $1{,}45$?
}
\answer{%
    $1 \cdot \sin \alpha = n \cdot \sin \beta \implies \beta \approx 26{,}6\degrees, L = \cbr{2 d \tg \alpha - 2 d \tg \beta} \cdot \cos \alpha \approx 9{,}1\,\text{мм}$
}
\solutionspace{80pt}

\tasknumber{5}%
\task{%
    В некотором прозрачном веществе свет распространяется со скоростью,
    вдвое меньшей скорости света в вакууме.
    Чему будет равен предельный угол
    внутреннего отражения для поверхности раздела этого вещества с водой?
}
\answer{%
    $n_1 = \frac c v = 2, n_1 \sin \varphi_\text{п.в.о.} = n_2 \sin \frac \pi 2 \implies \varphi_\text{п.в.о.} \approx 41{,}7\degrees.$
}
\solutionspace{80pt}

\tasknumber{6}%
\task{%
    Луч света падает на вертикально расположенную стеклянную пластинку толщиной $1{,}5\,\text{см}$.
    Пройдя через пластину, он выходит из неё в точке, смещённой по вертикали от точки падения на расстояние $5\,\text{мм}$.
    Показатель преломления стекла $1{,}6$.
    Найти синус угла падения.
}
\answer{%
    \begin{align*}
    \ctg \beta &= \frac{h}{d} \implies \\
    \implies \frac 1{\sin^2 \beta} &= \ctg^2 \beta + 1 = \sqr{\frac{h}{d}} + 1 \implies \\
    \implies \sin\alpha &= n\sin \beta = n\sqrt {\frac 1{\sqr{\frac{h}{d}} + 1}} \approx 0{,}51
    \end{align*}
}
% autogenerated
