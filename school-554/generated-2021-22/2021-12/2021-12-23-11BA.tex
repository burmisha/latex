\setdate{23~декабря~2021}
\setclass{11«БА»}

\addpersonalvariant{Михаил Бурмистров}

\tasknumber{1}%
\task{%
    Укажите, верны ли утверждения («да» или «нет» слева от каждого утверждения):
    \begin{itemize}
        \item  Изображение предмета в собирающей линзе всегда мнимое.
        \item  Изображение предмета в собирающей линзе всегда прямое.
        \item  Изображение предмета в собирающей линзе всегда увеличенное.
        \item  Оптическая сила собирающей линзы отрицательна.
    \end{itemize}
}
\answer{%
    $\text{ нет, нет, нет, да }$
}

\tasknumber{2}%
\task{%
    Запишите известные вам виды классификации изображений.
}
\solutionspace{60pt}

\tasknumber{3}%
\task{%
    В каких линзах можно получить мнимое изображение объекта?
}
\answer{%
    $\text{ собирающие и рассеивающие }$
}
\solutionspace{40pt}

\tasknumber{4}%
\task{%
    Какое изображение называют мнимым?
}
\solutionspace{40pt}

\tasknumber{5}%
\task{%
    Есть две линзы, обозначим их 1 и 2.
    Известно что оптическая сила линзы 2 больше, чем у линзы 1.
    Какая линза сильнее преломляет лучи?
}
\answer{%
    $2$
}
\solutionspace{40pt}

\tasknumber{6}%
\task{%
    Предмет находится на расстоянии $20\,\text{см}$ от собирающей линзы с фокусным расстоянием $12\,\text{см}$.
    Определите тип изображения, расстояние между предметом и его изображением, увеличение предмета.
    Сделайте схематичный рисунок (не обязательно в масштабе, но с сохранением свойств линзы и изображения).
}
\solutionspace{100pt}

\tasknumber{7}%
\task{%
    Объект находится на расстоянии $25\,\text{см}$ от линзы, а его мнимое изображение — в $30\,\text{см}$ от неё.
    Определите увеличение предмета, фокусное расстояние линзы, оптическую силу линзы и её тип.
}
\solutionspace{80pt}

\tasknumber{8}%
\task{%
    (Задача-«гроб»: решать на обратной стороне) Квадрат со стороной $d = 2\,\text{см}$ расположен так, что 2 его стороны параллельны главной оптической оси собирающей линзы,
    его центр удален на $h = 4\,\text{см}$ от этой оси и на $a = 15\,\text{см}$ от плоскости линзы.
    Определите площадь изображения квадрата, если фокусное расстояние линзы составляет $F = 18\,\text{см}$.
    % (и сравните с площадью объекта, умноженной на квадрат увеличения центра квадрата).
}
\answer{%
    \begin{align*}
    \frac 1 F &= \frac 1{a + \frac d2} + \frac 1b \implies b = \frac 1{\frac 1 F - \frac 1{a + \frac d2}} = \frac{F(a + \frac d2)}{a + \frac d2 - F}, \\
    \frac 1 F &= \frac 1{a - \frac d2} + \frac 1c \implies c = \frac 1{\frac 1 F - \frac 1{a - \frac d2}} = \frac{F(a - \frac d2)}{a - \frac d2 - F}, \\
    c - b &= \frac{F(a - \frac d2)}{a - \frac d2 - F} - \frac{F(a + \frac d2)}{a + \frac d2 - F} = F\cbr{ \frac{a - \frac d2}{a - \frac d2 - F} - \frac{a + \frac d2}{a + \frac d2 - F} } =  \\
    &= F \cdot \frac{a^2 + \frac {ad}2 - aF - \frac{ad}2 - \frac{d^2}4 + \frac{dF}2 - a^2 + \frac {ad}2 + aF - \frac{ad}2 + \frac{d^2}4 + \frac{dF}2}{\cbr{a + \frac d2 - F}\cbr{a - \frac d2 - F}}= F \cdot \frac {dF}{\cbr{a + \frac d2 - F}\cbr{a - \frac d2 - F}}.
    \\
    \Gamma_b &= \frac b{a + \frac d2} = \frac{ F }{a + \frac d2 - F}, \\
    \Gamma_c &= \frac c{a - \frac d2} = \frac{ F }{a - \frac d2 - F}, \\
    &\text{ тут интересно отметить, что } \Gamma_x = \frac{ c - b}{ d } = \frac{ F^2 }{\cbr{a + \frac d2 - F}\cbr{a - \frac d2 - F}} \ne \Gamma_b \text{ или } \Gamma_c \text{ даже при малых $d$}.
    \\
    S' &= \frac{d \cdot \Gamma_b + d \cdot \Gamma_c}2 \cdot (c - b) = \frac d2 \cbr{\frac{ F }{a + \frac d2 - F} + \frac{ F }{a - \frac d2 - F}} \cdot \cbr{c - b} =  \\
    &=\frac {dF}2 \cbr{\frac 1{a + \frac d2 - F} + \frac 1{a - \frac d2 - F}} \cdot \frac {dF^2}{\cbr{a + \frac d2 - F}\cbr{a - \frac d2 - F}} =  \\
    &=\frac {dF}2 \cdot \frac{a - \frac d2 - F + a + \frac d2 - F}{\cbr{a + \frac d2 - F}\cbr{a - \frac d2 - F}} \cdot \frac {dF^2}{\cbr{a + \frac d2 - F}\cbr{a - \frac d2 - F}} =  \\
    &= \frac {d^2F^3}{2\sqr{a + \frac d2 - F}\sqr{a - \frac d2 - F}} \cdot (2a - 2F) = \frac {d^2F^3(a - F)}{ \sqr{\sqr{a - F} - \frac{d^2}4} }.
    \end{align*}
}

\variantsplitter

\addpersonalvariant{Ирина Ан}

\tasknumber{1}%
\task{%
    Укажите, верны ли утверждения («да» или «нет» слева от каждого утверждения):
    \begin{itemize}
        \item  Изображение предмета в собирающей линзе всегда мнимое.
        \item  Изображение предмета в собирающей линзе всегда перевёрнутое.
        \item  Изображение предмета в собирающей линзе всегда уменьшенное.
        \item  Оптическая сила собирающей линзы положительна.
    \end{itemize}
}
\answer{%
    $\text{ нет, нет, нет, нет }$
}

\tasknumber{2}%
\task{%
    Запишите формулу тонкой линзы и сделайте рисунок, указав на нём физические величины из этой формулы.
}
\solutionspace{60pt}

\tasknumber{3}%
\task{%
    В каких линзах можно получить уменьшенное изображение объекта?
}
\answer{%
    $\text{ собирающие и рассеивающие }$
}
\solutionspace{40pt}

\tasknumber{4}%
\task{%
    Какое изображение называют мнимым?
}
\solutionspace{40pt}

\tasknumber{5}%
\task{%
    Есть две линзы, обозначим их 1 и 2.
    Известно что оптическая сила линзы 1 меньше, чем у линзы 2.
    Какая линза сильнее преломляет лучи?
}
\answer{%
    $2$
}
\solutionspace{40pt}

\tasknumber{6}%
\task{%
    Предмет находится на расстоянии $10\,\text{см}$ от рассеивающей линзы с фокусным расстоянием $15\,\text{см}$.
    Определите тип изображения, расстояние между предметом и его изображением, увеличение предмета.
    Сделайте схематичный рисунок (не обязательно в масштабе, но с сохранением свойств линзы и изображения).
}
\solutionspace{100pt}

\tasknumber{7}%
\task{%
    Объект находится на расстоянии $25\,\text{см}$ от линзы, а его мнимое изображение — в $30\,\text{см}$ от неё.
    Определите увеличение предмета, фокусное расстояние линзы, оптическую силу линзы и её тип.
}
\solutionspace{80pt}

\tasknumber{8}%
\task{%
    (Задача-«гроб»: решать на обратной стороне) Квадрат со стороной $d = 3\,\text{см}$ расположен так, что 2 его стороны параллельны главной оптической оси рассеивающей линзы,
    его центр удален на $h = 4\,\text{см}$ от этой оси и на $a = 12\,\text{см}$ от плоскости линзы.
    Определите площадь изображения квадрата, если фокусное расстояние линзы составляет $F = 20\,\text{см}$.
    % (и сравните с площадью объекта, умноженной на квадрат увеличения центра квадрата).
}
\answer{%
    \begin{align*}
    \frac 1 F &= \frac 1{a + \frac d2} + \frac 1b \implies b = \frac 1{\frac 1 F - \frac 1{a + \frac d2}} = \frac{F(a + \frac d2)}{a + \frac d2 - F}, \\
    \frac 1 F &= \frac 1{a - \frac d2} + \frac 1c \implies c = \frac 1{\frac 1 F - \frac 1{a - \frac d2}} = \frac{F(a - \frac d2)}{a - \frac d2 - F}, \\
    c - b &= \frac{F(a - \frac d2)}{a - \frac d2 - F} - \frac{F(a + \frac d2)}{a + \frac d2 - F} = F\cbr{ \frac{a - \frac d2}{a - \frac d2 - F} - \frac{a + \frac d2}{a + \frac d2 - F} } =  \\
    &= F \cdot \frac{a^2 + \frac {ad}2 - aF - \frac{ad}2 - \frac{d^2}4 + \frac{dF}2 - a^2 + \frac {ad}2 + aF - \frac{ad}2 + \frac{d^2}4 + \frac{dF}2}{\cbr{a + \frac d2 - F}\cbr{a - \frac d2 - F}}= F \cdot \frac {dF}{\cbr{a + \frac d2 - F}\cbr{a - \frac d2 - F}}.
    \\
    \Gamma_b &= \frac b{a + \frac d2} = \frac{ F }{a + \frac d2 - F}, \\
    \Gamma_c &= \frac c{a - \frac d2} = \frac{ F }{a - \frac d2 - F}, \\
    &\text{ тут интересно отметить, что } \Gamma_x = \frac{ c - b}{ d } = \frac{ F^2 }{\cbr{a + \frac d2 - F}\cbr{a - \frac d2 - F}} \ne \Gamma_b \text{ или } \Gamma_c \text{ даже при малых $d$}.
    \\
    S' &= \frac{d \cdot \Gamma_b + d \cdot \Gamma_c}2 \cdot (c - b) = \frac d2 \cbr{\frac{ F }{a + \frac d2 - F} + \frac{ F }{a - \frac d2 - F}} \cdot \cbr{c - b} =  \\
    &=\frac {dF}2 \cbr{\frac 1{a + \frac d2 - F} + \frac 1{a - \frac d2 - F}} \cdot \frac {dF^2}{\cbr{a + \frac d2 - F}\cbr{a - \frac d2 - F}} =  \\
    &=\frac {dF}2 \cdot \frac{a - \frac d2 - F + a + \frac d2 - F}{\cbr{a + \frac d2 - F}\cbr{a - \frac d2 - F}} \cdot \frac {dF^2}{\cbr{a + \frac d2 - F}\cbr{a - \frac d2 - F}} =  \\
    &= \frac {d^2F^3}{2\sqr{a + \frac d2 - F}\sqr{a - \frac d2 - F}} \cdot (2a - 2F) = \frac {d^2F^3(a - F)}{ \sqr{\sqr{a - F} - \frac{d^2}4} }.
    \end{align*}
}

\variantsplitter

\addpersonalvariant{Софья Андрианова}

\tasknumber{1}%
\task{%
    Укажите, верны ли утверждения («да» или «нет» слева от каждого утверждения):
    \begin{itemize}
        \item  Изображение предмета в собирающей линзе всегда действительное.
        \item  Изображение предмета в собирающей линзе всегда прямое.
        \item  Изображение предмета в собирающей линзе всегда уменьшенное.
        \item  Оптическая сила собирающей линзы отрицательна.
    \end{itemize}
}
\answer{%
    $\text{ нет, нет, нет, да }$
}

\tasknumber{2}%
\task{%
    Запишите известные вам виды классификации изображений.
}
\solutionspace{60pt}

\tasknumber{3}%
\task{%
    В каких линзах можно получить мнимое изображение объекта?
}
\answer{%
    $\text{ собирающие и рассеивающие }$
}
\solutionspace{40pt}

\tasknumber{4}%
\task{%
    Какое изображение называют мнимым?
}
\solutionspace{40pt}

\tasknumber{5}%
\task{%
    Есть две линзы, обозначим их 1 и 2.
    Известно что оптическая сила линзы 1 меньше, чем у линзы 2.
    Какая линза сильнее преломляет лучи?
}
\answer{%
    $2$
}
\solutionspace{40pt}

\tasknumber{6}%
\task{%
    Предмет находится на расстоянии $20\,\text{см}$ от рассеивающей линзы с фокусным расстоянием $8\,\text{см}$.
    Определите тип изображения, расстояние между предметом и его изображением, увеличение предмета.
    Сделайте схематичный рисунок (не обязательно в масштабе, но с сохранением свойств линзы и изображения).
}
\solutionspace{100pt}

\tasknumber{7}%
\task{%
    Объект находится на расстоянии $115\,\text{см}$ от линзы, а его мнимое изображение — в $20\,\text{см}$ от неё.
    Определите увеличение предмета, фокусное расстояние линзы, оптическую силу линзы и её тип.
}
\solutionspace{80pt}

\tasknumber{8}%
\task{%
    (Задача-«гроб»: решать на обратной стороне) Квадрат со стороной $d = 1\,\text{см}$ расположен так, что 2 его стороны параллельны главной оптической оси собирающей линзы,
    его центр удален на $h = 4\,\text{см}$ от этой оси и на $a = 12\,\text{см}$ от плоскости линзы.
    Определите площадь изображения квадрата, если фокусное расстояние линзы составляет $F = 25\,\text{см}$.
    % (и сравните с площадью объекта, умноженной на квадрат увеличения центра квадрата).
}
\answer{%
    \begin{align*}
    \frac 1 F &= \frac 1{a + \frac d2} + \frac 1b \implies b = \frac 1{\frac 1 F - \frac 1{a + \frac d2}} = \frac{F(a + \frac d2)}{a + \frac d2 - F}, \\
    \frac 1 F &= \frac 1{a - \frac d2} + \frac 1c \implies c = \frac 1{\frac 1 F - \frac 1{a - \frac d2}} = \frac{F(a - \frac d2)}{a - \frac d2 - F}, \\
    c - b &= \frac{F(a - \frac d2)}{a - \frac d2 - F} - \frac{F(a + \frac d2)}{a + \frac d2 - F} = F\cbr{ \frac{a - \frac d2}{a - \frac d2 - F} - \frac{a + \frac d2}{a + \frac d2 - F} } =  \\
    &= F \cdot \frac{a^2 + \frac {ad}2 - aF - \frac{ad}2 - \frac{d^2}4 + \frac{dF}2 - a^2 + \frac {ad}2 + aF - \frac{ad}2 + \frac{d^2}4 + \frac{dF}2}{\cbr{a + \frac d2 - F}\cbr{a - \frac d2 - F}}= F \cdot \frac {dF}{\cbr{a + \frac d2 - F}\cbr{a - \frac d2 - F}}.
    \\
    \Gamma_b &= \frac b{a + \frac d2} = \frac{ F }{a + \frac d2 - F}, \\
    \Gamma_c &= \frac c{a - \frac d2} = \frac{ F }{a - \frac d2 - F}, \\
    &\text{ тут интересно отметить, что } \Gamma_x = \frac{ c - b}{ d } = \frac{ F^2 }{\cbr{a + \frac d2 - F}\cbr{a - \frac d2 - F}} \ne \Gamma_b \text{ или } \Gamma_c \text{ даже при малых $d$}.
    \\
    S' &= \frac{d \cdot \Gamma_b + d \cdot \Gamma_c}2 \cdot (c - b) = \frac d2 \cbr{\frac{ F }{a + \frac d2 - F} + \frac{ F }{a - \frac d2 - F}} \cdot \cbr{c - b} =  \\
    &=\frac {dF}2 \cbr{\frac 1{a + \frac d2 - F} + \frac 1{a - \frac d2 - F}} \cdot \frac {dF^2}{\cbr{a + \frac d2 - F}\cbr{a - \frac d2 - F}} =  \\
    &=\frac {dF}2 \cdot \frac{a - \frac d2 - F + a + \frac d2 - F}{\cbr{a + \frac d2 - F}\cbr{a - \frac d2 - F}} \cdot \frac {dF^2}{\cbr{a + \frac d2 - F}\cbr{a - \frac d2 - F}} =  \\
    &= \frac {d^2F^3}{2\sqr{a + \frac d2 - F}\sqr{a - \frac d2 - F}} \cdot (2a - 2F) = \frac {d^2F^3(a - F)}{ \sqr{\sqr{a - F} - \frac{d^2}4} }.
    \end{align*}
}

\variantsplitter

\addpersonalvariant{Владимир Артемчук}

\tasknumber{1}%
\task{%
    Укажите, верны ли утверждения («да» или «нет» слева от каждого утверждения):
    \begin{itemize}
        \item  Изображение предмета в собирающей линзе всегда мнимое.
        \item  Изображение предмета в собирающей линзе всегда прямое.
        \item  Изображение предмета в собирающей линзе всегда уменьшенное.
        \item  Оптическая сила рассеивающей линзы положительна.
    \end{itemize}
}
\answer{%
    $\text{ нет, нет, нет, да }$
}

\tasknumber{2}%
\task{%
    Запишите известные вам виды классификации изображений.
}
\solutionspace{60pt}

\tasknumber{3}%
\task{%
    В каких линзах можно получить прямое изображение объекта?
}
\answer{%
    $\text{ собирающие и рассеивающие }$
}
\solutionspace{40pt}

\tasknumber{4}%
\task{%
    Какое изображение называют мнимым?
}
\solutionspace{40pt}

\tasknumber{5}%
\task{%
    Есть две линзы, обозначим их 1 и 2.
    Известно что фокусное расстояние линзы 2 меньше, чем у линзы 1.
    Какая линза сильнее преломляет лучи?
}
\answer{%
    $2$
}
\solutionspace{40pt}

\tasknumber{6}%
\task{%
    Предмет находится на расстоянии $30\,\text{см}$ от рассеивающей линзы с фокусным расстоянием $12\,\text{см}$.
    Определите тип изображения, расстояние между предметом и его изображением, увеличение предмета.
    Сделайте схематичный рисунок (не обязательно в масштабе, но с сохранением свойств линзы и изображения).
}
\solutionspace{100pt}

\tasknumber{7}%
\task{%
    Объект находится на расстоянии $25\,\text{см}$ от линзы, а его действительное изображение — в $50\,\text{см}$ от неё.
    Определите увеличение предмета, фокусное расстояние линзы, оптическую силу линзы и её тип.
}
\solutionspace{80pt}

\tasknumber{8}%
\task{%
    (Задача-«гроб»: решать на обратной стороне) Квадрат со стороной $d = 1\,\text{см}$ расположен так, что 2 его стороны параллельны главной оптической оси рассеивающей линзы,
    его центр удален на $h = 5\,\text{см}$ от этой оси и на $a = 15\,\text{см}$ от плоскости линзы.
    Определите площадь изображения квадрата, если фокусное расстояние линзы составляет $F = 18\,\text{см}$.
    % (и сравните с площадью объекта, умноженной на квадрат увеличения центра квадрата).
}
\answer{%
    \begin{align*}
    \frac 1 F &= \frac 1{a + \frac d2} + \frac 1b \implies b = \frac 1{\frac 1 F - \frac 1{a + \frac d2}} = \frac{F(a + \frac d2)}{a + \frac d2 - F}, \\
    \frac 1 F &= \frac 1{a - \frac d2} + \frac 1c \implies c = \frac 1{\frac 1 F - \frac 1{a - \frac d2}} = \frac{F(a - \frac d2)}{a - \frac d2 - F}, \\
    c - b &= \frac{F(a - \frac d2)}{a - \frac d2 - F} - \frac{F(a + \frac d2)}{a + \frac d2 - F} = F\cbr{ \frac{a - \frac d2}{a - \frac d2 - F} - \frac{a + \frac d2}{a + \frac d2 - F} } =  \\
    &= F \cdot \frac{a^2 + \frac {ad}2 - aF - \frac{ad}2 - \frac{d^2}4 + \frac{dF}2 - a^2 + \frac {ad}2 + aF - \frac{ad}2 + \frac{d^2}4 + \frac{dF}2}{\cbr{a + \frac d2 - F}\cbr{a - \frac d2 - F}}= F \cdot \frac {dF}{\cbr{a + \frac d2 - F}\cbr{a - \frac d2 - F}}.
    \\
    \Gamma_b &= \frac b{a + \frac d2} = \frac{ F }{a + \frac d2 - F}, \\
    \Gamma_c &= \frac c{a - \frac d2} = \frac{ F }{a - \frac d2 - F}, \\
    &\text{ тут интересно отметить, что } \Gamma_x = \frac{ c - b}{ d } = \frac{ F^2 }{\cbr{a + \frac d2 - F}\cbr{a - \frac d2 - F}} \ne \Gamma_b \text{ или } \Gamma_c \text{ даже при малых $d$}.
    \\
    S' &= \frac{d \cdot \Gamma_b + d \cdot \Gamma_c}2 \cdot (c - b) = \frac d2 \cbr{\frac{ F }{a + \frac d2 - F} + \frac{ F }{a - \frac d2 - F}} \cdot \cbr{c - b} =  \\
    &=\frac {dF}2 \cbr{\frac 1{a + \frac d2 - F} + \frac 1{a - \frac d2 - F}} \cdot \frac {dF^2}{\cbr{a + \frac d2 - F}\cbr{a - \frac d2 - F}} =  \\
    &=\frac {dF}2 \cdot \frac{a - \frac d2 - F + a + \frac d2 - F}{\cbr{a + \frac d2 - F}\cbr{a - \frac d2 - F}} \cdot \frac {dF^2}{\cbr{a + \frac d2 - F}\cbr{a - \frac d2 - F}} =  \\
    &= \frac {d^2F^3}{2\sqr{a + \frac d2 - F}\sqr{a - \frac d2 - F}} \cdot (2a - 2F) = \frac {d^2F^3(a - F)}{ \sqr{\sqr{a - F} - \frac{d^2}4} }.
    \end{align*}
}

\variantsplitter

\addpersonalvariant{Софья Белянкина}

\tasknumber{1}%
\task{%
    Укажите, верны ли утверждения («да» или «нет» слева от каждого утверждения):
    \begin{itemize}
        \item  Изображение предмета в собирающей линзе всегда действительное.
        \item  Изображение предмета в собирающей линзе всегда прямое.
        \item  Изображение предмета в собирающей линзе всегда уменьшенное.
        \item  Оптическая сила рассеивающей линзы отрицательна.
    \end{itemize}
}
\answer{%
    $\text{ нет, нет, нет, нет }$
}

\tasknumber{2}%
\task{%
    Запишите формулу тонкой линзы и сделайте рисунок, указав на нём физические величины из этой формулы.
}
\solutionspace{60pt}

\tasknumber{3}%
\task{%
    В каких линзах можно получить обратное изображение объекта?
}
\answer{%
    $\text{ собирающие }$
}
\solutionspace{40pt}

\tasknumber{4}%
\task{%
    Какое изображение называют действительным?
}
\solutionspace{40pt}

\tasknumber{5}%
\task{%
    Есть две линзы, обозначим их 1 и 2.
    Известно что оптическая сила линзы 2 меньше, чем у линзы 1.
    Какая линза сильнее преломляет лучи?
}
\answer{%
    $1$
}
\solutionspace{40pt}

\tasknumber{6}%
\task{%
    Предмет находится на расстоянии $20\,\text{см}$ от собирающей линзы с фокусным расстоянием $15\,\text{см}$.
    Определите тип изображения, расстояние между предметом и его изображением, увеличение предмета.
    Сделайте схематичный рисунок (не обязательно в масштабе, но с сохранением свойств линзы и изображения).
}
\solutionspace{100pt}

\tasknumber{7}%
\task{%
    Объект находится на расстоянии $45\,\text{см}$ от линзы, а его мнимое изображение — в $20\,\text{см}$ от неё.
    Определите увеличение предмета, фокусное расстояние линзы, оптическую силу линзы и её тип.
}
\solutionspace{80pt}

\tasknumber{8}%
\task{%
    (Задача-«гроб»: решать на обратной стороне) Квадрат со стороной $d = 1\,\text{см}$ расположен так, что 2 его стороны параллельны главной оптической оси собирающей линзы,
    его центр удален на $h = 4\,\text{см}$ от этой оси и на $a = 10\,\text{см}$ от плоскости линзы.
    Определите площадь изображения квадрата, если фокусное расстояние линзы составляет $F = 18\,\text{см}$.
    % (и сравните с площадью объекта, умноженной на квадрат увеличения центра квадрата).
}
\answer{%
    \begin{align*}
    \frac 1 F &= \frac 1{a + \frac d2} + \frac 1b \implies b = \frac 1{\frac 1 F - \frac 1{a + \frac d2}} = \frac{F(a + \frac d2)}{a + \frac d2 - F}, \\
    \frac 1 F &= \frac 1{a - \frac d2} + \frac 1c \implies c = \frac 1{\frac 1 F - \frac 1{a - \frac d2}} = \frac{F(a - \frac d2)}{a - \frac d2 - F}, \\
    c - b &= \frac{F(a - \frac d2)}{a - \frac d2 - F} - \frac{F(a + \frac d2)}{a + \frac d2 - F} = F\cbr{ \frac{a - \frac d2}{a - \frac d2 - F} - \frac{a + \frac d2}{a + \frac d2 - F} } =  \\
    &= F \cdot \frac{a^2 + \frac {ad}2 - aF - \frac{ad}2 - \frac{d^2}4 + \frac{dF}2 - a^2 + \frac {ad}2 + aF - \frac{ad}2 + \frac{d^2}4 + \frac{dF}2}{\cbr{a + \frac d2 - F}\cbr{a - \frac d2 - F}}= F \cdot \frac {dF}{\cbr{a + \frac d2 - F}\cbr{a - \frac d2 - F}}.
    \\
    \Gamma_b &= \frac b{a + \frac d2} = \frac{ F }{a + \frac d2 - F}, \\
    \Gamma_c &= \frac c{a - \frac d2} = \frac{ F }{a - \frac d2 - F}, \\
    &\text{ тут интересно отметить, что } \Gamma_x = \frac{ c - b}{ d } = \frac{ F^2 }{\cbr{a + \frac d2 - F}\cbr{a - \frac d2 - F}} \ne \Gamma_b \text{ или } \Gamma_c \text{ даже при малых $d$}.
    \\
    S' &= \frac{d \cdot \Gamma_b + d \cdot \Gamma_c}2 \cdot (c - b) = \frac d2 \cbr{\frac{ F }{a + \frac d2 - F} + \frac{ F }{a - \frac d2 - F}} \cdot \cbr{c - b} =  \\
    &=\frac {dF}2 \cbr{\frac 1{a + \frac d2 - F} + \frac 1{a - \frac d2 - F}} \cdot \frac {dF^2}{\cbr{a + \frac d2 - F}\cbr{a - \frac d2 - F}} =  \\
    &=\frac {dF}2 \cdot \frac{a - \frac d2 - F + a + \frac d2 - F}{\cbr{a + \frac d2 - F}\cbr{a - \frac d2 - F}} \cdot \frac {dF^2}{\cbr{a + \frac d2 - F}\cbr{a - \frac d2 - F}} =  \\
    &= \frac {d^2F^3}{2\sqr{a + \frac d2 - F}\sqr{a - \frac d2 - F}} \cdot (2a - 2F) = \frac {d^2F^3(a - F)}{ \sqr{\sqr{a - F} - \frac{d^2}4} }.
    \end{align*}
}

\variantsplitter

\addpersonalvariant{Варвара Егиазарян}

\tasknumber{1}%
\task{%
    Укажите, верны ли утверждения («да» или «нет» слева от каждого утверждения):
    \begin{itemize}
        \item  Изображение предмета в собирающей линзе всегда мнимое.
        \item  Изображение предмета в собирающей линзе всегда прямое.
        \item  Изображение предмета в собирающей линзе всегда уменьшенное.
        \item  Оптическая сила рассеивающей линзы положительна.
    \end{itemize}
}
\answer{%
    $\text{ нет, нет, нет, да }$
}

\tasknumber{2}%
\task{%
    Запишите формулу тонкой линзы и сделайте рисунок, указав на нём физические величины из этой формулы.
}
\solutionspace{60pt}

\tasknumber{3}%
\task{%
    В каких линзах можно получить прямое изображение объекта?
}
\answer{%
    $\text{ собирающие и рассеивающие }$
}
\solutionspace{40pt}

\tasknumber{4}%
\task{%
    Какое изображение называют мнимым?
}
\solutionspace{40pt}

\tasknumber{5}%
\task{%
    Есть две линзы, обозначим их 1 и 2.
    Известно что фокусное расстояние линзы 1 больше, чем у линзы 2.
    Какая линза сильнее преломляет лучи?
}
\answer{%
    $2$
}
\solutionspace{40pt}

\tasknumber{6}%
\task{%
    Предмет находится на расстоянии $30\,\text{см}$ от собирающей линзы с фокусным расстоянием $50\,\text{см}$.
    Определите тип изображения, расстояние между предметом и его изображением, увеличение предмета.
    Сделайте схематичный рисунок (не обязательно в масштабе, но с сохранением свойств линзы и изображения).
}
\solutionspace{100pt}

\tasknumber{7}%
\task{%
    Объект находится на расстоянии $115\,\text{см}$ от линзы, а его мнимое изображение — в $20\,\text{см}$ от неё.
    Определите увеличение предмета, фокусное расстояние линзы, оптическую силу линзы и её тип.
}
\solutionspace{80pt}

\tasknumber{8}%
\task{%
    (Задача-«гроб»: решать на обратной стороне) Квадрат со стороной $d = 2\,\text{см}$ расположен так, что 2 его стороны параллельны главной оптической оси рассеивающей линзы,
    его центр удален на $h = 5\,\text{см}$ от этой оси и на $a = 15\,\text{см}$ от плоскости линзы.
    Определите площадь изображения квадрата, если фокусное расстояние линзы составляет $F = 25\,\text{см}$.
    % (и сравните с площадью объекта, умноженной на квадрат увеличения центра квадрата).
}
\answer{%
    \begin{align*}
    \frac 1 F &= \frac 1{a + \frac d2} + \frac 1b \implies b = \frac 1{\frac 1 F - \frac 1{a + \frac d2}} = \frac{F(a + \frac d2)}{a + \frac d2 - F}, \\
    \frac 1 F &= \frac 1{a - \frac d2} + \frac 1c \implies c = \frac 1{\frac 1 F - \frac 1{a - \frac d2}} = \frac{F(a - \frac d2)}{a - \frac d2 - F}, \\
    c - b &= \frac{F(a - \frac d2)}{a - \frac d2 - F} - \frac{F(a + \frac d2)}{a + \frac d2 - F} = F\cbr{ \frac{a - \frac d2}{a - \frac d2 - F} - \frac{a + \frac d2}{a + \frac d2 - F} } =  \\
    &= F \cdot \frac{a^2 + \frac {ad}2 - aF - \frac{ad}2 - \frac{d^2}4 + \frac{dF}2 - a^2 + \frac {ad}2 + aF - \frac{ad}2 + \frac{d^2}4 + \frac{dF}2}{\cbr{a + \frac d2 - F}\cbr{a - \frac d2 - F}}= F \cdot \frac {dF}{\cbr{a + \frac d2 - F}\cbr{a - \frac d2 - F}}.
    \\
    \Gamma_b &= \frac b{a + \frac d2} = \frac{ F }{a + \frac d2 - F}, \\
    \Gamma_c &= \frac c{a - \frac d2} = \frac{ F }{a - \frac d2 - F}, \\
    &\text{ тут интересно отметить, что } \Gamma_x = \frac{ c - b}{ d } = \frac{ F^2 }{\cbr{a + \frac d2 - F}\cbr{a - \frac d2 - F}} \ne \Gamma_b \text{ или } \Gamma_c \text{ даже при малых $d$}.
    \\
    S' &= \frac{d \cdot \Gamma_b + d \cdot \Gamma_c}2 \cdot (c - b) = \frac d2 \cbr{\frac{ F }{a + \frac d2 - F} + \frac{ F }{a - \frac d2 - F}} \cdot \cbr{c - b} =  \\
    &=\frac {dF}2 \cbr{\frac 1{a + \frac d2 - F} + \frac 1{a - \frac d2 - F}} \cdot \frac {dF^2}{\cbr{a + \frac d2 - F}\cbr{a - \frac d2 - F}} =  \\
    &=\frac {dF}2 \cdot \frac{a - \frac d2 - F + a + \frac d2 - F}{\cbr{a + \frac d2 - F}\cbr{a - \frac d2 - F}} \cdot \frac {dF^2}{\cbr{a + \frac d2 - F}\cbr{a - \frac d2 - F}} =  \\
    &= \frac {d^2F^3}{2\sqr{a + \frac d2 - F}\sqr{a - \frac d2 - F}} \cdot (2a - 2F) = \frac {d^2F^3(a - F)}{ \sqr{\sqr{a - F} - \frac{d^2}4} }.
    \end{align*}
}

\variantsplitter

\addpersonalvariant{Владислав Емелин}

\tasknumber{1}%
\task{%
    Укажите, верны ли утверждения («да» или «нет» слева от каждого утверждения):
    \begin{itemize}
        \item  Изображение предмета в рассеивающей линзе всегда действительное.
        \item  Изображение предмета в рассеивающей линзе всегда перевёрнутое.
        \item  Изображение предмета в рассеивающей линзе всегда увеличенное.
        \item  Оптическая сила рассеивающей линзы положительна.
    \end{itemize}
}
\answer{%
    $\text{ нет, нет, нет, да }$
}

\tasknumber{2}%
\task{%
    Запишите формулу тонкой линзы и сделайте рисунок, указав на нём физические величины из этой формулы.
}
\solutionspace{60pt}

\tasknumber{3}%
\task{%
    В каких линзах можно получить действительное изображение объекта?
}
\answer{%
    $\text{ собирающие }$
}
\solutionspace{40pt}

\tasknumber{4}%
\task{%
    Какое изображение называют действительным?
}
\solutionspace{40pt}

\tasknumber{5}%
\task{%
    Есть две линзы, обозначим их 1 и 2.
    Известно что фокусное расстояние линзы 1 больше, чем у линзы 2.
    Какая линза сильнее преломляет лучи?
}
\answer{%
    $2$
}
\solutionspace{40pt}

\tasknumber{6}%
\task{%
    Предмет находится на расстоянии $30\,\text{см}$ от собирающей линзы с фокусным расстоянием $8\,\text{см}$.
    Определите тип изображения, расстояние между предметом и его изображением, увеличение предмета.
    Сделайте схематичный рисунок (не обязательно в масштабе, но с сохранением свойств линзы и изображения).
}
\solutionspace{100pt}

\tasknumber{7}%
\task{%
    Объект находится на расстоянии $115\,\text{см}$ от линзы, а его мнимое изображение — в $40\,\text{см}$ от неё.
    Определите увеличение предмета, фокусное расстояние линзы, оптическую силу линзы и её тип.
}
\solutionspace{80pt}

\tasknumber{8}%
\task{%
    (Задача-«гроб»: решать на обратной стороне) Квадрат со стороной $d = 2\,\text{см}$ расположен так, что 2 его стороны параллельны главной оптической оси собирающей линзы,
    его центр удален на $h = 6\,\text{см}$ от этой оси и на $a = 15\,\text{см}$ от плоскости линзы.
    Определите площадь изображения квадрата, если фокусное расстояние линзы составляет $F = 25\,\text{см}$.
    % (и сравните с площадью объекта, умноженной на квадрат увеличения центра квадрата).
}
\answer{%
    \begin{align*}
    \frac 1 F &= \frac 1{a + \frac d2} + \frac 1b \implies b = \frac 1{\frac 1 F - \frac 1{a + \frac d2}} = \frac{F(a + \frac d2)}{a + \frac d2 - F}, \\
    \frac 1 F &= \frac 1{a - \frac d2} + \frac 1c \implies c = \frac 1{\frac 1 F - \frac 1{a - \frac d2}} = \frac{F(a - \frac d2)}{a - \frac d2 - F}, \\
    c - b &= \frac{F(a - \frac d2)}{a - \frac d2 - F} - \frac{F(a + \frac d2)}{a + \frac d2 - F} = F\cbr{ \frac{a - \frac d2}{a - \frac d2 - F} - \frac{a + \frac d2}{a + \frac d2 - F} } =  \\
    &= F \cdot \frac{a^2 + \frac {ad}2 - aF - \frac{ad}2 - \frac{d^2}4 + \frac{dF}2 - a^2 + \frac {ad}2 + aF - \frac{ad}2 + \frac{d^2}4 + \frac{dF}2}{\cbr{a + \frac d2 - F}\cbr{a - \frac d2 - F}}= F \cdot \frac {dF}{\cbr{a + \frac d2 - F}\cbr{a - \frac d2 - F}}.
    \\
    \Gamma_b &= \frac b{a + \frac d2} = \frac{ F }{a + \frac d2 - F}, \\
    \Gamma_c &= \frac c{a - \frac d2} = \frac{ F }{a - \frac d2 - F}, \\
    &\text{ тут интересно отметить, что } \Gamma_x = \frac{ c - b}{ d } = \frac{ F^2 }{\cbr{a + \frac d2 - F}\cbr{a - \frac d2 - F}} \ne \Gamma_b \text{ или } \Gamma_c \text{ даже при малых $d$}.
    \\
    S' &= \frac{d \cdot \Gamma_b + d \cdot \Gamma_c}2 \cdot (c - b) = \frac d2 \cbr{\frac{ F }{a + \frac d2 - F} + \frac{ F }{a - \frac d2 - F}} \cdot \cbr{c - b} =  \\
    &=\frac {dF}2 \cbr{\frac 1{a + \frac d2 - F} + \frac 1{a - \frac d2 - F}} \cdot \frac {dF^2}{\cbr{a + \frac d2 - F}\cbr{a - \frac d2 - F}} =  \\
    &=\frac {dF}2 \cdot \frac{a - \frac d2 - F + a + \frac d2 - F}{\cbr{a + \frac d2 - F}\cbr{a - \frac d2 - F}} \cdot \frac {dF^2}{\cbr{a + \frac d2 - F}\cbr{a - \frac d2 - F}} =  \\
    &= \frac {d^2F^3}{2\sqr{a + \frac d2 - F}\sqr{a - \frac d2 - F}} \cdot (2a - 2F) = \frac {d^2F^3(a - F)}{ \sqr{\sqr{a - F} - \frac{d^2}4} }.
    \end{align*}
}

\variantsplitter

\addpersonalvariant{Артём Жичин}

\tasknumber{1}%
\task{%
    Укажите, верны ли утверждения («да» или «нет» слева от каждого утверждения):
    \begin{itemize}
        \item  Изображение предмета в рассеивающей линзе всегда действительное.
        \item  Изображение предмета в рассеивающей линзе всегда перевёрнутое.
        \item  Изображение предмета в рассеивающей линзе всегда уменьшенное.
        \item  Оптическая сила рассеивающей линзы положительна.
    \end{itemize}
}
\answer{%
    $\text{ нет, нет, да, да }$
}

\tasknumber{2}%
\task{%
    Запишите формулу тонкой линзы и сделайте рисунок, указав на нём физические величины из этой формулы.
}
\solutionspace{60pt}

\tasknumber{3}%
\task{%
    В каких линзах можно получить мнимое изображение объекта?
}
\answer{%
    $\text{ собирающие и рассеивающие }$
}
\solutionspace{40pt}

\tasknumber{4}%
\task{%
    Какое изображение называют мнимым?
}
\solutionspace{40pt}

\tasknumber{5}%
\task{%
    Есть две линзы, обозначим их 1 и 2.
    Известно что оптическая сила линзы 1 больше, чем у линзы 2.
    Какая линза сильнее преломляет лучи?
}
\answer{%
    $1$
}
\solutionspace{40pt}

\tasknumber{6}%
\task{%
    Предмет находится на расстоянии $30\,\text{см}$ от рассеивающей линзы с фокусным расстоянием $15\,\text{см}$.
    Определите тип изображения, расстояние между предметом и его изображением, увеличение предмета.
    Сделайте схематичный рисунок (не обязательно в масштабе, но с сохранением свойств линзы и изображения).
}
\solutionspace{100pt}

\tasknumber{7}%
\task{%
    Объект находится на расстоянии $45\,\text{см}$ от линзы, а его мнимое изображение — в $10\,\text{см}$ от неё.
    Определите увеличение предмета, фокусное расстояние линзы, оптическую силу линзы и её тип.
}
\solutionspace{80pt}

\tasknumber{8}%
\task{%
    (Задача-«гроб»: решать на обратной стороне) Квадрат со стороной $d = 3\,\text{см}$ расположен так, что 2 его стороны параллельны главной оптической оси рассеивающей линзы,
    его центр удален на $h = 4\,\text{см}$ от этой оси и на $a = 15\,\text{см}$ от плоскости линзы.
    Определите площадь изображения квадрата, если фокусное расстояние линзы составляет $F = 18\,\text{см}$.
    % (и сравните с площадью объекта, умноженной на квадрат увеличения центра квадрата).
}
\answer{%
    \begin{align*}
    \frac 1 F &= \frac 1{a + \frac d2} + \frac 1b \implies b = \frac 1{\frac 1 F - \frac 1{a + \frac d2}} = \frac{F(a + \frac d2)}{a + \frac d2 - F}, \\
    \frac 1 F &= \frac 1{a - \frac d2} + \frac 1c \implies c = \frac 1{\frac 1 F - \frac 1{a - \frac d2}} = \frac{F(a - \frac d2)}{a - \frac d2 - F}, \\
    c - b &= \frac{F(a - \frac d2)}{a - \frac d2 - F} - \frac{F(a + \frac d2)}{a + \frac d2 - F} = F\cbr{ \frac{a - \frac d2}{a - \frac d2 - F} - \frac{a + \frac d2}{a + \frac d2 - F} } =  \\
    &= F \cdot \frac{a^2 + \frac {ad}2 - aF - \frac{ad}2 - \frac{d^2}4 + \frac{dF}2 - a^2 + \frac {ad}2 + aF - \frac{ad}2 + \frac{d^2}4 + \frac{dF}2}{\cbr{a + \frac d2 - F}\cbr{a - \frac d2 - F}}= F \cdot \frac {dF}{\cbr{a + \frac d2 - F}\cbr{a - \frac d2 - F}}.
    \\
    \Gamma_b &= \frac b{a + \frac d2} = \frac{ F }{a + \frac d2 - F}, \\
    \Gamma_c &= \frac c{a - \frac d2} = \frac{ F }{a - \frac d2 - F}, \\
    &\text{ тут интересно отметить, что } \Gamma_x = \frac{ c - b}{ d } = \frac{ F^2 }{\cbr{a + \frac d2 - F}\cbr{a - \frac d2 - F}} \ne \Gamma_b \text{ или } \Gamma_c \text{ даже при малых $d$}.
    \\
    S' &= \frac{d \cdot \Gamma_b + d \cdot \Gamma_c}2 \cdot (c - b) = \frac d2 \cbr{\frac{ F }{a + \frac d2 - F} + \frac{ F }{a - \frac d2 - F}} \cdot \cbr{c - b} =  \\
    &=\frac {dF}2 \cbr{\frac 1{a + \frac d2 - F} + \frac 1{a - \frac d2 - F}} \cdot \frac {dF^2}{\cbr{a + \frac d2 - F}\cbr{a - \frac d2 - F}} =  \\
    &=\frac {dF}2 \cdot \frac{a - \frac d2 - F + a + \frac d2 - F}{\cbr{a + \frac d2 - F}\cbr{a - \frac d2 - F}} \cdot \frac {dF^2}{\cbr{a + \frac d2 - F}\cbr{a - \frac d2 - F}} =  \\
    &= \frac {d^2F^3}{2\sqr{a + \frac d2 - F}\sqr{a - \frac d2 - F}} \cdot (2a - 2F) = \frac {d^2F^3(a - F)}{ \sqr{\sqr{a - F} - \frac{d^2}4} }.
    \end{align*}
}

\variantsplitter

\addpersonalvariant{Дарья Кошман}

\tasknumber{1}%
\task{%
    Укажите, верны ли утверждения («да» или «нет» слева от каждого утверждения):
    \begin{itemize}
        \item  Изображение предмета в рассеивающей линзе всегда действительное.
        \item  Изображение предмета в рассеивающей линзе всегда перевёрнутое.
        \item  Изображение предмета в рассеивающей линзе всегда увеличенное.
        \item  Оптическая сила рассеивающей линзы отрицательна.
    \end{itemize}
}
\answer{%
    $\text{ нет, нет, нет, нет }$
}

\tasknumber{2}%
\task{%
    Запишите формулу тонкой линзы и сделайте рисунок, указав на нём физические величины из этой формулы.
}
\solutionspace{60pt}

\tasknumber{3}%
\task{%
    В каких линзах можно получить увеличенное изображение объекта?
}
\answer{%
    $\text{ рассеивающие }$
}
\solutionspace{40pt}

\tasknumber{4}%
\task{%
    Какое изображение называют действительным?
}
\solutionspace{40pt}

\tasknumber{5}%
\task{%
    Есть две линзы, обозначим их 1 и 2.
    Известно что оптическая сила линзы 1 больше, чем у линзы 2.
    Какая линза сильнее преломляет лучи?
}
\answer{%
    $1$
}
\solutionspace{40pt}

\tasknumber{6}%
\task{%
    Предмет находится на расстоянии $10\,\text{см}$ от собирающей линзы с фокусным расстоянием $15\,\text{см}$.
    Определите тип изображения, расстояние между предметом и его изображением, увеличение предмета.
    Сделайте схематичный рисунок (не обязательно в масштабе, но с сохранением свойств линзы и изображения).
}
\solutionspace{100pt}

\tasknumber{7}%
\task{%
    Объект находится на расстоянии $25\,\text{см}$ от линзы, а его действительное изображение — в $30\,\text{см}$ от неё.
    Определите увеличение предмета, фокусное расстояние линзы, оптическую силу линзы и её тип.
}
\solutionspace{80pt}

\tasknumber{8}%
\task{%
    (Задача-«гроб»: решать на обратной стороне) Квадрат со стороной $d = 2\,\text{см}$ расположен так, что 2 его стороны параллельны главной оптической оси собирающей линзы,
    его центр удален на $h = 5\,\text{см}$ от этой оси и на $a = 15\,\text{см}$ от плоскости линзы.
    Определите площадь изображения квадрата, если фокусное расстояние линзы составляет $F = 20\,\text{см}$.
    % (и сравните с площадью объекта, умноженной на квадрат увеличения центра квадрата).
}
\answer{%
    \begin{align*}
    \frac 1 F &= \frac 1{a + \frac d2} + \frac 1b \implies b = \frac 1{\frac 1 F - \frac 1{a + \frac d2}} = \frac{F(a + \frac d2)}{a + \frac d2 - F}, \\
    \frac 1 F &= \frac 1{a - \frac d2} + \frac 1c \implies c = \frac 1{\frac 1 F - \frac 1{a - \frac d2}} = \frac{F(a - \frac d2)}{a - \frac d2 - F}, \\
    c - b &= \frac{F(a - \frac d2)}{a - \frac d2 - F} - \frac{F(a + \frac d2)}{a + \frac d2 - F} = F\cbr{ \frac{a - \frac d2}{a - \frac d2 - F} - \frac{a + \frac d2}{a + \frac d2 - F} } =  \\
    &= F \cdot \frac{a^2 + \frac {ad}2 - aF - \frac{ad}2 - \frac{d^2}4 + \frac{dF}2 - a^2 + \frac {ad}2 + aF - \frac{ad}2 + \frac{d^2}4 + \frac{dF}2}{\cbr{a + \frac d2 - F}\cbr{a - \frac d2 - F}}= F \cdot \frac {dF}{\cbr{a + \frac d2 - F}\cbr{a - \frac d2 - F}}.
    \\
    \Gamma_b &= \frac b{a + \frac d2} = \frac{ F }{a + \frac d2 - F}, \\
    \Gamma_c &= \frac c{a - \frac d2} = \frac{ F }{a - \frac d2 - F}, \\
    &\text{ тут интересно отметить, что } \Gamma_x = \frac{ c - b}{ d } = \frac{ F^2 }{\cbr{a + \frac d2 - F}\cbr{a - \frac d2 - F}} \ne \Gamma_b \text{ или } \Gamma_c \text{ даже при малых $d$}.
    \\
    S' &= \frac{d \cdot \Gamma_b + d \cdot \Gamma_c}2 \cdot (c - b) = \frac d2 \cbr{\frac{ F }{a + \frac d2 - F} + \frac{ F }{a - \frac d2 - F}} \cdot \cbr{c - b} =  \\
    &=\frac {dF}2 \cbr{\frac 1{a + \frac d2 - F} + \frac 1{a - \frac d2 - F}} \cdot \frac {dF^2}{\cbr{a + \frac d2 - F}\cbr{a - \frac d2 - F}} =  \\
    &=\frac {dF}2 \cdot \frac{a - \frac d2 - F + a + \frac d2 - F}{\cbr{a + \frac d2 - F}\cbr{a - \frac d2 - F}} \cdot \frac {dF^2}{\cbr{a + \frac d2 - F}\cbr{a - \frac d2 - F}} =  \\
    &= \frac {d^2F^3}{2\sqr{a + \frac d2 - F}\sqr{a - \frac d2 - F}} \cdot (2a - 2F) = \frac {d^2F^3(a - F)}{ \sqr{\sqr{a - F} - \frac{d^2}4} }.
    \end{align*}
}

\variantsplitter

\addpersonalvariant{Анна Кузьмичёва}

\tasknumber{1}%
\task{%
    Укажите, верны ли утверждения («да» или «нет» слева от каждого утверждения):
    \begin{itemize}
        \item  Изображение предмета в собирающей линзе всегда мнимое.
        \item  Изображение предмета в собирающей линзе всегда прямое.
        \item  Изображение предмета в собирающей линзе всегда уменьшенное.
        \item  Оптическая сила рассеивающей линзы отрицательна.
    \end{itemize}
}
\answer{%
    $\text{ нет, нет, нет, нет }$
}

\tasknumber{2}%
\task{%
    Запишите формулу тонкой линзы и сделайте рисунок, указав на нём физические величины из этой формулы.
}
\solutionspace{60pt}

\tasknumber{3}%
\task{%
    В каких линзах можно получить мнимое изображение объекта?
}
\answer{%
    $\text{ собирающие и рассеивающие }$
}
\solutionspace{40pt}

\tasknumber{4}%
\task{%
    Какое изображение называют мнимым?
}
\solutionspace{40pt}

\tasknumber{5}%
\task{%
    Есть две линзы, обозначим их 1 и 2.
    Известно что оптическая сила линзы 2 больше, чем у линзы 1.
    Какая линза сильнее преломляет лучи?
}
\answer{%
    $2$
}
\solutionspace{40pt}

\tasknumber{6}%
\task{%
    Предмет находится на расстоянии $20\,\text{см}$ от собирающей линзы с фокусным расстоянием $50\,\text{см}$.
    Определите тип изображения, расстояние между предметом и его изображением, увеличение предмета.
    Сделайте схематичный рисунок (не обязательно в масштабе, но с сохранением свойств линзы и изображения).
}
\solutionspace{100pt}

\tasknumber{7}%
\task{%
    Объект находится на расстоянии $115\,\text{см}$ от линзы, а его действительное изображение — в $40\,\text{см}$ от неё.
    Определите увеличение предмета, фокусное расстояние линзы, оптическую силу линзы и её тип.
}
\solutionspace{80pt}

\tasknumber{8}%
\task{%
    (Задача-«гроб»: решать на обратной стороне) Квадрат со стороной $d = 1\,\text{см}$ расположен так, что 2 его стороны параллельны главной оптической оси рассеивающей линзы,
    его центр удален на $h = 6\,\text{см}$ от этой оси и на $a = 10\,\text{см}$ от плоскости линзы.
    Определите площадь изображения квадрата, если фокусное расстояние линзы составляет $F = 20\,\text{см}$.
    % (и сравните с площадью объекта, умноженной на квадрат увеличения центра квадрата).
}
\answer{%
    \begin{align*}
    \frac 1 F &= \frac 1{a + \frac d2} + \frac 1b \implies b = \frac 1{\frac 1 F - \frac 1{a + \frac d2}} = \frac{F(a + \frac d2)}{a + \frac d2 - F}, \\
    \frac 1 F &= \frac 1{a - \frac d2} + \frac 1c \implies c = \frac 1{\frac 1 F - \frac 1{a - \frac d2}} = \frac{F(a - \frac d2)}{a - \frac d2 - F}, \\
    c - b &= \frac{F(a - \frac d2)}{a - \frac d2 - F} - \frac{F(a + \frac d2)}{a + \frac d2 - F} = F\cbr{ \frac{a - \frac d2}{a - \frac d2 - F} - \frac{a + \frac d2}{a + \frac d2 - F} } =  \\
    &= F \cdot \frac{a^2 + \frac {ad}2 - aF - \frac{ad}2 - \frac{d^2}4 + \frac{dF}2 - a^2 + \frac {ad}2 + aF - \frac{ad}2 + \frac{d^2}4 + \frac{dF}2}{\cbr{a + \frac d2 - F}\cbr{a - \frac d2 - F}}= F \cdot \frac {dF}{\cbr{a + \frac d2 - F}\cbr{a - \frac d2 - F}}.
    \\
    \Gamma_b &= \frac b{a + \frac d2} = \frac{ F }{a + \frac d2 - F}, \\
    \Gamma_c &= \frac c{a - \frac d2} = \frac{ F }{a - \frac d2 - F}, \\
    &\text{ тут интересно отметить, что } \Gamma_x = \frac{ c - b}{ d } = \frac{ F^2 }{\cbr{a + \frac d2 - F}\cbr{a - \frac d2 - F}} \ne \Gamma_b \text{ или } \Gamma_c \text{ даже при малых $d$}.
    \\
    S' &= \frac{d \cdot \Gamma_b + d \cdot \Gamma_c}2 \cdot (c - b) = \frac d2 \cbr{\frac{ F }{a + \frac d2 - F} + \frac{ F }{a - \frac d2 - F}} \cdot \cbr{c - b} =  \\
    &=\frac {dF}2 \cbr{\frac 1{a + \frac d2 - F} + \frac 1{a - \frac d2 - F}} \cdot \frac {dF^2}{\cbr{a + \frac d2 - F}\cbr{a - \frac d2 - F}} =  \\
    &=\frac {dF}2 \cdot \frac{a - \frac d2 - F + a + \frac d2 - F}{\cbr{a + \frac d2 - F}\cbr{a - \frac d2 - F}} \cdot \frac {dF^2}{\cbr{a + \frac d2 - F}\cbr{a - \frac d2 - F}} =  \\
    &= \frac {d^2F^3}{2\sqr{a + \frac d2 - F}\sqr{a - \frac d2 - F}} \cdot (2a - 2F) = \frac {d^2F^3(a - F)}{ \sqr{\sqr{a - F} - \frac{d^2}4} }.
    \end{align*}
}

\variantsplitter

\addpersonalvariant{Алёна Куприянова}

\tasknumber{1}%
\task{%
    Укажите, верны ли утверждения («да» или «нет» слева от каждого утверждения):
    \begin{itemize}
        \item  Изображение предмета в рассеивающей линзе всегда мнимое.
        \item  Изображение предмета в рассеивающей линзе всегда перевёрнутое.
        \item  Изображение предмета в рассеивающей линзе всегда уменьшенное.
        \item  Оптическая сила рассеивающей линзы положительна.
    \end{itemize}
}
\answer{%
    $\text{ да, нет, да, да }$
}

\tasknumber{2}%
\task{%
    Запишите известные вам виды классификации изображений.
}
\solutionspace{60pt}

\tasknumber{3}%
\task{%
    В каких линзах можно получить действительное изображение объекта?
}
\answer{%
    $\text{ собирающие }$
}
\solutionspace{40pt}

\tasknumber{4}%
\task{%
    Какое изображение называют действительным?
}
\solutionspace{40pt}

\tasknumber{5}%
\task{%
    Есть две линзы, обозначим их 1 и 2.
    Известно что оптическая сила линзы 2 больше, чем у линзы 1.
    Какая линза сильнее преломляет лучи?
}
\answer{%
    $2$
}
\solutionspace{40pt}

\tasknumber{6}%
\task{%
    Предмет находится на расстоянии $10\,\text{см}$ от собирающей линзы с фокусным расстоянием $6\,\text{см}$.
    Определите тип изображения, расстояние между предметом и его изображением, увеличение предмета.
    Сделайте схематичный рисунок (не обязательно в масштабе, но с сохранением свойств линзы и изображения).
}
\solutionspace{100pt}

\tasknumber{7}%
\task{%
    Объект находится на расстоянии $115\,\text{см}$ от линзы, а его действительное изображение — в $30\,\text{см}$ от неё.
    Определите увеличение предмета, фокусное расстояние линзы, оптическую силу линзы и её тип.
}
\solutionspace{80pt}

\tasknumber{8}%
\task{%
    (Задача-«гроб»: решать на обратной стороне) Квадрат со стороной $d = 2\,\text{см}$ расположен так, что 2 его стороны параллельны главной оптической оси собирающей линзы,
    его центр удален на $h = 5\,\text{см}$ от этой оси и на $a = 15\,\text{см}$ от плоскости линзы.
    Определите площадь изображения квадрата, если фокусное расстояние линзы составляет $F = 20\,\text{см}$.
    % (и сравните с площадью объекта, умноженной на квадрат увеличения центра квадрата).
}
\answer{%
    \begin{align*}
    \frac 1 F &= \frac 1{a + \frac d2} + \frac 1b \implies b = \frac 1{\frac 1 F - \frac 1{a + \frac d2}} = \frac{F(a + \frac d2)}{a + \frac d2 - F}, \\
    \frac 1 F &= \frac 1{a - \frac d2} + \frac 1c \implies c = \frac 1{\frac 1 F - \frac 1{a - \frac d2}} = \frac{F(a - \frac d2)}{a - \frac d2 - F}, \\
    c - b &= \frac{F(a - \frac d2)}{a - \frac d2 - F} - \frac{F(a + \frac d2)}{a + \frac d2 - F} = F\cbr{ \frac{a - \frac d2}{a - \frac d2 - F} - \frac{a + \frac d2}{a + \frac d2 - F} } =  \\
    &= F \cdot \frac{a^2 + \frac {ad}2 - aF - \frac{ad}2 - \frac{d^2}4 + \frac{dF}2 - a^2 + \frac {ad}2 + aF - \frac{ad}2 + \frac{d^2}4 + \frac{dF}2}{\cbr{a + \frac d2 - F}\cbr{a - \frac d2 - F}}= F \cdot \frac {dF}{\cbr{a + \frac d2 - F}\cbr{a - \frac d2 - F}}.
    \\
    \Gamma_b &= \frac b{a + \frac d2} = \frac{ F }{a + \frac d2 - F}, \\
    \Gamma_c &= \frac c{a - \frac d2} = \frac{ F }{a - \frac d2 - F}, \\
    &\text{ тут интересно отметить, что } \Gamma_x = \frac{ c - b}{ d } = \frac{ F^2 }{\cbr{a + \frac d2 - F}\cbr{a - \frac d2 - F}} \ne \Gamma_b \text{ или } \Gamma_c \text{ даже при малых $d$}.
    \\
    S' &= \frac{d \cdot \Gamma_b + d \cdot \Gamma_c}2 \cdot (c - b) = \frac d2 \cbr{\frac{ F }{a + \frac d2 - F} + \frac{ F }{a - \frac d2 - F}} \cdot \cbr{c - b} =  \\
    &=\frac {dF}2 \cbr{\frac 1{a + \frac d2 - F} + \frac 1{a - \frac d2 - F}} \cdot \frac {dF^2}{\cbr{a + \frac d2 - F}\cbr{a - \frac d2 - F}} =  \\
    &=\frac {dF}2 \cdot \frac{a - \frac d2 - F + a + \frac d2 - F}{\cbr{a + \frac d2 - F}\cbr{a - \frac d2 - F}} \cdot \frac {dF^2}{\cbr{a + \frac d2 - F}\cbr{a - \frac d2 - F}} =  \\
    &= \frac {d^2F^3}{2\sqr{a + \frac d2 - F}\sqr{a - \frac d2 - F}} \cdot (2a - 2F) = \frac {d^2F^3(a - F)}{ \sqr{\sqr{a - F} - \frac{d^2}4} }.
    \end{align*}
}

\variantsplitter

\addpersonalvariant{Ярослав Лавровский}

\tasknumber{1}%
\task{%
    Укажите, верны ли утверждения («да» или «нет» слева от каждого утверждения):
    \begin{itemize}
        \item  Изображение предмета в рассеивающей линзе всегда действительное.
        \item  Изображение предмета в рассеивающей линзе всегда перевёрнутое.
        \item  Изображение предмета в рассеивающей линзе всегда увеличенное.
        \item  Оптическая сила собирающей линзы положительна.
    \end{itemize}
}
\answer{%
    $\text{ нет, нет, нет, нет }$
}

\tasknumber{2}%
\task{%
    Запишите известные вам виды классификации изображений.
}
\solutionspace{60pt}

\tasknumber{3}%
\task{%
    В каких линзах можно получить уменьшенное изображение объекта?
}
\answer{%
    $\text{ собирающие и рассеивающие }$
}
\solutionspace{40pt}

\tasknumber{4}%
\task{%
    Какое изображение называют мнимым?
}
\solutionspace{40pt}

\tasknumber{5}%
\task{%
    Есть две линзы, обозначим их 1 и 2.
    Известно что оптическая сила линзы 2 меньше, чем у линзы 1.
    Какая линза сильнее преломляет лучи?
}
\answer{%
    $1$
}
\solutionspace{40pt}

\tasknumber{6}%
\task{%
    Предмет находится на расстоянии $30\,\text{см}$ от собирающей линзы с фокусным расстоянием $8\,\text{см}$.
    Определите тип изображения, расстояние между предметом и его изображением, увеличение предмета.
    Сделайте схематичный рисунок (не обязательно в масштабе, но с сохранением свойств линзы и изображения).
}
\solutionspace{100pt}

\tasknumber{7}%
\task{%
    Объект находится на расстоянии $25\,\text{см}$ от линзы, а его мнимое изображение — в $40\,\text{см}$ от неё.
    Определите увеличение предмета, фокусное расстояние линзы, оптическую силу линзы и её тип.
}
\solutionspace{80pt}

\tasknumber{8}%
\task{%
    (Задача-«гроб»: решать на обратной стороне) Квадрат со стороной $d = 1\,\text{см}$ расположен так, что 2 его стороны параллельны главной оптической оси рассеивающей линзы,
    его центр удален на $h = 4\,\text{см}$ от этой оси и на $a = 12\,\text{см}$ от плоскости линзы.
    Определите площадь изображения квадрата, если фокусное расстояние линзы составляет $F = 25\,\text{см}$.
    % (и сравните с площадью объекта, умноженной на квадрат увеличения центра квадрата).
}
\answer{%
    \begin{align*}
    \frac 1 F &= \frac 1{a + \frac d2} + \frac 1b \implies b = \frac 1{\frac 1 F - \frac 1{a + \frac d2}} = \frac{F(a + \frac d2)}{a + \frac d2 - F}, \\
    \frac 1 F &= \frac 1{a - \frac d2} + \frac 1c \implies c = \frac 1{\frac 1 F - \frac 1{a - \frac d2}} = \frac{F(a - \frac d2)}{a - \frac d2 - F}, \\
    c - b &= \frac{F(a - \frac d2)}{a - \frac d2 - F} - \frac{F(a + \frac d2)}{a + \frac d2 - F} = F\cbr{ \frac{a - \frac d2}{a - \frac d2 - F} - \frac{a + \frac d2}{a + \frac d2 - F} } =  \\
    &= F \cdot \frac{a^2 + \frac {ad}2 - aF - \frac{ad}2 - \frac{d^2}4 + \frac{dF}2 - a^2 + \frac {ad}2 + aF - \frac{ad}2 + \frac{d^2}4 + \frac{dF}2}{\cbr{a + \frac d2 - F}\cbr{a - \frac d2 - F}}= F \cdot \frac {dF}{\cbr{a + \frac d2 - F}\cbr{a - \frac d2 - F}}.
    \\
    \Gamma_b &= \frac b{a + \frac d2} = \frac{ F }{a + \frac d2 - F}, \\
    \Gamma_c &= \frac c{a - \frac d2} = \frac{ F }{a - \frac d2 - F}, \\
    &\text{ тут интересно отметить, что } \Gamma_x = \frac{ c - b}{ d } = \frac{ F^2 }{\cbr{a + \frac d2 - F}\cbr{a - \frac d2 - F}} \ne \Gamma_b \text{ или } \Gamma_c \text{ даже при малых $d$}.
    \\
    S' &= \frac{d \cdot \Gamma_b + d \cdot \Gamma_c}2 \cdot (c - b) = \frac d2 \cbr{\frac{ F }{a + \frac d2 - F} + \frac{ F }{a - \frac d2 - F}} \cdot \cbr{c - b} =  \\
    &=\frac {dF}2 \cbr{\frac 1{a + \frac d2 - F} + \frac 1{a - \frac d2 - F}} \cdot \frac {dF^2}{\cbr{a + \frac d2 - F}\cbr{a - \frac d2 - F}} =  \\
    &=\frac {dF}2 \cdot \frac{a - \frac d2 - F + a + \frac d2 - F}{\cbr{a + \frac d2 - F}\cbr{a - \frac d2 - F}} \cdot \frac {dF^2}{\cbr{a + \frac d2 - F}\cbr{a - \frac d2 - F}} =  \\
    &= \frac {d^2F^3}{2\sqr{a + \frac d2 - F}\sqr{a - \frac d2 - F}} \cdot (2a - 2F) = \frac {d^2F^3(a - F)}{ \sqr{\sqr{a - F} - \frac{d^2}4} }.
    \end{align*}
}

\variantsplitter

\addpersonalvariant{Анастасия Ламанова}

\tasknumber{1}%
\task{%
    Укажите, верны ли утверждения («да» или «нет» слева от каждого утверждения):
    \begin{itemize}
        \item  Изображение предмета в собирающей линзе всегда действительное.
        \item  Изображение предмета в собирающей линзе всегда прямое.
        \item  Изображение предмета в собирающей линзе всегда уменьшенное.
        \item  Оптическая сила рассеивающей линзы отрицательна.
    \end{itemize}
}
\answer{%
    $\text{ нет, нет, нет, нет }$
}

\tasknumber{2}%
\task{%
    Запишите формулу тонкой линзы и сделайте рисунок, указав на нём физические величины из этой формулы.
}
\solutionspace{60pt}

\tasknumber{3}%
\task{%
    В каких линзах можно получить обратное изображение объекта?
}
\answer{%
    $\text{ собирающие }$
}
\solutionspace{40pt}

\tasknumber{4}%
\task{%
    Какое изображение называют действительным?
}
\solutionspace{40pt}

\tasknumber{5}%
\task{%
    Есть две линзы, обозначим их 1 и 2.
    Известно что фокусное расстояние линзы 1 меньше, чем у линзы 2.
    Какая линза сильнее преломляет лучи?
}
\answer{%
    $1$
}
\solutionspace{40pt}

\tasknumber{6}%
\task{%
    Предмет находится на расстоянии $10\,\text{см}$ от собирающей линзы с фокусным расстоянием $25\,\text{см}$.
    Определите тип изображения, расстояние между предметом и его изображением, увеличение предмета.
    Сделайте схематичный рисунок (не обязательно в масштабе, но с сохранением свойств линзы и изображения).
}
\solutionspace{100pt}

\tasknumber{7}%
\task{%
    Объект находится на расстоянии $25\,\text{см}$ от линзы, а его мнимое изображение — в $40\,\text{см}$ от неё.
    Определите увеличение предмета, фокусное расстояние линзы, оптическую силу линзы и её тип.
}
\solutionspace{80pt}

\tasknumber{8}%
\task{%
    (Задача-«гроб»: решать на обратной стороне) Квадрат со стороной $d = 1\,\text{см}$ расположен так, что 2 его стороны параллельны главной оптической оси рассеивающей линзы,
    его центр удален на $h = 5\,\text{см}$ от этой оси и на $a = 15\,\text{см}$ от плоскости линзы.
    Определите площадь изображения квадрата, если фокусное расстояние линзы составляет $F = 25\,\text{см}$.
    % (и сравните с площадью объекта, умноженной на квадрат увеличения центра квадрата).
}
\answer{%
    \begin{align*}
    \frac 1 F &= \frac 1{a + \frac d2} + \frac 1b \implies b = \frac 1{\frac 1 F - \frac 1{a + \frac d2}} = \frac{F(a + \frac d2)}{a + \frac d2 - F}, \\
    \frac 1 F &= \frac 1{a - \frac d2} + \frac 1c \implies c = \frac 1{\frac 1 F - \frac 1{a - \frac d2}} = \frac{F(a - \frac d2)}{a - \frac d2 - F}, \\
    c - b &= \frac{F(a - \frac d2)}{a - \frac d2 - F} - \frac{F(a + \frac d2)}{a + \frac d2 - F} = F\cbr{ \frac{a - \frac d2}{a - \frac d2 - F} - \frac{a + \frac d2}{a + \frac d2 - F} } =  \\
    &= F \cdot \frac{a^2 + \frac {ad}2 - aF - \frac{ad}2 - \frac{d^2}4 + \frac{dF}2 - a^2 + \frac {ad}2 + aF - \frac{ad}2 + \frac{d^2}4 + \frac{dF}2}{\cbr{a + \frac d2 - F}\cbr{a - \frac d2 - F}}= F \cdot \frac {dF}{\cbr{a + \frac d2 - F}\cbr{a - \frac d2 - F}}.
    \\
    \Gamma_b &= \frac b{a + \frac d2} = \frac{ F }{a + \frac d2 - F}, \\
    \Gamma_c &= \frac c{a - \frac d2} = \frac{ F }{a - \frac d2 - F}, \\
    &\text{ тут интересно отметить, что } \Gamma_x = \frac{ c - b}{ d } = \frac{ F^2 }{\cbr{a + \frac d2 - F}\cbr{a - \frac d2 - F}} \ne \Gamma_b \text{ или } \Gamma_c \text{ даже при малых $d$}.
    \\
    S' &= \frac{d \cdot \Gamma_b + d \cdot \Gamma_c}2 \cdot (c - b) = \frac d2 \cbr{\frac{ F }{a + \frac d2 - F} + \frac{ F }{a - \frac d2 - F}} \cdot \cbr{c - b} =  \\
    &=\frac {dF}2 \cbr{\frac 1{a + \frac d2 - F} + \frac 1{a - \frac d2 - F}} \cdot \frac {dF^2}{\cbr{a + \frac d2 - F}\cbr{a - \frac d2 - F}} =  \\
    &=\frac {dF}2 \cdot \frac{a - \frac d2 - F + a + \frac d2 - F}{\cbr{a + \frac d2 - F}\cbr{a - \frac d2 - F}} \cdot \frac {dF^2}{\cbr{a + \frac d2 - F}\cbr{a - \frac d2 - F}} =  \\
    &= \frac {d^2F^3}{2\sqr{a + \frac d2 - F}\sqr{a - \frac d2 - F}} \cdot (2a - 2F) = \frac {d^2F^3(a - F)}{ \sqr{\sqr{a - F} - \frac{d^2}4} }.
    \end{align*}
}

\variantsplitter

\addpersonalvariant{Виктория Легонькова}

\tasknumber{1}%
\task{%
    Укажите, верны ли утверждения («да» или «нет» слева от каждого утверждения):
    \begin{itemize}
        \item  Изображение предмета в собирающей линзе всегда мнимое.
        \item  Изображение предмета в собирающей линзе всегда перевёрнутое.
        \item  Изображение предмета в собирающей линзе всегда увеличенное.
        \item  Оптическая сила собирающей линзы отрицательна.
    \end{itemize}
}
\answer{%
    $\text{ нет, нет, нет, да }$
}

\tasknumber{2}%
\task{%
    Запишите известные вам виды классификации изображений.
}
\solutionspace{60pt}

\tasknumber{3}%
\task{%
    В каких линзах можно получить мнимое изображение объекта?
}
\answer{%
    $\text{ собирающие и рассеивающие }$
}
\solutionspace{40pt}

\tasknumber{4}%
\task{%
    Какое изображение называют мнимым?
}
\solutionspace{40pt}

\tasknumber{5}%
\task{%
    Есть две линзы, обозначим их 1 и 2.
    Известно что оптическая сила линзы 1 меньше, чем у линзы 2.
    Какая линза сильнее преломляет лучи?
}
\answer{%
    $2$
}
\solutionspace{40pt}

\tasknumber{6}%
\task{%
    Предмет находится на расстоянии $20\,\text{см}$ от рассеивающей линзы с фокусным расстоянием $15\,\text{см}$.
    Определите тип изображения, расстояние между предметом и его изображением, увеличение предмета.
    Сделайте схематичный рисунок (не обязательно в масштабе, но с сохранением свойств линзы и изображения).
}
\solutionspace{100pt}

\tasknumber{7}%
\task{%
    Объект находится на расстоянии $115\,\text{см}$ от линзы, а его действительное изображение — в $50\,\text{см}$ от неё.
    Определите увеличение предмета, фокусное расстояние линзы, оптическую силу линзы и её тип.
}
\solutionspace{80pt}

\tasknumber{8}%
\task{%
    (Задача-«гроб»: решать на обратной стороне) Квадрат со стороной $d = 2\,\text{см}$ расположен так, что 2 его стороны параллельны главной оптической оси рассеивающей линзы,
    его центр удален на $h = 6\,\text{см}$ от этой оси и на $a = 10\,\text{см}$ от плоскости линзы.
    Определите площадь изображения квадрата, если фокусное расстояние линзы составляет $F = 25\,\text{см}$.
    % (и сравните с площадью объекта, умноженной на квадрат увеличения центра квадрата).
}
\answer{%
    \begin{align*}
    \frac 1 F &= \frac 1{a + \frac d2} + \frac 1b \implies b = \frac 1{\frac 1 F - \frac 1{a + \frac d2}} = \frac{F(a + \frac d2)}{a + \frac d2 - F}, \\
    \frac 1 F &= \frac 1{a - \frac d2} + \frac 1c \implies c = \frac 1{\frac 1 F - \frac 1{a - \frac d2}} = \frac{F(a - \frac d2)}{a - \frac d2 - F}, \\
    c - b &= \frac{F(a - \frac d2)}{a - \frac d2 - F} - \frac{F(a + \frac d2)}{a + \frac d2 - F} = F\cbr{ \frac{a - \frac d2}{a - \frac d2 - F} - \frac{a + \frac d2}{a + \frac d2 - F} } =  \\
    &= F \cdot \frac{a^2 + \frac {ad}2 - aF - \frac{ad}2 - \frac{d^2}4 + \frac{dF}2 - a^2 + \frac {ad}2 + aF - \frac{ad}2 + \frac{d^2}4 + \frac{dF}2}{\cbr{a + \frac d2 - F}\cbr{a - \frac d2 - F}}= F \cdot \frac {dF}{\cbr{a + \frac d2 - F}\cbr{a - \frac d2 - F}}.
    \\
    \Gamma_b &= \frac b{a + \frac d2} = \frac{ F }{a + \frac d2 - F}, \\
    \Gamma_c &= \frac c{a - \frac d2} = \frac{ F }{a - \frac d2 - F}, \\
    &\text{ тут интересно отметить, что } \Gamma_x = \frac{ c - b}{ d } = \frac{ F^2 }{\cbr{a + \frac d2 - F}\cbr{a - \frac d2 - F}} \ne \Gamma_b \text{ или } \Gamma_c \text{ даже при малых $d$}.
    \\
    S' &= \frac{d \cdot \Gamma_b + d \cdot \Gamma_c}2 \cdot (c - b) = \frac d2 \cbr{\frac{ F }{a + \frac d2 - F} + \frac{ F }{a - \frac d2 - F}} \cdot \cbr{c - b} =  \\
    &=\frac {dF}2 \cbr{\frac 1{a + \frac d2 - F} + \frac 1{a - \frac d2 - F}} \cdot \frac {dF^2}{\cbr{a + \frac d2 - F}\cbr{a - \frac d2 - F}} =  \\
    &=\frac {dF}2 \cdot \frac{a - \frac d2 - F + a + \frac d2 - F}{\cbr{a + \frac d2 - F}\cbr{a - \frac d2 - F}} \cdot \frac {dF^2}{\cbr{a + \frac d2 - F}\cbr{a - \frac d2 - F}} =  \\
    &= \frac {d^2F^3}{2\sqr{a + \frac d2 - F}\sqr{a - \frac d2 - F}} \cdot (2a - 2F) = \frac {d^2F^3(a - F)}{ \sqr{\sqr{a - F} - \frac{d^2}4} }.
    \end{align*}
}

\variantsplitter

\addpersonalvariant{Семён Мартынов}

\tasknumber{1}%
\task{%
    Укажите, верны ли утверждения («да» или «нет» слева от каждого утверждения):
    \begin{itemize}
        \item  Изображение предмета в рассеивающей линзе всегда мнимое.
        \item  Изображение предмета в рассеивающей линзе всегда перевёрнутое.
        \item  Изображение предмета в рассеивающей линзе всегда увеличенное.
        \item  Оптическая сила рассеивающей линзы отрицательна.
    \end{itemize}
}
\answer{%
    $\text{ да, нет, нет, нет }$
}

\tasknumber{2}%
\task{%
    Запишите известные вам виды классификации изображений.
}
\solutionspace{60pt}

\tasknumber{3}%
\task{%
    В каких линзах можно получить действительное изображение объекта?
}
\answer{%
    $\text{ собирающие }$
}
\solutionspace{40pt}

\tasknumber{4}%
\task{%
    Какое изображение называют действительным?
}
\solutionspace{40pt}

\tasknumber{5}%
\task{%
    Есть две линзы, обозначим их 1 и 2.
    Известно что фокусное расстояние линзы 2 больше, чем у линзы 1.
    Какая линза сильнее преломляет лучи?
}
\answer{%
    $1$
}
\solutionspace{40pt}

\tasknumber{6}%
\task{%
    Предмет находится на расстоянии $10\,\text{см}$ от собирающей линзы с фокусным расстоянием $40\,\text{см}$.
    Определите тип изображения, расстояние между предметом и его изображением, увеличение предмета.
    Сделайте схематичный рисунок (не обязательно в масштабе, но с сохранением свойств линзы и изображения).
}
\solutionspace{100pt}

\tasknumber{7}%
\task{%
    Объект находится на расстоянии $25\,\text{см}$ от линзы, а его действительное изображение — в $40\,\text{см}$ от неё.
    Определите увеличение предмета, фокусное расстояние линзы, оптическую силу линзы и её тип.
}
\solutionspace{80pt}

\tasknumber{8}%
\task{%
    (Задача-«гроб»: решать на обратной стороне) Квадрат со стороной $d = 3\,\text{см}$ расположен так, что 2 его стороны параллельны главной оптической оси рассеивающей линзы,
    его центр удален на $h = 5\,\text{см}$ от этой оси и на $a = 12\,\text{см}$ от плоскости линзы.
    Определите площадь изображения квадрата, если фокусное расстояние линзы составляет $F = 18\,\text{см}$.
    % (и сравните с площадью объекта, умноженной на квадрат увеличения центра квадрата).
}
\answer{%
    \begin{align*}
    \frac 1 F &= \frac 1{a + \frac d2} + \frac 1b \implies b = \frac 1{\frac 1 F - \frac 1{a + \frac d2}} = \frac{F(a + \frac d2)}{a + \frac d2 - F}, \\
    \frac 1 F &= \frac 1{a - \frac d2} + \frac 1c \implies c = \frac 1{\frac 1 F - \frac 1{a - \frac d2}} = \frac{F(a - \frac d2)}{a - \frac d2 - F}, \\
    c - b &= \frac{F(a - \frac d2)}{a - \frac d2 - F} - \frac{F(a + \frac d2)}{a + \frac d2 - F} = F\cbr{ \frac{a - \frac d2}{a - \frac d2 - F} - \frac{a + \frac d2}{a + \frac d2 - F} } =  \\
    &= F \cdot \frac{a^2 + \frac {ad}2 - aF - \frac{ad}2 - \frac{d^2}4 + \frac{dF}2 - a^2 + \frac {ad}2 + aF - \frac{ad}2 + \frac{d^2}4 + \frac{dF}2}{\cbr{a + \frac d2 - F}\cbr{a - \frac d2 - F}}= F \cdot \frac {dF}{\cbr{a + \frac d2 - F}\cbr{a - \frac d2 - F}}.
    \\
    \Gamma_b &= \frac b{a + \frac d2} = \frac{ F }{a + \frac d2 - F}, \\
    \Gamma_c &= \frac c{a - \frac d2} = \frac{ F }{a - \frac d2 - F}, \\
    &\text{ тут интересно отметить, что } \Gamma_x = \frac{ c - b}{ d } = \frac{ F^2 }{\cbr{a + \frac d2 - F}\cbr{a - \frac d2 - F}} \ne \Gamma_b \text{ или } \Gamma_c \text{ даже при малых $d$}.
    \\
    S' &= \frac{d \cdot \Gamma_b + d \cdot \Gamma_c}2 \cdot (c - b) = \frac d2 \cbr{\frac{ F }{a + \frac d2 - F} + \frac{ F }{a - \frac d2 - F}} \cdot \cbr{c - b} =  \\
    &=\frac {dF}2 \cbr{\frac 1{a + \frac d2 - F} + \frac 1{a - \frac d2 - F}} \cdot \frac {dF^2}{\cbr{a + \frac d2 - F}\cbr{a - \frac d2 - F}} =  \\
    &=\frac {dF}2 \cdot \frac{a - \frac d2 - F + a + \frac d2 - F}{\cbr{a + \frac d2 - F}\cbr{a - \frac d2 - F}} \cdot \frac {dF^2}{\cbr{a + \frac d2 - F}\cbr{a - \frac d2 - F}} =  \\
    &= \frac {d^2F^3}{2\sqr{a + \frac d2 - F}\sqr{a - \frac d2 - F}} \cdot (2a - 2F) = \frac {d^2F^3(a - F)}{ \sqr{\sqr{a - F} - \frac{d^2}4} }.
    \end{align*}
}

\variantsplitter

\addpersonalvariant{Варвара Минаева}

\tasknumber{1}%
\task{%
    Укажите, верны ли утверждения («да» или «нет» слева от каждого утверждения):
    \begin{itemize}
        \item  Изображение предмета в рассеивающей линзе всегда мнимое.
        \item  Изображение предмета в рассеивающей линзе всегда прямое.
        \item  Изображение предмета в рассеивающей линзе всегда увеличенное.
        \item  Оптическая сила собирающей линзы отрицательна.
    \end{itemize}
}
\answer{%
    $\text{ да, да, нет, да }$
}

\tasknumber{2}%
\task{%
    Запишите известные вам виды классификации изображений.
}
\solutionspace{60pt}

\tasknumber{3}%
\task{%
    В каких линзах можно получить мнимое изображение объекта?
}
\answer{%
    $\text{ собирающие и рассеивающие }$
}
\solutionspace{40pt}

\tasknumber{4}%
\task{%
    Какое изображение называют мнимым?
}
\solutionspace{40pt}

\tasknumber{5}%
\task{%
    Есть две линзы, обозначим их 1 и 2.
    Известно что фокусное расстояние линзы 2 меньше, чем у линзы 1.
    Какая линза сильнее преломляет лучи?
}
\answer{%
    $2$
}
\solutionspace{40pt}

\tasknumber{6}%
\task{%
    Предмет находится на расстоянии $20\,\text{см}$ от собирающей линзы с фокусным расстоянием $6\,\text{см}$.
    Определите тип изображения, расстояние между предметом и его изображением, увеличение предмета.
    Сделайте схематичный рисунок (не обязательно в масштабе, но с сохранением свойств линзы и изображения).
}
\solutionspace{100pt}

\tasknumber{7}%
\task{%
    Объект находится на расстоянии $115\,\text{см}$ от линзы, а его мнимое изображение — в $30\,\text{см}$ от неё.
    Определите увеличение предмета, фокусное расстояние линзы, оптическую силу линзы и её тип.
}
\solutionspace{80pt}

\tasknumber{8}%
\task{%
    (Задача-«гроб»: решать на обратной стороне) Квадрат со стороной $d = 1\,\text{см}$ расположен так, что 2 его стороны параллельны главной оптической оси собирающей линзы,
    его центр удален на $h = 6\,\text{см}$ от этой оси и на $a = 15\,\text{см}$ от плоскости линзы.
    Определите площадь изображения квадрата, если фокусное расстояние линзы составляет $F = 25\,\text{см}$.
    % (и сравните с площадью объекта, умноженной на квадрат увеличения центра квадрата).
}
\answer{%
    \begin{align*}
    \frac 1 F &= \frac 1{a + \frac d2} + \frac 1b \implies b = \frac 1{\frac 1 F - \frac 1{a + \frac d2}} = \frac{F(a + \frac d2)}{a + \frac d2 - F}, \\
    \frac 1 F &= \frac 1{a - \frac d2} + \frac 1c \implies c = \frac 1{\frac 1 F - \frac 1{a - \frac d2}} = \frac{F(a - \frac d2)}{a - \frac d2 - F}, \\
    c - b &= \frac{F(a - \frac d2)}{a - \frac d2 - F} - \frac{F(a + \frac d2)}{a + \frac d2 - F} = F\cbr{ \frac{a - \frac d2}{a - \frac d2 - F} - \frac{a + \frac d2}{a + \frac d2 - F} } =  \\
    &= F \cdot \frac{a^2 + \frac {ad}2 - aF - \frac{ad}2 - \frac{d^2}4 + \frac{dF}2 - a^2 + \frac {ad}2 + aF - \frac{ad}2 + \frac{d^2}4 + \frac{dF}2}{\cbr{a + \frac d2 - F}\cbr{a - \frac d2 - F}}= F \cdot \frac {dF}{\cbr{a + \frac d2 - F}\cbr{a - \frac d2 - F}}.
    \\
    \Gamma_b &= \frac b{a + \frac d2} = \frac{ F }{a + \frac d2 - F}, \\
    \Gamma_c &= \frac c{a - \frac d2} = \frac{ F }{a - \frac d2 - F}, \\
    &\text{ тут интересно отметить, что } \Gamma_x = \frac{ c - b}{ d } = \frac{ F^2 }{\cbr{a + \frac d2 - F}\cbr{a - \frac d2 - F}} \ne \Gamma_b \text{ или } \Gamma_c \text{ даже при малых $d$}.
    \\
    S' &= \frac{d \cdot \Gamma_b + d \cdot \Gamma_c}2 \cdot (c - b) = \frac d2 \cbr{\frac{ F }{a + \frac d2 - F} + \frac{ F }{a - \frac d2 - F}} \cdot \cbr{c - b} =  \\
    &=\frac {dF}2 \cbr{\frac 1{a + \frac d2 - F} + \frac 1{a - \frac d2 - F}} \cdot \frac {dF^2}{\cbr{a + \frac d2 - F}\cbr{a - \frac d2 - F}} =  \\
    &=\frac {dF}2 \cdot \frac{a - \frac d2 - F + a + \frac d2 - F}{\cbr{a + \frac d2 - F}\cbr{a - \frac d2 - F}} \cdot \frac {dF^2}{\cbr{a + \frac d2 - F}\cbr{a - \frac d2 - F}} =  \\
    &= \frac {d^2F^3}{2\sqr{a + \frac d2 - F}\sqr{a - \frac d2 - F}} \cdot (2a - 2F) = \frac {d^2F^3(a - F)}{ \sqr{\sqr{a - F} - \frac{d^2}4} }.
    \end{align*}
}

\variantsplitter

\addpersonalvariant{Леонид Никитин}

\tasknumber{1}%
\task{%
    Укажите, верны ли утверждения («да» или «нет» слева от каждого утверждения):
    \begin{itemize}
        \item  Изображение предмета в рассеивающей линзе всегда действительное.
        \item  Изображение предмета в рассеивающей линзе всегда перевёрнутое.
        \item  Изображение предмета в рассеивающей линзе всегда увеличенное.
        \item  Оптическая сила собирающей линзы отрицательна.
    \end{itemize}
}
\answer{%
    $\text{ нет, нет, нет, да }$
}

\tasknumber{2}%
\task{%
    Запишите формулу тонкой линзы и сделайте рисунок, указав на нём физические величины из этой формулы.
}
\solutionspace{60pt}

\tasknumber{3}%
\task{%
    В каких линзах можно получить прямое изображение объекта?
}
\answer{%
    $\text{ собирающие и рассеивающие }$
}
\solutionspace{40pt}

\tasknumber{4}%
\task{%
    Какое изображение называют мнимым?
}
\solutionspace{40pt}

\tasknumber{5}%
\task{%
    Есть две линзы, обозначим их 1 и 2.
    Известно что фокусное расстояние линзы 2 больше, чем у линзы 1.
    Какая линза сильнее преломляет лучи?
}
\answer{%
    $1$
}
\solutionspace{40pt}

\tasknumber{6}%
\task{%
    Предмет находится на расстоянии $30\,\text{см}$ от рассеивающей линзы с фокусным расстоянием $25\,\text{см}$.
    Определите тип изображения, расстояние между предметом и его изображением, увеличение предмета.
    Сделайте схематичный рисунок (не обязательно в масштабе, но с сохранением свойств линзы и изображения).
}
\solutionspace{100pt}

\tasknumber{7}%
\task{%
    Объект находится на расстоянии $45\,\text{см}$ от линзы, а его действительное изображение — в $10\,\text{см}$ от неё.
    Определите увеличение предмета, фокусное расстояние линзы, оптическую силу линзы и её тип.
}
\solutionspace{80pt}

\tasknumber{8}%
\task{%
    (Задача-«гроб»: решать на обратной стороне) Квадрат со стороной $d = 1\,\text{см}$ расположен так, что 2 его стороны параллельны главной оптической оси собирающей линзы,
    его центр удален на $h = 5\,\text{см}$ от этой оси и на $a = 12\,\text{см}$ от плоскости линзы.
    Определите площадь изображения квадрата, если фокусное расстояние линзы составляет $F = 18\,\text{см}$.
    % (и сравните с площадью объекта, умноженной на квадрат увеличения центра квадрата).
}
\answer{%
    \begin{align*}
    \frac 1 F &= \frac 1{a + \frac d2} + \frac 1b \implies b = \frac 1{\frac 1 F - \frac 1{a + \frac d2}} = \frac{F(a + \frac d2)}{a + \frac d2 - F}, \\
    \frac 1 F &= \frac 1{a - \frac d2} + \frac 1c \implies c = \frac 1{\frac 1 F - \frac 1{a - \frac d2}} = \frac{F(a - \frac d2)}{a - \frac d2 - F}, \\
    c - b &= \frac{F(a - \frac d2)}{a - \frac d2 - F} - \frac{F(a + \frac d2)}{a + \frac d2 - F} = F\cbr{ \frac{a - \frac d2}{a - \frac d2 - F} - \frac{a + \frac d2}{a + \frac d2 - F} } =  \\
    &= F \cdot \frac{a^2 + \frac {ad}2 - aF - \frac{ad}2 - \frac{d^2}4 + \frac{dF}2 - a^2 + \frac {ad}2 + aF - \frac{ad}2 + \frac{d^2}4 + \frac{dF}2}{\cbr{a + \frac d2 - F}\cbr{a - \frac d2 - F}}= F \cdot \frac {dF}{\cbr{a + \frac d2 - F}\cbr{a - \frac d2 - F}}.
    \\
    \Gamma_b &= \frac b{a + \frac d2} = \frac{ F }{a + \frac d2 - F}, \\
    \Gamma_c &= \frac c{a - \frac d2} = \frac{ F }{a - \frac d2 - F}, \\
    &\text{ тут интересно отметить, что } \Gamma_x = \frac{ c - b}{ d } = \frac{ F^2 }{\cbr{a + \frac d2 - F}\cbr{a - \frac d2 - F}} \ne \Gamma_b \text{ или } \Gamma_c \text{ даже при малых $d$}.
    \\
    S' &= \frac{d \cdot \Gamma_b + d \cdot \Gamma_c}2 \cdot (c - b) = \frac d2 \cbr{\frac{ F }{a + \frac d2 - F} + \frac{ F }{a - \frac d2 - F}} \cdot \cbr{c - b} =  \\
    &=\frac {dF}2 \cbr{\frac 1{a + \frac d2 - F} + \frac 1{a - \frac d2 - F}} \cdot \frac {dF^2}{\cbr{a + \frac d2 - F}\cbr{a - \frac d2 - F}} =  \\
    &=\frac {dF}2 \cdot \frac{a - \frac d2 - F + a + \frac d2 - F}{\cbr{a + \frac d2 - F}\cbr{a - \frac d2 - F}} \cdot \frac {dF^2}{\cbr{a + \frac d2 - F}\cbr{a - \frac d2 - F}} =  \\
    &= \frac {d^2F^3}{2\sqr{a + \frac d2 - F}\sqr{a - \frac d2 - F}} \cdot (2a - 2F) = \frac {d^2F^3(a - F)}{ \sqr{\sqr{a - F} - \frac{d^2}4} }.
    \end{align*}
}

\variantsplitter

\addpersonalvariant{Тимофей Полетаев}

\tasknumber{1}%
\task{%
    Укажите, верны ли утверждения («да» или «нет» слева от каждого утверждения):
    \begin{itemize}
        \item  Изображение предмета в рассеивающей линзе всегда мнимое.
        \item  Изображение предмета в рассеивающей линзе всегда перевёрнутое.
        \item  Изображение предмета в рассеивающей линзе всегда увеличенное.
        \item  Оптическая сила собирающей линзы отрицательна.
    \end{itemize}
}
\answer{%
    $\text{ да, нет, нет, да }$
}

\tasknumber{2}%
\task{%
    Запишите формулу тонкой линзы и сделайте рисунок, указав на нём физические величины из этой формулы.
}
\solutionspace{60pt}

\tasknumber{3}%
\task{%
    В каких линзах можно получить мнимое изображение объекта?
}
\answer{%
    $\text{ собирающие и рассеивающие }$
}
\solutionspace{40pt}

\tasknumber{4}%
\task{%
    Какое изображение называют мнимым?
}
\solutionspace{40pt}

\tasknumber{5}%
\task{%
    Есть две линзы, обозначим их 1 и 2.
    Известно что фокусное расстояние линзы 1 меньше, чем у линзы 2.
    Какая линза сильнее преломляет лучи?
}
\answer{%
    $1$
}
\solutionspace{40pt}

\tasknumber{6}%
\task{%
    Предмет находится на расстоянии $30\,\text{см}$ от рассеивающей линзы с фокусным расстоянием $6\,\text{см}$.
    Определите тип изображения, расстояние между предметом и его изображением, увеличение предмета.
    Сделайте схематичный рисунок (не обязательно в масштабе, но с сохранением свойств линзы и изображения).
}
\solutionspace{100pt}

\tasknumber{7}%
\task{%
    Объект находится на расстоянии $45\,\text{см}$ от линзы, а его мнимое изображение — в $30\,\text{см}$ от неё.
    Определите увеличение предмета, фокусное расстояние линзы, оптическую силу линзы и её тип.
}
\solutionspace{80pt}

\tasknumber{8}%
\task{%
    (Задача-«гроб»: решать на обратной стороне) Квадрат со стороной $d = 3\,\text{см}$ расположен так, что 2 его стороны параллельны главной оптической оси собирающей линзы,
    его центр удален на $h = 6\,\text{см}$ от этой оси и на $a = 15\,\text{см}$ от плоскости линзы.
    Определите площадь изображения квадрата, если фокусное расстояние линзы составляет $F = 25\,\text{см}$.
    % (и сравните с площадью объекта, умноженной на квадрат увеличения центра квадрата).
}
\answer{%
    \begin{align*}
    \frac 1 F &= \frac 1{a + \frac d2} + \frac 1b \implies b = \frac 1{\frac 1 F - \frac 1{a + \frac d2}} = \frac{F(a + \frac d2)}{a + \frac d2 - F}, \\
    \frac 1 F &= \frac 1{a - \frac d2} + \frac 1c \implies c = \frac 1{\frac 1 F - \frac 1{a - \frac d2}} = \frac{F(a - \frac d2)}{a - \frac d2 - F}, \\
    c - b &= \frac{F(a - \frac d2)}{a - \frac d2 - F} - \frac{F(a + \frac d2)}{a + \frac d2 - F} = F\cbr{ \frac{a - \frac d2}{a - \frac d2 - F} - \frac{a + \frac d2}{a + \frac d2 - F} } =  \\
    &= F \cdot \frac{a^2 + \frac {ad}2 - aF - \frac{ad}2 - \frac{d^2}4 + \frac{dF}2 - a^2 + \frac {ad}2 + aF - \frac{ad}2 + \frac{d^2}4 + \frac{dF}2}{\cbr{a + \frac d2 - F}\cbr{a - \frac d2 - F}}= F \cdot \frac {dF}{\cbr{a + \frac d2 - F}\cbr{a - \frac d2 - F}}.
    \\
    \Gamma_b &= \frac b{a + \frac d2} = \frac{ F }{a + \frac d2 - F}, \\
    \Gamma_c &= \frac c{a - \frac d2} = \frac{ F }{a - \frac d2 - F}, \\
    &\text{ тут интересно отметить, что } \Gamma_x = \frac{ c - b}{ d } = \frac{ F^2 }{\cbr{a + \frac d2 - F}\cbr{a - \frac d2 - F}} \ne \Gamma_b \text{ или } \Gamma_c \text{ даже при малых $d$}.
    \\
    S' &= \frac{d \cdot \Gamma_b + d \cdot \Gamma_c}2 \cdot (c - b) = \frac d2 \cbr{\frac{ F }{a + \frac d2 - F} + \frac{ F }{a - \frac d2 - F}} \cdot \cbr{c - b} =  \\
    &=\frac {dF}2 \cbr{\frac 1{a + \frac d2 - F} + \frac 1{a - \frac d2 - F}} \cdot \frac {dF^2}{\cbr{a + \frac d2 - F}\cbr{a - \frac d2 - F}} =  \\
    &=\frac {dF}2 \cdot \frac{a - \frac d2 - F + a + \frac d2 - F}{\cbr{a + \frac d2 - F}\cbr{a - \frac d2 - F}} \cdot \frac {dF^2}{\cbr{a + \frac d2 - F}\cbr{a - \frac d2 - F}} =  \\
    &= \frac {d^2F^3}{2\sqr{a + \frac d2 - F}\sqr{a - \frac d2 - F}} \cdot (2a - 2F) = \frac {d^2F^3(a - F)}{ \sqr{\sqr{a - F} - \frac{d^2}4} }.
    \end{align*}
}

\variantsplitter

\addpersonalvariant{Андрей Рожков}

\tasknumber{1}%
\task{%
    Укажите, верны ли утверждения («да» или «нет» слева от каждого утверждения):
    \begin{itemize}
        \item  Изображение предмета в рассеивающей линзе всегда мнимое.
        \item  Изображение предмета в рассеивающей линзе всегда перевёрнутое.
        \item  Изображение предмета в рассеивающей линзе всегда увеличенное.
        \item  Оптическая сила рассеивающей линзы положительна.
    \end{itemize}
}
\answer{%
    $\text{ да, нет, нет, да }$
}

\tasknumber{2}%
\task{%
    Запишите известные вам виды классификации изображений.
}
\solutionspace{60pt}

\tasknumber{3}%
\task{%
    В каких линзах можно получить уменьшенное изображение объекта?
}
\answer{%
    $\text{ собирающие и рассеивающие }$
}
\solutionspace{40pt}

\tasknumber{4}%
\task{%
    Какое изображение называют мнимым?
}
\solutionspace{40pt}

\tasknumber{5}%
\task{%
    Есть две линзы, обозначим их 1 и 2.
    Известно что фокусное расстояние линзы 1 меньше, чем у линзы 2.
    Какая линза сильнее преломляет лучи?
}
\answer{%
    $1$
}
\solutionspace{40pt}

\tasknumber{6}%
\task{%
    Предмет находится на расстоянии $10\,\text{см}$ от собирающей линзы с фокусным расстоянием $12\,\text{см}$.
    Определите тип изображения, расстояние между предметом и его изображением, увеличение предмета.
    Сделайте схематичный рисунок (не обязательно в масштабе, но с сохранением свойств линзы и изображения).
}
\solutionspace{100pt}

\tasknumber{7}%
\task{%
    Объект находится на расстоянии $25\,\text{см}$ от линзы, а его действительное изображение — в $30\,\text{см}$ от неё.
    Определите увеличение предмета, фокусное расстояние линзы, оптическую силу линзы и её тип.
}
\solutionspace{80pt}

\tasknumber{8}%
\task{%
    (Задача-«гроб»: решать на обратной стороне) Квадрат со стороной $d = 3\,\text{см}$ расположен так, что 2 его стороны параллельны главной оптической оси рассеивающей линзы,
    его центр удален на $h = 4\,\text{см}$ от этой оси и на $a = 15\,\text{см}$ от плоскости линзы.
    Определите площадь изображения квадрата, если фокусное расстояние линзы составляет $F = 25\,\text{см}$.
    % (и сравните с площадью объекта, умноженной на квадрат увеличения центра квадрата).
}
\answer{%
    \begin{align*}
    \frac 1 F &= \frac 1{a + \frac d2} + \frac 1b \implies b = \frac 1{\frac 1 F - \frac 1{a + \frac d2}} = \frac{F(a + \frac d2)}{a + \frac d2 - F}, \\
    \frac 1 F &= \frac 1{a - \frac d2} + \frac 1c \implies c = \frac 1{\frac 1 F - \frac 1{a - \frac d2}} = \frac{F(a - \frac d2)}{a - \frac d2 - F}, \\
    c - b &= \frac{F(a - \frac d2)}{a - \frac d2 - F} - \frac{F(a + \frac d2)}{a + \frac d2 - F} = F\cbr{ \frac{a - \frac d2}{a - \frac d2 - F} - \frac{a + \frac d2}{a + \frac d2 - F} } =  \\
    &= F \cdot \frac{a^2 + \frac {ad}2 - aF - \frac{ad}2 - \frac{d^2}4 + \frac{dF}2 - a^2 + \frac {ad}2 + aF - \frac{ad}2 + \frac{d^2}4 + \frac{dF}2}{\cbr{a + \frac d2 - F}\cbr{a - \frac d2 - F}}= F \cdot \frac {dF}{\cbr{a + \frac d2 - F}\cbr{a - \frac d2 - F}}.
    \\
    \Gamma_b &= \frac b{a + \frac d2} = \frac{ F }{a + \frac d2 - F}, \\
    \Gamma_c &= \frac c{a - \frac d2} = \frac{ F }{a - \frac d2 - F}, \\
    &\text{ тут интересно отметить, что } \Gamma_x = \frac{ c - b}{ d } = \frac{ F^2 }{\cbr{a + \frac d2 - F}\cbr{a - \frac d2 - F}} \ne \Gamma_b \text{ или } \Gamma_c \text{ даже при малых $d$}.
    \\
    S' &= \frac{d \cdot \Gamma_b + d \cdot \Gamma_c}2 \cdot (c - b) = \frac d2 \cbr{\frac{ F }{a + \frac d2 - F} + \frac{ F }{a - \frac d2 - F}} \cdot \cbr{c - b} =  \\
    &=\frac {dF}2 \cbr{\frac 1{a + \frac d2 - F} + \frac 1{a - \frac d2 - F}} \cdot \frac {dF^2}{\cbr{a + \frac d2 - F}\cbr{a - \frac d2 - F}} =  \\
    &=\frac {dF}2 \cdot \frac{a - \frac d2 - F + a + \frac d2 - F}{\cbr{a + \frac d2 - F}\cbr{a - \frac d2 - F}} \cdot \frac {dF^2}{\cbr{a + \frac d2 - F}\cbr{a - \frac d2 - F}} =  \\
    &= \frac {d^2F^3}{2\sqr{a + \frac d2 - F}\sqr{a - \frac d2 - F}} \cdot (2a - 2F) = \frac {d^2F^3(a - F)}{ \sqr{\sqr{a - F} - \frac{d^2}4} }.
    \end{align*}
}

\variantsplitter

\addpersonalvariant{Рената Таржиманова}

\tasknumber{1}%
\task{%
    Укажите, верны ли утверждения («да» или «нет» слева от каждого утверждения):
    \begin{itemize}
        \item  Изображение предмета в рассеивающей линзе всегда мнимое.
        \item  Изображение предмета в рассеивающей линзе всегда перевёрнутое.
        \item  Изображение предмета в рассеивающей линзе всегда увеличенное.
        \item  Оптическая сила рассеивающей линзы положительна.
    \end{itemize}
}
\answer{%
    $\text{ да, нет, нет, да }$
}

\tasknumber{2}%
\task{%
    Запишите формулу тонкой линзы и сделайте рисунок, указав на нём физические величины из этой формулы.
}
\solutionspace{60pt}

\tasknumber{3}%
\task{%
    В каких линзах можно получить мнимое изображение объекта?
}
\answer{%
    $\text{ собирающие и рассеивающие }$
}
\solutionspace{40pt}

\tasknumber{4}%
\task{%
    Какое изображение называют мнимым?
}
\solutionspace{40pt}

\tasknumber{5}%
\task{%
    Есть две линзы, обозначим их 1 и 2.
    Известно что оптическая сила линзы 1 больше, чем у линзы 2.
    Какая линза сильнее преломляет лучи?
}
\answer{%
    $1$
}
\solutionspace{40pt}

\tasknumber{6}%
\task{%
    Предмет находится на расстоянии $30\,\text{см}$ от рассеивающей линзы с фокусным расстоянием $6\,\text{см}$.
    Определите тип изображения, расстояние между предметом и его изображением, увеличение предмета.
    Сделайте схематичный рисунок (не обязательно в масштабе, но с сохранением свойств линзы и изображения).
}
\solutionspace{100pt}

\tasknumber{7}%
\task{%
    Объект находится на расстоянии $45\,\text{см}$ от линзы, а его действительное изображение — в $50\,\text{см}$ от неё.
    Определите увеличение предмета, фокусное расстояние линзы, оптическую силу линзы и её тип.
}
\solutionspace{80pt}

\tasknumber{8}%
\task{%
    (Задача-«гроб»: решать на обратной стороне) Квадрат со стороной $d = 2\,\text{см}$ расположен так, что 2 его стороны параллельны главной оптической оси собирающей линзы,
    его центр удален на $h = 5\,\text{см}$ от этой оси и на $a = 10\,\text{см}$ от плоскости линзы.
    Определите площадь изображения квадрата, если фокусное расстояние линзы составляет $F = 20\,\text{см}$.
    % (и сравните с площадью объекта, умноженной на квадрат увеличения центра квадрата).
}
\answer{%
    \begin{align*}
    \frac 1 F &= \frac 1{a + \frac d2} + \frac 1b \implies b = \frac 1{\frac 1 F - \frac 1{a + \frac d2}} = \frac{F(a + \frac d2)}{a + \frac d2 - F}, \\
    \frac 1 F &= \frac 1{a - \frac d2} + \frac 1c \implies c = \frac 1{\frac 1 F - \frac 1{a - \frac d2}} = \frac{F(a - \frac d2)}{a - \frac d2 - F}, \\
    c - b &= \frac{F(a - \frac d2)}{a - \frac d2 - F} - \frac{F(a + \frac d2)}{a + \frac d2 - F} = F\cbr{ \frac{a - \frac d2}{a - \frac d2 - F} - \frac{a + \frac d2}{a + \frac d2 - F} } =  \\
    &= F \cdot \frac{a^2 + \frac {ad}2 - aF - \frac{ad}2 - \frac{d^2}4 + \frac{dF}2 - a^2 + \frac {ad}2 + aF - \frac{ad}2 + \frac{d^2}4 + \frac{dF}2}{\cbr{a + \frac d2 - F}\cbr{a - \frac d2 - F}}= F \cdot \frac {dF}{\cbr{a + \frac d2 - F}\cbr{a - \frac d2 - F}}.
    \\
    \Gamma_b &= \frac b{a + \frac d2} = \frac{ F }{a + \frac d2 - F}, \\
    \Gamma_c &= \frac c{a - \frac d2} = \frac{ F }{a - \frac d2 - F}, \\
    &\text{ тут интересно отметить, что } \Gamma_x = \frac{ c - b}{ d } = \frac{ F^2 }{\cbr{a + \frac d2 - F}\cbr{a - \frac d2 - F}} \ne \Gamma_b \text{ или } \Gamma_c \text{ даже при малых $d$}.
    \\
    S' &= \frac{d \cdot \Gamma_b + d \cdot \Gamma_c}2 \cdot (c - b) = \frac d2 \cbr{\frac{ F }{a + \frac d2 - F} + \frac{ F }{a - \frac d2 - F}} \cdot \cbr{c - b} =  \\
    &=\frac {dF}2 \cbr{\frac 1{a + \frac d2 - F} + \frac 1{a - \frac d2 - F}} \cdot \frac {dF^2}{\cbr{a + \frac d2 - F}\cbr{a - \frac d2 - F}} =  \\
    &=\frac {dF}2 \cdot \frac{a - \frac d2 - F + a + \frac d2 - F}{\cbr{a + \frac d2 - F}\cbr{a - \frac d2 - F}} \cdot \frac {dF^2}{\cbr{a + \frac d2 - F}\cbr{a - \frac d2 - F}} =  \\
    &= \frac {d^2F^3}{2\sqr{a + \frac d2 - F}\sqr{a - \frac d2 - F}} \cdot (2a - 2F) = \frac {d^2F^3(a - F)}{ \sqr{\sqr{a - F} - \frac{d^2}4} }.
    \end{align*}
}

\variantsplitter

\addpersonalvariant{Андрей Щербаков}

\tasknumber{1}%
\task{%
    Укажите, верны ли утверждения («да» или «нет» слева от каждого утверждения):
    \begin{itemize}
        \item  Изображение предмета в собирающей линзе всегда мнимое.
        \item  Изображение предмета в собирающей линзе всегда прямое.
        \item  Изображение предмета в собирающей линзе всегда увеличенное.
        \item  Оптическая сила рассеивающей линзы отрицательна.
    \end{itemize}
}
\answer{%
    $\text{ нет, нет, нет, нет }$
}

\tasknumber{2}%
\task{%
    Запишите известные вам виды классификации изображений.
}
\solutionspace{60pt}

\tasknumber{3}%
\task{%
    В каких линзах можно получить увеличенное изображение объекта?
}
\answer{%
    $\text{ рассеивающие }$
}
\solutionspace{40pt}

\tasknumber{4}%
\task{%
    Какое изображение называют действительным?
}
\solutionspace{40pt}

\tasknumber{5}%
\task{%
    Есть две линзы, обозначим их 1 и 2.
    Известно что фокусное расстояние линзы 2 больше, чем у линзы 1.
    Какая линза сильнее преломляет лучи?
}
\answer{%
    $1$
}
\solutionspace{40pt}

\tasknumber{6}%
\task{%
    Предмет находится на расстоянии $20\,\text{см}$ от рассеивающей линзы с фокусным расстоянием $40\,\text{см}$.
    Определите тип изображения, расстояние между предметом и его изображением, увеличение предмета.
    Сделайте схематичный рисунок (не обязательно в масштабе, но с сохранением свойств линзы и изображения).
}
\solutionspace{100pt}

\tasknumber{7}%
\task{%
    Объект находится на расстоянии $25\,\text{см}$ от линзы, а его мнимое изображение — в $20\,\text{см}$ от неё.
    Определите увеличение предмета, фокусное расстояние линзы, оптическую силу линзы и её тип.
}
\solutionspace{80pt}

\tasknumber{8}%
\task{%
    (Задача-«гроб»: решать на обратной стороне) Квадрат со стороной $d = 1\,\text{см}$ расположен так, что 2 его стороны параллельны главной оптической оси собирающей линзы,
    его центр удален на $h = 4\,\text{см}$ от этой оси и на $a = 12\,\text{см}$ от плоскости линзы.
    Определите площадь изображения квадрата, если фокусное расстояние линзы составляет $F = 20\,\text{см}$.
    % (и сравните с площадью объекта, умноженной на квадрат увеличения центра квадрата).
}
\answer{%
    \begin{align*}
    \frac 1 F &= \frac 1{a + \frac d2} + \frac 1b \implies b = \frac 1{\frac 1 F - \frac 1{a + \frac d2}} = \frac{F(a + \frac d2)}{a + \frac d2 - F}, \\
    \frac 1 F &= \frac 1{a - \frac d2} + \frac 1c \implies c = \frac 1{\frac 1 F - \frac 1{a - \frac d2}} = \frac{F(a - \frac d2)}{a - \frac d2 - F}, \\
    c - b &= \frac{F(a - \frac d2)}{a - \frac d2 - F} - \frac{F(a + \frac d2)}{a + \frac d2 - F} = F\cbr{ \frac{a - \frac d2}{a - \frac d2 - F} - \frac{a + \frac d2}{a + \frac d2 - F} } =  \\
    &= F \cdot \frac{a^2 + \frac {ad}2 - aF - \frac{ad}2 - \frac{d^2}4 + \frac{dF}2 - a^2 + \frac {ad}2 + aF - \frac{ad}2 + \frac{d^2}4 + \frac{dF}2}{\cbr{a + \frac d2 - F}\cbr{a - \frac d2 - F}}= F \cdot \frac {dF}{\cbr{a + \frac d2 - F}\cbr{a - \frac d2 - F}}.
    \\
    \Gamma_b &= \frac b{a + \frac d2} = \frac{ F }{a + \frac d2 - F}, \\
    \Gamma_c &= \frac c{a - \frac d2} = \frac{ F }{a - \frac d2 - F}, \\
    &\text{ тут интересно отметить, что } \Gamma_x = \frac{ c - b}{ d } = \frac{ F^2 }{\cbr{a + \frac d2 - F}\cbr{a - \frac d2 - F}} \ne \Gamma_b \text{ или } \Gamma_c \text{ даже при малых $d$}.
    \\
    S' &= \frac{d \cdot \Gamma_b + d \cdot \Gamma_c}2 \cdot (c - b) = \frac d2 \cbr{\frac{ F }{a + \frac d2 - F} + \frac{ F }{a - \frac d2 - F}} \cdot \cbr{c - b} =  \\
    &=\frac {dF}2 \cbr{\frac 1{a + \frac d2 - F} + \frac 1{a - \frac d2 - F}} \cdot \frac {dF^2}{\cbr{a + \frac d2 - F}\cbr{a - \frac d2 - F}} =  \\
    &=\frac {dF}2 \cdot \frac{a - \frac d2 - F + a + \frac d2 - F}{\cbr{a + \frac d2 - F}\cbr{a - \frac d2 - F}} \cdot \frac {dF^2}{\cbr{a + \frac d2 - F}\cbr{a - \frac d2 - F}} =  \\
    &= \frac {d^2F^3}{2\sqr{a + \frac d2 - F}\sqr{a - \frac d2 - F}} \cdot (2a - 2F) = \frac {d^2F^3(a - F)}{ \sqr{\sqr{a - F} - \frac{d^2}4} }.
    \end{align*}
}

\variantsplitter

\addpersonalvariant{Михаил Ярошевский}

\tasknumber{1}%
\task{%
    Укажите, верны ли утверждения («да» или «нет» слева от каждого утверждения):
    \begin{itemize}
        \item  Изображение предмета в рассеивающей линзе всегда мнимое.
        \item  Изображение предмета в рассеивающей линзе всегда прямое.
        \item  Изображение предмета в рассеивающей линзе всегда уменьшенное.
        \item  Оптическая сила собирающей линзы положительна.
    \end{itemize}
}
\answer{%
    $\text{ да, да, да, нет }$
}

\tasknumber{2}%
\task{%
    Запишите известные вам виды классификации изображений.
}
\solutionspace{60pt}

\tasknumber{3}%
\task{%
    В каких линзах можно получить увеличенное изображение объекта?
}
\answer{%
    $\text{ рассеивающие }$
}
\solutionspace{40pt}

\tasknumber{4}%
\task{%
    Какое изображение называют действительным?
}
\solutionspace{40pt}

\tasknumber{5}%
\task{%
    Есть две линзы, обозначим их 1 и 2.
    Известно что оптическая сила линзы 2 больше, чем у линзы 1.
    Какая линза сильнее преломляет лучи?
}
\answer{%
    $2$
}
\solutionspace{40pt}

\tasknumber{6}%
\task{%
    Предмет находится на расстоянии $30\,\text{см}$ от рассеивающей линзы с фокусным расстоянием $12\,\text{см}$.
    Определите тип изображения, расстояние между предметом и его изображением, увеличение предмета.
    Сделайте схематичный рисунок (не обязательно в масштабе, но с сохранением свойств линзы и изображения).
}
\solutionspace{100pt}

\tasknumber{7}%
\task{%
    Объект находится на расстоянии $25\,\text{см}$ от линзы, а его действительное изображение — в $30\,\text{см}$ от неё.
    Определите увеличение предмета, фокусное расстояние линзы, оптическую силу линзы и её тип.
}
\solutionspace{80pt}

\tasknumber{8}%
\task{%
    (Задача-«гроб»: решать на обратной стороне) Квадрат со стороной $d = 1\,\text{см}$ расположен так, что 2 его стороны параллельны главной оптической оси рассеивающей линзы,
    его центр удален на $h = 6\,\text{см}$ от этой оси и на $a = 10\,\text{см}$ от плоскости линзы.
    Определите площадь изображения квадрата, если фокусное расстояние линзы составляет $F = 25\,\text{см}$.
    % (и сравните с площадью объекта, умноженной на квадрат увеличения центра квадрата).
}
\answer{%
    \begin{align*}
    \frac 1 F &= \frac 1{a + \frac d2} + \frac 1b \implies b = \frac 1{\frac 1 F - \frac 1{a + \frac d2}} = \frac{F(a + \frac d2)}{a + \frac d2 - F}, \\
    \frac 1 F &= \frac 1{a - \frac d2} + \frac 1c \implies c = \frac 1{\frac 1 F - \frac 1{a - \frac d2}} = \frac{F(a - \frac d2)}{a - \frac d2 - F}, \\
    c - b &= \frac{F(a - \frac d2)}{a - \frac d2 - F} - \frac{F(a + \frac d2)}{a + \frac d2 - F} = F\cbr{ \frac{a - \frac d2}{a - \frac d2 - F} - \frac{a + \frac d2}{a + \frac d2 - F} } =  \\
    &= F \cdot \frac{a^2 + \frac {ad}2 - aF - \frac{ad}2 - \frac{d^2}4 + \frac{dF}2 - a^2 + \frac {ad}2 + aF - \frac{ad}2 + \frac{d^2}4 + \frac{dF}2}{\cbr{a + \frac d2 - F}\cbr{a - \frac d2 - F}}= F \cdot \frac {dF}{\cbr{a + \frac d2 - F}\cbr{a - \frac d2 - F}}.
    \\
    \Gamma_b &= \frac b{a + \frac d2} = \frac{ F }{a + \frac d2 - F}, \\
    \Gamma_c &= \frac c{a - \frac d2} = \frac{ F }{a - \frac d2 - F}, \\
    &\text{ тут интересно отметить, что } \Gamma_x = \frac{ c - b}{ d } = \frac{ F^2 }{\cbr{a + \frac d2 - F}\cbr{a - \frac d2 - F}} \ne \Gamma_b \text{ или } \Gamma_c \text{ даже при малых $d$}.
    \\
    S' &= \frac{d \cdot \Gamma_b + d \cdot \Gamma_c}2 \cdot (c - b) = \frac d2 \cbr{\frac{ F }{a + \frac d2 - F} + \frac{ F }{a - \frac d2 - F}} \cdot \cbr{c - b} =  \\
    &=\frac {dF}2 \cbr{\frac 1{a + \frac d2 - F} + \frac 1{a - \frac d2 - F}} \cdot \frac {dF^2}{\cbr{a + \frac d2 - F}\cbr{a - \frac d2 - F}} =  \\
    &=\frac {dF}2 \cdot \frac{a - \frac d2 - F + a + \frac d2 - F}{\cbr{a + \frac d2 - F}\cbr{a - \frac d2 - F}} \cdot \frac {dF^2}{\cbr{a + \frac d2 - F}\cbr{a - \frac d2 - F}} =  \\
    &= \frac {d^2F^3}{2\sqr{a + \frac d2 - F}\sqr{a - \frac d2 - F}} \cdot (2a - 2F) = \frac {d^2F^3(a - F)}{ \sqr{\sqr{a - F} - \frac{d^2}4} }.
    \end{align*}
}

\variantsplitter

\addpersonalvariant{Алексей Алимпиев}

\tasknumber{1}%
\task{%
    Укажите, верны ли утверждения («да» или «нет» слева от каждого утверждения):
    \begin{itemize}
        \item  Изображение предмета в рассеивающей линзе всегда мнимое.
        \item  Изображение предмета в рассеивающей линзе всегда перевёрнутое.
        \item  Изображение предмета в рассеивающей линзе всегда увеличенное.
        \item  Оптическая сила собирающей линзы положительна.
    \end{itemize}
}
\answer{%
    $\text{ да, нет, нет, нет }$
}

\tasknumber{2}%
\task{%
    Запишите формулу тонкой линзы и сделайте рисунок, указав на нём физические величины из этой формулы.
}
\solutionspace{60pt}

\tasknumber{3}%
\task{%
    В каких линзах можно получить мнимое изображение объекта?
}
\answer{%
    $\text{ собирающие и рассеивающие }$
}
\solutionspace{40pt}

\tasknumber{4}%
\task{%
    Какое изображение называют мнимым?
}
\solutionspace{40pt}

\tasknumber{5}%
\task{%
    Есть две линзы, обозначим их 1 и 2.
    Известно что оптическая сила линзы 1 больше, чем у линзы 2.
    Какая линза сильнее преломляет лучи?
}
\answer{%
    $1$
}
\solutionspace{40pt}

\tasknumber{6}%
\task{%
    Предмет находится на расстоянии $10\,\text{см}$ от собирающей линзы с фокусным расстоянием $40\,\text{см}$.
    Определите тип изображения, расстояние между предметом и его изображением, увеличение предмета.
    Сделайте схематичный рисунок (не обязательно в масштабе, но с сохранением свойств линзы и изображения).
}
\solutionspace{100pt}

\tasknumber{7}%
\task{%
    Объект находится на расстоянии $45\,\text{см}$ от линзы, а его мнимое изображение — в $30\,\text{см}$ от неё.
    Определите увеличение предмета, фокусное расстояние линзы, оптическую силу линзы и её тип.
}
\solutionspace{80pt}

\tasknumber{8}%
\task{%
    (Задача-«гроб»: решать на обратной стороне) Квадрат со стороной $d = 1\,\text{см}$ расположен так, что 2 его стороны параллельны главной оптической оси рассеивающей линзы,
    его центр удален на $h = 4\,\text{см}$ от этой оси и на $a = 15\,\text{см}$ от плоскости линзы.
    Определите площадь изображения квадрата, если фокусное расстояние линзы составляет $F = 18\,\text{см}$.
    % (и сравните с площадью объекта, умноженной на квадрат увеличения центра квадрата).
}
\answer{%
    \begin{align*}
    \frac 1 F &= \frac 1{a + \frac d2} + \frac 1b \implies b = \frac 1{\frac 1 F - \frac 1{a + \frac d2}} = \frac{F(a + \frac d2)}{a + \frac d2 - F}, \\
    \frac 1 F &= \frac 1{a - \frac d2} + \frac 1c \implies c = \frac 1{\frac 1 F - \frac 1{a - \frac d2}} = \frac{F(a - \frac d2)}{a - \frac d2 - F}, \\
    c - b &= \frac{F(a - \frac d2)}{a - \frac d2 - F} - \frac{F(a + \frac d2)}{a + \frac d2 - F} = F\cbr{ \frac{a - \frac d2}{a - \frac d2 - F} - \frac{a + \frac d2}{a + \frac d2 - F} } =  \\
    &= F \cdot \frac{a^2 + \frac {ad}2 - aF - \frac{ad}2 - \frac{d^2}4 + \frac{dF}2 - a^2 + \frac {ad}2 + aF - \frac{ad}2 + \frac{d^2}4 + \frac{dF}2}{\cbr{a + \frac d2 - F}\cbr{a - \frac d2 - F}}= F \cdot \frac {dF}{\cbr{a + \frac d2 - F}\cbr{a - \frac d2 - F}}.
    \\
    \Gamma_b &= \frac b{a + \frac d2} = \frac{ F }{a + \frac d2 - F}, \\
    \Gamma_c &= \frac c{a - \frac d2} = \frac{ F }{a - \frac d2 - F}, \\
    &\text{ тут интересно отметить, что } \Gamma_x = \frac{ c - b}{ d } = \frac{ F^2 }{\cbr{a + \frac d2 - F}\cbr{a - \frac d2 - F}} \ne \Gamma_b \text{ или } \Gamma_c \text{ даже при малых $d$}.
    \\
    S' &= \frac{d \cdot \Gamma_b + d \cdot \Gamma_c}2 \cdot (c - b) = \frac d2 \cbr{\frac{ F }{a + \frac d2 - F} + \frac{ F }{a - \frac d2 - F}} \cdot \cbr{c - b} =  \\
    &=\frac {dF}2 \cbr{\frac 1{a + \frac d2 - F} + \frac 1{a - \frac d2 - F}} \cdot \frac {dF^2}{\cbr{a + \frac d2 - F}\cbr{a - \frac d2 - F}} =  \\
    &=\frac {dF}2 \cdot \frac{a - \frac d2 - F + a + \frac d2 - F}{\cbr{a + \frac d2 - F}\cbr{a - \frac d2 - F}} \cdot \frac {dF^2}{\cbr{a + \frac d2 - F}\cbr{a - \frac d2 - F}} =  \\
    &= \frac {d^2F^3}{2\sqr{a + \frac d2 - F}\sqr{a - \frac d2 - F}} \cdot (2a - 2F) = \frac {d^2F^3(a - F)}{ \sqr{\sqr{a - F} - \frac{d^2}4} }.
    \end{align*}
}

\variantsplitter

\addpersonalvariant{Евгений Васин}

\tasknumber{1}%
\task{%
    Укажите, верны ли утверждения («да» или «нет» слева от каждого утверждения):
    \begin{itemize}
        \item  Изображение предмета в собирающей линзе всегда действительное.
        \item  Изображение предмета в собирающей линзе всегда перевёрнутое.
        \item  Изображение предмета в собирающей линзе всегда уменьшенное.
        \item  Оптическая сила собирающей линзы отрицательна.
    \end{itemize}
}
\answer{%
    $\text{ нет, нет, нет, да }$
}

\tasknumber{2}%
\task{%
    Запишите известные вам виды классификации изображений.
}
\solutionspace{60pt}

\tasknumber{3}%
\task{%
    В каких линзах можно получить мнимое изображение объекта?
}
\answer{%
    $\text{ собирающие и рассеивающие }$
}
\solutionspace{40pt}

\tasknumber{4}%
\task{%
    Какое изображение называют мнимым?
}
\solutionspace{40pt}

\tasknumber{5}%
\task{%
    Есть две линзы, обозначим их 1 и 2.
    Известно что оптическая сила линзы 2 меньше, чем у линзы 1.
    Какая линза сильнее преломляет лучи?
}
\answer{%
    $1$
}
\solutionspace{40pt}

\tasknumber{6}%
\task{%
    Предмет находится на расстоянии $30\,\text{см}$ от рассеивающей линзы с фокусным расстоянием $8\,\text{см}$.
    Определите тип изображения, расстояние между предметом и его изображением, увеличение предмета.
    Сделайте схематичный рисунок (не обязательно в масштабе, но с сохранением свойств линзы и изображения).
}
\solutionspace{100pt}

\tasknumber{7}%
\task{%
    Объект находится на расстоянии $115\,\text{см}$ от линзы, а его мнимое изображение — в $40\,\text{см}$ от неё.
    Определите увеличение предмета, фокусное расстояние линзы, оптическую силу линзы и её тип.
}
\solutionspace{80pt}

\tasknumber{8}%
\task{%
    (Задача-«гроб»: решать на обратной стороне) Квадрат со стороной $d = 3\,\text{см}$ расположен так, что 2 его стороны параллельны главной оптической оси рассеивающей линзы,
    его центр удален на $h = 6\,\text{см}$ от этой оси и на $a = 15\,\text{см}$ от плоскости линзы.
    Определите площадь изображения квадрата, если фокусное расстояние линзы составляет $F = 18\,\text{см}$.
    % (и сравните с площадью объекта, умноженной на квадрат увеличения центра квадрата).
}
\answer{%
    \begin{align*}
    \frac 1 F &= \frac 1{a + \frac d2} + \frac 1b \implies b = \frac 1{\frac 1 F - \frac 1{a + \frac d2}} = \frac{F(a + \frac d2)}{a + \frac d2 - F}, \\
    \frac 1 F &= \frac 1{a - \frac d2} + \frac 1c \implies c = \frac 1{\frac 1 F - \frac 1{a - \frac d2}} = \frac{F(a - \frac d2)}{a - \frac d2 - F}, \\
    c - b &= \frac{F(a - \frac d2)}{a - \frac d2 - F} - \frac{F(a + \frac d2)}{a + \frac d2 - F} = F\cbr{ \frac{a - \frac d2}{a - \frac d2 - F} - \frac{a + \frac d2}{a + \frac d2 - F} } =  \\
    &= F \cdot \frac{a^2 + \frac {ad}2 - aF - \frac{ad}2 - \frac{d^2}4 + \frac{dF}2 - a^2 + \frac {ad}2 + aF - \frac{ad}2 + \frac{d^2}4 + \frac{dF}2}{\cbr{a + \frac d2 - F}\cbr{a - \frac d2 - F}}= F \cdot \frac {dF}{\cbr{a + \frac d2 - F}\cbr{a - \frac d2 - F}}.
    \\
    \Gamma_b &= \frac b{a + \frac d2} = \frac{ F }{a + \frac d2 - F}, \\
    \Gamma_c &= \frac c{a - \frac d2} = \frac{ F }{a - \frac d2 - F}, \\
    &\text{ тут интересно отметить, что } \Gamma_x = \frac{ c - b}{ d } = \frac{ F^2 }{\cbr{a + \frac d2 - F}\cbr{a - \frac d2 - F}} \ne \Gamma_b \text{ или } \Gamma_c \text{ даже при малых $d$}.
    \\
    S' &= \frac{d \cdot \Gamma_b + d \cdot \Gamma_c}2 \cdot (c - b) = \frac d2 \cbr{\frac{ F }{a + \frac d2 - F} + \frac{ F }{a - \frac d2 - F}} \cdot \cbr{c - b} =  \\
    &=\frac {dF}2 \cbr{\frac 1{a + \frac d2 - F} + \frac 1{a - \frac d2 - F}} \cdot \frac {dF^2}{\cbr{a + \frac d2 - F}\cbr{a - \frac d2 - F}} =  \\
    &=\frac {dF}2 \cdot \frac{a - \frac d2 - F + a + \frac d2 - F}{\cbr{a + \frac d2 - F}\cbr{a - \frac d2 - F}} \cdot \frac {dF^2}{\cbr{a + \frac d2 - F}\cbr{a - \frac d2 - F}} =  \\
    &= \frac {d^2F^3}{2\sqr{a + \frac d2 - F}\sqr{a - \frac d2 - F}} \cdot (2a - 2F) = \frac {d^2F^3(a - F)}{ \sqr{\sqr{a - F} - \frac{d^2}4} }.
    \end{align*}
}

\variantsplitter

\addpersonalvariant{Вячеслав Волохов}

\tasknumber{1}%
\task{%
    Укажите, верны ли утверждения («да» или «нет» слева от каждого утверждения):
    \begin{itemize}
        \item  Изображение предмета в рассеивающей линзе всегда действительное.
        \item  Изображение предмета в рассеивающей линзе всегда прямое.
        \item  Изображение предмета в рассеивающей линзе всегда уменьшенное.
        \item  Оптическая сила собирающей линзы положительна.
    \end{itemize}
}
\answer{%
    $\text{ нет, да, да, нет }$
}

\tasknumber{2}%
\task{%
    Запишите формулу тонкой линзы и сделайте рисунок, указав на нём физические величины из этой формулы.
}
\solutionspace{60pt}

\tasknumber{3}%
\task{%
    В каких линзах можно получить действительное изображение объекта?
}
\answer{%
    $\text{ собирающие }$
}
\solutionspace{40pt}

\tasknumber{4}%
\task{%
    Какое изображение называют действительным?
}
\solutionspace{40pt}

\tasknumber{5}%
\task{%
    Есть две линзы, обозначим их 1 и 2.
    Известно что оптическая сила линзы 1 больше, чем у линзы 2.
    Какая линза сильнее преломляет лучи?
}
\answer{%
    $1$
}
\solutionspace{40pt}

\tasknumber{6}%
\task{%
    Предмет находится на расстоянии $20\,\text{см}$ от собирающей линзы с фокусным расстоянием $8\,\text{см}$.
    Определите тип изображения, расстояние между предметом и его изображением, увеличение предмета.
    Сделайте схематичный рисунок (не обязательно в масштабе, но с сохранением свойств линзы и изображения).
}
\solutionspace{100pt}

\tasknumber{7}%
\task{%
    Объект находится на расстоянии $25\,\text{см}$ от линзы, а его действительное изображение — в $10\,\text{см}$ от неё.
    Определите увеличение предмета, фокусное расстояние линзы, оптическую силу линзы и её тип.
}
\solutionspace{80pt}

\tasknumber{8}%
\task{%
    (Задача-«гроб»: решать на обратной стороне) Квадрат со стороной $d = 1\,\text{см}$ расположен так, что 2 его стороны параллельны главной оптической оси собирающей линзы,
    его центр удален на $h = 6\,\text{см}$ от этой оси и на $a = 15\,\text{см}$ от плоскости линзы.
    Определите площадь изображения квадрата, если фокусное расстояние линзы составляет $F = 18\,\text{см}$.
    % (и сравните с площадью объекта, умноженной на квадрат увеличения центра квадрата).
}
\answer{%
    \begin{align*}
    \frac 1 F &= \frac 1{a + \frac d2} + \frac 1b \implies b = \frac 1{\frac 1 F - \frac 1{a + \frac d2}} = \frac{F(a + \frac d2)}{a + \frac d2 - F}, \\
    \frac 1 F &= \frac 1{a - \frac d2} + \frac 1c \implies c = \frac 1{\frac 1 F - \frac 1{a - \frac d2}} = \frac{F(a - \frac d2)}{a - \frac d2 - F}, \\
    c - b &= \frac{F(a - \frac d2)}{a - \frac d2 - F} - \frac{F(a + \frac d2)}{a + \frac d2 - F} = F\cbr{ \frac{a - \frac d2}{a - \frac d2 - F} - \frac{a + \frac d2}{a + \frac d2 - F} } =  \\
    &= F \cdot \frac{a^2 + \frac {ad}2 - aF - \frac{ad}2 - \frac{d^2}4 + \frac{dF}2 - a^2 + \frac {ad}2 + aF - \frac{ad}2 + \frac{d^2}4 + \frac{dF}2}{\cbr{a + \frac d2 - F}\cbr{a - \frac d2 - F}}= F \cdot \frac {dF}{\cbr{a + \frac d2 - F}\cbr{a - \frac d2 - F}}.
    \\
    \Gamma_b &= \frac b{a + \frac d2} = \frac{ F }{a + \frac d2 - F}, \\
    \Gamma_c &= \frac c{a - \frac d2} = \frac{ F }{a - \frac d2 - F}, \\
    &\text{ тут интересно отметить, что } \Gamma_x = \frac{ c - b}{ d } = \frac{ F^2 }{\cbr{a + \frac d2 - F}\cbr{a - \frac d2 - F}} \ne \Gamma_b \text{ или } \Gamma_c \text{ даже при малых $d$}.
    \\
    S' &= \frac{d \cdot \Gamma_b + d \cdot \Gamma_c}2 \cdot (c - b) = \frac d2 \cbr{\frac{ F }{a + \frac d2 - F} + \frac{ F }{a - \frac d2 - F}} \cdot \cbr{c - b} =  \\
    &=\frac {dF}2 \cbr{\frac 1{a + \frac d2 - F} + \frac 1{a - \frac d2 - F}} \cdot \frac {dF^2}{\cbr{a + \frac d2 - F}\cbr{a - \frac d2 - F}} =  \\
    &=\frac {dF}2 \cdot \frac{a - \frac d2 - F + a + \frac d2 - F}{\cbr{a + \frac d2 - F}\cbr{a - \frac d2 - F}} \cdot \frac {dF^2}{\cbr{a + \frac d2 - F}\cbr{a - \frac d2 - F}} =  \\
    &= \frac {d^2F^3}{2\sqr{a + \frac d2 - F}\sqr{a - \frac d2 - F}} \cdot (2a - 2F) = \frac {d^2F^3(a - F)}{ \sqr{\sqr{a - F} - \frac{d^2}4} }.
    \end{align*}
}

\variantsplitter

\addpersonalvariant{Герман Говоров}

\tasknumber{1}%
\task{%
    Укажите, верны ли утверждения («да» или «нет» слева от каждого утверждения):
    \begin{itemize}
        \item  Изображение предмета в рассеивающей линзе всегда действительное.
        \item  Изображение предмета в рассеивающей линзе всегда прямое.
        \item  Изображение предмета в рассеивающей линзе всегда увеличенное.
        \item  Оптическая сила собирающей линзы положительна.
    \end{itemize}
}
\answer{%
    $\text{ нет, да, нет, нет }$
}

\tasknumber{2}%
\task{%
    Запишите формулу тонкой линзы и сделайте рисунок, указав на нём физические величины из этой формулы.
}
\solutionspace{60pt}

\tasknumber{3}%
\task{%
    В каких линзах можно получить мнимое изображение объекта?
}
\answer{%
    $\text{ собирающие и рассеивающие }$
}
\solutionspace{40pt}

\tasknumber{4}%
\task{%
    Какое изображение называют мнимым?
}
\solutionspace{40pt}

\tasknumber{5}%
\task{%
    Есть две линзы, обозначим их 1 и 2.
    Известно что оптическая сила линзы 1 меньше, чем у линзы 2.
    Какая линза сильнее преломляет лучи?
}
\answer{%
    $2$
}
\solutionspace{40pt}

\tasknumber{6}%
\task{%
    Предмет находится на расстоянии $30\,\text{см}$ от собирающей линзы с фокусным расстоянием $6\,\text{см}$.
    Определите тип изображения, расстояние между предметом и его изображением, увеличение предмета.
    Сделайте схематичный рисунок (не обязательно в масштабе, но с сохранением свойств линзы и изображения).
}
\solutionspace{100pt}

\tasknumber{7}%
\task{%
    Объект находится на расстоянии $45\,\text{см}$ от линзы, а его мнимое изображение — в $40\,\text{см}$ от неё.
    Определите увеличение предмета, фокусное расстояние линзы, оптическую силу линзы и её тип.
}
\solutionspace{80pt}

\tasknumber{8}%
\task{%
    (Задача-«гроб»: решать на обратной стороне) Квадрат со стороной $d = 2\,\text{см}$ расположен так, что 2 его стороны параллельны главной оптической оси собирающей линзы,
    его центр удален на $h = 4\,\text{см}$ от этой оси и на $a = 12\,\text{см}$ от плоскости линзы.
    Определите площадь изображения квадрата, если фокусное расстояние линзы составляет $F = 25\,\text{см}$.
    % (и сравните с площадью объекта, умноженной на квадрат увеличения центра квадрата).
}
\answer{%
    \begin{align*}
    \frac 1 F &= \frac 1{a + \frac d2} + \frac 1b \implies b = \frac 1{\frac 1 F - \frac 1{a + \frac d2}} = \frac{F(a + \frac d2)}{a + \frac d2 - F}, \\
    \frac 1 F &= \frac 1{a - \frac d2} + \frac 1c \implies c = \frac 1{\frac 1 F - \frac 1{a - \frac d2}} = \frac{F(a - \frac d2)}{a - \frac d2 - F}, \\
    c - b &= \frac{F(a - \frac d2)}{a - \frac d2 - F} - \frac{F(a + \frac d2)}{a + \frac d2 - F} = F\cbr{ \frac{a - \frac d2}{a - \frac d2 - F} - \frac{a + \frac d2}{a + \frac d2 - F} } =  \\
    &= F \cdot \frac{a^2 + \frac {ad}2 - aF - \frac{ad}2 - \frac{d^2}4 + \frac{dF}2 - a^2 + \frac {ad}2 + aF - \frac{ad}2 + \frac{d^2}4 + \frac{dF}2}{\cbr{a + \frac d2 - F}\cbr{a - \frac d2 - F}}= F \cdot \frac {dF}{\cbr{a + \frac d2 - F}\cbr{a - \frac d2 - F}}.
    \\
    \Gamma_b &= \frac b{a + \frac d2} = \frac{ F }{a + \frac d2 - F}, \\
    \Gamma_c &= \frac c{a - \frac d2} = \frac{ F }{a - \frac d2 - F}, \\
    &\text{ тут интересно отметить, что } \Gamma_x = \frac{ c - b}{ d } = \frac{ F^2 }{\cbr{a + \frac d2 - F}\cbr{a - \frac d2 - F}} \ne \Gamma_b \text{ или } \Gamma_c \text{ даже при малых $d$}.
    \\
    S' &= \frac{d \cdot \Gamma_b + d \cdot \Gamma_c}2 \cdot (c - b) = \frac d2 \cbr{\frac{ F }{a + \frac d2 - F} + \frac{ F }{a - \frac d2 - F}} \cdot \cbr{c - b} =  \\
    &=\frac {dF}2 \cbr{\frac 1{a + \frac d2 - F} + \frac 1{a - \frac d2 - F}} \cdot \frac {dF^2}{\cbr{a + \frac d2 - F}\cbr{a - \frac d2 - F}} =  \\
    &=\frac {dF}2 \cdot \frac{a - \frac d2 - F + a + \frac d2 - F}{\cbr{a + \frac d2 - F}\cbr{a - \frac d2 - F}} \cdot \frac {dF^2}{\cbr{a + \frac d2 - F}\cbr{a - \frac d2 - F}} =  \\
    &= \frac {d^2F^3}{2\sqr{a + \frac d2 - F}\sqr{a - \frac d2 - F}} \cdot (2a - 2F) = \frac {d^2F^3(a - F)}{ \sqr{\sqr{a - F} - \frac{d^2}4} }.
    \end{align*}
}

\variantsplitter

\addpersonalvariant{София Журавлёва}

\tasknumber{1}%
\task{%
    Укажите, верны ли утверждения («да» или «нет» слева от каждого утверждения):
    \begin{itemize}
        \item  Изображение предмета в собирающей линзе всегда действительное.
        \item  Изображение предмета в собирающей линзе всегда прямое.
        \item  Изображение предмета в собирающей линзе всегда уменьшенное.
        \item  Оптическая сила рассеивающей линзы положительна.
    \end{itemize}
}
\answer{%
    $\text{ нет, нет, нет, да }$
}

\tasknumber{2}%
\task{%
    Запишите формулу тонкой линзы и сделайте рисунок, указав на нём физические величины из этой формулы.
}
\solutionspace{60pt}

\tasknumber{3}%
\task{%
    В каких линзах можно получить мнимое изображение объекта?
}
\answer{%
    $\text{ собирающие и рассеивающие }$
}
\solutionspace{40pt}

\tasknumber{4}%
\task{%
    Какое изображение называют мнимым?
}
\solutionspace{40pt}

\tasknumber{5}%
\task{%
    Есть две линзы, обозначим их 1 и 2.
    Известно что фокусное расстояние линзы 2 больше, чем у линзы 1.
    Какая линза сильнее преломляет лучи?
}
\answer{%
    $1$
}
\solutionspace{40pt}

\tasknumber{6}%
\task{%
    Предмет находится на расстоянии $10\,\text{см}$ от собирающей линзы с фокусным расстоянием $12\,\text{см}$.
    Определите тип изображения, расстояние между предметом и его изображением, увеличение предмета.
    Сделайте схематичный рисунок (не обязательно в масштабе, но с сохранением свойств линзы и изображения).
}
\solutionspace{100pt}

\tasknumber{7}%
\task{%
    Объект находится на расстоянии $115\,\text{см}$ от линзы, а его мнимое изображение — в $30\,\text{см}$ от неё.
    Определите увеличение предмета, фокусное расстояние линзы, оптическую силу линзы и её тип.
}
\solutionspace{80pt}

\tasknumber{8}%
\task{%
    (Задача-«гроб»: решать на обратной стороне) Квадрат со стороной $d = 1\,\text{см}$ расположен так, что 2 его стороны параллельны главной оптической оси рассеивающей линзы,
    его центр удален на $h = 6\,\text{см}$ от этой оси и на $a = 12\,\text{см}$ от плоскости линзы.
    Определите площадь изображения квадрата, если фокусное расстояние линзы составляет $F = 25\,\text{см}$.
    % (и сравните с площадью объекта, умноженной на квадрат увеличения центра квадрата).
}
\answer{%
    \begin{align*}
    \frac 1 F &= \frac 1{a + \frac d2} + \frac 1b \implies b = \frac 1{\frac 1 F - \frac 1{a + \frac d2}} = \frac{F(a + \frac d2)}{a + \frac d2 - F}, \\
    \frac 1 F &= \frac 1{a - \frac d2} + \frac 1c \implies c = \frac 1{\frac 1 F - \frac 1{a - \frac d2}} = \frac{F(a - \frac d2)}{a - \frac d2 - F}, \\
    c - b &= \frac{F(a - \frac d2)}{a - \frac d2 - F} - \frac{F(a + \frac d2)}{a + \frac d2 - F} = F\cbr{ \frac{a - \frac d2}{a - \frac d2 - F} - \frac{a + \frac d2}{a + \frac d2 - F} } =  \\
    &= F \cdot \frac{a^2 + \frac {ad}2 - aF - \frac{ad}2 - \frac{d^2}4 + \frac{dF}2 - a^2 + \frac {ad}2 + aF - \frac{ad}2 + \frac{d^2}4 + \frac{dF}2}{\cbr{a + \frac d2 - F}\cbr{a - \frac d2 - F}}= F \cdot \frac {dF}{\cbr{a + \frac d2 - F}\cbr{a - \frac d2 - F}}.
    \\
    \Gamma_b &= \frac b{a + \frac d2} = \frac{ F }{a + \frac d2 - F}, \\
    \Gamma_c &= \frac c{a - \frac d2} = \frac{ F }{a - \frac d2 - F}, \\
    &\text{ тут интересно отметить, что } \Gamma_x = \frac{ c - b}{ d } = \frac{ F^2 }{\cbr{a + \frac d2 - F}\cbr{a - \frac d2 - F}} \ne \Gamma_b \text{ или } \Gamma_c \text{ даже при малых $d$}.
    \\
    S' &= \frac{d \cdot \Gamma_b + d \cdot \Gamma_c}2 \cdot (c - b) = \frac d2 \cbr{\frac{ F }{a + \frac d2 - F} + \frac{ F }{a - \frac d2 - F}} \cdot \cbr{c - b} =  \\
    &=\frac {dF}2 \cbr{\frac 1{a + \frac d2 - F} + \frac 1{a - \frac d2 - F}} \cdot \frac {dF^2}{\cbr{a + \frac d2 - F}\cbr{a - \frac d2 - F}} =  \\
    &=\frac {dF}2 \cdot \frac{a - \frac d2 - F + a + \frac d2 - F}{\cbr{a + \frac d2 - F}\cbr{a - \frac d2 - F}} \cdot \frac {dF^2}{\cbr{a + \frac d2 - F}\cbr{a - \frac d2 - F}} =  \\
    &= \frac {d^2F^3}{2\sqr{a + \frac d2 - F}\sqr{a - \frac d2 - F}} \cdot (2a - 2F) = \frac {d^2F^3(a - F)}{ \sqr{\sqr{a - F} - \frac{d^2}4} }.
    \end{align*}
}

\variantsplitter

\addpersonalvariant{Константин Козлов}

\tasknumber{1}%
\task{%
    Укажите, верны ли утверждения («да» или «нет» слева от каждого утверждения):
    \begin{itemize}
        \item  Изображение предмета в рассеивающей линзе всегда мнимое.
        \item  Изображение предмета в рассеивающей линзе всегда прямое.
        \item  Изображение предмета в рассеивающей линзе всегда увеличенное.
        \item  Оптическая сила рассеивающей линзы положительна.
    \end{itemize}
}
\answer{%
    $\text{ да, да, нет, да }$
}

\tasknumber{2}%
\task{%
    Запишите известные вам виды классификации изображений.
}
\solutionspace{60pt}

\tasknumber{3}%
\task{%
    В каких линзах можно получить увеличенное изображение объекта?
}
\answer{%
    $\text{ рассеивающие }$
}
\solutionspace{40pt}

\tasknumber{4}%
\task{%
    Какое изображение называют действительным?
}
\solutionspace{40pt}

\tasknumber{5}%
\task{%
    Есть две линзы, обозначим их 1 и 2.
    Известно что фокусное расстояние линзы 1 меньше, чем у линзы 2.
    Какая линза сильнее преломляет лучи?
}
\answer{%
    $1$
}
\solutionspace{40pt}

\tasknumber{6}%
\task{%
    Предмет находится на расстоянии $20\,\text{см}$ от собирающей линзы с фокусным расстоянием $12\,\text{см}$.
    Определите тип изображения, расстояние между предметом и его изображением, увеличение предмета.
    Сделайте схематичный рисунок (не обязательно в масштабе, но с сохранением свойств линзы и изображения).
}
\solutionspace{100pt}

\tasknumber{7}%
\task{%
    Объект находится на расстоянии $115\,\text{см}$ от линзы, а его действительное изображение — в $30\,\text{см}$ от неё.
    Определите увеличение предмета, фокусное расстояние линзы, оптическую силу линзы и её тип.
}
\solutionspace{80pt}

\tasknumber{8}%
\task{%
    (Задача-«гроб»: решать на обратной стороне) Квадрат со стороной $d = 1\,\text{см}$ расположен так, что 2 его стороны параллельны главной оптической оси собирающей линзы,
    его центр удален на $h = 5\,\text{см}$ от этой оси и на $a = 15\,\text{см}$ от плоскости линзы.
    Определите площадь изображения квадрата, если фокусное расстояние линзы составляет $F = 25\,\text{см}$.
    % (и сравните с площадью объекта, умноженной на квадрат увеличения центра квадрата).
}
\answer{%
    \begin{align*}
    \frac 1 F &= \frac 1{a + \frac d2} + \frac 1b \implies b = \frac 1{\frac 1 F - \frac 1{a + \frac d2}} = \frac{F(a + \frac d2)}{a + \frac d2 - F}, \\
    \frac 1 F &= \frac 1{a - \frac d2} + \frac 1c \implies c = \frac 1{\frac 1 F - \frac 1{a - \frac d2}} = \frac{F(a - \frac d2)}{a - \frac d2 - F}, \\
    c - b &= \frac{F(a - \frac d2)}{a - \frac d2 - F} - \frac{F(a + \frac d2)}{a + \frac d2 - F} = F\cbr{ \frac{a - \frac d2}{a - \frac d2 - F} - \frac{a + \frac d2}{a + \frac d2 - F} } =  \\
    &= F \cdot \frac{a^2 + \frac {ad}2 - aF - \frac{ad}2 - \frac{d^2}4 + \frac{dF}2 - a^2 + \frac {ad}2 + aF - \frac{ad}2 + \frac{d^2}4 + \frac{dF}2}{\cbr{a + \frac d2 - F}\cbr{a - \frac d2 - F}}= F \cdot \frac {dF}{\cbr{a + \frac d2 - F}\cbr{a - \frac d2 - F}}.
    \\
    \Gamma_b &= \frac b{a + \frac d2} = \frac{ F }{a + \frac d2 - F}, \\
    \Gamma_c &= \frac c{a - \frac d2} = \frac{ F }{a - \frac d2 - F}, \\
    &\text{ тут интересно отметить, что } \Gamma_x = \frac{ c - b}{ d } = \frac{ F^2 }{\cbr{a + \frac d2 - F}\cbr{a - \frac d2 - F}} \ne \Gamma_b \text{ или } \Gamma_c \text{ даже при малых $d$}.
    \\
    S' &= \frac{d \cdot \Gamma_b + d \cdot \Gamma_c}2 \cdot (c - b) = \frac d2 \cbr{\frac{ F }{a + \frac d2 - F} + \frac{ F }{a - \frac d2 - F}} \cdot \cbr{c - b} =  \\
    &=\frac {dF}2 \cbr{\frac 1{a + \frac d2 - F} + \frac 1{a - \frac d2 - F}} \cdot \frac {dF^2}{\cbr{a + \frac d2 - F}\cbr{a - \frac d2 - F}} =  \\
    &=\frac {dF}2 \cdot \frac{a - \frac d2 - F + a + \frac d2 - F}{\cbr{a + \frac d2 - F}\cbr{a - \frac d2 - F}} \cdot \frac {dF^2}{\cbr{a + \frac d2 - F}\cbr{a - \frac d2 - F}} =  \\
    &= \frac {d^2F^3}{2\sqr{a + \frac d2 - F}\sqr{a - \frac d2 - F}} \cdot (2a - 2F) = \frac {d^2F^3(a - F)}{ \sqr{\sqr{a - F} - \frac{d^2}4} }.
    \end{align*}
}

\variantsplitter

\addpersonalvariant{Наталья Кравченко}

\tasknumber{1}%
\task{%
    Укажите, верны ли утверждения («да» или «нет» слева от каждого утверждения):
    \begin{itemize}
        \item  Изображение предмета в рассеивающей линзе всегда действительное.
        \item  Изображение предмета в рассеивающей линзе всегда перевёрнутое.
        \item  Изображение предмета в рассеивающей линзе всегда увеличенное.
        \item  Оптическая сила рассеивающей линзы положительна.
    \end{itemize}
}
\answer{%
    $\text{ нет, нет, нет, да }$
}

\tasknumber{2}%
\task{%
    Запишите известные вам виды классификации изображений.
}
\solutionspace{60pt}

\tasknumber{3}%
\task{%
    В каких линзах можно получить обратное изображение объекта?
}
\answer{%
    $\text{ собирающие }$
}
\solutionspace{40pt}

\tasknumber{4}%
\task{%
    Какое изображение называют действительным?
}
\solutionspace{40pt}

\tasknumber{5}%
\task{%
    Есть две линзы, обозначим их 1 и 2.
    Известно что оптическая сила линзы 2 меньше, чем у линзы 1.
    Какая линза сильнее преломляет лучи?
}
\answer{%
    $1$
}
\solutionspace{40pt}

\tasknumber{6}%
\task{%
    Предмет находится на расстоянии $30\,\text{см}$ от рассеивающей линзы с фокусным расстоянием $25\,\text{см}$.
    Определите тип изображения, расстояние между предметом и его изображением, увеличение предмета.
    Сделайте схематичный рисунок (не обязательно в масштабе, но с сохранением свойств линзы и изображения).
}
\solutionspace{100pt}

\tasknumber{7}%
\task{%
    Объект находится на расстоянии $115\,\text{см}$ от линзы, а его действительное изображение — в $40\,\text{см}$ от неё.
    Определите увеличение предмета, фокусное расстояние линзы, оптическую силу линзы и её тип.
}
\solutionspace{80pt}

\tasknumber{8}%
\task{%
    (Задача-«гроб»: решать на обратной стороне) Квадрат со стороной $d = 2\,\text{см}$ расположен так, что 2 его стороны параллельны главной оптической оси собирающей линзы,
    его центр удален на $h = 6\,\text{см}$ от этой оси и на $a = 15\,\text{см}$ от плоскости линзы.
    Определите площадь изображения квадрата, если фокусное расстояние линзы составляет $F = 25\,\text{см}$.
    % (и сравните с площадью объекта, умноженной на квадрат увеличения центра квадрата).
}
\answer{%
    \begin{align*}
    \frac 1 F &= \frac 1{a + \frac d2} + \frac 1b \implies b = \frac 1{\frac 1 F - \frac 1{a + \frac d2}} = \frac{F(a + \frac d2)}{a + \frac d2 - F}, \\
    \frac 1 F &= \frac 1{a - \frac d2} + \frac 1c \implies c = \frac 1{\frac 1 F - \frac 1{a - \frac d2}} = \frac{F(a - \frac d2)}{a - \frac d2 - F}, \\
    c - b &= \frac{F(a - \frac d2)}{a - \frac d2 - F} - \frac{F(a + \frac d2)}{a + \frac d2 - F} = F\cbr{ \frac{a - \frac d2}{a - \frac d2 - F} - \frac{a + \frac d2}{a + \frac d2 - F} } =  \\
    &= F \cdot \frac{a^2 + \frac {ad}2 - aF - \frac{ad}2 - \frac{d^2}4 + \frac{dF}2 - a^2 + \frac {ad}2 + aF - \frac{ad}2 + \frac{d^2}4 + \frac{dF}2}{\cbr{a + \frac d2 - F}\cbr{a - \frac d2 - F}}= F \cdot \frac {dF}{\cbr{a + \frac d2 - F}\cbr{a - \frac d2 - F}}.
    \\
    \Gamma_b &= \frac b{a + \frac d2} = \frac{ F }{a + \frac d2 - F}, \\
    \Gamma_c &= \frac c{a - \frac d2} = \frac{ F }{a - \frac d2 - F}, \\
    &\text{ тут интересно отметить, что } \Gamma_x = \frac{ c - b}{ d } = \frac{ F^2 }{\cbr{a + \frac d2 - F}\cbr{a - \frac d2 - F}} \ne \Gamma_b \text{ или } \Gamma_c \text{ даже при малых $d$}.
    \\
    S' &= \frac{d \cdot \Gamma_b + d \cdot \Gamma_c}2 \cdot (c - b) = \frac d2 \cbr{\frac{ F }{a + \frac d2 - F} + \frac{ F }{a - \frac d2 - F}} \cdot \cbr{c - b} =  \\
    &=\frac {dF}2 \cbr{\frac 1{a + \frac d2 - F} + \frac 1{a - \frac d2 - F}} \cdot \frac {dF^2}{\cbr{a + \frac d2 - F}\cbr{a - \frac d2 - F}} =  \\
    &=\frac {dF}2 \cdot \frac{a - \frac d2 - F + a + \frac d2 - F}{\cbr{a + \frac d2 - F}\cbr{a - \frac d2 - F}} \cdot \frac {dF^2}{\cbr{a + \frac d2 - F}\cbr{a - \frac d2 - F}} =  \\
    &= \frac {d^2F^3}{2\sqr{a + \frac d2 - F}\sqr{a - \frac d2 - F}} \cdot (2a - 2F) = \frac {d^2F^3(a - F)}{ \sqr{\sqr{a - F} - \frac{d^2}4} }.
    \end{align*}
}

\variantsplitter

\addpersonalvariant{Матвей Кузьмин}

\tasknumber{1}%
\task{%
    Укажите, верны ли утверждения («да» или «нет» слева от каждого утверждения):
    \begin{itemize}
        \item  Изображение предмета в собирающей линзе всегда мнимое.
        \item  Изображение предмета в собирающей линзе всегда прямое.
        \item  Изображение предмета в собирающей линзе всегда уменьшенное.
        \item  Оптическая сила рассеивающей линзы положительна.
    \end{itemize}
}
\answer{%
    $\text{ нет, нет, нет, да }$
}

\tasknumber{2}%
\task{%
    Запишите известные вам виды классификации изображений.
}
\solutionspace{60pt}

\tasknumber{3}%
\task{%
    В каких линзах можно получить уменьшенное изображение объекта?
}
\answer{%
    $\text{ собирающие и рассеивающие }$
}
\solutionspace{40pt}

\tasknumber{4}%
\task{%
    Какое изображение называют мнимым?
}
\solutionspace{40pt}

\tasknumber{5}%
\task{%
    Есть две линзы, обозначим их 1 и 2.
    Известно что фокусное расстояние линзы 2 больше, чем у линзы 1.
    Какая линза сильнее преломляет лучи?
}
\answer{%
    $1$
}
\solutionspace{40pt}

\tasknumber{6}%
\task{%
    Предмет находится на расстоянии $10\,\text{см}$ от рассеивающей линзы с фокусным расстоянием $12\,\text{см}$.
    Определите тип изображения, расстояние между предметом и его изображением, увеличение предмета.
    Сделайте схематичный рисунок (не обязательно в масштабе, но с сохранением свойств линзы и изображения).
}
\solutionspace{100pt}

\tasknumber{7}%
\task{%
    Объект находится на расстоянии $115\,\text{см}$ от линзы, а его мнимое изображение — в $10\,\text{см}$ от неё.
    Определите увеличение предмета, фокусное расстояние линзы, оптическую силу линзы и её тип.
}
\solutionspace{80pt}

\tasknumber{8}%
\task{%
    (Задача-«гроб»: решать на обратной стороне) Квадрат со стороной $d = 3\,\text{см}$ расположен так, что 2 его стороны параллельны главной оптической оси рассеивающей линзы,
    его центр удален на $h = 6\,\text{см}$ от этой оси и на $a = 10\,\text{см}$ от плоскости линзы.
    Определите площадь изображения квадрата, если фокусное расстояние линзы составляет $F = 25\,\text{см}$.
    % (и сравните с площадью объекта, умноженной на квадрат увеличения центра квадрата).
}
\answer{%
    \begin{align*}
    \frac 1 F &= \frac 1{a + \frac d2} + \frac 1b \implies b = \frac 1{\frac 1 F - \frac 1{a + \frac d2}} = \frac{F(a + \frac d2)}{a + \frac d2 - F}, \\
    \frac 1 F &= \frac 1{a - \frac d2} + \frac 1c \implies c = \frac 1{\frac 1 F - \frac 1{a - \frac d2}} = \frac{F(a - \frac d2)}{a - \frac d2 - F}, \\
    c - b &= \frac{F(a - \frac d2)}{a - \frac d2 - F} - \frac{F(a + \frac d2)}{a + \frac d2 - F} = F\cbr{ \frac{a - \frac d2}{a - \frac d2 - F} - \frac{a + \frac d2}{a + \frac d2 - F} } =  \\
    &= F \cdot \frac{a^2 + \frac {ad}2 - aF - \frac{ad}2 - \frac{d^2}4 + \frac{dF}2 - a^2 + \frac {ad}2 + aF - \frac{ad}2 + \frac{d^2}4 + \frac{dF}2}{\cbr{a + \frac d2 - F}\cbr{a - \frac d2 - F}}= F \cdot \frac {dF}{\cbr{a + \frac d2 - F}\cbr{a - \frac d2 - F}}.
    \\
    \Gamma_b &= \frac b{a + \frac d2} = \frac{ F }{a + \frac d2 - F}, \\
    \Gamma_c &= \frac c{a - \frac d2} = \frac{ F }{a - \frac d2 - F}, \\
    &\text{ тут интересно отметить, что } \Gamma_x = \frac{ c - b}{ d } = \frac{ F^2 }{\cbr{a + \frac d2 - F}\cbr{a - \frac d2 - F}} \ne \Gamma_b \text{ или } \Gamma_c \text{ даже при малых $d$}.
    \\
    S' &= \frac{d \cdot \Gamma_b + d \cdot \Gamma_c}2 \cdot (c - b) = \frac d2 \cbr{\frac{ F }{a + \frac d2 - F} + \frac{ F }{a - \frac d2 - F}} \cdot \cbr{c - b} =  \\
    &=\frac {dF}2 \cbr{\frac 1{a + \frac d2 - F} + \frac 1{a - \frac d2 - F}} \cdot \frac {dF^2}{\cbr{a + \frac d2 - F}\cbr{a - \frac d2 - F}} =  \\
    &=\frac {dF}2 \cdot \frac{a - \frac d2 - F + a + \frac d2 - F}{\cbr{a + \frac d2 - F}\cbr{a - \frac d2 - F}} \cdot \frac {dF^2}{\cbr{a + \frac d2 - F}\cbr{a - \frac d2 - F}} =  \\
    &= \frac {d^2F^3}{2\sqr{a + \frac d2 - F}\sqr{a - \frac d2 - F}} \cdot (2a - 2F) = \frac {d^2F^3(a - F)}{ \sqr{\sqr{a - F} - \frac{d^2}4} }.
    \end{align*}
}

\variantsplitter

\addpersonalvariant{Сергей Малышев}

\tasknumber{1}%
\task{%
    Укажите, верны ли утверждения («да» или «нет» слева от каждого утверждения):
    \begin{itemize}
        \item  Изображение предмета в собирающей линзе всегда действительное.
        \item  Изображение предмета в собирающей линзе всегда перевёрнутое.
        \item  Изображение предмета в собирающей линзе всегда уменьшенное.
        \item  Оптическая сила собирающей линзы положительна.
    \end{itemize}
}
\answer{%
    $\text{ нет, нет, нет, нет }$
}

\tasknumber{2}%
\task{%
    Запишите формулу тонкой линзы и сделайте рисунок, указав на нём физические величины из этой формулы.
}
\solutionspace{60pt}

\tasknumber{3}%
\task{%
    В каких линзах можно получить обратное изображение объекта?
}
\answer{%
    $\text{ собирающие }$
}
\solutionspace{40pt}

\tasknumber{4}%
\task{%
    Какое изображение называют действительным?
}
\solutionspace{40pt}

\tasknumber{5}%
\task{%
    Есть две линзы, обозначим их 1 и 2.
    Известно что оптическая сила линзы 2 больше, чем у линзы 1.
    Какая линза сильнее преломляет лучи?
}
\answer{%
    $2$
}
\solutionspace{40pt}

\tasknumber{6}%
\task{%
    Предмет находится на расстоянии $20\,\text{см}$ от рассеивающей линзы с фокусным расстоянием $50\,\text{см}$.
    Определите тип изображения, расстояние между предметом и его изображением, увеличение предмета.
    Сделайте схематичный рисунок (не обязательно в масштабе, но с сохранением свойств линзы и изображения).
}
\solutionspace{100pt}

\tasknumber{7}%
\task{%
    Объект находится на расстоянии $115\,\text{см}$ от линзы, а его мнимое изображение — в $50\,\text{см}$ от неё.
    Определите увеличение предмета, фокусное расстояние линзы, оптическую силу линзы и её тип.
}
\solutionspace{80pt}

\tasknumber{8}%
\task{%
    (Задача-«гроб»: решать на обратной стороне) Квадрат со стороной $d = 3\,\text{см}$ расположен так, что 2 его стороны параллельны главной оптической оси рассеивающей линзы,
    его центр удален на $h = 4\,\text{см}$ от этой оси и на $a = 10\,\text{см}$ от плоскости линзы.
    Определите площадь изображения квадрата, если фокусное расстояние линзы составляет $F = 18\,\text{см}$.
    % (и сравните с площадью объекта, умноженной на квадрат увеличения центра квадрата).
}
\answer{%
    \begin{align*}
    \frac 1 F &= \frac 1{a + \frac d2} + \frac 1b \implies b = \frac 1{\frac 1 F - \frac 1{a + \frac d2}} = \frac{F(a + \frac d2)}{a + \frac d2 - F}, \\
    \frac 1 F &= \frac 1{a - \frac d2} + \frac 1c \implies c = \frac 1{\frac 1 F - \frac 1{a - \frac d2}} = \frac{F(a - \frac d2)}{a - \frac d2 - F}, \\
    c - b &= \frac{F(a - \frac d2)}{a - \frac d2 - F} - \frac{F(a + \frac d2)}{a + \frac d2 - F} = F\cbr{ \frac{a - \frac d2}{a - \frac d2 - F} - \frac{a + \frac d2}{a + \frac d2 - F} } =  \\
    &= F \cdot \frac{a^2 + \frac {ad}2 - aF - \frac{ad}2 - \frac{d^2}4 + \frac{dF}2 - a^2 + \frac {ad}2 + aF - \frac{ad}2 + \frac{d^2}4 + \frac{dF}2}{\cbr{a + \frac d2 - F}\cbr{a - \frac d2 - F}}= F \cdot \frac {dF}{\cbr{a + \frac d2 - F}\cbr{a - \frac d2 - F}}.
    \\
    \Gamma_b &= \frac b{a + \frac d2} = \frac{ F }{a + \frac d2 - F}, \\
    \Gamma_c &= \frac c{a - \frac d2} = \frac{ F }{a - \frac d2 - F}, \\
    &\text{ тут интересно отметить, что } \Gamma_x = \frac{ c - b}{ d } = \frac{ F^2 }{\cbr{a + \frac d2 - F}\cbr{a - \frac d2 - F}} \ne \Gamma_b \text{ или } \Gamma_c \text{ даже при малых $d$}.
    \\
    S' &= \frac{d \cdot \Gamma_b + d \cdot \Gamma_c}2 \cdot (c - b) = \frac d2 \cbr{\frac{ F }{a + \frac d2 - F} + \frac{ F }{a - \frac d2 - F}} \cdot \cbr{c - b} =  \\
    &=\frac {dF}2 \cbr{\frac 1{a + \frac d2 - F} + \frac 1{a - \frac d2 - F}} \cdot \frac {dF^2}{\cbr{a + \frac d2 - F}\cbr{a - \frac d2 - F}} =  \\
    &=\frac {dF}2 \cdot \frac{a - \frac d2 - F + a + \frac d2 - F}{\cbr{a + \frac d2 - F}\cbr{a - \frac d2 - F}} \cdot \frac {dF^2}{\cbr{a + \frac d2 - F}\cbr{a - \frac d2 - F}} =  \\
    &= \frac {d^2F^3}{2\sqr{a + \frac d2 - F}\sqr{a - \frac d2 - F}} \cdot (2a - 2F) = \frac {d^2F^3(a - F)}{ \sqr{\sqr{a - F} - \frac{d^2}4} }.
    \end{align*}
}

\variantsplitter

\addpersonalvariant{Алина Полканова}

\tasknumber{1}%
\task{%
    Укажите, верны ли утверждения («да» или «нет» слева от каждого утверждения):
    \begin{itemize}
        \item  Изображение предмета в собирающей линзе всегда действительное.
        \item  Изображение предмета в собирающей линзе всегда прямое.
        \item  Изображение предмета в собирающей линзе всегда уменьшенное.
        \item  Оптическая сила собирающей линзы положительна.
    \end{itemize}
}
\answer{%
    $\text{ нет, нет, нет, нет }$
}

\tasknumber{2}%
\task{%
    Запишите формулу тонкой линзы и сделайте рисунок, указав на нём физические величины из этой формулы.
}
\solutionspace{60pt}

\tasknumber{3}%
\task{%
    В каких линзах можно получить обратное изображение объекта?
}
\answer{%
    $\text{ собирающие }$
}
\solutionspace{40pt}

\tasknumber{4}%
\task{%
    Какое изображение называют действительным?
}
\solutionspace{40pt}

\tasknumber{5}%
\task{%
    Есть две линзы, обозначим их 1 и 2.
    Известно что оптическая сила линзы 1 меньше, чем у линзы 2.
    Какая линза сильнее преломляет лучи?
}
\answer{%
    $2$
}
\solutionspace{40pt}

\tasknumber{6}%
\task{%
    Предмет находится на расстоянии $30\,\text{см}$ от рассеивающей линзы с фокусным расстоянием $15\,\text{см}$.
    Определите тип изображения, расстояние между предметом и его изображением, увеличение предмета.
    Сделайте схематичный рисунок (не обязательно в масштабе, но с сохранением свойств линзы и изображения).
}
\solutionspace{100pt}

\tasknumber{7}%
\task{%
    Объект находится на расстоянии $25\,\text{см}$ от линзы, а его мнимое изображение — в $50\,\text{см}$ от неё.
    Определите увеличение предмета, фокусное расстояние линзы, оптическую силу линзы и её тип.
}
\solutionspace{80pt}

\tasknumber{8}%
\task{%
    (Задача-«гроб»: решать на обратной стороне) Квадрат со стороной $d = 2\,\text{см}$ расположен так, что 2 его стороны параллельны главной оптической оси собирающей линзы,
    его центр удален на $h = 6\,\text{см}$ от этой оси и на $a = 12\,\text{см}$ от плоскости линзы.
    Определите площадь изображения квадрата, если фокусное расстояние линзы составляет $F = 25\,\text{см}$.
    % (и сравните с площадью объекта, умноженной на квадрат увеличения центра квадрата).
}
\answer{%
    \begin{align*}
    \frac 1 F &= \frac 1{a + \frac d2} + \frac 1b \implies b = \frac 1{\frac 1 F - \frac 1{a + \frac d2}} = \frac{F(a + \frac d2)}{a + \frac d2 - F}, \\
    \frac 1 F &= \frac 1{a - \frac d2} + \frac 1c \implies c = \frac 1{\frac 1 F - \frac 1{a - \frac d2}} = \frac{F(a - \frac d2)}{a - \frac d2 - F}, \\
    c - b &= \frac{F(a - \frac d2)}{a - \frac d2 - F} - \frac{F(a + \frac d2)}{a + \frac d2 - F} = F\cbr{ \frac{a - \frac d2}{a - \frac d2 - F} - \frac{a + \frac d2}{a + \frac d2 - F} } =  \\
    &= F \cdot \frac{a^2 + \frac {ad}2 - aF - \frac{ad}2 - \frac{d^2}4 + \frac{dF}2 - a^2 + \frac {ad}2 + aF - \frac{ad}2 + \frac{d^2}4 + \frac{dF}2}{\cbr{a + \frac d2 - F}\cbr{a - \frac d2 - F}}= F \cdot \frac {dF}{\cbr{a + \frac d2 - F}\cbr{a - \frac d2 - F}}.
    \\
    \Gamma_b &= \frac b{a + \frac d2} = \frac{ F }{a + \frac d2 - F}, \\
    \Gamma_c &= \frac c{a - \frac d2} = \frac{ F }{a - \frac d2 - F}, \\
    &\text{ тут интересно отметить, что } \Gamma_x = \frac{ c - b}{ d } = \frac{ F^2 }{\cbr{a + \frac d2 - F}\cbr{a - \frac d2 - F}} \ne \Gamma_b \text{ или } \Gamma_c \text{ даже при малых $d$}.
    \\
    S' &= \frac{d \cdot \Gamma_b + d \cdot \Gamma_c}2 \cdot (c - b) = \frac d2 \cbr{\frac{ F }{a + \frac d2 - F} + \frac{ F }{a - \frac d2 - F}} \cdot \cbr{c - b} =  \\
    &=\frac {dF}2 \cbr{\frac 1{a + \frac d2 - F} + \frac 1{a - \frac d2 - F}} \cdot \frac {dF^2}{\cbr{a + \frac d2 - F}\cbr{a - \frac d2 - F}} =  \\
    &=\frac {dF}2 \cdot \frac{a - \frac d2 - F + a + \frac d2 - F}{\cbr{a + \frac d2 - F}\cbr{a - \frac d2 - F}} \cdot \frac {dF^2}{\cbr{a + \frac d2 - F}\cbr{a - \frac d2 - F}} =  \\
    &= \frac {d^2F^3}{2\sqr{a + \frac d2 - F}\sqr{a - \frac d2 - F}} \cdot (2a - 2F) = \frac {d^2F^3(a - F)}{ \sqr{\sqr{a - F} - \frac{d^2}4} }.
    \end{align*}
}

\variantsplitter

\addpersonalvariant{Сергей Пономарёв}

\tasknumber{1}%
\task{%
    Укажите, верны ли утверждения («да» или «нет» слева от каждого утверждения):
    \begin{itemize}
        \item  Изображение предмета в собирающей линзе всегда действительное.
        \item  Изображение предмета в собирающей линзе всегда перевёрнутое.
        \item  Изображение предмета в собирающей линзе всегда увеличенное.
        \item  Оптическая сила собирающей линзы отрицательна.
    \end{itemize}
}
\answer{%
    $\text{ нет, нет, нет, да }$
}

\tasknumber{2}%
\task{%
    Запишите известные вам виды классификации изображений.
}
\solutionspace{60pt}

\tasknumber{3}%
\task{%
    В каких линзах можно получить уменьшенное изображение объекта?
}
\answer{%
    $\text{ собирающие и рассеивающие }$
}
\solutionspace{40pt}

\tasknumber{4}%
\task{%
    Какое изображение называют мнимым?
}
\solutionspace{40pt}

\tasknumber{5}%
\task{%
    Есть две линзы, обозначим их 1 и 2.
    Известно что фокусное расстояние линзы 1 больше, чем у линзы 2.
    Какая линза сильнее преломляет лучи?
}
\answer{%
    $2$
}
\solutionspace{40pt}

\tasknumber{6}%
\task{%
    Предмет находится на расстоянии $20\,\text{см}$ от собирающей линзы с фокусным расстоянием $50\,\text{см}$.
    Определите тип изображения, расстояние между предметом и его изображением, увеличение предмета.
    Сделайте схематичный рисунок (не обязательно в масштабе, но с сохранением свойств линзы и изображения).
}
\solutionspace{100pt}

\tasknumber{7}%
\task{%
    Объект находится на расстоянии $25\,\text{см}$ от линзы, а его действительное изображение — в $50\,\text{см}$ от неё.
    Определите увеличение предмета, фокусное расстояние линзы, оптическую силу линзы и её тип.
}
\solutionspace{80pt}

\tasknumber{8}%
\task{%
    (Задача-«гроб»: решать на обратной стороне) Квадрат со стороной $d = 2\,\text{см}$ расположен так, что 2 его стороны параллельны главной оптической оси собирающей линзы,
    его центр удален на $h = 6\,\text{см}$ от этой оси и на $a = 10\,\text{см}$ от плоскости линзы.
    Определите площадь изображения квадрата, если фокусное расстояние линзы составляет $F = 25\,\text{см}$.
    % (и сравните с площадью объекта, умноженной на квадрат увеличения центра квадрата).
}
\answer{%
    \begin{align*}
    \frac 1 F &= \frac 1{a + \frac d2} + \frac 1b \implies b = \frac 1{\frac 1 F - \frac 1{a + \frac d2}} = \frac{F(a + \frac d2)}{a + \frac d2 - F}, \\
    \frac 1 F &= \frac 1{a - \frac d2} + \frac 1c \implies c = \frac 1{\frac 1 F - \frac 1{a - \frac d2}} = \frac{F(a - \frac d2)}{a - \frac d2 - F}, \\
    c - b &= \frac{F(a - \frac d2)}{a - \frac d2 - F} - \frac{F(a + \frac d2)}{a + \frac d2 - F} = F\cbr{ \frac{a - \frac d2}{a - \frac d2 - F} - \frac{a + \frac d2}{a + \frac d2 - F} } =  \\
    &= F \cdot \frac{a^2 + \frac {ad}2 - aF - \frac{ad}2 - \frac{d^2}4 + \frac{dF}2 - a^2 + \frac {ad}2 + aF - \frac{ad}2 + \frac{d^2}4 + \frac{dF}2}{\cbr{a + \frac d2 - F}\cbr{a - \frac d2 - F}}= F \cdot \frac {dF}{\cbr{a + \frac d2 - F}\cbr{a - \frac d2 - F}}.
    \\
    \Gamma_b &= \frac b{a + \frac d2} = \frac{ F }{a + \frac d2 - F}, \\
    \Gamma_c &= \frac c{a - \frac d2} = \frac{ F }{a - \frac d2 - F}, \\
    &\text{ тут интересно отметить, что } \Gamma_x = \frac{ c - b}{ d } = \frac{ F^2 }{\cbr{a + \frac d2 - F}\cbr{a - \frac d2 - F}} \ne \Gamma_b \text{ или } \Gamma_c \text{ даже при малых $d$}.
    \\
    S' &= \frac{d \cdot \Gamma_b + d \cdot \Gamma_c}2 \cdot (c - b) = \frac d2 \cbr{\frac{ F }{a + \frac d2 - F} + \frac{ F }{a - \frac d2 - F}} \cdot \cbr{c - b} =  \\
    &=\frac {dF}2 \cbr{\frac 1{a + \frac d2 - F} + \frac 1{a - \frac d2 - F}} \cdot \frac {dF^2}{\cbr{a + \frac d2 - F}\cbr{a - \frac d2 - F}} =  \\
    &=\frac {dF}2 \cdot \frac{a - \frac d2 - F + a + \frac d2 - F}{\cbr{a + \frac d2 - F}\cbr{a - \frac d2 - F}} \cdot \frac {dF^2}{\cbr{a + \frac d2 - F}\cbr{a - \frac d2 - F}} =  \\
    &= \frac {d^2F^3}{2\sqr{a + \frac d2 - F}\sqr{a - \frac d2 - F}} \cdot (2a - 2F) = \frac {d^2F^3(a - F)}{ \sqr{\sqr{a - F} - \frac{d^2}4} }.
    \end{align*}
}

\variantsplitter

\addpersonalvariant{Егор Свистушкин}

\tasknumber{1}%
\task{%
    Укажите, верны ли утверждения («да» или «нет» слева от каждого утверждения):
    \begin{itemize}
        \item  Изображение предмета в собирающей линзе всегда мнимое.
        \item  Изображение предмета в собирающей линзе всегда прямое.
        \item  Изображение предмета в собирающей линзе всегда увеличенное.
        \item  Оптическая сила рассеивающей линзы отрицательна.
    \end{itemize}
}
\answer{%
    $\text{ нет, нет, нет, нет }$
}

\tasknumber{2}%
\task{%
    Запишите формулу тонкой линзы и сделайте рисунок, указав на нём физические величины из этой формулы.
}
\solutionspace{60pt}

\tasknumber{3}%
\task{%
    В каких линзах можно получить мнимое изображение объекта?
}
\answer{%
    $\text{ собирающие и рассеивающие }$
}
\solutionspace{40pt}

\tasknumber{4}%
\task{%
    Какое изображение называют мнимым?
}
\solutionspace{40pt}

\tasknumber{5}%
\task{%
    Есть две линзы, обозначим их 1 и 2.
    Известно что оптическая сила линзы 1 больше, чем у линзы 2.
    Какая линза сильнее преломляет лучи?
}
\answer{%
    $1$
}
\solutionspace{40pt}

\tasknumber{6}%
\task{%
    Предмет находится на расстоянии $20\,\text{см}$ от рассеивающей линзы с фокусным расстоянием $40\,\text{см}$.
    Определите тип изображения, расстояние между предметом и его изображением, увеличение предмета.
    Сделайте схематичный рисунок (не обязательно в масштабе, но с сохранением свойств линзы и изображения).
}
\solutionspace{100pt}

\tasknumber{7}%
\task{%
    Объект находится на расстоянии $25\,\text{см}$ от линзы, а его действительное изображение — в $20\,\text{см}$ от неё.
    Определите увеличение предмета, фокусное расстояние линзы, оптическую силу линзы и её тип.
}
\solutionspace{80pt}

\tasknumber{8}%
\task{%
    (Задача-«гроб»: решать на обратной стороне) Квадрат со стороной $d = 1\,\text{см}$ расположен так, что 2 его стороны параллельны главной оптической оси рассеивающей линзы,
    его центр удален на $h = 5\,\text{см}$ от этой оси и на $a = 10\,\text{см}$ от плоскости линзы.
    Определите площадь изображения квадрата, если фокусное расстояние линзы составляет $F = 25\,\text{см}$.
    % (и сравните с площадью объекта, умноженной на квадрат увеличения центра квадрата).
}
\answer{%
    \begin{align*}
    \frac 1 F &= \frac 1{a + \frac d2} + \frac 1b \implies b = \frac 1{\frac 1 F - \frac 1{a + \frac d2}} = \frac{F(a + \frac d2)}{a + \frac d2 - F}, \\
    \frac 1 F &= \frac 1{a - \frac d2} + \frac 1c \implies c = \frac 1{\frac 1 F - \frac 1{a - \frac d2}} = \frac{F(a - \frac d2)}{a - \frac d2 - F}, \\
    c - b &= \frac{F(a - \frac d2)}{a - \frac d2 - F} - \frac{F(a + \frac d2)}{a + \frac d2 - F} = F\cbr{ \frac{a - \frac d2}{a - \frac d2 - F} - \frac{a + \frac d2}{a + \frac d2 - F} } =  \\
    &= F \cdot \frac{a^2 + \frac {ad}2 - aF - \frac{ad}2 - \frac{d^2}4 + \frac{dF}2 - a^2 + \frac {ad}2 + aF - \frac{ad}2 + \frac{d^2}4 + \frac{dF}2}{\cbr{a + \frac d2 - F}\cbr{a - \frac d2 - F}}= F \cdot \frac {dF}{\cbr{a + \frac d2 - F}\cbr{a - \frac d2 - F}}.
    \\
    \Gamma_b &= \frac b{a + \frac d2} = \frac{ F }{a + \frac d2 - F}, \\
    \Gamma_c &= \frac c{a - \frac d2} = \frac{ F }{a - \frac d2 - F}, \\
    &\text{ тут интересно отметить, что } \Gamma_x = \frac{ c - b}{ d } = \frac{ F^2 }{\cbr{a + \frac d2 - F}\cbr{a - \frac d2 - F}} \ne \Gamma_b \text{ или } \Gamma_c \text{ даже при малых $d$}.
    \\
    S' &= \frac{d \cdot \Gamma_b + d \cdot \Gamma_c}2 \cdot (c - b) = \frac d2 \cbr{\frac{ F }{a + \frac d2 - F} + \frac{ F }{a - \frac d2 - F}} \cdot \cbr{c - b} =  \\
    &=\frac {dF}2 \cbr{\frac 1{a + \frac d2 - F} + \frac 1{a - \frac d2 - F}} \cdot \frac {dF^2}{\cbr{a + \frac d2 - F}\cbr{a - \frac d2 - F}} =  \\
    &=\frac {dF}2 \cdot \frac{a - \frac d2 - F + a + \frac d2 - F}{\cbr{a + \frac d2 - F}\cbr{a - \frac d2 - F}} \cdot \frac {dF^2}{\cbr{a + \frac d2 - F}\cbr{a - \frac d2 - F}} =  \\
    &= \frac {d^2F^3}{2\sqr{a + \frac d2 - F}\sqr{a - \frac d2 - F}} \cdot (2a - 2F) = \frac {d^2F^3(a - F)}{ \sqr{\sqr{a - F} - \frac{d^2}4} }.
    \end{align*}
}

\variantsplitter

\addpersonalvariant{Дмитрий Соколов}

\tasknumber{1}%
\task{%
    Укажите, верны ли утверждения («да» или «нет» слева от каждого утверждения):
    \begin{itemize}
        \item  Изображение предмета в собирающей линзе всегда мнимое.
        \item  Изображение предмета в собирающей линзе всегда перевёрнутое.
        \item  Изображение предмета в собирающей линзе всегда увеличенное.
        \item  Оптическая сила собирающей линзы отрицательна.
    \end{itemize}
}
\answer{%
    $\text{ нет, нет, нет, да }$
}

\tasknumber{2}%
\task{%
    Запишите известные вам виды классификации изображений.
}
\solutionspace{60pt}

\tasknumber{3}%
\task{%
    В каких линзах можно получить прямое изображение объекта?
}
\answer{%
    $\text{ собирающие и рассеивающие }$
}
\solutionspace{40pt}

\tasknumber{4}%
\task{%
    Какое изображение называют мнимым?
}
\solutionspace{40pt}

\tasknumber{5}%
\task{%
    Есть две линзы, обозначим их 1 и 2.
    Известно что оптическая сила линзы 1 больше, чем у линзы 2.
    Какая линза сильнее преломляет лучи?
}
\answer{%
    $1$
}
\solutionspace{40pt}

\tasknumber{6}%
\task{%
    Предмет находится на расстоянии $30\,\text{см}$ от рассеивающей линзы с фокусным расстоянием $6\,\text{см}$.
    Определите тип изображения, расстояние между предметом и его изображением, увеличение предмета.
    Сделайте схематичный рисунок (не обязательно в масштабе, но с сохранением свойств линзы и изображения).
}
\solutionspace{100pt}

\tasknumber{7}%
\task{%
    Объект находится на расстоянии $25\,\text{см}$ от линзы, а его мнимое изображение — в $50\,\text{см}$ от неё.
    Определите увеличение предмета, фокусное расстояние линзы, оптическую силу линзы и её тип.
}
\solutionspace{80pt}

\tasknumber{8}%
\task{%
    (Задача-«гроб»: решать на обратной стороне) Квадрат со стороной $d = 3\,\text{см}$ расположен так, что 2 его стороны параллельны главной оптической оси рассеивающей линзы,
    его центр удален на $h = 5\,\text{см}$ от этой оси и на $a = 10\,\text{см}$ от плоскости линзы.
    Определите площадь изображения квадрата, если фокусное расстояние линзы составляет $F = 25\,\text{см}$.
    % (и сравните с площадью объекта, умноженной на квадрат увеличения центра квадрата).
}
\answer{%
    \begin{align*}
    \frac 1 F &= \frac 1{a + \frac d2} + \frac 1b \implies b = \frac 1{\frac 1 F - \frac 1{a + \frac d2}} = \frac{F(a + \frac d2)}{a + \frac d2 - F}, \\
    \frac 1 F &= \frac 1{a - \frac d2} + \frac 1c \implies c = \frac 1{\frac 1 F - \frac 1{a - \frac d2}} = \frac{F(a - \frac d2)}{a - \frac d2 - F}, \\
    c - b &= \frac{F(a - \frac d2)}{a - \frac d2 - F} - \frac{F(a + \frac d2)}{a + \frac d2 - F} = F\cbr{ \frac{a - \frac d2}{a - \frac d2 - F} - \frac{a + \frac d2}{a + \frac d2 - F} } =  \\
    &= F \cdot \frac{a^2 + \frac {ad}2 - aF - \frac{ad}2 - \frac{d^2}4 + \frac{dF}2 - a^2 + \frac {ad}2 + aF - \frac{ad}2 + \frac{d^2}4 + \frac{dF}2}{\cbr{a + \frac d2 - F}\cbr{a - \frac d2 - F}}= F \cdot \frac {dF}{\cbr{a + \frac d2 - F}\cbr{a - \frac d2 - F}}.
    \\
    \Gamma_b &= \frac b{a + \frac d2} = \frac{ F }{a + \frac d2 - F}, \\
    \Gamma_c &= \frac c{a - \frac d2} = \frac{ F }{a - \frac d2 - F}, \\
    &\text{ тут интересно отметить, что } \Gamma_x = \frac{ c - b}{ d } = \frac{ F^2 }{\cbr{a + \frac d2 - F}\cbr{a - \frac d2 - F}} \ne \Gamma_b \text{ или } \Gamma_c \text{ даже при малых $d$}.
    \\
    S' &= \frac{d \cdot \Gamma_b + d \cdot \Gamma_c}2 \cdot (c - b) = \frac d2 \cbr{\frac{ F }{a + \frac d2 - F} + \frac{ F }{a - \frac d2 - F}} \cdot \cbr{c - b} =  \\
    &=\frac {dF}2 \cbr{\frac 1{a + \frac d2 - F} + \frac 1{a - \frac d2 - F}} \cdot \frac {dF^2}{\cbr{a + \frac d2 - F}\cbr{a - \frac d2 - F}} =  \\
    &=\frac {dF}2 \cdot \frac{a - \frac d2 - F + a + \frac d2 - F}{\cbr{a + \frac d2 - F}\cbr{a - \frac d2 - F}} \cdot \frac {dF^2}{\cbr{a + \frac d2 - F}\cbr{a - \frac d2 - F}} =  \\
    &= \frac {d^2F^3}{2\sqr{a + \frac d2 - F}\sqr{a - \frac d2 - F}} \cdot (2a - 2F) = \frac {d^2F^3(a - F)}{ \sqr{\sqr{a - F} - \frac{d^2}4} }.
    \end{align*}
}

\variantsplitter

\addpersonalvariant{Арсений Трофимов}

\tasknumber{1}%
\task{%
    Укажите, верны ли утверждения («да» или «нет» слева от каждого утверждения):
    \begin{itemize}
        \item  Изображение предмета в собирающей линзе всегда мнимое.
        \item  Изображение предмета в собирающей линзе всегда прямое.
        \item  Изображение предмета в собирающей линзе всегда уменьшенное.
        \item  Оптическая сила рассеивающей линзы положительна.
    \end{itemize}
}
\answer{%
    $\text{ нет, нет, нет, да }$
}

\tasknumber{2}%
\task{%
    Запишите известные вам виды классификации изображений.
}
\solutionspace{60pt}

\tasknumber{3}%
\task{%
    В каких линзах можно получить действительное изображение объекта?
}
\answer{%
    $\text{ собирающие }$
}
\solutionspace{40pt}

\tasknumber{4}%
\task{%
    Какое изображение называют действительным?
}
\solutionspace{40pt}

\tasknumber{5}%
\task{%
    Есть две линзы, обозначим их 1 и 2.
    Известно что оптическая сила линзы 1 меньше, чем у линзы 2.
    Какая линза сильнее преломляет лучи?
}
\answer{%
    $2$
}
\solutionspace{40pt}

\tasknumber{6}%
\task{%
    Предмет находится на расстоянии $20\,\text{см}$ от рассеивающей линзы с фокусным расстоянием $40\,\text{см}$.
    Определите тип изображения, расстояние между предметом и его изображением, увеличение предмета.
    Сделайте схематичный рисунок (не обязательно в масштабе, но с сохранением свойств линзы и изображения).
}
\solutionspace{100pt}

\tasknumber{7}%
\task{%
    Объект находится на расстоянии $115\,\text{см}$ от линзы, а его действительное изображение — в $40\,\text{см}$ от неё.
    Определите увеличение предмета, фокусное расстояние линзы, оптическую силу линзы и её тип.
}
\solutionspace{80pt}

\tasknumber{8}%
\task{%
    (Задача-«гроб»: решать на обратной стороне) Квадрат со стороной $d = 3\,\text{см}$ расположен так, что 2 его стороны параллельны главной оптической оси рассеивающей линзы,
    его центр удален на $h = 6\,\text{см}$ от этой оси и на $a = 15\,\text{см}$ от плоскости линзы.
    Определите площадь изображения квадрата, если фокусное расстояние линзы составляет $F = 20\,\text{см}$.
    % (и сравните с площадью объекта, умноженной на квадрат увеличения центра квадрата).
}
\answer{%
    \begin{align*}
    \frac 1 F &= \frac 1{a + \frac d2} + \frac 1b \implies b = \frac 1{\frac 1 F - \frac 1{a + \frac d2}} = \frac{F(a + \frac d2)}{a + \frac d2 - F}, \\
    \frac 1 F &= \frac 1{a - \frac d2} + \frac 1c \implies c = \frac 1{\frac 1 F - \frac 1{a - \frac d2}} = \frac{F(a - \frac d2)}{a - \frac d2 - F}, \\
    c - b &= \frac{F(a - \frac d2)}{a - \frac d2 - F} - \frac{F(a + \frac d2)}{a + \frac d2 - F} = F\cbr{ \frac{a - \frac d2}{a - \frac d2 - F} - \frac{a + \frac d2}{a + \frac d2 - F} } =  \\
    &= F \cdot \frac{a^2 + \frac {ad}2 - aF - \frac{ad}2 - \frac{d^2}4 + \frac{dF}2 - a^2 + \frac {ad}2 + aF - \frac{ad}2 + \frac{d^2}4 + \frac{dF}2}{\cbr{a + \frac d2 - F}\cbr{a - \frac d2 - F}}= F \cdot \frac {dF}{\cbr{a + \frac d2 - F}\cbr{a - \frac d2 - F}}.
    \\
    \Gamma_b &= \frac b{a + \frac d2} = \frac{ F }{a + \frac d2 - F}, \\
    \Gamma_c &= \frac c{a - \frac d2} = \frac{ F }{a - \frac d2 - F}, \\
    &\text{ тут интересно отметить, что } \Gamma_x = \frac{ c - b}{ d } = \frac{ F^2 }{\cbr{a + \frac d2 - F}\cbr{a - \frac d2 - F}} \ne \Gamma_b \text{ или } \Gamma_c \text{ даже при малых $d$}.
    \\
    S' &= \frac{d \cdot \Gamma_b + d \cdot \Gamma_c}2 \cdot (c - b) = \frac d2 \cbr{\frac{ F }{a + \frac d2 - F} + \frac{ F }{a - \frac d2 - F}} \cdot \cbr{c - b} =  \\
    &=\frac {dF}2 \cbr{\frac 1{a + \frac d2 - F} + \frac 1{a - \frac d2 - F}} \cdot \frac {dF^2}{\cbr{a + \frac d2 - F}\cbr{a - \frac d2 - F}} =  \\
    &=\frac {dF}2 \cdot \frac{a - \frac d2 - F + a + \frac d2 - F}{\cbr{a + \frac d2 - F}\cbr{a - \frac d2 - F}} \cdot \frac {dF^2}{\cbr{a + \frac d2 - F}\cbr{a - \frac d2 - F}} =  \\
    &= \frac {d^2F^3}{2\sqr{a + \frac d2 - F}\sqr{a - \frac d2 - F}} \cdot (2a - 2F) = \frac {d^2F^3(a - F)}{ \sqr{\sqr{a - F} - \frac{d^2}4} }.
    \end{align*}
}
% autogenerated
