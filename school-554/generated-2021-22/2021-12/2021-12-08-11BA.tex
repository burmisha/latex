\setdate{8~декабря~2021}
\setclass{11«БА»}

\addpersonalvariant{Михаил Бурмистров}

\tasknumber{1}%
\task{%
    Сформулируйте:
    \begin{itemize}
        \item что такое интерференция,
        \item определение дисперсии,
        \item условие интерференционного максимума для света, падающего нормально на дифракционную решётку,
        \item условия наблюдения минимума и максимума в интерферeнционной картине,
    \end{itemize}
}
\solutionspace{40pt}

\tasknumber{2}%
\task{%
    Произведите вычисления и округлите
    \begin{itemize}
        \item $336{,}92\,\text{м} \cdot 0{,}097\,\text{кг} = \ldots$
        \item $\cfrac{0{,}204\,\text{с}}{3{,}3\,\frac{\text{с}}{\text{м}}} = \ldots$
        \item $0{,}0252\,\text{А} \cdot 0{,}12\,\text{с} = \ldots$
        \item $3 \pi^2 \cdot \cfrac{0{,}0252\,\text{А}}{0{,}12\,\text{с}} = \ldots$
    \end{itemize}
}
\answer{%
    \begin{align*}
    336{,}92\,\text{м} \cdot 0{,}097\,\text{кг} &= 33\,\text{кг}\cdot\text{м} \\
    \cfrac{0{,}204\,\text{с}}{3{,}3\,\frac{\text{с}}{\text{м}}} &= 0{,}062\,\text{м} \\
    0{,}0252\,\text{А} \cdot 0{,}12\,\text{с} &= 0{,}003\,\text{Кл} \\
    3 \pi^2 \cdot \cfrac{0{,}0252\,\text{А}}{0{,}12\,\text{с}} &= 6{,}2\,\frac{\text{А}}{\text{с}}
    \end{align*}
}

\tasknumber{3}%
\task{%
    Установка для наблюдения интерференции состоит
    из двух когерентных источников света и экрана.
    Расстояние между источниками $l = 2{,}4\,\text{мм}$,
    а от каждого источника до экрана — $L = 2\,\text{м}$.
    Сделайте рисунок и укажите положение нулевого максимума освещенности,
    а также определите расстояние между четвёртым минимумом и нулевым максимумом.
    Длина волны падающего света составляет $\lambda = 500\,\text{нм}$.
}
\answer{%
    \begin{align*}
    l_1^2 &= L^2 + \sqr{x - \frac \ell 2} \\
    l_2^2 &= L^2 + \sqr{x + \frac \ell 2} \\
    l_2^2 - l_1^2 &= 2x\ell \implies (l_2 - l_1)(l_2 + l_1) = 2x\ell \implies n\lambda \cdot 2L \approx 2x_n\ell \implies x_n = \frac{\lambda L}{\ell} n, n\in \mathbb{N} \\
    x &= \frac{\lambda L}{\ell} \cdot \frac72 = \frac{500\,\text{нм} \cdot 2\,\text{м}}{2{,}4\,\text{мм}} \cdot \frac72 \approx 1{,}46\,\text{мм}
    \end{align*}
}
\solutionspace{120pt}

\tasknumber{4}%
\task{%
    Каков наибольший порядок спектра, который можно наблюдать при дифракции света
    с длиной волны $\lambda$, на дифракционной решётке с периодом $d =  4{,}1 \lambda$?
    Под каким углом наблюдается последний максимум?
}
\answer{%
    $
        d\sin \varphi_k = k\lambda
        \implies k = \frac{d\sin \varphi_k}{\lambda} \le \frac{d \cdot 1}{\lambda} =  4{,}1
        \implies k_{\max} = 4
        \implies \alpha_{4} \approx 77{,}32\degrees
    $
}
\solutionspace{80pt}

\tasknumber{5}%
\task{%
    Вертикально стоящий шест высотой 1,1 м, освещенный солнцем,
    отбрасывает на горизонтальную поверхность земли тень длиной $2\,\text{м}$.
    Известно, что длина тени от телеграфного столба на $9\,\text{м}$ больше.
    Определить высоту столба.
}
\solutionspace{80pt}

\tasknumber{6}%
\task{%
    Определить абсолютный показатель преломления прозрачной среды,
    в которой распространяется свет с длиной волны $0{,}550\,\text{мкм}$ и частотой $320\,\text{ТГц}$.
    Скорость света в вакууме $3 \cdot 10^{8}\,\frac{\text{м}}{\text{с}}$.
}
\answer{%
    $
        n = \frac{c}{v}
        = \frac{c}{\frac \lambda T}
        = \frac{c}{\lambda \nu}
        = \frac{3 \cdot 10^{8}\,\frac{\text{м}}{\text{с}}}{0{,}550\,\text{мкм} \cdot {320\,\text{ТГц}}}
        \approx1{,}70
    $
}

\variantsplitter

\addpersonalvariant{Ирина Ан}

\tasknumber{1}%
\task{%
    Сформулируйте:
    \begin{itemize}
        \item что такое дифракция,
        \item определение дисперсии,
        \item условие интерференционного максимума для света, падающего нормально на дифракционную решётку,
        \item условия наблюдения минимума и максимума в интерферeнционной картине,
    \end{itemize}
}
\solutionspace{40pt}

\tasknumber{2}%
\task{%
    Произведите вычисления и округлите
    \begin{itemize}
        \item $671{,}8\,\text{м} \cdot 0{,}008\,\text{кг} = \ldots$
        \item $\cfrac{0{,}204\,\text{с}}{4{,}6\,\frac{\text{с}}{\text{м}}} = \ldots$
        \item $0{,}0824\,\text{А} \cdot 4{,}14\,\text{с} = \ldots$
        \item $3 \pi^2 \cdot \cfrac{0{,}0824\,\text{А}}{4{,}14\,\text{с}} = \ldots$
    \end{itemize}
}
\answer{%
    \begin{align*}
    671{,}8\,\text{м} \cdot 0{,}008\,\text{кг} &= 5\,\text{кг}\cdot\text{м} \\
    \cfrac{0{,}204\,\text{с}}{4{,}6\,\frac{\text{с}}{\text{м}}} &= 0{,}044\,\text{м} \\
    0{,}0824\,\text{А} \cdot 4{,}14\,\text{с} &= 0{,}341\,\text{Кл} \\
    3 \pi^2 \cdot \cfrac{0{,}0824\,\text{А}}{4{,}14\,\text{с}} &= 0{,}59\,\frac{\text{А}}{\text{с}}
    \end{align*}
}

\tasknumber{3}%
\task{%
    Установка для наблюдения интерференции состоит
    из двух когерентных источников света и экрана.
    Расстояние между источниками $l = 1{,}5\,\text{мм}$,
    а от каждого источника до экрана — $L = 2\,\text{м}$.
    Сделайте рисунок и укажите положение нулевого максимума освещенности,
    а также определите расстояние между четвёртым минимумом и нулевым максимумом.
    Длина волны падающего света составляет $\lambda = 500\,\text{нм}$.
}
\answer{%
    \begin{align*}
    l_1^2 &= L^2 + \sqr{x - \frac \ell 2} \\
    l_2^2 &= L^2 + \sqr{x + \frac \ell 2} \\
    l_2^2 - l_1^2 &= 2x\ell \implies (l_2 - l_1)(l_2 + l_1) = 2x\ell \implies n\lambda \cdot 2L \approx 2x_n\ell \implies x_n = \frac{\lambda L}{\ell} n, n\in \mathbb{N} \\
    x &= \frac{\lambda L}{\ell} \cdot \frac72 = \frac{500\,\text{нм} \cdot 2\,\text{м}}{1{,}5\,\text{мм}} \cdot \frac72 \approx 2{,}3\,\text{мм}
    \end{align*}
}
\solutionspace{120pt}

\tasknumber{4}%
\task{%
    Каков наибольший порядок спектра, который можно наблюдать при дифракции света
    с длиной волны $\lambda$, на дифракционной решётке с периодом $d =  3{,}3 \lambda$?
    Под каким углом наблюдается последний максимум?
}
\answer{%
    $
        d\sin \varphi_k = k\lambda
        \implies k = \frac{d\sin \varphi_k}{\lambda} \le \frac{d \cdot 1}{\lambda} =  3{,}3
        \implies k_{\max} = 3
        \implies \alpha_{3} \approx 65{,}38\degrees
    $
}
\solutionspace{80pt}

\tasknumber{5}%
\task{%
    Вертикально стоящий шест высотой 1,1 м, освещенный солнцем,
    отбрасывает на горизонтальную поверхность земли тень длиной $3\,\text{м}$.
    Известно, что длина тени от телеграфного столба на $9\,\text{м}$ больше.
    Определить высоту столба.
}
\solutionspace{80pt}

\tasknumber{6}%
\task{%
    Определить абсолютный показатель преломления прозрачной среды,
    в которой распространяется свет с длиной волны $0{,}500\,\text{мкм}$ и частотой $400\,\text{ТГц}$.
    Скорость света в вакууме $3 \cdot 10^{8}\,\frac{\text{м}}{\text{с}}$.
}
\answer{%
    $
        n = \frac{c}{v}
        = \frac{c}{\frac \lambda T}
        = \frac{c}{\lambda \nu}
        = \frac{3 \cdot 10^{8}\,\frac{\text{м}}{\text{с}}}{0{,}500\,\text{мкм} \cdot {400\,\text{ТГц}}}
        \approx1{,}50
    $
}

\variantsplitter

\addpersonalvariant{Софья Андрианова}

\tasknumber{1}%
\task{%
    Сформулируйте:
    \begin{itemize}
        \item что такое интерференция,
        \item определение дисперсии,
        \item условие интерференционного максимума для света, падающего нормально на дифракционную решётку,
        \item условия когерентности 2 источников света,
    \end{itemize}
}
\solutionspace{40pt}

\tasknumber{2}%
\task{%
    Произведите вычисления и округлите
    \begin{itemize}
        \item $671{,}8\,\text{м} \cdot 0{,}053\,\text{кг} = \ldots$
        \item $\cfrac{0{,}562\,\text{с}}{1{,}1\,\frac{\text{с}}{\text{м}}} = \ldots$
        \item $0{,}0252\,\text{А} \cdot 6{,}46\,\text{с} = \ldots$
        \item $3 \pi^2 \cdot \cfrac{0{,}0252\,\text{А}}{6{,}46\,\text{с}} = \ldots$
    \end{itemize}
}
\answer{%
    \begin{align*}
    671{,}8\,\text{м} \cdot 0{,}053\,\text{кг} &= 36\,\text{кг}\cdot\text{м} \\
    \cfrac{0{,}562\,\text{с}}{1{,}1\,\frac{\text{с}}{\text{м}}} &= 0{,}5\,\text{м} \\
    0{,}0252\,\text{А} \cdot 6{,}46\,\text{с} &= 0{,}1628\,\text{Кл} \\
    3 \pi^2 \cdot \cfrac{0{,}0252\,\text{А}}{6{,}46\,\text{с}} &= 0{,}116\,\frac{\text{А}}{\text{с}}
    \end{align*}
}

\tasknumber{3}%
\task{%
    Установка для наблюдения интерференции состоит
    из двух когерентных источников света и экрана.
    Расстояние между источниками $l = 0{,}8\,\text{мм}$,
    а от каждого источника до экрана — $L = 4\,\text{м}$.
    Сделайте рисунок и укажите положение нулевого максимума освещенности,
    а также определите расстояние между вторым максимумом и нулевым максимумом.
    Длина волны падающего света составляет $\lambda = 450\,\text{нм}$.
}
\answer{%
    \begin{align*}
    l_1^2 &= L^2 + \sqr{x - \frac \ell 2} \\
    l_2^2 &= L^2 + \sqr{x + \frac \ell 2} \\
    l_2^2 - l_1^2 &= 2x\ell \implies (l_2 - l_1)(l_2 + l_1) = 2x\ell \implies n\lambda \cdot 2L \approx 2x_n\ell \implies x_n = \frac{\lambda L}{\ell} n, n\in \mathbb{N} \\
    x &= \frac{\lambda L}{\ell} \cdot 2 = \frac{450\,\text{нм} \cdot 4\,\text{м}}{0{,}8\,\text{мм}} \cdot 2 \approx 4{,}5\,\text{мм}
    \end{align*}
}
\solutionspace{120pt}

\tasknumber{4}%
\task{%
    Каков наибольший порядок спектра, который можно наблюдать при дифракции света
    с длиной волны $\lambda$, на дифракционной решётке с периодом $d =  2{,}2 \lambda$?
    Под каким углом наблюдается последний максимум?
}
\answer{%
    $
        d\sin \varphi_k = k\lambda
        \implies k = \frac{d\sin \varphi_k}{\lambda} \le \frac{d \cdot 1}{\lambda} =  2{,}2
        \implies k_{\max} = 2
        \implies \alpha_{2} \approx 65{,}38\degrees
    $
}
\solutionspace{80pt}

\tasknumber{5}%
\task{%
    Вертикально стоящий шест высотой 1,1 м, освещенный солнцем,
    отбрасывает на горизонтальную поверхность земли тень длиной $1\,\text{м}$.
    Известно, что длина тени от телеграфного столба на $8\,\text{м}$ больше.
    Определить высоту столба.
}
\solutionspace{80pt}

\tasknumber{6}%
\task{%
    Определить абсолютный показатель преломления прозрачной среды,
    в которой распространяется свет с длиной волны $0{,}450\,\text{мкм}$ и частотой $480\,\text{ТГц}$.
    Скорость света в вакууме $3 \cdot 10^{8}\,\frac{\text{м}}{\text{с}}$.
}
\answer{%
    $
        n = \frac{c}{v}
        = \frac{c}{\frac \lambda T}
        = \frac{c}{\lambda \nu}
        = \frac{3 \cdot 10^{8}\,\frac{\text{м}}{\text{с}}}{0{,}450\,\text{мкм} \cdot {480\,\text{ТГц}}}
        \approx1{,}40
    $
}

\variantsplitter

\addpersonalvariant{Владимир Артемчук}

\tasknumber{1}%
\task{%
    Сформулируйте:
    \begin{itemize}
        \item что такое дифракция,
        \item определение дисперсии,
        \item условие интерференционного максимума для света, падающего нормально на дифракционную решётку,
        \item условия наблюдения минимума и максимума в интерферeнционной картине,
    \end{itemize}
}
\solutionspace{40pt}

\tasknumber{2}%
\task{%
    Произведите вычисления и округлите
    \begin{itemize}
        \item $55{,}58\,\text{м} \cdot 0{,}097\,\text{кг} = \ldots$
        \item $\cfrac{0{,}562\,\text{с}}{1{,}1\,\frac{\text{с}}{\text{м}}} = \ldots$
        \item $0{,}1286\,\text{А} \cdot 1{,}51\,\text{с} = \ldots$
        \item $3 \pi^2 \cdot \cfrac{0{,}1286\,\text{А}}{1{,}51\,\text{с}} = \ldots$
    \end{itemize}
}
\answer{%
    \begin{align*}
    55{,}58\,\text{м} \cdot 0{,}097\,\text{кг} &= 5{,}4\,\text{кг}\cdot\text{м} \\
    \cfrac{0{,}562\,\text{с}}{1{,}1\,\frac{\text{с}}{\text{м}}} &= 0{,}5\,\text{м} \\
    0{,}1286\,\text{А} \cdot 1{,}51\,\text{с} &= 0{,}194\,\text{Кл} \\
    3 \pi^2 \cdot \cfrac{0{,}1286\,\text{А}}{1{,}51\,\text{с}} &= 2{,}5\,\frac{\text{А}}{\text{с}}
    \end{align*}
}

\tasknumber{3}%
\task{%
    Установка для наблюдения интерференции состоит
    из двух когерентных источников света и экрана.
    Расстояние между источниками $l = 0{,}8\,\text{мм}$,
    а от каждого источника до экрана — $L = 4\,\text{м}$.
    Сделайте рисунок и укажите положение нулевого максимума освещенности,
    а также определите расстояние между третьим максимумом и нулевым максимумом.
    Длина волны падающего света составляет $\lambda = 600\,\text{нм}$.
}
\answer{%
    \begin{align*}
    l_1^2 &= L^2 + \sqr{x - \frac \ell 2} \\
    l_2^2 &= L^2 + \sqr{x + \frac \ell 2} \\
    l_2^2 - l_1^2 &= 2x\ell \implies (l_2 - l_1)(l_2 + l_1) = 2x\ell \implies n\lambda \cdot 2L \approx 2x_n\ell \implies x_n = \frac{\lambda L}{\ell} n, n\in \mathbb{N} \\
    x &= \frac{\lambda L}{\ell} \cdot 3 = \frac{600\,\text{нм} \cdot 4\,\text{м}}{0{,}8\,\text{мм}} \cdot 3 \approx 9\,\text{мм}
    \end{align*}
}
\solutionspace{120pt}

\tasknumber{4}%
\task{%
    Каков наибольший порядок спектра, который можно наблюдать при дифракции света
    с длиной волны $\lambda$, на дифракционной решётке с периодом $d =  2{,}7 \lambda$?
    Под каким углом наблюдается последний максимум?
}
\answer{%
    $
        d\sin \varphi_k = k\lambda
        \implies k = \frac{d\sin \varphi_k}{\lambda} \le \frac{d \cdot 1}{\lambda} =  2{,}7
        \implies k_{\max} = 2
        \implies \alpha_{2} \approx 47{,}79\degrees
    $
}
\solutionspace{80pt}

\tasknumber{5}%
\task{%
    Вертикально стоящий шест высотой 1,1 м, освещенный солнцем,
    отбрасывает на горизонтальную поверхность земли тень длиной $3\,\text{м}$.
    Известно, что длина тени от телеграфного столба на $6\,\text{м}$ больше.
    Определить высоту столба.
}
\solutionspace{80pt}

\tasknumber{6}%
\task{%
    Определить абсолютный показатель преломления прозрачной среды,
    в которой распространяется свет с длиной волны $0{,}650\,\text{мкм}$ и частотой $360\,\text{ТГц}$.
    Скорость света в вакууме $3 \cdot 10^{8}\,\frac{\text{м}}{\text{с}}$.
}
\answer{%
    $
        n = \frac{c}{v}
        = \frac{c}{\frac \lambda T}
        = \frac{c}{\lambda \nu}
        = \frac{3 \cdot 10^{8}\,\frac{\text{м}}{\text{с}}}{0{,}650\,\text{мкм} \cdot {360\,\text{ТГц}}}
        \approx1{,}30
    $
}

\variantsplitter

\addpersonalvariant{Софья Белянкина}

\tasknumber{1}%
\task{%
    Сформулируйте:
    \begin{itemize}
        \item что такое интерференция,
        \item определение дисперсии,
        \item условие интерференционного максимума для света, падающего нормально на дифракционную решётку,
        \item условия наблюдения минимума и максимума в интерферeнционной картине,
    \end{itemize}
}
\solutionspace{40pt}

\tasknumber{2}%
\task{%
    Произведите вычисления и округлите
    \begin{itemize}
        \item $885{,}36\,\text{м} \cdot 0{,}053\,\text{кг} = \ldots$
        \item $\cfrac{0{,}107\,\text{с}}{4{,}8\,\frac{\text{с}}{\text{м}}} = \ldots$
        \item $0{,}0252\,\text{А} \cdot 2{,}05\,\text{с} = \ldots$
        \item $3 \pi^2 \cdot \cfrac{0{,}0252\,\text{А}}{2{,}05\,\text{с}} = \ldots$
    \end{itemize}
}
\answer{%
    \begin{align*}
    885{,}36\,\text{м} \cdot 0{,}053\,\text{кг} &= 47\,\text{кг}\cdot\text{м} \\
    \cfrac{0{,}107\,\text{с}}{4{,}8\,\frac{\text{с}}{\text{м}}} &= 0{,}022\,\text{м} \\
    0{,}0252\,\text{А} \cdot 2{,}05\,\text{с} &= 0{,}0517\,\text{Кл} \\
    3 \pi^2 \cdot \cfrac{0{,}0252\,\text{А}}{2{,}05\,\text{с}} &= 0{,}36\,\frac{\text{А}}{\text{с}}
    \end{align*}
}

\tasknumber{3}%
\task{%
    Установка для наблюдения интерференции состоит
    из двух когерентных источников света и экрана.
    Расстояние между источниками $l = 0{,}8\,\text{мм}$,
    а от каждого источника до экрана — $L = 3\,\text{м}$.
    Сделайте рисунок и укажите положение нулевого максимума освещенности,
    а также определите расстояние между вторым минимумом и нулевым максимумом.
    Длина волны падающего света составляет $\lambda = 450\,\text{нм}$.
}
\answer{%
    \begin{align*}
    l_1^2 &= L^2 + \sqr{x - \frac \ell 2} \\
    l_2^2 &= L^2 + \sqr{x + \frac \ell 2} \\
    l_2^2 - l_1^2 &= 2x\ell \implies (l_2 - l_1)(l_2 + l_1) = 2x\ell \implies n\lambda \cdot 2L \approx 2x_n\ell \implies x_n = \frac{\lambda L}{\ell} n, n\in \mathbb{N} \\
    x &= \frac{\lambda L}{\ell} \cdot \frac32 = \frac{450\,\text{нм} \cdot 3\,\text{м}}{0{,}8\,\text{мм}} \cdot \frac32 \approx 2{,}5\,\text{мм}
    \end{align*}
}
\solutionspace{120pt}

\tasknumber{4}%
\task{%
    Каков наибольший порядок спектра, который можно наблюдать при дифракции света
    с длиной волны $\lambda$, на дифракционной решётке с периодом $d =  4{,}5 \lambda$?
    Под каким углом наблюдается последний максимум?
}
\answer{%
    $
        d\sin \varphi_k = k\lambda
        \implies k = \frac{d\sin \varphi_k}{\lambda} \le \frac{d \cdot 1}{\lambda} =  4{,}5
        \implies k_{\max} = 4
        \implies \alpha_{4} \approx 62{,}73\degrees
    $
}
\solutionspace{80pt}

\tasknumber{5}%
\task{%
    Вертикально стоящий шест высотой 1,1 м, освещенный солнцем,
    отбрасывает на горизонтальную поверхность земли тень длиной $2\,\text{м}$.
    Известно, что длина тени от телеграфного столба на $5\,\text{м}$ больше.
    Определить высоту столба.
}
\solutionspace{80pt}

\tasknumber{6}%
\task{%
    Определить абсолютный показатель преломления прозрачной среды,
    в которой распространяется свет с длиной волны $0{,}500\,\text{мкм}$ и частотой $460\,\text{ТГц}$.
    Скорость света в вакууме $3 \cdot 10^{8}\,\frac{\text{м}}{\text{с}}$.
}
\answer{%
    $
        n = \frac{c}{v}
        = \frac{c}{\frac \lambda T}
        = \frac{c}{\lambda \nu}
        = \frac{3 \cdot 10^{8}\,\frac{\text{м}}{\text{с}}}{0{,}500\,\text{мкм} \cdot {460\,\text{ТГц}}}
        \approx1{,}30
    $
}

\variantsplitter

\addpersonalvariant{Варвара Егиазарян}

\tasknumber{1}%
\task{%
    Сформулируйте:
    \begin{itemize}
        \item что такое интерференция,
        \item определение дисперсии,
        \item условие интерференционного максимума для света, падающего нормально на дифракционную решётку,
        \item условия наблюдения минимума и максимума в интерферeнционной картине,
    \end{itemize}
}
\solutionspace{40pt}

\tasknumber{2}%
\task{%
    Произведите вычисления и округлите
    \begin{itemize}
        \item $55{,}58\,\text{м} \cdot 0{,}053\,\text{кг} = \ldots$
        \item $\cfrac{0{,}204\,\text{с}}{3{,}3\,\frac{\text{с}}{\text{м}}} = \ldots$
        \item $0{,}0025\,\text{А} \cdot 3{,}66\,\text{с} = \ldots$
        \item $3 \pi^2 \cdot \cfrac{0{,}0025\,\text{А}}{3{,}66\,\text{с}} = \ldots$
    \end{itemize}
}
\answer{%
    \begin{align*}
    55{,}58\,\text{м} \cdot 0{,}053\,\text{кг} &= 2{,}9\,\text{кг}\cdot\text{м} \\
    \cfrac{0{,}204\,\text{с}}{3{,}3\,\frac{\text{с}}{\text{м}}} &= 0{,}062\,\text{м} \\
    0{,}0025\,\text{А} \cdot 3{,}66\,\text{с} &= 0{,}0092\,\text{Кл} \\
    3 \pi^2 \cdot \cfrac{0{,}0025\,\text{А}}{3{,}66\,\text{с}} &= 0{,}020\,\frac{\text{А}}{\text{с}}
    \end{align*}
}

\tasknumber{3}%
\task{%
    Установка для наблюдения интерференции состоит
    из двух когерентных источников света и экрана.
    Расстояние между источниками $l = 1{,}2\,\text{мм}$,
    а от каждого источника до экрана — $L = 3\,\text{м}$.
    Сделайте рисунок и укажите положение нулевого максимума освещенности,
    а также определите расстояние между третьим максимумом и нулевым максимумом.
    Длина волны падающего света составляет $\lambda = 550\,\text{нм}$.
}
\answer{%
    \begin{align*}
    l_1^2 &= L^2 + \sqr{x - \frac \ell 2} \\
    l_2^2 &= L^2 + \sqr{x + \frac \ell 2} \\
    l_2^2 - l_1^2 &= 2x\ell \implies (l_2 - l_1)(l_2 + l_1) = 2x\ell \implies n\lambda \cdot 2L \approx 2x_n\ell \implies x_n = \frac{\lambda L}{\ell} n, n\in \mathbb{N} \\
    x &= \frac{\lambda L}{\ell} \cdot 3 = \frac{550\,\text{нм} \cdot 3\,\text{м}}{1{,}2\,\text{мм}} \cdot 3 \approx 4{,}1\,\text{мм}
    \end{align*}
}
\solutionspace{120pt}

\tasknumber{4}%
\task{%
    Каков наибольший порядок спектра, который можно наблюдать при дифракции света
    с длиной волны $\lambda$, на дифракционной решётке с периодом $d =  2{,}5 \lambda$?
    Под каким углом наблюдается последний максимум?
}
\answer{%
    $
        d\sin \varphi_k = k\lambda
        \implies k = \frac{d\sin \varphi_k}{\lambda} \le \frac{d \cdot 1}{\lambda} =  2{,}5
        \implies k_{\max} = 2
        \implies \alpha_{2} \approx 53{,}13\degrees
    $
}
\solutionspace{80pt}

\tasknumber{5}%
\task{%
    Вертикально стоящий шест высотой 1,1 м, освещенный солнцем,
    отбрасывает на горизонтальную поверхность земли тень длиной $3\,\text{м}$.
    Известно, что длина тени от телеграфного столба на $7\,\text{м}$ больше.
    Определить высоту столба.
}
\solutionspace{80pt}

\tasknumber{6}%
\task{%
    Определить абсолютный показатель преломления прозрачной среды,
    в которой распространяется свет с длиной волны $0{,}550\,\text{мкм}$ и частотой $420\,\text{ТГц}$.
    Скорость света в вакууме $3 \cdot 10^{8}\,\frac{\text{м}}{\text{с}}$.
}
\answer{%
    $
        n = \frac{c}{v}
        = \frac{c}{\frac \lambda T}
        = \frac{c}{\lambda \nu}
        = \frac{3 \cdot 10^{8}\,\frac{\text{м}}{\text{с}}}{0{,}550\,\text{мкм} \cdot {420\,\text{ТГц}}}
        \approx1{,}30
    $
}

\variantsplitter

\addpersonalvariant{Владислав Емелин}

\tasknumber{1}%
\task{%
    Сформулируйте:
    \begin{itemize}
        \item что такое интерференция,
        \item определение дисперсии,
        \item условие интерференционного максимума для света, падающего нормально на дифракционную решётку,
        \item условия когерентности 2 источников света,
    \end{itemize}
}
\solutionspace{40pt}

\tasknumber{2}%
\task{%
    Произведите вычисления и округлите
    \begin{itemize}
        \item $59{,}11\,\text{м} \cdot 0{,}063\,\text{кг} = \ldots$
        \item $\cfrac{0{,}107\,\text{с}}{5{,}2\,\frac{\text{с}}{\text{м}}} = \ldots$
        \item $0{,}0288\,\text{А} \cdot 1{,}51\,\text{с} = \ldots$
        \item $3 \pi^2 \cdot \cfrac{0{,}0288\,\text{А}}{1{,}51\,\text{с}} = \ldots$
    \end{itemize}
}
\answer{%
    \begin{align*}
    59{,}11\,\text{м} \cdot 0{,}063\,\text{кг} &= 3{,}7\,\text{кг}\cdot\text{м} \\
    \cfrac{0{,}107\,\text{с}}{5{,}2\,\frac{\text{с}}{\text{м}}} &= 0{,}021\,\text{м} \\
    0{,}0288\,\text{А} \cdot 1{,}51\,\text{с} &= 0{,}043\,\text{Кл} \\
    3 \pi^2 \cdot \cfrac{0{,}0288\,\text{А}}{1{,}51\,\text{с}} &= 0{,}56\,\frac{\text{А}}{\text{с}}
    \end{align*}
}

\tasknumber{3}%
\task{%
    Установка для наблюдения интерференции состоит
    из двух когерентных источников света и экрана.
    Расстояние между источниками $l = 2{,}4\,\text{мм}$,
    а от каждого источника до экрана — $L = 3\,\text{м}$.
    Сделайте рисунок и укажите положение нулевого максимума освещенности,
    а также определите расстояние между третьим минимумом и нулевым максимумом.
    Длина волны падающего света составляет $\lambda = 600\,\text{нм}$.
}
\answer{%
    \begin{align*}
    l_1^2 &= L^2 + \sqr{x - \frac \ell 2} \\
    l_2^2 &= L^2 + \sqr{x + \frac \ell 2} \\
    l_2^2 - l_1^2 &= 2x\ell \implies (l_2 - l_1)(l_2 + l_1) = 2x\ell \implies n\lambda \cdot 2L \approx 2x_n\ell \implies x_n = \frac{\lambda L}{\ell} n, n\in \mathbb{N} \\
    x &= \frac{\lambda L}{\ell} \cdot \frac52 = \frac{600\,\text{нм} \cdot 3\,\text{м}}{2{,}4\,\text{мм}} \cdot \frac52 \approx 1{,}88\,\text{мм}
    \end{align*}
}
\solutionspace{120pt}

\tasknumber{4}%
\task{%
    Каков наибольший порядок спектра, который можно наблюдать при дифракции света
    с длиной волны $\lambda$, на дифракционной решётке с периодом $d =  3{,}5 \lambda$?
    Под каким углом наблюдается последний максимум?
}
\answer{%
    $
        d\sin \varphi_k = k\lambda
        \implies k = \frac{d\sin \varphi_k}{\lambda} \le \frac{d \cdot 1}{\lambda} =  3{,}5
        \implies k_{\max} = 3
        \implies \alpha_{3} \approx 59{,}00\degrees
    $
}
\solutionspace{80pt}

\tasknumber{5}%
\task{%
    Вертикально стоящий шест высотой 1,1 м, освещенный солнцем,
    отбрасывает на горизонтальную поверхность земли тень длиной $2\,\text{м}$.
    Известно, что длина тени от телеграфного столба на $8\,\text{м}$ больше.
    Определить высоту столба.
}
\solutionspace{80pt}

\tasknumber{6}%
\task{%
    Определить абсолютный показатель преломления прозрачной среды,
    в которой распространяется свет с длиной волны $0{,}650\,\text{мкм}$ и частотой $330\,\text{ТГц}$.
    Скорость света в вакууме $3 \cdot 10^{8}\,\frac{\text{м}}{\text{с}}$.
}
\answer{%
    $
        n = \frac{c}{v}
        = \frac{c}{\frac \lambda T}
        = \frac{c}{\lambda \nu}
        = \frac{3 \cdot 10^{8}\,\frac{\text{м}}{\text{с}}}{0{,}650\,\text{мкм} \cdot {330\,\text{ТГц}}}
        \approx1{,}40
    $
}

\variantsplitter

\addpersonalvariant{Артём Жичин}

\tasknumber{1}%
\task{%
    Сформулируйте:
    \begin{itemize}
        \item что такое дифракция,
        \item определение дисперсии,
        \item условие интерференционного максимума для света, падающего нормально на дифракционную решётку,
        \item условия наблюдения минимума и максимума в интерферeнционной картине,
    \end{itemize}
}
\solutionspace{40pt}

\tasknumber{2}%
\task{%
    Произведите вычисления и округлите
    \begin{itemize}
        \item $78{,}97\,\text{м} \cdot 0{,}082\,\text{кг} = \ldots$
        \item $\cfrac{0{,}562\,\text{с}}{3{,}2\,\frac{\text{с}}{\text{м}}} = \ldots$
        \item $0{,}0824\,\text{А} \cdot 0{,}12\,\text{с} = \ldots$
        \item $3 \pi^2 \cdot \cfrac{0{,}0824\,\text{А}}{0{,}12\,\text{с}} = \ldots$
    \end{itemize}
}
\answer{%
    \begin{align*}
    78{,}97\,\text{м} \cdot 0{,}082\,\text{кг} &= 6{,}5\,\text{кг}\cdot\text{м} \\
    \cfrac{0{,}562\,\text{с}}{3{,}2\,\frac{\text{с}}{\text{м}}} &= 0{,}176\,\text{м} \\
    0{,}0824\,\text{А} \cdot 0{,}12\,\text{с} &= 0{,}010\,\text{Кл} \\
    3 \pi^2 \cdot \cfrac{0{,}0824\,\text{А}}{0{,}12\,\text{с}} &= 20\,\frac{\text{А}}{\text{с}}
    \end{align*}
}

\tasknumber{3}%
\task{%
    Установка для наблюдения интерференции состоит
    из двух когерентных источников света и экрана.
    Расстояние между источниками $l = 1{,}2\,\text{мм}$,
    а от каждого источника до экрана — $L = 2\,\text{м}$.
    Сделайте рисунок и укажите положение нулевого максимума освещенности,
    а также определите расстояние между четвёртым максимумом и нулевым максимумом.
    Длина волны падающего света составляет $\lambda = 450\,\text{нм}$.
}
\answer{%
    \begin{align*}
    l_1^2 &= L^2 + \sqr{x - \frac \ell 2} \\
    l_2^2 &= L^2 + \sqr{x + \frac \ell 2} \\
    l_2^2 - l_1^2 &= 2x\ell \implies (l_2 - l_1)(l_2 + l_1) = 2x\ell \implies n\lambda \cdot 2L \approx 2x_n\ell \implies x_n = \frac{\lambda L}{\ell} n, n\in \mathbb{N} \\
    x &= \frac{\lambda L}{\ell} \cdot 4 = \frac{450\,\text{нм} \cdot 2\,\text{м}}{1{,}2\,\text{мм}} \cdot 4 \approx 3\,\text{мм}
    \end{align*}
}
\solutionspace{120pt}

\tasknumber{4}%
\task{%
    Каков наибольший порядок спектра, который можно наблюдать при дифракции света
    с длиной волны $\lambda$, на дифракционной решётке с периодом $d =  2{,}7 \lambda$?
    Под каким углом наблюдается последний максимум?
}
\answer{%
    $
        d\sin \varphi_k = k\lambda
        \implies k = \frac{d\sin \varphi_k}{\lambda} \le \frac{d \cdot 1}{\lambda} =  2{,}7
        \implies k_{\max} = 2
        \implies \alpha_{2} \approx 47{,}79\degrees
    $
}
\solutionspace{80pt}

\tasknumber{5}%
\task{%
    Вертикально стоящий шест высотой 1,1 м, освещенный солнцем,
    отбрасывает на горизонтальную поверхность земли тень длиной $3\,\text{м}$.
    Известно, что длина тени от телеграфного столба на $9\,\text{м}$ больше.
    Определить высоту столба.
}
\solutionspace{80pt}

\tasknumber{6}%
\task{%
    Определить абсолютный показатель преломления прозрачной среды,
    в которой распространяется свет с длиной волны $0{,}550\,\text{мкм}$ и частотой $390\,\text{ТГц}$.
    Скорость света в вакууме $3 \cdot 10^{8}\,\frac{\text{м}}{\text{с}}$.
}
\answer{%
    $
        n = \frac{c}{v}
        = \frac{c}{\frac \lambda T}
        = \frac{c}{\lambda \nu}
        = \frac{3 \cdot 10^{8}\,\frac{\text{м}}{\text{с}}}{0{,}550\,\text{мкм} \cdot {390\,\text{ТГц}}}
        \approx1{,}40
    $
}

\variantsplitter

\addpersonalvariant{Дарья Кошман}

\tasknumber{1}%
\task{%
    Сформулируйте:
    \begin{itemize}
        \item что такое дифракция,
        \item определение дисперсии,
        \item условие интерференционного максимума для света, падающего нормально на дифракционную решётку,
        \item условия когерентности 2 источников света,
    \end{itemize}
}
\solutionspace{40pt}

\tasknumber{2}%
\task{%
    Произведите вычисления и округлите
    \begin{itemize}
        \item $540\,\text{м} \cdot 0{,}008\,\text{кг} = \ldots$
        \item $\cfrac{0{,}954\,\text{с}}{4{,}9\,\frac{\text{с}}{\text{м}}} = \ldots$
        \item $0{,}0252\,\text{А} \cdot 3{,}66\,\text{с} = \ldots$
        \item $3 \pi^2 \cdot \cfrac{0{,}0252\,\text{А}}{3{,}66\,\text{с}} = \ldots$
    \end{itemize}
}
\answer{%
    \begin{align*}
    540\,\text{м} \cdot 0{,}008\,\text{кг} &= 4\,\text{кг}\cdot\text{м} \\
    \cfrac{0{,}954\,\text{с}}{4{,}9\,\frac{\text{с}}{\text{м}}} &= 0{,}195\,\text{м} \\
    0{,}0252\,\text{А} \cdot 3{,}66\,\text{с} &= 0{,}0922\,\text{Кл} \\
    3 \pi^2 \cdot \cfrac{0{,}0252\,\text{А}}{3{,}66\,\text{с}} &= 0{,}20\,\frac{\text{А}}{\text{с}}
    \end{align*}
}

\tasknumber{3}%
\task{%
    Установка для наблюдения интерференции состоит
    из двух когерентных источников света и экрана.
    Расстояние между источниками $l = 1{,}5\,\text{мм}$,
    а от каждого источника до экрана — $L = 4\,\text{м}$.
    Сделайте рисунок и укажите положение нулевого максимума освещенности,
    а также определите расстояние между вторым минимумом и нулевым максимумом.
    Длина волны падающего света составляет $\lambda = 550\,\text{нм}$.
}
\answer{%
    \begin{align*}
    l_1^2 &= L^2 + \sqr{x - \frac \ell 2} \\
    l_2^2 &= L^2 + \sqr{x + \frac \ell 2} \\
    l_2^2 - l_1^2 &= 2x\ell \implies (l_2 - l_1)(l_2 + l_1) = 2x\ell \implies n\lambda \cdot 2L \approx 2x_n\ell \implies x_n = \frac{\lambda L}{\ell} n, n\in \mathbb{N} \\
    x &= \frac{\lambda L}{\ell} \cdot \frac32 = \frac{550\,\text{нм} \cdot 4\,\text{м}}{1{,}5\,\text{мм}} \cdot \frac32 \approx 2{,}2\,\text{мм}
    \end{align*}
}
\solutionspace{120pt}

\tasknumber{4}%
\task{%
    Каков наибольший порядок спектра, который можно наблюдать при дифракции света
    с длиной волны $\lambda$, на дифракционной решётке с периодом $d =  4{,}1 \lambda$?
    Под каким углом наблюдается последний максимум?
}
\answer{%
    $
        d\sin \varphi_k = k\lambda
        \implies k = \frac{d\sin \varphi_k}{\lambda} \le \frac{d \cdot 1}{\lambda} =  4{,}1
        \implies k_{\max} = 4
        \implies \alpha_{4} \approx 77{,}32\degrees
    $
}
\solutionspace{80pt}

\tasknumber{5}%
\task{%
    Вертикально стоящий шест высотой 1,1 м, освещенный солнцем,
    отбрасывает на горизонтальную поверхность земли тень длиной $2\,\text{м}$.
    Известно, что длина тени от телеграфного столба на $6\,\text{м}$ больше.
    Определить высоту столба.
}
\solutionspace{80pt}

\tasknumber{6}%
\task{%
    Определить абсолютный показатель преломления прозрачной среды,
    в которой распространяется свет с длиной волны $0{,}600\,\text{мкм}$ и частотой $360\,\text{ТГц}$.
    Скорость света в вакууме $3 \cdot 10^{8}\,\frac{\text{м}}{\text{с}}$.
}
\answer{%
    $
        n = \frac{c}{v}
        = \frac{c}{\frac \lambda T}
        = \frac{c}{\lambda \nu}
        = \frac{3 \cdot 10^{8}\,\frac{\text{м}}{\text{с}}}{0{,}600\,\text{мкм} \cdot {360\,\text{ТГц}}}
        \approx1{,}40
    $
}

\variantsplitter

\addpersonalvariant{Анна Кузьмичёва}

\tasknumber{1}%
\task{%
    Сформулируйте:
    \begin{itemize}
        \item что такое интерференция,
        \item определение дисперсии,
        \item условие интерференционного максимума для света, падающего нормально на дифракционную решётку,
        \item условия когерентности 2 источников света,
    \end{itemize}
}
\solutionspace{40pt}

\tasknumber{2}%
\task{%
    Произведите вычисления и округлите
    \begin{itemize}
        \item $55{,}58\,\text{м} \cdot 0{,}051\,\text{кг} = \ldots$
        \item $\cfrac{0{,}705\,\text{с}}{4{,}6\,\frac{\text{с}}{\text{м}}} = \ldots$
        \item $0{,}0288\,\text{А} \cdot 5{,}89\,\text{с} = \ldots$
        \item $3 \pi^2 \cdot \cfrac{0{,}0288\,\text{А}}{5{,}89\,\text{с}} = \ldots$
    \end{itemize}
}
\answer{%
    \begin{align*}
    55{,}58\,\text{м} \cdot 0{,}051\,\text{кг} &= 2{,}8\,\text{кг}\cdot\text{м} \\
    \cfrac{0{,}705\,\text{с}}{4{,}6\,\frac{\text{с}}{\text{м}}} &= 0{,}153\,\text{м} \\
    0{,}0288\,\text{А} \cdot 5{,}89\,\text{с} &= 0{,}1696\,\text{Кл} \\
    3 \pi^2 \cdot \cfrac{0{,}0288\,\text{А}}{5{,}89\,\text{с}} &= 0{,}145\,\frac{\text{А}}{\text{с}}
    \end{align*}
}

\tasknumber{3}%
\task{%
    Установка для наблюдения интерференции состоит
    из двух когерентных источников света и экрана.
    Расстояние между источниками $l = 1{,}5\,\text{мм}$,
    а от каждого источника до экрана — $L = 4\,\text{м}$.
    Сделайте рисунок и укажите положение нулевого максимума освещенности,
    а также определите расстояние между четвёртым максимумом и нулевым максимумом.
    Длина волны падающего света составляет $\lambda = 450\,\text{нм}$.
}
\answer{%
    \begin{align*}
    l_1^2 &= L^2 + \sqr{x - \frac \ell 2} \\
    l_2^2 &= L^2 + \sqr{x + \frac \ell 2} \\
    l_2^2 - l_1^2 &= 2x\ell \implies (l_2 - l_1)(l_2 + l_1) = 2x\ell \implies n\lambda \cdot 2L \approx 2x_n\ell \implies x_n = \frac{\lambda L}{\ell} n, n\in \mathbb{N} \\
    x &= \frac{\lambda L}{\ell} \cdot 4 = \frac{450\,\text{нм} \cdot 4\,\text{м}}{1{,}5\,\text{мм}} \cdot 4 \approx 4{,}8\,\text{мм}
    \end{align*}
}
\solutionspace{120pt}

\tasknumber{4}%
\task{%
    Каков наибольший порядок спектра, который можно наблюдать при дифракции света
    с длиной волны $\lambda$, на дифракционной решётке с периодом $d =  3{,}3 \lambda$?
    Под каким углом наблюдается последний максимум?
}
\answer{%
    $
        d\sin \varphi_k = k\lambda
        \implies k = \frac{d\sin \varphi_k}{\lambda} \le \frac{d \cdot 1}{\lambda} =  3{,}3
        \implies k_{\max} = 3
        \implies \alpha_{3} \approx 65{,}38\degrees
    $
}
\solutionspace{80pt}

\tasknumber{5}%
\task{%
    Вертикально стоящий шест высотой 1,1 м, освещенный солнцем,
    отбрасывает на горизонтальную поверхность земли тень длиной $4\,\text{м}$.
    Известно, что длина тени от телеграфного столба на $8\,\text{м}$ больше.
    Определить высоту столба.
}
\solutionspace{80pt}

\tasknumber{6}%
\task{%
    Определить абсолютный показатель преломления прозрачной среды,
    в которой распространяется свет с длиной волны $0{,}500\,\text{мкм}$ и частотой $400\,\text{ТГц}$.
    Скорость света в вакууме $3 \cdot 10^{8}\,\frac{\text{м}}{\text{с}}$.
}
\answer{%
    $
        n = \frac{c}{v}
        = \frac{c}{\frac \lambda T}
        = \frac{c}{\lambda \nu}
        = \frac{3 \cdot 10^{8}\,\frac{\text{м}}{\text{с}}}{0{,}500\,\text{мкм} \cdot {400\,\text{ТГц}}}
        \approx1{,}50
    $
}

\variantsplitter

\addpersonalvariant{Алёна Куприянова}

\tasknumber{1}%
\task{%
    Сформулируйте:
    \begin{itemize}
        \item что такое дифракция,
        \item определение дисперсии,
        \item условие интерференционного максимума для света, падающего нормально на дифракционную решётку,
        \item условия когерентности 2 источников света,
    \end{itemize}
}
\solutionspace{40pt}

\tasknumber{2}%
\task{%
    Произведите вычисления и округлите
    \begin{itemize}
        \item $540\,\text{м} \cdot 0{,}063\,\text{кг} = \ldots$
        \item $\cfrac{0{,}107\,\text{с}}{1{,}1\,\frac{\text{с}}{\text{м}}} = \ldots$
        \item $0{,}1364\,\text{А} \cdot 2{,}05\,\text{с} = \ldots$
        \item $3 \pi^2 \cdot \cfrac{0{,}1364\,\text{А}}{2{,}05\,\text{с}} = \ldots$
    \end{itemize}
}
\answer{%
    \begin{align*}
    540\,\text{м} \cdot 0{,}063\,\text{кг} &= 34\,\text{кг}\cdot\text{м} \\
    \cfrac{0{,}107\,\text{с}}{1{,}1\,\frac{\text{с}}{\text{м}}} &= 0{,}10\,\text{м} \\
    0{,}1364\,\text{А} \cdot 2{,}05\,\text{с} &= 0{,}280\,\text{Кл} \\
    3 \pi^2 \cdot \cfrac{0{,}1364\,\text{А}}{2{,}05\,\text{с}} &= 1{,}97\,\frac{\text{А}}{\text{с}}
    \end{align*}
}

\tasknumber{3}%
\task{%
    Установка для наблюдения интерференции состоит
    из двух когерентных источников света и экрана.
    Расстояние между источниками $l = 0{,}8\,\text{мм}$,
    а от каждого источника до экрана — $L = 3\,\text{м}$.
    Сделайте рисунок и укажите положение нулевого максимума освещенности,
    а также определите расстояние между третьим максимумом и нулевым максимумом.
    Длина волны падающего света составляет $\lambda = 400\,\text{нм}$.
}
\answer{%
    \begin{align*}
    l_1^2 &= L^2 + \sqr{x - \frac \ell 2} \\
    l_2^2 &= L^2 + \sqr{x + \frac \ell 2} \\
    l_2^2 - l_1^2 &= 2x\ell \implies (l_2 - l_1)(l_2 + l_1) = 2x\ell \implies n\lambda \cdot 2L \approx 2x_n\ell \implies x_n = \frac{\lambda L}{\ell} n, n\in \mathbb{N} \\
    x &= \frac{\lambda L}{\ell} \cdot 3 = \frac{400\,\text{нм} \cdot 3\,\text{м}}{0{,}8\,\text{мм}} \cdot 3 \approx 4{,}5\,\text{мм}
    \end{align*}
}
\solutionspace{120pt}

\tasknumber{4}%
\task{%
    Каков наибольший порядок спектра, который можно наблюдать при дифракции света
    с длиной волны $\lambda$, на дифракционной решётке с периодом $d =  2{,}7 \lambda$?
    Под каким углом наблюдается последний максимум?
}
\answer{%
    $
        d\sin \varphi_k = k\lambda
        \implies k = \frac{d\sin \varphi_k}{\lambda} \le \frac{d \cdot 1}{\lambda} =  2{,}7
        \implies k_{\max} = 2
        \implies \alpha_{2} \approx 47{,}79\degrees
    $
}
\solutionspace{80pt}

\tasknumber{5}%
\task{%
    Вертикально стоящий шест высотой 1,1 м, освещенный солнцем,
    отбрасывает на горизонтальную поверхность земли тень длиной $2\,\text{м}$.
    Известно, что длина тени от телеграфного столба на $9\,\text{м}$ больше.
    Определить высоту столба.
}
\solutionspace{80pt}

\tasknumber{6}%
\task{%
    Определить абсолютный показатель преломления прозрачной среды,
    в которой распространяется свет с длиной волны $0{,}500\,\text{мкм}$ и частотой $400\,\text{ТГц}$.
    Скорость света в вакууме $3 \cdot 10^{8}\,\frac{\text{м}}{\text{с}}$.
}
\answer{%
    $
        n = \frac{c}{v}
        = \frac{c}{\frac \lambda T}
        = \frac{c}{\lambda \nu}
        = \frac{3 \cdot 10^{8}\,\frac{\text{м}}{\text{с}}}{0{,}500\,\text{мкм} \cdot {400\,\text{ТГц}}}
        \approx1{,}50
    $
}

\variantsplitter

\addpersonalvariant{Ярослав Лавровский}

\tasknumber{1}%
\task{%
    Сформулируйте:
    \begin{itemize}
        \item что такое дифракция,
        \item определение дисперсии,
        \item условие интерференционного максимума для света, падающего нормально на дифракционную решётку,
        \item условия когерентности 2 источников света,
    \end{itemize}
}
\solutionspace{40pt}

\tasknumber{2}%
\task{%
    Произведите вычисления и округлите
    \begin{itemize}
        \item $540\,\text{м} \cdot 0{,}046\,\text{кг} = \ldots$
        \item $\cfrac{0{,}858\,\text{с}}{3{,}3\,\frac{\text{с}}{\text{м}}} = \ldots$
        \item $0{,}1152\,\text{А} \cdot 6{,}46\,\text{с} = \ldots$
        \item $3 \pi^2 \cdot \cfrac{0{,}1152\,\text{А}}{6{,}46\,\text{с}} = \ldots$
    \end{itemize}
}
\answer{%
    \begin{align*}
    540\,\text{м} \cdot 0{,}046\,\text{кг} &= 25\,\text{кг}\cdot\text{м} \\
    \cfrac{0{,}858\,\text{с}}{3{,}3\,\frac{\text{с}}{\text{м}}} &= 0{,}26\,\text{м} \\
    0{,}1152\,\text{А} \cdot 6{,}46\,\text{с} &= 0{,}744\,\text{Кл} \\
    3 \pi^2 \cdot \cfrac{0{,}1152\,\text{А}}{6{,}46\,\text{с}} &= 0{,}53\,\frac{\text{А}}{\text{с}}
    \end{align*}
}

\tasknumber{3}%
\task{%
    Установка для наблюдения интерференции состоит
    из двух когерентных источников света и экрана.
    Расстояние между источниками $l = 1{,}5\,\text{мм}$,
    а от каждого источника до экрана — $L = 2\,\text{м}$.
    Сделайте рисунок и укажите положение нулевого максимума освещенности,
    а также определите расстояние между вторым минимумом и нулевым максимумом.
    Длина волны падающего света составляет $\lambda = 400\,\text{нм}$.
}
\answer{%
    \begin{align*}
    l_1^2 &= L^2 + \sqr{x - \frac \ell 2} \\
    l_2^2 &= L^2 + \sqr{x + \frac \ell 2} \\
    l_2^2 - l_1^2 &= 2x\ell \implies (l_2 - l_1)(l_2 + l_1) = 2x\ell \implies n\lambda \cdot 2L \approx 2x_n\ell \implies x_n = \frac{\lambda L}{\ell} n, n\in \mathbb{N} \\
    x &= \frac{\lambda L}{\ell} \cdot \frac32 = \frac{400\,\text{нм} \cdot 2\,\text{м}}{1{,}5\,\text{мм}} \cdot \frac32 \approx 0{,}80\,\text{мм}
    \end{align*}
}
\solutionspace{120pt}

\tasknumber{4}%
\task{%
    Каков наибольший порядок спектра, который можно наблюдать при дифракции света
    с длиной волны $\lambda$, на дифракционной решётке с периодом $d =  2{,}5 \lambda$?
    Под каким углом наблюдается последний максимум?
}
\answer{%
    $
        d\sin \varphi_k = k\lambda
        \implies k = \frac{d\sin \varphi_k}{\lambda} \le \frac{d \cdot 1}{\lambda} =  2{,}5
        \implies k_{\max} = 2
        \implies \alpha_{2} \approx 53{,}13\degrees
    $
}
\solutionspace{80pt}

\tasknumber{5}%
\task{%
    Вертикально стоящий шест высотой 1,1 м, освещенный солнцем,
    отбрасывает на горизонтальную поверхность земли тень длиной $2\,\text{м}$.
    Известно, что длина тени от телеграфного столба на $6\,\text{м}$ больше.
    Определить высоту столба.
}
\solutionspace{80pt}

\tasknumber{6}%
\task{%
    Определить абсолютный показатель преломления прозрачной среды,
    в которой распространяется свет с длиной волны $0{,}650\,\text{мкм}$ и частотой $310\,\text{ТГц}$.
    Скорость света в вакууме $3 \cdot 10^{8}\,\frac{\text{м}}{\text{с}}$.
}
\answer{%
    $
        n = \frac{c}{v}
        = \frac{c}{\frac \lambda T}
        = \frac{c}{\lambda \nu}
        = \frac{3 \cdot 10^{8}\,\frac{\text{м}}{\text{с}}}{0{,}650\,\text{мкм} \cdot {310\,\text{ТГц}}}
        \approx1{,}50
    $
}

\variantsplitter

\addpersonalvariant{Анастасия Ламанова}

\tasknumber{1}%
\task{%
    Сформулируйте:
    \begin{itemize}
        \item что такое дифракция,
        \item определение дисперсии,
        \item условие интерференционного максимума для света, падающего нормально на дифракционную решётку,
        \item условия когерентности 2 источников света,
    \end{itemize}
}
\solutionspace{40pt}

\tasknumber{2}%
\task{%
    Произведите вычисления и округлите
    \begin{itemize}
        \item $540\,\text{м} \cdot 0{,}051\,\text{кг} = \ldots$
        \item $\cfrac{0{,}562\,\text{с}}{4{,}8\,\frac{\text{с}}{\text{м}}} = \ldots$
        \item $0{,}1364\,\text{А} \cdot 2{,}05\,\text{с} = \ldots$
        \item $3 \pi^2 \cdot \cfrac{0{,}1364\,\text{А}}{2{,}05\,\text{с}} = \ldots$
    \end{itemize}
}
\answer{%
    \begin{align*}
    540\,\text{м} \cdot 0{,}051\,\text{кг} &= 28\,\text{кг}\cdot\text{м} \\
    \cfrac{0{,}562\,\text{с}}{4{,}8\,\frac{\text{с}}{\text{м}}} &= 0{,}117\,\text{м} \\
    0{,}1364\,\text{А} \cdot 2{,}05\,\text{с} &= 0{,}280\,\text{Кл} \\
    3 \pi^2 \cdot \cfrac{0{,}1364\,\text{А}}{2{,}05\,\text{с}} &= 1{,}97\,\frac{\text{А}}{\text{с}}
    \end{align*}
}

\tasknumber{3}%
\task{%
    Установка для наблюдения интерференции состоит
    из двух когерентных источников света и экрана.
    Расстояние между источниками $l = 1{,}2\,\text{мм}$,
    а от каждого источника до экрана — $L = 4\,\text{м}$.
    Сделайте рисунок и укажите положение нулевого максимума освещенности,
    а также определите расстояние между четвёртым минимумом и нулевым максимумом.
    Длина волны падающего света составляет $\lambda = 400\,\text{нм}$.
}
\answer{%
    \begin{align*}
    l_1^2 &= L^2 + \sqr{x - \frac \ell 2} \\
    l_2^2 &= L^2 + \sqr{x + \frac \ell 2} \\
    l_2^2 - l_1^2 &= 2x\ell \implies (l_2 - l_1)(l_2 + l_1) = 2x\ell \implies n\lambda \cdot 2L \approx 2x_n\ell \implies x_n = \frac{\lambda L}{\ell} n, n\in \mathbb{N} \\
    x &= \frac{\lambda L}{\ell} \cdot \frac72 = \frac{400\,\text{нм} \cdot 4\,\text{м}}{1{,}2\,\text{мм}} \cdot \frac72 \approx 4{,}7\,\text{мм}
    \end{align*}
}
\solutionspace{120pt}

\tasknumber{4}%
\task{%
    Каков наибольший порядок спектра, который можно наблюдать при дифракции света
    с длиной волны $\lambda$, на дифракционной решётке с периодом $d =  3{,}3 \lambda$?
    Под каким углом наблюдается последний максимум?
}
\answer{%
    $
        d\sin \varphi_k = k\lambda
        \implies k = \frac{d\sin \varphi_k}{\lambda} \le \frac{d \cdot 1}{\lambda} =  3{,}3
        \implies k_{\max} = 3
        \implies \alpha_{3} \approx 65{,}38\degrees
    $
}
\solutionspace{80pt}

\tasknumber{5}%
\task{%
    Вертикально стоящий шест высотой 1,1 м, освещенный солнцем,
    отбрасывает на горизонтальную поверхность земли тень длиной $3\,\text{м}$.
    Известно, что длина тени от телеграфного столба на $5\,\text{м}$ больше.
    Определить высоту столба.
}
\solutionspace{80pt}

\tasknumber{6}%
\task{%
    Определить абсолютный показатель преломления прозрачной среды,
    в которой распространяется свет с длиной волны $0{,}650\,\text{мкм}$ и частотой $310\,\text{ТГц}$.
    Скорость света в вакууме $3 \cdot 10^{8}\,\frac{\text{м}}{\text{с}}$.
}
\answer{%
    $
        n = \frac{c}{v}
        = \frac{c}{\frac \lambda T}
        = \frac{c}{\lambda \nu}
        = \frac{3 \cdot 10^{8}\,\frac{\text{м}}{\text{с}}}{0{,}650\,\text{мкм} \cdot {310\,\text{ТГц}}}
        \approx1{,}50
    $
}

\variantsplitter

\addpersonalvariant{Виктория Легонькова}

\tasknumber{1}%
\task{%
    Сформулируйте:
    \begin{itemize}
        \item что такое дифракция,
        \item определение дисперсии,
        \item условие интерференционного максимума для света, падающего нормально на дифракционную решётку,
        \item условия когерентности 2 источников света,
    \end{itemize}
}
\solutionspace{40pt}

\tasknumber{2}%
\task{%
    Произведите вычисления и округлите
    \begin{itemize}
        \item $89{,}89\,\text{м} \cdot 0{,}051\,\text{кг} = \ldots$
        \item $\cfrac{0{,}265\,\text{с}}{3{,}2\,\frac{\text{с}}{\text{м}}} = \ldots$
        \item $0{,}1152\,\text{А} \cdot 0{,}06\,\text{с} = \ldots$
        \item $3 \pi^2 \cdot \cfrac{0{,}1152\,\text{А}}{0{,}06\,\text{с}} = \ldots$
    \end{itemize}
}
\answer{%
    \begin{align*}
    89{,}89\,\text{м} \cdot 0{,}051\,\text{кг} &= 4{,}6\,\text{кг}\cdot\text{м} \\
    \cfrac{0{,}265\,\text{с}}{3{,}2\,\frac{\text{с}}{\text{м}}} &= 0{,}083\,\text{м} \\
    0{,}1152\,\text{А} \cdot 0{,}06\,\text{с} &= 0{,}007\,\text{Кл} \\
    3 \pi^2 \cdot \cfrac{0{,}1152\,\text{А}}{0{,}06\,\text{с}} &= 57\,\frac{\text{А}}{\text{с}}
    \end{align*}
}

\tasknumber{3}%
\task{%
    Установка для наблюдения интерференции состоит
    из двух когерентных источников света и экрана.
    Расстояние между источниками $l = 1{,}5\,\text{мм}$,
    а от каждого источника до экрана — $L = 4\,\text{м}$.
    Сделайте рисунок и укажите положение нулевого максимума освещенности,
    а также определите расстояние между вторым максимумом и нулевым максимумом.
    Длина волны падающего света составляет $\lambda = 400\,\text{нм}$.
}
\answer{%
    \begin{align*}
    l_1^2 &= L^2 + \sqr{x - \frac \ell 2} \\
    l_2^2 &= L^2 + \sqr{x + \frac \ell 2} \\
    l_2^2 - l_1^2 &= 2x\ell \implies (l_2 - l_1)(l_2 + l_1) = 2x\ell \implies n\lambda \cdot 2L \approx 2x_n\ell \implies x_n = \frac{\lambda L}{\ell} n, n\in \mathbb{N} \\
    x &= \frac{\lambda L}{\ell} \cdot 2 = \frac{400\,\text{нм} \cdot 4\,\text{м}}{1{,}5\,\text{мм}} \cdot 2 \approx 2{,}1\,\text{мм}
    \end{align*}
}
\solutionspace{120pt}

\tasknumber{4}%
\task{%
    Каков наибольший порядок спектра, который можно наблюдать при дифракции света
    с длиной волны $\lambda$, на дифракционной решётке с периодом $d =  3{,}9 \lambda$?
    Под каким углом наблюдается последний максимум?
}
\answer{%
    $
        d\sin \varphi_k = k\lambda
        \implies k = \frac{d\sin \varphi_k}{\lambda} \le \frac{d \cdot 1}{\lambda} =  3{,}9
        \implies k_{\max} = 3
        \implies \alpha_{3} \approx 50{,}28\degrees
    $
}
\solutionspace{80pt}

\tasknumber{5}%
\task{%
    Вертикально стоящий шест высотой 1,1 м, освещенный солнцем,
    отбрасывает на горизонтальную поверхность земли тень длиной $2\,\text{м}$.
    Известно, что длина тени от телеграфного столба на $9\,\text{м}$ больше.
    Определить высоту столба.
}
\solutionspace{80pt}

\tasknumber{6}%
\task{%
    Определить абсолютный показатель преломления прозрачной среды,
    в которой распространяется свет с длиной волны $0{,}650\,\text{мкм}$ и частотой $330\,\text{ТГц}$.
    Скорость света в вакууме $3 \cdot 10^{8}\,\frac{\text{м}}{\text{с}}$.
}
\answer{%
    $
        n = \frac{c}{v}
        = \frac{c}{\frac \lambda T}
        = \frac{c}{\lambda \nu}
        = \frac{3 \cdot 10^{8}\,\frac{\text{м}}{\text{с}}}{0{,}650\,\text{мкм} \cdot {330\,\text{ТГц}}}
        \approx1{,}40
    $
}

\variantsplitter

\addpersonalvariant{Семён Мартынов}

\tasknumber{1}%
\task{%
    Сформулируйте:
    \begin{itemize}
        \item что такое дифракция,
        \item определение дисперсии,
        \item условие интерференционного максимума для света, падающего нормально на дифракционную решётку,
        \item условия наблюдения минимума и максимума в интерферeнционной картине,
    \end{itemize}
}
\solutionspace{40pt}

\tasknumber{2}%
\task{%
    Произведите вычисления и округлите
    \begin{itemize}
        \item $336{,}92\,\text{м} \cdot 0{,}051\,\text{кг} = \ldots$
        \item $\cfrac{0{,}107\,\text{с}}{1{,}1\,\frac{\text{с}}{\text{м}}} = \ldots$
        \item $0{,}0205\,\text{А} \cdot 0{,}06\,\text{с} = \ldots$
        \item $3 \pi^2 \cdot \cfrac{0{,}0205\,\text{А}}{0{,}06\,\text{с}} = \ldots$
    \end{itemize}
}
\answer{%
    \begin{align*}
    336{,}92\,\text{м} \cdot 0{,}051\,\text{кг} &= 17{,}2\,\text{кг}\cdot\text{м} \\
    \cfrac{0{,}107\,\text{с}}{1{,}1\,\frac{\text{с}}{\text{м}}} &= 0{,}10\,\text{м} \\
    0{,}0205\,\text{А} \cdot 0{,}06\,\text{с} &= 0{,}0012\,\text{Кл} \\
    3 \pi^2 \cdot \cfrac{0{,}0205\,\text{А}}{0{,}06\,\text{с}} &= 10{,}1\,\frac{\text{А}}{\text{с}}
    \end{align*}
}

\tasknumber{3}%
\task{%
    Установка для наблюдения интерференции состоит
    из двух когерентных источников света и экрана.
    Расстояние между источниками $l = 1{,}5\,\text{мм}$,
    а от каждого источника до экрана — $L = 3\,\text{м}$.
    Сделайте рисунок и укажите положение нулевого максимума освещенности,
    а также определите расстояние между вторым минимумом и нулевым максимумом.
    Длина волны падающего света составляет $\lambda = 450\,\text{нм}$.
}
\answer{%
    \begin{align*}
    l_1^2 &= L^2 + \sqr{x - \frac \ell 2} \\
    l_2^2 &= L^2 + \sqr{x + \frac \ell 2} \\
    l_2^2 - l_1^2 &= 2x\ell \implies (l_2 - l_1)(l_2 + l_1) = 2x\ell \implies n\lambda \cdot 2L \approx 2x_n\ell \implies x_n = \frac{\lambda L}{\ell} n, n\in \mathbb{N} \\
    x &= \frac{\lambda L}{\ell} \cdot \frac32 = \frac{450\,\text{нм} \cdot 3\,\text{м}}{1{,}5\,\text{мм}} \cdot \frac32 \approx 1{,}35\,\text{мм}
    \end{align*}
}
\solutionspace{120pt}

\tasknumber{4}%
\task{%
    Каков наибольший порядок спектра, который можно наблюдать при дифракции света
    с длиной волны $\lambda$, на дифракционной решётке с периодом $d =  4{,}1 \lambda$?
    Под каким углом наблюдается последний максимум?
}
\answer{%
    $
        d\sin \varphi_k = k\lambda
        \implies k = \frac{d\sin \varphi_k}{\lambda} \le \frac{d \cdot 1}{\lambda} =  4{,}1
        \implies k_{\max} = 4
        \implies \alpha_{4} \approx 77{,}32\degrees
    $
}
\solutionspace{80pt}

\tasknumber{5}%
\task{%
    Вертикально стоящий шест высотой 1,1 м, освещенный солнцем,
    отбрасывает на горизонтальную поверхность земли тень длиной $1\,\text{м}$.
    Известно, что длина тени от телеграфного столба на $9\,\text{м}$ больше.
    Определить высоту столба.
}
\solutionspace{80pt}

\tasknumber{6}%
\task{%
    Определить абсолютный показатель преломления прозрачной среды,
    в которой распространяется свет с длиной волны $0{,}550\,\text{мкм}$ и частотой $340\,\text{ТГц}$.
    Скорость света в вакууме $3 \cdot 10^{8}\,\frac{\text{м}}{\text{с}}$.
}
\answer{%
    $
        n = \frac{c}{v}
        = \frac{c}{\frac \lambda T}
        = \frac{c}{\lambda \nu}
        = \frac{3 \cdot 10^{8}\,\frac{\text{м}}{\text{с}}}{0{,}550\,\text{мкм} \cdot {340\,\text{ТГц}}}
        \approx1{,}60
    $
}

\variantsplitter

\addpersonalvariant{Варвара Минаева}

\tasknumber{1}%
\task{%
    Сформулируйте:
    \begin{itemize}
        \item что такое дифракция,
        \item определение дисперсии,
        \item условие интерференционного максимума для света, падающего нормально на дифракционную решётку,
        \item условия наблюдения минимума и максимума в интерферeнционной картине,
    \end{itemize}
}
\solutionspace{40pt}

\tasknumber{2}%
\task{%
    Произведите вычисления и округлите
    \begin{itemize}
        \item $336{,}92\,\text{м} \cdot 0{,}074\,\text{кг} = \ldots$
        \item $\cfrac{0{,}694\,\text{с}}{1\,\frac{\text{с}}{\text{м}}} = \ldots$
        \item $0{,}0824\,\text{А} \cdot 1{,}51\,\text{с} = \ldots$
        \item $3 \pi^2 \cdot \cfrac{0{,}0824\,\text{А}}{1{,}51\,\text{с}} = \ldots$
    \end{itemize}
}
\answer{%
    \begin{align*}
    336{,}92\,\text{м} \cdot 0{,}074\,\text{кг} &= 25\,\text{кг}\cdot\text{м} \\
    \cfrac{0{,}694\,\text{с}}{1\,\frac{\text{с}}{\text{м}}} &= 0{,}7\,\text{м} \\
    0{,}0824\,\text{А} \cdot 1{,}51\,\text{с} &= 0{,}124\,\text{Кл} \\
    3 \pi^2 \cdot \cfrac{0{,}0824\,\text{А}}{1{,}51\,\text{с}} &= 1{,}62\,\frac{\text{А}}{\text{с}}
    \end{align*}
}

\tasknumber{3}%
\task{%
    Установка для наблюдения интерференции состоит
    из двух когерентных источников света и экрана.
    Расстояние между источниками $l = 1{,}2\,\text{мм}$,
    а от каждого источника до экрана — $L = 3\,\text{м}$.
    Сделайте рисунок и укажите положение нулевого максимума освещенности,
    а также определите расстояние между четвёртым максимумом и нулевым максимумом.
    Длина волны падающего света составляет $\lambda = 550\,\text{нм}$.
}
\answer{%
    \begin{align*}
    l_1^2 &= L^2 + \sqr{x - \frac \ell 2} \\
    l_2^2 &= L^2 + \sqr{x + \frac \ell 2} \\
    l_2^2 - l_1^2 &= 2x\ell \implies (l_2 - l_1)(l_2 + l_1) = 2x\ell \implies n\lambda \cdot 2L \approx 2x_n\ell \implies x_n = \frac{\lambda L}{\ell} n, n\in \mathbb{N} \\
    x &= \frac{\lambda L}{\ell} \cdot 4 = \frac{550\,\text{нм} \cdot 3\,\text{м}}{1{,}2\,\text{мм}} \cdot 4 \approx 5{,}5\,\text{мм}
    \end{align*}
}
\solutionspace{120pt}

\tasknumber{4}%
\task{%
    Каков наибольший порядок спектра, который можно наблюдать при дифракции света
    с длиной волны $\lambda$, на дифракционной решётке с периодом $d =  2{,}7 \lambda$?
    Под каким углом наблюдается последний максимум?
}
\answer{%
    $
        d\sin \varphi_k = k\lambda
        \implies k = \frac{d\sin \varphi_k}{\lambda} \le \frac{d \cdot 1}{\lambda} =  2{,}7
        \implies k_{\max} = 2
        \implies \alpha_{2} \approx 47{,}79\degrees
    $
}
\solutionspace{80pt}

\tasknumber{5}%
\task{%
    Вертикально стоящий шест высотой 1,1 м, освещенный солнцем,
    отбрасывает на горизонтальную поверхность земли тень длиной $1\,\text{м}$.
    Известно, что длина тени от телеграфного столба на $6\,\text{м}$ больше.
    Определить высоту столба.
}
\solutionspace{80pt}

\tasknumber{6}%
\task{%
    Определить абсолютный показатель преломления прозрачной среды,
    в которой распространяется свет с длиной волны $0{,}500\,\text{мкм}$ и частотой $430\,\text{ТГц}$.
    Скорость света в вакууме $3 \cdot 10^{8}\,\frac{\text{м}}{\text{с}}$.
}
\answer{%
    $
        n = \frac{c}{v}
        = \frac{c}{\frac \lambda T}
        = \frac{c}{\lambda \nu}
        = \frac{3 \cdot 10^{8}\,\frac{\text{м}}{\text{с}}}{0{,}500\,\text{мкм} \cdot {430\,\text{ТГц}}}
        \approx1{,}40
    $
}

\variantsplitter

\addpersonalvariant{Леонид Никитин}

\tasknumber{1}%
\task{%
    Сформулируйте:
    \begin{itemize}
        \item что такое интерференция,
        \item определение дисперсии,
        \item условие интерференционного максимума для света, падающего нормально на дифракционную решётку,
        \item условия когерентности 2 источников света,
    \end{itemize}
}
\solutionspace{40pt}

\tasknumber{2}%
\task{%
    Произведите вычисления и округлите
    \begin{itemize}
        \item $53{,}98\,\text{м} \cdot 0{,}051\,\text{кг} = \ldots$
        \item $\cfrac{0{,}694\,\text{с}}{1\,\frac{\text{с}}{\text{м}}} = \ldots$
        \item $0{,}0288\,\text{А} \cdot 6{,}46\,\text{с} = \ldots$
        \item $3 \pi^2 \cdot \cfrac{0{,}0288\,\text{А}}{6{,}46\,\text{с}} = \ldots$
    \end{itemize}
}
\answer{%
    \begin{align*}
    53{,}98\,\text{м} \cdot 0{,}051\,\text{кг} &= 2{,}8\,\text{кг}\cdot\text{м} \\
    \cfrac{0{,}694\,\text{с}}{1\,\frac{\text{с}}{\text{м}}} &= 0{,}7\,\text{м} \\
    0{,}0288\,\text{А} \cdot 6{,}46\,\text{с} &= 0{,}1860\,\text{Кл} \\
    3 \pi^2 \cdot \cfrac{0{,}0288\,\text{А}}{6{,}46\,\text{с}} &= 0{,}132\,\frac{\text{А}}{\text{с}}
    \end{align*}
}

\tasknumber{3}%
\task{%
    Установка для наблюдения интерференции состоит
    из двух когерентных источников света и экрана.
    Расстояние между источниками $l = 1{,}5\,\text{мм}$,
    а от каждого источника до экрана — $L = 2\,\text{м}$.
    Сделайте рисунок и укажите положение нулевого максимума освещенности,
    а также определите расстояние между третьим минимумом и нулевым максимумом.
    Длина волны падающего света составляет $\lambda = 400\,\text{нм}$.
}
\answer{%
    \begin{align*}
    l_1^2 &= L^2 + \sqr{x - \frac \ell 2} \\
    l_2^2 &= L^2 + \sqr{x + \frac \ell 2} \\
    l_2^2 - l_1^2 &= 2x\ell \implies (l_2 - l_1)(l_2 + l_1) = 2x\ell \implies n\lambda \cdot 2L \approx 2x_n\ell \implies x_n = \frac{\lambda L}{\ell} n, n\in \mathbb{N} \\
    x &= \frac{\lambda L}{\ell} \cdot \frac52 = \frac{400\,\text{нм} \cdot 2\,\text{м}}{1{,}5\,\text{мм}} \cdot \frac52 \approx 1{,}33\,\text{мм}
    \end{align*}
}
\solutionspace{120pt}

\tasknumber{4}%
\task{%
    Каков наибольший порядок спектра, который можно наблюдать при дифракции света
    с длиной волны $\lambda$, на дифракционной решётке с периодом $d =  2{,}2 \lambda$?
    Под каким углом наблюдается последний максимум?
}
\answer{%
    $
        d\sin \varphi_k = k\lambda
        \implies k = \frac{d\sin \varphi_k}{\lambda} \le \frac{d \cdot 1}{\lambda} =  2{,}2
        \implies k_{\max} = 2
        \implies \alpha_{2} \approx 65{,}38\degrees
    $
}
\solutionspace{80pt}

\tasknumber{5}%
\task{%
    Вертикально стоящий шест высотой 1,1 м, освещенный солнцем,
    отбрасывает на горизонтальную поверхность земли тень длиной $3\,\text{м}$.
    Известно, что длина тени от телеграфного столба на $7\,\text{м}$ больше.
    Определить высоту столба.
}
\solutionspace{80pt}

\tasknumber{6}%
\task{%
    Определить абсолютный показатель преломления прозрачной среды,
    в которой распространяется свет с длиной волны $0{,}650\,\text{мкм}$ и частотой $330\,\text{ТГц}$.
    Скорость света в вакууме $3 \cdot 10^{8}\,\frac{\text{м}}{\text{с}}$.
}
\answer{%
    $
        n = \frac{c}{v}
        = \frac{c}{\frac \lambda T}
        = \frac{c}{\lambda \nu}
        = \frac{3 \cdot 10^{8}\,\frac{\text{м}}{\text{с}}}{0{,}650\,\text{мкм} \cdot {330\,\text{ТГц}}}
        \approx1{,}40
    $
}

\variantsplitter

\addpersonalvariant{Тимофей Полетаев}

\tasknumber{1}%
\task{%
    Сформулируйте:
    \begin{itemize}
        \item что такое интерференция,
        \item определение дисперсии,
        \item условие интерференционного максимума для света, падающего нормально на дифракционную решётку,
        \item условия когерентности 2 источников света,
    \end{itemize}
}
\solutionspace{40pt}

\tasknumber{2}%
\task{%
    Произведите вычисления и округлите
    \begin{itemize}
        \item $885{,}36\,\text{м} \cdot 0{,}004\,\text{кг} = \ldots$
        \item $\cfrac{0{,}107\,\text{с}}{5{,}2\,\frac{\text{с}}{\text{м}}} = \ldots$
        \item $0{,}1152\,\text{А} \cdot 1{,}51\,\text{с} = \ldots$
        \item $3 \pi^2 \cdot \cfrac{0{,}1152\,\text{А}}{1{,}51\,\text{с}} = \ldots$
    \end{itemize}
}
\answer{%
    \begin{align*}
    885{,}36\,\text{м} \cdot 0{,}004\,\text{кг} &= 4\,\text{кг}\cdot\text{м} \\
    \cfrac{0{,}107\,\text{с}}{5{,}2\,\frac{\text{с}}{\text{м}}} &= 0{,}021\,\text{м} \\
    0{,}1152\,\text{А} \cdot 1{,}51\,\text{с} &= 0{,}174\,\text{Кл} \\
    3 \pi^2 \cdot \cfrac{0{,}1152\,\text{А}}{1{,}51\,\text{с}} &= 2{,}3\,\frac{\text{А}}{\text{с}}
    \end{align*}
}

\tasknumber{3}%
\task{%
    Установка для наблюдения интерференции состоит
    из двух когерентных источников света и экрана.
    Расстояние между источниками $l = 1{,}2\,\text{мм}$,
    а от каждого источника до экрана — $L = 2\,\text{м}$.
    Сделайте рисунок и укажите положение нулевого максимума освещенности,
    а также определите расстояние между четвёртым максимумом и нулевым максимумом.
    Длина волны падающего света составляет $\lambda = 400\,\text{нм}$.
}
\answer{%
    \begin{align*}
    l_1^2 &= L^2 + \sqr{x - \frac \ell 2} \\
    l_2^2 &= L^2 + \sqr{x + \frac \ell 2} \\
    l_2^2 - l_1^2 &= 2x\ell \implies (l_2 - l_1)(l_2 + l_1) = 2x\ell \implies n\lambda \cdot 2L \approx 2x_n\ell \implies x_n = \frac{\lambda L}{\ell} n, n\in \mathbb{N} \\
    x &= \frac{\lambda L}{\ell} \cdot 4 = \frac{400\,\text{нм} \cdot 2\,\text{м}}{1{,}2\,\text{мм}} \cdot 4 \approx 2{,}7\,\text{мм}
    \end{align*}
}
\solutionspace{120pt}

\tasknumber{4}%
\task{%
    Каков наибольший порядок спектра, который можно наблюдать при дифракции света
    с длиной волны $\lambda$, на дифракционной решётке с периодом $d =  4{,}6 \lambda$?
    Под каким углом наблюдается последний максимум?
}
\answer{%
    $
        d\sin \varphi_k = k\lambda
        \implies k = \frac{d\sin \varphi_k}{\lambda} \le \frac{d \cdot 1}{\lambda} =  4{,}6
        \implies k_{\max} = 4
        \implies \alpha_{4} \approx 60{,}41\degrees
    $
}
\solutionspace{80pt}

\tasknumber{5}%
\task{%
    Вертикально стоящий шест высотой 1,1 м, освещенный солнцем,
    отбрасывает на горизонтальную поверхность земли тень длиной $2\,\text{м}$.
    Известно, что длина тени от телеграфного столба на $8\,\text{м}$ больше.
    Определить высоту столба.
}
\solutionspace{80pt}

\tasknumber{6}%
\task{%
    Определить абсолютный показатель преломления прозрачной среды,
    в которой распространяется свет с длиной волны $0{,}650\,\text{мкм}$ и частотой $290\,\text{ТГц}$.
    Скорость света в вакууме $3 \cdot 10^{8}\,\frac{\text{м}}{\text{с}}$.
}
\answer{%
    $
        n = \frac{c}{v}
        = \frac{c}{\frac \lambda T}
        = \frac{c}{\lambda \nu}
        = \frac{3 \cdot 10^{8}\,\frac{\text{м}}{\text{с}}}{0{,}650\,\text{мкм} \cdot {290\,\text{ТГц}}}
        \approx1{,}60
    $
}

\variantsplitter

\addpersonalvariant{Андрей Рожков}

\tasknumber{1}%
\task{%
    Сформулируйте:
    \begin{itemize}
        \item что такое интерференция,
        \item определение дисперсии,
        \item условие интерференционного максимума для света, падающего нормально на дифракционную решётку,
        \item условия наблюдения минимума и максимума в интерферeнционной картине,
    \end{itemize}
}
\solutionspace{40pt}

\tasknumber{2}%
\task{%
    Произведите вычисления и округлите
    \begin{itemize}
        \item $78{,}97\,\text{м} \cdot 0{,}071\,\text{кг} = \ldots$
        \item $\cfrac{0{,}858\,\text{с}}{4{,}9\,\frac{\text{с}}{\text{м}}} = \ldots$
        \item $0{,}0824\,\text{А} \cdot 0{,}12\,\text{с} = \ldots$
        \item $3 \pi^2 \cdot \cfrac{0{,}0824\,\text{А}}{0{,}12\,\text{с}} = \ldots$
    \end{itemize}
}
\answer{%
    \begin{align*}
    78{,}97\,\text{м} \cdot 0{,}071\,\text{кг} &= 5{,}6\,\text{кг}\cdot\text{м} \\
    \cfrac{0{,}858\,\text{с}}{4{,}9\,\frac{\text{с}}{\text{м}}} &= 0{,}175\,\text{м} \\
    0{,}0824\,\text{А} \cdot 0{,}12\,\text{с} &= 0{,}010\,\text{Кл} \\
    3 \pi^2 \cdot \cfrac{0{,}0824\,\text{А}}{0{,}12\,\text{с}} &= 20\,\frac{\text{А}}{\text{с}}
    \end{align*}
}

\tasknumber{3}%
\task{%
    Установка для наблюдения интерференции состоит
    из двух когерентных источников света и экрана.
    Расстояние между источниками $l = 0{,}8\,\text{мм}$,
    а от каждого источника до экрана — $L = 2\,\text{м}$.
    Сделайте рисунок и укажите положение нулевого максимума освещенности,
    а также определите расстояние между третьим максимумом и нулевым максимумом.
    Длина волны падающего света составляет $\lambda = 550\,\text{нм}$.
}
\answer{%
    \begin{align*}
    l_1^2 &= L^2 + \sqr{x - \frac \ell 2} \\
    l_2^2 &= L^2 + \sqr{x + \frac \ell 2} \\
    l_2^2 - l_1^2 &= 2x\ell \implies (l_2 - l_1)(l_2 + l_1) = 2x\ell \implies n\lambda \cdot 2L \approx 2x_n\ell \implies x_n = \frac{\lambda L}{\ell} n, n\in \mathbb{N} \\
    x &= \frac{\lambda L}{\ell} \cdot 3 = \frac{550\,\text{нм} \cdot 2\,\text{м}}{0{,}8\,\text{мм}} \cdot 3 \approx 4{,}1\,\text{мм}
    \end{align*}
}
\solutionspace{120pt}

\tasknumber{4}%
\task{%
    Каков наибольший порядок спектра, который можно наблюдать при дифракции света
    с длиной волны $\lambda$, на дифракционной решётке с периодом $d =  3{,}3 \lambda$?
    Под каким углом наблюдается последний максимум?
}
\answer{%
    $
        d\sin \varphi_k = k\lambda
        \implies k = \frac{d\sin \varphi_k}{\lambda} \le \frac{d \cdot 1}{\lambda} =  3{,}3
        \implies k_{\max} = 3
        \implies \alpha_{3} \approx 65{,}38\degrees
    $
}
\solutionspace{80pt}

\tasknumber{5}%
\task{%
    Вертикально стоящий шест высотой 1,1 м, освещенный солнцем,
    отбрасывает на горизонтальную поверхность земли тень длиной $1\,\text{м}$.
    Известно, что длина тени от телеграфного столба на $8\,\text{м}$ больше.
    Определить высоту столба.
}
\solutionspace{80pt}

\tasknumber{6}%
\task{%
    Определить абсолютный показатель преломления прозрачной среды,
    в которой распространяется свет с длиной волны $0{,}550\,\text{мкм}$ и частотой $360\,\text{ТГц}$.
    Скорость света в вакууме $3 \cdot 10^{8}\,\frac{\text{м}}{\text{с}}$.
}
\answer{%
    $
        n = \frac{c}{v}
        = \frac{c}{\frac \lambda T}
        = \frac{c}{\lambda \nu}
        = \frac{3 \cdot 10^{8}\,\frac{\text{м}}{\text{с}}}{0{,}550\,\text{мкм} \cdot {360\,\text{ТГц}}}
        \approx1{,}50
    $
}

\variantsplitter

\addpersonalvariant{Рената Таржиманова}

\tasknumber{1}%
\task{%
    Сформулируйте:
    \begin{itemize}
        \item что такое дифракция,
        \item определение дисперсии,
        \item условие интерференционного максимума для света, падающего нормально на дифракционную решётку,
        \item условия когерентности 2 источников света,
    \end{itemize}
}
\solutionspace{40pt}

\tasknumber{2}%
\task{%
    Произведите вычисления и округлите
    \begin{itemize}
        \item $59{,}11\,\text{м} \cdot 0{,}074\,\text{кг} = \ldots$
        \item $\cfrac{0{,}204\,\text{с}}{4{,}6\,\frac{\text{с}}{\text{м}}} = \ldots$
        \item $0{,}0025\,\text{А} \cdot 1{,}51\,\text{с} = \ldots$
        \item $3 \pi^2 \cdot \cfrac{0{,}0025\,\text{А}}{1{,}51\,\text{с}} = \ldots$
    \end{itemize}
}
\answer{%
    \begin{align*}
    59{,}11\,\text{м} \cdot 0{,}074\,\text{кг} &= 4{,}4\,\text{кг}\cdot\text{м} \\
    \cfrac{0{,}204\,\text{с}}{4{,}6\,\frac{\text{с}}{\text{м}}} &= 0{,}044\,\text{м} \\
    0{,}0025\,\text{А} \cdot 1{,}51\,\text{с} &= 0{,}0038\,\text{Кл} \\
    3 \pi^2 \cdot \cfrac{0{,}0025\,\text{А}}{1{,}51\,\text{с}} &= 0{,}049\,\frac{\text{А}}{\text{с}}
    \end{align*}
}

\tasknumber{3}%
\task{%
    Установка для наблюдения интерференции состоит
    из двух когерентных источников света и экрана.
    Расстояние между источниками $l = 2{,}4\,\text{мм}$,
    а от каждого источника до экрана — $L = 2\,\text{м}$.
    Сделайте рисунок и укажите положение нулевого максимума освещенности,
    а также определите расстояние между вторым минимумом и нулевым максимумом.
    Длина волны падающего света составляет $\lambda = 500\,\text{нм}$.
}
\answer{%
    \begin{align*}
    l_1^2 &= L^2 + \sqr{x - \frac \ell 2} \\
    l_2^2 &= L^2 + \sqr{x + \frac \ell 2} \\
    l_2^2 - l_1^2 &= 2x\ell \implies (l_2 - l_1)(l_2 + l_1) = 2x\ell \implies n\lambda \cdot 2L \approx 2x_n\ell \implies x_n = \frac{\lambda L}{\ell} n, n\in \mathbb{N} \\
    x &= \frac{\lambda L}{\ell} \cdot \frac32 = \frac{500\,\text{нм} \cdot 2\,\text{м}}{2{,}4\,\text{мм}} \cdot \frac32 \approx 0{,}63\,\text{мм}
    \end{align*}
}
\solutionspace{120pt}

\tasknumber{4}%
\task{%
    Каков наибольший порядок спектра, который можно наблюдать при дифракции света
    с длиной волны $\lambda$, на дифракционной решётке с периодом $d =  2{,}7 \lambda$?
    Под каким углом наблюдается последний максимум?
}
\answer{%
    $
        d\sin \varphi_k = k\lambda
        \implies k = \frac{d\sin \varphi_k}{\lambda} \le \frac{d \cdot 1}{\lambda} =  2{,}7
        \implies k_{\max} = 2
        \implies \alpha_{2} \approx 47{,}79\degrees
    $
}
\solutionspace{80pt}

\tasknumber{5}%
\task{%
    Вертикально стоящий шест высотой 1,1 м, освещенный солнцем,
    отбрасывает на горизонтальную поверхность земли тень длиной $4\,\text{м}$.
    Известно, что длина тени от телеграфного столба на $6\,\text{м}$ больше.
    Определить высоту столба.
}
\solutionspace{80pt}

\tasknumber{6}%
\task{%
    Определить абсолютный показатель преломления прозрачной среды,
    в которой распространяется свет с длиной волны $0{,}450\,\text{мкм}$ и частотой $510\,\text{ТГц}$.
    Скорость света в вакууме $3 \cdot 10^{8}\,\frac{\text{м}}{\text{с}}$.
}
\answer{%
    $
        n = \frac{c}{v}
        = \frac{c}{\frac \lambda T}
        = \frac{c}{\lambda \nu}
        = \frac{3 \cdot 10^{8}\,\frac{\text{м}}{\text{с}}}{0{,}450\,\text{мкм} \cdot {510\,\text{ТГц}}}
        \approx1{,}30
    $
}

\variantsplitter

\addpersonalvariant{Андрей Щербаков}

\tasknumber{1}%
\task{%
    Сформулируйте:
    \begin{itemize}
        \item что такое дифракция,
        \item определение дисперсии,
        \item условие интерференционного максимума для света, падающего нормально на дифракционную решётку,
        \item условия когерентности 2 источников света,
    \end{itemize}
}
\solutionspace{40pt}

\tasknumber{2}%
\task{%
    Произведите вычисления и округлите
    \begin{itemize}
        \item $78{,}97\,\text{м} \cdot 0{,}053\,\text{кг} = \ldots$
        \item $\cfrac{0{,}107\,\text{с}}{4{,}8\,\frac{\text{с}}{\text{м}}} = \ldots$
        \item $0{,}1364\,\text{А} \cdot 5{,}89\,\text{с} = \ldots$
        \item $3 \pi^2 \cdot \cfrac{0{,}1364\,\text{А}}{5{,}89\,\text{с}} = \ldots$
    \end{itemize}
}
\answer{%
    \begin{align*}
    78{,}97\,\text{м} \cdot 0{,}053\,\text{кг} &= 4{,}2\,\text{кг}\cdot\text{м} \\
    \cfrac{0{,}107\,\text{с}}{4{,}8\,\frac{\text{с}}{\text{м}}} &= 0{,}022\,\text{м} \\
    0{,}1364\,\text{А} \cdot 5{,}89\,\text{с} &= 0{,}803\,\text{Кл} \\
    3 \pi^2 \cdot \cfrac{0{,}1364\,\text{А}}{5{,}89\,\text{с}} &= 0{,}69\,\frac{\text{А}}{\text{с}}
    \end{align*}
}

\tasknumber{3}%
\task{%
    Установка для наблюдения интерференции состоит
    из двух когерентных источников света и экрана.
    Расстояние между источниками $l = 1{,}2\,\text{мм}$,
    а от каждого источника до экрана — $L = 3\,\text{м}$.
    Сделайте рисунок и укажите положение нулевого максимума освещенности,
    а также определите расстояние между вторым минимумом и нулевым максимумом.
    Длина волны падающего света составляет $\lambda = 550\,\text{нм}$.
}
\answer{%
    \begin{align*}
    l_1^2 &= L^2 + \sqr{x - \frac \ell 2} \\
    l_2^2 &= L^2 + \sqr{x + \frac \ell 2} \\
    l_2^2 - l_1^2 &= 2x\ell \implies (l_2 - l_1)(l_2 + l_1) = 2x\ell \implies n\lambda \cdot 2L \approx 2x_n\ell \implies x_n = \frac{\lambda L}{\ell} n, n\in \mathbb{N} \\
    x &= \frac{\lambda L}{\ell} \cdot \frac32 = \frac{550\,\text{нм} \cdot 3\,\text{м}}{1{,}2\,\text{мм}} \cdot \frac32 \approx 2{,}1\,\text{мм}
    \end{align*}
}
\solutionspace{120pt}

\tasknumber{4}%
\task{%
    Каков наибольший порядок спектра, который можно наблюдать при дифракции света
    с длиной волны $\lambda$, на дифракционной решётке с периодом $d =  2{,}5 \lambda$?
    Под каким углом наблюдается последний максимум?
}
\answer{%
    $
        d\sin \varphi_k = k\lambda
        \implies k = \frac{d\sin \varphi_k}{\lambda} \le \frac{d \cdot 1}{\lambda} =  2{,}5
        \implies k_{\max} = 2
        \implies \alpha_{2} \approx 53{,}13\degrees
    $
}
\solutionspace{80pt}

\tasknumber{5}%
\task{%
    Вертикально стоящий шест высотой 1,1 м, освещенный солнцем,
    отбрасывает на горизонтальную поверхность земли тень длиной $4\,\text{м}$.
    Известно, что длина тени от телеграфного столба на $7\,\text{м}$ больше.
    Определить высоту столба.
}
\solutionspace{80pt}

\tasknumber{6}%
\task{%
    Определить абсолютный показатель преломления прозрачной среды,
    в которой распространяется свет с длиной волны $0{,}600\,\text{мкм}$ и частотой $360\,\text{ТГц}$.
    Скорость света в вакууме $3 \cdot 10^{8}\,\frac{\text{м}}{\text{с}}$.
}
\answer{%
    $
        n = \frac{c}{v}
        = \frac{c}{\frac \lambda T}
        = \frac{c}{\lambda \nu}
        = \frac{3 \cdot 10^{8}\,\frac{\text{м}}{\text{с}}}{0{,}600\,\text{мкм} \cdot {360\,\text{ТГц}}}
        \approx1{,}40
    $
}

\variantsplitter

\addpersonalvariant{Михаил Ярошевский}

\tasknumber{1}%
\task{%
    Сформулируйте:
    \begin{itemize}
        \item что такое интерференция,
        \item определение дисперсии,
        \item условие интерференционного максимума для света, падающего нормально на дифракционную решётку,
        \item условия когерентности 2 источников света,
    \end{itemize}
}
\solutionspace{40pt}

\tasknumber{2}%
\task{%
    Произведите вычисления и округлите
    \begin{itemize}
        \item $76{,}19\,\text{м} \cdot 0{,}082\,\text{кг} = \ldots$
        \item $\cfrac{0{,}265\,\text{с}}{5{,}2\,\frac{\text{с}}{\text{м}}} = \ldots$
        \item $0{,}1286\,\text{А} \cdot 6{,}18\,\text{с} = \ldots$
        \item $3 \pi^2 \cdot \cfrac{0{,}1286\,\text{А}}{6{,}18\,\text{с}} = \ldots$
    \end{itemize}
}
\answer{%
    \begin{align*}
    76{,}19\,\text{м} \cdot 0{,}082\,\text{кг} &= 6{,}2\,\text{кг}\cdot\text{м} \\
    \cfrac{0{,}265\,\text{с}}{5{,}2\,\frac{\text{с}}{\text{м}}} &= 0{,}051\,\text{м} \\
    0{,}1286\,\text{А} \cdot 6{,}18\,\text{с} &= 0{,}795\,\text{Кл} \\
    3 \pi^2 \cdot \cfrac{0{,}1286\,\text{А}}{6{,}18\,\text{с}} &= 0{,}62\,\frac{\text{А}}{\text{с}}
    \end{align*}
}

\tasknumber{3}%
\task{%
    Установка для наблюдения интерференции состоит
    из двух когерентных источников света и экрана.
    Расстояние между источниками $l = 1{,}2\,\text{мм}$,
    а от каждого источника до экрана — $L = 3\,\text{м}$.
    Сделайте рисунок и укажите положение нулевого максимума освещенности,
    а также определите расстояние между третьим минимумом и нулевым максимумом.
    Длина волны падающего света составляет $\lambda = 550\,\text{нм}$.
}
\answer{%
    \begin{align*}
    l_1^2 &= L^2 + \sqr{x - \frac \ell 2} \\
    l_2^2 &= L^2 + \sqr{x + \frac \ell 2} \\
    l_2^2 - l_1^2 &= 2x\ell \implies (l_2 - l_1)(l_2 + l_1) = 2x\ell \implies n\lambda \cdot 2L \approx 2x_n\ell \implies x_n = \frac{\lambda L}{\ell} n, n\in \mathbb{N} \\
    x &= \frac{\lambda L}{\ell} \cdot \frac52 = \frac{550\,\text{нм} \cdot 3\,\text{м}}{1{,}2\,\text{мм}} \cdot \frac52 \approx 3{,}4\,\text{мм}
    \end{align*}
}
\solutionspace{120pt}

\tasknumber{4}%
\task{%
    Каков наибольший порядок спектра, который можно наблюдать при дифракции света
    с длиной волны $\lambda$, на дифракционной решётке с периодом $d =  2{,}7 \lambda$?
    Под каким углом наблюдается последний максимум?
}
\answer{%
    $
        d\sin \varphi_k = k\lambda
        \implies k = \frac{d\sin \varphi_k}{\lambda} \le \frac{d \cdot 1}{\lambda} =  2{,}7
        \implies k_{\max} = 2
        \implies \alpha_{2} \approx 47{,}79\degrees
    $
}
\solutionspace{80pt}

\tasknumber{5}%
\task{%
    Вертикально стоящий шест высотой 1,1 м, освещенный солнцем,
    отбрасывает на горизонтальную поверхность земли тень длиной $2\,\text{м}$.
    Известно, что длина тени от телеграфного столба на $8\,\text{м}$ больше.
    Определить высоту столба.
}
\solutionspace{80pt}

\tasknumber{6}%
\task{%
    Определить абсолютный показатель преломления прозрачной среды,
    в которой распространяется свет с длиной волны $0{,}600\,\text{мкм}$ и частотой $360\,\text{ТГц}$.
    Скорость света в вакууме $3 \cdot 10^{8}\,\frac{\text{м}}{\text{с}}$.
}
\answer{%
    $
        n = \frac{c}{v}
        = \frac{c}{\frac \lambda T}
        = \frac{c}{\lambda \nu}
        = \frac{3 \cdot 10^{8}\,\frac{\text{м}}{\text{с}}}{0{,}600\,\text{мкм} \cdot {360\,\text{ТГц}}}
        \approx1{,}40
    $
}

\variantsplitter

\addpersonalvariant{Алексей Алимпиев}

\tasknumber{1}%
\task{%
    Сформулируйте:
    \begin{itemize}
        \item что такое интерференция,
        \item определение дисперсии,
        \item условие интерференционного максимума для света, падающего нормально на дифракционную решётку,
        \item условия наблюдения минимума и максимума в интерферeнционной картине,
    \end{itemize}
}
\solutionspace{40pt}

\tasknumber{2}%
\task{%
    Произведите вычисления и округлите
    \begin{itemize}
        \item $671{,}8\,\text{м} \cdot 0{,}046\,\text{кг} = \ldots$
        \item $\cfrac{0{,}562\,\text{с}}{1{,}1\,\frac{\text{с}}{\text{м}}} = \ldots$
        \item $0{,}1286\,\text{А} \cdot 5{,}89\,\text{с} = \ldots$
        \item $3 \pi^2 \cdot \cfrac{0{,}1286\,\text{А}}{5{,}89\,\text{с}} = \ldots$
    \end{itemize}
}
\answer{%
    \begin{align*}
    671{,}8\,\text{м} \cdot 0{,}046\,\text{кг} &= 31\,\text{кг}\cdot\text{м} \\
    \cfrac{0{,}562\,\text{с}}{1{,}1\,\frac{\text{с}}{\text{м}}} &= 0{,}5\,\text{м} \\
    0{,}1286\,\text{А} \cdot 5{,}89\,\text{с} &= 0{,}757\,\text{Кл} \\
    3 \pi^2 \cdot \cfrac{0{,}1286\,\text{А}}{5{,}89\,\text{с}} &= 0{,}65\,\frac{\text{А}}{\text{с}}
    \end{align*}
}

\tasknumber{3}%
\task{%
    Установка для наблюдения интерференции состоит
    из двух когерентных источников света и экрана.
    Расстояние между источниками $l = 0{,}8\,\text{мм}$,
    а от каждого источника до экрана — $L = 4\,\text{м}$.
    Сделайте рисунок и укажите положение нулевого максимума освещенности,
    а также определите расстояние между третьим минимумом и нулевым максимумом.
    Длина волны падающего света составляет $\lambda = 450\,\text{нм}$.
}
\answer{%
    \begin{align*}
    l_1^2 &= L^2 + \sqr{x - \frac \ell 2} \\
    l_2^2 &= L^2 + \sqr{x + \frac \ell 2} \\
    l_2^2 - l_1^2 &= 2x\ell \implies (l_2 - l_1)(l_2 + l_1) = 2x\ell \implies n\lambda \cdot 2L \approx 2x_n\ell \implies x_n = \frac{\lambda L}{\ell} n, n\in \mathbb{N} \\
    x &= \frac{\lambda L}{\ell} \cdot \frac52 = \frac{450\,\text{нм} \cdot 4\,\text{м}}{0{,}8\,\text{мм}} \cdot \frac52 \approx 5{,}6\,\text{мм}
    \end{align*}
}
\solutionspace{120pt}

\tasknumber{4}%
\task{%
    Каков наибольший порядок спектра, который можно наблюдать при дифракции света
    с длиной волны $\lambda$, на дифракционной решётке с периодом $d =  4{,}1 \lambda$?
    Под каким углом наблюдается последний максимум?
}
\answer{%
    $
        d\sin \varphi_k = k\lambda
        \implies k = \frac{d\sin \varphi_k}{\lambda} \le \frac{d \cdot 1}{\lambda} =  4{,}1
        \implies k_{\max} = 4
        \implies \alpha_{4} \approx 77{,}32\degrees
    $
}
\solutionspace{80pt}

\tasknumber{5}%
\task{%
    Вертикально стоящий шест высотой 1,1 м, освещенный солнцем,
    отбрасывает на горизонтальную поверхность земли тень длиной $4\,\text{м}$.
    Известно, что длина тени от телеграфного столба на $7\,\text{м}$ больше.
    Определить высоту столба.
}
\solutionspace{80pt}

\tasknumber{6}%
\task{%
    Определить абсолютный показатель преломления прозрачной среды,
    в которой распространяется свет с длиной волны $0{,}650\,\text{мкм}$ и частотой $270\,\text{ТГц}$.
    Скорость света в вакууме $3 \cdot 10^{8}\,\frac{\text{м}}{\text{с}}$.
}
\answer{%
    $
        n = \frac{c}{v}
        = \frac{c}{\frac \lambda T}
        = \frac{c}{\lambda \nu}
        = \frac{3 \cdot 10^{8}\,\frac{\text{м}}{\text{с}}}{0{,}650\,\text{мкм} \cdot {270\,\text{ТГц}}}
        \approx1{,}70
    $
}

\variantsplitter

\addpersonalvariant{Евгений Васин}

\tasknumber{1}%
\task{%
    Сформулируйте:
    \begin{itemize}
        \item что такое интерференция,
        \item определение дисперсии,
        \item условие интерференционного максимума для света, падающего нормально на дифракционную решётку,
        \item условия когерентности 2 источников света,
    \end{itemize}
}
\solutionspace{40pt}

\tasknumber{2}%
\task{%
    Произведите вычисления и округлите
    \begin{itemize}
        \item $885{,}36\,\text{м} \cdot 0{,}053\,\text{кг} = \ldots$
        \item $\cfrac{0{,}562\,\text{с}}{4{,}3\,\frac{\text{с}}{\text{м}}} = \ldots$
        \item $0{,}1245\,\text{А} \cdot 4{,}14\,\text{с} = \ldots$
        \item $3 \pi^2 \cdot \cfrac{0{,}1245\,\text{А}}{4{,}14\,\text{с}} = \ldots$
    \end{itemize}
}
\answer{%
    \begin{align*}
    885{,}36\,\text{м} \cdot 0{,}053\,\text{кг} &= 47\,\text{кг}\cdot\text{м} \\
    \cfrac{0{,}562\,\text{с}}{4{,}3\,\frac{\text{с}}{\text{м}}} &= 0{,}131\,\text{м} \\
    0{,}1245\,\text{А} \cdot 4{,}14\,\text{с} &= 0{,}515\,\text{Кл} \\
    3 \pi^2 \cdot \cfrac{0{,}1245\,\text{А}}{4{,}14\,\text{с}} &= 0{,}89\,\frac{\text{А}}{\text{с}}
    \end{align*}
}

\tasknumber{3}%
\task{%
    Установка для наблюдения интерференции состоит
    из двух когерентных источников света и экрана.
    Расстояние между источниками $l = 1{,}2\,\text{мм}$,
    а от каждого источника до экрана — $L = 3\,\text{м}$.
    Сделайте рисунок и укажите положение нулевого максимума освещенности,
    а также определите расстояние между четвёртым максимумом и нулевым максимумом.
    Длина волны падающего света составляет $\lambda = 500\,\text{нм}$.
}
\answer{%
    \begin{align*}
    l_1^2 &= L^2 + \sqr{x - \frac \ell 2} \\
    l_2^2 &= L^2 + \sqr{x + \frac \ell 2} \\
    l_2^2 - l_1^2 &= 2x\ell \implies (l_2 - l_1)(l_2 + l_1) = 2x\ell \implies n\lambda \cdot 2L \approx 2x_n\ell \implies x_n = \frac{\lambda L}{\ell} n, n\in \mathbb{N} \\
    x &= \frac{\lambda L}{\ell} \cdot 4 = \frac{500\,\text{нм} \cdot 3\,\text{м}}{1{,}2\,\text{мм}} \cdot 4 \approx 5\,\text{мм}
    \end{align*}
}
\solutionspace{120pt}

\tasknumber{4}%
\task{%
    Каков наибольший порядок спектра, который можно наблюдать при дифракции света
    с длиной волны $\lambda$, на дифракционной решётке с периодом $d =  3{,}5 \lambda$?
    Под каким углом наблюдается последний максимум?
}
\answer{%
    $
        d\sin \varphi_k = k\lambda
        \implies k = \frac{d\sin \varphi_k}{\lambda} \le \frac{d \cdot 1}{\lambda} =  3{,}5
        \implies k_{\max} = 3
        \implies \alpha_{3} \approx 59{,}00\degrees
    $
}
\solutionspace{80pt}

\tasknumber{5}%
\task{%
    Вертикально стоящий шест высотой 1,1 м, освещенный солнцем,
    отбрасывает на горизонтальную поверхность земли тень длиной $1\,\text{м}$.
    Известно, что длина тени от телеграфного столба на $8\,\text{м}$ больше.
    Определить высоту столба.
}
\solutionspace{80pt}

\tasknumber{6}%
\task{%
    Определить абсолютный показатель преломления прозрачной среды,
    в которой распространяется свет с длиной волны $0{,}650\,\text{мкм}$ и частотой $270\,\text{ТГц}$.
    Скорость света в вакууме $3 \cdot 10^{8}\,\frac{\text{м}}{\text{с}}$.
}
\answer{%
    $
        n = \frac{c}{v}
        = \frac{c}{\frac \lambda T}
        = \frac{c}{\lambda \nu}
        = \frac{3 \cdot 10^{8}\,\frac{\text{м}}{\text{с}}}{0{,}650\,\text{мкм} \cdot {270\,\text{ТГц}}}
        \approx1{,}70
    $
}

\variantsplitter

\addpersonalvariant{Вячеслав Волохов}

\tasknumber{1}%
\task{%
    Сформулируйте:
    \begin{itemize}
        \item что такое дифракция,
        \item определение дисперсии,
        \item условие интерференционного максимума для света, падающего нормально на дифракционную решётку,
        \item условия когерентности 2 источников света,
    \end{itemize}
}
\solutionspace{40pt}

\tasknumber{2}%
\task{%
    Произведите вычисления и округлите
    \begin{itemize}
        \item $76{,}19\,\text{м} \cdot 0{,}074\,\text{кг} = \ldots$
        \item $\cfrac{0{,}562\,\text{с}}{4{,}8\,\frac{\text{с}}{\text{м}}} = \ldots$
        \item $0{,}1245\,\text{А} \cdot 2{,}05\,\text{с} = \ldots$
        \item $3 \pi^2 \cdot \cfrac{0{,}1245\,\text{А}}{2{,}05\,\text{с}} = \ldots$
    \end{itemize}
}
\answer{%
    \begin{align*}
    76{,}19\,\text{м} \cdot 0{,}074\,\text{кг} &= 5{,}6\,\text{кг}\cdot\text{м} \\
    \cfrac{0{,}562\,\text{с}}{4{,}8\,\frac{\text{с}}{\text{м}}} &= 0{,}117\,\text{м} \\
    0{,}1245\,\text{А} \cdot 2{,}05\,\text{с} &= 0{,}255\,\text{Кл} \\
    3 \pi^2 \cdot \cfrac{0{,}1245\,\text{А}}{2{,}05\,\text{с}} &= 1{,}80\,\frac{\text{А}}{\text{с}}
    \end{align*}
}

\tasknumber{3}%
\task{%
    Установка для наблюдения интерференции состоит
    из двух когерентных источников света и экрана.
    Расстояние между источниками $l = 0{,}8\,\text{мм}$,
    а от каждого источника до экрана — $L = 2\,\text{м}$.
    Сделайте рисунок и укажите положение нулевого максимума освещенности,
    а также определите расстояние между вторым максимумом и нулевым максимумом.
    Длина волны падающего света составляет $\lambda = 500\,\text{нм}$.
}
\answer{%
    \begin{align*}
    l_1^2 &= L^2 + \sqr{x - \frac \ell 2} \\
    l_2^2 &= L^2 + \sqr{x + \frac \ell 2} \\
    l_2^2 - l_1^2 &= 2x\ell \implies (l_2 - l_1)(l_2 + l_1) = 2x\ell \implies n\lambda \cdot 2L \approx 2x_n\ell \implies x_n = \frac{\lambda L}{\ell} n, n\in \mathbb{N} \\
    x &= \frac{\lambda L}{\ell} \cdot 2 = \frac{500\,\text{нм} \cdot 2\,\text{м}}{0{,}8\,\text{мм}} \cdot 2 \approx 2{,}5\,\text{мм}
    \end{align*}
}
\solutionspace{120pt}

\tasknumber{4}%
\task{%
    Каков наибольший порядок спектра, который можно наблюдать при дифракции света
    с длиной волны $\lambda$, на дифракционной решётке с периодом $d =  2{,}5 \lambda$?
    Под каким углом наблюдается последний максимум?
}
\answer{%
    $
        d\sin \varphi_k = k\lambda
        \implies k = \frac{d\sin \varphi_k}{\lambda} \le \frac{d \cdot 1}{\lambda} =  2{,}5
        \implies k_{\max} = 2
        \implies \alpha_{2} \approx 53{,}13\degrees
    $
}
\solutionspace{80pt}

\tasknumber{5}%
\task{%
    Вертикально стоящий шест высотой 1,1 м, освещенный солнцем,
    отбрасывает на горизонтальную поверхность земли тень длиной $3\,\text{м}$.
    Известно, что длина тени от телеграфного столба на $6\,\text{м}$ больше.
    Определить высоту столба.
}
\solutionspace{80pt}

\tasknumber{6}%
\task{%
    Определить абсолютный показатель преломления прозрачной среды,
    в которой распространяется свет с длиной волны $0{,}450\,\text{мкм}$ и частотой $390\,\text{ТГц}$.
    Скорость света в вакууме $3 \cdot 10^{8}\,\frac{\text{м}}{\text{с}}$.
}
\answer{%
    $
        n = \frac{c}{v}
        = \frac{c}{\frac \lambda T}
        = \frac{c}{\lambda \nu}
        = \frac{3 \cdot 10^{8}\,\frac{\text{м}}{\text{с}}}{0{,}450\,\text{мкм} \cdot {390\,\text{ТГц}}}
        \approx1{,}70
    $
}

\variantsplitter

\addpersonalvariant{Герман Говоров}

\tasknumber{1}%
\task{%
    Сформулируйте:
    \begin{itemize}
        \item что такое интерференция,
        \item определение дисперсии,
        \item условие интерференционного максимума для света, падающего нормально на дифракционную решётку,
        \item условия наблюдения минимума и максимума в интерферeнционной картине,
    \end{itemize}
}
\solutionspace{40pt}

\tasknumber{2}%
\task{%
    Произведите вычисления и округлите
    \begin{itemize}
        \item $540\,\text{м} \cdot 0{,}074\,\text{кг} = \ldots$
        \item $\cfrac{0{,}107\,\text{с}}{1\,\frac{\text{с}}{\text{м}}} = \ldots$
        \item $0{,}0025\,\text{А} \cdot 3{,}66\,\text{с} = \ldots$
        \item $3 \pi^2 \cdot \cfrac{0{,}0025\,\text{А}}{3{,}66\,\text{с}} = \ldots$
    \end{itemize}
}
\answer{%
    \begin{align*}
    540\,\text{м} \cdot 0{,}074\,\text{кг} &= 40\,\text{кг}\cdot\text{м} \\
    \cfrac{0{,}107\,\text{с}}{1\,\frac{\text{с}}{\text{м}}} &= 0{,}11\,\text{м} \\
    0{,}0025\,\text{А} \cdot 3{,}66\,\text{с} &= 0{,}0092\,\text{Кл} \\
    3 \pi^2 \cdot \cfrac{0{,}0025\,\text{А}}{3{,}66\,\text{с}} &= 0{,}020\,\frac{\text{А}}{\text{с}}
    \end{align*}
}

\tasknumber{3}%
\task{%
    Установка для наблюдения интерференции состоит
    из двух когерентных источников света и экрана.
    Расстояние между источниками $l = 2{,}4\,\text{мм}$,
    а от каждого источника до экрана — $L = 2\,\text{м}$.
    Сделайте рисунок и укажите положение нулевого максимума освещенности,
    а также определите расстояние между третьим максимумом и нулевым максимумом.
    Длина волны падающего света составляет $\lambda = 600\,\text{нм}$.
}
\answer{%
    \begin{align*}
    l_1^2 &= L^2 + \sqr{x - \frac \ell 2} \\
    l_2^2 &= L^2 + \sqr{x + \frac \ell 2} \\
    l_2^2 - l_1^2 &= 2x\ell \implies (l_2 - l_1)(l_2 + l_1) = 2x\ell \implies n\lambda \cdot 2L \approx 2x_n\ell \implies x_n = \frac{\lambda L}{\ell} n, n\in \mathbb{N} \\
    x &= \frac{\lambda L}{\ell} \cdot 3 = \frac{600\,\text{нм} \cdot 2\,\text{м}}{2{,}4\,\text{мм}} \cdot 3 \approx 1{,}50\,\text{мм}
    \end{align*}
}
\solutionspace{120pt}

\tasknumber{4}%
\task{%
    Каков наибольший порядок спектра, который можно наблюдать при дифракции света
    с длиной волны $\lambda$, на дифракционной решётке с периодом $d =  4{,}5 \lambda$?
    Под каким углом наблюдается последний максимум?
}
\answer{%
    $
        d\sin \varphi_k = k\lambda
        \implies k = \frac{d\sin \varphi_k}{\lambda} \le \frac{d \cdot 1}{\lambda} =  4{,}5
        \implies k_{\max} = 4
        \implies \alpha_{4} \approx 62{,}73\degrees
    $
}
\solutionspace{80pt}

\tasknumber{5}%
\task{%
    Вертикально стоящий шест высотой 1,1 м, освещенный солнцем,
    отбрасывает на горизонтальную поверхность земли тень длиной $3\,\text{м}$.
    Известно, что длина тени от телеграфного столба на $7\,\text{м}$ больше.
    Определить высоту столба.
}
\solutionspace{80pt}

\tasknumber{6}%
\task{%
    Определить абсолютный показатель преломления прозрачной среды,
    в которой распространяется свет с длиной волны $0{,}550\,\text{мкм}$ и частотой $390\,\text{ТГц}$.
    Скорость света в вакууме $3 \cdot 10^{8}\,\frac{\text{м}}{\text{с}}$.
}
\answer{%
    $
        n = \frac{c}{v}
        = \frac{c}{\frac \lambda T}
        = \frac{c}{\lambda \nu}
        = \frac{3 \cdot 10^{8}\,\frac{\text{м}}{\text{с}}}{0{,}550\,\text{мкм} \cdot {390\,\text{ТГц}}}
        \approx1{,}40
    $
}

\variantsplitter

\addpersonalvariant{София Журавлёва}

\tasknumber{1}%
\task{%
    Сформулируйте:
    \begin{itemize}
        \item что такое дифракция,
        \item определение дисперсии,
        \item условие интерференционного максимума для света, падающего нормально на дифракционную решётку,
        \item условия наблюдения минимума и максимума в интерферeнционной картине,
    \end{itemize}
}
\solutionspace{40pt}

\tasknumber{2}%
\task{%
    Произведите вычисления и округлите
    \begin{itemize}
        \item $540\,\text{м} \cdot 0{,}046\,\text{кг} = \ldots$
        \item $\cfrac{0{,}360\,\text{с}}{4{,}6\,\frac{\text{с}}{\text{м}}} = \ldots$
        \item $0{,}0288\,\text{А} \cdot 2{,}05\,\text{с} = \ldots$
        \item $3 \pi^2 \cdot \cfrac{0{,}0288\,\text{А}}{2{,}05\,\text{с}} = \ldots$
    \end{itemize}
}
\answer{%
    \begin{align*}
    540\,\text{м} \cdot 0{,}046\,\text{кг} &= 25\,\text{кг}\cdot\text{м} \\
    \cfrac{0{,}360\,\text{с}}{4{,}6\,\frac{\text{с}}{\text{м}}} &= 0{,}078\,\text{м} \\
    0{,}0288\,\text{А} \cdot 2{,}05\,\text{с} &= 0{,}0590\,\text{Кл} \\
    3 \pi^2 \cdot \cfrac{0{,}0288\,\text{А}}{2{,}05\,\text{с}} &= 0{,}42\,\frac{\text{А}}{\text{с}}
    \end{align*}
}

\tasknumber{3}%
\task{%
    Установка для наблюдения интерференции состоит
    из двух когерентных источников света и экрана.
    Расстояние между источниками $l = 2{,}4\,\text{мм}$,
    а от каждого источника до экрана — $L = 2\,\text{м}$.
    Сделайте рисунок и укажите положение нулевого максимума освещенности,
    а также определите расстояние между четвёртым минимумом и нулевым максимумом.
    Длина волны падающего света составляет $\lambda = 600\,\text{нм}$.
}
\answer{%
    \begin{align*}
    l_1^2 &= L^2 + \sqr{x - \frac \ell 2} \\
    l_2^2 &= L^2 + \sqr{x + \frac \ell 2} \\
    l_2^2 - l_1^2 &= 2x\ell \implies (l_2 - l_1)(l_2 + l_1) = 2x\ell \implies n\lambda \cdot 2L \approx 2x_n\ell \implies x_n = \frac{\lambda L}{\ell} n, n\in \mathbb{N} \\
    x &= \frac{\lambda L}{\ell} \cdot \frac72 = \frac{600\,\text{нм} \cdot 2\,\text{м}}{2{,}4\,\text{мм}} \cdot \frac72 \approx 1{,}75\,\text{мм}
    \end{align*}
}
\solutionspace{120pt}

\tasknumber{4}%
\task{%
    Каков наибольший порядок спектра, который можно наблюдать при дифракции света
    с длиной волны $\lambda$, на дифракционной решётке с периодом $d =  3{,}5 \lambda$?
    Под каким углом наблюдается последний максимум?
}
\answer{%
    $
        d\sin \varphi_k = k\lambda
        \implies k = \frac{d\sin \varphi_k}{\lambda} \le \frac{d \cdot 1}{\lambda} =  3{,}5
        \implies k_{\max} = 3
        \implies \alpha_{3} \approx 59{,}00\degrees
    $
}
\solutionspace{80pt}

\tasknumber{5}%
\task{%
    Вертикально стоящий шест высотой 1,1 м, освещенный солнцем,
    отбрасывает на горизонтальную поверхность земли тень длиной $3\,\text{м}$.
    Известно, что длина тени от телеграфного столба на $8\,\text{м}$ больше.
    Определить высоту столба.
}
\solutionspace{80pt}

\tasknumber{6}%
\task{%
    Определить абсолютный показатель преломления прозрачной среды,
    в которой распространяется свет с длиной волны $0{,}650\,\text{мкм}$ и частотой $360\,\text{ТГц}$.
    Скорость света в вакууме $3 \cdot 10^{8}\,\frac{\text{м}}{\text{с}}$.
}
\answer{%
    $
        n = \frac{c}{v}
        = \frac{c}{\frac \lambda T}
        = \frac{c}{\lambda \nu}
        = \frac{3 \cdot 10^{8}\,\frac{\text{м}}{\text{с}}}{0{,}650\,\text{мкм} \cdot {360\,\text{ТГц}}}
        \approx1{,}30
    $
}

\variantsplitter

\addpersonalvariant{Константин Козлов}

\tasknumber{1}%
\task{%
    Сформулируйте:
    \begin{itemize}
        \item что такое интерференция,
        \item определение дисперсии,
        \item условие интерференционного максимума для света, падающего нормально на дифракционную решётку,
        \item условия когерентности 2 источников света,
    \end{itemize}
}
\solutionspace{40pt}

\tasknumber{2}%
\task{%
    Произведите вычисления и округлите
    \begin{itemize}
        \item $671{,}8\,\text{м} \cdot 0{,}082\,\text{кг} = \ldots$
        \item $\cfrac{0{,}204\,\text{с}}{3{,}2\,\frac{\text{с}}{\text{м}}} = \ldots$
        \item $0{,}0252\,\text{А} \cdot 0{,}12\,\text{с} = \ldots$
        \item $3 \pi^2 \cdot \cfrac{0{,}0252\,\text{А}}{0{,}12\,\text{с}} = \ldots$
    \end{itemize}
}
\answer{%
    \begin{align*}
    671{,}8\,\text{м} \cdot 0{,}082\,\text{кг} &= 55\,\text{кг}\cdot\text{м} \\
    \cfrac{0{,}204\,\text{с}}{3{,}2\,\frac{\text{с}}{\text{м}}} &= 0{,}064\,\text{м} \\
    0{,}0252\,\text{А} \cdot 0{,}12\,\text{с} &= 0{,}003\,\text{Кл} \\
    3 \pi^2 \cdot \cfrac{0{,}0252\,\text{А}}{0{,}12\,\text{с}} &= 6{,}2\,\frac{\text{А}}{\text{с}}
    \end{align*}
}

\tasknumber{3}%
\task{%
    Установка для наблюдения интерференции состоит
    из двух когерентных источников света и экрана.
    Расстояние между источниками $l = 1{,}5\,\text{мм}$,
    а от каждого источника до экрана — $L = 2\,\text{м}$.
    Сделайте рисунок и укажите положение нулевого максимума освещенности,
    а также определите расстояние между вторым минимумом и нулевым максимумом.
    Длина волны падающего света составляет $\lambda = 500\,\text{нм}$.
}
\answer{%
    \begin{align*}
    l_1^2 &= L^2 + \sqr{x - \frac \ell 2} \\
    l_2^2 &= L^2 + \sqr{x + \frac \ell 2} \\
    l_2^2 - l_1^2 &= 2x\ell \implies (l_2 - l_1)(l_2 + l_1) = 2x\ell \implies n\lambda \cdot 2L \approx 2x_n\ell \implies x_n = \frac{\lambda L}{\ell} n, n\in \mathbb{N} \\
    x &= \frac{\lambda L}{\ell} \cdot \frac32 = \frac{500\,\text{нм} \cdot 2\,\text{м}}{1{,}5\,\text{мм}} \cdot \frac32 \approx 1\,\text{мм}
    \end{align*}
}
\solutionspace{120pt}

\tasknumber{4}%
\task{%
    Каков наибольший порядок спектра, который можно наблюдать при дифракции света
    с длиной волны $\lambda$, на дифракционной решётке с периодом $d =  3{,}5 \lambda$?
    Под каким углом наблюдается последний максимум?
}
\answer{%
    $
        d\sin \varphi_k = k\lambda
        \implies k = \frac{d\sin \varphi_k}{\lambda} \le \frac{d \cdot 1}{\lambda} =  3{,}5
        \implies k_{\max} = 3
        \implies \alpha_{3} \approx 59{,}00\degrees
    $
}
\solutionspace{80pt}

\tasknumber{5}%
\task{%
    Вертикально стоящий шест высотой 1,1 м, освещенный солнцем,
    отбрасывает на горизонтальную поверхность земли тень длиной $3\,\text{м}$.
    Известно, что длина тени от телеграфного столба на $9\,\text{м}$ больше.
    Определить высоту столба.
}
\solutionspace{80pt}

\tasknumber{6}%
\task{%
    Определить абсолютный показатель преломления прозрачной среды,
    в которой распространяется свет с длиной волны $0{,}450\,\text{мкм}$ и частотой $390\,\text{ТГц}$.
    Скорость света в вакууме $3 \cdot 10^{8}\,\frac{\text{м}}{\text{с}}$.
}
\answer{%
    $
        n = \frac{c}{v}
        = \frac{c}{\frac \lambda T}
        = \frac{c}{\lambda \nu}
        = \frac{3 \cdot 10^{8}\,\frac{\text{м}}{\text{с}}}{0{,}450\,\text{мкм} \cdot {390\,\text{ТГц}}}
        \approx1{,}70
    $
}

\variantsplitter

\addpersonalvariant{Наталья Кравченко}

\tasknumber{1}%
\task{%
    Сформулируйте:
    \begin{itemize}
        \item что такое интерференция,
        \item определение дисперсии,
        \item условие интерференционного максимума для света, падающего нормально на дифракционную решётку,
        \item условия наблюдения минимума и максимума в интерферeнционной картине,
    \end{itemize}
}
\solutionspace{40pt}

\tasknumber{2}%
\task{%
    Произведите вычисления и округлите
    \begin{itemize}
        \item $76{,}19\,\text{м} \cdot 0{,}046\,\text{кг} = \ldots$
        \item $\cfrac{0{,}562\,\text{с}}{3{,}2\,\frac{\text{с}}{\text{м}}} = \ldots$
        \item $0{,}0824\,\text{А} \cdot 6{,}46\,\text{с} = \ldots$
        \item $3 \pi^2 \cdot \cfrac{0{,}0824\,\text{А}}{6{,}46\,\text{с}} = \ldots$
    \end{itemize}
}
\answer{%
    \begin{align*}
    76{,}19\,\text{м} \cdot 0{,}046\,\text{кг} &= 3{,}5\,\text{кг}\cdot\text{м} \\
    \cfrac{0{,}562\,\text{с}}{3{,}2\,\frac{\text{с}}{\text{м}}} &= 0{,}176\,\text{м} \\
    0{,}0824\,\text{А} \cdot 6{,}46\,\text{с} &= 0{,}532\,\text{Кл} \\
    3 \pi^2 \cdot \cfrac{0{,}0824\,\text{А}}{6{,}46\,\text{с}} &= 0{,}38\,\frac{\text{А}}{\text{с}}
    \end{align*}
}

\tasknumber{3}%
\task{%
    Установка для наблюдения интерференции состоит
    из двух когерентных источников света и экрана.
    Расстояние между источниками $l = 1{,}2\,\text{мм}$,
    а от каждого источника до экрана — $L = 2\,\text{м}$.
    Сделайте рисунок и укажите положение нулевого максимума освещенности,
    а также определите расстояние между четвёртым максимумом и нулевым максимумом.
    Длина волны падающего света составляет $\lambda = 600\,\text{нм}$.
}
\answer{%
    \begin{align*}
    l_1^2 &= L^2 + \sqr{x - \frac \ell 2} \\
    l_2^2 &= L^2 + \sqr{x + \frac \ell 2} \\
    l_2^2 - l_1^2 &= 2x\ell \implies (l_2 - l_1)(l_2 + l_1) = 2x\ell \implies n\lambda \cdot 2L \approx 2x_n\ell \implies x_n = \frac{\lambda L}{\ell} n, n\in \mathbb{N} \\
    x &= \frac{\lambda L}{\ell} \cdot 4 = \frac{600\,\text{нм} \cdot 2\,\text{м}}{1{,}2\,\text{мм}} \cdot 4 \approx 4\,\text{мм}
    \end{align*}
}
\solutionspace{120pt}

\tasknumber{4}%
\task{%
    Каков наибольший порядок спектра, который можно наблюдать при дифракции света
    с длиной волны $\lambda$, на дифракционной решётке с периодом $d =  3{,}9 \lambda$?
    Под каким углом наблюдается последний максимум?
}
\answer{%
    $
        d\sin \varphi_k = k\lambda
        \implies k = \frac{d\sin \varphi_k}{\lambda} \le \frac{d \cdot 1}{\lambda} =  3{,}9
        \implies k_{\max} = 3
        \implies \alpha_{3} \approx 50{,}28\degrees
    $
}
\solutionspace{80pt}

\tasknumber{5}%
\task{%
    Вертикально стоящий шест высотой 1,1 м, освещенный солнцем,
    отбрасывает на горизонтальную поверхность земли тень длиной $3\,\text{м}$.
    Известно, что длина тени от телеграфного столба на $6\,\text{м}$ больше.
    Определить высоту столба.
}
\solutionspace{80pt}

\tasknumber{6}%
\task{%
    Определить абсолютный показатель преломления прозрачной среды,
    в которой распространяется свет с длиной волны $0{,}550\,\text{мкм}$ и частотой $340\,\text{ТГц}$.
    Скорость света в вакууме $3 \cdot 10^{8}\,\frac{\text{м}}{\text{с}}$.
}
\answer{%
    $
        n = \frac{c}{v}
        = \frac{c}{\frac \lambda T}
        = \frac{c}{\lambda \nu}
        = \frac{3 \cdot 10^{8}\,\frac{\text{м}}{\text{с}}}{0{,}550\,\text{мкм} \cdot {340\,\text{ТГц}}}
        \approx1{,}60
    $
}

\variantsplitter

\addpersonalvariant{Матвей Кузьмин}

\tasknumber{1}%
\task{%
    Сформулируйте:
    \begin{itemize}
        \item что такое дифракция,
        \item определение дисперсии,
        \item условие интерференционного максимума для света, падающего нормально на дифракционную решётку,
        \item условия наблюдения минимума и максимума в интерферeнционной картине,
    \end{itemize}
}
\solutionspace{40pt}

\tasknumber{2}%
\task{%
    Произведите вычисления и округлите
    \begin{itemize}
        \item $89{,}89\,\text{м} \cdot 0{,}071\,\text{кг} = \ldots$
        \item $\cfrac{0{,}204\,\text{с}}{4{,}6\,\frac{\text{с}}{\text{м}}} = \ldots$
        \item $0{,}0824\,\text{А} \cdot 1{,}51\,\text{с} = \ldots$
        \item $3 \pi^2 \cdot \cfrac{0{,}0824\,\text{А}}{1{,}51\,\text{с}} = \ldots$
    \end{itemize}
}
\answer{%
    \begin{align*}
    89{,}89\,\text{м} \cdot 0{,}071\,\text{кг} &= 6{,}4\,\text{кг}\cdot\text{м} \\
    \cfrac{0{,}204\,\text{с}}{4{,}6\,\frac{\text{с}}{\text{м}}} &= 0{,}044\,\text{м} \\
    0{,}0824\,\text{А} \cdot 1{,}51\,\text{с} &= 0{,}124\,\text{Кл} \\
    3 \pi^2 \cdot \cfrac{0{,}0824\,\text{А}}{1{,}51\,\text{с}} &= 1{,}62\,\frac{\text{А}}{\text{с}}
    \end{align*}
}

\tasknumber{3}%
\task{%
    Установка для наблюдения интерференции состоит
    из двух когерентных источников света и экрана.
    Расстояние между источниками $l = 0{,}8\,\text{мм}$,
    а от каждого источника до экрана — $L = 3\,\text{м}$.
    Сделайте рисунок и укажите положение нулевого максимума освещенности,
    а также определите расстояние между вторым минимумом и нулевым максимумом.
    Длина волны падающего света составляет $\lambda = 400\,\text{нм}$.
}
\answer{%
    \begin{align*}
    l_1^2 &= L^2 + \sqr{x - \frac \ell 2} \\
    l_2^2 &= L^2 + \sqr{x + \frac \ell 2} \\
    l_2^2 - l_1^2 &= 2x\ell \implies (l_2 - l_1)(l_2 + l_1) = 2x\ell \implies n\lambda \cdot 2L \approx 2x_n\ell \implies x_n = \frac{\lambda L}{\ell} n, n\in \mathbb{N} \\
    x &= \frac{\lambda L}{\ell} \cdot \frac32 = \frac{400\,\text{нм} \cdot 3\,\text{м}}{0{,}8\,\text{мм}} \cdot \frac32 \approx 2{,}3\,\text{мм}
    \end{align*}
}
\solutionspace{120pt}

\tasknumber{4}%
\task{%
    Каков наибольший порядок спектра, который можно наблюдать при дифракции света
    с длиной волны $\lambda$, на дифракционной решётке с периодом $d =  3{,}3 \lambda$?
    Под каким углом наблюдается последний максимум?
}
\answer{%
    $
        d\sin \varphi_k = k\lambda
        \implies k = \frac{d\sin \varphi_k}{\lambda} \le \frac{d \cdot 1}{\lambda} =  3{,}3
        \implies k_{\max} = 3
        \implies \alpha_{3} \approx 65{,}38\degrees
    $
}
\solutionspace{80pt}

\tasknumber{5}%
\task{%
    Вертикально стоящий шест высотой 1,1 м, освещенный солнцем,
    отбрасывает на горизонтальную поверхность земли тень длиной $2\,\text{м}$.
    Известно, что длина тени от телеграфного столба на $6\,\text{м}$ больше.
    Определить высоту столба.
}
\solutionspace{80pt}

\tasknumber{6}%
\task{%
    Определить абсолютный показатель преломления прозрачной среды,
    в которой распространяется свет с длиной волны $0{,}650\,\text{мкм}$ и частотой $360\,\text{ТГц}$.
    Скорость света в вакууме $3 \cdot 10^{8}\,\frac{\text{м}}{\text{с}}$.
}
\answer{%
    $
        n = \frac{c}{v}
        = \frac{c}{\frac \lambda T}
        = \frac{c}{\lambda \nu}
        = \frac{3 \cdot 10^{8}\,\frac{\text{м}}{\text{с}}}{0{,}650\,\text{мкм} \cdot {360\,\text{ТГц}}}
        \approx1{,}30
    $
}

\variantsplitter

\addpersonalvariant{Сергей Малышев}

\tasknumber{1}%
\task{%
    Сформулируйте:
    \begin{itemize}
        \item что такое интерференция,
        \item определение дисперсии,
        \item условие интерференционного максимума для света, падающего нормально на дифракционную решётку,
        \item условия когерентности 2 источников света,
    \end{itemize}
}
\solutionspace{40pt}

\tasknumber{2}%
\task{%
    Произведите вычисления и округлите
    \begin{itemize}
        \item $336{,}92\,\text{м} \cdot 0{,}082\,\text{кг} = \ldots$
        \item $\cfrac{0{,}204\,\text{с}}{4{,}8\,\frac{\text{с}}{\text{м}}} = \ldots$
        \item $0{,}1152\,\text{А} \cdot 1{,}51\,\text{с} = \ldots$
        \item $3 \pi^2 \cdot \cfrac{0{,}1152\,\text{А}}{1{,}51\,\text{с}} = \ldots$
    \end{itemize}
}
\answer{%
    \begin{align*}
    336{,}92\,\text{м} \cdot 0{,}082\,\text{кг} &= 28\,\text{кг}\cdot\text{м} \\
    \cfrac{0{,}204\,\text{с}}{4{,}8\,\frac{\text{с}}{\text{м}}} &= 0{,}043\,\text{м} \\
    0{,}1152\,\text{А} \cdot 1{,}51\,\text{с} &= 0{,}174\,\text{Кл} \\
    3 \pi^2 \cdot \cfrac{0{,}1152\,\text{А}}{1{,}51\,\text{с}} &= 2{,}3\,\frac{\text{А}}{\text{с}}
    \end{align*}
}

\tasknumber{3}%
\task{%
    Установка для наблюдения интерференции состоит
    из двух когерентных источников света и экрана.
    Расстояние между источниками $l = 1{,}5\,\text{мм}$,
    а от каждого источника до экрана — $L = 2\,\text{м}$.
    Сделайте рисунок и укажите положение нулевого максимума освещенности,
    а также определите расстояние между четвёртым минимумом и нулевым максимумом.
    Длина волны падающего света составляет $\lambda = 600\,\text{нм}$.
}
\answer{%
    \begin{align*}
    l_1^2 &= L^2 + \sqr{x - \frac \ell 2} \\
    l_2^2 &= L^2 + \sqr{x + \frac \ell 2} \\
    l_2^2 - l_1^2 &= 2x\ell \implies (l_2 - l_1)(l_2 + l_1) = 2x\ell \implies n\lambda \cdot 2L \approx 2x_n\ell \implies x_n = \frac{\lambda L}{\ell} n, n\in \mathbb{N} \\
    x &= \frac{\lambda L}{\ell} \cdot \frac72 = \frac{600\,\text{нм} \cdot 2\,\text{м}}{1{,}5\,\text{мм}} \cdot \frac72 \approx 2{,}8\,\text{мм}
    \end{align*}
}
\solutionspace{120pt}

\tasknumber{4}%
\task{%
    Каков наибольший порядок спектра, который можно наблюдать при дифракции света
    с длиной волны $\lambda$, на дифракционной решётке с периодом $d =  4{,}1 \lambda$?
    Под каким углом наблюдается последний максимум?
}
\answer{%
    $
        d\sin \varphi_k = k\lambda
        \implies k = \frac{d\sin \varphi_k}{\lambda} \le \frac{d \cdot 1}{\lambda} =  4{,}1
        \implies k_{\max} = 4
        \implies \alpha_{4} \approx 77{,}32\degrees
    $
}
\solutionspace{80pt}

\tasknumber{5}%
\task{%
    Вертикально стоящий шест высотой 1,1 м, освещенный солнцем,
    отбрасывает на горизонтальную поверхность земли тень длиной $4\,\text{м}$.
    Известно, что длина тени от телеграфного столба на $8\,\text{м}$ больше.
    Определить высоту столба.
}
\solutionspace{80pt}

\tasknumber{6}%
\task{%
    Определить абсолютный показатель преломления прозрачной среды,
    в которой распространяется свет с длиной волны $0{,}600\,\text{мкм}$ и частотой $330\,\text{ТГц}$.
    Скорость света в вакууме $3 \cdot 10^{8}\,\frac{\text{м}}{\text{с}}$.
}
\answer{%
    $
        n = \frac{c}{v}
        = \frac{c}{\frac \lambda T}
        = \frac{c}{\lambda \nu}
        = \frac{3 \cdot 10^{8}\,\frac{\text{м}}{\text{с}}}{0{,}600\,\text{мкм} \cdot {330\,\text{ТГц}}}
        \approx1{,}50
    $
}

\variantsplitter

\addpersonalvariant{Алина Полканова}

\tasknumber{1}%
\task{%
    Сформулируйте:
    \begin{itemize}
        \item что такое интерференция,
        \item определение дисперсии,
        \item условие интерференционного максимума для света, падающего нормально на дифракционную решётку,
        \item условия когерентности 2 источников света,
    \end{itemize}
}
\solutionspace{40pt}

\tasknumber{2}%
\task{%
    Произведите вычисления и округлите
    \begin{itemize}
        \item $55{,}58\,\text{м} \cdot 0{,}063\,\text{кг} = \ldots$
        \item $\cfrac{0{,}107\,\text{с}}{4{,}3\,\frac{\text{с}}{\text{м}}} = \ldots$
        \item $0{,}1245\,\text{А} \cdot 1{,}43\,\text{с} = \ldots$
        \item $3 \pi^2 \cdot \cfrac{0{,}1245\,\text{А}}{1{,}43\,\text{с}} = \ldots$
    \end{itemize}
}
\answer{%
    \begin{align*}
    55{,}58\,\text{м} \cdot 0{,}063\,\text{кг} &= 3{,}5\,\text{кг}\cdot\text{м} \\
    \cfrac{0{,}107\,\text{с}}{4{,}3\,\frac{\text{с}}{\text{м}}} &= 0{,}025\,\text{м} \\
    0{,}1245\,\text{А} \cdot 1{,}43\,\text{с} &= 0{,}178\,\text{Кл} \\
    3 \pi^2 \cdot \cfrac{0{,}1245\,\text{А}}{1{,}43\,\text{с}} &= 2{,}6\,\frac{\text{А}}{\text{с}}
    \end{align*}
}

\tasknumber{3}%
\task{%
    Установка для наблюдения интерференции состоит
    из двух когерентных источников света и экрана.
    Расстояние между источниками $l = 2{,}4\,\text{мм}$,
    а от каждого источника до экрана — $L = 3\,\text{м}$.
    Сделайте рисунок и укажите положение нулевого максимума освещенности,
    а также определите расстояние между четвёртым максимумом и нулевым максимумом.
    Длина волны падающего света составляет $\lambda = 600\,\text{нм}$.
}
\answer{%
    \begin{align*}
    l_1^2 &= L^2 + \sqr{x - \frac \ell 2} \\
    l_2^2 &= L^2 + \sqr{x + \frac \ell 2} \\
    l_2^2 - l_1^2 &= 2x\ell \implies (l_2 - l_1)(l_2 + l_1) = 2x\ell \implies n\lambda \cdot 2L \approx 2x_n\ell \implies x_n = \frac{\lambda L}{\ell} n, n\in \mathbb{N} \\
    x &= \frac{\lambda L}{\ell} \cdot 4 = \frac{600\,\text{нм} \cdot 3\,\text{м}}{2{,}4\,\text{мм}} \cdot 4 \approx 3\,\text{мм}
    \end{align*}
}
\solutionspace{120pt}

\tasknumber{4}%
\task{%
    Каков наибольший порядок спектра, который можно наблюдать при дифракции света
    с длиной волны $\lambda$, на дифракционной решётке с периодом $d =  3{,}9 \lambda$?
    Под каким углом наблюдается последний максимум?
}
\answer{%
    $
        d\sin \varphi_k = k\lambda
        \implies k = \frac{d\sin \varphi_k}{\lambda} \le \frac{d \cdot 1}{\lambda} =  3{,}9
        \implies k_{\max} = 3
        \implies \alpha_{3} \approx 50{,}28\degrees
    $
}
\solutionspace{80pt}

\tasknumber{5}%
\task{%
    Вертикально стоящий шест высотой 1,1 м, освещенный солнцем,
    отбрасывает на горизонтальную поверхность земли тень длиной $2\,\text{м}$.
    Известно, что длина тени от телеграфного столба на $6\,\text{м}$ больше.
    Определить высоту столба.
}
\solutionspace{80pt}

\tasknumber{6}%
\task{%
    Определить абсолютный показатель преломления прозрачной среды,
    в которой распространяется свет с длиной волны $0{,}500\,\text{мкм}$ и частотой $400\,\text{ТГц}$.
    Скорость света в вакууме $3 \cdot 10^{8}\,\frac{\text{м}}{\text{с}}$.
}
\answer{%
    $
        n = \frac{c}{v}
        = \frac{c}{\frac \lambda T}
        = \frac{c}{\lambda \nu}
        = \frac{3 \cdot 10^{8}\,\frac{\text{м}}{\text{с}}}{0{,}500\,\text{мкм} \cdot {400\,\text{ТГц}}}
        \approx1{,}50
    $
}

\variantsplitter

\addpersonalvariant{Сергей Пономарёв}

\tasknumber{1}%
\task{%
    Сформулируйте:
    \begin{itemize}
        \item что такое интерференция,
        \item определение дисперсии,
        \item условие интерференционного максимума для света, падающего нормально на дифракционную решётку,
        \item условия наблюдения минимума и максимума в интерферeнционной картине,
    \end{itemize}
}
\solutionspace{40pt}

\tasknumber{2}%
\task{%
    Произведите вычисления и округлите
    \begin{itemize}
        \item $59{,}11\,\text{м} \cdot 0{,}097\,\text{кг} = \ldots$
        \item $\cfrac{0{,}360\,\text{с}}{4{,}6\,\frac{\text{с}}{\text{м}}} = \ldots$
        \item $0{,}1286\,\text{А} \cdot 1{,}43\,\text{с} = \ldots$
        \item $3 \pi^2 \cdot \cfrac{0{,}1286\,\text{А}}{1{,}43\,\text{с}} = \ldots$
    \end{itemize}
}
\answer{%
    \begin{align*}
    59{,}11\,\text{м} \cdot 0{,}097\,\text{кг} &= 5{,}7\,\text{кг}\cdot\text{м} \\
    \cfrac{0{,}360\,\text{с}}{4{,}6\,\frac{\text{с}}{\text{м}}} &= 0{,}078\,\text{м} \\
    0{,}1286\,\text{А} \cdot 1{,}43\,\text{с} &= 0{,}184\,\text{Кл} \\
    3 \pi^2 \cdot \cfrac{0{,}1286\,\text{А}}{1{,}43\,\text{с}} &= 2{,}7\,\frac{\text{А}}{\text{с}}
    \end{align*}
}

\tasknumber{3}%
\task{%
    Установка для наблюдения интерференции состоит
    из двух когерентных источников света и экрана.
    Расстояние между источниками $l = 1{,}2\,\text{мм}$,
    а от каждого источника до экрана — $L = 2\,\text{м}$.
    Сделайте рисунок и укажите положение нулевого максимума освещенности,
    а также определите расстояние между вторым максимумом и нулевым максимумом.
    Длина волны падающего света составляет $\lambda = 450\,\text{нм}$.
}
\answer{%
    \begin{align*}
    l_1^2 &= L^2 + \sqr{x - \frac \ell 2} \\
    l_2^2 &= L^2 + \sqr{x + \frac \ell 2} \\
    l_2^2 - l_1^2 &= 2x\ell \implies (l_2 - l_1)(l_2 + l_1) = 2x\ell \implies n\lambda \cdot 2L \approx 2x_n\ell \implies x_n = \frac{\lambda L}{\ell} n, n\in \mathbb{N} \\
    x &= \frac{\lambda L}{\ell} \cdot 2 = \frac{450\,\text{нм} \cdot 2\,\text{м}}{1{,}2\,\text{мм}} \cdot 2 \approx 1{,}50\,\text{мм}
    \end{align*}
}
\solutionspace{120pt}

\tasknumber{4}%
\task{%
    Каков наибольший порядок спектра, который можно наблюдать при дифракции света
    с длиной волны $\lambda$, на дифракционной решётке с периодом $d =  3{,}9 \lambda$?
    Под каким углом наблюдается последний максимум?
}
\answer{%
    $
        d\sin \varphi_k = k\lambda
        \implies k = \frac{d\sin \varphi_k}{\lambda} \le \frac{d \cdot 1}{\lambda} =  3{,}9
        \implies k_{\max} = 3
        \implies \alpha_{3} \approx 50{,}28\degrees
    $
}
\solutionspace{80pt}

\tasknumber{5}%
\task{%
    Вертикально стоящий шест высотой 1,1 м, освещенный солнцем,
    отбрасывает на горизонтальную поверхность земли тень длиной $2\,\text{м}$.
    Известно, что длина тени от телеграфного столба на $7\,\text{м}$ больше.
    Определить высоту столба.
}
\solutionspace{80pt}

\tasknumber{6}%
\task{%
    Определить абсолютный показатель преломления прозрачной среды,
    в которой распространяется свет с длиной волны $0{,}500\,\text{мкм}$ и частотой $400\,\text{ТГц}$.
    Скорость света в вакууме $3 \cdot 10^{8}\,\frac{\text{м}}{\text{с}}$.
}
\answer{%
    $
        n = \frac{c}{v}
        = \frac{c}{\frac \lambda T}
        = \frac{c}{\lambda \nu}
        = \frac{3 \cdot 10^{8}\,\frac{\text{м}}{\text{с}}}{0{,}500\,\text{мкм} \cdot {400\,\text{ТГц}}}
        \approx1{,}50
    $
}

\variantsplitter

\addpersonalvariant{Егор Свистушкин}

\tasknumber{1}%
\task{%
    Сформулируйте:
    \begin{itemize}
        \item что такое интерференция,
        \item определение дисперсии,
        \item условие интерференционного максимума для света, падающего нормально на дифракционную решётку,
        \item условия когерентности 2 источников света,
    \end{itemize}
}
\solutionspace{40pt}

\tasknumber{2}%
\task{%
    Произведите вычисления и округлите
    \begin{itemize}
        \item $55{,}58\,\text{м} \cdot 0{,}053\,\text{кг} = \ldots$
        \item $\cfrac{0{,}107\,\text{с}}{3{,}3\,\frac{\text{с}}{\text{м}}} = \ldots$
        \item $0{,}1245\,\text{А} \cdot 5{,}89\,\text{с} = \ldots$
        \item $3 \pi^2 \cdot \cfrac{0{,}1245\,\text{А}}{5{,}89\,\text{с}} = \ldots$
    \end{itemize}
}
\answer{%
    \begin{align*}
    55{,}58\,\text{м} \cdot 0{,}053\,\text{кг} &= 2{,}9\,\text{кг}\cdot\text{м} \\
    \cfrac{0{,}107\,\text{с}}{3{,}3\,\frac{\text{с}}{\text{м}}} &= 0{,}032\,\text{м} \\
    0{,}1245\,\text{А} \cdot 5{,}89\,\text{с} &= 0{,}733\,\text{Кл} \\
    3 \pi^2 \cdot \cfrac{0{,}1245\,\text{А}}{5{,}89\,\text{с}} &= 0{,}63\,\frac{\text{А}}{\text{с}}
    \end{align*}
}

\tasknumber{3}%
\task{%
    Установка для наблюдения интерференции состоит
    из двух когерентных источников света и экрана.
    Расстояние между источниками $l = 2{,}4\,\text{мм}$,
    а от каждого источника до экрана — $L = 2\,\text{м}$.
    Сделайте рисунок и укажите положение нулевого максимума освещенности,
    а также определите расстояние между третьим минимумом и нулевым максимумом.
    Длина волны падающего света составляет $\lambda = 450\,\text{нм}$.
}
\answer{%
    \begin{align*}
    l_1^2 &= L^2 + \sqr{x - \frac \ell 2} \\
    l_2^2 &= L^2 + \sqr{x + \frac \ell 2} \\
    l_2^2 - l_1^2 &= 2x\ell \implies (l_2 - l_1)(l_2 + l_1) = 2x\ell \implies n\lambda \cdot 2L \approx 2x_n\ell \implies x_n = \frac{\lambda L}{\ell} n, n\in \mathbb{N} \\
    x &= \frac{\lambda L}{\ell} \cdot \frac52 = \frac{450\,\text{нм} \cdot 2\,\text{м}}{2{,}4\,\text{мм}} \cdot \frac52 \approx 0{,}94\,\text{мм}
    \end{align*}
}
\solutionspace{120pt}

\tasknumber{4}%
\task{%
    Каков наибольший порядок спектра, который можно наблюдать при дифракции света
    с длиной волны $\lambda$, на дифракционной решётке с периодом $d =  3{,}3 \lambda$?
    Под каким углом наблюдается последний максимум?
}
\answer{%
    $
        d\sin \varphi_k = k\lambda
        \implies k = \frac{d\sin \varphi_k}{\lambda} \le \frac{d \cdot 1}{\lambda} =  3{,}3
        \implies k_{\max} = 3
        \implies \alpha_{3} \approx 65{,}38\degrees
    $
}
\solutionspace{80pt}

\tasknumber{5}%
\task{%
    Вертикально стоящий шест высотой 1,1 м, освещенный солнцем,
    отбрасывает на горизонтальную поверхность земли тень длиной $4\,\text{м}$.
    Известно, что длина тени от телеграфного столба на $7\,\text{м}$ больше.
    Определить высоту столба.
}
\solutionspace{80pt}

\tasknumber{6}%
\task{%
    Определить абсолютный показатель преломления прозрачной среды,
    в которой распространяется свет с длиной волны $0{,}550\,\text{мкм}$ и частотой $360\,\text{ТГц}$.
    Скорость света в вакууме $3 \cdot 10^{8}\,\frac{\text{м}}{\text{с}}$.
}
\answer{%
    $
        n = \frac{c}{v}
        = \frac{c}{\frac \lambda T}
        = \frac{c}{\lambda \nu}
        = \frac{3 \cdot 10^{8}\,\frac{\text{м}}{\text{с}}}{0{,}550\,\text{мкм} \cdot {360\,\text{ТГц}}}
        \approx1{,}50
    $
}

\variantsplitter

\addpersonalvariant{Дмитрий Соколов}

\tasknumber{1}%
\task{%
    Сформулируйте:
    \begin{itemize}
        \item что такое интерференция,
        \item определение дисперсии,
        \item условие интерференционного максимума для света, падающего нормально на дифракционную решётку,
        \item условия когерентности 2 источников света,
    \end{itemize}
}
\solutionspace{40pt}

\tasknumber{2}%
\task{%
    Произведите вычисления и округлите
    \begin{itemize}
        \item $59{,}11\,\text{м} \cdot 0{,}074\,\text{кг} = \ldots$
        \item $\cfrac{0{,}360\,\text{с}}{4{,}6\,\frac{\text{с}}{\text{м}}} = \ldots$
        \item $0{,}0069\,\text{А} \cdot 0{,}06\,\text{с} = \ldots$
        \item $3 \pi^2 \cdot \cfrac{0{,}0069\,\text{А}}{0{,}06\,\text{с}} = \ldots$
    \end{itemize}
}
\answer{%
    \begin{align*}
    59{,}11\,\text{м} \cdot 0{,}074\,\text{кг} &= 4{,}4\,\text{кг}\cdot\text{м} \\
    \cfrac{0{,}360\,\text{с}}{4{,}6\,\frac{\text{с}}{\text{м}}} &= 0{,}078\,\text{м} \\
    0{,}0069\,\text{А} \cdot 0{,}06\,\text{с} &= 0{,}0004\,\text{Кл} \\
    3 \pi^2 \cdot \cfrac{0{,}0069\,\text{А}}{0{,}06\,\text{с}} &= 3{,}4\,\frac{\text{А}}{\text{с}}
    \end{align*}
}

\tasknumber{3}%
\task{%
    Установка для наблюдения интерференции состоит
    из двух когерентных источников света и экрана.
    Расстояние между источниками $l = 1{,}5\,\text{мм}$,
    а от каждого источника до экрана — $L = 4\,\text{м}$.
    Сделайте рисунок и укажите положение нулевого максимума освещенности,
    а также определите расстояние между третьим максимумом и нулевым максимумом.
    Длина волны падающего света составляет $\lambda = 450\,\text{нм}$.
}
\answer{%
    \begin{align*}
    l_1^2 &= L^2 + \sqr{x - \frac \ell 2} \\
    l_2^2 &= L^2 + \sqr{x + \frac \ell 2} \\
    l_2^2 - l_1^2 &= 2x\ell \implies (l_2 - l_1)(l_2 + l_1) = 2x\ell \implies n\lambda \cdot 2L \approx 2x_n\ell \implies x_n = \frac{\lambda L}{\ell} n, n\in \mathbb{N} \\
    x &= \frac{\lambda L}{\ell} \cdot 3 = \frac{450\,\text{нм} \cdot 4\,\text{м}}{1{,}5\,\text{мм}} \cdot 3 \approx 3{,}6\,\text{мм}
    \end{align*}
}
\solutionspace{120pt}

\tasknumber{4}%
\task{%
    Каков наибольший порядок спектра, который можно наблюдать при дифракции света
    с длиной волны $\lambda$, на дифракционной решётке с периодом $d =  2{,}7 \lambda$?
    Под каким углом наблюдается последний максимум?
}
\answer{%
    $
        d\sin \varphi_k = k\lambda
        \implies k = \frac{d\sin \varphi_k}{\lambda} \le \frac{d \cdot 1}{\lambda} =  2{,}7
        \implies k_{\max} = 2
        \implies \alpha_{2} \approx 47{,}79\degrees
    $
}
\solutionspace{80pt}

\tasknumber{5}%
\task{%
    Вертикально стоящий шест высотой 1,1 м, освещенный солнцем,
    отбрасывает на горизонтальную поверхность земли тень длиной $4\,\text{м}$.
    Известно, что длина тени от телеграфного столба на $8\,\text{м}$ больше.
    Определить высоту столба.
}
\solutionspace{80pt}

\tasknumber{6}%
\task{%
    Определить абсолютный показатель преломления прозрачной среды,
    в которой распространяется свет с длиной волны $0{,}550\,\text{мкм}$ и частотой $320\,\text{ТГц}$.
    Скорость света в вакууме $3 \cdot 10^{8}\,\frac{\text{м}}{\text{с}}$.
}
\answer{%
    $
        n = \frac{c}{v}
        = \frac{c}{\frac \lambda T}
        = \frac{c}{\lambda \nu}
        = \frac{3 \cdot 10^{8}\,\frac{\text{м}}{\text{с}}}{0{,}550\,\text{мкм} \cdot {320\,\text{ТГц}}}
        \approx1{,}70
    $
}

\variantsplitter

\addpersonalvariant{Арсений Трофимов}

\tasknumber{1}%
\task{%
    Сформулируйте:
    \begin{itemize}
        \item что такое дифракция,
        \item определение дисперсии,
        \item условие интерференционного максимума для света, падающего нормально на дифракционную решётку,
        \item условия когерентности 2 источников света,
    \end{itemize}
}
\solutionspace{40pt}

\tasknumber{2}%
\task{%
    Произведите вычисления и округлите
    \begin{itemize}
        \item $885{,}36\,\text{м} \cdot 0{,}004\,\text{кг} = \ldots$
        \item $\cfrac{0{,}694\,\text{с}}{4{,}6\,\frac{\text{с}}{\text{м}}} = \ldots$
        \item $0{,}0205\,\text{А} \cdot 0{,}06\,\text{с} = \ldots$
        \item $3 \pi^2 \cdot \cfrac{0{,}0205\,\text{А}}{0{,}06\,\text{с}} = \ldots$
    \end{itemize}
}
\answer{%
    \begin{align*}
    885{,}36\,\text{м} \cdot 0{,}004\,\text{кг} &= 4\,\text{кг}\cdot\text{м} \\
    \cfrac{0{,}694\,\text{с}}{4{,}6\,\frac{\text{с}}{\text{м}}} &= 0{,}151\,\text{м} \\
    0{,}0205\,\text{А} \cdot 0{,}06\,\text{с} &= 0{,}0012\,\text{Кл} \\
    3 \pi^2 \cdot \cfrac{0{,}0205\,\text{А}}{0{,}06\,\text{с}} &= 10{,}1\,\frac{\text{А}}{\text{с}}
    \end{align*}
}

\tasknumber{3}%
\task{%
    Установка для наблюдения интерференции состоит
    из двух когерентных источников света и экрана.
    Расстояние между источниками $l = 2{,}4\,\text{мм}$,
    а от каждого источника до экрана — $L = 2\,\text{м}$.
    Сделайте рисунок и укажите положение нулевого максимума освещенности,
    а также определите расстояние между третьим максимумом и нулевым максимумом.
    Длина волны падающего света составляет $\lambda = 450\,\text{нм}$.
}
\answer{%
    \begin{align*}
    l_1^2 &= L^2 + \sqr{x - \frac \ell 2} \\
    l_2^2 &= L^2 + \sqr{x + \frac \ell 2} \\
    l_2^2 - l_1^2 &= 2x\ell \implies (l_2 - l_1)(l_2 + l_1) = 2x\ell \implies n\lambda \cdot 2L \approx 2x_n\ell \implies x_n = \frac{\lambda L}{\ell} n, n\in \mathbb{N} \\
    x &= \frac{\lambda L}{\ell} \cdot 3 = \frac{450\,\text{нм} \cdot 2\,\text{м}}{2{,}4\,\text{мм}} \cdot 3 \approx 1{,}13\,\text{мм}
    \end{align*}
}
\solutionspace{120pt}

\tasknumber{4}%
\task{%
    Каков наибольший порядок спектра, который можно наблюдать при дифракции света
    с длиной волны $\lambda$, на дифракционной решётке с периодом $d =  4{,}6 \lambda$?
    Под каким углом наблюдается последний максимум?
}
\answer{%
    $
        d\sin \varphi_k = k\lambda
        \implies k = \frac{d\sin \varphi_k}{\lambda} \le \frac{d \cdot 1}{\lambda} =  4{,}6
        \implies k_{\max} = 4
        \implies \alpha_{4} \approx 60{,}41\degrees
    $
}
\solutionspace{80pt}

\tasknumber{5}%
\task{%
    Вертикально стоящий шест высотой 1,1 м, освещенный солнцем,
    отбрасывает на горизонтальную поверхность земли тень длиной $2\,\text{м}$.
    Известно, что длина тени от телеграфного столба на $6\,\text{м}$ больше.
    Определить высоту столба.
}
\solutionspace{80pt}

\tasknumber{6}%
\task{%
    Определить абсолютный показатель преломления прозрачной среды,
    в которой распространяется свет с длиной волны $0{,}450\,\text{мкм}$ и частотой $440\,\text{ТГц}$.
    Скорость света в вакууме $3 \cdot 10^{8}\,\frac{\text{м}}{\text{с}}$.
}
\answer{%
    $
        n = \frac{c}{v}
        = \frac{c}{\frac \lambda T}
        = \frac{c}{\lambda \nu}
        = \frac{3 \cdot 10^{8}\,\frac{\text{м}}{\text{с}}}{0{,}450\,\text{мкм} \cdot {440\,\text{ТГц}}}
        \approx1{,}50
    $
}
% autogenerated
