\setdate{15~декабря~2021}
\setclass{11«Б»}

\addpersonalvariant{Михаил Бурмистров}

\tasknumber{1}%
\task{%
    Сформулируйте:
    \begin{itemize}
        \item что такое дифракция,
        \item определение дисперсии,
        \item условие интерференционного максимума для света, падающего нормально на дифракционную решётку,
        \item условия наблюдения минимума и максимума в интерферeнционной картине,
    \end{itemize}
}
\solutionspace{40pt}

\tasknumber{2}%
\task{%
    На дифракционную решётку, имеющую период $3 \cdot 10^{-4}\,\text{см}$, нормально падает монохроматическая световая волна.
    Под углом $ 25 \degrees$ наблюдается дифракционный максимум второго порядка.
    Какова длина волны падающего света?
}
\answer{%
    $
        d\sin \varphi_k = k\lambda
        \implies \lambda = \frac{d \sin \varphi_k}k
        = \frac{3 \cdot 10^{-4}\,\text{см} \cdot \sin  25 \degrees}{2} \approx 630\,\text{нм}
    $
}
\solutionspace{150pt}

\tasknumber{3}%
\task{%
    Свет с длиной волны $0{,}6\,\text{мкм}$ падает нормально на дифракционную решётку с периодом, равным $1\,\text{мкм}$.
    Под каким углом наблюдается дифракционный максимум первого порядка?
}
\answer{%
    $
        d\sin \varphi_k = k\lambda
        \implies \sin \varphi_k = \frac{k\lambda}{ d }
        = \frac{1 \cdot 0{,}6\,\text{мкм}}{1\,\text{мкм}} \approx 0{,}6 \implies \varphi_k \approx 36{,}9\degrees
    $
}
\solutionspace{150pt}

\tasknumber{4}%
\task{%
    При нормальном падении белого света на дифракционную решётку зелёная линия ($520\,\text{нм}$)
    в спектре третьего порядка видна под углом дифракции $25\degrees$.
    Определить число штрихов на $1\,\text{см}$ длины этой решётки.
}
\answer{%
    $
        d\sin \varphi_k = k\lambda
        \implies d = \frac{k\lambda}{\sin \varphi_k}.
        \qquad N = \frac{ l }{d} = \frac{ l \sin \varphi_k}{k\lambda}
        = \frac{1\,\text{см} \cdot \sin 25\degrees}{3 \cdot 520\,\text{нм}} \approx 2700
    $
}
\solutionspace{150pt}

\tasknumber{5}%
\task{%
    Каков наибольший порядок спектра, который можно наблюдать при дифракции света
    с длиной волны $\lambda$, на дифракционной решётке с периодом $d =  3{,}5 \lambda$?
    Под каким углом наблюдается последний максимум?
}
\answer{%
    $
        d\sin \varphi_k = k\lambda
        \implies k = \frac{d\sin \varphi_k}{\lambda} \le \frac{d \cdot 1}{\lambda} =  3{,}5
        \implies k_{\max} = 3
        \implies \alpha_{3} \approx 59{,}00\degrees
    $
}

\variantsplitter

\addpersonalvariant{Снежана Авдошина}

\tasknumber{1}%
\task{%
    Сформулируйте:
    \begin{itemize}
        \item что такое интерференция,
        \item определение дисперсии,
        \item условие интерференционного максимума для света, падающего нормально на дифракционную решётку,
        \item условия наблюдения минимума и максимума в интерферeнционной картине,
    \end{itemize}
}
\solutionspace{40pt}

\tasknumber{2}%
\task{%
    На дифракционную решётку, имеющую период $4 \cdot 10^{-4}\,\text{см}$, нормально падает монохроматическая световая волна.
    Под углом $ 25 \degrees$ наблюдается дифракционный максимум третьего порядка.
    Какова длина волны падающего света?
}
\answer{%
    $
        d\sin \varphi_k = k\lambda
        \implies \lambda = \frac{d \sin \varphi_k}k
        = \frac{4 \cdot 10^{-4}\,\text{см} \cdot \sin  25 \degrees}{3} \approx 560\,\text{нм}
    $
}
\solutionspace{150pt}

\tasknumber{3}%
\task{%
    Свет с длиной волны $0{,}7\,\text{мкм}$ падает нормально на дифракционную решётку с периодом, равным $2\,\text{мкм}$.
    Под каким углом наблюдается дифракционный максимум первого порядка?
}
\answer{%
    $
        d\sin \varphi_k = k\lambda
        \implies \sin \varphi_k = \frac{k\lambda}{ d }
        = \frac{1 \cdot 0{,}7\,\text{мкм}}{2\,\text{мкм}} \approx 0{,}4 \implies \varphi_k \approx 20{,}5\degrees
    $
}
\solutionspace{150pt}

\tasknumber{4}%
\task{%
    При нормальном падении белого света на дифракционную решётку красная линия ($720\,\text{нм}$)
    в спектре четвёртого порядка видна под углом дифракции $25\degrees$.
    Определить число штрихов на $1\,\text{мм}$ длины этой решётки.
}
\answer{%
    $
        d\sin \varphi_k = k\lambda
        \implies d = \frac{k\lambda}{\sin \varphi_k}.
        \qquad N = \frac{ l }{d} = \frac{ l \sin \varphi_k}{k\lambda}
        = \frac{1\,\text{мм} \cdot \sin 25\degrees}{4 \cdot 720\,\text{нм}} \approx 147
    $
}
\solutionspace{150pt}

\tasknumber{5}%
\task{%
    Каков наибольший порядок спектра, который можно наблюдать при дифракции света
    с длиной волны $\lambda$, на дифракционной решётке с периодом $d =  4{,}1 \lambda$?
    Под каким углом наблюдается последний максимум?
}
\answer{%
    $
        d\sin \varphi_k = k\lambda
        \implies k = \frac{d\sin \varphi_k}{\lambda} \le \frac{d \cdot 1}{\lambda} =  4{,}1
        \implies k_{\max} = 4
        \implies \alpha_{4} \approx 77{,}32\degrees
    $
}

\variantsplitter

\addpersonalvariant{Марьяна Аристова}

\tasknumber{1}%
\task{%
    Сформулируйте:
    \begin{itemize}
        \item что такое дифракция,
        \item определение дисперсии,
        \item условие интерференционного максимума для света, падающего нормально на дифракционную решётку,
        \item условия наблюдения минимума и максимума в интерферeнционной картине,
    \end{itemize}
}
\solutionspace{40pt}

\tasknumber{2}%
\task{%
    На дифракционную решётку, имеющую период $3 \cdot 10^{-4}\,\text{см}$, нормально падает монохроматическая световая волна.
    Под углом $ 20 \degrees$ наблюдается дифракционный максимум четвёртого порядка.
    Какова длина волны падающего света?
}
\answer{%
    $
        d\sin \varphi_k = k\lambda
        \implies \lambda = \frac{d \sin \varphi_k}k
        = \frac{3 \cdot 10^{-4}\,\text{см} \cdot \sin  20 \degrees}{4} \approx 260\,\text{нм}
    $
}
\solutionspace{150pt}

\tasknumber{3}%
\task{%
    Свет с длиной волны $0{,}6\,\text{мкм}$ падает нормально на дифракционную решётку с периодом, равным $2\,\text{мкм}$.
    Под каким углом наблюдается дифракционный максимум первого порядка?
}
\answer{%
    $
        d\sin \varphi_k = k\lambda
        \implies \sin \varphi_k = \frac{k\lambda}{ d }
        = \frac{1 \cdot 0{,}6\,\text{мкм}}{2\,\text{мкм}} \approx 0{,}3 \implies \varphi_k \approx 17{,}5\degrees
    $
}
\solutionspace{150pt}

\tasknumber{4}%
\task{%
    При нормальном падении белого света на дифракционную решётку синяя линия ($450\,\text{нм}$)
    в спектре третьего порядка видна под углом дифракции $18\degrees$.
    Определить число штрихов на $1\,\text{мм}$ длины этой решётки.
}
\answer{%
    $
        d\sin \varphi_k = k\lambda
        \implies d = \frac{k\lambda}{\sin \varphi_k}.
        \qquad N = \frac{ l }{d} = \frac{ l \sin \varphi_k}{k\lambda}
        = \frac{1\,\text{мм} \cdot \sin 18\degrees}{3 \cdot 450\,\text{нм}} \approx 230
    $
}
\solutionspace{150pt}

\tasknumber{5}%
\task{%
    Каков наибольший порядок спектра, который можно наблюдать при дифракции света
    с длиной волны $\lambda$, на дифракционной решётке с периодом $d =  2{,}5 \lambda$?
    Под каким углом наблюдается последний максимум?
}
\answer{%
    $
        d\sin \varphi_k = k\lambda
        \implies k = \frac{d\sin \varphi_k}{\lambda} \le \frac{d \cdot 1}{\lambda} =  2{,}5
        \implies k_{\max} = 2
        \implies \alpha_{2} \approx 53{,}13\degrees
    $
}

\variantsplitter

\addpersonalvariant{Никита Иванов}

\tasknumber{1}%
\task{%
    Сформулируйте:
    \begin{itemize}
        \item что такое дифракция,
        \item определение дисперсии,
        \item условие интерференционного максимума для света, падающего нормально на дифракционную решётку,
        \item условия когерентности 2 источников света,
    \end{itemize}
}
\solutionspace{40pt}

\tasknumber{2}%
\task{%
    На дифракционную решётку, имеющую период $3 \cdot 10^{-4}\,\text{см}$, нормально падает монохроматическая световая волна.
    Под углом $ 35 \degrees$ наблюдается дифракционный максимум четвёртого порядка.
    Какова длина волны падающего света?
}
\answer{%
    $
        d\sin \varphi_k = k\lambda
        \implies \lambda = \frac{d \sin \varphi_k}k
        = \frac{3 \cdot 10^{-4}\,\text{см} \cdot \sin  35 \degrees}{4} \approx 430\,\text{нм}
    $
}
\solutionspace{150pt}

\tasknumber{3}%
\task{%
    Свет с длиной волны $0{,}4\,\text{мкм}$ падает нормально на дифракционную решётку с периодом, равным $2\,\text{мкм}$.
    Под каким углом наблюдается дифракционный максимум первого порядка?
}
\answer{%
    $
        d\sin \varphi_k = k\lambda
        \implies \sin \varphi_k = \frac{k\lambda}{ d }
        = \frac{1 \cdot 0{,}4\,\text{мкм}}{2\,\text{мкм}} \approx 0{,}2 \implies \varphi_k \approx 11{,}5\degrees
    $
}
\solutionspace{150pt}

\tasknumber{4}%
\task{%
    При нормальном падении белого света на дифракционную решётку красная линия ($680\,\text{нм}$)
    в спектре второго порядка видна под углом дифракции $18\degrees$.
    Определить число штрихов на $1\,\text{мм}$ длины этой решётки.
}
\answer{%
    $
        d\sin \varphi_k = k\lambda
        \implies d = \frac{k\lambda}{\sin \varphi_k}.
        \qquad N = \frac{ l }{d} = \frac{ l \sin \varphi_k}{k\lambda}
        = \frac{1\,\text{мм} \cdot \sin 18\degrees}{2 \cdot 680\,\text{нм}} \approx 230
    $
}
\solutionspace{150pt}

\tasknumber{5}%
\task{%
    Каков наибольший порядок спектра, который можно наблюдать при дифракции света
    с длиной волны $\lambda$, на дифракционной решётке с периодом $d =  4{,}5 \lambda$?
    Под каким углом наблюдается последний максимум?
}
\answer{%
    $
        d\sin \varphi_k = k\lambda
        \implies k = \frac{d\sin \varphi_k}{\lambda} \le \frac{d \cdot 1}{\lambda} =  4{,}5
        \implies k_{\max} = 4
        \implies \alpha_{4} \approx 62{,}73\degrees
    $
}

\variantsplitter

\addpersonalvariant{Анастасия Князева}

\tasknumber{1}%
\task{%
    Сформулируйте:
    \begin{itemize}
        \item что такое интерференция,
        \item определение дисперсии,
        \item условие интерференционного максимума для света, падающего нормально на дифракционную решётку,
        \item условия наблюдения минимума и максимума в интерферeнционной картине,
    \end{itemize}
}
\solutionspace{40pt}

\tasknumber{2}%
\task{%
    На дифракционную решётку, имеющую период $3 \cdot 10^{-4}\,\text{см}$, нормально падает монохроматическая световая волна.
    Под углом $ 40 \degrees$ наблюдается дифракционный максимум третьего порядка.
    Какова длина волны падающего света?
}
\answer{%
    $
        d\sin \varphi_k = k\lambda
        \implies \lambda = \frac{d \sin \varphi_k}k
        = \frac{3 \cdot 10^{-4}\,\text{см} \cdot \sin  40 \degrees}{3} \approx 640\,\text{нм}
    $
}
\solutionspace{150pt}

\tasknumber{3}%
\task{%
    Свет с длиной волны $0{,}4\,\text{мкм}$ падает нормально на дифракционную решётку с периодом, равным $3\,\text{мкм}$.
    Под каким углом наблюдается дифракционный максимум первого порядка?
}
\answer{%
    $
        d\sin \varphi_k = k\lambda
        \implies \sin \varphi_k = \frac{k\lambda}{ d }
        = \frac{1 \cdot 0{,}4\,\text{мкм}}{3\,\text{мкм}} \approx 0{,}13 \implies \varphi_k \approx 7{,}7\degrees
    $
}
\solutionspace{150pt}

\tasknumber{4}%
\task{%
    При нормальном падении белого света на дифракционную решётку синяя линия ($480\,\text{нм}$)
    в спектре четвёртого порядка видна под углом дифракции $25\degrees$.
    Определить число штрихов на $1\,\text{см}$ длины этой решётки.
}
\answer{%
    $
        d\sin \varphi_k = k\lambda
        \implies d = \frac{k\lambda}{\sin \varphi_k}.
        \qquad N = \frac{ l }{d} = \frac{ l \sin \varphi_k}{k\lambda}
        = \frac{1\,\text{см} \cdot \sin 25\degrees}{4 \cdot 480\,\text{нм}} \approx 2200
    $
}
\solutionspace{150pt}

\tasknumber{5}%
\task{%
    Каков наибольший порядок спектра, который можно наблюдать при дифракции света
    с длиной волны $\lambda$, на дифракционной решётке с периодом $d =  2{,}2 \lambda$?
    Под каким углом наблюдается последний максимум?
}
\answer{%
    $
        d\sin \varphi_k = k\lambda
        \implies k = \frac{d\sin \varphi_k}{\lambda} \le \frac{d \cdot 1}{\lambda} =  2{,}2
        \implies k_{\max} = 2
        \implies \alpha_{2} \approx 65{,}38\degrees
    $
}

\variantsplitter

\addpersonalvariant{Елизавета Кутумова}

\tasknumber{1}%
\task{%
    Сформулируйте:
    \begin{itemize}
        \item что такое дифракция,
        \item определение дисперсии,
        \item условие интерференционного максимума для света, падающего нормально на дифракционную решётку,
        \item условия когерентности 2 источников света,
    \end{itemize}
}
\solutionspace{40pt}

\tasknumber{2}%
\task{%
    На дифракционную решётку, имеющую период $2 \cdot 10^{-4}\,\text{см}$, нормально падает монохроматическая световая волна.
    Под углом $ 20 \degrees$ наблюдается дифракционный максимум третьего порядка.
    Какова длина волны падающего света?
}
\answer{%
    $
        d\sin \varphi_k = k\lambda
        \implies \lambda = \frac{d \sin \varphi_k}k
        = \frac{2 \cdot 10^{-4}\,\text{см} \cdot \sin  20 \degrees}{3} \approx 230\,\text{нм}
    $
}
\solutionspace{150pt}

\tasknumber{3}%
\task{%
    Свет с длиной волны $0{,}4\,\text{мкм}$ падает нормально на дифракционную решётку с периодом, равным $2\,\text{мкм}$.
    Под каким углом наблюдается дифракционный максимум первого порядка?
}
\answer{%
    $
        d\sin \varphi_k = k\lambda
        \implies \sin \varphi_k = \frac{k\lambda}{ d }
        = \frac{1 \cdot 0{,}4\,\text{мкм}}{2\,\text{мкм}} \approx 0{,}2 \implies \varphi_k \approx 11{,}5\degrees
    $
}
\solutionspace{150pt}

\tasknumber{4}%
\task{%
    При нормальном падении белого света на дифракционную решётку зелёная линия ($520\,\text{нм}$)
    в спектре третьего порядка видна под углом дифракции $5\degrees$.
    Определить число штрихов на $1\,\text{см}$ длины этой решётки.
}
\answer{%
    $
        d\sin \varphi_k = k\lambda
        \implies d = \frac{k\lambda}{\sin \varphi_k}.
        \qquad N = \frac{ l }{d} = \frac{ l \sin \varphi_k}{k\lambda}
        = \frac{1\,\text{см} \cdot \sin 5\degrees}{3 \cdot 520\,\text{нм}} \approx 560
    $
}
\solutionspace{150pt}

\tasknumber{5}%
\task{%
    Каков наибольший порядок спектра, который можно наблюдать при дифракции света
    с длиной волны $\lambda$, на дифракционной решётке с периодом $d =  4{,}1 \lambda$?
    Под каким углом наблюдается последний максимум?
}
\answer{%
    $
        d\sin \varphi_k = k\lambda
        \implies k = \frac{d\sin \varphi_k}{\lambda} \le \frac{d \cdot 1}{\lambda} =  4{,}1
        \implies k_{\max} = 4
        \implies \alpha_{4} \approx 77{,}32\degrees
    $
}

\variantsplitter

\addpersonalvariant{Роксана Мехтиева}

\tasknumber{1}%
\task{%
    Сформулируйте:
    \begin{itemize}
        \item что такое дифракция,
        \item определение дисперсии,
        \item условие интерференционного максимума для света, падающего нормально на дифракционную решётку,
        \item условия наблюдения минимума и максимума в интерферeнционной картине,
    \end{itemize}
}
\solutionspace{40pt}

\tasknumber{2}%
\task{%
    На дифракционную решётку, имеющую период $2 \cdot 10^{-4}\,\text{см}$, нормально падает монохроматическая световая волна.
    Под углом $ 40 \degrees$ наблюдается дифракционный максимум второго порядка.
    Какова длина волны падающего света?
}
\answer{%
    $
        d\sin \varphi_k = k\lambda
        \implies \lambda = \frac{d \sin \varphi_k}k
        = \frac{2 \cdot 10^{-4}\,\text{см} \cdot \sin  40 \degrees}{2} \approx 640\,\text{нм}
    $
}
\solutionspace{150pt}

\tasknumber{3}%
\task{%
    Свет с длиной волны $0{,}7\,\text{мкм}$ падает нормально на дифракционную решётку с периодом, равным $3\,\text{мкм}$.
    Под каким углом наблюдается дифракционный максимум первого порядка?
}
\answer{%
    $
        d\sin \varphi_k = k\lambda
        \implies \sin \varphi_k = \frac{k\lambda}{ d }
        = \frac{1 \cdot 0{,}7\,\text{мкм}}{3\,\text{мкм}} \approx 0{,}2 \implies \varphi_k \approx 13{,}5\degrees
    $
}
\solutionspace{150pt}

\tasknumber{4}%
\task{%
    При нормальном падении белого света на дифракционную решётку синяя линия ($480\,\text{нм}$)
    в спектре четвёртого порядка видна под углом дифракции $5\degrees$.
    Определить число штрихов на $1\,\text{мм}$ длины этой решётки.
}
\answer{%
    $
        d\sin \varphi_k = k\lambda
        \implies d = \frac{k\lambda}{\sin \varphi_k}.
        \qquad N = \frac{ l }{d} = \frac{ l \sin \varphi_k}{k\lambda}
        = \frac{1\,\text{мм} \cdot \sin 5\degrees}{4 \cdot 480\,\text{нм}} \approx 45
    $
}
\solutionspace{150pt}

\tasknumber{5}%
\task{%
    Каков наибольший порядок спектра, который можно наблюдать при дифракции света
    с длиной волны $\lambda$, на дифракционной решётке с периодом $d =  2{,}2 \lambda$?
    Под каким углом наблюдается последний максимум?
}
\answer{%
    $
        d\sin \varphi_k = k\lambda
        \implies k = \frac{d\sin \varphi_k}{\lambda} \le \frac{d \cdot 1}{\lambda} =  2{,}2
        \implies k_{\max} = 2
        \implies \alpha_{2} \approx 65{,}38\degrees
    $
}

\variantsplitter

\addpersonalvariant{Дилноза Нодиршоева}

\tasknumber{1}%
\task{%
    Сформулируйте:
    \begin{itemize}
        \item что такое дифракция,
        \item определение дисперсии,
        \item условие интерференционного максимума для света, падающего нормально на дифракционную решётку,
        \item условия когерентности 2 источников света,
    \end{itemize}
}
\solutionspace{40pt}

\tasknumber{2}%
\task{%
    На дифракционную решётку, имеющую период $4 \cdot 10^{-4}\,\text{см}$, нормально падает монохроматическая световая волна.
    Под углом $ 30 \degrees$ наблюдается дифракционный максимум второго порядка.
    Какова длина волны падающего света?
}
\answer{%
    $
        d\sin \varphi_k = k\lambda
        \implies \lambda = \frac{d \sin \varphi_k}k
        = \frac{4 \cdot 10^{-4}\,\text{см} \cdot \sin  30 \degrees}{2} \approx 1000\,\text{нм}
    $
}
\solutionspace{150pt}

\tasknumber{3}%
\task{%
    Свет с длиной волны $0{,}4\,\text{мкм}$ падает нормально на дифракционную решётку с периодом, равным $3\,\text{мкм}$.
    Под каким углом наблюдается дифракционный максимум первого порядка?
}
\answer{%
    $
        d\sin \varphi_k = k\lambda
        \implies \sin \varphi_k = \frac{k\lambda}{ d }
        = \frac{1 \cdot 0{,}4\,\text{мкм}}{3\,\text{мкм}} \approx 0{,}13 \implies \varphi_k \approx 7{,}7\degrees
    $
}
\solutionspace{150pt}

\tasknumber{4}%
\task{%
    При нормальном падении белого света на дифракционную решётку зелёная линия ($550\,\text{нм}$)
    в спектре третьего порядка видна под углом дифракции $25\degrees$.
    Определить число штрихов на $1\,\text{мм}$ длины этой решётки.
}
\answer{%
    $
        d\sin \varphi_k = k\lambda
        \implies d = \frac{k\lambda}{\sin \varphi_k}.
        \qquad N = \frac{ l }{d} = \frac{ l \sin \varphi_k}{k\lambda}
        = \frac{1\,\text{мм} \cdot \sin 25\degrees}{3 \cdot 550\,\text{нм}} \approx 260
    $
}
\solutionspace{150pt}

\tasknumber{5}%
\task{%
    Каков наибольший порядок спектра, который можно наблюдать при дифракции света
    с длиной волны $\lambda$, на дифракционной решётке с периодом $d =  4{,}1 \lambda$?
    Под каким углом наблюдается последний максимум?
}
\answer{%
    $
        d\sin \varphi_k = k\lambda
        \implies k = \frac{d\sin \varphi_k}{\lambda} \le \frac{d \cdot 1}{\lambda} =  4{,}1
        \implies k_{\max} = 4
        \implies \alpha_{4} \approx 77{,}32\degrees
    $
}

\variantsplitter

\addpersonalvariant{Жаклин Пантелеева}

\tasknumber{1}%
\task{%
    Сформулируйте:
    \begin{itemize}
        \item что такое интерференция,
        \item определение дисперсии,
        \item условие интерференционного максимума для света, падающего нормально на дифракционную решётку,
        \item условия наблюдения минимума и максимума в интерферeнционной картине,
    \end{itemize}
}
\solutionspace{40pt}

\tasknumber{2}%
\task{%
    На дифракционную решётку, имеющую период $4 \cdot 10^{-4}\,\text{см}$, нормально падает монохроматическая световая волна.
    Под углом $ 35 \degrees$ наблюдается дифракционный максимум второго порядка.
    Какова длина волны падающего света?
}
\answer{%
    $
        d\sin \varphi_k = k\lambda
        \implies \lambda = \frac{d \sin \varphi_k}k
        = \frac{4 \cdot 10^{-4}\,\text{см} \cdot \sin  35 \degrees}{2} \approx 1150\,\text{нм}
    $
}
\solutionspace{150pt}

\tasknumber{3}%
\task{%
    Свет с длиной волны $0{,}5\,\text{мкм}$ падает нормально на дифракционную решётку с периодом, равным $2\,\text{мкм}$.
    Под каким углом наблюдается дифракционный максимум первого порядка?
}
\answer{%
    $
        d\sin \varphi_k = k\lambda
        \implies \sin \varphi_k = \frac{k\lambda}{ d }
        = \frac{1 \cdot 0{,}5\,\text{мкм}}{2\,\text{мкм}} \approx 0{,}3 \implies \varphi_k \approx 14{,}5\degrees
    $
}
\solutionspace{150pt}

\tasknumber{4}%
\task{%
    При нормальном падении белого света на дифракционную решётку синяя линия ($450\,\text{нм}$)
    в спектре второго порядка видна под углом дифракции $5\degrees$.
    Определить число штрихов на $1\,\text{мм}$ длины этой решётки.
}
\answer{%
    $
        d\sin \varphi_k = k\lambda
        \implies d = \frac{k\lambda}{\sin \varphi_k}.
        \qquad N = \frac{ l }{d} = \frac{ l \sin \varphi_k}{k\lambda}
        = \frac{1\,\text{мм} \cdot \sin 5\degrees}{2 \cdot 450\,\text{нм}} \approx 97
    $
}
\solutionspace{150pt}

\tasknumber{5}%
\task{%
    Каков наибольший порядок спектра, который можно наблюдать при дифракции света
    с длиной волны $\lambda$, на дифракционной решётке с периодом $d =  4{,}5 \lambda$?
    Под каким углом наблюдается последний максимум?
}
\answer{%
    $
        d\sin \varphi_k = k\lambda
        \implies k = \frac{d\sin \varphi_k}{\lambda} \le \frac{d \cdot 1}{\lambda} =  4{,}5
        \implies k_{\max} = 4
        \implies \alpha_{4} \approx 62{,}73\degrees
    $
}

\variantsplitter

\addpersonalvariant{Артём Переверзев}

\tasknumber{1}%
\task{%
    Сформулируйте:
    \begin{itemize}
        \item что такое интерференция,
        \item определение дисперсии,
        \item условие интерференционного максимума для света, падающего нормально на дифракционную решётку,
        \item условия наблюдения минимума и максимума в интерферeнционной картине,
    \end{itemize}
}
\solutionspace{40pt}

\tasknumber{2}%
\task{%
    На дифракционную решётку, имеющую период $4 \cdot 10^{-4}\,\text{см}$, нормально падает монохроматическая световая волна.
    Под углом $ 35 \degrees$ наблюдается дифракционный максимум второго порядка.
    Какова длина волны падающего света?
}
\answer{%
    $
        d\sin \varphi_k = k\lambda
        \implies \lambda = \frac{d \sin \varphi_k}k
        = \frac{4 \cdot 10^{-4}\,\text{см} \cdot \sin  35 \degrees}{2} \approx 1150\,\text{нм}
    $
}
\solutionspace{150pt}

\tasknumber{3}%
\task{%
    Свет с длиной волны $0{,}6\,\text{мкм}$ падает нормально на дифракционную решётку с периодом, равным $2\,\text{мкм}$.
    Под каким углом наблюдается дифракционный максимум первого порядка?
}
\answer{%
    $
        d\sin \varphi_k = k\lambda
        \implies \sin \varphi_k = \frac{k\lambda}{ d }
        = \frac{1 \cdot 0{,}6\,\text{мкм}}{2\,\text{мкм}} \approx 0{,}3 \implies \varphi_k \approx 17{,}5\degrees
    $
}
\solutionspace{150pt}

\tasknumber{4}%
\task{%
    При нормальном падении белого света на дифракционную решётку синяя линия ($450\,\text{нм}$)
    в спектре четвёртого порядка видна под углом дифракции $12\degrees$.
    Определить число штрихов на $1\,\text{мм}$ длины этой решётки.
}
\answer{%
    $
        d\sin \varphi_k = k\lambda
        \implies d = \frac{k\lambda}{\sin \varphi_k}.
        \qquad N = \frac{ l }{d} = \frac{ l \sin \varphi_k}{k\lambda}
        = \frac{1\,\text{мм} \cdot \sin 12\degrees}{4 \cdot 450\,\text{нм}} \approx 116
    $
}
\solutionspace{150pt}

\tasknumber{5}%
\task{%
    Каков наибольший порядок спектра, который можно наблюдать при дифракции света
    с длиной волны $\lambda$, на дифракционной решётке с периодом $d =  2{,}7 \lambda$?
    Под каким углом наблюдается последний максимум?
}
\answer{%
    $
        d\sin \varphi_k = k\lambda
        \implies k = \frac{d\sin \varphi_k}{\lambda} \le \frac{d \cdot 1}{\lambda} =  2{,}7
        \implies k_{\max} = 2
        \implies \alpha_{2} \approx 47{,}79\degrees
    $
}

\variantsplitter

\addpersonalvariant{Варвара Пранова}

\tasknumber{1}%
\task{%
    Сформулируйте:
    \begin{itemize}
        \item что такое дифракция,
        \item определение дисперсии,
        \item условие интерференционного максимума для света, падающего нормально на дифракционную решётку,
        \item условия наблюдения минимума и максимума в интерферeнционной картине,
    \end{itemize}
}
\solutionspace{40pt}

\tasknumber{2}%
\task{%
    На дифракционную решётку, имеющую период $4 \cdot 10^{-4}\,\text{см}$, нормально падает монохроматическая световая волна.
    Под углом $ 30 \degrees$ наблюдается дифракционный максимум четвёртого порядка.
    Какова длина волны падающего света?
}
\answer{%
    $
        d\sin \varphi_k = k\lambda
        \implies \lambda = \frac{d \sin \varphi_k}k
        = \frac{4 \cdot 10^{-4}\,\text{см} \cdot \sin  30 \degrees}{4} \approx 500\,\text{нм}
    $
}
\solutionspace{150pt}

\tasknumber{3}%
\task{%
    Свет с длиной волны $0{,}6\,\text{мкм}$ падает нормально на дифракционную решётку с периодом, равным $3\,\text{мкм}$.
    Под каким углом наблюдается дифракционный максимум первого порядка?
}
\answer{%
    $
        d\sin \varphi_k = k\lambda
        \implies \sin \varphi_k = \frac{k\lambda}{ d }
        = \frac{1 \cdot 0{,}6\,\text{мкм}}{3\,\text{мкм}} \approx 0{,}2 \implies \varphi_k \approx 11{,}5\degrees
    $
}
\solutionspace{150pt}

\tasknumber{4}%
\task{%
    При нормальном падении белого света на дифракционную решётку зелёная линия ($520\,\text{нм}$)
    в спектре четвёртого порядка видна под углом дифракции $12\degrees$.
    Определить число штрихов на $1\,\text{мм}$ длины этой решётки.
}
\answer{%
    $
        d\sin \varphi_k = k\lambda
        \implies d = \frac{k\lambda}{\sin \varphi_k}.
        \qquad N = \frac{ l }{d} = \frac{ l \sin \varphi_k}{k\lambda}
        = \frac{1\,\text{мм} \cdot \sin 12\degrees}{4 \cdot 520\,\text{нм}} \approx 100
    $
}
\solutionspace{150pt}

\tasknumber{5}%
\task{%
    Каков наибольший порядок спектра, который можно наблюдать при дифракции света
    с длиной волны $\lambda$, на дифракционной решётке с периодом $d =  2{,}2 \lambda$?
    Под каким углом наблюдается последний максимум?
}
\answer{%
    $
        d\sin \varphi_k = k\lambda
        \implies k = \frac{d\sin \varphi_k}{\lambda} \le \frac{d \cdot 1}{\lambda} =  2{,}2
        \implies k_{\max} = 2
        \implies \alpha_{2} \approx 65{,}38\degrees
    $
}

\variantsplitter

\addpersonalvariant{Марьям Салимова}

\tasknumber{1}%
\task{%
    Сформулируйте:
    \begin{itemize}
        \item что такое дифракция,
        \item определение дисперсии,
        \item условие интерференционного максимума для света, падающего нормально на дифракционную решётку,
        \item условия наблюдения минимума и максимума в интерферeнционной картине,
    \end{itemize}
}
\solutionspace{40pt}

\tasknumber{2}%
\task{%
    На дифракционную решётку, имеющую период $3 \cdot 10^{-4}\,\text{см}$, нормально падает монохроматическая световая волна.
    Под углом $ 25 \degrees$ наблюдается дифракционный максимум второго порядка.
    Какова длина волны падающего света?
}
\answer{%
    $
        d\sin \varphi_k = k\lambda
        \implies \lambda = \frac{d \sin \varphi_k}k
        = \frac{3 \cdot 10^{-4}\,\text{см} \cdot \sin  25 \degrees}{2} \approx 630\,\text{нм}
    $
}
\solutionspace{150pt}

\tasknumber{3}%
\task{%
    Свет с длиной волны $0{,}4\,\text{мкм}$ падает нормально на дифракционную решётку с периодом, равным $2\,\text{мкм}$.
    Под каким углом наблюдается дифракционный максимум первого порядка?
}
\answer{%
    $
        d\sin \varphi_k = k\lambda
        \implies \sin \varphi_k = \frac{k\lambda}{ d }
        = \frac{1 \cdot 0{,}4\,\text{мкм}}{2\,\text{мкм}} \approx 0{,}2 \implies \varphi_k \approx 11{,}5\degrees
    $
}
\solutionspace{150pt}

\tasknumber{4}%
\task{%
    При нормальном падении белого света на дифракционную решётку жёлтая линия ($580\,\text{нм}$)
    в спектре второго порядка видна под углом дифракции $12\degrees$.
    Определить число штрихов на $1\,\text{мм}$ длины этой решётки.
}
\answer{%
    $
        d\sin \varphi_k = k\lambda
        \implies d = \frac{k\lambda}{\sin \varphi_k}.
        \qquad N = \frac{ l }{d} = \frac{ l \sin \varphi_k}{k\lambda}
        = \frac{1\,\text{мм} \cdot \sin 12\degrees}{2 \cdot 580\,\text{нм}} \approx 179
    $
}
\solutionspace{150pt}

\tasknumber{5}%
\task{%
    Каков наибольший порядок спектра, который можно наблюдать при дифракции света
    с длиной волны $\lambda$, на дифракционной решётке с периодом $d =  2{,}7 \lambda$?
    Под каким углом наблюдается последний максимум?
}
\answer{%
    $
        d\sin \varphi_k = k\lambda
        \implies k = \frac{d\sin \varphi_k}{\lambda} \le \frac{d \cdot 1}{\lambda} =  2{,}7
        \implies k_{\max} = 2
        \implies \alpha_{2} \approx 47{,}79\degrees
    $
}

\variantsplitter

\addpersonalvariant{Юлия Шевченко}

\tasknumber{1}%
\task{%
    Сформулируйте:
    \begin{itemize}
        \item что такое дифракция,
        \item определение дисперсии,
        \item условие интерференционного максимума для света, падающего нормально на дифракционную решётку,
        \item условия когерентности 2 источников света,
    \end{itemize}
}
\solutionspace{40pt}

\tasknumber{2}%
\task{%
    На дифракционную решётку, имеющую период $2 \cdot 10^{-4}\,\text{см}$, нормально падает монохроматическая световая волна.
    Под углом $ 40 \degrees$ наблюдается дифракционный максимум четвёртого порядка.
    Какова длина волны падающего света?
}
\answer{%
    $
        d\sin \varphi_k = k\lambda
        \implies \lambda = \frac{d \sin \varphi_k}k
        = \frac{2 \cdot 10^{-4}\,\text{см} \cdot \sin  40 \degrees}{4} \approx 320\,\text{нм}
    $
}
\solutionspace{150pt}

\tasknumber{3}%
\task{%
    Свет с длиной волны $0{,}5\,\text{мкм}$ падает нормально на дифракционную решётку с периодом, равным $2\,\text{мкм}$.
    Под каким углом наблюдается дифракционный максимум первого порядка?
}
\answer{%
    $
        d\sin \varphi_k = k\lambda
        \implies \sin \varphi_k = \frac{k\lambda}{ d }
        = \frac{1 \cdot 0{,}5\,\text{мкм}}{2\,\text{мкм}} \approx 0{,}3 \implies \varphi_k \approx 14{,}5\degrees
    $
}
\solutionspace{150pt}

\tasknumber{4}%
\task{%
    При нормальном падении белого света на дифракционную решётку красная линия ($720\,\text{нм}$)
    в спектре третьего порядка видна под углом дифракции $12\degrees$.
    Определить число штрихов на $1\,\text{см}$ длины этой решётки.
}
\answer{%
    $
        d\sin \varphi_k = k\lambda
        \implies d = \frac{k\lambda}{\sin \varphi_k}.
        \qquad N = \frac{ l }{d} = \frac{ l \sin \varphi_k}{k\lambda}
        = \frac{1\,\text{см} \cdot \sin 12\degrees}{3 \cdot 720\,\text{нм}} \approx 960
    $
}
\solutionspace{150pt}

\tasknumber{5}%
\task{%
    Каков наибольший порядок спектра, который можно наблюдать при дифракции света
    с длиной волны $\lambda$, на дифракционной решётке с периодом $d =  4{,}5 \lambda$?
    Под каким углом наблюдается последний максимум?
}
\answer{%
    $
        d\sin \varphi_k = k\lambda
        \implies k = \frac{d\sin \varphi_k}{\lambda} \le \frac{d \cdot 1}{\lambda} =  4{,}5
        \implies k_{\max} = 4
        \implies \alpha_{4} \approx 62{,}73\degrees
    $
}
% autogenerated
