\setdate{14~декабря~2021}
\setclass{11«БА»}

\addpersonalvariant{Михаил Бурмистров}

\tasknumber{1}%
\task{%
    Cформулируйте принцип Гюйгенса-Френеля, запишите формулой закон преломления
    и выведите из принципа этот закон.
}
\solutionspace{80pt}

\tasknumber{2}%
\task{%
    Постройте изображения $B'F'$ и $M'{V}'$ стрелок $BF$ и $M{V}$
    в 2 плоских зеркалах соответственно (см.
    рис.
    на доске).
}
\solutionspace{180pt}

\tasknumber{3}%
\task{%
    Постройте область видимости стрелки $AM$ в плоском зеркале (см.
    рис.
    на доске).
}
\solutionspace{120pt}

\tasknumber{4}%
\task{%
    Докажите, что при повороте плоского зеркала на угол $\varphi$ вокруг оси, лежащей в плоскости зеркала
        и перпендикулярной падающему лучу, этот луч повернётся на угол $2\varphi$.
}
\solutionspace{120pt}

\tasknumber{5}%
\task{%
    Плоское зеркало вращается с угловой скоростью $0{,}25\,\frac{\text{рад}}{\text{с}}$.
    Ось вращения лежит в плоскости зеркала.
    На зеркало падает луч перпендикулярно оси вращения.
    Определите угловую скорость вращения отражённого луча.
}
\answer{%
    $\omega' = 2 \omega = 0{,}5\,\text{Гц}$
}
\solutionspace{80pt}

\tasknumber{6}%
\task{%
    Плоское зеркало приближается к стационарному предмету размером $7\,\text{см}$
    со скоростью $2\,\frac{\text{см}}{\text{с}}$.
    Определите размер изображения предмета через $4\,\text{с}$ после начала движения,
    если изначальное расстояние между зеркалом и предметом было равно $70\,\text{см}$.
}
\answer{%
    $7\,\text{см}$
}
\solutionspace{80pt}

\tasknumber{7}%
\task{%
    Предмет приближается к плоскому зеркалу со скоростью $3\,\frac{\text{см}}{\text{с}}$.
    Определите скорость изображения, приняв зеркало стационарным.
}
\answer{%
    $3\,\frac{\text{см}}{\text{с}}$
}
\solutionspace{80pt}

\tasknumber{8}%
\task{%
    Запишите своё имя (не фамилию) печатными буквами
    и постройте их изображение в 2 зеркалах: вертикальном и горизонтальном.
    Не забудьте отметить зеркала.
}
\solutionspace{120pt}

\tasknumber{9}%
\task{%
    Луч падает из {what} на стекло с показателем преломления  1{,}65 .
    Сделайте рисунок (без рисунка и отмеченных углов задача не проверяется) и определите:
    \begin{itemize}
        \item угол отражения,
        \item угол преломления,
        \item угол между падающим и отраженным лучом,
        \item угол между падающим и преломленным лучом,
        \item угол отклонения луча при преломлении,
    \end{itemize}
    если угол падения равен $22\degrees$.
}
\answer{%
    \begin{align*}
    \alpha &= 22\degrees, \\
    1 \cdot \sin \alpha &= n \sin \beta \implies \beta = \arcsin\cbr{ \frac{\sin \alpha}{ n } } \approx 13{,}12\degrees, \\
    \varphi_1 &= \alpha \approx 22\degrees, \\
    \varphi_2 &= \beta \approx 13{,}12\degrees, \\
    \varphi_3 &= 2\alpha = 44\degrees, \\
    \varphi_4 &= 180\degrees - \alpha + \beta \approx 171{,}12\degrees, \\
    \varphi_5 &= \alpha - \beta \approx 8{,}88\degrees.
    \end{align*}
}
\solutionspace{100pt}

\tasknumber{10}%
\task{%
    Определите радиус полутени на экране диска размером $D = 3\,\text{см}$ от протяжённого источника, также обладающего формой диска размером $d = 4\,\text{см}$ (см.
    рис.
    на доске, вид сбоку).
    Расстояние от источника до диска равно $l = 15\,\text{см}$, а расстояние от диска до экрана — $L = 30\,\text{см}$.
}
\answer{%
    $\cfrac{\frac d2 + \frac D2}l = \cfrac{\frac d2 + r}{l + L} \implies r = \cfrac{Dl + dL + DL}{2l} = \cfrac D2 + \cfrac{ L }{ l } \cdot \cfrac{d+D}2 \approx 8{,}5\,\text{см} \implies 2r \approx 17\,\text{см}$
}

\variantsplitter

\addpersonalvariant{Ирина Ан}

\tasknumber{1}%
\task{%
    Cформулируйте принцип Гюйгенса-Френеля, запишите формулой закон преломления
    и выведите из принципа этот закон.
}
\solutionspace{80pt}

\tasknumber{2}%
\task{%
    Постройте изображения $B'D'$ и $K'{V}'$ стрелок $BD$ и $K{V}$
    в 2 плоских зеркалах соответственно (см.
    рис.
    на доске).
}
\solutionspace{180pt}

\tasknumber{3}%
\task{%
    Постройте область видимости стрелки $BN$ в плоском зеркале (см.
    рис.
    на доске).
}
\solutionspace{120pt}

\tasknumber{4}%
\task{%
    Докажите, что изображение точечного источника света в плоском зеркале можно получить,
        «удвоив» (в векторном смысле) перпендикуляр, опущенный из источника на плоскость зеркала.
}
\solutionspace{120pt}

\tasknumber{5}%
\task{%
    Плоское зеркало вращается с угловой скоростью $0{,}14\,\frac{\text{рад}}{\text{с}}$.
    Ось вращения лежит в плоскости зеркала.
    На зеркало падает луч перпендикулярно оси вращения.
    Определите угловую скорость вращения отражённого луча.
}
\answer{%
    $\omega' = 2 \omega = 0{,}3\,\text{Гц}$
}
\solutionspace{80pt}

\tasknumber{6}%
\task{%
    Плоское зеркало приближается к стационарному предмету размером $5\,\text{см}$
    со скоростью $1\,\frac{\text{см}}{\text{с}}$.
    Определите размер изображения предмета через $2\,\text{с}$ после начала движения,
    если изначальное расстояние между зеркалом и предметом было равно $60\,\text{см}$.
}
\answer{%
    $5\,\text{см}$
}
\solutionspace{80pt}

\tasknumber{7}%
\task{%
    Предмет приближается к плоскому зеркалу со скоростью $2\,\frac{\text{см}}{\text{с}}$.
    Определите скорость изображения, приняв зеркало стационарным.
}
\answer{%
    $2\,\frac{\text{см}}{\text{с}}$
}
\solutionspace{80pt}

\tasknumber{8}%
\task{%
    Запишите своё имя (не фамилию) печатными буквами
    и постройте их изображение в 2 зеркалах: вертикальном и горизонтальном.
    Не забудьте отметить зеркала.
}
\solutionspace{120pt}

\tasknumber{9}%
\task{%
    Луч падает из {what} на стекло с показателем преломления  1{,}65 .
    Сделайте рисунок (без рисунка и отмеченных углов задача не проверяется) и определите:
    \begin{itemize}
        \item угол отражения,
        \item угол преломления,
        \item угол между падающим и отраженным лучом,
        \item угол между падающим и преломленным лучом,
        \item угол отклонения луча при преломлении,
    \end{itemize}
    если между падающим лучом и границей раздела сред равен $22\degrees$.
}
\answer{%
    \begin{align*}
    \alpha &= 68\degrees, \\
    1 \cdot \sin \alpha &= n \sin \beta \implies \beta = \arcsin\cbr{ \frac{\sin \alpha}{ n } } \approx 13{,}12\degrees, \\
    \varphi_1 &= \alpha \approx 68\degrees, \\
    \varphi_2 &= \beta \approx 13{,}12\degrees, \\
    \varphi_3 &= 2\alpha = 136\degrees, \\
    \varphi_4 &= 180\degrees - \alpha + \beta \approx 125{,}12\degrees, \\
    \varphi_5 &= \alpha - \beta \approx 54{,}88\degrees.
    \end{align*}
}
\solutionspace{100pt}

\tasknumber{10}%
\task{%
    Определите радиус полутени на экране диска размером $D = 3{,}5\,\text{см}$ от протяжённого источника, также обладающего формой диска размером $d = 3\,\text{см}$ (см.
    рис.
    на доске, вид сбоку).
    Расстояние от источника до диска равно $l = 12\,\text{см}$, а расстояние от диска до экрана — $L = 10\,\text{см}$.
}
\answer{%
    $\cfrac{\frac d2 + \frac D2}l = \cfrac{\frac d2 + r}{l + L} \implies r = \cfrac{Dl + dL + DL}{2l} = \cfrac D2 + \cfrac{ L }{ l } \cdot \cfrac{d+D}2 \approx 4{,}5\,\text{см} \implies 2r \approx 8{,}9\,\text{см}$
}

\variantsplitter

\addpersonalvariant{Софья Андрианова}

\tasknumber{1}%
\task{%
    Cформулируйте принцип Гюйгенса-Френеля, запишите формулой закон отражения
    и выведите из принципа этот закон.
}
\solutionspace{80pt}

\tasknumber{2}%
\task{%
    Постройте изображения $B'D'$ и $L'{V}'$ стрелок $BD$ и $L{V}$
    в 2 плоских зеркалах соответственно (см.
    рис.
    на доске).
}
\solutionspace{180pt}

\tasknumber{3}%
\task{%
    Постройте область видимости стрелки $DN$ в плоском зеркале (см.
    рис.
    на доске).
}
\solutionspace{120pt}

\tasknumber{4}%
\task{%
    Докажите, что при повороте плоского зеркала на угол $\varphi$ вокруг оси, лежащей в плоскости зеркала
        и перпендикулярной падающему лучу, этот луч повернётся на угол $2\varphi$.
}
\solutionspace{120pt}

\tasknumber{5}%
\task{%
    Плоское зеркало вращается с угловой скоростью $0{,}12\,\frac{\text{рад}}{\text{с}}$.
    Ось вращения лежит в плоскости зеркала.
    На зеркало падает луч перпендикулярно оси вращения.
    Определите угловую скорость вращения отражённого луча.
}
\answer{%
    $\omega' = 2 \omega = 0{,}2\,\text{Гц}$
}
\solutionspace{80pt}

\tasknumber{6}%
\task{%
    Плоское зеркало приближается к стационарному предмету размером $5\,\text{см}$
    со скоростью $2\,\frac{\text{см}}{\text{с}}$.
    Определите размер изображения предмета через $4\,\text{с}$ после начала движения,
    если изначальное расстояние между зеркалом и предметом было равно $70\,\text{см}$.
}
\answer{%
    $5\,\text{см}$
}
\solutionspace{80pt}

\tasknumber{7}%
\task{%
    Предмет отдаляется от плоского зеркала со скоростью $2\,\frac{\text{см}}{\text{с}}$.
    Определите скорость изображения, приняв зеркало стационарным.
}
\answer{%
    $2\,\frac{\text{см}}{\text{с}}$
}
\solutionspace{80pt}

\tasknumber{8}%
\task{%
    Запишите своё имя (не фамилию) печатными буквами
    и постройте их изображение в 2 зеркалах: вертикальном и горизонтальном.
    Не забудьте отметить зеркала.
}
\solutionspace{120pt}

\tasknumber{9}%
\task{%
    Луч падает из {what} на стекло с показателем преломления  1{,}55 .
    Сделайте рисунок (без рисунка и отмеченных углов задача не проверяется) и определите:
    \begin{itemize}
        \item угол отражения,
        \item угол преломления,
        \item угол между падающим и отраженным лучом,
        \item угол между падающим и преломленным лучом,
        \item угол отклонения луча при преломлении,
    \end{itemize}
    если между падающим лучом и границей раздела сред равен $65\degrees$.
}
\answer{%
    \begin{align*}
    \alpha &= 25\degrees, \\
    1 \cdot \sin \alpha &= n \sin \beta \implies \beta = \arcsin\cbr{ \frac{\sin \alpha}{ n } } \approx 35{,}78\degrees, \\
    \varphi_1 &= \alpha \approx 25\degrees, \\
    \varphi_2 &= \beta \approx 35{,}78\degrees, \\
    \varphi_3 &= 2\alpha = 50\degrees, \\
    \varphi_4 &= 180\degrees - \alpha + \beta \approx 190{,}78\degrees, \\
    \varphi_5 &= \alpha - \beta \approx -10{,}78\degrees.
    \end{align*}
}
\solutionspace{100pt}

\tasknumber{10}%
\task{%
    Определите радиус полутени на экране диска размером $D = 2{,}5\,\text{см}$ от протяжённого источника, также обладающего формой диска размером $d = 4\,\text{см}$ (см.
    рис.
    на доске, вид сбоку).
    Расстояние от источника до диска равно $l = 18\,\text{см}$, а расстояние от диска до экрана — $L = 10\,\text{см}$.
}
\answer{%
    $\cfrac{\frac d2 + \frac D2}l = \cfrac{\frac d2 + r}{l + L} \implies r = \cfrac{Dl + dL + DL}{2l} = \cfrac D2 + \cfrac{ L }{ l } \cdot \cfrac{d+D}2 \approx 3{,}1\,\text{см} \implies 2r \approx 6{,}1\,\text{см}$
}

\variantsplitter

\addpersonalvariant{Владимир Артемчук}

\tasknumber{1}%
\task{%
    Cформулируйте принцип Гюйгенса-Френеля, запишите формулой закон отражения
    и выведите из принципа этот закон.
}
\solutionspace{80pt}

\tasknumber{2}%
\task{%
    Постройте изображения $A'E'$ и $L'{V}'$ стрелок $AE$ и $L{V}$
    в 2 плоских зеркалах соответственно (см.
    рис.
    на доске).
}
\solutionspace{180pt}

\tasknumber{3}%
\task{%
    Постройте область видимости стрелки $AL$ в плоском зеркале (см.
    рис.
    на доске).
}
\solutionspace{120pt}

\tasknumber{4}%
\task{%
    Докажите, что при повороте плоского зеркала на угол $\varphi$ вокруг оси, лежащей в плоскости зеркала
        и перпендикулярной падающему лучу, этот луч повернётся на угол $2\varphi$.
}
\solutionspace{120pt}

\tasknumber{5}%
\task{%
    Плоское зеркало вращается с угловой скоростью $0{,}23\,\frac{\text{рад}}{\text{с}}$.
    Ось вращения лежит в плоскости зеркала.
    На зеркало падает луч перпендикулярно оси вращения.
    Определите угловую скорость вращения отражённого луча.
}
\answer{%
    $\omega' = 2 \omega = 0{,}5\,\text{Гц}$
}
\solutionspace{80pt}

\tasknumber{6}%
\task{%
    Плоское зеркало приближается к стационарному предмету размером $5\,\text{см}$
    со скоростью $1\,\frac{\text{см}}{\text{с}}$.
    Определите размер изображения предмета через $2\,\text{с}$ после начала движения,
    если изначальное расстояние между зеркалом и предметом было равно $60\,\text{см}$.
}
\answer{%
    $5\,\text{см}$
}
\solutionspace{80pt}

\tasknumber{7}%
\task{%
    Предмет приближается к плоскому зеркалу со скоростью $3\,\frac{\text{см}}{\text{с}}$.
    Определите скорость изображения, приняв зеркало стационарным.
}
\answer{%
    $3\,\frac{\text{см}}{\text{с}}$
}
\solutionspace{80pt}

\tasknumber{8}%
\task{%
    Запишите своё имя (не фамилию) печатными буквами
    и постройте их изображение в 2 зеркалах: вертикальном и горизонтальном.
    Не забудьте отметить зеркала.
}
\solutionspace{120pt}

\tasknumber{9}%
\task{%
    Луч падает из {what} на стекло с показателем преломления  1{,}35 .
    Сделайте рисунок (без рисунка и отмеченных углов задача не проверяется) и определите:
    \begin{itemize}
        \item угол отражения,
        \item угол преломления,
        \item угол между падающим и отраженным лучом,
        \item угол между падающим и преломленным лучом,
        \item угол отклонения луча при преломлении,
    \end{itemize}
    если между падающим лучом и границей раздела сред равен $65\degrees$.
}
\answer{%
    \begin{align*}
    \alpha &= 25\degrees, \\
    1 \cdot \sin \alpha &= n \sin \beta \implies \beta = \arcsin\cbr{ \frac{\sin \alpha}{ n } } \approx 42{,}17\degrees, \\
    \varphi_1 &= \alpha \approx 25\degrees, \\
    \varphi_2 &= \beta \approx 42{,}17\degrees, \\
    \varphi_3 &= 2\alpha = 50\degrees, \\
    \varphi_4 &= 180\degrees - \alpha + \beta \approx 197{,}17\degrees, \\
    \varphi_5 &= \alpha - \beta \approx -17{,}17\degrees.
    \end{align*}
}
\solutionspace{100pt}

\tasknumber{10}%
\task{%
    Определите диаметр полутени на экране диска размером $D = 2{,}5\,\text{см}$ от протяжённого источника, также обладающего формой диска размером $d = 2\,\text{см}$ (см.
    рис.
    на доске, вид сбоку).
    Расстояние от источника до диска равно $l = 12\,\text{см}$, а расстояние от диска до экрана — $L = 30\,\text{см}$.
}
\answer{%
    $\cfrac{\frac d2 + \frac D2}l = \cfrac{\frac d2 + r}{l + L} \implies r = \cfrac{Dl + dL + DL}{2l} = \cfrac D2 + \cfrac{ L }{ l } \cdot \cfrac{d+D}2 \approx 6{,}9\,\text{см} \implies 2r \approx 13{,}8\,\text{см}$
}

\variantsplitter

\addpersonalvariant{Софья Белянкина}

\tasknumber{1}%
\task{%
    Cформулируйте принцип Гюйгенса-Френеля, запишите формулой закон преломления
    и выведите из принципа этот закон.
}
\solutionspace{80pt}

\tasknumber{2}%
\task{%
    Постройте изображения $A'F'$ и $L'{V}'$ стрелок $AF$ и $L{V}$
    в 2 плоских зеркалах соответственно (см.
    рис.
    на доске).
}
\solutionspace{180pt}

\tasknumber{3}%
\task{%
    Постройте область видимости стрелки $AK$ в плоском зеркале (см.
    рис.
    на доске).
}
\solutionspace{120pt}

\tasknumber{4}%
\task{%
    Докажите, что при повороте плоского зеркала на угол $\varphi$ вокруг оси, лежащей в плоскости зеркала
        и перпендикулярной падающему лучу, этот луч повернётся на угол $2\varphi$.
}
\solutionspace{120pt}

\tasknumber{5}%
\task{%
    Плоское зеркало вращается с угловой скоростью $0{,}14\,\frac{\text{рад}}{\text{с}}$.
    Ось вращения лежит в плоскости зеркала.
    На зеркало падает луч перпендикулярно оси вращения.
    Определите угловую скорость вращения отражённого луча.
}
\answer{%
    $\omega' = 2 \omega = 0{,}3\,\text{Гц}$
}
\solutionspace{80pt}

\tasknumber{6}%
\task{%
    Плоское зеркало приближается к стационарному предмету размером $7\,\text{см}$
    со скоростью $2\,\frac{\text{см}}{\text{с}}$.
    Определите размер изображения предмета через $4\,\text{с}$ после начала движения,
    если изначальное расстояние между зеркалом и предметом было равно $60\,\text{см}$.
}
\answer{%
    $7\,\text{см}$
}
\solutionspace{80pt}

\tasknumber{7}%
\task{%
    Предмет приближается к плоскому зеркалу со скоростью $3\,\frac{\text{см}}{\text{с}}$.
    Определите скорость изображения, приняв зеркало стационарным.
}
\answer{%
    $3\,\frac{\text{см}}{\text{с}}$
}
\solutionspace{80pt}

\tasknumber{8}%
\task{%
    Запишите своё имя (не фамилию) печатными буквами
    и постройте их изображение в 2 зеркалах: вертикальном и горизонтальном.
    Не забудьте отметить зеркала.
}
\solutionspace{120pt}

\tasknumber{9}%
\task{%
    Луч падает из {what} на стекло с показателем преломления  1{,}65 .
    Сделайте рисунок (без рисунка и отмеченных углов задача не проверяется) и определите:
    \begin{itemize}
        \item угол отражения,
        \item угол преломления,
        \item угол между падающим и отраженным лучом,
        \item угол между падающим и преломленным лучом,
        \item угол отклонения луча при преломлении,
    \end{itemize}
    если между падающим лучом и границей раздела сред равен $22\degrees$.
}
\answer{%
    \begin{align*}
    \alpha &= 68\degrees, \\
    1 \cdot \sin \alpha &= n \sin \beta \implies \beta = \arcsin\cbr{ \frac{\sin \alpha}{ n } } \approx 13{,}12\degrees, \\
    \varphi_1 &= \alpha \approx 68\degrees, \\
    \varphi_2 &= \beta \approx 13{,}12\degrees, \\
    \varphi_3 &= 2\alpha = 136\degrees, \\
    \varphi_4 &= 180\degrees - \alpha + \beta \approx 125{,}12\degrees, \\
    \varphi_5 &= \alpha - \beta \approx 54{,}88\degrees.
    \end{align*}
}
\solutionspace{100pt}

\tasknumber{10}%
\task{%
    Определите диаметр полутени на экране диска размером $D = 3\,\text{см}$ от протяжённого источника, также обладающего формой диска размером $d = 3\,\text{см}$ (см.
    рис.
    на доске, вид сбоку).
    Расстояние от источника до диска равно $l = 18\,\text{см}$, а расстояние от диска до экрана — $L = 20\,\text{см}$.
}
\answer{%
    $\cfrac{\frac d2 + \frac D2}l = \cfrac{\frac d2 + r}{l + L} \implies r = \cfrac{Dl + dL + DL}{2l} = \cfrac D2 + \cfrac{ L }{ l } \cdot \cfrac{d+D}2 \approx 4{,}8\,\text{см} \implies 2r \approx 9{,}7\,\text{см}$
}

\variantsplitter

\addpersonalvariant{Варвара Егиазарян}

\tasknumber{1}%
\task{%
    Cформулируйте принцип Гюйгенса-Френеля, запишите формулой закон отражения
    и выведите из принципа этот закон.
}
\solutionspace{80pt}

\tasknumber{2}%
\task{%
    Постройте изображения $A'E'$ и $K'{V}'$ стрелок $AE$ и $K{V}$
    в 2 плоских зеркалах соответственно (см.
    рис.
    на доске).
}
\solutionspace{180pt}

\tasknumber{3}%
\task{%
    Постройте область видимости стрелки $BL$ в плоском зеркале (см.
    рис.
    на доске).
}
\solutionspace{120pt}

\tasknumber{4}%
\task{%
    Докажите, что изображение точечного источника света в плоском зеркале можно получить,
        «удвоив» (в векторном смысле) перпендикуляр, опущенный из источника на плоскость зеркала.
}
\solutionspace{120pt}

\tasknumber{5}%
\task{%
    Плоское зеркало вращается с угловой скоростью $0{,}25\,\frac{\text{рад}}{\text{с}}$.
    Ось вращения лежит в плоскости зеркала.
    На зеркало падает луч перпендикулярно оси вращения.
    Определите угловую скорость вращения отражённого луча.
}
\answer{%
    $\omega' = 2 \omega = 0{,}5\,\text{Гц}$
}
\solutionspace{80pt}

\tasknumber{6}%
\task{%
    Плоское зеркало приближается к стационарному предмету размером $5\,\text{см}$
    со скоростью $3\,\frac{\text{см}}{\text{с}}$.
    Определите размер изображения предмета через $2\,\text{с}$ после начала движения,
    если изначальное расстояние между зеркалом и предметом было равно $50\,\text{см}$.
}
\answer{%
    $5\,\text{см}$
}
\solutionspace{80pt}

\tasknumber{7}%
\task{%
    Предмет приближается к плоскому зеркалу со скоростью $4\,\frac{\text{см}}{\text{с}}$.
    Определите скорость изображения, приняв зеркало стационарным.
}
\answer{%
    $4\,\frac{\text{см}}{\text{с}}$
}
\solutionspace{80pt}

\tasknumber{8}%
\task{%
    Запишите своё имя (не фамилию) печатными буквами
    и постройте их изображение в 2 зеркалах: вертикальном и горизонтальном.
    Не забудьте отметить зеркала.
}
\solutionspace{120pt}

\tasknumber{9}%
\task{%
    Луч падает из {what} на стекло с показателем преломления  1{,}55 .
    Сделайте рисунок (без рисунка и отмеченных углов задача не проверяется) и определите:
    \begin{itemize}
        \item угол отражения,
        \item угол преломления,
        \item угол между падающим и отраженным лучом,
        \item угол между падающим и преломленным лучом,
        \item угол отклонения луча при преломлении,
    \end{itemize}
    если угол падения равен $65\degrees$.
}
\answer{%
    \begin{align*}
    \alpha &= 65\degrees, \\
    1 \cdot \sin \alpha &= n \sin \beta \implies \beta = \arcsin\cbr{ \frac{\sin \alpha}{ n } } \approx 35{,}78\degrees, \\
    \varphi_1 &= \alpha \approx 65\degrees, \\
    \varphi_2 &= \beta \approx 35{,}78\degrees, \\
    \varphi_3 &= 2\alpha = 130\degrees, \\
    \varphi_4 &= 180\degrees - \alpha + \beta \approx 150{,}78\degrees, \\
    \varphi_5 &= \alpha - \beta \approx 29{,}22\degrees.
    \end{align*}
}
\solutionspace{100pt}

\tasknumber{10}%
\task{%
    Определите радиус полутени на экране диска размером $D = 3\,\text{см}$ от протяжённого источника, также обладающего формой диска размером $d = 4\,\text{см}$ (см.
    рис.
    на доске, вид сбоку).
    Расстояние от источника до диска равно $l = 18\,\text{см}$, а расстояние от диска до экрана — $L = 20\,\text{см}$.
}
\answer{%
    $\cfrac{\frac d2 + \frac D2}l = \cfrac{\frac d2 + r}{l + L} \implies r = \cfrac{Dl + dL + DL}{2l} = \cfrac D2 + \cfrac{ L }{ l } \cdot \cfrac{d+D}2 \approx 5{,}4\,\text{см} \implies 2r \approx 10{,}8\,\text{см}$
}

\variantsplitter

\addpersonalvariant{Владислав Емелин}

\tasknumber{1}%
\task{%
    Cформулируйте принцип Гюйгенса-Френеля, запишите формулой закон отражения
    и выведите из принципа этот закон.
}
\solutionspace{80pt}

\tasknumber{2}%
\task{%
    Постройте изображения $B'D'$ и $L'{V}'$ стрелок $BD$ и $L{V}$
    в 2 плоских зеркалах соответственно (см.
    рис.
    на доске).
}
\solutionspace{180pt}

\tasknumber{3}%
\task{%
    Постройте область видимости стрелки $AL$ в плоском зеркале (см.
    рис.
    на доске).
}
\solutionspace{120pt}

\tasknumber{4}%
\task{%
    Докажите, что при повороте плоского зеркала на угол $\varphi$ вокруг оси, лежащей в плоскости зеркала
        и перпендикулярной падающему лучу, этот луч повернётся на угол $2\varphi$.
}
\solutionspace{120pt}

\tasknumber{5}%
\task{%
    Плоское зеркало вращается с угловой скоростью $0{,}17\,\frac{\text{рад}}{\text{с}}$.
    Ось вращения лежит в плоскости зеркала.
    На зеркало падает луч перпендикулярно оси вращения.
    Определите угловую скорость вращения отражённого луча.
}
\answer{%
    $\omega' = 2 \omega = 0{,}3\,\text{Гц}$
}
\solutionspace{80pt}

\tasknumber{6}%
\task{%
    Плоское зеркало приближается к стационарному предмету размером $6\,\text{см}$
    со скоростью $2\,\frac{\text{см}}{\text{с}}$.
    Определите размер изображения предмета через $5\,\text{с}$ после начала движения,
    если изначальное расстояние между зеркалом и предметом было равно $50\,\text{см}$.
}
\answer{%
    $6\,\text{см}$
}
\solutionspace{80pt}

\tasknumber{7}%
\task{%
    Предмет отдаляется от плоского зеркала со скоростью $4\,\frac{\text{см}}{\text{с}}$.
    Определите скорость изображения, приняв зеркало стационарным.
}
\answer{%
    $4\,\frac{\text{см}}{\text{с}}$
}
\solutionspace{80pt}

\tasknumber{8}%
\task{%
    Запишите своё имя (не фамилию) печатными буквами
    и постройте их изображение в 2 зеркалах: вертикальном и горизонтальном.
    Не забудьте отметить зеркала.
}
\solutionspace{120pt}

\tasknumber{9}%
\task{%
    Луч падает из {what} на стекло с показателем преломления  1{,}55 .
    Сделайте рисунок (без рисунка и отмеченных углов задача не проверяется) и определите:
    \begin{itemize}
        \item угол отражения,
        \item угол преломления,
        \item угол между падающим и отраженным лучом,
        \item угол между падающим и преломленным лучом,
        \item угол отклонения луча при преломлении,
    \end{itemize}
    если между падающим лучом и границей раздела сред равен $65\degrees$.
}
\answer{%
    \begin{align*}
    \alpha &= 25\degrees, \\
    1 \cdot \sin \alpha &= n \sin \beta \implies \beta = \arcsin\cbr{ \frac{\sin \alpha}{ n } } \approx 35{,}78\degrees, \\
    \varphi_1 &= \alpha \approx 25\degrees, \\
    \varphi_2 &= \beta \approx 35{,}78\degrees, \\
    \varphi_3 &= 2\alpha = 50\degrees, \\
    \varphi_4 &= 180\degrees - \alpha + \beta \approx 190{,}78\degrees, \\
    \varphi_5 &= \alpha - \beta \approx -10{,}78\degrees.
    \end{align*}
}
\solutionspace{100pt}

\tasknumber{10}%
\task{%
    Определите радиус полутени на экране диска размером $D = 2{,}5\,\text{см}$ от протяжённого источника, также обладающего формой диска размером $d = 2\,\text{см}$ (см.
    рис.
    на доске, вид сбоку).
    Расстояние от источника до диска равно $l = 15\,\text{см}$, а расстояние от диска до экрана — $L = 10\,\text{см}$.
}
\answer{%
    $\cfrac{\frac d2 + \frac D2}l = \cfrac{\frac d2 + r}{l + L} \implies r = \cfrac{Dl + dL + DL}{2l} = \cfrac D2 + \cfrac{ L }{ l } \cdot \cfrac{d+D}2 \approx 2{,}8\,\text{см} \implies 2r \approx 5{,}5\,\text{см}$
}

\variantsplitter

\addpersonalvariant{Артём Жичин}

\tasknumber{1}%
\task{%
    Cформулируйте принцип Гюйгенса-Френеля, запишите формулой закон отражения
    и выведите из принципа этот закон.
}
\solutionspace{80pt}

\tasknumber{2}%
\task{%
    Постройте изображения $C'D'$ и $L'{V}'$ стрелок $CD$ и $L{V}$
    в 2 плоских зеркалах соответственно (см.
    рис.
    на доске).
}
\solutionspace{180pt}

\tasknumber{3}%
\task{%
    Постройте область видимости стрелки $DM$ в плоском зеркале (см.
    рис.
    на доске).
}
\solutionspace{120pt}

\tasknumber{4}%
\task{%
    Докажите, что при повороте плоского зеркала на угол $\varphi$ вокруг оси, лежащей в плоскости зеркала
        и перпендикулярной падающему лучу, этот луч повернётся на угол $2\varphi$.
}
\solutionspace{120pt}

\tasknumber{5}%
\task{%
    Плоское зеркало вращается с угловой скоростью $0{,}15\,\frac{\text{рад}}{\text{с}}$.
    Ось вращения лежит в плоскости зеркала.
    На зеркало падает луч перпендикулярно оси вращения.
    Определите угловую скорость вращения отражённого луча.
}
\answer{%
    $\omega' = 2 \omega = 0{,}3\,\text{Гц}$
}
\solutionspace{80pt}

\tasknumber{6}%
\task{%
    Плоское зеркало приближается к стационарному предмету размером $5\,\text{см}$
    со скоростью $1\,\frac{\text{см}}{\text{с}}$.
    Определите размер изображения предмета через $3\,\text{с}$ после начала движения,
    если изначальное расстояние между зеркалом и предметом было равно $60\,\text{см}$.
}
\answer{%
    $5\,\text{см}$
}
\solutionspace{80pt}

\tasknumber{7}%
\task{%
    Предмет отдаляется от плоского зеркала со скоростью $3\,\frac{\text{см}}{\text{с}}$.
    Определите скорость изображения, приняв зеркало стационарным.
}
\answer{%
    $3\,\frac{\text{см}}{\text{с}}$
}
\solutionspace{80pt}

\tasknumber{8}%
\task{%
    Запишите своё имя (не фамилию) печатными буквами
    и постройте их изображение в 2 зеркалах: вертикальном и горизонтальном.
    Не забудьте отметить зеркала.
}
\solutionspace{120pt}

\tasknumber{9}%
\task{%
    Луч падает из {what} на стекло с показателем преломления  1{,}65 .
    Сделайте рисунок (без рисунка и отмеченных углов задача не проверяется) и определите:
    \begin{itemize}
        \item угол отражения,
        \item угол преломления,
        \item угол между падающим и отраженным лучом,
        \item угол между падающим и преломленным лучом,
        \item угол отклонения луча при преломлении,
    \end{itemize}
    если между падающим лучом и границей раздела сред равен $22\degrees$.
}
\answer{%
    \begin{align*}
    \alpha &= 68\degrees, \\
    1 \cdot \sin \alpha &= n \sin \beta \implies \beta = \arcsin\cbr{ \frac{\sin \alpha}{ n } } \approx 13{,}12\degrees, \\
    \varphi_1 &= \alpha \approx 68\degrees, \\
    \varphi_2 &= \beta \approx 13{,}12\degrees, \\
    \varphi_3 &= 2\alpha = 136\degrees, \\
    \varphi_4 &= 180\degrees - \alpha + \beta \approx 125{,}12\degrees, \\
    \varphi_5 &= \alpha - \beta \approx 54{,}88\degrees.
    \end{align*}
}
\solutionspace{100pt}

\tasknumber{10}%
\task{%
    Определите диаметр полутени на экране диска размером $D = 3\,\text{см}$ от протяжённого источника, также обладающего формой диска размером $d = 3\,\text{см}$ (см.
    рис.
    на доске, вид сбоку).
    Расстояние от источника до диска равно $l = 15\,\text{см}$, а расстояние от диска до экрана — $L = 10\,\text{см}$.
}
\answer{%
    $\cfrac{\frac d2 + \frac D2}l = \cfrac{\frac d2 + r}{l + L} \implies r = \cfrac{Dl + dL + DL}{2l} = \cfrac D2 + \cfrac{ L }{ l } \cdot \cfrac{d+D}2 \approx 3{,}5\,\text{см} \implies 2r \approx 7\,\text{см}$
}

\variantsplitter

\addpersonalvariant{Дарья Кошман}

\tasknumber{1}%
\task{%
    Cформулируйте принцип Гюйгенса-Френеля, запишите формулой закон отражения
    и выведите из принципа этот закон.
}
\solutionspace{80pt}

\tasknumber{2}%
\task{%
    Постройте изображения $B'D'$ и $M'{V}'$ стрелок $BD$ и $M{V}$
    в 2 плоских зеркалах соответственно (см.
    рис.
    на доске).
}
\solutionspace{180pt}

\tasknumber{3}%
\task{%
    Постройте область видимости стрелки $CM$ в плоском зеркале (см.
    рис.
    на доске).
}
\solutionspace{120pt}

\tasknumber{4}%
\task{%
    Докажите, что изображение точечного источника света в плоском зеркале можно получить,
        «удвоив» (в векторном смысле) перпендикуляр, опущенный из источника на плоскость зеркала.
}
\solutionspace{120pt}

\tasknumber{5}%
\task{%
    Плоское зеркало вращается с угловой скоростью $0{,}14\,\frac{\text{рад}}{\text{с}}$.
    Ось вращения лежит в плоскости зеркала.
    На зеркало падает луч перпендикулярно оси вращения.
    Определите угловую скорость вращения отражённого луча.
}
\answer{%
    $\omega' = 2 \omega = 0{,}3\,\text{Гц}$
}
\solutionspace{80pt}

\tasknumber{6}%
\task{%
    Плоское зеркало приближается к стационарному предмету размером $7\,\text{см}$
    со скоростью $2\,\frac{\text{см}}{\text{с}}$.
    Определите размер изображения предмета через $3\,\text{с}$ после начала движения,
    если изначальное расстояние между зеркалом и предметом было равно $50\,\text{см}$.
}
\answer{%
    $7\,\text{см}$
}
\solutionspace{80pt}

\tasknumber{7}%
\task{%
    Предмет приближается к плоскому зеркалу со скоростью $2\,\frac{\text{см}}{\text{с}}$.
    Определите скорость изображения, приняв зеркало стационарным.
}
\answer{%
    $2\,\frac{\text{см}}{\text{с}}$
}
\solutionspace{80pt}

\tasknumber{8}%
\task{%
    Запишите своё имя (не фамилию) печатными буквами
    и постройте их изображение в 2 зеркалах: вертикальном и горизонтальном.
    Не забудьте отметить зеркала.
}
\solutionspace{120pt}

\tasknumber{9}%
\task{%
    Луч падает из {what} на стекло с показателем преломления  1{,}65 .
    Сделайте рисунок (без рисунка и отмеченных углов задача не проверяется) и определите:
    \begin{itemize}
        \item угол отражения,
        \item угол преломления,
        \item угол между падающим и отраженным лучом,
        \item угол между падающим и преломленным лучом,
        \item угол отклонения луча при преломлении,
    \end{itemize}
    если угол падения равен $50\degrees$.
}
\answer{%
    \begin{align*}
    \alpha &= 50\degrees, \\
    1 \cdot \sin \alpha &= n \sin \beta \implies \beta = \arcsin\cbr{ \frac{\sin \alpha}{ n } } \approx 27{,}66\degrees, \\
    \varphi_1 &= \alpha \approx 50\degrees, \\
    \varphi_2 &= \beta \approx 27{,}66\degrees, \\
    \varphi_3 &= 2\alpha = 100\degrees, \\
    \varphi_4 &= 180\degrees - \alpha + \beta \approx 157{,}66\degrees, \\
    \varphi_5 &= \alpha - \beta \approx 22{,}34\degrees.
    \end{align*}
}
\solutionspace{100pt}

\tasknumber{10}%
\task{%
    Определите диаметр полутени на экране диска размером $D = 4\,\text{см}$ от протяжённого источника, также обладающего формой диска размером $d = 4\,\text{см}$ (см.
    рис.
    на доске, вид сбоку).
    Расстояние от источника до диска равно $l = 15\,\text{см}$, а расстояние от диска до экрана — $L = 10\,\text{см}$.
}
\answer{%
    $\cfrac{\frac d2 + \frac D2}l = \cfrac{\frac d2 + r}{l + L} \implies r = \cfrac{Dl + dL + DL}{2l} = \cfrac D2 + \cfrac{ L }{ l } \cdot \cfrac{d+D}2 \approx 4{,}7\,\text{см} \implies 2r \approx 9{,}3\,\text{см}$
}

\variantsplitter

\addpersonalvariant{Анна Кузьмичёва}

\tasknumber{1}%
\task{%
    Cформулируйте принцип Гюйгенса-Френеля, запишите формулой закон преломления
    и выведите из принципа этот закон.
}
\solutionspace{80pt}

\tasknumber{2}%
\task{%
    Постройте изображения $A'D'$ и $M'{V}'$ стрелок $AD$ и $M{V}$
    в 2 плоских зеркалах соответственно (см.
    рис.
    на доске).
}
\solutionspace{180pt}

\tasknumber{3}%
\task{%
    Постройте область видимости стрелки $AN$ в плоском зеркале (см.
    рис.
    на доске).
}
\solutionspace{120pt}

\tasknumber{4}%
\task{%
    Докажите, что при повороте плоского зеркала на угол $\varphi$ вокруг оси, лежащей в плоскости зеркала
        и перпендикулярной падающему лучу, этот луч повернётся на угол $2\varphi$.
}
\solutionspace{120pt}

\tasknumber{5}%
\task{%
    Плоское зеркало вращается с угловой скоростью $0{,}19\,\frac{\text{рад}}{\text{с}}$.
    Ось вращения лежит в плоскости зеркала.
    На зеркало падает луч перпендикулярно оси вращения.
    Определите угловую скорость вращения отражённого луча.
}
\answer{%
    $\omega' = 2 \omega = 0{,}4\,\text{Гц}$
}
\solutionspace{80pt}

\tasknumber{6}%
\task{%
    Плоское зеркало приближается к стационарному предмету размером $7\,\text{см}$
    со скоростью $1\,\frac{\text{см}}{\text{с}}$.
    Определите размер изображения предмета через $5\,\text{с}$ после начала движения,
    если изначальное расстояние между зеркалом и предметом было равно $60\,\text{см}$.
}
\answer{%
    $7\,\text{см}$
}
\solutionspace{80pt}

\tasknumber{7}%
\task{%
    Предмет отдаляется от плоского зеркала со скоростью $4\,\frac{\text{см}}{\text{с}}$.
    Определите скорость изображения, приняв зеркало стационарным.
}
\answer{%
    $4\,\frac{\text{см}}{\text{с}}$
}
\solutionspace{80pt}

\tasknumber{8}%
\task{%
    Запишите своё имя (не фамилию) печатными буквами
    и постройте их изображение в 2 зеркалах: вертикальном и горизонтальном.
    Не забудьте отметить зеркала.
}
\solutionspace{120pt}

\tasknumber{9}%
\task{%
    Луч падает из {what} на стекло с показателем преломления  1{,}35 .
    Сделайте рисунок (без рисунка и отмеченных углов задача не проверяется) и определите:
    \begin{itemize}
        \item угол отражения,
        \item угол преломления,
        \item угол между падающим и отраженным лучом,
        \item угол между падающим и преломленным лучом,
        \item угол отклонения луча при преломлении,
    \end{itemize}
    если между падающим лучом и границей раздела сред равен $22\degrees$.
}
\answer{%
    \begin{align*}
    \alpha &= 68\degrees, \\
    1 \cdot \sin \alpha &= n \sin \beta \implies \beta = \arcsin\cbr{ \frac{\sin \alpha}{ n } } \approx 16{,}11\degrees, \\
    \varphi_1 &= \alpha \approx 68\degrees, \\
    \varphi_2 &= \beta \approx 16{,}11\degrees, \\
    \varphi_3 &= 2\alpha = 136\degrees, \\
    \varphi_4 &= 180\degrees - \alpha + \beta \approx 128{,}11\degrees, \\
    \varphi_5 &= \alpha - \beta \approx 51{,}89\degrees.
    \end{align*}
}
\solutionspace{100pt}

\tasknumber{10}%
\task{%
    Определите радиус полутени на экране диска размером $D = 3{,}5\,\text{см}$ от протяжённого источника, также обладающего формой диска размером $d = 2\,\text{см}$ (см.
    рис.
    на доске, вид сбоку).
    Расстояние от источника до диска равно $l = 15\,\text{см}$, а расстояние от диска до экрана — $L = 30\,\text{см}$.
}
\answer{%
    $\cfrac{\frac d2 + \frac D2}l = \cfrac{\frac d2 + r}{l + L} \implies r = \cfrac{Dl + dL + DL}{2l} = \cfrac D2 + \cfrac{ L }{ l } \cdot \cfrac{d+D}2 \approx 7{,}2\,\text{см} \implies 2r \approx 14{,}5\,\text{см}$
}

\variantsplitter

\addpersonalvariant{Алёна Куприянова}

\tasknumber{1}%
\task{%
    Cформулируйте принцип Гюйгенса-Френеля, запишите формулой закон отражения
    и выведите из принципа этот закон.
}
\solutionspace{80pt}

\tasknumber{2}%
\task{%
    Постройте изображения $A'E'$ и $K'{V}'$ стрелок $AE$ и $K{V}$
    в 2 плоских зеркалах соответственно (см.
    рис.
    на доске).
}
\solutionspace{180pt}

\tasknumber{3}%
\task{%
    Постройте область видимости стрелки $AM$ в плоском зеркале (см.
    рис.
    на доске).
}
\solutionspace{120pt}

\tasknumber{4}%
\task{%
    Докажите, что изображение точечного источника света в плоском зеркале можно получить,
        «удвоив» (в векторном смысле) перпендикуляр, опущенный из источника на плоскость зеркала.
}
\solutionspace{120pt}

\tasknumber{5}%
\task{%
    Плоское зеркало вращается с угловой скоростью $0{,}12\,\frac{\text{рад}}{\text{с}}$.
    Ось вращения лежит в плоскости зеркала.
    На зеркало падает луч перпендикулярно оси вращения.
    Определите угловую скорость вращения отражённого луча.
}
\answer{%
    $\omega' = 2 \omega = 0{,}2\,\text{Гц}$
}
\solutionspace{80pt}

\tasknumber{6}%
\task{%
    Плоское зеркало приближается к стационарному предмету размером $7\,\text{см}$
    со скоростью $2\,\frac{\text{см}}{\text{с}}$.
    Определите размер изображения предмета через $4\,\text{с}$ после начала движения,
    если изначальное расстояние между зеркалом и предметом было равно $50\,\text{см}$.
}
\answer{%
    $7\,\text{см}$
}
\solutionspace{80pt}

\tasknumber{7}%
\task{%
    Предмет приближается к плоскому зеркалу со скоростью $3\,\frac{\text{см}}{\text{с}}$.
    Определите скорость изображения, приняв зеркало стационарным.
}
\answer{%
    $3\,\frac{\text{см}}{\text{с}}$
}
\solutionspace{80pt}

\tasknumber{8}%
\task{%
    Запишите своё имя (не фамилию) печатными буквами
    и постройте их изображение в 2 зеркалах: вертикальном и горизонтальном.
    Не забудьте отметить зеркала.
}
\solutionspace{120pt}

\tasknumber{9}%
\task{%
    Луч падает из {what} на стекло с показателем преломления  1{,}35 .
    Сделайте рисунок (без рисунка и отмеченных углов задача не проверяется) и определите:
    \begin{itemize}
        \item угол отражения,
        \item угол преломления,
        \item угол между падающим и отраженным лучом,
        \item угол между падающим и преломленным лучом,
        \item угол отклонения луча при преломлении,
    \end{itemize}
    если между падающим лучом и границей раздела сред равен $65\degrees$.
}
\answer{%
    \begin{align*}
    \alpha &= 25\degrees, \\
    1 \cdot \sin \alpha &= n \sin \beta \implies \beta = \arcsin\cbr{ \frac{\sin \alpha}{ n } } \approx 42{,}17\degrees, \\
    \varphi_1 &= \alpha \approx 25\degrees, \\
    \varphi_2 &= \beta \approx 42{,}17\degrees, \\
    \varphi_3 &= 2\alpha = 50\degrees, \\
    \varphi_4 &= 180\degrees - \alpha + \beta \approx 197{,}17\degrees, \\
    \varphi_5 &= \alpha - \beta \approx -17{,}17\degrees.
    \end{align*}
}
\solutionspace{100pt}

\tasknumber{10}%
\task{%
    Определите диаметр полутени на экране диска размером $D = 3\,\text{см}$ от протяжённого источника, также обладающего формой диска размером $d = 3\,\text{см}$ (см.
    рис.
    на доске, вид сбоку).
    Расстояние от источника до диска равно $l = 12\,\text{см}$, а расстояние от диска до экрана — $L = 10\,\text{см}$.
}
\answer{%
    $\cfrac{\frac d2 + \frac D2}l = \cfrac{\frac d2 + r}{l + L} \implies r = \cfrac{Dl + dL + DL}{2l} = \cfrac D2 + \cfrac{ L }{ l } \cdot \cfrac{d+D}2 \approx 4\,\text{см} \implies 2r \approx 8\,\text{см}$
}

\variantsplitter

\addpersonalvariant{Ярослав Лавровский}

\tasknumber{1}%
\task{%
    Cформулируйте принцип Гюйгенса-Френеля, запишите формулой закон отражения
    и выведите из принципа этот закон.
}
\solutionspace{80pt}

\tasknumber{2}%
\task{%
    Постройте изображения $B'D'$ и $K'{V}'$ стрелок $BD$ и $K{V}$
    в 2 плоских зеркалах соответственно (см.
    рис.
    на доске).
}
\solutionspace{180pt}

\tasknumber{3}%
\task{%
    Постройте область видимости стрелки $CL$ в плоском зеркале (см.
    рис.
    на доске).
}
\solutionspace{120pt}

\tasknumber{4}%
\task{%
    Докажите, что при повороте плоского зеркала на угол $\varphi$ вокруг оси, лежащей в плоскости зеркала
        и перпендикулярной падающему лучу, этот луч повернётся на угол $2\varphi$.
}
\solutionspace{120pt}

\tasknumber{5}%
\task{%
    Плоское зеркало вращается с угловой скоростью $0{,}23\,\frac{\text{рад}}{\text{с}}$.
    Ось вращения лежит в плоскости зеркала.
    На зеркало падает луч перпендикулярно оси вращения.
    Определите угловую скорость вращения отражённого луча.
}
\answer{%
    $\omega' = 2 \omega = 0{,}5\,\text{Гц}$
}
\solutionspace{80pt}

\tasknumber{6}%
\task{%
    Плоское зеркало приближается к стационарному предмету размером $5\,\text{см}$
    со скоростью $3\,\frac{\text{см}}{\text{с}}$.
    Определите размер изображения предмета через $5\,\text{с}$ после начала движения,
    если изначальное расстояние между зеркалом и предметом было равно $70\,\text{см}$.
}
\answer{%
    $5\,\text{см}$
}
\solutionspace{80pt}

\tasknumber{7}%
\task{%
    Предмет приближается к плоскому зеркалу со скоростью $3\,\frac{\text{см}}{\text{с}}$.
    Определите скорость изображения, приняв зеркало стационарным.
}
\answer{%
    $3\,\frac{\text{см}}{\text{с}}$
}
\solutionspace{80pt}

\tasknumber{8}%
\task{%
    Запишите своё имя (не фамилию) печатными буквами
    и постройте их изображение в 2 зеркалах: вертикальном и горизонтальном.
    Не забудьте отметить зеркала.
}
\solutionspace{120pt}

\tasknumber{9}%
\task{%
    Луч падает из {what} на стекло с показателем преломления  1{,}35 .
    Сделайте рисунок (без рисунка и отмеченных углов задача не проверяется) и определите:
    \begin{itemize}
        \item угол отражения,
        \item угол преломления,
        \item угол между падающим и отраженным лучом,
        \item угол между падающим и преломленным лучом,
        \item угол отклонения луча при преломлении,
    \end{itemize}
    если между падающим лучом и границей раздела сред равен $50\degrees$.
}
\answer{%
    \begin{align*}
    \alpha &= 40\degrees, \\
    1 \cdot \sin \alpha &= n \sin \beta \implies \beta = \arcsin\cbr{ \frac{\sin \alpha}{ n } } \approx 34{,}57\degrees, \\
    \varphi_1 &= \alpha \approx 40\degrees, \\
    \varphi_2 &= \beta \approx 34{,}57\degrees, \\
    \varphi_3 &= 2\alpha = 80\degrees, \\
    \varphi_4 &= 180\degrees - \alpha + \beta \approx 174{,}57\degrees, \\
    \varphi_5 &= \alpha - \beta \approx 5{,}43\degrees.
    \end{align*}
}
\solutionspace{100pt}

\tasknumber{10}%
\task{%
    Определите радиус полутени на экране диска размером $D = 4\,\text{см}$ от протяжённого источника, также обладающего формой диска размером $d = 2\,\text{см}$ (см.
    рис.
    на доске, вид сбоку).
    Расстояние от источника до диска равно $l = 12\,\text{см}$, а расстояние от диска до экрана — $L = 10\,\text{см}$.
}
\answer{%
    $\cfrac{\frac d2 + \frac D2}l = \cfrac{\frac d2 + r}{l + L} \implies r = \cfrac{Dl + dL + DL}{2l} = \cfrac D2 + \cfrac{ L }{ l } \cdot \cfrac{d+D}2 \approx 4{,}5\,\text{см} \implies 2r \approx 9\,\text{см}$
}

\variantsplitter

\addpersonalvariant{Анастасия Ламанова}

\tasknumber{1}%
\task{%
    Cформулируйте принцип Гюйгенса-Френеля, запишите формулой закон отражения
    и выведите из принципа этот закон.
}
\solutionspace{80pt}

\tasknumber{2}%
\task{%
    Постройте изображения $A'E'$ и $M'{V}'$ стрелок $AE$ и $M{V}$
    в 2 плоских зеркалах соответственно (см.
    рис.
    на доске).
}
\solutionspace{180pt}

\tasknumber{3}%
\task{%
    Постройте область видимости стрелки $DL$ в плоском зеркале (см.
    рис.
    на доске).
}
\solutionspace{120pt}

\tasknumber{4}%
\task{%
    Докажите, что изображение точечного источника света в плоском зеркале можно получить,
        «удвоив» (в векторном смысле) перпендикуляр, опущенный из источника на плоскость зеркала.
}
\solutionspace{120pt}

\tasknumber{5}%
\task{%
    Плоское зеркало вращается с угловой скоростью $0{,}16\,\frac{\text{рад}}{\text{с}}$.
    Ось вращения лежит в плоскости зеркала.
    На зеркало падает луч перпендикулярно оси вращения.
    Определите угловую скорость вращения отражённого луча.
}
\answer{%
    $\omega' = 2 \omega = 0{,}3\,\text{Гц}$
}
\solutionspace{80pt}

\tasknumber{6}%
\task{%
    Плоское зеркало приближается к стационарному предмету размером $7\,\text{см}$
    со скоростью $2\,\frac{\text{см}}{\text{с}}$.
    Определите размер изображения предмета через $2\,\text{с}$ после начала движения,
    если изначальное расстояние между зеркалом и предметом было равно $60\,\text{см}$.
}
\answer{%
    $7\,\text{см}$
}
\solutionspace{80pt}

\tasknumber{7}%
\task{%
    Предмет отдаляется от плоского зеркала со скоростью $2\,\frac{\text{см}}{\text{с}}$.
    Определите скорость изображения, приняв зеркало стационарным.
}
\answer{%
    $2\,\frac{\text{см}}{\text{с}}$
}
\solutionspace{80pt}

\tasknumber{8}%
\task{%
    Запишите своё имя (не фамилию) печатными буквами
    и постройте их изображение в 2 зеркалах: вертикальном и горизонтальном.
    Не забудьте отметить зеркала.
}
\solutionspace{120pt}

\tasknumber{9}%
\task{%
    Луч падает из {what} на стекло с показателем преломления  1{,}35 .
    Сделайте рисунок (без рисунка и отмеченных углов задача не проверяется) и определите:
    \begin{itemize}
        \item угол отражения,
        \item угол преломления,
        \item угол между падающим и отраженным лучом,
        \item угол между падающим и преломленным лучом,
        \item угол отклонения луча при преломлении,
    \end{itemize}
    если угол падения равен $22\degrees$.
}
\answer{%
    \begin{align*}
    \alpha &= 22\degrees, \\
    1 \cdot \sin \alpha &= n \sin \beta \implies \beta = \arcsin\cbr{ \frac{\sin \alpha}{ n } } \approx 16{,}11\degrees, \\
    \varphi_1 &= \alpha \approx 22\degrees, \\
    \varphi_2 &= \beta \approx 16{,}11\degrees, \\
    \varphi_3 &= 2\alpha = 44\degrees, \\
    \varphi_4 &= 180\degrees - \alpha + \beta \approx 174{,}11\degrees, \\
    \varphi_5 &= \alpha - \beta \approx 5{,}89\degrees.
    \end{align*}
}
\solutionspace{100pt}

\tasknumber{10}%
\task{%
    Определите радиус полутени на экране диска размером $D = 3{,}5\,\text{см}$ от протяжённого источника, также обладающего формой диска размером $d = 4\,\text{см}$ (см.
    рис.
    на доске, вид сбоку).
    Расстояние от источника до диска равно $l = 12\,\text{см}$, а расстояние от диска до экрана — $L = 30\,\text{см}$.
}
\answer{%
    $\cfrac{\frac d2 + \frac D2}l = \cfrac{\frac d2 + r}{l + L} \implies r = \cfrac{Dl + dL + DL}{2l} = \cfrac D2 + \cfrac{ L }{ l } \cdot \cfrac{d+D}2 \approx 11{,}1\,\text{см} \implies 2r \approx 22{,}2\,\text{см}$
}

\variantsplitter

\addpersonalvariant{Виктория Легонькова}

\tasknumber{1}%
\task{%
    Cформулируйте принцип Гюйгенса-Френеля, запишите формулой закон преломления
    и выведите из принципа этот закон.
}
\solutionspace{80pt}

\tasknumber{2}%
\task{%
    Постройте изображения $C'F'$ и $K'{V}'$ стрелок $CF$ и $K{V}$
    в 2 плоских зеркалах соответственно (см.
    рис.
    на доске).
}
\solutionspace{180pt}

\tasknumber{3}%
\task{%
    Постройте область видимости стрелки $BL$ в плоском зеркале (см.
    рис.
    на доске).
}
\solutionspace{120pt}

\tasknumber{4}%
\task{%
    Докажите, что изображение точечного источника света в плоском зеркале можно получить,
        «удвоив» (в векторном смысле) перпендикуляр, опущенный из источника на плоскость зеркала.
}
\solutionspace{120pt}

\tasknumber{5}%
\task{%
    Плоское зеркало вращается с угловой скоростью $0{,}28\,\frac{\text{рад}}{\text{с}}$.
    Ось вращения лежит в плоскости зеркала.
    На зеркало падает луч перпендикулярно оси вращения.
    Определите угловую скорость вращения отражённого луча.
}
\answer{%
    $\omega' = 2 \omega = 0{,}6\,\text{Гц}$
}
\solutionspace{80pt}

\tasknumber{6}%
\task{%
    Плоское зеркало приближается к стационарному предмету размером $6\,\text{см}$
    со скоростью $2\,\frac{\text{см}}{\text{с}}$.
    Определите размер изображения предмета через $3\,\text{с}$ после начала движения,
    если изначальное расстояние между зеркалом и предметом было равно $70\,\text{см}$.
}
\answer{%
    $6\,\text{см}$
}
\solutionspace{80pt}

\tasknumber{7}%
\task{%
    Предмет отдаляется от плоского зеркала со скоростью $3\,\frac{\text{см}}{\text{с}}$.
    Определите скорость изображения, приняв зеркало стационарным.
}
\answer{%
    $3\,\frac{\text{см}}{\text{с}}$
}
\solutionspace{80pt}

\tasknumber{8}%
\task{%
    Запишите своё имя (не фамилию) печатными буквами
    и постройте их изображение в 2 зеркалах: вертикальном и горизонтальном.
    Не забудьте отметить зеркала.
}
\solutionspace{120pt}

\tasknumber{9}%
\task{%
    Луч падает из {what} на стекло с показателем преломления  1{,}35 .
    Сделайте рисунок (без рисунка и отмеченных углов задача не проверяется) и определите:
    \begin{itemize}
        \item угол отражения,
        \item угол преломления,
        \item угол между падающим и отраженным лучом,
        \item угол между падающим и преломленным лучом,
        \item угол отклонения луча при преломлении,
    \end{itemize}
    если угол падения равен $65\degrees$.
}
\answer{%
    \begin{align*}
    \alpha &= 65\degrees, \\
    1 \cdot \sin \alpha &= n \sin \beta \implies \beta = \arcsin\cbr{ \frac{\sin \alpha}{ n } } \approx 42{,}17\degrees, \\
    \varphi_1 &= \alpha \approx 65\degrees, \\
    \varphi_2 &= \beta \approx 42{,}17\degrees, \\
    \varphi_3 &= 2\alpha = 130\degrees, \\
    \varphi_4 &= 180\degrees - \alpha + \beta \approx 157{,}17\degrees, \\
    \varphi_5 &= \alpha - \beta \approx 22{,}83\degrees.
    \end{align*}
}
\solutionspace{100pt}

\tasknumber{10}%
\task{%
    Определите радиус полутени на экране диска размером $D = 2{,}5\,\text{см}$ от протяжённого источника, также обладающего формой диска размером $d = 4\,\text{см}$ (см.
    рис.
    на доске, вид сбоку).
    Расстояние от источника до диска равно $l = 15\,\text{см}$, а расстояние от диска до экрана — $L = 20\,\text{см}$.
}
\answer{%
    $\cfrac{\frac d2 + \frac D2}l = \cfrac{\frac d2 + r}{l + L} \implies r = \cfrac{Dl + dL + DL}{2l} = \cfrac D2 + \cfrac{ L }{ l } \cdot \cfrac{d+D}2 \approx 5{,}6\,\text{см} \implies 2r \approx 11{,}2\,\text{см}$
}

\variantsplitter

\addpersonalvariant{Семён Мартынов}

\tasknumber{1}%
\task{%
    Cформулируйте принцип Гюйгенса-Френеля, запишите формулой закон преломления
    и выведите из принципа этот закон.
}
\solutionspace{80pt}

\tasknumber{2}%
\task{%
    Постройте изображения $C'F'$ и $L'{V}'$ стрелок $CF$ и $L{V}$
    в 2 плоских зеркалах соответственно (см.
    рис.
    на доске).
}
\solutionspace{180pt}

\tasknumber{3}%
\task{%
    Постройте область видимости стрелки $CN$ в плоском зеркале (см.
    рис.
    на доске).
}
\solutionspace{120pt}

\tasknumber{4}%
\task{%
    Докажите, что изображение точечного источника света в плоском зеркале можно получить,
        «удвоив» (в векторном смысле) перпендикуляр, опущенный из источника на плоскость зеркала.
}
\solutionspace{120pt}

\tasknumber{5}%
\task{%
    Плоское зеркало вращается с угловой скоростью $0{,}18\,\frac{\text{рад}}{\text{с}}$.
    Ось вращения лежит в плоскости зеркала.
    На зеркало падает луч перпендикулярно оси вращения.
    Определите угловую скорость вращения отражённого луча.
}
\answer{%
    $\omega' = 2 \omega = 0{,}4\,\text{Гц}$
}
\solutionspace{80pt}

\tasknumber{6}%
\task{%
    Плоское зеркало приближается к стационарному предмету размером $5\,\text{см}$
    со скоростью $3\,\frac{\text{см}}{\text{с}}$.
    Определите размер изображения предмета через $2\,\text{с}$ после начала движения,
    если изначальное расстояние между зеркалом и предметом было равно $60\,\text{см}$.
}
\answer{%
    $5\,\text{см}$
}
\solutionspace{80pt}

\tasknumber{7}%
\task{%
    Предмет отдаляется от плоского зеркала со скоростью $2\,\frac{\text{см}}{\text{с}}$.
    Определите скорость изображения, приняв зеркало стационарным.
}
\answer{%
    $2\,\frac{\text{см}}{\text{с}}$
}
\solutionspace{80pt}

\tasknumber{8}%
\task{%
    Запишите своё имя (не фамилию) печатными буквами
    и постройте их изображение в 2 зеркалах: вертикальном и горизонтальном.
    Не забудьте отметить зеркала.
}
\solutionspace{120pt}

\tasknumber{9}%
\task{%
    Луч падает из {what} на стекло с показателем преломления  1{,}45 .
    Сделайте рисунок (без рисунка и отмеченных углов задача не проверяется) и определите:
    \begin{itemize}
        \item угол отражения,
        \item угол преломления,
        \item угол между падающим и отраженным лучом,
        \item угол между падающим и преломленным лучом,
        \item угол отклонения луча при преломлении,
    \end{itemize}
    если между падающим лучом и границей раздела сред равен $22\degrees$.
}
\answer{%
    \begin{align*}
    \alpha &= 68\degrees, \\
    1 \cdot \sin \alpha &= n \sin \beta \implies \beta = \arcsin\cbr{ \frac{\sin \alpha}{ n } } \approx 14{,}97\degrees, \\
    \varphi_1 &= \alpha \approx 68\degrees, \\
    \varphi_2 &= \beta \approx 14{,}97\degrees, \\
    \varphi_3 &= 2\alpha = 136\degrees, \\
    \varphi_4 &= 180\degrees - \alpha + \beta \approx 126{,}97\degrees, \\
    \varphi_5 &= \alpha - \beta \approx 53{,}03\degrees.
    \end{align*}
}
\solutionspace{100pt}

\tasknumber{10}%
\task{%
    Определите радиус полутени на экране диска размером $D = 3\,\text{см}$ от протяжённого источника, также обладающего формой диска размером $d = 4\,\text{см}$ (см.
    рис.
    на доске, вид сбоку).
    Расстояние от источника до диска равно $l = 12\,\text{см}$, а расстояние от диска до экрана — $L = 20\,\text{см}$.
}
\answer{%
    $\cfrac{\frac d2 + \frac D2}l = \cfrac{\frac d2 + r}{l + L} \implies r = \cfrac{Dl + dL + DL}{2l} = \cfrac D2 + \cfrac{ L }{ l } \cdot \cfrac{d+D}2 \approx 7{,}3\,\text{см} \implies 2r \approx 14{,}7\,\text{см}$
}

\variantsplitter

\addpersonalvariant{Варвара Минаева}

\tasknumber{1}%
\task{%
    Cформулируйте принцип Гюйгенса-Френеля, запишите формулой закон преломления
    и выведите из принципа этот закон.
}
\solutionspace{80pt}

\tasknumber{2}%
\task{%
    Постройте изображения $C'F'$ и $M'{V}'$ стрелок $CF$ и $M{V}$
    в 2 плоских зеркалах соответственно (см.
    рис.
    на доске).
}
\solutionspace{180pt}

\tasknumber{3}%
\task{%
    Постройте область видимости стрелки $AL$ в плоском зеркале (см.
    рис.
    на доске).
}
\solutionspace{120pt}

\tasknumber{4}%
\task{%
    Докажите, что изображение точечного источника света в плоском зеркале можно получить,
        «удвоив» (в векторном смысле) перпендикуляр, опущенный из источника на плоскость зеркала.
}
\solutionspace{120pt}

\tasknumber{5}%
\task{%
    Плоское зеркало вращается с угловой скоростью $0{,}18\,\frac{\text{рад}}{\text{с}}$.
    Ось вращения лежит в плоскости зеркала.
    На зеркало падает луч перпендикулярно оси вращения.
    Определите угловую скорость вращения отражённого луча.
}
\answer{%
    $\omega' = 2 \omega = 0{,}4\,\text{Гц}$
}
\solutionspace{80pt}

\tasknumber{6}%
\task{%
    Плоское зеркало приближается к стационарному предмету размером $5\,\text{см}$
    со скоростью $2\,\frac{\text{см}}{\text{с}}$.
    Определите размер изображения предмета через $4\,\text{с}$ после начала движения,
    если изначальное расстояние между зеркалом и предметом было равно $70\,\text{см}$.
}
\answer{%
    $5\,\text{см}$
}
\solutionspace{80pt}

\tasknumber{7}%
\task{%
    Предмет приближается к плоскому зеркалу со скоростью $2\,\frac{\text{см}}{\text{с}}$.
    Определите скорость изображения, приняв зеркало стационарным.
}
\answer{%
    $2\,\frac{\text{см}}{\text{с}}$
}
\solutionspace{80pt}

\tasknumber{8}%
\task{%
    Запишите своё имя (не фамилию) печатными буквами
    и постройте их изображение в 2 зеркалах: вертикальном и горизонтальном.
    Не забудьте отметить зеркала.
}
\solutionspace{120pt}

\tasknumber{9}%
\task{%
    Луч падает из {what} на стекло с показателем преломления  1{,}45 .
    Сделайте рисунок (без рисунка и отмеченных углов задача не проверяется) и определите:
    \begin{itemize}
        \item угол отражения,
        \item угол преломления,
        \item угол между падающим и отраженным лучом,
        \item угол между падающим и преломленным лучом,
        \item угол отклонения луча при преломлении,
    \end{itemize}
    если угол падения равен $55\degrees$.
}
\answer{%
    \begin{align*}
    \alpha &= 55\degrees, \\
    1 \cdot \sin \alpha &= n \sin \beta \implies \beta = \arcsin\cbr{ \frac{\sin \alpha}{ n } } \approx 34{,}40\degrees, \\
    \varphi_1 &= \alpha \approx 55\degrees, \\
    \varphi_2 &= \beta \approx 34{,}40\degrees, \\
    \varphi_3 &= 2\alpha = 110\degrees, \\
    \varphi_4 &= 180\degrees - \alpha + \beta \approx 159{,}40\degrees, \\
    \varphi_5 &= \alpha - \beta \approx 20{,}60\degrees.
    \end{align*}
}
\solutionspace{100pt}

\tasknumber{10}%
\task{%
    Определите диаметр полутени на экране диска размером $D = 2\,\text{см}$ от протяжённого источника, также обладающего формой диска размером $d = 4\,\text{см}$ (см.
    рис.
    на доске, вид сбоку).
    Расстояние от источника до диска равно $l = 15\,\text{см}$, а расстояние от диска до экрана — $L = 10\,\text{см}$.
}
\answer{%
    $\cfrac{\frac d2 + \frac D2}l = \cfrac{\frac d2 + r}{l + L} \implies r = \cfrac{Dl + dL + DL}{2l} = \cfrac D2 + \cfrac{ L }{ l } \cdot \cfrac{d+D}2 \approx 3\,\text{см} \implies 2r \approx 6\,\text{см}$
}

\variantsplitter

\addpersonalvariant{Леонид Никитин}

\tasknumber{1}%
\task{%
    Cформулируйте принцип Гюйгенса-Френеля, запишите формулой закон отражения
    и выведите из принципа этот закон.
}
\solutionspace{80pt}

\tasknumber{2}%
\task{%
    Постройте изображения $C'F'$ и $M'{V}'$ стрелок $CF$ и $M{V}$
    в 2 плоских зеркалах соответственно (см.
    рис.
    на доске).
}
\solutionspace{180pt}

\tasknumber{3}%
\task{%
    Постройте область видимости стрелки $CM$ в плоском зеркале (см.
    рис.
    на доске).
}
\solutionspace{120pt}

\tasknumber{4}%
\task{%
    Докажите, что изображение точечного источника света в плоском зеркале можно получить,
        «удвоив» (в векторном смысле) перпендикуляр, опущенный из источника на плоскость зеркала.
}
\solutionspace{120pt}

\tasknumber{5}%
\task{%
    Плоское зеркало вращается с угловой скоростью $0{,}23\,\frac{\text{рад}}{\text{с}}$.
    Ось вращения лежит в плоскости зеркала.
    На зеркало падает луч перпендикулярно оси вращения.
    Определите угловую скорость вращения отражённого луча.
}
\answer{%
    $\omega' = 2 \omega = 0{,}5\,\text{Гц}$
}
\solutionspace{80pt}

\tasknumber{6}%
\task{%
    Плоское зеркало приближается к стационарному предмету размером $7\,\text{см}$
    со скоростью $1\,\frac{\text{см}}{\text{с}}$.
    Определите размер изображения предмета через $5\,\text{с}$ после начала движения,
    если изначальное расстояние между зеркалом и предметом было равно $50\,\text{см}$.
}
\answer{%
    $7\,\text{см}$
}
\solutionspace{80pt}

\tasknumber{7}%
\task{%
    Предмет отдаляется от плоского зеркала со скоростью $3\,\frac{\text{см}}{\text{с}}$.
    Определите скорость изображения, приняв зеркало стационарным.
}
\answer{%
    $3\,\frac{\text{см}}{\text{с}}$
}
\solutionspace{80pt}

\tasknumber{8}%
\task{%
    Запишите своё имя (не фамилию) печатными буквами
    и постройте их изображение в 2 зеркалах: вертикальном и горизонтальном.
    Не забудьте отметить зеркала.
}
\solutionspace{120pt}

\tasknumber{9}%
\task{%
    Луч падает из {what} на стекло с показателем преломления  1{,}45 .
    Сделайте рисунок (без рисунка и отмеченных углов задача не проверяется) и определите:
    \begin{itemize}
        \item угол отражения,
        \item угол преломления,
        \item угол между падающим и отраженным лучом,
        \item угол между падающим и преломленным лучом,
        \item угол отклонения луча при преломлении,
    \end{itemize}
    если между падающим лучом и границей раздела сред равен $35\degrees$.
}
\answer{%
    \begin{align*}
    \alpha &= 55\degrees, \\
    1 \cdot \sin \alpha &= n \sin \beta \implies \beta = \arcsin\cbr{ \frac{\sin \alpha}{ n } } \approx 23{,}30\degrees, \\
    \varphi_1 &= \alpha \approx 55\degrees, \\
    \varphi_2 &= \beta \approx 23{,}30\degrees, \\
    \varphi_3 &= 2\alpha = 110\degrees, \\
    \varphi_4 &= 180\degrees - \alpha + \beta \approx 148{,}30\degrees, \\
    \varphi_5 &= \alpha - \beta \approx 31{,}70\degrees.
    \end{align*}
}
\solutionspace{100pt}

\tasknumber{10}%
\task{%
    Определите диаметр полутени на экране диска размером $D = 3\,\text{см}$ от протяжённого источника, также обладающего формой диска размером $d = 4\,\text{см}$ (см.
    рис.
    на доске, вид сбоку).
    Расстояние от источника до диска равно $l = 18\,\text{см}$, а расстояние от диска до экрана — $L = 10\,\text{см}$.
}
\answer{%
    $\cfrac{\frac d2 + \frac D2}l = \cfrac{\frac d2 + r}{l + L} \implies r = \cfrac{Dl + dL + DL}{2l} = \cfrac D2 + \cfrac{ L }{ l } \cdot \cfrac{d+D}2 \approx 3{,}4\,\text{см} \implies 2r \approx 6{,}9\,\text{см}$
}

\variantsplitter

\addpersonalvariant{Тимофей Полетаев}

\tasknumber{1}%
\task{%
    Cформулируйте принцип Гюйгенса-Френеля, запишите формулой закон отражения
    и выведите из принципа этот закон.
}
\solutionspace{80pt}

\tasknumber{2}%
\task{%
    Постройте изображения $A'F'$ и $L'{V}'$ стрелок $AF$ и $L{V}$
    в 2 плоских зеркалах соответственно (см.
    рис.
    на доске).
}
\solutionspace{180pt}

\tasknumber{3}%
\task{%
    Постройте область видимости стрелки $BL$ в плоском зеркале (см.
    рис.
    на доске).
}
\solutionspace{120pt}

\tasknumber{4}%
\task{%
    Докажите, что при повороте плоского зеркала на угол $\varphi$ вокруг оси, лежащей в плоскости зеркала
        и перпендикулярной падающему лучу, этот луч повернётся на угол $2\varphi$.
}
\solutionspace{120pt}

\tasknumber{5}%
\task{%
    Плоское зеркало вращается с угловой скоростью $0{,}25\,\frac{\text{рад}}{\text{с}}$.
    Ось вращения лежит в плоскости зеркала.
    На зеркало падает луч перпендикулярно оси вращения.
    Определите угловую скорость вращения отражённого луча.
}
\answer{%
    $\omega' = 2 \omega = 0{,}5\,\text{Гц}$
}
\solutionspace{80pt}

\tasknumber{6}%
\task{%
    Плоское зеркало приближается к стационарному предмету размером $7\,\text{см}$
    со скоростью $2\,\frac{\text{см}}{\text{с}}$.
    Определите размер изображения предмета через $2\,\text{с}$ после начала движения,
    если изначальное расстояние между зеркалом и предметом было равно $50\,\text{см}$.
}
\answer{%
    $7\,\text{см}$
}
\solutionspace{80pt}

\tasknumber{7}%
\task{%
    Предмет отдаляется от плоского зеркала со скоростью $4\,\frac{\text{см}}{\text{с}}$.
    Определите скорость изображения, приняв зеркало стационарным.
}
\answer{%
    $4\,\frac{\text{см}}{\text{с}}$
}
\solutionspace{80pt}

\tasknumber{8}%
\task{%
    Запишите своё имя (не фамилию) печатными буквами
    и постройте их изображение в 2 зеркалах: вертикальном и горизонтальном.
    Не забудьте отметить зеркала.
}
\solutionspace{120pt}

\tasknumber{9}%
\task{%
    Луч падает из {what} на стекло с показателем преломления  1{,}55 .
    Сделайте рисунок (без рисунка и отмеченных углов задача не проверяется) и определите:
    \begin{itemize}
        \item угол отражения,
        \item угол преломления,
        \item угол между падающим и отраженным лучом,
        \item угол между падающим и преломленным лучом,
        \item угол отклонения луча при преломлении,
    \end{itemize}
    если угол падения равен $55\degrees$.
}
\answer{%
    \begin{align*}
    \alpha &= 55\degrees, \\
    1 \cdot \sin \alpha &= n \sin \beta \implies \beta = \arcsin\cbr{ \frac{\sin \alpha}{ n } } \approx 31{,}90\degrees, \\
    \varphi_1 &= \alpha \approx 55\degrees, \\
    \varphi_2 &= \beta \approx 31{,}90\degrees, \\
    \varphi_3 &= 2\alpha = 110\degrees, \\
    \varphi_4 &= 180\degrees - \alpha + \beta \approx 156{,}90\degrees, \\
    \varphi_5 &= \alpha - \beta \approx 23{,}10\degrees.
    \end{align*}
}
\solutionspace{100pt}

\tasknumber{10}%
\task{%
    Определите радиус полутени на экране диска размером $D = 3{,}5\,\text{см}$ от протяжённого источника, также обладающего формой диска размером $d = 2\,\text{см}$ (см.
    рис.
    на доске, вид сбоку).
    Расстояние от источника до диска равно $l = 12\,\text{см}$, а расстояние от диска до экрана — $L = 20\,\text{см}$.
}
\answer{%
    $\cfrac{\frac d2 + \frac D2}l = \cfrac{\frac d2 + r}{l + L} \implies r = \cfrac{Dl + dL + DL}{2l} = \cfrac D2 + \cfrac{ L }{ l } \cdot \cfrac{d+D}2 \approx 6{,}3\,\text{см} \implies 2r \approx 12{,}7\,\text{см}$
}

\variantsplitter

\addpersonalvariant{Андрей Рожков}

\tasknumber{1}%
\task{%
    Cформулируйте принцип Гюйгенса-Френеля, запишите формулой закон преломления
    и выведите из принципа этот закон.
}
\solutionspace{80pt}

\tasknumber{2}%
\task{%
    Постройте изображения $C'E'$ и $M'{V}'$ стрелок $CE$ и $M{V}$
    в 2 плоских зеркалах соответственно (см.
    рис.
    на доске).
}
\solutionspace{180pt}

\tasknumber{3}%
\task{%
    Постройте область видимости стрелки $DK$ в плоском зеркале (см.
    рис.
    на доске).
}
\solutionspace{120pt}

\tasknumber{4}%
\task{%
    Докажите, что при повороте плоского зеркала на угол $\varphi$ вокруг оси, лежащей в плоскости зеркала
        и перпендикулярной падающему лучу, этот луч повернётся на угол $2\varphi$.
}
\solutionspace{120pt}

\tasknumber{5}%
\task{%
    Плоское зеркало вращается с угловой скоростью $0{,}15\,\frac{\text{рад}}{\text{с}}$.
    Ось вращения лежит в плоскости зеркала.
    На зеркало падает луч перпендикулярно оси вращения.
    Определите угловую скорость вращения отражённого луча.
}
\answer{%
    $\omega' = 2 \omega = 0{,}3\,\text{Гц}$
}
\solutionspace{80pt}

\tasknumber{6}%
\task{%
    Плоское зеркало приближается к стационарному предмету размером $7\,\text{см}$
    со скоростью $3\,\frac{\text{см}}{\text{с}}$.
    Определите размер изображения предмета через $2\,\text{с}$ после начала движения,
    если изначальное расстояние между зеркалом и предметом было равно $60\,\text{см}$.
}
\answer{%
    $7\,\text{см}$
}
\solutionspace{80pt}

\tasknumber{7}%
\task{%
    Предмет отдаляется от плоского зеркала со скоростью $2\,\frac{\text{см}}{\text{с}}$.
    Определите скорость изображения, приняв зеркало стационарным.
}
\answer{%
    $2\,\frac{\text{см}}{\text{с}}$
}
\solutionspace{80pt}

\tasknumber{8}%
\task{%
    Запишите своё имя (не фамилию) печатными буквами
    и постройте их изображение в 2 зеркалах: вертикальном и горизонтальном.
    Не забудьте отметить зеркала.
}
\solutionspace{120pt}

\tasknumber{9}%
\task{%
    Луч падает из {what} на стекло с показателем преломления  1{,}55 .
    Сделайте рисунок (без рисунка и отмеченных углов задача не проверяется) и определите:
    \begin{itemize}
        \item угол отражения,
        \item угол преломления,
        \item угол между падающим и отраженным лучом,
        \item угол между падающим и преломленным лучом,
        \item угол отклонения луча при преломлении,
    \end{itemize}
    если между падающим лучом и границей раздела сред равен $55\degrees$.
}
\answer{%
    \begin{align*}
    \alpha &= 35\degrees, \\
    1 \cdot \sin \alpha &= n \sin \beta \implies \beta = \arcsin\cbr{ \frac{\sin \alpha}{ n } } \approx 31{,}90\degrees, \\
    \varphi_1 &= \alpha \approx 35\degrees, \\
    \varphi_2 &= \beta \approx 31{,}90\degrees, \\
    \varphi_3 &= 2\alpha = 70\degrees, \\
    \varphi_4 &= 180\degrees - \alpha + \beta \approx 176{,}90\degrees, \\
    \varphi_5 &= \alpha - \beta \approx 3{,}10\degrees.
    \end{align*}
}
\solutionspace{100pt}

\tasknumber{10}%
\task{%
    Определите диаметр полутени на экране диска размером $D = 2{,}5\,\text{см}$ от протяжённого источника, также обладающего формой диска размером $d = 2\,\text{см}$ (см.
    рис.
    на доске, вид сбоку).
    Расстояние от источника до диска равно $l = 18\,\text{см}$, а расстояние от диска до экрана — $L = 10\,\text{см}$.
}
\answer{%
    $\cfrac{\frac d2 + \frac D2}l = \cfrac{\frac d2 + r}{l + L} \implies r = \cfrac{Dl + dL + DL}{2l} = \cfrac D2 + \cfrac{ L }{ l } \cdot \cfrac{d+D}2 \approx 2{,}5\,\text{см} \implies 2r \approx 5\,\text{см}$
}

\variantsplitter

\addpersonalvariant{Рената Таржиманова}

\tasknumber{1}%
\task{%
    Cформулируйте принцип Гюйгенса-Френеля, запишите формулой закон отражения
    и выведите из принципа этот закон.
}
\solutionspace{80pt}

\tasknumber{2}%
\task{%
    Постройте изображения $C'F'$ и $L'{V}'$ стрелок $CF$ и $L{V}$
    в 2 плоских зеркалах соответственно (см.
    рис.
    на доске).
}
\solutionspace{180pt}

\tasknumber{3}%
\task{%
    Постройте область видимости стрелки $AL$ в плоском зеркале (см.
    рис.
    на доске).
}
\solutionspace{120pt}

\tasknumber{4}%
\task{%
    Докажите, что при повороте плоского зеркала на угол $\varphi$ вокруг оси, лежащей в плоскости зеркала
        и перпендикулярной падающему лучу, этот луч повернётся на угол $2\varphi$.
}
\solutionspace{120pt}

\tasknumber{5}%
\task{%
    Плоское зеркало вращается с угловой скоростью $0{,}15\,\frac{\text{рад}}{\text{с}}$.
    Ось вращения лежит в плоскости зеркала.
    На зеркало падает луч перпендикулярно оси вращения.
    Определите угловую скорость вращения отражённого луча.
}
\answer{%
    $\omega' = 2 \omega = 0{,}3\,\text{Гц}$
}
\solutionspace{80pt}

\tasknumber{6}%
\task{%
    Плоское зеркало приближается к стационарному предмету размером $5\,\text{см}$
    со скоростью $3\,\frac{\text{см}}{\text{с}}$.
    Определите размер изображения предмета через $4\,\text{с}$ после начала движения,
    если изначальное расстояние между зеркалом и предметом было равно $70\,\text{см}$.
}
\answer{%
    $5\,\text{см}$
}
\solutionspace{80pt}

\tasknumber{7}%
\task{%
    Предмет приближается к плоскому зеркалу со скоростью $4\,\frac{\text{см}}{\text{с}}$.
    Определите скорость изображения, приняв зеркало стационарным.
}
\answer{%
    $4\,\frac{\text{см}}{\text{с}}$
}
\solutionspace{80pt}

\tasknumber{8}%
\task{%
    Запишите своё имя (не фамилию) печатными буквами
    и постройте их изображение в 2 зеркалах: вертикальном и горизонтальном.
    Не забудьте отметить зеркала.
}
\solutionspace{120pt}

\tasknumber{9}%
\task{%
    Луч падает из {what} на стекло с показателем преломления  1{,}35 .
    Сделайте рисунок (без рисунка и отмеченных углов задача не проверяется) и определите:
    \begin{itemize}
        \item угол отражения,
        \item угол преломления,
        \item угол между падающим и отраженным лучом,
        \item угол между падающим и преломленным лучом,
        \item угол отклонения луча при преломлении,
    \end{itemize}
    если между падающим лучом и границей раздела сред равен $65\degrees$.
}
\answer{%
    \begin{align*}
    \alpha &= 25\degrees, \\
    1 \cdot \sin \alpha &= n \sin \beta \implies \beta = \arcsin\cbr{ \frac{\sin \alpha}{ n } } \approx 42{,}17\degrees, \\
    \varphi_1 &= \alpha \approx 25\degrees, \\
    \varphi_2 &= \beta \approx 42{,}17\degrees, \\
    \varphi_3 &= 2\alpha = 50\degrees, \\
    \varphi_4 &= 180\degrees - \alpha + \beta \approx 197{,}17\degrees, \\
    \varphi_5 &= \alpha - \beta \approx -17{,}17\degrees.
    \end{align*}
}
\solutionspace{100pt}

\tasknumber{10}%
\task{%
    Определите диаметр полутени на экране диска размером $D = 2{,}5\,\text{см}$ от протяжённого источника, также обладающего формой диска размером $d = 4\,\text{см}$ (см.
    рис.
    на доске, вид сбоку).
    Расстояние от источника до диска равно $l = 15\,\text{см}$, а расстояние от диска до экрана — $L = 20\,\text{см}$.
}
\answer{%
    $\cfrac{\frac d2 + \frac D2}l = \cfrac{\frac d2 + r}{l + L} \implies r = \cfrac{Dl + dL + DL}{2l} = \cfrac D2 + \cfrac{ L }{ l } \cdot \cfrac{d+D}2 \approx 5{,}6\,\text{см} \implies 2r \approx 11{,}2\,\text{см}$
}

\variantsplitter

\addpersonalvariant{Андрей Щербаков}

\tasknumber{1}%
\task{%
    Cформулируйте принцип Гюйгенса-Френеля, запишите формулой закон отражения
    и выведите из принципа этот закон.
}
\solutionspace{80pt}

\tasknumber{2}%
\task{%
    Постройте изображения $B'F'$ и $K'{V}'$ стрелок $BF$ и $K{V}$
    в 2 плоских зеркалах соответственно (см.
    рис.
    на доске).
}
\solutionspace{180pt}

\tasknumber{3}%
\task{%
    Постройте область видимости стрелки $BM$ в плоском зеркале (см.
    рис.
    на доске).
}
\solutionspace{120pt}

\tasknumber{4}%
\task{%
    Докажите, что при повороте плоского зеркала на угол $\varphi$ вокруг оси, лежащей в плоскости зеркала
        и перпендикулярной падающему лучу, этот луч повернётся на угол $2\varphi$.
}
\solutionspace{120pt}

\tasknumber{5}%
\task{%
    Плоское зеркало вращается с угловой скоростью $0{,}25\,\frac{\text{рад}}{\text{с}}$.
    Ось вращения лежит в плоскости зеркала.
    На зеркало падает луч перпендикулярно оси вращения.
    Определите угловую скорость вращения отражённого луча.
}
\answer{%
    $\omega' = 2 \omega = 0{,}5\,\text{Гц}$
}
\solutionspace{80pt}

\tasknumber{6}%
\task{%
    Плоское зеркало приближается к стационарному предмету размером $6\,\text{см}$
    со скоростью $1\,\frac{\text{см}}{\text{с}}$.
    Определите размер изображения предмета через $5\,\text{с}$ после начала движения,
    если изначальное расстояние между зеркалом и предметом было равно $60\,\text{см}$.
}
\answer{%
    $6\,\text{см}$
}
\solutionspace{80pt}

\tasknumber{7}%
\task{%
    Предмет отдаляется от плоского зеркала со скоростью $4\,\frac{\text{см}}{\text{с}}$.
    Определите скорость изображения, приняв зеркало стационарным.
}
\answer{%
    $4\,\frac{\text{см}}{\text{с}}$
}
\solutionspace{80pt}

\tasknumber{8}%
\task{%
    Запишите своё имя (не фамилию) печатными буквами
    и постройте их изображение в 2 зеркалах: вертикальном и горизонтальном.
    Не забудьте отметить зеркала.
}
\solutionspace{120pt}

\tasknumber{9}%
\task{%
    Луч падает из {what} на стекло с показателем преломления  1{,}65 .
    Сделайте рисунок (без рисунка и отмеченных углов задача не проверяется) и определите:
    \begin{itemize}
        \item угол отражения,
        \item угол преломления,
        \item угол между падающим и отраженным лучом,
        \item угол между падающим и преломленным лучом,
        \item угол отклонения луча при преломлении,
    \end{itemize}
    если между падающим лучом и границей раздела сред равен $22\degrees$.
}
\answer{%
    \begin{align*}
    \alpha &= 68\degrees, \\
    1 \cdot \sin \alpha &= n \sin \beta \implies \beta = \arcsin\cbr{ \frac{\sin \alpha}{ n } } \approx 13{,}12\degrees, \\
    \varphi_1 &= \alpha \approx 68\degrees, \\
    \varphi_2 &= \beta \approx 13{,}12\degrees, \\
    \varphi_3 &= 2\alpha = 136\degrees, \\
    \varphi_4 &= 180\degrees - \alpha + \beta \approx 125{,}12\degrees, \\
    \varphi_5 &= \alpha - \beta \approx 54{,}88\degrees.
    \end{align*}
}
\solutionspace{100pt}

\tasknumber{10}%
\task{%
    Определите диаметр полутени на экране диска размером $D = 3{,}5\,\text{см}$ от протяжённого источника, также обладающего формой диска размером $d = 4\,\text{см}$ (см.
    рис.
    на доске, вид сбоку).
    Расстояние от источника до диска равно $l = 12\,\text{см}$, а расстояние от диска до экрана — $L = 30\,\text{см}$.
}
\answer{%
    $\cfrac{\frac d2 + \frac D2}l = \cfrac{\frac d2 + r}{l + L} \implies r = \cfrac{Dl + dL + DL}{2l} = \cfrac D2 + \cfrac{ L }{ l } \cdot \cfrac{d+D}2 \approx 11{,}1\,\text{см} \implies 2r \approx 22{,}2\,\text{см}$
}

\variantsplitter

\addpersonalvariant{Михаил Ярошевский}

\tasknumber{1}%
\task{%
    Cформулируйте принцип Гюйгенса-Френеля, запишите формулой закон отражения
    и выведите из принципа этот закон.
}
\solutionspace{80pt}

\tasknumber{2}%
\task{%
    Постройте изображения $C'D'$ и $K'{V}'$ стрелок $CD$ и $K{V}$
    в 2 плоских зеркалах соответственно (см.
    рис.
    на доске).
}
\solutionspace{180pt}

\tasknumber{3}%
\task{%
    Постройте область видимости стрелки $CL$ в плоском зеркале (см.
    рис.
    на доске).
}
\solutionspace{120pt}

\tasknumber{4}%
\task{%
    Докажите, что изображение точечного источника света в плоском зеркале можно получить,
        «удвоив» (в векторном смысле) перпендикуляр, опущенный из источника на плоскость зеркала.
}
\solutionspace{120pt}

\tasknumber{5}%
\task{%
    Плоское зеркало вращается с угловой скоростью $0{,}16\,\frac{\text{рад}}{\text{с}}$.
    Ось вращения лежит в плоскости зеркала.
    На зеркало падает луч перпендикулярно оси вращения.
    Определите угловую скорость вращения отражённого луча.
}
\answer{%
    $\omega' = 2 \omega = 0{,}3\,\text{Гц}$
}
\solutionspace{80pt}

\tasknumber{6}%
\task{%
    Плоское зеркало приближается к стационарному предмету размером $7\,\text{см}$
    со скоростью $3\,\frac{\text{см}}{\text{с}}$.
    Определите размер изображения предмета через $5\,\text{с}$ после начала движения,
    если изначальное расстояние между зеркалом и предметом было равно $50\,\text{см}$.
}
\answer{%
    $7\,\text{см}$
}
\solutionspace{80pt}

\tasknumber{7}%
\task{%
    Предмет отдаляется от плоского зеркала со скоростью $2\,\frac{\text{см}}{\text{с}}$.
    Определите скорость изображения, приняв зеркало стационарным.
}
\answer{%
    $2\,\frac{\text{см}}{\text{с}}$
}
\solutionspace{80pt}

\tasknumber{8}%
\task{%
    Запишите своё имя (не фамилию) печатными буквами
    и постройте их изображение в 2 зеркалах: вертикальном и горизонтальном.
    Не забудьте отметить зеркала.
}
\solutionspace{120pt}

\tasknumber{9}%
\task{%
    Луч падает из {what} на стекло с показателем преломления  1{,}65 .
    Сделайте рисунок (без рисунка и отмеченных углов задача не проверяется) и определите:
    \begin{itemize}
        \item угол отражения,
        \item угол преломления,
        \item угол между падающим и отраженным лучом,
        \item угол между падающим и преломленным лучом,
        \item угол отклонения луча при преломлении,
    \end{itemize}
    если между падающим лучом и границей раздела сред равен $40\degrees$.
}
\answer{%
    \begin{align*}
    \alpha &= 50\degrees, \\
    1 \cdot \sin \alpha &= n \sin \beta \implies \beta = \arcsin\cbr{ \frac{\sin \alpha}{ n } } \approx 22{,}93\degrees, \\
    \varphi_1 &= \alpha \approx 50\degrees, \\
    \varphi_2 &= \beta \approx 22{,}93\degrees, \\
    \varphi_3 &= 2\alpha = 100\degrees, \\
    \varphi_4 &= 180\degrees - \alpha + \beta \approx 152{,}93\degrees, \\
    \varphi_5 &= \alpha - \beta \approx 27{,}07\degrees.
    \end{align*}
}
\solutionspace{100pt}

\tasknumber{10}%
\task{%
    Определите диаметр полутени на экране диска размером $D = 2\,\text{см}$ от протяжённого источника, также обладающего формой диска размером $d = 4\,\text{см}$ (см.
    рис.
    на доске, вид сбоку).
    Расстояние от источника до диска равно $l = 15\,\text{см}$, а расстояние от диска до экрана — $L = 30\,\text{см}$.
}
\answer{%
    $\cfrac{\frac d2 + \frac D2}l = \cfrac{\frac d2 + r}{l + L} \implies r = \cfrac{Dl + dL + DL}{2l} = \cfrac D2 + \cfrac{ L }{ l } \cdot \cfrac{d+D}2 \approx 7\,\text{см} \implies 2r \approx 14\,\text{см}$
}

\variantsplitter

\addpersonalvariant{Алексей Алимпиев}

\tasknumber{1}%
\task{%
    Cформулируйте принцип Гюйгенса-Френеля, запишите формулой закон преломления
    и выведите из принципа этот закон.
}
\solutionspace{80pt}

\tasknumber{2}%
\task{%
    Постройте изображения $B'D'$ и $L'{V}'$ стрелок $BD$ и $L{V}$
    в 2 плоских зеркалах соответственно (см.
    рис.
    на доске).
}
\solutionspace{180pt}

\tasknumber{3}%
\task{%
    Постройте область видимости стрелки $DN$ в плоском зеркале (см.
    рис.
    на доске).
}
\solutionspace{120pt}

\tasknumber{4}%
\task{%
    Докажите, что изображение точечного источника света в плоском зеркале можно получить,
        «удвоив» (в векторном смысле) перпендикуляр, опущенный из источника на плоскость зеркала.
}
\solutionspace{120pt}

\tasknumber{5}%
\task{%
    Плоское зеркало вращается с угловой скоростью $0{,}15\,\frac{\text{рад}}{\text{с}}$.
    Ось вращения лежит в плоскости зеркала.
    На зеркало падает луч перпендикулярно оси вращения.
    Определите угловую скорость вращения отражённого луча.
}
\answer{%
    $\omega' = 2 \omega = 0{,}3\,\text{Гц}$
}
\solutionspace{80pt}

\tasknumber{6}%
\task{%
    Плоское зеркало приближается к стационарному предмету размером $5\,\text{см}$
    со скоростью $1\,\frac{\text{см}}{\text{с}}$.
    Определите размер изображения предмета через $3\,\text{с}$ после начала движения,
    если изначальное расстояние между зеркалом и предметом было равно $70\,\text{см}$.
}
\answer{%
    $5\,\text{см}$
}
\solutionspace{80pt}

\tasknumber{7}%
\task{%
    Предмет отдаляется от плоского зеркала со скоростью $4\,\frac{\text{см}}{\text{с}}$.
    Определите скорость изображения, приняв зеркало стационарным.
}
\answer{%
    $4\,\frac{\text{см}}{\text{с}}$
}
\solutionspace{80pt}

\tasknumber{8}%
\task{%
    Запишите своё имя (не фамилию) печатными буквами
    и постройте их изображение в 2 зеркалах: вертикальном и горизонтальном.
    Не забудьте отметить зеркала.
}
\solutionspace{120pt}

\tasknumber{9}%
\task{%
    Луч падает из {what} на стекло с показателем преломления  1{,}65 .
    Сделайте рисунок (без рисунка и отмеченных углов задача не проверяется) и определите:
    \begin{itemize}
        \item угол отражения,
        \item угол преломления,
        \item угол между падающим и отраженным лучом,
        \item угол между падающим и преломленным лучом,
        \item угол отклонения луча при преломлении,
    \end{itemize}
    если между падающим лучом и границей раздела сред равен $50\degrees$.
}
\answer{%
    \begin{align*}
    \alpha &= 40\degrees, \\
    1 \cdot \sin \alpha &= n \sin \beta \implies \beta = \arcsin\cbr{ \frac{\sin \alpha}{ n } } \approx 27{,}66\degrees, \\
    \varphi_1 &= \alpha \approx 40\degrees, \\
    \varphi_2 &= \beta \approx 27{,}66\degrees, \\
    \varphi_3 &= 2\alpha = 80\degrees, \\
    \varphi_4 &= 180\degrees - \alpha + \beta \approx 167{,}66\degrees, \\
    \varphi_5 &= \alpha - \beta \approx 12{,}34\degrees.
    \end{align*}
}
\solutionspace{100pt}

\tasknumber{10}%
\task{%
    Определите радиус полутени на экране диска размером $D = 3\,\text{см}$ от протяжённого источника, также обладающего формой диска размером $d = 3\,\text{см}$ (см.
    рис.
    на доске, вид сбоку).
    Расстояние от источника до диска равно $l = 18\,\text{см}$, а расстояние от диска до экрана — $L = 20\,\text{см}$.
}
\answer{%
    $\cfrac{\frac d2 + \frac D2}l = \cfrac{\frac d2 + r}{l + L} \implies r = \cfrac{Dl + dL + DL}{2l} = \cfrac D2 + \cfrac{ L }{ l } \cdot \cfrac{d+D}2 \approx 4{,}8\,\text{см} \implies 2r \approx 9{,}7\,\text{см}$
}

\variantsplitter

\addpersonalvariant{Евгений Васин}

\tasknumber{1}%
\task{%
    Cформулируйте принцип Гюйгенса-Френеля, запишите формулой закон преломления
    и выведите из принципа этот закон.
}
\solutionspace{80pt}

\tasknumber{2}%
\task{%
    Постройте изображения $C'E'$ и $M'{V}'$ стрелок $CE$ и $M{V}$
    в 2 плоских зеркалах соответственно (см.
    рис.
    на доске).
}
\solutionspace{180pt}

\tasknumber{3}%
\task{%
    Постройте область видимости стрелки $CL$ в плоском зеркале (см.
    рис.
    на доске).
}
\solutionspace{120pt}

\tasknumber{4}%
\task{%
    Докажите, что изображение точечного источника света в плоском зеркале можно получить,
        «удвоив» (в векторном смысле) перпендикуляр, опущенный из источника на плоскость зеркала.
}
\solutionspace{120pt}

\tasknumber{5}%
\task{%
    Плоское зеркало вращается с угловой скоростью $0{,}12\,\frac{\text{рад}}{\text{с}}$.
    Ось вращения лежит в плоскости зеркала.
    На зеркало падает луч перпендикулярно оси вращения.
    Определите угловую скорость вращения отражённого луча.
}
\answer{%
    $\omega' = 2 \omega = 0{,}2\,\text{Гц}$
}
\solutionspace{80pt}

\tasknumber{6}%
\task{%
    Плоское зеркало приближается к стационарному предмету размером $6\,\text{см}$
    со скоростью $3\,\frac{\text{см}}{\text{с}}$.
    Определите размер изображения предмета через $3\,\text{с}$ после начала движения,
    если изначальное расстояние между зеркалом и предметом было равно $60\,\text{см}$.
}
\answer{%
    $6\,\text{см}$
}
\solutionspace{80pt}

\tasknumber{7}%
\task{%
    Предмет приближается к плоскому зеркалу со скоростью $4\,\frac{\text{см}}{\text{с}}$.
    Определите скорость изображения, приняв зеркало стационарным.
}
\answer{%
    $4\,\frac{\text{см}}{\text{с}}$
}
\solutionspace{80pt}

\tasknumber{8}%
\task{%
    Запишите своё имя (не фамилию) печатными буквами
    и постройте их изображение в 2 зеркалах: вертикальном и горизонтальном.
    Не забудьте отметить зеркала.
}
\solutionspace{120pt}

\tasknumber{9}%
\task{%
    Луч падает из {what} на стекло с показателем преломления  1{,}45 .
    Сделайте рисунок (без рисунка и отмеченных углов задача не проверяется) и определите:
    \begin{itemize}
        \item угол отражения,
        \item угол преломления,
        \item угол между падающим и отраженным лучом,
        \item угол между падающим и преломленным лучом,
        \item угол отклонения луча при преломлении,
    \end{itemize}
    если между падающим лучом и границей раздела сред равен $65\degrees$.
}
\answer{%
    \begin{align*}
    \alpha &= 25\degrees, \\
    1 \cdot \sin \alpha &= n \sin \beta \implies \beta = \arcsin\cbr{ \frac{\sin \alpha}{ n } } \approx 38{,}69\degrees, \\
    \varphi_1 &= \alpha \approx 25\degrees, \\
    \varphi_2 &= \beta \approx 38{,}69\degrees, \\
    \varphi_3 &= 2\alpha = 50\degrees, \\
    \varphi_4 &= 180\degrees - \alpha + \beta \approx 193{,}69\degrees, \\
    \varphi_5 &= \alpha - \beta \approx -13{,}69\degrees.
    \end{align*}
}
\solutionspace{100pt}

\tasknumber{10}%
\task{%
    Определите радиус полутени на экране диска размером $D = 4\,\text{см}$ от протяжённого источника, также обладающего формой диска размером $d = 3\,\text{см}$ (см.
    рис.
    на доске, вид сбоку).
    Расстояние от источника до диска равно $l = 15\,\text{см}$, а расстояние от диска до экрана — $L = 10\,\text{см}$.
}
\answer{%
    $\cfrac{\frac d2 + \frac D2}l = \cfrac{\frac d2 + r}{l + L} \implies r = \cfrac{Dl + dL + DL}{2l} = \cfrac D2 + \cfrac{ L }{ l } \cdot \cfrac{d+D}2 \approx 4{,}3\,\text{см} \implies 2r \approx 8{,}7\,\text{см}$
}

\variantsplitter

\addpersonalvariant{Вячеслав Волохов}

\tasknumber{1}%
\task{%
    Cформулируйте принцип Гюйгенса-Френеля, запишите формулой закон преломления
    и выведите из принципа этот закон.
}
\solutionspace{80pt}

\tasknumber{2}%
\task{%
    Постройте изображения $C'D'$ и $K'{V}'$ стрелок $CD$ и $K{V}$
    в 2 плоских зеркалах соответственно (см.
    рис.
    на доске).
}
\solutionspace{180pt}

\tasknumber{3}%
\task{%
    Постройте область видимости стрелки $AL$ в плоском зеркале (см.
    рис.
    на доске).
}
\solutionspace{120pt}

\tasknumber{4}%
\task{%
    Докажите, что при повороте плоского зеркала на угол $\varphi$ вокруг оси, лежащей в плоскости зеркала
        и перпендикулярной падающему лучу, этот луч повернётся на угол $2\varphi$.
}
\solutionspace{120pt}

\tasknumber{5}%
\task{%
    Плоское зеркало вращается с угловой скоростью $0{,}19\,\frac{\text{рад}}{\text{с}}$.
    Ось вращения лежит в плоскости зеркала.
    На зеркало падает луч перпендикулярно оси вращения.
    Определите угловую скорость вращения отражённого луча.
}
\answer{%
    $\omega' = 2 \omega = 0{,}4\,\text{Гц}$
}
\solutionspace{80pt}

\tasknumber{6}%
\task{%
    Плоское зеркало приближается к стационарному предмету размером $5\,\text{см}$
    со скоростью $3\,\frac{\text{см}}{\text{с}}$.
    Определите размер изображения предмета через $5\,\text{с}$ после начала движения,
    если изначальное расстояние между зеркалом и предметом было равно $50\,\text{см}$.
}
\answer{%
    $5\,\text{см}$
}
\solutionspace{80pt}

\tasknumber{7}%
\task{%
    Предмет отдаляется от плоского зеркала со скоростью $4\,\frac{\text{см}}{\text{с}}$.
    Определите скорость изображения, приняв зеркало стационарным.
}
\answer{%
    $4\,\frac{\text{см}}{\text{с}}$
}
\solutionspace{80pt}

\tasknumber{8}%
\task{%
    Запишите своё имя (не фамилию) печатными буквами
    и постройте их изображение в 2 зеркалах: вертикальном и горизонтальном.
    Не забудьте отметить зеркала.
}
\solutionspace{120pt}

\tasknumber{9}%
\task{%
    Луч падает из {what} на стекло с показателем преломления  1{,}65 .
    Сделайте рисунок (без рисунка и отмеченных углов задача не проверяется) и определите:
    \begin{itemize}
        \item угол отражения,
        \item угол преломления,
        \item угол между падающим и отраженным лучом,
        \item угол между падающим и преломленным лучом,
        \item угол отклонения луча при преломлении,
    \end{itemize}
    если между падающим лучом и границей раздела сред равен $40\degrees$.
}
\answer{%
    \begin{align*}
    \alpha &= 50\degrees, \\
    1 \cdot \sin \alpha &= n \sin \beta \implies \beta = \arcsin\cbr{ \frac{\sin \alpha}{ n } } \approx 22{,}93\degrees, \\
    \varphi_1 &= \alpha \approx 50\degrees, \\
    \varphi_2 &= \beta \approx 22{,}93\degrees, \\
    \varphi_3 &= 2\alpha = 100\degrees, \\
    \varphi_4 &= 180\degrees - \alpha + \beta \approx 152{,}93\degrees, \\
    \varphi_5 &= \alpha - \beta \approx 27{,}07\degrees.
    \end{align*}
}
\solutionspace{100pt}

\tasknumber{10}%
\task{%
    Определите диаметр полутени на экране диска размером $D = 2{,}5\,\text{см}$ от протяжённого источника, также обладающего формой диска размером $d = 3\,\text{см}$ (см.
    рис.
    на доске, вид сбоку).
    Расстояние от источника до диска равно $l = 15\,\text{см}$, а расстояние от диска до экрана — $L = 10\,\text{см}$.
}
\answer{%
    $\cfrac{\frac d2 + \frac D2}l = \cfrac{\frac d2 + r}{l + L} \implies r = \cfrac{Dl + dL + DL}{2l} = \cfrac D2 + \cfrac{ L }{ l } \cdot \cfrac{d+D}2 \approx 3{,}1\,\text{см} \implies 2r \approx 6{,}2\,\text{см}$
}

\variantsplitter

\addpersonalvariant{Герман Говоров}

\tasknumber{1}%
\task{%
    Cформулируйте принцип Гюйгенса-Френеля, запишите формулой закон преломления
    и выведите из принципа этот закон.
}
\solutionspace{80pt}

\tasknumber{2}%
\task{%
    Постройте изображения $C'D'$ и $M'{V}'$ стрелок $CD$ и $M{V}$
    в 2 плоских зеркалах соответственно (см.
    рис.
    на доске).
}
\solutionspace{180pt}

\tasknumber{3}%
\task{%
    Постройте область видимости стрелки $AN$ в плоском зеркале (см.
    рис.
    на доске).
}
\solutionspace{120pt}

\tasknumber{4}%
\task{%
    Докажите, что изображение точечного источника света в плоском зеркале можно получить,
        «удвоив» (в векторном смысле) перпендикуляр, опущенный из источника на плоскость зеркала.
}
\solutionspace{120pt}

\tasknumber{5}%
\task{%
    Плоское зеркало вращается с угловой скоростью $0{,}16\,\frac{\text{рад}}{\text{с}}$.
    Ось вращения лежит в плоскости зеркала.
    На зеркало падает луч перпендикулярно оси вращения.
    Определите угловую скорость вращения отражённого луча.
}
\answer{%
    $\omega' = 2 \omega = 0{,}3\,\text{Гц}$
}
\solutionspace{80pt}

\tasknumber{6}%
\task{%
    Плоское зеркало приближается к стационарному предмету размером $6\,\text{см}$
    со скоростью $2\,\frac{\text{см}}{\text{с}}$.
    Определите размер изображения предмета через $5\,\text{с}$ после начала движения,
    если изначальное расстояние между зеркалом и предметом было равно $70\,\text{см}$.
}
\answer{%
    $6\,\text{см}$
}
\solutionspace{80pt}

\tasknumber{7}%
\task{%
    Предмет отдаляется от плоского зеркала со скоростью $3\,\frac{\text{см}}{\text{с}}$.
    Определите скорость изображения, приняв зеркало стационарным.
}
\answer{%
    $3\,\frac{\text{см}}{\text{с}}$
}
\solutionspace{80pt}

\tasknumber{8}%
\task{%
    Запишите своё имя (не фамилию) печатными буквами
    и постройте их изображение в 2 зеркалах: вертикальном и горизонтальном.
    Не забудьте отметить зеркала.
}
\solutionspace{120pt}

\tasknumber{9}%
\task{%
    Луч падает из {what} на стекло с показателем преломления  1{,}55 .
    Сделайте рисунок (без рисунка и отмеченных углов задача не проверяется) и определите:
    \begin{itemize}
        \item угол отражения,
        \item угол преломления,
        \item угол между падающим и отраженным лучом,
        \item угол между падающим и преломленным лучом,
        \item угол отклонения луча при преломлении,
    \end{itemize}
    если между падающим лучом и границей раздела сред равен $35\degrees$.
}
\answer{%
    \begin{align*}
    \alpha &= 55\degrees, \\
    1 \cdot \sin \alpha &= n \sin \beta \implies \beta = \arcsin\cbr{ \frac{\sin \alpha}{ n } } \approx 21{,}72\degrees, \\
    \varphi_1 &= \alpha \approx 55\degrees, \\
    \varphi_2 &= \beta \approx 21{,}72\degrees, \\
    \varphi_3 &= 2\alpha = 110\degrees, \\
    \varphi_4 &= 180\degrees - \alpha + \beta \approx 146{,}72\degrees, \\
    \varphi_5 &= \alpha - \beta \approx 33{,}28\degrees.
    \end{align*}
}
\solutionspace{100pt}

\tasknumber{10}%
\task{%
    Определите диаметр полутени на экране диска размером $D = 4\,\text{см}$ от протяжённого источника, также обладающего формой диска размером $d = 2\,\text{см}$ (см.
    рис.
    на доске, вид сбоку).
    Расстояние от источника до диска равно $l = 18\,\text{см}$, а расстояние от диска до экрана — $L = 10\,\text{см}$.
}
\answer{%
    $\cfrac{\frac d2 + \frac D2}l = \cfrac{\frac d2 + r}{l + L} \implies r = \cfrac{Dl + dL + DL}{2l} = \cfrac D2 + \cfrac{ L }{ l } \cdot \cfrac{d+D}2 \approx 3{,}7\,\text{см} \implies 2r \approx 7{,}3\,\text{см}$
}

\variantsplitter

\addpersonalvariant{София Журавлёва}

\tasknumber{1}%
\task{%
    Cформулируйте принцип Гюйгенса-Френеля, запишите формулой закон отражения
    и выведите из принципа этот закон.
}
\solutionspace{80pt}

\tasknumber{2}%
\task{%
    Постройте изображения $C'E'$ и $K'{V}'$ стрелок $CE$ и $K{V}$
    в 2 плоских зеркалах соответственно (см.
    рис.
    на доске).
}
\solutionspace{180pt}

\tasknumber{3}%
\task{%
    Постройте область видимости стрелки $AK$ в плоском зеркале (см.
    рис.
    на доске).
}
\solutionspace{120pt}

\tasknumber{4}%
\task{%
    Докажите, что изображение точечного источника света в плоском зеркале можно получить,
        «удвоив» (в векторном смысле) перпендикуляр, опущенный из источника на плоскость зеркала.
}
\solutionspace{120pt}

\tasknumber{5}%
\task{%
    Плоское зеркало вращается с угловой скоростью $0{,}15\,\frac{\text{рад}}{\text{с}}$.
    Ось вращения лежит в плоскости зеркала.
    На зеркало падает луч перпендикулярно оси вращения.
    Определите угловую скорость вращения отражённого луча.
}
\answer{%
    $\omega' = 2 \omega = 0{,}3\,\text{Гц}$
}
\solutionspace{80pt}

\tasknumber{6}%
\task{%
    Плоское зеркало приближается к стационарному предмету размером $7\,\text{см}$
    со скоростью $1\,\frac{\text{см}}{\text{с}}$.
    Определите размер изображения предмета через $4\,\text{с}$ после начала движения,
    если изначальное расстояние между зеркалом и предметом было равно $50\,\text{см}$.
}
\answer{%
    $7\,\text{см}$
}
\solutionspace{80pt}

\tasknumber{7}%
\task{%
    Предмет отдаляется от плоского зеркала со скоростью $2\,\frac{\text{см}}{\text{с}}$.
    Определите скорость изображения, приняв зеркало стационарным.
}
\answer{%
    $2\,\frac{\text{см}}{\text{с}}$
}
\solutionspace{80pt}

\tasknumber{8}%
\task{%
    Запишите своё имя (не фамилию) печатными буквами
    и постройте их изображение в 2 зеркалах: вертикальном и горизонтальном.
    Не забудьте отметить зеркала.
}
\solutionspace{120pt}

\tasknumber{9}%
\task{%
    Луч падает из {what} на стекло с показателем преломления  1{,}45 .
    Сделайте рисунок (без рисунка и отмеченных углов задача не проверяется) и определите:
    \begin{itemize}
        \item угол отражения,
        \item угол преломления,
        \item угол между падающим и отраженным лучом,
        \item угол между падающим и преломленным лучом,
        \item угол отклонения луча при преломлении,
    \end{itemize}
    если угол падения равен $22\degrees$.
}
\answer{%
    \begin{align*}
    \alpha &= 22\degrees, \\
    1 \cdot \sin \alpha &= n \sin \beta \implies \beta = \arcsin\cbr{ \frac{\sin \alpha}{ n } } \approx 14{,}97\degrees, \\
    \varphi_1 &= \alpha \approx 22\degrees, \\
    \varphi_2 &= \beta \approx 14{,}97\degrees, \\
    \varphi_3 &= 2\alpha = 44\degrees, \\
    \varphi_4 &= 180\degrees - \alpha + \beta \approx 172{,}97\degrees, \\
    \varphi_5 &= \alpha - \beta \approx 7{,}03\degrees.
    \end{align*}
}
\solutionspace{100pt}

\tasknumber{10}%
\task{%
    Определите диаметр полутени на экране диска размером $D = 3{,}5\,\text{см}$ от протяжённого источника, также обладающего формой диска размером $d = 4\,\text{см}$ (см.
    рис.
    на доске, вид сбоку).
    Расстояние от источника до диска равно $l = 12\,\text{см}$, а расстояние от диска до экрана — $L = 30\,\text{см}$.
}
\answer{%
    $\cfrac{\frac d2 + \frac D2}l = \cfrac{\frac d2 + r}{l + L} \implies r = \cfrac{Dl + dL + DL}{2l} = \cfrac D2 + \cfrac{ L }{ l } \cdot \cfrac{d+D}2 \approx 11{,}1\,\text{см} \implies 2r \approx 22{,}2\,\text{см}$
}

\variantsplitter

\addpersonalvariant{Константин Козлов}

\tasknumber{1}%
\task{%
    Cформулируйте принцип Гюйгенса-Френеля, запишите формулой закон отражения
    и выведите из принципа этот закон.
}
\solutionspace{80pt}

\tasknumber{2}%
\task{%
    Постройте изображения $A'F'$ и $M'{V}'$ стрелок $AF$ и $M{V}$
    в 2 плоских зеркалах соответственно (см.
    рис.
    на доске).
}
\solutionspace{180pt}

\tasknumber{3}%
\task{%
    Постройте область видимости стрелки $BK$ в плоском зеркале (см.
    рис.
    на доске).
}
\solutionspace{120pt}

\tasknumber{4}%
\task{%
    Докажите, что изображение точечного источника света в плоском зеркале можно получить,
        «удвоив» (в векторном смысле) перпендикуляр, опущенный из источника на плоскость зеркала.
}
\solutionspace{120pt}

\tasknumber{5}%
\task{%
    Плоское зеркало вращается с угловой скоростью $0{,}23\,\frac{\text{рад}}{\text{с}}$.
    Ось вращения лежит в плоскости зеркала.
    На зеркало падает луч перпендикулярно оси вращения.
    Определите угловую скорость вращения отражённого луча.
}
\answer{%
    $\omega' = 2 \omega = 0{,}5\,\text{Гц}$
}
\solutionspace{80pt}

\tasknumber{6}%
\task{%
    Плоское зеркало приближается к стационарному предмету размером $5\,\text{см}$
    со скоростью $3\,\frac{\text{см}}{\text{с}}$.
    Определите размер изображения предмета через $3\,\text{с}$ после начала движения,
    если изначальное расстояние между зеркалом и предметом было равно $70\,\text{см}$.
}
\answer{%
    $5\,\text{см}$
}
\solutionspace{80pt}

\tasknumber{7}%
\task{%
    Предмет приближается к плоскому зеркалу со скоростью $3\,\frac{\text{см}}{\text{с}}$.
    Определите скорость изображения, приняв зеркало стационарным.
}
\answer{%
    $3\,\frac{\text{см}}{\text{с}}$
}
\solutionspace{80pt}

\tasknumber{8}%
\task{%
    Запишите своё имя (не фамилию) печатными буквами
    и постройте их изображение в 2 зеркалах: вертикальном и горизонтальном.
    Не забудьте отметить зеркала.
}
\solutionspace{120pt}

\tasknumber{9}%
\task{%
    Луч падает из {what} на стекло с показателем преломления  1{,}55 .
    Сделайте рисунок (без рисунка и отмеченных углов задача не проверяется) и определите:
    \begin{itemize}
        \item угол отражения,
        \item угол преломления,
        \item угол между падающим и отраженным лучом,
        \item угол между падающим и преломленным лучом,
        \item угол отклонения луча при преломлении,
    \end{itemize}
    если между падающим лучом и границей раздела сред равен $65\degrees$.
}
\answer{%
    \begin{align*}
    \alpha &= 25\degrees, \\
    1 \cdot \sin \alpha &= n \sin \beta \implies \beta = \arcsin\cbr{ \frac{\sin \alpha}{ n } } \approx 35{,}78\degrees, \\
    \varphi_1 &= \alpha \approx 25\degrees, \\
    \varphi_2 &= \beta \approx 35{,}78\degrees, \\
    \varphi_3 &= 2\alpha = 50\degrees, \\
    \varphi_4 &= 180\degrees - \alpha + \beta \approx 190{,}78\degrees, \\
    \varphi_5 &= \alpha - \beta \approx -10{,}78\degrees.
    \end{align*}
}
\solutionspace{100pt}

\tasknumber{10}%
\task{%
    Определите диаметр полутени на экране диска размером $D = 2\,\text{см}$ от протяжённого источника, также обладающего формой диска размером $d = 4\,\text{см}$ (см.
    рис.
    на доске, вид сбоку).
    Расстояние от источника до диска равно $l = 15\,\text{см}$, а расстояние от диска до экрана — $L = 20\,\text{см}$.
}
\answer{%
    $\cfrac{\frac d2 + \frac D2}l = \cfrac{\frac d2 + r}{l + L} \implies r = \cfrac{Dl + dL + DL}{2l} = \cfrac D2 + \cfrac{ L }{ l } \cdot \cfrac{d+D}2 \approx 5\,\text{см} \implies 2r \approx 10\,\text{см}$
}

\variantsplitter

\addpersonalvariant{Наталья Кравченко}

\tasknumber{1}%
\task{%
    Cформулируйте принцип Гюйгенса-Френеля, запишите формулой закон преломления
    и выведите из принципа этот закон.
}
\solutionspace{80pt}

\tasknumber{2}%
\task{%
    Постройте изображения $B'E'$ и $K'{V}'$ стрелок $BE$ и $K{V}$
    в 2 плоских зеркалах соответственно (см.
    рис.
    на доске).
}
\solutionspace{180pt}

\tasknumber{3}%
\task{%
    Постройте область видимости стрелки $BM$ в плоском зеркале (см.
    рис.
    на доске).
}
\solutionspace{120pt}

\tasknumber{4}%
\task{%
    Докажите, что изображение точечного источника света в плоском зеркале можно получить,
        «удвоив» (в векторном смысле) перпендикуляр, опущенный из источника на плоскость зеркала.
}
\solutionspace{120pt}

\tasknumber{5}%
\task{%
    Плоское зеркало вращается с угловой скоростью $0{,}15\,\frac{\text{рад}}{\text{с}}$.
    Ось вращения лежит в плоскости зеркала.
    На зеркало падает луч перпендикулярно оси вращения.
    Определите угловую скорость вращения отражённого луча.
}
\answer{%
    $\omega' = 2 \omega = 0{,}3\,\text{Гц}$
}
\solutionspace{80pt}

\tasknumber{6}%
\task{%
    Плоское зеркало приближается к стационарному предмету размером $7\,\text{см}$
    со скоростью $1\,\frac{\text{см}}{\text{с}}$.
    Определите размер изображения предмета через $3\,\text{с}$ после начала движения,
    если изначальное расстояние между зеркалом и предметом было равно $50\,\text{см}$.
}
\answer{%
    $7\,\text{см}$
}
\solutionspace{80pt}

\tasknumber{7}%
\task{%
    Предмет приближается к плоскому зеркалу со скоростью $3\,\frac{\text{см}}{\text{с}}$.
    Определите скорость изображения, приняв зеркало стационарным.
}
\answer{%
    $3\,\frac{\text{см}}{\text{с}}$
}
\solutionspace{80pt}

\tasknumber{8}%
\task{%
    Запишите своё имя (не фамилию) печатными буквами
    и постройте их изображение в 2 зеркалах: вертикальном и горизонтальном.
    Не забудьте отметить зеркала.
}
\solutionspace{120pt}

\tasknumber{9}%
\task{%
    Луч падает из {what} на стекло с показателем преломления  1{,}55 .
    Сделайте рисунок (без рисунка и отмеченных углов задача не проверяется) и определите:
    \begin{itemize}
        \item угол отражения,
        \item угол преломления,
        \item угол между падающим и отраженным лучом,
        \item угол между падающим и преломленным лучом,
        \item угол отклонения луча при преломлении,
    \end{itemize}
    если между падающим лучом и границей раздела сред равен $55\degrees$.
}
\answer{%
    \begin{align*}
    \alpha &= 35\degrees, \\
    1 \cdot \sin \alpha &= n \sin \beta \implies \beta = \arcsin\cbr{ \frac{\sin \alpha}{ n } } \approx 31{,}90\degrees, \\
    \varphi_1 &= \alpha \approx 35\degrees, \\
    \varphi_2 &= \beta \approx 31{,}90\degrees, \\
    \varphi_3 &= 2\alpha = 70\degrees, \\
    \varphi_4 &= 180\degrees - \alpha + \beta \approx 176{,}90\degrees, \\
    \varphi_5 &= \alpha - \beta \approx 3{,}10\degrees.
    \end{align*}
}
\solutionspace{100pt}

\tasknumber{10}%
\task{%
    Определите радиус полутени на экране диска размером $D = 4\,\text{см}$ от протяжённого источника, также обладающего формой диска размером $d = 3\,\text{см}$ (см.
    рис.
    на доске, вид сбоку).
    Расстояние от источника до диска равно $l = 18\,\text{см}$, а расстояние от диска до экрана — $L = 20\,\text{см}$.
}
\answer{%
    $\cfrac{\frac d2 + \frac D2}l = \cfrac{\frac d2 + r}{l + L} \implies r = \cfrac{Dl + dL + DL}{2l} = \cfrac D2 + \cfrac{ L }{ l } \cdot \cfrac{d+D}2 \approx 5{,}9\,\text{см} \implies 2r \approx 11{,}8\,\text{см}$
}

\variantsplitter

\addpersonalvariant{Матвей Кузьмин}

\tasknumber{1}%
\task{%
    Cформулируйте принцип Гюйгенса-Френеля, запишите формулой закон отражения
    и выведите из принципа этот закон.
}
\solutionspace{80pt}

\tasknumber{2}%
\task{%
    Постройте изображения $C'E'$ и $K'{V}'$ стрелок $CE$ и $K{V}$
    в 2 плоских зеркалах соответственно (см.
    рис.
    на доске).
}
\solutionspace{180pt}

\tasknumber{3}%
\task{%
    Постройте область видимости стрелки $CK$ в плоском зеркале (см.
    рис.
    на доске).
}
\solutionspace{120pt}

\tasknumber{4}%
\task{%
    Докажите, что при повороте плоского зеркала на угол $\varphi$ вокруг оси, лежащей в плоскости зеркала
        и перпендикулярной падающему лучу, этот луч повернётся на угол $2\varphi$.
}
\solutionspace{120pt}

\tasknumber{5}%
\task{%
    Плоское зеркало вращается с угловой скоростью $0{,}16\,\frac{\text{рад}}{\text{с}}$.
    Ось вращения лежит в плоскости зеркала.
    На зеркало падает луч перпендикулярно оси вращения.
    Определите угловую скорость вращения отражённого луча.
}
\answer{%
    $\omega' = 2 \omega = 0{,}3\,\text{Гц}$
}
\solutionspace{80pt}

\tasknumber{6}%
\task{%
    Плоское зеркало приближается к стационарному предмету размером $5\,\text{см}$
    со скоростью $2\,\frac{\text{см}}{\text{с}}$.
    Определите размер изображения предмета через $3\,\text{с}$ после начала движения,
    если изначальное расстояние между зеркалом и предметом было равно $50\,\text{см}$.
}
\answer{%
    $5\,\text{см}$
}
\solutionspace{80pt}

\tasknumber{7}%
\task{%
    Предмет отдаляется от плоского зеркала со скоростью $3\,\frac{\text{см}}{\text{с}}$.
    Определите скорость изображения, приняв зеркало стационарным.
}
\answer{%
    $3\,\frac{\text{см}}{\text{с}}$
}
\solutionspace{80pt}

\tasknumber{8}%
\task{%
    Запишите своё имя (не фамилию) печатными буквами
    и постройте их изображение в 2 зеркалах: вертикальном и горизонтальном.
    Не забудьте отметить зеркала.
}
\solutionspace{120pt}

\tasknumber{9}%
\task{%
    Луч падает из {what} на стекло с показателем преломления  1{,}45 .
    Сделайте рисунок (без рисунка и отмеченных углов задача не проверяется) и определите:
    \begin{itemize}
        \item угол отражения,
        \item угол преломления,
        \item угол между падающим и отраженным лучом,
        \item угол между падающим и преломленным лучом,
        \item угол отклонения луча при преломлении,
    \end{itemize}
    если между падающим лучом и границей раздела сред равен $35\degrees$.
}
\answer{%
    \begin{align*}
    \alpha &= 55\degrees, \\
    1 \cdot \sin \alpha &= n \sin \beta \implies \beta = \arcsin\cbr{ \frac{\sin \alpha}{ n } } \approx 23{,}30\degrees, \\
    \varphi_1 &= \alpha \approx 55\degrees, \\
    \varphi_2 &= \beta \approx 23{,}30\degrees, \\
    \varphi_3 &= 2\alpha = 110\degrees, \\
    \varphi_4 &= 180\degrees - \alpha + \beta \approx 148{,}30\degrees, \\
    \varphi_5 &= \alpha - \beta \approx 31{,}70\degrees.
    \end{align*}
}
\solutionspace{100pt}

\tasknumber{10}%
\task{%
    Определите диаметр полутени на экране диска размером $D = 3{,}5\,\text{см}$ от протяжённого источника, также обладающего формой диска размером $d = 3\,\text{см}$ (см.
    рис.
    на доске, вид сбоку).
    Расстояние от источника до диска равно $l = 18\,\text{см}$, а расстояние от диска до экрана — $L = 10\,\text{см}$.
}
\answer{%
    $\cfrac{\frac d2 + \frac D2}l = \cfrac{\frac d2 + r}{l + L} \implies r = \cfrac{Dl + dL + DL}{2l} = \cfrac D2 + \cfrac{ L }{ l } \cdot \cfrac{d+D}2 \approx 3{,}6\,\text{см} \implies 2r \approx 7{,}1\,\text{см}$
}

\variantsplitter

\addpersonalvariant{Сергей Малышев}

\tasknumber{1}%
\task{%
    Cформулируйте принцип Гюйгенса-Френеля, запишите формулой закон преломления
    и выведите из принципа этот закон.
}
\solutionspace{80pt}

\tasknumber{2}%
\task{%
    Постройте изображения $B'F'$ и $L'{V}'$ стрелок $BF$ и $L{V}$
    в 2 плоских зеркалах соответственно (см.
    рис.
    на доске).
}
\solutionspace{180pt}

\tasknumber{3}%
\task{%
    Постройте область видимости стрелки $BN$ в плоском зеркале (см.
    рис.
    на доске).
}
\solutionspace{120pt}

\tasknumber{4}%
\task{%
    Докажите, что изображение точечного источника света в плоском зеркале можно получить,
        «удвоив» (в векторном смысле) перпендикуляр, опущенный из источника на плоскость зеркала.
}
\solutionspace{120pt}

\tasknumber{5}%
\task{%
    Плоское зеркало вращается с угловой скоростью $0{,}14\,\frac{\text{рад}}{\text{с}}$.
    Ось вращения лежит в плоскости зеркала.
    На зеркало падает луч перпендикулярно оси вращения.
    Определите угловую скорость вращения отражённого луча.
}
\answer{%
    $\omega' = 2 \omega = 0{,}3\,\text{Гц}$
}
\solutionspace{80pt}

\tasknumber{6}%
\task{%
    Плоское зеркало приближается к стационарному предмету размером $5\,\text{см}$
    со скоростью $1\,\frac{\text{см}}{\text{с}}$.
    Определите размер изображения предмета через $5\,\text{с}$ после начала движения,
    если изначальное расстояние между зеркалом и предметом было равно $70\,\text{см}$.
}
\answer{%
    $5\,\text{см}$
}
\solutionspace{80pt}

\tasknumber{7}%
\task{%
    Предмет отдаляется от плоского зеркала со скоростью $4\,\frac{\text{см}}{\text{с}}$.
    Определите скорость изображения, приняв зеркало стационарным.
}
\answer{%
    $4\,\frac{\text{см}}{\text{с}}$
}
\solutionspace{80pt}

\tasknumber{8}%
\task{%
    Запишите своё имя (не фамилию) печатными буквами
    и постройте их изображение в 2 зеркалах: вертикальном и горизонтальном.
    Не забудьте отметить зеркала.
}
\solutionspace{120pt}

\tasknumber{9}%
\task{%
    Луч падает из {what} на стекло с показателем преломления  1{,}55 .
    Сделайте рисунок (без рисунка и отмеченных углов задача не проверяется) и определите:
    \begin{itemize}
        \item угол отражения,
        \item угол преломления,
        \item угол между падающим и отраженным лучом,
        \item угол между падающим и преломленным лучом,
        \item угол отклонения луча при преломлении,
    \end{itemize}
    если между падающим лучом и границей раздела сред равен $65\degrees$.
}
\answer{%
    \begin{align*}
    \alpha &= 25\degrees, \\
    1 \cdot \sin \alpha &= n \sin \beta \implies \beta = \arcsin\cbr{ \frac{\sin \alpha}{ n } } \approx 35{,}78\degrees, \\
    \varphi_1 &= \alpha \approx 25\degrees, \\
    \varphi_2 &= \beta \approx 35{,}78\degrees, \\
    \varphi_3 &= 2\alpha = 50\degrees, \\
    \varphi_4 &= 180\degrees - \alpha + \beta \approx 190{,}78\degrees, \\
    \varphi_5 &= \alpha - \beta \approx -10{,}78\degrees.
    \end{align*}
}
\solutionspace{100pt}

\tasknumber{10}%
\task{%
    Определите диаметр полутени на экране диска размером $D = 2\,\text{см}$ от протяжённого источника, также обладающего формой диска размером $d = 4\,\text{см}$ (см.
    рис.
    на доске, вид сбоку).
    Расстояние от источника до диска равно $l = 18\,\text{см}$, а расстояние от диска до экрана — $L = 30\,\text{см}$.
}
\answer{%
    $\cfrac{\frac d2 + \frac D2}l = \cfrac{\frac d2 + r}{l + L} \implies r = \cfrac{Dl + dL + DL}{2l} = \cfrac D2 + \cfrac{ L }{ l } \cdot \cfrac{d+D}2 \approx 6\,\text{см} \implies 2r \approx 12\,\text{см}$
}

\variantsplitter

\addpersonalvariant{Алина Полканова}

\tasknumber{1}%
\task{%
    Cформулируйте принцип Гюйгенса-Френеля, запишите формулой закон отражения
    и выведите из принципа этот закон.
}
\solutionspace{80pt}

\tasknumber{2}%
\task{%
    Постройте изображения $A'E'$ и $M'{V}'$ стрелок $AE$ и $M{V}$
    в 2 плоских зеркалах соответственно (см.
    рис.
    на доске).
}
\solutionspace{180pt}

\tasknumber{3}%
\task{%
    Постройте область видимости стрелки $DN$ в плоском зеркале (см.
    рис.
    на доске).
}
\solutionspace{120pt}

\tasknumber{4}%
\task{%
    Докажите, что при повороте плоского зеркала на угол $\varphi$ вокруг оси, лежащей в плоскости зеркала
        и перпендикулярной падающему лучу, этот луч повернётся на угол $2\varphi$.
}
\solutionspace{120pt}

\tasknumber{5}%
\task{%
    Плоское зеркало вращается с угловой скоростью $0{,}17\,\frac{\text{рад}}{\text{с}}$.
    Ось вращения лежит в плоскости зеркала.
    На зеркало падает луч перпендикулярно оси вращения.
    Определите угловую скорость вращения отражённого луча.
}
\answer{%
    $\omega' = 2 \omega = 0{,}3\,\text{Гц}$
}
\solutionspace{80pt}

\tasknumber{6}%
\task{%
    Плоское зеркало приближается к стационарному предмету размером $7\,\text{см}$
    со скоростью $1\,\frac{\text{см}}{\text{с}}$.
    Определите размер изображения предмета через $3\,\text{с}$ после начала движения,
    если изначальное расстояние между зеркалом и предметом было равно $50\,\text{см}$.
}
\answer{%
    $7\,\text{см}$
}
\solutionspace{80pt}

\tasknumber{7}%
\task{%
    Предмет приближается к плоскому зеркалу со скоростью $4\,\frac{\text{см}}{\text{с}}$.
    Определите скорость изображения, приняв зеркало стационарным.
}
\answer{%
    $4\,\frac{\text{см}}{\text{с}}$
}
\solutionspace{80pt}

\tasknumber{8}%
\task{%
    Запишите своё имя (не фамилию) печатными буквами
    и постройте их изображение в 2 зеркалах: вертикальном и горизонтальном.
    Не забудьте отметить зеркала.
}
\solutionspace{120pt}

\tasknumber{9}%
\task{%
    Луч падает из {what} на стекло с показателем преломления  1{,}35 .
    Сделайте рисунок (без рисунка и отмеченных углов задача не проверяется) и определите:
    \begin{itemize}
        \item угол отражения,
        \item угол преломления,
        \item угол между падающим и отраженным лучом,
        \item угол между падающим и преломленным лучом,
        \item угол отклонения луча при преломлении,
    \end{itemize}
    если угол падения равен $40\degrees$.
}
\answer{%
    \begin{align*}
    \alpha &= 40\degrees, \\
    1 \cdot \sin \alpha &= n \sin \beta \implies \beta = \arcsin\cbr{ \frac{\sin \alpha}{ n } } \approx 28{,}43\degrees, \\
    \varphi_1 &= \alpha \approx 40\degrees, \\
    \varphi_2 &= \beta \approx 28{,}43\degrees, \\
    \varphi_3 &= 2\alpha = 80\degrees, \\
    \varphi_4 &= 180\degrees - \alpha + \beta \approx 168{,}43\degrees, \\
    \varphi_5 &= \alpha - \beta \approx 11{,}57\degrees.
    \end{align*}
}
\solutionspace{100pt}

\tasknumber{10}%
\task{%
    Определите диаметр полутени на экране диска размером $D = 3\,\text{см}$ от протяжённого источника, также обладающего формой диска размером $d = 3\,\text{см}$ (см.
    рис.
    на доске, вид сбоку).
    Расстояние от источника до диска равно $l = 12\,\text{см}$, а расстояние от диска до экрана — $L = 20\,\text{см}$.
}
\answer{%
    $\cfrac{\frac d2 + \frac D2}l = \cfrac{\frac d2 + r}{l + L} \implies r = \cfrac{Dl + dL + DL}{2l} = \cfrac D2 + \cfrac{ L }{ l } \cdot \cfrac{d+D}2 \approx 6{,}5\,\text{см} \implies 2r \approx 13\,\text{см}$
}

\variantsplitter

\addpersonalvariant{Сергей Пономарёв}

\tasknumber{1}%
\task{%
    Cформулируйте принцип Гюйгенса-Френеля, запишите формулой закон отражения
    и выведите из принципа этот закон.
}
\solutionspace{80pt}

\tasknumber{2}%
\task{%
    Постройте изображения $C'F'$ и $M'{V}'$ стрелок $CF$ и $M{V}$
    в 2 плоских зеркалах соответственно (см.
    рис.
    на доске).
}
\solutionspace{180pt}

\tasknumber{3}%
\task{%
    Постройте область видимости стрелки $AM$ в плоском зеркале (см.
    рис.
    на доске).
}
\solutionspace{120pt}

\tasknumber{4}%
\task{%
    Докажите, что изображение точечного источника света в плоском зеркале можно получить,
        «удвоив» (в векторном смысле) перпендикуляр, опущенный из источника на плоскость зеркала.
}
\solutionspace{120pt}

\tasknumber{5}%
\task{%
    Плоское зеркало вращается с угловой скоростью $0{,}19\,\frac{\text{рад}}{\text{с}}$.
    Ось вращения лежит в плоскости зеркала.
    На зеркало падает луч перпендикулярно оси вращения.
    Определите угловую скорость вращения отражённого луча.
}
\answer{%
    $\omega' = 2 \omega = 0{,}4\,\text{Гц}$
}
\solutionspace{80pt}

\tasknumber{6}%
\task{%
    Плоское зеркало приближается к стационарному предмету размером $7\,\text{см}$
    со скоростью $2\,\frac{\text{см}}{\text{с}}$.
    Определите размер изображения предмета через $3\,\text{с}$ после начала движения,
    если изначальное расстояние между зеркалом и предметом было равно $60\,\text{см}$.
}
\answer{%
    $7\,\text{см}$
}
\solutionspace{80pt}

\tasknumber{7}%
\task{%
    Предмет приближается к плоскому зеркалу со скоростью $2\,\frac{\text{см}}{\text{с}}$.
    Определите скорость изображения, приняв зеркало стационарным.
}
\answer{%
    $2\,\frac{\text{см}}{\text{с}}$
}
\solutionspace{80pt}

\tasknumber{8}%
\task{%
    Запишите своё имя (не фамилию) печатными буквами
    и постройте их изображение в 2 зеркалах: вертикальном и горизонтальном.
    Не забудьте отметить зеркала.
}
\solutionspace{120pt}

\tasknumber{9}%
\task{%
    Луч падает из {what} на стекло с показателем преломления  1{,}55 .
    Сделайте рисунок (без рисунка и отмеченных углов задача не проверяется) и определите:
    \begin{itemize}
        \item угол отражения,
        \item угол преломления,
        \item угол между падающим и отраженным лучом,
        \item угол между падающим и преломленным лучом,
        \item угол отклонения луча при преломлении,
    \end{itemize}
    если между падающим лучом и границей раздела сред равен $50\degrees$.
}
\answer{%
    \begin{align*}
    \alpha &= 40\degrees, \\
    1 \cdot \sin \alpha &= n \sin \beta \implies \beta = \arcsin\cbr{ \frac{\sin \alpha}{ n } } \approx 29{,}62\degrees, \\
    \varphi_1 &= \alpha \approx 40\degrees, \\
    \varphi_2 &= \beta \approx 29{,}62\degrees, \\
    \varphi_3 &= 2\alpha = 80\degrees, \\
    \varphi_4 &= 180\degrees - \alpha + \beta \approx 169{,}62\degrees, \\
    \varphi_5 &= \alpha - \beta \approx 10{,}38\degrees.
    \end{align*}
}
\solutionspace{100pt}

\tasknumber{10}%
\task{%
    Определите диаметр полутени на экране диска размером $D = 2{,}5\,\text{см}$ от протяжённого источника, также обладающего формой диска размером $d = 3\,\text{см}$ (см.
    рис.
    на доске, вид сбоку).
    Расстояние от источника до диска равно $l = 15\,\text{см}$, а расстояние от диска до экрана — $L = 10\,\text{см}$.
}
\answer{%
    $\cfrac{\frac d2 + \frac D2}l = \cfrac{\frac d2 + r}{l + L} \implies r = \cfrac{Dl + dL + DL}{2l} = \cfrac D2 + \cfrac{ L }{ l } \cdot \cfrac{d+D}2 \approx 3{,}1\,\text{см} \implies 2r \approx 6{,}2\,\text{см}$
}

\variantsplitter

\addpersonalvariant{Егор Свистушкин}

\tasknumber{1}%
\task{%
    Cформулируйте принцип Гюйгенса-Френеля, запишите формулой закон отражения
    и выведите из принципа этот закон.
}
\solutionspace{80pt}

\tasknumber{2}%
\task{%
    Постройте изображения $C'D'$ и $K'{V}'$ стрелок $CD$ и $K{V}$
    в 2 плоских зеркалах соответственно (см.
    рис.
    на доске).
}
\solutionspace{180pt}

\tasknumber{3}%
\task{%
    Постройте область видимости стрелки $BN$ в плоском зеркале (см.
    рис.
    на доске).
}
\solutionspace{120pt}

\tasknumber{4}%
\task{%
    Докажите, что изображение точечного источника света в плоском зеркале можно получить,
        «удвоив» (в векторном смысле) перпендикуляр, опущенный из источника на плоскость зеркала.
}
\solutionspace{120pt}

\tasknumber{5}%
\task{%
    Плоское зеркало вращается с угловой скоростью $0{,}18\,\frac{\text{рад}}{\text{с}}$.
    Ось вращения лежит в плоскости зеркала.
    На зеркало падает луч перпендикулярно оси вращения.
    Определите угловую скорость вращения отражённого луча.
}
\answer{%
    $\omega' = 2 \omega = 0{,}4\,\text{Гц}$
}
\solutionspace{80pt}

\tasknumber{6}%
\task{%
    Плоское зеркало приближается к стационарному предмету размером $6\,\text{см}$
    со скоростью $2\,\frac{\text{см}}{\text{с}}$.
    Определите размер изображения предмета через $2\,\text{с}$ после начала движения,
    если изначальное расстояние между зеркалом и предметом было равно $60\,\text{см}$.
}
\answer{%
    $6\,\text{см}$
}
\solutionspace{80pt}

\tasknumber{7}%
\task{%
    Предмет приближается к плоскому зеркалу со скоростью $2\,\frac{\text{см}}{\text{с}}$.
    Определите скорость изображения, приняв зеркало стационарным.
}
\answer{%
    $2\,\frac{\text{см}}{\text{с}}$
}
\solutionspace{80pt}

\tasknumber{8}%
\task{%
    Запишите своё имя (не фамилию) печатными буквами
    и постройте их изображение в 2 зеркалах: вертикальном и горизонтальном.
    Не забудьте отметить зеркала.
}
\solutionspace{120pt}

\tasknumber{9}%
\task{%
    Луч падает из {what} на стекло с показателем преломления  1{,}45 .
    Сделайте рисунок (без рисунка и отмеченных углов задача не проверяется) и определите:
    \begin{itemize}
        \item угол отражения,
        \item угол преломления,
        \item угол между падающим и отраженным лучом,
        \item угол между падающим и преломленным лучом,
        \item угол отклонения луча при преломлении,
    \end{itemize}
    если угол падения равен $22\degrees$.
}
\answer{%
    \begin{align*}
    \alpha &= 22\degrees, \\
    1 \cdot \sin \alpha &= n \sin \beta \implies \beta = \arcsin\cbr{ \frac{\sin \alpha}{ n } } \approx 14{,}97\degrees, \\
    \varphi_1 &= \alpha \approx 22\degrees, \\
    \varphi_2 &= \beta \approx 14{,}97\degrees, \\
    \varphi_3 &= 2\alpha = 44\degrees, \\
    \varphi_4 &= 180\degrees - \alpha + \beta \approx 172{,}97\degrees, \\
    \varphi_5 &= \alpha - \beta \approx 7{,}03\degrees.
    \end{align*}
}
\solutionspace{100pt}

\tasknumber{10}%
\task{%
    Определите диаметр полутени на экране диска размером $D = 3\,\text{см}$ от протяжённого источника, также обладающего формой диска размером $d = 3\,\text{см}$ (см.
    рис.
    на доске, вид сбоку).
    Расстояние от источника до диска равно $l = 15\,\text{см}$, а расстояние от диска до экрана — $L = 20\,\text{см}$.
}
\answer{%
    $\cfrac{\frac d2 + \frac D2}l = \cfrac{\frac d2 + r}{l + L} \implies r = \cfrac{Dl + dL + DL}{2l} = \cfrac D2 + \cfrac{ L }{ l } \cdot \cfrac{d+D}2 \approx 5{,}5\,\text{см} \implies 2r \approx 11\,\text{см}$
}

\variantsplitter

\addpersonalvariant{Дмитрий Соколов}

\tasknumber{1}%
\task{%
    Cформулируйте принцип Гюйгенса-Френеля, запишите формулой закон отражения
    и выведите из принципа этот закон.
}
\solutionspace{80pt}

\tasknumber{2}%
\task{%
    Постройте изображения $C'E'$ и $M'{V}'$ стрелок $CE$ и $M{V}$
    в 2 плоских зеркалах соответственно (см.
    рис.
    на доске).
}
\solutionspace{180pt}

\tasknumber{3}%
\task{%
    Постройте область видимости стрелки $DM$ в плоском зеркале (см.
    рис.
    на доске).
}
\solutionspace{120pt}

\tasknumber{4}%
\task{%
    Докажите, что при повороте плоского зеркала на угол $\varphi$ вокруг оси, лежащей в плоскости зеркала
        и перпендикулярной падающему лучу, этот луч повернётся на угол $2\varphi$.
}
\solutionspace{120pt}

\tasknumber{5}%
\task{%
    Плоское зеркало вращается с угловой скоростью $0{,}15\,\frac{\text{рад}}{\text{с}}$.
    Ось вращения лежит в плоскости зеркала.
    На зеркало падает луч перпендикулярно оси вращения.
    Определите угловую скорость вращения отражённого луча.
}
\answer{%
    $\omega' = 2 \omega = 0{,}3\,\text{Гц}$
}
\solutionspace{80pt}

\tasknumber{6}%
\task{%
    Плоское зеркало приближается к стационарному предмету размером $5\,\text{см}$
    со скоростью $3\,\frac{\text{см}}{\text{с}}$.
    Определите размер изображения предмета через $5\,\text{с}$ после начала движения,
    если изначальное расстояние между зеркалом и предметом было равно $70\,\text{см}$.
}
\answer{%
    $5\,\text{см}$
}
\solutionspace{80pt}

\tasknumber{7}%
\task{%
    Предмет приближается к плоскому зеркалу со скоростью $4\,\frac{\text{см}}{\text{с}}$.
    Определите скорость изображения, приняв зеркало стационарным.
}
\answer{%
    $4\,\frac{\text{см}}{\text{с}}$
}
\solutionspace{80pt}

\tasknumber{8}%
\task{%
    Запишите своё имя (не фамилию) печатными буквами
    и постройте их изображение в 2 зеркалах: вертикальном и горизонтальном.
    Не забудьте отметить зеркала.
}
\solutionspace{120pt}

\tasknumber{9}%
\task{%
    Луч падает из {what} на стекло с показателем преломления  1{,}45 .
    Сделайте рисунок (без рисунка и отмеченных углов задача не проверяется) и определите:
    \begin{itemize}
        \item угол отражения,
        \item угол преломления,
        \item угол между падающим и отраженным лучом,
        \item угол между падающим и преломленным лучом,
        \item угол отклонения луча при преломлении,
    \end{itemize}
    если между падающим лучом и границей раздела сред равен $22\degrees$.
}
\answer{%
    \begin{align*}
    \alpha &= 68\degrees, \\
    1 \cdot \sin \alpha &= n \sin \beta \implies \beta = \arcsin\cbr{ \frac{\sin \alpha}{ n } } \approx 14{,}97\degrees, \\
    \varphi_1 &= \alpha \approx 68\degrees, \\
    \varphi_2 &= \beta \approx 14{,}97\degrees, \\
    \varphi_3 &= 2\alpha = 136\degrees, \\
    \varphi_4 &= 180\degrees - \alpha + \beta \approx 126{,}97\degrees, \\
    \varphi_5 &= \alpha - \beta \approx 53{,}03\degrees.
    \end{align*}
}
\solutionspace{100pt}

\tasknumber{10}%
\task{%
    Определите диаметр полутени на экране диска размером $D = 3{,}5\,\text{см}$ от протяжённого источника, также обладающего формой диска размером $d = 3\,\text{см}$ (см.
    рис.
    на доске, вид сбоку).
    Расстояние от источника до диска равно $l = 18\,\text{см}$, а расстояние от диска до экрана — $L = 20\,\text{см}$.
}
\answer{%
    $\cfrac{\frac d2 + \frac D2}l = \cfrac{\frac d2 + r}{l + L} \implies r = \cfrac{Dl + dL + DL}{2l} = \cfrac D2 + \cfrac{ L }{ l } \cdot \cfrac{d+D}2 \approx 5{,}4\,\text{см} \implies 2r \approx 10{,}7\,\text{см}$
}

\variantsplitter

\addpersonalvariant{Арсений Трофимов}

\tasknumber{1}%
\task{%
    Cформулируйте принцип Гюйгенса-Френеля, запишите формулой закон преломления
    и выведите из принципа этот закон.
}
\solutionspace{80pt}

\tasknumber{2}%
\task{%
    Постройте изображения $A'E'$ и $M'{V}'$ стрелок $AE$ и $M{V}$
    в 2 плоских зеркалах соответственно (см.
    рис.
    на доске).
}
\solutionspace{180pt}

\tasknumber{3}%
\task{%
    Постройте область видимости стрелки $CN$ в плоском зеркале (см.
    рис.
    на доске).
}
\solutionspace{120pt}

\tasknumber{4}%
\task{%
    Докажите, что изображение точечного источника света в плоском зеркале можно получить,
        «удвоив» (в векторном смысле) перпендикуляр, опущенный из источника на плоскость зеркала.
}
\solutionspace{120pt}

\tasknumber{5}%
\task{%
    Плоское зеркало вращается с угловой скоростью $0{,}28\,\frac{\text{рад}}{\text{с}}$.
    Ось вращения лежит в плоскости зеркала.
    На зеркало падает луч перпендикулярно оси вращения.
    Определите угловую скорость вращения отражённого луча.
}
\answer{%
    $\omega' = 2 \omega = 0{,}6\,\text{Гц}$
}
\solutionspace{80pt}

\tasknumber{6}%
\task{%
    Плоское зеркало приближается к стационарному предмету размером $5\,\text{см}$
    со скоростью $1\,\frac{\text{см}}{\text{с}}$.
    Определите размер изображения предмета через $2\,\text{с}$ после начала движения,
    если изначальное расстояние между зеркалом и предметом было равно $60\,\text{см}$.
}
\answer{%
    $5\,\text{см}$
}
\solutionspace{80pt}

\tasknumber{7}%
\task{%
    Предмет приближается к плоскому зеркалу со скоростью $4\,\frac{\text{см}}{\text{с}}$.
    Определите скорость изображения, приняв зеркало стационарным.
}
\answer{%
    $4\,\frac{\text{см}}{\text{с}}$
}
\solutionspace{80pt}

\tasknumber{8}%
\task{%
    Запишите своё имя (не фамилию) печатными буквами
    и постройте их изображение в 2 зеркалах: вертикальном и горизонтальном.
    Не забудьте отметить зеркала.
}
\solutionspace{120pt}

\tasknumber{9}%
\task{%
    Луч падает из {what} на стекло с показателем преломления  1{,}65 .
    Сделайте рисунок (без рисунка и отмеченных углов задача не проверяется) и определите:
    \begin{itemize}
        \item угол отражения,
        \item угол преломления,
        \item угол между падающим и отраженным лучом,
        \item угол между падающим и преломленным лучом,
        \item угол отклонения луча при преломлении,
    \end{itemize}
    если между падающим лучом и границей раздела сред равен $40\degrees$.
}
\answer{%
    \begin{align*}
    \alpha &= 50\degrees, \\
    1 \cdot \sin \alpha &= n \sin \beta \implies \beta = \arcsin\cbr{ \frac{\sin \alpha}{ n } } \approx 22{,}93\degrees, \\
    \varphi_1 &= \alpha \approx 50\degrees, \\
    \varphi_2 &= \beta \approx 22{,}93\degrees, \\
    \varphi_3 &= 2\alpha = 100\degrees, \\
    \varphi_4 &= 180\degrees - \alpha + \beta \approx 152{,}93\degrees, \\
    \varphi_5 &= \alpha - \beta \approx 27{,}07\degrees.
    \end{align*}
}
\solutionspace{100pt}

\tasknumber{10}%
\task{%
    Определите диаметр полутени на экране диска размером $D = 3\,\text{см}$ от протяжённого источника, также обладающего формой диска размером $d = 4\,\text{см}$ (см.
    рис.
    на доске, вид сбоку).
    Расстояние от источника до диска равно $l = 18\,\text{см}$, а расстояние от диска до экрана — $L = 30\,\text{см}$.
}
\answer{%
    $\cfrac{\frac d2 + \frac D2}l = \cfrac{\frac d2 + r}{l + L} \implies r = \cfrac{Dl + dL + DL}{2l} = \cfrac D2 + \cfrac{ L }{ l } \cdot \cfrac{d+D}2 \approx 7{,}3\,\text{см} \implies 2r \approx 14{,}7\,\text{см}$
}
% autogenerated
