\setdate{2~декабря~2021}
\setclass{11«БА»}

\addpersonalvariant{Михаил Бурмистров}

\tasknumber{1}%
\task{%
    На дифракционную решётку, имеющую период $2 \cdot 10^{-4}\,\text{см}$, нормально падает монохроматическая световая волна.
    Под углом $ 35 \degrees$ наблюдается дифракционный максимум четвёртого порядка.
    Какова длина волны падающего света?
}
\solutionspace{150pt}

\tasknumber{2}%
\task{%
    Свет с длиной волны $0{,}4\,\text{мкм}$ падает нормально на дифракционную решётку с периодом, равным $1\,\text{мкм}$.
    Под каким углом наблюдается дифракционный максимум первого порядка?
}
\solutionspace{150pt}

\tasknumber{3}%
\task{%
    При нормальном падении белого света на дифракционную решётку зелёная линия ($550\,\text{нм}$)
    в спектре четвёртого порядка видна под углом дифракции $5\degrees$.
    Определить число штрихов на $1\,\text{мм}$ длины этой решётки.
}
\solutionspace{150pt}

\tasknumber{4}%
\task{%
    Каков наибольший порядок спектра, который можно наблюдать при дифракции света
    с длиной волны $\lambda$, на дифракционной решётке с периодом $d =  4{,}5 \lambda$?
}

\variantsplitter

\addpersonalvariant{Ирина Ан}

\tasknumber{1}%
\task{%
    На дифракционную решётку, имеющую период $3 \cdot 10^{-4}\,\text{см}$, нормально падает монохроматическая световая волна.
    Под углом $ 40 \degrees$ наблюдается дифракционный максимум третьего порядка.
    Какова длина волны падающего света?
}
\solutionspace{150pt}

\tasknumber{2}%
\task{%
    Свет с длиной волны $0{,}5\,\text{мкм}$ падает нормально на дифракционную решётку с периодом, равным $2\,\text{мкм}$.
    Под каким углом наблюдается дифракционный максимум первого порядка?
}
\solutionspace{150pt}

\tasknumber{3}%
\task{%
    При нормальном падении белого света на дифракционную решётку оранжевая линия ($600\,\text{нм}$)
    в спектре четвёртого порядка видна под углом дифракции $25\degrees$.
    Определить число штрихов на $1\,\text{мм}$ длины этой решётки.
}
\solutionspace{150pt}

\tasknumber{4}%
\task{%
    Каков наибольший порядок спектра, который можно наблюдать при дифракции света
    с длиной волны $\lambda$, на дифракционной решётке с периодом $d =  4{,}1 \lambda$?
}

\variantsplitter

\addpersonalvariant{Софья Андрианова}

\tasknumber{1}%
\task{%
    На дифракционную решётку, имеющую период $2 \cdot 10^{-4}\,\text{см}$, нормально падает монохроматическая световая волна.
    Под углом $ 25 \degrees$ наблюдается дифракционный максимум второго порядка.
    Какова длина волны падающего света?
}
\solutionspace{150pt}

\tasknumber{2}%
\task{%
    Свет с длиной волны $0{,}5\,\text{мкм}$ падает нормально на дифракционную решётку с периодом, равным $1\,\text{мкм}$.
    Под каким углом наблюдается дифракционный максимум первого порядка?
}
\solutionspace{150pt}

\tasknumber{3}%
\task{%
    При нормальном падении белого света на дифракционную решётку красная линия ($720\,\text{нм}$)
    в спектре третьего порядка видна под углом дифракции $25\degrees$.
    Определить число штрихов на $1\,\text{мм}$ длины этой решётки.
}
\solutionspace{150pt}

\tasknumber{4}%
\task{%
    Каков наибольший порядок спектра, который можно наблюдать при дифракции света
    с длиной волны $\lambda$, на дифракционной решётке с периодом $d =  2{,}5 \lambda$?
}

\variantsplitter

\addpersonalvariant{Владимир Артемчук}

\tasknumber{1}%
\task{%
    На дифракционную решётку, имеющую период $3 \cdot 10^{-4}\,\text{см}$, нормально падает монохроматическая световая волна.
    Под углом $ 20 \degrees$ наблюдается дифракционный максимум третьего порядка.
    Какова длина волны падающего света?
}
\solutionspace{150pt}

\tasknumber{2}%
\task{%
    Свет с длиной волны $0{,}4\,\text{мкм}$ падает нормально на дифракционную решётку с периодом, равным $2\,\text{мкм}$.
    Под каким углом наблюдается дифракционный максимум первого порядка?
}
\solutionspace{150pt}

\tasknumber{3}%
\task{%
    При нормальном падении белого света на дифракционную решётку оранжевая линия ($600\,\text{нм}$)
    в спектре третьего порядка видна под углом дифракции $25\degrees$.
    Определить число штрихов на $1\,\text{см}$ длины этой решётки.
}
\solutionspace{150pt}

\tasknumber{4}%
\task{%
    Каков наибольший порядок спектра, который можно наблюдать при дифракции света
    с длиной волны $\lambda$, на дифракционной решётке с периодом $d =  3{,}3 \lambda$?
}

\variantsplitter

\addpersonalvariant{Софья Белянкина}

\tasknumber{1}%
\task{%
    На дифракционную решётку, имеющую период $2 \cdot 10^{-4}\,\text{см}$, нормально падает монохроматическая световая волна.
    Под углом $ 20 \degrees$ наблюдается дифракционный максимум третьего порядка.
    Какова длина волны падающего света?
}
\solutionspace{150pt}

\tasknumber{2}%
\task{%
    Свет с длиной волны $0{,}7\,\text{мкм}$ падает нормально на дифракционную решётку с периодом, равным $2\,\text{мкм}$.
    Под каким углом наблюдается дифракционный максимум первого порядка?
}
\solutionspace{150pt}

\tasknumber{3}%
\task{%
    При нормальном падении белого света на дифракционную решётку синяя линия ($480\,\text{нм}$)
    в спектре четвёртого порядка видна под углом дифракции $18\degrees$.
    Определить число штрихов на $1\,\text{см}$ длины этой решётки.
}
\solutionspace{150pt}

\tasknumber{4}%
\task{%
    Каков наибольший порядок спектра, который можно наблюдать при дифракции света
    с длиной волны $\lambda$, на дифракционной решётке с периодом $d =  2{,}2 \lambda$?
}

\variantsplitter

\addpersonalvariant{Варвара Егиазарян}

\tasknumber{1}%
\task{%
    На дифракционную решётку, имеющую период $2 \cdot 10^{-4}\,\text{см}$, нормально падает монохроматическая световая волна.
    Под углом $ 25 \degrees$ наблюдается дифракционный максимум третьего порядка.
    Какова длина волны падающего света?
}
\solutionspace{150pt}

\tasknumber{2}%
\task{%
    Свет с длиной волны $0{,}4\,\text{мкм}$ падает нормально на дифракционную решётку с периодом, равным $1\,\text{мкм}$.
    Под каким углом наблюдается дифракционный максимум первого порядка?
}
\solutionspace{150pt}

\tasknumber{3}%
\task{%
    При нормальном падении белого света на дифракционную решётку жёлтая линия ($580\,\text{нм}$)
    в спектре четвёртого порядка видна под углом дифракции $18\degrees$.
    Определить число штрихов на $1\,\text{см}$ длины этой решётки.
}
\solutionspace{150pt}

\tasknumber{4}%
\task{%
    Каков наибольший порядок спектра, который можно наблюдать при дифракции света
    с длиной волны $\lambda$, на дифракционной решётке с периодом $d =  4{,}1 \lambda$?
}

\variantsplitter

\addpersonalvariant{Владислав Емелин}

\tasknumber{1}%
\task{%
    На дифракционную решётку, имеющую период $2 \cdot 10^{-4}\,\text{см}$, нормально падает монохроматическая световая волна.
    Под углом $ 25 \degrees$ наблюдается дифракционный максимум второго порядка.
    Какова длина волны падающего света?
}
\solutionspace{150pt}

\tasknumber{2}%
\task{%
    Свет с длиной волны $0{,}7\,\text{мкм}$ падает нормально на дифракционную решётку с периодом, равным $2\,\text{мкм}$.
    Под каким углом наблюдается дифракционный максимум первого порядка?
}
\solutionspace{150pt}

\tasknumber{3}%
\task{%
    При нормальном падении белого света на дифракционную решётку жёлтая линия ($580\,\text{нм}$)
    в спектре четвёртого порядка видна под углом дифракции $12\degrees$.
    Определить число штрихов на $1\,\text{мм}$ длины этой решётки.
}
\solutionspace{150pt}

\tasknumber{4}%
\task{%
    Каков наибольший порядок спектра, который можно наблюдать при дифракции света
    с длиной волны $\lambda$, на дифракционной решётке с периодом $d =  2{,}7 \lambda$?
}

\variantsplitter

\addpersonalvariant{Артём Жичин}

\tasknumber{1}%
\task{%
    На дифракционную решётку, имеющую период $4 \cdot 10^{-4}\,\text{см}$, нормально падает монохроматическая световая волна.
    Под углом $ 30 \degrees$ наблюдается дифракционный максимум второго порядка.
    Какова длина волны падающего света?
}
\solutionspace{150pt}

\tasknumber{2}%
\task{%
    Свет с длиной волны $0{,}7\,\text{мкм}$ падает нормально на дифракционную решётку с периодом, равным $3\,\text{мкм}$.
    Под каким углом наблюдается дифракционный максимум первого порядка?
}
\solutionspace{150pt}

\tasknumber{3}%
\task{%
    При нормальном падении белого света на дифракционную решётку жёлтая линия ($570\,\text{нм}$)
    в спектре второго порядка видна под углом дифракции $12\degrees$.
    Определить число штрихов на $1\,\text{см}$ длины этой решётки.
}
\solutionspace{150pt}

\tasknumber{4}%
\task{%
    Каков наибольший порядок спектра, который можно наблюдать при дифракции света
    с длиной волны $\lambda$, на дифракционной решётке с периодом $d =  3{,}9 \lambda$?
}

\variantsplitter

\addpersonalvariant{Дарья Кошман}

\tasknumber{1}%
\task{%
    На дифракционную решётку, имеющую период $2 \cdot 10^{-4}\,\text{см}$, нормально падает монохроматическая световая волна.
    Под углом $ 35 \degrees$ наблюдается дифракционный максимум четвёртого порядка.
    Какова длина волны падающего света?
}
\solutionspace{150pt}

\tasknumber{2}%
\task{%
    Свет с длиной волны $0{,}7\,\text{мкм}$ падает нормально на дифракционную решётку с периодом, равным $2\,\text{мкм}$.
    Под каким углом наблюдается дифракционный максимум первого порядка?
}
\solutionspace{150pt}

\tasknumber{3}%
\task{%
    При нормальном падении белого света на дифракционную решётку синяя линия ($450\,\text{нм}$)
    в спектре третьего порядка видна под углом дифракции $18\degrees$.
    Определить число штрихов на $1\,\text{мм}$ длины этой решётки.
}
\solutionspace{150pt}

\tasknumber{4}%
\task{%
    Каков наибольший порядок спектра, который можно наблюдать при дифракции света
    с длиной волны $\lambda$, на дифракционной решётке с периодом $d =  2{,}7 \lambda$?
}

\variantsplitter

\addpersonalvariant{Анна Кузьмичёва}

\tasknumber{1}%
\task{%
    На дифракционную решётку, имеющую период $4 \cdot 10^{-4}\,\text{см}$, нормально падает монохроматическая световая волна.
    Под углом $ 35 \degrees$ наблюдается дифракционный максимум второго порядка.
    Какова длина волны падающего света?
}
\solutionspace{150pt}

\tasknumber{2}%
\task{%
    Свет с длиной волны $0{,}4\,\text{мкм}$ падает нормально на дифракционную решётку с периодом, равным $3\,\text{мкм}$.
    Под каким углом наблюдается дифракционный максимум первого порядка?
}
\solutionspace{150pt}

\tasknumber{3}%
\task{%
    При нормальном падении белого света на дифракционную решётку жёлтая линия ($580\,\text{нм}$)
    в спектре второго порядка видна под углом дифракции $5\degrees$.
    Определить число штрихов на $1\,\text{см}$ длины этой решётки.
}
\solutionspace{150pt}

\tasknumber{4}%
\task{%
    Каков наибольший порядок спектра, который можно наблюдать при дифракции света
    с длиной волны $\lambda$, на дифракционной решётке с периодом $d =  3{,}9 \lambda$?
}

\variantsplitter

\addpersonalvariant{Алёна Куприянова}

\tasknumber{1}%
\task{%
    На дифракционную решётку, имеющую период $4 \cdot 10^{-4}\,\text{см}$, нормально падает монохроматическая световая волна.
    Под углом $ 25 \degrees$ наблюдается дифракционный максимум четвёртого порядка.
    Какова длина волны падающего света?
}
\solutionspace{150pt}

\tasknumber{2}%
\task{%
    Свет с длиной волны $0{,}5\,\text{мкм}$ падает нормально на дифракционную решётку с периодом, равным $1\,\text{мкм}$.
    Под каким углом наблюдается дифракционный максимум первого порядка?
}
\solutionspace{150pt}

\tasknumber{3}%
\task{%
    При нормальном падении белого света на дифракционную решётку зелёная линия ($520\,\text{нм}$)
    в спектре второго порядка видна под углом дифракции $5\degrees$.
    Определить число штрихов на $1\,\text{см}$ длины этой решётки.
}
\solutionspace{150pt}

\tasknumber{4}%
\task{%
    Каков наибольший порядок спектра, который можно наблюдать при дифракции света
    с длиной волны $\lambda$, на дифракционной решётке с периодом $d =  4{,}5 \lambda$?
}

\variantsplitter

\addpersonalvariant{Ярослав Лавровский}

\tasknumber{1}%
\task{%
    На дифракционную решётку, имеющую период $2 \cdot 10^{-4}\,\text{см}$, нормально падает монохроматическая световая волна.
    Под углом $ 20 \degrees$ наблюдается дифракционный максимум третьего порядка.
    Какова длина волны падающего света?
}
\solutionspace{150pt}

\tasknumber{2}%
\task{%
    Свет с длиной волны $0{,}5\,\text{мкм}$ падает нормально на дифракционную решётку с периодом, равным $2\,\text{мкм}$.
    Под каким углом наблюдается дифракционный максимум первого порядка?
}
\solutionspace{150pt}

\tasknumber{3}%
\task{%
    При нормальном падении белого света на дифракционную решётку зелёная линия ($550\,\text{нм}$)
    в спектре третьего порядка видна под углом дифракции $12\degrees$.
    Определить число штрихов на $1\,\text{мм}$ длины этой решётки.
}
\solutionspace{150pt}

\tasknumber{4}%
\task{%
    Каков наибольший порядок спектра, который можно наблюдать при дифракции света
    с длиной волны $\lambda$, на дифракционной решётке с периодом $d =  4{,}1 \lambda$?
}

\variantsplitter

\addpersonalvariant{Анастасия Ламанова}

\tasknumber{1}%
\task{%
    На дифракционную решётку, имеющую период $4 \cdot 10^{-4}\,\text{см}$, нормально падает монохроматическая световая волна.
    Под углом $ 35 \degrees$ наблюдается дифракционный максимум второго порядка.
    Какова длина волны падающего света?
}
\solutionspace{150pt}

\tasknumber{2}%
\task{%
    Свет с длиной волны $0{,}7\,\text{мкм}$ падает нормально на дифракционную решётку с периодом, равным $2\,\text{мкм}$.
    Под каким углом наблюдается дифракционный максимум первого порядка?
}
\solutionspace{150pt}

\tasknumber{3}%
\task{%
    При нормальном падении белого света на дифракционную решётку красная линия ($680\,\text{нм}$)
    в спектре четвёртого порядка видна под углом дифракции $18\degrees$.
    Определить число штрихов на $1\,\text{см}$ длины этой решётки.
}
\solutionspace{150pt}

\tasknumber{4}%
\task{%
    Каков наибольший порядок спектра, который можно наблюдать при дифракции света
    с длиной волны $\lambda$, на дифракционной решётке с периодом $d =  2{,}2 \lambda$?
}

\variantsplitter

\addpersonalvariant{Виктория Легонькова}

\tasknumber{1}%
\task{%
    На дифракционную решётку, имеющую период $3 \cdot 10^{-4}\,\text{см}$, нормально падает монохроматическая световая волна.
    Под углом $ 25 \degrees$ наблюдается дифракционный максимум четвёртого порядка.
    Какова длина волны падающего света?
}
\solutionspace{150pt}

\tasknumber{2}%
\task{%
    Свет с длиной волны $0{,}6\,\text{мкм}$ падает нормально на дифракционную решётку с периодом, равным $3\,\text{мкм}$.
    Под каким углом наблюдается дифракционный максимум первого порядка?
}
\solutionspace{150pt}

\tasknumber{3}%
\task{%
    При нормальном падении белого света на дифракционную решётку синяя линия ($450\,\text{нм}$)
    в спектре четвёртого порядка видна под углом дифракции $5\degrees$.
    Определить число штрихов на $1\,\text{мм}$ длины этой решётки.
}
\solutionspace{150pt}

\tasknumber{4}%
\task{%
    Каков наибольший порядок спектра, который можно наблюдать при дифракции света
    с длиной волны $\lambda$, на дифракционной решётке с периодом $d =  4{,}1 \lambda$?
}

\variantsplitter

\addpersonalvariant{Семён Мартынов}

\tasknumber{1}%
\task{%
    На дифракционную решётку, имеющую период $3 \cdot 10^{-4}\,\text{см}$, нормально падает монохроматическая световая волна.
    Под углом $ 25 \degrees$ наблюдается дифракционный максимум второго порядка.
    Какова длина волны падающего света?
}
\solutionspace{150pt}

\tasknumber{2}%
\task{%
    Свет с длиной волны $0{,}6\,\text{мкм}$ падает нормально на дифракционную решётку с периодом, равным $2\,\text{мкм}$.
    Под каким углом наблюдается дифракционный максимум первого порядка?
}
\solutionspace{150pt}

\tasknumber{3}%
\task{%
    При нормальном падении белого света на дифракционную решётку синяя линия ($480\,\text{нм}$)
    в спектре третьего порядка видна под углом дифракции $18\degrees$.
    Определить число штрихов на $1\,\text{мм}$ длины этой решётки.
}
\solutionspace{150pt}

\tasknumber{4}%
\task{%
    Каков наибольший порядок спектра, который можно наблюдать при дифракции света
    с длиной волны $\lambda$, на дифракционной решётке с периодом $d =  4{,}6 \lambda$?
}

\variantsplitter

\addpersonalvariant{Варвара Минаева}

\tasknumber{1}%
\task{%
    На дифракционную решётку, имеющую период $4 \cdot 10^{-4}\,\text{см}$, нормально падает монохроматическая световая волна.
    Под углом $ 40 \degrees$ наблюдается дифракционный максимум четвёртого порядка.
    Какова длина волны падающего света?
}
\solutionspace{150pt}

\tasknumber{2}%
\task{%
    Свет с длиной волны $0{,}6\,\text{мкм}$ падает нормально на дифракционную решётку с периодом, равным $1\,\text{мкм}$.
    Под каким углом наблюдается дифракционный максимум первого порядка?
}
\solutionspace{150pt}

\tasknumber{3}%
\task{%
    При нормальном падении белого света на дифракционную решётку синяя линия ($480\,\text{нм}$)
    в спектре третьего порядка видна под углом дифракции $18\degrees$.
    Определить число штрихов на $1\,\text{мм}$ длины этой решётки.
}
\solutionspace{150pt}

\tasknumber{4}%
\task{%
    Каков наибольший порядок спектра, который можно наблюдать при дифракции света
    с длиной волны $\lambda$, на дифракционной решётке с периодом $d =  3{,}5 \lambda$?
}

\variantsplitter

\addpersonalvariant{Леонид Никитин}

\tasknumber{1}%
\task{%
    На дифракционную решётку, имеющую период $4 \cdot 10^{-4}\,\text{см}$, нормально падает монохроматическая световая волна.
    Под углом $ 20 \degrees$ наблюдается дифракционный максимум четвёртого порядка.
    Какова длина волны падающего света?
}
\solutionspace{150pt}

\tasknumber{2}%
\task{%
    Свет с длиной волны $0{,}7\,\text{мкм}$ падает нормально на дифракционную решётку с периодом, равным $3\,\text{мкм}$.
    Под каким углом наблюдается дифракционный максимум первого порядка?
}
\solutionspace{150pt}

\tasknumber{3}%
\task{%
    При нормальном падении белого света на дифракционную решётку жёлтая линия ($580\,\text{нм}$)
    в спектре четвёртого порядка видна под углом дифракции $5\degrees$.
    Определить число штрихов на $1\,\text{см}$ длины этой решётки.
}
\solutionspace{150pt}

\tasknumber{4}%
\task{%
    Каков наибольший порядок спектра, который можно наблюдать при дифракции света
    с длиной волны $\lambda$, на дифракционной решётке с периодом $d =  2{,}2 \lambda$?
}

\variantsplitter

\addpersonalvariant{Тимофей Полетаев}

\tasknumber{1}%
\task{%
    На дифракционную решётку, имеющую период $3 \cdot 10^{-4}\,\text{см}$, нормально падает монохроматическая световая волна.
    Под углом $ 30 \degrees$ наблюдается дифракционный максимум второго порядка.
    Какова длина волны падающего света?
}
\solutionspace{150pt}

\tasknumber{2}%
\task{%
    Свет с длиной волны $0{,}5\,\text{мкм}$ падает нормально на дифракционную решётку с периодом, равным $2\,\text{мкм}$.
    Под каким углом наблюдается дифракционный максимум первого порядка?
}
\solutionspace{150pt}

\tasknumber{3}%
\task{%
    При нормальном падении белого света на дифракционную решётку жёлтая линия ($570\,\text{нм}$)
    в спектре второго порядка видна под углом дифракции $25\degrees$.
    Определить число штрихов на $1\,\text{мм}$ длины этой решётки.
}
\solutionspace{150pt}

\tasknumber{4}%
\task{%
    Каков наибольший порядок спектра, который можно наблюдать при дифракции света
    с длиной волны $\lambda$, на дифракционной решётке с периодом $d =  3{,}3 \lambda$?
}

\variantsplitter

\addpersonalvariant{Андрей Рожков}

\tasknumber{1}%
\task{%
    На дифракционную решётку, имеющую период $4 \cdot 10^{-4}\,\text{см}$, нормально падает монохроматическая световая волна.
    Под углом $ 25 \degrees$ наблюдается дифракционный максимум третьего порядка.
    Какова длина волны падающего света?
}
\solutionspace{150pt}

\tasknumber{2}%
\task{%
    Свет с длиной волны $0{,}5\,\text{мкм}$ падает нормально на дифракционную решётку с периодом, равным $2\,\text{мкм}$.
    Под каким углом наблюдается дифракционный максимум первого порядка?
}
\solutionspace{150pt}

\tasknumber{3}%
\task{%
    При нормальном падении белого света на дифракционную решётку красная линия ($720\,\text{нм}$)
    в спектре третьего порядка видна под углом дифракции $25\degrees$.
    Определить число штрихов на $1\,\text{мм}$ длины этой решётки.
}
\solutionspace{150pt}

\tasknumber{4}%
\task{%
    Каков наибольший порядок спектра, который можно наблюдать при дифракции света
    с длиной волны $\lambda$, на дифракционной решётке с периодом $d =  3{,}9 \lambda$?
}

\variantsplitter

\addpersonalvariant{Рената Таржиманова}

\tasknumber{1}%
\task{%
    На дифракционную решётку, имеющую период $2 \cdot 10^{-4}\,\text{см}$, нормально падает монохроматическая световая волна.
    Под углом $ 25 \degrees$ наблюдается дифракционный максимум второго порядка.
    Какова длина волны падающего света?
}
\solutionspace{150pt}

\tasknumber{2}%
\task{%
    Свет с длиной волны $0{,}6\,\text{мкм}$ падает нормально на дифракционную решётку с периодом, равным $1\,\text{мкм}$.
    Под каким углом наблюдается дифракционный максимум первого порядка?
}
\solutionspace{150pt}

\tasknumber{3}%
\task{%
    При нормальном падении белого света на дифракционную решётку жёлтая линия ($570\,\text{нм}$)
    в спектре третьего порядка видна под углом дифракции $25\degrees$.
    Определить число штрихов на $1\,\text{см}$ длины этой решётки.
}
\solutionspace{150pt}

\tasknumber{4}%
\task{%
    Каков наибольший порядок спектра, который можно наблюдать при дифракции света
    с длиной волны $\lambda$, на дифракционной решётке с периодом $d =  4{,}5 \lambda$?
}

\variantsplitter

\addpersonalvariant{Андрей Щербаков}

\tasknumber{1}%
\task{%
    На дифракционную решётку, имеющую период $3 \cdot 10^{-4}\,\text{см}$, нормально падает монохроматическая световая волна.
    Под углом $ 25 \degrees$ наблюдается дифракционный максимум третьего порядка.
    Какова длина волны падающего света?
}
\solutionspace{150pt}

\tasknumber{2}%
\task{%
    Свет с длиной волны $0{,}7\,\text{мкм}$ падает нормально на дифракционную решётку с периодом, равным $3\,\text{мкм}$.
    Под каким углом наблюдается дифракционный максимум первого порядка?
}
\solutionspace{150pt}

\tasknumber{3}%
\task{%
    При нормальном падении белого света на дифракционную решётку зелёная линия ($520\,\text{нм}$)
    в спектре второго порядка видна под углом дифракции $25\degrees$.
    Определить число штрихов на $1\,\text{мм}$ длины этой решётки.
}
\solutionspace{150pt}

\tasknumber{4}%
\task{%
    Каков наибольший порядок спектра, который можно наблюдать при дифракции света
    с длиной волны $\lambda$, на дифракционной решётке с периодом $d =  4{,}5 \lambda$?
}

\variantsplitter

\addpersonalvariant{Михаил Ярошевский}

\tasknumber{1}%
\task{%
    На дифракционную решётку, имеющую период $2 \cdot 10^{-4}\,\text{см}$, нормально падает монохроматическая световая волна.
    Под углом $ 40 \degrees$ наблюдается дифракционный максимум второго порядка.
    Какова длина волны падающего света?
}
\solutionspace{150pt}

\tasknumber{2}%
\task{%
    Свет с длиной волны $0{,}5\,\text{мкм}$ падает нормально на дифракционную решётку с периодом, равным $2\,\text{мкм}$.
    Под каким углом наблюдается дифракционный максимум первого порядка?
}
\solutionspace{150pt}

\tasknumber{3}%
\task{%
    При нормальном падении белого света на дифракционную решётку жёлтая линия ($570\,\text{нм}$)
    в спектре третьего порядка видна под углом дифракции $12\degrees$.
    Определить число штрихов на $1\,\text{мм}$ длины этой решётки.
}
\solutionspace{150pt}

\tasknumber{4}%
\task{%
    Каков наибольший порядок спектра, который можно наблюдать при дифракции света
    с длиной волны $\lambda$, на дифракционной решётке с периодом $d =  3{,}9 \lambda$?
}

\variantsplitter

\addpersonalvariant{Алексей Алимпиев}

\tasknumber{1}%
\task{%
    На дифракционную решётку, имеющую период $3 \cdot 10^{-4}\,\text{см}$, нормально падает монохроматическая световая волна.
    Под углом $ 35 \degrees$ наблюдается дифракционный максимум четвёртого порядка.
    Какова длина волны падающего света?
}
\solutionspace{150pt}

\tasknumber{2}%
\task{%
    Свет с длиной волны $0{,}6\,\text{мкм}$ падает нормально на дифракционную решётку с периодом, равным $3\,\text{мкм}$.
    Под каким углом наблюдается дифракционный максимум первого порядка?
}
\solutionspace{150pt}

\tasknumber{3}%
\task{%
    При нормальном падении белого света на дифракционную решётку красная линия ($720\,\text{нм}$)
    в спектре второго порядка видна под углом дифракции $18\degrees$.
    Определить число штрихов на $1\,\text{мм}$ длины этой решётки.
}
\solutionspace{150pt}

\tasknumber{4}%
\task{%
    Каков наибольший порядок спектра, который можно наблюдать при дифракции света
    с длиной волны $\lambda$, на дифракционной решётке с периодом $d =  4{,}1 \lambda$?
}

\variantsplitter

\addpersonalvariant{Евгений Васин}

\tasknumber{1}%
\task{%
    На дифракционную решётку, имеющую период $3 \cdot 10^{-4}\,\text{см}$, нормально падает монохроматическая световая волна.
    Под углом $ 30 \degrees$ наблюдается дифракционный максимум второго порядка.
    Какова длина волны падающего света?
}
\solutionspace{150pt}

\tasknumber{2}%
\task{%
    Свет с длиной волны $0{,}6\,\text{мкм}$ падает нормально на дифракционную решётку с периодом, равным $3\,\text{мкм}$.
    Под каким углом наблюдается дифракционный максимум первого порядка?
}
\solutionspace{150pt}

\tasknumber{3}%
\task{%
    При нормальном падении белого света на дифракционную решётку зелёная линия ($550\,\text{нм}$)
    в спектре третьего порядка видна под углом дифракции $18\degrees$.
    Определить число штрихов на $1\,\text{см}$ длины этой решётки.
}
\solutionspace{150pt}

\tasknumber{4}%
\task{%
    Каков наибольший порядок спектра, который можно наблюдать при дифракции света
    с длиной волны $\lambda$, на дифракционной решётке с периодом $d =  2{,}5 \lambda$?
}

\variantsplitter

\addpersonalvariant{Вячеслав Волохов}

\tasknumber{1}%
\task{%
    На дифракционную решётку, имеющую период $3 \cdot 10^{-4}\,\text{см}$, нормально падает монохроматическая световая волна.
    Под углом $ 20 \degrees$ наблюдается дифракционный максимум четвёртого порядка.
    Какова длина волны падающего света?
}
\solutionspace{150pt}

\tasknumber{2}%
\task{%
    Свет с длиной волны $0{,}5\,\text{мкм}$ падает нормально на дифракционную решётку с периодом, равным $2\,\text{мкм}$.
    Под каким углом наблюдается дифракционный максимум первого порядка?
}
\solutionspace{150pt}

\tasknumber{3}%
\task{%
    При нормальном падении белого света на дифракционную решётку зелёная линия ($550\,\text{нм}$)
    в спектре четвёртого порядка видна под углом дифракции $5\degrees$.
    Определить число штрихов на $1\,\text{см}$ длины этой решётки.
}
\solutionspace{150pt}

\tasknumber{4}%
\task{%
    Каков наибольший порядок спектра, который можно наблюдать при дифракции света
    с длиной волны $\lambda$, на дифракционной решётке с периодом $d =  4{,}6 \lambda$?
}

\variantsplitter

\addpersonalvariant{Герман Говоров}

\tasknumber{1}%
\task{%
    На дифракционную решётку, имеющую период $3 \cdot 10^{-4}\,\text{см}$, нормально падает монохроматическая световая волна.
    Под углом $ 20 \degrees$ наблюдается дифракционный максимум третьего порядка.
    Какова длина волны падающего света?
}
\solutionspace{150pt}

\tasknumber{2}%
\task{%
    Свет с длиной волны $0{,}4\,\text{мкм}$ падает нормально на дифракционную решётку с периодом, равным $3\,\text{мкм}$.
    Под каким углом наблюдается дифракционный максимум первого порядка?
}
\solutionspace{150pt}

\tasknumber{3}%
\task{%
    При нормальном падении белого света на дифракционную решётку красная линия ($720\,\text{нм}$)
    в спектре третьего порядка видна под углом дифракции $5\degrees$.
    Определить число штрихов на $1\,\text{см}$ длины этой решётки.
}
\solutionspace{150pt}

\tasknumber{4}%
\task{%
    Каков наибольший порядок спектра, который можно наблюдать при дифракции света
    с длиной волны $\lambda$, на дифракционной решётке с периодом $d =  3{,}5 \lambda$?
}

\variantsplitter

\addpersonalvariant{София Журавлёва}

\tasknumber{1}%
\task{%
    На дифракционную решётку, имеющую период $4 \cdot 10^{-4}\,\text{см}$, нормально падает монохроматическая световая волна.
    Под углом $ 20 \degrees$ наблюдается дифракционный максимум второго порядка.
    Какова длина волны падающего света?
}
\solutionspace{150pt}

\tasknumber{2}%
\task{%
    Свет с длиной волны $0{,}7\,\text{мкм}$ падает нормально на дифракционную решётку с периодом, равным $1\,\text{мкм}$.
    Под каким углом наблюдается дифракционный максимум первого порядка?
}
\solutionspace{150pt}

\tasknumber{3}%
\task{%
    При нормальном падении белого света на дифракционную решётку синяя линия ($480\,\text{нм}$)
    в спектре четвёртого порядка видна под углом дифракции $18\degrees$.
    Определить число штрихов на $1\,\text{мм}$ длины этой решётки.
}
\solutionspace{150pt}

\tasknumber{4}%
\task{%
    Каков наибольший порядок спектра, который можно наблюдать при дифракции света
    с длиной волны $\lambda$, на дифракционной решётке с периодом $d =  3{,}9 \lambda$?
}

\variantsplitter

\addpersonalvariant{Константин Козлов}

\tasknumber{1}%
\task{%
    На дифракционную решётку, имеющую период $2 \cdot 10^{-4}\,\text{см}$, нормально падает монохроматическая световая волна.
    Под углом $ 35 \degrees$ наблюдается дифракционный максимум четвёртого порядка.
    Какова длина волны падающего света?
}
\solutionspace{150pt}

\tasknumber{2}%
\task{%
    Свет с длиной волны $0{,}7\,\text{мкм}$ падает нормально на дифракционную решётку с периодом, равным $3\,\text{мкм}$.
    Под каким углом наблюдается дифракционный максимум первого порядка?
}
\solutionspace{150pt}

\tasknumber{3}%
\task{%
    При нормальном падении белого света на дифракционную решётку синяя линия ($480\,\text{нм}$)
    в спектре второго порядка видна под углом дифракции $5\degrees$.
    Определить число штрихов на $1\,\text{мм}$ длины этой решётки.
}
\solutionspace{150pt}

\tasknumber{4}%
\task{%
    Каков наибольший порядок спектра, который можно наблюдать при дифракции света
    с длиной волны $\lambda$, на дифракционной решётке с периодом $d =  3{,}9 \lambda$?
}

\variantsplitter

\addpersonalvariant{Наталья Кравченко}

\tasknumber{1}%
\task{%
    На дифракционную решётку, имеющую период $4 \cdot 10^{-4}\,\text{см}$, нормально падает монохроматическая световая волна.
    Под углом $ 20 \degrees$ наблюдается дифракционный максимум четвёртого порядка.
    Какова длина волны падающего света?
}
\solutionspace{150pt}

\tasknumber{2}%
\task{%
    Свет с длиной волны $0{,}4\,\text{мкм}$ падает нормально на дифракционную решётку с периодом, равным $3\,\text{мкм}$.
    Под каким углом наблюдается дифракционный максимум первого порядка?
}
\solutionspace{150pt}

\tasknumber{3}%
\task{%
    При нормальном падении белого света на дифракционную решётку жёлтая линия ($580\,\text{нм}$)
    в спектре четвёртого порядка видна под углом дифракции $25\degrees$.
    Определить число штрихов на $1\,\text{мм}$ длины этой решётки.
}
\solutionspace{150pt}

\tasknumber{4}%
\task{%
    Каков наибольший порядок спектра, который можно наблюдать при дифракции света
    с длиной волны $\lambda$, на дифракционной решётке с периодом $d =  4{,}6 \lambda$?
}

\variantsplitter

\addpersonalvariant{Матвей Кузьмин}

\tasknumber{1}%
\task{%
    На дифракционную решётку, имеющую период $2 \cdot 10^{-4}\,\text{см}$, нормально падает монохроматическая световая волна.
    Под углом $ 20 \degrees$ наблюдается дифракционный максимум третьего порядка.
    Какова длина волны падающего света?
}
\solutionspace{150pt}

\tasknumber{2}%
\task{%
    Свет с длиной волны $0{,}6\,\text{мкм}$ падает нормально на дифракционную решётку с периодом, равным $3\,\text{мкм}$.
    Под каким углом наблюдается дифракционный максимум первого порядка?
}
\solutionspace{150pt}

\tasknumber{3}%
\task{%
    При нормальном падении белого света на дифракционную решётку жёлтая линия ($570\,\text{нм}$)
    в спектре третьего порядка видна под углом дифракции $25\degrees$.
    Определить число штрихов на $1\,\text{мм}$ длины этой решётки.
}
\solutionspace{150pt}

\tasknumber{4}%
\task{%
    Каков наибольший порядок спектра, который можно наблюдать при дифракции света
    с длиной волны $\lambda$, на дифракционной решётке с периодом $d =  4{,}5 \lambda$?
}

\variantsplitter

\addpersonalvariant{Сергей Малышев}

\tasknumber{1}%
\task{%
    На дифракционную решётку, имеющую период $3 \cdot 10^{-4}\,\text{см}$, нормально падает монохроматическая световая волна.
    Под углом $ 20 \degrees$ наблюдается дифракционный максимум третьего порядка.
    Какова длина волны падающего света?
}
\solutionspace{150pt}

\tasknumber{2}%
\task{%
    Свет с длиной волны $0{,}5\,\text{мкм}$ падает нормально на дифракционную решётку с периодом, равным $1\,\text{мкм}$.
    Под каким углом наблюдается дифракционный максимум первого порядка?
}
\solutionspace{150pt}

\tasknumber{3}%
\task{%
    При нормальном падении белого света на дифракционную решётку зелёная линия ($520\,\text{нм}$)
    в спектре четвёртого порядка видна под углом дифракции $12\degrees$.
    Определить число штрихов на $1\,\text{мм}$ длины этой решётки.
}
\solutionspace{150pt}

\tasknumber{4}%
\task{%
    Каков наибольший порядок спектра, который можно наблюдать при дифракции света
    с длиной волны $\lambda$, на дифракционной решётке с периодом $d =  2{,}5 \lambda$?
}

\variantsplitter

\addpersonalvariant{Алина Полканова}

\tasknumber{1}%
\task{%
    На дифракционную решётку, имеющую период $2 \cdot 10^{-4}\,\text{см}$, нормально падает монохроматическая световая волна.
    Под углом $ 20 \degrees$ наблюдается дифракционный максимум третьего порядка.
    Какова длина волны падающего света?
}
\solutionspace{150pt}

\tasknumber{2}%
\task{%
    Свет с длиной волны $0{,}7\,\text{мкм}$ падает нормально на дифракционную решётку с периодом, равным $2\,\text{мкм}$.
    Под каким углом наблюдается дифракционный максимум первого порядка?
}
\solutionspace{150pt}

\tasknumber{3}%
\task{%
    При нормальном падении белого света на дифракционную решётку красная линия ($720\,\text{нм}$)
    в спектре четвёртого порядка видна под углом дифракции $12\degrees$.
    Определить число штрихов на $1\,\text{см}$ длины этой решётки.
}
\solutionspace{150pt}

\tasknumber{4}%
\task{%
    Каков наибольший порядок спектра, который можно наблюдать при дифракции света
    с длиной волны $\lambda$, на дифракционной решётке с периодом $d =  4{,}5 \lambda$?
}

\variantsplitter

\addpersonalvariant{Сергей Пономарёв}

\tasknumber{1}%
\task{%
    На дифракционную решётку, имеющую период $4 \cdot 10^{-4}\,\text{см}$, нормально падает монохроматическая световая волна.
    Под углом $ 20 \degrees$ наблюдается дифракционный максимум третьего порядка.
    Какова длина волны падающего света?
}
\solutionspace{150pt}

\tasknumber{2}%
\task{%
    Свет с длиной волны $0{,}5\,\text{мкм}$ падает нормально на дифракционную решётку с периодом, равным $2\,\text{мкм}$.
    Под каким углом наблюдается дифракционный максимум первого порядка?
}
\solutionspace{150pt}

\tasknumber{3}%
\task{%
    При нормальном падении белого света на дифракционную решётку жёлтая линия ($580\,\text{нм}$)
    в спектре третьего порядка видна под углом дифракции $5\degrees$.
    Определить число штрихов на $1\,\text{мм}$ длины этой решётки.
}
\solutionspace{150pt}

\tasknumber{4}%
\task{%
    Каков наибольший порядок спектра, который можно наблюдать при дифракции света
    с длиной волны $\lambda$, на дифракционной решётке с периодом $d =  4{,}6 \lambda$?
}

\variantsplitter

\addpersonalvariant{Егор Свистушкин}

\tasknumber{1}%
\task{%
    На дифракционную решётку, имеющую период $3 \cdot 10^{-4}\,\text{см}$, нормально падает монохроматическая световая волна.
    Под углом $ 30 \degrees$ наблюдается дифракционный максимум четвёртого порядка.
    Какова длина волны падающего света?
}
\solutionspace{150pt}

\tasknumber{2}%
\task{%
    Свет с длиной волны $0{,}5\,\text{мкм}$ падает нормально на дифракционную решётку с периодом, равным $3\,\text{мкм}$.
    Под каким углом наблюдается дифракционный максимум первого порядка?
}
\solutionspace{150pt}

\tasknumber{3}%
\task{%
    При нормальном падении белого света на дифракционную решётку жёлтая линия ($580\,\text{нм}$)
    в спектре второго порядка видна под углом дифракции $18\degrees$.
    Определить число штрихов на $1\,\text{мм}$ длины этой решётки.
}
\solutionspace{150pt}

\tasknumber{4}%
\task{%
    Каков наибольший порядок спектра, который можно наблюдать при дифракции света
    с длиной волны $\lambda$, на дифракционной решётке с периодом $d =  3{,}5 \lambda$?
}

\variantsplitter

\addpersonalvariant{Дмитрий Соколов}

\tasknumber{1}%
\task{%
    На дифракционную решётку, имеющую период $2 \cdot 10^{-4}\,\text{см}$, нормально падает монохроматическая световая волна.
    Под углом $ 35 \degrees$ наблюдается дифракционный максимум четвёртого порядка.
    Какова длина волны падающего света?
}
\solutionspace{150pt}

\tasknumber{2}%
\task{%
    Свет с длиной волны $0{,}7\,\text{мкм}$ падает нормально на дифракционную решётку с периодом, равным $2\,\text{мкм}$.
    Под каким углом наблюдается дифракционный максимум первого порядка?
}
\solutionspace{150pt}

\tasknumber{3}%
\task{%
    При нормальном падении белого света на дифракционную решётку красная линия ($680\,\text{нм}$)
    в спектре третьего порядка видна под углом дифракции $5\degrees$.
    Определить число штрихов на $1\,\text{мм}$ длины этой решётки.
}
\solutionspace{150pt}

\tasknumber{4}%
\task{%
    Каков наибольший порядок спектра, который можно наблюдать при дифракции света
    с длиной волны $\lambda$, на дифракционной решётке с периодом $d =  2{,}5 \lambda$?
}

\variantsplitter

\addpersonalvariant{Арсений Трофимов}

\tasknumber{1}%
\task{%
    На дифракционную решётку, имеющую период $4 \cdot 10^{-4}\,\text{см}$, нормально падает монохроматическая световая волна.
    Под углом $ 35 \degrees$ наблюдается дифракционный максимум четвёртого порядка.
    Какова длина волны падающего света?
}
\solutionspace{150pt}

\tasknumber{2}%
\task{%
    Свет с длиной волны $0{,}7\,\text{мкм}$ падает нормально на дифракционную решётку с периодом, равным $2\,\text{мкм}$.
    Под каким углом наблюдается дифракционный максимум первого порядка?
}
\solutionspace{150pt}

\tasknumber{3}%
\task{%
    При нормальном падении белого света на дифракционную решётку синяя линия ($480\,\text{нм}$)
    в спектре четвёртого порядка видна под углом дифракции $25\degrees$.
    Определить число штрихов на $1\,\text{мм}$ длины этой решётки.
}
\solutionspace{150pt}

\tasknumber{4}%
\task{%
    Каков наибольший порядок спектра, который можно наблюдать при дифракции света
    с длиной волны $\lambda$, на дифракционной решётке с периодом $d =  4{,}1 \lambda$?
}
% autogenerated
