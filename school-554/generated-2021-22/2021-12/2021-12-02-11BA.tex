\setdate{2~декабря~2021}
\setclass{11«БА»}

\addpersonalvariant{Михаил Бурмистров}

\tasknumber{1}%
\task{%
    На дифракционную решётку, имеющую период $2 \cdot 10^{-4}\,\text{см}$, нормально падает монохроматическая световая волна.
    Под углом $ 35 \degrees$ наблюдается дифракционный максимум четвёртого порядка.
    Какова длина волны падающего света?
}
\answer{%
    $
        d\sin \varphi_k = k\lambda
        \implies \lambda = \frac{d \sin \varphi_k}k
        = \frac{2 \cdot 10^{-4}\,\text{см} \cdot \sin  35 \degrees}{4} \approx 290\,\text{нм}
    $
}
\solutionspace{150pt}

\tasknumber{2}%
\task{%
    Свет с длиной волны $0{,}4\,\text{мкм}$ падает нормально на дифракционную решётку с периодом, равным $1\,\text{мкм}$.
    Под каким углом наблюдается дифракционный максимум первого порядка?
}
\answer{%
    $
        d\sin \varphi_k = k\lambda
        \implies \sin \varphi_k = \frac{k\lambda}{ d }
        = \frac{1 \cdot 0{,}4\,\text{мкм}}{1\,\text{мкм}} \approx 0{,}4 \implies \varphi_k \approx 23{,}6\degrees
    $
}
\solutionspace{150pt}

\tasknumber{3}%
\task{%
    При нормальном падении белого света на дифракционную решётку зелёная линия ($550\,\text{нм}$)
    в спектре четвёртого порядка видна под углом дифракции $5\degrees$.
    Определить число штрихов на $1\,\text{мм}$ длины этой решётки.
}
\answer{%
    $
        d\sin \varphi_k = k\lambda
        \implies d = \frac{k\lambda}{\sin \varphi_k}.
        \qquad N = \frac{ l }{d} = \frac{ l \sin \varphi_k}{k\lambda}
        = \frac{1\,\text{мм} \cdot \sin 5\degrees}{4 \cdot 550\,\text{нм}} \approx 40
    $
}
\solutionspace{150pt}

\tasknumber{4}%
\task{%
    Каков наибольший порядок спектра, который можно наблюдать при дифракции света
    с длиной волны $\lambda$, на дифракционной решётке с периодом $d =  4{,}5 \lambda$?
    Под каким углом наблюдается последний максимум?
}
\answer{%
    $
        d\sin \varphi_k = k\lambda
        \implies k = \frac{d\sin \varphi_k}{\lambda} \le \frac{d \cdot 1}{\lambda} =  4{,}5
        \implies k_{\max} = 4
    $
}

\variantsplitter

\addpersonalvariant{Ирина Ан}

\tasknumber{1}%
\task{%
    На дифракционную решётку, имеющую период $3 \cdot 10^{-4}\,\text{см}$, нормально падает монохроматическая световая волна.
    Под углом $ 40 \degrees$ наблюдается дифракционный максимум третьего порядка.
    Какова длина волны падающего света?
}
\answer{%
    $
        d\sin \varphi_k = k\lambda
        \implies \lambda = \frac{d \sin \varphi_k}k
        = \frac{3 \cdot 10^{-4}\,\text{см} \cdot \sin  40 \degrees}{3} \approx 640\,\text{нм}
    $
}
\solutionspace{150pt}

\tasknumber{2}%
\task{%
    Свет с длиной волны $0{,}5\,\text{мкм}$ падает нормально на дифракционную решётку с периодом, равным $2\,\text{мкм}$.
    Под каким углом наблюдается дифракционный максимум первого порядка?
}
\answer{%
    $
        d\sin \varphi_k = k\lambda
        \implies \sin \varphi_k = \frac{k\lambda}{ d }
        = \frac{1 \cdot 0{,}5\,\text{мкм}}{2\,\text{мкм}} \approx 0{,}3 \implies \varphi_k \approx 14{,}5\degrees
    $
}
\solutionspace{150pt}

\tasknumber{3}%
\task{%
    При нормальном падении белого света на дифракционную решётку оранжевая линия ($600\,\text{нм}$)
    в спектре четвёртого порядка видна под углом дифракции $25\degrees$.
    Определить число штрихов на $1\,\text{мм}$ длины этой решётки.
}
\answer{%
    $
        d\sin \varphi_k = k\lambda
        \implies d = \frac{k\lambda}{\sin \varphi_k}.
        \qquad N = \frac{ l }{d} = \frac{ l \sin \varphi_k}{k\lambda}
        = \frac{1\,\text{мм} \cdot \sin 25\degrees}{4 \cdot 600\,\text{нм}} \approx 176
    $
}
\solutionspace{150pt}

\tasknumber{4}%
\task{%
    Каков наибольший порядок спектра, который можно наблюдать при дифракции света
    с длиной волны $\lambda$, на дифракционной решётке с периодом $d =  4{,}1 \lambda$?
    Под каким углом наблюдается последний максимум?
}
\answer{%
    $
        d\sin \varphi_k = k\lambda
        \implies k = \frac{d\sin \varphi_k}{\lambda} \le \frac{d \cdot 1}{\lambda} =  4{,}1
        \implies k_{\max} = 4
    $
}

\variantsplitter

\addpersonalvariant{Софья Андрианова}

\tasknumber{1}%
\task{%
    На дифракционную решётку, имеющую период $2 \cdot 10^{-4}\,\text{см}$, нормально падает монохроматическая световая волна.
    Под углом $ 25 \degrees$ наблюдается дифракционный максимум второго порядка.
    Какова длина волны падающего света?
}
\answer{%
    $
        d\sin \varphi_k = k\lambda
        \implies \lambda = \frac{d \sin \varphi_k}k
        = \frac{2 \cdot 10^{-4}\,\text{см} \cdot \sin  25 \degrees}{2} \approx 420\,\text{нм}
    $
}
\solutionspace{150pt}

\tasknumber{2}%
\task{%
    Свет с длиной волны $0{,}5\,\text{мкм}$ падает нормально на дифракционную решётку с периодом, равным $1\,\text{мкм}$.
    Под каким углом наблюдается дифракционный максимум первого порядка?
}
\answer{%
    $
        d\sin \varphi_k = k\lambda
        \implies \sin \varphi_k = \frac{k\lambda}{ d }
        = \frac{1 \cdot 0{,}5\,\text{мкм}}{1\,\text{мкм}} \approx 0{,}5 \implies \varphi_k \approx 30{,}0\degrees
    $
}
\solutionspace{150pt}

\tasknumber{3}%
\task{%
    При нормальном падении белого света на дифракционную решётку красная линия ($720\,\text{нм}$)
    в спектре третьего порядка видна под углом дифракции $25\degrees$.
    Определить число штрихов на $1\,\text{мм}$ длины этой решётки.
}
\answer{%
    $
        d\sin \varphi_k = k\lambda
        \implies d = \frac{k\lambda}{\sin \varphi_k}.
        \qquad N = \frac{ l }{d} = \frac{ l \sin \varphi_k}{k\lambda}
        = \frac{1\,\text{мм} \cdot \sin 25\degrees}{3 \cdot 720\,\text{нм}} \approx 196
    $
}
\solutionspace{150pt}

\tasknumber{4}%
\task{%
    Каков наибольший порядок спектра, который можно наблюдать при дифракции света
    с длиной волны $\lambda$, на дифракционной решётке с периодом $d =  2{,}5 \lambda$?
    Под каким углом наблюдается последний максимум?
}
\answer{%
    $
        d\sin \varphi_k = k\lambda
        \implies k = \frac{d\sin \varphi_k}{\lambda} \le \frac{d \cdot 1}{\lambda} =  2{,}5
        \implies k_{\max} = 2
    $
}

\variantsplitter

\addpersonalvariant{Владимир Артемчук}

\tasknumber{1}%
\task{%
    На дифракционную решётку, имеющую период $3 \cdot 10^{-4}\,\text{см}$, нормально падает монохроматическая световая волна.
    Под углом $ 20 \degrees$ наблюдается дифракционный максимум третьего порядка.
    Какова длина волны падающего света?
}
\answer{%
    $
        d\sin \varphi_k = k\lambda
        \implies \lambda = \frac{d \sin \varphi_k}k
        = \frac{3 \cdot 10^{-4}\,\text{см} \cdot \sin  20 \degrees}{3} \approx 340\,\text{нм}
    $
}
\solutionspace{150pt}

\tasknumber{2}%
\task{%
    Свет с длиной волны $0{,}4\,\text{мкм}$ падает нормально на дифракционную решётку с периодом, равным $2\,\text{мкм}$.
    Под каким углом наблюдается дифракционный максимум первого порядка?
}
\answer{%
    $
        d\sin \varphi_k = k\lambda
        \implies \sin \varphi_k = \frac{k\lambda}{ d }
        = \frac{1 \cdot 0{,}4\,\text{мкм}}{2\,\text{мкм}} \approx 0{,}2 \implies \varphi_k \approx 11{,}5\degrees
    $
}
\solutionspace{150pt}

\tasknumber{3}%
\task{%
    При нормальном падении белого света на дифракционную решётку оранжевая линия ($600\,\text{нм}$)
    в спектре третьего порядка видна под углом дифракции $25\degrees$.
    Определить число штрихов на $1\,\text{см}$ длины этой решётки.
}
\answer{%
    $
        d\sin \varphi_k = k\lambda
        \implies d = \frac{k\lambda}{\sin \varphi_k}.
        \qquad N = \frac{ l }{d} = \frac{ l \sin \varphi_k}{k\lambda}
        = \frac{1\,\text{см} \cdot \sin 25\degrees}{3 \cdot 600\,\text{нм}} \approx 2300
    $
}
\solutionspace{150pt}

\tasknumber{4}%
\task{%
    Каков наибольший порядок спектра, который можно наблюдать при дифракции света
    с длиной волны $\lambda$, на дифракционной решётке с периодом $d =  3{,}3 \lambda$?
    Под каким углом наблюдается последний максимум?
}
\answer{%
    $
        d\sin \varphi_k = k\lambda
        \implies k = \frac{d\sin \varphi_k}{\lambda} \le \frac{d \cdot 1}{\lambda} =  3{,}3
        \implies k_{\max} = 3
    $
}

\variantsplitter

\addpersonalvariant{Софья Белянкина}

\tasknumber{1}%
\task{%
    На дифракционную решётку, имеющую период $2 \cdot 10^{-4}\,\text{см}$, нормально падает монохроматическая световая волна.
    Под углом $ 20 \degrees$ наблюдается дифракционный максимум третьего порядка.
    Какова длина волны падающего света?
}
\answer{%
    $
        d\sin \varphi_k = k\lambda
        \implies \lambda = \frac{d \sin \varphi_k}k
        = \frac{2 \cdot 10^{-4}\,\text{см} \cdot \sin  20 \degrees}{3} \approx 230\,\text{нм}
    $
}
\solutionspace{150pt}

\tasknumber{2}%
\task{%
    Свет с длиной волны $0{,}7\,\text{мкм}$ падает нормально на дифракционную решётку с периодом, равным $2\,\text{мкм}$.
    Под каким углом наблюдается дифракционный максимум первого порядка?
}
\answer{%
    $
        d\sin \varphi_k = k\lambda
        \implies \sin \varphi_k = \frac{k\lambda}{ d }
        = \frac{1 \cdot 0{,}7\,\text{мкм}}{2\,\text{мкм}} \approx 0{,}4 \implies \varphi_k \approx 20{,}5\degrees
    $
}
\solutionspace{150pt}

\tasknumber{3}%
\task{%
    При нормальном падении белого света на дифракционную решётку синяя линия ($480\,\text{нм}$)
    в спектре четвёртого порядка видна под углом дифракции $18\degrees$.
    Определить число штрихов на $1\,\text{см}$ длины этой решётки.
}
\answer{%
    $
        d\sin \varphi_k = k\lambda
        \implies d = \frac{k\lambda}{\sin \varphi_k}.
        \qquad N = \frac{ l }{d} = \frac{ l \sin \varphi_k}{k\lambda}
        = \frac{1\,\text{см} \cdot \sin 18\degrees}{4 \cdot 480\,\text{нм}} \approx 1610
    $
}
\solutionspace{150pt}

\tasknumber{4}%
\task{%
    Каков наибольший порядок спектра, который можно наблюдать при дифракции света
    с длиной волны $\lambda$, на дифракционной решётке с периодом $d =  2{,}2 \lambda$?
    Под каким углом наблюдается последний максимум?
}
\answer{%
    $
        d\sin \varphi_k = k\lambda
        \implies k = \frac{d\sin \varphi_k}{\lambda} \le \frac{d \cdot 1}{\lambda} =  2{,}2
        \implies k_{\max} = 2
    $
}

\variantsplitter

\addpersonalvariant{Варвара Егиазарян}

\tasknumber{1}%
\task{%
    На дифракционную решётку, имеющую период $2 \cdot 10^{-4}\,\text{см}$, нормально падает монохроматическая световая волна.
    Под углом $ 25 \degrees$ наблюдается дифракционный максимум третьего порядка.
    Какова длина волны падающего света?
}
\answer{%
    $
        d\sin \varphi_k = k\lambda
        \implies \lambda = \frac{d \sin \varphi_k}k
        = \frac{2 \cdot 10^{-4}\,\text{см} \cdot \sin  25 \degrees}{3} \approx 280\,\text{нм}
    $
}
\solutionspace{150pt}

\tasknumber{2}%
\task{%
    Свет с длиной волны $0{,}4\,\text{мкм}$ падает нормально на дифракционную решётку с периодом, равным $1\,\text{мкм}$.
    Под каким углом наблюдается дифракционный максимум первого порядка?
}
\answer{%
    $
        d\sin \varphi_k = k\lambda
        \implies \sin \varphi_k = \frac{k\lambda}{ d }
        = \frac{1 \cdot 0{,}4\,\text{мкм}}{1\,\text{мкм}} \approx 0{,}4 \implies \varphi_k \approx 23{,}6\degrees
    $
}
\solutionspace{150pt}

\tasknumber{3}%
\task{%
    При нормальном падении белого света на дифракционную решётку жёлтая линия ($580\,\text{нм}$)
    в спектре четвёртого порядка видна под углом дифракции $18\degrees$.
    Определить число штрихов на $1\,\text{см}$ длины этой решётки.
}
\answer{%
    $
        d\sin \varphi_k = k\lambda
        \implies d = \frac{k\lambda}{\sin \varphi_k}.
        \qquad N = \frac{ l }{d} = \frac{ l \sin \varphi_k}{k\lambda}
        = \frac{1\,\text{см} \cdot \sin 18\degrees}{4 \cdot 580\,\text{нм}} \approx 1330
    $
}
\solutionspace{150pt}

\tasknumber{4}%
\task{%
    Каков наибольший порядок спектра, который можно наблюдать при дифракции света
    с длиной волны $\lambda$, на дифракционной решётке с периодом $d =  4{,}1 \lambda$?
    Под каким углом наблюдается последний максимум?
}
\answer{%
    $
        d\sin \varphi_k = k\lambda
        \implies k = \frac{d\sin \varphi_k}{\lambda} \le \frac{d \cdot 1}{\lambda} =  4{,}1
        \implies k_{\max} = 4
    $
}

\variantsplitter

\addpersonalvariant{Владислав Емелин}

\tasknumber{1}%
\task{%
    На дифракционную решётку, имеющую период $2 \cdot 10^{-4}\,\text{см}$, нормально падает монохроматическая световая волна.
    Под углом $ 25 \degrees$ наблюдается дифракционный максимум второго порядка.
    Какова длина волны падающего света?
}
\answer{%
    $
        d\sin \varphi_k = k\lambda
        \implies \lambda = \frac{d \sin \varphi_k}k
        = \frac{2 \cdot 10^{-4}\,\text{см} \cdot \sin  25 \degrees}{2} \approx 420\,\text{нм}
    $
}
\solutionspace{150pt}

\tasknumber{2}%
\task{%
    Свет с длиной волны $0{,}7\,\text{мкм}$ падает нормально на дифракционную решётку с периодом, равным $2\,\text{мкм}$.
    Под каким углом наблюдается дифракционный максимум первого порядка?
}
\answer{%
    $
        d\sin \varphi_k = k\lambda
        \implies \sin \varphi_k = \frac{k\lambda}{ d }
        = \frac{1 \cdot 0{,}7\,\text{мкм}}{2\,\text{мкм}} \approx 0{,}4 \implies \varphi_k \approx 20{,}5\degrees
    $
}
\solutionspace{150pt}

\tasknumber{3}%
\task{%
    При нормальном падении белого света на дифракционную решётку жёлтая линия ($580\,\text{нм}$)
    в спектре четвёртого порядка видна под углом дифракции $12\degrees$.
    Определить число штрихов на $1\,\text{мм}$ длины этой решётки.
}
\answer{%
    $
        d\sin \varphi_k = k\lambda
        \implies d = \frac{k\lambda}{\sin \varphi_k}.
        \qquad N = \frac{ l }{d} = \frac{ l \sin \varphi_k}{k\lambda}
        = \frac{1\,\text{мм} \cdot \sin 12\degrees}{4 \cdot 580\,\text{нм}} \approx 90
    $
}
\solutionspace{150pt}

\tasknumber{4}%
\task{%
    Каков наибольший порядок спектра, который можно наблюдать при дифракции света
    с длиной волны $\lambda$, на дифракционной решётке с периодом $d =  2{,}7 \lambda$?
    Под каким углом наблюдается последний максимум?
}
\answer{%
    $
        d\sin \varphi_k = k\lambda
        \implies k = \frac{d\sin \varphi_k}{\lambda} \le \frac{d \cdot 1}{\lambda} =  2{,}7
        \implies k_{\max} = 2
    $
}

\variantsplitter

\addpersonalvariant{Артём Жичин}

\tasknumber{1}%
\task{%
    На дифракционную решётку, имеющую период $4 \cdot 10^{-4}\,\text{см}$, нормально падает монохроматическая световая волна.
    Под углом $ 30 \degrees$ наблюдается дифракционный максимум второго порядка.
    Какова длина волны падающего света?
}
\answer{%
    $
        d\sin \varphi_k = k\lambda
        \implies \lambda = \frac{d \sin \varphi_k}k
        = \frac{4 \cdot 10^{-4}\,\text{см} \cdot \sin  30 \degrees}{2} \approx 1000\,\text{нм}
    $
}
\solutionspace{150pt}

\tasknumber{2}%
\task{%
    Свет с длиной волны $0{,}7\,\text{мкм}$ падает нормально на дифракционную решётку с периодом, равным $3\,\text{мкм}$.
    Под каким углом наблюдается дифракционный максимум первого порядка?
}
\answer{%
    $
        d\sin \varphi_k = k\lambda
        \implies \sin \varphi_k = \frac{k\lambda}{ d }
        = \frac{1 \cdot 0{,}7\,\text{мкм}}{3\,\text{мкм}} \approx 0{,}2 \implies \varphi_k \approx 13{,}5\degrees
    $
}
\solutionspace{150pt}

\tasknumber{3}%
\task{%
    При нормальном падении белого света на дифракционную решётку жёлтая линия ($570\,\text{нм}$)
    в спектре второго порядка видна под углом дифракции $12\degrees$.
    Определить число штрихов на $1\,\text{см}$ длины этой решётки.
}
\answer{%
    $
        d\sin \varphi_k = k\lambda
        \implies d = \frac{k\lambda}{\sin \varphi_k}.
        \qquad N = \frac{ l }{d} = \frac{ l \sin \varphi_k}{k\lambda}
        = \frac{1\,\text{см} \cdot \sin 12\degrees}{2 \cdot 570\,\text{нм}} \approx 1820
    $
}
\solutionspace{150pt}

\tasknumber{4}%
\task{%
    Каков наибольший порядок спектра, который можно наблюдать при дифракции света
    с длиной волны $\lambda$, на дифракционной решётке с периодом $d =  3{,}9 \lambda$?
    Под каким углом наблюдается последний максимум?
}
\answer{%
    $
        d\sin \varphi_k = k\lambda
        \implies k = \frac{d\sin \varphi_k}{\lambda} \le \frac{d \cdot 1}{\lambda} =  3{,}9
        \implies k_{\max} = 3
    $
}

\variantsplitter

\addpersonalvariant{Дарья Кошман}

\tasknumber{1}%
\task{%
    На дифракционную решётку, имеющую период $2 \cdot 10^{-4}\,\text{см}$, нормально падает монохроматическая световая волна.
    Под углом $ 35 \degrees$ наблюдается дифракционный максимум четвёртого порядка.
    Какова длина волны падающего света?
}
\answer{%
    $
        d\sin \varphi_k = k\lambda
        \implies \lambda = \frac{d \sin \varphi_k}k
        = \frac{2 \cdot 10^{-4}\,\text{см} \cdot \sin  35 \degrees}{4} \approx 290\,\text{нм}
    $
}
\solutionspace{150pt}

\tasknumber{2}%
\task{%
    Свет с длиной волны $0{,}7\,\text{мкм}$ падает нормально на дифракционную решётку с периодом, равным $2\,\text{мкм}$.
    Под каким углом наблюдается дифракционный максимум первого порядка?
}
\answer{%
    $
        d\sin \varphi_k = k\lambda
        \implies \sin \varphi_k = \frac{k\lambda}{ d }
        = \frac{1 \cdot 0{,}7\,\text{мкм}}{2\,\text{мкм}} \approx 0{,}4 \implies \varphi_k \approx 20{,}5\degrees
    $
}
\solutionspace{150pt}

\tasknumber{3}%
\task{%
    При нормальном падении белого света на дифракционную решётку синяя линия ($450\,\text{нм}$)
    в спектре третьего порядка видна под углом дифракции $18\degrees$.
    Определить число штрихов на $1\,\text{мм}$ длины этой решётки.
}
\answer{%
    $
        d\sin \varphi_k = k\lambda
        \implies d = \frac{k\lambda}{\sin \varphi_k}.
        \qquad N = \frac{ l }{d} = \frac{ l \sin \varphi_k}{k\lambda}
        = \frac{1\,\text{мм} \cdot \sin 18\degrees}{3 \cdot 450\,\text{нм}} \approx 230
    $
}
\solutionspace{150pt}

\tasknumber{4}%
\task{%
    Каков наибольший порядок спектра, который можно наблюдать при дифракции света
    с длиной волны $\lambda$, на дифракционной решётке с периодом $d =  2{,}7 \lambda$?
    Под каким углом наблюдается последний максимум?
}
\answer{%
    $
        d\sin \varphi_k = k\lambda
        \implies k = \frac{d\sin \varphi_k}{\lambda} \le \frac{d \cdot 1}{\lambda} =  2{,}7
        \implies k_{\max} = 2
    $
}

\variantsplitter

\addpersonalvariant{Анна Кузьмичёва}

\tasknumber{1}%
\task{%
    На дифракционную решётку, имеющую период $4 \cdot 10^{-4}\,\text{см}$, нормально падает монохроматическая световая волна.
    Под углом $ 35 \degrees$ наблюдается дифракционный максимум второго порядка.
    Какова длина волны падающего света?
}
\answer{%
    $
        d\sin \varphi_k = k\lambda
        \implies \lambda = \frac{d \sin \varphi_k}k
        = \frac{4 \cdot 10^{-4}\,\text{см} \cdot \sin  35 \degrees}{2} \approx 1150\,\text{нм}
    $
}
\solutionspace{150pt}

\tasknumber{2}%
\task{%
    Свет с длиной волны $0{,}4\,\text{мкм}$ падает нормально на дифракционную решётку с периодом, равным $3\,\text{мкм}$.
    Под каким углом наблюдается дифракционный максимум первого порядка?
}
\answer{%
    $
        d\sin \varphi_k = k\lambda
        \implies \sin \varphi_k = \frac{k\lambda}{ d }
        = \frac{1 \cdot 0{,}4\,\text{мкм}}{3\,\text{мкм}} \approx 0{,}13 \implies \varphi_k \approx 7{,}7\degrees
    $
}
\solutionspace{150pt}

\tasknumber{3}%
\task{%
    При нормальном падении белого света на дифракционную решётку жёлтая линия ($580\,\text{нм}$)
    в спектре второго порядка видна под углом дифракции $5\degrees$.
    Определить число штрихов на $1\,\text{см}$ длины этой решётки.
}
\answer{%
    $
        d\sin \varphi_k = k\lambda
        \implies d = \frac{k\lambda}{\sin \varphi_k}.
        \qquad N = \frac{ l }{d} = \frac{ l \sin \varphi_k}{k\lambda}
        = \frac{1\,\text{см} \cdot \sin 5\degrees}{2 \cdot 580\,\text{нм}} \approx 750
    $
}
\solutionspace{150pt}

\tasknumber{4}%
\task{%
    Каков наибольший порядок спектра, который можно наблюдать при дифракции света
    с длиной волны $\lambda$, на дифракционной решётке с периодом $d =  3{,}9 \lambda$?
    Под каким углом наблюдается последний максимум?
}
\answer{%
    $
        d\sin \varphi_k = k\lambda
        \implies k = \frac{d\sin \varphi_k}{\lambda} \le \frac{d \cdot 1}{\lambda} =  3{,}9
        \implies k_{\max} = 3
    $
}

\variantsplitter

\addpersonalvariant{Алёна Куприянова}

\tasknumber{1}%
\task{%
    На дифракционную решётку, имеющую период $4 \cdot 10^{-4}\,\text{см}$, нормально падает монохроматическая световая волна.
    Под углом $ 25 \degrees$ наблюдается дифракционный максимум четвёртого порядка.
    Какова длина волны падающего света?
}
\answer{%
    $
        d\sin \varphi_k = k\lambda
        \implies \lambda = \frac{d \sin \varphi_k}k
        = \frac{4 \cdot 10^{-4}\,\text{см} \cdot \sin  25 \degrees}{4} \approx 420\,\text{нм}
    $
}
\solutionspace{150pt}

\tasknumber{2}%
\task{%
    Свет с длиной волны $0{,}5\,\text{мкм}$ падает нормально на дифракционную решётку с периодом, равным $1\,\text{мкм}$.
    Под каким углом наблюдается дифракционный максимум первого порядка?
}
\answer{%
    $
        d\sin \varphi_k = k\lambda
        \implies \sin \varphi_k = \frac{k\lambda}{ d }
        = \frac{1 \cdot 0{,}5\,\text{мкм}}{1\,\text{мкм}} \approx 0{,}5 \implies \varphi_k \approx 30{,}0\degrees
    $
}
\solutionspace{150pt}

\tasknumber{3}%
\task{%
    При нормальном падении белого света на дифракционную решётку зелёная линия ($520\,\text{нм}$)
    в спектре второго порядка видна под углом дифракции $5\degrees$.
    Определить число штрихов на $1\,\text{см}$ длины этой решётки.
}
\answer{%
    $
        d\sin \varphi_k = k\lambda
        \implies d = \frac{k\lambda}{\sin \varphi_k}.
        \qquad N = \frac{ l }{d} = \frac{ l \sin \varphi_k}{k\lambda}
        = \frac{1\,\text{см} \cdot \sin 5\degrees}{2 \cdot 520\,\text{нм}} \approx 840
    $
}
\solutionspace{150pt}

\tasknumber{4}%
\task{%
    Каков наибольший порядок спектра, который можно наблюдать при дифракции света
    с длиной волны $\lambda$, на дифракционной решётке с периодом $d =  4{,}5 \lambda$?
    Под каким углом наблюдается последний максимум?
}
\answer{%
    $
        d\sin \varphi_k = k\lambda
        \implies k = \frac{d\sin \varphi_k}{\lambda} \le \frac{d \cdot 1}{\lambda} =  4{,}5
        \implies k_{\max} = 4
    $
}

\variantsplitter

\addpersonalvariant{Ярослав Лавровский}

\tasknumber{1}%
\task{%
    На дифракционную решётку, имеющую период $2 \cdot 10^{-4}\,\text{см}$, нормально падает монохроматическая световая волна.
    Под углом $ 20 \degrees$ наблюдается дифракционный максимум третьего порядка.
    Какова длина волны падающего света?
}
\answer{%
    $
        d\sin \varphi_k = k\lambda
        \implies \lambda = \frac{d \sin \varphi_k}k
        = \frac{2 \cdot 10^{-4}\,\text{см} \cdot \sin  20 \degrees}{3} \approx 230\,\text{нм}
    $
}
\solutionspace{150pt}

\tasknumber{2}%
\task{%
    Свет с длиной волны $0{,}5\,\text{мкм}$ падает нормально на дифракционную решётку с периодом, равным $2\,\text{мкм}$.
    Под каким углом наблюдается дифракционный максимум первого порядка?
}
\answer{%
    $
        d\sin \varphi_k = k\lambda
        \implies \sin \varphi_k = \frac{k\lambda}{ d }
        = \frac{1 \cdot 0{,}5\,\text{мкм}}{2\,\text{мкм}} \approx 0{,}3 \implies \varphi_k \approx 14{,}5\degrees
    $
}
\solutionspace{150pt}

\tasknumber{3}%
\task{%
    При нормальном падении белого света на дифракционную решётку зелёная линия ($550\,\text{нм}$)
    в спектре третьего порядка видна под углом дифракции $12\degrees$.
    Определить число штрихов на $1\,\text{мм}$ длины этой решётки.
}
\answer{%
    $
        d\sin \varphi_k = k\lambda
        \implies d = \frac{k\lambda}{\sin \varphi_k}.
        \qquad N = \frac{ l }{d} = \frac{ l \sin \varphi_k}{k\lambda}
        = \frac{1\,\text{мм} \cdot \sin 12\degrees}{3 \cdot 550\,\text{нм}} \approx 126
    $
}
\solutionspace{150pt}

\tasknumber{4}%
\task{%
    Каков наибольший порядок спектра, который можно наблюдать при дифракции света
    с длиной волны $\lambda$, на дифракционной решётке с периодом $d =  4{,}1 \lambda$?
    Под каким углом наблюдается последний максимум?
}
\answer{%
    $
        d\sin \varphi_k = k\lambda
        \implies k = \frac{d\sin \varphi_k}{\lambda} \le \frac{d \cdot 1}{\lambda} =  4{,}1
        \implies k_{\max} = 4
    $
}

\variantsplitter

\addpersonalvariant{Анастасия Ламанова}

\tasknumber{1}%
\task{%
    На дифракционную решётку, имеющую период $4 \cdot 10^{-4}\,\text{см}$, нормально падает монохроматическая световая волна.
    Под углом $ 35 \degrees$ наблюдается дифракционный максимум второго порядка.
    Какова длина волны падающего света?
}
\answer{%
    $
        d\sin \varphi_k = k\lambda
        \implies \lambda = \frac{d \sin \varphi_k}k
        = \frac{4 \cdot 10^{-4}\,\text{см} \cdot \sin  35 \degrees}{2} \approx 1150\,\text{нм}
    $
}
\solutionspace{150pt}

\tasknumber{2}%
\task{%
    Свет с длиной волны $0{,}7\,\text{мкм}$ падает нормально на дифракционную решётку с периодом, равным $2\,\text{мкм}$.
    Под каким углом наблюдается дифракционный максимум первого порядка?
}
\answer{%
    $
        d\sin \varphi_k = k\lambda
        \implies \sin \varphi_k = \frac{k\lambda}{ d }
        = \frac{1 \cdot 0{,}7\,\text{мкм}}{2\,\text{мкм}} \approx 0{,}4 \implies \varphi_k \approx 20{,}5\degrees
    $
}
\solutionspace{150pt}

\tasknumber{3}%
\task{%
    При нормальном падении белого света на дифракционную решётку красная линия ($680\,\text{нм}$)
    в спектре четвёртого порядка видна под углом дифракции $18\degrees$.
    Определить число штрихов на $1\,\text{см}$ длины этой решётки.
}
\answer{%
    $
        d\sin \varphi_k = k\lambda
        \implies d = \frac{k\lambda}{\sin \varphi_k}.
        \qquad N = \frac{ l }{d} = \frac{ l \sin \varphi_k}{k\lambda}
        = \frac{1\,\text{см} \cdot \sin 18\degrees}{4 \cdot 680\,\text{нм}} \approx 1140
    $
}
\solutionspace{150pt}

\tasknumber{4}%
\task{%
    Каков наибольший порядок спектра, который можно наблюдать при дифракции света
    с длиной волны $\lambda$, на дифракционной решётке с периодом $d =  2{,}2 \lambda$?
    Под каким углом наблюдается последний максимум?
}
\answer{%
    $
        d\sin \varphi_k = k\lambda
        \implies k = \frac{d\sin \varphi_k}{\lambda} \le \frac{d \cdot 1}{\lambda} =  2{,}2
        \implies k_{\max} = 2
    $
}

\variantsplitter

\addpersonalvariant{Виктория Легонькова}

\tasknumber{1}%
\task{%
    На дифракционную решётку, имеющую период $3 \cdot 10^{-4}\,\text{см}$, нормально падает монохроматическая световая волна.
    Под углом $ 25 \degrees$ наблюдается дифракционный максимум четвёртого порядка.
    Какова длина волны падающего света?
}
\answer{%
    $
        d\sin \varphi_k = k\lambda
        \implies \lambda = \frac{d \sin \varphi_k}k
        = \frac{3 \cdot 10^{-4}\,\text{см} \cdot \sin  25 \degrees}{4} \approx 320\,\text{нм}
    $
}
\solutionspace{150pt}

\tasknumber{2}%
\task{%
    Свет с длиной волны $0{,}6\,\text{мкм}$ падает нормально на дифракционную решётку с периодом, равным $3\,\text{мкм}$.
    Под каким углом наблюдается дифракционный максимум первого порядка?
}
\answer{%
    $
        d\sin \varphi_k = k\lambda
        \implies \sin \varphi_k = \frac{k\lambda}{ d }
        = \frac{1 \cdot 0{,}6\,\text{мкм}}{3\,\text{мкм}} \approx 0{,}2 \implies \varphi_k \approx 11{,}5\degrees
    $
}
\solutionspace{150pt}

\tasknumber{3}%
\task{%
    При нормальном падении белого света на дифракционную решётку синяя линия ($450\,\text{нм}$)
    в спектре четвёртого порядка видна под углом дифракции $5\degrees$.
    Определить число штрихов на $1\,\text{мм}$ длины этой решётки.
}
\answer{%
    $
        d\sin \varphi_k = k\lambda
        \implies d = \frac{k\lambda}{\sin \varphi_k}.
        \qquad N = \frac{ l }{d} = \frac{ l \sin \varphi_k}{k\lambda}
        = \frac{1\,\text{мм} \cdot \sin 5\degrees}{4 \cdot 450\,\text{нм}} \approx 48
    $
}
\solutionspace{150pt}

\tasknumber{4}%
\task{%
    Каков наибольший порядок спектра, который можно наблюдать при дифракции света
    с длиной волны $\lambda$, на дифракционной решётке с периодом $d =  4{,}1 \lambda$?
    Под каким углом наблюдается последний максимум?
}
\answer{%
    $
        d\sin \varphi_k = k\lambda
        \implies k = \frac{d\sin \varphi_k}{\lambda} \le \frac{d \cdot 1}{\lambda} =  4{,}1
        \implies k_{\max} = 4
    $
}

\variantsplitter

\addpersonalvariant{Семён Мартынов}

\tasknumber{1}%
\task{%
    На дифракционную решётку, имеющую период $3 \cdot 10^{-4}\,\text{см}$, нормально падает монохроматическая световая волна.
    Под углом $ 25 \degrees$ наблюдается дифракционный максимум второго порядка.
    Какова длина волны падающего света?
}
\answer{%
    $
        d\sin \varphi_k = k\lambda
        \implies \lambda = \frac{d \sin \varphi_k}k
        = \frac{3 \cdot 10^{-4}\,\text{см} \cdot \sin  25 \degrees}{2} \approx 630\,\text{нм}
    $
}
\solutionspace{150pt}

\tasknumber{2}%
\task{%
    Свет с длиной волны $0{,}6\,\text{мкм}$ падает нормально на дифракционную решётку с периодом, равным $2\,\text{мкм}$.
    Под каким углом наблюдается дифракционный максимум первого порядка?
}
\answer{%
    $
        d\sin \varphi_k = k\lambda
        \implies \sin \varphi_k = \frac{k\lambda}{ d }
        = \frac{1 \cdot 0{,}6\,\text{мкм}}{2\,\text{мкм}} \approx 0{,}3 \implies \varphi_k \approx 17{,}5\degrees
    $
}
\solutionspace{150pt}

\tasknumber{3}%
\task{%
    При нормальном падении белого света на дифракционную решётку синяя линия ($480\,\text{нм}$)
    в спектре третьего порядка видна под углом дифракции $18\degrees$.
    Определить число штрихов на $1\,\text{мм}$ длины этой решётки.
}
\answer{%
    $
        d\sin \varphi_k = k\lambda
        \implies d = \frac{k\lambda}{\sin \varphi_k}.
        \qquad N = \frac{ l }{d} = \frac{ l \sin \varphi_k}{k\lambda}
        = \frac{1\,\text{мм} \cdot \sin 18\degrees}{3 \cdot 480\,\text{нм}} \approx 210
    $
}
\solutionspace{150pt}

\tasknumber{4}%
\task{%
    Каков наибольший порядок спектра, который можно наблюдать при дифракции света
    с длиной волны $\lambda$, на дифракционной решётке с периодом $d =  4{,}6 \lambda$?
    Под каким углом наблюдается последний максимум?
}
\answer{%
    $
        d\sin \varphi_k = k\lambda
        \implies k = \frac{d\sin \varphi_k}{\lambda} \le \frac{d \cdot 1}{\lambda} =  4{,}6
        \implies k_{\max} = 4
    $
}

\variantsplitter

\addpersonalvariant{Варвара Минаева}

\tasknumber{1}%
\task{%
    На дифракционную решётку, имеющую период $4 \cdot 10^{-4}\,\text{см}$, нормально падает монохроматическая световая волна.
    Под углом $ 40 \degrees$ наблюдается дифракционный максимум четвёртого порядка.
    Какова длина волны падающего света?
}
\answer{%
    $
        d\sin \varphi_k = k\lambda
        \implies \lambda = \frac{d \sin \varphi_k}k
        = \frac{4 \cdot 10^{-4}\,\text{см} \cdot \sin  40 \degrees}{4} \approx 640\,\text{нм}
    $
}
\solutionspace{150pt}

\tasknumber{2}%
\task{%
    Свет с длиной волны $0{,}6\,\text{мкм}$ падает нормально на дифракционную решётку с периодом, равным $1\,\text{мкм}$.
    Под каким углом наблюдается дифракционный максимум первого порядка?
}
\answer{%
    $
        d\sin \varphi_k = k\lambda
        \implies \sin \varphi_k = \frac{k\lambda}{ d }
        = \frac{1 \cdot 0{,}6\,\text{мкм}}{1\,\text{мкм}} \approx 0{,}6 \implies \varphi_k \approx 36{,}9\degrees
    $
}
\solutionspace{150pt}

\tasknumber{3}%
\task{%
    При нормальном падении белого света на дифракционную решётку синяя линия ($480\,\text{нм}$)
    в спектре третьего порядка видна под углом дифракции $18\degrees$.
    Определить число штрихов на $1\,\text{мм}$ длины этой решётки.
}
\answer{%
    $
        d\sin \varphi_k = k\lambda
        \implies d = \frac{k\lambda}{\sin \varphi_k}.
        \qquad N = \frac{ l }{d} = \frac{ l \sin \varphi_k}{k\lambda}
        = \frac{1\,\text{мм} \cdot \sin 18\degrees}{3 \cdot 480\,\text{нм}} \approx 210
    $
}
\solutionspace{150pt}

\tasknumber{4}%
\task{%
    Каков наибольший порядок спектра, который можно наблюдать при дифракции света
    с длиной волны $\lambda$, на дифракционной решётке с периодом $d =  3{,}5 \lambda$?
    Под каким углом наблюдается последний максимум?
}
\answer{%
    $
        d\sin \varphi_k = k\lambda
        \implies k = \frac{d\sin \varphi_k}{\lambda} \le \frac{d \cdot 1}{\lambda} =  3{,}5
        \implies k_{\max} = 3
    $
}

\variantsplitter

\addpersonalvariant{Леонид Никитин}

\tasknumber{1}%
\task{%
    На дифракционную решётку, имеющую период $4 \cdot 10^{-4}\,\text{см}$, нормально падает монохроматическая световая волна.
    Под углом $ 20 \degrees$ наблюдается дифракционный максимум четвёртого порядка.
    Какова длина волны падающего света?
}
\answer{%
    $
        d\sin \varphi_k = k\lambda
        \implies \lambda = \frac{d \sin \varphi_k}k
        = \frac{4 \cdot 10^{-4}\,\text{см} \cdot \sin  20 \degrees}{4} \approx 340\,\text{нм}
    $
}
\solutionspace{150pt}

\tasknumber{2}%
\task{%
    Свет с длиной волны $0{,}7\,\text{мкм}$ падает нормально на дифракционную решётку с периодом, равным $3\,\text{мкм}$.
    Под каким углом наблюдается дифракционный максимум первого порядка?
}
\answer{%
    $
        d\sin \varphi_k = k\lambda
        \implies \sin \varphi_k = \frac{k\lambda}{ d }
        = \frac{1 \cdot 0{,}7\,\text{мкм}}{3\,\text{мкм}} \approx 0{,}2 \implies \varphi_k \approx 13{,}5\degrees
    $
}
\solutionspace{150pt}

\tasknumber{3}%
\task{%
    При нормальном падении белого света на дифракционную решётку жёлтая линия ($580\,\text{нм}$)
    в спектре четвёртого порядка видна под углом дифракции $5\degrees$.
    Определить число штрихов на $1\,\text{см}$ длины этой решётки.
}
\answer{%
    $
        d\sin \varphi_k = k\lambda
        \implies d = \frac{k\lambda}{\sin \varphi_k}.
        \qquad N = \frac{ l }{d} = \frac{ l \sin \varphi_k}{k\lambda}
        = \frac{1\,\text{см} \cdot \sin 5\degrees}{4 \cdot 580\,\text{нм}} \approx 380
    $
}
\solutionspace{150pt}

\tasknumber{4}%
\task{%
    Каков наибольший порядок спектра, который можно наблюдать при дифракции света
    с длиной волны $\lambda$, на дифракционной решётке с периодом $d =  2{,}2 \lambda$?
    Под каким углом наблюдается последний максимум?
}
\answer{%
    $
        d\sin \varphi_k = k\lambda
        \implies k = \frac{d\sin \varphi_k}{\lambda} \le \frac{d \cdot 1}{\lambda} =  2{,}2
        \implies k_{\max} = 2
    $
}

\variantsplitter

\addpersonalvariant{Тимофей Полетаев}

\tasknumber{1}%
\task{%
    На дифракционную решётку, имеющую период $3 \cdot 10^{-4}\,\text{см}$, нормально падает монохроматическая световая волна.
    Под углом $ 30 \degrees$ наблюдается дифракционный максимум второго порядка.
    Какова длина волны падающего света?
}
\answer{%
    $
        d\sin \varphi_k = k\lambda
        \implies \lambda = \frac{d \sin \varphi_k}k
        = \frac{3 \cdot 10^{-4}\,\text{см} \cdot \sin  30 \degrees}{2} \approx 750\,\text{нм}
    $
}
\solutionspace{150pt}

\tasknumber{2}%
\task{%
    Свет с длиной волны $0{,}5\,\text{мкм}$ падает нормально на дифракционную решётку с периодом, равным $2\,\text{мкм}$.
    Под каким углом наблюдается дифракционный максимум первого порядка?
}
\answer{%
    $
        d\sin \varphi_k = k\lambda
        \implies \sin \varphi_k = \frac{k\lambda}{ d }
        = \frac{1 \cdot 0{,}5\,\text{мкм}}{2\,\text{мкм}} \approx 0{,}3 \implies \varphi_k \approx 14{,}5\degrees
    $
}
\solutionspace{150pt}

\tasknumber{3}%
\task{%
    При нормальном падении белого света на дифракционную решётку жёлтая линия ($570\,\text{нм}$)
    в спектре второго порядка видна под углом дифракции $25\degrees$.
    Определить число штрихов на $1\,\text{мм}$ длины этой решётки.
}
\answer{%
    $
        d\sin \varphi_k = k\lambda
        \implies d = \frac{k\lambda}{\sin \varphi_k}.
        \qquad N = \frac{ l }{d} = \frac{ l \sin \varphi_k}{k\lambda}
        = \frac{1\,\text{мм} \cdot \sin 25\degrees}{2 \cdot 570\,\text{нм}} \approx 370
    $
}
\solutionspace{150pt}

\tasknumber{4}%
\task{%
    Каков наибольший порядок спектра, который можно наблюдать при дифракции света
    с длиной волны $\lambda$, на дифракционной решётке с периодом $d =  3{,}3 \lambda$?
    Под каким углом наблюдается последний максимум?
}
\answer{%
    $
        d\sin \varphi_k = k\lambda
        \implies k = \frac{d\sin \varphi_k}{\lambda} \le \frac{d \cdot 1}{\lambda} =  3{,}3
        \implies k_{\max} = 3
    $
}

\variantsplitter

\addpersonalvariant{Андрей Рожков}

\tasknumber{1}%
\task{%
    На дифракционную решётку, имеющую период $4 \cdot 10^{-4}\,\text{см}$, нормально падает монохроматическая световая волна.
    Под углом $ 25 \degrees$ наблюдается дифракционный максимум третьего порядка.
    Какова длина волны падающего света?
}
\answer{%
    $
        d\sin \varphi_k = k\lambda
        \implies \lambda = \frac{d \sin \varphi_k}k
        = \frac{4 \cdot 10^{-4}\,\text{см} \cdot \sin  25 \degrees}{3} \approx 560\,\text{нм}
    $
}
\solutionspace{150pt}

\tasknumber{2}%
\task{%
    Свет с длиной волны $0{,}5\,\text{мкм}$ падает нормально на дифракционную решётку с периодом, равным $2\,\text{мкм}$.
    Под каким углом наблюдается дифракционный максимум первого порядка?
}
\answer{%
    $
        d\sin \varphi_k = k\lambda
        \implies \sin \varphi_k = \frac{k\lambda}{ d }
        = \frac{1 \cdot 0{,}5\,\text{мкм}}{2\,\text{мкм}} \approx 0{,}3 \implies \varphi_k \approx 14{,}5\degrees
    $
}
\solutionspace{150pt}

\tasknumber{3}%
\task{%
    При нормальном падении белого света на дифракционную решётку красная линия ($720\,\text{нм}$)
    в спектре третьего порядка видна под углом дифракции $25\degrees$.
    Определить число штрихов на $1\,\text{мм}$ длины этой решётки.
}
\answer{%
    $
        d\sin \varphi_k = k\lambda
        \implies d = \frac{k\lambda}{\sin \varphi_k}.
        \qquad N = \frac{ l }{d} = \frac{ l \sin \varphi_k}{k\lambda}
        = \frac{1\,\text{мм} \cdot \sin 25\degrees}{3 \cdot 720\,\text{нм}} \approx 196
    $
}
\solutionspace{150pt}

\tasknumber{4}%
\task{%
    Каков наибольший порядок спектра, который можно наблюдать при дифракции света
    с длиной волны $\lambda$, на дифракционной решётке с периодом $d =  3{,}9 \lambda$?
    Под каким углом наблюдается последний максимум?
}
\answer{%
    $
        d\sin \varphi_k = k\lambda
        \implies k = \frac{d\sin \varphi_k}{\lambda} \le \frac{d \cdot 1}{\lambda} =  3{,}9
        \implies k_{\max} = 3
    $
}

\variantsplitter

\addpersonalvariant{Рената Таржиманова}

\tasknumber{1}%
\task{%
    На дифракционную решётку, имеющую период $2 \cdot 10^{-4}\,\text{см}$, нормально падает монохроматическая световая волна.
    Под углом $ 25 \degrees$ наблюдается дифракционный максимум второго порядка.
    Какова длина волны падающего света?
}
\answer{%
    $
        d\sin \varphi_k = k\lambda
        \implies \lambda = \frac{d \sin \varphi_k}k
        = \frac{2 \cdot 10^{-4}\,\text{см} \cdot \sin  25 \degrees}{2} \approx 420\,\text{нм}
    $
}
\solutionspace{150pt}

\tasknumber{2}%
\task{%
    Свет с длиной волны $0{,}6\,\text{мкм}$ падает нормально на дифракционную решётку с периодом, равным $1\,\text{мкм}$.
    Под каким углом наблюдается дифракционный максимум первого порядка?
}
\answer{%
    $
        d\sin \varphi_k = k\lambda
        \implies \sin \varphi_k = \frac{k\lambda}{ d }
        = \frac{1 \cdot 0{,}6\,\text{мкм}}{1\,\text{мкм}} \approx 0{,}6 \implies \varphi_k \approx 36{,}9\degrees
    $
}
\solutionspace{150pt}

\tasknumber{3}%
\task{%
    При нормальном падении белого света на дифракционную решётку жёлтая линия ($570\,\text{нм}$)
    в спектре третьего порядка видна под углом дифракции $25\degrees$.
    Определить число штрихов на $1\,\text{см}$ длины этой решётки.
}
\answer{%
    $
        d\sin \varphi_k = k\lambda
        \implies d = \frac{k\lambda}{\sin \varphi_k}.
        \qquad N = \frac{ l }{d} = \frac{ l \sin \varphi_k}{k\lambda}
        = \frac{1\,\text{см} \cdot \sin 25\degrees}{3 \cdot 570\,\text{нм}} \approx 2500
    $
}
\solutionspace{150pt}

\tasknumber{4}%
\task{%
    Каков наибольший порядок спектра, который можно наблюдать при дифракции света
    с длиной волны $\lambda$, на дифракционной решётке с периодом $d =  4{,}5 \lambda$?
    Под каким углом наблюдается последний максимум?
}
\answer{%
    $
        d\sin \varphi_k = k\lambda
        \implies k = \frac{d\sin \varphi_k}{\lambda} \le \frac{d \cdot 1}{\lambda} =  4{,}5
        \implies k_{\max} = 4
    $
}

\variantsplitter

\addpersonalvariant{Андрей Щербаков}

\tasknumber{1}%
\task{%
    На дифракционную решётку, имеющую период $3 \cdot 10^{-4}\,\text{см}$, нормально падает монохроматическая световая волна.
    Под углом $ 25 \degrees$ наблюдается дифракционный максимум третьего порядка.
    Какова длина волны падающего света?
}
\answer{%
    $
        d\sin \varphi_k = k\lambda
        \implies \lambda = \frac{d \sin \varphi_k}k
        = \frac{3 \cdot 10^{-4}\,\text{см} \cdot \sin  25 \degrees}{3} \approx 420\,\text{нм}
    $
}
\solutionspace{150pt}

\tasknumber{2}%
\task{%
    Свет с длиной волны $0{,}7\,\text{мкм}$ падает нормально на дифракционную решётку с периодом, равным $3\,\text{мкм}$.
    Под каким углом наблюдается дифракционный максимум первого порядка?
}
\answer{%
    $
        d\sin \varphi_k = k\lambda
        \implies \sin \varphi_k = \frac{k\lambda}{ d }
        = \frac{1 \cdot 0{,}7\,\text{мкм}}{3\,\text{мкм}} \approx 0{,}2 \implies \varphi_k \approx 13{,}5\degrees
    $
}
\solutionspace{150pt}

\tasknumber{3}%
\task{%
    При нормальном падении белого света на дифракционную решётку зелёная линия ($520\,\text{нм}$)
    в спектре второго порядка видна под углом дифракции $25\degrees$.
    Определить число штрихов на $1\,\text{мм}$ длины этой решётки.
}
\answer{%
    $
        d\sin \varphi_k = k\lambda
        \implies d = \frac{k\lambda}{\sin \varphi_k}.
        \qquad N = \frac{ l }{d} = \frac{ l \sin \varphi_k}{k\lambda}
        = \frac{1\,\text{мм} \cdot \sin 25\degrees}{2 \cdot 520\,\text{нм}} \approx 410
    $
}
\solutionspace{150pt}

\tasknumber{4}%
\task{%
    Каков наибольший порядок спектра, который можно наблюдать при дифракции света
    с длиной волны $\lambda$, на дифракционной решётке с периодом $d =  4{,}5 \lambda$?
    Под каким углом наблюдается последний максимум?
}
\answer{%
    $
        d\sin \varphi_k = k\lambda
        \implies k = \frac{d\sin \varphi_k}{\lambda} \le \frac{d \cdot 1}{\lambda} =  4{,}5
        \implies k_{\max} = 4
    $
}

\variantsplitter

\addpersonalvariant{Михаил Ярошевский}

\tasknumber{1}%
\task{%
    На дифракционную решётку, имеющую период $2 \cdot 10^{-4}\,\text{см}$, нормально падает монохроматическая световая волна.
    Под углом $ 40 \degrees$ наблюдается дифракционный максимум второго порядка.
    Какова длина волны падающего света?
}
\answer{%
    $
        d\sin \varphi_k = k\lambda
        \implies \lambda = \frac{d \sin \varphi_k}k
        = \frac{2 \cdot 10^{-4}\,\text{см} \cdot \sin  40 \degrees}{2} \approx 640\,\text{нм}
    $
}
\solutionspace{150pt}

\tasknumber{2}%
\task{%
    Свет с длиной волны $0{,}5\,\text{мкм}$ падает нормально на дифракционную решётку с периодом, равным $2\,\text{мкм}$.
    Под каким углом наблюдается дифракционный максимум первого порядка?
}
\answer{%
    $
        d\sin \varphi_k = k\lambda
        \implies \sin \varphi_k = \frac{k\lambda}{ d }
        = \frac{1 \cdot 0{,}5\,\text{мкм}}{2\,\text{мкм}} \approx 0{,}3 \implies \varphi_k \approx 14{,}5\degrees
    $
}
\solutionspace{150pt}

\tasknumber{3}%
\task{%
    При нормальном падении белого света на дифракционную решётку жёлтая линия ($570\,\text{нм}$)
    в спектре третьего порядка видна под углом дифракции $12\degrees$.
    Определить число штрихов на $1\,\text{мм}$ длины этой решётки.
}
\answer{%
    $
        d\sin \varphi_k = k\lambda
        \implies d = \frac{k\lambda}{\sin \varphi_k}.
        \qquad N = \frac{ l }{d} = \frac{ l \sin \varphi_k}{k\lambda}
        = \frac{1\,\text{мм} \cdot \sin 12\degrees}{3 \cdot 570\,\text{нм}} \approx 122
    $
}
\solutionspace{150pt}

\tasknumber{4}%
\task{%
    Каков наибольший порядок спектра, который можно наблюдать при дифракции света
    с длиной волны $\lambda$, на дифракционной решётке с периодом $d =  3{,}9 \lambda$?
    Под каким углом наблюдается последний максимум?
}
\answer{%
    $
        d\sin \varphi_k = k\lambda
        \implies k = \frac{d\sin \varphi_k}{\lambda} \le \frac{d \cdot 1}{\lambda} =  3{,}9
        \implies k_{\max} = 3
    $
}

\variantsplitter

\addpersonalvariant{Алексей Алимпиев}

\tasknumber{1}%
\task{%
    На дифракционную решётку, имеющую период $3 \cdot 10^{-4}\,\text{см}$, нормально падает монохроматическая световая волна.
    Под углом $ 35 \degrees$ наблюдается дифракционный максимум четвёртого порядка.
    Какова длина волны падающего света?
}
\answer{%
    $
        d\sin \varphi_k = k\lambda
        \implies \lambda = \frac{d \sin \varphi_k}k
        = \frac{3 \cdot 10^{-4}\,\text{см} \cdot \sin  35 \degrees}{4} \approx 430\,\text{нм}
    $
}
\solutionspace{150pt}

\tasknumber{2}%
\task{%
    Свет с длиной волны $0{,}6\,\text{мкм}$ падает нормально на дифракционную решётку с периодом, равным $3\,\text{мкм}$.
    Под каким углом наблюдается дифракционный максимум первого порядка?
}
\answer{%
    $
        d\sin \varphi_k = k\lambda
        \implies \sin \varphi_k = \frac{k\lambda}{ d }
        = \frac{1 \cdot 0{,}6\,\text{мкм}}{3\,\text{мкм}} \approx 0{,}2 \implies \varphi_k \approx 11{,}5\degrees
    $
}
\solutionspace{150pt}

\tasknumber{3}%
\task{%
    При нормальном падении белого света на дифракционную решётку красная линия ($720\,\text{нм}$)
    в спектре второго порядка видна под углом дифракции $18\degrees$.
    Определить число штрихов на $1\,\text{мм}$ длины этой решётки.
}
\answer{%
    $
        d\sin \varphi_k = k\lambda
        \implies d = \frac{k\lambda}{\sin \varphi_k}.
        \qquad N = \frac{ l }{d} = \frac{ l \sin \varphi_k}{k\lambda}
        = \frac{1\,\text{мм} \cdot \sin 18\degrees}{2 \cdot 720\,\text{нм}} \approx 210
    $
}
\solutionspace{150pt}

\tasknumber{4}%
\task{%
    Каков наибольший порядок спектра, который можно наблюдать при дифракции света
    с длиной волны $\lambda$, на дифракционной решётке с периодом $d =  4{,}1 \lambda$?
    Под каким углом наблюдается последний максимум?
}
\answer{%
    $
        d\sin \varphi_k = k\lambda
        \implies k = \frac{d\sin \varphi_k}{\lambda} \le \frac{d \cdot 1}{\lambda} =  4{,}1
        \implies k_{\max} = 4
    $
}

\variantsplitter

\addpersonalvariant{Евгений Васин}

\tasknumber{1}%
\task{%
    На дифракционную решётку, имеющую период $3 \cdot 10^{-4}\,\text{см}$, нормально падает монохроматическая световая волна.
    Под углом $ 30 \degrees$ наблюдается дифракционный максимум второго порядка.
    Какова длина волны падающего света?
}
\answer{%
    $
        d\sin \varphi_k = k\lambda
        \implies \lambda = \frac{d \sin \varphi_k}k
        = \frac{3 \cdot 10^{-4}\,\text{см} \cdot \sin  30 \degrees}{2} \approx 750\,\text{нм}
    $
}
\solutionspace{150pt}

\tasknumber{2}%
\task{%
    Свет с длиной волны $0{,}6\,\text{мкм}$ падает нормально на дифракционную решётку с периодом, равным $3\,\text{мкм}$.
    Под каким углом наблюдается дифракционный максимум первого порядка?
}
\answer{%
    $
        d\sin \varphi_k = k\lambda
        \implies \sin \varphi_k = \frac{k\lambda}{ d }
        = \frac{1 \cdot 0{,}6\,\text{мкм}}{3\,\text{мкм}} \approx 0{,}2 \implies \varphi_k \approx 11{,}5\degrees
    $
}
\solutionspace{150pt}

\tasknumber{3}%
\task{%
    При нормальном падении белого света на дифракционную решётку зелёная линия ($550\,\text{нм}$)
    в спектре третьего порядка видна под углом дифракции $18\degrees$.
    Определить число штрихов на $1\,\text{см}$ длины этой решётки.
}
\answer{%
    $
        d\sin \varphi_k = k\lambda
        \implies d = \frac{k\lambda}{\sin \varphi_k}.
        \qquad N = \frac{ l }{d} = \frac{ l \sin \varphi_k}{k\lambda}
        = \frac{1\,\text{см} \cdot \sin 18\degrees}{3 \cdot 550\,\text{нм}} \approx 1870
    $
}
\solutionspace{150pt}

\tasknumber{4}%
\task{%
    Каков наибольший порядок спектра, который можно наблюдать при дифракции света
    с длиной волны $\lambda$, на дифракционной решётке с периодом $d =  2{,}5 \lambda$?
    Под каким углом наблюдается последний максимум?
}
\answer{%
    $
        d\sin \varphi_k = k\lambda
        \implies k = \frac{d\sin \varphi_k}{\lambda} \le \frac{d \cdot 1}{\lambda} =  2{,}5
        \implies k_{\max} = 2
    $
}

\variantsplitter

\addpersonalvariant{Вячеслав Волохов}

\tasknumber{1}%
\task{%
    На дифракционную решётку, имеющую период $3 \cdot 10^{-4}\,\text{см}$, нормально падает монохроматическая световая волна.
    Под углом $ 20 \degrees$ наблюдается дифракционный максимум четвёртого порядка.
    Какова длина волны падающего света?
}
\answer{%
    $
        d\sin \varphi_k = k\lambda
        \implies \lambda = \frac{d \sin \varphi_k}k
        = \frac{3 \cdot 10^{-4}\,\text{см} \cdot \sin  20 \degrees}{4} \approx 260\,\text{нм}
    $
}
\solutionspace{150pt}

\tasknumber{2}%
\task{%
    Свет с длиной волны $0{,}5\,\text{мкм}$ падает нормально на дифракционную решётку с периодом, равным $2\,\text{мкм}$.
    Под каким углом наблюдается дифракционный максимум первого порядка?
}
\answer{%
    $
        d\sin \varphi_k = k\lambda
        \implies \sin \varphi_k = \frac{k\lambda}{ d }
        = \frac{1 \cdot 0{,}5\,\text{мкм}}{2\,\text{мкм}} \approx 0{,}3 \implies \varphi_k \approx 14{,}5\degrees
    $
}
\solutionspace{150pt}

\tasknumber{3}%
\task{%
    При нормальном падении белого света на дифракционную решётку зелёная линия ($550\,\text{нм}$)
    в спектре четвёртого порядка видна под углом дифракции $5\degrees$.
    Определить число штрихов на $1\,\text{см}$ длины этой решётки.
}
\answer{%
    $
        d\sin \varphi_k = k\lambda
        \implies d = \frac{k\lambda}{\sin \varphi_k}.
        \qquad N = \frac{ l }{d} = \frac{ l \sin \varphi_k}{k\lambda}
        = \frac{1\,\text{см} \cdot \sin 5\degrees}{4 \cdot 550\,\text{нм}} \approx 400
    $
}
\solutionspace{150pt}

\tasknumber{4}%
\task{%
    Каков наибольший порядок спектра, который можно наблюдать при дифракции света
    с длиной волны $\lambda$, на дифракционной решётке с периодом $d =  4{,}6 \lambda$?
    Под каким углом наблюдается последний максимум?
}
\answer{%
    $
        d\sin \varphi_k = k\lambda
        \implies k = \frac{d\sin \varphi_k}{\lambda} \le \frac{d \cdot 1}{\lambda} =  4{,}6
        \implies k_{\max} = 4
    $
}

\variantsplitter

\addpersonalvariant{Герман Говоров}

\tasknumber{1}%
\task{%
    На дифракционную решётку, имеющую период $3 \cdot 10^{-4}\,\text{см}$, нормально падает монохроматическая световая волна.
    Под углом $ 20 \degrees$ наблюдается дифракционный максимум третьего порядка.
    Какова длина волны падающего света?
}
\answer{%
    $
        d\sin \varphi_k = k\lambda
        \implies \lambda = \frac{d \sin \varphi_k}k
        = \frac{3 \cdot 10^{-4}\,\text{см} \cdot \sin  20 \degrees}{3} \approx 340\,\text{нм}
    $
}
\solutionspace{150pt}

\tasknumber{2}%
\task{%
    Свет с длиной волны $0{,}4\,\text{мкм}$ падает нормально на дифракционную решётку с периодом, равным $3\,\text{мкм}$.
    Под каким углом наблюдается дифракционный максимум первого порядка?
}
\answer{%
    $
        d\sin \varphi_k = k\lambda
        \implies \sin \varphi_k = \frac{k\lambda}{ d }
        = \frac{1 \cdot 0{,}4\,\text{мкм}}{3\,\text{мкм}} \approx 0{,}13 \implies \varphi_k \approx 7{,}7\degrees
    $
}
\solutionspace{150pt}

\tasknumber{3}%
\task{%
    При нормальном падении белого света на дифракционную решётку красная линия ($720\,\text{нм}$)
    в спектре третьего порядка видна под углом дифракции $5\degrees$.
    Определить число штрихов на $1\,\text{см}$ длины этой решётки.
}
\answer{%
    $
        d\sin \varphi_k = k\lambda
        \implies d = \frac{k\lambda}{\sin \varphi_k}.
        \qquad N = \frac{ l }{d} = \frac{ l \sin \varphi_k}{k\lambda}
        = \frac{1\,\text{см} \cdot \sin 5\degrees}{3 \cdot 720\,\text{нм}} \approx 400
    $
}
\solutionspace{150pt}

\tasknumber{4}%
\task{%
    Каков наибольший порядок спектра, который можно наблюдать при дифракции света
    с длиной волны $\lambda$, на дифракционной решётке с периодом $d =  3{,}5 \lambda$?
    Под каким углом наблюдается последний максимум?
}
\answer{%
    $
        d\sin \varphi_k = k\lambda
        \implies k = \frac{d\sin \varphi_k}{\lambda} \le \frac{d \cdot 1}{\lambda} =  3{,}5
        \implies k_{\max} = 3
    $
}

\variantsplitter

\addpersonalvariant{София Журавлёва}

\tasknumber{1}%
\task{%
    На дифракционную решётку, имеющую период $4 \cdot 10^{-4}\,\text{см}$, нормально падает монохроматическая световая волна.
    Под углом $ 20 \degrees$ наблюдается дифракционный максимум второго порядка.
    Какова длина волны падающего света?
}
\answer{%
    $
        d\sin \varphi_k = k\lambda
        \implies \lambda = \frac{d \sin \varphi_k}k
        = \frac{4 \cdot 10^{-4}\,\text{см} \cdot \sin  20 \degrees}{2} \approx 680\,\text{нм}
    $
}
\solutionspace{150pt}

\tasknumber{2}%
\task{%
    Свет с длиной волны $0{,}7\,\text{мкм}$ падает нормально на дифракционную решётку с периодом, равным $1\,\text{мкм}$.
    Под каким углом наблюдается дифракционный максимум первого порядка?
}
\answer{%
    $
        d\sin \varphi_k = k\lambda
        \implies \sin \varphi_k = \frac{k\lambda}{ d }
        = \frac{1 \cdot 0{,}7\,\text{мкм}}{1\,\text{мкм}} \approx 0{,}7 \implies \varphi_k \approx 44{,}4\degrees
    $
}
\solutionspace{150pt}

\tasknumber{3}%
\task{%
    При нормальном падении белого света на дифракционную решётку синяя линия ($480\,\text{нм}$)
    в спектре четвёртого порядка видна под углом дифракции $18\degrees$.
    Определить число штрихов на $1\,\text{мм}$ длины этой решётки.
}
\answer{%
    $
        d\sin \varphi_k = k\lambda
        \implies d = \frac{k\lambda}{\sin \varphi_k}.
        \qquad N = \frac{ l }{d} = \frac{ l \sin \varphi_k}{k\lambda}
        = \frac{1\,\text{мм} \cdot \sin 18\degrees}{4 \cdot 480\,\text{нм}} \approx 161
    $
}
\solutionspace{150pt}

\tasknumber{4}%
\task{%
    Каков наибольший порядок спектра, который можно наблюдать при дифракции света
    с длиной волны $\lambda$, на дифракционной решётке с периодом $d =  3{,}9 \lambda$?
    Под каким углом наблюдается последний максимум?
}
\answer{%
    $
        d\sin \varphi_k = k\lambda
        \implies k = \frac{d\sin \varphi_k}{\lambda} \le \frac{d \cdot 1}{\lambda} =  3{,}9
        \implies k_{\max} = 3
    $
}

\variantsplitter

\addpersonalvariant{Константин Козлов}

\tasknumber{1}%
\task{%
    На дифракционную решётку, имеющую период $2 \cdot 10^{-4}\,\text{см}$, нормально падает монохроматическая световая волна.
    Под углом $ 35 \degrees$ наблюдается дифракционный максимум четвёртого порядка.
    Какова длина волны падающего света?
}
\answer{%
    $
        d\sin \varphi_k = k\lambda
        \implies \lambda = \frac{d \sin \varphi_k}k
        = \frac{2 \cdot 10^{-4}\,\text{см} \cdot \sin  35 \degrees}{4} \approx 290\,\text{нм}
    $
}
\solutionspace{150pt}

\tasknumber{2}%
\task{%
    Свет с длиной волны $0{,}7\,\text{мкм}$ падает нормально на дифракционную решётку с периодом, равным $3\,\text{мкм}$.
    Под каким углом наблюдается дифракционный максимум первого порядка?
}
\answer{%
    $
        d\sin \varphi_k = k\lambda
        \implies \sin \varphi_k = \frac{k\lambda}{ d }
        = \frac{1 \cdot 0{,}7\,\text{мкм}}{3\,\text{мкм}} \approx 0{,}2 \implies \varphi_k \approx 13{,}5\degrees
    $
}
\solutionspace{150pt}

\tasknumber{3}%
\task{%
    При нормальном падении белого света на дифракционную решётку синяя линия ($480\,\text{нм}$)
    в спектре второго порядка видна под углом дифракции $5\degrees$.
    Определить число штрихов на $1\,\text{мм}$ длины этой решётки.
}
\answer{%
    $
        d\sin \varphi_k = k\lambda
        \implies d = \frac{k\lambda}{\sin \varphi_k}.
        \qquad N = \frac{ l }{d} = \frac{ l \sin \varphi_k}{k\lambda}
        = \frac{1\,\text{мм} \cdot \sin 5\degrees}{2 \cdot 480\,\text{нм}} \approx 91
    $
}
\solutionspace{150pt}

\tasknumber{4}%
\task{%
    Каков наибольший порядок спектра, который можно наблюдать при дифракции света
    с длиной волны $\lambda$, на дифракционной решётке с периодом $d =  3{,}9 \lambda$?
    Под каким углом наблюдается последний максимум?
}
\answer{%
    $
        d\sin \varphi_k = k\lambda
        \implies k = \frac{d\sin \varphi_k}{\lambda} \le \frac{d \cdot 1}{\lambda} =  3{,}9
        \implies k_{\max} = 3
    $
}

\variantsplitter

\addpersonalvariant{Наталья Кравченко}

\tasknumber{1}%
\task{%
    На дифракционную решётку, имеющую период $4 \cdot 10^{-4}\,\text{см}$, нормально падает монохроматическая световая волна.
    Под углом $ 20 \degrees$ наблюдается дифракционный максимум четвёртого порядка.
    Какова длина волны падающего света?
}
\answer{%
    $
        d\sin \varphi_k = k\lambda
        \implies \lambda = \frac{d \sin \varphi_k}k
        = \frac{4 \cdot 10^{-4}\,\text{см} \cdot \sin  20 \degrees}{4} \approx 340\,\text{нм}
    $
}
\solutionspace{150pt}

\tasknumber{2}%
\task{%
    Свет с длиной волны $0{,}4\,\text{мкм}$ падает нормально на дифракционную решётку с периодом, равным $3\,\text{мкм}$.
    Под каким углом наблюдается дифракционный максимум первого порядка?
}
\answer{%
    $
        d\sin \varphi_k = k\lambda
        \implies \sin \varphi_k = \frac{k\lambda}{ d }
        = \frac{1 \cdot 0{,}4\,\text{мкм}}{3\,\text{мкм}} \approx 0{,}13 \implies \varphi_k \approx 7{,}7\degrees
    $
}
\solutionspace{150pt}

\tasknumber{3}%
\task{%
    При нормальном падении белого света на дифракционную решётку жёлтая линия ($580\,\text{нм}$)
    в спектре четвёртого порядка видна под углом дифракции $25\degrees$.
    Определить число штрихов на $1\,\text{мм}$ длины этой решётки.
}
\answer{%
    $
        d\sin \varphi_k = k\lambda
        \implies d = \frac{k\lambda}{\sin \varphi_k}.
        \qquad N = \frac{ l }{d} = \frac{ l \sin \varphi_k}{k\lambda}
        = \frac{1\,\text{мм} \cdot \sin 25\degrees}{4 \cdot 580\,\text{нм}} \approx 182
    $
}
\solutionspace{150pt}

\tasknumber{4}%
\task{%
    Каков наибольший порядок спектра, который можно наблюдать при дифракции света
    с длиной волны $\lambda$, на дифракционной решётке с периодом $d =  4{,}6 \lambda$?
    Под каким углом наблюдается последний максимум?
}
\answer{%
    $
        d\sin \varphi_k = k\lambda
        \implies k = \frac{d\sin \varphi_k}{\lambda} \le \frac{d \cdot 1}{\lambda} =  4{,}6
        \implies k_{\max} = 4
    $
}

\variantsplitter

\addpersonalvariant{Матвей Кузьмин}

\tasknumber{1}%
\task{%
    На дифракционную решётку, имеющую период $2 \cdot 10^{-4}\,\text{см}$, нормально падает монохроматическая световая волна.
    Под углом $ 20 \degrees$ наблюдается дифракционный максимум третьего порядка.
    Какова длина волны падающего света?
}
\answer{%
    $
        d\sin \varphi_k = k\lambda
        \implies \lambda = \frac{d \sin \varphi_k}k
        = \frac{2 \cdot 10^{-4}\,\text{см} \cdot \sin  20 \degrees}{3} \approx 230\,\text{нм}
    $
}
\solutionspace{150pt}

\tasknumber{2}%
\task{%
    Свет с длиной волны $0{,}6\,\text{мкм}$ падает нормально на дифракционную решётку с периодом, равным $3\,\text{мкм}$.
    Под каким углом наблюдается дифракционный максимум первого порядка?
}
\answer{%
    $
        d\sin \varphi_k = k\lambda
        \implies \sin \varphi_k = \frac{k\lambda}{ d }
        = \frac{1 \cdot 0{,}6\,\text{мкм}}{3\,\text{мкм}} \approx 0{,}2 \implies \varphi_k \approx 11{,}5\degrees
    $
}
\solutionspace{150pt}

\tasknumber{3}%
\task{%
    При нормальном падении белого света на дифракционную решётку жёлтая линия ($570\,\text{нм}$)
    в спектре третьего порядка видна под углом дифракции $25\degrees$.
    Определить число штрихов на $1\,\text{мм}$ длины этой решётки.
}
\answer{%
    $
        d\sin \varphi_k = k\lambda
        \implies d = \frac{k\lambda}{\sin \varphi_k}.
        \qquad N = \frac{ l }{d} = \frac{ l \sin \varphi_k}{k\lambda}
        = \frac{1\,\text{мм} \cdot \sin 25\degrees}{3 \cdot 570\,\text{нм}} \approx 250
    $
}
\solutionspace{150pt}

\tasknumber{4}%
\task{%
    Каков наибольший порядок спектра, который можно наблюдать при дифракции света
    с длиной волны $\lambda$, на дифракционной решётке с периодом $d =  4{,}5 \lambda$?
    Под каким углом наблюдается последний максимум?
}
\answer{%
    $
        d\sin \varphi_k = k\lambda
        \implies k = \frac{d\sin \varphi_k}{\lambda} \le \frac{d \cdot 1}{\lambda} =  4{,}5
        \implies k_{\max} = 4
    $
}

\variantsplitter

\addpersonalvariant{Сергей Малышев}

\tasknumber{1}%
\task{%
    На дифракционную решётку, имеющую период $3 \cdot 10^{-4}\,\text{см}$, нормально падает монохроматическая световая волна.
    Под углом $ 20 \degrees$ наблюдается дифракционный максимум третьего порядка.
    Какова длина волны падающего света?
}
\answer{%
    $
        d\sin \varphi_k = k\lambda
        \implies \lambda = \frac{d \sin \varphi_k}k
        = \frac{3 \cdot 10^{-4}\,\text{см} \cdot \sin  20 \degrees}{3} \approx 340\,\text{нм}
    $
}
\solutionspace{150pt}

\tasknumber{2}%
\task{%
    Свет с длиной волны $0{,}5\,\text{мкм}$ падает нормально на дифракционную решётку с периодом, равным $1\,\text{мкм}$.
    Под каким углом наблюдается дифракционный максимум первого порядка?
}
\answer{%
    $
        d\sin \varphi_k = k\lambda
        \implies \sin \varphi_k = \frac{k\lambda}{ d }
        = \frac{1 \cdot 0{,}5\,\text{мкм}}{1\,\text{мкм}} \approx 0{,}5 \implies \varphi_k \approx 30{,}0\degrees
    $
}
\solutionspace{150pt}

\tasknumber{3}%
\task{%
    При нормальном падении белого света на дифракционную решётку зелёная линия ($520\,\text{нм}$)
    в спектре четвёртого порядка видна под углом дифракции $12\degrees$.
    Определить число штрихов на $1\,\text{мм}$ длины этой решётки.
}
\answer{%
    $
        d\sin \varphi_k = k\lambda
        \implies d = \frac{k\lambda}{\sin \varphi_k}.
        \qquad N = \frac{ l }{d} = \frac{ l \sin \varphi_k}{k\lambda}
        = \frac{1\,\text{мм} \cdot \sin 12\degrees}{4 \cdot 520\,\text{нм}} \approx 100
    $
}
\solutionspace{150pt}

\tasknumber{4}%
\task{%
    Каков наибольший порядок спектра, который можно наблюдать при дифракции света
    с длиной волны $\lambda$, на дифракционной решётке с периодом $d =  2{,}5 \lambda$?
    Под каким углом наблюдается последний максимум?
}
\answer{%
    $
        d\sin \varphi_k = k\lambda
        \implies k = \frac{d\sin \varphi_k}{\lambda} \le \frac{d \cdot 1}{\lambda} =  2{,}5
        \implies k_{\max} = 2
    $
}

\variantsplitter

\addpersonalvariant{Алина Полканова}

\tasknumber{1}%
\task{%
    На дифракционную решётку, имеющую период $2 \cdot 10^{-4}\,\text{см}$, нормально падает монохроматическая световая волна.
    Под углом $ 20 \degrees$ наблюдается дифракционный максимум третьего порядка.
    Какова длина волны падающего света?
}
\answer{%
    $
        d\sin \varphi_k = k\lambda
        \implies \lambda = \frac{d \sin \varphi_k}k
        = \frac{2 \cdot 10^{-4}\,\text{см} \cdot \sin  20 \degrees}{3} \approx 230\,\text{нм}
    $
}
\solutionspace{150pt}

\tasknumber{2}%
\task{%
    Свет с длиной волны $0{,}7\,\text{мкм}$ падает нормально на дифракционную решётку с периодом, равным $2\,\text{мкм}$.
    Под каким углом наблюдается дифракционный максимум первого порядка?
}
\answer{%
    $
        d\sin \varphi_k = k\lambda
        \implies \sin \varphi_k = \frac{k\lambda}{ d }
        = \frac{1 \cdot 0{,}7\,\text{мкм}}{2\,\text{мкм}} \approx 0{,}4 \implies \varphi_k \approx 20{,}5\degrees
    $
}
\solutionspace{150pt}

\tasknumber{3}%
\task{%
    При нормальном падении белого света на дифракционную решётку красная линия ($720\,\text{нм}$)
    в спектре четвёртого порядка видна под углом дифракции $12\degrees$.
    Определить число штрихов на $1\,\text{см}$ длины этой решётки.
}
\answer{%
    $
        d\sin \varphi_k = k\lambda
        \implies d = \frac{k\lambda}{\sin \varphi_k}.
        \qquad N = \frac{ l }{d} = \frac{ l \sin \varphi_k}{k\lambda}
        = \frac{1\,\text{см} \cdot \sin 12\degrees}{4 \cdot 720\,\text{нм}} \approx 720
    $
}
\solutionspace{150pt}

\tasknumber{4}%
\task{%
    Каков наибольший порядок спектра, который можно наблюдать при дифракции света
    с длиной волны $\lambda$, на дифракционной решётке с периодом $d =  4{,}5 \lambda$?
    Под каким углом наблюдается последний максимум?
}
\answer{%
    $
        d\sin \varphi_k = k\lambda
        \implies k = \frac{d\sin \varphi_k}{\lambda} \le \frac{d \cdot 1}{\lambda} =  4{,}5
        \implies k_{\max} = 4
    $
}

\variantsplitter

\addpersonalvariant{Сергей Пономарёв}

\tasknumber{1}%
\task{%
    На дифракционную решётку, имеющую период $4 \cdot 10^{-4}\,\text{см}$, нормально падает монохроматическая световая волна.
    Под углом $ 20 \degrees$ наблюдается дифракционный максимум третьего порядка.
    Какова длина волны падающего света?
}
\answer{%
    $
        d\sin \varphi_k = k\lambda
        \implies \lambda = \frac{d \sin \varphi_k}k
        = \frac{4 \cdot 10^{-4}\,\text{см} \cdot \sin  20 \degrees}{3} \approx 460\,\text{нм}
    $
}
\solutionspace{150pt}

\tasknumber{2}%
\task{%
    Свет с длиной волны $0{,}5\,\text{мкм}$ падает нормально на дифракционную решётку с периодом, равным $2\,\text{мкм}$.
    Под каким углом наблюдается дифракционный максимум первого порядка?
}
\answer{%
    $
        d\sin \varphi_k = k\lambda
        \implies \sin \varphi_k = \frac{k\lambda}{ d }
        = \frac{1 \cdot 0{,}5\,\text{мкм}}{2\,\text{мкм}} \approx 0{,}3 \implies \varphi_k \approx 14{,}5\degrees
    $
}
\solutionspace{150pt}

\tasknumber{3}%
\task{%
    При нормальном падении белого света на дифракционную решётку жёлтая линия ($580\,\text{нм}$)
    в спектре третьего порядка видна под углом дифракции $5\degrees$.
    Определить число штрихов на $1\,\text{мм}$ длины этой решётки.
}
\answer{%
    $
        d\sin \varphi_k = k\lambda
        \implies d = \frac{k\lambda}{\sin \varphi_k}.
        \qquad N = \frac{ l }{d} = \frac{ l \sin \varphi_k}{k\lambda}
        = \frac{1\,\text{мм} \cdot \sin 5\degrees}{3 \cdot 580\,\text{нм}} \approx 50
    $
}
\solutionspace{150pt}

\tasknumber{4}%
\task{%
    Каков наибольший порядок спектра, который можно наблюдать при дифракции света
    с длиной волны $\lambda$, на дифракционной решётке с периодом $d =  4{,}6 \lambda$?
    Под каким углом наблюдается последний максимум?
}
\answer{%
    $
        d\sin \varphi_k = k\lambda
        \implies k = \frac{d\sin \varphi_k}{\lambda} \le \frac{d \cdot 1}{\lambda} =  4{,}6
        \implies k_{\max} = 4
    $
}

\variantsplitter

\addpersonalvariant{Егор Свистушкин}

\tasknumber{1}%
\task{%
    На дифракционную решётку, имеющую период $3 \cdot 10^{-4}\,\text{см}$, нормально падает монохроматическая световая волна.
    Под углом $ 30 \degrees$ наблюдается дифракционный максимум четвёртого порядка.
    Какова длина волны падающего света?
}
\answer{%
    $
        d\sin \varphi_k = k\lambda
        \implies \lambda = \frac{d \sin \varphi_k}k
        = \frac{3 \cdot 10^{-4}\,\text{см} \cdot \sin  30 \degrees}{4} \approx 370\,\text{нм}
    $
}
\solutionspace{150pt}

\tasknumber{2}%
\task{%
    Свет с длиной волны $0{,}5\,\text{мкм}$ падает нормально на дифракционную решётку с периодом, равным $3\,\text{мкм}$.
    Под каким углом наблюдается дифракционный максимум первого порядка?
}
\answer{%
    $
        d\sin \varphi_k = k\lambda
        \implies \sin \varphi_k = \frac{k\lambda}{ d }
        = \frac{1 \cdot 0{,}5\,\text{мкм}}{3\,\text{мкм}} \approx 0{,}17 \implies \varphi_k \approx 9{,}6\degrees
    $
}
\solutionspace{150pt}

\tasknumber{3}%
\task{%
    При нормальном падении белого света на дифракционную решётку жёлтая линия ($580\,\text{нм}$)
    в спектре второго порядка видна под углом дифракции $18\degrees$.
    Определить число штрихов на $1\,\text{мм}$ длины этой решётки.
}
\answer{%
    $
        d\sin \varphi_k = k\lambda
        \implies d = \frac{k\lambda}{\sin \varphi_k}.
        \qquad N = \frac{ l }{d} = \frac{ l \sin \varphi_k}{k\lambda}
        = \frac{1\,\text{мм} \cdot \sin 18\degrees}{2 \cdot 580\,\text{нм}} \approx 270
    $
}
\solutionspace{150pt}

\tasknumber{4}%
\task{%
    Каков наибольший порядок спектра, который можно наблюдать при дифракции света
    с длиной волны $\lambda$, на дифракционной решётке с периодом $d =  3{,}5 \lambda$?
    Под каким углом наблюдается последний максимум?
}
\answer{%
    $
        d\sin \varphi_k = k\lambda
        \implies k = \frac{d\sin \varphi_k}{\lambda} \le \frac{d \cdot 1}{\lambda} =  3{,}5
        \implies k_{\max} = 3
    $
}

\variantsplitter

\addpersonalvariant{Дмитрий Соколов}

\tasknumber{1}%
\task{%
    На дифракционную решётку, имеющую период $2 \cdot 10^{-4}\,\text{см}$, нормально падает монохроматическая световая волна.
    Под углом $ 35 \degrees$ наблюдается дифракционный максимум четвёртого порядка.
    Какова длина волны падающего света?
}
\answer{%
    $
        d\sin \varphi_k = k\lambda
        \implies \lambda = \frac{d \sin \varphi_k}k
        = \frac{2 \cdot 10^{-4}\,\text{см} \cdot \sin  35 \degrees}{4} \approx 290\,\text{нм}
    $
}
\solutionspace{150pt}

\tasknumber{2}%
\task{%
    Свет с длиной волны $0{,}7\,\text{мкм}$ падает нормально на дифракционную решётку с периодом, равным $2\,\text{мкм}$.
    Под каким углом наблюдается дифракционный максимум первого порядка?
}
\answer{%
    $
        d\sin \varphi_k = k\lambda
        \implies \sin \varphi_k = \frac{k\lambda}{ d }
        = \frac{1 \cdot 0{,}7\,\text{мкм}}{2\,\text{мкм}} \approx 0{,}4 \implies \varphi_k \approx 20{,}5\degrees
    $
}
\solutionspace{150pt}

\tasknumber{3}%
\task{%
    При нормальном падении белого света на дифракционную решётку красная линия ($680\,\text{нм}$)
    в спектре третьего порядка видна под углом дифракции $5\degrees$.
    Определить число штрихов на $1\,\text{мм}$ длины этой решётки.
}
\answer{%
    $
        d\sin \varphi_k = k\lambda
        \implies d = \frac{k\lambda}{\sin \varphi_k}.
        \qquad N = \frac{ l }{d} = \frac{ l \sin \varphi_k}{k\lambda}
        = \frac{1\,\text{мм} \cdot \sin 5\degrees}{3 \cdot 680\,\text{нм}} \approx 43
    $
}
\solutionspace{150pt}

\tasknumber{4}%
\task{%
    Каков наибольший порядок спектра, который можно наблюдать при дифракции света
    с длиной волны $\lambda$, на дифракционной решётке с периодом $d =  2{,}5 \lambda$?
    Под каким углом наблюдается последний максимум?
}
\answer{%
    $
        d\sin \varphi_k = k\lambda
        \implies k = \frac{d\sin \varphi_k}{\lambda} \le \frac{d \cdot 1}{\lambda} =  2{,}5
        \implies k_{\max} = 2
    $
}

\variantsplitter

\addpersonalvariant{Арсений Трофимов}

\tasknumber{1}%
\task{%
    На дифракционную решётку, имеющую период $4 \cdot 10^{-4}\,\text{см}$, нормально падает монохроматическая световая волна.
    Под углом $ 35 \degrees$ наблюдается дифракционный максимум четвёртого порядка.
    Какова длина волны падающего света?
}
\answer{%
    $
        d\sin \varphi_k = k\lambda
        \implies \lambda = \frac{d \sin \varphi_k}k
        = \frac{4 \cdot 10^{-4}\,\text{см} \cdot \sin  35 \degrees}{4} \approx 570\,\text{нм}
    $
}
\solutionspace{150pt}

\tasknumber{2}%
\task{%
    Свет с длиной волны $0{,}7\,\text{мкм}$ падает нормально на дифракционную решётку с периодом, равным $2\,\text{мкм}$.
    Под каким углом наблюдается дифракционный максимум первого порядка?
}
\answer{%
    $
        d\sin \varphi_k = k\lambda
        \implies \sin \varphi_k = \frac{k\lambda}{ d }
        = \frac{1 \cdot 0{,}7\,\text{мкм}}{2\,\text{мкм}} \approx 0{,}4 \implies \varphi_k \approx 20{,}5\degrees
    $
}
\solutionspace{150pt}

\tasknumber{3}%
\task{%
    При нормальном падении белого света на дифракционную решётку синяя линия ($480\,\text{нм}$)
    в спектре четвёртого порядка видна под углом дифракции $25\degrees$.
    Определить число штрихов на $1\,\text{мм}$ длины этой решётки.
}
\answer{%
    $
        d\sin \varphi_k = k\lambda
        \implies d = \frac{k\lambda}{\sin \varphi_k}.
        \qquad N = \frac{ l }{d} = \frac{ l \sin \varphi_k}{k\lambda}
        = \frac{1\,\text{мм} \cdot \sin 25\degrees}{4 \cdot 480\,\text{нм}} \approx 220
    $
}
\solutionspace{150pt}

\tasknumber{4}%
\task{%
    Каков наибольший порядок спектра, который можно наблюдать при дифракции света
    с длиной волны $\lambda$, на дифракционной решётке с периодом $d =  4{,}1 \lambda$?
    Под каким углом наблюдается последний максимум?
}
\answer{%
    $
        d\sin \varphi_k = k\lambda
        \implies k = \frac{d\sin \varphi_k}{\lambda} \le \frac{d \cdot 1}{\lambda} =  4{,}1
        \implies k_{\max} = 4
    $
}
% autogenerated
