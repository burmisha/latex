\setdate{10~ноября~2021}
\setclass{11«БА»}

\addpersonalvariant{Михаил Бурмистров}

\tasknumber{1}%
\task{%
    Электрический колебательный контур состоит
    из катушки индуктивностью $L$ и конденсатора ёмкостью $C$.
    Последовательно катушке подключают ещё одну катушку индуктивностью $2L$.
    Как изменится частота свободных колебаний в контуре?
}
\answer{%
    $
        T = 2\pi\sqrt{LC}, \quad
        T' = 2\pi\sqrt{L'C'}
            = T \sqrt{\frac{L'}L \cdot \frac{C'}C}
            = T \sqrt{ 3 \cdot 1 }
        \implies \frac{\nu'}{\nu} = \frac{T}{T'} = \frac1{\sqrt{ 3 \cdot 1 }} \approx 0{,}577.
    $
}
\solutionspace{100pt}

\tasknumber{2}%
\task{%
    В $LC$-контуре ёмкость конденсатора $2\,\text{мкФ}$, а максимальное напряжение на нём $9\,\text{В}$.
    Определите энергию магнитного поля катушки в момент времени, когда напряжение на конденсаторе оказалось равным $3\,\text{В}$.
}
\answer{%
    \begin{align*}
    W &= \frac{LI^2}2 + \frac{CU^2}2 = \frac{CU_m^2}2 \implies W_L = \frac{CU_m^2}2 - \frac{CU^2}2 = \frac C2(U_m^2 - U^2) = \\
    &= \frac {2\,\text{мкФ}}2 \cbr{\sqr{9\,\text{В}} - \sqr{3\,\text{В}}} \approx 72\,\text{мкДж}.
    \end{align*}
}
\solutionspace{100pt}

\tasknumber{3}%
\task{%
    Конденсатор ёмкостью $4\,\text{мкФ}$ зарядили до напряжения $10\,\text{В}$ и подключили к катушке индуктивностью $40\,\text{мГн}$.
    Определите максимальную энергию электрического поля конденсатора и период колебаний в $LC$-контуре.
}
\answer{%
    \begin{align*}
    W &= \frac{CU^2}2 = \frac{4\,\text{мкФ} \cdot \sqr{10\,\text{В}}}2 \approx 200\,\text{мкДж}, \\
    T &= 2\pi\sqrt{LC}= 2\pi\sqrt{40\,\text{мГн} \cdot 4\,\text{мкФ}} \approx 2{,}51\,\text{мс.}
    \end{align*}
}
\solutionspace{100pt}

\tasknumber{4}%
\task{%
    Во сколько раз (и как) изменится частота свободных незатухающих колебаний в контуре,
    если его индуктивность увеличить в четыре раза, а ёмкость уменьшить в три раза раз?
}
\answer{%
    $\text{уменьшится в $1{,}15$ раз}$
}
\solutionspace{100pt}

\tasknumber{5}%
\task{%
    В схеме (см.
    рис.
    на доске) при разомкнутом ключе $K$ конденсатор ёмкостью $C = 10\,\text{мкФ}$ заряжен до напряжения $U_0 = 10\,\text{В}$.
    ЭДС батареи $\ele = 15\,\text{В}$, индуктивность катушки $L = 0{,}10\,\text{В}$.
    Определите
    \begin{itemize}
        \item чему равен установившийся ток в цепи после замыкания ключа?
        \item чему равен максимальный ток в цепи после замыкания ключа?
    \end{itemize}
    Внутренним сопротивлением батареи и омическим сопротивлением катушки пренебречь, $D$ — идеальный диод.
}

\variantsplitter

\addpersonalvariant{Ирина Ан}

\tasknumber{1}%
\task{%
    Электрический колебательный контур состоит
    из катушки индуктивностью $L$ и конденсатора ёмкостью $C$.
    Параллельно конденсатору подключают ещё один конденсатор ёмкостью $2C$.
    Как изменится частота свободных колебаний в контуре?
}
\answer{%
    $
        T = 2\pi\sqrt{LC}, \quad
        T' = 2\pi\sqrt{L'C'}
            = T \sqrt{\frac{L'}L \cdot \frac{C'}C}
            = T \sqrt{ 1 \cdot 3 }
        \implies \frac{\nu'}{\nu} = \frac{T}{T'} = \frac1{\sqrt{ 1 \cdot 3 }} \approx 0{,}577.
    $
}
\solutionspace{100pt}

\tasknumber{2}%
\task{%
    В $LC$-контуре ёмкость конденсатора $8\,\text{мкФ}$, а максимальное напряжение на нём $7\,\text{В}$.
    Определите энергию магнитного поля катушки в момент времени, когда напряжение на конденсаторе оказалось равным $1\,\text{В}$.
}
\answer{%
    \begin{align*}
    W &= \frac{LI^2}2 + \frac{CU^2}2 = \frac{CU_m^2}2 \implies W_L = \frac{CU_m^2}2 - \frac{CU^2}2 = \frac C2(U_m^2 - U^2) = \\
    &= \frac {8\,\text{мкФ}}2 \cbr{\sqr{7\,\text{В}} - \sqr{1\,\text{В}}} \approx 192\,\text{мкДж}.
    \end{align*}
}
\solutionspace{100pt}

\tasknumber{3}%
\task{%
    Конденсатор ёмкостью $2\,\text{мкФ}$ зарядили до напряжения $10\,\text{В}$ и подключили к катушке индуктивностью $25\,\text{мГн}$.
    Определите максимальную энергию магнитного поля катушки и период колебаний в $LC$-контуре.
}
\answer{%
    \begin{align*}
    W &= \frac{CU^2}2 = \frac{2\,\text{мкФ} \cdot \sqr{10\,\text{В}}}2 \approx 100\,\text{мкДж}, \\
    T &= 2\pi\sqrt{LC}= 2\pi\sqrt{25\,\text{мГн} \cdot 2\,\text{мкФ}} \approx 1{,}40\,\text{мс.}
    \end{align*}
}
\solutionspace{100pt}

\tasknumber{4}%
\task{%
    Во сколько раз (и как) изменится период свободных незатухающих колебаний в контуре,
    если его индуктивность уменьшить в четыре раза, а ёмкость увеличить в семь раз раз?
}
\answer{%
    $\text{увеличится в $1{,}32$ раз}$
}
\solutionspace{100pt}

\tasknumber{5}%
\task{%
    В схеме (см.
    рис.
    на доске) при разомкнутом ключе $K$ конденсатор ёмкостью $C = 10\,\text{мкФ}$ заряжен до напряжения $U_0 = 10\,\text{В}$.
    ЭДС батареи $\ele = 15\,\text{В}$, индуктивность катушки $L = 0{,}10\,\text{В}$.
    Определите
    \begin{itemize}
        \item чему равен установившийся ток в цепи после замыкания ключа?
        \item чему равен максимальный ток в цепи после замыкания ключа?
    \end{itemize}
    Внутренним сопротивлением батареи и омическим сопротивлением катушки пренебречь, $D$ — идеальный диод.
}

\variantsplitter

\addpersonalvariant{Софья Андрианова}

\tasknumber{1}%
\task{%
    Электрический колебательный контур состоит
    из катушки индуктивностью $L$ и конденсатора ёмкостью $C$.
    Параллельно конденсатору подключают ещё один конденсатор ёмкостью $2C$.
    Как изменится частота свободных колебаний в контуре?
}
\answer{%
    $
        T = 2\pi\sqrt{LC}, \quad
        T' = 2\pi\sqrt{L'C'}
            = T \sqrt{\frac{L'}L \cdot \frac{C'}C}
            = T \sqrt{ 1 \cdot 3 }
        \implies \frac{\nu'}{\nu} = \frac{T}{T'} = \frac1{\sqrt{ 1 \cdot 3 }} \approx 0{,}577.
    $
}
\solutionspace{100pt}

\tasknumber{2}%
\task{%
    В $LC$-контуре ёмкость конденсатора $2\,\text{мкФ}$, а максимальное напряжение на нём $12\,\text{В}$.
    Определите энергию магнитного поля катушки в момент времени, когда напряжение на конденсаторе оказалось равным $1\,\text{В}$.
}
\answer{%
    \begin{align*}
    W &= \frac{LI^2}2 + \frac{CU^2}2 = \frac{CU_m^2}2 \implies W_L = \frac{CU_m^2}2 - \frac{CU^2}2 = \frac C2(U_m^2 - U^2) = \\
    &= \frac {2\,\text{мкФ}}2 \cbr{\sqr{12\,\text{В}} - \sqr{1\,\text{В}}} \approx 143\,\text{мкДж}.
    \end{align*}
}
\solutionspace{100pt}

\tasknumber{3}%
\task{%
    Конденсатор ёмкостью $6\,\text{мкФ}$ зарядили до напряжения $5\,\text{В}$ и подключили к катушке индуктивностью $20\,\text{мГн}$.
    Определите максимальную энергию электрического поля конденсатора и период колебаний в $LC$-контуре.
}
\answer{%
    \begin{align*}
    W &= \frac{CU^2}2 = \frac{6\,\text{мкФ} \cdot \sqr{5\,\text{В}}}2 \approx 75\,\text{мкДж}, \\
    T &= 2\pi\sqrt{LC}= 2\pi\sqrt{20\,\text{мГн} \cdot 6\,\text{мкФ}} \approx 2{,}18\,\text{мс.}
    \end{align*}
}
\solutionspace{100pt}

\tasknumber{4}%
\task{%
    Во сколько раз (и как) изменится циклическая частота свободных незатухающих колебаний в контуре,
    если его индуктивность уменьшить в три раза, а ёмкость уменьшить в пять раз раз?
}
\answer{%
    $\text{увеличится в $3{,}87$ раз}$
}
\solutionspace{100pt}

\tasknumber{5}%
\task{%
    В схеме (см.
    рис.
    на доске) при разомкнутом ключе $K$ конденсатор ёмкостью $C = 10\,\text{мкФ}$ заряжен до напряжения $U_0 = 10\,\text{В}$.
    ЭДС батареи $\ele = 15\,\text{В}$, индуктивность катушки $L = 0{,}10\,\text{В}$.
    Определите
    \begin{itemize}
        \item чему равен установившийся ток в цепи после замыкания ключа?
        \item чему равен максимальный ток в цепи после замыкания ключа?
    \end{itemize}
    Внутренним сопротивлением батареи и омическим сопротивлением катушки пренебречь, $D$ — идеальный диод.
}

\variantsplitter

\addpersonalvariant{Владимир Артемчук}

\tasknumber{1}%
\task{%
    Электрический колебательный контур состоит
    из катушки индуктивностью $L$ и конденсатора ёмкостью $C$.
    Параллельно катушке подключают ещё одну катушку индуктивностью $3L$.
    Как изменится частота свободных колебаний в контуре?
}
\answer{%
    $
        T = 2\pi\sqrt{LC}, \quad
        T' = 2\pi\sqrt{L'C'}
            = T \sqrt{\frac{L'}L \cdot \frac{C'}C}
            = T \sqrt{ \frac34 \cdot 1 }
        \implies \frac{\nu'}{\nu} = \frac{T}{T'} = \frac1{\sqrt{ \frac34 \cdot 1 }} \approx 1{,}155.
    $
}
\solutionspace{100pt}

\tasknumber{2}%
\task{%
    В $LC$-контуре ёмкость конденсатора $6\,\text{мкФ}$, а максимальное напряжение на нём $9\,\text{В}$.
    Определите энергию магнитного поля катушки в момент времени, когда напряжение на конденсаторе оказалось равным $5\,\text{В}$.
}
\answer{%
    \begin{align*}
    W &= \frac{LI^2}2 + \frac{CU^2}2 = \frac{CU_m^2}2 \implies W_L = \frac{CU_m^2}2 - \frac{CU^2}2 = \frac C2(U_m^2 - U^2) = \\
    &= \frac {6\,\text{мкФ}}2 \cbr{\sqr{9\,\text{В}} - \sqr{5\,\text{В}}} \approx 168\,\text{мкДж}.
    \end{align*}
}
\solutionspace{100pt}

\tasknumber{3}%
\task{%
    Конденсатор ёмкостью $4\,\text{мкФ}$ зарядили до напряжения $5\,\text{В}$ и подключили к катушке индуктивностью $25\,\text{мГн}$.
    Определите максимальную энергию электрического поля конденсатора и период колебаний в $LC$-контуре.
}
\answer{%
    \begin{align*}
    W &= \frac{CU^2}2 = \frac{4\,\text{мкФ} \cdot \sqr{5\,\text{В}}}2 \approx 50\,\text{мкДж}, \\
    T &= 2\pi\sqrt{LC}= 2\pi\sqrt{25\,\text{мГн} \cdot 4\,\text{мкФ}} \approx 1{,}99\,\text{мс.}
    \end{align*}
}
\solutionspace{100pt}

\tasknumber{4}%
\task{%
    Во сколько раз (и как) изменится частота свободных незатухающих колебаний в контуре,
    если его индуктивность увеличить в пять раз, а ёмкость уменьшить в четыре раза раз?
}
\answer{%
    $\text{уменьшится в $1{,}12$ раз}$
}
\solutionspace{100pt}

\tasknumber{5}%
\task{%
    В схеме (см.
    рис.
    на доске) при разомкнутом ключе $K$ конденсатор ёмкостью $C = 10\,\text{мкФ}$ заряжен до напряжения $U_0 = 10\,\text{В}$.
    ЭДС батареи $\ele = 15\,\text{В}$, индуктивность катушки $L = 0{,}10\,\text{В}$.
    Определите
    \begin{itemize}
        \item чему равен установившийся ток в цепи после замыкания ключа?
        \item чему равен максимальный ток в цепи после замыкания ключа?
    \end{itemize}
    Внутренним сопротивлением батареи и омическим сопротивлением катушки пренебречь, $D$ — идеальный диод.
}

\variantsplitter

\addpersonalvariant{Софья Белянкина}

\tasknumber{1}%
\task{%
    Электрический колебательный контур состоит
    из катушки индуктивностью $L$ и конденсатора ёмкостью $C$.
    Последовательно конденсатору подключают ещё один конденсатор ёмкостью $2C$.
    Как изменится частота свободных колебаний в контуре?
}
\answer{%
    $
        T = 2\pi\sqrt{LC}, \quad
        T' = 2\pi\sqrt{L'C'}
            = T \sqrt{\frac{L'}L \cdot \frac{C'}C}
            = T \sqrt{ 1 \cdot \frac23 }
        \implies \frac{\nu'}{\nu} = \frac{T}{T'} = \frac1{\sqrt{ 1 \cdot \frac23 }} \approx 1{,}225.
    $
}
\solutionspace{100pt}

\tasknumber{2}%
\task{%
    В $LC$-контуре ёмкость конденсатора $2\,\text{мкФ}$, а максимальное напряжение на нём $7\,\text{В}$.
    Определите энергию магнитного поля катушки в момент времени, когда напряжение на конденсаторе оказалось равным $1\,\text{В}$.
}
\answer{%
    \begin{align*}
    W &= \frac{LI^2}2 + \frac{CU^2}2 = \frac{CU_m^2}2 \implies W_L = \frac{CU_m^2}2 - \frac{CU^2}2 = \frac C2(U_m^2 - U^2) = \\
    &= \frac {2\,\text{мкФ}}2 \cbr{\sqr{7\,\text{В}} - \sqr{1\,\text{В}}} \approx 48\,\text{мкДж}.
    \end{align*}
}
\solutionspace{100pt}

\tasknumber{3}%
\task{%
    Конденсатор ёмкостью $2\,\text{мкФ}$ зарядили до напряжения $8\,\text{В}$ и подключили к катушке индуктивностью $25\,\text{мГн}$.
    Определите максимальную энергию электрического поля конденсатора и период колебаний в $LC$-контуре.
}
\answer{%
    \begin{align*}
    W &= \frac{CU^2}2 = \frac{2\,\text{мкФ} \cdot \sqr{8\,\text{В}}}2 \approx 64\,\text{мкДж}, \\
    T &= 2\pi\sqrt{LC}= 2\pi\sqrt{25\,\text{мГн} \cdot 2\,\text{мкФ}} \approx 1{,}40\,\text{мс.}
    \end{align*}
}
\solutionspace{100pt}

\tasknumber{4}%
\task{%
    Во сколько раз (и как) изменится период свободных незатухающих колебаний в контуре,
    если его индуктивность увеличить в пять раз, а ёмкость увеличить в три раза раз?
}
\answer{%
    $\text{увеличится в $3{,}87$ раз}$
}
\solutionspace{100pt}

\tasknumber{5}%
\task{%
    В схеме (см.
    рис.
    на доске) при разомкнутом ключе $K$ конденсатор ёмкостью $C = 10\,\text{мкФ}$ заряжен до напряжения $U_0 = 10\,\text{В}$.
    ЭДС батареи $\ele = 15\,\text{В}$, индуктивность катушки $L = 0{,}10\,\text{В}$.
    Определите
    \begin{itemize}
        \item чему равен установившийся ток в цепи после замыкания ключа?
        \item чему равен максимальный ток в цепи после замыкания ключа?
    \end{itemize}
    Внутренним сопротивлением батареи и омическим сопротивлением катушки пренебречь, $D$ — идеальный диод.
}

\variantsplitter

\addpersonalvariant{Варвара Егиазарян}

\tasknumber{1}%
\task{%
    Электрический колебательный контур состоит
    из катушки индуктивностью $L$ и конденсатора ёмкостью $C$.
    Параллельно конденсатору подключают ещё один конденсатор ёмкостью $2C$.
    Как изменится частота свободных колебаний в контуре?
}
\answer{%
    $
        T = 2\pi\sqrt{LC}, \quad
        T' = 2\pi\sqrt{L'C'}
            = T \sqrt{\frac{L'}L \cdot \frac{C'}C}
            = T \sqrt{ 1 \cdot 3 }
        \implies \frac{\nu'}{\nu} = \frac{T}{T'} = \frac1{\sqrt{ 1 \cdot 3 }} \approx 0{,}577.
    $
}
\solutionspace{100pt}

\tasknumber{2}%
\task{%
    В $LC$-контуре ёмкость конденсатора $4\,\text{мкФ}$, а максимальное напряжение на нём $7\,\text{В}$.
    Определите энергию магнитного поля катушки в момент времени, когда напряжение на конденсаторе оказалось равным $5\,\text{В}$.
}
\answer{%
    \begin{align*}
    W &= \frac{LI^2}2 + \frac{CU^2}2 = \frac{CU_m^2}2 \implies W_L = \frac{CU_m^2}2 - \frac{CU^2}2 = \frac C2(U_m^2 - U^2) = \\
    &= \frac {4\,\text{мкФ}}2 \cbr{\sqr{7\,\text{В}} - \sqr{5\,\text{В}}} \approx 48\,\text{мкДж}.
    \end{align*}
}
\solutionspace{100pt}

\tasknumber{3}%
\task{%
    Конденсатор ёмкостью $2\,\text{мкФ}$ зарядили до напряжения $8\,\text{В}$ и подключили к катушке индуктивностью $20\,\text{мГн}$.
    Определите максимальную энергию электрического поля конденсатора и период колебаний в $LC$-контуре.
}
\answer{%
    \begin{align*}
    W &= \frac{CU^2}2 = \frac{2\,\text{мкФ} \cdot \sqr{8\,\text{В}}}2 \approx 64\,\text{мкДж}, \\
    T &= 2\pi\sqrt{LC}= 2\pi\sqrt{20\,\text{мГн} \cdot 2\,\text{мкФ}} \approx 1{,}26\,\text{мс.}
    \end{align*}
}
\solutionspace{100pt}

\tasknumber{4}%
\task{%
    Во сколько раз (и как) изменится частота свободных незатухающих колебаний в контуре,
    если его индуктивность уменьшить в три раза, а ёмкость увеличить в семь раз раз?
}
\answer{%
    $\text{уменьшится в $1{,}53$ раз}$
}
\solutionspace{100pt}

\tasknumber{5}%
\task{%
    В схеме (см.
    рис.
    на доске) при разомкнутом ключе $K$ конденсатор ёмкостью $C = 10\,\text{мкФ}$ заряжен до напряжения $U_0 = 10\,\text{В}$.
    ЭДС батареи $\ele = 15\,\text{В}$, индуктивность катушки $L = 0{,}10\,\text{В}$.
    Определите
    \begin{itemize}
        \item чему равен установившийся ток в цепи после замыкания ключа?
        \item чему равен максимальный ток в цепи после замыкания ключа?
    \end{itemize}
    Внутренним сопротивлением батареи и омическим сопротивлением катушки пренебречь, $D$ — идеальный диод.
}

\variantsplitter

\addpersonalvariant{Владислав Емелин}

\tasknumber{1}%
\task{%
    Электрический колебательный контур состоит
    из катушки индуктивностью $L$ и конденсатора ёмкостью $C$.
    Последовательно конденсатору подключают ещё один конденсатор ёмкостью $2C$.
    Как изменится частота свободных колебаний в контуре?
}
\answer{%
    $
        T = 2\pi\sqrt{LC}, \quad
        T' = 2\pi\sqrt{L'C'}
            = T \sqrt{\frac{L'}L \cdot \frac{C'}C}
            = T \sqrt{ 1 \cdot \frac23 }
        \implies \frac{\nu'}{\nu} = \frac{T}{T'} = \frac1{\sqrt{ 1 \cdot \frac23 }} \approx 1{,}225.
    $
}
\solutionspace{100pt}

\tasknumber{2}%
\task{%
    В $LC$-контуре ёмкость конденсатора $8\,\text{мкФ}$, а максимальное напряжение на нём $9\,\text{В}$.
    Определите энергию магнитного поля катушки в момент времени, когда напряжение на конденсаторе оказалось равным $1\,\text{В}$.
}
\answer{%
    \begin{align*}
    W &= \frac{LI^2}2 + \frac{CU^2}2 = \frac{CU_m^2}2 \implies W_L = \frac{CU_m^2}2 - \frac{CU^2}2 = \frac C2(U_m^2 - U^2) = \\
    &= \frac {8\,\text{мкФ}}2 \cbr{\sqr{9\,\text{В}} - \sqr{1\,\text{В}}} \approx 320\,\text{мкДж}.
    \end{align*}
}
\solutionspace{100pt}

\tasknumber{3}%
\task{%
    Конденсатор ёмкостью $6\,\text{мкФ}$ зарядили до напряжения $5\,\text{В}$ и подключили к катушке индуктивностью $20\,\text{мГн}$.
    Определите максимальную энергию электрического поля конденсатора и период колебаний в $LC$-контуре.
}
\answer{%
    \begin{align*}
    W &= \frac{CU^2}2 = \frac{6\,\text{мкФ} \cdot \sqr{5\,\text{В}}}2 \approx 75\,\text{мкДж}, \\
    T &= 2\pi\sqrt{LC}= 2\pi\sqrt{20\,\text{мГн} \cdot 6\,\text{мкФ}} \approx 2{,}18\,\text{мс.}
    \end{align*}
}
\solutionspace{100pt}

\tasknumber{4}%
\task{%
    Во сколько раз (и как) изменится период свободных незатухающих колебаний в контуре,
    если его индуктивность увеличить в четыре раза, а ёмкость увеличить в шесть раз раз?
}
\answer{%
    $\text{увеличится в $4{,}90$ раз}$
}
\solutionspace{100pt}

\tasknumber{5}%
\task{%
    В схеме (см.
    рис.
    на доске) при разомкнутом ключе $K$ конденсатор ёмкостью $C = 10\,\text{мкФ}$ заряжен до напряжения $U_0 = 10\,\text{В}$.
    ЭДС батареи $\ele = 15\,\text{В}$, индуктивность катушки $L = 0{,}10\,\text{В}$.
    Определите
    \begin{itemize}
        \item чему равен установившийся ток в цепи после замыкания ключа?
        \item чему равен максимальный ток в цепи после замыкания ключа?
    \end{itemize}
    Внутренним сопротивлением батареи и омическим сопротивлением катушки пренебречь, $D$ — идеальный диод.
}

\variantsplitter

\addpersonalvariant{Артём Жичин}

\tasknumber{1}%
\task{%
    Электрический колебательный контур состоит
    из катушки индуктивностью $L$ и конденсатора ёмкостью $C$.
    Параллельно катушке подключают ещё одну катушку индуктивностью $\frac13L$.
    Как изменится частота свободных колебаний в контуре?
}
\answer{%
    $
        T = 2\pi\sqrt{LC}, \quad
        T' = 2\pi\sqrt{L'C'}
            = T \sqrt{\frac{L'}L \cdot \frac{C'}C}
            = T \sqrt{ \frac14 \cdot 1 }
        \implies \frac{\nu'}{\nu} = \frac{T}{T'} = \frac1{\sqrt{ \frac14 \cdot 1 }} \approx 2{,}000.
    $
}
\solutionspace{100pt}

\tasknumber{2}%
\task{%
    В $LC$-контуре ёмкость конденсатора $6\,\text{мкФ}$, а максимальное напряжение на нём $7\,\text{В}$.
    Определите энергию магнитного поля катушки в момент времени, когда напряжение на конденсаторе оказалось равным $1\,\text{В}$.
}
\answer{%
    \begin{align*}
    W &= \frac{LI^2}2 + \frac{CU^2}2 = \frac{CU_m^2}2 \implies W_L = \frac{CU_m^2}2 - \frac{CU^2}2 = \frac C2(U_m^2 - U^2) = \\
    &= \frac {6\,\text{мкФ}}2 \cbr{\sqr{7\,\text{В}} - \sqr{1\,\text{В}}} \approx 144\,\text{мкДж}.
    \end{align*}
}
\solutionspace{100pt}

\tasknumber{3}%
\task{%
    Конденсатор ёмкостью $4\,\text{мкФ}$ зарядили до напряжения $10\,\text{В}$ и подключили к катушке индуктивностью $30\,\text{мГн}$.
    Определите максимальную энергию магнитного поля катушки и период колебаний в $LC$-контуре.
}
\answer{%
    \begin{align*}
    W &= \frac{CU^2}2 = \frac{4\,\text{мкФ} \cdot \sqr{10\,\text{В}}}2 \approx 200\,\text{мкДж}, \\
    T &= 2\pi\sqrt{LC}= 2\pi\sqrt{30\,\text{мГн} \cdot 4\,\text{мкФ}} \approx 2{,}18\,\text{мс.}
    \end{align*}
}
\solutionspace{100pt}

\tasknumber{4}%
\task{%
    Во сколько раз (и как) изменится частота свободных незатухающих колебаний в контуре,
    если его индуктивность увеличить в три раза, а ёмкость уменьшить в три раза раз?
}
\answer{%
    $\text{не изменится}$
}
\solutionspace{100pt}

\tasknumber{5}%
\task{%
    В схеме (см.
    рис.
    на доске) при разомкнутом ключе $K$ конденсатор ёмкостью $C = 10\,\text{мкФ}$ заряжен до напряжения $U_0 = 10\,\text{В}$.
    ЭДС батареи $\ele = 15\,\text{В}$, индуктивность катушки $L = 0{,}10\,\text{В}$.
    Определите
    \begin{itemize}
        \item чему равен установившийся ток в цепи после замыкания ключа?
        \item чему равен максимальный ток в цепи после замыкания ключа?
    \end{itemize}
    Внутренним сопротивлением батареи и омическим сопротивлением катушки пренебречь, $D$ — идеальный диод.
}

\variantsplitter

\addpersonalvariant{Дарья Кошман}

\tasknumber{1}%
\task{%
    Электрический колебательный контур состоит
    из катушки индуктивностью $L$ и конденсатора ёмкостью $C$.
    Последовательно катушке подключают ещё одну катушку индуктивностью $\frac13L$.
    Как изменится частота свободных колебаний в контуре?
}
\answer{%
    $
        T = 2\pi\sqrt{LC}, \quad
        T' = 2\pi\sqrt{L'C'}
            = T \sqrt{\frac{L'}L \cdot \frac{C'}C}
            = T \sqrt{ \frac43 \cdot 1 }
        \implies \frac{\nu'}{\nu} = \frac{T}{T'} = \frac1{\sqrt{ \frac43 \cdot 1 }} \approx 0{,}866.
    $
}
\solutionspace{100pt}

\tasknumber{2}%
\task{%
    В $LC$-контуре ёмкость конденсатора $2\,\text{мкФ}$, а максимальное напряжение на нём $7\,\text{В}$.
    Определите энергию магнитного поля катушки в момент времени, когда напряжение на конденсаторе оказалось равным $3\,\text{В}$.
}
\answer{%
    \begin{align*}
    W &= \frac{LI^2}2 + \frac{CU^2}2 = \frac{CU_m^2}2 \implies W_L = \frac{CU_m^2}2 - \frac{CU^2}2 = \frac C2(U_m^2 - U^2) = \\
    &= \frac {2\,\text{мкФ}}2 \cbr{\sqr{7\,\text{В}} - \sqr{3\,\text{В}}} \approx 40\,\text{мкДж}.
    \end{align*}
}
\solutionspace{100pt}

\tasknumber{3}%
\task{%
    Конденсатор ёмкостью $6\,\text{мкФ}$ зарядили до напряжения $8\,\text{В}$ и подключили к катушке индуктивностью $30\,\text{мГн}$.
    Определите максимальную энергию магнитного поля катушки и период колебаний в $LC$-контуре.
}
\answer{%
    \begin{align*}
    W &= \frac{CU^2}2 = \frac{6\,\text{мкФ} \cdot \sqr{8\,\text{В}}}2 \approx 192\,\text{мкДж}, \\
    T &= 2\pi\sqrt{LC}= 2\pi\sqrt{30\,\text{мГн} \cdot 6\,\text{мкФ}} \approx 2{,}67\,\text{мс.}
    \end{align*}
}
\solutionspace{100pt}

\tasknumber{4}%
\task{%
    Во сколько раз (и как) изменится частота свободных незатухающих колебаний в контуре,
    если его индуктивность увеличить в пять раз, а ёмкость уменьшить в пять раз раз?
}
\answer{%
    $\text{не изменится}$
}
\solutionspace{100pt}

\tasknumber{5}%
\task{%
    В схеме (см.
    рис.
    на доске) при разомкнутом ключе $K$ конденсатор ёмкостью $C = 10\,\text{мкФ}$ заряжен до напряжения $U_0 = 10\,\text{В}$.
    ЭДС батареи $\ele = 15\,\text{В}$, индуктивность катушки $L = 0{,}10\,\text{В}$.
    Определите
    \begin{itemize}
        \item чему равен установившийся ток в цепи после замыкания ключа?
        \item чему равен максимальный ток в цепи после замыкания ключа?
    \end{itemize}
    Внутренним сопротивлением батареи и омическим сопротивлением катушки пренебречь, $D$ — идеальный диод.
}

\variantsplitter

\addpersonalvariant{Анна Кузьмичёва}

\tasknumber{1}%
\task{%
    Электрический колебательный контур состоит
    из катушки индуктивностью $L$ и конденсатора ёмкостью $C$.
    Последовательно конденсатору подключают ещё один конденсатор ёмкостью $\frac13C$.
    Как изменится частота свободных колебаний в контуре?
}
\answer{%
    $
        T = 2\pi\sqrt{LC}, \quad
        T' = 2\pi\sqrt{L'C'}
            = T \sqrt{\frac{L'}L \cdot \frac{C'}C}
            = T \sqrt{ 1 \cdot \frac14 }
        \implies \frac{\nu'}{\nu} = \frac{T}{T'} = \frac1{\sqrt{ 1 \cdot \frac14 }} \approx 2{,}000.
    $
}
\solutionspace{100pt}

\tasknumber{2}%
\task{%
    В $LC$-контуре ёмкость конденсатора $4\,\text{мкФ}$, а максимальное напряжение на нём $9\,\text{В}$.
    Определите энергию магнитного поля катушки в момент времени, когда напряжение на конденсаторе оказалось равным $3\,\text{В}$.
}
\answer{%
    \begin{align*}
    W &= \frac{LI^2}2 + \frac{CU^2}2 = \frac{CU_m^2}2 \implies W_L = \frac{CU_m^2}2 - \frac{CU^2}2 = \frac C2(U_m^2 - U^2) = \\
    &= \frac {4\,\text{мкФ}}2 \cbr{\sqr{9\,\text{В}} - \sqr{3\,\text{В}}} \approx 144\,\text{мкДж}.
    \end{align*}
}
\solutionspace{100pt}

\tasknumber{3}%
\task{%
    Конденсатор ёмкостью $8\,\text{мкФ}$ зарядили до напряжения $8\,\text{В}$ и подключили к катушке индуктивностью $40\,\text{мГн}$.
    Определите максимальную энергию электрического поля конденсатора и период колебаний в $LC$-контуре.
}
\answer{%
    \begin{align*}
    W &= \frac{CU^2}2 = \frac{8\,\text{мкФ} \cdot \sqr{8\,\text{В}}}2 \approx 256\,\text{мкДж}, \\
    T &= 2\pi\sqrt{LC}= 2\pi\sqrt{40\,\text{мГн} \cdot 8\,\text{мкФ}} \approx 3{,}55\,\text{мс.}
    \end{align*}
}
\solutionspace{100pt}

\tasknumber{4}%
\task{%
    Во сколько раз (и как) изменится частота свободных незатухающих колебаний в контуре,
    если его индуктивность увеличить в пять раз, а ёмкость увеличить в семь раз раз?
}
\answer{%
    $\text{уменьшится в $5{,}92$ раз}$
}
\solutionspace{100pt}

\tasknumber{5}%
\task{%
    В схеме (см.
    рис.
    на доске) при разомкнутом ключе $K$ конденсатор ёмкостью $C = 10\,\text{мкФ}$ заряжен до напряжения $U_0 = 10\,\text{В}$.
    ЭДС батареи $\ele = 15\,\text{В}$, индуктивность катушки $L = 0{,}10\,\text{В}$.
    Определите
    \begin{itemize}
        \item чему равен установившийся ток в цепи после замыкания ключа?
        \item чему равен максимальный ток в цепи после замыкания ключа?
    \end{itemize}
    Внутренним сопротивлением батареи и омическим сопротивлением катушки пренебречь, $D$ — идеальный диод.
}

\variantsplitter

\addpersonalvariant{Алёна Куприянова}

\tasknumber{1}%
\task{%
    Электрический колебательный контур состоит
    из катушки индуктивностью $L$ и конденсатора ёмкостью $C$.
    Параллельно конденсатору подключают ещё один конденсатор ёмкостью $\frac13C$.
    Как изменится частота свободных колебаний в контуре?
}
\answer{%
    $
        T = 2\pi\sqrt{LC}, \quad
        T' = 2\pi\sqrt{L'C'}
            = T \sqrt{\frac{L'}L \cdot \frac{C'}C}
            = T \sqrt{ 1 \cdot \frac43 }
        \implies \frac{\nu'}{\nu} = \frac{T}{T'} = \frac1{\sqrt{ 1 \cdot \frac43 }} \approx 0{,}866.
    $
}
\solutionspace{100pt}

\tasknumber{2}%
\task{%
    В $LC$-контуре ёмкость конденсатора $6\,\text{мкФ}$, а максимальное напряжение на нём $9\,\text{В}$.
    Определите энергию магнитного поля катушки в момент времени, когда напряжение на конденсаторе оказалось равным $1\,\text{В}$.
}
\answer{%
    \begin{align*}
    W &= \frac{LI^2}2 + \frac{CU^2}2 = \frac{CU_m^2}2 \implies W_L = \frac{CU_m^2}2 - \frac{CU^2}2 = \frac C2(U_m^2 - U^2) = \\
    &= \frac {6\,\text{мкФ}}2 \cbr{\sqr{9\,\text{В}} - \sqr{1\,\text{В}}} \approx 240\,\text{мкДж}.
    \end{align*}
}
\solutionspace{100pt}

\tasknumber{3}%
\task{%
    Конденсатор ёмкостью $2\,\text{мкФ}$ зарядили до напряжения $8\,\text{В}$ и подключили к катушке индуктивностью $25\,\text{мГн}$.
    Определите максимальную энергию магнитного поля катушки и период колебаний в $LC$-контуре.
}
\answer{%
    \begin{align*}
    W &= \frac{CU^2}2 = \frac{2\,\text{мкФ} \cdot \sqr{8\,\text{В}}}2 \approx 64\,\text{мкДж}, \\
    T &= 2\pi\sqrt{LC}= 2\pi\sqrt{25\,\text{мГн} \cdot 2\,\text{мкФ}} \approx 1{,}40\,\text{мс.}
    \end{align*}
}
\solutionspace{100pt}

\tasknumber{4}%
\task{%
    Во сколько раз (и как) изменится частота свободных незатухающих колебаний в контуре,
    если его индуктивность увеличить в пять раз, а ёмкость увеличить в шесть раз раз?
}
\answer{%
    $\text{уменьшится в $5{,}48$ раз}$
}
\solutionspace{100pt}

\tasknumber{5}%
\task{%
    В схеме (см.
    рис.
    на доске) при разомкнутом ключе $K$ конденсатор ёмкостью $C = 10\,\text{мкФ}$ заряжен до напряжения $U_0 = 10\,\text{В}$.
    ЭДС батареи $\ele = 15\,\text{В}$, индуктивность катушки $L = 0{,}10\,\text{В}$.
    Определите
    \begin{itemize}
        \item чему равен установившийся ток в цепи после замыкания ключа?
        \item чему равен максимальный ток в цепи после замыкания ключа?
    \end{itemize}
    Внутренним сопротивлением батареи и омическим сопротивлением катушки пренебречь, $D$ — идеальный диод.
}

\variantsplitter

\addpersonalvariant{Ярослав Лавровский}

\tasknumber{1}%
\task{%
    Электрический колебательный контур состоит
    из катушки индуктивностью $L$ и конденсатора ёмкостью $C$.
    Последовательно катушке подключают ещё одну катушку индуктивностью $\frac12L$.
    Как изменится частота свободных колебаний в контуре?
}
\answer{%
    $
        T = 2\pi\sqrt{LC}, \quad
        T' = 2\pi\sqrt{L'C'}
            = T \sqrt{\frac{L'}L \cdot \frac{C'}C}
            = T \sqrt{ \frac32 \cdot 1 }
        \implies \frac{\nu'}{\nu} = \frac{T}{T'} = \frac1{\sqrt{ \frac32 \cdot 1 }} \approx 0{,}816.
    $
}
\solutionspace{100pt}

\tasknumber{2}%
\task{%
    В $LC$-контуре ёмкость конденсатора $8\,\text{мкФ}$, а максимальное напряжение на нём $12\,\text{В}$.
    Определите энергию магнитного поля катушки в момент времени, когда напряжение на конденсаторе оказалось равным $5\,\text{В}$.
}
\answer{%
    \begin{align*}
    W &= \frac{LI^2}2 + \frac{CU^2}2 = \frac{CU_m^2}2 \implies W_L = \frac{CU_m^2}2 - \frac{CU^2}2 = \frac C2(U_m^2 - U^2) = \\
    &= \frac {8\,\text{мкФ}}2 \cbr{\sqr{12\,\text{В}} - \sqr{5\,\text{В}}} \approx 476\,\text{мкДж}.
    \end{align*}
}
\solutionspace{100pt}

\tasknumber{3}%
\task{%
    Конденсатор ёмкостью $2\,\text{мкФ}$ зарядили до напряжения $10\,\text{В}$ и подключили к катушке индуктивностью $25\,\text{мГн}$.
    Определите максимальную энергию электрического поля конденсатора и период колебаний в $LC$-контуре.
}
\answer{%
    \begin{align*}
    W &= \frac{CU^2}2 = \frac{2\,\text{мкФ} \cdot \sqr{10\,\text{В}}}2 \approx 100\,\text{мкДж}, \\
    T &= 2\pi\sqrt{LC}= 2\pi\sqrt{25\,\text{мГн} \cdot 2\,\text{мкФ}} \approx 1{,}40\,\text{мс.}
    \end{align*}
}
\solutionspace{100pt}

\tasknumber{4}%
\task{%
    Во сколько раз (и как) изменится период свободных незатухающих колебаний в контуре,
    если его индуктивность уменьшить в пять раз, а ёмкость уменьшить в пять раз раз?
}
\answer{%
    $\text{уменьшится в $5{,}00$ раз}$
}
\solutionspace{100pt}

\tasknumber{5}%
\task{%
    В схеме (см.
    рис.
    на доске) при разомкнутом ключе $K$ конденсатор ёмкостью $C = 10\,\text{мкФ}$ заряжен до напряжения $U_0 = 10\,\text{В}$.
    ЭДС батареи $\ele = 15\,\text{В}$, индуктивность катушки $L = 0{,}10\,\text{В}$.
    Определите
    \begin{itemize}
        \item чему равен установившийся ток в цепи после замыкания ключа?
        \item чему равен максимальный ток в цепи после замыкания ключа?
    \end{itemize}
    Внутренним сопротивлением батареи и омическим сопротивлением катушки пренебречь, $D$ — идеальный диод.
}

\variantsplitter

\addpersonalvariant{Анастасия Ламанова}

\tasknumber{1}%
\task{%
    Электрический колебательный контур состоит
    из катушки индуктивностью $L$ и конденсатора ёмкостью $C$.
    Параллельно конденсатору подключают ещё один конденсатор ёмкостью $\frac13C$.
    Как изменится частота свободных колебаний в контуре?
}
\answer{%
    $
        T = 2\pi\sqrt{LC}, \quad
        T' = 2\pi\sqrt{L'C'}
            = T \sqrt{\frac{L'}L \cdot \frac{C'}C}
            = T \sqrt{ 1 \cdot \frac43 }
        \implies \frac{\nu'}{\nu} = \frac{T}{T'} = \frac1{\sqrt{ 1 \cdot \frac43 }} \approx 0{,}866.
    $
}
\solutionspace{100pt}

\tasknumber{2}%
\task{%
    В $LC$-контуре ёмкость конденсатора $4\,\text{мкФ}$, а максимальное напряжение на нём $7\,\text{В}$.
    Определите энергию магнитного поля катушки в момент времени, когда напряжение на конденсаторе оказалось равным $5\,\text{В}$.
}
\answer{%
    \begin{align*}
    W &= \frac{LI^2}2 + \frac{CU^2}2 = \frac{CU_m^2}2 \implies W_L = \frac{CU_m^2}2 - \frac{CU^2}2 = \frac C2(U_m^2 - U^2) = \\
    &= \frac {4\,\text{мкФ}}2 \cbr{\sqr{7\,\text{В}} - \sqr{5\,\text{В}}} \approx 48\,\text{мкДж}.
    \end{align*}
}
\solutionspace{100pt}

\tasknumber{3}%
\task{%
    Конденсатор ёмкостью $8\,\text{мкФ}$ зарядили до напряжения $8\,\text{В}$ и подключили к катушке индуктивностью $30\,\text{мГн}$.
    Определите максимальную энергию магнитного поля катушки и период колебаний в $LC$-контуре.
}
\answer{%
    \begin{align*}
    W &= \frac{CU^2}2 = \frac{8\,\text{мкФ} \cdot \sqr{8\,\text{В}}}2 \approx 256\,\text{мкДж}, \\
    T &= 2\pi\sqrt{LC}= 2\pi\sqrt{30\,\text{мГн} \cdot 8\,\text{мкФ}} \approx 3{,}08\,\text{мс.}
    \end{align*}
}
\solutionspace{100pt}

\tasknumber{4}%
\task{%
    Во сколько раз (и как) изменится частота свободных незатухающих колебаний в контуре,
    если его индуктивность увеличить в шесть раз, а ёмкость увеличить в четыре раза раз?
}
\answer{%
    $\text{уменьшится в $4{,}90$ раз}$
}
\solutionspace{100pt}

\tasknumber{5}%
\task{%
    В схеме (см.
    рис.
    на доске) при разомкнутом ключе $K$ конденсатор ёмкостью $C = 10\,\text{мкФ}$ заряжен до напряжения $U_0 = 10\,\text{В}$.
    ЭДС батареи $\ele = 15\,\text{В}$, индуктивность катушки $L = 0{,}10\,\text{В}$.
    Определите
    \begin{itemize}
        \item чему равен установившийся ток в цепи после замыкания ключа?
        \item чему равен максимальный ток в цепи после замыкания ключа?
    \end{itemize}
    Внутренним сопротивлением батареи и омическим сопротивлением катушки пренебречь, $D$ — идеальный диод.
}

\variantsplitter

\addpersonalvariant{Виктория Легонькова}

\tasknumber{1}%
\task{%
    Электрический колебательный контур состоит
    из катушки индуктивностью $L$ и конденсатора ёмкостью $C$.
    Параллельно конденсатору подключают ещё один конденсатор ёмкостью $3C$.
    Как изменится частота свободных колебаний в контуре?
}
\answer{%
    $
        T = 2\pi\sqrt{LC}, \quad
        T' = 2\pi\sqrt{L'C'}
            = T \sqrt{\frac{L'}L \cdot \frac{C'}C}
            = T \sqrt{ 1 \cdot 4 }
        \implies \frac{\nu'}{\nu} = \frac{T}{T'} = \frac1{\sqrt{ 1 \cdot 4 }} \approx 0{,}500.
    $
}
\solutionspace{100pt}

\tasknumber{2}%
\task{%
    В $LC$-контуре ёмкость конденсатора $4\,\text{мкФ}$, а максимальное напряжение на нём $7\,\text{В}$.
    Определите энергию магнитного поля катушки в момент времени, когда напряжение на конденсаторе оказалось равным $5\,\text{В}$.
}
\answer{%
    \begin{align*}
    W &= \frac{LI^2}2 + \frac{CU^2}2 = \frac{CU_m^2}2 \implies W_L = \frac{CU_m^2}2 - \frac{CU^2}2 = \frac C2(U_m^2 - U^2) = \\
    &= \frac {4\,\text{мкФ}}2 \cbr{\sqr{7\,\text{В}} - \sqr{5\,\text{В}}} \approx 48\,\text{мкДж}.
    \end{align*}
}
\solutionspace{100pt}

\tasknumber{3}%
\task{%
    Конденсатор ёмкостью $8\,\text{мкФ}$ зарядили до напряжения $8\,\text{В}$ и подключили к катушке индуктивностью $30\,\text{мГн}$.
    Определите максимальную энергию электрического поля конденсатора и период колебаний в $LC$-контуре.
}
\answer{%
    \begin{align*}
    W &= \frac{CU^2}2 = \frac{8\,\text{мкФ} \cdot \sqr{8\,\text{В}}}2 \approx 256\,\text{мкДж}, \\
    T &= 2\pi\sqrt{LC}= 2\pi\sqrt{30\,\text{мГн} \cdot 8\,\text{мкФ}} \approx 3{,}08\,\text{мс.}
    \end{align*}
}
\solutionspace{100pt}

\tasknumber{4}%
\task{%
    Во сколько раз (и как) изменится частота свободных незатухающих колебаний в контуре,
    если его индуктивность увеличить в три раза, а ёмкость уменьшить в пять раз раз?
}
\answer{%
    $\text{увеличится в $1{,}29$ раз}$
}
\solutionspace{100pt}

\tasknumber{5}%
\task{%
    В схеме (см.
    рис.
    на доске) при разомкнутом ключе $K$ конденсатор ёмкостью $C = 10\,\text{мкФ}$ заряжен до напряжения $U_0 = 10\,\text{В}$.
    ЭДС батареи $\ele = 15\,\text{В}$, индуктивность катушки $L = 0{,}10\,\text{В}$.
    Определите
    \begin{itemize}
        \item чему равен установившийся ток в цепи после замыкания ключа?
        \item чему равен максимальный ток в цепи после замыкания ключа?
    \end{itemize}
    Внутренним сопротивлением батареи и омическим сопротивлением катушки пренебречь, $D$ — идеальный диод.
}

\variantsplitter

\addpersonalvariant{Семён Мартынов}

\tasknumber{1}%
\task{%
    Электрический колебательный контур состоит
    из катушки индуктивностью $L$ и конденсатора ёмкостью $C$.
    Параллельно катушке подключают ещё одну катушку индуктивностью $3L$.
    Как изменится частота свободных колебаний в контуре?
}
\answer{%
    $
        T = 2\pi\sqrt{LC}, \quad
        T' = 2\pi\sqrt{L'C'}
            = T \sqrt{\frac{L'}L \cdot \frac{C'}C}
            = T \sqrt{ \frac34 \cdot 1 }
        \implies \frac{\nu'}{\nu} = \frac{T}{T'} = \frac1{\sqrt{ \frac34 \cdot 1 }} \approx 1{,}155.
    $
}
\solutionspace{100pt}

\tasknumber{2}%
\task{%
    В $LC$-контуре ёмкость конденсатора $2\,\text{мкФ}$, а максимальное напряжение на нём $12\,\text{В}$.
    Определите энергию магнитного поля катушки в момент времени, когда напряжение на конденсаторе оказалось равным $3\,\text{В}$.
}
\answer{%
    \begin{align*}
    W &= \frac{LI^2}2 + \frac{CU^2}2 = \frac{CU_m^2}2 \implies W_L = \frac{CU_m^2}2 - \frac{CU^2}2 = \frac C2(U_m^2 - U^2) = \\
    &= \frac {2\,\text{мкФ}}2 \cbr{\sqr{12\,\text{В}} - \sqr{3\,\text{В}}} \approx 135\,\text{мкДж}.
    \end{align*}
}
\solutionspace{100pt}

\tasknumber{3}%
\task{%
    Конденсатор ёмкостью $8\,\text{мкФ}$ зарядили до напряжения $10\,\text{В}$ и подключили к катушке индуктивностью $30\,\text{мГн}$.
    Определите максимальную энергию электрического поля конденсатора и период колебаний в $LC$-контуре.
}
\answer{%
    \begin{align*}
    W &= \frac{CU^2}2 = \frac{8\,\text{мкФ} \cdot \sqr{10\,\text{В}}}2 \approx 400\,\text{мкДж}, \\
    T &= 2\pi\sqrt{LC}= 2\pi\sqrt{30\,\text{мГн} \cdot 8\,\text{мкФ}} \approx 3{,}08\,\text{мс.}
    \end{align*}
}
\solutionspace{100pt}

\tasknumber{4}%
\task{%
    Во сколько раз (и как) изменится циклическая частота свободных незатухающих колебаний в контуре,
    если его индуктивность увеличить в три раза, а ёмкость увеличить в четыре раза раз?
}
\answer{%
    $\text{уменьшится в $3{,}46$ раз}$
}
\solutionspace{100pt}

\tasknumber{5}%
\task{%
    В схеме (см.
    рис.
    на доске) при разомкнутом ключе $K$ конденсатор ёмкостью $C = 10\,\text{мкФ}$ заряжен до напряжения $U_0 = 10\,\text{В}$.
    ЭДС батареи $\ele = 15\,\text{В}$, индуктивность катушки $L = 0{,}10\,\text{В}$.
    Определите
    \begin{itemize}
        \item чему равен установившийся ток в цепи после замыкания ключа?
        \item чему равен максимальный ток в цепи после замыкания ключа?
    \end{itemize}
    Внутренним сопротивлением батареи и омическим сопротивлением катушки пренебречь, $D$ — идеальный диод.
}

\variantsplitter

\addpersonalvariant{Варвара Минаева}

\tasknumber{1}%
\task{%
    Электрический колебательный контур состоит
    из катушки индуктивностью $L$ и конденсатора ёмкостью $C$.
    Последовательно катушке подключают ещё одну катушку индуктивностью $3L$.
    Как изменится частота свободных колебаний в контуре?
}
\answer{%
    $
        T = 2\pi\sqrt{LC}, \quad
        T' = 2\pi\sqrt{L'C'}
            = T \sqrt{\frac{L'}L \cdot \frac{C'}C}
            = T \sqrt{ 4 \cdot 1 }
        \implies \frac{\nu'}{\nu} = \frac{T}{T'} = \frac1{\sqrt{ 4 \cdot 1 }} \approx 0{,}500.
    $
}
\solutionspace{100pt}

\tasknumber{2}%
\task{%
    В $LC$-контуре ёмкость конденсатора $8\,\text{мкФ}$, а максимальное напряжение на нём $12\,\text{В}$.
    Определите энергию магнитного поля катушки в момент времени, когда напряжение на конденсаторе оказалось равным $5\,\text{В}$.
}
\answer{%
    \begin{align*}
    W &= \frac{LI^2}2 + \frac{CU^2}2 = \frac{CU_m^2}2 \implies W_L = \frac{CU_m^2}2 - \frac{CU^2}2 = \frac C2(U_m^2 - U^2) = \\
    &= \frac {8\,\text{мкФ}}2 \cbr{\sqr{12\,\text{В}} - \sqr{5\,\text{В}}} \approx 476\,\text{мкДж}.
    \end{align*}
}
\solutionspace{100pt}

\tasknumber{3}%
\task{%
    Конденсатор ёмкостью $4\,\text{мкФ}$ зарядили до напряжения $5\,\text{В}$ и подключили к катушке индуктивностью $25\,\text{мГн}$.
    Определите максимальную энергию электрического поля конденсатора и период колебаний в $LC$-контуре.
}
\answer{%
    \begin{align*}
    W &= \frac{CU^2}2 = \frac{4\,\text{мкФ} \cdot \sqr{5\,\text{В}}}2 \approx 50\,\text{мкДж}, \\
    T &= 2\pi\sqrt{LC}= 2\pi\sqrt{25\,\text{мГн} \cdot 4\,\text{мкФ}} \approx 1{,}99\,\text{мс.}
    \end{align*}
}
\solutionspace{100pt}

\tasknumber{4}%
\task{%
    Во сколько раз (и как) изменится период свободных незатухающих колебаний в контуре,
    если его индуктивность уменьшить в два раза, а ёмкость уменьшить в семь раз раз?
}
\answer{%
    $\text{уменьшится в $3{,}74$ раз}$
}
\solutionspace{100pt}

\tasknumber{5}%
\task{%
    В схеме (см.
    рис.
    на доске) при разомкнутом ключе $K$ конденсатор ёмкостью $C = 10\,\text{мкФ}$ заряжен до напряжения $U_0 = 10\,\text{В}$.
    ЭДС батареи $\ele = 15\,\text{В}$, индуктивность катушки $L = 0{,}10\,\text{В}$.
    Определите
    \begin{itemize}
        \item чему равен установившийся ток в цепи после замыкания ключа?
        \item чему равен максимальный ток в цепи после замыкания ключа?
    \end{itemize}
    Внутренним сопротивлением батареи и омическим сопротивлением катушки пренебречь, $D$ — идеальный диод.
}

\variantsplitter

\addpersonalvariant{Леонид Никитин}

\tasknumber{1}%
\task{%
    Электрический колебательный контур состоит
    из катушки индуктивностью $L$ и конденсатора ёмкостью $C$.
    Последовательно катушке подключают ещё одну катушку индуктивностью $2L$.
    Как изменится частота свободных колебаний в контуре?
}
\answer{%
    $
        T = 2\pi\sqrt{LC}, \quad
        T' = 2\pi\sqrt{L'C'}
            = T \sqrt{\frac{L'}L \cdot \frac{C'}C}
            = T \sqrt{ 3 \cdot 1 }
        \implies \frac{\nu'}{\nu} = \frac{T}{T'} = \frac1{\sqrt{ 3 \cdot 1 }} \approx 0{,}577.
    $
}
\solutionspace{100pt}

\tasknumber{2}%
\task{%
    В $LC$-контуре ёмкость конденсатора $2\,\text{мкФ}$, а максимальное напряжение на нём $7\,\text{В}$.
    Определите энергию магнитного поля катушки в момент времени, когда напряжение на конденсаторе оказалось равным $5\,\text{В}$.
}
\answer{%
    \begin{align*}
    W &= \frac{LI^2}2 + \frac{CU^2}2 = \frac{CU_m^2}2 \implies W_L = \frac{CU_m^2}2 - \frac{CU^2}2 = \frac C2(U_m^2 - U^2) = \\
    &= \frac {2\,\text{мкФ}}2 \cbr{\sqr{7\,\text{В}} - \sqr{5\,\text{В}}} \approx 24\,\text{мкДж}.
    \end{align*}
}
\solutionspace{100pt}

\tasknumber{3}%
\task{%
    Конденсатор ёмкостью $2\,\text{мкФ}$ зарядили до напряжения $10\,\text{В}$ и подключили к катушке индуктивностью $25\,\text{мГн}$.
    Определите максимальную энергию магнитного поля катушки и период колебаний в $LC$-контуре.
}
\answer{%
    \begin{align*}
    W &= \frac{CU^2}2 = \frac{2\,\text{мкФ} \cdot \sqr{10\,\text{В}}}2 \approx 100\,\text{мкДж}, \\
    T &= 2\pi\sqrt{LC}= 2\pi\sqrt{25\,\text{мГн} \cdot 2\,\text{мкФ}} \approx 1{,}40\,\text{мс.}
    \end{align*}
}
\solutionspace{100pt}

\tasknumber{4}%
\task{%
    Во сколько раз (и как) изменится частота свободных незатухающих колебаний в контуре,
    если его индуктивность уменьшить в два раза, а ёмкость уменьшить в четыре раза раз?
}
\answer{%
    $\text{увеличится в $2{,}83$ раз}$
}
\solutionspace{100pt}

\tasknumber{5}%
\task{%
    В схеме (см.
    рис.
    на доске) при разомкнутом ключе $K$ конденсатор ёмкостью $C = 10\,\text{мкФ}$ заряжен до напряжения $U_0 = 10\,\text{В}$.
    ЭДС батареи $\ele = 15\,\text{В}$, индуктивность катушки $L = 0{,}10\,\text{В}$.
    Определите
    \begin{itemize}
        \item чему равен установившийся ток в цепи после замыкания ключа?
        \item чему равен максимальный ток в цепи после замыкания ключа?
    \end{itemize}
    Внутренним сопротивлением батареи и омическим сопротивлением катушки пренебречь, $D$ — идеальный диод.
}

\variantsplitter

\addpersonalvariant{Тимофей Полетаев}

\tasknumber{1}%
\task{%
    Электрический колебательный контур состоит
    из катушки индуктивностью $L$ и конденсатора ёмкостью $C$.
    Параллельно катушке подключают ещё одну катушку индуктивностью $\frac12L$.
    Как изменится частота свободных колебаний в контуре?
}
\answer{%
    $
        T = 2\pi\sqrt{LC}, \quad
        T' = 2\pi\sqrt{L'C'}
            = T \sqrt{\frac{L'}L \cdot \frac{C'}C}
            = T \sqrt{ \frac13 \cdot 1 }
        \implies \frac{\nu'}{\nu} = \frac{T}{T'} = \frac1{\sqrt{ \frac13 \cdot 1 }} \approx 1{,}732.
    $
}
\solutionspace{100pt}

\tasknumber{2}%
\task{%
    В $LC$-контуре ёмкость конденсатора $2\,\text{мкФ}$, а максимальное напряжение на нём $12\,\text{В}$.
    Определите энергию магнитного поля катушки в момент времени, когда напряжение на конденсаторе оказалось равным $1\,\text{В}$.
}
\answer{%
    \begin{align*}
    W &= \frac{LI^2}2 + \frac{CU^2}2 = \frac{CU_m^2}2 \implies W_L = \frac{CU_m^2}2 - \frac{CU^2}2 = \frac C2(U_m^2 - U^2) = \\
    &= \frac {2\,\text{мкФ}}2 \cbr{\sqr{12\,\text{В}} - \sqr{1\,\text{В}}} \approx 143\,\text{мкДж}.
    \end{align*}
}
\solutionspace{100pt}

\tasknumber{3}%
\task{%
    Конденсатор ёмкостью $6\,\text{мкФ}$ зарядили до напряжения $8\,\text{В}$ и подключили к катушке индуктивностью $30\,\text{мГн}$.
    Определите максимальную энергию электрического поля конденсатора и период колебаний в $LC$-контуре.
}
\answer{%
    \begin{align*}
    W &= \frac{CU^2}2 = \frac{6\,\text{мкФ} \cdot \sqr{8\,\text{В}}}2 \approx 192\,\text{мкДж}, \\
    T &= 2\pi\sqrt{LC}= 2\pi\sqrt{30\,\text{мГн} \cdot 6\,\text{мкФ}} \approx 2{,}67\,\text{мс.}
    \end{align*}
}
\solutionspace{100pt}

\tasknumber{4}%
\task{%
    Во сколько раз (и как) изменится частота свободных незатухающих колебаний в контуре,
    если его индуктивность уменьшить в шесть раз, а ёмкость уменьшить в три раза раз?
}
\answer{%
    $\text{увеличится в $4{,}24$ раз}$
}
\solutionspace{100pt}

\tasknumber{5}%
\task{%
    В схеме (см.
    рис.
    на доске) при разомкнутом ключе $K$ конденсатор ёмкостью $C = 10\,\text{мкФ}$ заряжен до напряжения $U_0 = 10\,\text{В}$.
    ЭДС батареи $\ele = 15\,\text{В}$, индуктивность катушки $L = 0{,}10\,\text{В}$.
    Определите
    \begin{itemize}
        \item чему равен установившийся ток в цепи после замыкания ключа?
        \item чему равен максимальный ток в цепи после замыкания ключа?
    \end{itemize}
    Внутренним сопротивлением батареи и омическим сопротивлением катушки пренебречь, $D$ — идеальный диод.
}

\variantsplitter

\addpersonalvariant{Андрей Рожков}

\tasknumber{1}%
\task{%
    Электрический колебательный контур состоит
    из катушки индуктивностью $L$ и конденсатора ёмкостью $C$.
    Параллельно катушке подключают ещё одну катушку индуктивностью $\frac12L$.
    Как изменится частота свободных колебаний в контуре?
}
\answer{%
    $
        T = 2\pi\sqrt{LC}, \quad
        T' = 2\pi\sqrt{L'C'}
            = T \sqrt{\frac{L'}L \cdot \frac{C'}C}
            = T \sqrt{ \frac13 \cdot 1 }
        \implies \frac{\nu'}{\nu} = \frac{T}{T'} = \frac1{\sqrt{ \frac13 \cdot 1 }} \approx 1{,}732.
    $
}
\solutionspace{100pt}

\tasknumber{2}%
\task{%
    В $LC$-контуре ёмкость конденсатора $6\,\text{мкФ}$, а максимальное напряжение на нём $9\,\text{В}$.
    Определите энергию магнитного поля катушки в момент времени, когда напряжение на конденсаторе оказалось равным $3\,\text{В}$.
}
\answer{%
    \begin{align*}
    W &= \frac{LI^2}2 + \frac{CU^2}2 = \frac{CU_m^2}2 \implies W_L = \frac{CU_m^2}2 - \frac{CU^2}2 = \frac C2(U_m^2 - U^2) = \\
    &= \frac {6\,\text{мкФ}}2 \cbr{\sqr{9\,\text{В}} - \sqr{3\,\text{В}}} \approx 216\,\text{мкДж}.
    \end{align*}
}
\solutionspace{100pt}

\tasknumber{3}%
\task{%
    Конденсатор ёмкостью $2\,\text{мкФ}$ зарядили до напряжения $10\,\text{В}$ и подключили к катушке индуктивностью $40\,\text{мГн}$.
    Определите максимальную энергию магнитного поля катушки и период колебаний в $LC$-контуре.
}
\answer{%
    \begin{align*}
    W &= \frac{CU^2}2 = \frac{2\,\text{мкФ} \cdot \sqr{10\,\text{В}}}2 \approx 100\,\text{мкДж}, \\
    T &= 2\pi\sqrt{LC}= 2\pi\sqrt{40\,\text{мГн} \cdot 2\,\text{мкФ}} \approx 1{,}78\,\text{мс.}
    \end{align*}
}
\solutionspace{100pt}

\tasknumber{4}%
\task{%
    Во сколько раз (и как) изменится циклическая частота свободных незатухающих колебаний в контуре,
    если его индуктивность увеличить в три раза, а ёмкость уменьшить в пять раз раз?
}
\answer{%
    $\text{увеличится в $1{,}29$ раз}$
}
\solutionspace{100pt}

\tasknumber{5}%
\task{%
    В схеме (см.
    рис.
    на доске) при разомкнутом ключе $K$ конденсатор ёмкостью $C = 10\,\text{мкФ}$ заряжен до напряжения $U_0 = 10\,\text{В}$.
    ЭДС батареи $\ele = 15\,\text{В}$, индуктивность катушки $L = 0{,}10\,\text{В}$.
    Определите
    \begin{itemize}
        \item чему равен установившийся ток в цепи после замыкания ключа?
        \item чему равен максимальный ток в цепи после замыкания ключа?
    \end{itemize}
    Внутренним сопротивлением батареи и омическим сопротивлением катушки пренебречь, $D$ — идеальный диод.
}

\variantsplitter

\addpersonalvariant{Рената Таржиманова}

\tasknumber{1}%
\task{%
    Электрический колебательный контур состоит
    из катушки индуктивностью $L$ и конденсатора ёмкостью $C$.
    Последовательно катушке подключают ещё одну катушку индуктивностью $\frac13L$.
    Как изменится частота свободных колебаний в контуре?
}
\answer{%
    $
        T = 2\pi\sqrt{LC}, \quad
        T' = 2\pi\sqrt{L'C'}
            = T \sqrt{\frac{L'}L \cdot \frac{C'}C}
            = T \sqrt{ \frac43 \cdot 1 }
        \implies \frac{\nu'}{\nu} = \frac{T}{T'} = \frac1{\sqrt{ \frac43 \cdot 1 }} \approx 0{,}866.
    $
}
\solutionspace{100pt}

\tasknumber{2}%
\task{%
    В $LC$-контуре ёмкость конденсатора $4\,\text{мкФ}$, а максимальное напряжение на нём $9\,\text{В}$.
    Определите энергию магнитного поля катушки в момент времени, когда напряжение на конденсаторе оказалось равным $5\,\text{В}$.
}
\answer{%
    \begin{align*}
    W &= \frac{LI^2}2 + \frac{CU^2}2 = \frac{CU_m^2}2 \implies W_L = \frac{CU_m^2}2 - \frac{CU^2}2 = \frac C2(U_m^2 - U^2) = \\
    &= \frac {4\,\text{мкФ}}2 \cbr{\sqr{9\,\text{В}} - \sqr{5\,\text{В}}} \approx 112\,\text{мкДж}.
    \end{align*}
}
\solutionspace{100pt}

\tasknumber{3}%
\task{%
    Конденсатор ёмкостью $2\,\text{мкФ}$ зарядили до напряжения $8\,\text{В}$ и подключили к катушке индуктивностью $30\,\text{мГн}$.
    Определите максимальную энергию магнитного поля катушки и период колебаний в $LC$-контуре.
}
\answer{%
    \begin{align*}
    W &= \frac{CU^2}2 = \frac{2\,\text{мкФ} \cdot \sqr{8\,\text{В}}}2 \approx 64\,\text{мкДж}, \\
    T &= 2\pi\sqrt{LC}= 2\pi\sqrt{30\,\text{мГн} \cdot 2\,\text{мкФ}} \approx 1{,}54\,\text{мс.}
    \end{align*}
}
\solutionspace{100pt}

\tasknumber{4}%
\task{%
    Во сколько раз (и как) изменится циклическая частота свободных незатухающих колебаний в контуре,
    если его индуктивность увеличить в два раза, а ёмкость уменьшить в шесть раз раз?
}
\answer{%
    $\text{увеличится в $1{,}73$ раз}$
}
\solutionspace{100pt}

\tasknumber{5}%
\task{%
    В схеме (см.
    рис.
    на доске) при разомкнутом ключе $K$ конденсатор ёмкостью $C = 10\,\text{мкФ}$ заряжен до напряжения $U_0 = 10\,\text{В}$.
    ЭДС батареи $\ele = 15\,\text{В}$, индуктивность катушки $L = 0{,}10\,\text{В}$.
    Определите
    \begin{itemize}
        \item чему равен установившийся ток в цепи после замыкания ключа?
        \item чему равен максимальный ток в цепи после замыкания ключа?
    \end{itemize}
    Внутренним сопротивлением батареи и омическим сопротивлением катушки пренебречь, $D$ — идеальный диод.
}

\variantsplitter

\addpersonalvariant{Андрей Щербаков}

\tasknumber{1}%
\task{%
    Электрический колебательный контур состоит
    из катушки индуктивностью $L$ и конденсатора ёмкостью $C$.
    Последовательно конденсатору подключают ещё один конденсатор ёмкостью $\frac12C$.
    Как изменится частота свободных колебаний в контуре?
}
\answer{%
    $
        T = 2\pi\sqrt{LC}, \quad
        T' = 2\pi\sqrt{L'C'}
            = T \sqrt{\frac{L'}L \cdot \frac{C'}C}
            = T \sqrt{ 1 \cdot \frac13 }
        \implies \frac{\nu'}{\nu} = \frac{T}{T'} = \frac1{\sqrt{ 1 \cdot \frac13 }} \approx 1{,}732.
    $
}
\solutionspace{100pt}

\tasknumber{2}%
\task{%
    В $LC$-контуре ёмкость конденсатора $2\,\text{мкФ}$, а максимальное напряжение на нём $9\,\text{В}$.
    Определите энергию магнитного поля катушки в момент времени, когда напряжение на конденсаторе оказалось равным $3\,\text{В}$.
}
\answer{%
    \begin{align*}
    W &= \frac{LI^2}2 + \frac{CU^2}2 = \frac{CU_m^2}2 \implies W_L = \frac{CU_m^2}2 - \frac{CU^2}2 = \frac C2(U_m^2 - U^2) = \\
    &= \frac {2\,\text{мкФ}}2 \cbr{\sqr{9\,\text{В}} - \sqr{3\,\text{В}}} \approx 72\,\text{мкДж}.
    \end{align*}
}
\solutionspace{100pt}

\tasknumber{3}%
\task{%
    Конденсатор ёмкостью $8\,\text{мкФ}$ зарядили до напряжения $10\,\text{В}$ и подключили к катушке индуктивностью $40\,\text{мГн}$.
    Определите максимальную энергию магнитного поля катушки и период колебаний в $LC$-контуре.
}
\answer{%
    \begin{align*}
    W &= \frac{CU^2}2 = \frac{8\,\text{мкФ} \cdot \sqr{10\,\text{В}}}2 \approx 400\,\text{мкДж}, \\
    T &= 2\pi\sqrt{LC}= 2\pi\sqrt{40\,\text{мГн} \cdot 8\,\text{мкФ}} \approx 3{,}55\,\text{мс.}
    \end{align*}
}
\solutionspace{100pt}

\tasknumber{4}%
\task{%
    Во сколько раз (и как) изменится частота свободных незатухающих колебаний в контуре,
    если его индуктивность уменьшить в два раза, а ёмкость увеличить в пять раз раз?
}
\answer{%
    $\text{уменьшится в $1{,}58$ раз}$
}
\solutionspace{100pt}

\tasknumber{5}%
\task{%
    В схеме (см.
    рис.
    на доске) при разомкнутом ключе $K$ конденсатор ёмкостью $C = 10\,\text{мкФ}$ заряжен до напряжения $U_0 = 10\,\text{В}$.
    ЭДС батареи $\ele = 15\,\text{В}$, индуктивность катушки $L = 0{,}10\,\text{В}$.
    Определите
    \begin{itemize}
        \item чему равен установившийся ток в цепи после замыкания ключа?
        \item чему равен максимальный ток в цепи после замыкания ключа?
    \end{itemize}
    Внутренним сопротивлением батареи и омическим сопротивлением катушки пренебречь, $D$ — идеальный диод.
}

\variantsplitter

\addpersonalvariant{Михаил Ярошевский}

\tasknumber{1}%
\task{%
    Электрический колебательный контур состоит
    из катушки индуктивностью $L$ и конденсатора ёмкостью $C$.
    Последовательно конденсатору подключают ещё один конденсатор ёмкостью $\frac12C$.
    Как изменится частота свободных колебаний в контуре?
}
\answer{%
    $
        T = 2\pi\sqrt{LC}, \quad
        T' = 2\pi\sqrt{L'C'}
            = T \sqrt{\frac{L'}L \cdot \frac{C'}C}
            = T \sqrt{ 1 \cdot \frac13 }
        \implies \frac{\nu'}{\nu} = \frac{T}{T'} = \frac1{\sqrt{ 1 \cdot \frac13 }} \approx 1{,}732.
    $
}
\solutionspace{100pt}

\tasknumber{2}%
\task{%
    В $LC$-контуре ёмкость конденсатора $6\,\text{мкФ}$, а максимальное напряжение на нём $12\,\text{В}$.
    Определите энергию магнитного поля катушки в момент времени, когда напряжение на конденсаторе оказалось равным $3\,\text{В}$.
}
\answer{%
    \begin{align*}
    W &= \frac{LI^2}2 + \frac{CU^2}2 = \frac{CU_m^2}2 \implies W_L = \frac{CU_m^2}2 - \frac{CU^2}2 = \frac C2(U_m^2 - U^2) = \\
    &= \frac {6\,\text{мкФ}}2 \cbr{\sqr{12\,\text{В}} - \sqr{3\,\text{В}}} \approx 405\,\text{мкДж}.
    \end{align*}
}
\solutionspace{100pt}

\tasknumber{3}%
\task{%
    Конденсатор ёмкостью $6\,\text{мкФ}$ зарядили до напряжения $10\,\text{В}$ и подключили к катушке индуктивностью $20\,\text{мГн}$.
    Определите максимальную энергию электрического поля конденсатора и период колебаний в $LC$-контуре.
}
\answer{%
    \begin{align*}
    W &= \frac{CU^2}2 = \frac{6\,\text{мкФ} \cdot \sqr{10\,\text{В}}}2 \approx 300\,\text{мкДж}, \\
    T &= 2\pi\sqrt{LC}= 2\pi\sqrt{20\,\text{мГн} \cdot 6\,\text{мкФ}} \approx 2{,}18\,\text{мс.}
    \end{align*}
}
\solutionspace{100pt}

\tasknumber{4}%
\task{%
    Во сколько раз (и как) изменится циклическая частота свободных незатухающих колебаний в контуре,
    если его индуктивность уменьшить в два раза, а ёмкость увеличить в семь раз раз?
}
\answer{%
    $\text{уменьшится в $1{,}87$ раз}$
}
\solutionspace{100pt}

\tasknumber{5}%
\task{%
    В схеме (см.
    рис.
    на доске) при разомкнутом ключе $K$ конденсатор ёмкостью $C = 10\,\text{мкФ}$ заряжен до напряжения $U_0 = 10\,\text{В}$.
    ЭДС батареи $\ele = 15\,\text{В}$, индуктивность катушки $L = 0{,}10\,\text{В}$.
    Определите
    \begin{itemize}
        \item чему равен установившийся ток в цепи после замыкания ключа?
        \item чему равен максимальный ток в цепи после замыкания ключа?
    \end{itemize}
    Внутренним сопротивлением батареи и омическим сопротивлением катушки пренебречь, $D$ — идеальный диод.
}

\variantsplitter

\addpersonalvariant{Алексей Алимпиев}

\tasknumber{1}%
\task{%
    Электрический колебательный контур состоит
    из катушки индуктивностью $L$ и конденсатора ёмкостью $C$.
    Последовательно конденсатору подключают ещё один конденсатор ёмкостью $\frac12C$.
    Как изменится частота свободных колебаний в контуре?
}
\answer{%
    $
        T = 2\pi\sqrt{LC}, \quad
        T' = 2\pi\sqrt{L'C'}
            = T \sqrt{\frac{L'}L \cdot \frac{C'}C}
            = T \sqrt{ 1 \cdot \frac13 }
        \implies \frac{\nu'}{\nu} = \frac{T}{T'} = \frac1{\sqrt{ 1 \cdot \frac13 }} \approx 1{,}732.
    $
}
\solutionspace{100pt}

\tasknumber{2}%
\task{%
    В $LC$-контуре ёмкость конденсатора $6\,\text{мкФ}$, а максимальное напряжение на нём $7\,\text{В}$.
    Определите энергию магнитного поля катушки в момент времени, когда напряжение на конденсаторе оказалось равным $3\,\text{В}$.
}
\answer{%
    \begin{align*}
    W &= \frac{LI^2}2 + \frac{CU^2}2 = \frac{CU_m^2}2 \implies W_L = \frac{CU_m^2}2 - \frac{CU^2}2 = \frac C2(U_m^2 - U^2) = \\
    &= \frac {6\,\text{мкФ}}2 \cbr{\sqr{7\,\text{В}} - \sqr{3\,\text{В}}} \approx 120\,\text{мкДж}.
    \end{align*}
}
\solutionspace{100pt}

\tasknumber{3}%
\task{%
    Конденсатор ёмкостью $2\,\text{мкФ}$ зарядили до напряжения $8\,\text{В}$ и подключили к катушке индуктивностью $40\,\text{мГн}$.
    Определите максимальную энергию магнитного поля катушки и период колебаний в $LC$-контуре.
}
\answer{%
    \begin{align*}
    W &= \frac{CU^2}2 = \frac{2\,\text{мкФ} \cdot \sqr{8\,\text{В}}}2 \approx 64\,\text{мкДж}, \\
    T &= 2\pi\sqrt{LC}= 2\pi\sqrt{40\,\text{мГн} \cdot 2\,\text{мкФ}} \approx 1{,}78\,\text{мс.}
    \end{align*}
}
\solutionspace{100pt}

\tasknumber{4}%
\task{%
    Во сколько раз (и как) изменится циклическая частота свободных незатухающих колебаний в контуре,
    если его индуктивность уменьшить в пять раз, а ёмкость увеличить в четыре раза раз?
}
\answer{%
    $\text{увеличится в $1{,}12$ раз}$
}
\solutionspace{100pt}

\tasknumber{5}%
\task{%
    В схеме (см.
    рис.
    на доске) при разомкнутом ключе $K$ конденсатор ёмкостью $C = 10\,\text{мкФ}$ заряжен до напряжения $U_0 = 10\,\text{В}$.
    ЭДС батареи $\ele = 15\,\text{В}$, индуктивность катушки $L = 0{,}10\,\text{В}$.
    Определите
    \begin{itemize}
        \item чему равен установившийся ток в цепи после замыкания ключа?
        \item чему равен максимальный ток в цепи после замыкания ключа?
    \end{itemize}
    Внутренним сопротивлением батареи и омическим сопротивлением катушки пренебречь, $D$ — идеальный диод.
}

\variantsplitter

\addpersonalvariant{Евгений Васин}

\tasknumber{1}%
\task{%
    Электрический колебательный контур состоит
    из катушки индуктивностью $L$ и конденсатора ёмкостью $C$.
    Последовательно конденсатору подключают ещё один конденсатор ёмкостью $\frac13C$.
    Как изменится частота свободных колебаний в контуре?
}
\answer{%
    $
        T = 2\pi\sqrt{LC}, \quad
        T' = 2\pi\sqrt{L'C'}
            = T \sqrt{\frac{L'}L \cdot \frac{C'}C}
            = T \sqrt{ 1 \cdot \frac14 }
        \implies \frac{\nu'}{\nu} = \frac{T}{T'} = \frac1{\sqrt{ 1 \cdot \frac14 }} \approx 2{,}000.
    $
}
\solutionspace{100pt}

\tasknumber{2}%
\task{%
    В $LC$-контуре ёмкость конденсатора $8\,\text{мкФ}$, а максимальное напряжение на нём $12\,\text{В}$.
    Определите энергию магнитного поля катушки в момент времени, когда напряжение на конденсаторе оказалось равным $3\,\text{В}$.
}
\answer{%
    \begin{align*}
    W &= \frac{LI^2}2 + \frac{CU^2}2 = \frac{CU_m^2}2 \implies W_L = \frac{CU_m^2}2 - \frac{CU^2}2 = \frac C2(U_m^2 - U^2) = \\
    &= \frac {8\,\text{мкФ}}2 \cbr{\sqr{12\,\text{В}} - \sqr{3\,\text{В}}} \approx 540\,\text{мкДж}.
    \end{align*}
}
\solutionspace{100pt}

\tasknumber{3}%
\task{%
    Конденсатор ёмкостью $2\,\text{мкФ}$ зарядили до напряжения $5\,\text{В}$ и подключили к катушке индуктивностью $30\,\text{мГн}$.
    Определите максимальную энергию электрического поля конденсатора и период колебаний в $LC$-контуре.
}
\answer{%
    \begin{align*}
    W &= \frac{CU^2}2 = \frac{2\,\text{мкФ} \cdot \sqr{5\,\text{В}}}2 \approx 25\,\text{мкДж}, \\
    T &= 2\pi\sqrt{LC}= 2\pi\sqrt{30\,\text{мГн} \cdot 2\,\text{мкФ}} \approx 1{,}54\,\text{мс.}
    \end{align*}
}
\solutionspace{100pt}

\tasknumber{4}%
\task{%
    Во сколько раз (и как) изменится период свободных незатухающих колебаний в контуре,
    если его индуктивность уменьшить в шесть раз, а ёмкость увеличить в три раза раз?
}
\answer{%
    $\text{уменьшится в $1{,}41$ раз}$
}
\solutionspace{100pt}

\tasknumber{5}%
\task{%
    В схеме (см.
    рис.
    на доске) при разомкнутом ключе $K$ конденсатор ёмкостью $C = 10\,\text{мкФ}$ заряжен до напряжения $U_0 = 10\,\text{В}$.
    ЭДС батареи $\ele = 15\,\text{В}$, индуктивность катушки $L = 0{,}10\,\text{В}$.
    Определите
    \begin{itemize}
        \item чему равен установившийся ток в цепи после замыкания ключа?
        \item чему равен максимальный ток в цепи после замыкания ключа?
    \end{itemize}
    Внутренним сопротивлением батареи и омическим сопротивлением катушки пренебречь, $D$ — идеальный диод.
}

\variantsplitter

\addpersonalvariant{Вячеслав Волохов}

\tasknumber{1}%
\task{%
    Электрический колебательный контур состоит
    из катушки индуктивностью $L$ и конденсатора ёмкостью $C$.
    Последовательно конденсатору подключают ещё один конденсатор ёмкостью $3C$.
    Как изменится частота свободных колебаний в контуре?
}
\answer{%
    $
        T = 2\pi\sqrt{LC}, \quad
        T' = 2\pi\sqrt{L'C'}
            = T \sqrt{\frac{L'}L \cdot \frac{C'}C}
            = T \sqrt{ 1 \cdot \frac34 }
        \implies \frac{\nu'}{\nu} = \frac{T}{T'} = \frac1{\sqrt{ 1 \cdot \frac34 }} \approx 1{,}155.
    $
}
\solutionspace{100pt}

\tasknumber{2}%
\task{%
    В $LC$-контуре ёмкость конденсатора $8\,\text{мкФ}$, а максимальное напряжение на нём $7\,\text{В}$.
    Определите энергию магнитного поля катушки в момент времени, когда напряжение на конденсаторе оказалось равным $1\,\text{В}$.
}
\answer{%
    \begin{align*}
    W &= \frac{LI^2}2 + \frac{CU^2}2 = \frac{CU_m^2}2 \implies W_L = \frac{CU_m^2}2 - \frac{CU^2}2 = \frac C2(U_m^2 - U^2) = \\
    &= \frac {8\,\text{мкФ}}2 \cbr{\sqr{7\,\text{В}} - \sqr{1\,\text{В}}} \approx 192\,\text{мкДж}.
    \end{align*}
}
\solutionspace{100pt}

\tasknumber{3}%
\task{%
    Конденсатор ёмкостью $8\,\text{мкФ}$ зарядили до напряжения $8\,\text{В}$ и подключили к катушке индуктивностью $20\,\text{мГн}$.
    Определите максимальную энергию электрического поля конденсатора и период колебаний в $LC$-контуре.
}
\answer{%
    \begin{align*}
    W &= \frac{CU^2}2 = \frac{8\,\text{мкФ} \cdot \sqr{8\,\text{В}}}2 \approx 256\,\text{мкДж}, \\
    T &= 2\pi\sqrt{LC}= 2\pi\sqrt{20\,\text{мГн} \cdot 8\,\text{мкФ}} \approx 2{,}51\,\text{мс.}
    \end{align*}
}
\solutionspace{100pt}

\tasknumber{4}%
\task{%
    Во сколько раз (и как) изменится период свободных незатухающих колебаний в контуре,
    если его индуктивность увеличить в три раза, а ёмкость уменьшить в три раза раз?
}
\answer{%
    $\text{не изменится}$
}
\solutionspace{100pt}

\tasknumber{5}%
\task{%
    В схеме (см.
    рис.
    на доске) при разомкнутом ключе $K$ конденсатор ёмкостью $C = 10\,\text{мкФ}$ заряжен до напряжения $U_0 = 10\,\text{В}$.
    ЭДС батареи $\ele = 15\,\text{В}$, индуктивность катушки $L = 0{,}10\,\text{В}$.
    Определите
    \begin{itemize}
        \item чему равен установившийся ток в цепи после замыкания ключа?
        \item чему равен максимальный ток в цепи после замыкания ключа?
    \end{itemize}
    Внутренним сопротивлением батареи и омическим сопротивлением катушки пренебречь, $D$ — идеальный диод.
}

\variantsplitter

\addpersonalvariant{Герман Говоров}

\tasknumber{1}%
\task{%
    Электрический колебательный контур состоит
    из катушки индуктивностью $L$ и конденсатора ёмкостью $C$.
    Параллельно катушке подключают ещё одну катушку индуктивностью $\frac12L$.
    Как изменится частота свободных колебаний в контуре?
}
\answer{%
    $
        T = 2\pi\sqrt{LC}, \quad
        T' = 2\pi\sqrt{L'C'}
            = T \sqrt{\frac{L'}L \cdot \frac{C'}C}
            = T \sqrt{ \frac13 \cdot 1 }
        \implies \frac{\nu'}{\nu} = \frac{T}{T'} = \frac1{\sqrt{ \frac13 \cdot 1 }} \approx 1{,}732.
    $
}
\solutionspace{100pt}

\tasknumber{2}%
\task{%
    В $LC$-контуре ёмкость конденсатора $4\,\text{мкФ}$, а максимальное напряжение на нём $7\,\text{В}$.
    Определите энергию магнитного поля катушки в момент времени, когда напряжение на конденсаторе оказалось равным $3\,\text{В}$.
}
\answer{%
    \begin{align*}
    W &= \frac{LI^2}2 + \frac{CU^2}2 = \frac{CU_m^2}2 \implies W_L = \frac{CU_m^2}2 - \frac{CU^2}2 = \frac C2(U_m^2 - U^2) = \\
    &= \frac {4\,\text{мкФ}}2 \cbr{\sqr{7\,\text{В}} - \sqr{3\,\text{В}}} \approx 80\,\text{мкДж}.
    \end{align*}
}
\solutionspace{100pt}

\tasknumber{3}%
\task{%
    Конденсатор ёмкостью $6\,\text{мкФ}$ зарядили до напряжения $8\,\text{В}$ и подключили к катушке индуктивностью $25\,\text{мГн}$.
    Определите максимальную энергию электрического поля конденсатора и период колебаний в $LC$-контуре.
}
\answer{%
    \begin{align*}
    W &= \frac{CU^2}2 = \frac{6\,\text{мкФ} \cdot \sqr{8\,\text{В}}}2 \approx 192\,\text{мкДж}, \\
    T &= 2\pi\sqrt{LC}= 2\pi\sqrt{25\,\text{мГн} \cdot 6\,\text{мкФ}} \approx 2{,}43\,\text{мс.}
    \end{align*}
}
\solutionspace{100pt}

\tasknumber{4}%
\task{%
    Во сколько раз (и как) изменится циклическая частота свободных незатухающих колебаний в контуре,
    если его индуктивность уменьшить в два раза, а ёмкость уменьшить в шесть раз раз?
}
\answer{%
    $\text{увеличится в $3{,}46$ раз}$
}
\solutionspace{100pt}

\tasknumber{5}%
\task{%
    В схеме (см.
    рис.
    на доске) при разомкнутом ключе $K$ конденсатор ёмкостью $C = 10\,\text{мкФ}$ заряжен до напряжения $U_0 = 10\,\text{В}$.
    ЭДС батареи $\ele = 15\,\text{В}$, индуктивность катушки $L = 0{,}10\,\text{В}$.
    Определите
    \begin{itemize}
        \item чему равен установившийся ток в цепи после замыкания ключа?
        \item чему равен максимальный ток в цепи после замыкания ключа?
    \end{itemize}
    Внутренним сопротивлением батареи и омическим сопротивлением катушки пренебречь, $D$ — идеальный диод.
}

\variantsplitter

\addpersonalvariant{София Журавлёва}

\tasknumber{1}%
\task{%
    Электрический колебательный контур состоит
    из катушки индуктивностью $L$ и конденсатора ёмкостью $C$.
    Последовательно конденсатору подключают ещё один конденсатор ёмкостью $\frac13C$.
    Как изменится частота свободных колебаний в контуре?
}
\answer{%
    $
        T = 2\pi\sqrt{LC}, \quad
        T' = 2\pi\sqrt{L'C'}
            = T \sqrt{\frac{L'}L \cdot \frac{C'}C}
            = T \sqrt{ 1 \cdot \frac14 }
        \implies \frac{\nu'}{\nu} = \frac{T}{T'} = \frac1{\sqrt{ 1 \cdot \frac14 }} \approx 2{,}000.
    $
}
\solutionspace{100pt}

\tasknumber{2}%
\task{%
    В $LC$-контуре ёмкость конденсатора $2\,\text{мкФ}$, а максимальное напряжение на нём $9\,\text{В}$.
    Определите энергию магнитного поля катушки в момент времени, когда напряжение на конденсаторе оказалось равным $5\,\text{В}$.
}
\answer{%
    \begin{align*}
    W &= \frac{LI^2}2 + \frac{CU^2}2 = \frac{CU_m^2}2 \implies W_L = \frac{CU_m^2}2 - \frac{CU^2}2 = \frac C2(U_m^2 - U^2) = \\
    &= \frac {2\,\text{мкФ}}2 \cbr{\sqr{9\,\text{В}} - \sqr{5\,\text{В}}} \approx 56\,\text{мкДж}.
    \end{align*}
}
\solutionspace{100pt}

\tasknumber{3}%
\task{%
    Конденсатор ёмкостью $6\,\text{мкФ}$ зарядили до напряжения $8\,\text{В}$ и подключили к катушке индуктивностью $25\,\text{мГн}$.
    Определите максимальную энергию электрического поля конденсатора и период колебаний в $LC$-контуре.
}
\answer{%
    \begin{align*}
    W &= \frac{CU^2}2 = \frac{6\,\text{мкФ} \cdot \sqr{8\,\text{В}}}2 \approx 192\,\text{мкДж}, \\
    T &= 2\pi\sqrt{LC}= 2\pi\sqrt{25\,\text{мГн} \cdot 6\,\text{мкФ}} \approx 2{,}43\,\text{мс.}
    \end{align*}
}
\solutionspace{100pt}

\tasknumber{4}%
\task{%
    Во сколько раз (и как) изменится циклическая частота свободных незатухающих колебаний в контуре,
    если его индуктивность увеличить в два раза, а ёмкость увеличить в пять раз раз?
}
\answer{%
    $\text{уменьшится в $3{,}16$ раз}$
}
\solutionspace{100pt}

\tasknumber{5}%
\task{%
    В схеме (см.
    рис.
    на доске) при разомкнутом ключе $K$ конденсатор ёмкостью $C = 10\,\text{мкФ}$ заряжен до напряжения $U_0 = 10\,\text{В}$.
    ЭДС батареи $\ele = 15\,\text{В}$, индуктивность катушки $L = 0{,}10\,\text{В}$.
    Определите
    \begin{itemize}
        \item чему равен установившийся ток в цепи после замыкания ключа?
        \item чему равен максимальный ток в цепи после замыкания ключа?
    \end{itemize}
    Внутренним сопротивлением батареи и омическим сопротивлением катушки пренебречь, $D$ — идеальный диод.
}

\variantsplitter

\addpersonalvariant{Константин Козлов}

\tasknumber{1}%
\task{%
    Электрический колебательный контур состоит
    из катушки индуктивностью $L$ и конденсатора ёмкостью $C$.
    Параллельно катушке подключают ещё одну катушку индуктивностью $\frac12L$.
    Как изменится частота свободных колебаний в контуре?
}
\answer{%
    $
        T = 2\pi\sqrt{LC}, \quad
        T' = 2\pi\sqrt{L'C'}
            = T \sqrt{\frac{L'}L \cdot \frac{C'}C}
            = T \sqrt{ \frac13 \cdot 1 }
        \implies \frac{\nu'}{\nu} = \frac{T}{T'} = \frac1{\sqrt{ \frac13 \cdot 1 }} \approx 1{,}732.
    $
}
\solutionspace{100pt}

\tasknumber{2}%
\task{%
    В $LC$-контуре ёмкость конденсатора $8\,\text{мкФ}$, а максимальное напряжение на нём $7\,\text{В}$.
    Определите энергию магнитного поля катушки в момент времени, когда напряжение на конденсаторе оказалось равным $3\,\text{В}$.
}
\answer{%
    \begin{align*}
    W &= \frac{LI^2}2 + \frac{CU^2}2 = \frac{CU_m^2}2 \implies W_L = \frac{CU_m^2}2 - \frac{CU^2}2 = \frac C2(U_m^2 - U^2) = \\
    &= \frac {8\,\text{мкФ}}2 \cbr{\sqr{7\,\text{В}} - \sqr{3\,\text{В}}} \approx 160\,\text{мкДж}.
    \end{align*}
}
\solutionspace{100pt}

\tasknumber{3}%
\task{%
    Конденсатор ёмкостью $6\,\text{мкФ}$ зарядили до напряжения $8\,\text{В}$ и подключили к катушке индуктивностью $40\,\text{мГн}$.
    Определите максимальную энергию электрического поля конденсатора и период колебаний в $LC$-контуре.
}
\answer{%
    \begin{align*}
    W &= \frac{CU^2}2 = \frac{6\,\text{мкФ} \cdot \sqr{8\,\text{В}}}2 \approx 192\,\text{мкДж}, \\
    T &= 2\pi\sqrt{LC}= 2\pi\sqrt{40\,\text{мГн} \cdot 6\,\text{мкФ}} \approx 3{,}08\,\text{мс.}
    \end{align*}
}
\solutionspace{100pt}

\tasknumber{4}%
\task{%
    Во сколько раз (и как) изменится частота свободных незатухающих колебаний в контуре,
    если его индуктивность уменьшить в два раза, а ёмкость уменьшить в четыре раза раз?
}
\answer{%
    $\text{увеличится в $2{,}83$ раз}$
}
\solutionspace{100pt}

\tasknumber{5}%
\task{%
    В схеме (см.
    рис.
    на доске) при разомкнутом ключе $K$ конденсатор ёмкостью $C = 10\,\text{мкФ}$ заряжен до напряжения $U_0 = 10\,\text{В}$.
    ЭДС батареи $\ele = 15\,\text{В}$, индуктивность катушки $L = 0{,}10\,\text{В}$.
    Определите
    \begin{itemize}
        \item чему равен установившийся ток в цепи после замыкания ключа?
        \item чему равен максимальный ток в цепи после замыкания ключа?
    \end{itemize}
    Внутренним сопротивлением батареи и омическим сопротивлением катушки пренебречь, $D$ — идеальный диод.
}

\variantsplitter

\addpersonalvariant{Наталья Кравченко}

\tasknumber{1}%
\task{%
    Электрический колебательный контур состоит
    из катушки индуктивностью $L$ и конденсатора ёмкостью $C$.
    Параллельно катушке подключают ещё одну катушку индуктивностью $\frac12L$.
    Как изменится частота свободных колебаний в контуре?
}
\answer{%
    $
        T = 2\pi\sqrt{LC}, \quad
        T' = 2\pi\sqrt{L'C'}
            = T \sqrt{\frac{L'}L \cdot \frac{C'}C}
            = T \sqrt{ \frac13 \cdot 1 }
        \implies \frac{\nu'}{\nu} = \frac{T}{T'} = \frac1{\sqrt{ \frac13 \cdot 1 }} \approx 1{,}732.
    $
}
\solutionspace{100pt}

\tasknumber{2}%
\task{%
    В $LC$-контуре ёмкость конденсатора $8\,\text{мкФ}$, а максимальное напряжение на нём $9\,\text{В}$.
    Определите энергию магнитного поля катушки в момент времени, когда напряжение на конденсаторе оказалось равным $1\,\text{В}$.
}
\answer{%
    \begin{align*}
    W &= \frac{LI^2}2 + \frac{CU^2}2 = \frac{CU_m^2}2 \implies W_L = \frac{CU_m^2}2 - \frac{CU^2}2 = \frac C2(U_m^2 - U^2) = \\
    &= \frac {8\,\text{мкФ}}2 \cbr{\sqr{9\,\text{В}} - \sqr{1\,\text{В}}} \approx 320\,\text{мкДж}.
    \end{align*}
}
\solutionspace{100pt}

\tasknumber{3}%
\task{%
    Конденсатор ёмкостью $2\,\text{мкФ}$ зарядили до напряжения $10\,\text{В}$ и подключили к катушке индуктивностью $40\,\text{мГн}$.
    Определите максимальную энергию магнитного поля катушки и период колебаний в $LC$-контуре.
}
\answer{%
    \begin{align*}
    W &= \frac{CU^2}2 = \frac{2\,\text{мкФ} \cdot \sqr{10\,\text{В}}}2 \approx 100\,\text{мкДж}, \\
    T &= 2\pi\sqrt{LC}= 2\pi\sqrt{40\,\text{мГн} \cdot 2\,\text{мкФ}} \approx 1{,}78\,\text{мс.}
    \end{align*}
}
\solutionspace{100pt}

\tasknumber{4}%
\task{%
    Во сколько раз (и как) изменится циклическая частота свободных незатухающих колебаний в контуре,
    если его индуктивность увеличить в шесть раз, а ёмкость уменьшить в пять раз раз?
}
\answer{%
    $\text{уменьшится в $1{,}10$ раз}$
}
\solutionspace{100pt}

\tasknumber{5}%
\task{%
    В схеме (см.
    рис.
    на доске) при разомкнутом ключе $K$ конденсатор ёмкостью $C = 10\,\text{мкФ}$ заряжен до напряжения $U_0 = 10\,\text{В}$.
    ЭДС батареи $\ele = 15\,\text{В}$, индуктивность катушки $L = 0{,}10\,\text{В}$.
    Определите
    \begin{itemize}
        \item чему равен установившийся ток в цепи после замыкания ключа?
        \item чему равен максимальный ток в цепи после замыкания ключа?
    \end{itemize}
    Внутренним сопротивлением батареи и омическим сопротивлением катушки пренебречь, $D$ — идеальный диод.
}

\variantsplitter

\addpersonalvariant{Сергей Малышев}

\tasknumber{1}%
\task{%
    Электрический колебательный контур состоит
    из катушки индуктивностью $L$ и конденсатора ёмкостью $C$.
    Параллельно катушке подключают ещё одну катушку индуктивностью $\frac12L$.
    Как изменится частота свободных колебаний в контуре?
}
\answer{%
    $
        T = 2\pi\sqrt{LC}, \quad
        T' = 2\pi\sqrt{L'C'}
            = T \sqrt{\frac{L'}L \cdot \frac{C'}C}
            = T \sqrt{ \frac13 \cdot 1 }
        \implies \frac{\nu'}{\nu} = \frac{T}{T'} = \frac1{\sqrt{ \frac13 \cdot 1 }} \approx 1{,}732.
    $
}
\solutionspace{100pt}

\tasknumber{2}%
\task{%
    В $LC$-контуре ёмкость конденсатора $4\,\text{мкФ}$, а максимальное напряжение на нём $7\,\text{В}$.
    Определите энергию магнитного поля катушки в момент времени, когда напряжение на конденсаторе оказалось равным $1\,\text{В}$.
}
\answer{%
    \begin{align*}
    W &= \frac{LI^2}2 + \frac{CU^2}2 = \frac{CU_m^2}2 \implies W_L = \frac{CU_m^2}2 - \frac{CU^2}2 = \frac C2(U_m^2 - U^2) = \\
    &= \frac {4\,\text{мкФ}}2 \cbr{\sqr{7\,\text{В}} - \sqr{1\,\text{В}}} \approx 96\,\text{мкДж}.
    \end{align*}
}
\solutionspace{100pt}

\tasknumber{3}%
\task{%
    Конденсатор ёмкостью $8\,\text{мкФ}$ зарядили до напряжения $5\,\text{В}$ и подключили к катушке индуктивностью $25\,\text{мГн}$.
    Определите максимальную энергию электрического поля конденсатора и период колебаний в $LC$-контуре.
}
\answer{%
    \begin{align*}
    W &= \frac{CU^2}2 = \frac{8\,\text{мкФ} \cdot \sqr{5\,\text{В}}}2 \approx 100\,\text{мкДж}, \\
    T &= 2\pi\sqrt{LC}= 2\pi\sqrt{25\,\text{мГн} \cdot 8\,\text{мкФ}} \approx 2{,}81\,\text{мс.}
    \end{align*}
}
\solutionspace{100pt}

\tasknumber{4}%
\task{%
    Во сколько раз (и как) изменится период свободных незатухающих колебаний в контуре,
    если его индуктивность увеличить в три раза, а ёмкость увеличить в семь раз раз?
}
\answer{%
    $\text{увеличится в $4{,}58$ раз}$
}
\solutionspace{100pt}

\tasknumber{5}%
\task{%
    В схеме (см.
    рис.
    на доске) при разомкнутом ключе $K$ конденсатор ёмкостью $C = 10\,\text{мкФ}$ заряжен до напряжения $U_0 = 10\,\text{В}$.
    ЭДС батареи $\ele = 15\,\text{В}$, индуктивность катушки $L = 0{,}10\,\text{В}$.
    Определите
    \begin{itemize}
        \item чему равен установившийся ток в цепи после замыкания ключа?
        \item чему равен максимальный ток в цепи после замыкания ключа?
    \end{itemize}
    Внутренним сопротивлением батареи и омическим сопротивлением катушки пренебречь, $D$ — идеальный диод.
}

\variantsplitter

\addpersonalvariant{Алина Полканова}

\tasknumber{1}%
\task{%
    Электрический колебательный контур состоит
    из катушки индуктивностью $L$ и конденсатора ёмкостью $C$.
    Параллельно конденсатору подключают ещё один конденсатор ёмкостью $2C$.
    Как изменится частота свободных колебаний в контуре?
}
\answer{%
    $
        T = 2\pi\sqrt{LC}, \quad
        T' = 2\pi\sqrt{L'C'}
            = T \sqrt{\frac{L'}L \cdot \frac{C'}C}
            = T \sqrt{ 1 \cdot 3 }
        \implies \frac{\nu'}{\nu} = \frac{T}{T'} = \frac1{\sqrt{ 1 \cdot 3 }} \approx 0{,}577.
    $
}
\solutionspace{100pt}

\tasknumber{2}%
\task{%
    В $LC$-контуре ёмкость конденсатора $4\,\text{мкФ}$, а максимальное напряжение на нём $12\,\text{В}$.
    Определите энергию магнитного поля катушки в момент времени, когда напряжение на конденсаторе оказалось равным $1\,\text{В}$.
}
\answer{%
    \begin{align*}
    W &= \frac{LI^2}2 + \frac{CU^2}2 = \frac{CU_m^2}2 \implies W_L = \frac{CU_m^2}2 - \frac{CU^2}2 = \frac C2(U_m^2 - U^2) = \\
    &= \frac {4\,\text{мкФ}}2 \cbr{\sqr{12\,\text{В}} - \sqr{1\,\text{В}}} \approx 286\,\text{мкДж}.
    \end{align*}
}
\solutionspace{100pt}

\tasknumber{3}%
\task{%
    Конденсатор ёмкостью $4\,\text{мкФ}$ зарядили до напряжения $5\,\text{В}$ и подключили к катушке индуктивностью $20\,\text{мГн}$.
    Определите максимальную энергию магнитного поля катушки и период колебаний в $LC$-контуре.
}
\answer{%
    \begin{align*}
    W &= \frac{CU^2}2 = \frac{4\,\text{мкФ} \cdot \sqr{5\,\text{В}}}2 \approx 50\,\text{мкДж}, \\
    T &= 2\pi\sqrt{LC}= 2\pi\sqrt{20\,\text{мГн} \cdot 4\,\text{мкФ}} \approx 1{,}78\,\text{мс.}
    \end{align*}
}
\solutionspace{100pt}

\tasknumber{4}%
\task{%
    Во сколько раз (и как) изменится циклическая частота свободных незатухающих колебаний в контуре,
    если его индуктивность уменьшить в два раза, а ёмкость увеличить в три раза раз?
}
\answer{%
    $\text{уменьшится в $1{,}22$ раз}$
}
\solutionspace{100pt}

\tasknumber{5}%
\task{%
    В схеме (см.
    рис.
    на доске) при разомкнутом ключе $K$ конденсатор ёмкостью $C = 10\,\text{мкФ}$ заряжен до напряжения $U_0 = 10\,\text{В}$.
    ЭДС батареи $\ele = 15\,\text{В}$, индуктивность катушки $L = 0{,}10\,\text{В}$.
    Определите
    \begin{itemize}
        \item чему равен установившийся ток в цепи после замыкания ключа?
        \item чему равен максимальный ток в цепи после замыкания ключа?
    \end{itemize}
    Внутренним сопротивлением батареи и омическим сопротивлением катушки пренебречь, $D$ — идеальный диод.
}

\variantsplitter

\addpersonalvariant{Сергей Пономарёв}

\tasknumber{1}%
\task{%
    Электрический колебательный контур состоит
    из катушки индуктивностью $L$ и конденсатора ёмкостью $C$.
    Последовательно катушке подключают ещё одну катушку индуктивностью $3L$.
    Как изменится частота свободных колебаний в контуре?
}
\answer{%
    $
        T = 2\pi\sqrt{LC}, \quad
        T' = 2\pi\sqrt{L'C'}
            = T \sqrt{\frac{L'}L \cdot \frac{C'}C}
            = T \sqrt{ 4 \cdot 1 }
        \implies \frac{\nu'}{\nu} = \frac{T}{T'} = \frac1{\sqrt{ 4 \cdot 1 }} \approx 0{,}500.
    $
}
\solutionspace{100pt}

\tasknumber{2}%
\task{%
    В $LC$-контуре ёмкость конденсатора $2\,\text{мкФ}$, а максимальное напряжение на нём $12\,\text{В}$.
    Определите энергию магнитного поля катушки в момент времени, когда напряжение на конденсаторе оказалось равным $3\,\text{В}$.
}
\answer{%
    \begin{align*}
    W &= \frac{LI^2}2 + \frac{CU^2}2 = \frac{CU_m^2}2 \implies W_L = \frac{CU_m^2}2 - \frac{CU^2}2 = \frac C2(U_m^2 - U^2) = \\
    &= \frac {2\,\text{мкФ}}2 \cbr{\sqr{12\,\text{В}} - \sqr{3\,\text{В}}} \approx 135\,\text{мкДж}.
    \end{align*}
}
\solutionspace{100pt}

\tasknumber{3}%
\task{%
    Конденсатор ёмкостью $6\,\text{мкФ}$ зарядили до напряжения $5\,\text{В}$ и подключили к катушке индуктивностью $20\,\text{мГн}$.
    Определите максимальную энергию магнитного поля катушки и период колебаний в $LC$-контуре.
}
\answer{%
    \begin{align*}
    W &= \frac{CU^2}2 = \frac{6\,\text{мкФ} \cdot \sqr{5\,\text{В}}}2 \approx 75\,\text{мкДж}, \\
    T &= 2\pi\sqrt{LC}= 2\pi\sqrt{20\,\text{мГн} \cdot 6\,\text{мкФ}} \approx 2{,}18\,\text{мс.}
    \end{align*}
}
\solutionspace{100pt}

\tasknumber{4}%
\task{%
    Во сколько раз (и как) изменится период свободных незатухающих колебаний в контуре,
    если его индуктивность уменьшить в три раза, а ёмкость увеличить в пять раз раз?
}
\answer{%
    $\text{увеличится в $1{,}29$ раз}$
}
\solutionspace{100pt}

\tasknumber{5}%
\task{%
    В схеме (см.
    рис.
    на доске) при разомкнутом ключе $K$ конденсатор ёмкостью $C = 10\,\text{мкФ}$ заряжен до напряжения $U_0 = 10\,\text{В}$.
    ЭДС батареи $\ele = 15\,\text{В}$, индуктивность катушки $L = 0{,}10\,\text{В}$.
    Определите
    \begin{itemize}
        \item чему равен установившийся ток в цепи после замыкания ключа?
        \item чему равен максимальный ток в цепи после замыкания ключа?
    \end{itemize}
    Внутренним сопротивлением батареи и омическим сопротивлением катушки пренебречь, $D$ — идеальный диод.
}

\variantsplitter

\addpersonalvariant{Егор Свистушкин}

\tasknumber{1}%
\task{%
    Электрический колебательный контур состоит
    из катушки индуктивностью $L$ и конденсатора ёмкостью $C$.
    Последовательно конденсатору подключают ещё один конденсатор ёмкостью $3C$.
    Как изменится частота свободных колебаний в контуре?
}
\answer{%
    $
        T = 2\pi\sqrt{LC}, \quad
        T' = 2\pi\sqrt{L'C'}
            = T \sqrt{\frac{L'}L \cdot \frac{C'}C}
            = T \sqrt{ 1 \cdot \frac34 }
        \implies \frac{\nu'}{\nu} = \frac{T}{T'} = \frac1{\sqrt{ 1 \cdot \frac34 }} \approx 1{,}155.
    $
}
\solutionspace{100pt}

\tasknumber{2}%
\task{%
    В $LC$-контуре ёмкость конденсатора $8\,\text{мкФ}$, а максимальное напряжение на нём $9\,\text{В}$.
    Определите энергию магнитного поля катушки в момент времени, когда напряжение на конденсаторе оказалось равным $5\,\text{В}$.
}
\answer{%
    \begin{align*}
    W &= \frac{LI^2}2 + \frac{CU^2}2 = \frac{CU_m^2}2 \implies W_L = \frac{CU_m^2}2 - \frac{CU^2}2 = \frac C2(U_m^2 - U^2) = \\
    &= \frac {8\,\text{мкФ}}2 \cbr{\sqr{9\,\text{В}} - \sqr{5\,\text{В}}} \approx 224\,\text{мкДж}.
    \end{align*}
}
\solutionspace{100pt}

\tasknumber{3}%
\task{%
    Конденсатор ёмкостью $6\,\text{мкФ}$ зарядили до напряжения $5\,\text{В}$ и подключили к катушке индуктивностью $40\,\text{мГн}$.
    Определите максимальную энергию электрического поля конденсатора и период колебаний в $LC$-контуре.
}
\answer{%
    \begin{align*}
    W &= \frac{CU^2}2 = \frac{6\,\text{мкФ} \cdot \sqr{5\,\text{В}}}2 \approx 75\,\text{мкДж}, \\
    T &= 2\pi\sqrt{LC}= 2\pi\sqrt{40\,\text{мГн} \cdot 6\,\text{мкФ}} \approx 3{,}08\,\text{мс.}
    \end{align*}
}
\solutionspace{100pt}

\tasknumber{4}%
\task{%
    Во сколько раз (и как) изменится период свободных незатухающих колебаний в контуре,
    если его индуктивность уменьшить в два раза, а ёмкость увеличить в шесть раз раз?
}
\answer{%
    $\text{увеличится в $1{,}73$ раз}$
}
\solutionspace{100pt}

\tasknumber{5}%
\task{%
    В схеме (см.
    рис.
    на доске) при разомкнутом ключе $K$ конденсатор ёмкостью $C = 10\,\text{мкФ}$ заряжен до напряжения $U_0 = 10\,\text{В}$.
    ЭДС батареи $\ele = 15\,\text{В}$, индуктивность катушки $L = 0{,}10\,\text{В}$.
    Определите
    \begin{itemize}
        \item чему равен установившийся ток в цепи после замыкания ключа?
        \item чему равен максимальный ток в цепи после замыкания ключа?
    \end{itemize}
    Внутренним сопротивлением батареи и омическим сопротивлением катушки пренебречь, $D$ — идеальный диод.
}

\variantsplitter

\addpersonalvariant{Дмитрий Соколов}

\tasknumber{1}%
\task{%
    Электрический колебательный контур состоит
    из катушки индуктивностью $L$ и конденсатора ёмкостью $C$.
    Параллельно конденсатору подключают ещё один конденсатор ёмкостью $\frac12C$.
    Как изменится частота свободных колебаний в контуре?
}
\answer{%
    $
        T = 2\pi\sqrt{LC}, \quad
        T' = 2\pi\sqrt{L'C'}
            = T \sqrt{\frac{L'}L \cdot \frac{C'}C}
            = T \sqrt{ 1 \cdot \frac32 }
        \implies \frac{\nu'}{\nu} = \frac{T}{T'} = \frac1{\sqrt{ 1 \cdot \frac32 }} \approx 0{,}816.
    $
}
\solutionspace{100pt}

\tasknumber{2}%
\task{%
    В $LC$-контуре ёмкость конденсатора $4\,\text{мкФ}$, а максимальное напряжение на нём $9\,\text{В}$.
    Определите энергию магнитного поля катушки в момент времени, когда напряжение на конденсаторе оказалось равным $5\,\text{В}$.
}
\answer{%
    \begin{align*}
    W &= \frac{LI^2}2 + \frac{CU^2}2 = \frac{CU_m^2}2 \implies W_L = \frac{CU_m^2}2 - \frac{CU^2}2 = \frac C2(U_m^2 - U^2) = \\
    &= \frac {4\,\text{мкФ}}2 \cbr{\sqr{9\,\text{В}} - \sqr{5\,\text{В}}} \approx 112\,\text{мкДж}.
    \end{align*}
}
\solutionspace{100pt}

\tasknumber{3}%
\task{%
    Конденсатор ёмкостью $4\,\text{мкФ}$ зарядили до напряжения $5\,\text{В}$ и подключили к катушке индуктивностью $20\,\text{мГн}$.
    Определите максимальную энергию магнитного поля катушки и период колебаний в $LC$-контуре.
}
\answer{%
    \begin{align*}
    W &= \frac{CU^2}2 = \frac{4\,\text{мкФ} \cdot \sqr{5\,\text{В}}}2 \approx 50\,\text{мкДж}, \\
    T &= 2\pi\sqrt{LC}= 2\pi\sqrt{20\,\text{мГн} \cdot 4\,\text{мкФ}} \approx 1{,}78\,\text{мс.}
    \end{align*}
}
\solutionspace{100pt}

\tasknumber{4}%
\task{%
    Во сколько раз (и как) изменится период свободных незатухающих колебаний в контуре,
    если его индуктивность увеличить в четыре раза, а ёмкость увеличить в четыре раза раз?
}
\answer{%
    $\text{увеличится в $4{,}00$ раз}$
}
\solutionspace{100pt}

\tasknumber{5}%
\task{%
    В схеме (см.
    рис.
    на доске) при разомкнутом ключе $K$ конденсатор ёмкостью $C = 10\,\text{мкФ}$ заряжен до напряжения $U_0 = 10\,\text{В}$.
    ЭДС батареи $\ele = 15\,\text{В}$, индуктивность катушки $L = 0{,}10\,\text{В}$.
    Определите
    \begin{itemize}
        \item чему равен установившийся ток в цепи после замыкания ключа?
        \item чему равен максимальный ток в цепи после замыкания ключа?
    \end{itemize}
    Внутренним сопротивлением батареи и омическим сопротивлением катушки пренебречь, $D$ — идеальный диод.
}

\variantsplitter

\addpersonalvariant{Арсений Трофимов}

\tasknumber{1}%
\task{%
    Электрический колебательный контур состоит
    из катушки индуктивностью $L$ и конденсатора ёмкостью $C$.
    Последовательно катушке подключают ещё одну катушку индуктивностью $2L$.
    Как изменится частота свободных колебаний в контуре?
}
\answer{%
    $
        T = 2\pi\sqrt{LC}, \quad
        T' = 2\pi\sqrt{L'C'}
            = T \sqrt{\frac{L'}L \cdot \frac{C'}C}
            = T \sqrt{ 3 \cdot 1 }
        \implies \frac{\nu'}{\nu} = \frac{T}{T'} = \frac1{\sqrt{ 3 \cdot 1 }} \approx 0{,}577.
    $
}
\solutionspace{100pt}

\tasknumber{2}%
\task{%
    В $LC$-контуре ёмкость конденсатора $2\,\text{мкФ}$, а максимальное напряжение на нём $12\,\text{В}$.
    Определите энергию магнитного поля катушки в момент времени, когда напряжение на конденсаторе оказалось равным $5\,\text{В}$.
}
\answer{%
    \begin{align*}
    W &= \frac{LI^2}2 + \frac{CU^2}2 = \frac{CU_m^2}2 \implies W_L = \frac{CU_m^2}2 - \frac{CU^2}2 = \frac C2(U_m^2 - U^2) = \\
    &= \frac {2\,\text{мкФ}}2 \cbr{\sqr{12\,\text{В}} - \sqr{5\,\text{В}}} \approx 119\,\text{мкДж}.
    \end{align*}
}
\solutionspace{100pt}

\tasknumber{3}%
\task{%
    Конденсатор ёмкостью $8\,\text{мкФ}$ зарядили до напряжения $5\,\text{В}$ и подключили к катушке индуктивностью $25\,\text{мГн}$.
    Определите максимальную энергию магнитного поля катушки и период колебаний в $LC$-контуре.
}
\answer{%
    \begin{align*}
    W &= \frac{CU^2}2 = \frac{8\,\text{мкФ} \cdot \sqr{5\,\text{В}}}2 \approx 100\,\text{мкДж}, \\
    T &= 2\pi\sqrt{LC}= 2\pi\sqrt{25\,\text{мГн} \cdot 8\,\text{мкФ}} \approx 2{,}81\,\text{мс.}
    \end{align*}
}
\solutionspace{100pt}

\tasknumber{4}%
\task{%
    Во сколько раз (и как) изменится период свободных незатухающих колебаний в контуре,
    если его индуктивность уменьшить в два раза, а ёмкость увеличить в три раза раз?
}
\answer{%
    $\text{увеличится в $1{,}22$ раз}$
}
\solutionspace{100pt}

\tasknumber{5}%
\task{%
    В схеме (см.
    рис.
    на доске) при разомкнутом ключе $K$ конденсатор ёмкостью $C = 10\,\text{мкФ}$ заряжен до напряжения $U_0 = 10\,\text{В}$.
    ЭДС батареи $\ele = 15\,\text{В}$, индуктивность катушки $L = 0{,}10\,\text{В}$.
    Определите
    \begin{itemize}
        \item чему равен установившийся ток в цепи после замыкания ключа?
        \item чему равен максимальный ток в цепи после замыкания ключа?
    \end{itemize}
    Внутренним сопротивлением батареи и омическим сопротивлением катушки пренебречь, $D$ — идеальный диод.
}
% autogenerated
