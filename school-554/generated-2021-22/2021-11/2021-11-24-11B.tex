\setdate{24~ноября~2021}
\setclass{11«Б»}

\addpersonalvariant{Михаил Бурмистров}

\tasknumber{1}%
\task{%
    Длина волны света в~вакууме $\lambda = 700\,\text{нм}$.
    Какова частота этой световой волны?
    Какова длина этой волны в среде с показателем преломления $n = 1{,}5$?
    Может ли человек увидеть такую волну света, и если да, то какой именно цвет соответствует этим волнам в вакууме и в этой среде?
}
\answer{%
    \begin{align*}
    \nu &= \frac 1T = \frac 1{\lambda/c} = \frac c\lambda = \frac{3 \cdot 10^{8}\,\frac{\text{м}}{\text{с}}}{700\,\text{нм}} \approx 429 \cdot 10^{12}\,\text{Гц}, \\
    \nu' &= \nu \cbr{\text{или } T' = T} \implies \lambda' = v'T' = \frac vn T = \frac{ vt }n = \frac \lambda n = \frac{700\,\text{нм}}{1{,}5} \approx 470\,\text{нм}.
    \\
    &\text{380 нм---фиол---440---син---485---гол---500---зел---565---жёл---590---оранж---625---крас---780 нм}, \text{увидит}
    \end{align*}
}
\solutionspace{60pt}

\tasknumber{2}%
\task{%
    Укажите букву, соответствующую физическую величину (из текущего раздела), её единицы измерения в СИ и выразите её из какого-либо физического закона:
    \begin{enumerate}
        \item «л'амбда»,
        \item «цэ»,
        \item «н'у»,
        \item «тэ».
    \end{enumerate}
}

\tasknumber{3}%
\task{%
    На какую длину волны настроен радиоприемник, если его колебательный контур
    обладает индуктивностью $300\,\text{мкГн}$ и ёмкостью $800\,\text{пФ}$?
}
\answer{%
    \begin{align*}
    T = 2\pi\sqrt{LC} \implies \nu &= \frac 1T = \frac 1{ 2\pi\sqrt{LC} } = \frac 1{ 2\pi\sqrt{300\,\text{мкГн} \cdot 800\,\text{пФ}}} \approx 0{,}325\,\text{МГц}, \\
    \lambda &= cT = c \cdot 2\pi\sqrt{LC} = 3 \cdot 10^{8}\,\frac{\text{м}}{\text{с}} \cdot 2\pi\sqrt{300\,\text{мкГн} \cdot 800\,\text{пФ}} \approx 923\,\text{м}.
    \end{align*}
}
\solutionspace{80pt}

\tasknumber{4}%
\task{%
    Колебательный контур настроен на частоту $0{,}5 \cdot 10^{7}\,\text{Гц}$.
    Во сколько раз и как надо изменить ёмкость конденсатора для перенастройки контура на длину волны $40\,\text{м}$?
}
\answer{%
    \begin{align*}
    T_1 &= 2\pi\sqrt{L_1C_1} \implies \nu_1 = \frac 1{T_1} = \frac 1{ 2\pi\sqrt{L_1C_1} } \implies L_1C_1 = \frac 1{\sqr{2\pi \nu_1}}, \\
    L_2C_2 &= \frac 1{\sqr{2\pi \nu_2}} = \frac 1{\sqr{2\pi \frac 1{T_2}}} = \frac 1{\sqr{2\pi \frac c{\lambda_2}}}, \\
    \frac{L_2C_2}{L_1C_1} &= \frac {\sqr{2\pi \nu}}{\sqr{2\pi \frac c{\lambda_2}}} =  \sqr{ \frac {\nu}{\frac c{\lambda_2}} } = \sqr{ \frac {\nu\lambda_2}{c} } = \sqr{ \frac { 0{,}5 \cdot 10^{7}\,\text{Гц} \cdot 40\,\text{м} }{3 \cdot 10^{8}\,\frac{\text{м}}{\text{с}}} } \approx 0{,}44.
    \end{align*}
}
\solutionspace{80pt}

\tasknumber{5}%
\task{%
    В колебательном контуре сила тока изменяется
    по закону $\eli=0{,}30\sin(18t)$ (в СИ).
    Индуктивность катушки при этом равна $50\,\text{мГн}$.
    Определите:
    \begin{itemize}
        \item период колебаний,
        \item ёмкость конденсатора,
        \item максимальный заряд конденсатора.
    \end{itemize}
}
\answer{%
    \begin{align*}
    \omega &= 18\funits{рад}{c}, \qquad \eli_{\max} = 0{,}30\,\text{A}, \\
    T &= \frac{2\pi}\omega \approx 349{,}1\,\text{мc}, \\
    C &= \frac 1{\omega^2 L} \approx 61{,}7\,\text{мФ}, \\
    q_{\max} &= \frac{\eli_{\max}}\omega  \approx 16{,}7\,\text{мКл}.
    \end{align*}
}

\variantsplitter

\addpersonalvariant{Снежана Авдошина}

\tasknumber{1}%
\task{%
    Длина волны света в~вакууме $\lambda = 600\,\text{нм}$.
    Какова частота этой световой волны?
    Какова длина этой волны в среде с показателем преломления $n = 1{,}6$?
    Может ли человек увидеть такую волну света, и если да, то какой именно цвет соответствует этим волнам в вакууме и в этой среде?
}
\answer{%
    \begin{align*}
    \nu &= \frac 1T = \frac 1{\lambda/c} = \frac c\lambda = \frac{3 \cdot 10^{8}\,\frac{\text{м}}{\text{с}}}{600\,\text{нм}} \approx 500 \cdot 10^{12}\,\text{Гц}, \\
    \nu' &= \nu \cbr{\text{или } T' = T} \implies \lambda' = v'T' = \frac vn T = \frac{ vt }n = \frac \lambda n = \frac{600\,\text{нм}}{1{,}6} \approx 375\,\text{нм}.
    \\
    &\text{380 нм---фиол---440---син---485---гол---500---зел---565---жёл---590---оранж---625---крас---780 нм}, \text{увидит}
    \end{align*}
}
\solutionspace{60pt}

\tasknumber{2}%
\task{%
    Укажите букву, соответствующую физическую величину (из текущего раздела), её единицы измерения в СИ и выразите её из какого-либо физического закона:
    \begin{enumerate}
        \item «л'амбда»,
        \item «цэ»,
        \item «н'у»,
        \item «эн».
    \end{enumerate}
}

\tasknumber{3}%
\task{%
    На какую частоту волны настроен радиоприемник, если его колебательный контур
    обладает индуктивностью $300\,\text{мкГн}$ и ёмкостью $750\,\text{пФ}$?
}
\answer{%
    \begin{align*}
    T = 2\pi\sqrt{LC} \implies \nu &= \frac 1T = \frac 1{ 2\pi\sqrt{LC} } = \frac 1{ 2\pi\sqrt{300\,\text{мкГн} \cdot 750\,\text{пФ}}} \approx 0{,}336\,\text{МГц}, \\
    \lambda &= cT = c \cdot 2\pi\sqrt{LC} = 3 \cdot 10^{8}\,\frac{\text{м}}{\text{с}} \cdot 2\pi\sqrt{300\,\text{мкГн} \cdot 750\,\text{пФ}} \approx 894\,\text{м}.
    \end{align*}
}
\solutionspace{80pt}

\tasknumber{4}%
\task{%
    Колебательный контур настроен на частоту $4{,}5 \cdot 10^{7}\,\text{Гц}$.
    Во сколько раз и как надо изменить индуктивность катушки для перенастройки контура на длину волны $50\,\text{м}$?
}
\answer{%
    \begin{align*}
    T_1 &= 2\pi\sqrt{L_1C_1} \implies \nu_1 = \frac 1{T_1} = \frac 1{ 2\pi\sqrt{L_1C_1} } \implies L_1C_1 = \frac 1{\sqr{2\pi \nu_1}}, \\
    L_2C_2 &= \frac 1{\sqr{2\pi \nu_2}} = \frac 1{\sqr{2\pi \frac 1{T_2}}} = \frac 1{\sqr{2\pi \frac c{\lambda_2}}}, \\
    \frac{L_2C_2}{L_1C_1} &= \frac {\sqr{2\pi \nu}}{\sqr{2\pi \frac c{\lambda_2}}} =  \sqr{ \frac {\nu}{\frac c{\lambda_2}} } = \sqr{ \frac {\nu\lambda_2}{c} } = \sqr{ \frac { 4{,}5 \cdot 10^{7}\,\text{Гц} \cdot 50\,\text{м} }{3 \cdot 10^{8}\,\frac{\text{м}}{\text{с}}} } \approx 56{,}3.
    \end{align*}
}
\solutionspace{80pt}

\tasknumber{5}%
\task{%
    В колебательном контуре сила тока изменяется
    по закону $\eli=0{,}05\cos(15t)$ (в СИ).
    Индуктивность катушки при этом равна $50\,\text{мГн}$.
    Определите:
    \begin{itemize}
        \item период колебаний,
        \item ёмкость конденсатора,
        \item максимальный заряд конденсатора.
    \end{itemize}
}
\answer{%
    \begin{align*}
    \omega &= 15\funits{рад}{c}, \qquad \eli_{\max} = 0{,}05\,\text{A}, \\
    T &= \frac{2\pi}\omega \approx 418{,}9\,\text{мc}, \\
    C &= \frac 1{\omega^2 L} \approx 88{,}9\,\text{мФ}, \\
    q_{\max} &= \frac{\eli_{\max}}\omega  \approx 3{,}3\,\text{мКл}.
    \end{align*}
}

\variantsplitter

\addpersonalvariant{Марьяна Аристова}

\tasknumber{1}%
\task{%
    Длина волны света в~вакууме $\lambda = 500\,\text{нм}$.
    Какова частота этой световой волны?
    Какова длина этой волны в среде с показателем преломления $n = 1{,}7$?
    Может ли человек увидеть такую волну света, и если да, то какой именно цвет соответствует этим волнам в вакууме и в этой среде?
}
\answer{%
    \begin{align*}
    \nu &= \frac 1T = \frac 1{\lambda/c} = \frac c\lambda = \frac{3 \cdot 10^{8}\,\frac{\text{м}}{\text{с}}}{500\,\text{нм}} \approx 600 \cdot 10^{12}\,\text{Гц}, \\
    \nu' &= \nu \cbr{\text{или } T' = T} \implies \lambda' = v'T' = \frac vn T = \frac{ vt }n = \frac \lambda n = \frac{500\,\text{нм}}{1{,}7} \approx 290\,\text{нм}.
    \\
    &\text{380 нм---фиол---440---син---485---гол---500---зел---565---жёл---590---оранж---625---крас---780 нм}, \text{увидит}
    \end{align*}
}
\solutionspace{60pt}

\tasknumber{2}%
\task{%
    Укажите букву, соответствующую физическую величину (из текущего раздела), её единицы измерения в СИ и выразите её из какого-либо физического закона:
    \begin{enumerate}
        \item «эл'»,
        \item «цэ»,
        \item «бал'шайа цэ»,
        \item «эн».
    \end{enumerate}
}

\tasknumber{3}%
\task{%
    На какую частоту волны настроен радиоприемник, если его колебательный контур
    обладает индуктивностью $600\,\text{мкГн}$ и ёмкостью $700\,\text{пФ}$?
}
\answer{%
    \begin{align*}
    T = 2\pi\sqrt{LC} \implies \nu &= \frac 1T = \frac 1{ 2\pi\sqrt{LC} } = \frac 1{ 2\pi\sqrt{600\,\text{мкГн} \cdot 700\,\text{пФ}}} \approx 0{,}246\,\text{МГц}, \\
    \lambda &= cT = c \cdot 2\pi\sqrt{LC} = 3 \cdot 10^{8}\,\frac{\text{м}}{\text{с}} \cdot 2\pi\sqrt{600\,\text{мкГн} \cdot 700\,\text{пФ}} \approx 1222\,\text{м}.
    \end{align*}
}
\solutionspace{80pt}

\tasknumber{4}%
\task{%
    Колебательный контур настроен на частоту $3{,}2 \cdot 10^{7}\,\text{Гц}$.
    Во сколько раз и как надо изменить ёмкость конденсатора для перенастройки контура на длину волны $25\,\text{м}$?
}
\answer{%
    \begin{align*}
    T_1 &= 2\pi\sqrt{L_1C_1} \implies \nu_1 = \frac 1{T_1} = \frac 1{ 2\pi\sqrt{L_1C_1} } \implies L_1C_1 = \frac 1{\sqr{2\pi \nu_1}}, \\
    L_2C_2 &= \frac 1{\sqr{2\pi \nu_2}} = \frac 1{\sqr{2\pi \frac 1{T_2}}} = \frac 1{\sqr{2\pi \frac c{\lambda_2}}}, \\
    \frac{L_2C_2}{L_1C_1} &= \frac {\sqr{2\pi \nu}}{\sqr{2\pi \frac c{\lambda_2}}} =  \sqr{ \frac {\nu}{\frac c{\lambda_2}} } = \sqr{ \frac {\nu\lambda_2}{c} } = \sqr{ \frac { 3{,}2 \cdot 10^{7}\,\text{Гц} \cdot 25\,\text{м} }{3 \cdot 10^{8}\,\frac{\text{м}}{\text{с}}} } \approx 7{,}11.
    \end{align*}
}
\solutionspace{80pt}

\tasknumber{5}%
\task{%
    В колебательном контуре сила тока изменяется
    по закону $\eli=0{,}25\cos(15t)$ (в СИ).
    Индуктивность катушки при этом равна $50\,\text{мГн}$.
    Определите:
    \begin{itemize}
        \item период колебаний,
        \item ёмкость конденсатора,
        \item максимальный заряд конденсатора.
    \end{itemize}
}
\answer{%
    \begin{align*}
    \omega &= 15\funits{рад}{c}, \qquad \eli_{\max} = 0{,}25\,\text{A}, \\
    T &= \frac{2\pi}\omega \approx 418{,}9\,\text{мc}, \\
    C &= \frac 1{\omega^2 L} \approx 88{,}9\,\text{мФ}, \\
    q_{\max} &= \frac{\eli_{\max}}\omega  \approx 16{,}7\,\text{мКл}.
    \end{align*}
}

\variantsplitter

\addpersonalvariant{Никита Иванов}

\tasknumber{1}%
\task{%
    Длина волны света в~вакууме $\lambda = 600\,\text{нм}$.
    Какова частота этой световой волны?
    Какова длина этой волны в среде с показателем преломления $n = 1{,}6$?
    Может ли человек увидеть такую волну света, и если да, то какой именно цвет соответствует этим волнам в вакууме и в этой среде?
}
\answer{%
    \begin{align*}
    \nu &= \frac 1T = \frac 1{\lambda/c} = \frac c\lambda = \frac{3 \cdot 10^{8}\,\frac{\text{м}}{\text{с}}}{600\,\text{нм}} \approx 500 \cdot 10^{12}\,\text{Гц}, \\
    \nu' &= \nu \cbr{\text{или } T' = T} \implies \lambda' = v'T' = \frac vn T = \frac{ vt }n = \frac \lambda n = \frac{600\,\text{нм}}{1{,}6} \approx 375\,\text{нм}.
    \\
    &\text{380 нм---фиол---440---син---485---гол---500---зел---565---жёл---590---оранж---625---крас---780 нм}, \text{увидит}
    \end{align*}
}
\solutionspace{60pt}

\tasknumber{2}%
\task{%
    Укажите букву, соответствующую физическую величину (из текущего раздела), её единицы измерения в СИ и выразите её из какого-либо физического закона:
    \begin{enumerate}
        \item «л'амбда»,
        \item «цэ»,
        \item «н'у»,
        \item «эн».
    \end{enumerate}
}

\tasknumber{3}%
\task{%
    На какую частоту волны настроен радиоприемник, если его колебательный контур
    обладает индуктивностью $300\,\text{мкГн}$ и ёмкостью $800\,\text{пФ}$?
}
\answer{%
    \begin{align*}
    T = 2\pi\sqrt{LC} \implies \nu &= \frac 1T = \frac 1{ 2\pi\sqrt{LC} } = \frac 1{ 2\pi\sqrt{300\,\text{мкГн} \cdot 800\,\text{пФ}}} \approx 0{,}325\,\text{МГц}, \\
    \lambda &= cT = c \cdot 2\pi\sqrt{LC} = 3 \cdot 10^{8}\,\frac{\text{м}}{\text{с}} \cdot 2\pi\sqrt{300\,\text{мкГн} \cdot 800\,\text{пФ}} \approx 923\,\text{м}.
    \end{align*}
}
\solutionspace{80pt}

\tasknumber{4}%
\task{%
    Колебательный контур настроен на частоту $1{,}8 \cdot 10^{7}\,\text{Гц}$.
    Во сколько раз и как надо изменить ёмкость конденсатора для перенастройки контура на длину волны $50\,\text{м}$?
}
\answer{%
    \begin{align*}
    T_1 &= 2\pi\sqrt{L_1C_1} \implies \nu_1 = \frac 1{T_1} = \frac 1{ 2\pi\sqrt{L_1C_1} } \implies L_1C_1 = \frac 1{\sqr{2\pi \nu_1}}, \\
    L_2C_2 &= \frac 1{\sqr{2\pi \nu_2}} = \frac 1{\sqr{2\pi \frac 1{T_2}}} = \frac 1{\sqr{2\pi \frac c{\lambda_2}}}, \\
    \frac{L_2C_2}{L_1C_1} &= \frac {\sqr{2\pi \nu}}{\sqr{2\pi \frac c{\lambda_2}}} =  \sqr{ \frac {\nu}{\frac c{\lambda_2}} } = \sqr{ \frac {\nu\lambda_2}{c} } = \sqr{ \frac { 1{,}8 \cdot 10^{7}\,\text{Гц} \cdot 50\,\text{м} }{3 \cdot 10^{8}\,\frac{\text{м}}{\text{с}}} } \approx 9.
    \end{align*}
}
\solutionspace{80pt}

\tasknumber{5}%
\task{%
    В колебательном контуре сила тока изменяется
    по закону $\eli=0{,}25\cos(12t)$ (в СИ).
    Индуктивность катушки при этом равна $70\,\text{мГн}$.
    Определите:
    \begin{itemize}
        \item период колебаний,
        \item ёмкость конденсатора,
        \item максимальный заряд конденсатора.
    \end{itemize}
}
\answer{%
    \begin{align*}
    \omega &= 12\funits{рад}{c}, \qquad \eli_{\max} = 0{,}25\,\text{A}, \\
    T &= \frac{2\pi}\omega \approx 523{,}6\,\text{мc}, \\
    C &= \frac 1{\omega^2 L} \approx 99{,}2\,\text{мФ}, \\
    q_{\max} &= \frac{\eli_{\max}}\omega  \approx 20{,}8\,\text{мКл}.
    \end{align*}
}

\variantsplitter

\addpersonalvariant{Анастасия Князева}

\tasknumber{1}%
\task{%
    Длина волны света в~вакууме $\lambda = 600\,\text{нм}$.
    Какова частота этой световой волны?
    Какова длина этой волны в среде с показателем преломления $n = 1{,}6$?
    Может ли человек увидеть такую волну света, и если да, то какой именно цвет соответствует этим волнам в вакууме и в этой среде?
}
\answer{%
    \begin{align*}
    \nu &= \frac 1T = \frac 1{\lambda/c} = \frac c\lambda = \frac{3 \cdot 10^{8}\,\frac{\text{м}}{\text{с}}}{600\,\text{нм}} \approx 500 \cdot 10^{12}\,\text{Гц}, \\
    \nu' &= \nu \cbr{\text{или } T' = T} \implies \lambda' = v'T' = \frac vn T = \frac{ vt }n = \frac \lambda n = \frac{600\,\text{нм}}{1{,}6} \approx 375\,\text{нм}.
    \\
    &\text{380 нм---фиол---440---син---485---гол---500---зел---565---жёл---590---оранж---625---крас---780 нм}, \text{увидит}
    \end{align*}
}
\solutionspace{60pt}

\tasknumber{2}%
\task{%
    Укажите букву, соответствующую физическую величину (из текущего раздела), её единицы измерения в СИ и выразите её из какого-либо физического закона:
    \begin{enumerate}
        \item «л'амбда»,
        \item «цэ»,
        \item «бал'шайа цэ»,
        \item «тэ».
    \end{enumerate}
}

\tasknumber{3}%
\task{%
    На какую длину волны настроен радиоприемник, если его колебательный контур
    обладает индуктивностью $600\,\text{мкГн}$ и ёмкостью $750\,\text{пФ}$?
}
\answer{%
    \begin{align*}
    T = 2\pi\sqrt{LC} \implies \nu &= \frac 1T = \frac 1{ 2\pi\sqrt{LC} } = \frac 1{ 2\pi\sqrt{600\,\text{мкГн} \cdot 750\,\text{пФ}}} \approx 0{,}237\,\text{МГц}, \\
    \lambda &= cT = c \cdot 2\pi\sqrt{LC} = 3 \cdot 10^{8}\,\frac{\text{м}}{\text{с}} \cdot 2\pi\sqrt{600\,\text{мкГн} \cdot 750\,\text{пФ}} \approx 1265\,\text{м}.
    \end{align*}
}
\solutionspace{80pt}

\tasknumber{4}%
\task{%
    Колебательный контур настроен на частоту $1{,}5 \cdot 10^{7}\,\text{Гц}$.
    Во сколько раз и как надо изменить индуктивность катушки для перенастройки контура на длину волны $20\,\text{м}$?
}
\answer{%
    \begin{align*}
    T_1 &= 2\pi\sqrt{L_1C_1} \implies \nu_1 = \frac 1{T_1} = \frac 1{ 2\pi\sqrt{L_1C_1} } \implies L_1C_1 = \frac 1{\sqr{2\pi \nu_1}}, \\
    L_2C_2 &= \frac 1{\sqr{2\pi \nu_2}} = \frac 1{\sqr{2\pi \frac 1{T_2}}} = \frac 1{\sqr{2\pi \frac c{\lambda_2}}}, \\
    \frac{L_2C_2}{L_1C_1} &= \frac {\sqr{2\pi \nu}}{\sqr{2\pi \frac c{\lambda_2}}} =  \sqr{ \frac {\nu}{\frac c{\lambda_2}} } = \sqr{ \frac {\nu\lambda_2}{c} } = \sqr{ \frac { 1{,}5 \cdot 10^{7}\,\text{Гц} \cdot 20\,\text{м} }{3 \cdot 10^{8}\,\frac{\text{м}}{\text{с}}} } \approx 1.
    \end{align*}
}
\solutionspace{80pt}

\tasknumber{5}%
\task{%
    В колебательном контуре сила тока изменяется
    по закону $\eli=0{,}30\cos(15t)$ (в СИ).
    Индуктивность катушки при этом равна $70\,\text{мГн}$.
    Определите:
    \begin{itemize}
        \item период колебаний,
        \item ёмкость конденсатора,
        \item максимальный заряд конденсатора.
    \end{itemize}
}
\answer{%
    \begin{align*}
    \omega &= 15\funits{рад}{c}, \qquad \eli_{\max} = 0{,}30\,\text{A}, \\
    T &= \frac{2\pi}\omega \approx 418{,}9\,\text{мc}, \\
    C &= \frac 1{\omega^2 L} \approx 63{,}5\,\text{мФ}, \\
    q_{\max} &= \frac{\eli_{\max}}\omega  \approx 20\,\text{мКл}.
    \end{align*}
}

\variantsplitter

\addpersonalvariant{Елизавета Кутумова}

\tasknumber{1}%
\task{%
    Длина волны света в~вакууме $\lambda = 500\,\text{нм}$.
    Какова частота этой световой волны?
    Какова длина этой волны в среде с показателем преломления $n = 1{,}7$?
    Может ли человек увидеть такую волну света, и если да, то какой именно цвет соответствует этим волнам в вакууме и в этой среде?
}
\answer{%
    \begin{align*}
    \nu &= \frac 1T = \frac 1{\lambda/c} = \frac c\lambda = \frac{3 \cdot 10^{8}\,\frac{\text{м}}{\text{с}}}{500\,\text{нм}} \approx 600 \cdot 10^{12}\,\text{Гц}, \\
    \nu' &= \nu \cbr{\text{или } T' = T} \implies \lambda' = v'T' = \frac vn T = \frac{ vt }n = \frac \lambda n = \frac{500\,\text{нм}}{1{,}7} \approx 290\,\text{нм}.
    \\
    &\text{380 нм---фиол---440---син---485---гол---500---зел---565---жёл---590---оранж---625---крас---780 нм}, \text{увидит}
    \end{align*}
}
\solutionspace{60pt}

\tasknumber{2}%
\task{%
    Укажите букву, соответствующую физическую величину (из текущего раздела), её единицы измерения в СИ и выразите её из какого-либо физического закона:
    \begin{enumerate}
        \item «л'амбда»,
        \item «цэ»,
        \item «н'у»,
        \item «эн».
    \end{enumerate}
}

\tasknumber{3}%
\task{%
    На какую частоту волны настроен радиоприемник, если его колебательный контур
    обладает индуктивностью $300\,\text{мкГн}$ и ёмкостью $700\,\text{пФ}$?
}
\answer{%
    \begin{align*}
    T = 2\pi\sqrt{LC} \implies \nu &= \frac 1T = \frac 1{ 2\pi\sqrt{LC} } = \frac 1{ 2\pi\sqrt{300\,\text{мкГн} \cdot 700\,\text{пФ}}} \approx 0{,}347\,\text{МГц}, \\
    \lambda &= cT = c \cdot 2\pi\sqrt{LC} = 3 \cdot 10^{8}\,\frac{\text{м}}{\text{с}} \cdot 2\pi\sqrt{300\,\text{мкГн} \cdot 700\,\text{пФ}} \approx 864\,\text{м}.
    \end{align*}
}
\solutionspace{80pt}

\tasknumber{4}%
\task{%
    Колебательный контур настроен на частоту $3{,}2 \cdot 10^{7}\,\text{Гц}$.
    Во сколько раз и как надо изменить ёмкость конденсатора для перенастройки контура на длину волны $30\,\text{м}$?
}
\answer{%
    \begin{align*}
    T_1 &= 2\pi\sqrt{L_1C_1} \implies \nu_1 = \frac 1{T_1} = \frac 1{ 2\pi\sqrt{L_1C_1} } \implies L_1C_1 = \frac 1{\sqr{2\pi \nu_1}}, \\
    L_2C_2 &= \frac 1{\sqr{2\pi \nu_2}} = \frac 1{\sqr{2\pi \frac 1{T_2}}} = \frac 1{\sqr{2\pi \frac c{\lambda_2}}}, \\
    \frac{L_2C_2}{L_1C_1} &= \frac {\sqr{2\pi \nu}}{\sqr{2\pi \frac c{\lambda_2}}} =  \sqr{ \frac {\nu}{\frac c{\lambda_2}} } = \sqr{ \frac {\nu\lambda_2}{c} } = \sqr{ \frac { 3{,}2 \cdot 10^{7}\,\text{Гц} \cdot 30\,\text{м} }{3 \cdot 10^{8}\,\frac{\text{м}}{\text{с}}} } \approx 10{,}24.
    \end{align*}
}
\solutionspace{80pt}

\tasknumber{5}%
\task{%
    В колебательном контуре сила тока изменяется
    по закону $\eli=0{,}25\cos(12t)$ (в СИ).
    Индуктивность катушки при этом равна $70\,\text{мГн}$.
    Определите:
    \begin{itemize}
        \item период колебаний,
        \item ёмкость конденсатора,
        \item максимальный заряд конденсатора.
    \end{itemize}
}
\answer{%
    \begin{align*}
    \omega &= 12\funits{рад}{c}, \qquad \eli_{\max} = 0{,}25\,\text{A}, \\
    T &= \frac{2\pi}\omega \approx 523{,}6\,\text{мc}, \\
    C &= \frac 1{\omega^2 L} \approx 99{,}2\,\text{мФ}, \\
    q_{\max} &= \frac{\eli_{\max}}\omega  \approx 20{,}8\,\text{мКл}.
    \end{align*}
}

\variantsplitter

\addpersonalvariant{Роксана Мехтиева}

\tasknumber{1}%
\task{%
    Длина волны света в~вакууме $\lambda = 500\,\text{нм}$.
    Какова частота этой световой волны?
    Какова длина этой волны в среде с показателем преломления $n = 1{,}4$?
    Может ли человек увидеть такую волну света, и если да, то какой именно цвет соответствует этим волнам в вакууме и в этой среде?
}
\answer{%
    \begin{align*}
    \nu &= \frac 1T = \frac 1{\lambda/c} = \frac c\lambda = \frac{3 \cdot 10^{8}\,\frac{\text{м}}{\text{с}}}{500\,\text{нм}} \approx 600 \cdot 10^{12}\,\text{Гц}, \\
    \nu' &= \nu \cbr{\text{или } T' = T} \implies \lambda' = v'T' = \frac vn T = \frac{ vt }n = \frac \lambda n = \frac{500\,\text{нм}}{1{,}4} \approx 360\,\text{нм}.
    \\
    &\text{380 нм---фиол---440---син---485---гол---500---зел---565---жёл---590---оранж---625---крас---780 нм}, \text{увидит}
    \end{align*}
}
\solutionspace{60pt}

\tasknumber{2}%
\task{%
    Укажите букву, соответствующую физическую величину (из текущего раздела), её единицы измерения в СИ и выразите её из какого-либо физического закона:
    \begin{enumerate}
        \item «л'амбда»,
        \item «цэ»,
        \item «н'у»,
        \item «тэ».
    \end{enumerate}
}

\tasknumber{3}%
\task{%
    На какую длину волны настроен радиоприемник, если его колебательный контур
    обладает индуктивностью $200\,\text{мкГн}$ и ёмкостью $700\,\text{пФ}$?
}
\answer{%
    \begin{align*}
    T = 2\pi\sqrt{LC} \implies \nu &= \frac 1T = \frac 1{ 2\pi\sqrt{LC} } = \frac 1{ 2\pi\sqrt{200\,\text{мкГн} \cdot 700\,\text{пФ}}} \approx 0{,}425\,\text{МГц}, \\
    \lambda &= cT = c \cdot 2\pi\sqrt{LC} = 3 \cdot 10^{8}\,\frac{\text{м}}{\text{с}} \cdot 2\pi\sqrt{200\,\text{мкГн} \cdot 700\,\text{пФ}} \approx 705\,\text{м}.
    \end{align*}
}
\solutionspace{80pt}

\tasknumber{4}%
\task{%
    Колебательный контур настроен на частоту $1{,}5 \cdot 10^{7}\,\text{Гц}$.
    Во сколько раз и как надо изменить ёмкость конденсатора для перенастройки контура на длину волны $30\,\text{м}$?
}
\answer{%
    \begin{align*}
    T_1 &= 2\pi\sqrt{L_1C_1} \implies \nu_1 = \frac 1{T_1} = \frac 1{ 2\pi\sqrt{L_1C_1} } \implies L_1C_1 = \frac 1{\sqr{2\pi \nu_1}}, \\
    L_2C_2 &= \frac 1{\sqr{2\pi \nu_2}} = \frac 1{\sqr{2\pi \frac 1{T_2}}} = \frac 1{\sqr{2\pi \frac c{\lambda_2}}}, \\
    \frac{L_2C_2}{L_1C_1} &= \frac {\sqr{2\pi \nu}}{\sqr{2\pi \frac c{\lambda_2}}} =  \sqr{ \frac {\nu}{\frac c{\lambda_2}} } = \sqr{ \frac {\nu\lambda_2}{c} } = \sqr{ \frac { 1{,}5 \cdot 10^{7}\,\text{Гц} \cdot 30\,\text{м} }{3 \cdot 10^{8}\,\frac{\text{м}}{\text{с}}} } \approx 2{,}3.
    \end{align*}
}
\solutionspace{80pt}

\tasknumber{5}%
\task{%
    В колебательном контуре сила тока изменяется
    по закону $\eli=0{,}05\cos(18t)$ (в СИ).
    Индуктивность катушки при этом равна $60\,\text{мГн}$.
    Определите:
    \begin{itemize}
        \item период колебаний,
        \item ёмкость конденсатора,
        \item максимальный заряд конденсатора.
    \end{itemize}
}
\answer{%
    \begin{align*}
    \omega &= 18\funits{рад}{c}, \qquad \eli_{\max} = 0{,}05\,\text{A}, \\
    T &= \frac{2\pi}\omega \approx 349{,}1\,\text{мc}, \\
    C &= \frac 1{\omega^2 L} \approx 51{,}4\,\text{мФ}, \\
    q_{\max} &= \frac{\eli_{\max}}\omega  \approx 2{,}8\,\text{мКл}.
    \end{align*}
}

\variantsplitter

\addpersonalvariant{Дилноза Нодиршоева}

\tasknumber{1}%
\task{%
    Длина волны света в~вакууме $\lambda = 600\,\text{нм}$.
    Какова частота этой световой волны?
    Какова длина этой волны в среде с показателем преломления $n = 1{,}6$?
    Может ли человек увидеть такую волну света, и если да, то какой именно цвет соответствует этим волнам в вакууме и в этой среде?
}
\answer{%
    \begin{align*}
    \nu &= \frac 1T = \frac 1{\lambda/c} = \frac c\lambda = \frac{3 \cdot 10^{8}\,\frac{\text{м}}{\text{с}}}{600\,\text{нм}} \approx 500 \cdot 10^{12}\,\text{Гц}, \\
    \nu' &= \nu \cbr{\text{или } T' = T} \implies \lambda' = v'T' = \frac vn T = \frac{ vt }n = \frac \lambda n = \frac{600\,\text{нм}}{1{,}6} \approx 375\,\text{нм}.
    \\
    &\text{380 нм---фиол---440---син---485---гол---500---зел---565---жёл---590---оранж---625---крас---780 нм}, \text{увидит}
    \end{align*}
}
\solutionspace{60pt}

\tasknumber{2}%
\task{%
    Укажите букву, соответствующую физическую величину (из текущего раздела), её единицы измерения в СИ и выразите её из какого-либо физического закона:
    \begin{enumerate}
        \item «л'амбда»,
        \item «вэ»,
        \item «н'у»,
        \item «тэ».
    \end{enumerate}
}

\tasknumber{3}%
\task{%
    На какую длину волны настроен радиоприемник, если его колебательный контур
    обладает индуктивностью $300\,\text{мкГн}$ и ёмкостью $600\,\text{пФ}$?
}
\answer{%
    \begin{align*}
    T = 2\pi\sqrt{LC} \implies \nu &= \frac 1T = \frac 1{ 2\pi\sqrt{LC} } = \frac 1{ 2\pi\sqrt{300\,\text{мкГн} \cdot 600\,\text{пФ}}} \approx 0{,}375\,\text{МГц}, \\
    \lambda &= cT = c \cdot 2\pi\sqrt{LC} = 3 \cdot 10^{8}\,\frac{\text{м}}{\text{с}} \cdot 2\pi\sqrt{300\,\text{мкГн} \cdot 600\,\text{пФ}} \approx 800\,\text{м}.
    \end{align*}
}
\solutionspace{80pt}

\tasknumber{4}%
\task{%
    Колебательный контур настроен на частоту $1{,}8 \cdot 10^{7}\,\text{Гц}$.
    Во сколько раз и как надо изменить индуктивность катушки для перенастройки контура на длину волны $50\,\text{м}$?
}
\answer{%
    \begin{align*}
    T_1 &= 2\pi\sqrt{L_1C_1} \implies \nu_1 = \frac 1{T_1} = \frac 1{ 2\pi\sqrt{L_1C_1} } \implies L_1C_1 = \frac 1{\sqr{2\pi \nu_1}}, \\
    L_2C_2 &= \frac 1{\sqr{2\pi \nu_2}} = \frac 1{\sqr{2\pi \frac 1{T_2}}} = \frac 1{\sqr{2\pi \frac c{\lambda_2}}}, \\
    \frac{L_2C_2}{L_1C_1} &= \frac {\sqr{2\pi \nu}}{\sqr{2\pi \frac c{\lambda_2}}} =  \sqr{ \frac {\nu}{\frac c{\lambda_2}} } = \sqr{ \frac {\nu\lambda_2}{c} } = \sqr{ \frac { 1{,}8 \cdot 10^{7}\,\text{Гц} \cdot 50\,\text{м} }{3 \cdot 10^{8}\,\frac{\text{м}}{\text{с}}} } \approx 9.
    \end{align*}
}
\solutionspace{80pt}

\tasknumber{5}%
\task{%
    В колебательном контуре сила тока изменяется
    по закону $\eli=0{,}30\cos(12t)$ (в СИ).
    Индуктивность катушки при этом равна $70\,\text{мГн}$.
    Определите:
    \begin{itemize}
        \item период колебаний,
        \item ёмкость конденсатора,
        \item максимальный заряд конденсатора.
    \end{itemize}
}
\answer{%
    \begin{align*}
    \omega &= 12\funits{рад}{c}, \qquad \eli_{\max} = 0{,}30\,\text{A}, \\
    T &= \frac{2\pi}\omega \approx 523{,}6\,\text{мc}, \\
    C &= \frac 1{\omega^2 L} \approx 99{,}2\,\text{мФ}, \\
    q_{\max} &= \frac{\eli_{\max}}\omega  \approx 25\,\text{мКл}.
    \end{align*}
}

\variantsplitter

\addpersonalvariant{Жаклин Пантелеева}

\tasknumber{1}%
\task{%
    Длина волны света в~вакууме $\lambda = 700\,\text{нм}$.
    Какова частота этой световой волны?
    Какова длина этой волны в среде с показателем преломления $n = 1{,}4$?
    Может ли человек увидеть такую волну света, и если да, то какой именно цвет соответствует этим волнам в вакууме и в этой среде?
}
\answer{%
    \begin{align*}
    \nu &= \frac 1T = \frac 1{\lambda/c} = \frac c\lambda = \frac{3 \cdot 10^{8}\,\frac{\text{м}}{\text{с}}}{700\,\text{нм}} \approx 429 \cdot 10^{12}\,\text{Гц}, \\
    \nu' &= \nu \cbr{\text{или } T' = T} \implies \lambda' = v'T' = \frac vn T = \frac{ vt }n = \frac \lambda n = \frac{700\,\text{нм}}{1{,}4} \approx 500\,\text{нм}.
    \\
    &\text{380 нм---фиол---440---син---485---гол---500---зел---565---жёл---590---оранж---625---крас---780 нм}, \text{увидит}
    \end{align*}
}
\solutionspace{60pt}

\tasknumber{2}%
\task{%
    Укажите букву, соответствующую физическую величину (из текущего раздела), её единицы измерения в СИ и выразите её из какого-либо физического закона:
    \begin{enumerate}
        \item «л'амбда»,
        \item «цэ»,
        \item «н'у»,
        \item «тэ».
    \end{enumerate}
}

\tasknumber{3}%
\task{%
    На какую частоту волны настроен радиоприемник, если его колебательный контур
    обладает индуктивностью $200\,\text{мкГн}$ и ёмкостью $650\,\text{пФ}$?
}
\answer{%
    \begin{align*}
    T = 2\pi\sqrt{LC} \implies \nu &= \frac 1T = \frac 1{ 2\pi\sqrt{LC} } = \frac 1{ 2\pi\sqrt{200\,\text{мкГн} \cdot 650\,\text{пФ}}} \approx 0{,}441\,\text{МГц}, \\
    \lambda &= cT = c \cdot 2\pi\sqrt{LC} = 3 \cdot 10^{8}\,\frac{\text{м}}{\text{с}} \cdot 2\pi\sqrt{200\,\text{мкГн} \cdot 650\,\text{пФ}} \approx 680\,\text{м}.
    \end{align*}
}
\solutionspace{80pt}

\tasknumber{4}%
\task{%
    Колебательный контур настроен на частоту $2{,}5 \cdot 10^{7}\,\text{Гц}$.
    Во сколько раз и как надо изменить индуктивность катушки для перенастройки контура на длину волны $30\,\text{м}$?
}
\answer{%
    \begin{align*}
    T_1 &= 2\pi\sqrt{L_1C_1} \implies \nu_1 = \frac 1{T_1} = \frac 1{ 2\pi\sqrt{L_1C_1} } \implies L_1C_1 = \frac 1{\sqr{2\pi \nu_1}}, \\
    L_2C_2 &= \frac 1{\sqr{2\pi \nu_2}} = \frac 1{\sqr{2\pi \frac 1{T_2}}} = \frac 1{\sqr{2\pi \frac c{\lambda_2}}}, \\
    \frac{L_2C_2}{L_1C_1} &= \frac {\sqr{2\pi \nu}}{\sqr{2\pi \frac c{\lambda_2}}} =  \sqr{ \frac {\nu}{\frac c{\lambda_2}} } = \sqr{ \frac {\nu\lambda_2}{c} } = \sqr{ \frac { 2{,}5 \cdot 10^{7}\,\text{Гц} \cdot 30\,\text{м} }{3 \cdot 10^{8}\,\frac{\text{м}}{\text{с}}} } \approx 6{,}25.
    \end{align*}
}
\solutionspace{80pt}

\tasknumber{5}%
\task{%
    В колебательном контуре сила тока изменяется
    по закону $\eli=0{,}05\cos(12t)$ (в СИ).
    Индуктивность катушки при этом равна $50\,\text{мГн}$.
    Определите:
    \begin{itemize}
        \item период колебаний,
        \item ёмкость конденсатора,
        \item максимальный заряд конденсатора.
    \end{itemize}
}
\answer{%
    \begin{align*}
    \omega &= 12\funits{рад}{c}, \qquad \eli_{\max} = 0{,}05\,\text{A}, \\
    T &= \frac{2\pi}\omega \approx 523{,}6\,\text{мc}, \\
    C &= \frac 1{\omega^2 L} \approx 138{,}9\,\text{мФ}, \\
    q_{\max} &= \frac{\eli_{\max}}\omega  \approx 4{,}2\,\text{мКл}.
    \end{align*}
}

\variantsplitter

\addpersonalvariant{Артём Переверзев}

\tasknumber{1}%
\task{%
    Длина волны света в~вакууме $\lambda = 700\,\text{нм}$.
    Какова частота этой световой волны?
    Какова длина этой волны в среде с показателем преломления $n = 1{,}4$?
    Может ли человек увидеть такую волну света, и если да, то какой именно цвет соответствует этим волнам в вакууме и в этой среде?
}
\answer{%
    \begin{align*}
    \nu &= \frac 1T = \frac 1{\lambda/c} = \frac c\lambda = \frac{3 \cdot 10^{8}\,\frac{\text{м}}{\text{с}}}{700\,\text{нм}} \approx 429 \cdot 10^{12}\,\text{Гц}, \\
    \nu' &= \nu \cbr{\text{или } T' = T} \implies \lambda' = v'T' = \frac vn T = \frac{ vt }n = \frac \lambda n = \frac{700\,\text{нм}}{1{,}4} \approx 500\,\text{нм}.
    \\
    &\text{380 нм---фиол---440---син---485---гол---500---зел---565---жёл---590---оранж---625---крас---780 нм}, \text{увидит}
    \end{align*}
}
\solutionspace{60pt}

\tasknumber{2}%
\task{%
    Укажите букву, соответствующую физическую величину (из текущего раздела), её единицы измерения в СИ и выразите её из какого-либо физического закона:
    \begin{enumerate}
        \item «л'амбда»,
        \item «вэ»,
        \item «бал'шайа цэ»,
        \item «тэ».
    \end{enumerate}
}

\tasknumber{3}%
\task{%
    На какую частоту волны настроен радиоприемник, если его колебательный контур
    обладает индуктивностью $600\,\text{мкГн}$ и ёмкостью $700\,\text{пФ}$?
}
\answer{%
    \begin{align*}
    T = 2\pi\sqrt{LC} \implies \nu &= \frac 1T = \frac 1{ 2\pi\sqrt{LC} } = \frac 1{ 2\pi\sqrt{600\,\text{мкГн} \cdot 700\,\text{пФ}}} \approx 0{,}246\,\text{МГц}, \\
    \lambda &= cT = c \cdot 2\pi\sqrt{LC} = 3 \cdot 10^{8}\,\frac{\text{м}}{\text{с}} \cdot 2\pi\sqrt{600\,\text{мкГн} \cdot 700\,\text{пФ}} \approx 1222\,\text{м}.
    \end{align*}
}
\solutionspace{80pt}

\tasknumber{4}%
\task{%
    Колебательный контур настроен на частоту $1{,}8 \cdot 10^{7}\,\text{Гц}$.
    Во сколько раз и как надо изменить ёмкость конденсатора для перенастройки контура на длину волны $30\,\text{м}$?
}
\answer{%
    \begin{align*}
    T_1 &= 2\pi\sqrt{L_1C_1} \implies \nu_1 = \frac 1{T_1} = \frac 1{ 2\pi\sqrt{L_1C_1} } \implies L_1C_1 = \frac 1{\sqr{2\pi \nu_1}}, \\
    L_2C_2 &= \frac 1{\sqr{2\pi \nu_2}} = \frac 1{\sqr{2\pi \frac 1{T_2}}} = \frac 1{\sqr{2\pi \frac c{\lambda_2}}}, \\
    \frac{L_2C_2}{L_1C_1} &= \frac {\sqr{2\pi \nu}}{\sqr{2\pi \frac c{\lambda_2}}} =  \sqr{ \frac {\nu}{\frac c{\lambda_2}} } = \sqr{ \frac {\nu\lambda_2}{c} } = \sqr{ \frac { 1{,}8 \cdot 10^{7}\,\text{Гц} \cdot 30\,\text{м} }{3 \cdot 10^{8}\,\frac{\text{м}}{\text{с}}} } \approx 3{,}2.
    \end{align*}
}
\solutionspace{80pt}

\tasknumber{5}%
\task{%
    В колебательном контуре сила тока изменяется
    по закону $\eli=0{,}30\cos(12t)$ (в СИ).
    Индуктивность катушки при этом равна $80\,\text{мГн}$.
    Определите:
    \begin{itemize}
        \item период колебаний,
        \item ёмкость конденсатора,
        \item максимальный заряд конденсатора.
    \end{itemize}
}
\answer{%
    \begin{align*}
    \omega &= 12\funits{рад}{c}, \qquad \eli_{\max} = 0{,}30\,\text{A}, \\
    T &= \frac{2\pi}\omega \approx 523{,}6\,\text{мc}, \\
    C &= \frac 1{\omega^2 L} \approx 86{,}8\,\text{мФ}, \\
    q_{\max} &= \frac{\eli_{\max}}\omega  \approx 25\,\text{мКл}.
    \end{align*}
}

\variantsplitter

\addpersonalvariant{Варвара Пранова}

\tasknumber{1}%
\task{%
    Длина волны света в~вакууме $\lambda = 600\,\text{нм}$.
    Какова частота этой световой волны?
    Какова длина этой волны в среде с показателем преломления $n = 1{,}7$?
    Может ли человек увидеть такую волну света, и если да, то какой именно цвет соответствует этим волнам в вакууме и в этой среде?
}
\answer{%
    \begin{align*}
    \nu &= \frac 1T = \frac 1{\lambda/c} = \frac c\lambda = \frac{3 \cdot 10^{8}\,\frac{\text{м}}{\text{с}}}{600\,\text{нм}} \approx 500 \cdot 10^{12}\,\text{Гц}, \\
    \nu' &= \nu \cbr{\text{или } T' = T} \implies \lambda' = v'T' = \frac vn T = \frac{ vt }n = \frac \lambda n = \frac{600\,\text{нм}}{1{,}7} \approx 350\,\text{нм}.
    \\
    &\text{380 нм---фиол---440---син---485---гол---500---зел---565---жёл---590---оранж---625---крас---780 нм}, \text{увидит}
    \end{align*}
}
\solutionspace{60pt}

\tasknumber{2}%
\task{%
    Укажите букву, соответствующую физическую величину (из текущего раздела), её единицы измерения в СИ и выразите её из какого-либо физического закона:
    \begin{enumerate}
        \item «л'амбда»,
        \item «цэ»,
        \item «бал'шайа цэ»,
        \item «эн».
    \end{enumerate}
}

\tasknumber{3}%
\task{%
    На какую частоту волны настроен радиоприемник, если его колебательный контур
    обладает индуктивностью $200\,\text{мкГн}$ и ёмкостью $700\,\text{пФ}$?
}
\answer{%
    \begin{align*}
    T = 2\pi\sqrt{LC} \implies \nu &= \frac 1T = \frac 1{ 2\pi\sqrt{LC} } = \frac 1{ 2\pi\sqrt{200\,\text{мкГн} \cdot 700\,\text{пФ}}} \approx 0{,}425\,\text{МГц}, \\
    \lambda &= cT = c \cdot 2\pi\sqrt{LC} = 3 \cdot 10^{8}\,\frac{\text{м}}{\text{с}} \cdot 2\pi\sqrt{200\,\text{мкГн} \cdot 700\,\text{пФ}} \approx 705\,\text{м}.
    \end{align*}
}
\solutionspace{80pt}

\tasknumber{4}%
\task{%
    Колебательный контур настроен на частоту $0{,}5 \cdot 10^{7}\,\text{Гц}$.
    Во сколько раз и как надо изменить ёмкость конденсатора для перенастройки контура на длину волны $30\,\text{м}$?
}
\answer{%
    \begin{align*}
    T_1 &= 2\pi\sqrt{L_1C_1} \implies \nu_1 = \frac 1{T_1} = \frac 1{ 2\pi\sqrt{L_1C_1} } \implies L_1C_1 = \frac 1{\sqr{2\pi \nu_1}}, \\
    L_2C_2 &= \frac 1{\sqr{2\pi \nu_2}} = \frac 1{\sqr{2\pi \frac 1{T_2}}} = \frac 1{\sqr{2\pi \frac c{\lambda_2}}}, \\
    \frac{L_2C_2}{L_1C_1} &= \frac {\sqr{2\pi \nu}}{\sqr{2\pi \frac c{\lambda_2}}} =  \sqr{ \frac {\nu}{\frac c{\lambda_2}} } = \sqr{ \frac {\nu\lambda_2}{c} } = \sqr{ \frac { 0{,}5 \cdot 10^{7}\,\text{Гц} \cdot 30\,\text{м} }{3 \cdot 10^{8}\,\frac{\text{м}}{\text{с}}} } \approx 0{,}25.
    \end{align*}
}
\solutionspace{80pt}

\tasknumber{5}%
\task{%
    В колебательном контуре сила тока изменяется
    по закону $\eli=0{,}05\cos(12t)$ (в СИ).
    Индуктивность катушки при этом равна $70\,\text{мГн}$.
    Определите:
    \begin{itemize}
        \item период колебаний,
        \item ёмкость конденсатора,
        \item максимальный заряд конденсатора.
    \end{itemize}
}
\answer{%
    \begin{align*}
    \omega &= 12\funits{рад}{c}, \qquad \eli_{\max} = 0{,}05\,\text{A}, \\
    T &= \frac{2\pi}\omega \approx 523{,}6\,\text{мc}, \\
    C &= \frac 1{\omega^2 L} \approx 99{,}2\,\text{мФ}, \\
    q_{\max} &= \frac{\eli_{\max}}\omega  \approx 4{,}2\,\text{мКл}.
    \end{align*}
}

\variantsplitter

\addpersonalvariant{Марьям Салимова}

\tasknumber{1}%
\task{%
    Длина волны света в~вакууме $\lambda = 700\,\text{нм}$.
    Какова частота этой световой волны?
    Какова длина этой волны в среде с показателем преломления $n = 1{,}7$?
    Может ли человек увидеть такую волну света, и если да, то какой именно цвет соответствует этим волнам в вакууме и в этой среде?
}
\answer{%
    \begin{align*}
    \nu &= \frac 1T = \frac 1{\lambda/c} = \frac c\lambda = \frac{3 \cdot 10^{8}\,\frac{\text{м}}{\text{с}}}{700\,\text{нм}} \approx 429 \cdot 10^{12}\,\text{Гц}, \\
    \nu' &= \nu \cbr{\text{или } T' = T} \implies \lambda' = v'T' = \frac vn T = \frac{ vt }n = \frac \lambda n = \frac{700\,\text{нм}}{1{,}7} \approx 410\,\text{нм}.
    \\
    &\text{380 нм---фиол---440---син---485---гол---500---зел---565---жёл---590---оранж---625---крас---780 нм}, \text{увидит}
    \end{align*}
}
\solutionspace{60pt}

\tasknumber{2}%
\task{%
    Укажите букву, соответствующую физическую величину (из текущего раздела), её единицы измерения в СИ и выразите её из какого-либо физического закона:
    \begin{enumerate}
        \item «л'амбда»,
        \item «цэ»,
        \item «бал'шайа цэ»,
        \item «эн».
    \end{enumerate}
}

\tasknumber{3}%
\task{%
    На какую длину волны настроен радиоприемник, если его колебательный контур
    обладает индуктивностью $200\,\text{мкГн}$ и ёмкостью $700\,\text{пФ}$?
}
\answer{%
    \begin{align*}
    T = 2\pi\sqrt{LC} \implies \nu &= \frac 1T = \frac 1{ 2\pi\sqrt{LC} } = \frac 1{ 2\pi\sqrt{200\,\text{мкГн} \cdot 700\,\text{пФ}}} \approx 0{,}425\,\text{МГц}, \\
    \lambda &= cT = c \cdot 2\pi\sqrt{LC} = 3 \cdot 10^{8}\,\frac{\text{м}}{\text{с}} \cdot 2\pi\sqrt{200\,\text{мкГн} \cdot 700\,\text{пФ}} \approx 705\,\text{м}.
    \end{align*}
}
\solutionspace{80pt}

\tasknumber{4}%
\task{%
    Колебательный контур настроен на частоту $4{,}5 \cdot 10^{7}\,\text{Гц}$.
    Во сколько раз и как надо изменить индуктивность катушки для перенастройки контура на длину волны $30\,\text{м}$?
}
\answer{%
    \begin{align*}
    T_1 &= 2\pi\sqrt{L_1C_1} \implies \nu_1 = \frac 1{T_1} = \frac 1{ 2\pi\sqrt{L_1C_1} } \implies L_1C_1 = \frac 1{\sqr{2\pi \nu_1}}, \\
    L_2C_2 &= \frac 1{\sqr{2\pi \nu_2}} = \frac 1{\sqr{2\pi \frac 1{T_2}}} = \frac 1{\sqr{2\pi \frac c{\lambda_2}}}, \\
    \frac{L_2C_2}{L_1C_1} &= \frac {\sqr{2\pi \nu}}{\sqr{2\pi \frac c{\lambda_2}}} =  \sqr{ \frac {\nu}{\frac c{\lambda_2}} } = \sqr{ \frac {\nu\lambda_2}{c} } = \sqr{ \frac { 4{,}5 \cdot 10^{7}\,\text{Гц} \cdot 30\,\text{м} }{3 \cdot 10^{8}\,\frac{\text{м}}{\text{с}}} } \approx 20{,}3.
    \end{align*}
}
\solutionspace{80pt}

\tasknumber{5}%
\task{%
    В колебательном контуре сила тока изменяется
    по закону $\eli=0{,}05\cos(18t)$ (в СИ).
    Индуктивность катушки при этом равна $50\,\text{мГн}$.
    Определите:
    \begin{itemize}
        \item период колебаний,
        \item ёмкость конденсатора,
        \item максимальный заряд конденсатора.
    \end{itemize}
}
\answer{%
    \begin{align*}
    \omega &= 18\funits{рад}{c}, \qquad \eli_{\max} = 0{,}05\,\text{A}, \\
    T &= \frac{2\pi}\omega \approx 349{,}1\,\text{мc}, \\
    C &= \frac 1{\omega^2 L} \approx 61{,}7\,\text{мФ}, \\
    q_{\max} &= \frac{\eli_{\max}}\omega  \approx 2{,}8\,\text{мКл}.
    \end{align*}
}

\variantsplitter

\addpersonalvariant{Юлия Шевченко}

\tasknumber{1}%
\task{%
    Длина волны света в~вакууме $\lambda = 600\,\text{нм}$.
    Какова частота этой световой волны?
    Какова длина этой волны в среде с показателем преломления $n = 1{,}6$?
    Может ли человек увидеть такую волну света, и если да, то какой именно цвет соответствует этим волнам в вакууме и в этой среде?
}
\answer{%
    \begin{align*}
    \nu &= \frac 1T = \frac 1{\lambda/c} = \frac c\lambda = \frac{3 \cdot 10^{8}\,\frac{\text{м}}{\text{с}}}{600\,\text{нм}} \approx 500 \cdot 10^{12}\,\text{Гц}, \\
    \nu' &= \nu \cbr{\text{или } T' = T} \implies \lambda' = v'T' = \frac vn T = \frac{ vt }n = \frac \lambda n = \frac{600\,\text{нм}}{1{,}6} \approx 375\,\text{нм}.
    \\
    &\text{380 нм---фиол---440---син---485---гол---500---зел---565---жёл---590---оранж---625---крас---780 нм}, \text{увидит}
    \end{align*}
}
\solutionspace{60pt}

\tasknumber{2}%
\task{%
    Укажите букву, соответствующую физическую величину (из текущего раздела), её единицы измерения в СИ и выразите её из какого-либо физического закона:
    \begin{enumerate}
        \item «л'амбда»,
        \item «вэ»,
        \item «н'у»,
        \item «тэ».
    \end{enumerate}
}

\tasknumber{3}%
\task{%
    На какую длину волны настроен радиоприемник, если его колебательный контур
    обладает индуктивностью $600\,\text{мкГн}$ и ёмкостью $700\,\text{пФ}$?
}
\answer{%
    \begin{align*}
    T = 2\pi\sqrt{LC} \implies \nu &= \frac 1T = \frac 1{ 2\pi\sqrt{LC} } = \frac 1{ 2\pi\sqrt{600\,\text{мкГн} \cdot 700\,\text{пФ}}} \approx 0{,}246\,\text{МГц}, \\
    \lambda &= cT = c \cdot 2\pi\sqrt{LC} = 3 \cdot 10^{8}\,\frac{\text{м}}{\text{с}} \cdot 2\pi\sqrt{600\,\text{мкГн} \cdot 700\,\text{пФ}} \approx 1222\,\text{м}.
    \end{align*}
}
\solutionspace{80pt}

\tasknumber{4}%
\task{%
    Колебательный контур настроен на частоту $0{,}8 \cdot 10^{7}\,\text{Гц}$.
    Во сколько раз и как надо изменить индуктивность катушки для перенастройки контура на длину волны $30\,\text{м}$?
}
\answer{%
    \begin{align*}
    T_1 &= 2\pi\sqrt{L_1C_1} \implies \nu_1 = \frac 1{T_1} = \frac 1{ 2\pi\sqrt{L_1C_1} } \implies L_1C_1 = \frac 1{\sqr{2\pi \nu_1}}, \\
    L_2C_2 &= \frac 1{\sqr{2\pi \nu_2}} = \frac 1{\sqr{2\pi \frac 1{T_2}}} = \frac 1{\sqr{2\pi \frac c{\lambda_2}}}, \\
    \frac{L_2C_2}{L_1C_1} &= \frac {\sqr{2\pi \nu}}{\sqr{2\pi \frac c{\lambda_2}}} =  \sqr{ \frac {\nu}{\frac c{\lambda_2}} } = \sqr{ \frac {\nu\lambda_2}{c} } = \sqr{ \frac { 0{,}8 \cdot 10^{7}\,\text{Гц} \cdot 30\,\text{м} }{3 \cdot 10^{8}\,\frac{\text{м}}{\text{с}}} } \approx 0{,}64.
    \end{align*}
}
\solutionspace{80pt}

\tasknumber{5}%
\task{%
    В колебательном контуре сила тока изменяется
    по закону $\eli=0{,}30\cos(18t)$ (в СИ).
    Индуктивность катушки при этом равна $60\,\text{мГн}$.
    Определите:
    \begin{itemize}
        \item период колебаний,
        \item ёмкость конденсатора,
        \item максимальный заряд конденсатора.
    \end{itemize}
}
\answer{%
    \begin{align*}
    \omega &= 18\funits{рад}{c}, \qquad \eli_{\max} = 0{,}30\,\text{A}, \\
    T &= \frac{2\pi}\omega \approx 349{,}1\,\text{мc}, \\
    C &= \frac 1{\omega^2 L} \approx 51{,}4\,\text{мФ}, \\
    q_{\max} &= \frac{\eli_{\max}}\omega  \approx 16{,}7\,\text{мКл}.
    \end{align*}
}
% autogenerated
