\setdate{17~ноября~2021}
\setclass{11«Б»}

\addpersonalvariant{Михаил Бурмистров}

\tasknumber{1}%
\task{%
    Схематично изобразите колебательный контур.
    Запишите формулу для периода колебаний в колебательном контуре и ...
    \begin{itemize}
        \item подпишите все физические величины,
        \item укажите их единицы измерения,
        \item выразите из формулы периода циклическую частоту,
        \item выразите из формулы периода ёмкость конденсатора.
    \end{itemize}
}
\answer{%
    $
        T = 2\pi\sqrt{LC},
        \nu = \frac 1{2\pi\sqrt{LC}},
        \omega = \frac 1{\sqrt{LC}},
        L = \frac 1C \sqr{\frac T{2\pi}},
        C = \frac 1L \sqr{\frac T{2\pi}}.
    $
}
\solutionspace{80pt}

\tasknumber{2}%
\task{%
    Во сколько раз (и как) изменится циклическая частота свободных незатухающих колебаний в контуре,
    если его индуктивность увеличить в шесть раз, а ёмкость увеличить в три раза?
}
\answer{%
    $\text{уменьшится в $4{,}24$ раз}$
}
\solutionspace{100pt}

\tasknumber{3}%
\task{%
    Оказалось, что наибольший заряд конденсатора в колебательном контуре равен $80\,\text{мкКл}$,
    а максимальный ток — $180\,\text{мА}$.
    Определите частоту колебаний.
}
\answer{%
    $
        \eli_{\max} = q_{\max}\omega \implies \nu = \frac{\omega}{2\pi} = \frac{\eli_{\max}}{2\pi q} \approx 358{,}1\,\text{Гц}.
    $
}
\solutionspace{80pt}

\tasknumber{4}%
\task{%
    В колебательном контуре сила тока изменяется
    по закону $\eli=0{,}30\cos(18t)$ (в СИ).
    Индуктивность катушки при этом равна $60\,\text{мГн}$.
    Определите:
    \begin{itemize}
        \item период колебаний,
        \item ёмкость конденсатора,
        \item максимальный заряд конденсатора.
    \end{itemize}
}
\answer{%
    \begin{align*}
    \omega &= 18\funits{рад}{c}, \qquad \eli_{\max} = 0{,}30\,\text{A}, \\
    T &= \frac{2\pi}\omega \approx 349{,}1\,\text{мc}, \\
    C &= \frac 1{\omega^2 L} \approx 51{,}4\,\text{мФ}, \\
    q_{\max} &= \frac{\eli_{\max}}\omega  \approx 16{,}7\,\text{мКл}.
    \end{align*}
}

\variantsplitter

\addpersonalvariant{Снежана Авдошина}

\tasknumber{1}%
\task{%
    Схематично изобразите колебательный контур.
    Запишите формулу для периода колебаний в колебательном контуре и ...
    \begin{itemize}
        \item подпишите все физические величины,
        \item укажите их единицы измерения,
        \item выразите из формулы периода циклическую частоту,
        \item выразите из формулы периода ёмкость конденсатора.
    \end{itemize}
}
\answer{%
    $
        T = 2\pi\sqrt{LC},
        \nu = \frac 1{2\pi\sqrt{LC}},
        \omega = \frac 1{\sqrt{LC}},
        L = \frac 1C \sqr{\frac T{2\pi}},
        C = \frac 1L \sqr{\frac T{2\pi}}.
    $
}
\solutionspace{80pt}

\tasknumber{2}%
\task{%
    Во сколько раз (и как) изменится циклическая частота свободных незатухающих колебаний в контуре,
    если его индуктивность увеличить в шесть раз, а ёмкость уменьшить в три раза?
}
\answer{%
    $\text{уменьшится в $1{,}41$ раз}$
}
\solutionspace{100pt}

\tasknumber{3}%
\task{%
    Оказалось, что наибольший заряд конденсатора в колебательном контуре равен $40\,\text{мкКл}$,
    а максимальный ток — $240\,\text{мА}$.
    Определите частоту колебаний.
}
\answer{%
    $
        \eli_{\max} = q_{\max}\omega \implies \nu = \frac{\omega}{2\pi} = \frac{\eli_{\max}}{2\pi q} \approx 954{,}9\,\text{Гц}.
    $
}
\solutionspace{80pt}

\tasknumber{4}%
\task{%
    В колебательном контуре сила тока изменяется
    по закону $\eli=0{,}05\cos(12t)$ (в СИ).
    Индуктивность катушки при этом равна $60\,\text{мГн}$.
    Определите:
    \begin{itemize}
        \item период колебаний,
        \item ёмкость конденсатора,
        \item максимальный заряд конденсатора.
    \end{itemize}
}
\answer{%
    \begin{align*}
    \omega &= 12\funits{рад}{c}, \qquad \eli_{\max} = 0{,}05\,\text{A}, \\
    T &= \frac{2\pi}\omega \approx 523{,}6\,\text{мc}, \\
    C &= \frac 1{\omega^2 L} \approx 115{,}7\,\text{мФ}, \\
    q_{\max} &= \frac{\eli_{\max}}\omega  \approx 4{,}2\,\text{мКл}.
    \end{align*}
}

\variantsplitter

\addpersonalvariant{Марьяна Аристова}

\tasknumber{1}%
\task{%
    Схематично изобразите колебательный контур.
    Запишите формулу для периода колебаний в колебательном контуре и ...
    \begin{itemize}
        \item подпишите все физические величины,
        \item укажите их единицы измерения,
        \item выразите из формулы периода циклическую частоту,
        \item выразите из формулы периода ёмкость конденсатора.
    \end{itemize}
}
\answer{%
    $
        T = 2\pi\sqrt{LC},
        \nu = \frac 1{2\pi\sqrt{LC}},
        \omega = \frac 1{\sqrt{LC}},
        L = \frac 1C \sqr{\frac T{2\pi}},
        C = \frac 1L \sqr{\frac T{2\pi}}.
    $
}
\solutionspace{80pt}

\tasknumber{2}%
\task{%
    Во сколько раз (и как) изменится период свободных незатухающих колебаний в контуре,
    если его индуктивность уменьшить в четыре раза, а ёмкость уменьшить в три раза?
}
\answer{%
    $\text{уменьшится в $3{,}46$ раз}$
}
\solutionspace{100pt}

\tasknumber{3}%
\task{%
    Оказалось, что наибольший заряд конденсатора в колебательном контуре равен $80\,\text{мкКл}$,
    а максимальный ток — $240\,\text{мА}$.
    Определите частоту колебаний.
}
\answer{%
    $
        \eli_{\max} = q_{\max}\omega \implies \nu = \frac{\omega}{2\pi} = \frac{\eli_{\max}}{2\pi q} \approx 477{,}5\,\text{Гц}.
    $
}
\solutionspace{80pt}

\tasknumber{4}%
\task{%
    В колебательном контуре сила тока изменяется
    по закону $\eli=0{,}05\cos(12t)$ (в СИ).
    Индуктивность катушки при этом равна $50\,\text{мГн}$.
    Определите:
    \begin{itemize}
        \item период колебаний,
        \item ёмкость конденсатора,
        \item максимальный заряд конденсатора.
    \end{itemize}
}
\answer{%
    \begin{align*}
    \omega &= 12\funits{рад}{c}, \qquad \eli_{\max} = 0{,}05\,\text{A}, \\
    T &= \frac{2\pi}\omega \approx 523{,}6\,\text{мc}, \\
    C &= \frac 1{\omega^2 L} \approx 138{,}9\,\text{мФ}, \\
    q_{\max} &= \frac{\eli_{\max}}\omega  \approx 4{,}2\,\text{мКл}.
    \end{align*}
}

\variantsplitter

\addpersonalvariant{Никита Иванов}

\tasknumber{1}%
\task{%
    Схематично изобразите колебательный контур.
    Запишите формулу для периода колебаний в колебательном контуре и ...
    \begin{itemize}
        \item подпишите все физические величины,
        \item укажите их единицы измерения,
        \item выразите из формулы периода частоту,
        \item выразите из формулы периода ёмкость конденсатора.
    \end{itemize}
}
\answer{%
    $
        T = 2\pi\sqrt{LC},
        \nu = \frac 1{2\pi\sqrt{LC}},
        \omega = \frac 1{\sqrt{LC}},
        L = \frac 1C \sqr{\frac T{2\pi}},
        C = \frac 1L \sqr{\frac T{2\pi}}.
    $
}
\solutionspace{80pt}

\tasknumber{2}%
\task{%
    Во сколько раз (и как) изменится частота свободных незатухающих колебаний в контуре,
    если его индуктивность увеличить в два раза, а ёмкость увеличить в четыре раза?
}
\answer{%
    $\text{уменьшится в $2{,}83$ раз}$
}
\solutionspace{100pt}

\tasknumber{3}%
\task{%
    Оказалось, что наибольший заряд конденсатора в колебательном контуре равен $40\,\text{мкКл}$,
    а максимальный ток — $180\,\text{мА}$.
    Определите частоту колебаний.
}
\answer{%
    $
        \eli_{\max} = q_{\max}\omega \implies \nu = \frac{\omega}{2\pi} = \frac{\eli_{\max}}{2\pi q} \approx 716{,}2\,\text{Гц}.
    $
}
\solutionspace{80pt}

\tasknumber{4}%
\task{%
    В колебательном контуре сила тока изменяется
    по закону $\eli=0{,}25\sin(18t)$ (в СИ).
    Индуктивность катушки при этом равна $50\,\text{мГн}$.
    Определите:
    \begin{itemize}
        \item период колебаний,
        \item ёмкость конденсатора,
        \item максимальный заряд конденсатора.
    \end{itemize}
}
\answer{%
    \begin{align*}
    \omega &= 18\funits{рад}{c}, \qquad \eli_{\max} = 0{,}25\,\text{A}, \\
    T &= \frac{2\pi}\omega \approx 349{,}1\,\text{мc}, \\
    C &= \frac 1{\omega^2 L} \approx 61{,}7\,\text{мФ}, \\
    q_{\max} &= \frac{\eli_{\max}}\omega  \approx 13{,}9\,\text{мКл}.
    \end{align*}
}

\variantsplitter

\addpersonalvariant{Анастасия Князева}

\tasknumber{1}%
\task{%
    Схематично изобразите колебательный контур.
    Запишите формулу для периода колебаний в колебательном контуре и ...
    \begin{itemize}
        \item подпишите все физические величины,
        \item укажите их единицы измерения,
        \item выразите из формулы периода циклическую частоту,
        \item выразите из формулы периода ёмкость конденсатора.
    \end{itemize}
}
\answer{%
    $
        T = 2\pi\sqrt{LC},
        \nu = \frac 1{2\pi\sqrt{LC}},
        \omega = \frac 1{\sqrt{LC}},
        L = \frac 1C \sqr{\frac T{2\pi}},
        C = \frac 1L \sqr{\frac T{2\pi}}.
    $
}
\solutionspace{80pt}

\tasknumber{2}%
\task{%
    Во сколько раз (и как) изменится период свободных незатухающих колебаний в контуре,
    если его индуктивность увеличить в три раза, а ёмкость уменьшить в семь раз?
}
\answer{%
    $\text{уменьшится в $1{,}53$ раз}$
}
\solutionspace{100pt}

\tasknumber{3}%
\task{%
    Оказалось, что наибольший заряд конденсатора в колебательном контуре равен $60\,\text{мкКл}$,
    а максимальный ток — $180\,\text{мА}$.
    Определите частоту колебаний.
}
\answer{%
    $
        \eli_{\max} = q_{\max}\omega \implies \nu = \frac{\omega}{2\pi} = \frac{\eli_{\max}}{2\pi q} \approx 477{,}5\,\text{Гц}.
    $
}
\solutionspace{80pt}

\tasknumber{4}%
\task{%
    В колебательном контуре сила тока изменяется
    по закону $\eli=0{,}30\cos(15t)$ (в СИ).
    Индуктивность катушки при этом равна $50\,\text{мГн}$.
    Определите:
    \begin{itemize}
        \item период колебаний,
        \item ёмкость конденсатора,
        \item максимальный заряд конденсатора.
    \end{itemize}
}
\answer{%
    \begin{align*}
    \omega &= 15\funits{рад}{c}, \qquad \eli_{\max} = 0{,}30\,\text{A}, \\
    T &= \frac{2\pi}\omega \approx 418{,}9\,\text{мc}, \\
    C &= \frac 1{\omega^2 L} \approx 88{,}9\,\text{мФ}, \\
    q_{\max} &= \frac{\eli_{\max}}\omega  \approx 20\,\text{мКл}.
    \end{align*}
}

\variantsplitter

\addpersonalvariant{Елизавета Кутумова}

\tasknumber{1}%
\task{%
    Схематично изобразите колебательный контур.
    Запишите формулу для периода колебаний в колебательном контуре и ...
    \begin{itemize}
        \item подпишите все физические величины,
        \item укажите их единицы измерения,
        \item выразите из формулы периода циклическую частоту,
        \item выразите из формулы периода индуктивность катушки.
    \end{itemize}
}
\answer{%
    $
        T = 2\pi\sqrt{LC},
        \nu = \frac 1{2\pi\sqrt{LC}},
        \omega = \frac 1{\sqrt{LC}},
        L = \frac 1C \sqr{\frac T{2\pi}},
        C = \frac 1L \sqr{\frac T{2\pi}}.
    $
}
\solutionspace{80pt}

\tasknumber{2}%
\task{%
    Во сколько раз (и как) изменится циклическая частота свободных незатухающих колебаний в контуре,
    если его индуктивность уменьшить в пять раз, а ёмкость уменьшить в пять раз?
}
\answer{%
    $\text{увеличится в $5{,}00$ раз}$
}
\solutionspace{100pt}

\tasknumber{3}%
\task{%
    Оказалось, что наибольший заряд конденсатора в колебательном контуре равен $80\,\text{мкКл}$,
    а максимальный ток — $150\,\text{мА}$.
    Определите частоту колебаний.
}
\answer{%
    $
        \eli_{\max} = q_{\max}\omega \implies \nu = \frac{\omega}{2\pi} = \frac{\eli_{\max}}{2\pi q} \approx 298{,}4\,\text{Гц}.
    $
}
\solutionspace{80pt}

\tasknumber{4}%
\task{%
    В колебательном контуре сила тока изменяется
    по закону $\eli=0{,}05\cos(12t)$ (в СИ).
    Индуктивность катушки при этом равна $50\,\text{мГн}$.
    Определите:
    \begin{itemize}
        \item период колебаний,
        \item ёмкость конденсатора,
        \item максимальный заряд конденсатора.
    \end{itemize}
}
\answer{%
    \begin{align*}
    \omega &= 12\funits{рад}{c}, \qquad \eli_{\max} = 0{,}05\,\text{A}, \\
    T &= \frac{2\pi}\omega \approx 523{,}6\,\text{мc}, \\
    C &= \frac 1{\omega^2 L} \approx 138{,}9\,\text{мФ}, \\
    q_{\max} &= \frac{\eli_{\max}}\omega  \approx 4{,}2\,\text{мКл}.
    \end{align*}
}

\variantsplitter

\addpersonalvariant{Роксана Мехтиева}

\tasknumber{1}%
\task{%
    Схематично изобразите колебательный контур.
    Запишите формулу для периода колебаний в колебательном контуре и ...
    \begin{itemize}
        \item подпишите все физические величины,
        \item укажите их единицы измерения,
        \item выразите из формулы периода частоту,
        \item выразите из формулы периода индуктивность катушки.
    \end{itemize}
}
\answer{%
    $
        T = 2\pi\sqrt{LC},
        \nu = \frac 1{2\pi\sqrt{LC}},
        \omega = \frac 1{\sqrt{LC}},
        L = \frac 1C \sqr{\frac T{2\pi}},
        C = \frac 1L \sqr{\frac T{2\pi}}.
    $
}
\solutionspace{80pt}

\tasknumber{2}%
\task{%
    Во сколько раз (и как) изменится циклическая частота свободных незатухающих колебаний в контуре,
    если его индуктивность увеличить в три раза, а ёмкость увеличить в пять раз?
}
\answer{%
    $\text{уменьшится в $3{,}87$ раз}$
}
\solutionspace{100pt}

\tasknumber{3}%
\task{%
    Оказалось, что наибольший заряд конденсатора в колебательном контуре равен $40\,\text{мкКл}$,
    а максимальный ток — $180\,\text{мА}$.
    Определите частоту колебаний.
}
\answer{%
    $
        \eli_{\max} = q_{\max}\omega \implies \nu = \frac{\omega}{2\pi} = \frac{\eli_{\max}}{2\pi q} \approx 716{,}2\,\text{Гц}.
    $
}
\solutionspace{80pt}

\tasknumber{4}%
\task{%
    В колебательном контуре сила тока изменяется
    по закону $\eli=0{,}05\cos(12t)$ (в СИ).
    Индуктивность катушки при этом равна $80\,\text{мГн}$.
    Определите:
    \begin{itemize}
        \item период колебаний,
        \item ёмкость конденсатора,
        \item максимальный заряд конденсатора.
    \end{itemize}
}
\answer{%
    \begin{align*}
    \omega &= 12\funits{рад}{c}, \qquad \eli_{\max} = 0{,}05\,\text{A}, \\
    T &= \frac{2\pi}\omega \approx 523{,}6\,\text{мc}, \\
    C &= \frac 1{\omega^2 L} \approx 86{,}8\,\text{мФ}, \\
    q_{\max} &= \frac{\eli_{\max}}\omega  \approx 4{,}2\,\text{мКл}.
    \end{align*}
}

\variantsplitter

\addpersonalvariant{Дилноза Нодиршоева}

\tasknumber{1}%
\task{%
    Схематично изобразите колебательный контур.
    Запишите формулу для периода колебаний в колебательном контуре и ...
    \begin{itemize}
        \item подпишите все физические величины,
        \item укажите их единицы измерения,
        \item выразите из формулы периода циклическую частоту,
        \item выразите из формулы периода индуктивность катушки.
    \end{itemize}
}
\answer{%
    $
        T = 2\pi\sqrt{LC},
        \nu = \frac 1{2\pi\sqrt{LC}},
        \omega = \frac 1{\sqrt{LC}},
        L = \frac 1C \sqr{\frac T{2\pi}},
        C = \frac 1L \sqr{\frac T{2\pi}}.
    $
}
\solutionspace{80pt}

\tasknumber{2}%
\task{%
    Во сколько раз (и как) изменится циклическая частота свободных незатухающих колебаний в контуре,
    если его индуктивность увеличить в шесть раз, а ёмкость уменьшить в три раза?
}
\answer{%
    $\text{уменьшится в $1{,}41$ раз}$
}
\solutionspace{100pt}

\tasknumber{3}%
\task{%
    Оказалось, что наибольший заряд конденсатора в колебательном контуре равен $60\,\text{мкКл}$,
    а максимальный ток — $180\,\text{мА}$.
    Определите частоту колебаний.
}
\answer{%
    $
        \eli_{\max} = q_{\max}\omega \implies \nu = \frac{\omega}{2\pi} = \frac{\eli_{\max}}{2\pi q} \approx 477{,}5\,\text{Гц}.
    $
}
\solutionspace{80pt}

\tasknumber{4}%
\task{%
    В колебательном контуре сила тока изменяется
    по закону $\eli=0{,}30\cos(18t)$ (в СИ).
    Индуктивность катушки при этом равна $80\,\text{мГн}$.
    Определите:
    \begin{itemize}
        \item период колебаний,
        \item ёмкость конденсатора,
        \item максимальный заряд конденсатора.
    \end{itemize}
}
\answer{%
    \begin{align*}
    \omega &= 18\funits{рад}{c}, \qquad \eli_{\max} = 0{,}30\,\text{A}, \\
    T &= \frac{2\pi}\omega \approx 349{,}1\,\text{мc}, \\
    C &= \frac 1{\omega^2 L} \approx 38{,}6\,\text{мФ}, \\
    q_{\max} &= \frac{\eli_{\max}}\omega  \approx 16{,}7\,\text{мКл}.
    \end{align*}
}

\variantsplitter

\addpersonalvariant{Жаклин Пантелеева}

\tasknumber{1}%
\task{%
    Схематично изобразите колебательный контур.
    Запишите формулу для периода колебаний в колебательном контуре и ...
    \begin{itemize}
        \item подпишите все физические величины,
        \item укажите их единицы измерения,
        \item выразите из формулы периода циклическую частоту,
        \item выразите из формулы периода ёмкость конденсатора.
    \end{itemize}
}
\answer{%
    $
        T = 2\pi\sqrt{LC},
        \nu = \frac 1{2\pi\sqrt{LC}},
        \omega = \frac 1{\sqrt{LC}},
        L = \frac 1C \sqr{\frac T{2\pi}},
        C = \frac 1L \sqr{\frac T{2\pi}}.
    $
}
\solutionspace{80pt}

\tasknumber{2}%
\task{%
    Во сколько раз (и как) изменится циклическая частота свободных незатухающих колебаний в контуре,
    если его индуктивность уменьшить в пять раз, а ёмкость увеличить в шесть раз?
}
\answer{%
    $\text{уменьшится в $1{,}10$ раз}$
}
\solutionspace{100pt}

\tasknumber{3}%
\task{%
    Оказалось, что наибольший заряд конденсатора в колебательном контуре равен $80\,\text{мкКл}$,
    а максимальный ток — $240\,\text{мА}$.
    Определите частоту колебаний.
}
\answer{%
    $
        \eli_{\max} = q_{\max}\omega \implies \nu = \frac{\omega}{2\pi} = \frac{\eli_{\max}}{2\pi q} \approx 477{,}5\,\text{Гц}.
    $
}
\solutionspace{80pt}

\tasknumber{4}%
\task{%
    В колебательном контуре сила тока изменяется
    по закону $\eli=0{,}30\cos(12t)$ (в СИ).
    Индуктивность катушки при этом равна $80\,\text{мГн}$.
    Определите:
    \begin{itemize}
        \item период колебаний,
        \item ёмкость конденсатора,
        \item максимальный заряд конденсатора.
    \end{itemize}
}
\answer{%
    \begin{align*}
    \omega &= 12\funits{рад}{c}, \qquad \eli_{\max} = 0{,}30\,\text{A}, \\
    T &= \frac{2\pi}\omega \approx 523{,}6\,\text{мc}, \\
    C &= \frac 1{\omega^2 L} \approx 86{,}8\,\text{мФ}, \\
    q_{\max} &= \frac{\eli_{\max}}\omega  \approx 25\,\text{мКл}.
    \end{align*}
}

\variantsplitter

\addpersonalvariant{Артём Переверзев}

\tasknumber{1}%
\task{%
    Схематично изобразите колебательный контур.
    Запишите формулу для периода колебаний в колебательном контуре и ...
    \begin{itemize}
        \item подпишите все физические величины,
        \item укажите их единицы измерения,
        \item выразите из формулы периода циклическую частоту,
        \item выразите из формулы периода индуктивность катушки.
    \end{itemize}
}
\answer{%
    $
        T = 2\pi\sqrt{LC},
        \nu = \frac 1{2\pi\sqrt{LC}},
        \omega = \frac 1{\sqrt{LC}},
        L = \frac 1C \sqr{\frac T{2\pi}},
        C = \frac 1L \sqr{\frac T{2\pi}}.
    $
}
\solutionspace{80pt}

\tasknumber{2}%
\task{%
    Во сколько раз (и как) изменится период свободных незатухающих колебаний в контуре,
    если его индуктивность увеличить в пять раз, а ёмкость увеличить в шесть раз?
}
\answer{%
    $\text{увеличится в $5{,}48$ раз}$
}
\solutionspace{100pt}

\tasknumber{3}%
\task{%
    Оказалось, что наибольший заряд конденсатора в колебательном контуре равен $40\,\text{мкКл}$,
    а максимальный ток — $180\,\text{мА}$.
    Определите частоту колебаний.
}
\answer{%
    $
        \eli_{\max} = q_{\max}\omega \implies \nu = \frac{\omega}{2\pi} = \frac{\eli_{\max}}{2\pi q} \approx 716{,}2\,\text{Гц}.
    $
}
\solutionspace{80pt}

\tasknumber{4}%
\task{%
    В колебательном контуре сила тока изменяется
    по закону $\eli=0{,}05\cos(18t)$ (в СИ).
    Индуктивность катушки при этом равна $70\,\text{мГн}$.
    Определите:
    \begin{itemize}
        \item период колебаний,
        \item ёмкость конденсатора,
        \item максимальный заряд конденсатора.
    \end{itemize}
}
\answer{%
    \begin{align*}
    \omega &= 18\funits{рад}{c}, \qquad \eli_{\max} = 0{,}05\,\text{A}, \\
    T &= \frac{2\pi}\omega \approx 349{,}1\,\text{мc}, \\
    C &= \frac 1{\omega^2 L} \approx 44{,}1\,\text{мФ}, \\
    q_{\max} &= \frac{\eli_{\max}}\omega  \approx 2{,}8\,\text{мКл}.
    \end{align*}
}

\variantsplitter

\addpersonalvariant{Варвара Пранова}

\tasknumber{1}%
\task{%
    Схематично изобразите колебательный контур.
    Запишите формулу для периода колебаний в колебательном контуре и ...
    \begin{itemize}
        \item подпишите все физические величины,
        \item укажите их единицы измерения,
        \item выразите из формулы периода частоту,
        \item выразите из формулы периода ёмкость конденсатора.
    \end{itemize}
}
\answer{%
    $
        T = 2\pi\sqrt{LC},
        \nu = \frac 1{2\pi\sqrt{LC}},
        \omega = \frac 1{\sqrt{LC}},
        L = \frac 1C \sqr{\frac T{2\pi}},
        C = \frac 1L \sqr{\frac T{2\pi}}.
    $
}
\solutionspace{80pt}

\tasknumber{2}%
\task{%
    Во сколько раз (и как) изменится период свободных незатухающих колебаний в контуре,
    если его индуктивность уменьшить в пять раз, а ёмкость увеличить в четыре раза?
}
\answer{%
    $\text{уменьшится в $1{,}12$ раз}$
}
\solutionspace{100pt}

\tasknumber{3}%
\task{%
    Оказалось, что наибольший заряд конденсатора в колебательном контуре равен $40\,\text{мкКл}$,
    а максимальный ток — $270\,\text{мА}$.
    Определите частоту колебаний.
}
\answer{%
    $
        \eli_{\max} = q_{\max}\omega \implies \nu = \frac{\omega}{2\pi} = \frac{\eli_{\max}}{2\pi q} \approx 1074{,}3\,\text{Гц}.
    $
}
\solutionspace{80pt}

\tasknumber{4}%
\task{%
    В колебательном контуре сила тока изменяется
    по закону $\eli=0{,}30\sin(18t)$ (в СИ).
    Индуктивность катушки при этом равна $60\,\text{мГн}$.
    Определите:
    \begin{itemize}
        \item период колебаний,
        \item ёмкость конденсатора,
        \item максимальный заряд конденсатора.
    \end{itemize}
}
\answer{%
    \begin{align*}
    \omega &= 18\funits{рад}{c}, \qquad \eli_{\max} = 0{,}30\,\text{A}, \\
    T &= \frac{2\pi}\omega \approx 349{,}1\,\text{мc}, \\
    C &= \frac 1{\omega^2 L} \approx 51{,}4\,\text{мФ}, \\
    q_{\max} &= \frac{\eli_{\max}}\omega  \approx 16{,}7\,\text{мКл}.
    \end{align*}
}

\variantsplitter

\addpersonalvariant{Марьям Салимова}

\tasknumber{1}%
\task{%
    Схематично изобразите колебательный контур.
    Запишите формулу для периода колебаний в колебательном контуре и ...
    \begin{itemize}
        \item подпишите все физические величины,
        \item укажите их единицы измерения,
        \item выразите из формулы периода циклическую частоту,
        \item выразите из формулы периода индуктивность катушки.
    \end{itemize}
}
\answer{%
    $
        T = 2\pi\sqrt{LC},
        \nu = \frac 1{2\pi\sqrt{LC}},
        \omega = \frac 1{\sqrt{LC}},
        L = \frac 1C \sqr{\frac T{2\pi}},
        C = \frac 1L \sqr{\frac T{2\pi}}.
    $
}
\solutionspace{80pt}

\tasknumber{2}%
\task{%
    Во сколько раз (и как) изменится циклическая частота свободных незатухающих колебаний в контуре,
    если его индуктивность увеличить в два раза, а ёмкость увеличить в три раза?
}
\answer{%
    $\text{уменьшится в $2{,}45$ раз}$
}
\solutionspace{100pt}

\tasknumber{3}%
\task{%
    Оказалось, что наибольший заряд конденсатора в колебательном контуре равен $40\,\text{мкКл}$,
    а максимальный ток — $180\,\text{мА}$.
    Определите частоту колебаний.
}
\answer{%
    $
        \eli_{\max} = q_{\max}\omega \implies \nu = \frac{\omega}{2\pi} = \frac{\eli_{\max}}{2\pi q} \approx 716{,}2\,\text{Гц}.
    $
}
\solutionspace{80pt}

\tasknumber{4}%
\task{%
    В колебательном контуре сила тока изменяется
    по закону $\eli=0{,}25\sin(18t)$ (в СИ).
    Индуктивность катушки при этом равна $70\,\text{мГн}$.
    Определите:
    \begin{itemize}
        \item период колебаний,
        \item ёмкость конденсатора,
        \item максимальный заряд конденсатора.
    \end{itemize}
}
\answer{%
    \begin{align*}
    \omega &= 18\funits{рад}{c}, \qquad \eli_{\max} = 0{,}25\,\text{A}, \\
    T &= \frac{2\pi}\omega \approx 349{,}1\,\text{мc}, \\
    C &= \frac 1{\omega^2 L} \approx 44{,}1\,\text{мФ}, \\
    q_{\max} &= \frac{\eli_{\max}}\omega  \approx 13{,}9\,\text{мКл}.
    \end{align*}
}

\variantsplitter

\addpersonalvariant{Юлия Шевченко}

\tasknumber{1}%
\task{%
    Схематично изобразите колебательный контур.
    Запишите формулу для периода колебаний в колебательном контуре и ...
    \begin{itemize}
        \item подпишите все физические величины,
        \item укажите их единицы измерения,
        \item выразите из формулы периода циклическую частоту,
        \item выразите из формулы периода индуктивность катушки.
    \end{itemize}
}
\answer{%
    $
        T = 2\pi\sqrt{LC},
        \nu = \frac 1{2\pi\sqrt{LC}},
        \omega = \frac 1{\sqrt{LC}},
        L = \frac 1C \sqr{\frac T{2\pi}},
        C = \frac 1L \sqr{\frac T{2\pi}}.
    $
}
\solutionspace{80pt}

\tasknumber{2}%
\task{%
    Во сколько раз (и как) изменится период свободных незатухающих колебаний в контуре,
    если его индуктивность уменьшить в пять раз, а ёмкость уменьшить в три раза?
}
\answer{%
    $\text{уменьшится в $3{,}87$ раз}$
}
\solutionspace{100pt}

\tasknumber{3}%
\task{%
    Оказалось, что наибольший заряд конденсатора в колебательном контуре равен $40\,\text{мкКл}$,
    а максимальный ток — $150\,\text{мА}$.
    Определите частоту колебаний.
}
\answer{%
    $
        \eli_{\max} = q_{\max}\omega \implies \nu = \frac{\omega}{2\pi} = \frac{\eli_{\max}}{2\pi q} \approx 596{,}8\,\text{Гц}.
    $
}
\solutionspace{80pt}

\tasknumber{4}%
\task{%
    В колебательном контуре сила тока изменяется
    по закону $\eli=0{,}25\cos(15t)$ (в СИ).
    Индуктивность катушки при этом равна $60\,\text{мГн}$.
    Определите:
    \begin{itemize}
        \item период колебаний,
        \item ёмкость конденсатора,
        \item максимальный заряд конденсатора.
    \end{itemize}
}
\answer{%
    \begin{align*}
    \omega &= 15\funits{рад}{c}, \qquad \eli_{\max} = 0{,}25\,\text{A}, \\
    T &= \frac{2\pi}\omega \approx 418{,}9\,\text{мc}, \\
    C &= \frac 1{\omega^2 L} \approx 74{,}1\,\text{мФ}, \\
    q_{\max} &= \frac{\eli_{\max}}\omega  \approx 16{,}7\,\text{мКл}.
    \end{align*}
}
% autogenerated
