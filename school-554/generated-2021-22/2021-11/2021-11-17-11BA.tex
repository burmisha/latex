\setdate{17~ноября~2021}
\setclass{11«БА»}

\addpersonalvariant{Михаил Бурмистров}

\tasknumber{1}%
\task{%
    На какую частоту волны настроен радиоприемник, если его колебательный контур
    обладает индуктивностью $200\,\text{мкГн}$ и емкостью $750\,\text{пФ}$?
}
\solutionspace{100pt}

\tasknumber{2}%
\task{%
    Колебательный контур настроен на частоту $4{,}5 \cdot 10^{7}\,\text{Гц}$.
    Во сколько раз и как надо изменить емкость конденсатора для перестройки контура на длину волны $40\,\text{м}$?
}
\solutionspace{100pt}

\tasknumber{3}%
\task{%
    Колебательный контур, состоящий из катушки индуктивности
    и воздушного конденсатора, настроен на длину волны $20\,\text{м}$.
    При этом расстояние между пластинами конденсатора $4{,}5\,\text{мм}$.
    Каким должно быть это расстояние, чтобы контур был настроен на длину волны $80\,\text{м}$?
}

\variantsplitter

\addpersonalvariant{Ирина Ан}

\tasknumber{1}%
\task{%
    На какую длину волны настроен радиоприемник, если его колебательный контур
    обладает индуктивностью $600\,\text{мкГн}$ и емкостью $800\,\text{пФ}$?
}
\solutionspace{100pt}

\tasknumber{2}%
\task{%
    Колебательный контур настроен на частоту $2{,}5 \cdot 10^{7}\,\text{Гц}$.
    Во сколько раз и как надо изменить емкость конденсатора для перестройки контура на длину волны $20\,\text{м}$?
}
\solutionspace{100pt}

\tasknumber{3}%
\task{%
    Колебательный контур, состоящий из катушки индуктивности
    и воздушного конденсатора, настроен на длину волны $120\,\text{м}$.
    При этом расстояние между пластинами конденсатора $3\,\text{мм}$.
    Каким должно быть это расстояние, чтобы контур был настроен на длину волны $60\,\text{м}$?
}

\variantsplitter

\addpersonalvariant{Софья Андрианова}

\tasknumber{1}%
\task{%
    На какую длину волны настроен радиоприемник, если его колебательный контур
    обладает индуктивностью $200\,\text{мкГн}$ и емкостью $700\,\text{пФ}$?
}
\solutionspace{100pt}

\tasknumber{2}%
\task{%
    Колебательный контур настроен на частоту $4{,}5 \cdot 10^{7}\,\text{Гц}$.
    Во сколько раз и как надо изменить емкость конденсатора для перестройки контура на длину волны $30\,\text{м}$?
}
\solutionspace{100pt}

\tasknumber{3}%
\task{%
    Колебательный контур, состоящий из катушки индуктивности
    и воздушного конденсатора, настроен на длину волны $30\,\text{м}$.
    При этом расстояние между пластинами конденсатора $3\,\text{мм}$.
    Каким должно быть это расстояние, чтобы контур был настроен на длину волны $100\,\text{м}$?
}

\variantsplitter

\addpersonalvariant{Владимир Артемчук}

\tasknumber{1}%
\task{%
    На какую частоту волны настроен радиоприемник, если его колебательный контур
    обладает индуктивностью $300\,\text{мкГн}$ и емкостью $800\,\text{пФ}$?
}
\solutionspace{100pt}

\tasknumber{2}%
\task{%
    Колебательный контур настроен на частоту $1{,}5 \cdot 10^{7}\,\text{Гц}$.
    Во сколько раз и как надо изменить емкость конденсатора для перестройки контура на длину волны $25\,\text{м}$?
}
\solutionspace{100pt}

\tasknumber{3}%
\task{%
    Колебательный контур, состоящий из катушки индуктивности
    и воздушного конденсатора, настроен на длину волны $120\,\text{м}$.
    При этом расстояние между пластинами конденсатора $2{,}5\,\text{мм}$.
    Каким должно быть это расстояние, чтобы контур был настроен на длину волны $100\,\text{м}$?
}

\variantsplitter

\addpersonalvariant{Софья Белянкина}

\tasknumber{1}%
\task{%
    На какую частоту волны настроен радиоприемник, если его колебательный контур
    обладает индуктивностью $600\,\text{мкГн}$ и емкостью $700\,\text{пФ}$?
}
\solutionspace{100pt}

\tasknumber{2}%
\task{%
    Колебательный контур настроен на частоту $3{,}2 \cdot 10^{7}\,\text{Гц}$.
    Во сколько раз и как надо изменить емкость конденсатора для перестройки контура на длину волны $25\,\text{м}$?
}
\solutionspace{100pt}

\tasknumber{3}%
\task{%
    Колебательный контур, состоящий из катушки индуктивности
    и воздушного конденсатора, настроен на длину волны $50\,\text{м}$.
    При этом расстояние между пластинами конденсатора $5\,\text{мм}$.
    Каким должно быть это расстояние, чтобы контур был настроен на длину волны $45\,\text{м}$?
}

\variantsplitter

\addpersonalvariant{Варвара Егиазарян}

\tasknumber{1}%
\task{%
    На какую длину волны настроен радиоприемник, если его колебательный контур
    обладает индуктивностью $200\,\text{мкГн}$ и емкостью $650\,\text{пФ}$?
}
\solutionspace{100pt}

\tasknumber{2}%
\task{%
    Колебательный контур настроен на частоту $0{,}8 \cdot 10^{7}\,\text{Гц}$.
    Во сколько раз и как надо изменить емкость конденсатора для перестройки контура на длину волны $20\,\text{м}$?
}
\solutionspace{100pt}

\tasknumber{3}%
\task{%
    Колебательный контур, состоящий из катушки индуктивности
    и воздушного конденсатора, настроен на длину волны $20\,\text{м}$.
    При этом расстояние между пластинами конденсатора $4{,}5\,\text{мм}$.
    Каким должно быть это расстояние, чтобы контур был настроен на длину волны $150\,\text{м}$?
}

\variantsplitter

\addpersonalvariant{Владислав Емелин}

\tasknumber{1}%
\task{%
    На какую длину волны настроен радиоприемник, если его колебательный контур
    обладает индуктивностью $600\,\text{мкГн}$ и емкостью $800\,\text{пФ}$?
}
\solutionspace{100pt}

\tasknumber{2}%
\task{%
    Колебательный контур настроен на частоту $1{,}8 \cdot 10^{7}\,\text{Гц}$.
    Во сколько раз и как надо изменить емкость конденсатора для перестройки контура на длину волны $50\,\text{м}$?
}
\solutionspace{100pt}

\tasknumber{3}%
\task{%
    Колебательный контур, состоящий из катушки индуктивности
    и воздушного конденсатора, настроен на длину волны $180\,\text{м}$.
    При этом расстояние между пластинами конденсатора $2\,\text{мм}$.
    Каким должно быть это расстояние, чтобы контур был настроен на длину волны $150\,\text{м}$?
}

\variantsplitter

\addpersonalvariant{Артём Жичин}

\tasknumber{1}%
\task{%
    На какую частоту волны настроен радиоприемник, если его колебательный контур
    обладает индуктивностью $300\,\text{мкГн}$ и емкостью $700\,\text{пФ}$?
}
\solutionspace{100pt}

\tasknumber{2}%
\task{%
    Колебательный контур настроен на частоту $1{,}8 \cdot 10^{7}\,\text{Гц}$.
    Во сколько раз и как надо изменить емкость конденсатора для перестройки контура на длину волны $50\,\text{м}$?
}
\solutionspace{100pt}

\tasknumber{3}%
\task{%
    Колебательный контур, состоящий из катушки индуктивности
    и воздушного конденсатора, настроен на длину волны $50\,\text{м}$.
    При этом расстояние между пластинами конденсатора $4\,\text{мм}$.
    Каким должно быть это расстояние, чтобы контур был настроен на длину волны $45\,\text{м}$?
}

\variantsplitter

\addpersonalvariant{Дарья Кошман}

\tasknumber{1}%
\task{%
    На какую частоту волны настроен радиоприемник, если его колебательный контур
    обладает индуктивностью $300\,\text{мкГн}$ и емкостью $800\,\text{пФ}$?
}
\solutionspace{100pt}

\tasknumber{2}%
\task{%
    Колебательный контур настроен на частоту $4{,}5 \cdot 10^{7}\,\text{Гц}$.
    Во сколько раз и как надо изменить емкость конденсатора для перестройки контура на длину волны $30\,\text{м}$?
}
\solutionspace{100pt}

\tasknumber{3}%
\task{%
    Колебательный контур, состоящий из катушки индуктивности
    и воздушного конденсатора, настроен на длину волны $20\,\text{м}$.
    При этом расстояние между пластинами конденсатора $2{,}5\,\text{мм}$.
    Каким должно быть это расстояние, чтобы контур был настроен на длину волны $100\,\text{м}$?
}

\variantsplitter

\addpersonalvariant{Анна Кузьмичёва}

\tasknumber{1}%
\task{%
    На какую частоту волны настроен радиоприемник, если его колебательный контур
    обладает индуктивностью $600\,\text{мкГн}$ и емкостью $800\,\text{пФ}$?
}
\solutionspace{100pt}

\tasknumber{2}%
\task{%
    Колебательный контур настроен на частоту $2{,}5 \cdot 10^{7}\,\text{Гц}$.
    Во сколько раз и как надо изменить емкость конденсатора для перестройки контура на длину волны $30\,\text{м}$?
}
\solutionspace{100pt}

\tasknumber{3}%
\task{%
    Колебательный контур, состоящий из катушки индуктивности
    и воздушного конденсатора, настроен на длину волны $120\,\text{м}$.
    При этом расстояние между пластинами конденсатора $2\,\text{мм}$.
    Каким должно быть это расстояние, чтобы контур был настроен на длину волны $80\,\text{м}$?
}

\variantsplitter

\addpersonalvariant{Алёна Куприянова}

\tasknumber{1}%
\task{%
    На какую длину волны настроен радиоприемник, если его колебательный контур
    обладает индуктивностью $600\,\text{мкГн}$ и емкостью $600\,\text{пФ}$?
}
\solutionspace{100pt}

\tasknumber{2}%
\task{%
    Колебательный контур настроен на частоту $4{,}5 \cdot 10^{7}\,\text{Гц}$.
    Во сколько раз и как надо изменить емкость конденсатора для перестройки контура на длину волны $20\,\text{м}$?
}
\solutionspace{100pt}

\tasknumber{3}%
\task{%
    Колебательный контур, состоящий из катушки индуктивности
    и воздушного конденсатора, настроен на длину волны $120\,\text{м}$.
    При этом расстояние между пластинами конденсатора $4{,}5\,\text{мм}$.
    Каким должно быть это расстояние, чтобы контур был настроен на длину волны $80\,\text{м}$?
}

\variantsplitter

\addpersonalvariant{Ярослав Лавровский}

\tasknumber{1}%
\task{%
    На какую частоту волны настроен радиоприемник, если его колебательный контур
    обладает индуктивностью $600\,\text{мкГн}$ и емкостью $750\,\text{пФ}$?
}
\solutionspace{100pt}

\tasknumber{2}%
\task{%
    Колебательный контур настроен на частоту $1{,}5 \cdot 10^{7}\,\text{Гц}$.
    Во сколько раз и как надо изменить емкость конденсатора для перестройки контура на длину волны $20\,\text{м}$?
}
\solutionspace{100pt}

\tasknumber{3}%
\task{%
    Колебательный контур, состоящий из катушки индуктивности
    и воздушного конденсатора, настроен на длину волны $50\,\text{м}$.
    При этом расстояние между пластинами конденсатора $5\,\text{мм}$.
    Каким должно быть это расстояние, чтобы контур был настроен на длину волны $150\,\text{м}$?
}

\variantsplitter

\addpersonalvariant{Анастасия Ламанова}

\tasknumber{1}%
\task{%
    На какую частоту волны настроен радиоприемник, если его колебательный контур
    обладает индуктивностью $200\,\text{мкГн}$ и емкостью $750\,\text{пФ}$?
}
\solutionspace{100pt}

\tasknumber{2}%
\task{%
    Колебательный контур настроен на частоту $4{,}5 \cdot 10^{7}\,\text{Гц}$.
    Во сколько раз и как надо изменить емкость конденсатора для перестройки контура на длину волны $40\,\text{м}$?
}
\solutionspace{100pt}

\tasknumber{3}%
\task{%
    Колебательный контур, состоящий из катушки индуктивности
    и воздушного конденсатора, настроен на длину волны $180\,\text{м}$.
    При этом расстояние между пластинами конденсатора $3{,}5\,\text{мм}$.
    Каким должно быть это расстояние, чтобы контур был настроен на длину волны $80\,\text{м}$?
}

\variantsplitter

\addpersonalvariant{Виктория Легонькова}

\tasknumber{1}%
\task{%
    На какую частоту волны настроен радиоприемник, если его колебательный контур
    обладает индуктивностью $200\,\text{мкГн}$ и емкостью $750\,\text{пФ}$?
}
\solutionspace{100pt}

\tasknumber{2}%
\task{%
    Колебательный контур настроен на частоту $0{,}5 \cdot 10^{7}\,\text{Гц}$.
    Во сколько раз и как надо изменить емкость конденсатора для перестройки контура на длину волны $50\,\text{м}$?
}
\solutionspace{100pt}

\tasknumber{3}%
\task{%
    Колебательный контур, состоящий из катушки индуктивности
    и воздушного конденсатора, настроен на длину волны $20\,\text{м}$.
    При этом расстояние между пластинами конденсатора $3\,\text{мм}$.
    Каким должно быть это расстояние, чтобы контур был настроен на длину волны $80\,\text{м}$?
}

\variantsplitter

\addpersonalvariant{Семён Мартынов}

\tasknumber{1}%
\task{%
    На какую длину волны настроен радиоприемник, если его колебательный контур
    обладает индуктивностью $200\,\text{мкГн}$ и емкостью $800\,\text{пФ}$?
}
\solutionspace{100pt}

\tasknumber{2}%
\task{%
    Колебательный контур настроен на частоту $1{,}8 \cdot 10^{7}\,\text{Гц}$.
    Во сколько раз и как надо изменить емкость конденсатора для перестройки контура на длину волны $20\,\text{м}$?
}
\solutionspace{100pt}

\tasknumber{3}%
\task{%
    Колебательный контур, состоящий из катушки индуктивности
    и воздушного конденсатора, настроен на длину волны $20\,\text{м}$.
    При этом расстояние между пластинами конденсатора $5\,\text{мм}$.
    Каким должно быть это расстояние, чтобы контур был настроен на длину волны $45\,\text{м}$?
}

\variantsplitter

\addpersonalvariant{Варвара Минаева}

\tasknumber{1}%
\task{%
    На какую длину волны настроен радиоприемник, если его колебательный контур
    обладает индуктивностью $300\,\text{мкГн}$ и емкостью $750\,\text{пФ}$?
}
\solutionspace{100pt}

\tasknumber{2}%
\task{%
    Колебательный контур настроен на частоту $0{,}5 \cdot 10^{7}\,\text{Гц}$.
    Во сколько раз и как надо изменить емкость конденсатора для перестройки контура на длину волны $30\,\text{м}$?
}
\solutionspace{100pt}

\tasknumber{3}%
\task{%
    Колебательный контур, состоящий из катушки индуктивности
    и воздушного конденсатора, настроен на длину волны $120\,\text{м}$.
    При этом расстояние между пластинами конденсатора $4{,}5\,\text{мм}$.
    Каким должно быть это расстояние, чтобы контур был настроен на длину волны $45\,\text{м}$?
}

\variantsplitter

\addpersonalvariant{Леонид Никитин}

\tasknumber{1}%
\task{%
    На какую частоту волны настроен радиоприемник, если его колебательный контур
    обладает индуктивностью $200\,\text{мкГн}$ и емкостью $600\,\text{пФ}$?
}
\solutionspace{100pt}

\tasknumber{2}%
\task{%
    Колебательный контур настроен на частоту $0{,}5 \cdot 10^{7}\,\text{Гц}$.
    Во сколько раз и как надо изменить емкость конденсатора для перестройки контура на длину волны $50\,\text{м}$?
}
\solutionspace{100pt}

\tasknumber{3}%
\task{%
    Колебательный контур, состоящий из катушки индуктивности
    и воздушного конденсатора, настроен на длину волны $20\,\text{м}$.
    При этом расстояние между пластинами конденсатора $2\,\text{мм}$.
    Каким должно быть это расстояние, чтобы контур был настроен на длину волны $80\,\text{м}$?
}

\variantsplitter

\addpersonalvariant{Тимофей Полетаев}

\tasknumber{1}%
\task{%
    На какую длину волны настроен радиоприемник, если его колебательный контур
    обладает индуктивностью $200\,\text{мкГн}$ и емкостью $650\,\text{пФ}$?
}
\solutionspace{100pt}

\tasknumber{2}%
\task{%
    Колебательный контур настроен на частоту $1{,}8 \cdot 10^{7}\,\text{Гц}$.
    Во сколько раз и как надо изменить емкость конденсатора для перестройки контура на длину волны $40\,\text{м}$?
}
\solutionspace{100pt}

\tasknumber{3}%
\task{%
    Колебательный контур, состоящий из катушки индуктивности
    и воздушного конденсатора, настроен на длину волны $50\,\text{м}$.
    При этом расстояние между пластинами конденсатора $4{,}5\,\text{мм}$.
    Каким должно быть это расстояние, чтобы контур был настроен на длину волны $150\,\text{м}$?
}

\variantsplitter

\addpersonalvariant{Андрей Рожков}

\tasknumber{1}%
\task{%
    На какую длину волны настроен радиоприемник, если его колебательный контур
    обладает индуктивностью $300\,\text{мкГн}$ и емкостью $650\,\text{пФ}$?
}
\solutionspace{100pt}

\tasknumber{2}%
\task{%
    Колебательный контур настроен на частоту $1{,}8 \cdot 10^{7}\,\text{Гц}$.
    Во сколько раз и как надо изменить емкость конденсатора для перестройки контура на длину волны $40\,\text{м}$?
}
\solutionspace{100pt}

\tasknumber{3}%
\task{%
    Колебательный контур, состоящий из катушки индуктивности
    и воздушного конденсатора, настроен на длину волны $180\,\text{м}$.
    При этом расстояние между пластинами конденсатора $3\,\text{мм}$.
    Каким должно быть это расстояние, чтобы контур был настроен на длину волны $60\,\text{м}$?
}

\variantsplitter

\addpersonalvariant{Рената Таржиманова}

\tasknumber{1}%
\task{%
    На какую длину волны настроен радиоприемник, если его колебательный контур
    обладает индуктивностью $600\,\text{мкГн}$ и емкостью $800\,\text{пФ}$?
}
\solutionspace{100pt}

\tasknumber{2}%
\task{%
    Колебательный контур настроен на частоту $1{,}5 \cdot 10^{7}\,\text{Гц}$.
    Во сколько раз и как надо изменить емкость конденсатора для перестройки контура на длину волны $25\,\text{м}$?
}
\solutionspace{100pt}

\tasknumber{3}%
\task{%
    Колебательный контур, состоящий из катушки индуктивности
    и воздушного конденсатора, настроен на длину волны $120\,\text{м}$.
    При этом расстояние между пластинами конденсатора $3{,}5\,\text{мм}$.
    Каким должно быть это расстояние, чтобы контур был настроен на длину волны $100\,\text{м}$?
}

\variantsplitter

\addpersonalvariant{Андрей Щербаков}

\tasknumber{1}%
\task{%
    На какую длину волны настроен радиоприемник, если его колебательный контур
    обладает индуктивностью $300\,\text{мкГн}$ и емкостью $750\,\text{пФ}$?
}
\solutionspace{100pt}

\tasknumber{2}%
\task{%
    Колебательный контур настроен на частоту $0{,}8 \cdot 10^{7}\,\text{Гц}$.
    Во сколько раз и как надо изменить емкость конденсатора для перестройки контура на длину волны $20\,\text{м}$?
}
\solutionspace{100pt}

\tasknumber{3}%
\task{%
    Колебательный контур, состоящий из катушки индуктивности
    и воздушного конденсатора, настроен на длину волны $20\,\text{м}$.
    При этом расстояние между пластинами конденсатора $4{,}5\,\text{мм}$.
    Каким должно быть это расстояние, чтобы контур был настроен на длину волны $80\,\text{м}$?
}

\variantsplitter

\addpersonalvariant{Михаил Ярошевский}

\tasknumber{1}%
\task{%
    На какую частоту волны настроен радиоприемник, если его колебательный контур
    обладает индуктивностью $600\,\text{мкГн}$ и емкостью $750\,\text{пФ}$?
}
\solutionspace{100pt}

\tasknumber{2}%
\task{%
    Колебательный контур настроен на частоту $2{,}5 \cdot 10^{7}\,\text{Гц}$.
    Во сколько раз и как надо изменить емкость конденсатора для перестройки контура на длину волны $20\,\text{м}$?
}
\solutionspace{100pt}

\tasknumber{3}%
\task{%
    Колебательный контур, состоящий из катушки индуктивности
    и воздушного конденсатора, настроен на длину волны $180\,\text{м}$.
    При этом расстояние между пластинами конденсатора $4\,\text{мм}$.
    Каким должно быть это расстояние, чтобы контур был настроен на длину волны $60\,\text{м}$?
}

\variantsplitter

\addpersonalvariant{Алексей Алимпиев}

\tasknumber{1}%
\task{%
    На какую длину волны настроен радиоприемник, если его колебательный контур
    обладает индуктивностью $300\,\text{мкГн}$ и емкостью $800\,\text{пФ}$?
}
\solutionspace{100pt}

\tasknumber{2}%
\task{%
    Колебательный контур настроен на частоту $0{,}5 \cdot 10^{7}\,\text{Гц}$.
    Во сколько раз и как надо изменить емкость конденсатора для перестройки контура на длину волны $25\,\text{м}$?
}
\solutionspace{100pt}

\tasknumber{3}%
\task{%
    Колебательный контур, состоящий из катушки индуктивности
    и воздушного конденсатора, настроен на длину волны $20\,\text{м}$.
    При этом расстояние между пластинами конденсатора $2\,\text{мм}$.
    Каким должно быть это расстояние, чтобы контур был настроен на длину волны $100\,\text{м}$?
}

\variantsplitter

\addpersonalvariant{Евгений Васин}

\tasknumber{1}%
\task{%
    На какую длину волны настроен радиоприемник, если его колебательный контур
    обладает индуктивностью $200\,\text{мкГн}$ и емкостью $600\,\text{пФ}$?
}
\solutionspace{100pt}

\tasknumber{2}%
\task{%
    Колебательный контур настроен на частоту $4{,}5 \cdot 10^{7}\,\text{Гц}$.
    Во сколько раз и как надо изменить емкость конденсатора для перестройки контура на длину волны $30\,\text{м}$?
}
\solutionspace{100pt}

\tasknumber{3}%
\task{%
    Колебательный контур, состоящий из катушки индуктивности
    и воздушного конденсатора, настроен на длину волны $20\,\text{м}$.
    При этом расстояние между пластинами конденсатора $4\,\text{мм}$.
    Каким должно быть это расстояние, чтобы контур был настроен на длину волны $100\,\text{м}$?
}

\variantsplitter

\addpersonalvariant{Вячеслав Волохов}

\tasknumber{1}%
\task{%
    На какую частоту волны настроен радиоприемник, если его колебательный контур
    обладает индуктивностью $600\,\text{мкГн}$ и емкостью $700\,\text{пФ}$?
}
\solutionspace{100pt}

\tasknumber{2}%
\task{%
    Колебательный контур настроен на частоту $4{,}5 \cdot 10^{7}\,\text{Гц}$.
    Во сколько раз и как надо изменить емкость конденсатора для перестройки контура на длину волны $30\,\text{м}$?
}
\solutionspace{100pt}

\tasknumber{3}%
\task{%
    Колебательный контур, состоящий из катушки индуктивности
    и воздушного конденсатора, настроен на длину волны $50\,\text{м}$.
    При этом расстояние между пластинами конденсатора $4\,\text{мм}$.
    Каким должно быть это расстояние, чтобы контур был настроен на длину волны $60\,\text{м}$?
}

\variantsplitter

\addpersonalvariant{Герман Говоров}

\tasknumber{1}%
\task{%
    На какую частоту волны настроен радиоприемник, если его колебательный контур
    обладает индуктивностью $600\,\text{мкГн}$ и емкостью $800\,\text{пФ}$?
}
\solutionspace{100pt}

\tasknumber{2}%
\task{%
    Колебательный контур настроен на частоту $1{,}8 \cdot 10^{7}\,\text{Гц}$.
    Во сколько раз и как надо изменить емкость конденсатора для перестройки контура на длину волны $30\,\text{м}$?
}
\solutionspace{100pt}

\tasknumber{3}%
\task{%
    Колебательный контур, состоящий из катушки индуктивности
    и воздушного конденсатора, настроен на длину волны $20\,\text{м}$.
    При этом расстояние между пластинами конденсатора $3\,\text{мм}$.
    Каким должно быть это расстояние, чтобы контур был настроен на длину волны $60\,\text{м}$?
}

\variantsplitter

\addpersonalvariant{София Журавлёва}

\tasknumber{1}%
\task{%
    На какую частоту волны настроен радиоприемник, если его колебательный контур
    обладает индуктивностью $300\,\text{мкГн}$ и емкостью $700\,\text{пФ}$?
}
\solutionspace{100pt}

\tasknumber{2}%
\task{%
    Колебательный контур настроен на частоту $0{,}5 \cdot 10^{7}\,\text{Гц}$.
    Во сколько раз и как надо изменить емкость конденсатора для перестройки контура на длину волны $20\,\text{м}$?
}
\solutionspace{100pt}

\tasknumber{3}%
\task{%
    Колебательный контур, состоящий из катушки индуктивности
    и воздушного конденсатора, настроен на длину волны $50\,\text{м}$.
    При этом расстояние между пластинами конденсатора $3{,}5\,\text{мм}$.
    Каким должно быть это расстояние, чтобы контур был настроен на длину волны $45\,\text{м}$?
}

\variantsplitter

\addpersonalvariant{Константин Козлов}

\tasknumber{1}%
\task{%
    На какую частоту волны настроен радиоприемник, если его колебательный контур
    обладает индуктивностью $600\,\text{мкГн}$ и емкостью $750\,\text{пФ}$?
}
\solutionspace{100pt}

\tasknumber{2}%
\task{%
    Колебательный контур настроен на частоту $4{,}5 \cdot 10^{7}\,\text{Гц}$.
    Во сколько раз и как надо изменить емкость конденсатора для перестройки контура на длину волны $50\,\text{м}$?
}
\solutionspace{100pt}

\tasknumber{3}%
\task{%
    Колебательный контур, состоящий из катушки индуктивности
    и воздушного конденсатора, настроен на длину волны $20\,\text{м}$.
    При этом расстояние между пластинами конденсатора $2{,}5\,\text{мм}$.
    Каким должно быть это расстояние, чтобы контур был настроен на длину волны $150\,\text{м}$?
}

\variantsplitter

\addpersonalvariant{Наталья Кравченко}

\tasknumber{1}%
\task{%
    На какую частоту волны настроен радиоприемник, если его колебательный контур
    обладает индуктивностью $600\,\text{мкГн}$ и емкостью $800\,\text{пФ}$?
}
\solutionspace{100pt}

\tasknumber{2}%
\task{%
    Колебательный контур настроен на частоту $1{,}8 \cdot 10^{7}\,\text{Гц}$.
    Во сколько раз и как надо изменить емкость конденсатора для перестройки контура на длину волны $25\,\text{м}$?
}
\solutionspace{100pt}

\tasknumber{3}%
\task{%
    Колебательный контур, состоящий из катушки индуктивности
    и воздушного конденсатора, настроен на длину волны $180\,\text{м}$.
    При этом расстояние между пластинами конденсатора $2\,\text{мм}$.
    Каким должно быть это расстояние, чтобы контур был настроен на длину волны $80\,\text{м}$?
}

\variantsplitter

\addpersonalvariant{Матвей Кузьмин}

\tasknumber{1}%
\task{%
    На какую длину волны настроен радиоприемник, если его колебательный контур
    обладает индуктивностью $300\,\text{мкГн}$ и емкостью $650\,\text{пФ}$?
}
\solutionspace{100pt}

\tasknumber{2}%
\task{%
    Колебательный контур настроен на частоту $4{,}5 \cdot 10^{7}\,\text{Гц}$.
    Во сколько раз и как надо изменить емкость конденсатора для перестройки контура на длину волны $50\,\text{м}$?
}
\solutionspace{100pt}

\tasknumber{3}%
\task{%
    Колебательный контур, состоящий из катушки индуктивности
    и воздушного конденсатора, настроен на длину волны $20\,\text{м}$.
    При этом расстояние между пластинами конденсатора $3\,\text{мм}$.
    Каким должно быть это расстояние, чтобы контур был настроен на длину волны $45\,\text{м}$?
}

\variantsplitter

\addpersonalvariant{Сергей Малышев}

\tasknumber{1}%
\task{%
    На какую длину волны настроен радиоприемник, если его колебательный контур
    обладает индуктивностью $600\,\text{мкГн}$ и емкостью $600\,\text{пФ}$?
}
\solutionspace{100pt}

\tasknumber{2}%
\task{%
    Колебательный контур настроен на частоту $1{,}5 \cdot 10^{7}\,\text{Гц}$.
    Во сколько раз и как надо изменить емкость конденсатора для перестройки контура на длину волны $40\,\text{м}$?
}
\solutionspace{100pt}

\tasknumber{3}%
\task{%
    Колебательный контур, состоящий из катушки индуктивности
    и воздушного конденсатора, настроен на длину волны $20\,\text{м}$.
    При этом расстояние между пластинами конденсатора $2{,}5\,\text{мм}$.
    Каким должно быть это расстояние, чтобы контур был настроен на длину волны $150\,\text{м}$?
}

\variantsplitter

\addpersonalvariant{Алина Полканова}

\tasknumber{1}%
\task{%
    На какую частоту волны настроен радиоприемник, если его колебательный контур
    обладает индуктивностью $300\,\text{мкГн}$ и емкостью $800\,\text{пФ}$?
}
\solutionspace{100pt}

\tasknumber{2}%
\task{%
    Колебательный контур настроен на частоту $1{,}5 \cdot 10^{7}\,\text{Гц}$.
    Во сколько раз и как надо изменить емкость конденсатора для перестройки контура на длину волны $40\,\text{м}$?
}
\solutionspace{100pt}

\tasknumber{3}%
\task{%
    Колебательный контур, состоящий из катушки индуктивности
    и воздушного конденсатора, настроен на длину волны $120\,\text{м}$.
    При этом расстояние между пластинами конденсатора $4{,}5\,\text{мм}$.
    Каким должно быть это расстояние, чтобы контур был настроен на длину волны $45\,\text{м}$?
}

\variantsplitter

\addpersonalvariant{Сергей Пономарёв}

\tasknumber{1}%
\task{%
    На какую длину волны настроен радиоприемник, если его колебательный контур
    обладает индуктивностью $600\,\text{мкГн}$ и емкостью $600\,\text{пФ}$?
}
\solutionspace{100pt}

\tasknumber{2}%
\task{%
    Колебательный контур настроен на частоту $1{,}8 \cdot 10^{7}\,\text{Гц}$.
    Во сколько раз и как надо изменить емкость конденсатора для перестройки контура на длину волны $30\,\text{м}$?
}
\solutionspace{100pt}

\tasknumber{3}%
\task{%
    Колебательный контур, состоящий из катушки индуктивности
    и воздушного конденсатора, настроен на длину волны $120\,\text{м}$.
    При этом расстояние между пластинами конденсатора $4{,}5\,\text{мм}$.
    Каким должно быть это расстояние, чтобы контур был настроен на длину волны $150\,\text{м}$?
}

\variantsplitter

\addpersonalvariant{Егор Свистушкин}

\tasknumber{1}%
\task{%
    На какую длину волны настроен радиоприемник, если его колебательный контур
    обладает индуктивностью $300\,\text{мкГн}$ и емкостью $700\,\text{пФ}$?
}
\solutionspace{100pt}

\tasknumber{2}%
\task{%
    Колебательный контур настроен на частоту $0{,}8 \cdot 10^{7}\,\text{Гц}$.
    Во сколько раз и как надо изменить емкость конденсатора для перестройки контура на длину волны $20\,\text{м}$?
}
\solutionspace{100pt}

\tasknumber{3}%
\task{%
    Колебательный контур, состоящий из катушки индуктивности
    и воздушного конденсатора, настроен на длину волны $120\,\text{м}$.
    При этом расстояние между пластинами конденсатора $3\,\text{мм}$.
    Каким должно быть это расстояние, чтобы контур был настроен на длину волны $100\,\text{м}$?
}

\variantsplitter

\addpersonalvariant{Дмитрий Соколов}

\tasknumber{1}%
\task{%
    На какую частоту волны настроен радиоприемник, если его колебательный контур
    обладает индуктивностью $200\,\text{мкГн}$ и емкостью $750\,\text{пФ}$?
}
\solutionspace{100pt}

\tasknumber{2}%
\task{%
    Колебательный контур настроен на частоту $0{,}8 \cdot 10^{7}\,\text{Гц}$.
    Во сколько раз и как надо изменить емкость конденсатора для перестройки контура на длину волны $20\,\text{м}$?
}
\solutionspace{100pt}

\tasknumber{3}%
\task{%
    Колебательный контур, состоящий из катушки индуктивности
    и воздушного конденсатора, настроен на длину волны $180\,\text{м}$.
    При этом расстояние между пластинами конденсатора $3\,\text{мм}$.
    Каким должно быть это расстояние, чтобы контур был настроен на длину волны $80\,\text{м}$?
}

\variantsplitter

\addpersonalvariant{Арсений Трофимов}

\tasknumber{1}%
\task{%
    На какую частоту волны настроен радиоприемник, если его колебательный контур
    обладает индуктивностью $300\,\text{мкГн}$ и емкостью $600\,\text{пФ}$?
}
\solutionspace{100pt}

\tasknumber{2}%
\task{%
    Колебательный контур настроен на частоту $4{,}5 \cdot 10^{7}\,\text{Гц}$.
    Во сколько раз и как надо изменить емкость конденсатора для перестройки контура на длину волны $25\,\text{м}$?
}
\solutionspace{100pt}

\tasknumber{3}%
\task{%
    Колебательный контур, состоящий из катушки индуктивности
    и воздушного конденсатора, настроен на длину волны $180\,\text{м}$.
    При этом расстояние между пластинами конденсатора $4\,\text{мм}$.
    Каким должно быть это расстояние, чтобы контур был настроен на длину волны $45\,\text{м}$?
}
% autogenerated
