\setdate{18~ноября~2021}
\setclass{11«БА»}

\addpersonalvariant{Михаил Бурмистров}

\tasknumber{1}%
\task{%
    Разность фаз двух интерферирующих световых волн равна $3\pi$, а разность хода между ними равна $9{,}5 \cdot 10^{-7}\,\text{м}$.
    Определить длину волны.
}
\solutionspace{100pt}

\tasknumber{2}%
\task{%
    Расстояние между двумя точечными когерентными источниками света $S_1$ u $S_2$ равно $2{,}5\,\text{мм}$.
    Источники расположены в плоскости, параллельной экрану, на расстоянии $9\,\text{м}$ от него.
    На экране в точках, лежащих на перпендикулярах, опущенных из источников света $S_1$ и $S_2$,
    находятся два ближайших минимума (тёмные полосы).
    Определите длину световой волны.
    Ответ дать в нанометрах.
}
\solutionspace{100pt}

\tasknumber{3}%
\task{%
    На дифракционную решетку, имеющую период $3 \cdot 10^{-4}\,\text{см}$, нормально падает монохроматическая световая волна.
    Под углом $ 40 \degrees$ наблюдается дифракционный максимум второго порядка.
    Какова длина волны падающего света?
}
\solutionspace{100pt}

\tasknumber{4}%
\task{%
    Свет с длиной волны $0{,}6\,\text{мкм}$ падает нормально на дифракционную решетку с периодом, равным $2\,\text{мкм}$.
    Под каким углом наблюдается дифракционный максимум первого порядка?
}
\solutionspace{100pt}

\tasknumber{5}%
\task{%
    При нормальном падении белого света на дифракционную решетку зелёная линия ($500\,\text{нм}$)
    в спектре второго порядка видна под углом дифракции $25\degrees$.
    Определить число штрихов на $1\,\text{мм}$ длины этой решетки.
}
\solutionspace{100pt}

\tasknumber{6}%
\task{%
    Каков наибольший порядок спектра, который можно наблюдать при дифракции света
    с длиной волны $\lambda$, на дифракционной решетке с периодом $d =  2{,}5 \lambda$?
}
\solutionspace{100pt}

\tasknumber{7}%
\task{%
    Вертикально стоящий шест высотой 1,1 м, освещенный солнцем,
    отбрасывает на горизонтальную поверхность земли тень длиной $1\,\text{м}$.
    Известно, что длина тени от телеграфного столба на $6\,\text{м}$ больше.
    Определить высоту столба.
}
\solutionspace{180pt}

\tasknumber{8}%
\task{%
    Определить абсолютный показатель преломления прозрачной среды,
    в которой распространяется свет с длиной волны $0{,}500\,\text{мкм}$ и частотой $600\,\text{ТГц}$.
    Скорость света в вакууме $3 \cdot 10^{8}\,\frac{\text{м}}{\text{с}}$.
}
\solutionspace{180pt}

\tasknumber{9}%
\task{%
    На дне водоема глубиной $2\,\text{м}$ лежит зеркало.
    Луч света, пройдя через воду, отражается от зеркала и выходит из воды.
    Найти расстояние между точкой входа луча в воду и точкой выхода луча из воды,
    если показатель преломления воды $1{,}33$, а угол падения луча $30\degrees$.
}
\solutionspace{180pt}

\tasknumber{10}%
\task{%
    Луч света падает на горизонтально расположенную стеклянную пластинку толщиной $4\,\text{мм}$.
    Пройдя через пластину, он выходит из неё в точке, смещённой по горизонтали от точки падения на расстояние $1{,}2\,\text{мм}$
    Показатель преломления стекла $1{,}4$.
    Найти синус угла падения, округлив его значение до двух знаков после запятой.
}
\solutionspace{180pt}

\tasknumber{11}%
\task{%
    Найти оптическую силу собирающей линзы, если действительное изображение предмета,
    помещённого в $35\,\text{см}$ от линзы, получается на расстоянии $40\,\text{см}$ от неё.
}
\solutionspace{180pt}

\tasknumber{12}%
\task{%
    Найти увеличение изображения, если изображение предмета, находящегося
    на расстоянии $15\,\text{см}$ от линзы, получается на расстоянии $30\,\text{см}$ от неё.
}
\solutionspace{180pt}

\tasknumber{13}%
\task{%
    Расстояние от предмета до линзы $8\,\text{см}$, а от линзы до мнимого изображения $30\,\text{см}$.
    Чему равно фокусное расстояние линзы?
}
\solutionspace{180pt}

\tasknumber{14}%
\task{%
    Две тонкие линзы с фокусными расстояниями $18\,\text{см}$ и $20\,\text{см}$ сложены вместе.
    Чему равно фокусное расстояние такой оптической системы?
}
\solutionspace{180pt}

\tasknumber{15}%
\task{%
    Линейные размеры прямого изображения предмета, полученного в собирающей линзе,
    в три раза больше линейных размеров предмета.
    Зная, что предмет находится на $30\,\text{см}$ ближе к линзе,
    чем его изображение, найти оптическую силу линзы.
}
\solutionspace{180pt}

\tasknumber{16}%
\task{%
    Оптическая сила объектива фотоаппарата равна $3\,\text{дптр}$.
    При фотографировании чертежа с расстояния $0{,}8\,\text{м}$ площадь изображения
    чертежа на фотопластинке оказалась равной $4\,\text{см}^{2}$.
    Какова площадь самого чертежа? Ответ выразите в квадратных сантиметрах.
}

\variantsplitter

\addpersonalvariant{Ирина Ан}

\tasknumber{1}%
\task{%
    Разность фаз двух интерферирующих световых волн равна $9\pi$, а разность хода между ними равна $7{,}5 \cdot 10^{-7}\,\text{м}$.
    Определить длину волны.
}
\solutionspace{100pt}

\tasknumber{2}%
\task{%
    Расстояние между двумя точечными когерентными источниками света $S_1$ u $S_2$ равно $2\,\text{мм}$.
    Источники расположены в плоскости, параллельной экрану, на расстоянии $8\,\text{м}$ от него.
    На экране в точках, лежащих на перпендикулярах, опущенных из источников света $S_1$ и $S_2$,
    находятся два ближайших минимума (тёмные полосы).
    Определите длину световой волны.
    Ответ дать в нанометрах.
}
\solutionspace{100pt}

\tasknumber{3}%
\task{%
    На дифракционную решетку, имеющую период $3 \cdot 10^{-4}\,\text{см}$, нормально падает монохроматическая световая волна.
    Под углом $ 40 \degrees$ наблюдается дифракционный максимум второго порядка.
    Какова длина волны падающего света?
}
\solutionspace{100pt}

\tasknumber{4}%
\task{%
    Свет с длиной волны $0{,}4\,\text{мкм}$ падает нормально на дифракционную решетку с периодом, равным $1\,\text{мкм}$.
    Под каким углом наблюдается дифракционный максимум первого порядка?
}
\solutionspace{100pt}

\tasknumber{5}%
\task{%
    При нормальном падении белого света на дифракционную решетку зелёная линия ($500\,\text{нм}$)
    в спектре второго порядка видна под углом дифракции $20\degrees$.
    Определить число штрихов на $1\,\text{см}$ длины этой решетки.
}
\solutionspace{100pt}

\tasknumber{6}%
\task{%
    Каков наибольший порядок спектра, который можно наблюдать при дифракции света
    с длиной волны $\lambda$, на дифракционной решетке с периодом $d =  2{,}5 \lambda$?
}
\solutionspace{100pt}

\tasknumber{7}%
\task{%
    Вертикально стоящий шест высотой 1,1 м, освещенный солнцем,
    отбрасывает на горизонтальную поверхность земли тень длиной $2\,\text{м}$.
    Известно, что длина тени от телеграфного столба на $9\,\text{м}$ больше.
    Определить высоту столба.
}
\solutionspace{180pt}

\tasknumber{8}%
\task{%
    Определить абсолютный показатель преломления прозрачной среды,
    в которой распространяется свет с длиной волны $0{,}500\,\text{мкм}$ и частотой $600\,\text{ТГц}$.
    Скорость света в вакууме $3 \cdot 10^{8}\,\frac{\text{м}}{\text{с}}$.
}
\solutionspace{180pt}

\tasknumber{9}%
\task{%
    На дне водоема глубиной $3\,\text{м}$ лежит зеркало.
    Луч света, пройдя через воду, отражается от зеркала и выходит из воды.
    Найти расстояние между точкой входа луча в воду и точкой выхода луча из воды,
    если показатель преломления воды $1{,}33$, а угол падения луча $35\degrees$.
}
\solutionspace{180pt}

\tasknumber{10}%
\task{%
    Луч света падает на горизонтально расположенную стеклянную пластинку толщиной $4\,\text{мм}$.
    Пройдя через пластину, он выходит из неё в точке, смещённой по горизонтали от точки падения на расстояние $1{,}6\,\text{мм}$
    Показатель преломления стекла $1{,}5$.
    Найти синус угла падения, округлив его значение до двух знаков после запятой.
}
\solutionspace{180pt}

\tasknumber{11}%
\task{%
    Найти оптическую силу собирающей линзы, если действительное изображение предмета,
    помещённого в $35\,\text{см}$ от линзы, получается на расстоянии $20\,\text{см}$ от неё.
}
\solutionspace{180pt}

\tasknumber{12}%
\task{%
    Найти увеличение изображения, если изображение предмета, находящегося
    на расстоянии $20\,\text{см}$ от линзы, получается на расстоянии $30\,\text{см}$ от неё.
}
\solutionspace{180pt}

\tasknumber{13}%
\task{%
    Расстояние от предмета до линзы $10\,\text{см}$, а от линзы до мнимого изображения $30\,\text{см}$.
    Чему равно фокусное расстояние линзы?
}
\solutionspace{180pt}

\tasknumber{14}%
\task{%
    Две тонкие линзы с фокусными расстояниями $18\,\text{см}$ и $20\,\text{см}$ сложены вместе.
    Чему равно фокусное расстояние такой оптической системы?
}
\solutionspace{180pt}

\tasknumber{15}%
\task{%
    Линейные размеры прямого изображения предмета, полученного в собирающей линзе,
    в три раза больше линейных размеров предмета.
    Зная, что предмет находится на $20\,\text{см}$ ближе к линзе,
    чем его изображение, найти оптическую силу линзы.
}
\solutionspace{180pt}

\tasknumber{16}%
\task{%
    Оптическая сила объектива фотоаппарата равна $4\,\text{дптр}$.
    При фотографировании чертежа с расстояния $0{,}9\,\text{м}$ площадь изображения
    чертежа на фотопластинке оказалась равной $16\,\text{см}^{2}$.
    Какова площадь самого чертежа? Ответ выразите в квадратных сантиметрах.
}

\variantsplitter

\addpersonalvariant{Софья Андрианова}

\tasknumber{1}%
\task{%
    Разность фаз двух интерферирующих световых волн равна $7\pi$, а разность хода между ними равна $10{,}5 \cdot 10^{-7}\,\text{м}$.
    Определить длину волны.
}
\solutionspace{100pt}

\tasknumber{2}%
\task{%
    Расстояние между двумя точечными когерентными источниками света $S_1$ u $S_2$ равно $1{,}5\,\text{мм}$.
    Источники расположены в плоскости, параллельной экрану, на расстоянии $8\,\text{м}$ от него.
    На экране в точках, лежащих на перпендикулярах, опущенных из источников света $S_1$ и $S_2$,
    находятся два ближайших минимума (тёмные полосы).
    Определите длину световой волны.
    Ответ дать в нанометрах.
}
\solutionspace{100pt}

\tasknumber{3}%
\task{%
    На дифракционную решетку, имеющую период $3 \cdot 10^{-4}\,\text{см}$, нормально падает монохроматическая световая волна.
    Под углом $ 35 \degrees$ наблюдается дифракционный максимум второго порядка.
    Какова длина волны падающего света?
}
\solutionspace{100pt}

\tasknumber{4}%
\task{%
    Свет с длиной волны $0{,}7\,\text{мкм}$ падает нормально на дифракционную решетку с периодом, равным $3\,\text{мкм}$.
    Под каким углом наблюдается дифракционный максимум первого порядка?
}
\solutionspace{100pt}

\tasknumber{5}%
\task{%
    При нормальном падении белого света на дифракционную решетку зелёная линия ($500\,\text{нм}$)
    в спектре второго порядка видна под углом дифракции $25\degrees$.
    Определить число штрихов на $1\,\text{мм}$ длины этой решетки.
}
\solutionspace{100pt}

\tasknumber{6}%
\task{%
    Каков наибольший порядок спектра, который можно наблюдать при дифракции света
    с длиной волны $\lambda$, на дифракционной решетке с периодом $d =  3{,}5 \lambda$?
}
\solutionspace{100pt}

\tasknumber{7}%
\task{%
    Вертикально стоящий шест высотой 1,1 м, освещенный солнцем,
    отбрасывает на горизонтальную поверхность земли тень длиной $3\,\text{м}$.
    Известно, что длина тени от телеграфного столба на $9\,\text{м}$ больше.
    Определить высоту столба.
}
\solutionspace{180pt}

\tasknumber{8}%
\task{%
    Определить абсолютный показатель преломления прозрачной среды,
    в которой распространяется свет с длиной волны $0{,}650\,\text{мкм}$ и частотой $462\,\text{ТГц}$.
    Скорость света в вакууме $3 \cdot 10^{8}\,\frac{\text{м}}{\text{с}}$.
}
\solutionspace{180pt}

\tasknumber{9}%
\task{%
    На дне водоема глубиной $4\,\text{м}$ лежит зеркало.
    Луч света, пройдя через воду, отражается от зеркала и выходит из воды.
    Найти расстояние между точкой входа луча в воду и точкой выхода луча из воды,
    если показатель преломления воды $1{,}33$, а угол падения луча $35\degrees$.
}
\solutionspace{180pt}

\tasknumber{10}%
\task{%
    Луч света падает на горизонтально расположенную стеклянную пластинку толщиной $5\,\text{мм}$.
    Пройдя через пластину, он выходит из неё в точке, смещённой по горизонтали от точки падения на расстояние $1{,}4\,\text{мм}$
    Показатель преломления стекла $1{,}5$.
    Найти синус угла падения, округлив его значение до двух знаков после запятой.
}
\solutionspace{180pt}

\tasknumber{11}%
\task{%
    Найти оптическую силу собирающей линзы, если действительное изображение предмета,
    помещённого в $55\,\text{см}$ от линзы, получается на расстоянии $40\,\text{см}$ от неё.
}
\solutionspace{180pt}

\tasknumber{12}%
\task{%
    Найти увеличение изображения, если изображение предмета, находящегося
    на расстоянии $25\,\text{см}$ от линзы, получается на расстоянии $18\,\text{см}$ от неё.
}
\solutionspace{180pt}

\tasknumber{13}%
\task{%
    Расстояние от предмета до линзы $12\,\text{см}$, а от линзы до мнимого изображения $25\,\text{см}$.
    Чему равно фокусное расстояние линзы?
}
\solutionspace{180pt}

\tasknumber{14}%
\task{%
    Две тонкие линзы с фокусными расстояниями $18\,\text{см}$ и $20\,\text{см}$ сложены вместе.
    Чему равно фокусное расстояние такой оптической системы?
}
\solutionspace{180pt}

\tasknumber{15}%
\task{%
    Линейные размеры прямого изображения предмета, полученного в собирающей линзе,
    в четыре раза больше линейных размеров предмета.
    Зная, что предмет находится на $35\,\text{см}$ ближе к линзе,
    чем его изображение, найти оптическую силу линзы.
}
\solutionspace{180pt}

\tasknumber{16}%
\task{%
    Оптическая сила объектива фотоаппарата равна $6\,\text{дптр}$.
    При фотографировании чертежа с расстояния $0{,}9\,\text{м}$ площадь изображения
    чертежа на фотопластинке оказалась равной $4\,\text{см}^{2}$.
    Какова площадь самого чертежа? Ответ выразите в квадратных сантиметрах.
}

\variantsplitter

\addpersonalvariant{Владимир Артемчук}

\tasknumber{1}%
\task{%
    Разность фаз двух интерферирующих световых волн равна $7\pi$, а разность хода между ними равна $15{,}5 \cdot 10^{-7}\,\text{м}$.
    Определить длину волны.
}
\solutionspace{100pt}

\tasknumber{2}%
\task{%
    Расстояние между двумя точечными когерентными источниками света $S_1$ u $S_2$ равно $2{,}5\,\text{мм}$.
    Источники расположены в плоскости, параллельной экрану, на расстоянии $7\,\text{м}$ от него.
    На экране в точках, лежащих на перпендикулярах, опущенных из источников света $S_1$ и $S_2$,
    находятся два ближайших минимума (тёмные полосы).
    Определите длину световой волны.
    Ответ дать в нанометрах.
}
\solutionspace{100pt}

\tasknumber{3}%
\task{%
    На дифракционную решетку, имеющую период $4 \cdot 10^{-4}\,\text{см}$, нормально падает монохроматическая световая волна.
    Под углом $ 35 \degrees$ наблюдается дифракционный максимум второго порядка.
    Какова длина волны падающего света?
}
\solutionspace{100pt}

\tasknumber{4}%
\task{%
    Свет с длиной волны $0{,}6\,\text{мкм}$ падает нормально на дифракционную решетку с периодом, равным $1\,\text{мкм}$.
    Под каким углом наблюдается дифракционный максимум первого порядка?
}
\solutionspace{100pt}

\tasknumber{5}%
\task{%
    При нормальном падении белого света на дифракционную решетку зелёная линия ($500\,\text{нм}$)
    в спектре второго порядка видна под углом дифракции $53\degrees$.
    Определить число штрихов на $1\,\text{см}$ длины этой решетки.
}
\solutionspace{100pt}

\tasknumber{6}%
\task{%
    Каков наибольший порядок спектра, который можно наблюдать при дифракции света
    с длиной волны $\lambda$, на дифракционной решетке с периодом $d =  4{,}5 \lambda$?
}
\solutionspace{100pt}

\tasknumber{7}%
\task{%
    Вертикально стоящий шест высотой 1,1 м, освещенный солнцем,
    отбрасывает на горизонтальную поверхность земли тень длиной $1\,\text{м}$.
    Известно, что длина тени от телеграфного столба на $7\,\text{м}$ больше.
    Определить высоту столба.
}
\solutionspace{180pt}

\tasknumber{8}%
\task{%
    Определить абсолютный показатель преломления прозрачной среды,
    в которой распространяется свет с длиной волны $0{,}650\,\text{мкм}$ и частотой $462\,\text{ТГц}$.
    Скорость света в вакууме $3 \cdot 10^{8}\,\frac{\text{м}}{\text{с}}$.
}
\solutionspace{180pt}

\tasknumber{9}%
\task{%
    На дне водоема глубиной $3\,\text{м}$ лежит зеркало.
    Луч света, пройдя через воду, отражается от зеркала и выходит из воды.
    Найти расстояние между точкой входа луча в воду и точкой выхода луча из воды,
    если показатель преломления воды $1{,}33$, а угол падения луча $25\degrees$.
}
\solutionspace{180pt}

\tasknumber{10}%
\task{%
    Луч света падает на горизонтально расположенную стеклянную пластинку толщиной $6\,\text{мм}$.
    Пройдя через пластину, он выходит из неё в точке, смещённой по горизонтали от точки падения на расстояние $1{,}2\,\text{мм}$
    Показатель преломления стекла $1{,}5$.
    Найти синус угла падения, округлив его значение до двух знаков после запятой.
}
\solutionspace{180pt}

\tasknumber{11}%
\task{%
    Найти оптическую силу собирающей линзы, если действительное изображение предмета,
    помещённого в $15\,\text{см}$ от линзы, получается на расстоянии $20\,\text{см}$ от неё.
}
\solutionspace{180pt}

\tasknumber{12}%
\task{%
    Найти увеличение изображения, если изображение предмета, находящегося
    на расстоянии $15\,\text{см}$ от линзы, получается на расстоянии $18\,\text{см}$ от неё.
}
\solutionspace{180pt}

\tasknumber{13}%
\task{%
    Расстояние от предмета до линзы $8\,\text{см}$, а от линзы до мнимого изображения $25\,\text{см}$.
    Чему равно фокусное расстояние линзы?
}
\solutionspace{180pt}

\tasknumber{14}%
\task{%
    Две тонкие линзы с фокусными расстояниями $12\,\text{см}$ и $20\,\text{см}$ сложены вместе.
    Чему равно фокусное расстояние такой оптической системы?
}
\solutionspace{180pt}

\tasknumber{15}%
\task{%
    Линейные размеры прямого изображения предмета, полученного в собирающей линзе,
    в четыре раза больше линейных размеров предмета.
    Зная, что предмет находится на $25\,\text{см}$ ближе к линзе,
    чем его изображение, найти оптическую силу линзы.
}
\solutionspace{180pt}

\tasknumber{16}%
\task{%
    Оптическая сила объектива фотоаппарата равна $5\,\text{дптр}$.
    При фотографировании чертежа с расстояния $1{,}2\,\text{м}$ площадь изображения
    чертежа на фотопластинке оказалась равной $16\,\text{см}^{2}$.
    Какова площадь самого чертежа? Ответ выразите в квадратных сантиметрах.
}

\variantsplitter

\addpersonalvariant{Софья Белянкина}

\tasknumber{1}%
\task{%
    Разность фаз двух интерферирующих световых волн равна $3\pi$, а разность хода между ними равна $10{,}5 \cdot 10^{-7}\,\text{м}$.
    Определить длину волны.
}
\solutionspace{100pt}

\tasknumber{2}%
\task{%
    Расстояние между двумя точечными когерентными источниками света $S_1$ u $S_2$ равно $1{,}5\,\text{мм}$.
    Источники расположены в плоскости, параллельной экрану, на расстоянии $8\,\text{м}$ от него.
    На экране в точках, лежащих на перпендикулярах, опущенных из источников света $S_1$ и $S_2$,
    находятся два ближайших минимума (тёмные полосы).
    Определите длину световой волны.
    Ответ дать в нанометрах.
}
\solutionspace{100pt}

\tasknumber{3}%
\task{%
    На дифракционную решетку, имеющую период $2 \cdot 10^{-4}\,\text{см}$, нормально падает монохроматическая световая волна.
    Под углом $ 25 \degrees$ наблюдается дифракционный максимум второго порядка.
    Какова длина волны падающего света?
}
\solutionspace{100pt}

\tasknumber{4}%
\task{%
    Свет с длиной волны $0{,}4\,\text{мкм}$ падает нормально на дифракционную решетку с периодом, равным $1\,\text{мкм}$.
    Под каким углом наблюдается дифракционный максимум первого порядка?
}
\solutionspace{100pt}

\tasknumber{5}%
\task{%
    При нормальном падении белого света на дифракционную решетку зелёная линия ($500\,\text{нм}$)
    в спектре второго порядка видна под углом дифракции $20\degrees$.
    Определить число штрихов на $1\,\text{мм}$ длины этой решетки.
}
\solutionspace{100pt}

\tasknumber{6}%
\task{%
    Каков наибольший порядок спектра, который можно наблюдать при дифракции света
    с длиной волны $\lambda$, на дифракционной решетке с периодом $d =  4{,}5 \lambda$?
}
\solutionspace{100pt}

\tasknumber{7}%
\task{%
    Вертикально стоящий шест высотой 1,1 м, освещенный солнцем,
    отбрасывает на горизонтальную поверхность земли тень длиной $2\,\text{м}$.
    Известно, что длина тени от телеграфного столба на $7\,\text{м}$ больше.
    Определить высоту столба.
}
\solutionspace{180pt}

\tasknumber{8}%
\task{%
    Определить абсолютный показатель преломления прозрачной среды,
    в которой распространяется свет с длиной волны $0{,}650\,\text{мкм}$ и частотой $462\,\text{ТГц}$.
    Скорость света в вакууме $3 \cdot 10^{8}\,\frac{\text{м}}{\text{с}}$.
}
\solutionspace{180pt}

\tasknumber{9}%
\task{%
    На дне водоема глубиной $4\,\text{м}$ лежит зеркало.
    Луч света, пройдя через воду, отражается от зеркала и выходит из воды.
    Найти расстояние между точкой входа луча в воду и точкой выхода луча из воды,
    если показатель преломления воды $1{,}33$, а угол падения луча $30\degrees$.
}
\solutionspace{180pt}

\tasknumber{10}%
\task{%
    Луч света падает на горизонтально расположенную стеклянную пластинку толщиной $4\,\text{мм}$.
    Пройдя через пластину, он выходит из неё в точке, смещённой по горизонтали от точки падения на расстояние $1{,}2\,\text{мм}$
    Показатель преломления стекла $1{,}4$.
    Найти синус угла падения, округлив его значение до двух знаков после запятой.
}
\solutionspace{180pt}

\tasknumber{11}%
\task{%
    Найти оптическую силу собирающей линзы, если действительное изображение предмета,
    помещённого в $55\,\text{см}$ от линзы, получается на расстоянии $20\,\text{см}$ от неё.
}
\solutionspace{180pt}

\tasknumber{12}%
\task{%
    Найти увеличение изображения, если изображение предмета, находящегося
    на расстоянии $15\,\text{см}$ от линзы, получается на расстоянии $18\,\text{см}$ от неё.
}
\solutionspace{180pt}

\tasknumber{13}%
\task{%
    Расстояние от предмета до линзы $8\,\text{см}$, а от линзы до мнимого изображения $25\,\text{см}$.
    Чему равно фокусное расстояние линзы?
}
\solutionspace{180pt}

\tasknumber{14}%
\task{%
    Две тонкие линзы с фокусными расстояниями $18\,\text{см}$ и $20\,\text{см}$ сложены вместе.
    Чему равно фокусное расстояние такой оптической системы?
}
\solutionspace{180pt}

\tasknumber{15}%
\task{%
    Линейные размеры прямого изображения предмета, полученного в собирающей линзе,
    в два раза больше линейных размеров предмета.
    Зная, что предмет находится на $35\,\text{см}$ ближе к линзе,
    чем его изображение, найти оптическую силу линзы.
}
\solutionspace{180pt}

\tasknumber{16}%
\task{%
    Оптическая сила объектива фотоаппарата равна $3\,\text{дптр}$.
    При фотографировании чертежа с расстояния $1{,}1\,\text{м}$ площадь изображения
    чертежа на фотопластинке оказалась равной $9\,\text{см}^{2}$.
    Какова площадь самого чертежа? Ответ выразите в квадратных сантиметрах.
}

\variantsplitter

\addpersonalvariant{Варвара Егиазарян}

\tasknumber{1}%
\task{%
    Разность фаз двух интерферирующих световых волн равна $7\pi$, а разность хода между ними равна $12{,}5 \cdot 10^{-7}\,\text{м}$.
    Определить длину волны.
}
\solutionspace{100pt}

\tasknumber{2}%
\task{%
    Расстояние между двумя точечными когерентными источниками света $S_1$ u $S_2$ равно $1{,}5\,\text{мм}$.
    Источники расположены в плоскости, параллельной экрану, на расстоянии $9\,\text{м}$ от него.
    На экране в точках, лежащих на перпендикулярах, опущенных из источников света $S_1$ и $S_2$,
    находятся два ближайших минимума (тёмные полосы).
    Определите длину световой волны.
    Ответ дать в нанометрах.
}
\solutionspace{100pt}

\tasknumber{3}%
\task{%
    На дифракционную решетку, имеющую период $3 \cdot 10^{-4}\,\text{см}$, нормально падает монохроматическая световая волна.
    Под углом $ 30 \degrees$ наблюдается дифракционный максимум второго порядка.
    Какова длина волны падающего света?
}
\solutionspace{100pt}

\tasknumber{4}%
\task{%
    Свет с длиной волны $0{,}5\,\text{мкм}$ падает нормально на дифракционную решетку с периодом, равным $1\,\text{мкм}$.
    Под каким углом наблюдается дифракционный максимум первого порядка?
}
\solutionspace{100pt}

\tasknumber{5}%
\task{%
    При нормальном падении белого света на дифракционную решетку зелёная линия ($500\,\text{нм}$)
    в спектре второго порядка видна под углом дифракции $53\degrees$.
    Определить число штрихов на $1\,\text{мм}$ длины этой решетки.
}
\solutionspace{100pt}

\tasknumber{6}%
\task{%
    Каков наибольший порядок спектра, который можно наблюдать при дифракции света
    с длиной волны $\lambda$, на дифракционной решетке с периодом $d =  3{,}5 \lambda$?
}
\solutionspace{100pt}

\tasknumber{7}%
\task{%
    Вертикально стоящий шест высотой 1,1 м, освещенный солнцем,
    отбрасывает на горизонтальную поверхность земли тень длиной $4\,\text{м}$.
    Известно, что длина тени от телеграфного столба на $5\,\text{м}$ больше.
    Определить высоту столба.
}
\solutionspace{180pt}

\tasknumber{8}%
\task{%
    Определить абсолютный показатель преломления прозрачной среды,
    в которой распространяется свет с длиной волны $0{,}500\,\text{мкм}$ и частотой $600\,\text{ТГц}$.
    Скорость света в вакууме $3 \cdot 10^{8}\,\frac{\text{м}}{\text{с}}$.
}
\solutionspace{180pt}

\tasknumber{9}%
\task{%
    На дне водоема глубиной $2\,\text{м}$ лежит зеркало.
    Луч света, пройдя через воду, отражается от зеркала и выходит из воды.
    Найти расстояние между точкой входа луча в воду и точкой выхода луча из воды,
    если показатель преломления воды $1{,}33$, а угол падения луча $30\degrees$.
}
\solutionspace{180pt}

\tasknumber{10}%
\task{%
    Луч света падает на горизонтально расположенную стеклянную пластинку толщиной $6\,\text{мм}$.
    Пройдя через пластину, он выходит из неё в точке, смещённой по горизонтали от точки падения на расстояние $1{,}4\,\text{мм}$
    Показатель преломления стекла $1{,}4$.
    Найти синус угла падения, округлив его значение до двух знаков после запятой.
}
\solutionspace{180pt}

\tasknumber{11}%
\task{%
    Найти оптическую силу собирающей линзы, если действительное изображение предмета,
    помещённого в $35\,\text{см}$ от линзы, получается на расстоянии $20\,\text{см}$ от неё.
}
\solutionspace{180pt}

\tasknumber{12}%
\task{%
    Найти увеличение изображения, если изображение предмета, находящегося
    на расстоянии $20\,\text{см}$ от линзы, получается на расстоянии $30\,\text{см}$ от неё.
}
\solutionspace{180pt}

\tasknumber{13}%
\task{%
    Расстояние от предмета до линзы $10\,\text{см}$, а от линзы до мнимого изображения $30\,\text{см}$.
    Чему равно фокусное расстояние линзы?
}
\solutionspace{180pt}

\tasknumber{14}%
\task{%
    Две тонкие линзы с фокусными расстояниями $18\,\text{см}$ и $20\,\text{см}$ сложены вместе.
    Чему равно фокусное расстояние такой оптической системы?
}
\solutionspace{180pt}

\tasknumber{15}%
\task{%
    Линейные размеры прямого изображения предмета, полученного в собирающей линзе,
    в два раза больше линейных размеров предмета.
    Зная, что предмет находится на $35\,\text{см}$ ближе к линзе,
    чем его изображение, найти оптическую силу линзы.
}
\solutionspace{180pt}

\tasknumber{16}%
\task{%
    Оптическая сила объектива фотоаппарата равна $5\,\text{дптр}$.
    При фотографировании чертежа с расстояния $1{,}2\,\text{м}$ площадь изображения
    чертежа на фотопластинке оказалась равной $16\,\text{см}^{2}$.
    Какова площадь самого чертежа? Ответ выразите в квадратных сантиметрах.
}

\variantsplitter

\addpersonalvariant{Владислав Емелин}

\tasknumber{1}%
\task{%
    Разность фаз двух интерферирующих световых волн равна $5\pi$, а разность хода между ними равна $10{,}5 \cdot 10^{-7}\,\text{м}$.
    Определить длину волны.
}
\solutionspace{100pt}

\tasknumber{2}%
\task{%
    Расстояние между двумя точечными когерентными источниками света $S_1$ u $S_2$ равно $2{,}5\,\text{мм}$.
    Источники расположены в плоскости, параллельной экрану, на расстоянии $8\,\text{м}$ от него.
    На экране в точках, лежащих на перпендикулярах, опущенных из источников света $S_1$ и $S_2$,
    находятся два ближайших минимума (тёмные полосы).
    Определите длину световой волны.
    Ответ дать в нанометрах.
}
\solutionspace{100pt}

\tasknumber{3}%
\task{%
    На дифракционную решетку, имеющую период $4 \cdot 10^{-4}\,\text{см}$, нормально падает монохроматическая световая волна.
    Под углом $ 40 \degrees$ наблюдается дифракционный максимум второго порядка.
    Какова длина волны падающего света?
}
\solutionspace{100pt}

\tasknumber{4}%
\task{%
    Свет с длиной волны $0{,}6\,\text{мкм}$ падает нормально на дифракционную решетку с периодом, равным $3\,\text{мкм}$.
    Под каким углом наблюдается дифракционный максимум первого порядка?
}
\solutionspace{100pt}

\tasknumber{5}%
\task{%
    При нормальном падении белого света на дифракционную решетку зелёная линия ($500\,\text{нм}$)
    в спектре второго порядка видна под углом дифракции $25\degrees$.
    Определить число штрихов на $1\,\text{см}$ длины этой решетки.
}
\solutionspace{100pt}

\tasknumber{6}%
\task{%
    Каков наибольший порядок спектра, который можно наблюдать при дифракции света
    с длиной волны $\lambda$, на дифракционной решетке с периодом $d =  2{,}5 \lambda$?
}
\solutionspace{100pt}

\tasknumber{7}%
\task{%
    Вертикально стоящий шест высотой 1,1 м, освещенный солнцем,
    отбрасывает на горизонтальную поверхность земли тень длиной $3\,\text{м}$.
    Известно, что длина тени от телеграфного столба на $5\,\text{м}$ больше.
    Определить высоту столба.
}
\solutionspace{180pt}

\tasknumber{8}%
\task{%
    Определить абсолютный показатель преломления прозрачной среды,
    в которой распространяется свет с длиной волны $0{,}500\,\text{мкм}$ и частотой $600\,\text{ТГц}$.
    Скорость света в вакууме $3 \cdot 10^{8}\,\frac{\text{м}}{\text{с}}$.
}
\solutionspace{180pt}

\tasknumber{9}%
\task{%
    На дне водоема глубиной $3\,\text{м}$ лежит зеркало.
    Луч света, пройдя через воду, отражается от зеркала и выходит из воды.
    Найти расстояние между точкой входа луча в воду и точкой выхода луча из воды,
    если показатель преломления воды $1{,}33$, а угол падения луча $30\degrees$.
}
\solutionspace{180pt}

\tasknumber{10}%
\task{%
    Луч света падает на горизонтально расположенную стеклянную пластинку толщиной $5\,\text{мм}$.
    Пройдя через пластину, он выходит из неё в точке, смещённой по горизонтали от точки падения на расстояние $1{,}5\,\text{мм}$
    Показатель преломления стекла $1{,}5$.
    Найти синус угла падения, округлив его значение до двух знаков после запятой.
}
\solutionspace{180pt}

\tasknumber{11}%
\task{%
    Найти оптическую силу собирающей линзы, если действительное изображение предмета,
    помещённого в $15\,\text{см}$ от линзы, получается на расстоянии $30\,\text{см}$ от неё.
}
\solutionspace{180pt}

\tasknumber{12}%
\task{%
    Найти увеличение изображения, если изображение предмета, находящегося
    на расстоянии $20\,\text{см}$ от линзы, получается на расстоянии $12\,\text{см}$ от неё.
}
\solutionspace{180pt}

\tasknumber{13}%
\task{%
    Расстояние от предмета до линзы $10\,\text{см}$, а от линзы до мнимого изображения $20\,\text{см}$.
    Чему равно фокусное расстояние линзы?
}
\solutionspace{180pt}

\tasknumber{14}%
\task{%
    Две тонкие линзы с фокусными расстояниями $25\,\text{см}$ и $20\,\text{см}$ сложены вместе.
    Чему равно фокусное расстояние такой оптической системы?
}
\solutionspace{180pt}

\tasknumber{15}%
\task{%
    Линейные размеры прямого изображения предмета, полученного в собирающей линзе,
    в два раза больше линейных размеров предмета.
    Зная, что предмет находится на $30\,\text{см}$ ближе к линзе,
    чем его изображение, найти оптическую силу линзы.
}
\solutionspace{180pt}

\tasknumber{16}%
\task{%
    Оптическая сила объектива фотоаппарата равна $6\,\text{дптр}$.
    При фотографировании чертежа с расстояния $1{,}2\,\text{м}$ площадь изображения
    чертежа на фотопластинке оказалась равной $4\,\text{см}^{2}$.
    Какова площадь самого чертежа? Ответ выразите в квадратных сантиметрах.
}

\variantsplitter

\addpersonalvariant{Артём Жичин}

\tasknumber{1}%
\task{%
    Разность фаз двух интерферирующих световых волн равна $3\pi$, а разность хода между ними равна $10{,}5 \cdot 10^{-7}\,\text{м}$.
    Определить длину волны.
}
\solutionspace{100pt}

\tasknumber{2}%
\task{%
    Расстояние между двумя точечными когерентными источниками света $S_1$ u $S_2$ равно $2{,}5\,\text{мм}$.
    Источники расположены в плоскости, параллельной экрану, на расстоянии $8\,\text{м}$ от него.
    На экране в точках, лежащих на перпендикулярах, опущенных из источников света $S_1$ и $S_2$,
    находятся два ближайших минимума (тёмные полосы).
    Определите длину световой волны.
    Ответ дать в нанометрах.
}
\solutionspace{100pt}

\tasknumber{3}%
\task{%
    На дифракционную решетку, имеющую период $2 \cdot 10^{-4}\,\text{см}$, нормально падает монохроматическая световая волна.
    Под углом $ 30 \degrees$ наблюдается дифракционный максимум второго порядка.
    Какова длина волны падающего света?
}
\solutionspace{100pt}

\tasknumber{4}%
\task{%
    Свет с длиной волны $0{,}4\,\text{мкм}$ падает нормально на дифракционную решетку с периодом, равным $1\,\text{мкм}$.
    Под каким углом наблюдается дифракционный максимум первого порядка?
}
\solutionspace{100pt}

\tasknumber{5}%
\task{%
    При нормальном падении белого света на дифракционную решетку зелёная линия ($500\,\text{нм}$)
    в спектре второго порядка видна под углом дифракции $20\degrees$.
    Определить число штрихов на $1\,\text{мм}$ длины этой решетки.
}
\solutionspace{100pt}

\tasknumber{6}%
\task{%
    Каков наибольший порядок спектра, который можно наблюдать при дифракции света
    с длиной волны $\lambda$, на дифракционной решетке с периодом $d =  4{,}5 \lambda$?
}
\solutionspace{100pt}

\tasknumber{7}%
\task{%
    Вертикально стоящий шест высотой 1,1 м, освещенный солнцем,
    отбрасывает на горизонтальную поверхность земли тень длиной $4\,\text{м}$.
    Известно, что длина тени от телеграфного столба на $6\,\text{м}$ больше.
    Определить высоту столба.
}
\solutionspace{180pt}

\tasknumber{8}%
\task{%
    Определить абсолютный показатель преломления прозрачной среды,
    в которой распространяется свет с длиной волны $0{,}550\,\text{мкм}$ и частотой $545\,\text{ТГц}$.
    Скорость света в вакууме $3 \cdot 10^{8}\,\frac{\text{м}}{\text{с}}$.
}
\solutionspace{180pt}

\tasknumber{9}%
\task{%
    На дне водоема глубиной $3\,\text{м}$ лежит зеркало.
    Луч света, пройдя через воду, отражается от зеркала и выходит из воды.
    Найти расстояние между точкой входа луча в воду и точкой выхода луча из воды,
    если показатель преломления воды $1{,}33$, а угол падения луча $35\degrees$.
}
\solutionspace{180pt}

\tasknumber{10}%
\task{%
    Луч света падает на горизонтально расположенную стеклянную пластинку толщиной $5\,\text{мм}$.
    Пройдя через пластину, он выходит из неё в точке, смещённой по горизонтали от точки падения на расстояние $1{,}4\,\text{мм}$
    Показатель преломления стекла $1{,}6$.
    Найти синус угла падения, округлив его значение до двух знаков после запятой.
}
\solutionspace{180pt}

\tasknumber{11}%
\task{%
    Найти оптическую силу собирающей линзы, если действительное изображение предмета,
    помещённого в $35\,\text{см}$ от линзы, получается на расстоянии $40\,\text{см}$ от неё.
}
\solutionspace{180pt}

\tasknumber{12}%
\task{%
    Найти увеличение изображения, если изображение предмета, находящегося
    на расстоянии $15\,\text{см}$ от линзы, получается на расстоянии $18\,\text{см}$ от неё.
}
\solutionspace{180pt}

\tasknumber{13}%
\task{%
    Расстояние от предмета до линзы $8\,\text{см}$, а от линзы до мнимого изображения $25\,\text{см}$.
    Чему равно фокусное расстояние линзы?
}
\solutionspace{180pt}

\tasknumber{14}%
\task{%
    Две тонкие линзы с фокусными расстояниями $25\,\text{см}$ и $30\,\text{см}$ сложены вместе.
    Чему равно фокусное расстояние такой оптической системы?
}
\solutionspace{180pt}

\tasknumber{15}%
\task{%
    Линейные размеры прямого изображения предмета, полученного в собирающей линзе,
    в четыре раза больше линейных размеров предмета.
    Зная, что предмет находится на $25\,\text{см}$ ближе к линзе,
    чем его изображение, найти оптическую силу линзы.
}
\solutionspace{180pt}

\tasknumber{16}%
\task{%
    Оптическая сила объектива фотоаппарата равна $3\,\text{дптр}$.
    При фотографировании чертежа с расстояния $0{,}8\,\text{м}$ площадь изображения
    чертежа на фотопластинке оказалась равной $9\,\text{см}^{2}$.
    Какова площадь самого чертежа? Ответ выразите в квадратных сантиметрах.
}

\variantsplitter

\addpersonalvariant{Дарья Кошман}

\tasknumber{1}%
\task{%
    Разность фаз двух интерферирующих световых волн равна $9\pi$, а разность хода между ними равна $12{,}5 \cdot 10^{-7}\,\text{м}$.
    Определить длину волны.
}
\solutionspace{100pt}

\tasknumber{2}%
\task{%
    Расстояние между двумя точечными когерентными источниками света $S_1$ u $S_2$ равно $2\,\text{мм}$.
    Источники расположены в плоскости, параллельной экрану, на расстоянии $8\,\text{м}$ от него.
    На экране в точках, лежащих на перпендикулярах, опущенных из источников света $S_1$ и $S_2$,
    находятся два ближайших минимума (тёмные полосы).
    Определите длину световой волны.
    Ответ дать в нанометрах.
}
\solutionspace{100pt}

\tasknumber{3}%
\task{%
    На дифракционную решетку, имеющую период $4 \cdot 10^{-4}\,\text{см}$, нормально падает монохроматическая световая волна.
    Под углом $ 25 \degrees$ наблюдается дифракционный максимум второго порядка.
    Какова длина волны падающего света?
}
\solutionspace{100pt}

\tasknumber{4}%
\task{%
    Свет с длиной волны $0{,}5\,\text{мкм}$ падает нормально на дифракционную решетку с периодом, равным $2\,\text{мкм}$.
    Под каким углом наблюдается дифракционный максимум первого порядка?
}
\solutionspace{100pt}

\tasknumber{5}%
\task{%
    При нормальном падении белого света на дифракционную решетку зелёная линия ($500\,\text{нм}$)
    в спектре второго порядка видна под углом дифракции $20\degrees$.
    Определить число штрихов на $1\,\text{см}$ длины этой решетки.
}
\solutionspace{100pt}

\tasknumber{6}%
\task{%
    Каков наибольший порядок спектра, который можно наблюдать при дифракции света
    с длиной волны $\lambda$, на дифракционной решетке с периодом $d =  2{,}5 \lambda$?
}
\solutionspace{100pt}

\tasknumber{7}%
\task{%
    Вертикально стоящий шест высотой 1,1 м, освещенный солнцем,
    отбрасывает на горизонтальную поверхность земли тень длиной $3\,\text{м}$.
    Известно, что длина тени от телеграфного столба на $7\,\text{м}$ больше.
    Определить высоту столба.
}
\solutionspace{180pt}

\tasknumber{8}%
\task{%
    Определить абсолютный показатель преломления прозрачной среды,
    в которой распространяется свет с длиной волны $0{,}550\,\text{мкм}$ и частотой $545\,\text{ТГц}$.
    Скорость света в вакууме $3 \cdot 10^{8}\,\frac{\text{м}}{\text{с}}$.
}
\solutionspace{180pt}

\tasknumber{9}%
\task{%
    На дне водоема глубиной $2\,\text{м}$ лежит зеркало.
    Луч света, пройдя через воду, отражается от зеркала и выходит из воды.
    Найти расстояние между точкой входа луча в воду и точкой выхода луча из воды,
    если показатель преломления воды $1{,}33$, а угол падения луча $35\degrees$.
}
\solutionspace{180pt}

\tasknumber{10}%
\task{%
    Луч света падает на горизонтально расположенную стеклянную пластинку толщиной $4\,\text{мм}$.
    Пройдя через пластину, он выходит из неё в точке, смещённой по горизонтали от точки падения на расстояние $1{,}2\,\text{мм}$
    Показатель преломления стекла $1{,}5$.
    Найти синус угла падения, округлив его значение до двух знаков после запятой.
}
\solutionspace{180pt}

\tasknumber{11}%
\task{%
    Найти оптическую силу собирающей линзы, если действительное изображение предмета,
    помещённого в $15\,\text{см}$ от линзы, получается на расстоянии $40\,\text{см}$ от неё.
}
\solutionspace{180pt}

\tasknumber{12}%
\task{%
    Найти увеличение изображения, если изображение предмета, находящегося
    на расстоянии $25\,\text{см}$ от линзы, получается на расстоянии $12\,\text{см}$ от неё.
}
\solutionspace{180pt}

\tasknumber{13}%
\task{%
    Расстояние от предмета до линзы $12\,\text{см}$, а от линзы до мнимого изображения $20\,\text{см}$.
    Чему равно фокусное расстояние линзы?
}
\solutionspace{180pt}

\tasknumber{14}%
\task{%
    Две тонкие линзы с фокусными расстояниями $25\,\text{см}$ и $20\,\text{см}$ сложены вместе.
    Чему равно фокусное расстояние такой оптической системы?
}
\solutionspace{180pt}

\tasknumber{15}%
\task{%
    Линейные размеры прямого изображения предмета, полученного в собирающей линзе,
    в три раза больше линейных размеров предмета.
    Зная, что предмет находится на $25\,\text{см}$ ближе к линзе,
    чем его изображение, найти оптическую силу линзы.
}
\solutionspace{180pt}

\tasknumber{16}%
\task{%
    Оптическая сила объектива фотоаппарата равна $6\,\text{дптр}$.
    При фотографировании чертежа с расстояния $1{,}2\,\text{м}$ площадь изображения
    чертежа на фотопластинке оказалась равной $4\,\text{см}^{2}$.
    Какова площадь самого чертежа? Ответ выразите в квадратных сантиметрах.
}

\variantsplitter

\addpersonalvariant{Анна Кузьмичёва}

\tasknumber{1}%
\task{%
    Разность фаз двух интерферирующих световых волн равна $7\pi$, а разность хода между ними равна $10{,}5 \cdot 10^{-7}\,\text{м}$.
    Определить длину волны.
}
\solutionspace{100pt}

\tasknumber{2}%
\task{%
    Расстояние между двумя точечными когерентными источниками света $S_1$ u $S_2$ равно $2{,}5\,\text{мм}$.
    Источники расположены в плоскости, параллельной экрану, на расстоянии $9\,\text{м}$ от него.
    На экране в точках, лежащих на перпендикулярах, опущенных из источников света $S_1$ и $S_2$,
    находятся два ближайших минимума (тёмные полосы).
    Определите длину световой волны.
    Ответ дать в нанометрах.
}
\solutionspace{100pt}

\tasknumber{3}%
\task{%
    На дифракционную решетку, имеющую период $3 \cdot 10^{-4}\,\text{см}$, нормально падает монохроматическая световая волна.
    Под углом $ 30 \degrees$ наблюдается дифракционный максимум второго порядка.
    Какова длина волны падающего света?
}
\solutionspace{100pt}

\tasknumber{4}%
\task{%
    Свет с длиной волны $0{,}5\,\text{мкм}$ падает нормально на дифракционную решетку с периодом, равным $2\,\text{мкм}$.
    Под каким углом наблюдается дифракционный максимум первого порядка?
}
\solutionspace{100pt}

\tasknumber{5}%
\task{%
    При нормальном падении белого света на дифракционную решетку зелёная линия ($500\,\text{нм}$)
    в спектре второго порядка видна под углом дифракции $25\degrees$.
    Определить число штрихов на $1\,\text{мм}$ длины этой решетки.
}
\solutionspace{100pt}

\tasknumber{6}%
\task{%
    Каков наибольший порядок спектра, который можно наблюдать при дифракции света
    с длиной волны $\lambda$, на дифракционной решетке с периодом $d =  2{,}5 \lambda$?
}
\solutionspace{100pt}

\tasknumber{7}%
\task{%
    Вертикально стоящий шест высотой 1,1 м, освещенный солнцем,
    отбрасывает на горизонтальную поверхность земли тень длиной $1\,\text{м}$.
    Известно, что длина тени от телеграфного столба на $6\,\text{м}$ больше.
    Определить высоту столба.
}
\solutionspace{180pt}

\tasknumber{8}%
\task{%
    Определить абсолютный показатель преломления прозрачной среды,
    в которой распространяется свет с длиной волны $0{,}550\,\text{мкм}$ и частотой $545\,\text{ТГц}$.
    Скорость света в вакууме $3 \cdot 10^{8}\,\frac{\text{м}}{\text{с}}$.
}
\solutionspace{180pt}

\tasknumber{9}%
\task{%
    На дне водоема глубиной $4\,\text{м}$ лежит зеркало.
    Луч света, пройдя через воду, отражается от зеркала и выходит из воды.
    Найти расстояние между точкой входа луча в воду и точкой выхода луча из воды,
    если показатель преломления воды $1{,}33$, а угол падения луча $35\degrees$.
}
\solutionspace{180pt}

\tasknumber{10}%
\task{%
    Луч света падает на горизонтально расположенную стеклянную пластинку толщиной $4\,\text{мм}$.
    Пройдя через пластину, он выходит из неё в точке, смещённой по горизонтали от точки падения на расстояние $1{,}6\,\text{мм}$
    Показатель преломления стекла $1{,}6$.
    Найти синус угла падения, округлив его значение до двух знаков после запятой.
}
\solutionspace{180pt}

\tasknumber{11}%
\task{%
    Найти оптическую силу собирающей линзы, если действительное изображение предмета,
    помещённого в $35\,\text{см}$ от линзы, получается на расстоянии $20\,\text{см}$ от неё.
}
\solutionspace{180pt}

\tasknumber{12}%
\task{%
    Найти увеличение изображения, если изображение предмета, находящегося
    на расстоянии $25\,\text{см}$ от линзы, получается на расстоянии $12\,\text{см}$ от неё.
}
\solutionspace{180pt}

\tasknumber{13}%
\task{%
    Расстояние от предмета до линзы $12\,\text{см}$, а от линзы до мнимого изображения $20\,\text{см}$.
    Чему равно фокусное расстояние линзы?
}
\solutionspace{180pt}

\tasknumber{14}%
\task{%
    Две тонкие линзы с фокусными расстояниями $12\,\text{см}$ и $20\,\text{см}$ сложены вместе.
    Чему равно фокусное расстояние такой оптической системы?
}
\solutionspace{180pt}

\tasknumber{15}%
\task{%
    Линейные размеры прямого изображения предмета, полученного в собирающей линзе,
    в четыре раза больше линейных размеров предмета.
    Зная, что предмет находится на $35\,\text{см}$ ближе к линзе,
    чем его изображение, найти оптическую силу линзы.
}
\solutionspace{180pt}

\tasknumber{16}%
\task{%
    Оптическая сила объектива фотоаппарата равна $6\,\text{дптр}$.
    При фотографировании чертежа с расстояния $1{,}2\,\text{м}$ площадь изображения
    чертежа на фотопластинке оказалась равной $16\,\text{см}^{2}$.
    Какова площадь самого чертежа? Ответ выразите в квадратных сантиметрах.
}

\variantsplitter

\addpersonalvariant{Алёна Куприянова}

\tasknumber{1}%
\task{%
    Разность фаз двух интерферирующих световых волн равна $5\pi$, а разность хода между ними равна $15{,}5 \cdot 10^{-7}\,\text{м}$.
    Определить длину волны.
}
\solutionspace{100pt}

\tasknumber{2}%
\task{%
    Расстояние между двумя точечными когерентными источниками света $S_1$ u $S_2$ равно $1{,}5\,\text{мм}$.
    Источники расположены в плоскости, параллельной экрану, на расстоянии $8\,\text{м}$ от него.
    На экране в точках, лежащих на перпендикулярах, опущенных из источников света $S_1$ и $S_2$,
    находятся два ближайших минимума (тёмные полосы).
    Определите длину световой волны.
    Ответ дать в нанометрах.
}
\solutionspace{100pt}

\tasknumber{3}%
\task{%
    На дифракционную решетку, имеющую период $4 \cdot 10^{-4}\,\text{см}$, нормально падает монохроматическая световая волна.
    Под углом $ 40 \degrees$ наблюдается дифракционный максимум второго порядка.
    Какова длина волны падающего света?
}
\solutionspace{100pt}

\tasknumber{4}%
\task{%
    Свет с длиной волны $0{,}7\,\text{мкм}$ падает нормально на дифракционную решетку с периодом, равным $2\,\text{мкм}$.
    Под каким углом наблюдается дифракционный максимум первого порядка?
}
\solutionspace{100pt}

\tasknumber{5}%
\task{%
    При нормальном падении белого света на дифракционную решетку зелёная линия ($500\,\text{нм}$)
    в спектре второго порядка видна под углом дифракции $25\degrees$.
    Определить число штрихов на $1\,\text{см}$ длины этой решетки.
}
\solutionspace{100pt}

\tasknumber{6}%
\task{%
    Каков наибольший порядок спектра, который можно наблюдать при дифракции света
    с длиной волны $\lambda$, на дифракционной решетке с периодом $d =  3{,}5 \lambda$?
}
\solutionspace{100pt}

\tasknumber{7}%
\task{%
    Вертикально стоящий шест высотой 1,1 м, освещенный солнцем,
    отбрасывает на горизонтальную поверхность земли тень длиной $1\,\text{м}$.
    Известно, что длина тени от телеграфного столба на $9\,\text{м}$ больше.
    Определить высоту столба.
}
\solutionspace{180pt}

\tasknumber{8}%
\task{%
    Определить абсолютный показатель преломления прозрачной среды,
    в которой распространяется свет с длиной волны $0{,}500\,\text{мкм}$ и частотой $600\,\text{ТГц}$.
    Скорость света в вакууме $3 \cdot 10^{8}\,\frac{\text{м}}{\text{с}}$.
}
\solutionspace{180pt}

\tasknumber{9}%
\task{%
    На дне водоема глубиной $3\,\text{м}$ лежит зеркало.
    Луч света, пройдя через воду, отражается от зеркала и выходит из воды.
    Найти расстояние между точкой входа луча в воду и точкой выхода луча из воды,
    если показатель преломления воды $1{,}33$, а угол падения луча $30\degrees$.
}
\solutionspace{180pt}

\tasknumber{10}%
\task{%
    Луч света падает на горизонтально расположенную стеклянную пластинку толщиной $6\,\text{мм}$.
    Пройдя через пластину, он выходит из неё в точке, смещённой по горизонтали от точки падения на расстояние $1{,}2\,\text{мм}$
    Показатель преломления стекла $1{,}4$.
    Найти синус угла падения, округлив его значение до двух знаков после запятой.
}
\solutionspace{180pt}

\tasknumber{11}%
\task{%
    Найти оптическую силу собирающей линзы, если действительное изображение предмета,
    помещённого в $35\,\text{см}$ от линзы, получается на расстоянии $20\,\text{см}$ от неё.
}
\solutionspace{180pt}

\tasknumber{12}%
\task{%
    Найти увеличение изображения, если изображение предмета, находящегося
    на расстоянии $25\,\text{см}$ от линзы, получается на расстоянии $18\,\text{см}$ от неё.
}
\solutionspace{180pt}

\tasknumber{13}%
\task{%
    Расстояние от предмета до линзы $12\,\text{см}$, а от линзы до мнимого изображения $25\,\text{см}$.
    Чему равно фокусное расстояние линзы?
}
\solutionspace{180pt}

\tasknumber{14}%
\task{%
    Две тонкие линзы с фокусными расстояниями $18\,\text{см}$ и $30\,\text{см}$ сложены вместе.
    Чему равно фокусное расстояние такой оптической системы?
}
\solutionspace{180pt}

\tasknumber{15}%
\task{%
    Линейные размеры прямого изображения предмета, полученного в собирающей линзе,
    в два раза больше линейных размеров предмета.
    Зная, что предмет находится на $40\,\text{см}$ ближе к линзе,
    чем его изображение, найти оптическую силу линзы.
}
\solutionspace{180pt}

\tasknumber{16}%
\task{%
    Оптическая сила объектива фотоаппарата равна $4\,\text{дптр}$.
    При фотографировании чертежа с расстояния $0{,}9\,\text{м}$ площадь изображения
    чертежа на фотопластинке оказалась равной $9\,\text{см}^{2}$.
    Какова площадь самого чертежа? Ответ выразите в квадратных сантиметрах.
}

\variantsplitter

\addpersonalvariant{Ярослав Лавровский}

\tasknumber{1}%
\task{%
    Разность фаз двух интерферирующих световых волн равна $5\pi$, а разность хода между ними равна $7{,}5 \cdot 10^{-7}\,\text{м}$.
    Определить длину волны.
}
\solutionspace{100pt}

\tasknumber{2}%
\task{%
    Расстояние между двумя точечными когерентными источниками света $S_1$ u $S_2$ равно $2\,\text{мм}$.
    Источники расположены в плоскости, параллельной экрану, на расстоянии $8\,\text{м}$ от него.
    На экране в точках, лежащих на перпендикулярах, опущенных из источников света $S_1$ и $S_2$,
    находятся два ближайших минимума (тёмные полосы).
    Определите длину световой волны.
    Ответ дать в нанометрах.
}
\solutionspace{100pt}

\tasknumber{3}%
\task{%
    На дифракционную решетку, имеющую период $2 \cdot 10^{-4}\,\text{см}$, нормально падает монохроматическая световая волна.
    Под углом $ 35 \degrees$ наблюдается дифракционный максимум второго порядка.
    Какова длина волны падающего света?
}
\solutionspace{100pt}

\tasknumber{4}%
\task{%
    Свет с длиной волны $0{,}6\,\text{мкм}$ падает нормально на дифракционную решетку с периодом, равным $2\,\text{мкм}$.
    Под каким углом наблюдается дифракционный максимум первого порядка?
}
\solutionspace{100pt}

\tasknumber{5}%
\task{%
    При нормальном падении белого света на дифракционную решетку зелёная линия ($500\,\text{нм}$)
    в спектре второго порядка видна под углом дифракции $53\degrees$.
    Определить число штрихов на $1\,\text{см}$ длины этой решетки.
}
\solutionspace{100pt}

\tasknumber{6}%
\task{%
    Каков наибольший порядок спектра, который можно наблюдать при дифракции света
    с длиной волны $\lambda$, на дифракционной решетке с периодом $d =  2{,}5 \lambda$?
}
\solutionspace{100pt}

\tasknumber{7}%
\task{%
    Вертикально стоящий шест высотой 1,1 м, освещенный солнцем,
    отбрасывает на горизонтальную поверхность земли тень длиной $3\,\text{м}$.
    Известно, что длина тени от телеграфного столба на $8\,\text{м}$ больше.
    Определить высоту столба.
}
\solutionspace{180pt}

\tasknumber{8}%
\task{%
    Определить абсолютный показатель преломления прозрачной среды,
    в которой распространяется свет с длиной волны $0{,}650\,\text{мкм}$ и частотой $462\,\text{ТГц}$.
    Скорость света в вакууме $3 \cdot 10^{8}\,\frac{\text{м}}{\text{с}}$.
}
\solutionspace{180pt}

\tasknumber{9}%
\task{%
    На дне водоема глубиной $2\,\text{м}$ лежит зеркало.
    Луч света, пройдя через воду, отражается от зеркала и выходит из воды.
    Найти расстояние между точкой входа луча в воду и точкой выхода луча из воды,
    если показатель преломления воды $1{,}33$, а угол падения луча $25\degrees$.
}
\solutionspace{180pt}

\tasknumber{10}%
\task{%
    Луч света падает на горизонтально расположенную стеклянную пластинку толщиной $5\,\text{мм}$.
    Пройдя через пластину, он выходит из неё в точке, смещённой по горизонтали от точки падения на расстояние $1{,}3\,\text{мм}$
    Показатель преломления стекла $1{,}6$.
    Найти синус угла падения, округлив его значение до двух знаков после запятой.
}
\solutionspace{180pt}

\tasknumber{11}%
\task{%
    Найти оптическую силу собирающей линзы, если действительное изображение предмета,
    помещённого в $55\,\text{см}$ от линзы, получается на расстоянии $30\,\text{см}$ от неё.
}
\solutionspace{180pt}

\tasknumber{12}%
\task{%
    Найти увеличение изображения, если изображение предмета, находящегося
    на расстоянии $20\,\text{см}$ от линзы, получается на расстоянии $30\,\text{см}$ от неё.
}
\solutionspace{180pt}

\tasknumber{13}%
\task{%
    Расстояние от предмета до линзы $10\,\text{см}$, а от линзы до мнимого изображения $30\,\text{см}$.
    Чему равно фокусное расстояние линзы?
}
\solutionspace{180pt}

\tasknumber{14}%
\task{%
    Две тонкие линзы с фокусными расстояниями $12\,\text{см}$ и $20\,\text{см}$ сложены вместе.
    Чему равно фокусное расстояние такой оптической системы?
}
\solutionspace{180pt}

\tasknumber{15}%
\task{%
    Линейные размеры прямого изображения предмета, полученного в собирающей линзе,
    в четыре раза больше линейных размеров предмета.
    Зная, что предмет находится на $30\,\text{см}$ ближе к линзе,
    чем его изображение, найти оптическую силу линзы.
}
\solutionspace{180pt}

\tasknumber{16}%
\task{%
    Оптическая сила объектива фотоаппарата равна $6\,\text{дптр}$.
    При фотографировании чертежа с расстояния $1{,}1\,\text{м}$ площадь изображения
    чертежа на фотопластинке оказалась равной $16\,\text{см}^{2}$.
    Какова площадь самого чертежа? Ответ выразите в квадратных сантиметрах.
}

\variantsplitter

\addpersonalvariant{Анастасия Ламанова}

\tasknumber{1}%
\task{%
    Разность фаз двух интерферирующих световых волн равна $3\pi$, а разность хода между ними равна $12{,}5 \cdot 10^{-7}\,\text{м}$.
    Определить длину волны.
}
\solutionspace{100pt}

\tasknumber{2}%
\task{%
    Расстояние между двумя точечными когерентными источниками света $S_1$ u $S_2$ равно $2\,\text{мм}$.
    Источники расположены в плоскости, параллельной экрану, на расстоянии $7\,\text{м}$ от него.
    На экране в точках, лежащих на перпендикулярах, опущенных из источников света $S_1$ и $S_2$,
    находятся два ближайших минимума (тёмные полосы).
    Определите длину световой волны.
    Ответ дать в нанометрах.
}
\solutionspace{100pt}

\tasknumber{3}%
\task{%
    На дифракционную решетку, имеющую период $3 \cdot 10^{-4}\,\text{см}$, нормально падает монохроматическая световая волна.
    Под углом $ 20 \degrees$ наблюдается дифракционный максимум второго порядка.
    Какова длина волны падающего света?
}
\solutionspace{100pt}

\tasknumber{4}%
\task{%
    Свет с длиной волны $0{,}4\,\text{мкм}$ падает нормально на дифракционную решетку с периодом, равным $3\,\text{мкм}$.
    Под каким углом наблюдается дифракционный максимум первого порядка?
}
\solutionspace{100pt}

\tasknumber{5}%
\task{%
    При нормальном падении белого света на дифракционную решетку зелёная линия ($500\,\text{нм}$)
    в спектре второго порядка видна под углом дифракции $25\degrees$.
    Определить число штрихов на $1\,\text{см}$ длины этой решетки.
}
\solutionspace{100pt}

\tasknumber{6}%
\task{%
    Каков наибольший порядок спектра, который можно наблюдать при дифракции света
    с длиной волны $\lambda$, на дифракционной решетке с периодом $d =  2{,}5 \lambda$?
}
\solutionspace{100pt}

\tasknumber{7}%
\task{%
    Вертикально стоящий шест высотой 1,1 м, освещенный солнцем,
    отбрасывает на горизонтальную поверхность земли тень длиной $2\,\text{м}$.
    Известно, что длина тени от телеграфного столба на $8\,\text{м}$ больше.
    Определить высоту столба.
}
\solutionspace{180pt}

\tasknumber{8}%
\task{%
    Определить абсолютный показатель преломления прозрачной среды,
    в которой распространяется свет с длиной волны $0{,}450\,\text{мкм}$ и частотой $667\,\text{ТГц}$.
    Скорость света в вакууме $3 \cdot 10^{8}\,\frac{\text{м}}{\text{с}}$.
}
\solutionspace{180pt}

\tasknumber{9}%
\task{%
    На дне водоема глубиной $2\,\text{м}$ лежит зеркало.
    Луч света, пройдя через воду, отражается от зеркала и выходит из воды.
    Найти расстояние между точкой входа луча в воду и точкой выхода луча из воды,
    если показатель преломления воды $1{,}33$, а угол падения луча $25\degrees$.
}
\solutionspace{180pt}

\tasknumber{10}%
\task{%
    Луч света падает на горизонтально расположенную стеклянную пластинку толщиной $5\,\text{мм}$.
    Пройдя через пластину, он выходит из неё в точке, смещённой по горизонтали от точки падения на расстояние $1{,}6\,\text{мм}$
    Показатель преломления стекла $1{,}6$.
    Найти синус угла падения, округлив его значение до двух знаков после запятой.
}
\solutionspace{180pt}

\tasknumber{11}%
\task{%
    Найти оптическую силу собирающей линзы, если действительное изображение предмета,
    помещённого в $15\,\text{см}$ от линзы, получается на расстоянии $20\,\text{см}$ от неё.
}
\solutionspace{180pt}

\tasknumber{12}%
\task{%
    Найти увеличение изображения, если изображение предмета, находящегося
    на расстоянии $15\,\text{см}$ от линзы, получается на расстоянии $18\,\text{см}$ от неё.
}
\solutionspace{180pt}

\tasknumber{13}%
\task{%
    Расстояние от предмета до линзы $8\,\text{см}$, а от линзы до мнимого изображения $25\,\text{см}$.
    Чему равно фокусное расстояние линзы?
}
\solutionspace{180pt}

\tasknumber{14}%
\task{%
    Две тонкие линзы с фокусными расстояниями $25\,\text{см}$ и $20\,\text{см}$ сложены вместе.
    Чему равно фокусное расстояние такой оптической системы?
}
\solutionspace{180pt}

\tasknumber{15}%
\task{%
    Линейные размеры прямого изображения предмета, полученного в собирающей линзе,
    в четыре раза больше линейных размеров предмета.
    Зная, что предмет находится на $40\,\text{см}$ ближе к линзе,
    чем его изображение, найти оптическую силу линзы.
}
\solutionspace{180pt}

\tasknumber{16}%
\task{%
    Оптическая сила объектива фотоаппарата равна $3\,\text{дптр}$.
    При фотографировании чертежа с расстояния $1{,}2\,\text{м}$ площадь изображения
    чертежа на фотопластинке оказалась равной $4\,\text{см}^{2}$.
    Какова площадь самого чертежа? Ответ выразите в квадратных сантиметрах.
}

\variantsplitter

\addpersonalvariant{Виктория Легонькова}

\tasknumber{1}%
\task{%
    Разность фаз двух интерферирующих световых волн равна $9\pi$, а разность хода между ними равна $15{,}5 \cdot 10^{-7}\,\text{м}$.
    Определить длину волны.
}
\solutionspace{100pt}

\tasknumber{2}%
\task{%
    Расстояние между двумя точечными когерентными источниками света $S_1$ u $S_2$ равно $2\,\text{мм}$.
    Источники расположены в плоскости, параллельной экрану, на расстоянии $7\,\text{м}$ от него.
    На экране в точках, лежащих на перпендикулярах, опущенных из источников света $S_1$ и $S_2$,
    находятся два ближайших минимума (тёмные полосы).
    Определите длину световой волны.
    Ответ дать в нанометрах.
}
\solutionspace{100pt}

\tasknumber{3}%
\task{%
    На дифракционную решетку, имеющую период $2 \cdot 10^{-4}\,\text{см}$, нормально падает монохроматическая световая волна.
    Под углом $ 35 \degrees$ наблюдается дифракционный максимум второго порядка.
    Какова длина волны падающего света?
}
\solutionspace{100pt}

\tasknumber{4}%
\task{%
    Свет с длиной волны $0{,}6\,\text{мкм}$ падает нормально на дифракционную решетку с периодом, равным $3\,\text{мкм}$.
    Под каким углом наблюдается дифракционный максимум первого порядка?
}
\solutionspace{100pt}

\tasknumber{5}%
\task{%
    При нормальном падении белого света на дифракционную решетку зелёная линия ($500\,\text{нм}$)
    в спектре второго порядка видна под углом дифракции $20\degrees$.
    Определить число штрихов на $1\,\text{см}$ длины этой решетки.
}
\solutionspace{100pt}

\tasknumber{6}%
\task{%
    Каков наибольший порядок спектра, который можно наблюдать при дифракции света
    с длиной волны $\lambda$, на дифракционной решетке с периодом $d =  3{,}5 \lambda$?
}
\solutionspace{100pt}

\tasknumber{7}%
\task{%
    Вертикально стоящий шест высотой 1,1 м, освещенный солнцем,
    отбрасывает на горизонтальную поверхность земли тень длиной $3\,\text{м}$.
    Известно, что длина тени от телеграфного столба на $9\,\text{м}$ больше.
    Определить высоту столба.
}
\solutionspace{180pt}

\tasknumber{8}%
\task{%
    Определить абсолютный показатель преломления прозрачной среды,
    в которой распространяется свет с длиной волны $0{,}650\,\text{мкм}$ и частотой $462\,\text{ТГц}$.
    Скорость света в вакууме $3 \cdot 10^{8}\,\frac{\text{м}}{\text{с}}$.
}
\solutionspace{180pt}

\tasknumber{9}%
\task{%
    На дне водоема глубиной $4\,\text{м}$ лежит зеркало.
    Луч света, пройдя через воду, отражается от зеркала и выходит из воды.
    Найти расстояние между точкой входа луча в воду и точкой выхода луча из воды,
    если показатель преломления воды $1{,}33$, а угол падения луча $30\degrees$.
}
\solutionspace{180pt}

\tasknumber{10}%
\task{%
    Луч света падает на горизонтально расположенную стеклянную пластинку толщиной $4\,\text{мм}$.
    Пройдя через пластину, он выходит из неё в точке, смещённой по горизонтали от точки падения на расстояние $1{,}4\,\text{мм}$
    Показатель преломления стекла $1{,}5$.
    Найти синус угла падения, округлив его значение до двух знаков после запятой.
}
\solutionspace{180pt}

\tasknumber{11}%
\task{%
    Найти оптическую силу собирающей линзы, если действительное изображение предмета,
    помещённого в $35\,\text{см}$ от линзы, получается на расстоянии $30\,\text{см}$ от неё.
}
\solutionspace{180pt}

\tasknumber{12}%
\task{%
    Найти увеличение изображения, если изображение предмета, находящегося
    на расстоянии $15\,\text{см}$ от линзы, получается на расстоянии $18\,\text{см}$ от неё.
}
\solutionspace{180pt}

\tasknumber{13}%
\task{%
    Расстояние от предмета до линзы $8\,\text{см}$, а от линзы до мнимого изображения $25\,\text{см}$.
    Чему равно фокусное расстояние линзы?
}
\solutionspace{180pt}

\tasknumber{14}%
\task{%
    Две тонкие линзы с фокусными расстояниями $25\,\text{см}$ и $20\,\text{см}$ сложены вместе.
    Чему равно фокусное расстояние такой оптической системы?
}
\solutionspace{180pt}

\tasknumber{15}%
\task{%
    Линейные размеры прямого изображения предмета, полученного в собирающей линзе,
    в три раза больше линейных размеров предмета.
    Зная, что предмет находится на $20\,\text{см}$ ближе к линзе,
    чем его изображение, найти оптическую силу линзы.
}
\solutionspace{180pt}

\tasknumber{16}%
\task{%
    Оптическая сила объектива фотоаппарата равна $3\,\text{дптр}$.
    При фотографировании чертежа с расстояния $0{,}8\,\text{м}$ площадь изображения
    чертежа на фотопластинке оказалась равной $16\,\text{см}^{2}$.
    Какова площадь самого чертежа? Ответ выразите в квадратных сантиметрах.
}

\variantsplitter

\addpersonalvariant{Семён Мартынов}

\tasknumber{1}%
\task{%
    Разность фаз двух интерферирующих световых волн равна $5\pi$, а разность хода между ними равна $7{,}5 \cdot 10^{-7}\,\text{м}$.
    Определить длину волны.
}
\solutionspace{100pt}

\tasknumber{2}%
\task{%
    Расстояние между двумя точечными когерентными источниками света $S_1$ u $S_2$ равно $1{,}5\,\text{мм}$.
    Источники расположены в плоскости, параллельной экрану, на расстоянии $9\,\text{м}$ от него.
    На экране в точках, лежащих на перпендикулярах, опущенных из источников света $S_1$ и $S_2$,
    находятся два ближайших минимума (тёмные полосы).
    Определите длину световой волны.
    Ответ дать в нанометрах.
}
\solutionspace{100pt}

\tasknumber{3}%
\task{%
    На дифракционную решетку, имеющую период $4 \cdot 10^{-4}\,\text{см}$, нормально падает монохроматическая световая волна.
    Под углом $ 30 \degrees$ наблюдается дифракционный максимум второго порядка.
    Какова длина волны падающего света?
}
\solutionspace{100pt}

\tasknumber{4}%
\task{%
    Свет с длиной волны $0{,}5\,\text{мкм}$ падает нормально на дифракционную решетку с периодом, равным $2\,\text{мкм}$.
    Под каким углом наблюдается дифракционный максимум первого порядка?
}
\solutionspace{100pt}

\tasknumber{5}%
\task{%
    При нормальном падении белого света на дифракционную решетку зелёная линия ($500\,\text{нм}$)
    в спектре второго порядка видна под углом дифракции $20\degrees$.
    Определить число штрихов на $1\,\text{см}$ длины этой решетки.
}
\solutionspace{100pt}

\tasknumber{6}%
\task{%
    Каков наибольший порядок спектра, который можно наблюдать при дифракции света
    с длиной волны $\lambda$, на дифракционной решетке с периодом $d =  2{,}5 \lambda$?
}
\solutionspace{100pt}

\tasknumber{7}%
\task{%
    Вертикально стоящий шест высотой 1,1 м, освещенный солнцем,
    отбрасывает на горизонтальную поверхность земли тень длиной $4\,\text{м}$.
    Известно, что длина тени от телеграфного столба на $7\,\text{м}$ больше.
    Определить высоту столба.
}
\solutionspace{180pt}

\tasknumber{8}%
\task{%
    Определить абсолютный показатель преломления прозрачной среды,
    в которой распространяется свет с длиной волны $0{,}500\,\text{мкм}$ и частотой $600\,\text{ТГц}$.
    Скорость света в вакууме $3 \cdot 10^{8}\,\frac{\text{м}}{\text{с}}$.
}
\solutionspace{180pt}

\tasknumber{9}%
\task{%
    На дне водоема глубиной $3\,\text{м}$ лежит зеркало.
    Луч света, пройдя через воду, отражается от зеркала и выходит из воды.
    Найти расстояние между точкой входа луча в воду и точкой выхода луча из воды,
    если показатель преломления воды $1{,}33$, а угол падения луча $35\degrees$.
}
\solutionspace{180pt}

\tasknumber{10}%
\task{%
    Луч света падает на горизонтально расположенную стеклянную пластинку толщиной $6\,\text{мм}$.
    Пройдя через пластину, он выходит из неё в точке, смещённой по горизонтали от точки падения на расстояние $1{,}6\,\text{мм}$
    Показатель преломления стекла $1{,}6$.
    Найти синус угла падения, округлив его значение до двух знаков после запятой.
}
\solutionspace{180pt}

\tasknumber{11}%
\task{%
    Найти оптическую силу собирающей линзы, если действительное изображение предмета,
    помещённого в $15\,\text{см}$ от линзы, получается на расстоянии $20\,\text{см}$ от неё.
}
\solutionspace{180pt}

\tasknumber{12}%
\task{%
    Найти увеличение изображения, если изображение предмета, находящегося
    на расстоянии $15\,\text{см}$ от линзы, получается на расстоянии $30\,\text{см}$ от неё.
}
\solutionspace{180pt}

\tasknumber{13}%
\task{%
    Расстояние от предмета до линзы $8\,\text{см}$, а от линзы до мнимого изображения $30\,\text{см}$.
    Чему равно фокусное расстояние линзы?
}
\solutionspace{180pt}

\tasknumber{14}%
\task{%
    Две тонкие линзы с фокусными расстояниями $18\,\text{см}$ и $30\,\text{см}$ сложены вместе.
    Чему равно фокусное расстояние такой оптической системы?
}
\solutionspace{180pt}

\tasknumber{15}%
\task{%
    Линейные размеры прямого изображения предмета, полученного в собирающей линзе,
    в четыре раза больше линейных размеров предмета.
    Зная, что предмет находится на $25\,\text{см}$ ближе к линзе,
    чем его изображение, найти оптическую силу линзы.
}
\solutionspace{180pt}

\tasknumber{16}%
\task{%
    Оптическая сила объектива фотоаппарата равна $3\,\text{дптр}$.
    При фотографировании чертежа с расстояния $1{,}1\,\text{м}$ площадь изображения
    чертежа на фотопластинке оказалась равной $4\,\text{см}^{2}$.
    Какова площадь самого чертежа? Ответ выразите в квадратных сантиметрах.
}

\variantsplitter

\addpersonalvariant{Варвара Минаева}

\tasknumber{1}%
\task{%
    Разность фаз двух интерферирующих световых волн равна $9\pi$, а разность хода между ними равна $10{,}5 \cdot 10^{-7}\,\text{м}$.
    Определить длину волны.
}
\solutionspace{100pt}

\tasknumber{2}%
\task{%
    Расстояние между двумя точечными когерентными источниками света $S_1$ u $S_2$ равно $1{,}5\,\text{мм}$.
    Источники расположены в плоскости, параллельной экрану, на расстоянии $7\,\text{м}$ от него.
    На экране в точках, лежащих на перпендикулярах, опущенных из источников света $S_1$ и $S_2$,
    находятся два ближайших минимума (тёмные полосы).
    Определите длину световой волны.
    Ответ дать в нанометрах.
}
\solutionspace{100pt}

\tasknumber{3}%
\task{%
    На дифракционную решетку, имеющую период $3 \cdot 10^{-4}\,\text{см}$, нормально падает монохроматическая световая волна.
    Под углом $ 35 \degrees$ наблюдается дифракционный максимум второго порядка.
    Какова длина волны падающего света?
}
\solutionspace{100pt}

\tasknumber{4}%
\task{%
    Свет с длиной волны $0{,}4\,\text{мкм}$ падает нормально на дифракционную решетку с периодом, равным $3\,\text{мкм}$.
    Под каким углом наблюдается дифракционный максимум первого порядка?
}
\solutionspace{100pt}

\tasknumber{5}%
\task{%
    При нормальном падении белого света на дифракционную решетку зелёная линия ($500\,\text{нм}$)
    в спектре второго порядка видна под углом дифракции $53\degrees$.
    Определить число штрихов на $1\,\text{мм}$ длины этой решетки.
}
\solutionspace{100pt}

\tasknumber{6}%
\task{%
    Каков наибольший порядок спектра, который можно наблюдать при дифракции света
    с длиной волны $\lambda$, на дифракционной решетке с периодом $d =  2{,}5 \lambda$?
}
\solutionspace{100pt}

\tasknumber{7}%
\task{%
    Вертикально стоящий шест высотой 1,1 м, освещенный солнцем,
    отбрасывает на горизонтальную поверхность земли тень длиной $1\,\text{м}$.
    Известно, что длина тени от телеграфного столба на $8\,\text{м}$ больше.
    Определить высоту столба.
}
\solutionspace{180pt}

\tasknumber{8}%
\task{%
    Определить абсолютный показатель преломления прозрачной среды,
    в которой распространяется свет с длиной волны $0{,}450\,\text{мкм}$ и частотой $667\,\text{ТГц}$.
    Скорость света в вакууме $3 \cdot 10^{8}\,\frac{\text{м}}{\text{с}}$.
}
\solutionspace{180pt}

\tasknumber{9}%
\task{%
    На дне водоема глубиной $3\,\text{м}$ лежит зеркало.
    Луч света, пройдя через воду, отражается от зеркала и выходит из воды.
    Найти расстояние между точкой входа луча в воду и точкой выхода луча из воды,
    если показатель преломления воды $1{,}33$, а угол падения луча $25\degrees$.
}
\solutionspace{180pt}

\tasknumber{10}%
\task{%
    Луч света падает на горизонтально расположенную стеклянную пластинку толщиной $4\,\text{мм}$.
    Пройдя через пластину, он выходит из неё в точке, смещённой по горизонтали от точки падения на расстояние $1{,}2\,\text{мм}$
    Показатель преломления стекла $1{,}5$.
    Найти синус угла падения, округлив его значение до двух знаков после запятой.
}
\solutionspace{180pt}

\tasknumber{11}%
\task{%
    Найти оптическую силу собирающей линзы, если действительное изображение предмета,
    помещённого в $15\,\text{см}$ от линзы, получается на расстоянии $20\,\text{см}$ от неё.
}
\solutionspace{180pt}

\tasknumber{12}%
\task{%
    Найти увеличение изображения, если изображение предмета, находящегося
    на расстоянии $15\,\text{см}$ от линзы, получается на расстоянии $30\,\text{см}$ от неё.
}
\solutionspace{180pt}

\tasknumber{13}%
\task{%
    Расстояние от предмета до линзы $8\,\text{см}$, а от линзы до мнимого изображения $30\,\text{см}$.
    Чему равно фокусное расстояние линзы?
}
\solutionspace{180pt}

\tasknumber{14}%
\task{%
    Две тонкие линзы с фокусными расстояниями $12\,\text{см}$ и $20\,\text{см}$ сложены вместе.
    Чему равно фокусное расстояние такой оптической системы?
}
\solutionspace{180pt}

\tasknumber{15}%
\task{%
    Линейные размеры прямого изображения предмета, полученного в собирающей линзе,
    в два раза больше линейных размеров предмета.
    Зная, что предмет находится на $25\,\text{см}$ ближе к линзе,
    чем его изображение, найти оптическую силу линзы.
}
\solutionspace{180pt}

\tasknumber{16}%
\task{%
    Оптическая сила объектива фотоаппарата равна $3\,\text{дптр}$.
    При фотографировании чертежа с расстояния $0{,}8\,\text{м}$ площадь изображения
    чертежа на фотопластинке оказалась равной $16\,\text{см}^{2}$.
    Какова площадь самого чертежа? Ответ выразите в квадратных сантиметрах.
}

\variantsplitter

\addpersonalvariant{Леонид Никитин}

\tasknumber{1}%
\task{%
    Разность фаз двух интерферирующих световых волн равна $9\pi$, а разность хода между ними равна $12{,}5 \cdot 10^{-7}\,\text{м}$.
    Определить длину волны.
}
\solutionspace{100pt}

\tasknumber{2}%
\task{%
    Расстояние между двумя точечными когерентными источниками света $S_1$ u $S_2$ равно $2{,}5\,\text{мм}$.
    Источники расположены в плоскости, параллельной экрану, на расстоянии $9\,\text{м}$ от него.
    На экране в точках, лежащих на перпендикулярах, опущенных из источников света $S_1$ и $S_2$,
    находятся два ближайших минимума (тёмные полосы).
    Определите длину световой волны.
    Ответ дать в нанометрах.
}
\solutionspace{100pt}

\tasknumber{3}%
\task{%
    На дифракционную решетку, имеющую период $3 \cdot 10^{-4}\,\text{см}$, нормально падает монохроматическая световая волна.
    Под углом $ 20 \degrees$ наблюдается дифракционный максимум второго порядка.
    Какова длина волны падающего света?
}
\solutionspace{100pt}

\tasknumber{4}%
\task{%
    Свет с длиной волны $0{,}6\,\text{мкм}$ падает нормально на дифракционную решетку с периодом, равным $3\,\text{мкм}$.
    Под каким углом наблюдается дифракционный максимум первого порядка?
}
\solutionspace{100pt}

\tasknumber{5}%
\task{%
    При нормальном падении белого света на дифракционную решетку зелёная линия ($500\,\text{нм}$)
    в спектре второго порядка видна под углом дифракции $53\degrees$.
    Определить число штрихов на $1\,\text{см}$ длины этой решетки.
}
\solutionspace{100pt}

\tasknumber{6}%
\task{%
    Каков наибольший порядок спектра, который можно наблюдать при дифракции света
    с длиной волны $\lambda$, на дифракционной решетке с периодом $d =  4{,}5 \lambda$?
}
\solutionspace{100pt}

\tasknumber{7}%
\task{%
    Вертикально стоящий шест высотой 1,1 м, освещенный солнцем,
    отбрасывает на горизонтальную поверхность земли тень длиной $3\,\text{м}$.
    Известно, что длина тени от телеграфного столба на $8\,\text{м}$ больше.
    Определить высоту столба.
}
\solutionspace{180pt}

\tasknumber{8}%
\task{%
    Определить абсолютный показатель преломления прозрачной среды,
    в которой распространяется свет с длиной волны $0{,}550\,\text{мкм}$ и частотой $545\,\text{ТГц}$.
    Скорость света в вакууме $3 \cdot 10^{8}\,\frac{\text{м}}{\text{с}}$.
}
\solutionspace{180pt}

\tasknumber{9}%
\task{%
    На дне водоема глубиной $2\,\text{м}$ лежит зеркало.
    Луч света, пройдя через воду, отражается от зеркала и выходит из воды.
    Найти расстояние между точкой входа луча в воду и точкой выхода луча из воды,
    если показатель преломления воды $1{,}33$, а угол падения луча $30\degrees$.
}
\solutionspace{180pt}

\tasknumber{10}%
\task{%
    Луч света падает на горизонтально расположенную стеклянную пластинку толщиной $6\,\text{мм}$.
    Пройдя через пластину, он выходит из неё в точке, смещённой по горизонтали от точки падения на расстояние $1{,}5\,\text{мм}$
    Показатель преломления стекла $1{,}4$.
    Найти синус угла падения, округлив его значение до двух знаков после запятой.
}
\solutionspace{180pt}

\tasknumber{11}%
\task{%
    Найти оптическую силу собирающей линзы, если действительное изображение предмета,
    помещённого в $55\,\text{см}$ от линзы, получается на расстоянии $30\,\text{см}$ от неё.
}
\solutionspace{180pt}

\tasknumber{12}%
\task{%
    Найти увеличение изображения, если изображение предмета, находящегося
    на расстоянии $20\,\text{см}$ от линзы, получается на расстоянии $30\,\text{см}$ от неё.
}
\solutionspace{180pt}

\tasknumber{13}%
\task{%
    Расстояние от предмета до линзы $10\,\text{см}$, а от линзы до мнимого изображения $30\,\text{см}$.
    Чему равно фокусное расстояние линзы?
}
\solutionspace{180pt}

\tasknumber{14}%
\task{%
    Две тонкие линзы с фокусными расстояниями $12\,\text{см}$ и $30\,\text{см}$ сложены вместе.
    Чему равно фокусное расстояние такой оптической системы?
}
\solutionspace{180pt}

\tasknumber{15}%
\task{%
    Линейные размеры прямого изображения предмета, полученного в собирающей линзе,
    в четыре раза больше линейных размеров предмета.
    Зная, что предмет находится на $25\,\text{см}$ ближе к линзе,
    чем его изображение, найти оптическую силу линзы.
}
\solutionspace{180pt}

\tasknumber{16}%
\task{%
    Оптическая сила объектива фотоаппарата равна $4\,\text{дптр}$.
    При фотографировании чертежа с расстояния $1{,}2\,\text{м}$ площадь изображения
    чертежа на фотопластинке оказалась равной $4\,\text{см}^{2}$.
    Какова площадь самого чертежа? Ответ выразите в квадратных сантиметрах.
}

\variantsplitter

\addpersonalvariant{Тимофей Полетаев}

\tasknumber{1}%
\task{%
    Разность фаз двух интерферирующих световых волн равна $5\pi$, а разность хода между ними равна $7{,}5 \cdot 10^{-7}\,\text{м}$.
    Определить длину волны.
}
\solutionspace{100pt}

\tasknumber{2}%
\task{%
    Расстояние между двумя точечными когерентными источниками света $S_1$ u $S_2$ равно $2{,}5\,\text{мм}$.
    Источники расположены в плоскости, параллельной экрану, на расстоянии $8\,\text{м}$ от него.
    На экране в точках, лежащих на перпендикулярах, опущенных из источников света $S_1$ и $S_2$,
    находятся два ближайших минимума (тёмные полосы).
    Определите длину световой волны.
    Ответ дать в нанометрах.
}
\solutionspace{100pt}

\tasknumber{3}%
\task{%
    На дифракционную решетку, имеющую период $3 \cdot 10^{-4}\,\text{см}$, нормально падает монохроматическая световая волна.
    Под углом $ 30 \degrees$ наблюдается дифракционный максимум второго порядка.
    Какова длина волны падающего света?
}
\solutionspace{100pt}

\tasknumber{4}%
\task{%
    Свет с длиной волны $0{,}4\,\text{мкм}$ падает нормально на дифракционную решетку с периодом, равным $3\,\text{мкм}$.
    Под каким углом наблюдается дифракционный максимум первого порядка?
}
\solutionspace{100pt}

\tasknumber{5}%
\task{%
    При нормальном падении белого света на дифракционную решетку зелёная линия ($500\,\text{нм}$)
    в спектре второго порядка видна под углом дифракции $25\degrees$.
    Определить число штрихов на $1\,\text{см}$ длины этой решетки.
}
\solutionspace{100pt}

\tasknumber{6}%
\task{%
    Каков наибольший порядок спектра, который можно наблюдать при дифракции света
    с длиной волны $\lambda$, на дифракционной решетке с периодом $d =  2{,}5 \lambda$?
}
\solutionspace{100pt}

\tasknumber{7}%
\task{%
    Вертикально стоящий шест высотой 1,1 м, освещенный солнцем,
    отбрасывает на горизонтальную поверхность земли тень длиной $3\,\text{м}$.
    Известно, что длина тени от телеграфного столба на $6\,\text{м}$ больше.
    Определить высоту столба.
}
\solutionspace{180pt}

\tasknumber{8}%
\task{%
    Определить абсолютный показатель преломления прозрачной среды,
    в которой распространяется свет с длиной волны $0{,}550\,\text{мкм}$ и частотой $545\,\text{ТГц}$.
    Скорость света в вакууме $3 \cdot 10^{8}\,\frac{\text{м}}{\text{с}}$.
}
\solutionspace{180pt}

\tasknumber{9}%
\task{%
    На дне водоема глубиной $2\,\text{м}$ лежит зеркало.
    Луч света, пройдя через воду, отражается от зеркала и выходит из воды.
    Найти расстояние между точкой входа луча в воду и точкой выхода луча из воды,
    если показатель преломления воды $1{,}33$, а угол падения луча $35\degrees$.
}
\solutionspace{180pt}

\tasknumber{10}%
\task{%
    Луч света падает на горизонтально расположенную стеклянную пластинку толщиной $6\,\text{мм}$.
    Пройдя через пластину, он выходит из неё в точке, смещённой по горизонтали от точки падения на расстояние $1{,}4\,\text{мм}$
    Показатель преломления стекла $1{,}4$.
    Найти синус угла падения, округлив его значение до двух знаков после запятой.
}
\solutionspace{180pt}

\tasknumber{11}%
\task{%
    Найти оптическую силу собирающей линзы, если действительное изображение предмета,
    помещённого в $55\,\text{см}$ от линзы, получается на расстоянии $20\,\text{см}$ от неё.
}
\solutionspace{180pt}

\tasknumber{12}%
\task{%
    Найти увеличение изображения, если изображение предмета, находящегося
    на расстоянии $15\,\text{см}$ от линзы, получается на расстоянии $12\,\text{см}$ от неё.
}
\solutionspace{180pt}

\tasknumber{13}%
\task{%
    Расстояние от предмета до линзы $8\,\text{см}$, а от линзы до мнимого изображения $20\,\text{см}$.
    Чему равно фокусное расстояние линзы?
}
\solutionspace{180pt}

\tasknumber{14}%
\task{%
    Две тонкие линзы с фокусными расстояниями $25\,\text{см}$ и $20\,\text{см}$ сложены вместе.
    Чему равно фокусное расстояние такой оптической системы?
}
\solutionspace{180pt}

\tasknumber{15}%
\task{%
    Линейные размеры прямого изображения предмета, полученного в собирающей линзе,
    в четыре раза больше линейных размеров предмета.
    Зная, что предмет находится на $20\,\text{см}$ ближе к линзе,
    чем его изображение, найти оптическую силу линзы.
}
\solutionspace{180pt}

\tasknumber{16}%
\task{%
    Оптическая сила объектива фотоаппарата равна $4\,\text{дптр}$.
    При фотографировании чертежа с расстояния $0{,}8\,\text{м}$ площадь изображения
    чертежа на фотопластинке оказалась равной $4\,\text{см}^{2}$.
    Какова площадь самого чертежа? Ответ выразите в квадратных сантиметрах.
}

\variantsplitter

\addpersonalvariant{Андрей Рожков}

\tasknumber{1}%
\task{%
    Разность фаз двух интерферирующих световых волн равна $7\pi$, а разность хода между ними равна $15{,}5 \cdot 10^{-7}\,\text{м}$.
    Определить длину волны.
}
\solutionspace{100pt}

\tasknumber{2}%
\task{%
    Расстояние между двумя точечными когерентными источниками света $S_1$ u $S_2$ равно $1{,}5\,\text{мм}$.
    Источники расположены в плоскости, параллельной экрану, на расстоянии $8\,\text{м}$ от него.
    На экране в точках, лежащих на перпендикулярах, опущенных из источников света $S_1$ и $S_2$,
    находятся два ближайших минимума (тёмные полосы).
    Определите длину световой волны.
    Ответ дать в нанометрах.
}
\solutionspace{100pt}

\tasknumber{3}%
\task{%
    На дифракционную решетку, имеющую период $2 \cdot 10^{-4}\,\text{см}$, нормально падает монохроматическая световая волна.
    Под углом $ 35 \degrees$ наблюдается дифракционный максимум второго порядка.
    Какова длина волны падающего света?
}
\solutionspace{100pt}

\tasknumber{4}%
\task{%
    Свет с длиной волны $0{,}5\,\text{мкм}$ падает нормально на дифракционную решетку с периодом, равным $3\,\text{мкм}$.
    Под каким углом наблюдается дифракционный максимум первого порядка?
}
\solutionspace{100pt}

\tasknumber{5}%
\task{%
    При нормальном падении белого света на дифракционную решетку зелёная линия ($500\,\text{нм}$)
    в спектре второго порядка видна под углом дифракции $20\degrees$.
    Определить число штрихов на $1\,\text{мм}$ длины этой решетки.
}
\solutionspace{100pt}

\tasknumber{6}%
\task{%
    Каков наибольший порядок спектра, который можно наблюдать при дифракции света
    с длиной волны $\lambda$, на дифракционной решетке с периодом $d =  3{,}5 \lambda$?
}
\solutionspace{100pt}

\tasknumber{7}%
\task{%
    Вертикально стоящий шест высотой 1,1 м, освещенный солнцем,
    отбрасывает на горизонтальную поверхность земли тень длиной $3\,\text{м}$.
    Известно, что длина тени от телеграфного столба на $6\,\text{м}$ больше.
    Определить высоту столба.
}
\solutionspace{180pt}

\tasknumber{8}%
\task{%
    Определить абсолютный показатель преломления прозрачной среды,
    в которой распространяется свет с длиной волны $0{,}600\,\text{мкм}$ и частотой $500\,\text{ТГц}$.
    Скорость света в вакууме $3 \cdot 10^{8}\,\frac{\text{м}}{\text{с}}$.
}
\solutionspace{180pt}

\tasknumber{9}%
\task{%
    На дне водоема глубиной $3\,\text{м}$ лежит зеркало.
    Луч света, пройдя через воду, отражается от зеркала и выходит из воды.
    Найти расстояние между точкой входа луча в воду и точкой выхода луча из воды,
    если показатель преломления воды $1{,}33$, а угол падения луча $35\degrees$.
}
\solutionspace{180pt}

\tasknumber{10}%
\task{%
    Луч света падает на горизонтально расположенную стеклянную пластинку толщиной $4\,\text{мм}$.
    Пройдя через пластину, он выходит из неё в точке, смещённой по горизонтали от точки падения на расстояние $1{,}2\,\text{мм}$
    Показатель преломления стекла $1{,}4$.
    Найти синус угла падения, округлив его значение до двух знаков после запятой.
}
\solutionspace{180pt}

\tasknumber{11}%
\task{%
    Найти оптическую силу собирающей линзы, если действительное изображение предмета,
    помещённого в $35\,\text{см}$ от линзы, получается на расстоянии $40\,\text{см}$ от неё.
}
\solutionspace{180pt}

\tasknumber{12}%
\task{%
    Найти увеличение изображения, если изображение предмета, находящегося
    на расстоянии $20\,\text{см}$ от линзы, получается на расстоянии $12\,\text{см}$ от неё.
}
\solutionspace{180pt}

\tasknumber{13}%
\task{%
    Расстояние от предмета до линзы $10\,\text{см}$, а от линзы до мнимого изображения $20\,\text{см}$.
    Чему равно фокусное расстояние линзы?
}
\solutionspace{180pt}

\tasknumber{14}%
\task{%
    Две тонкие линзы с фокусными расстояниями $25\,\text{см}$ и $30\,\text{см}$ сложены вместе.
    Чему равно фокусное расстояние такой оптической системы?
}
\solutionspace{180pt}

\tasknumber{15}%
\task{%
    Линейные размеры прямого изображения предмета, полученного в собирающей линзе,
    в четыре раза больше линейных размеров предмета.
    Зная, что предмет находится на $20\,\text{см}$ ближе к линзе,
    чем его изображение, найти оптическую силу линзы.
}
\solutionspace{180pt}

\tasknumber{16}%
\task{%
    Оптическая сила объектива фотоаппарата равна $4\,\text{дптр}$.
    При фотографировании чертежа с расстояния $0{,}9\,\text{м}$ площадь изображения
    чертежа на фотопластинке оказалась равной $4\,\text{см}^{2}$.
    Какова площадь самого чертежа? Ответ выразите в квадратных сантиметрах.
}

\variantsplitter

\addpersonalvariant{Рената Таржиманова}

\tasknumber{1}%
\task{%
    Разность фаз двух интерферирующих световых волн равна $9\pi$, а разность хода между ними равна $10{,}5 \cdot 10^{-7}\,\text{м}$.
    Определить длину волны.
}
\solutionspace{100pt}

\tasknumber{2}%
\task{%
    Расстояние между двумя точечными когерентными источниками света $S_1$ u $S_2$ равно $2\,\text{мм}$.
    Источники расположены в плоскости, параллельной экрану, на расстоянии $8\,\text{м}$ от него.
    На экране в точках, лежащих на перпендикулярах, опущенных из источников света $S_1$ и $S_2$,
    находятся два ближайших минимума (тёмные полосы).
    Определите длину световой волны.
    Ответ дать в нанометрах.
}
\solutionspace{100pt}

\tasknumber{3}%
\task{%
    На дифракционную решетку, имеющую период $3 \cdot 10^{-4}\,\text{см}$, нормально падает монохроматическая световая волна.
    Под углом $ 30 \degrees$ наблюдается дифракционный максимум второго порядка.
    Какова длина волны падающего света?
}
\solutionspace{100pt}

\tasknumber{4}%
\task{%
    Свет с длиной волны $0{,}6\,\text{мкм}$ падает нормально на дифракционную решетку с периодом, равным $3\,\text{мкм}$.
    Под каким углом наблюдается дифракционный максимум первого порядка?
}
\solutionspace{100pt}

\tasknumber{5}%
\task{%
    При нормальном падении белого света на дифракционную решетку зелёная линия ($500\,\text{нм}$)
    в спектре второго порядка видна под углом дифракции $20\degrees$.
    Определить число штрихов на $1\,\text{см}$ длины этой решетки.
}
\solutionspace{100pt}

\tasknumber{6}%
\task{%
    Каков наибольший порядок спектра, который можно наблюдать при дифракции света
    с длиной волны $\lambda$, на дифракционной решетке с периодом $d =  4{,}5 \lambda$?
}
\solutionspace{100pt}

\tasknumber{7}%
\task{%
    Вертикально стоящий шест высотой 1,1 м, освещенный солнцем,
    отбрасывает на горизонтальную поверхность земли тень длиной $2\,\text{м}$.
    Известно, что длина тени от телеграфного столба на $5\,\text{м}$ больше.
    Определить высоту столба.
}
\solutionspace{180pt}

\tasknumber{8}%
\task{%
    Определить абсолютный показатель преломления прозрачной среды,
    в которой распространяется свет с длиной волны $0{,}650\,\text{мкм}$ и частотой $462\,\text{ТГц}$.
    Скорость света в вакууме $3 \cdot 10^{8}\,\frac{\text{м}}{\text{с}}$.
}
\solutionspace{180pt}

\tasknumber{9}%
\task{%
    На дне водоема глубиной $3\,\text{м}$ лежит зеркало.
    Луч света, пройдя через воду, отражается от зеркала и выходит из воды.
    Найти расстояние между точкой входа луча в воду и точкой выхода луча из воды,
    если показатель преломления воды $1{,}33$, а угол падения луча $25\degrees$.
}
\solutionspace{180pt}

\tasknumber{10}%
\task{%
    Луч света падает на горизонтально расположенную стеклянную пластинку толщиной $6\,\text{мм}$.
    Пройдя через пластину, он выходит из неё в точке, смещённой по горизонтали от точки падения на расстояние $1{,}3\,\text{мм}$
    Показатель преломления стекла $1{,}4$.
    Найти синус угла падения, округлив его значение до двух знаков после запятой.
}
\solutionspace{180pt}

\tasknumber{11}%
\task{%
    Найти оптическую силу собирающей линзы, если действительное изображение предмета,
    помещённого в $35\,\text{см}$ от линзы, получается на расстоянии $40\,\text{см}$ от неё.
}
\solutionspace{180pt}

\tasknumber{12}%
\task{%
    Найти увеличение изображения, если изображение предмета, находящегося
    на расстоянии $15\,\text{см}$ от линзы, получается на расстоянии $12\,\text{см}$ от неё.
}
\solutionspace{180pt}

\tasknumber{13}%
\task{%
    Расстояние от предмета до линзы $8\,\text{см}$, а от линзы до мнимого изображения $20\,\text{см}$.
    Чему равно фокусное расстояние линзы?
}
\solutionspace{180pt}

\tasknumber{14}%
\task{%
    Две тонкие линзы с фокусными расстояниями $25\,\text{см}$ и $20\,\text{см}$ сложены вместе.
    Чему равно фокусное расстояние такой оптической системы?
}
\solutionspace{180pt}

\tasknumber{15}%
\task{%
    Линейные размеры прямого изображения предмета, полученного в собирающей линзе,
    в четыре раза больше линейных размеров предмета.
    Зная, что предмет находится на $25\,\text{см}$ ближе к линзе,
    чем его изображение, найти оптическую силу линзы.
}
\solutionspace{180pt}

\tasknumber{16}%
\task{%
    Оптическая сила объектива фотоаппарата равна $4\,\text{дптр}$.
    При фотографировании чертежа с расстояния $1{,}2\,\text{м}$ площадь изображения
    чертежа на фотопластинке оказалась равной $4\,\text{см}^{2}$.
    Какова площадь самого чертежа? Ответ выразите в квадратных сантиметрах.
}

\variantsplitter

\addpersonalvariant{Андрей Щербаков}

\tasknumber{1}%
\task{%
    Разность фаз двух интерферирующих световых волн равна $3\pi$, а разность хода между ними равна $15{,}5 \cdot 10^{-7}\,\text{м}$.
    Определить длину волны.
}
\solutionspace{100pt}

\tasknumber{2}%
\task{%
    Расстояние между двумя точечными когерентными источниками света $S_1$ u $S_2$ равно $2\,\text{мм}$.
    Источники расположены в плоскости, параллельной экрану, на расстоянии $9\,\text{м}$ от него.
    На экране в точках, лежащих на перпендикулярах, опущенных из источников света $S_1$ и $S_2$,
    находятся два ближайших минимума (тёмные полосы).
    Определите длину световой волны.
    Ответ дать в нанометрах.
}
\solutionspace{100pt}

\tasknumber{3}%
\task{%
    На дифракционную решетку, имеющую период $2 \cdot 10^{-4}\,\text{см}$, нормально падает монохроматическая световая волна.
    Под углом $ 25 \degrees$ наблюдается дифракционный максимум второго порядка.
    Какова длина волны падающего света?
}
\solutionspace{100pt}

\tasknumber{4}%
\task{%
    Свет с длиной волны $0{,}5\,\text{мкм}$ падает нормально на дифракционную решетку с периодом, равным $3\,\text{мкм}$.
    Под каким углом наблюдается дифракционный максимум первого порядка?
}
\solutionspace{100pt}

\tasknumber{5}%
\task{%
    При нормальном падении белого света на дифракционную решетку зелёная линия ($500\,\text{нм}$)
    в спектре второго порядка видна под углом дифракции $25\degrees$.
    Определить число штрихов на $1\,\text{см}$ длины этой решетки.
}
\solutionspace{100pt}

\tasknumber{6}%
\task{%
    Каков наибольший порядок спектра, который можно наблюдать при дифракции света
    с длиной волны $\lambda$, на дифракционной решетке с периодом $d =  2{,}5 \lambda$?
}
\solutionspace{100pt}

\tasknumber{7}%
\task{%
    Вертикально стоящий шест высотой 1,1 м, освещенный солнцем,
    отбрасывает на горизонтальную поверхность земли тень длиной $2\,\text{м}$.
    Известно, что длина тени от телеграфного столба на $5\,\text{м}$ больше.
    Определить высоту столба.
}
\solutionspace{180pt}

\tasknumber{8}%
\task{%
    Определить абсолютный показатель преломления прозрачной среды,
    в которой распространяется свет с длиной волны $0{,}500\,\text{мкм}$ и частотой $600\,\text{ТГц}$.
    Скорость света в вакууме $3 \cdot 10^{8}\,\frac{\text{м}}{\text{с}}$.
}
\solutionspace{180pt}

\tasknumber{9}%
\task{%
    На дне водоема глубиной $3\,\text{м}$ лежит зеркало.
    Луч света, пройдя через воду, отражается от зеркала и выходит из воды.
    Найти расстояние между точкой входа луча в воду и точкой выхода луча из воды,
    если показатель преломления воды $1{,}33$, а угол падения луча $30\degrees$.
}
\solutionspace{180pt}

\tasknumber{10}%
\task{%
    Луч света падает на горизонтально расположенную стеклянную пластинку толщиной $5\,\text{мм}$.
    Пройдя через пластину, он выходит из неё в точке, смещённой по горизонтали от точки падения на расстояние $1{,}4\,\text{мм}$
    Показатель преломления стекла $1{,}5$.
    Найти синус угла падения, округлив его значение до двух знаков после запятой.
}
\solutionspace{180pt}

\tasknumber{11}%
\task{%
    Найти оптическую силу собирающей линзы, если действительное изображение предмета,
    помещённого в $35\,\text{см}$ от линзы, получается на расстоянии $20\,\text{см}$ от неё.
}
\solutionspace{180pt}

\tasknumber{12}%
\task{%
    Найти увеличение изображения, если изображение предмета, находящегося
    на расстоянии $20\,\text{см}$ от линзы, получается на расстоянии $12\,\text{см}$ от неё.
}
\solutionspace{180pt}

\tasknumber{13}%
\task{%
    Расстояние от предмета до линзы $10\,\text{см}$, а от линзы до мнимого изображения $20\,\text{см}$.
    Чему равно фокусное расстояние линзы?
}
\solutionspace{180pt}

\tasknumber{14}%
\task{%
    Две тонкие линзы с фокусными расстояниями $25\,\text{см}$ и $20\,\text{см}$ сложены вместе.
    Чему равно фокусное расстояние такой оптической системы?
}
\solutionspace{180pt}

\tasknumber{15}%
\task{%
    Линейные размеры прямого изображения предмета, полученного в собирающей линзе,
    в три раза больше линейных размеров предмета.
    Зная, что предмет находится на $20\,\text{см}$ ближе к линзе,
    чем его изображение, найти оптическую силу линзы.
}
\solutionspace{180pt}

\tasknumber{16}%
\task{%
    Оптическая сила объектива фотоаппарата равна $6\,\text{дптр}$.
    При фотографировании чертежа с расстояния $1{,}1\,\text{м}$ площадь изображения
    чертежа на фотопластинке оказалась равной $9\,\text{см}^{2}$.
    Какова площадь самого чертежа? Ответ выразите в квадратных сантиметрах.
}

\variantsplitter

\addpersonalvariant{Михаил Ярошевский}

\tasknumber{1}%
\task{%
    Разность фаз двух интерферирующих световых волн равна $3\pi$, а разность хода между ними равна $7{,}5 \cdot 10^{-7}\,\text{м}$.
    Определить длину волны.
}
\solutionspace{100pt}

\tasknumber{2}%
\task{%
    Расстояние между двумя точечными когерентными источниками света $S_1$ u $S_2$ равно $2{,}5\,\text{мм}$.
    Источники расположены в плоскости, параллельной экрану, на расстоянии $9\,\text{м}$ от него.
    На экране в точках, лежащих на перпендикулярах, опущенных из источников света $S_1$ и $S_2$,
    находятся два ближайших минимума (тёмные полосы).
    Определите длину световой волны.
    Ответ дать в нанометрах.
}
\solutionspace{100pt}

\tasknumber{3}%
\task{%
    На дифракционную решетку, имеющую период $4 \cdot 10^{-4}\,\text{см}$, нормально падает монохроматическая световая волна.
    Под углом $ 25 \degrees$ наблюдается дифракционный максимум второго порядка.
    Какова длина волны падающего света?
}
\solutionspace{100pt}

\tasknumber{4}%
\task{%
    Свет с длиной волны $0{,}4\,\text{мкм}$ падает нормально на дифракционную решетку с периодом, равным $2\,\text{мкм}$.
    Под каким углом наблюдается дифракционный максимум первого порядка?
}
\solutionspace{100pt}

\tasknumber{5}%
\task{%
    При нормальном падении белого света на дифракционную решетку зелёная линия ($500\,\text{нм}$)
    в спектре второго порядка видна под углом дифракции $20\degrees$.
    Определить число штрихов на $1\,\text{мм}$ длины этой решетки.
}
\solutionspace{100pt}

\tasknumber{6}%
\task{%
    Каков наибольший порядок спектра, который можно наблюдать при дифракции света
    с длиной волны $\lambda$, на дифракционной решетке с периодом $d =  2{,}5 \lambda$?
}
\solutionspace{100pt}

\tasknumber{7}%
\task{%
    Вертикально стоящий шест высотой 1,1 м, освещенный солнцем,
    отбрасывает на горизонтальную поверхность земли тень длиной $3\,\text{м}$.
    Известно, что длина тени от телеграфного столба на $5\,\text{м}$ больше.
    Определить высоту столба.
}
\solutionspace{180pt}

\tasknumber{8}%
\task{%
    Определить абсолютный показатель преломления прозрачной среды,
    в которой распространяется свет с длиной волны $0{,}600\,\text{мкм}$ и частотой $500\,\text{ТГц}$.
    Скорость света в вакууме $3 \cdot 10^{8}\,\frac{\text{м}}{\text{с}}$.
}
\solutionspace{180pt}

\tasknumber{9}%
\task{%
    На дне водоема глубиной $4\,\text{м}$ лежит зеркало.
    Луч света, пройдя через воду, отражается от зеркала и выходит из воды.
    Найти расстояние между точкой входа луча в воду и точкой выхода луча из воды,
    если показатель преломления воды $1{,}33$, а угол падения луча $35\degrees$.
}
\solutionspace{180pt}

\tasknumber{10}%
\task{%
    Луч света падает на горизонтально расположенную стеклянную пластинку толщиной $5\,\text{мм}$.
    Пройдя через пластину, он выходит из неё в точке, смещённой по горизонтали от точки падения на расстояние $1{,}2\,\text{мм}$
    Показатель преломления стекла $1{,}6$.
    Найти синус угла падения, округлив его значение до двух знаков после запятой.
}
\solutionspace{180pt}

\tasknumber{11}%
\task{%
    Найти оптическую силу собирающей линзы, если действительное изображение предмета,
    помещённого в $55\,\text{см}$ от линзы, получается на расстоянии $20\,\text{см}$ от неё.
}
\solutionspace{180pt}

\tasknumber{12}%
\task{%
    Найти увеличение изображения, если изображение предмета, находящегося
    на расстоянии $15\,\text{см}$ от линзы, получается на расстоянии $12\,\text{см}$ от неё.
}
\solutionspace{180pt}

\tasknumber{13}%
\task{%
    Расстояние от предмета до линзы $8\,\text{см}$, а от линзы до мнимого изображения $20\,\text{см}$.
    Чему равно фокусное расстояние линзы?
}
\solutionspace{180pt}

\tasknumber{14}%
\task{%
    Две тонкие линзы с фокусными расстояниями $18\,\text{см}$ и $20\,\text{см}$ сложены вместе.
    Чему равно фокусное расстояние такой оптической системы?
}
\solutionspace{180pt}

\tasknumber{15}%
\task{%
    Линейные размеры прямого изображения предмета, полученного в собирающей линзе,
    в три раза больше линейных размеров предмета.
    Зная, что предмет находится на $20\,\text{см}$ ближе к линзе,
    чем его изображение, найти оптическую силу линзы.
}
\solutionspace{180pt}

\tasknumber{16}%
\task{%
    Оптическая сила объектива фотоаппарата равна $6\,\text{дптр}$.
    При фотографировании чертежа с расстояния $1{,}2\,\text{м}$ площадь изображения
    чертежа на фотопластинке оказалась равной $9\,\text{см}^{2}$.
    Какова площадь самого чертежа? Ответ выразите в квадратных сантиметрах.
}

\variantsplitter

\addpersonalvariant{Алексей Алимпиев}

\tasknumber{1}%
\task{%
    Разность фаз двух интерферирующих световых волн равна $7\pi$, а разность хода между ними равна $7{,}5 \cdot 10^{-7}\,\text{м}$.
    Определить длину волны.
}
\solutionspace{100pt}

\tasknumber{2}%
\task{%
    Расстояние между двумя точечными когерентными источниками света $S_1$ u $S_2$ равно $2\,\text{мм}$.
    Источники расположены в плоскости, параллельной экрану, на расстоянии $9\,\text{м}$ от него.
    На экране в точках, лежащих на перпендикулярах, опущенных из источников света $S_1$ и $S_2$,
    находятся два ближайших минимума (тёмные полосы).
    Определите длину световой волны.
    Ответ дать в нанометрах.
}
\solutionspace{100pt}

\tasknumber{3}%
\task{%
    На дифракционную решетку, имеющую период $4 \cdot 10^{-4}\,\text{см}$, нормально падает монохроматическая световая волна.
    Под углом $ 25 \degrees$ наблюдается дифракционный максимум второго порядка.
    Какова длина волны падающего света?
}
\solutionspace{100pt}

\tasknumber{4}%
\task{%
    Свет с длиной волны $0{,}6\,\text{мкм}$ падает нормально на дифракционную решетку с периодом, равным $1\,\text{мкм}$.
    Под каким углом наблюдается дифракционный максимум первого порядка?
}
\solutionspace{100pt}

\tasknumber{5}%
\task{%
    При нормальном падении белого света на дифракционную решетку зелёная линия ($500\,\text{нм}$)
    в спектре второго порядка видна под углом дифракции $53\degrees$.
    Определить число штрихов на $1\,\text{см}$ длины этой решетки.
}
\solutionspace{100pt}

\tasknumber{6}%
\task{%
    Каков наибольший порядок спектра, который можно наблюдать при дифракции света
    с длиной волны $\lambda$, на дифракционной решетке с периодом $d =  3{,}5 \lambda$?
}
\solutionspace{100pt}

\tasknumber{7}%
\task{%
    Вертикально стоящий шест высотой 1,1 м, освещенный солнцем,
    отбрасывает на горизонтальную поверхность земли тень длиной $4\,\text{м}$.
    Известно, что длина тени от телеграфного столба на $8\,\text{м}$ больше.
    Определить высоту столба.
}
\solutionspace{180pt}

\tasknumber{8}%
\task{%
    Определить абсолютный показатель преломления прозрачной среды,
    в которой распространяется свет с длиной волны $0{,}500\,\text{мкм}$ и частотой $600\,\text{ТГц}$.
    Скорость света в вакууме $3 \cdot 10^{8}\,\frac{\text{м}}{\text{с}}$.
}
\solutionspace{180pt}

\tasknumber{9}%
\task{%
    На дне водоема глубиной $4\,\text{м}$ лежит зеркало.
    Луч света, пройдя через воду, отражается от зеркала и выходит из воды.
    Найти расстояние между точкой входа луча в воду и точкой выхода луча из воды,
    если показатель преломления воды $1{,}33$, а угол падения луча $25\degrees$.
}
\solutionspace{180pt}

\tasknumber{10}%
\task{%
    Луч света падает на горизонтально расположенную стеклянную пластинку толщиной $6\,\text{мм}$.
    Пройдя через пластину, он выходит из неё в точке, смещённой по горизонтали от точки падения на расстояние $1{,}6\,\text{мм}$
    Показатель преломления стекла $1{,}6$.
    Найти синус угла падения, округлив его значение до двух знаков после запятой.
}
\solutionspace{180pt}

\tasknumber{11}%
\task{%
    Найти оптическую силу собирающей линзы, если действительное изображение предмета,
    помещённого в $15\,\text{см}$ от линзы, получается на расстоянии $20\,\text{см}$ от неё.
}
\solutionspace{180pt}

\tasknumber{12}%
\task{%
    Найти увеличение изображения, если изображение предмета, находящегося
    на расстоянии $15\,\text{см}$ от линзы, получается на расстоянии $12\,\text{см}$ от неё.
}
\solutionspace{180pt}

\tasknumber{13}%
\task{%
    Расстояние от предмета до линзы $8\,\text{см}$, а от линзы до мнимого изображения $20\,\text{см}$.
    Чему равно фокусное расстояние линзы?
}
\solutionspace{180pt}

\tasknumber{14}%
\task{%
    Две тонкие линзы с фокусными расстояниями $25\,\text{см}$ и $20\,\text{см}$ сложены вместе.
    Чему равно фокусное расстояние такой оптической системы?
}
\solutionspace{180pt}

\tasknumber{15}%
\task{%
    Линейные размеры прямого изображения предмета, полученного в собирающей линзе,
    в четыре раза больше линейных размеров предмета.
    Зная, что предмет находится на $40\,\text{см}$ ближе к линзе,
    чем его изображение, найти оптическую силу линзы.
}
\solutionspace{180pt}

\tasknumber{16}%
\task{%
    Оптическая сила объектива фотоаппарата равна $5\,\text{дптр}$.
    При фотографировании чертежа с расстояния $0{,}9\,\text{м}$ площадь изображения
    чертежа на фотопластинке оказалась равной $9\,\text{см}^{2}$.
    Какова площадь самого чертежа? Ответ выразите в квадратных сантиметрах.
}

\variantsplitter

\addpersonalvariant{Евгений Васин}

\tasknumber{1}%
\task{%
    Разность фаз двух интерферирующих световых волн равна $3\pi$, а разность хода между ними равна $12{,}5 \cdot 10^{-7}\,\text{м}$.
    Определить длину волны.
}
\solutionspace{100pt}

\tasknumber{2}%
\task{%
    Расстояние между двумя точечными когерентными источниками света $S_1$ u $S_2$ равно $2{,}5\,\text{мм}$.
    Источники расположены в плоскости, параллельной экрану, на расстоянии $8\,\text{м}$ от него.
    На экране в точках, лежащих на перпендикулярах, опущенных из источников света $S_1$ и $S_2$,
    находятся два ближайших минимума (тёмные полосы).
    Определите длину световой волны.
    Ответ дать в нанометрах.
}
\solutionspace{100pt}

\tasknumber{3}%
\task{%
    На дифракционную решетку, имеющую период $2 \cdot 10^{-4}\,\text{см}$, нормально падает монохроматическая световая волна.
    Под углом $ 30 \degrees$ наблюдается дифракционный максимум второго порядка.
    Какова длина волны падающего света?
}
\solutionspace{100pt}

\tasknumber{4}%
\task{%
    Свет с длиной волны $0{,}6\,\text{мкм}$ падает нормально на дифракционную решетку с периодом, равным $2\,\text{мкм}$.
    Под каким углом наблюдается дифракционный максимум первого порядка?
}
\solutionspace{100pt}

\tasknumber{5}%
\task{%
    При нормальном падении белого света на дифракционную решетку зелёная линия ($500\,\text{нм}$)
    в спектре второго порядка видна под углом дифракции $20\degrees$.
    Определить число штрихов на $1\,\text{см}$ длины этой решетки.
}
\solutionspace{100pt}

\tasknumber{6}%
\task{%
    Каков наибольший порядок спектра, который можно наблюдать при дифракции света
    с длиной волны $\lambda$, на дифракционной решетке с периодом $d =  4{,}5 \lambda$?
}
\solutionspace{100pt}

\tasknumber{7}%
\task{%
    Вертикально стоящий шест высотой 1,1 м, освещенный солнцем,
    отбрасывает на горизонтальную поверхность земли тень длиной $4\,\text{м}$.
    Известно, что длина тени от телеграфного столба на $8\,\text{м}$ больше.
    Определить высоту столба.
}
\solutionspace{180pt}

\tasknumber{8}%
\task{%
    Определить абсолютный показатель преломления прозрачной среды,
    в которой распространяется свет с длиной волны $0{,}600\,\text{мкм}$ и частотой $500\,\text{ТГц}$.
    Скорость света в вакууме $3 \cdot 10^{8}\,\frac{\text{м}}{\text{с}}$.
}
\solutionspace{180pt}

\tasknumber{9}%
\task{%
    На дне водоема глубиной $3\,\text{м}$ лежит зеркало.
    Луч света, пройдя через воду, отражается от зеркала и выходит из воды.
    Найти расстояние между точкой входа луча в воду и точкой выхода луча из воды,
    если показатель преломления воды $1{,}33$, а угол падения луча $25\degrees$.
}
\solutionspace{180pt}

\tasknumber{10}%
\task{%
    Луч света падает на горизонтально расположенную стеклянную пластинку толщиной $5\,\text{мм}$.
    Пройдя через пластину, он выходит из неё в точке, смещённой по горизонтали от точки падения на расстояние $1{,}3\,\text{мм}$
    Показатель преломления стекла $1{,}5$.
    Найти синус угла падения, округлив его значение до двух знаков после запятой.
}
\solutionspace{180pt}

\tasknumber{11}%
\task{%
    Найти оптическую силу собирающей линзы, если действительное изображение предмета,
    помещённого в $35\,\text{см}$ от линзы, получается на расстоянии $40\,\text{см}$ от неё.
}
\solutionspace{180pt}

\tasknumber{12}%
\task{%
    Найти увеличение изображения, если изображение предмета, находящегося
    на расстоянии $20\,\text{см}$ от линзы, получается на расстоянии $12\,\text{см}$ от неё.
}
\solutionspace{180pt}

\tasknumber{13}%
\task{%
    Расстояние от предмета до линзы $10\,\text{см}$, а от линзы до мнимого изображения $20\,\text{см}$.
    Чему равно фокусное расстояние линзы?
}
\solutionspace{180pt}

\tasknumber{14}%
\task{%
    Две тонкие линзы с фокусными расстояниями $12\,\text{см}$ и $30\,\text{см}$ сложены вместе.
    Чему равно фокусное расстояние такой оптической системы?
}
\solutionspace{180pt}

\tasknumber{15}%
\task{%
    Линейные размеры прямого изображения предмета, полученного в собирающей линзе,
    в четыре раза больше линейных размеров предмета.
    Зная, что предмет находится на $40\,\text{см}$ ближе к линзе,
    чем его изображение, найти оптическую силу линзы.
}
\solutionspace{180pt}

\tasknumber{16}%
\task{%
    Оптическая сила объектива фотоаппарата равна $3\,\text{дптр}$.
    При фотографировании чертежа с расстояния $1{,}2\,\text{м}$ площадь изображения
    чертежа на фотопластинке оказалась равной $4\,\text{см}^{2}$.
    Какова площадь самого чертежа? Ответ выразите в квадратных сантиметрах.
}

\variantsplitter

\addpersonalvariant{Вячеслав Волохов}

\tasknumber{1}%
\task{%
    Разность фаз двух интерферирующих световых волн равна $3\pi$, а разность хода между ними равна $7{,}5 \cdot 10^{-7}\,\text{м}$.
    Определить длину волны.
}
\solutionspace{100pt}

\tasknumber{2}%
\task{%
    Расстояние между двумя точечными когерентными источниками света $S_1$ u $S_2$ равно $2\,\text{мм}$.
    Источники расположены в плоскости, параллельной экрану, на расстоянии $8\,\text{м}$ от него.
    На экране в точках, лежащих на перпендикулярах, опущенных из источников света $S_1$ и $S_2$,
    находятся два ближайших минимума (тёмные полосы).
    Определите длину световой волны.
    Ответ дать в нанометрах.
}
\solutionspace{100pt}

\tasknumber{3}%
\task{%
    На дифракционную решетку, имеющую период $2 \cdot 10^{-4}\,\text{см}$, нормально падает монохроматическая световая волна.
    Под углом $ 35 \degrees$ наблюдается дифракционный максимум второго порядка.
    Какова длина волны падающего света?
}
\solutionspace{100pt}

\tasknumber{4}%
\task{%
    Свет с длиной волны $0{,}6\,\text{мкм}$ падает нормально на дифракционную решетку с периодом, равным $3\,\text{мкм}$.
    Под каким углом наблюдается дифракционный максимум первого порядка?
}
\solutionspace{100pt}

\tasknumber{5}%
\task{%
    При нормальном падении белого света на дифракционную решетку зелёная линия ($500\,\text{нм}$)
    в спектре второго порядка видна под углом дифракции $25\degrees$.
    Определить число штрихов на $1\,\text{см}$ длины этой решетки.
}
\solutionspace{100pt}

\tasknumber{6}%
\task{%
    Каков наибольший порядок спектра, который можно наблюдать при дифракции света
    с длиной волны $\lambda$, на дифракционной решетке с периодом $d =  2{,}5 \lambda$?
}
\solutionspace{100pt}

\tasknumber{7}%
\task{%
    Вертикально стоящий шест высотой 1,1 м, освещенный солнцем,
    отбрасывает на горизонтальную поверхность земли тень длиной $2\,\text{м}$.
    Известно, что длина тени от телеграфного столба на $5\,\text{м}$ больше.
    Определить высоту столба.
}
\solutionspace{180pt}

\tasknumber{8}%
\task{%
    Определить абсолютный показатель преломления прозрачной среды,
    в которой распространяется свет с длиной волны $0{,}600\,\text{мкм}$ и частотой $500\,\text{ТГц}$.
    Скорость света в вакууме $3 \cdot 10^{8}\,\frac{\text{м}}{\text{с}}$.
}
\solutionspace{180pt}

\tasknumber{9}%
\task{%
    На дне водоема глубиной $2\,\text{м}$ лежит зеркало.
    Луч света, пройдя через воду, отражается от зеркала и выходит из воды.
    Найти расстояние между точкой входа луча в воду и точкой выхода луча из воды,
    если показатель преломления воды $1{,}33$, а угол падения луча $35\degrees$.
}
\solutionspace{180pt}

\tasknumber{10}%
\task{%
    Луч света падает на горизонтально расположенную стеклянную пластинку толщиной $6\,\text{мм}$.
    Пройдя через пластину, он выходит из неё в точке, смещённой по горизонтали от точки падения на расстояние $1{,}2\,\text{мм}$
    Показатель преломления стекла $1{,}6$.
    Найти синус угла падения, округлив его значение до двух знаков после запятой.
}
\solutionspace{180pt}

\tasknumber{11}%
\task{%
    Найти оптическую силу собирающей линзы, если действительное изображение предмета,
    помещённого в $15\,\text{см}$ от линзы, получается на расстоянии $20\,\text{см}$ от неё.
}
\solutionspace{180pt}

\tasknumber{12}%
\task{%
    Найти увеличение изображения, если изображение предмета, находящегося
    на расстоянии $15\,\text{см}$ от линзы, получается на расстоянии $30\,\text{см}$ от неё.
}
\solutionspace{180pt}

\tasknumber{13}%
\task{%
    Расстояние от предмета до линзы $8\,\text{см}$, а от линзы до мнимого изображения $30\,\text{см}$.
    Чему равно фокусное расстояние линзы?
}
\solutionspace{180pt}

\tasknumber{14}%
\task{%
    Две тонкие линзы с фокусными расстояниями $18\,\text{см}$ и $30\,\text{см}$ сложены вместе.
    Чему равно фокусное расстояние такой оптической системы?
}
\solutionspace{180pt}

\tasknumber{15}%
\task{%
    Линейные размеры прямого изображения предмета, полученного в собирающей линзе,
    в два раза больше линейных размеров предмета.
    Зная, что предмет находится на $20\,\text{см}$ ближе к линзе,
    чем его изображение, найти оптическую силу линзы.
}
\solutionspace{180pt}

\tasknumber{16}%
\task{%
    Оптическая сила объектива фотоаппарата равна $5\,\text{дптр}$.
    При фотографировании чертежа с расстояния $0{,}9\,\text{м}$ площадь изображения
    чертежа на фотопластинке оказалась равной $9\,\text{см}^{2}$.
    Какова площадь самого чертежа? Ответ выразите в квадратных сантиметрах.
}

\variantsplitter

\addpersonalvariant{Герман Говоров}

\tasknumber{1}%
\task{%
    Разность фаз двух интерферирующих световых волн равна $5\pi$, а разность хода между ними равна $15{,}5 \cdot 10^{-7}\,\text{м}$.
    Определить длину волны.
}
\solutionspace{100pt}

\tasknumber{2}%
\task{%
    Расстояние между двумя точечными когерентными источниками света $S_1$ u $S_2$ равно $2{,}5\,\text{мм}$.
    Источники расположены в плоскости, параллельной экрану, на расстоянии $8\,\text{м}$ от него.
    На экране в точках, лежащих на перпендикулярах, опущенных из источников света $S_1$ и $S_2$,
    находятся два ближайших минимума (тёмные полосы).
    Определите длину световой волны.
    Ответ дать в нанометрах.
}
\solutionspace{100pt}

\tasknumber{3}%
\task{%
    На дифракционную решетку, имеющую период $2 \cdot 10^{-4}\,\text{см}$, нормально падает монохроматическая световая волна.
    Под углом $ 35 \degrees$ наблюдается дифракционный максимум второго порядка.
    Какова длина волны падающего света?
}
\solutionspace{100pt}

\tasknumber{4}%
\task{%
    Свет с длиной волны $0{,}7\,\text{мкм}$ падает нормально на дифракционную решетку с периодом, равным $3\,\text{мкм}$.
    Под каким углом наблюдается дифракционный максимум первого порядка?
}
\solutionspace{100pt}

\tasknumber{5}%
\task{%
    При нормальном падении белого света на дифракционную решетку зелёная линия ($500\,\text{нм}$)
    в спектре второго порядка видна под углом дифракции $20\degrees$.
    Определить число штрихов на $1\,\text{мм}$ длины этой решетки.
}
\solutionspace{100pt}

\tasknumber{6}%
\task{%
    Каков наибольший порядок спектра, который можно наблюдать при дифракции света
    с длиной волны $\lambda$, на дифракционной решетке с периодом $d =  3{,}5 \lambda$?
}
\solutionspace{100pt}

\tasknumber{7}%
\task{%
    Вертикально стоящий шест высотой 1,1 м, освещенный солнцем,
    отбрасывает на горизонтальную поверхность земли тень длиной $3\,\text{м}$.
    Известно, что длина тени от телеграфного столба на $5\,\text{м}$ больше.
    Определить высоту столба.
}
\solutionspace{180pt}

\tasknumber{8}%
\task{%
    Определить абсолютный показатель преломления прозрачной среды,
    в которой распространяется свет с длиной волны $0{,}500\,\text{мкм}$ и частотой $600\,\text{ТГц}$.
    Скорость света в вакууме $3 \cdot 10^{8}\,\frac{\text{м}}{\text{с}}$.
}
\solutionspace{180pt}

\tasknumber{9}%
\task{%
    На дне водоема глубиной $2\,\text{м}$ лежит зеркало.
    Луч света, пройдя через воду, отражается от зеркала и выходит из воды.
    Найти расстояние между точкой входа луча в воду и точкой выхода луча из воды,
    если показатель преломления воды $1{,}33$, а угол падения луча $25\degrees$.
}
\solutionspace{180pt}

\tasknumber{10}%
\task{%
    Луч света падает на горизонтально расположенную стеклянную пластинку толщиной $5\,\text{мм}$.
    Пройдя через пластину, он выходит из неё в точке, смещённой по горизонтали от точки падения на расстояние $1{,}6\,\text{мм}$
    Показатель преломления стекла $1{,}6$.
    Найти синус угла падения, округлив его значение до двух знаков после запятой.
}
\solutionspace{180pt}

\tasknumber{11}%
\task{%
    Найти оптическую силу собирающей линзы, если действительное изображение предмета,
    помещённого в $15\,\text{см}$ от линзы, получается на расстоянии $40\,\text{см}$ от неё.
}
\solutionspace{180pt}

\tasknumber{12}%
\task{%
    Найти увеличение изображения, если изображение предмета, находящегося
    на расстоянии $25\,\text{см}$ от линзы, получается на расстоянии $18\,\text{см}$ от неё.
}
\solutionspace{180pt}

\tasknumber{13}%
\task{%
    Расстояние от предмета до линзы $12\,\text{см}$, а от линзы до мнимого изображения $25\,\text{см}$.
    Чему равно фокусное расстояние линзы?
}
\solutionspace{180pt}

\tasknumber{14}%
\task{%
    Две тонкие линзы с фокусными расстояниями $25\,\text{см}$ и $20\,\text{см}$ сложены вместе.
    Чему равно фокусное расстояние такой оптической системы?
}
\solutionspace{180pt}

\tasknumber{15}%
\task{%
    Линейные размеры прямого изображения предмета, полученного в собирающей линзе,
    в четыре раза больше линейных размеров предмета.
    Зная, что предмет находится на $35\,\text{см}$ ближе к линзе,
    чем его изображение, найти оптическую силу линзы.
}
\solutionspace{180pt}

\tasknumber{16}%
\task{%
    Оптическая сила объектива фотоаппарата равна $6\,\text{дптр}$.
    При фотографировании чертежа с расстояния $1{,}2\,\text{м}$ площадь изображения
    чертежа на фотопластинке оказалась равной $9\,\text{см}^{2}$.
    Какова площадь самого чертежа? Ответ выразите в квадратных сантиметрах.
}

\variantsplitter

\addpersonalvariant{София Журавлёва}

\tasknumber{1}%
\task{%
    Разность фаз двух интерферирующих световых волн равна $7\pi$, а разность хода между ними равна $12{,}5 \cdot 10^{-7}\,\text{м}$.
    Определить длину волны.
}
\solutionspace{100pt}

\tasknumber{2}%
\task{%
    Расстояние между двумя точечными когерентными источниками света $S_1$ u $S_2$ равно $2\,\text{мм}$.
    Источники расположены в плоскости, параллельной экрану, на расстоянии $9\,\text{м}$ от него.
    На экране в точках, лежащих на перпендикулярах, опущенных из источников света $S_1$ и $S_2$,
    находятся два ближайших минимума (тёмные полосы).
    Определите длину световой волны.
    Ответ дать в нанометрах.
}
\solutionspace{100pt}

\tasknumber{3}%
\task{%
    На дифракционную решетку, имеющую период $4 \cdot 10^{-4}\,\text{см}$, нормально падает монохроматическая световая волна.
    Под углом $ 40 \degrees$ наблюдается дифракционный максимум второго порядка.
    Какова длина волны падающего света?
}
\solutionspace{100pt}

\tasknumber{4}%
\task{%
    Свет с длиной волны $0{,}7\,\text{мкм}$ падает нормально на дифракционную решетку с периодом, равным $1\,\text{мкм}$.
    Под каким углом наблюдается дифракционный максимум первого порядка?
}
\solutionspace{100pt}

\tasknumber{5}%
\task{%
    При нормальном падении белого света на дифракционную решетку зелёная линия ($500\,\text{нм}$)
    в спектре второго порядка видна под углом дифракции $53\degrees$.
    Определить число штрихов на $1\,\text{мм}$ длины этой решетки.
}
\solutionspace{100pt}

\tasknumber{6}%
\task{%
    Каков наибольший порядок спектра, который можно наблюдать при дифракции света
    с длиной волны $\lambda$, на дифракционной решетке с периодом $d =  3{,}5 \lambda$?
}
\solutionspace{100pt}

\tasknumber{7}%
\task{%
    Вертикально стоящий шест высотой 1,1 м, освещенный солнцем,
    отбрасывает на горизонтальную поверхность земли тень длиной $2\,\text{м}$.
    Известно, что длина тени от телеграфного столба на $9\,\text{м}$ больше.
    Определить высоту столба.
}
\solutionspace{180pt}

\tasknumber{8}%
\task{%
    Определить абсолютный показатель преломления прозрачной среды,
    в которой распространяется свет с длиной волны $0{,}450\,\text{мкм}$ и частотой $667\,\text{ТГц}$.
    Скорость света в вакууме $3 \cdot 10^{8}\,\frac{\text{м}}{\text{с}}$.
}
\solutionspace{180pt}

\tasknumber{9}%
\task{%
    На дне водоема глубиной $4\,\text{м}$ лежит зеркало.
    Луч света, пройдя через воду, отражается от зеркала и выходит из воды.
    Найти расстояние между точкой входа луча в воду и точкой выхода луча из воды,
    если показатель преломления воды $1{,}33$, а угол падения луча $35\degrees$.
}
\solutionspace{180pt}

\tasknumber{10}%
\task{%
    Луч света падает на горизонтально расположенную стеклянную пластинку толщиной $5\,\text{мм}$.
    Пройдя через пластину, он выходит из неё в точке, смещённой по горизонтали от точки падения на расстояние $1{,}3\,\text{мм}$
    Показатель преломления стекла $1{,}6$.
    Найти синус угла падения, округлив его значение до двух знаков после запятой.
}
\solutionspace{180pt}

\tasknumber{11}%
\task{%
    Найти оптическую силу собирающей линзы, если действительное изображение предмета,
    помещённого в $35\,\text{см}$ от линзы, получается на расстоянии $40\,\text{см}$ от неё.
}
\solutionspace{180pt}

\tasknumber{12}%
\task{%
    Найти увеличение изображения, если изображение предмета, находящегося
    на расстоянии $15\,\text{см}$ от линзы, получается на расстоянии $30\,\text{см}$ от неё.
}
\solutionspace{180pt}

\tasknumber{13}%
\task{%
    Расстояние от предмета до линзы $8\,\text{см}$, а от линзы до мнимого изображения $30\,\text{см}$.
    Чему равно фокусное расстояние линзы?
}
\solutionspace{180pt}

\tasknumber{14}%
\task{%
    Две тонкие линзы с фокусными расстояниями $12\,\text{см}$ и $20\,\text{см}$ сложены вместе.
    Чему равно фокусное расстояние такой оптической системы?
}
\solutionspace{180pt}

\tasknumber{15}%
\task{%
    Линейные размеры прямого изображения предмета, полученного в собирающей линзе,
    в четыре раза больше линейных размеров предмета.
    Зная, что предмет находится на $40\,\text{см}$ ближе к линзе,
    чем его изображение, найти оптическую силу линзы.
}
\solutionspace{180pt}

\tasknumber{16}%
\task{%
    Оптическая сила объектива фотоаппарата равна $4\,\text{дптр}$.
    При фотографировании чертежа с расстояния $1{,}1\,\text{м}$ площадь изображения
    чертежа на фотопластинке оказалась равной $16\,\text{см}^{2}$.
    Какова площадь самого чертежа? Ответ выразите в квадратных сантиметрах.
}

\variantsplitter

\addpersonalvariant{Константин Козлов}

\tasknumber{1}%
\task{%
    Разность фаз двух интерферирующих световых волн равна $3\pi$, а разность хода между ними равна $12{,}5 \cdot 10^{-7}\,\text{м}$.
    Определить длину волны.
}
\solutionspace{100pt}

\tasknumber{2}%
\task{%
    Расстояние между двумя точечными когерентными источниками света $S_1$ u $S_2$ равно $2{,}5\,\text{мм}$.
    Источники расположены в плоскости, параллельной экрану, на расстоянии $9\,\text{м}$ от него.
    На экране в точках, лежащих на перпендикулярах, опущенных из источников света $S_1$ и $S_2$,
    находятся два ближайших минимума (тёмные полосы).
    Определите длину световой волны.
    Ответ дать в нанометрах.
}
\solutionspace{100pt}

\tasknumber{3}%
\task{%
    На дифракционную решетку, имеющую период $3 \cdot 10^{-4}\,\text{см}$, нормально падает монохроматическая световая волна.
    Под углом $ 20 \degrees$ наблюдается дифракционный максимум второго порядка.
    Какова длина волны падающего света?
}
\solutionspace{100pt}

\tasknumber{4}%
\task{%
    Свет с длиной волны $0{,}7\,\text{мкм}$ падает нормально на дифракционную решетку с периодом, равным $3\,\text{мкм}$.
    Под каким углом наблюдается дифракционный максимум первого порядка?
}
\solutionspace{100pt}

\tasknumber{5}%
\task{%
    При нормальном падении белого света на дифракционную решетку зелёная линия ($500\,\text{нм}$)
    в спектре второго порядка видна под углом дифракции $53\degrees$.
    Определить число штрихов на $1\,\text{мм}$ длины этой решетки.
}
\solutionspace{100pt}

\tasknumber{6}%
\task{%
    Каков наибольший порядок спектра, который можно наблюдать при дифракции света
    с длиной волны $\lambda$, на дифракционной решетке с периодом $d =  2{,}5 \lambda$?
}
\solutionspace{100pt}

\tasknumber{7}%
\task{%
    Вертикально стоящий шест высотой 1,1 м, освещенный солнцем,
    отбрасывает на горизонтальную поверхность земли тень длиной $4\,\text{м}$.
    Известно, что длина тени от телеграфного столба на $8\,\text{м}$ больше.
    Определить высоту столба.
}
\solutionspace{180pt}

\tasknumber{8}%
\task{%
    Определить абсолютный показатель преломления прозрачной среды,
    в которой распространяется свет с длиной волны $0{,}500\,\text{мкм}$ и частотой $600\,\text{ТГц}$.
    Скорость света в вакууме $3 \cdot 10^{8}\,\frac{\text{м}}{\text{с}}$.
}
\solutionspace{180pt}

\tasknumber{9}%
\task{%
    На дне водоема глубиной $4\,\text{м}$ лежит зеркало.
    Луч света, пройдя через воду, отражается от зеркала и выходит из воды.
    Найти расстояние между точкой входа луча в воду и точкой выхода луча из воды,
    если показатель преломления воды $1{,}33$, а угол падения луча $30\degrees$.
}
\solutionspace{180pt}

\tasknumber{10}%
\task{%
    Луч света падает на горизонтально расположенную стеклянную пластинку толщиной $5\,\text{мм}$.
    Пройдя через пластину, он выходит из неё в точке, смещённой по горизонтали от точки падения на расстояние $1{,}3\,\text{мм}$
    Показатель преломления стекла $1{,}4$.
    Найти синус угла падения, округлив его значение до двух знаков после запятой.
}
\solutionspace{180pt}

\tasknumber{11}%
\task{%
    Найти оптическую силу собирающей линзы, если действительное изображение предмета,
    помещённого в $35\,\text{см}$ от линзы, получается на расстоянии $30\,\text{см}$ от неё.
}
\solutionspace{180pt}

\tasknumber{12}%
\task{%
    Найти увеличение изображения, если изображение предмета, находящегося
    на расстоянии $15\,\text{см}$ от линзы, получается на расстоянии $18\,\text{см}$ от неё.
}
\solutionspace{180pt}

\tasknumber{13}%
\task{%
    Расстояние от предмета до линзы $8\,\text{см}$, а от линзы до мнимого изображения $25\,\text{см}$.
    Чему равно фокусное расстояние линзы?
}
\solutionspace{180pt}

\tasknumber{14}%
\task{%
    Две тонкие линзы с фокусными расстояниями $18\,\text{см}$ и $20\,\text{см}$ сложены вместе.
    Чему равно фокусное расстояние такой оптической системы?
}
\solutionspace{180pt}

\tasknumber{15}%
\task{%
    Линейные размеры прямого изображения предмета, полученного в собирающей линзе,
    в три раза больше линейных размеров предмета.
    Зная, что предмет находится на $35\,\text{см}$ ближе к линзе,
    чем его изображение, найти оптическую силу линзы.
}
\solutionspace{180pt}

\tasknumber{16}%
\task{%
    Оптическая сила объектива фотоаппарата равна $3\,\text{дптр}$.
    При фотографировании чертежа с расстояния $1{,}1\,\text{м}$ площадь изображения
    чертежа на фотопластинке оказалась равной $16\,\text{см}^{2}$.
    Какова площадь самого чертежа? Ответ выразите в квадратных сантиметрах.
}

\variantsplitter

\addpersonalvariant{Наталья Кравченко}

\tasknumber{1}%
\task{%
    Разность фаз двух интерферирующих световых волн равна $9\pi$, а разность хода между ними равна $12{,}5 \cdot 10^{-7}\,\text{м}$.
    Определить длину волны.
}
\solutionspace{100pt}

\tasknumber{2}%
\task{%
    Расстояние между двумя точечными когерентными источниками света $S_1$ u $S_2$ равно $2{,}5\,\text{мм}$.
    Источники расположены в плоскости, параллельной экрану, на расстоянии $7\,\text{м}$ от него.
    На экране в точках, лежащих на перпендикулярах, опущенных из источников света $S_1$ и $S_2$,
    находятся два ближайших минимума (тёмные полосы).
    Определите длину световой волны.
    Ответ дать в нанометрах.
}
\solutionspace{100pt}

\tasknumber{3}%
\task{%
    На дифракционную решетку, имеющую период $2 \cdot 10^{-4}\,\text{см}$, нормально падает монохроматическая световая волна.
    Под углом $ 40 \degrees$ наблюдается дифракционный максимум второго порядка.
    Какова длина волны падающего света?
}
\solutionspace{100pt}

\tasknumber{4}%
\task{%
    Свет с длиной волны $0{,}5\,\text{мкм}$ падает нормально на дифракционную решетку с периодом, равным $1\,\text{мкм}$.
    Под каким углом наблюдается дифракционный максимум первого порядка?
}
\solutionspace{100pt}

\tasknumber{5}%
\task{%
    При нормальном падении белого света на дифракционную решетку зелёная линия ($500\,\text{нм}$)
    в спектре второго порядка видна под углом дифракции $25\degrees$.
    Определить число штрихов на $1\,\text{мм}$ длины этой решетки.
}
\solutionspace{100pt}

\tasknumber{6}%
\task{%
    Каков наибольший порядок спектра, который можно наблюдать при дифракции света
    с длиной волны $\lambda$, на дифракционной решетке с периодом $d =  2{,}5 \lambda$?
}
\solutionspace{100pt}

\tasknumber{7}%
\task{%
    Вертикально стоящий шест высотой 1,1 м, освещенный солнцем,
    отбрасывает на горизонтальную поверхность земли тень длиной $4\,\text{м}$.
    Известно, что длина тени от телеграфного столба на $7\,\text{м}$ больше.
    Определить высоту столба.
}
\solutionspace{180pt}

\tasknumber{8}%
\task{%
    Определить абсолютный показатель преломления прозрачной среды,
    в которой распространяется свет с длиной волны $0{,}650\,\text{мкм}$ и частотой $462\,\text{ТГц}$.
    Скорость света в вакууме $3 \cdot 10^{8}\,\frac{\text{м}}{\text{с}}$.
}
\solutionspace{180pt}

\tasknumber{9}%
\task{%
    На дне водоема глубиной $4\,\text{м}$ лежит зеркало.
    Луч света, пройдя через воду, отражается от зеркала и выходит из воды.
    Найти расстояние между точкой входа луча в воду и точкой выхода луча из воды,
    если показатель преломления воды $1{,}33$, а угол падения луча $35\degrees$.
}
\solutionspace{180pt}

\tasknumber{10}%
\task{%
    Луч света падает на горизонтально расположенную стеклянную пластинку толщиной $6\,\text{мм}$.
    Пройдя через пластину, он выходит из неё в точке, смещённой по горизонтали от точки падения на расстояние $1{,}3\,\text{мм}$
    Показатель преломления стекла $1{,}6$.
    Найти синус угла падения, округлив его значение до двух знаков после запятой.
}
\solutionspace{180pt}

\tasknumber{11}%
\task{%
    Найти оптическую силу собирающей линзы, если действительное изображение предмета,
    помещённого в $15\,\text{см}$ от линзы, получается на расстоянии $20\,\text{см}$ от неё.
}
\solutionspace{180pt}

\tasknumber{12}%
\task{%
    Найти увеличение изображения, если изображение предмета, находящегося
    на расстоянии $15\,\text{см}$ от линзы, получается на расстоянии $30\,\text{см}$ от неё.
}
\solutionspace{180pt}

\tasknumber{13}%
\task{%
    Расстояние от предмета до линзы $8\,\text{см}$, а от линзы до мнимого изображения $30\,\text{см}$.
    Чему равно фокусное расстояние линзы?
}
\solutionspace{180pt}

\tasknumber{14}%
\task{%
    Две тонкие линзы с фокусными расстояниями $18\,\text{см}$ и $20\,\text{см}$ сложены вместе.
    Чему равно фокусное расстояние такой оптической системы?
}
\solutionspace{180pt}

\tasknumber{15}%
\task{%
    Линейные размеры прямого изображения предмета, полученного в собирающей линзе,
    в два раза больше линейных размеров предмета.
    Зная, что предмет находится на $35\,\text{см}$ ближе к линзе,
    чем его изображение, найти оптическую силу линзы.
}
\solutionspace{180pt}

\tasknumber{16}%
\task{%
    Оптическая сила объектива фотоаппарата равна $6\,\text{дптр}$.
    При фотографировании чертежа с расстояния $1{,}2\,\text{м}$ площадь изображения
    чертежа на фотопластинке оказалась равной $16\,\text{см}^{2}$.
    Какова площадь самого чертежа? Ответ выразите в квадратных сантиметрах.
}

\variantsplitter

\addpersonalvariant{Матвей Кузьмин}

\tasknumber{1}%
\task{%
    Разность фаз двух интерферирующих световых волн равна $5\pi$, а разность хода между ними равна $7{,}5 \cdot 10^{-7}\,\text{м}$.
    Определить длину волны.
}
\solutionspace{100pt}

\tasknumber{2}%
\task{%
    Расстояние между двумя точечными когерентными источниками света $S_1$ u $S_2$ равно $2\,\text{мм}$.
    Источники расположены в плоскости, параллельной экрану, на расстоянии $7\,\text{м}$ от него.
    На экране в точках, лежащих на перпендикулярах, опущенных из источников света $S_1$ и $S_2$,
    находятся два ближайших минимума (тёмные полосы).
    Определите длину световой волны.
    Ответ дать в нанометрах.
}
\solutionspace{100pt}

\tasknumber{3}%
\task{%
    На дифракционную решетку, имеющую период $3 \cdot 10^{-4}\,\text{см}$, нормально падает монохроматическая световая волна.
    Под углом $ 25 \degrees$ наблюдается дифракционный максимум второго порядка.
    Какова длина волны падающего света?
}
\solutionspace{100pt}

\tasknumber{4}%
\task{%
    Свет с длиной волны $0{,}6\,\text{мкм}$ падает нормально на дифракционную решетку с периодом, равным $2\,\text{мкм}$.
    Под каким углом наблюдается дифракционный максимум первого порядка?
}
\solutionspace{100pt}

\tasknumber{5}%
\task{%
    При нормальном падении белого света на дифракционную решетку зелёная линия ($500\,\text{нм}$)
    в спектре второго порядка видна под углом дифракции $25\degrees$.
    Определить число штрихов на $1\,\text{мм}$ длины этой решетки.
}
\solutionspace{100pt}

\tasknumber{6}%
\task{%
    Каков наибольший порядок спектра, который можно наблюдать при дифракции света
    с длиной волны $\lambda$, на дифракционной решетке с периодом $d =  3{,}5 \lambda$?
}
\solutionspace{100pt}

\tasknumber{7}%
\task{%
    Вертикально стоящий шест высотой 1,1 м, освещенный солнцем,
    отбрасывает на горизонтальную поверхность земли тень длиной $2\,\text{м}$.
    Известно, что длина тени от телеграфного столба на $6\,\text{м}$ больше.
    Определить высоту столба.
}
\solutionspace{180pt}

\tasknumber{8}%
\task{%
    Определить абсолютный показатель преломления прозрачной среды,
    в которой распространяется свет с длиной волны $0{,}600\,\text{мкм}$ и частотой $500\,\text{ТГц}$.
    Скорость света в вакууме $3 \cdot 10^{8}\,\frac{\text{м}}{\text{с}}$.
}
\solutionspace{180pt}

\tasknumber{9}%
\task{%
    На дне водоема глубиной $4\,\text{м}$ лежит зеркало.
    Луч света, пройдя через воду, отражается от зеркала и выходит из воды.
    Найти расстояние между точкой входа луча в воду и точкой выхода луча из воды,
    если показатель преломления воды $1{,}33$, а угол падения луча $35\degrees$.
}
\solutionspace{180pt}

\tasknumber{10}%
\task{%
    Луч света падает на горизонтально расположенную стеклянную пластинку толщиной $5\,\text{мм}$.
    Пройдя через пластину, он выходит из неё в точке, смещённой по горизонтали от точки падения на расстояние $1{,}3\,\text{мм}$
    Показатель преломления стекла $1{,}5$.
    Найти синус угла падения, округлив его значение до двух знаков после запятой.
}
\solutionspace{180pt}

\tasknumber{11}%
\task{%
    Найти оптическую силу собирающей линзы, если действительное изображение предмета,
    помещённого в $55\,\text{см}$ от линзы, получается на расстоянии $40\,\text{см}$ от неё.
}
\solutionspace{180pt}

\tasknumber{12}%
\task{%
    Найти увеличение изображения, если изображение предмета, находящегося
    на расстоянии $25\,\text{см}$ от линзы, получается на расстоянии $18\,\text{см}$ от неё.
}
\solutionspace{180pt}

\tasknumber{13}%
\task{%
    Расстояние от предмета до линзы $12\,\text{см}$, а от линзы до мнимого изображения $25\,\text{см}$.
    Чему равно фокусное расстояние линзы?
}
\solutionspace{180pt}

\tasknumber{14}%
\task{%
    Две тонкие линзы с фокусными расстояниями $18\,\text{см}$ и $30\,\text{см}$ сложены вместе.
    Чему равно фокусное расстояние такой оптической системы?
}
\solutionspace{180pt}

\tasknumber{15}%
\task{%
    Линейные размеры прямого изображения предмета, полученного в собирающей линзе,
    в четыре раза больше линейных размеров предмета.
    Зная, что предмет находится на $30\,\text{см}$ ближе к линзе,
    чем его изображение, найти оптическую силу линзы.
}
\solutionspace{180pt}

\tasknumber{16}%
\task{%
    Оптическая сила объектива фотоаппарата равна $3\,\text{дптр}$.
    При фотографировании чертежа с расстояния $0{,}9\,\text{м}$ площадь изображения
    чертежа на фотопластинке оказалась равной $4\,\text{см}^{2}$.
    Какова площадь самого чертежа? Ответ выразите в квадратных сантиметрах.
}

\variantsplitter

\addpersonalvariant{Сергей Малышев}

\tasknumber{1}%
\task{%
    Разность фаз двух интерферирующих световых волн равна $7\pi$, а разность хода между ними равна $10{,}5 \cdot 10^{-7}\,\text{м}$.
    Определить длину волны.
}
\solutionspace{100pt}

\tasknumber{2}%
\task{%
    Расстояние между двумя точечными когерентными источниками света $S_1$ u $S_2$ равно $1{,}5\,\text{мм}$.
    Источники расположены в плоскости, параллельной экрану, на расстоянии $8\,\text{м}$ от него.
    На экране в точках, лежащих на перпендикулярах, опущенных из источников света $S_1$ и $S_2$,
    находятся два ближайших минимума (тёмные полосы).
    Определите длину световой волны.
    Ответ дать в нанометрах.
}
\solutionspace{100pt}

\tasknumber{3}%
\task{%
    На дифракционную решетку, имеющую период $3 \cdot 10^{-4}\,\text{см}$, нормально падает монохроматическая световая волна.
    Под углом $ 40 \degrees$ наблюдается дифракционный максимум второго порядка.
    Какова длина волны падающего света?
}
\solutionspace{100pt}

\tasknumber{4}%
\task{%
    Свет с длиной волны $0{,}5\,\text{мкм}$ падает нормально на дифракционную решетку с периодом, равным $1\,\text{мкм}$.
    Под каким углом наблюдается дифракционный максимум первого порядка?
}
\solutionspace{100pt}

\tasknumber{5}%
\task{%
    При нормальном падении белого света на дифракционную решетку зелёная линия ($500\,\text{нм}$)
    в спектре второго порядка видна под углом дифракции $20\degrees$.
    Определить число штрихов на $1\,\text{см}$ длины этой решетки.
}
\solutionspace{100pt}

\tasknumber{6}%
\task{%
    Каков наибольший порядок спектра, который можно наблюдать при дифракции света
    с длиной волны $\lambda$, на дифракционной решетке с периодом $d =  4{,}5 \lambda$?
}
\solutionspace{100pt}

\tasknumber{7}%
\task{%
    Вертикально стоящий шест высотой 1,1 м, освещенный солнцем,
    отбрасывает на горизонтальную поверхность земли тень длиной $4\,\text{м}$.
    Известно, что длина тени от телеграфного столба на $8\,\text{м}$ больше.
    Определить высоту столба.
}
\solutionspace{180pt}

\tasknumber{8}%
\task{%
    Определить абсолютный показатель преломления прозрачной среды,
    в которой распространяется свет с длиной волны $0{,}650\,\text{мкм}$ и частотой $462\,\text{ТГц}$.
    Скорость света в вакууме $3 \cdot 10^{8}\,\frac{\text{м}}{\text{с}}$.
}
\solutionspace{180pt}

\tasknumber{9}%
\task{%
    На дне водоема глубиной $2\,\text{м}$ лежит зеркало.
    Луч света, пройдя через воду, отражается от зеркала и выходит из воды.
    Найти расстояние между точкой входа луча в воду и точкой выхода луча из воды,
    если показатель преломления воды $1{,}33$, а угол падения луча $35\degrees$.
}
\solutionspace{180pt}

\tasknumber{10}%
\task{%
    Луч света падает на горизонтально расположенную стеклянную пластинку толщиной $6\,\text{мм}$.
    Пройдя через пластину, он выходит из неё в точке, смещённой по горизонтали от точки падения на расстояние $1{,}6\,\text{мм}$
    Показатель преломления стекла $1{,}4$.
    Найти синус угла падения, округлив его значение до двух знаков после запятой.
}
\solutionspace{180pt}

\tasknumber{11}%
\task{%
    Найти оптическую силу собирающей линзы, если действительное изображение предмета,
    помещённого в $55\,\text{см}$ от линзы, получается на расстоянии $40\,\text{см}$ от неё.
}
\solutionspace{180pt}

\tasknumber{12}%
\task{%
    Найти увеличение изображения, если изображение предмета, находящегося
    на расстоянии $25\,\text{см}$ от линзы, получается на расстоянии $12\,\text{см}$ от неё.
}
\solutionspace{180pt}

\tasknumber{13}%
\task{%
    Расстояние от предмета до линзы $12\,\text{см}$, а от линзы до мнимого изображения $20\,\text{см}$.
    Чему равно фокусное расстояние линзы?
}
\solutionspace{180pt}

\tasknumber{14}%
\task{%
    Две тонкие линзы с фокусными расстояниями $18\,\text{см}$ и $20\,\text{см}$ сложены вместе.
    Чему равно фокусное расстояние такой оптической системы?
}
\solutionspace{180pt}

\tasknumber{15}%
\task{%
    Линейные размеры прямого изображения предмета, полученного в собирающей линзе,
    в три раза больше линейных размеров предмета.
    Зная, что предмет находится на $30\,\text{см}$ ближе к линзе,
    чем его изображение, найти оптическую силу линзы.
}
\solutionspace{180pt}

\tasknumber{16}%
\task{%
    Оптическая сила объектива фотоаппарата равна $4\,\text{дптр}$.
    При фотографировании чертежа с расстояния $1{,}1\,\text{м}$ площадь изображения
    чертежа на фотопластинке оказалась равной $4\,\text{см}^{2}$.
    Какова площадь самого чертежа? Ответ выразите в квадратных сантиметрах.
}

\variantsplitter

\addpersonalvariant{Алина Полканова}

\tasknumber{1}%
\task{%
    Разность фаз двух интерферирующих световых волн равна $5\pi$, а разность хода между ними равна $9{,}5 \cdot 10^{-7}\,\text{м}$.
    Определить длину волны.
}
\solutionspace{100pt}

\tasknumber{2}%
\task{%
    Расстояние между двумя точечными когерентными источниками света $S_1$ u $S_2$ равно $2\,\text{мм}$.
    Источники расположены в плоскости, параллельной экрану, на расстоянии $9\,\text{м}$ от него.
    На экране в точках, лежащих на перпендикулярах, опущенных из источников света $S_1$ и $S_2$,
    находятся два ближайших минимума (тёмные полосы).
    Определите длину световой волны.
    Ответ дать в нанометрах.
}
\solutionspace{100pt}

\tasknumber{3}%
\task{%
    На дифракционную решетку, имеющую период $3 \cdot 10^{-4}\,\text{см}$, нормально падает монохроматическая световая волна.
    Под углом $ 25 \degrees$ наблюдается дифракционный максимум второго порядка.
    Какова длина волны падающего света?
}
\solutionspace{100pt}

\tasknumber{4}%
\task{%
    Свет с длиной волны $0{,}4\,\text{мкм}$ падает нормально на дифракционную решетку с периодом, равным $1\,\text{мкм}$.
    Под каким углом наблюдается дифракционный максимум первого порядка?
}
\solutionspace{100pt}

\tasknumber{5}%
\task{%
    При нормальном падении белого света на дифракционную решетку зелёная линия ($500\,\text{нм}$)
    в спектре второго порядка видна под углом дифракции $25\degrees$.
    Определить число штрихов на $1\,\text{мм}$ длины этой решетки.
}
\solutionspace{100pt}

\tasknumber{6}%
\task{%
    Каков наибольший порядок спектра, который можно наблюдать при дифракции света
    с длиной волны $\lambda$, на дифракционной решетке с периодом $d =  4{,}5 \lambda$?
}
\solutionspace{100pt}

\tasknumber{7}%
\task{%
    Вертикально стоящий шест высотой 1,1 м, освещенный солнцем,
    отбрасывает на горизонтальную поверхность земли тень длиной $4\,\text{м}$.
    Известно, что длина тени от телеграфного столба на $7\,\text{м}$ больше.
    Определить высоту столба.
}
\solutionspace{180pt}

\tasknumber{8}%
\task{%
    Определить абсолютный показатель преломления прозрачной среды,
    в которой распространяется свет с длиной волны $0{,}450\,\text{мкм}$ и частотой $667\,\text{ТГц}$.
    Скорость света в вакууме $3 \cdot 10^{8}\,\frac{\text{м}}{\text{с}}$.
}
\solutionspace{180pt}

\tasknumber{9}%
\task{%
    На дне водоема глубиной $2\,\text{м}$ лежит зеркало.
    Луч света, пройдя через воду, отражается от зеркала и выходит из воды.
    Найти расстояние между точкой входа луча в воду и точкой выхода луча из воды,
    если показатель преломления воды $1{,}33$, а угол падения луча $35\degrees$.
}
\solutionspace{180pt}

\tasknumber{10}%
\task{%
    Луч света падает на горизонтально расположенную стеклянную пластинку толщиной $4\,\text{мм}$.
    Пройдя через пластину, он выходит из неё в точке, смещённой по горизонтали от точки падения на расстояние $1{,}3\,\text{мм}$
    Показатель преломления стекла $1{,}4$.
    Найти синус угла падения, округлив его значение до двух знаков после запятой.
}
\solutionspace{180pt}

\tasknumber{11}%
\task{%
    Найти оптическую силу собирающей линзы, если действительное изображение предмета,
    помещённого в $15\,\text{см}$ от линзы, получается на расстоянии $30\,\text{см}$ от неё.
}
\solutionspace{180pt}

\tasknumber{12}%
\task{%
    Найти увеличение изображения, если изображение предмета, находящегося
    на расстоянии $20\,\text{см}$ от линзы, получается на расстоянии $12\,\text{см}$ от неё.
}
\solutionspace{180pt}

\tasknumber{13}%
\task{%
    Расстояние от предмета до линзы $10\,\text{см}$, а от линзы до мнимого изображения $20\,\text{см}$.
    Чему равно фокусное расстояние линзы?
}
\solutionspace{180pt}

\tasknumber{14}%
\task{%
    Две тонкие линзы с фокусными расстояниями $18\,\text{см}$ и $20\,\text{см}$ сложены вместе.
    Чему равно фокусное расстояние такой оптической системы?
}
\solutionspace{180pt}

\tasknumber{15}%
\task{%
    Линейные размеры прямого изображения предмета, полученного в собирающей линзе,
    в два раза больше линейных размеров предмета.
    Зная, что предмет находится на $20\,\text{см}$ ближе к линзе,
    чем его изображение, найти оптическую силу линзы.
}
\solutionspace{180pt}

\tasknumber{16}%
\task{%
    Оптическая сила объектива фотоаппарата равна $3\,\text{дптр}$.
    При фотографировании чертежа с расстояния $0{,}9\,\text{м}$ площадь изображения
    чертежа на фотопластинке оказалась равной $9\,\text{см}^{2}$.
    Какова площадь самого чертежа? Ответ выразите в квадратных сантиметрах.
}

\variantsplitter

\addpersonalvariant{Сергей Пономарёв}

\tasknumber{1}%
\task{%
    Разность фаз двух интерферирующих световых волн равна $3\pi$, а разность хода между ними равна $9{,}5 \cdot 10^{-7}\,\text{м}$.
    Определить длину волны.
}
\solutionspace{100pt}

\tasknumber{2}%
\task{%
    Расстояние между двумя точечными когерентными источниками света $S_1$ u $S_2$ равно $1{,}5\,\text{мм}$.
    Источники расположены в плоскости, параллельной экрану, на расстоянии $7\,\text{м}$ от него.
    На экране в точках, лежащих на перпендикулярах, опущенных из источников света $S_1$ и $S_2$,
    находятся два ближайших минимума (тёмные полосы).
    Определите длину световой волны.
    Ответ дать в нанометрах.
}
\solutionspace{100pt}

\tasknumber{3}%
\task{%
    На дифракционную решетку, имеющую период $2 \cdot 10^{-4}\,\text{см}$, нормально падает монохроматическая световая волна.
    Под углом $ 40 \degrees$ наблюдается дифракционный максимум второго порядка.
    Какова длина волны падающего света?
}
\solutionspace{100pt}

\tasknumber{4}%
\task{%
    Свет с длиной волны $0{,}7\,\text{мкм}$ падает нормально на дифракционную решетку с периодом, равным $3\,\text{мкм}$.
    Под каким углом наблюдается дифракционный максимум первого порядка?
}
\solutionspace{100pt}

\tasknumber{5}%
\task{%
    При нормальном падении белого света на дифракционную решетку зелёная линия ($500\,\text{нм}$)
    в спектре второго порядка видна под углом дифракции $25\degrees$.
    Определить число штрихов на $1\,\text{мм}$ длины этой решетки.
}
\solutionspace{100pt}

\tasknumber{6}%
\task{%
    Каков наибольший порядок спектра, который можно наблюдать при дифракции света
    с длиной волны $\lambda$, на дифракционной решетке с периодом $d =  3{,}5 \lambda$?
}
\solutionspace{100pt}

\tasknumber{7}%
\task{%
    Вертикально стоящий шест высотой 1,1 м, освещенный солнцем,
    отбрасывает на горизонтальную поверхность земли тень длиной $4\,\text{м}$.
    Известно, что длина тени от телеграфного столба на $6\,\text{м}$ больше.
    Определить высоту столба.
}
\solutionspace{180pt}

\tasknumber{8}%
\task{%
    Определить абсолютный показатель преломления прозрачной среды,
    в которой распространяется свет с длиной волны $0{,}600\,\text{мкм}$ и частотой $500\,\text{ТГц}$.
    Скорость света в вакууме $3 \cdot 10^{8}\,\frac{\text{м}}{\text{с}}$.
}
\solutionspace{180pt}

\tasknumber{9}%
\task{%
    На дне водоема глубиной $2\,\text{м}$ лежит зеркало.
    Луч света, пройдя через воду, отражается от зеркала и выходит из воды.
    Найти расстояние между точкой входа луча в воду и точкой выхода луча из воды,
    если показатель преломления воды $1{,}33$, а угол падения луча $30\degrees$.
}
\solutionspace{180pt}

\tasknumber{10}%
\task{%
    Луч света падает на горизонтально расположенную стеклянную пластинку толщиной $5\,\text{мм}$.
    Пройдя через пластину, он выходит из неё в точке, смещённой по горизонтали от точки падения на расстояние $1{,}3\,\text{мм}$
    Показатель преломления стекла $1{,}6$.
    Найти синус угла падения, округлив его значение до двух знаков после запятой.
}
\solutionspace{180pt}

\tasknumber{11}%
\task{%
    Найти оптическую силу собирающей линзы, если действительное изображение предмета,
    помещённого в $15\,\text{см}$ от линзы, получается на расстоянии $30\,\text{см}$ от неё.
}
\solutionspace{180pt}

\tasknumber{12}%
\task{%
    Найти увеличение изображения, если изображение предмета, находящегося
    на расстоянии $20\,\text{см}$ от линзы, получается на расстоянии $12\,\text{см}$ от неё.
}
\solutionspace{180pt}

\tasknumber{13}%
\task{%
    Расстояние от предмета до линзы $10\,\text{см}$, а от линзы до мнимого изображения $20\,\text{см}$.
    Чему равно фокусное расстояние линзы?
}
\solutionspace{180pt}

\tasknumber{14}%
\task{%
    Две тонкие линзы с фокусными расстояниями $12\,\text{см}$ и $30\,\text{см}$ сложены вместе.
    Чему равно фокусное расстояние такой оптической системы?
}
\solutionspace{180pt}

\tasknumber{15}%
\task{%
    Линейные размеры прямого изображения предмета, полученного в собирающей линзе,
    в четыре раза больше линейных размеров предмета.
    Зная, что предмет находится на $30\,\text{см}$ ближе к линзе,
    чем его изображение, найти оптическую силу линзы.
}
\solutionspace{180pt}

\tasknumber{16}%
\task{%
    Оптическая сила объектива фотоаппарата равна $3\,\text{дптр}$.
    При фотографировании чертежа с расстояния $0{,}9\,\text{м}$ площадь изображения
    чертежа на фотопластинке оказалась равной $4\,\text{см}^{2}$.
    Какова площадь самого чертежа? Ответ выразите в квадратных сантиметрах.
}

\variantsplitter

\addpersonalvariant{Егор Свистушкин}

\tasknumber{1}%
\task{%
    Разность фаз двух интерферирующих световых волн равна $3\pi$, а разность хода между ними равна $15{,}5 \cdot 10^{-7}\,\text{м}$.
    Определить длину волны.
}
\solutionspace{100pt}

\tasknumber{2}%
\task{%
    Расстояние между двумя точечными когерентными источниками света $S_1$ u $S_2$ равно $2\,\text{мм}$.
    Источники расположены в плоскости, параллельной экрану, на расстоянии $8\,\text{м}$ от него.
    На экране в точках, лежащих на перпендикулярах, опущенных из источников света $S_1$ и $S_2$,
    находятся два ближайших минимума (тёмные полосы).
    Определите длину световой волны.
    Ответ дать в нанометрах.
}
\solutionspace{100pt}

\tasknumber{3}%
\task{%
    На дифракционную решетку, имеющую период $3 \cdot 10^{-4}\,\text{см}$, нормально падает монохроматическая световая волна.
    Под углом $ 20 \degrees$ наблюдается дифракционный максимум второго порядка.
    Какова длина волны падающего света?
}
\solutionspace{100pt}

\tasknumber{4}%
\task{%
    Свет с длиной волны $0{,}5\,\text{мкм}$ падает нормально на дифракционную решетку с периодом, равным $2\,\text{мкм}$.
    Под каким углом наблюдается дифракционный максимум первого порядка?
}
\solutionspace{100pt}

\tasknumber{5}%
\task{%
    При нормальном падении белого света на дифракционную решетку зелёная линия ($500\,\text{нм}$)
    в спектре второго порядка видна под углом дифракции $53\degrees$.
    Определить число штрихов на $1\,\text{мм}$ длины этой решетки.
}
\solutionspace{100pt}

\tasknumber{6}%
\task{%
    Каков наибольший порядок спектра, который можно наблюдать при дифракции света
    с длиной волны $\lambda$, на дифракционной решетке с периодом $d =  2{,}5 \lambda$?
}
\solutionspace{100pt}

\tasknumber{7}%
\task{%
    Вертикально стоящий шест высотой 1,1 м, освещенный солнцем,
    отбрасывает на горизонтальную поверхность земли тень длиной $4\,\text{м}$.
    Известно, что длина тени от телеграфного столба на $9\,\text{м}$ больше.
    Определить высоту столба.
}
\solutionspace{180pt}

\tasknumber{8}%
\task{%
    Определить абсолютный показатель преломления прозрачной среды,
    в которой распространяется свет с длиной волны $0{,}650\,\text{мкм}$ и частотой $462\,\text{ТГц}$.
    Скорость света в вакууме $3 \cdot 10^{8}\,\frac{\text{м}}{\text{с}}$.
}
\solutionspace{180pt}

\tasknumber{9}%
\task{%
    На дне водоема глубиной $2\,\text{м}$ лежит зеркало.
    Луч света, пройдя через воду, отражается от зеркала и выходит из воды.
    Найти расстояние между точкой входа луча в воду и точкой выхода луча из воды,
    если показатель преломления воды $1{,}33$, а угол падения луча $25\degrees$.
}
\solutionspace{180pt}

\tasknumber{10}%
\task{%
    Луч света падает на горизонтально расположенную стеклянную пластинку толщиной $5\,\text{мм}$.
    Пройдя через пластину, он выходит из неё в точке, смещённой по горизонтали от точки падения на расстояние $1{,}6\,\text{мм}$
    Показатель преломления стекла $1{,}5$.
    Найти синус угла падения, округлив его значение до двух знаков после запятой.
}
\solutionspace{180pt}

\tasknumber{11}%
\task{%
    Найти оптическую силу собирающей линзы, если действительное изображение предмета,
    помещённого в $15\,\text{см}$ от линзы, получается на расстоянии $20\,\text{см}$ от неё.
}
\solutionspace{180pt}

\tasknumber{12}%
\task{%
    Найти увеличение изображения, если изображение предмета, находящегося
    на расстоянии $15\,\text{см}$ от линзы, получается на расстоянии $30\,\text{см}$ от неё.
}
\solutionspace{180pt}

\tasknumber{13}%
\task{%
    Расстояние от предмета до линзы $8\,\text{см}$, а от линзы до мнимого изображения $30\,\text{см}$.
    Чему равно фокусное расстояние линзы?
}
\solutionspace{180pt}

\tasknumber{14}%
\task{%
    Две тонкие линзы с фокусными расстояниями $12\,\text{см}$ и $20\,\text{см}$ сложены вместе.
    Чему равно фокусное расстояние такой оптической системы?
}
\solutionspace{180pt}

\tasknumber{15}%
\task{%
    Линейные размеры прямого изображения предмета, полученного в собирающей линзе,
    в три раза больше линейных размеров предмета.
    Зная, что предмет находится на $35\,\text{см}$ ближе к линзе,
    чем его изображение, найти оптическую силу линзы.
}
\solutionspace{180pt}

\tasknumber{16}%
\task{%
    Оптическая сила объектива фотоаппарата равна $3\,\text{дптр}$.
    При фотографировании чертежа с расстояния $0{,}9\,\text{м}$ площадь изображения
    чертежа на фотопластинке оказалась равной $4\,\text{см}^{2}$.
    Какова площадь самого чертежа? Ответ выразите в квадратных сантиметрах.
}

\variantsplitter

\addpersonalvariant{Дмитрий Соколов}

\tasknumber{1}%
\task{%
    Разность фаз двух интерферирующих световых волн равна $7\pi$, а разность хода между ними равна $12{,}5 \cdot 10^{-7}\,\text{м}$.
    Определить длину волны.
}
\solutionspace{100pt}

\tasknumber{2}%
\task{%
    Расстояние между двумя точечными когерентными источниками света $S_1$ u $S_2$ равно $1{,}5\,\text{мм}$.
    Источники расположены в плоскости, параллельной экрану, на расстоянии $7\,\text{м}$ от него.
    На экране в точках, лежащих на перпендикулярах, опущенных из источников света $S_1$ и $S_2$,
    находятся два ближайших минимума (тёмные полосы).
    Определите длину световой волны.
    Ответ дать в нанометрах.
}
\solutionspace{100pt}

\tasknumber{3}%
\task{%
    На дифракционную решетку, имеющую период $2 \cdot 10^{-4}\,\text{см}$, нормально падает монохроматическая световая волна.
    Под углом $ 40 \degrees$ наблюдается дифракционный максимум второго порядка.
    Какова длина волны падающего света?
}
\solutionspace{100pt}

\tasknumber{4}%
\task{%
    Свет с длиной волны $0{,}7\,\text{мкм}$ падает нормально на дифракционную решетку с периодом, равным $3\,\text{мкм}$.
    Под каким углом наблюдается дифракционный максимум первого порядка?
}
\solutionspace{100pt}

\tasknumber{5}%
\task{%
    При нормальном падении белого света на дифракционную решетку зелёная линия ($500\,\text{нм}$)
    в спектре второго порядка видна под углом дифракции $25\degrees$.
    Определить число штрихов на $1\,\text{мм}$ длины этой решетки.
}
\solutionspace{100pt}

\tasknumber{6}%
\task{%
    Каков наибольший порядок спектра, который можно наблюдать при дифракции света
    с длиной волны $\lambda$, на дифракционной решетке с периодом $d =  2{,}5 \lambda$?
}
\solutionspace{100pt}

\tasknumber{7}%
\task{%
    Вертикально стоящий шест высотой 1,1 м, освещенный солнцем,
    отбрасывает на горизонтальную поверхность земли тень длиной $3\,\text{м}$.
    Известно, что длина тени от телеграфного столба на $9\,\text{м}$ больше.
    Определить высоту столба.
}
\solutionspace{180pt}

\tasknumber{8}%
\task{%
    Определить абсолютный показатель преломления прозрачной среды,
    в которой распространяется свет с длиной волны $0{,}600\,\text{мкм}$ и частотой $500\,\text{ТГц}$.
    Скорость света в вакууме $3 \cdot 10^{8}\,\frac{\text{м}}{\text{с}}$.
}
\solutionspace{180pt}

\tasknumber{9}%
\task{%
    На дне водоема глубиной $3\,\text{м}$ лежит зеркало.
    Луч света, пройдя через воду, отражается от зеркала и выходит из воды.
    Найти расстояние между точкой входа луча в воду и точкой выхода луча из воды,
    если показатель преломления воды $1{,}33$, а угол падения луча $30\degrees$.
}
\solutionspace{180pt}

\tasknumber{10}%
\task{%
    Луч света падает на горизонтально расположенную стеклянную пластинку толщиной $6\,\text{мм}$.
    Пройдя через пластину, он выходит из неё в точке, смещённой по горизонтали от точки падения на расстояние $1{,}6\,\text{мм}$
    Показатель преломления стекла $1{,}5$.
    Найти синус угла падения, округлив его значение до двух знаков после запятой.
}
\solutionspace{180pt}

\tasknumber{11}%
\task{%
    Найти оптическую силу собирающей линзы, если действительное изображение предмета,
    помещённого в $15\,\text{см}$ от линзы, получается на расстоянии $30\,\text{см}$ от неё.
}
\solutionspace{180pt}

\tasknumber{12}%
\task{%
    Найти увеличение изображения, если изображение предмета, находящегося
    на расстоянии $20\,\text{см}$ от линзы, получается на расстоянии $18\,\text{см}$ от неё.
}
\solutionspace{180pt}

\tasknumber{13}%
\task{%
    Расстояние от предмета до линзы $10\,\text{см}$, а от линзы до мнимого изображения $25\,\text{см}$.
    Чему равно фокусное расстояние линзы?
}
\solutionspace{180pt}

\tasknumber{14}%
\task{%
    Две тонкие линзы с фокусными расстояниями $12\,\text{см}$ и $30\,\text{см}$ сложены вместе.
    Чему равно фокусное расстояние такой оптической системы?
}
\solutionspace{180pt}

\tasknumber{15}%
\task{%
    Линейные размеры прямого изображения предмета, полученного в собирающей линзе,
    в два раза больше линейных размеров предмета.
    Зная, что предмет находится на $40\,\text{см}$ ближе к линзе,
    чем его изображение, найти оптическую силу линзы.
}
\solutionspace{180pt}

\tasknumber{16}%
\task{%
    Оптическая сила объектива фотоаппарата равна $4\,\text{дптр}$.
    При фотографировании чертежа с расстояния $1{,}1\,\text{м}$ площадь изображения
    чертежа на фотопластинке оказалась равной $9\,\text{см}^{2}$.
    Какова площадь самого чертежа? Ответ выразите в квадратных сантиметрах.
}

\variantsplitter

\addpersonalvariant{Арсений Трофимов}

\tasknumber{1}%
\task{%
    Разность фаз двух интерферирующих световых волн равна $9\pi$, а разность хода между ними равна $9{,}5 \cdot 10^{-7}\,\text{м}$.
    Определить длину волны.
}
\solutionspace{100pt}

\tasknumber{2}%
\task{%
    Расстояние между двумя точечными когерентными источниками света $S_1$ u $S_2$ равно $2\,\text{мм}$.
    Источники расположены в плоскости, параллельной экрану, на расстоянии $9\,\text{м}$ от него.
    На экране в точках, лежащих на перпендикулярах, опущенных из источников света $S_1$ и $S_2$,
    находятся два ближайших минимума (тёмные полосы).
    Определите длину световой волны.
    Ответ дать в нанометрах.
}
\solutionspace{100pt}

\tasknumber{3}%
\task{%
    На дифракционную решетку, имеющую период $3 \cdot 10^{-4}\,\text{см}$, нормально падает монохроматическая световая волна.
    Под углом $ 40 \degrees$ наблюдается дифракционный максимум второго порядка.
    Какова длина волны падающего света?
}
\solutionspace{100pt}

\tasknumber{4}%
\task{%
    Свет с длиной волны $0{,}4\,\text{мкм}$ падает нормально на дифракционную решетку с периодом, равным $1\,\text{мкм}$.
    Под каким углом наблюдается дифракционный максимум первого порядка?
}
\solutionspace{100pt}

\tasknumber{5}%
\task{%
    При нормальном падении белого света на дифракционную решетку зелёная линия ($500\,\text{нм}$)
    в спектре второго порядка видна под углом дифракции $53\degrees$.
    Определить число штрихов на $1\,\text{мм}$ длины этой решетки.
}
\solutionspace{100pt}

\tasknumber{6}%
\task{%
    Каков наибольший порядок спектра, который можно наблюдать при дифракции света
    с длиной волны $\lambda$, на дифракционной решетке с периодом $d =  4{,}5 \lambda$?
}
\solutionspace{100pt}

\tasknumber{7}%
\task{%
    Вертикально стоящий шест высотой 1,1 м, освещенный солнцем,
    отбрасывает на горизонтальную поверхность земли тень длиной $3\,\text{м}$.
    Известно, что длина тени от телеграфного столба на $8\,\text{м}$ больше.
    Определить высоту столба.
}
\solutionspace{180pt}

\tasknumber{8}%
\task{%
    Определить абсолютный показатель преломления прозрачной среды,
    в которой распространяется свет с длиной волны $0{,}500\,\text{мкм}$ и частотой $600\,\text{ТГц}$.
    Скорость света в вакууме $3 \cdot 10^{8}\,\frac{\text{м}}{\text{с}}$.
}
\solutionspace{180pt}

\tasknumber{9}%
\task{%
    На дне водоема глубиной $3\,\text{м}$ лежит зеркало.
    Луч света, пройдя через воду, отражается от зеркала и выходит из воды.
    Найти расстояние между точкой входа луча в воду и точкой выхода луча из воды,
    если показатель преломления воды $1{,}33$, а угол падения луча $30\degrees$.
}
\solutionspace{180pt}

\tasknumber{10}%
\task{%
    Луч света падает на горизонтально расположенную стеклянную пластинку толщиной $4\,\text{мм}$.
    Пройдя через пластину, он выходит из неё в точке, смещённой по горизонтали от точки падения на расстояние $1{,}4\,\text{мм}$
    Показатель преломления стекла $1{,}6$.
    Найти синус угла падения, округлив его значение до двух знаков после запятой.
}
\solutionspace{180pt}

\tasknumber{11}%
\task{%
    Найти оптическую силу собирающей линзы, если действительное изображение предмета,
    помещённого в $15\,\text{см}$ от линзы, получается на расстоянии $40\,\text{см}$ от неё.
}
\solutionspace{180pt}

\tasknumber{12}%
\task{%
    Найти увеличение изображения, если изображение предмета, находящегося
    на расстоянии $25\,\text{см}$ от линзы, получается на расстоянии $18\,\text{см}$ от неё.
}
\solutionspace{180pt}

\tasknumber{13}%
\task{%
    Расстояние от предмета до линзы $12\,\text{см}$, а от линзы до мнимого изображения $25\,\text{см}$.
    Чему равно фокусное расстояние линзы?
}
\solutionspace{180pt}

\tasknumber{14}%
\task{%
    Две тонкие линзы с фокусными расстояниями $12\,\text{см}$ и $20\,\text{см}$ сложены вместе.
    Чему равно фокусное расстояние такой оптической системы?
}
\solutionspace{180pt}

\tasknumber{15}%
\task{%
    Линейные размеры прямого изображения предмета, полученного в собирающей линзе,
    в четыре раза больше линейных размеров предмета.
    Зная, что предмет находится на $40\,\text{см}$ ближе к линзе,
    чем его изображение, найти оптическую силу линзы.
}
\solutionspace{180pt}

\tasknumber{16}%
\task{%
    Оптическая сила объектива фотоаппарата равна $6\,\text{дптр}$.
    При фотографировании чертежа с расстояния $1{,}2\,\text{м}$ площадь изображения
    чертежа на фотопластинке оказалась равной $4\,\text{см}^{2}$.
    Какова площадь самого чертежа? Ответ выразите в квадратных сантиметрах.
}
% autogenerated
