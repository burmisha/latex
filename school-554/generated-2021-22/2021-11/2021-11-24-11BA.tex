\setdate{24~ноября~2021}
\setclass{11«БА»}

\addpersonalvariant{Михаил Бурмистров}

\tasknumber{1}%
\task{%
    В некоторую точку пространства приходят две когерентные световые волны
    с разностью хода $2{,}25\,\text{мкм}$.
    Определите, что наблюдается в этой точке.
    Длина волны равна $500\,\text{нм}$.
}
\answer{%
    $\text{точка минимума}$
}
\solutionspace{80pt}

\tasknumber{2}%
\task{%
    Установка для наблюдения интерференции состоит
    из двух когерентных источников света и экрана.
    Расстояние между источниками $l = 2{,}4\,\text{мм}$,
    а от каждого источника до экрана — $L = 4\,\text{м}$.
    Сделайте рисунок и укажите положение нулевого максимума освещенности,
    а также определите расстояние между вторым максимумом и нулевым максимумом.
    Длина волны падающего света составляет $\lambda = 550\,\text{нм}$.
}
\answer{%
    \begin{align*}
    l_1^2 &= L^2 + \sqr{x - \frac \ell 2} \\
    l_2^2 &= L^2 + \sqr{x + \frac \ell 2} \\
    l_2^2 - l_1^2 &= 2x\ell \implies (l_2 - l_1)(l_2 + l_1) = 2x\ell \implies n\lambda \cdot 2L \approx 2x_n\ell \implies x_n = \frac{\lambda L}{\ell} n, n\in \mathbb{N} \\
    x &= \frac{\lambda L}{\ell} \cdot 2 = \frac{550\,\text{нм} \cdot 4\,\text{м}}{2{,}4\,\text{мм}} \cdot 2 \approx 1{,}83\,\text{мм}
    \end{align*}
}
\solutionspace{120pt}

\tasknumber{3}%
\task{%
    Разность фаз двух интерферирующих световых волн равна $7\pi$,
    а разность хода между ними равна $15{,}5 \cdot 10^{-7}\,\text{м}$.
    Определить длину и частоту волны.
}
\answer{%
    \begin{align*}
    \Delta \varphi &= k\Delta l = \frac{2 \pi}{\lambda} \Delta l = 7\pi \implies \lambda = \frac27\Delta l \approx 443\,\text{нм}, \\
    &\nu = \frac 1T = \frac c\lambda = \frac72 \frac c{\Delta l} = \frac72 \cdot \frac{3 \cdot 10^{8}\,\frac{\text{м}}{\text{с}}}{15{,}5 \cdot 10^{-7}\,\text{м}} \approx 677\,\text{ТГц}.
    \end{align*}
}
\solutionspace{100pt}

\tasknumber{4}%
\task{%
    Два точечных когерентных источника света $S_1$ u $S_2$ расположены в плоскости, параллельной экрану, на расстоянии $4{,}5\,\text{м}$ от него.
    На экране в точках, лежащих на перпендикулярах, опущенных из источников света $S_1$ и $S_2$,
    находятся первые светлые полосы.
    Определите расстояние $S_1S_2$ между источниками, результат выразите в миллиметрах.
    Длина волны равна $640\,\text{нм}$.
}
\answer{%
    \begin{align*}
    &x_n = \frac{\lambda L}{\ell} n, n\in \mathbb{N} \leftarrow \text{точки максимума}, \\
    &2 \frac{\lambda L}{\ell} = \ell \implies \ell = \sqrt{ 2 \cdot \lambda L } = \sqrt{ 2 \cdot 640\,\text{нм} \cdot 4{,}5\,\text{м} } \approx 3{,}39\,\text{мм}.
    \end{align*}
}
\solutionspace{100pt}

\tasknumber{5}%
\task{%
    На стеклянную пластинку ($\hat n = 1{,}6$) нанесена прозрачная пленка ($n = 1{,}8$).
    На плёнку нормально к поверхности падает монохроматический свет с длиной волны $540\,\text{нм}$.
    Какова должна быть минимальная толщина пленки, чтобы в результате интерференции отражённый свет имел наибольшую интенсивность?
}
\answer{%
    $2 \cdot h \cdot n = \frac12 \lambda \implies h \approx 75\,\text{нм}$
}

\variantsplitter

\addpersonalvariant{Ирина Ан}

\tasknumber{1}%
\task{%
    В некоторую точку пространства приходят две когерентные световые волны
    с разностью хода $1{,}800\,\text{мкм}$.
    Определите, что наблюдается в этой точке.
    Длина волны равна $400\,\text{нм}$.
}
\answer{%
    $\text{точка минимума}$
}
\solutionspace{80pt}

\tasknumber{2}%
\task{%
    Установка для наблюдения интерференции состоит
    из двух когерентных источников света и экрана.
    Расстояние между источниками $l = 0{,}8\,\text{мм}$,
    а от каждого источника до экрана — $L = 2\,\text{м}$.
    Сделайте рисунок и укажите положение нулевого максимума освещенности,
    а также определите расстояние между четвёртым максимумом и нулевым максимумом.
    Длина волны падающего света составляет $\lambda = 500\,\text{нм}$.
}
\answer{%
    \begin{align*}
    l_1^2 &= L^2 + \sqr{x - \frac \ell 2} \\
    l_2^2 &= L^2 + \sqr{x + \frac \ell 2} \\
    l_2^2 - l_1^2 &= 2x\ell \implies (l_2 - l_1)(l_2 + l_1) = 2x\ell \implies n\lambda \cdot 2L \approx 2x_n\ell \implies x_n = \frac{\lambda L}{\ell} n, n\in \mathbb{N} \\
    x &= \frac{\lambda L}{\ell} \cdot 4 = \frac{500\,\text{нм} \cdot 2\,\text{м}}{0{,}8\,\text{мм}} \cdot 4 \approx 5\,\text{мм}
    \end{align*}
}
\solutionspace{120pt}

\tasknumber{3}%
\task{%
    Разность фаз двух интерферирующих световых волн равна $7\pi$,
    а разность хода между ними равна $12{,}5 \cdot 10^{-7}\,\text{м}$.
    Определить длину и частоту волны.
}
\answer{%
    \begin{align*}
    \Delta \varphi &= k\Delta l = \frac{2 \pi}{\lambda} \Delta l = 7\pi \implies \lambda = \frac27\Delta l \approx 357\,\text{нм}, \\
    &\nu = \frac 1T = \frac c\lambda = \frac72 \frac c{\Delta l} = \frac72 \cdot \frac{3 \cdot 10^{8}\,\frac{\text{м}}{\text{с}}}{12{,}5 \cdot 10^{-7}\,\text{м}} \approx 840\,\text{ТГц}.
    \end{align*}
}
\solutionspace{100pt}

\tasknumber{4}%
\task{%
    Два точечных когерентных источника света $S_1$ u $S_2$ расположены в плоскости, параллельной экрану, на расстоянии $3{,}2\,\text{м}$ от него.
    На экране в точках, лежащих на перпендикулярах, опущенных из источников света $S_1$ и $S_2$,
    находятся первые светлые полосы.
    Определите расстояние $S_1S_2$ между источниками, результат выразите в миллиметрах.
    Длина волны равна $640\,\text{нм}$.
}
\answer{%
    \begin{align*}
    &x_n = \frac{\lambda L}{\ell} n, n\in \mathbb{N} \leftarrow \text{точки максимума}, \\
    &2 \frac{\lambda L}{\ell} = \ell \implies \ell = \sqrt{ 2 \cdot \lambda L } = \sqrt{ 2 \cdot 640\,\text{нм} \cdot 3{,}2\,\text{м} } \approx 2{,}86\,\text{мм}.
    \end{align*}
}
\solutionspace{100pt}

\tasknumber{5}%
\task{%
    На стеклянную пластинку ($\hat n = 1{,}6$) нанесена прозрачная пленка ($n = 1{,}4$).
    На плёнку нормально к поверхности падает монохроматический свет с длиной волны $420\,\text{нм}$.
    Какова должна быть минимальная толщина пленки, чтобы в результате интерференции отражённый свет имел наименьшую интенсивность?
}
\answer{%
    $2 \cdot h \cdot n = \frac12 \lambda \implies h \approx 75\,\text{нм}$
}

\variantsplitter

\addpersonalvariant{Софья Андрианова}

\tasknumber{1}%
\task{%
    В некоторую точку пространства приходят две когерентные световые волны
    с разностью хода $2{,}40\,\text{мкм}$.
    Определите, что наблюдается в этой точке.
    Длина волны равна $400\,\text{нм}$.
}
\answer{%
    $\text{точка максимума}$
}
\solutionspace{80pt}

\tasknumber{2}%
\task{%
    Установка для наблюдения интерференции состоит
    из двух когерентных источников света и экрана.
    Расстояние между источниками $l = 1{,}2\,\text{мм}$,
    а от каждого источника до экрана — $L = 4\,\text{м}$.
    Сделайте рисунок и укажите положение нулевого максимума освещенности,
    а также определите расстояние между третьим максимумом и нулевым максимумом.
    Длина волны падающего света составляет $\lambda = 500\,\text{нм}$.
}
\answer{%
    \begin{align*}
    l_1^2 &= L^2 + \sqr{x - \frac \ell 2} \\
    l_2^2 &= L^2 + \sqr{x + \frac \ell 2} \\
    l_2^2 - l_1^2 &= 2x\ell \implies (l_2 - l_1)(l_2 + l_1) = 2x\ell \implies n\lambda \cdot 2L \approx 2x_n\ell \implies x_n = \frac{\lambda L}{\ell} n, n\in \mathbb{N} \\
    x &= \frac{\lambda L}{\ell} \cdot 3 = \frac{500\,\text{нм} \cdot 4\,\text{м}}{1{,}2\,\text{мм}} \cdot 3 \approx 5{,}0\,\text{мм}
    \end{align*}
}
\solutionspace{120pt}

\tasknumber{3}%
\task{%
    Разность фаз двух интерферирующих световых волн равна $7\pi$,
    а разность хода между ними равна $12{,}5 \cdot 10^{-7}\,\text{м}$.
    Определить длину и частоту волны.
}
\answer{%
    \begin{align*}
    \Delta \varphi &= k\Delta l = \frac{2 \pi}{\lambda} \Delta l = 7\pi \implies \lambda = \frac27\Delta l \approx 357\,\text{нм}, \\
    &\nu = \frac 1T = \frac c\lambda = \frac72 \frac c{\Delta l} = \frac72 \cdot \frac{3 \cdot 10^{8}\,\frac{\text{м}}{\text{с}}}{12{,}5 \cdot 10^{-7}\,\text{м}} \approx 840\,\text{ТГц}.
    \end{align*}
}
\solutionspace{100pt}

\tasknumber{4}%
\task{%
    Два точечных когерентных источника света $S_1$ u $S_2$ расположены в плоскости, параллельной экрану, на расстоянии $3{,}2\,\text{м}$ от него.
    На экране в точках, лежащих на перпендикулярах, опущенных из источников света $S_1$ и $S_2$,
    находятся первые светлые полосы.
    Определите расстояние $S_1S_2$ между источниками, результат выразите в миллиметрах.
    Длина волны равна $640\,\text{нм}$.
}
\answer{%
    \begin{align*}
    &x_n = \frac{\lambda L}{\ell} n, n\in \mathbb{N} \leftarrow \text{точки максимума}, \\
    &2 \frac{\lambda L}{\ell} = \ell \implies \ell = \sqrt{ 2 \cdot \lambda L } = \sqrt{ 2 \cdot 640\,\text{нм} \cdot 3{,}2\,\text{м} } \approx 2{,}86\,\text{мм}.
    \end{align*}
}
\solutionspace{100pt}

\tasknumber{5}%
\task{%
    На стеклянную пластинку ($\hat n = 1{,}5$) нанесена прозрачная пленка ($n = 1{,}8$).
    На плёнку нормально к поверхности падает монохроматический свет с длиной волны $540\,\text{нм}$.
    Какова должна быть минимальная толщина пленки, чтобы в результате интерференции отражённый свет имел наибольшую интенсивность?
}
\answer{%
    $2 \cdot h \cdot n = \frac12 \lambda \implies h \approx 75\,\text{нм}$
}

\variantsplitter

\addpersonalvariant{Владимир Артемчук}

\tasknumber{1}%
\task{%
    В некоторую точку пространства приходят две когерентные световые волны
    с разностью хода $2{,}45\,\text{мкм}$.
    Определите, что наблюдается в этой точке.
    Длина волны равна $700\,\text{нм}$.
}
\answer{%
    $\text{точка минимума}$
}
\solutionspace{80pt}

\tasknumber{2}%
\task{%
    Установка для наблюдения интерференции состоит
    из двух когерентных источников света и экрана.
    Расстояние между источниками $l = 1{,}2\,\text{мм}$,
    а от каждого источника до экрана — $L = 3\,\text{м}$.
    Сделайте рисунок и укажите положение нулевого максимума освещенности,
    а также определите расстояние между третьим максимумом и нулевым максимумом.
    Длина волны падающего света составляет $\lambda = 500\,\text{нм}$.
}
\answer{%
    \begin{align*}
    l_1^2 &= L^2 + \sqr{x - \frac \ell 2} \\
    l_2^2 &= L^2 + \sqr{x + \frac \ell 2} \\
    l_2^2 - l_1^2 &= 2x\ell \implies (l_2 - l_1)(l_2 + l_1) = 2x\ell \implies n\lambda \cdot 2L \approx 2x_n\ell \implies x_n = \frac{\lambda L}{\ell} n, n\in \mathbb{N} \\
    x &= \frac{\lambda L}{\ell} \cdot 3 = \frac{500\,\text{нм} \cdot 3\,\text{м}}{1{,}2\,\text{мм}} \cdot 3 \approx 3{,}8\,\text{мм}
    \end{align*}
}
\solutionspace{120pt}

\tasknumber{3}%
\task{%
    Разность фаз двух интерферирующих световых волн равна $6\pi$,
    а разность хода между ними равна $15{,}5 \cdot 10^{-7}\,\text{м}$.
    Определить длину и частоту волны.
}
\answer{%
    \begin{align*}
    \Delta \varphi &= k\Delta l = \frac{2 \pi}{\lambda} \Delta l = 6\pi \implies \lambda = \frac13\Delta l \approx 517\,\text{нм}, \\
    &\nu = \frac 1T = \frac c\lambda = 3 \frac c{\Delta l} = 3 \cdot \frac{3 \cdot 10^{8}\,\frac{\text{м}}{\text{с}}}{15{,}5 \cdot 10^{-7}\,\text{м}} \approx 581\,\text{ТГц}.
    \end{align*}
}
\solutionspace{100pt}

\tasknumber{4}%
\task{%
    Два точечных когерентных источника света $S_1$ u $S_2$ расположены в плоскости, параллельной экрану, на расстоянии $5{,}4\,\text{м}$ от него.
    На экране в точках, лежащих на перпендикулярах, опущенных из источников света $S_1$ и $S_2$,
    находятся первые тёмные полосы.
    Определите расстояние $S_1S_2$ между источниками, результат выразите в миллиметрах.
    Длина волны равна $420\,\text{нм}$.
}
\answer{%
    \begin{align*}
    &x_n = \frac{\lambda L}{\ell} n, n\in \mathbb{N} \leftarrow \text{точки максимума}, \\
    &1 \frac{\lambda L}{\ell} = \ell \implies \ell = \sqrt{ 1 \cdot \lambda L } = \sqrt{ 1 \cdot 420\,\text{нм} \cdot 5{,}4\,\text{м} } \approx 1{,}51\,\text{мм}.
    \end{align*}
}
\solutionspace{100pt}

\tasknumber{5}%
\task{%
    На стеклянную пластинку ($\hat n = 1{,}6$) нанесена прозрачная пленка ($n = 1{,}4$).
    На плёнку нормально к поверхности падает монохроматический свет с длиной волны $540\,\text{нм}$.
    Какова должна быть минимальная толщина пленки, чтобы в результате интерференции отражённый свет имел наибольшую интенсивность?
}
\answer{%
    $2 \cdot h \cdot n = 1 \lambda \implies h \approx 193\,\text{нм}$
}

\variantsplitter

\addpersonalvariant{Софья Белянкина}

\tasknumber{1}%
\task{%
    В некоторую точку пространства приходят две когерентные световые волны
    с разностью хода $3\,\text{мкм}$.
    Определите, что наблюдается в этой точке.
    Длина волны равна $500\,\text{нм}$.
}
\answer{%
    $\text{точка максимума}$
}
\solutionspace{80pt}

\tasknumber{2}%
\task{%
    Установка для наблюдения интерференции состоит
    из двух когерентных источников света и экрана.
    Расстояние между источниками $l = 0{,}8\,\text{мм}$,
    а от каждого источника до экрана — $L = 2\,\text{м}$.
    Сделайте рисунок и укажите положение нулевого максимума освещенности,
    а также определите расстояние между третьим максимумом и нулевым максимумом.
    Длина волны падающего света составляет $\lambda = 400\,\text{нм}$.
}
\answer{%
    \begin{align*}
    l_1^2 &= L^2 + \sqr{x - \frac \ell 2} \\
    l_2^2 &= L^2 + \sqr{x + \frac \ell 2} \\
    l_2^2 - l_1^2 &= 2x\ell \implies (l_2 - l_1)(l_2 + l_1) = 2x\ell \implies n\lambda \cdot 2L \approx 2x_n\ell \implies x_n = \frac{\lambda L}{\ell} n, n\in \mathbb{N} \\
    x &= \frac{\lambda L}{\ell} \cdot 3 = \frac{400\,\text{нм} \cdot 2\,\text{м}}{0{,}8\,\text{мм}} \cdot 3 \approx 3\,\text{мм}
    \end{align*}
}
\solutionspace{120pt}

\tasknumber{3}%
\task{%
    Разность фаз двух интерферирующих световых волн равна $6\pi$,
    а разность хода между ними равна $10{,}5 \cdot 10^{-7}\,\text{м}$.
    Определить длину и частоту волны.
}
\answer{%
    \begin{align*}
    \Delta \varphi &= k\Delta l = \frac{2 \pi}{\lambda} \Delta l = 6\pi \implies \lambda = \frac13\Delta l \approx 350\,\text{нм}, \\
    &\nu = \frac 1T = \frac c\lambda = 3 \frac c{\Delta l} = 3 \cdot \frac{3 \cdot 10^{8}\,\frac{\text{м}}{\text{с}}}{10{,}5 \cdot 10^{-7}\,\text{м}} \approx 857\,\text{ТГц}.
    \end{align*}
}
\solutionspace{100pt}

\tasknumber{4}%
\task{%
    Два точечных когерентных источника света $S_1$ u $S_2$ расположены в плоскости, параллельной экрану, на расстоянии $7{,}2\,\text{м}$ от него.
    На экране в точках, лежащих на перпендикулярах, опущенных из источников света $S_1$ и $S_2$,
    находятся первые светлые полосы.
    Определите расстояние $S_1S_2$ между источниками, результат выразите в миллиметрах.
    Длина волны равна $420\,\text{нм}$.
}
\answer{%
    \begin{align*}
    &x_n = \frac{\lambda L}{\ell} n, n\in \mathbb{N} \leftarrow \text{точки максимума}, \\
    &2 \frac{\lambda L}{\ell} = \ell \implies \ell = \sqrt{ 2 \cdot \lambda L } = \sqrt{ 2 \cdot 420\,\text{нм} \cdot 7{,}2\,\text{м} } \approx 3{,}48\,\text{мм}.
    \end{align*}
}
\solutionspace{100pt}

\tasknumber{5}%
\task{%
    На стеклянную пластинку ($\hat n = 1{,}6$) нанесена прозрачная пленка ($n = 1{,}7$).
    На плёнку нормально к поверхности падает монохроматический свет с длиной волны $480\,\text{нм}$.
    Какова должна быть минимальная толщина пленки, чтобы в результате интерференции отражённый свет имел наименьшую интенсивность?
}
\answer{%
    $2 \cdot h \cdot n = 1 \lambda \implies h \approx 141\,\text{нм}$
}

\variantsplitter

\addpersonalvariant{Варвара Егиазарян}

\tasknumber{1}%
\task{%
    В некоторую точку пространства приходят две когерентные световые волны
    с разностью хода $2{,}80\,\text{мкм}$.
    Определите, что наблюдается в этой точке.
    Длина волны равна $700\,\text{нм}$.
}
\answer{%
    $\text{точка максимума}$
}
\solutionspace{80pt}

\tasknumber{2}%
\task{%
    Установка для наблюдения интерференции состоит
    из двух когерентных источников света и экрана.
    Расстояние между источниками $l = 1{,}5\,\text{мм}$,
    а от каждого источника до экрана — $L = 3\,\text{м}$.
    Сделайте рисунок и укажите положение нулевого максимума освещенности,
    а также определите расстояние между вторым минимумом и нулевым максимумом.
    Длина волны падающего света составляет $\lambda = 550\,\text{нм}$.
}
\answer{%
    \begin{align*}
    l_1^2 &= L^2 + \sqr{x - \frac \ell 2} \\
    l_2^2 &= L^2 + \sqr{x + \frac \ell 2} \\
    l_2^2 - l_1^2 &= 2x\ell \implies (l_2 - l_1)(l_2 + l_1) = 2x\ell \implies n\lambda \cdot 2L \approx 2x_n\ell \implies x_n = \frac{\lambda L}{\ell} n, n\in \mathbb{N} \\
    x &= \frac{\lambda L}{\ell} \cdot \frac32 = \frac{550\,\text{нм} \cdot 3\,\text{м}}{1{,}5\,\text{мм}} \cdot \frac32 \approx 1{,}65\,\text{мм}
    \end{align*}
}
\solutionspace{120pt}

\tasknumber{3}%
\task{%
    Разность фаз двух интерферирующих световых волн равна $3\pi$,
    а разность хода между ними равна $15{,}5 \cdot 10^{-7}\,\text{м}$.
    Определить длину и частоту волны.
}
\answer{%
    \begin{align*}
    \Delta \varphi &= k\Delta l = \frac{2 \pi}{\lambda} \Delta l = 3\pi \implies \lambda = \frac23\Delta l \approx 1033\,\text{нм}, \\
    &\nu = \frac 1T = \frac c\lambda = \frac32 \frac c{\Delta l} = \frac32 \cdot \frac{3 \cdot 10^{8}\,\frac{\text{м}}{\text{с}}}{15{,}5 \cdot 10^{-7}\,\text{м}} \approx 290\,\text{ТГц}.
    \end{align*}
}
\solutionspace{100pt}

\tasknumber{4}%
\task{%
    Два точечных когерентных источника света $S_1$ u $S_2$ расположены в плоскости, параллельной экрану, на расстоянии $2{,}4\,\text{м}$ от него.
    На экране в точках, лежащих на перпендикулярах, опущенных из источников света $S_1$ и $S_2$,
    находятся первые светлые полосы.
    Определите расстояние $S_1S_2$ между источниками, результат выразите в миллиметрах.
    Длина волны равна $350\,\text{нм}$.
}
\answer{%
    \begin{align*}
    &x_n = \frac{\lambda L}{\ell} n, n\in \mathbb{N} \leftarrow \text{точки максимума}, \\
    &2 \frac{\lambda L}{\ell} = \ell \implies \ell = \sqrt{ 2 \cdot \lambda L } = \sqrt{ 2 \cdot 350\,\text{нм} \cdot 2{,}4\,\text{м} } \approx 1{,}83\,\text{мм}.
    \end{align*}
}
\solutionspace{100pt}

\tasknumber{5}%
\task{%
    На стеклянную пластинку ($\hat n = 1{,}5$) нанесена прозрачная пленка ($n = 1{,}8$).
    На плёнку нормально к поверхности падает монохроматический свет с длиной волны $480\,\text{нм}$.
    Какова должна быть минимальная толщина пленки, чтобы в результате интерференции отражённый свет имел наименьшую интенсивность?
}
\answer{%
    $2 \cdot h \cdot n = 1 \lambda \implies h \approx 133\,\text{нм}$
}

\variantsplitter

\addpersonalvariant{Владислав Емелин}

\tasknumber{1}%
\task{%
    В некоторую точку пространства приходят две когерентные световые волны
    с разностью хода $3{,}90\,\text{мкм}$.
    Определите, что наблюдается в этой точке.
    Длина волны равна $600\,\text{нм}$.
}
\answer{%
    $\text{точка минимума}$
}
\solutionspace{80pt}

\tasknumber{2}%
\task{%
    Установка для наблюдения интерференции состоит
    из двух когерентных источников света и экрана.
    Расстояние между источниками $l = 1{,}2\,\text{мм}$,
    а от каждого источника до экрана — $L = 2\,\text{м}$.
    Сделайте рисунок и укажите положение нулевого максимума освещенности,
    а также определите расстояние между четвёртым максимумом и нулевым максимумом.
    Длина волны падающего света составляет $\lambda = 450\,\text{нм}$.
}
\answer{%
    \begin{align*}
    l_1^2 &= L^2 + \sqr{x - \frac \ell 2} \\
    l_2^2 &= L^2 + \sqr{x + \frac \ell 2} \\
    l_2^2 - l_1^2 &= 2x\ell \implies (l_2 - l_1)(l_2 + l_1) = 2x\ell \implies n\lambda \cdot 2L \approx 2x_n\ell \implies x_n = \frac{\lambda L}{\ell} n, n\in \mathbb{N} \\
    x &= \frac{\lambda L}{\ell} \cdot 4 = \frac{450\,\text{нм} \cdot 2\,\text{м}}{1{,}2\,\text{мм}} \cdot 4 \approx 3\,\text{мм}
    \end{align*}
}
\solutionspace{120pt}

\tasknumber{3}%
\task{%
    Разность фаз двух интерферирующих световых волн равна $6\pi$,
    а разность хода между ними равна $10{,}5 \cdot 10^{-7}\,\text{м}$.
    Определить длину и частоту волны.
}
\answer{%
    \begin{align*}
    \Delta \varphi &= k\Delta l = \frac{2 \pi}{\lambda} \Delta l = 6\pi \implies \lambda = \frac13\Delta l \approx 350\,\text{нм}, \\
    &\nu = \frac 1T = \frac c\lambda = 3 \frac c{\Delta l} = 3 \cdot \frac{3 \cdot 10^{8}\,\frac{\text{м}}{\text{с}}}{10{,}5 \cdot 10^{-7}\,\text{м}} \approx 857\,\text{ТГц}.
    \end{align*}
}
\solutionspace{100pt}

\tasknumber{4}%
\task{%
    Два точечных когерентных источника света $S_1$ u $S_2$ расположены в плоскости, параллельной экрану, на расстоянии $4{,}5\,\text{м}$ от него.
    На экране в точках, лежащих на перпендикулярах, опущенных из источников света $S_1$ и $S_2$,
    находятся первые тёмные полосы.
    Определите расстояние $S_1S_2$ между источниками, результат выразите в миллиметрах.
    Длина волны равна $480\,\text{нм}$.
}
\answer{%
    \begin{align*}
    &x_n = \frac{\lambda L}{\ell} n, n\in \mathbb{N} \leftarrow \text{точки максимума}, \\
    &1 \frac{\lambda L}{\ell} = \ell \implies \ell = \sqrt{ 1 \cdot \lambda L } = \sqrt{ 1 \cdot 480\,\text{нм} \cdot 4{,}5\,\text{м} } \approx 1{,}47\,\text{мм}.
    \end{align*}
}
\solutionspace{100pt}

\tasknumber{5}%
\task{%
    На стеклянную пластинку ($\hat n = 1{,}5$) нанесена прозрачная пленка ($n = 1{,}7$).
    На плёнку нормально к поверхности падает монохроматический свет с длиной волны $420\,\text{нм}$.
    Какова должна быть минимальная толщина пленки, чтобы в результате интерференции отражённый свет имел наименьшую интенсивность?
}
\answer{%
    $2 \cdot h \cdot n = 1 \lambda \implies h \approx 124\,\text{нм}$
}

\variantsplitter

\addpersonalvariant{Артём Жичин}

\tasknumber{1}%
\task{%
    В некоторую точку пространства приходят две когерентные световые волны
    с разностью хода $3\,\text{мкм}$.
    Определите, что наблюдается в этой точке.
    Длина волны равна $600\,\text{нм}$.
}
\answer{%
    $\text{точка максимума}$
}
\solutionspace{80pt}

\tasknumber{2}%
\task{%
    Установка для наблюдения интерференции состоит
    из двух когерентных источников света и экрана.
    Расстояние между источниками $l = 1{,}2\,\text{мм}$,
    а от каждого источника до экрана — $L = 4\,\text{м}$.
    Сделайте рисунок и укажите положение нулевого максимума освещенности,
    а также определите расстояние между вторым максимумом и нулевым максимумом.
    Длина волны падающего света составляет $\lambda = 550\,\text{нм}$.
}
\answer{%
    \begin{align*}
    l_1^2 &= L^2 + \sqr{x - \frac \ell 2} \\
    l_2^2 &= L^2 + \sqr{x + \frac \ell 2} \\
    l_2^2 - l_1^2 &= 2x\ell \implies (l_2 - l_1)(l_2 + l_1) = 2x\ell \implies n\lambda \cdot 2L \approx 2x_n\ell \implies x_n = \frac{\lambda L}{\ell} n, n\in \mathbb{N} \\
    x &= \frac{\lambda L}{\ell} \cdot 2 = \frac{550\,\text{нм} \cdot 4\,\text{м}}{1{,}2\,\text{мм}} \cdot 2 \approx 3{,}7\,\text{мм}
    \end{align*}
}
\solutionspace{120pt}

\tasknumber{3}%
\task{%
    Разность фаз двух интерферирующих световых волн равна $8\pi$,
    а разность хода между ними равна $7{,}5 \cdot 10^{-7}\,\text{м}$.
    Определить длину и частоту волны.
}
\answer{%
    \begin{align*}
    \Delta \varphi &= k\Delta l = \frac{2 \pi}{\lambda} \Delta l = 8\pi \implies \lambda = \frac14\Delta l \approx 187{,}5\,\text{нм}, \\
    &\nu = \frac 1T = \frac c\lambda = 4 \frac c{\Delta l} = 4 \cdot \frac{3 \cdot 10^{8}\,\frac{\text{м}}{\text{с}}}{7{,}5 \cdot 10^{-7}\,\text{м}} \approx 1600\,\text{ТГц}.
    \end{align*}
}
\solutionspace{100pt}

\tasknumber{4}%
\task{%
    Два точечных когерентных источника света $S_1$ u $S_2$ расположены в плоскости, параллельной экрану, на расстоянии $5{,}4\,\text{м}$ от него.
    На экране в точках, лежащих на перпендикулярах, опущенных из источников света $S_1$ и $S_2$,
    находятся первые тёмные полосы.
    Определите расстояние $S_1S_2$ между источниками, результат выразите в миллиметрах.
    Длина волны равна $480\,\text{нм}$.
}
\answer{%
    \begin{align*}
    &x_n = \frac{\lambda L}{\ell} n, n\in \mathbb{N} \leftarrow \text{точки максимума}, \\
    &1 \frac{\lambda L}{\ell} = \ell \implies \ell = \sqrt{ 1 \cdot \lambda L } = \sqrt{ 1 \cdot 480\,\text{нм} \cdot 5{,}4\,\text{м} } \approx 1{,}61\,\text{мм}.
    \end{align*}
}
\solutionspace{100pt}

\tasknumber{5}%
\task{%
    На стеклянную пластинку ($\hat n = 1{,}6$) нанесена прозрачная пленка ($n = 1{,}8$).
    На плёнку нормально к поверхности падает монохроматический свет с длиной волны $640\,\text{нм}$.
    Какова должна быть минимальная толщина пленки, чтобы в результате интерференции отражённый свет имел наибольшую интенсивность?
}
\answer{%
    $2 \cdot h \cdot n = \frac12 \lambda \implies h \approx 89\,\text{нм}$
}

\variantsplitter

\addpersonalvariant{Дарья Кошман}

\tasknumber{1}%
\task{%
    В некоторую точку пространства приходят две когерентные световые волны
    с разностью хода $2\,\text{мкм}$.
    Определите, что наблюдается в этой точке.
    Длина волны равна $400\,\text{нм}$.
}
\answer{%
    $\text{точка максимума}$
}
\solutionspace{80pt}

\tasknumber{2}%
\task{%
    Установка для наблюдения интерференции состоит
    из двух когерентных источников света и экрана.
    Расстояние между источниками $l = 1{,}5\,\text{мм}$,
    а от каждого источника до экрана — $L = 3\,\text{м}$.
    Сделайте рисунок и укажите положение нулевого максимума освещенности,
    а также определите расстояние между вторым минимумом и нулевым максимумом.
    Длина волны падающего света составляет $\lambda = 550\,\text{нм}$.
}
\answer{%
    \begin{align*}
    l_1^2 &= L^2 + \sqr{x - \frac \ell 2} \\
    l_2^2 &= L^2 + \sqr{x + \frac \ell 2} \\
    l_2^2 - l_1^2 &= 2x\ell \implies (l_2 - l_1)(l_2 + l_1) = 2x\ell \implies n\lambda \cdot 2L \approx 2x_n\ell \implies x_n = \frac{\lambda L}{\ell} n, n\in \mathbb{N} \\
    x &= \frac{\lambda L}{\ell} \cdot \frac32 = \frac{550\,\text{нм} \cdot 3\,\text{м}}{1{,}5\,\text{мм}} \cdot \frac32 \approx 1{,}65\,\text{мм}
    \end{align*}
}
\solutionspace{120pt}

\tasknumber{3}%
\task{%
    Разность фаз двух интерферирующих световых волн равна $5\pi$,
    а разность хода между ними равна $15{,}5 \cdot 10^{-7}\,\text{м}$.
    Определить длину и частоту волны.
}
\answer{%
    \begin{align*}
    \Delta \varphi &= k\Delta l = \frac{2 \pi}{\lambda} \Delta l = 5\pi \implies \lambda = \frac25\Delta l \approx 620\,\text{нм}, \\
    &\nu = \frac 1T = \frac c\lambda = \frac52 \frac c{\Delta l} = \frac52 \cdot \frac{3 \cdot 10^{8}\,\frac{\text{м}}{\text{с}}}{15{,}5 \cdot 10^{-7}\,\text{м}} \approx 484\,\text{ТГц}.
    \end{align*}
}
\solutionspace{100pt}

\tasknumber{4}%
\task{%
    Два точечных когерентных источника света $S_1$ u $S_2$ расположены в плоскости, параллельной экрану, на расстоянии $3{,}2\,\text{м}$ от него.
    На экране в точках, лежащих на перпендикулярах, опущенных из источников света $S_1$ и $S_2$,
    находятся первые тёмные полосы.
    Определите расстояние $S_1S_2$ между источниками, результат выразите в миллиметрах.
    Длина волны равна $350\,\text{нм}$.
}
\answer{%
    \begin{align*}
    &x_n = \frac{\lambda L}{\ell} n, n\in \mathbb{N} \leftarrow \text{точки максимума}, \\
    &1 \frac{\lambda L}{\ell} = \ell \implies \ell = \sqrt{ 1 \cdot \lambda L } = \sqrt{ 1 \cdot 350\,\text{нм} \cdot 3{,}2\,\text{м} } \approx 1{,}06\,\text{мм}.
    \end{align*}
}
\solutionspace{100pt}

\tasknumber{5}%
\task{%
    На стеклянную пластинку ($\hat n = 1{,}6$) нанесена прозрачная пленка ($n = 1{,}3$).
    На плёнку нормально к поверхности падает монохроматический свет с длиной волны $540\,\text{нм}$.
    Какова должна быть минимальная толщина пленки, чтобы в результате интерференции отражённый свет имел наибольшую интенсивность?
}
\answer{%
    $2 \cdot h \cdot n = 1 \lambda \implies h \approx 210\,\text{нм}$
}

\variantsplitter

\addpersonalvariant{Анна Кузьмичёва}

\tasknumber{1}%
\task{%
    В некоторую точку пространства приходят две когерентные световые волны
    с разностью хода $3{,}25\,\text{мкм}$.
    Определите, что наблюдается в этой точке.
    Длина волны равна $500\,\text{нм}$.
}
\answer{%
    $\text{точка минимума}$
}
\solutionspace{80pt}

\tasknumber{2}%
\task{%
    Установка для наблюдения интерференции состоит
    из двух когерентных источников света и экрана.
    Расстояние между источниками $l = 1{,}5\,\text{мм}$,
    а от каждого источника до экрана — $L = 3\,\text{м}$.
    Сделайте рисунок и укажите положение нулевого максимума освещенности,
    а также определите расстояние между третьим минимумом и нулевым максимумом.
    Длина волны падающего света составляет $\lambda = 550\,\text{нм}$.
}
\answer{%
    \begin{align*}
    l_1^2 &= L^2 + \sqr{x - \frac \ell 2} \\
    l_2^2 &= L^2 + \sqr{x + \frac \ell 2} \\
    l_2^2 - l_1^2 &= 2x\ell \implies (l_2 - l_1)(l_2 + l_1) = 2x\ell \implies n\lambda \cdot 2L \approx 2x_n\ell \implies x_n = \frac{\lambda L}{\ell} n, n\in \mathbb{N} \\
    x &= \frac{\lambda L}{\ell} \cdot \frac52 = \frac{550\,\text{нм} \cdot 3\,\text{м}}{1{,}5\,\text{мм}} \cdot \frac52 \approx 2{,}8\,\text{мм}
    \end{align*}
}
\solutionspace{120pt}

\tasknumber{3}%
\task{%
    Разность фаз двух интерферирующих световых волн равна $4\pi$,
    а разность хода между ними равна $10{,}5 \cdot 10^{-7}\,\text{м}$.
    Определить длину и частоту волны.
}
\answer{%
    \begin{align*}
    \Delta \varphi &= k\Delta l = \frac{2 \pi}{\lambda} \Delta l = 4\pi \implies \lambda = \frac12\Delta l \approx 525\,\text{нм}, \\
    &\nu = \frac 1T = \frac c\lambda = 2 \frac c{\Delta l} = 2 \cdot \frac{3 \cdot 10^{8}\,\frac{\text{м}}{\text{с}}}{10{,}5 \cdot 10^{-7}\,\text{м}} \approx 571\,\text{ТГц}.
    \end{align*}
}
\solutionspace{100pt}

\tasknumber{4}%
\task{%
    Два точечных когерентных источника света $S_1$ u $S_2$ расположены в плоскости, параллельной экрану, на расстоянии $7{,}2\,\text{м}$ от него.
    На экране в точках, лежащих на перпендикулярах, опущенных из источников света $S_1$ и $S_2$,
    находятся первые тёмные полосы.
    Определите расстояние $S_1S_2$ между источниками, результат выразите в миллиметрах.
    Длина волны равна $420\,\text{нм}$.
}
\answer{%
    \begin{align*}
    &x_n = \frac{\lambda L}{\ell} n, n\in \mathbb{N} \leftarrow \text{точки максимума}, \\
    &1 \frac{\lambda L}{\ell} = \ell \implies \ell = \sqrt{ 1 \cdot \lambda L } = \sqrt{ 1 \cdot 420\,\text{нм} \cdot 7{,}2\,\text{м} } \approx 1{,}74\,\text{мм}.
    \end{align*}
}
\solutionspace{100pt}

\tasknumber{5}%
\task{%
    На стеклянную пластинку ($\hat n = 1{,}6$) нанесена прозрачная пленка ($n = 1{,}7$).
    На плёнку нормально к поверхности падает монохроматический свет с длиной волны $540\,\text{нм}$.
    Какова должна быть минимальная толщина пленки, чтобы в результате интерференции отражённый свет имел наибольшую интенсивность?
}
\answer{%
    $2 \cdot h \cdot n = \frac12 \lambda \implies h \approx 79\,\text{нм}$
}

\variantsplitter

\addpersonalvariant{Алёна Куприянова}

\tasknumber{1}%
\task{%
    В некоторую точку пространства приходят две когерентные световые волны
    с разностью хода $3\,\text{мкм}$.
    Определите, что наблюдается в этой точке.
    Длина волны равна $500\,\text{нм}$.
}
\answer{%
    $\text{точка максимума}$
}
\solutionspace{80pt}

\tasknumber{2}%
\task{%
    Установка для наблюдения интерференции состоит
    из двух когерентных источников света и экрана.
    Расстояние между источниками $l = 1{,}2\,\text{мм}$,
    а от каждого источника до экрана — $L = 2\,\text{м}$.
    Сделайте рисунок и укажите положение нулевого максимума освещенности,
    а также определите расстояние между вторым максимумом и нулевым максимумом.
    Длина волны падающего света составляет $\lambda = 500\,\text{нм}$.
}
\answer{%
    \begin{align*}
    l_1^2 &= L^2 + \sqr{x - \frac \ell 2} \\
    l_2^2 &= L^2 + \sqr{x + \frac \ell 2} \\
    l_2^2 - l_1^2 &= 2x\ell \implies (l_2 - l_1)(l_2 + l_1) = 2x\ell \implies n\lambda \cdot 2L \approx 2x_n\ell \implies x_n = \frac{\lambda L}{\ell} n, n\in \mathbb{N} \\
    x &= \frac{\lambda L}{\ell} \cdot 2 = \frac{500\,\text{нм} \cdot 2\,\text{м}}{1{,}2\,\text{мм}} \cdot 2 \approx 1{,}67\,\text{мм}
    \end{align*}
}
\solutionspace{120pt}

\tasknumber{3}%
\task{%
    Разность фаз двух интерферирующих световых волн равна $4\pi$,
    а разность хода между ними равна $15{,}5 \cdot 10^{-7}\,\text{м}$.
    Определить длину и частоту волны.
}
\answer{%
    \begin{align*}
    \Delta \varphi &= k\Delta l = \frac{2 \pi}{\lambda} \Delta l = 4\pi \implies \lambda = \frac12\Delta l \approx 775\,\text{нм}, \\
    &\nu = \frac 1T = \frac c\lambda = 2 \frac c{\Delta l} = 2 \cdot \frac{3 \cdot 10^{8}\,\frac{\text{м}}{\text{с}}}{15{,}5 \cdot 10^{-7}\,\text{м}} \approx 387\,\text{ТГц}.
    \end{align*}
}
\solutionspace{100pt}

\tasknumber{4}%
\task{%
    Два точечных когерентных источника света $S_1$ u $S_2$ расположены в плоскости, параллельной экрану, на расстоянии $5{,}4\,\text{м}$ от него.
    На экране в точках, лежащих на перпендикулярах, опущенных из источников света $S_1$ и $S_2$,
    находятся первые светлые полосы.
    Определите расстояние $S_1S_2$ между источниками, результат выразите в миллиметрах.
    Длина волны равна $350\,\text{нм}$.
}
\answer{%
    \begin{align*}
    &x_n = \frac{\lambda L}{\ell} n, n\in \mathbb{N} \leftarrow \text{точки максимума}, \\
    &2 \frac{\lambda L}{\ell} = \ell \implies \ell = \sqrt{ 2 \cdot \lambda L } = \sqrt{ 2 \cdot 350\,\text{нм} \cdot 5{,}4\,\text{м} } \approx 2{,}75\,\text{мм}.
    \end{align*}
}
\solutionspace{100pt}

\tasknumber{5}%
\task{%
    На стеклянную пластинку ($\hat n = 1{,}5$) нанесена прозрачная пленка ($n = 1{,}3$).
    На плёнку нормально к поверхности падает монохроматический свет с длиной волны $420\,\text{нм}$.
    Какова должна быть минимальная толщина пленки, чтобы в результате интерференции отражённый свет имел наибольшую интенсивность?
}
\answer{%
    $2 \cdot h \cdot n = 1 \lambda \implies h \approx 162\,\text{нм}$
}

\variantsplitter

\addpersonalvariant{Ярослав Лавровский}

\tasknumber{1}%
\task{%
    В некоторую точку пространства приходят две когерентные световые волны
    с разностью хода $2{,}40\,\text{мкм}$.
    Определите, что наблюдается в этой точке.
    Длина волны равна $600\,\text{нм}$.
}
\answer{%
    $\text{точка максимума}$
}
\solutionspace{80pt}

\tasknumber{2}%
\task{%
    Установка для наблюдения интерференции состоит
    из двух когерентных источников света и экрана.
    Расстояние между источниками $l = 0{,}8\,\text{мм}$,
    а от каждого источника до экрана — $L = 2\,\text{м}$.
    Сделайте рисунок и укажите положение нулевого максимума освещенности,
    а также определите расстояние между четвёртым минимумом и нулевым максимумом.
    Длина волны падающего света составляет $\lambda = 400\,\text{нм}$.
}
\answer{%
    \begin{align*}
    l_1^2 &= L^2 + \sqr{x - \frac \ell 2} \\
    l_2^2 &= L^2 + \sqr{x + \frac \ell 2} \\
    l_2^2 - l_1^2 &= 2x\ell \implies (l_2 - l_1)(l_2 + l_1) = 2x\ell \implies n\lambda \cdot 2L \approx 2x_n\ell \implies x_n = \frac{\lambda L}{\ell} n, n\in \mathbb{N} \\
    x &= \frac{\lambda L}{\ell} \cdot \frac72 = \frac{400\,\text{нм} \cdot 2\,\text{м}}{0{,}8\,\text{мм}} \cdot \frac72 \approx 3{,}5\,\text{мм}
    \end{align*}
}
\solutionspace{120pt}

\tasknumber{3}%
\task{%
    Разность фаз двух интерферирующих световых волн равна $8\pi$,
    а разность хода между ними равна $10{,}5 \cdot 10^{-7}\,\text{м}$.
    Определить длину и частоту волны.
}
\answer{%
    \begin{align*}
    \Delta \varphi &= k\Delta l = \frac{2 \pi}{\lambda} \Delta l = 8\pi \implies \lambda = \frac14\Delta l \approx 263\,\text{нм}, \\
    &\nu = \frac 1T = \frac c\lambda = 4 \frac c{\Delta l} = 4 \cdot \frac{3 \cdot 10^{8}\,\frac{\text{м}}{\text{с}}}{10{,}5 \cdot 10^{-7}\,\text{м}} \approx 1143\,\text{ТГц}.
    \end{align*}
}
\solutionspace{100pt}

\tasknumber{4}%
\task{%
    Два точечных когерентных источника света $S_1$ u $S_2$ расположены в плоскости, параллельной экрану, на расстоянии $7{,}2\,\text{м}$ от него.
    На экране в точках, лежащих на перпендикулярах, опущенных из источников света $S_1$ и $S_2$,
    находятся первые светлые полосы.
    Определите расстояние $S_1S_2$ между источниками, результат выразите в миллиметрах.
    Длина волны равна $480\,\text{нм}$.
}
\answer{%
    \begin{align*}
    &x_n = \frac{\lambda L}{\ell} n, n\in \mathbb{N} \leftarrow \text{точки максимума}, \\
    &2 \frac{\lambda L}{\ell} = \ell \implies \ell = \sqrt{ 2 \cdot \lambda L } = \sqrt{ 2 \cdot 480\,\text{нм} \cdot 7{,}2\,\text{м} } \approx 3{,}72\,\text{мм}.
    \end{align*}
}
\solutionspace{100pt}

\tasknumber{5}%
\task{%
    На стеклянную пластинку ($\hat n = 1{,}6$) нанесена прозрачная пленка ($n = 1{,}8$).
    На плёнку нормально к поверхности падает монохроматический свет с длиной волны $640\,\text{нм}$.
    Какова должна быть минимальная толщина пленки, чтобы в результате интерференции отражённый свет имел наименьшую интенсивность?
}
\answer{%
    $2 \cdot h \cdot n = 1 \lambda \implies h \approx 178\,\text{нм}$
}

\variantsplitter

\addpersonalvariant{Анастасия Ламанова}

\tasknumber{1}%
\task{%
    В некоторую точку пространства приходят две когерентные световые волны
    с разностью хода $1{,}800\,\text{мкм}$.
    Определите, что наблюдается в этой точке.
    Длина волны равна $400\,\text{нм}$.
}
\answer{%
    $\text{точка минимума}$
}
\solutionspace{80pt}

\tasknumber{2}%
\task{%
    Установка для наблюдения интерференции состоит
    из двух когерентных источников света и экрана.
    Расстояние между источниками $l = 1{,}5\,\text{мм}$,
    а от каждого источника до экрана — $L = 2\,\text{м}$.
    Сделайте рисунок и укажите положение нулевого максимума освещенности,
    а также определите расстояние между четвёртым минимумом и нулевым максимумом.
    Длина волны падающего света составляет $\lambda = 500\,\text{нм}$.
}
\answer{%
    \begin{align*}
    l_1^2 &= L^2 + \sqr{x - \frac \ell 2} \\
    l_2^2 &= L^2 + \sqr{x + \frac \ell 2} \\
    l_2^2 - l_1^2 &= 2x\ell \implies (l_2 - l_1)(l_2 + l_1) = 2x\ell \implies n\lambda \cdot 2L \approx 2x_n\ell \implies x_n = \frac{\lambda L}{\ell} n, n\in \mathbb{N} \\
    x &= \frac{\lambda L}{\ell} \cdot \frac72 = \frac{500\,\text{нм} \cdot 2\,\text{м}}{1{,}5\,\text{мм}} \cdot \frac72 \approx 2{,}3\,\text{мм}
    \end{align*}
}
\solutionspace{120pt}

\tasknumber{3}%
\task{%
    Разность фаз двух интерферирующих световых волн равна $6\pi$,
    а разность хода между ними равна $15{,}5 \cdot 10^{-7}\,\text{м}$.
    Определить длину и частоту волны.
}
\answer{%
    \begin{align*}
    \Delta \varphi &= k\Delta l = \frac{2 \pi}{\lambda} \Delta l = 6\pi \implies \lambda = \frac13\Delta l \approx 517\,\text{нм}, \\
    &\nu = \frac 1T = \frac c\lambda = 3 \frac c{\Delta l} = 3 \cdot \frac{3 \cdot 10^{8}\,\frac{\text{м}}{\text{с}}}{15{,}5 \cdot 10^{-7}\,\text{м}} \approx 581\,\text{ТГц}.
    \end{align*}
}
\solutionspace{100pt}

\tasknumber{4}%
\task{%
    Два точечных когерентных источника света $S_1$ u $S_2$ расположены в плоскости, параллельной экрану, на расстоянии $3{,}2\,\text{м}$ от него.
    На экране в точках, лежащих на перпендикулярах, опущенных из источников света $S_1$ и $S_2$,
    находятся первые тёмные полосы.
    Определите расстояние $S_1S_2$ между источниками, результат выразите в миллиметрах.
    Длина волны равна $550\,\text{нм}$.
}
\answer{%
    \begin{align*}
    &x_n = \frac{\lambda L}{\ell} n, n\in \mathbb{N} \leftarrow \text{точки максимума}, \\
    &1 \frac{\lambda L}{\ell} = \ell \implies \ell = \sqrt{ 1 \cdot \lambda L } = \sqrt{ 1 \cdot 550\,\text{нм} \cdot 3{,}2\,\text{м} } \approx 1{,}33\,\text{мм}.
    \end{align*}
}
\solutionspace{100pt}

\tasknumber{5}%
\task{%
    На стеклянную пластинку ($\hat n = 1{,}5$) нанесена прозрачная пленка ($n = 1{,}4$).
    На плёнку нормально к поверхности падает монохроматический свет с длиной волны $480\,\text{нм}$.
    Какова должна быть минимальная толщина пленки, чтобы в результате интерференции отражённый свет имел наибольшую интенсивность?
}
\answer{%
    $2 \cdot h \cdot n = 1 \lambda \implies h \approx 171\,\text{нм}$
}

\variantsplitter

\addpersonalvariant{Виктория Легонькова}

\tasknumber{1}%
\task{%
    В некоторую точку пространства приходят две когерентные световые волны
    с разностью хода $1{,}500\,\text{мкм}$.
    Определите, что наблюдается в этой точке.
    Длина волны равна $500\,\text{нм}$.
}
\answer{%
    $\text{точка максимума}$
}
\solutionspace{80pt}

\tasknumber{2}%
\task{%
    Установка для наблюдения интерференции состоит
    из двух когерентных источников света и экрана.
    Расстояние между источниками $l = 1{,}5\,\text{мм}$,
    а от каждого источника до экрана — $L = 2\,\text{м}$.
    Сделайте рисунок и укажите положение нулевого максимума освещенности,
    а также определите расстояние между третьим минимумом и нулевым максимумом.
    Длина волны падающего света составляет $\lambda = 550\,\text{нм}$.
}
\answer{%
    \begin{align*}
    l_1^2 &= L^2 + \sqr{x - \frac \ell 2} \\
    l_2^2 &= L^2 + \sqr{x + \frac \ell 2} \\
    l_2^2 - l_1^2 &= 2x\ell \implies (l_2 - l_1)(l_2 + l_1) = 2x\ell \implies n\lambda \cdot 2L \approx 2x_n\ell \implies x_n = \frac{\lambda L}{\ell} n, n\in \mathbb{N} \\
    x &= \frac{\lambda L}{\ell} \cdot \frac52 = \frac{550\,\text{нм} \cdot 2\,\text{м}}{1{,}5\,\text{мм}} \cdot \frac52 \approx 1{,}83\,\text{мм}
    \end{align*}
}
\solutionspace{120pt}

\tasknumber{3}%
\task{%
    Разность фаз двух интерферирующих световых волн равна $6\pi$,
    а разность хода между ними равна $15{,}5 \cdot 10^{-7}\,\text{м}$.
    Определить длину и частоту волны.
}
\answer{%
    \begin{align*}
    \Delta \varphi &= k\Delta l = \frac{2 \pi}{\lambda} \Delta l = 6\pi \implies \lambda = \frac13\Delta l \approx 517\,\text{нм}, \\
    &\nu = \frac 1T = \frac c\lambda = 3 \frac c{\Delta l} = 3 \cdot \frac{3 \cdot 10^{8}\,\frac{\text{м}}{\text{с}}}{15{,}5 \cdot 10^{-7}\,\text{м}} \approx 581\,\text{ТГц}.
    \end{align*}
}
\solutionspace{100pt}

\tasknumber{4}%
\task{%
    Два точечных когерентных источника света $S_1$ u $S_2$ расположены в плоскости, параллельной экрану, на расстоянии $7{,}2\,\text{м}$ от него.
    На экране в точках, лежащих на перпендикулярах, опущенных из источников света $S_1$ и $S_2$,
    находятся первые тёмные полосы.
    Определите расстояние $S_1S_2$ между источниками, результат выразите в миллиметрах.
    Длина волны равна $350\,\text{нм}$.
}
\answer{%
    \begin{align*}
    &x_n = \frac{\lambda L}{\ell} n, n\in \mathbb{N} \leftarrow \text{точки максимума}, \\
    &1 \frac{\lambda L}{\ell} = \ell \implies \ell = \sqrt{ 1 \cdot \lambda L } = \sqrt{ 1 \cdot 350\,\text{нм} \cdot 7{,}2\,\text{м} } \approx 1{,}59\,\text{мм}.
    \end{align*}
}
\solutionspace{100pt}

\tasknumber{5}%
\task{%
    На стеклянную пластинку ($\hat n = 1{,}5$) нанесена прозрачная пленка ($n = 1{,}7$).
    На плёнку нормально к поверхности падает монохроматический свет с длиной волны $420\,\text{нм}$.
    Какова должна быть минимальная толщина пленки, чтобы в результате интерференции отражённый свет имел наибольшую интенсивность?
}
\answer{%
    $2 \cdot h \cdot n = \frac12 \lambda \implies h \approx 62\,\text{нм}$
}

\variantsplitter

\addpersonalvariant{Семён Мартынов}

\tasknumber{1}%
\task{%
    В некоторую точку пространства приходят две когерентные световые волны
    с разностью хода $2{,}50\,\text{мкм}$.
    Определите, что наблюдается в этой точке.
    Длина волны равна $500\,\text{нм}$.
}
\answer{%
    $\text{точка максимума}$
}
\solutionspace{80pt}

\tasknumber{2}%
\task{%
    Установка для наблюдения интерференции состоит
    из двух когерентных источников света и экрана.
    Расстояние между источниками $l = 2{,}4\,\text{мм}$,
    а от каждого источника до экрана — $L = 4\,\text{м}$.
    Сделайте рисунок и укажите положение нулевого максимума освещенности,
    а также определите расстояние между четвёртым минимумом и нулевым максимумом.
    Длина волны падающего света составляет $\lambda = 600\,\text{нм}$.
}
\answer{%
    \begin{align*}
    l_1^2 &= L^2 + \sqr{x - \frac \ell 2} \\
    l_2^2 &= L^2 + \sqr{x + \frac \ell 2} \\
    l_2^2 - l_1^2 &= 2x\ell \implies (l_2 - l_1)(l_2 + l_1) = 2x\ell \implies n\lambda \cdot 2L \approx 2x_n\ell \implies x_n = \frac{\lambda L}{\ell} n, n\in \mathbb{N} \\
    x &= \frac{\lambda L}{\ell} \cdot \frac72 = \frac{600\,\text{нм} \cdot 4\,\text{м}}{2{,}4\,\text{мм}} \cdot \frac72 \approx 3{,}5\,\text{мм}
    \end{align*}
}
\solutionspace{120pt}

\tasknumber{3}%
\task{%
    Разность фаз двух интерферирующих световых волн равна $3\pi$,
    а разность хода между ними равна $9{,}5 \cdot 10^{-7}\,\text{м}$.
    Определить длину и частоту волны.
}
\answer{%
    \begin{align*}
    \Delta \varphi &= k\Delta l = \frac{2 \pi}{\lambda} \Delta l = 3\pi \implies \lambda = \frac23\Delta l \approx 633\,\text{нм}, \\
    &\nu = \frac 1T = \frac c\lambda = \frac32 \frac c{\Delta l} = \frac32 \cdot \frac{3 \cdot 10^{8}\,\frac{\text{м}}{\text{с}}}{9{,}5 \cdot 10^{-7}\,\text{м}} \approx 470\,\text{ТГц}.
    \end{align*}
}
\solutionspace{100pt}

\tasknumber{4}%
\task{%
    Два точечных когерентных источника света $S_1$ u $S_2$ расположены в плоскости, параллельной экрану, на расстоянии $7{,}2\,\text{м}$ от него.
    На экране в точках, лежащих на перпендикулярах, опущенных из источников света $S_1$ и $S_2$,
    находятся первые светлые полосы.
    Определите расстояние $S_1S_2$ между источниками, результат выразите в миллиметрах.
    Длина волны равна $350\,\text{нм}$.
}
\answer{%
    \begin{align*}
    &x_n = \frac{\lambda L}{\ell} n, n\in \mathbb{N} \leftarrow \text{точки максимума}, \\
    &2 \frac{\lambda L}{\ell} = \ell \implies \ell = \sqrt{ 2 \cdot \lambda L } = \sqrt{ 2 \cdot 350\,\text{нм} \cdot 7{,}2\,\text{м} } \approx 3{,}17\,\text{мм}.
    \end{align*}
}
\solutionspace{100pt}

\tasknumber{5}%
\task{%
    На стеклянную пластинку ($\hat n = 1{,}6$) нанесена прозрачная пленка ($n = 1{,}3$).
    На плёнку нормально к поверхности падает монохроматический свет с длиной волны $420\,\text{нм}$.
    Какова должна быть минимальная толщина пленки, чтобы в результате интерференции отражённый свет имел наименьшую интенсивность?
}
\answer{%
    $2 \cdot h \cdot n = \frac12 \lambda \implies h \approx 81\,\text{нм}$
}

\variantsplitter

\addpersonalvariant{Варвара Минаева}

\tasknumber{1}%
\task{%
    В некоторую точку пространства приходят две когерентные световые волны
    с разностью хода $1{,}800\,\text{мкм}$.
    Определите, что наблюдается в этой точке.
    Длина волны равна $400\,\text{нм}$.
}
\answer{%
    $\text{точка минимума}$
}
\solutionspace{80pt}

\tasknumber{2}%
\task{%
    Установка для наблюдения интерференции состоит
    из двух когерентных источников света и экрана.
    Расстояние между источниками $l = 2{,}4\,\text{мм}$,
    а от каждого источника до экрана — $L = 2\,\text{м}$.
    Сделайте рисунок и укажите положение нулевого максимума освещенности,
    а также определите расстояние между четвёртым минимумом и нулевым максимумом.
    Длина волны падающего света составляет $\lambda = 400\,\text{нм}$.
}
\answer{%
    \begin{align*}
    l_1^2 &= L^2 + \sqr{x - \frac \ell 2} \\
    l_2^2 &= L^2 + \sqr{x + \frac \ell 2} \\
    l_2^2 - l_1^2 &= 2x\ell \implies (l_2 - l_1)(l_2 + l_1) = 2x\ell \implies n\lambda \cdot 2L \approx 2x_n\ell \implies x_n = \frac{\lambda L}{\ell} n, n\in \mathbb{N} \\
    x &= \frac{\lambda L}{\ell} \cdot \frac72 = \frac{400\,\text{нм} \cdot 2\,\text{м}}{2{,}4\,\text{мм}} \cdot \frac72 \approx 1{,}17\,\text{мм}
    \end{align*}
}
\solutionspace{120pt}

\tasknumber{3}%
\task{%
    Разность фаз двух интерферирующих световых волн равна $3\pi$,
    а разность хода между ними равна $9{,}5 \cdot 10^{-7}\,\text{м}$.
    Определить длину и частоту волны.
}
\answer{%
    \begin{align*}
    \Delta \varphi &= k\Delta l = \frac{2 \pi}{\lambda} \Delta l = 3\pi \implies \lambda = \frac23\Delta l \approx 633\,\text{нм}, \\
    &\nu = \frac 1T = \frac c\lambda = \frac32 \frac c{\Delta l} = \frac32 \cdot \frac{3 \cdot 10^{8}\,\frac{\text{м}}{\text{с}}}{9{,}5 \cdot 10^{-7}\,\text{м}} \approx 470\,\text{ТГц}.
    \end{align*}
}
\solutionspace{100pt}

\tasknumber{4}%
\task{%
    Два точечных когерентных источника света $S_1$ u $S_2$ расположены в плоскости, параллельной экрану, на расстоянии $7{,}2\,\text{м}$ от него.
    На экране в точках, лежащих на перпендикулярах, опущенных из источников света $S_1$ и $S_2$,
    находятся первые тёмные полосы.
    Определите расстояние $S_1S_2$ между источниками, результат выразите в миллиметрах.
    Длина волны равна $420\,\text{нм}$.
}
\answer{%
    \begin{align*}
    &x_n = \frac{\lambda L}{\ell} n, n\in \mathbb{N} \leftarrow \text{точки максимума}, \\
    &1 \frac{\lambda L}{\ell} = \ell \implies \ell = \sqrt{ 1 \cdot \lambda L } = \sqrt{ 1 \cdot 420\,\text{нм} \cdot 7{,}2\,\text{м} } \approx 1{,}74\,\text{мм}.
    \end{align*}
}
\solutionspace{100pt}

\tasknumber{5}%
\task{%
    На стеклянную пластинку ($\hat n = 1{,}6$) нанесена прозрачная пленка ($n = 1{,}7$).
    На плёнку нормально к поверхности падает монохроматический свет с длиной волны $640\,\text{нм}$.
    Какова должна быть минимальная толщина пленки, чтобы в результате интерференции отражённый свет имел наибольшую интенсивность?
}
\answer{%
    $2 \cdot h \cdot n = \frac12 \lambda \implies h \approx 94\,\text{нм}$
}

\variantsplitter

\addpersonalvariant{Леонид Никитин}

\tasknumber{1}%
\task{%
    В некоторую точку пространства приходят две когерентные световые волны
    с разностью хода $4{,}20\,\text{мкм}$.
    Определите, что наблюдается в этой точке.
    Длина волны равна $700\,\text{нм}$.
}
\answer{%
    $\text{точка максимума}$
}
\solutionspace{80pt}

\tasknumber{2}%
\task{%
    Установка для наблюдения интерференции состоит
    из двух когерентных источников света и экрана.
    Расстояние между источниками $l = 2{,}4\,\text{мм}$,
    а от каждого источника до экрана — $L = 4\,\text{м}$.
    Сделайте рисунок и укажите положение нулевого максимума освещенности,
    а также определите расстояние между четвёртым максимумом и нулевым максимумом.
    Длина волны падающего света составляет $\lambda = 550\,\text{нм}$.
}
\answer{%
    \begin{align*}
    l_1^2 &= L^2 + \sqr{x - \frac \ell 2} \\
    l_2^2 &= L^2 + \sqr{x + \frac \ell 2} \\
    l_2^2 - l_1^2 &= 2x\ell \implies (l_2 - l_1)(l_2 + l_1) = 2x\ell \implies n\lambda \cdot 2L \approx 2x_n\ell \implies x_n = \frac{\lambda L}{\ell} n, n\in \mathbb{N} \\
    x &= \frac{\lambda L}{\ell} \cdot 4 = \frac{550\,\text{нм} \cdot 4\,\text{м}}{2{,}4\,\text{мм}} \cdot 4 \approx 3{,}7\,\text{мм}
    \end{align*}
}
\solutionspace{120pt}

\tasknumber{3}%
\task{%
    Разность фаз двух интерферирующих световых волн равна $8\pi$,
    а разность хода между ними равна $7{,}5 \cdot 10^{-7}\,\text{м}$.
    Определить длину и частоту волны.
}
\answer{%
    \begin{align*}
    \Delta \varphi &= k\Delta l = \frac{2 \pi}{\lambda} \Delta l = 8\pi \implies \lambda = \frac14\Delta l \approx 187{,}5\,\text{нм}, \\
    &\nu = \frac 1T = \frac c\lambda = 4 \frac c{\Delta l} = 4 \cdot \frac{3 \cdot 10^{8}\,\frac{\text{м}}{\text{с}}}{7{,}5 \cdot 10^{-7}\,\text{м}} \approx 1600\,\text{ТГц}.
    \end{align*}
}
\solutionspace{100pt}

\tasknumber{4}%
\task{%
    Два точечных когерентных источника света $S_1$ u $S_2$ расположены в плоскости, параллельной экрану, на расстоянии $5{,}4\,\text{м}$ от него.
    На экране в точках, лежащих на перпендикулярах, опущенных из источников света $S_1$ и $S_2$,
    находятся первые светлые полосы.
    Определите расстояние $S_1S_2$ между источниками, результат выразите в миллиметрах.
    Длина волны равна $640\,\text{нм}$.
}
\answer{%
    \begin{align*}
    &x_n = \frac{\lambda L}{\ell} n, n\in \mathbb{N} \leftarrow \text{точки максимума}, \\
    &2 \frac{\lambda L}{\ell} = \ell \implies \ell = \sqrt{ 2 \cdot \lambda L } = \sqrt{ 2 \cdot 640\,\text{нм} \cdot 5{,}4\,\text{м} } \approx 3{,}72\,\text{мм}.
    \end{align*}
}
\solutionspace{100pt}

\tasknumber{5}%
\task{%
    На стеклянную пластинку ($\hat n = 1{,}5$) нанесена прозрачная пленка ($n = 1{,}8$).
    На плёнку нормально к поверхности падает монохроматический свет с длиной волны $540\,\text{нм}$.
    Какова должна быть минимальная толщина пленки, чтобы в результате интерференции отражённый свет имел наибольшую интенсивность?
}
\answer{%
    $2 \cdot h \cdot n = \frac12 \lambda \implies h \approx 75\,\text{нм}$
}

\variantsplitter

\addpersonalvariant{Тимофей Полетаев}

\tasknumber{1}%
\task{%
    В некоторую точку пространства приходят две когерентные световые волны
    с разностью хода $2{,}50\,\text{мкм}$.
    Определите, что наблюдается в этой точке.
    Длина волны равна $500\,\text{нм}$.
}
\answer{%
    $\text{точка максимума}$
}
\solutionspace{80pt}

\tasknumber{2}%
\task{%
    Установка для наблюдения интерференции состоит
    из двух когерентных источников света и экрана.
    Расстояние между источниками $l = 1{,}2\,\text{мм}$,
    а от каждого источника до экрана — $L = 3\,\text{м}$.
    Сделайте рисунок и укажите положение нулевого максимума освещенности,
    а также определите расстояние между вторым максимумом и нулевым максимумом.
    Длина волны падающего света составляет $\lambda = 600\,\text{нм}$.
}
\answer{%
    \begin{align*}
    l_1^2 &= L^2 + \sqr{x - \frac \ell 2} \\
    l_2^2 &= L^2 + \sqr{x + \frac \ell 2} \\
    l_2^2 - l_1^2 &= 2x\ell \implies (l_2 - l_1)(l_2 + l_1) = 2x\ell \implies n\lambda \cdot 2L \approx 2x_n\ell \implies x_n = \frac{\lambda L}{\ell} n, n\in \mathbb{N} \\
    x &= \frac{\lambda L}{\ell} \cdot 2 = \frac{600\,\text{нм} \cdot 3\,\text{м}}{1{,}2\,\text{мм}} \cdot 2 \approx 3\,\text{мм}
    \end{align*}
}
\solutionspace{120pt}

\tasknumber{3}%
\task{%
    Разность фаз двух интерферирующих световых волн равна $4\pi$,
    а разность хода между ними равна $7{,}5 \cdot 10^{-7}\,\text{м}$.
    Определить длину и частоту волны.
}
\answer{%
    \begin{align*}
    \Delta \varphi &= k\Delta l = \frac{2 \pi}{\lambda} \Delta l = 4\pi \implies \lambda = \frac12\Delta l \approx 375\,\text{нм}, \\
    &\nu = \frac 1T = \frac c\lambda = 2 \frac c{\Delta l} = 2 \cdot \frac{3 \cdot 10^{8}\,\frac{\text{м}}{\text{с}}}{7{,}5 \cdot 10^{-7}\,\text{м}} \approx 800\,\text{ТГц}.
    \end{align*}
}
\solutionspace{100pt}

\tasknumber{4}%
\task{%
    Два точечных когерентных источника света $S_1$ u $S_2$ расположены в плоскости, параллельной экрану, на расстоянии $5{,}4\,\text{м}$ от него.
    На экране в точках, лежащих на перпендикулярах, опущенных из источников света $S_1$ и $S_2$,
    находятся первые тёмные полосы.
    Определите расстояние $S_1S_2$ между источниками, результат выразите в миллиметрах.
    Длина волны равна $640\,\text{нм}$.
}
\answer{%
    \begin{align*}
    &x_n = \frac{\lambda L}{\ell} n, n\in \mathbb{N} \leftarrow \text{точки максимума}, \\
    &1 \frac{\lambda L}{\ell} = \ell \implies \ell = \sqrt{ 1 \cdot \lambda L } = \sqrt{ 1 \cdot 640\,\text{нм} \cdot 5{,}4\,\text{м} } \approx 1{,}86\,\text{мм}.
    \end{align*}
}
\solutionspace{100pt}

\tasknumber{5}%
\task{%
    На стеклянную пластинку ($\hat n = 1{,}6$) нанесена прозрачная пленка ($n = 1{,}4$).
    На плёнку нормально к поверхности падает монохроматический свет с длиной волны $540\,\text{нм}$.
    Какова должна быть минимальная толщина пленки, чтобы в результате интерференции отражённый свет имел наименьшую интенсивность?
}
\answer{%
    $2 \cdot h \cdot n = \frac12 \lambda \implies h \approx 96\,\text{нм}$
}

\variantsplitter

\addpersonalvariant{Андрей Рожков}

\tasknumber{1}%
\task{%
    В некоторую точку пространства приходят две когерентные световые волны
    с разностью хода $2{,}10\,\text{мкм}$.
    Определите, что наблюдается в этой точке.
    Длина волны равна $600\,\text{нм}$.
}
\answer{%
    $\text{точка минимума}$
}
\solutionspace{80pt}

\tasknumber{2}%
\task{%
    Установка для наблюдения интерференции состоит
    из двух когерентных источников света и экрана.
    Расстояние между источниками $l = 0{,}8\,\text{мм}$,
    а от каждого источника до экрана — $L = 3\,\text{м}$.
    Сделайте рисунок и укажите положение нулевого максимума освещенности,
    а также определите расстояние между третьим максимумом и нулевым максимумом.
    Длина волны падающего света составляет $\lambda = 500\,\text{нм}$.
}
\answer{%
    \begin{align*}
    l_1^2 &= L^2 + \sqr{x - \frac \ell 2} \\
    l_2^2 &= L^2 + \sqr{x + \frac \ell 2} \\
    l_2^2 - l_1^2 &= 2x\ell \implies (l_2 - l_1)(l_2 + l_1) = 2x\ell \implies n\lambda \cdot 2L \approx 2x_n\ell \implies x_n = \frac{\lambda L}{\ell} n, n\in \mathbb{N} \\
    x &= \frac{\lambda L}{\ell} \cdot 3 = \frac{500\,\text{нм} \cdot 3\,\text{м}}{0{,}8\,\text{мм}} \cdot 3 \approx 5{,}6\,\text{мм}
    \end{align*}
}
\solutionspace{120pt}

\tasknumber{3}%
\task{%
    Разность фаз двух интерферирующих световых волн равна $8\pi$,
    а разность хода между ними равна $12{,}5 \cdot 10^{-7}\,\text{м}$.
    Определить длину и частоту волны.
}
\answer{%
    \begin{align*}
    \Delta \varphi &= k\Delta l = \frac{2 \pi}{\lambda} \Delta l = 8\pi \implies \lambda = \frac14\Delta l \approx 313\,\text{нм}, \\
    &\nu = \frac 1T = \frac c\lambda = 4 \frac c{\Delta l} = 4 \cdot \frac{3 \cdot 10^{8}\,\frac{\text{м}}{\text{с}}}{12{,}5 \cdot 10^{-7}\,\text{м}} \approx 960\,\text{ТГц}.
    \end{align*}
}
\solutionspace{100pt}

\tasknumber{4}%
\task{%
    Два точечных когерентных источника света $S_1$ u $S_2$ расположены в плоскости, параллельной экрану, на расстоянии $3{,}2\,\text{м}$ от него.
    На экране в точках, лежащих на перпендикулярах, опущенных из источников света $S_1$ и $S_2$,
    находятся первые тёмные полосы.
    Определите расстояние $S_1S_2$ между источниками, результат выразите в миллиметрах.
    Длина волны равна $350\,\text{нм}$.
}
\answer{%
    \begin{align*}
    &x_n = \frac{\lambda L}{\ell} n, n\in \mathbb{N} \leftarrow \text{точки максимума}, \\
    &1 \frac{\lambda L}{\ell} = \ell \implies \ell = \sqrt{ 1 \cdot \lambda L } = \sqrt{ 1 \cdot 350\,\text{нм} \cdot 3{,}2\,\text{м} } \approx 1{,}06\,\text{мм}.
    \end{align*}
}
\solutionspace{100pt}

\tasknumber{5}%
\task{%
    На стеклянную пластинку ($\hat n = 1{,}5$) нанесена прозрачная пленка ($n = 1{,}7$).
    На плёнку нормально к поверхности падает монохроматический свет с длиной волны $540\,\text{нм}$.
    Какова должна быть минимальная толщина пленки, чтобы в результате интерференции отражённый свет имел наибольшую интенсивность?
}
\answer{%
    $2 \cdot h \cdot n = \frac12 \lambda \implies h \approx 79\,\text{нм}$
}

\variantsplitter

\addpersonalvariant{Рената Таржиманова}

\tasknumber{1}%
\task{%
    В некоторую точку пространства приходят две когерентные световые волны
    с разностью хода $4{,}55\,\text{мкм}$.
    Определите, что наблюдается в этой точке.
    Длина волны равна $700\,\text{нм}$.
}
\answer{%
    $\text{точка минимума}$
}
\solutionspace{80pt}

\tasknumber{2}%
\task{%
    Установка для наблюдения интерференции состоит
    из двух когерентных источников света и экрана.
    Расстояние между источниками $l = 2{,}4\,\text{мм}$,
    а от каждого источника до экрана — $L = 3\,\text{м}$.
    Сделайте рисунок и укажите положение нулевого максимума освещенности,
    а также определите расстояние между вторым минимумом и нулевым максимумом.
    Длина волны падающего света составляет $\lambda = 450\,\text{нм}$.
}
\answer{%
    \begin{align*}
    l_1^2 &= L^2 + \sqr{x - \frac \ell 2} \\
    l_2^2 &= L^2 + \sqr{x + \frac \ell 2} \\
    l_2^2 - l_1^2 &= 2x\ell \implies (l_2 - l_1)(l_2 + l_1) = 2x\ell \implies n\lambda \cdot 2L \approx 2x_n\ell \implies x_n = \frac{\lambda L}{\ell} n, n\in \mathbb{N} \\
    x &= \frac{\lambda L}{\ell} \cdot \frac32 = \frac{450\,\text{нм} \cdot 3\,\text{м}}{2{,}4\,\text{мм}} \cdot \frac32 \approx 0{,}84\,\text{мм}
    \end{align*}
}
\solutionspace{120pt}

\tasknumber{3}%
\task{%
    Разность фаз двух интерферирующих световых волн равна $5\pi$,
    а разность хода между ними равна $7{,}5 \cdot 10^{-7}\,\text{м}$.
    Определить длину и частоту волны.
}
\answer{%
    \begin{align*}
    \Delta \varphi &= k\Delta l = \frac{2 \pi}{\lambda} \Delta l = 5\pi \implies \lambda = \frac25\Delta l \approx 300\,\text{нм}, \\
    &\nu = \frac 1T = \frac c\lambda = \frac52 \frac c{\Delta l} = \frac52 \cdot \frac{3 \cdot 10^{8}\,\frac{\text{м}}{\text{с}}}{7{,}5 \cdot 10^{-7}\,\text{м}} \approx 1000\,\text{ТГц}.
    \end{align*}
}
\solutionspace{100pt}

\tasknumber{4}%
\task{%
    Два точечных когерентных источника света $S_1$ u $S_2$ расположены в плоскости, параллельной экрану, на расстоянии $2{,}4\,\text{м}$ от него.
    На экране в точках, лежащих на перпендикулярах, опущенных из источников света $S_1$ и $S_2$,
    находятся первые тёмные полосы.
    Определите расстояние $S_1S_2$ между источниками, результат выразите в миллиметрах.
    Длина волны равна $420\,\text{нм}$.
}
\answer{%
    \begin{align*}
    &x_n = \frac{\lambda L}{\ell} n, n\in \mathbb{N} \leftarrow \text{точки максимума}, \\
    &1 \frac{\lambda L}{\ell} = \ell \implies \ell = \sqrt{ 1 \cdot \lambda L } = \sqrt{ 1 \cdot 420\,\text{нм} \cdot 2{,}4\,\text{м} } \approx 1\,\text{мм}.
    \end{align*}
}
\solutionspace{100pt}

\tasknumber{5}%
\task{%
    На стеклянную пластинку ($\hat n = 1{,}6$) нанесена прозрачная пленка ($n = 1{,}7$).
    На плёнку нормально к поверхности падает монохроматический свет с длиной волны $640\,\text{нм}$.
    Какова должна быть минимальная толщина пленки, чтобы в результате интерференции отражённый свет имел наименьшую интенсивность?
}
\answer{%
    $2 \cdot h \cdot n = 1 \lambda \implies h \approx 188\,\text{нм}$
}

\variantsplitter

\addpersonalvariant{Андрей Щербаков}

\tasknumber{1}%
\task{%
    В некоторую точку пространства приходят две когерентные световые волны
    с разностью хода $2{,}80\,\text{мкм}$.
    Определите, что наблюдается в этой точке.
    Длина волны равна $700\,\text{нм}$.
}
\answer{%
    $\text{точка максимума}$
}
\solutionspace{80pt}

\tasknumber{2}%
\task{%
    Установка для наблюдения интерференции состоит
    из двух когерентных источников света и экрана.
    Расстояние между источниками $l = 1{,}2\,\text{мм}$,
    а от каждого источника до экрана — $L = 3\,\text{м}$.
    Сделайте рисунок и укажите положение нулевого максимума освещенности,
    а также определите расстояние между третьим минимумом и нулевым максимумом.
    Длина волны падающего света составляет $\lambda = 450\,\text{нм}$.
}
\answer{%
    \begin{align*}
    l_1^2 &= L^2 + \sqr{x - \frac \ell 2} \\
    l_2^2 &= L^2 + \sqr{x + \frac \ell 2} \\
    l_2^2 - l_1^2 &= 2x\ell \implies (l_2 - l_1)(l_2 + l_1) = 2x\ell \implies n\lambda \cdot 2L \approx 2x_n\ell \implies x_n = \frac{\lambda L}{\ell} n, n\in \mathbb{N} \\
    x &= \frac{\lambda L}{\ell} \cdot \frac52 = \frac{450\,\text{нм} \cdot 3\,\text{м}}{1{,}2\,\text{мм}} \cdot \frac52 \approx 2{,}8\,\text{мм}
    \end{align*}
}
\solutionspace{120pt}

\tasknumber{3}%
\task{%
    Разность фаз двух интерферирующих световых волн равна $7\pi$,
    а разность хода между ними равна $7{,}5 \cdot 10^{-7}\,\text{м}$.
    Определить длину и частоту волны.
}
\answer{%
    \begin{align*}
    \Delta \varphi &= k\Delta l = \frac{2 \pi}{\lambda} \Delta l = 7\pi \implies \lambda = \frac27\Delta l \approx 214\,\text{нм}, \\
    &\nu = \frac 1T = \frac c\lambda = \frac72 \frac c{\Delta l} = \frac72 \cdot \frac{3 \cdot 10^{8}\,\frac{\text{м}}{\text{с}}}{7{,}5 \cdot 10^{-7}\,\text{м}} \approx 1400\,\text{ТГц}.
    \end{align*}
}
\solutionspace{100pt}

\tasknumber{4}%
\task{%
    Два точечных когерентных источника света $S_1$ u $S_2$ расположены в плоскости, параллельной экрану, на расстоянии $2{,}4\,\text{м}$ от него.
    На экране в точках, лежащих на перпендикулярах, опущенных из источников света $S_1$ и $S_2$,
    находятся первые светлые полосы.
    Определите расстояние $S_1S_2$ между источниками, результат выразите в миллиметрах.
    Длина волны равна $550\,\text{нм}$.
}
\answer{%
    \begin{align*}
    &x_n = \frac{\lambda L}{\ell} n, n\in \mathbb{N} \leftarrow \text{точки максимума}, \\
    &2 \frac{\lambda L}{\ell} = \ell \implies \ell = \sqrt{ 2 \cdot \lambda L } = \sqrt{ 2 \cdot 550\,\text{нм} \cdot 2{,}4\,\text{м} } \approx 2{,}30\,\text{мм}.
    \end{align*}
}
\solutionspace{100pt}

\tasknumber{5}%
\task{%
    На стеклянную пластинку ($\hat n = 1{,}5$) нанесена прозрачная пленка ($n = 1{,}8$).
    На плёнку нормально к поверхности падает монохроматический свет с длиной волны $540\,\text{нм}$.
    Какова должна быть минимальная толщина пленки, чтобы в результате интерференции отражённый свет имел наибольшую интенсивность?
}
\answer{%
    $2 \cdot h \cdot n = \frac12 \lambda \implies h \approx 75\,\text{нм}$
}

\variantsplitter

\addpersonalvariant{Михаил Ярошевский}

\tasknumber{1}%
\task{%
    В некоторую точку пространства приходят две когерентные световые волны
    с разностью хода $1{,}750\,\text{мкм}$.
    Определите, что наблюдается в этой точке.
    Длина волны равна $500\,\text{нм}$.
}
\answer{%
    $\text{точка минимума}$
}
\solutionspace{80pt}

\tasknumber{2}%
\task{%
    Установка для наблюдения интерференции состоит
    из двух когерентных источников света и экрана.
    Расстояние между источниками $l = 2{,}4\,\text{мм}$,
    а от каждого источника до экрана — $L = 3\,\text{м}$.
    Сделайте рисунок и укажите положение нулевого максимума освещенности,
    а также определите расстояние между четвёртым максимумом и нулевым максимумом.
    Длина волны падающего света составляет $\lambda = 550\,\text{нм}$.
}
\answer{%
    \begin{align*}
    l_1^2 &= L^2 + \sqr{x - \frac \ell 2} \\
    l_2^2 &= L^2 + \sqr{x + \frac \ell 2} \\
    l_2^2 - l_1^2 &= 2x\ell \implies (l_2 - l_1)(l_2 + l_1) = 2x\ell \implies n\lambda \cdot 2L \approx 2x_n\ell \implies x_n = \frac{\lambda L}{\ell} n, n\in \mathbb{N} \\
    x &= \frac{\lambda L}{\ell} \cdot 4 = \frac{550\,\text{нм} \cdot 3\,\text{м}}{2{,}4\,\text{мм}} \cdot 4 \approx 2{,}8\,\text{мм}
    \end{align*}
}
\solutionspace{120pt}

\tasknumber{3}%
\task{%
    Разность фаз двух интерферирующих световых волн равна $7\pi$,
    а разность хода между ними равна $12{,}5 \cdot 10^{-7}\,\text{м}$.
    Определить длину и частоту волны.
}
\answer{%
    \begin{align*}
    \Delta \varphi &= k\Delta l = \frac{2 \pi}{\lambda} \Delta l = 7\pi \implies \lambda = \frac27\Delta l \approx 357\,\text{нм}, \\
    &\nu = \frac 1T = \frac c\lambda = \frac72 \frac c{\Delta l} = \frac72 \cdot \frac{3 \cdot 10^{8}\,\frac{\text{м}}{\text{с}}}{12{,}5 \cdot 10^{-7}\,\text{м}} \approx 840\,\text{ТГц}.
    \end{align*}
}
\solutionspace{100pt}

\tasknumber{4}%
\task{%
    Два точечных когерентных источника света $S_1$ u $S_2$ расположены в плоскости, параллельной экрану, на расстоянии $7{,}2\,\text{м}$ от него.
    На экране в точках, лежащих на перпендикулярах, опущенных из источников света $S_1$ и $S_2$,
    находятся первые светлые полосы.
    Определите расстояние $S_1S_2$ между источниками, результат выразите в миллиметрах.
    Длина волны равна $480\,\text{нм}$.
}
\answer{%
    \begin{align*}
    &x_n = \frac{\lambda L}{\ell} n, n\in \mathbb{N} \leftarrow \text{точки максимума}, \\
    &2 \frac{\lambda L}{\ell} = \ell \implies \ell = \sqrt{ 2 \cdot \lambda L } = \sqrt{ 2 \cdot 480\,\text{нм} \cdot 7{,}2\,\text{м} } \approx 3{,}72\,\text{мм}.
    \end{align*}
}
\solutionspace{100pt}

\tasknumber{5}%
\task{%
    На стеклянную пластинку ($\hat n = 1{,}5$) нанесена прозрачная пленка ($n = 1{,}8$).
    На плёнку нормально к поверхности падает монохроматический свет с длиной волны $540\,\text{нм}$.
    Какова должна быть минимальная толщина пленки, чтобы в результате интерференции отражённый свет имел наименьшую интенсивность?
}
\answer{%
    $2 \cdot h \cdot n = 1 \lambda \implies h \approx 150\,\text{нм}$
}

\variantsplitter

\addpersonalvariant{Алексей Алимпиев}

\tasknumber{1}%
\task{%
    В некоторую точку пространства приходят две когерентные световые волны
    с разностью хода $3{,}50\,\text{мкм}$.
    Определите, что наблюдается в этой точке.
    Длина волны равна $700\,\text{нм}$.
}
\answer{%
    $\text{точка максимума}$
}
\solutionspace{80pt}

\tasknumber{2}%
\task{%
    Установка для наблюдения интерференции состоит
    из двух когерентных источников света и экрана.
    Расстояние между источниками $l = 1{,}5\,\text{мм}$,
    а от каждого источника до экрана — $L = 2\,\text{м}$.
    Сделайте рисунок и укажите положение нулевого максимума освещенности,
    а также определите расстояние между четвёртым минимумом и нулевым максимумом.
    Длина волны падающего света составляет $\lambda = 500\,\text{нм}$.
}
\answer{%
    \begin{align*}
    l_1^2 &= L^2 + \sqr{x - \frac \ell 2} \\
    l_2^2 &= L^2 + \sqr{x + \frac \ell 2} \\
    l_2^2 - l_1^2 &= 2x\ell \implies (l_2 - l_1)(l_2 + l_1) = 2x\ell \implies n\lambda \cdot 2L \approx 2x_n\ell \implies x_n = \frac{\lambda L}{\ell} n, n\in \mathbb{N} \\
    x &= \frac{\lambda L}{\ell} \cdot \frac72 = \frac{500\,\text{нм} \cdot 2\,\text{м}}{1{,}5\,\text{мм}} \cdot \frac72 \approx 2{,}3\,\text{мм}
    \end{align*}
}
\solutionspace{120pt}

\tasknumber{3}%
\task{%
    Разность фаз двух интерферирующих световых волн равна $8\pi$,
    а разность хода между ними равна $7{,}5 \cdot 10^{-7}\,\text{м}$.
    Определить длину и частоту волны.
}
\answer{%
    \begin{align*}
    \Delta \varphi &= k\Delta l = \frac{2 \pi}{\lambda} \Delta l = 8\pi \implies \lambda = \frac14\Delta l \approx 187{,}5\,\text{нм}, \\
    &\nu = \frac 1T = \frac c\lambda = 4 \frac c{\Delta l} = 4 \cdot \frac{3 \cdot 10^{8}\,\frac{\text{м}}{\text{с}}}{7{,}5 \cdot 10^{-7}\,\text{м}} \approx 1600\,\text{ТГц}.
    \end{align*}
}
\solutionspace{100pt}

\tasknumber{4}%
\task{%
    Два точечных когерентных источника света $S_1$ u $S_2$ расположены в плоскости, параллельной экрану, на расстоянии $5{,}4\,\text{м}$ от него.
    На экране в точках, лежащих на перпендикулярах, опущенных из источников света $S_1$ и $S_2$,
    находятся первые светлые полосы.
    Определите расстояние $S_1S_2$ между источниками, результат выразите в миллиметрах.
    Длина волны равна $480\,\text{нм}$.
}
\answer{%
    \begin{align*}
    &x_n = \frac{\lambda L}{\ell} n, n\in \mathbb{N} \leftarrow \text{точки максимума}, \\
    &2 \frac{\lambda L}{\ell} = \ell \implies \ell = \sqrt{ 2 \cdot \lambda L } = \sqrt{ 2 \cdot 480\,\text{нм} \cdot 5{,}4\,\text{м} } \approx 3{,}22\,\text{мм}.
    \end{align*}
}
\solutionspace{100pt}

\tasknumber{5}%
\task{%
    На стеклянную пластинку ($\hat n = 1{,}6$) нанесена прозрачная пленка ($n = 1{,}3$).
    На плёнку нормально к поверхности падает монохроматический свет с длиной волны $480\,\text{нм}$.
    Какова должна быть минимальная толщина пленки, чтобы в результате интерференции отражённый свет имел наименьшую интенсивность?
}
\answer{%
    $2 \cdot h \cdot n = \frac12 \lambda \implies h \approx 92\,\text{нм}$
}

\variantsplitter

\addpersonalvariant{Евгений Васин}

\tasknumber{1}%
\task{%
    В некоторую точку пространства приходят две когерентные световые волны
    с разностью хода $4{,}55\,\text{мкм}$.
    Определите, что наблюдается в этой точке.
    Длина волны равна $700\,\text{нм}$.
}
\answer{%
    $\text{точка минимума}$
}
\solutionspace{80pt}

\tasknumber{2}%
\task{%
    Установка для наблюдения интерференции состоит
    из двух когерентных источников света и экрана.
    Расстояние между источниками $l = 2{,}4\,\text{мм}$,
    а от каждого источника до экрана — $L = 3\,\text{м}$.
    Сделайте рисунок и укажите положение нулевого максимума освещенности,
    а также определите расстояние между вторым максимумом и нулевым максимумом.
    Длина волны падающего света составляет $\lambda = 500\,\text{нм}$.
}
\answer{%
    \begin{align*}
    l_1^2 &= L^2 + \sqr{x - \frac \ell 2} \\
    l_2^2 &= L^2 + \sqr{x + \frac \ell 2} \\
    l_2^2 - l_1^2 &= 2x\ell \implies (l_2 - l_1)(l_2 + l_1) = 2x\ell \implies n\lambda \cdot 2L \approx 2x_n\ell \implies x_n = \frac{\lambda L}{\ell} n, n\in \mathbb{N} \\
    x &= \frac{\lambda L}{\ell} \cdot 2 = \frac{500\,\text{нм} \cdot 3\,\text{м}}{2{,}4\,\text{мм}} \cdot 2 \approx 1{,}25\,\text{мм}
    \end{align*}
}
\solutionspace{120pt}

\tasknumber{3}%
\task{%
    Разность фаз двух интерферирующих световых волн равна $3\pi$,
    а разность хода между ними равна $9{,}5 \cdot 10^{-7}\,\text{м}$.
    Определить длину и частоту волны.
}
\answer{%
    \begin{align*}
    \Delta \varphi &= k\Delta l = \frac{2 \pi}{\lambda} \Delta l = 3\pi \implies \lambda = \frac23\Delta l \approx 633\,\text{нм}, \\
    &\nu = \frac 1T = \frac c\lambda = \frac32 \frac c{\Delta l} = \frac32 \cdot \frac{3 \cdot 10^{8}\,\frac{\text{м}}{\text{с}}}{9{,}5 \cdot 10^{-7}\,\text{м}} \approx 470\,\text{ТГц}.
    \end{align*}
}
\solutionspace{100pt}

\tasknumber{4}%
\task{%
    Два точечных когерентных источника света $S_1$ u $S_2$ расположены в плоскости, параллельной экрану, на расстоянии $5{,}4\,\text{м}$ от него.
    На экране в точках, лежащих на перпендикулярах, опущенных из источников света $S_1$ и $S_2$,
    находятся первые светлые полосы.
    Определите расстояние $S_1S_2$ между источниками, результат выразите в миллиметрах.
    Длина волны равна $640\,\text{нм}$.
}
\answer{%
    \begin{align*}
    &x_n = \frac{\lambda L}{\ell} n, n\in \mathbb{N} \leftarrow \text{точки максимума}, \\
    &2 \frac{\lambda L}{\ell} = \ell \implies \ell = \sqrt{ 2 \cdot \lambda L } = \sqrt{ 2 \cdot 640\,\text{нм} \cdot 5{,}4\,\text{м} } \approx 3{,}72\,\text{мм}.
    \end{align*}
}
\solutionspace{100pt}

\tasknumber{5}%
\task{%
    На стеклянную пластинку ($\hat n = 1{,}5$) нанесена прозрачная пленка ($n = 1{,}4$).
    На плёнку нормально к поверхности падает монохроматический свет с длиной волны $480\,\text{нм}$.
    Какова должна быть минимальная толщина пленки, чтобы в результате интерференции отражённый свет имел наибольшую интенсивность?
}
\answer{%
    $2 \cdot h \cdot n = 1 \lambda \implies h \approx 171\,\text{нм}$
}

\variantsplitter

\addpersonalvariant{Вячеслав Волохов}

\tasknumber{1}%
\task{%
    В некоторую точку пространства приходят две когерентные световые волны
    с разностью хода $3{,}90\,\text{мкм}$.
    Определите, что наблюдается в этой точке.
    Длина волны равна $600\,\text{нм}$.
}
\answer{%
    $\text{точка минимума}$
}
\solutionspace{80pt}

\tasknumber{2}%
\task{%
    Установка для наблюдения интерференции состоит
    из двух когерентных источников света и экрана.
    Расстояние между источниками $l = 2{,}4\,\text{мм}$,
    а от каждого источника до экрана — $L = 3\,\text{м}$.
    Сделайте рисунок и укажите положение нулевого максимума освещенности,
    а также определите расстояние между третьим максимумом и нулевым максимумом.
    Длина волны падающего света составляет $\lambda = 600\,\text{нм}$.
}
\answer{%
    \begin{align*}
    l_1^2 &= L^2 + \sqr{x - \frac \ell 2} \\
    l_2^2 &= L^2 + \sqr{x + \frac \ell 2} \\
    l_2^2 - l_1^2 &= 2x\ell \implies (l_2 - l_1)(l_2 + l_1) = 2x\ell \implies n\lambda \cdot 2L \approx 2x_n\ell \implies x_n = \frac{\lambda L}{\ell} n, n\in \mathbb{N} \\
    x &= \frac{\lambda L}{\ell} \cdot 3 = \frac{600\,\text{нм} \cdot 3\,\text{м}}{2{,}4\,\text{мм}} \cdot 3 \approx 2{,}3\,\text{мм}
    \end{align*}
}
\solutionspace{120pt}

\tasknumber{3}%
\task{%
    Разность фаз двух интерферирующих световых волн равна $6\pi$,
    а разность хода между ними равна $9{,}5 \cdot 10^{-7}\,\text{м}$.
    Определить длину и частоту волны.
}
\answer{%
    \begin{align*}
    \Delta \varphi &= k\Delta l = \frac{2 \pi}{\lambda} \Delta l = 6\pi \implies \lambda = \frac13\Delta l \approx 317\,\text{нм}, \\
    &\nu = \frac 1T = \frac c\lambda = 3 \frac c{\Delta l} = 3 \cdot \frac{3 \cdot 10^{8}\,\frac{\text{м}}{\text{с}}}{9{,}5 \cdot 10^{-7}\,\text{м}} \approx 947\,\text{ТГц}.
    \end{align*}
}
\solutionspace{100pt}

\tasknumber{4}%
\task{%
    Два точечных когерентных источника света $S_1$ u $S_2$ расположены в плоскости, параллельной экрану, на расстоянии $4{,}5\,\text{м}$ от него.
    На экране в точках, лежащих на перпендикулярах, опущенных из источников света $S_1$ и $S_2$,
    находятся первые светлые полосы.
    Определите расстояние $S_1S_2$ между источниками, результат выразите в миллиметрах.
    Длина волны равна $550\,\text{нм}$.
}
\answer{%
    \begin{align*}
    &x_n = \frac{\lambda L}{\ell} n, n\in \mathbb{N} \leftarrow \text{точки максимума}, \\
    &2 \frac{\lambda L}{\ell} = \ell \implies \ell = \sqrt{ 2 \cdot \lambda L } = \sqrt{ 2 \cdot 550\,\text{нм} \cdot 4{,}5\,\text{м} } \approx 3{,}15\,\text{мм}.
    \end{align*}
}
\solutionspace{100pt}

\tasknumber{5}%
\task{%
    На стеклянную пластинку ($\hat n = 1{,}5$) нанесена прозрачная пленка ($n = 1{,}3$).
    На плёнку нормально к поверхности падает монохроматический свет с длиной волны $540\,\text{нм}$.
    Какова должна быть минимальная толщина пленки, чтобы в результате интерференции отражённый свет имел наименьшую интенсивность?
}
\answer{%
    $2 \cdot h \cdot n = \frac12 \lambda \implies h \approx 104\,\text{нм}$
}

\variantsplitter

\addpersonalvariant{Герман Говоров}

\tasknumber{1}%
\task{%
    В некоторую точку пространства приходят две когерентные световые волны
    с разностью хода $2{,}75\,\text{мкм}$.
    Определите, что наблюдается в этой точке.
    Длина волны равна $500\,\text{нм}$.
}
\answer{%
    $\text{точка минимума}$
}
\solutionspace{80pt}

\tasknumber{2}%
\task{%
    Установка для наблюдения интерференции состоит
    из двух когерентных источников света и экрана.
    Расстояние между источниками $l = 0{,}8\,\text{мм}$,
    а от каждого источника до экрана — $L = 3\,\text{м}$.
    Сделайте рисунок и укажите положение нулевого максимума освещенности,
    а также определите расстояние между третьим минимумом и нулевым максимумом.
    Длина волны падающего света составляет $\lambda = 400\,\text{нм}$.
}
\answer{%
    \begin{align*}
    l_1^2 &= L^2 + \sqr{x - \frac \ell 2} \\
    l_2^2 &= L^2 + \sqr{x + \frac \ell 2} \\
    l_2^2 - l_1^2 &= 2x\ell \implies (l_2 - l_1)(l_2 + l_1) = 2x\ell \implies n\lambda \cdot 2L \approx 2x_n\ell \implies x_n = \frac{\lambda L}{\ell} n, n\in \mathbb{N} \\
    x &= \frac{\lambda L}{\ell} \cdot \frac52 = \frac{400\,\text{нм} \cdot 3\,\text{м}}{0{,}8\,\text{мм}} \cdot \frac52 \approx 3{,}8\,\text{мм}
    \end{align*}
}
\solutionspace{120pt}

\tasknumber{3}%
\task{%
    Разность фаз двух интерферирующих световых волн равна $7\pi$,
    а разность хода между ними равна $7{,}5 \cdot 10^{-7}\,\text{м}$.
    Определить длину и частоту волны.
}
\answer{%
    \begin{align*}
    \Delta \varphi &= k\Delta l = \frac{2 \pi}{\lambda} \Delta l = 7\pi \implies \lambda = \frac27\Delta l \approx 214\,\text{нм}, \\
    &\nu = \frac 1T = \frac c\lambda = \frac72 \frac c{\Delta l} = \frac72 \cdot \frac{3 \cdot 10^{8}\,\frac{\text{м}}{\text{с}}}{7{,}5 \cdot 10^{-7}\,\text{м}} \approx 1400\,\text{ТГц}.
    \end{align*}
}
\solutionspace{100pt}

\tasknumber{4}%
\task{%
    Два точечных когерентных источника света $S_1$ u $S_2$ расположены в плоскости, параллельной экрану, на расстоянии $2{,}4\,\text{м}$ от него.
    На экране в точках, лежащих на перпендикулярах, опущенных из источников света $S_1$ и $S_2$,
    находятся первые тёмные полосы.
    Определите расстояние $S_1S_2$ между источниками, результат выразите в миллиметрах.
    Длина волны равна $640\,\text{нм}$.
}
\answer{%
    \begin{align*}
    &x_n = \frac{\lambda L}{\ell} n, n\in \mathbb{N} \leftarrow \text{точки максимума}, \\
    &1 \frac{\lambda L}{\ell} = \ell \implies \ell = \sqrt{ 1 \cdot \lambda L } = \sqrt{ 1 \cdot 640\,\text{нм} \cdot 2{,}4\,\text{м} } \approx 1{,}24\,\text{мм}.
    \end{align*}
}
\solutionspace{100pt}

\tasknumber{5}%
\task{%
    На стеклянную пластинку ($\hat n = 1{,}5$) нанесена прозрачная пленка ($n = 1{,}7$).
    На плёнку нормально к поверхности падает монохроматический свет с длиной волны $480\,\text{нм}$.
    Какова должна быть минимальная толщина пленки, чтобы в результате интерференции отражённый свет имел наибольшую интенсивность?
}
\answer{%
    $2 \cdot h \cdot n = \frac12 \lambda \implies h \approx 71\,\text{нм}$
}

\variantsplitter

\addpersonalvariant{София Журавлёва}

\tasknumber{1}%
\task{%
    В некоторую точку пространства приходят две когерентные световые волны
    с разностью хода $1{,}500\,\text{мкм}$.
    Определите, что наблюдается в этой точке.
    Длина волны равна $500\,\text{нм}$.
}
\answer{%
    $\text{точка максимума}$
}
\solutionspace{80pt}

\tasknumber{2}%
\task{%
    Установка для наблюдения интерференции состоит
    из двух когерентных источников света и экрана.
    Расстояние между источниками $l = 2{,}4\,\text{мм}$,
    а от каждого источника до экрана — $L = 2\,\text{м}$.
    Сделайте рисунок и укажите положение нулевого максимума освещенности,
    а также определите расстояние между четвёртым максимумом и нулевым максимумом.
    Длина волны падающего света составляет $\lambda = 400\,\text{нм}$.
}
\answer{%
    \begin{align*}
    l_1^2 &= L^2 + \sqr{x - \frac \ell 2} \\
    l_2^2 &= L^2 + \sqr{x + \frac \ell 2} \\
    l_2^2 - l_1^2 &= 2x\ell \implies (l_2 - l_1)(l_2 + l_1) = 2x\ell \implies n\lambda \cdot 2L \approx 2x_n\ell \implies x_n = \frac{\lambda L}{\ell} n, n\in \mathbb{N} \\
    x &= \frac{\lambda L}{\ell} \cdot 4 = \frac{400\,\text{нм} \cdot 2\,\text{м}}{2{,}4\,\text{мм}} \cdot 4 \approx 1{,}33\,\text{мм}
    \end{align*}
}
\solutionspace{120pt}

\tasknumber{3}%
\task{%
    Разность фаз двух интерферирующих световых волн равна $8\pi$,
    а разность хода между ними равна $10{,}5 \cdot 10^{-7}\,\text{м}$.
    Определить длину и частоту волны.
}
\answer{%
    \begin{align*}
    \Delta \varphi &= k\Delta l = \frac{2 \pi}{\lambda} \Delta l = 8\pi \implies \lambda = \frac14\Delta l \approx 263\,\text{нм}, \\
    &\nu = \frac 1T = \frac c\lambda = 4 \frac c{\Delta l} = 4 \cdot \frac{3 \cdot 10^{8}\,\frac{\text{м}}{\text{с}}}{10{,}5 \cdot 10^{-7}\,\text{м}} \approx 1143\,\text{ТГц}.
    \end{align*}
}
\solutionspace{100pt}

\tasknumber{4}%
\task{%
    Два точечных когерентных источника света $S_1$ u $S_2$ расположены в плоскости, параллельной экрану, на расстоянии $4{,}5\,\text{м}$ от него.
    На экране в точках, лежащих на перпендикулярах, опущенных из источников света $S_1$ и $S_2$,
    находятся первые светлые полосы.
    Определите расстояние $S_1S_2$ между источниками, результат выразите в миллиметрах.
    Длина волны равна $550\,\text{нм}$.
}
\answer{%
    \begin{align*}
    &x_n = \frac{\lambda L}{\ell} n, n\in \mathbb{N} \leftarrow \text{точки максимума}, \\
    &2 \frac{\lambda L}{\ell} = \ell \implies \ell = \sqrt{ 2 \cdot \lambda L } = \sqrt{ 2 \cdot 550\,\text{нм} \cdot 4{,}5\,\text{м} } \approx 3{,}15\,\text{мм}.
    \end{align*}
}
\solutionspace{100pt}

\tasknumber{5}%
\task{%
    На стеклянную пластинку ($\hat n = 1{,}6$) нанесена прозрачная пленка ($n = 1{,}7$).
    На плёнку нормально к поверхности падает монохроматический свет с длиной волны $420\,\text{нм}$.
    Какова должна быть минимальная толщина пленки, чтобы в результате интерференции отражённый свет имел наименьшую интенсивность?
}
\answer{%
    $2 \cdot h \cdot n = 1 \lambda \implies h \approx 124\,\text{нм}$
}

\variantsplitter

\addpersonalvariant{Константин Козлов}

\tasknumber{1}%
\task{%
    В некоторую точку пространства приходят две когерентные световые волны
    с разностью хода $2{,}20\,\text{мкм}$.
    Определите, что наблюдается в этой точке.
    Длина волны равна $400\,\text{нм}$.
}
\answer{%
    $\text{точка минимума}$
}
\solutionspace{80pt}

\tasknumber{2}%
\task{%
    Установка для наблюдения интерференции состоит
    из двух когерентных источников света и экрана.
    Расстояние между источниками $l = 1{,}2\,\text{мм}$,
    а от каждого источника до экрана — $L = 3\,\text{м}$.
    Сделайте рисунок и укажите положение нулевого максимума освещенности,
    а также определите расстояние между четвёртым минимумом и нулевым максимумом.
    Длина волны падающего света составляет $\lambda = 400\,\text{нм}$.
}
\answer{%
    \begin{align*}
    l_1^2 &= L^2 + \sqr{x - \frac \ell 2} \\
    l_2^2 &= L^2 + \sqr{x + \frac \ell 2} \\
    l_2^2 - l_1^2 &= 2x\ell \implies (l_2 - l_1)(l_2 + l_1) = 2x\ell \implies n\lambda \cdot 2L \approx 2x_n\ell \implies x_n = \frac{\lambda L}{\ell} n, n\in \mathbb{N} \\
    x &= \frac{\lambda L}{\ell} \cdot \frac72 = \frac{400\,\text{нм} \cdot 3\,\text{м}}{1{,}2\,\text{мм}} \cdot \frac72 \approx 3{,}5\,\text{мм}
    \end{align*}
}
\solutionspace{120pt}

\tasknumber{3}%
\task{%
    Разность фаз двух интерферирующих световых волн равна $4\pi$,
    а разность хода между ними равна $9{,}5 \cdot 10^{-7}\,\text{м}$.
    Определить длину и частоту волны.
}
\answer{%
    \begin{align*}
    \Delta \varphi &= k\Delta l = \frac{2 \pi}{\lambda} \Delta l = 4\pi \implies \lambda = \frac12\Delta l \approx 475\,\text{нм}, \\
    &\nu = \frac 1T = \frac c\lambda = 2 \frac c{\Delta l} = 2 \cdot \frac{3 \cdot 10^{8}\,\frac{\text{м}}{\text{с}}}{9{,}5 \cdot 10^{-7}\,\text{м}} \approx 632\,\text{ТГц}.
    \end{align*}
}
\solutionspace{100pt}

\tasknumber{4}%
\task{%
    Два точечных когерентных источника света $S_1$ u $S_2$ расположены в плоскости, параллельной экрану, на расстоянии $2{,}4\,\text{м}$ от него.
    На экране в точках, лежащих на перпендикулярах, опущенных из источников света $S_1$ и $S_2$,
    находятся первые светлые полосы.
    Определите расстояние $S_1S_2$ между источниками, результат выразите в миллиметрах.
    Длина волны равна $640\,\text{нм}$.
}
\answer{%
    \begin{align*}
    &x_n = \frac{\lambda L}{\ell} n, n\in \mathbb{N} \leftarrow \text{точки максимума}, \\
    &2 \frac{\lambda L}{\ell} = \ell \implies \ell = \sqrt{ 2 \cdot \lambda L } = \sqrt{ 2 \cdot 640\,\text{нм} \cdot 2{,}4\,\text{м} } \approx 2{,}48\,\text{мм}.
    \end{align*}
}
\solutionspace{100pt}

\tasknumber{5}%
\task{%
    На стеклянную пластинку ($\hat n = 1{,}6$) нанесена прозрачная пленка ($n = 1{,}3$).
    На плёнку нормально к поверхности падает монохроматический свет с длиной волны $640\,\text{нм}$.
    Какова должна быть минимальная толщина пленки, чтобы в результате интерференции отражённый свет имел наибольшую интенсивность?
}
\answer{%
    $2 \cdot h \cdot n = 1 \lambda \implies h \approx 250\,\text{нм}$
}

\variantsplitter

\addpersonalvariant{Наталья Кравченко}

\tasknumber{1}%
\task{%
    В некоторую точку пространства приходят две когерентные световые волны
    с разностью хода $1{,}750\,\text{мкм}$.
    Определите, что наблюдается в этой точке.
    Длина волны равна $500\,\text{нм}$.
}
\answer{%
    $\text{точка минимума}$
}
\solutionspace{80pt}

\tasknumber{2}%
\task{%
    Установка для наблюдения интерференции состоит
    из двух когерентных источников света и экрана.
    Расстояние между источниками $l = 2{,}4\,\text{мм}$,
    а от каждого источника до экрана — $L = 2\,\text{м}$.
    Сделайте рисунок и укажите положение нулевого максимума освещенности,
    а также определите расстояние между третьим максимумом и нулевым максимумом.
    Длина волны падающего света составляет $\lambda = 450\,\text{нм}$.
}
\answer{%
    \begin{align*}
    l_1^2 &= L^2 + \sqr{x - \frac \ell 2} \\
    l_2^2 &= L^2 + \sqr{x + \frac \ell 2} \\
    l_2^2 - l_1^2 &= 2x\ell \implies (l_2 - l_1)(l_2 + l_1) = 2x\ell \implies n\lambda \cdot 2L \approx 2x_n\ell \implies x_n = \frac{\lambda L}{\ell} n, n\in \mathbb{N} \\
    x &= \frac{\lambda L}{\ell} \cdot 3 = \frac{450\,\text{нм} \cdot 2\,\text{м}}{2{,}4\,\text{мм}} \cdot 3 \approx 1{,}13\,\text{мм}
    \end{align*}
}
\solutionspace{120pt}

\tasknumber{3}%
\task{%
    Разность фаз двух интерферирующих световых волн равна $5\pi$,
    а разность хода между ними равна $9{,}5 \cdot 10^{-7}\,\text{м}$.
    Определить длину и частоту волны.
}
\answer{%
    \begin{align*}
    \Delta \varphi &= k\Delta l = \frac{2 \pi}{\lambda} \Delta l = 5\pi \implies \lambda = \frac25\Delta l \approx 380\,\text{нм}, \\
    &\nu = \frac 1T = \frac c\lambda = \frac52 \frac c{\Delta l} = \frac52 \cdot \frac{3 \cdot 10^{8}\,\frac{\text{м}}{\text{с}}}{9{,}5 \cdot 10^{-7}\,\text{м}} \approx 789\,\text{ТГц}.
    \end{align*}
}
\solutionspace{100pt}

\tasknumber{4}%
\task{%
    Два точечных когерентных источника света $S_1$ u $S_2$ расположены в плоскости, параллельной экрану, на расстоянии $3{,}2\,\text{м}$ от него.
    На экране в точках, лежащих на перпендикулярах, опущенных из источников света $S_1$ и $S_2$,
    находятся первые тёмные полосы.
    Определите расстояние $S_1S_2$ между источниками, результат выразите в миллиметрах.
    Длина волны равна $480\,\text{нм}$.
}
\answer{%
    \begin{align*}
    &x_n = \frac{\lambda L}{\ell} n, n\in \mathbb{N} \leftarrow \text{точки максимума}, \\
    &1 \frac{\lambda L}{\ell} = \ell \implies \ell = \sqrt{ 1 \cdot \lambda L } = \sqrt{ 1 \cdot 480\,\text{нм} \cdot 3{,}2\,\text{м} } \approx 1{,}24\,\text{мм}.
    \end{align*}
}
\solutionspace{100pt}

\tasknumber{5}%
\task{%
    На стеклянную пластинку ($\hat n = 1{,}5$) нанесена прозрачная пленка ($n = 1{,}3$).
    На плёнку нормально к поверхности падает монохроматический свет с длиной волны $640\,\text{нм}$.
    Какова должна быть минимальная толщина пленки, чтобы в результате интерференции отражённый свет имел наименьшую интенсивность?
}
\answer{%
    $2 \cdot h \cdot n = \frac12 \lambda \implies h \approx 123\,\text{нм}$
}

\variantsplitter

\addpersonalvariant{Матвей Кузьмин}

\tasknumber{1}%
\task{%
    В некоторую точку пространства приходят две когерентные световые волны
    с разностью хода $1{,}750\,\text{мкм}$.
    Определите, что наблюдается в этой точке.
    Длина волны равна $500\,\text{нм}$.
}
\answer{%
    $\text{точка минимума}$
}
\solutionspace{80pt}

\tasknumber{2}%
\task{%
    Установка для наблюдения интерференции состоит
    из двух когерентных источников света и экрана.
    Расстояние между источниками $l = 0{,}8\,\text{мм}$,
    а от каждого источника до экрана — $L = 4\,\text{м}$.
    Сделайте рисунок и укажите положение нулевого максимума освещенности,
    а также определите расстояние между третьим минимумом и нулевым максимумом.
    Длина волны падающего света составляет $\lambda = 450\,\text{нм}$.
}
\answer{%
    \begin{align*}
    l_1^2 &= L^2 + \sqr{x - \frac \ell 2} \\
    l_2^2 &= L^2 + \sqr{x + \frac \ell 2} \\
    l_2^2 - l_1^2 &= 2x\ell \implies (l_2 - l_1)(l_2 + l_1) = 2x\ell \implies n\lambda \cdot 2L \approx 2x_n\ell \implies x_n = \frac{\lambda L}{\ell} n, n\in \mathbb{N} \\
    x &= \frac{\lambda L}{\ell} \cdot \frac52 = \frac{450\,\text{нм} \cdot 4\,\text{м}}{0{,}8\,\text{мм}} \cdot \frac52 \approx 5{,}6\,\text{мм}
    \end{align*}
}
\solutionspace{120pt}

\tasknumber{3}%
\task{%
    Разность фаз двух интерферирующих световых волн равна $5\pi$,
    а разность хода между ними равна $12{,}5 \cdot 10^{-7}\,\text{м}$.
    Определить длину и частоту волны.
}
\answer{%
    \begin{align*}
    \Delta \varphi &= k\Delta l = \frac{2 \pi}{\lambda} \Delta l = 5\pi \implies \lambda = \frac25\Delta l \approx 500\,\text{нм}, \\
    &\nu = \frac 1T = \frac c\lambda = \frac52 \frac c{\Delta l} = \frac52 \cdot \frac{3 \cdot 10^{8}\,\frac{\text{м}}{\text{с}}}{12{,}5 \cdot 10^{-7}\,\text{м}} \approx 600\,\text{ТГц}.
    \end{align*}
}
\solutionspace{100pt}

\tasknumber{4}%
\task{%
    Два точечных когерентных источника света $S_1$ u $S_2$ расположены в плоскости, параллельной экрану, на расстоянии $3{,}2\,\text{м}$ от него.
    На экране в точках, лежащих на перпендикулярах, опущенных из источников света $S_1$ и $S_2$,
    находятся первые тёмные полосы.
    Определите расстояние $S_1S_2$ между источниками, результат выразите в миллиметрах.
    Длина волны равна $550\,\text{нм}$.
}
\answer{%
    \begin{align*}
    &x_n = \frac{\lambda L}{\ell} n, n\in \mathbb{N} \leftarrow \text{точки максимума}, \\
    &1 \frac{\lambda L}{\ell} = \ell \implies \ell = \sqrt{ 1 \cdot \lambda L } = \sqrt{ 1 \cdot 550\,\text{нм} \cdot 3{,}2\,\text{м} } \approx 1{,}33\,\text{мм}.
    \end{align*}
}
\solutionspace{100pt}

\tasknumber{5}%
\task{%
    На стеклянную пластинку ($\hat n = 1{,}6$) нанесена прозрачная пленка ($n = 1{,}7$).
    На плёнку нормально к поверхности падает монохроматический свет с длиной волны $480\,\text{нм}$.
    Какова должна быть минимальная толщина пленки, чтобы в результате интерференции отражённый свет имел наименьшую интенсивность?
}
\answer{%
    $2 \cdot h \cdot n = 1 \lambda \implies h \approx 141\,\text{нм}$
}

\variantsplitter

\addpersonalvariant{Сергей Малышев}

\tasknumber{1}%
\task{%
    В некоторую точку пространства приходят две когерентные световые волны
    с разностью хода $3{,}30\,\text{мкм}$.
    Определите, что наблюдается в этой точке.
    Длина волны равна $600\,\text{нм}$.
}
\answer{%
    $\text{точка минимума}$
}
\solutionspace{80pt}

\tasknumber{2}%
\task{%
    Установка для наблюдения интерференции состоит
    из двух когерентных источников света и экрана.
    Расстояние между источниками $l = 1{,}2\,\text{мм}$,
    а от каждого источника до экрана — $L = 3\,\text{м}$.
    Сделайте рисунок и укажите положение нулевого максимума освещенности,
    а также определите расстояние между третьим минимумом и нулевым максимумом.
    Длина волны падающего света составляет $\lambda = 550\,\text{нм}$.
}
\answer{%
    \begin{align*}
    l_1^2 &= L^2 + \sqr{x - \frac \ell 2} \\
    l_2^2 &= L^2 + \sqr{x + \frac \ell 2} \\
    l_2^2 - l_1^2 &= 2x\ell \implies (l_2 - l_1)(l_2 + l_1) = 2x\ell \implies n\lambda \cdot 2L \approx 2x_n\ell \implies x_n = \frac{\lambda L}{\ell} n, n\in \mathbb{N} \\
    x &= \frac{\lambda L}{\ell} \cdot \frac52 = \frac{550\,\text{нм} \cdot 3\,\text{м}}{1{,}2\,\text{мм}} \cdot \frac52 \approx 3{,}4\,\text{мм}
    \end{align*}
}
\solutionspace{120pt}

\tasknumber{3}%
\task{%
    Разность фаз двух интерферирующих световых волн равна $8\pi$,
    а разность хода между ними равна $15{,}5 \cdot 10^{-7}\,\text{м}$.
    Определить длину и частоту волны.
}
\answer{%
    \begin{align*}
    \Delta \varphi &= k\Delta l = \frac{2 \pi}{\lambda} \Delta l = 8\pi \implies \lambda = \frac14\Delta l \approx 388\,\text{нм}, \\
    &\nu = \frac 1T = \frac c\lambda = 4 \frac c{\Delta l} = 4 \cdot \frac{3 \cdot 10^{8}\,\frac{\text{м}}{\text{с}}}{15{,}5 \cdot 10^{-7}\,\text{м}} \approx 774\,\text{ТГц}.
    \end{align*}
}
\solutionspace{100pt}

\tasknumber{4}%
\task{%
    Два точечных когерентных источника света $S_1$ u $S_2$ расположены в плоскости, параллельной экрану, на расстоянии $5{,}4\,\text{м}$ от него.
    На экране в точках, лежащих на перпендикулярах, опущенных из источников света $S_1$ и $S_2$,
    находятся первые тёмные полосы.
    Определите расстояние $S_1S_2$ между источниками, результат выразите в миллиметрах.
    Длина волны равна $350\,\text{нм}$.
}
\answer{%
    \begin{align*}
    &x_n = \frac{\lambda L}{\ell} n, n\in \mathbb{N} \leftarrow \text{точки максимума}, \\
    &1 \frac{\lambda L}{\ell} = \ell \implies \ell = \sqrt{ 1 \cdot \lambda L } = \sqrt{ 1 \cdot 350\,\text{нм} \cdot 5{,}4\,\text{м} } \approx 1{,}37\,\text{мм}.
    \end{align*}
}
\solutionspace{100pt}

\tasknumber{5}%
\task{%
    На стеклянную пластинку ($\hat n = 1{,}5$) нанесена прозрачная пленка ($n = 1{,}3$).
    На плёнку нормально к поверхности падает монохроматический свет с длиной волны $420\,\text{нм}$.
    Какова должна быть минимальная толщина пленки, чтобы в результате интерференции отражённый свет имел наименьшую интенсивность?
}
\answer{%
    $2 \cdot h \cdot n = \frac12 \lambda \implies h \approx 81\,\text{нм}$
}

\variantsplitter

\addpersonalvariant{Алина Полканова}

\tasknumber{1}%
\task{%
    В некоторую точку пространства приходят две когерентные световые волны
    с разностью хода $3{,}90\,\text{мкм}$.
    Определите, что наблюдается в этой точке.
    Длина волны равна $600\,\text{нм}$.
}
\answer{%
    $\text{точка минимума}$
}
\solutionspace{80pt}

\tasknumber{2}%
\task{%
    Установка для наблюдения интерференции состоит
    из двух когерентных источников света и экрана.
    Расстояние между источниками $l = 2{,}4\,\text{мм}$,
    а от каждого источника до экрана — $L = 2\,\text{м}$.
    Сделайте рисунок и укажите положение нулевого максимума освещенности,
    а также определите расстояние между третьим минимумом и нулевым максимумом.
    Длина волны падающего света составляет $\lambda = 500\,\text{нм}$.
}
\answer{%
    \begin{align*}
    l_1^2 &= L^2 + \sqr{x - \frac \ell 2} \\
    l_2^2 &= L^2 + \sqr{x + \frac \ell 2} \\
    l_2^2 - l_1^2 &= 2x\ell \implies (l_2 - l_1)(l_2 + l_1) = 2x\ell \implies n\lambda \cdot 2L \approx 2x_n\ell \implies x_n = \frac{\lambda L}{\ell} n, n\in \mathbb{N} \\
    x &= \frac{\lambda L}{\ell} \cdot \frac52 = \frac{500\,\text{нм} \cdot 2\,\text{м}}{2{,}4\,\text{мм}} \cdot \frac52 \approx 1{,}04\,\text{мм}
    \end{align*}
}
\solutionspace{120pt}

\tasknumber{3}%
\task{%
    Разность фаз двух интерферирующих световых волн равна $3\pi$,
    а разность хода между ними равна $15{,}5 \cdot 10^{-7}\,\text{м}$.
    Определить длину и частоту волны.
}
\answer{%
    \begin{align*}
    \Delta \varphi &= k\Delta l = \frac{2 \pi}{\lambda} \Delta l = 3\pi \implies \lambda = \frac23\Delta l \approx 1033\,\text{нм}, \\
    &\nu = \frac 1T = \frac c\lambda = \frac32 \frac c{\Delta l} = \frac32 \cdot \frac{3 \cdot 10^{8}\,\frac{\text{м}}{\text{с}}}{15{,}5 \cdot 10^{-7}\,\text{м}} \approx 290\,\text{ТГц}.
    \end{align*}
}
\solutionspace{100pt}

\tasknumber{4}%
\task{%
    Два точечных когерентных источника света $S_1$ u $S_2$ расположены в плоскости, параллельной экрану, на расстоянии $4{,}5\,\text{м}$ от него.
    На экране в точках, лежащих на перпендикулярах, опущенных из источников света $S_1$ и $S_2$,
    находятся первые тёмные полосы.
    Определите расстояние $S_1S_2$ между источниками, результат выразите в миллиметрах.
    Длина волны равна $420\,\text{нм}$.
}
\answer{%
    \begin{align*}
    &x_n = \frac{\lambda L}{\ell} n, n\in \mathbb{N} \leftarrow \text{точки максимума}, \\
    &1 \frac{\lambda L}{\ell} = \ell \implies \ell = \sqrt{ 1 \cdot \lambda L } = \sqrt{ 1 \cdot 420\,\text{нм} \cdot 4{,}5\,\text{м} } \approx 1{,}37\,\text{мм}.
    \end{align*}
}
\solutionspace{100pt}

\tasknumber{5}%
\task{%
    На стеклянную пластинку ($\hat n = 1{,}5$) нанесена прозрачная пленка ($n = 1{,}8$).
    На плёнку нормально к поверхности падает монохроматический свет с длиной волны $420\,\text{нм}$.
    Какова должна быть минимальная толщина пленки, чтобы в результате интерференции отражённый свет имел наименьшую интенсивность?
}
\answer{%
    $2 \cdot h \cdot n = 1 \lambda \implies h \approx 117\,\text{нм}$
}

\variantsplitter

\addpersonalvariant{Сергей Пономарёв}

\tasknumber{1}%
\task{%
    В некоторую точку пространства приходят две когерентные световые волны
    с разностью хода $3\,\text{мкм}$.
    Определите, что наблюдается в этой точке.
    Длина волны равна $600\,\text{нм}$.
}
\answer{%
    $\text{точка максимума}$
}
\solutionspace{80pt}

\tasknumber{2}%
\task{%
    Установка для наблюдения интерференции состоит
    из двух когерентных источников света и экрана.
    Расстояние между источниками $l = 1{,}2\,\text{мм}$,
    а от каждого источника до экрана — $L = 2\,\text{м}$.
    Сделайте рисунок и укажите положение нулевого максимума освещенности,
    а также определите расстояние между вторым минимумом и нулевым максимумом.
    Длина волны падающего света составляет $\lambda = 450\,\text{нм}$.
}
\answer{%
    \begin{align*}
    l_1^2 &= L^2 + \sqr{x - \frac \ell 2} \\
    l_2^2 &= L^2 + \sqr{x + \frac \ell 2} \\
    l_2^2 - l_1^2 &= 2x\ell \implies (l_2 - l_1)(l_2 + l_1) = 2x\ell \implies n\lambda \cdot 2L \approx 2x_n\ell \implies x_n = \frac{\lambda L}{\ell} n, n\in \mathbb{N} \\
    x &= \frac{\lambda L}{\ell} \cdot \frac32 = \frac{450\,\text{нм} \cdot 2\,\text{м}}{1{,}2\,\text{мм}} \cdot \frac32 \approx 1{,}13\,\text{мм}
    \end{align*}
}
\solutionspace{120pt}

\tasknumber{3}%
\task{%
    Разность фаз двух интерферирующих световых волн равна $3\pi$,
    а разность хода между ними равна $7{,}5 \cdot 10^{-7}\,\text{м}$.
    Определить длину и частоту волны.
}
\answer{%
    \begin{align*}
    \Delta \varphi &= k\Delta l = \frac{2 \pi}{\lambda} \Delta l = 3\pi \implies \lambda = \frac23\Delta l \approx 500\,\text{нм}, \\
    &\nu = \frac 1T = \frac c\lambda = \frac32 \frac c{\Delta l} = \frac32 \cdot \frac{3 \cdot 10^{8}\,\frac{\text{м}}{\text{с}}}{7{,}5 \cdot 10^{-7}\,\text{м}} \approx 600\,\text{ТГц}.
    \end{align*}
}
\solutionspace{100pt}

\tasknumber{4}%
\task{%
    Два точечных когерентных источника света $S_1$ u $S_2$ расположены в плоскости, параллельной экрану, на расстоянии $3{,}2\,\text{м}$ от него.
    На экране в точках, лежащих на перпендикулярах, опущенных из источников света $S_1$ и $S_2$,
    находятся первые светлые полосы.
    Определите расстояние $S_1S_2$ между источниками, результат выразите в миллиметрах.
    Длина волны равна $350\,\text{нм}$.
}
\answer{%
    \begin{align*}
    &x_n = \frac{\lambda L}{\ell} n, n\in \mathbb{N} \leftarrow \text{точки максимума}, \\
    &2 \frac{\lambda L}{\ell} = \ell \implies \ell = \sqrt{ 2 \cdot \lambda L } = \sqrt{ 2 \cdot 350\,\text{нм} \cdot 3{,}2\,\text{м} } \approx 2{,}12\,\text{мм}.
    \end{align*}
}
\solutionspace{100pt}

\tasknumber{5}%
\task{%
    На стеклянную пластинку ($\hat n = 1{,}5$) нанесена прозрачная пленка ($n = 1{,}4$).
    На плёнку нормально к поверхности падает монохроматический свет с длиной волны $540\,\text{нм}$.
    Какова должна быть минимальная толщина пленки, чтобы в результате интерференции отражённый свет имел наибольшую интенсивность?
}
\answer{%
    $2 \cdot h \cdot n = 1 \lambda \implies h \approx 193\,\text{нм}$
}

\variantsplitter

\addpersonalvariant{Егор Свистушкин}

\tasknumber{1}%
\task{%
    В некоторую точку пространства приходят две когерентные световые волны
    с разностью хода $2{,}75\,\text{мкм}$.
    Определите, что наблюдается в этой точке.
    Длина волны равна $500\,\text{нм}$.
}
\answer{%
    $\text{точка минимума}$
}
\solutionspace{80pt}

\tasknumber{2}%
\task{%
    Установка для наблюдения интерференции состоит
    из двух когерентных источников света и экрана.
    Расстояние между источниками $l = 1{,}5\,\text{мм}$,
    а от каждого источника до экрана — $L = 4\,\text{м}$.
    Сделайте рисунок и укажите положение нулевого максимума освещенности,
    а также определите расстояние между вторым максимумом и нулевым максимумом.
    Длина волны падающего света составляет $\lambda = 450\,\text{нм}$.
}
\answer{%
    \begin{align*}
    l_1^2 &= L^2 + \sqr{x - \frac \ell 2} \\
    l_2^2 &= L^2 + \sqr{x + \frac \ell 2} \\
    l_2^2 - l_1^2 &= 2x\ell \implies (l_2 - l_1)(l_2 + l_1) = 2x\ell \implies n\lambda \cdot 2L \approx 2x_n\ell \implies x_n = \frac{\lambda L}{\ell} n, n\in \mathbb{N} \\
    x &= \frac{\lambda L}{\ell} \cdot 2 = \frac{450\,\text{нм} \cdot 4\,\text{м}}{1{,}5\,\text{мм}} \cdot 2 \approx 2{,}4\,\text{мм}
    \end{align*}
}
\solutionspace{120pt}

\tasknumber{3}%
\task{%
    Разность фаз двух интерферирующих световых волн равна $3\pi$,
    а разность хода между ними равна $12{,}5 \cdot 10^{-7}\,\text{м}$.
    Определить длину и частоту волны.
}
\answer{%
    \begin{align*}
    \Delta \varphi &= k\Delta l = \frac{2 \pi}{\lambda} \Delta l = 3\pi \implies \lambda = \frac23\Delta l \approx 833\,\text{нм}, \\
    &\nu = \frac 1T = \frac c\lambda = \frac32 \frac c{\Delta l} = \frac32 \cdot \frac{3 \cdot 10^{8}\,\frac{\text{м}}{\text{с}}}{12{,}5 \cdot 10^{-7}\,\text{м}} \approx 360\,\text{ТГц}.
    \end{align*}
}
\solutionspace{100pt}

\tasknumber{4}%
\task{%
    Два точечных когерентных источника света $S_1$ u $S_2$ расположены в плоскости, параллельной экрану, на расстоянии $5{,}4\,\text{м}$ от него.
    На экране в точках, лежащих на перпендикулярах, опущенных из источников света $S_1$ и $S_2$,
    находятся первые светлые полосы.
    Определите расстояние $S_1S_2$ между источниками, результат выразите в миллиметрах.
    Длина волны равна $640\,\text{нм}$.
}
\answer{%
    \begin{align*}
    &x_n = \frac{\lambda L}{\ell} n, n\in \mathbb{N} \leftarrow \text{точки максимума}, \\
    &2 \frac{\lambda L}{\ell} = \ell \implies \ell = \sqrt{ 2 \cdot \lambda L } = \sqrt{ 2 \cdot 640\,\text{нм} \cdot 5{,}4\,\text{м} } \approx 3{,}72\,\text{мм}.
    \end{align*}
}
\solutionspace{100pt}

\tasknumber{5}%
\task{%
    На стеклянную пластинку ($\hat n = 1{,}5$) нанесена прозрачная пленка ($n = 1{,}7$).
    На плёнку нормально к поверхности падает монохроматический свет с длиной волны $420\,\text{нм}$.
    Какова должна быть минимальная толщина пленки, чтобы в результате интерференции отражённый свет имел наибольшую интенсивность?
}
\answer{%
    $2 \cdot h \cdot n = \frac12 \lambda \implies h \approx 62\,\text{нм}$
}

\variantsplitter

\addpersonalvariant{Дмитрий Соколов}

\tasknumber{1}%
\task{%
    В некоторую точку пространства приходят две когерентные световые волны
    с разностью хода $1{,}200\,\text{мкм}$.
    Определите, что наблюдается в этой точке.
    Длина волны равна $400\,\text{нм}$.
}
\answer{%
    $\text{точка максимума}$
}
\solutionspace{80pt}

\tasknumber{2}%
\task{%
    Установка для наблюдения интерференции состоит
    из двух когерентных источников света и экрана.
    Расстояние между источниками $l = 2{,}4\,\text{мм}$,
    а от каждого источника до экрана — $L = 2\,\text{м}$.
    Сделайте рисунок и укажите положение нулевого максимума освещенности,
    а также определите расстояние между четвёртым минимумом и нулевым максимумом.
    Длина волны падающего света составляет $\lambda = 600\,\text{нм}$.
}
\answer{%
    \begin{align*}
    l_1^2 &= L^2 + \sqr{x - \frac \ell 2} \\
    l_2^2 &= L^2 + \sqr{x + \frac \ell 2} \\
    l_2^2 - l_1^2 &= 2x\ell \implies (l_2 - l_1)(l_2 + l_1) = 2x\ell \implies n\lambda \cdot 2L \approx 2x_n\ell \implies x_n = \frac{\lambda L}{\ell} n, n\in \mathbb{N} \\
    x &= \frac{\lambda L}{\ell} \cdot \frac72 = \frac{600\,\text{нм} \cdot 2\,\text{м}}{2{,}4\,\text{мм}} \cdot \frac72 \approx 1{,}75\,\text{мм}
    \end{align*}
}
\solutionspace{120pt}

\tasknumber{3}%
\task{%
    Разность фаз двух интерферирующих световых волн равна $5\pi$,
    а разность хода между ними равна $15{,}5 \cdot 10^{-7}\,\text{м}$.
    Определить длину и частоту волны.
}
\answer{%
    \begin{align*}
    \Delta \varphi &= k\Delta l = \frac{2 \pi}{\lambda} \Delta l = 5\pi \implies \lambda = \frac25\Delta l \approx 620\,\text{нм}, \\
    &\nu = \frac 1T = \frac c\lambda = \frac52 \frac c{\Delta l} = \frac52 \cdot \frac{3 \cdot 10^{8}\,\frac{\text{м}}{\text{с}}}{15{,}5 \cdot 10^{-7}\,\text{м}} \approx 484\,\text{ТГц}.
    \end{align*}
}
\solutionspace{100pt}

\tasknumber{4}%
\task{%
    Два точечных когерентных источника света $S_1$ u $S_2$ расположены в плоскости, параллельной экрану, на расстоянии $2{,}4\,\text{м}$ от него.
    На экране в точках, лежащих на перпендикулярах, опущенных из источников света $S_1$ и $S_2$,
    находятся первые светлые полосы.
    Определите расстояние $S_1S_2$ между источниками, результат выразите в миллиметрах.
    Длина волны равна $350\,\text{нм}$.
}
\answer{%
    \begin{align*}
    &x_n = \frac{\lambda L}{\ell} n, n\in \mathbb{N} \leftarrow \text{точки максимума}, \\
    &2 \frac{\lambda L}{\ell} = \ell \implies \ell = \sqrt{ 2 \cdot \lambda L } = \sqrt{ 2 \cdot 350\,\text{нм} \cdot 2{,}4\,\text{м} } \approx 1{,}83\,\text{мм}.
    \end{align*}
}
\solutionspace{100pt}

\tasknumber{5}%
\task{%
    На стеклянную пластинку ($\hat n = 1{,}5$) нанесена прозрачная пленка ($n = 1{,}8$).
    На плёнку нормально к поверхности падает монохроматический свет с длиной волны $540\,\text{нм}$.
    Какова должна быть минимальная толщина пленки, чтобы в результате интерференции отражённый свет имел наименьшую интенсивность?
}
\answer{%
    $2 \cdot h \cdot n = 1 \lambda \implies h \approx 150\,\text{нм}$
}

\variantsplitter

\addpersonalvariant{Арсений Трофимов}

\tasknumber{1}%
\task{%
    В некоторую точку пространства приходят две когерентные световые волны
    с разностью хода $3\,\text{мкм}$.
    Определите, что наблюдается в этой точке.
    Длина волны равна $600\,\text{нм}$.
}
\answer{%
    $\text{точка максимума}$
}
\solutionspace{80pt}

\tasknumber{2}%
\task{%
    Установка для наблюдения интерференции состоит
    из двух когерентных источников света и экрана.
    Расстояние между источниками $l = 1{,}5\,\text{мм}$,
    а от каждого источника до экрана — $L = 2\,\text{м}$.
    Сделайте рисунок и укажите положение нулевого максимума освещенности,
    а также определите расстояние между третьим минимумом и нулевым максимумом.
    Длина волны падающего света составляет $\lambda = 400\,\text{нм}$.
}
\answer{%
    \begin{align*}
    l_1^2 &= L^2 + \sqr{x - \frac \ell 2} \\
    l_2^2 &= L^2 + \sqr{x + \frac \ell 2} \\
    l_2^2 - l_1^2 &= 2x\ell \implies (l_2 - l_1)(l_2 + l_1) = 2x\ell \implies n\lambda \cdot 2L \approx 2x_n\ell \implies x_n = \frac{\lambda L}{\ell} n, n\in \mathbb{N} \\
    x &= \frac{\lambda L}{\ell} \cdot \frac52 = \frac{400\,\text{нм} \cdot 2\,\text{м}}{1{,}5\,\text{мм}} \cdot \frac52 \approx 1{,}33\,\text{мм}
    \end{align*}
}
\solutionspace{120pt}

\tasknumber{3}%
\task{%
    Разность фаз двух интерферирующих световых волн равна $4\pi$,
    а разность хода между ними равна $7{,}5 \cdot 10^{-7}\,\text{м}$.
    Определить длину и частоту волны.
}
\answer{%
    \begin{align*}
    \Delta \varphi &= k\Delta l = \frac{2 \pi}{\lambda} \Delta l = 4\pi \implies \lambda = \frac12\Delta l \approx 375\,\text{нм}, \\
    &\nu = \frac 1T = \frac c\lambda = 2 \frac c{\Delta l} = 2 \cdot \frac{3 \cdot 10^{8}\,\frac{\text{м}}{\text{с}}}{7{,}5 \cdot 10^{-7}\,\text{м}} \approx 800\,\text{ТГц}.
    \end{align*}
}
\solutionspace{100pt}

\tasknumber{4}%
\task{%
    Два точечных когерентных источника света $S_1$ u $S_2$ расположены в плоскости, параллельной экрану, на расстоянии $7{,}2\,\text{м}$ от него.
    На экране в точках, лежащих на перпендикулярах, опущенных из источников света $S_1$ и $S_2$,
    находятся первые тёмные полосы.
    Определите расстояние $S_1S_2$ между источниками, результат выразите в миллиметрах.
    Длина волны равна $420\,\text{нм}$.
}
\answer{%
    \begin{align*}
    &x_n = \frac{\lambda L}{\ell} n, n\in \mathbb{N} \leftarrow \text{точки максимума}, \\
    &1 \frac{\lambda L}{\ell} = \ell \implies \ell = \sqrt{ 1 \cdot \lambda L } = \sqrt{ 1 \cdot 420\,\text{нм} \cdot 7{,}2\,\text{м} } \approx 1{,}74\,\text{мм}.
    \end{align*}
}
\solutionspace{100pt}

\tasknumber{5}%
\task{%
    На стеклянную пластинку ($\hat n = 1{,}5$) нанесена прозрачная пленка ($n = 1{,}4$).
    На плёнку нормально к поверхности падает монохроматический свет с длиной волны $420\,\text{нм}$.
    Какова должна быть минимальная толщина пленки, чтобы в результате интерференции отражённый свет имел наименьшую интенсивность?
}
\answer{%
    $2 \cdot h \cdot n = \frac12 \lambda \implies h \approx 75\,\text{нм}$
}
% autogenerated
