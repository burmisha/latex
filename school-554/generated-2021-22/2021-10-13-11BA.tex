\setdate{13~октября~2021}
\setclass{11«БА»}

\addpersonalvariant{Михаил Бурмистров}

\tasknumber{1}%
\task{%
    \begin{itemize}
        \item Запишите линейное однородное дифференциальное уравнение второго порядка,
            описывающее свободные незатухающие колебания гармонического осциллятора,
        \item запишите общее решение этого уравнения,
        \item подпишите в выписанном решении фазу и амплитуду колебаний,
        \item запишите выражение для скорости,
        \item запишите выражение для ускорения.
    \end{itemize}
}
\answer{%
    \begin{align*}
    &\ddot x + \omega^2 x = 0 \Longleftrightarrow a_x + \omega^2 x = 0, \\
    &x = A \cos(\omega t + \varphi_0) \text{ или же } x = A \sin(\omega t + \varphi_0) \text{ или же } x = a \cos(\omega t) + b \sin(\omega t), \\
    &A \text{\, или \,} \sqrt{a^2 + b^2} \text{ --- это амплитуда}, \omega t + \varphi_0\text{ --- это фаза}, \\
    &v = \dot x = -\omega A \sin(\omega t + \varphi_0), \\
    &a = \dot v = \ddot x = -\omega^2 A \cos(\omega t + \varphi_0) = -\omega^2 x,
    \end{align*}
}
\solutionspace{135pt}

\tasknumber{2}%
\task{%
    Тело колеблется по гармоническому закону,
    амплитуда этих колебаний $18\,\text{см}$, период $6\,\text{c}$.
    Чему равно смещение тела относительно положения равновесия через $22\,\text{c}$
    после прохождения положения максимального отклонения?
}
\answer{%
    $x = A \cos \omega t = A \cos \cbr{ \frac {2\pi}T t } = A \cos \cbr{ 2\pi \frac tT } = 18\,\text{см} \cdot \cos \cbr{ 2\pi \cdot \frac {22\,\text{c}}{6\,\text{c}}}\approx -9{,}00\,\text{см}.$
}
\solutionspace{120pt}

\tasknumber{3}%
\task{%
    Тело совершает гармонические колебания с периодом $6\,\text{c}$.
    За какое время тело смещается от положения наибольшего отклонения до смещения в половину амплитуды?
}
\answer{%
    $t = \frac T{6} \approx 1{,}0\,\text{c}.$
}
\solutionspace{120pt}

\tasknumber{4}%
\task{%
    Запишите формулу для периода колебаний математического маятника и ...
    \begin{itemize}
        \item укажите названия всех физических величин в формуле,
        \item выразите из неё частоту колебаний
        \item выразите из неё ускорение свободного падения.
    \end{itemize}
}
\answer{%
    \begin{align*}
    T &= 2\pi \sqrt{\frac lg} \implies \nu = \frac 1T = \frac 1{2\pi}\sqrt{\frac gl}, \omega = 2\pi\nu = \sqrt{\frac gl}, l = g\sqr{\frac T{2\pi}}, g = l\sqr{\frac {2\pi}T} \\
    T &= 2\pi \sqrt{\frac mk} \implies \nu = \frac 1T = \frac 1{2\pi}\sqrt{\frac km}, \omega = 2\pi\nu = \sqrt{\frac km}, m = k\sqr{\frac T{2\pi}}, k = m\sqr{\frac {2\pi}T}
    \end{align*}
}
\solutionspace{120pt}

\tasknumber{5}%
\task{%
    Частота колебаний пружинного маятника равна $10\,\text{Гц}$.
    Определите периоды колебаний
    \begin{itemize}
        \item потенциальной энергии системы,
        \item ускорения груза,
        \item модуля ускорения груза.
    \end{itemize}
}
\answer{%
    $T = \frac 1\nu \approx 100{,}0\,\text{мc}, T_1 = \frac T2 \approx 50{,}0\,\text{мc}, T_2 = T \approx 100{,}0\,\text{мc}, T_3 = \frac T2 \approx 50{,}0\,\text{мc}.$
}
\solutionspace{80pt}

\tasknumber{6}%
\task{%
    Тело колеблется по гармоническому закону с амплитудой $6\,\text{см}$.
    Какой минимальный путь тело может пройти за половину периода?
}
\answer{%
    $2A \approx 12{,}0\,\text{см}.$
}
\solutionspace{80pt}

\tasknumber{7}%
\task{%
    Период колебаний математического маятника равен $3\,\text{с}$,
    а их амплитуда — $20\,\text{см}$.
    Определите максимальную скорость маятника.
}
\answer{%
    $
        T = \frac{2\pi}{\omega}
        \implies \omega = \frac{2\pi}{T}
        \implies v_{\max} = \omega A = \frac{2\pi}{T}A
        \approx 41{,}9\,\frac{\text{см}}{\text{с}}.
    $
}
\solutionspace{100pt}

\tasknumber{8}%
\task{%
    Определите период колебаний груза массой $m$, подвешенного к пружине жёсткостью $k$.
    Ускорение свободного падения $g$.
}
\answer{%
    \begin{align*}
    &-kx_0 + mg = 0, \\
    F &= -k(x_0 + \Delta x), \\
    ma &= -k(x_0 + \Delta x) + mg, \\
    ma &= -kx_0 -k \Delta x + mg = -mg -k \Delta x + mg = -k \Delta x, \\
    a &+ \frac k m x = 0, \\
    \omega^2 &= \frac k m \implies T = \frac{2\pi}\omega = 2\pi\sqrt{\frac m k}.
    \end{align*}
}
\solutionspace{150pt}

\tasknumber{9}%
\task{%
    Куб со стороной $a$ и плотности $\rho$ плавает в жидкости плотностью $\rho_0$.
    Определите частоту колебаний куба, считая, что 4 грани куба всегда вертикальны.
}
\answer{%
    \begin{align*}
    \Delta F &= -\rho_0 g a^2 \Delta x,\Delta F = m\ddot x, m = \rho a^3\implies \rho a^3 \ddot x = -\rho_0 g a^2 \Delta x \implies \\
    \implies \ddot x &= -\frac{\rho_0}{\rho} \frac g a \Delta x\implies \omega^2 = \frac{\rho_0}{\rho} \cdot \frac g a\implies T = \frac{2 \pi}{\omega} = 2 \pi\sqrt{\frac{\rho}{\rho_0} \cdot \frac a g}.
    \end{align*}
}
\solutionspace{150pt}

\tasknumber{10}%
\task{%
    Математический маятник с нитью длиной $43\,\text{см}$ подвешен к потолку в лифте.
    За $30\,\text{с}$ маятник совершил $26$ колебаний.
    Определите модуль и направление ускорения лифта.
    Куда движется лифт?
}
\answer{%
    $
        T = 2\pi\sqrt{\frac\ell {a + g}}, T = \frac {t}{N}
        \implies a + g = \ell \cdot \frac{4 \pi ^ 2}{T^2},
        a = \ell \cdot \frac{4 \pi ^ 2}{T^2} - g = \ell \cdot \frac{4 \pi ^ 2 N^2}{t^2} - g \approx 2{,}75\,\frac{\text{м}}{\text{c}^{2}},
        \text{вниз}.
    $
}
\solutionspace{120pt}

\tasknumber{11}%
\task{%
    Масса груза в пружинном маятнике равна $500\,\text{г}$, при этом период его колебаний равен $1{,}3\,\text{с}$.
    Груз утяжеляют на $150\,\text{г}$.
    Определите новый период колебаний маятника.
}
\answer{%
    $
        T'
            = 2\pi\sqrt{\frac{M + m}{k}}
            = 2\pi\sqrt{\frac{M}{k} \cdot \frac{M + m}{M}}
            = T\sqrt{\frac{M + m}{M}} =  T\sqrt{1 + \frac{m}{M}} \approx 1{,}48\,\text{с}.
    $
}
\solutionspace{120pt}

\tasknumber{12}%
\task{%
    При какой длине нити математического маятника период колебаний груза массой $200\,\text{г}$
    окажется равен периоду колебаний этого же груза в пружинном маятнике с пружиной жёсткостью $60\,\frac{\text{Н}}{\text{м}}$?
}
\answer{%
    $
        2\pi \sqrt{\frac \ell g} = 2\pi \sqrt{\frac m k}
        \implies \frac \ell g = \frac m k
        \implies \ell = g \frac m k \approx 3{,}3\,\text{см}.
    $
}
\solutionspace{120pt}

\tasknumber{13}%
\task{%
    Груз подвесили к пружине, при этом удлинение пружины составило $30\,\text{мм}$.
    Определите частоту колебаний пружинного маятника, собранного из этой пружины и этого груза.
}
\answer{%
    $
        mg -k\Delta x = 0 \implies \frac m k = \frac{\Delta x} g
        \implies T = 2\pi \sqrt{\frac m k } = 2\pi \sqrt{\frac{\Delta x} g } \approx $0{,}34\,\text{с}$,
        \nu = \frac 1T \approx $2{,}91\,\text{Гц}$.
    $
}
\solutionspace{120pt}

\tasknumber{14}%
\task{%
    Определите период колебаний системы: математический маятник ограничен с одной стороны стенкой (см.
    рис.
    на доске).
    Удары маятника о стенку абсолютно упругие, $n = \sqrt{2}$.
    Длина маятника $\ell$, ускорение свободного падения $g$.
}
\answer{%
    $
        T = 2\pi\sqrt{\frac\ell g}, \qquad
        T' = 2 \cdot \frac T 4 + 2 \cdot \frac T{8} = \frac34T = \frac32\pi\sqrt{\frac\ell g}
    $
}

\variantsplitter

\addpersonalvariant{Ирина Ан}

\tasknumber{1}%
\task{%
    \begin{itemize}
        \item Запишите линейное однородное дифференциальное уравнение второго порядка,
            описывающее свободные незатухающие колебания гармонического осциллятора,
        \item запишите общее решение этого уравнения,
        \item подпишите в выписанном решении фазу и амплитуду колебаний,
        \item запишите выражение для скорости,
        \item запишите выражение для ускорения.
    \end{itemize}
}
\answer{%
    \begin{align*}
    &\ddot x + \omega^2 x = 0 \Longleftrightarrow a_x + \omega^2 x = 0, \\
    &x = A \cos(\omega t + \varphi_0) \text{ или же } x = A \sin(\omega t + \varphi_0) \text{ или же } x = a \cos(\omega t) + b \sin(\omega t), \\
    &A \text{\, или \,} \sqrt{a^2 + b^2} \text{ --- это амплитуда}, \omega t + \varphi_0\text{ --- это фаза}, \\
    &v = \dot x = -\omega A \sin(\omega t + \varphi_0), \\
    &a = \dot v = \ddot x = -\omega^2 A \cos(\omega t + \varphi_0) = -\omega^2 x,
    \end{align*}
}
\solutionspace{135pt}

\tasknumber{2}%
\task{%
    Тело колеблется по гармоническому закону,
    амплитуда этих колебаний $16\,\text{см}$, период $2\,\text{c}$.
    Чему равно смещение тела относительно положения равновесия через $24\,\text{c}$
    после прохождения положения равновесия?
}
\answer{%
    $x = A \sin \omega t = A \sin \cbr{ \frac {2\pi}T t } = A \sin \cbr{ 2\pi \frac tT } = 16\,\text{см} \cdot \sin \cbr{ 2\pi \cdot \frac {24\,\text{c}}{2\,\text{c}}}\approx 0\,\text{см}.$
}
\solutionspace{120pt}

\tasknumber{3}%
\task{%
    Тело совершает гармонические колебания с периодом $4\,\text{c}$.
    За какое время тело смещается от положения равновесия до смещения в половину амплитуды?
}
\answer{%
    $t = \frac T{12} \approx 0{,}3\,\text{c}.$
}
\solutionspace{120pt}

\tasknumber{4}%
\task{%
    Запишите формулу для периода колебаний математического маятника и ...
    \begin{itemize}
        \item укажите названия всех физических величин в формуле,
        \item выразите из неё циклическую частоту колебаний
        \item выразите из неё длину маятника.
    \end{itemize}
}
\answer{%
    \begin{align*}
    T &= 2\pi \sqrt{\frac lg} \implies \nu = \frac 1T = \frac 1{2\pi}\sqrt{\frac gl}, \omega = 2\pi\nu = \sqrt{\frac gl}, l = g\sqr{\frac T{2\pi}}, g = l\sqr{\frac {2\pi}T} \\
    T &= 2\pi \sqrt{\frac mk} \implies \nu = \frac 1T = \frac 1{2\pi}\sqrt{\frac km}, \omega = 2\pi\nu = \sqrt{\frac km}, m = k\sqr{\frac T{2\pi}}, k = m\sqr{\frac {2\pi}T}
    \end{align*}
}
\solutionspace{120pt}

\tasknumber{5}%
\task{%
    Частота колебаний математического маятника равна $15\,\text{Гц}$.
    Определите периоды колебаний
    \begin{itemize}
        \item кинетической энергии системы,
        \item ускорения груза,
        \item модуля ускорения груза.
    \end{itemize}
}
\answer{%
    $T = \frac 1\nu \approx 66{,}7\,\text{мc}, T_1 = \frac T2 \approx 33{,}3\,\text{мc}, T_2 = T \approx 66{,}7\,\text{мc}, T_3 = \frac T2 \approx 33{,}3\,\text{мc}.$
}
\solutionspace{80pt}

\tasknumber{6}%
\task{%
    Тело колеблется по гармоническому закону с амплитудой $12\,\text{см}$.
    Какой минимальный путь тело может пройти за половину периода?
}
\answer{%
    $2A \approx 24{,}0\,\text{см}.$
}
\solutionspace{80pt}

\tasknumber{7}%
\task{%
    Период колебаний математического маятника равен $5\,\text{с}$,
    а их амплитуда — $20\,\text{см}$.
    Определите максимальную скорость маятника.
}
\answer{%
    $
        T = \frac{2\pi}{\omega}
        \implies \omega = \frac{2\pi}{T}
        \implies v_{\max} = \omega A = \frac{2\pi}{T}A
        \approx 25{,}1\,\frac{\text{см}}{\text{с}}.
    $
}
\solutionspace{100pt}

\tasknumber{8}%
\task{%
    Определите период колебаний груза массой $m$, подвешенного к пружине жёсткостью $k$.
    Ускорение свободного падения $g$.
}
\answer{%
    \begin{align*}
    &-kx_0 + mg = 0, \\
    F &= -k(x_0 + \Delta x), \\
    ma &= -k(x_0 + \Delta x) + mg, \\
    ma &= -kx_0 -k \Delta x + mg = -mg -k \Delta x + mg = -k \Delta x, \\
    a &+ \frac k m x = 0, \\
    \omega^2 &= \frac k m \implies T = \frac{2\pi}\omega = 2\pi\sqrt{\frac m k}.
    \end{align*}
}
\solutionspace{150pt}

\tasknumber{9}%
\task{%
    Куб со стороной $a$ и плотности $\rho$ плавает в жидкости плотностью $\rho_0$.
    Определите частоту колебаний куба, считая, что 4 грани куба всегда вертикальны.
}
\answer{%
    \begin{align*}
    \Delta F &= -\rho_0 g a^2 \Delta x,\Delta F = m\ddot x, m = \rho a^3\implies \rho a^3 \ddot x = -\rho_0 g a^2 \Delta x \implies \\
    \implies \ddot x &= -\frac{\rho_0}{\rho} \frac g a \Delta x\implies \omega^2 = \frac{\rho_0}{\rho} \cdot \frac g a\implies T = \frac{2 \pi}{\omega} = 2 \pi\sqrt{\frac{\rho}{\rho_0} \cdot \frac a g}.
    \end{align*}
}
\solutionspace{150pt}

\tasknumber{10}%
\task{%
    Математический маятник с нитью длиной $43\,\text{см}$ подвешен к потолку в лифте.
    За $30\,\text{с}$ маятник совершил $24$ колебаний.
    Определите модуль и направление ускорения лифта.
    Куда движется лифт?
}
\answer{%
    $
        T = 2\pi\sqrt{\frac\ell {a + g}}, T = \frac {t}{N}
        \implies a + g = \ell \cdot \frac{4 \pi ^ 2}{T^2},
        a = \ell \cdot \frac{4 \pi ^ 2}{T^2} - g = \ell \cdot \frac{4 \pi ^ 2 N^2}{t^2} - g \approx 0{,}86\,\frac{\text{м}}{\text{c}^{2}},
        \text{вниз}.
    $
}
\solutionspace{120pt}

\tasknumber{11}%
\task{%
    Масса груза в пружинном маятнике равна $400\,\text{г}$, при этом период его колебаний равен $1{,}2\,\text{с}$.
    Груз облегчают на $100\,\text{г}$.
    Определите новый период колебаний маятника.
}
\answer{%
    $
        T'
            = 2\pi\sqrt{\frac{M - m}{k}}
            = 2\pi\sqrt{\frac{M}{k} \cdot \frac{M - m}{M}}
            = T\sqrt{\frac{M - m}{M}} =  T\sqrt{1 - \frac{m}{M}} \approx 1{,}04\,\text{с}.
    $
}
\solutionspace{120pt}

\tasknumber{12}%
\task{%
    При какой длине нити математического маятника период колебаний груза массой $400\,\text{г}$
    окажется равен периоду колебаний этого же груза в пружинном маятнике с пружиной жёсткостью $50\,\frac{\text{Н}}{\text{м}}$?
}
\answer{%
    $
        2\pi \sqrt{\frac \ell g} = 2\pi \sqrt{\frac m k}
        \implies \frac \ell g = \frac m k
        \implies \ell = g \frac m k \approx 8{,}0\,\text{см}.
    $
}
\solutionspace{120pt}

\tasknumber{13}%
\task{%
    Груз подвесили к пружине, при этом удлинение пружины составило $30\,\text{мм}$.
    Определите частоту колебаний пружинного маятника, собранного из этой пружины и этого груза.
}
\answer{%
    $
        mg -k\Delta x = 0 \implies \frac m k = \frac{\Delta x} g
        \implies T = 2\pi \sqrt{\frac m k } = 2\pi \sqrt{\frac{\Delta x} g } \approx $0{,}34\,\text{с}$,
        \nu = \frac 1T \approx $2{,}91\,\text{Гц}$.
    $
}
\solutionspace{120pt}

\tasknumber{14}%
\task{%
    Определите период колебаний системы: математический маятник ограничен с одной стороны стенкой (см.
    рис.
    на доске).
    Удары маятника о стенку абсолютно упругие, $n = \frac{2}{\sqrt{3}}$.
    Длина маятника $\ell$, ускорение свободного падения $g$.
}
\answer{%
    $
        T = 2\pi\sqrt{\frac\ell g}, \qquad
        T' = 2 \cdot \frac T 4 + 2 \cdot \frac T{6} = \frac56T = \frac53\pi\sqrt{\frac\ell g}
    $
}

\variantsplitter

\addpersonalvariant{Софья Андрианова}

\tasknumber{1}%
\task{%
    \begin{itemize}
        \item Запишите линейное однородное дифференциальное уравнение второго порядка,
            описывающее свободные незатухающие колебания гармонического осциллятора,
        \item запишите общее решение этого уравнения,
        \item подпишите в выписанном решении фазу и амплитуду колебаний,
        \item запишите выражение для скорости,
        \item запишите выражение для ускорения.
    \end{itemize}
}
\answer{%
    \begin{align*}
    &\ddot x + \omega^2 x = 0 \Longleftrightarrow a_x + \omega^2 x = 0, \\
    &x = A \cos(\omega t + \varphi_0) \text{ или же } x = A \sin(\omega t + \varphi_0) \text{ или же } x = a \cos(\omega t) + b \sin(\omega t), \\
    &A \text{\, или \,} \sqrt{a^2 + b^2} \text{ --- это амплитуда}, \omega t + \varphi_0\text{ --- это фаза}, \\
    &v = \dot x = -\omega A \sin(\omega t + \varphi_0), \\
    &a = \dot v = \ddot x = -\omega^2 A \cos(\omega t + \varphi_0) = -\omega^2 x,
    \end{align*}
}
\solutionspace{135pt}

\tasknumber{2}%
\task{%
    Тело колеблется по гармоническому закону,
    амплитуда этих колебаний $18\,\text{см}$, период $2\,\text{c}$.
    Чему равно смещение тела относительно положения равновесия через $24\,\text{c}$
    после прохождения положения максимального отклонения?
}
\answer{%
    $x = A \cos \omega t = A \cos \cbr{ \frac {2\pi}T t } = A \cos \cbr{ 2\pi \frac tT } = 18\,\text{см} \cdot \cos \cbr{ 2\pi \cdot \frac {24\,\text{c}}{2\,\text{c}}}\approx 18{,}0\,\text{см}.$
}
\solutionspace{120pt}

\tasknumber{3}%
\task{%
    Тело совершает гармонические колебания с периодом $4\,\text{c}$.
    За какое время тело смещается от положения равновесия до смещения в половину амплитуды?
}
\answer{%
    $t = \frac T{12} \approx 0{,}3\,\text{c}.$
}
\solutionspace{120pt}

\tasknumber{4}%
\task{%
    Запишите формулу для периода колебаний математического маятника и ...
    \begin{itemize}
        \item укажите названия всех физических величин в формуле,
        \item выразите из неё частоту колебаний
        \item выразите из неё длину маятника.
    \end{itemize}
}
\answer{%
    \begin{align*}
    T &= 2\pi \sqrt{\frac lg} \implies \nu = \frac 1T = \frac 1{2\pi}\sqrt{\frac gl}, \omega = 2\pi\nu = \sqrt{\frac gl}, l = g\sqr{\frac T{2\pi}}, g = l\sqr{\frac {2\pi}T} \\
    T &= 2\pi \sqrt{\frac mk} \implies \nu = \frac 1T = \frac 1{2\pi}\sqrt{\frac km}, \omega = 2\pi\nu = \sqrt{\frac km}, m = k\sqr{\frac T{2\pi}}, k = m\sqr{\frac {2\pi}T}
    \end{align*}
}
\solutionspace{120pt}

\tasknumber{5}%
\task{%
    Частота колебаний пружинного маятника равна $15\,\text{Гц}$.
    Определите периоды колебаний
    \begin{itemize}
        \item кинетической энергии системы,
        \item ускорения груза,
        \item модуля ускорения груза.
    \end{itemize}
}
\answer{%
    $T = \frac 1\nu \approx 66{,}7\,\text{мc}, T_1 = \frac T2 \approx 33{,}3\,\text{мc}, T_2 = T \approx 66{,}7\,\text{мc}, T_3 = \frac T2 \approx 33{,}3\,\text{мc}.$
}
\solutionspace{80pt}

\tasknumber{6}%
\task{%
    Тело колеблется по гармоническому закону с амплитудой $4\,\text{см}$.
    Какой максимальный путь тело может пройти за четверть периода?
}
\answer{%
    $2 * A \frac{1}{\sqrt 2} = A\sqrt{2} \approx 5{,}7\,\text{см}.$
}
\solutionspace{80pt}

\tasknumber{7}%
\task{%
    Период колебаний математического маятника равен $2\,\text{с}$,
    а их амплитуда — $10\,\text{см}$.
    Определите максимальную скорость маятника.
}
\answer{%
    $
        T = \frac{2\pi}{\omega}
        \implies \omega = \frac{2\pi}{T}
        \implies v_{\max} = \omega A = \frac{2\pi}{T}A
        \approx 31{,}4\,\frac{\text{см}}{\text{с}}.
    $
}
\solutionspace{100pt}

\tasknumber{8}%
\task{%
    Определите период колебаний груза массой $m$, подвешенного к пружине жёсткостью $k$.
    Ускорение свободного падения $g$.
}
\answer{%
    \begin{align*}
    &-kx_0 + mg = 0, \\
    F &= -k(x_0 + \Delta x), \\
    ma &= -k(x_0 + \Delta x) + mg, \\
    ma &= -kx_0 -k \Delta x + mg = -mg -k \Delta x + mg = -k \Delta x, \\
    a &+ \frac k m x = 0, \\
    \omega^2 &= \frac k m \implies T = \frac{2\pi}\omega = 2\pi\sqrt{\frac m k}.
    \end{align*}
}
\solutionspace{150pt}

\tasknumber{9}%
\task{%
    Куб со стороной $a$ и плотности $\rho$ плавает в жидкости плотностью $\rho_0$.
    Определите частоту колебаний куба, считая, что 4 грани куба всегда вертикальны.
}
\answer{%
    \begin{align*}
    \Delta F &= -\rho_0 g a^2 \Delta x,\Delta F = m\ddot x, m = \rho a^3\implies \rho a^3 \ddot x = -\rho_0 g a^2 \Delta x \implies \\
    \implies \ddot x &= -\frac{\rho_0}{\rho} \frac g a \Delta x\implies \omega^2 = \frac{\rho_0}{\rho} \cdot \frac g a\implies T = \frac{2 \pi}{\omega} = 2 \pi\sqrt{\frac{\rho}{\rho_0} \cdot \frac a g}.
    \end{align*}
}
\solutionspace{150pt}

\tasknumber{10}%
\task{%
    Математический маятник с нитью длиной $40\,\text{см}$ подвешен к потолку в лифте.
    За $20\,\text{с}$ маятник совершил $16$ колебаний.
    Определите модуль и направление ускорения лифта.
    Куда движется лифт?
}
\answer{%
    $
        T = 2\pi\sqrt{\frac\ell {a + g}}, T = \frac {t}{N}
        \implies a + g = \ell \cdot \frac{4 \pi ^ 2}{T^2},
        a = \ell \cdot \frac{4 \pi ^ 2}{T^2} - g = \ell \cdot \frac{4 \pi ^ 2 N^2}{t^2} - g \approx 0{,}11\,\frac{\text{м}}{\text{c}^{2}},
        \text{вниз}.
    $
}
\solutionspace{120pt}

\tasknumber{11}%
\task{%
    Масса груза в пружинном маятнике равна $500\,\text{г}$, при этом период его колебаний равен $1{,}4\,\text{с}$.
    Груз утяжеляют на $100\,\text{г}$.
    Определите новый период колебаний маятника.
}
\answer{%
    $
        T'
            = 2\pi\sqrt{\frac{M + m}{k}}
            = 2\pi\sqrt{\frac{M}{k} \cdot \frac{M + m}{M}}
            = T\sqrt{\frac{M + m}{M}} =  T\sqrt{1 + \frac{m}{M}} \approx 1{,}53\,\text{с}.
    $
}
\solutionspace{120pt}

\tasknumber{12}%
\task{%
    При какой длине нити математического маятника период колебаний груза массой $300\,\text{г}$
    окажется равен периоду колебаний этого же груза в пружинном маятнике с пружиной жёсткостью $50\,\frac{\text{Н}}{\text{м}}$?
}
\answer{%
    $
        2\pi \sqrt{\frac \ell g} = 2\pi \sqrt{\frac m k}
        \implies \frac \ell g = \frac m k
        \implies \ell = g \frac m k \approx 6{,}0\,\text{см}.
    $
}
\solutionspace{120pt}

\tasknumber{13}%
\task{%
    Груз подвесили к пружине, при этом удлинение пружины составило $30\,\text{мм}$.
    Определите частоту колебаний пружинного маятника, собранного из этой пружины и этого груза.
}
\answer{%
    $
        mg -k\Delta x = 0 \implies \frac m k = \frac{\Delta x} g
        \implies T = 2\pi \sqrt{\frac m k } = 2\pi \sqrt{\frac{\Delta x} g } \approx $0{,}34\,\text{с}$,
        \nu = \frac 1T \approx $2{,}91\,\text{Гц}$.
    $
}
\solutionspace{120pt}

\tasknumber{14}%
\task{%
    Определите период колебаний системы: математический маятник ограничен с одной стороны стенкой (см.
    рис.
    на доске).
    Удары маятника о стенку абсолютно упругие, $n = 2$.
    Длина маятника $\ell$, ускорение свободного падения $g$.
}
\answer{%
    $
        T = 2\pi\sqrt{\frac\ell g}, \qquad
        T' = 2 \cdot \frac T 4 + 2 \cdot \frac T{12} = \frac23T = \frac43\pi\sqrt{\frac\ell g}
    $
}

\variantsplitter

\addpersonalvariant{Владимир Артемчук}

\tasknumber{1}%
\task{%
    \begin{itemize}
        \item Запишите линейное однородное дифференциальное уравнение второго порядка,
            описывающее свободные незатухающие колебания гармонического осциллятора,
        \item запишите общее решение этого уравнения,
        \item подпишите в выписанном решении фазу и амплитуду колебаний,
        \item запишите выражение для скорости,
        \item запишите выражение для ускорения.
    \end{itemize}
}
\answer{%
    \begin{align*}
    &\ddot x + \omega^2 x = 0 \Longleftrightarrow a_x + \omega^2 x = 0, \\
    &x = A \cos(\omega t + \varphi_0) \text{ или же } x = A \sin(\omega t + \varphi_0) \text{ или же } x = a \cos(\omega t) + b \sin(\omega t), \\
    &A \text{\, или \,} \sqrt{a^2 + b^2} \text{ --- это амплитуда}, \omega t + \varphi_0\text{ --- это фаза}, \\
    &v = \dot x = -\omega A \sin(\omega t + \varphi_0), \\
    &a = \dot v = \ddot x = -\omega^2 A \cos(\omega t + \varphi_0) = -\omega^2 x,
    \end{align*}
}
\solutionspace{135pt}

\tasknumber{2}%
\task{%
    Тело колеблется по гармоническому закону,
    амплитуда этих колебаний $18\,\text{см}$, период $6\,\text{c}$.
    Чему равно смещение тела относительно положения равновесия через $23\,\text{c}$
    после прохождения положения максимального отклонения?
}
\answer{%
    $x = A \cos \omega t = A \cos \cbr{ \frac {2\pi}T t } = A \cos \cbr{ 2\pi \frac tT } = 18\,\text{см} \cdot \cos \cbr{ 2\pi \cdot \frac {23\,\text{c}}{6\,\text{c}}}\approx 9{,}0\,\text{см}.$
}
\solutionspace{120pt}

\tasknumber{3}%
\task{%
    Тело совершает гармонические колебания с периодом $6\,\text{c}$.
    За какое время тело смещается от положения равновесия до смещения в половину амплитуды?
}
\answer{%
    $t = \frac T{12} \approx 0{,}5\,\text{c}.$
}
\solutionspace{120pt}

\tasknumber{4}%
\task{%
    Запишите формулу для периода колебаний пружинного маятника и ...
    \begin{itemize}
        \item укажите названия всех физических величин в формуле,
        \item выразите из неё циклическую частоту колебаний
        \item выразите из неё жёсткость пружины.
    \end{itemize}
}
\answer{%
    \begin{align*}
    T &= 2\pi \sqrt{\frac lg} \implies \nu = \frac 1T = \frac 1{2\pi}\sqrt{\frac gl}, \omega = 2\pi\nu = \sqrt{\frac gl}, l = g\sqr{\frac T{2\pi}}, g = l\sqr{\frac {2\pi}T} \\
    T &= 2\pi \sqrt{\frac mk} \implies \nu = \frac 1T = \frac 1{2\pi}\sqrt{\frac km}, \omega = 2\pi\nu = \sqrt{\frac km}, m = k\sqr{\frac T{2\pi}}, k = m\sqr{\frac {2\pi}T}
    \end{align*}
}
\solutionspace{120pt}

\tasknumber{5}%
\task{%
    Частота колебаний математического маятника равна $15\,\text{Гц}$.
    Определите периоды колебаний
    \begin{itemize}
        \item кинетической энергии системы,
        \item скорости груза,
        \item модуля ускорения груза.
    \end{itemize}
}
\answer{%
    $T = \frac 1\nu \approx 66{,}7\,\text{мc}, T_1 = \frac T2 \approx 33{,}3\,\text{мc}, T_2 = T \approx 66{,}7\,\text{мc}, T_3 = \frac T2 \approx 33{,}3\,\text{мc}.$
}
\solutionspace{80pt}

\tasknumber{6}%
\task{%
    Тело колеблется по гармоническому закону с амплитудой $4\,\text{см}$.
    Какой максимальный путь тело может пройти за одну шестую долю периода?
}
\answer{%
    $2 * A \frac{1}{2} = A \approx 4{,}0\,\text{см}.$
}
\solutionspace{80pt}

\tasknumber{7}%
\task{%
    Период колебаний математического маятника равен $3\,\text{с}$,
    а их амплитуда — $20\,\text{см}$.
    Определите амплитуду колебаний скорости маятника.
}
\answer{%
    $
        T = \frac{2\pi}{\omega}
        \implies \omega = \frac{2\pi}{T}
        \implies v_{\max} = \omega A = \frac{2\pi}{T}A
        \approx 41{,}9\,\frac{\text{см}}{\text{с}}.
    $
}
\solutionspace{100pt}

\tasknumber{8}%
\task{%
    Определите период колебаний груза массой $m$, подвешенного к пружине жёсткостью $k$.
    Ускорение свободного падения $g$.
}
\answer{%
    \begin{align*}
    &-kx_0 + mg = 0, \\
    F &= -k(x_0 + \Delta x), \\
    ma &= -k(x_0 + \Delta x) + mg, \\
    ma &= -kx_0 -k \Delta x + mg = -mg -k \Delta x + mg = -k \Delta x, \\
    a &+ \frac k m x = 0, \\
    \omega^2 &= \frac k m \implies T = \frac{2\pi}\omega = 2\pi\sqrt{\frac m k}.
    \end{align*}
}
\solutionspace{150pt}

\tasknumber{9}%
\task{%
    Куб со стороной $a$ и плотности $\rho$ плавает в жидкости плотностью $\rho_0$.
    Определите частоту колебаний куба, считая, что 4 грани куба всегда вертикальны.
}
\answer{%
    \begin{align*}
    \Delta F &= -\rho_0 g a^2 \Delta x,\Delta F = m\ddot x, m = \rho a^3\implies \rho a^3 \ddot x = -\rho_0 g a^2 \Delta x \implies \\
    \implies \ddot x &= -\frac{\rho_0}{\rho} \frac g a \Delta x\implies \omega^2 = \frac{\rho_0}{\rho} \cdot \frac g a\implies T = \frac{2 \pi}{\omega} = 2 \pi\sqrt{\frac{\rho}{\rho_0} \cdot \frac a g}.
    \end{align*}
}
\solutionspace{150pt}

\tasknumber{10}%
\task{%
    Математический маятник с нитью длиной $40\,\text{см}$ подвешен к потолку в лифте.
    За $25\,\text{с}$ маятник совершил $18$ колебаний.
    Определите модуль и направление ускорения лифта.
    Куда движется лифт?
}
\answer{%
    $
        T = 2\pi\sqrt{\frac\ell {a + g}}, T = \frac {t}{N}
        \implies a + g = \ell \cdot \frac{4 \pi ^ 2}{T^2},
        a = \ell \cdot \frac{4 \pi ^ 2}{T^2} - g = \ell \cdot \frac{4 \pi ^ 2 N^2}{t^2} - g \approx -1{,}8100\,\frac{\text{м}}{\text{c}^{2}},
        \text{вверх}.
    $
}
\solutionspace{120pt}

\tasknumber{11}%
\task{%
    Масса груза в пружинном маятнике равна $600\,\text{г}$, при этом период его колебаний равен $1{,}3\,\text{с}$.
    Груз утяжеляют на $150\,\text{г}$.
    Определите новый период колебаний маятника.
}
\answer{%
    $
        T'
            = 2\pi\sqrt{\frac{M + m}{k}}
            = 2\pi\sqrt{\frac{M}{k} \cdot \frac{M + m}{M}}
            = T\sqrt{\frac{M + m}{M}} =  T\sqrt{1 + \frac{m}{M}} \approx 1{,}45\,\text{с}.
    $
}
\solutionspace{120pt}

\tasknumber{12}%
\task{%
    При какой длине нити математического маятника период колебаний груза массой $300\,\text{г}$
    окажется равен периоду колебаний этого же груза в пружинном маятнике с пружиной жёсткостью $50\,\frac{\text{Н}}{\text{м}}$?
}
\answer{%
    $
        2\pi \sqrt{\frac \ell g} = 2\pi \sqrt{\frac m k}
        \implies \frac \ell g = \frac m k
        \implies \ell = g \frac m k \approx 6{,}0\,\text{см}.
    $
}
\solutionspace{120pt}

\tasknumber{13}%
\task{%
    Груз подвесили к пружине, при этом удлинение пружины составило $60\,\text{мм}$.
    Определите частоту колебаний пружинного маятника, собранного из этой пружины и этого груза.
}
\answer{%
    $
        mg -k\Delta x = 0 \implies \frac m k = \frac{\Delta x} g
        \implies T = 2\pi \sqrt{\frac m k } = 2\pi \sqrt{\frac{\Delta x} g } \approx $0{,}49\,\text{с}$,
        \nu = \frac 1T \approx $2{,}05\,\text{Гц}$.
    $
}
\solutionspace{120pt}

\tasknumber{14}%
\task{%
    Определите период колебаний системы: математический маятник ограничен с одной стороны стенкой (см.
    рис.
    на доске).
    Удары маятника о стенку абсолютно упругие, $n = \sqrt{2}$.
    Длина маятника $\ell$, ускорение свободного падения $g$.
}
\answer{%
    $
        T = 2\pi\sqrt{\frac\ell g}, \qquad
        T' = 2 \cdot \frac T 4 + 2 \cdot \frac T{8} = \frac34T = \frac32\pi\sqrt{\frac\ell g}
    $
}

\variantsplitter

\addpersonalvariant{Софья Белянкина}

\tasknumber{1}%
\task{%
    \begin{itemize}
        \item Запишите линейное однородное дифференциальное уравнение второго порядка,
            описывающее свободные незатухающие колебания гармонического осциллятора,
        \item запишите общее решение этого уравнения,
        \item подпишите в выписанном решении фазу и амплитуду колебаний,
        \item запишите выражение для скорости,
        \item запишите выражение для ускорения.
    \end{itemize}
}
\answer{%
    \begin{align*}
    &\ddot x + \omega^2 x = 0 \Longleftrightarrow a_x + \omega^2 x = 0, \\
    &x = A \cos(\omega t + \varphi_0) \text{ или же } x = A \sin(\omega t + \varphi_0) \text{ или же } x = a \cos(\omega t) + b \sin(\omega t), \\
    &A \text{\, или \,} \sqrt{a^2 + b^2} \text{ --- это амплитуда}, \omega t + \varphi_0\text{ --- это фаза}, \\
    &v = \dot x = -\omega A \sin(\omega t + \varphi_0), \\
    &a = \dot v = \ddot x = -\omega^2 A \cos(\omega t + \varphi_0) = -\omega^2 x,
    \end{align*}
}
\solutionspace{135pt}

\tasknumber{2}%
\task{%
    Тело колеблется по гармоническому закону,
    амплитуда этих колебаний $20\,\text{см}$, период $6\,\text{c}$.
    Чему равно смещение тела относительно положения равновесия через $20\,\text{c}$
    после прохождения положения равновесия?
}
\answer{%
    $x = A \sin \omega t = A \sin \cbr{ \frac {2\pi}T t } = A \sin \cbr{ 2\pi \frac tT } = 20\,\text{см} \cdot \sin \cbr{ 2\pi \cdot \frac {20\,\text{c}}{6\,\text{c}}}\approx 17{,}3\,\text{см}.$
}
\solutionspace{120pt}

\tasknumber{3}%
\task{%
    Тело совершает гармонические колебания с периодом $4\,\text{c}$.
    За какое время тело смещается от положения наибольшего отклонения до смещения в половину амплитуды?
}
\answer{%
    $t = \frac T{6} \approx 0{,}7\,\text{c}.$
}
\solutionspace{120pt}

\tasknumber{4}%
\task{%
    Запишите формулу для периода колебаний математического маятника и ...
    \begin{itemize}
        \item укажите названия всех физических величин в формуле,
        \item выразите из неё частоту колебаний
        \item выразите из неё ускорение свободного падения.
    \end{itemize}
}
\answer{%
    \begin{align*}
    T &= 2\pi \sqrt{\frac lg} \implies \nu = \frac 1T = \frac 1{2\pi}\sqrt{\frac gl}, \omega = 2\pi\nu = \sqrt{\frac gl}, l = g\sqr{\frac T{2\pi}}, g = l\sqr{\frac {2\pi}T} \\
    T &= 2\pi \sqrt{\frac mk} \implies \nu = \frac 1T = \frac 1{2\pi}\sqrt{\frac km}, \omega = 2\pi\nu = \sqrt{\frac km}, m = k\sqr{\frac T{2\pi}}, k = m\sqr{\frac {2\pi}T}
    \end{align*}
}
\solutionspace{120pt}

\tasknumber{5}%
\task{%
    Частота колебаний математического маятника равна $8\,\text{Гц}$.
    Определите периоды колебаний
    \begin{itemize}
        \item потенциальной энергии системы,
        \item скорости груза,
        \item модуля ускорения груза.
    \end{itemize}
}
\answer{%
    $T = \frac 1\nu \approx 125{,}0\,\text{мc}, T_1 = \frac T2 \approx 62{,}5\,\text{мc}, T_2 = T \approx 125{,}0\,\text{мc}, T_3 = \frac T2 \approx 62{,}5\,\text{мc}.$
}
\solutionspace{80pt}

\tasknumber{6}%
\task{%
    Тело колеблется по гармоническому закону с амплитудой $4\,\text{см}$.
    Какой максимальный путь тело может пройти за половину периода?
}
\answer{%
    $2A \approx 8{,}0\,\text{см}.$
}
\solutionspace{80pt}

\tasknumber{7}%
\task{%
    Период колебаний математического маятника равен $2\,\text{с}$,
    а их амплитуда — $20\,\text{см}$.
    Определите максимальную скорость маятника.
}
\answer{%
    $
        T = \frac{2\pi}{\omega}
        \implies \omega = \frac{2\pi}{T}
        \implies v_{\max} = \omega A = \frac{2\pi}{T}A
        \approx 62{,}8\,\frac{\text{см}}{\text{с}}.
    $
}
\solutionspace{100pt}

\tasknumber{8}%
\task{%
    Определите период колебаний груза массой $m$, подвешенного к пружине жёсткостью $k$.
    Ускорение свободного падения $g$.
}
\answer{%
    \begin{align*}
    &-kx_0 + mg = 0, \\
    F &= -k(x_0 + \Delta x), \\
    ma &= -k(x_0 + \Delta x) + mg, \\
    ma &= -kx_0 -k \Delta x + mg = -mg -k \Delta x + mg = -k \Delta x, \\
    a &+ \frac k m x = 0, \\
    \omega^2 &= \frac k m \implies T = \frac{2\pi}\omega = 2\pi\sqrt{\frac m k}.
    \end{align*}
}
\solutionspace{150pt}

\tasknumber{9}%
\task{%
    Куб со стороной $a$ и плотности $\rho$ плавает в жидкости плотностью $\rho_0$.
    Определите частоту колебаний куба, считая, что 4 грани куба всегда вертикальны.
}
\answer{%
    \begin{align*}
    \Delta F &= -\rho_0 g a^2 \Delta x,\Delta F = m\ddot x, m = \rho a^3\implies \rho a^3 \ddot x = -\rho_0 g a^2 \Delta x \implies \\
    \implies \ddot x &= -\frac{\rho_0}{\rho} \frac g a \Delta x\implies \omega^2 = \frac{\rho_0}{\rho} \cdot \frac g a\implies T = \frac{2 \pi}{\omega} = 2 \pi\sqrt{\frac{\rho}{\rho_0} \cdot \frac a g}.
    \end{align*}
}
\solutionspace{150pt}

\tasknumber{10}%
\task{%
    Математический маятник с нитью длиной $40\,\text{см}$ подвешен к потолку в лифте.
    За $25\,\text{с}$ маятник совершил $19$ колебаний.
    Определите модуль и направление ускорения лифта.
    Куда движется лифт?
}
\answer{%
    $
        T = 2\pi\sqrt{\frac\ell {a + g}}, T = \frac {t}{N}
        \implies a + g = \ell \cdot \frac{4 \pi ^ 2}{T^2},
        a = \ell \cdot \frac{4 \pi ^ 2}{T^2} - g = \ell \cdot \frac{4 \pi ^ 2 N^2}{t^2} - g \approx -0{,}8800\,\frac{\text{м}}{\text{c}^{2}},
        \text{вверх}.
    $
}
\solutionspace{120pt}

\tasknumber{11}%
\task{%
    Масса груза в пружинном маятнике равна $500\,\text{г}$, при этом период его колебаний равен $1{,}3\,\text{с}$.
    Груз утяжеляют на $100\,\text{г}$.
    Определите новый период колебаний маятника.
}
\answer{%
    $
        T'
            = 2\pi\sqrt{\frac{M + m}{k}}
            = 2\pi\sqrt{\frac{M}{k} \cdot \frac{M + m}{M}}
            = T\sqrt{\frac{M + m}{M}} =  T\sqrt{1 + \frac{m}{M}} \approx 1{,}42\,\text{с}.
    $
}
\solutionspace{120pt}

\tasknumber{12}%
\task{%
    При какой длине нити математического маятника период колебаний груза массой $300\,\text{г}$
    окажется равен периоду колебаний этого же груза в пружинном маятнике с пружиной жёсткостью $50\,\frac{\text{Н}}{\text{м}}$?
}
\answer{%
    $
        2\pi \sqrt{\frac \ell g} = 2\pi \sqrt{\frac m k}
        \implies \frac \ell g = \frac m k
        \implies \ell = g \frac m k \approx 6{,}0\,\text{см}.
    $
}
\solutionspace{120pt}

\tasknumber{13}%
\task{%
    Груз подвесили к пружине, при этом удлинение пружины составило $30\,\text{мм}$.
    Определите частоту колебаний пружинного маятника, собранного из этой пружины и этого груза.
}
\answer{%
    $
        mg -k\Delta x = 0 \implies \frac m k = \frac{\Delta x} g
        \implies T = 2\pi \sqrt{\frac m k } = 2\pi \sqrt{\frac{\Delta x} g } \approx $0{,}34\,\text{с}$,
        \nu = \frac 1T \approx $2{,}91\,\text{Гц}$.
    $
}
\solutionspace{120pt}

\tasknumber{14}%
\task{%
    Определите период колебаний системы: математический маятник ограничен с одной стороны стенкой (см.
    рис.
    на доске).
    Удары маятника о стенку абсолютно упругие, $n = 2$.
    Длина маятника $\ell$, ускорение свободного падения $g$.
}
\answer{%
    $
        T = 2\pi\sqrt{\frac\ell g}, \qquad
        T' = 2 \cdot \frac T 4 + 2 \cdot \frac T{12} = \frac23T = \frac43\pi\sqrt{\frac\ell g}
    $
}

\variantsplitter

\addpersonalvariant{Варвара Егиазарян}

\tasknumber{1}%
\task{%
    \begin{itemize}
        \item Запишите линейное однородное дифференциальное уравнение второго порядка,
            описывающее свободные незатухающие колебания гармонического осциллятора,
        \item запишите общее решение этого уравнения,
        \item подпишите в выписанном решении фазу и амплитуду колебаний,
        \item запишите выражение для скорости,
        \item запишите выражение для ускорения.
    \end{itemize}
}
\answer{%
    \begin{align*}
    &\ddot x + \omega^2 x = 0 \Longleftrightarrow a_x + \omega^2 x = 0, \\
    &x = A \cos(\omega t + \varphi_0) \text{ или же } x = A \sin(\omega t + \varphi_0) \text{ или же } x = a \cos(\omega t) + b \sin(\omega t), \\
    &A \text{\, или \,} \sqrt{a^2 + b^2} \text{ --- это амплитуда}, \omega t + \varphi_0\text{ --- это фаза}, \\
    &v = \dot x = -\omega A \sin(\omega t + \varphi_0), \\
    &a = \dot v = \ddot x = -\omega^2 A \cos(\omega t + \varphi_0) = -\omega^2 x,
    \end{align*}
}
\solutionspace{135pt}

\tasknumber{2}%
\task{%
    Тело колеблется по гармоническому закону,
    амплитуда этих колебаний $16\,\text{см}$, период $6\,\text{c}$.
    Чему равно смещение тела относительно положения равновесия через $23\,\text{c}$
    после прохождения положения максимального отклонения?
}
\answer{%
    $x = A \cos \omega t = A \cos \cbr{ \frac {2\pi}T t } = A \cos \cbr{ 2\pi \frac tT } = 16\,\text{см} \cdot \cos \cbr{ 2\pi \cdot \frac {23\,\text{c}}{6\,\text{c}}}\approx 8{,}0\,\text{см}.$
}
\solutionspace{120pt}

\tasknumber{3}%
\task{%
    Тело совершает гармонические колебания с периодом $5\,\text{c}$.
    За какое время тело смещается от положения равновесия до смещения в половину амплитуды?
}
\answer{%
    $t = \frac T{12} \approx 0{,}4\,\text{c}.$
}
\solutionspace{120pt}

\tasknumber{4}%
\task{%
    Запишите формулу для периода колебаний пружинного маятника и ...
    \begin{itemize}
        \item укажите названия всех физических величин в формуле,
        \item выразите из неё частоту колебаний
        \item выразите из неё жёсткость пружины.
    \end{itemize}
}
\answer{%
    \begin{align*}
    T &= 2\pi \sqrt{\frac lg} \implies \nu = \frac 1T = \frac 1{2\pi}\sqrt{\frac gl}, \omega = 2\pi\nu = \sqrt{\frac gl}, l = g\sqr{\frac T{2\pi}}, g = l\sqr{\frac {2\pi}T} \\
    T &= 2\pi \sqrt{\frac mk} \implies \nu = \frac 1T = \frac 1{2\pi}\sqrt{\frac km}, \omega = 2\pi\nu = \sqrt{\frac km}, m = k\sqr{\frac T{2\pi}}, k = m\sqr{\frac {2\pi}T}
    \end{align*}
}
\solutionspace{120pt}

\tasknumber{5}%
\task{%
    Частота колебаний математического маятника равна $15\,\text{Гц}$.
    Определите периоды колебаний
    \begin{itemize}
        \item кинетической энергии системы,
        \item ускорения груза,
        \item модуля ускорения груза.
    \end{itemize}
}
\answer{%
    $T = \frac 1\nu \approx 66{,}7\,\text{мc}, T_1 = \frac T2 \approx 33{,}3\,\text{мc}, T_2 = T \approx 66{,}7\,\text{мc}, T_3 = \frac T2 \approx 33{,}3\,\text{мc}.$
}
\solutionspace{80pt}

\tasknumber{6}%
\task{%
    Тело колеблется по гармоническому закону с амплитудой $4\,\text{см}$.
    Какой минимальный путь тело может пройти за половину периода?
}
\answer{%
    $2A \approx 8{,}0\,\text{см}.$
}
\solutionspace{80pt}

\tasknumber{7}%
\task{%
    Период колебаний математического маятника равен $2\,\text{с}$,
    а их амплитуда — $20\,\text{см}$.
    Определите амплитуду колебаний скорости маятника.
}
\answer{%
    $
        T = \frac{2\pi}{\omega}
        \implies \omega = \frac{2\pi}{T}
        \implies v_{\max} = \omega A = \frac{2\pi}{T}A
        \approx 62{,}8\,\frac{\text{см}}{\text{с}}.
    $
}
\solutionspace{100pt}

\tasknumber{8}%
\task{%
    Определите период колебаний груза массой $m$, подвешенного к пружине жёсткостью $k$.
    Ускорение свободного падения $g$.
}
\answer{%
    \begin{align*}
    &-kx_0 + mg = 0, \\
    F &= -k(x_0 + \Delta x), \\
    ma &= -k(x_0 + \Delta x) + mg, \\
    ma &= -kx_0 -k \Delta x + mg = -mg -k \Delta x + mg = -k \Delta x, \\
    a &+ \frac k m x = 0, \\
    \omega^2 &= \frac k m \implies T = \frac{2\pi}\omega = 2\pi\sqrt{\frac m k}.
    \end{align*}
}
\solutionspace{150pt}

\tasknumber{9}%
\task{%
    Куб со стороной $a$ и плотности $\rho$ плавает в жидкости плотностью $\rho_0$.
    Определите частоту колебаний куба, считая, что 4 грани куба всегда вертикальны.
}
\answer{%
    \begin{align*}
    \Delta F &= -\rho_0 g a^2 \Delta x,\Delta F = m\ddot x, m = \rho a^3\implies \rho a^3 \ddot x = -\rho_0 g a^2 \Delta x \implies \\
    \implies \ddot x &= -\frac{\rho_0}{\rho} \frac g a \Delta x\implies \omega^2 = \frac{\rho_0}{\rho} \cdot \frac g a\implies T = \frac{2 \pi}{\omega} = 2 \pi\sqrt{\frac{\rho}{\rho_0} \cdot \frac a g}.
    \end{align*}
}
\solutionspace{150pt}

\tasknumber{10}%
\task{%
    Математический маятник с нитью длиной $37\,\text{см}$ подвешен к потолку в лифте.
    За $20\,\text{с}$ маятник совершил $17$ колебаний.
    Определите модуль и направление ускорения лифта.
    Куда движется лифт?
}
\answer{%
    $
        T = 2\pi\sqrt{\frac\ell {a + g}}, T = \frac {t}{N}
        \implies a + g = \ell \cdot \frac{4 \pi ^ 2}{T^2},
        a = \ell \cdot \frac{4 \pi ^ 2}{T^2} - g = \ell \cdot \frac{4 \pi ^ 2 N^2}{t^2} - g \approx 0{,}55\,\frac{\text{м}}{\text{c}^{2}},
        \text{вниз}.
    $
}
\solutionspace{120pt}

\tasknumber{11}%
\task{%
    Масса груза в пружинном маятнике равна $500\,\text{г}$, при этом период его колебаний равен $1{,}2\,\text{с}$.
    Груз утяжеляют на $50\,\text{г}$.
    Определите новый период колебаний маятника.
}
\answer{%
    $
        T'
            = 2\pi\sqrt{\frac{M + m}{k}}
            = 2\pi\sqrt{\frac{M}{k} \cdot \frac{M + m}{M}}
            = T\sqrt{\frac{M + m}{M}} =  T\sqrt{1 + \frac{m}{M}} \approx 1{,}26\,\text{с}.
    $
}
\solutionspace{120pt}

\tasknumber{12}%
\task{%
    При какой длине нити математического маятника период колебаний груза массой $400\,\text{г}$
    окажется равен периоду колебаний этого же груза в пружинном маятнике с пружиной жёсткостью $50\,\frac{\text{Н}}{\text{м}}$?
}
\answer{%
    $
        2\pi \sqrt{\frac \ell g} = 2\pi \sqrt{\frac m k}
        \implies \frac \ell g = \frac m k
        \implies \ell = g \frac m k \approx 8{,}0\,\text{см}.
    $
}
\solutionspace{120pt}

\tasknumber{13}%
\task{%
    Груз подвесили к пружине, при этом удлинение пружины составило $30\,\text{мм}$.
    Определите частоту колебаний пружинного маятника, собранного из этой пружины и этого груза.
}
\answer{%
    $
        mg -k\Delta x = 0 \implies \frac m k = \frac{\Delta x} g
        \implies T = 2\pi \sqrt{\frac m k } = 2\pi \sqrt{\frac{\Delta x} g } \approx $0{,}34\,\text{с}$,
        \nu = \frac 1T \approx $2{,}91\,\text{Гц}$.
    $
}
\solutionspace{120pt}

\tasknumber{14}%
\task{%
    Определите период колебаний системы: математический маятник ограничен с одной стороны стенкой (см.
    рис.
    на доске).
    Удары маятника о стенку абсолютно упругие, $n = \frac{2}{\sqrt{3}}$.
    Длина маятника $\ell$, ускорение свободного падения $g$.
}
\answer{%
    $
        T = 2\pi\sqrt{\frac\ell g}, \qquad
        T' = 2 \cdot \frac T 4 + 2 \cdot \frac T{6} = \frac56T = \frac53\pi\sqrt{\frac\ell g}
    $
}

\variantsplitter

\addpersonalvariant{Владислав Емелин}

\tasknumber{1}%
\task{%
    \begin{itemize}
        \item Запишите линейное однородное дифференциальное уравнение второго порядка,
            описывающее свободные незатухающие колебания гармонического осциллятора,
        \item запишите общее решение этого уравнения,
        \item подпишите в выписанном решении фазу и амплитуду колебаний,
        \item запишите выражение для скорости,
        \item запишите выражение для ускорения.
    \end{itemize}
}
\answer{%
    \begin{align*}
    &\ddot x + \omega^2 x = 0 \Longleftrightarrow a_x + \omega^2 x = 0, \\
    &x = A \cos(\omega t + \varphi_0) \text{ или же } x = A \sin(\omega t + \varphi_0) \text{ или же } x = a \cos(\omega t) + b \sin(\omega t), \\
    &A \text{\, или \,} \sqrt{a^2 + b^2} \text{ --- это амплитуда}, \omega t + \varphi_0\text{ --- это фаза}, \\
    &v = \dot x = -\omega A \sin(\omega t + \varphi_0), \\
    &a = \dot v = \ddot x = -\omega^2 A \cos(\omega t + \varphi_0) = -\omega^2 x,
    \end{align*}
}
\solutionspace{135pt}

\tasknumber{2}%
\task{%
    Тело колеблется по гармоническому закону,
    амплитуда этих колебаний $18\,\text{см}$, период $2\,\text{c}$.
    Чему равно смещение тела относительно положения равновесия через $21\,\text{c}$
    после прохождения положения максимального отклонения?
}
\answer{%
    $x = A \cos \omega t = A \cos \cbr{ \frac {2\pi}T t } = A \cos \cbr{ 2\pi \frac tT } = 18\,\text{см} \cdot \cos \cbr{ 2\pi \cdot \frac {21\,\text{c}}{2\,\text{c}}}\approx -18{,}000\,\text{см}.$
}
\solutionspace{120pt}

\tasknumber{3}%
\task{%
    Тело совершает гармонические колебания с периодом $6\,\text{c}$.
    За какое время тело смещается от положения наибольшего отклонения до смещения в половину амплитуды?
}
\answer{%
    $t = \frac T{6} \approx 1{,}0\,\text{c}.$
}
\solutionspace{120pt}

\tasknumber{4}%
\task{%
    Запишите формулу для периода колебаний пружинного маятника и ...
    \begin{itemize}
        \item укажите названия всех физических величин в формуле,
        \item выразите из неё циклическую частоту колебаний
        \item выразите из неё жёсткость пружины.
    \end{itemize}
}
\answer{%
    \begin{align*}
    T &= 2\pi \sqrt{\frac lg} \implies \nu = \frac 1T = \frac 1{2\pi}\sqrt{\frac gl}, \omega = 2\pi\nu = \sqrt{\frac gl}, l = g\sqr{\frac T{2\pi}}, g = l\sqr{\frac {2\pi}T} \\
    T &= 2\pi \sqrt{\frac mk} \implies \nu = \frac 1T = \frac 1{2\pi}\sqrt{\frac km}, \omega = 2\pi\nu = \sqrt{\frac km}, m = k\sqr{\frac T{2\pi}}, k = m\sqr{\frac {2\pi}T}
    \end{align*}
}
\solutionspace{120pt}

\tasknumber{5}%
\task{%
    Частота колебаний пружинного маятника равна $8\,\text{Гц}$.
    Определите периоды колебаний
    \begin{itemize}
        \item кинетической энергии системы,
        \item скорости груза,
        \item модуля скорости груза.
    \end{itemize}
}
\answer{%
    $T = \frac 1\nu \approx 125{,}0\,\text{мc}, T_1 = \frac T2 \approx 62{,}5\,\text{мc}, T_2 = T \approx 125{,}0\,\text{мc}, T_3 = \frac T2 \approx 62{,}5\,\text{мc}.$
}
\solutionspace{80pt}

\tasknumber{6}%
\task{%
    Тело колеблется по гармоническому закону с амплитудой $6\,\text{см}$.
    Какой максимальный путь тело может пройти за одну шестую долю периода?
}
\answer{%
    $2 * A \frac{1}{2} = A \approx 6{,}0\,\text{см}.$
}
\solutionspace{80pt}

\tasknumber{7}%
\task{%
    Период колебаний математического маятника равен $5\,\text{с}$,
    а их амплитуда — $20\,\text{см}$.
    Определите амплитуду колебаний скорости маятника.
}
\answer{%
    $
        T = \frac{2\pi}{\omega}
        \implies \omega = \frac{2\pi}{T}
        \implies v_{\max} = \omega A = \frac{2\pi}{T}A
        \approx 25{,}1\,\frac{\text{см}}{\text{с}}.
    $
}
\solutionspace{100pt}

\tasknumber{8}%
\task{%
    Определите период колебаний груза массой $m$, подвешенного к пружине жёсткостью $k$.
    Ускорение свободного падения $g$.
}
\answer{%
    \begin{align*}
    &-kx_0 + mg = 0, \\
    F &= -k(x_0 + \Delta x), \\
    ma &= -k(x_0 + \Delta x) + mg, \\
    ma &= -kx_0 -k \Delta x + mg = -mg -k \Delta x + mg = -k \Delta x, \\
    a &+ \frac k m x = 0, \\
    \omega^2 &= \frac k m \implies T = \frac{2\pi}\omega = 2\pi\sqrt{\frac m k}.
    \end{align*}
}
\solutionspace{150pt}

\tasknumber{9}%
\task{%
    Куб со стороной $a$ и плотности $\rho$ плавает в жидкости плотностью $\rho_0$.
    Определите частоту колебаний куба, считая, что 4 грани куба всегда вертикальны.
}
\answer{%
    \begin{align*}
    \Delta F &= -\rho_0 g a^2 \Delta x,\Delta F = m\ddot x, m = \rho a^3\implies \rho a^3 \ddot x = -\rho_0 g a^2 \Delta x \implies \\
    \implies \ddot x &= -\frac{\rho_0}{\rho} \frac g a \Delta x\implies \omega^2 = \frac{\rho_0}{\rho} \cdot \frac g a\implies T = \frac{2 \pi}{\omega} = 2 \pi\sqrt{\frac{\rho}{\rho_0} \cdot \frac a g}.
    \end{align*}
}
\solutionspace{150pt}

\tasknumber{10}%
\task{%
    Математический маятник с нитью длиной $37\,\text{см}$ подвешен к потолку в лифте.
    За $20\,\text{с}$ маятник совершил $19$ колебаний.
    Определите модуль и направление ускорения лифта.
    Куда движется лифт?
}
\answer{%
    $
        T = 2\pi\sqrt{\frac\ell {a + g}}, T = \frac {t}{N}
        \implies a + g = \ell \cdot \frac{4 \pi ^ 2}{T^2},
        a = \ell \cdot \frac{4 \pi ^ 2}{T^2} - g = \ell \cdot \frac{4 \pi ^ 2 N^2}{t^2} - g \approx 3{,}18\,\frac{\text{м}}{\text{c}^{2}},
        \text{вниз}.
    $
}
\solutionspace{120pt}

\tasknumber{11}%
\task{%
    Масса груза в пружинном маятнике равна $600\,\text{г}$, при этом период его колебаний равен $1{,}2\,\text{с}$.
    Груз утяжеляют на $50\,\text{г}$.
    Определите новый период колебаний маятника.
}
\answer{%
    $
        T'
            = 2\pi\sqrt{\frac{M + m}{k}}
            = 2\pi\sqrt{\frac{M}{k} \cdot \frac{M + m}{M}}
            = T\sqrt{\frac{M + m}{M}} =  T\sqrt{1 + \frac{m}{M}} \approx 1{,}25\,\text{с}.
    $
}
\solutionspace{120pt}

\tasknumber{12}%
\task{%
    При какой длине нити математического маятника период колебаний груза массой $400\,\text{г}$
    окажется равен периоду колебаний этого же груза в пружинном маятнике с пружиной жёсткостью $40\,\frac{\text{Н}}{\text{м}}$?
}
\answer{%
    $
        2\pi \sqrt{\frac \ell g} = 2\pi \sqrt{\frac m k}
        \implies \frac \ell g = \frac m k
        \implies \ell = g \frac m k \approx 10{,}0\,\text{см}.
    $
}
\solutionspace{120pt}

\tasknumber{13}%
\task{%
    Груз подвесили к пружине, при этом удлинение пружины составило $60\,\text{мм}$.
    Определите частоту колебаний пружинного маятника, собранного из этой пружины и этого груза.
}
\answer{%
    $
        mg -k\Delta x = 0 \implies \frac m k = \frac{\Delta x} g
        \implies T = 2\pi \sqrt{\frac m k } = 2\pi \sqrt{\frac{\Delta x} g } \approx $0{,}49\,\text{с}$,
        \nu = \frac 1T \approx $2{,}05\,\text{Гц}$.
    $
}
\solutionspace{120pt}

\tasknumber{14}%
\task{%
    Определите период колебаний системы: математический маятник ограничен с одной стороны стенкой (см.
    рис.
    на доске).
    Удары маятника о стенку абсолютно упругие, $n = \frac{2}{\sqrt{3}}$.
    Длина маятника $\ell$, ускорение свободного падения $g$.
}
\answer{%
    $
        T = 2\pi\sqrt{\frac\ell g}, \qquad
        T' = 2 \cdot \frac T 4 + 2 \cdot \frac T{6} = \frac56T = \frac53\pi\sqrt{\frac\ell g}
    $
}

\variantsplitter

\addpersonalvariant{Артём Жичин}

\tasknumber{1}%
\task{%
    \begin{itemize}
        \item Запишите линейное однородное дифференциальное уравнение второго порядка,
            описывающее свободные незатухающие колебания гармонического осциллятора,
        \item запишите общее решение этого уравнения,
        \item подпишите в выписанном решении фазу и амплитуду колебаний,
        \item запишите выражение для скорости,
        \item запишите выражение для ускорения.
    \end{itemize}
}
\answer{%
    \begin{align*}
    &\ddot x + \omega^2 x = 0 \Longleftrightarrow a_x + \omega^2 x = 0, \\
    &x = A \cos(\omega t + \varphi_0) \text{ или же } x = A \sin(\omega t + \varphi_0) \text{ или же } x = a \cos(\omega t) + b \sin(\omega t), \\
    &A \text{\, или \,} \sqrt{a^2 + b^2} \text{ --- это амплитуда}, \omega t + \varphi_0\text{ --- это фаза}, \\
    &v = \dot x = -\omega A \sin(\omega t + \varphi_0), \\
    &a = \dot v = \ddot x = -\omega^2 A \cos(\omega t + \varphi_0) = -\omega^2 x,
    \end{align*}
}
\solutionspace{135pt}

\tasknumber{2}%
\task{%
    Тело колеблется по гармоническому закону,
    амплитуда этих колебаний $12\,\text{см}$, период $4\,\text{c}$.
    Чему равно смещение тела относительно положения равновесия через $20\,\text{c}$
    после прохождения положения максимального отклонения?
}
\answer{%
    $x = A \cos \omega t = A \cos \cbr{ \frac {2\pi}T t } = A \cos \cbr{ 2\pi \frac tT } = 12\,\text{см} \cdot \cos \cbr{ 2\pi \cdot \frac {20\,\text{c}}{4\,\text{c}}}\approx 12{,}0\,\text{см}.$
}
\solutionspace{120pt}

\tasknumber{3}%
\task{%
    Тело совершает гармонические колебания с периодом $4\,\text{c}$.
    За какое время тело смещается от положения наибольшего отклонения до смещения в половину амплитуды?
}
\answer{%
    $t = \frac T{6} \approx 0{,}7\,\text{c}.$
}
\solutionspace{120pt}

\tasknumber{4}%
\task{%
    Запишите формулу для периода колебаний математического маятника и ...
    \begin{itemize}
        \item укажите названия всех физических величин в формуле,
        \item выразите из неё циклическую частоту колебаний
        \item выразите из неё длину маятника.
    \end{itemize}
}
\answer{%
    \begin{align*}
    T &= 2\pi \sqrt{\frac lg} \implies \nu = \frac 1T = \frac 1{2\pi}\sqrt{\frac gl}, \omega = 2\pi\nu = \sqrt{\frac gl}, l = g\sqr{\frac T{2\pi}}, g = l\sqr{\frac {2\pi}T} \\
    T &= 2\pi \sqrt{\frac mk} \implies \nu = \frac 1T = \frac 1{2\pi}\sqrt{\frac km}, \omega = 2\pi\nu = \sqrt{\frac km}, m = k\sqr{\frac T{2\pi}}, k = m\sqr{\frac {2\pi}T}
    \end{align*}
}
\solutionspace{120pt}

\tasknumber{5}%
\task{%
    Частота колебаний пружинного маятника равна $10\,\text{Гц}$.
    Определите периоды колебаний
    \begin{itemize}
        \item потенциальной энергии системы,
        \item скорости груза,
        \item модуля скорости груза.
    \end{itemize}
}
\answer{%
    $T = \frac 1\nu \approx 100{,}0\,\text{мc}, T_1 = \frac T2 \approx 50{,}0\,\text{мc}, T_2 = T \approx 100{,}0\,\text{мc}, T_3 = \frac T2 \approx 50{,}0\,\text{мc}.$
}
\solutionspace{80pt}

\tasknumber{6}%
\task{%
    Тело колеблется по гармоническому закону с амплитудой $6\,\text{см}$.
    Какой минимальный путь тело может пройти за одну шестую долю периода?
}
\answer{%
    $2 * A \frac{1}{12} = \frac{A:L:s}3 \approx 1{,}0\,\text{см}.$
}
\solutionspace{80pt}

\tasknumber{7}%
\task{%
    Период колебаний математического маятника равен $3\,\text{с}$,
    а их амплитуда — $20\,\text{см}$.
    Определите максимальную скорость маятника.
}
\answer{%
    $
        T = \frac{2\pi}{\omega}
        \implies \omega = \frac{2\pi}{T}
        \implies v_{\max} = \omega A = \frac{2\pi}{T}A
        \approx 41{,}9\,\frac{\text{см}}{\text{с}}.
    $
}
\solutionspace{100pt}

\tasknumber{8}%
\task{%
    Определите период колебаний груза массой $m$, подвешенного к пружине жёсткостью $k$.
    Ускорение свободного падения $g$.
}
\answer{%
    \begin{align*}
    &-kx_0 + mg = 0, \\
    F &= -k(x_0 + \Delta x), \\
    ma &= -k(x_0 + \Delta x) + mg, \\
    ma &= -kx_0 -k \Delta x + mg = -mg -k \Delta x + mg = -k \Delta x, \\
    a &+ \frac k m x = 0, \\
    \omega^2 &= \frac k m \implies T = \frac{2\pi}\omega = 2\pi\sqrt{\frac m k}.
    \end{align*}
}
\solutionspace{150pt}

\tasknumber{9}%
\task{%
    Куб со стороной $a$ и плотности $\rho$ плавает в жидкости плотностью $\rho_0$.
    Определите частоту колебаний куба, считая, что 4 грани куба всегда вертикальны.
}
\answer{%
    \begin{align*}
    \Delta F &= -\rho_0 g a^2 \Delta x,\Delta F = m\ddot x, m = \rho a^3\implies \rho a^3 \ddot x = -\rho_0 g a^2 \Delta x \implies \\
    \implies \ddot x &= -\frac{\rho_0}{\rho} \frac g a \Delta x\implies \omega^2 = \frac{\rho_0}{\rho} \cdot \frac g a\implies T = \frac{2 \pi}{\omega} = 2 \pi\sqrt{\frac{\rho}{\rho_0} \cdot \frac a g}.
    \end{align*}
}
\solutionspace{150pt}

\tasknumber{10}%
\task{%
    Математический маятник с нитью длиной $43\,\text{см}$ подвешен к потолку в лифте.
    За $25\,\text{с}$ маятник совершил $18$ колебаний.
    Определите модуль и направление ускорения лифта.
    Куда движется лифт?
}
\answer{%
    $
        T = 2\pi\sqrt{\frac\ell {a + g}}, T = \frac {t}{N}
        \implies a + g = \ell \cdot \frac{4 \pi ^ 2}{T^2},
        a = \ell \cdot \frac{4 \pi ^ 2}{T^2} - g = \ell \cdot \frac{4 \pi ^ 2 N^2}{t^2} - g \approx -1{,}2000\,\frac{\text{м}}{\text{c}^{2}},
        \text{вверх}.
    $
}
\solutionspace{120pt}

\tasknumber{11}%
\task{%
    Масса груза в пружинном маятнике равна $400\,\text{г}$, при этом период его колебаний равен $1{,}5\,\text{с}$.
    Груз утяжеляют на $150\,\text{г}$.
    Определите новый период колебаний маятника.
}
\answer{%
    $
        T'
            = 2\pi\sqrt{\frac{M + m}{k}}
            = 2\pi\sqrt{\frac{M}{k} \cdot \frac{M + m}{M}}
            = T\sqrt{\frac{M + m}{M}} =  T\sqrt{1 + \frac{m}{M}} \approx 1{,}76\,\text{с}.
    $
}
\solutionspace{120pt}

\tasknumber{12}%
\task{%
    При какой длине нити математического маятника период колебаний груза массой $300\,\text{г}$
    окажется равен периоду колебаний этого же груза в пружинном маятнике с пружиной жёсткостью $40\,\frac{\text{Н}}{\text{м}}$?
}
\answer{%
    $
        2\pi \sqrt{\frac \ell g} = 2\pi \sqrt{\frac m k}
        \implies \frac \ell g = \frac m k
        \implies \ell = g \frac m k \approx 7{,}5\,\text{см}.
    $
}
\solutionspace{120pt}

\tasknumber{13}%
\task{%
    Груз подвесили к пружине, при этом удлинение пружины составило $45\,\text{мм}$.
    Определите частоту колебаний пружинного маятника, собранного из этой пружины и этого груза.
}
\answer{%
    $
        mg -k\Delta x = 0 \implies \frac m k = \frac{\Delta x} g
        \implies T = 2\pi \sqrt{\frac m k } = 2\pi \sqrt{\frac{\Delta x} g } \approx $0{,}42\,\text{с}$,
        \nu = \frac 1T \approx $2{,}37\,\text{Гц}$.
    $
}
\solutionspace{120pt}

\tasknumber{14}%
\task{%
    Определите период колебаний системы: математический маятник ограничен с одной стороны стенкой (см.
    рис.
    на доске).
    Удары маятника о стенку абсолютно упругие, $n = \frac{2}{\sqrt{3}}$.
    Длина маятника $\ell$, ускорение свободного падения $g$.
}
\answer{%
    $
        T = 2\pi\sqrt{\frac\ell g}, \qquad
        T' = 2 \cdot \frac T 4 + 2 \cdot \frac T{6} = \frac56T = \frac53\pi\sqrt{\frac\ell g}
    $
}

\variantsplitter

\addpersonalvariant{Дарья Кошман}

\tasknumber{1}%
\task{%
    \begin{itemize}
        \item Запишите линейное однородное дифференциальное уравнение второго порядка,
            описывающее свободные незатухающие колебания гармонического осциллятора,
        \item запишите общее решение этого уравнения,
        \item подпишите в выписанном решении фазу и амплитуду колебаний,
        \item запишите выражение для скорости,
        \item запишите выражение для ускорения.
    \end{itemize}
}
\answer{%
    \begin{align*}
    &\ddot x + \omega^2 x = 0 \Longleftrightarrow a_x + \omega^2 x = 0, \\
    &x = A \cos(\omega t + \varphi_0) \text{ или же } x = A \sin(\omega t + \varphi_0) \text{ или же } x = a \cos(\omega t) + b \sin(\omega t), \\
    &A \text{\, или \,} \sqrt{a^2 + b^2} \text{ --- это амплитуда}, \omega t + \varphi_0\text{ --- это фаза}, \\
    &v = \dot x = -\omega A \sin(\omega t + \varphi_0), \\
    &a = \dot v = \ddot x = -\omega^2 A \cos(\omega t + \varphi_0) = -\omega^2 x,
    \end{align*}
}
\solutionspace{135pt}

\tasknumber{2}%
\task{%
    Тело колеблется по гармоническому закону,
    амплитуда этих колебаний $18\,\text{см}$, период $4\,\text{c}$.
    Чему равно смещение тела относительно положения равновесия через $20\,\text{c}$
    после прохождения положения максимального отклонения?
}
\answer{%
    $x = A \cos \omega t = A \cos \cbr{ \frac {2\pi}T t } = A \cos \cbr{ 2\pi \frac tT } = 18\,\text{см} \cdot \cos \cbr{ 2\pi \cdot \frac {20\,\text{c}}{4\,\text{c}}}\approx 18{,}0\,\text{см}.$
}
\solutionspace{120pt}

\tasknumber{3}%
\task{%
    Тело совершает гармонические колебания с периодом $4\,\text{c}$.
    За какое время тело смещается от положения наибольшего отклонения до смещения в половину амплитуды?
}
\answer{%
    $t = \frac T{6} \approx 0{,}7\,\text{c}.$
}
\solutionspace{120pt}

\tasknumber{4}%
\task{%
    Запишите формулу для периода колебаний пружинного маятника и ...
    \begin{itemize}
        \item укажите названия всех физических величин в формуле,
        \item выразите из неё циклическую частоту колебаний
        \item выразите из неё жёсткость пружины.
    \end{itemize}
}
\answer{%
    \begin{align*}
    T &= 2\pi \sqrt{\frac lg} \implies \nu = \frac 1T = \frac 1{2\pi}\sqrt{\frac gl}, \omega = 2\pi\nu = \sqrt{\frac gl}, l = g\sqr{\frac T{2\pi}}, g = l\sqr{\frac {2\pi}T} \\
    T &= 2\pi \sqrt{\frac mk} \implies \nu = \frac 1T = \frac 1{2\pi}\sqrt{\frac km}, \omega = 2\pi\nu = \sqrt{\frac km}, m = k\sqr{\frac T{2\pi}}, k = m\sqr{\frac {2\pi}T}
    \end{align*}
}
\solutionspace{120pt}

\tasknumber{5}%
\task{%
    Частота колебаний пружинного маятника равна $12\,\text{Гц}$.
    Определите периоды колебаний
    \begin{itemize}
        \item кинетической энергии системы,
        \item скорости груза,
        \item модуля скорости груза.
    \end{itemize}
}
\answer{%
    $T = \frac 1\nu \approx 83{,}3\,\text{мc}, T_1 = \frac T2 \approx 41{,}7\,\text{мc}, T_2 = T \approx 83{,}3\,\text{мc}, T_3 = \frac T2 \approx 41{,}7\,\text{мc}.$
}
\solutionspace{80pt}

\tasknumber{6}%
\task{%
    Тело колеблется по гармоническому закону с амплитудой $4\,\text{см}$.
    Какой минимальный путь тело может пройти за четверть периода?
}
\answer{%
    $2 * A \cbr{1 - \frac 1{\sqrt 2}} = A\cbr{2 - \sqrt{2}} \approx 2{,}3\,\text{см}.$
}
\solutionspace{80pt}

\tasknumber{7}%
\task{%
    Период колебаний математического маятника равен $3\,\text{с}$,
    а их амплитуда — $15\,\text{см}$.
    Определите максимальную скорость маятника.
}
\answer{%
    $
        T = \frac{2\pi}{\omega}
        \implies \omega = \frac{2\pi}{T}
        \implies v_{\max} = \omega A = \frac{2\pi}{T}A
        \approx 31{,}4\,\frac{\text{см}}{\text{с}}.
    $
}
\solutionspace{100pt}

\tasknumber{8}%
\task{%
    Определите период колебаний груза массой $m$, подвешенного к пружине жёсткостью $k$.
    Ускорение свободного падения $g$.
}
\answer{%
    \begin{align*}
    &-kx_0 + mg = 0, \\
    F &= -k(x_0 + \Delta x), \\
    ma &= -k(x_0 + \Delta x) + mg, \\
    ma &= -kx_0 -k \Delta x + mg = -mg -k \Delta x + mg = -k \Delta x, \\
    a &+ \frac k m x = 0, \\
    \omega^2 &= \frac k m \implies T = \frac{2\pi}\omega = 2\pi\sqrt{\frac m k}.
    \end{align*}
}
\solutionspace{150pt}

\tasknumber{9}%
\task{%
    Куб со стороной $a$ и плотности $\rho$ плавает в жидкости плотностью $\rho_0$.
    Определите частоту колебаний куба, считая, что 4 грани куба всегда вертикальны.
}
\answer{%
    \begin{align*}
    \Delta F &= -\rho_0 g a^2 \Delta x,\Delta F = m\ddot x, m = \rho a^3\implies \rho a^3 \ddot x = -\rho_0 g a^2 \Delta x \implies \\
    \implies \ddot x &= -\frac{\rho_0}{\rho} \frac g a \Delta x\implies \omega^2 = \frac{\rho_0}{\rho} \cdot \frac g a\implies T = \frac{2 \pi}{\omega} = 2 \pi\sqrt{\frac{\rho}{\rho_0} \cdot \frac a g}.
    \end{align*}
}
\solutionspace{150pt}

\tasknumber{10}%
\task{%
    Математический маятник с нитью длиной $40\,\text{см}$ подвешен к потолку в лифте.
    За $30\,\text{с}$ маятник совершил $22$ колебаний.
    Определите модуль и направление ускорения лифта.
    Куда движется лифт?
}
\answer{%
    $
        T = 2\pi\sqrt{\frac\ell {a + g}}, T = \frac {t}{N}
        \implies a + g = \ell \cdot \frac{4 \pi ^ 2}{T^2},
        a = \ell \cdot \frac{4 \pi ^ 2}{T^2} - g = \ell \cdot \frac{4 \pi ^ 2 N^2}{t^2} - g \approx -1{,}5100\,\frac{\text{м}}{\text{c}^{2}},
        \text{вверх}.
    $
}
\solutionspace{120pt}

\tasknumber{11}%
\task{%
    Масса груза в пружинном маятнике равна $600\,\text{г}$, при этом период его колебаний равен $1{,}4\,\text{с}$.
    Груз облегчают на $50\,\text{г}$.
    Определите новый период колебаний маятника.
}
\answer{%
    $
        T'
            = 2\pi\sqrt{\frac{M - m}{k}}
            = 2\pi\sqrt{\frac{M}{k} \cdot \frac{M - m}{M}}
            = T\sqrt{\frac{M - m}{M}} =  T\sqrt{1 - \frac{m}{M}} \approx 1{,}34\,\text{с}.
    $
}
\solutionspace{120pt}

\tasknumber{12}%
\task{%
    При какой длине нити математического маятника период колебаний груза массой $300\,\text{г}$
    окажется равен периоду колебаний этого же груза в пружинном маятнике с пружиной жёсткостью $40\,\frac{\text{Н}}{\text{м}}$?
}
\answer{%
    $
        2\pi \sqrt{\frac \ell g} = 2\pi \sqrt{\frac m k}
        \implies \frac \ell g = \frac m k
        \implies \ell = g \frac m k \approx 7{,}5\,\text{см}.
    $
}
\solutionspace{120pt}

\tasknumber{13}%
\task{%
    Груз подвесили к пружине, при этом удлинение пружины составило $30\,\text{мм}$.
    Определите частоту колебаний пружинного маятника, собранного из этой пружины и этого груза.
}
\answer{%
    $
        mg -k\Delta x = 0 \implies \frac m k = \frac{\Delta x} g
        \implies T = 2\pi \sqrt{\frac m k } = 2\pi \sqrt{\frac{\Delta x} g } \approx $0{,}34\,\text{с}$,
        \nu = \frac 1T \approx $2{,}91\,\text{Гц}$.
    $
}
\solutionspace{120pt}

\tasknumber{14}%
\task{%
    Определите период колебаний системы: математический маятник ограничен с одной стороны стенкой (см.
    рис.
    на доске).
    Удары маятника о стенку абсолютно упругие, $n = \frac{2}{\sqrt{3}}$.
    Длина маятника $\ell$, ускорение свободного падения $g$.
}
\answer{%
    $
        T = 2\pi\sqrt{\frac\ell g}, \qquad
        T' = 2 \cdot \frac T 4 + 2 \cdot \frac T{6} = \frac56T = \frac53\pi\sqrt{\frac\ell g}
    $
}

\variantsplitter

\addpersonalvariant{Анна Кузьмичёва}

\tasknumber{1}%
\task{%
    \begin{itemize}
        \item Запишите линейное однородное дифференциальное уравнение второго порядка,
            описывающее свободные незатухающие колебания гармонического осциллятора,
        \item запишите общее решение этого уравнения,
        \item подпишите в выписанном решении фазу и амплитуду колебаний,
        \item запишите выражение для скорости,
        \item запишите выражение для ускорения.
    \end{itemize}
}
\answer{%
    \begin{align*}
    &\ddot x + \omega^2 x = 0 \Longleftrightarrow a_x + \omega^2 x = 0, \\
    &x = A \cos(\omega t + \varphi_0) \text{ или же } x = A \sin(\omega t + \varphi_0) \text{ или же } x = a \cos(\omega t) + b \sin(\omega t), \\
    &A \text{\, или \,} \sqrt{a^2 + b^2} \text{ --- это амплитуда}, \omega t + \varphi_0\text{ --- это фаза}, \\
    &v = \dot x = -\omega A \sin(\omega t + \varphi_0), \\
    &a = \dot v = \ddot x = -\omega^2 A \cos(\omega t + \varphi_0) = -\omega^2 x,
    \end{align*}
}
\solutionspace{135pt}

\tasknumber{2}%
\task{%
    Тело колеблется по гармоническому закону,
    амплитуда этих колебаний $14\,\text{см}$, период $2\,\text{c}$.
    Чему равно смещение тела относительно положения равновесия через $23\,\text{c}$
    после прохождения положения максимального отклонения?
}
\answer{%
    $x = A \cos \omega t = A \cos \cbr{ \frac {2\pi}T t } = A \cos \cbr{ 2\pi \frac tT } = 14\,\text{см} \cdot \cos \cbr{ 2\pi \cdot \frac {23\,\text{c}}{2\,\text{c}}}\approx -14{,}000\,\text{см}.$
}
\solutionspace{120pt}

\tasknumber{3}%
\task{%
    Тело совершает гармонические колебания с периодом $4\,\text{c}$.
    За какое время тело смещается от положения равновесия до смещения в половину амплитуды?
}
\answer{%
    $t = \frac T{12} \approx 0{,}3\,\text{c}.$
}
\solutionspace{120pt}

\tasknumber{4}%
\task{%
    Запишите формулу для периода колебаний пружинного маятника и ...
    \begin{itemize}
        \item укажите названия всех физических величин в формуле,
        \item выразите из неё частоту колебаний
        \item выразите из неё жёсткость пружины.
    \end{itemize}
}
\answer{%
    \begin{align*}
    T &= 2\pi \sqrt{\frac lg} \implies \nu = \frac 1T = \frac 1{2\pi}\sqrt{\frac gl}, \omega = 2\pi\nu = \sqrt{\frac gl}, l = g\sqr{\frac T{2\pi}}, g = l\sqr{\frac {2\pi}T} \\
    T &= 2\pi \sqrt{\frac mk} \implies \nu = \frac 1T = \frac 1{2\pi}\sqrt{\frac km}, \omega = 2\pi\nu = \sqrt{\frac km}, m = k\sqr{\frac T{2\pi}}, k = m\sqr{\frac {2\pi}T}
    \end{align*}
}
\solutionspace{120pt}

\tasknumber{5}%
\task{%
    Частота колебаний математического маятника равна $10\,\text{Гц}$.
    Определите периоды колебаний
    \begin{itemize}
        \item кинетической энергии системы,
        \item ускорения груза,
        \item модуля скорости груза.
    \end{itemize}
}
\answer{%
    $T = \frac 1\nu \approx 100{,}0\,\text{мc}, T_1 = \frac T2 \approx 50{,}0\,\text{мc}, T_2 = T \approx 100{,}0\,\text{мc}, T_3 = \frac T2 \approx 50{,}0\,\text{мc}.$
}
\solutionspace{80pt}

\tasknumber{6}%
\task{%
    Тело колеблется по гармоническому закону с амплитудой $12\,\text{см}$.
    Какой минимальный путь тело может пройти за одну шестую долю периода?
}
\answer{%
    $2 * A \frac{1}{12} = \frac{A:L:s}3 \approx 2{,}0\,\text{см}.$
}
\solutionspace{80pt}

\tasknumber{7}%
\task{%
    Период колебаний математического маятника равен $3\,\text{с}$,
    а их амплитуда — $10\,\text{см}$.
    Определите максимальную скорость маятника.
}
\answer{%
    $
        T = \frac{2\pi}{\omega}
        \implies \omega = \frac{2\pi}{T}
        \implies v_{\max} = \omega A = \frac{2\pi}{T}A
        \approx 20{,}9\,\frac{\text{см}}{\text{с}}.
    $
}
\solutionspace{100pt}

\tasknumber{8}%
\task{%
    Определите период колебаний груза массой $m$, подвешенного к пружине жёсткостью $k$.
    Ускорение свободного падения $g$.
}
\answer{%
    \begin{align*}
    &-kx_0 + mg = 0, \\
    F &= -k(x_0 + \Delta x), \\
    ma &= -k(x_0 + \Delta x) + mg, \\
    ma &= -kx_0 -k \Delta x + mg = -mg -k \Delta x + mg = -k \Delta x, \\
    a &+ \frac k m x = 0, \\
    \omega^2 &= \frac k m \implies T = \frac{2\pi}\omega = 2\pi\sqrt{\frac m k}.
    \end{align*}
}
\solutionspace{150pt}

\tasknumber{9}%
\task{%
    Куб со стороной $a$ и плотности $\rho$ плавает в жидкости плотностью $\rho_0$.
    Определите частоту колебаний куба, считая, что 4 грани куба всегда вертикальны.
}
\answer{%
    \begin{align*}
    \Delta F &= -\rho_0 g a^2 \Delta x,\Delta F = m\ddot x, m = \rho a^3\implies \rho a^3 \ddot x = -\rho_0 g a^2 \Delta x \implies \\
    \implies \ddot x &= -\frac{\rho_0}{\rho} \frac g a \Delta x\implies \omega^2 = \frac{\rho_0}{\rho} \cdot \frac g a\implies T = \frac{2 \pi}{\omega} = 2 \pi\sqrt{\frac{\rho}{\rho_0} \cdot \frac a g}.
    \end{align*}
}
\solutionspace{150pt}

\tasknumber{10}%
\task{%
    Математический маятник с нитью длиной $37\,\text{см}$ подвешен к потолку в лифте.
    За $25\,\text{с}$ маятник совершил $19$ колебаний.
    Определите модуль и направление ускорения лифта.
    Куда движется лифт?
}
\answer{%
    $
        T = 2\pi\sqrt{\frac\ell {a + g}}, T = \frac {t}{N}
        \implies a + g = \ell \cdot \frac{4 \pi ^ 2}{T^2},
        a = \ell \cdot \frac{4 \pi ^ 2}{T^2} - g = \ell \cdot \frac{4 \pi ^ 2 N^2}{t^2} - g \approx -1{,}5600\,\frac{\text{м}}{\text{c}^{2}},
        \text{вверх}.
    $
}
\solutionspace{120pt}

\tasknumber{11}%
\task{%
    Масса груза в пружинном маятнике равна $400\,\text{г}$, при этом период его колебаний равен $1{,}4\,\text{с}$.
    Груз утяжеляют на $50\,\text{г}$.
    Определите новый период колебаний маятника.
}
\answer{%
    $
        T'
            = 2\pi\sqrt{\frac{M + m}{k}}
            = 2\pi\sqrt{\frac{M}{k} \cdot \frac{M + m}{M}}
            = T\sqrt{\frac{M + m}{M}} =  T\sqrt{1 + \frac{m}{M}} \approx 1{,}48\,\text{с}.
    $
}
\solutionspace{120pt}

\tasknumber{12}%
\task{%
    При какой длине нити математического маятника период колебаний груза массой $200\,\text{г}$
    окажется равен периоду колебаний этого же груза в пружинном маятнике с пружиной жёсткостью $40\,\frac{\text{Н}}{\text{м}}$?
}
\answer{%
    $
        2\pi \sqrt{\frac \ell g} = 2\pi \sqrt{\frac m k}
        \implies \frac \ell g = \frac m k
        \implies \ell = g \frac m k \approx 5{,}0\,\text{см}.
    $
}
\solutionspace{120pt}

\tasknumber{13}%
\task{%
    Груз подвесили к пружине, при этом удлинение пружины составило $45\,\text{мм}$.
    Определите частоту колебаний пружинного маятника, собранного из этой пружины и этого груза.
}
\answer{%
    $
        mg -k\Delta x = 0 \implies \frac m k = \frac{\Delta x} g
        \implies T = 2\pi \sqrt{\frac m k } = 2\pi \sqrt{\frac{\Delta x} g } \approx $0{,}42\,\text{с}$,
        \nu = \frac 1T \approx $2{,}37\,\text{Гц}$.
    $
}
\solutionspace{120pt}

\tasknumber{14}%
\task{%
    Определите период колебаний системы: математический маятник ограничен с одной стороны стенкой (см.
    рис.
    на доске).
    Удары маятника о стенку абсолютно упругие, $n = 2$.
    Длина маятника $\ell$, ускорение свободного падения $g$.
}
\answer{%
    $
        T = 2\pi\sqrt{\frac\ell g}, \qquad
        T' = 2 \cdot \frac T 4 + 2 \cdot \frac T{12} = \frac23T = \frac43\pi\sqrt{\frac\ell g}
    $
}

\variantsplitter

\addpersonalvariant{Алёна Куприянова}

\tasknumber{1}%
\task{%
    \begin{itemize}
        \item Запишите линейное однородное дифференциальное уравнение второго порядка,
            описывающее свободные незатухающие колебания гармонического осциллятора,
        \item запишите общее решение этого уравнения,
        \item подпишите в выписанном решении фазу и амплитуду колебаний,
        \item запишите выражение для скорости,
        \item запишите выражение для ускорения.
    \end{itemize}
}
\answer{%
    \begin{align*}
    &\ddot x + \omega^2 x = 0 \Longleftrightarrow a_x + \omega^2 x = 0, \\
    &x = A \cos(\omega t + \varphi_0) \text{ или же } x = A \sin(\omega t + \varphi_0) \text{ или же } x = a \cos(\omega t) + b \sin(\omega t), \\
    &A \text{\, или \,} \sqrt{a^2 + b^2} \text{ --- это амплитуда}, \omega t + \varphi_0\text{ --- это фаза}, \\
    &v = \dot x = -\omega A \sin(\omega t + \varphi_0), \\
    &a = \dot v = \ddot x = -\omega^2 A \cos(\omega t + \varphi_0) = -\omega^2 x,
    \end{align*}
}
\solutionspace{135pt}

\tasknumber{2}%
\task{%
    Тело колеблется по гармоническому закону,
    амплитуда этих колебаний $10\,\text{см}$, период $2\,\text{c}$.
    Чему равно смещение тела относительно положения равновесия через $26\,\text{c}$
    после прохождения положения максимального отклонения?
}
\answer{%
    $x = A \cos \omega t = A \cos \cbr{ \frac {2\pi}T t } = A \cos \cbr{ 2\pi \frac tT } = 10\,\text{см} \cdot \cos \cbr{ 2\pi \cdot \frac {26\,\text{c}}{2\,\text{c}}}\approx 10{,}0\,\text{см}.$
}
\solutionspace{120pt}

\tasknumber{3}%
\task{%
    Тело совершает гармонические колебания с периодом $5\,\text{c}$.
    За какое время тело смещается от положения равновесия до смещения в половину амплитуды?
}
\answer{%
    $t = \frac T{12} \approx 0{,}4\,\text{c}.$
}
\solutionspace{120pt}

\tasknumber{4}%
\task{%
    Запишите формулу для периода колебаний математического маятника и ...
    \begin{itemize}
        \item укажите названия всех физических величин в формуле,
        \item выразите из неё циклическую частоту колебаний
        \item выразите из неё ускорение свободного падения.
    \end{itemize}
}
\answer{%
    \begin{align*}
    T &= 2\pi \sqrt{\frac lg} \implies \nu = \frac 1T = \frac 1{2\pi}\sqrt{\frac gl}, \omega = 2\pi\nu = \sqrt{\frac gl}, l = g\sqr{\frac T{2\pi}}, g = l\sqr{\frac {2\pi}T} \\
    T &= 2\pi \sqrt{\frac mk} \implies \nu = \frac 1T = \frac 1{2\pi}\sqrt{\frac km}, \omega = 2\pi\nu = \sqrt{\frac km}, m = k\sqr{\frac T{2\pi}}, k = m\sqr{\frac {2\pi}T}
    \end{align*}
}
\solutionspace{120pt}

\tasknumber{5}%
\task{%
    Частота колебаний пружинного маятника равна $15\,\text{Гц}$.
    Определите периоды колебаний
    \begin{itemize}
        \item кинетической энергии системы,
        \item ускорения груза,
        \item модуля скорости груза.
    \end{itemize}
}
\answer{%
    $T = \frac 1\nu \approx 66{,}7\,\text{мc}, T_1 = \frac T2 \approx 33{,}3\,\text{мc}, T_2 = T \approx 66{,}7\,\text{мc}, T_3 = \frac T2 \approx 33{,}3\,\text{мc}.$
}
\solutionspace{80pt}

\tasknumber{6}%
\task{%
    Тело колеблется по гармоническому закону с амплитудой $12\,\text{см}$.
    Какой максимальный путь тело может пройти за одну шестую долю периода?
}
\answer{%
    $2 * A \frac{1}{2} = A \approx 12{,}0\,\text{см}.$
}
\solutionspace{80pt}

\tasknumber{7}%
\task{%
    Период колебаний математического маятника равен $3\,\text{с}$,
    а их амплитуда — $20\,\text{см}$.
    Определите максимальную скорость маятника.
}
\answer{%
    $
        T = \frac{2\pi}{\omega}
        \implies \omega = \frac{2\pi}{T}
        \implies v_{\max} = \omega A = \frac{2\pi}{T}A
        \approx 41{,}9\,\frac{\text{см}}{\text{с}}.
    $
}
\solutionspace{100pt}

\tasknumber{8}%
\task{%
    Определите период колебаний груза массой $m$, подвешенного к пружине жёсткостью $k$.
    Ускорение свободного падения $g$.
}
\answer{%
    \begin{align*}
    &-kx_0 + mg = 0, \\
    F &= -k(x_0 + \Delta x), \\
    ma &= -k(x_0 + \Delta x) + mg, \\
    ma &= -kx_0 -k \Delta x + mg = -mg -k \Delta x + mg = -k \Delta x, \\
    a &+ \frac k m x = 0, \\
    \omega^2 &= \frac k m \implies T = \frac{2\pi}\omega = 2\pi\sqrt{\frac m k}.
    \end{align*}
}
\solutionspace{150pt}

\tasknumber{9}%
\task{%
    Куб со стороной $a$ и плотности $\rho$ плавает в жидкости плотностью $\rho_0$.
    Определите частоту колебаний куба, считая, что 4 грани куба всегда вертикальны.
}
\answer{%
    \begin{align*}
    \Delta F &= -\rho_0 g a^2 \Delta x,\Delta F = m\ddot x, m = \rho a^3\implies \rho a^3 \ddot x = -\rho_0 g a^2 \Delta x \implies \\
    \implies \ddot x &= -\frac{\rho_0}{\rho} \frac g a \Delta x\implies \omega^2 = \frac{\rho_0}{\rho} \cdot \frac g a\implies T = \frac{2 \pi}{\omega} = 2 \pi\sqrt{\frac{\rho}{\rho_0} \cdot \frac a g}.
    \end{align*}
}
\solutionspace{150pt}

\tasknumber{10}%
\task{%
    Математический маятник с нитью длиной $40\,\text{см}$ подвешен к потолку в лифте.
    За $20\,\text{с}$ маятник совершил $16$ колебаний.
    Определите модуль и направление ускорения лифта.
    Куда движется лифт?
}
\answer{%
    $
        T = 2\pi\sqrt{\frac\ell {a + g}}, T = \frac {t}{N}
        \implies a + g = \ell \cdot \frac{4 \pi ^ 2}{T^2},
        a = \ell \cdot \frac{4 \pi ^ 2}{T^2} - g = \ell \cdot \frac{4 \pi ^ 2 N^2}{t^2} - g \approx 0{,}11\,\frac{\text{м}}{\text{c}^{2}},
        \text{вниз}.
    $
}
\solutionspace{120pt}

\tasknumber{11}%
\task{%
    Масса груза в пружинном маятнике равна $500\,\text{г}$, при этом период его колебаний равен $1{,}2\,\text{с}$.
    Груз облегчают на $150\,\text{г}$.
    Определите новый период колебаний маятника.
}
\answer{%
    $
        T'
            = 2\pi\sqrt{\frac{M - m}{k}}
            = 2\pi\sqrt{\frac{M}{k} \cdot \frac{M - m}{M}}
            = T\sqrt{\frac{M - m}{M}} =  T\sqrt{1 - \frac{m}{M}} \approx 1{,}00\,\text{с}.
    $
}
\solutionspace{120pt}

\tasknumber{12}%
\task{%
    При какой длине нити математического маятника период колебаний груза массой $400\,\text{г}$
    окажется равен периоду колебаний этого же груза в пружинном маятнике с пружиной жёсткостью $60\,\frac{\text{Н}}{\text{м}}$?
}
\answer{%
    $
        2\pi \sqrt{\frac \ell g} = 2\pi \sqrt{\frac m k}
        \implies \frac \ell g = \frac m k
        \implies \ell = g \frac m k \approx 6{,}7\,\text{см}.
    $
}
\solutionspace{120pt}

\tasknumber{13}%
\task{%
    Груз подвесили к пружине, при этом удлинение пружины составило $45\,\text{мм}$.
    Определите частоту колебаний пружинного маятника, собранного из этой пружины и этого груза.
}
\answer{%
    $
        mg -k\Delta x = 0 \implies \frac m k = \frac{\Delta x} g
        \implies T = 2\pi \sqrt{\frac m k } = 2\pi \sqrt{\frac{\Delta x} g } \approx $0{,}42\,\text{с}$,
        \nu = \frac 1T \approx $2{,}37\,\text{Гц}$.
    $
}
\solutionspace{120pt}

\tasknumber{14}%
\task{%
    Определите период колебаний системы: математический маятник ограничен с одной стороны стенкой (см.
    рис.
    на доске).
    Удары маятника о стенку абсолютно упругие, $n = \sqrt{2}$.
    Длина маятника $\ell$, ускорение свободного падения $g$.
}
\answer{%
    $
        T = 2\pi\sqrt{\frac\ell g}, \qquad
        T' = 2 \cdot \frac T 4 + 2 \cdot \frac T{8} = \frac34T = \frac32\pi\sqrt{\frac\ell g}
    $
}

\variantsplitter

\addpersonalvariant{Ярослав Лавровский}

\tasknumber{1}%
\task{%
    \begin{itemize}
        \item Запишите линейное однородное дифференциальное уравнение второго порядка,
            описывающее свободные незатухающие колебания гармонического осциллятора,
        \item запишите общее решение этого уравнения,
        \item подпишите в выписанном решении фазу и амплитуду колебаний,
        \item запишите выражение для скорости,
        \item запишите выражение для ускорения.
    \end{itemize}
}
\answer{%
    \begin{align*}
    &\ddot x + \omega^2 x = 0 \Longleftrightarrow a_x + \omega^2 x = 0, \\
    &x = A \cos(\omega t + \varphi_0) \text{ или же } x = A \sin(\omega t + \varphi_0) \text{ или же } x = a \cos(\omega t) + b \sin(\omega t), \\
    &A \text{\, или \,} \sqrt{a^2 + b^2} \text{ --- это амплитуда}, \omega t + \varphi_0\text{ --- это фаза}, \\
    &v = \dot x = -\omega A \sin(\omega t + \varphi_0), \\
    &a = \dot v = \ddot x = -\omega^2 A \cos(\omega t + \varphi_0) = -\omega^2 x,
    \end{align*}
}
\solutionspace{135pt}

\tasknumber{2}%
\task{%
    Тело колеблется по гармоническому закону,
    амплитуда этих колебаний $18\,\text{см}$, период $4\,\text{c}$.
    Чему равно смещение тела относительно положения равновесия через $24\,\text{c}$
    после прохождения положения равновесия?
}
\answer{%
    $x = A \sin \omega t = A \sin \cbr{ \frac {2\pi}T t } = A \sin \cbr{ 2\pi \frac tT } = 18\,\text{см} \cdot \sin \cbr{ 2\pi \cdot \frac {24\,\text{c}}{4\,\text{c}}}\approx 0\,\text{см}.$
}
\solutionspace{120pt}

\tasknumber{3}%
\task{%
    Тело совершает гармонические колебания с периодом $6\,\text{c}$.
    За какое время тело смещается от положения наибольшего отклонения до смещения в половину амплитуды?
}
\answer{%
    $t = \frac T{6} \approx 1{,}0\,\text{c}.$
}
\solutionspace{120pt}

\tasknumber{4}%
\task{%
    Запишите формулу для периода колебаний математического маятника и ...
    \begin{itemize}
        \item укажите названия всех физических величин в формуле,
        \item выразите из неё частоту колебаний
        \item выразите из неё ускорение свободного падения.
    \end{itemize}
}
\answer{%
    \begin{align*}
    T &= 2\pi \sqrt{\frac lg} \implies \nu = \frac 1T = \frac 1{2\pi}\sqrt{\frac gl}, \omega = 2\pi\nu = \sqrt{\frac gl}, l = g\sqr{\frac T{2\pi}}, g = l\sqr{\frac {2\pi}T} \\
    T &= 2\pi \sqrt{\frac mk} \implies \nu = \frac 1T = \frac 1{2\pi}\sqrt{\frac km}, \omega = 2\pi\nu = \sqrt{\frac km}, m = k\sqr{\frac T{2\pi}}, k = m\sqr{\frac {2\pi}T}
    \end{align*}
}
\solutionspace{120pt}

\tasknumber{5}%
\task{%
    Частота колебаний пружинного маятника равна $12\,\text{Гц}$.
    Определите периоды колебаний
    \begin{itemize}
        \item потенциальной энергии системы,
        \item ускорения груза,
        \item модуля скорости груза.
    \end{itemize}
}
\answer{%
    $T = \frac 1\nu \approx 83{,}3\,\text{мc}, T_1 = \frac T2 \approx 41{,}7\,\text{мc}, T_2 = T \approx 83{,}3\,\text{мc}, T_3 = \frac T2 \approx 41{,}7\,\text{мc}.$
}
\solutionspace{80pt}

\tasknumber{6}%
\task{%
    Тело колеблется по гармоническому закону с амплитудой $4\,\text{см}$.
    Какой минимальный путь тело может пройти за половину периода?
}
\answer{%
    $2A \approx 8{,}0\,\text{см}.$
}
\solutionspace{80pt}

\tasknumber{7}%
\task{%
    Период колебаний математического маятника равен $3\,\text{с}$,
    а их амплитуда — $15\,\text{см}$.
    Определите амплитуду колебаний скорости маятника.
}
\answer{%
    $
        T = \frac{2\pi}{\omega}
        \implies \omega = \frac{2\pi}{T}
        \implies v_{\max} = \omega A = \frac{2\pi}{T}A
        \approx 31{,}4\,\frac{\text{см}}{\text{с}}.
    $
}
\solutionspace{100pt}

\tasknumber{8}%
\task{%
    Определите период колебаний груза массой $m$, подвешенного к пружине жёсткостью $k$.
    Ускорение свободного падения $g$.
}
\answer{%
    \begin{align*}
    &-kx_0 + mg = 0, \\
    F &= -k(x_0 + \Delta x), \\
    ma &= -k(x_0 + \Delta x) + mg, \\
    ma &= -kx_0 -k \Delta x + mg = -mg -k \Delta x + mg = -k \Delta x, \\
    a &+ \frac k m x = 0, \\
    \omega^2 &= \frac k m \implies T = \frac{2\pi}\omega = 2\pi\sqrt{\frac m k}.
    \end{align*}
}
\solutionspace{150pt}

\tasknumber{9}%
\task{%
    Куб со стороной $a$ и плотности $\rho$ плавает в жидкости плотностью $\rho_0$.
    Определите частоту колебаний куба, считая, что 4 грани куба всегда вертикальны.
}
\answer{%
    \begin{align*}
    \Delta F &= -\rho_0 g a^2 \Delta x,\Delta F = m\ddot x, m = \rho a^3\implies \rho a^3 \ddot x = -\rho_0 g a^2 \Delta x \implies \\
    \implies \ddot x &= -\frac{\rho_0}{\rho} \frac g a \Delta x\implies \omega^2 = \frac{\rho_0}{\rho} \cdot \frac g a\implies T = \frac{2 \pi}{\omega} = 2 \pi\sqrt{\frac{\rho}{\rho_0} \cdot \frac a g}.
    \end{align*}
}
\solutionspace{150pt}

\tasknumber{10}%
\task{%
    Математический маятник с нитью длиной $37\,\text{см}$ подвешен к потолку в лифте.
    За $30\,\text{с}$ маятник совершил $22$ колебаний.
    Определите модуль и направление ускорения лифта.
    Куда движется лифт?
}
\answer{%
    $
        T = 2\pi\sqrt{\frac\ell {a + g}}, T = \frac {t}{N}
        \implies a + g = \ell \cdot \frac{4 \pi ^ 2}{T^2},
        a = \ell \cdot \frac{4 \pi ^ 2}{T^2} - g = \ell \cdot \frac{4 \pi ^ 2 N^2}{t^2} - g \approx -2{,}140\,\frac{\text{м}}{\text{c}^{2}},
        \text{вверх}.
    $
}
\solutionspace{120pt}

\tasknumber{11}%
\task{%
    Масса груза в пружинном маятнике равна $500\,\text{г}$, при этом период его колебаний равен $1{,}4\,\text{с}$.
    Груз утяжеляют на $100\,\text{г}$.
    Определите новый период колебаний маятника.
}
\answer{%
    $
        T'
            = 2\pi\sqrt{\frac{M + m}{k}}
            = 2\pi\sqrt{\frac{M}{k} \cdot \frac{M + m}{M}}
            = T\sqrt{\frac{M + m}{M}} =  T\sqrt{1 + \frac{m}{M}} \approx 1{,}53\,\text{с}.
    $
}
\solutionspace{120pt}

\tasknumber{12}%
\task{%
    При какой длине нити математического маятника период колебаний груза массой $400\,\text{г}$
    окажется равен периоду колебаний этого же груза в пружинном маятнике с пружиной жёсткостью $40\,\frac{\text{Н}}{\text{м}}$?
}
\answer{%
    $
        2\pi \sqrt{\frac \ell g} = 2\pi \sqrt{\frac m k}
        \implies \frac \ell g = \frac m k
        \implies \ell = g \frac m k \approx 10{,}0\,\text{см}.
    $
}
\solutionspace{120pt}

\tasknumber{13}%
\task{%
    Груз подвесили к пружине, при этом удлинение пружины составило $45\,\text{мм}$.
    Определите частоту колебаний пружинного маятника, собранного из этой пружины и этого груза.
}
\answer{%
    $
        mg -k\Delta x = 0 \implies \frac m k = \frac{\Delta x} g
        \implies T = 2\pi \sqrt{\frac m k } = 2\pi \sqrt{\frac{\Delta x} g } \approx $0{,}42\,\text{с}$,
        \nu = \frac 1T \approx $2{,}37\,\text{Гц}$.
    $
}
\solutionspace{120pt}

\tasknumber{14}%
\task{%
    Определите период колебаний системы: математический маятник ограничен с одной стороны стенкой (см.
    рис.
    на доске).
    Удары маятника о стенку абсолютно упругие, $n = 2$.
    Длина маятника $\ell$, ускорение свободного падения $g$.
}
\answer{%
    $
        T = 2\pi\sqrt{\frac\ell g}, \qquad
        T' = 2 \cdot \frac T 4 + 2 \cdot \frac T{12} = \frac23T = \frac43\pi\sqrt{\frac\ell g}
    $
}

\variantsplitter

\addpersonalvariant{Анастасия Ламанова}

\tasknumber{1}%
\task{%
    \begin{itemize}
        \item Запишите линейное однородное дифференциальное уравнение второго порядка,
            описывающее свободные незатухающие колебания гармонического осциллятора,
        \item запишите общее решение этого уравнения,
        \item подпишите в выписанном решении фазу и амплитуду колебаний,
        \item запишите выражение для скорости,
        \item запишите выражение для ускорения.
    \end{itemize}
}
\answer{%
    \begin{align*}
    &\ddot x + \omega^2 x = 0 \Longleftrightarrow a_x + \omega^2 x = 0, \\
    &x = A \cos(\omega t + \varphi_0) \text{ или же } x = A \sin(\omega t + \varphi_0) \text{ или же } x = a \cos(\omega t) + b \sin(\omega t), \\
    &A \text{\, или \,} \sqrt{a^2 + b^2} \text{ --- это амплитуда}, \omega t + \varphi_0\text{ --- это фаза}, \\
    &v = \dot x = -\omega A \sin(\omega t + \varphi_0), \\
    &a = \dot v = \ddot x = -\omega^2 A \cos(\omega t + \varphi_0) = -\omega^2 x,
    \end{align*}
}
\solutionspace{135pt}

\tasknumber{2}%
\task{%
    Тело колеблется по гармоническому закону,
    амплитуда этих колебаний $18\,\text{см}$, период $2\,\text{c}$.
    Чему равно смещение тела относительно положения равновесия через $20\,\text{c}$
    после прохождения положения максимального отклонения?
}
\answer{%
    $x = A \cos \omega t = A \cos \cbr{ \frac {2\pi}T t } = A \cos \cbr{ 2\pi \frac tT } = 18\,\text{см} \cdot \cos \cbr{ 2\pi \cdot \frac {20\,\text{c}}{2\,\text{c}}}\approx 18{,}0\,\text{см}.$
}
\solutionspace{120pt}

\tasknumber{3}%
\task{%
    Тело совершает гармонические колебания с периодом $5\,\text{c}$.
    За какое время тело смещается от положения равновесия до смещения в половину амплитуды?
}
\answer{%
    $t = \frac T{12} \approx 0{,}4\,\text{c}.$
}
\solutionspace{120pt}

\tasknumber{4}%
\task{%
    Запишите формулу для периода колебаний математического маятника и ...
    \begin{itemize}
        \item укажите названия всех физических величин в формуле,
        \item выразите из неё частоту колебаний
        \item выразите из неё длину маятника.
    \end{itemize}
}
\answer{%
    \begin{align*}
    T &= 2\pi \sqrt{\frac lg} \implies \nu = \frac 1T = \frac 1{2\pi}\sqrt{\frac gl}, \omega = 2\pi\nu = \sqrt{\frac gl}, l = g\sqr{\frac T{2\pi}}, g = l\sqr{\frac {2\pi}T} \\
    T &= 2\pi \sqrt{\frac mk} \implies \nu = \frac 1T = \frac 1{2\pi}\sqrt{\frac km}, \omega = 2\pi\nu = \sqrt{\frac km}, m = k\sqr{\frac T{2\pi}}, k = m\sqr{\frac {2\pi}T}
    \end{align*}
}
\solutionspace{120pt}

\tasknumber{5}%
\task{%
    Частота колебаний математического маятника равна $15\,\text{Гц}$.
    Определите периоды колебаний
    \begin{itemize}
        \item кинетической энергии системы,
        \item ускорения груза,
        \item модуля скорости груза.
    \end{itemize}
}
\answer{%
    $T = \frac 1\nu \approx 66{,}7\,\text{мc}, T_1 = \frac T2 \approx 33{,}3\,\text{мc}, T_2 = T \approx 66{,}7\,\text{мc}, T_3 = \frac T2 \approx 33{,}3\,\text{мc}.$
}
\solutionspace{80pt}

\tasknumber{6}%
\task{%
    Тело колеблется по гармоническому закону с амплитудой $12\,\text{см}$.
    Какой минимальный путь тело может пройти за половину периода?
}
\answer{%
    $2A \approx 24{,}0\,\text{см}.$
}
\solutionspace{80pt}

\tasknumber{7}%
\task{%
    Период колебаний математического маятника равен $5\,\text{с}$,
    а их амплитуда — $15\,\text{см}$.
    Определите максимальную скорость маятника.
}
\answer{%
    $
        T = \frac{2\pi}{\omega}
        \implies \omega = \frac{2\pi}{T}
        \implies v_{\max} = \omega A = \frac{2\pi}{T}A
        \approx 18{,}8\,\frac{\text{см}}{\text{с}}.
    $
}
\solutionspace{100pt}

\tasknumber{8}%
\task{%
    Определите период колебаний груза массой $m$, подвешенного к пружине жёсткостью $k$.
    Ускорение свободного падения $g$.
}
\answer{%
    \begin{align*}
    &-kx_0 + mg = 0, \\
    F &= -k(x_0 + \Delta x), \\
    ma &= -k(x_0 + \Delta x) + mg, \\
    ma &= -kx_0 -k \Delta x + mg = -mg -k \Delta x + mg = -k \Delta x, \\
    a &+ \frac k m x = 0, \\
    \omega^2 &= \frac k m \implies T = \frac{2\pi}\omega = 2\pi\sqrt{\frac m k}.
    \end{align*}
}
\solutionspace{150pt}

\tasknumber{9}%
\task{%
    Куб со стороной $a$ и плотности $\rho$ плавает в жидкости плотностью $\rho_0$.
    Определите частоту колебаний куба, считая, что 4 грани куба всегда вертикальны.
}
\answer{%
    \begin{align*}
    \Delta F &= -\rho_0 g a^2 \Delta x,\Delta F = m\ddot x, m = \rho a^3\implies \rho a^3 \ddot x = -\rho_0 g a^2 \Delta x \implies \\
    \implies \ddot x &= -\frac{\rho_0}{\rho} \frac g a \Delta x\implies \omega^2 = \frac{\rho_0}{\rho} \cdot \frac g a\implies T = \frac{2 \pi}{\omega} = 2 \pi\sqrt{\frac{\rho}{\rho_0} \cdot \frac a g}.
    \end{align*}
}
\solutionspace{150pt}

\tasknumber{10}%
\task{%
    Математический маятник с нитью длиной $40\,\text{см}$ подвешен к потолку в лифте.
    За $30\,\text{с}$ маятник совершил $24$ колебаний.
    Определите модуль и направление ускорения лифта.
    Куда движется лифт?
}
\answer{%
    $
        T = 2\pi\sqrt{\frac\ell {a + g}}, T = \frac {t}{N}
        \implies a + g = \ell \cdot \frac{4 \pi ^ 2}{T^2},
        a = \ell \cdot \frac{4 \pi ^ 2}{T^2} - g = \ell \cdot \frac{4 \pi ^ 2 N^2}{t^2} - g \approx 0{,}11\,\frac{\text{м}}{\text{c}^{2}},
        \text{вниз}.
    $
}
\solutionspace{120pt}

\tasknumber{11}%
\task{%
    Масса груза в пружинном маятнике равна $500\,\text{г}$, при этом период его колебаний равен $1{,}5\,\text{с}$.
    Груз утяжеляют на $100\,\text{г}$.
    Определите новый период колебаний маятника.
}
\answer{%
    $
        T'
            = 2\pi\sqrt{\frac{M + m}{k}}
            = 2\pi\sqrt{\frac{M}{k} \cdot \frac{M + m}{M}}
            = T\sqrt{\frac{M + m}{M}} =  T\sqrt{1 + \frac{m}{M}} \approx 1{,}64\,\text{с}.
    $
}
\solutionspace{120pt}

\tasknumber{12}%
\task{%
    При какой длине нити математического маятника период колебаний груза массой $200\,\text{г}$
    окажется равен периоду колебаний этого же груза в пружинном маятнике с пружиной жёсткостью $60\,\frac{\text{Н}}{\text{м}}$?
}
\answer{%
    $
        2\pi \sqrt{\frac \ell g} = 2\pi \sqrt{\frac m k}
        \implies \frac \ell g = \frac m k
        \implies \ell = g \frac m k \approx 3{,}3\,\text{см}.
    $
}
\solutionspace{120pt}

\tasknumber{13}%
\task{%
    Груз подвесили к пружине, при этом удлинение пружины составило $45\,\text{мм}$.
    Определите частоту колебаний пружинного маятника, собранного из этой пружины и этого груза.
}
\answer{%
    $
        mg -k\Delta x = 0 \implies \frac m k = \frac{\Delta x} g
        \implies T = 2\pi \sqrt{\frac m k } = 2\pi \sqrt{\frac{\Delta x} g } \approx $0{,}42\,\text{с}$,
        \nu = \frac 1T \approx $2{,}37\,\text{Гц}$.
    $
}
\solutionspace{120pt}

\tasknumber{14}%
\task{%
    Определите период колебаний системы: математический маятник ограничен с одной стороны стенкой (см.
    рис.
    на доске).
    Удары маятника о стенку абсолютно упругие, $n = \frac{2}{\sqrt{3}}$.
    Длина маятника $\ell$, ускорение свободного падения $g$.
}
\answer{%
    $
        T = 2\pi\sqrt{\frac\ell g}, \qquad
        T' = 2 \cdot \frac T 4 + 2 \cdot \frac T{6} = \frac56T = \frac53\pi\sqrt{\frac\ell g}
    $
}

\variantsplitter

\addpersonalvariant{Виктория Легонькова}

\tasknumber{1}%
\task{%
    \begin{itemize}
        \item Запишите линейное однородное дифференциальное уравнение второго порядка,
            описывающее свободные незатухающие колебания гармонического осциллятора,
        \item запишите общее решение этого уравнения,
        \item подпишите в выписанном решении фазу и амплитуду колебаний,
        \item запишите выражение для скорости,
        \item запишите выражение для ускорения.
    \end{itemize}
}
\answer{%
    \begin{align*}
    &\ddot x + \omega^2 x = 0 \Longleftrightarrow a_x + \omega^2 x = 0, \\
    &x = A \cos(\omega t + \varphi_0) \text{ или же } x = A \sin(\omega t + \varphi_0) \text{ или же } x = a \cos(\omega t) + b \sin(\omega t), \\
    &A \text{\, или \,} \sqrt{a^2 + b^2} \text{ --- это амплитуда}, \omega t + \varphi_0\text{ --- это фаза}, \\
    &v = \dot x = -\omega A \sin(\omega t + \varphi_0), \\
    &a = \dot v = \ddot x = -\omega^2 A \cos(\omega t + \varphi_0) = -\omega^2 x,
    \end{align*}
}
\solutionspace{135pt}

\tasknumber{2}%
\task{%
    Тело колеблется по гармоническому закону,
    амплитуда этих колебаний $16\,\text{см}$, период $2\,\text{c}$.
    Чему равно смещение тела относительно положения равновесия через $22\,\text{c}$
    после прохождения положения равновесия?
}
\answer{%
    $x = A \sin \omega t = A \sin \cbr{ \frac {2\pi}T t } = A \sin \cbr{ 2\pi \frac tT } = 16\,\text{см} \cdot \sin \cbr{ 2\pi \cdot \frac {22\,\text{c}}{2\,\text{c}}}\approx 0\,\text{см}.$
}
\solutionspace{120pt}

\tasknumber{3}%
\task{%
    Тело совершает гармонические колебания с периодом $5\,\text{c}$.
    За какое время тело смещается от положения равновесия до смещения в половину амплитуды?
}
\answer{%
    $t = \frac T{12} \approx 0{,}4\,\text{c}.$
}
\solutionspace{120pt}

\tasknumber{4}%
\task{%
    Запишите формулу для периода колебаний математического маятника и ...
    \begin{itemize}
        \item укажите названия всех физических величин в формуле,
        \item выразите из неё частоту колебаний
        \item выразите из неё длину маятника.
    \end{itemize}
}
\answer{%
    \begin{align*}
    T &= 2\pi \sqrt{\frac lg} \implies \nu = \frac 1T = \frac 1{2\pi}\sqrt{\frac gl}, \omega = 2\pi\nu = \sqrt{\frac gl}, l = g\sqr{\frac T{2\pi}}, g = l\sqr{\frac {2\pi}T} \\
    T &= 2\pi \sqrt{\frac mk} \implies \nu = \frac 1T = \frac 1{2\pi}\sqrt{\frac km}, \omega = 2\pi\nu = \sqrt{\frac km}, m = k\sqr{\frac T{2\pi}}, k = m\sqr{\frac {2\pi}T}
    \end{align*}
}
\solutionspace{120pt}

\tasknumber{5}%
\task{%
    Частота колебаний математического маятника равна $8\,\text{Гц}$.
    Определите периоды колебаний
    \begin{itemize}
        \item потенциальной энергии системы,
        \item скорости груза,
        \item модуля скорости груза.
    \end{itemize}
}
\answer{%
    $T = \frac 1\nu \approx 125{,}0\,\text{мc}, T_1 = \frac T2 \approx 62{,}5\,\text{мc}, T_2 = T \approx 125{,}0\,\text{мc}, T_3 = \frac T2 \approx 62{,}5\,\text{мc}.$
}
\solutionspace{80pt}

\tasknumber{6}%
\task{%
    Тело колеблется по гармоническому закону с амплитудой $4\,\text{см}$.
    Какой максимальный путь тело может пройти за половину периода?
}
\answer{%
    $2A \approx 8{,}0\,\text{см}.$
}
\solutionspace{80pt}

\tasknumber{7}%
\task{%
    Период колебаний математического маятника равен $3\,\text{с}$,
    а их амплитуда — $20\,\text{см}$.
    Определите максимальную скорость маятника.
}
\answer{%
    $
        T = \frac{2\pi}{\omega}
        \implies \omega = \frac{2\pi}{T}
        \implies v_{\max} = \omega A = \frac{2\pi}{T}A
        \approx 41{,}9\,\frac{\text{см}}{\text{с}}.
    $
}
\solutionspace{100pt}

\tasknumber{8}%
\task{%
    Определите период колебаний груза массой $m$, подвешенного к пружине жёсткостью $k$.
    Ускорение свободного падения $g$.
}
\answer{%
    \begin{align*}
    &-kx_0 + mg = 0, \\
    F &= -k(x_0 + \Delta x), \\
    ma &= -k(x_0 + \Delta x) + mg, \\
    ma &= -kx_0 -k \Delta x + mg = -mg -k \Delta x + mg = -k \Delta x, \\
    a &+ \frac k m x = 0, \\
    \omega^2 &= \frac k m \implies T = \frac{2\pi}\omega = 2\pi\sqrt{\frac m k}.
    \end{align*}
}
\solutionspace{150pt}

\tasknumber{9}%
\task{%
    Куб со стороной $a$ и плотности $\rho$ плавает в жидкости плотностью $\rho_0$.
    Определите частоту колебаний куба, считая, что 4 грани куба всегда вертикальны.
}
\answer{%
    \begin{align*}
    \Delta F &= -\rho_0 g a^2 \Delta x,\Delta F = m\ddot x, m = \rho a^3\implies \rho a^3 \ddot x = -\rho_0 g a^2 \Delta x \implies \\
    \implies \ddot x &= -\frac{\rho_0}{\rho} \frac g a \Delta x\implies \omega^2 = \frac{\rho_0}{\rho} \cdot \frac g a\implies T = \frac{2 \pi}{\omega} = 2 \pi\sqrt{\frac{\rho}{\rho_0} \cdot \frac a g}.
    \end{align*}
}
\solutionspace{150pt}

\tasknumber{10}%
\task{%
    Математический маятник с нитью длиной $37\,\text{см}$ подвешен к потолку в лифте.
    За $30\,\text{с}$ маятник совершил $27$ колебаний.
    Определите модуль и направление ускорения лифта.
    Куда движется лифт?
}
\answer{%
    $
        T = 2\pi\sqrt{\frac\ell {a + g}}, T = \frac {t}{N}
        \implies a + g = \ell \cdot \frac{4 \pi ^ 2}{T^2},
        a = \ell \cdot \frac{4 \pi ^ 2}{T^2} - g = \ell \cdot \frac{4 \pi ^ 2 N^2}{t^2} - g \approx 1{,}83\,\frac{\text{м}}{\text{c}^{2}},
        \text{вниз}.
    $
}
\solutionspace{120pt}

\tasknumber{11}%
\task{%
    Масса груза в пружинном маятнике равна $400\,\text{г}$, при этом период его колебаний равен $1{,}2\,\text{с}$.
    Груз облегчают на $150\,\text{г}$.
    Определите новый период колебаний маятника.
}
\answer{%
    $
        T'
            = 2\pi\sqrt{\frac{M - m}{k}}
            = 2\pi\sqrt{\frac{M}{k} \cdot \frac{M - m}{M}}
            = T\sqrt{\frac{M - m}{M}} =  T\sqrt{1 - \frac{m}{M}} \approx 0{,}95\,\text{с}.
    $
}
\solutionspace{120pt}

\tasknumber{12}%
\task{%
    При какой длине нити математического маятника период колебаний груза массой $300\,\text{г}$
    окажется равен периоду колебаний этого же груза в пружинном маятнике с пружиной жёсткостью $50\,\frac{\text{Н}}{\text{м}}$?
}
\answer{%
    $
        2\pi \sqrt{\frac \ell g} = 2\pi \sqrt{\frac m k}
        \implies \frac \ell g = \frac m k
        \implies \ell = g \frac m k \approx 6{,}0\,\text{см}.
    $
}
\solutionspace{120pt}

\tasknumber{13}%
\task{%
    Груз подвесили к пружине, при этом удлинение пружины составило $45\,\text{мм}$.
    Определите частоту колебаний пружинного маятника, собранного из этой пружины и этого груза.
}
\answer{%
    $
        mg -k\Delta x = 0 \implies \frac m k = \frac{\Delta x} g
        \implies T = 2\pi \sqrt{\frac m k } = 2\pi \sqrt{\frac{\Delta x} g } \approx $0{,}42\,\text{с}$,
        \nu = \frac 1T \approx $2{,}37\,\text{Гц}$.
    $
}
\solutionspace{120pt}

\tasknumber{14}%
\task{%
    Определите период колебаний системы: математический маятник ограничен с одной стороны стенкой (см.
    рис.
    на доске).
    Удары маятника о стенку абсолютно упругие, $n = \frac{2}{\sqrt{3}}$.
    Длина маятника $\ell$, ускорение свободного падения $g$.
}
\answer{%
    $
        T = 2\pi\sqrt{\frac\ell g}, \qquad
        T' = 2 \cdot \frac T 4 + 2 \cdot \frac T{6} = \frac56T = \frac53\pi\sqrt{\frac\ell g}
    $
}

\variantsplitter

\addpersonalvariant{Семён Мартынов}

\tasknumber{1}%
\task{%
    \begin{itemize}
        \item Запишите линейное однородное дифференциальное уравнение второго порядка,
            описывающее свободные незатухающие колебания гармонического осциллятора,
        \item запишите общее решение этого уравнения,
        \item подпишите в выписанном решении фазу и амплитуду колебаний,
        \item запишите выражение для скорости,
        \item запишите выражение для ускорения.
    \end{itemize}
}
\answer{%
    \begin{align*}
    &\ddot x + \omega^2 x = 0 \Longleftrightarrow a_x + \omega^2 x = 0, \\
    &x = A \cos(\omega t + \varphi_0) \text{ или же } x = A \sin(\omega t + \varphi_0) \text{ или же } x = a \cos(\omega t) + b \sin(\omega t), \\
    &A \text{\, или \,} \sqrt{a^2 + b^2} \text{ --- это амплитуда}, \omega t + \varphi_0\text{ --- это фаза}, \\
    &v = \dot x = -\omega A \sin(\omega t + \varphi_0), \\
    &a = \dot v = \ddot x = -\omega^2 A \cos(\omega t + \varphi_0) = -\omega^2 x,
    \end{align*}
}
\solutionspace{135pt}

\tasknumber{2}%
\task{%
    Тело колеблется по гармоническому закону,
    амплитуда этих колебаний $16\,\text{см}$, период $4\,\text{c}$.
    Чему равно смещение тела относительно положения равновесия через $23\,\text{c}$
    после прохождения положения максимального отклонения?
}
\answer{%
    $x = A \cos \omega t = A \cos \cbr{ \frac {2\pi}T t } = A \cos \cbr{ 2\pi \frac tT } = 16\,\text{см} \cdot \cos \cbr{ 2\pi \cdot \frac {23\,\text{c}}{4\,\text{c}}}\approx 0\,\text{см}.$
}
\solutionspace{120pt}

\tasknumber{3}%
\task{%
    Тело совершает гармонические колебания с периодом $6\,\text{c}$.
    За какое время тело смещается от положения равновесия до смещения в половину амплитуды?
}
\answer{%
    $t = \frac T{12} \approx 0{,}5\,\text{c}.$
}
\solutionspace{120pt}

\tasknumber{4}%
\task{%
    Запишите формулу для периода колебаний пружинного маятника и ...
    \begin{itemize}
        \item укажите названия всех физических величин в формуле,
        \item выразите из неё циклическую частоту колебаний
        \item выразите из неё массу груза.
    \end{itemize}
}
\answer{%
    \begin{align*}
    T &= 2\pi \sqrt{\frac lg} \implies \nu = \frac 1T = \frac 1{2\pi}\sqrt{\frac gl}, \omega = 2\pi\nu = \sqrt{\frac gl}, l = g\sqr{\frac T{2\pi}}, g = l\sqr{\frac {2\pi}T} \\
    T &= 2\pi \sqrt{\frac mk} \implies \nu = \frac 1T = \frac 1{2\pi}\sqrt{\frac km}, \omega = 2\pi\nu = \sqrt{\frac km}, m = k\sqr{\frac T{2\pi}}, k = m\sqr{\frac {2\pi}T}
    \end{align*}
}
\solutionspace{120pt}

\tasknumber{5}%
\task{%
    Частота колебаний пружинного маятника равна $15\,\text{Гц}$.
    Определите периоды колебаний
    \begin{itemize}
        \item потенциальной энергии системы,
        \item скорости груза,
        \item модуля скорости груза.
    \end{itemize}
}
\answer{%
    $T = \frac 1\nu \approx 66{,}7\,\text{мc}, T_1 = \frac T2 \approx 33{,}3\,\text{мc}, T_2 = T \approx 66{,}7\,\text{мc}, T_3 = \frac T2 \approx 33{,}3\,\text{мc}.$
}
\solutionspace{80pt}

\tasknumber{6}%
\task{%
    Тело колеблется по гармоническому закону с амплитудой $4\,\text{см}$.
    Какой минимальный путь тело может пройти за половину периода?
}
\answer{%
    $2A \approx 8{,}0\,\text{см}.$
}
\solutionspace{80pt}

\tasknumber{7}%
\task{%
    Период колебаний математического маятника равен $3\,\text{с}$,
    а их амплитуда — $10\,\text{см}$.
    Определите максимальную скорость маятника.
}
\answer{%
    $
        T = \frac{2\pi}{\omega}
        \implies \omega = \frac{2\pi}{T}
        \implies v_{\max} = \omega A = \frac{2\pi}{T}A
        \approx 20{,}9\,\frac{\text{см}}{\text{с}}.
    $
}
\solutionspace{100pt}

\tasknumber{8}%
\task{%
    Определите период колебаний груза массой $m$, подвешенного к пружине жёсткостью $k$.
    Ускорение свободного падения $g$.
}
\answer{%
    \begin{align*}
    &-kx_0 + mg = 0, \\
    F &= -k(x_0 + \Delta x), \\
    ma &= -k(x_0 + \Delta x) + mg, \\
    ma &= -kx_0 -k \Delta x + mg = -mg -k \Delta x + mg = -k \Delta x, \\
    a &+ \frac k m x = 0, \\
    \omega^2 &= \frac k m \implies T = \frac{2\pi}\omega = 2\pi\sqrt{\frac m k}.
    \end{align*}
}
\solutionspace{150pt}

\tasknumber{9}%
\task{%
    Куб со стороной $a$ и плотности $\rho$ плавает в жидкости плотностью $\rho_0$.
    Определите частоту колебаний куба, считая, что 4 грани куба всегда вертикальны.
}
\answer{%
    \begin{align*}
    \Delta F &= -\rho_0 g a^2 \Delta x,\Delta F = m\ddot x, m = \rho a^3\implies \rho a^3 \ddot x = -\rho_0 g a^2 \Delta x \implies \\
    \implies \ddot x &= -\frac{\rho_0}{\rho} \frac g a \Delta x\implies \omega^2 = \frac{\rho_0}{\rho} \cdot \frac g a\implies T = \frac{2 \pi}{\omega} = 2 \pi\sqrt{\frac{\rho}{\rho_0} \cdot \frac a g}.
    \end{align*}
}
\solutionspace{150pt}

\tasknumber{10}%
\task{%
    Математический маятник с нитью длиной $37\,\text{см}$ подвешен к потолку в лифте.
    За $25\,\text{с}$ маятник совершил $20$ колебаний.
    Определите модуль и направление ускорения лифта.
    Куда движется лифт?
}
\answer{%
    $
        T = 2\pi\sqrt{\frac\ell {a + g}}, T = \frac {t}{N}
        \implies a + g = \ell \cdot \frac{4 \pi ^ 2}{T^2},
        a = \ell \cdot \frac{4 \pi ^ 2}{T^2} - g = \ell \cdot \frac{4 \pi ^ 2 N^2}{t^2} - g \approx -0{,}6500\,\frac{\text{м}}{\text{c}^{2}},
        \text{вверх}.
    $
}
\solutionspace{120pt}

\tasknumber{11}%
\task{%
    Масса груза в пружинном маятнике равна $600\,\text{г}$, при этом период его колебаний равен $1{,}3\,\text{с}$.
    Груз утяжеляют на $150\,\text{г}$.
    Определите новый период колебаний маятника.
}
\answer{%
    $
        T'
            = 2\pi\sqrt{\frac{M + m}{k}}
            = 2\pi\sqrt{\frac{M}{k} \cdot \frac{M + m}{M}}
            = T\sqrt{\frac{M + m}{M}} =  T\sqrt{1 + \frac{m}{M}} \approx 1{,}45\,\text{с}.
    $
}
\solutionspace{120pt}

\tasknumber{12}%
\task{%
    При какой длине нити математического маятника период колебаний груза массой $300\,\text{г}$
    окажется равен периоду колебаний этого же груза в пружинном маятнике с пружиной жёсткостью $50\,\frac{\text{Н}}{\text{м}}$?
}
\answer{%
    $
        2\pi \sqrt{\frac \ell g} = 2\pi \sqrt{\frac m k}
        \implies \frac \ell g = \frac m k
        \implies \ell = g \frac m k \approx 6{,}0\,\text{см}.
    $
}
\solutionspace{120pt}

\tasknumber{13}%
\task{%
    Груз подвесили к пружине, при этом удлинение пружины составило $60\,\text{мм}$.
    Определите частоту колебаний пружинного маятника, собранного из этой пружины и этого груза.
}
\answer{%
    $
        mg -k\Delta x = 0 \implies \frac m k = \frac{\Delta x} g
        \implies T = 2\pi \sqrt{\frac m k } = 2\pi \sqrt{\frac{\Delta x} g } \approx $0{,}49\,\text{с}$,
        \nu = \frac 1T \approx $2{,}05\,\text{Гц}$.
    $
}
\solutionspace{120pt}

\tasknumber{14}%
\task{%
    Определите период колебаний системы: математический маятник ограничен с одной стороны стенкой (см.
    рис.
    на доске).
    Удары маятника о стенку абсолютно упругие, $n = \frac{2}{\sqrt{3}}$.
    Длина маятника $\ell$, ускорение свободного падения $g$.
}
\answer{%
    $
        T = 2\pi\sqrt{\frac\ell g}, \qquad
        T' = 2 \cdot \frac T 4 + 2 \cdot \frac T{6} = \frac56T = \frac53\pi\sqrt{\frac\ell g}
    $
}

\variantsplitter

\addpersonalvariant{Варвара Минаева}

\tasknumber{1}%
\task{%
    \begin{itemize}
        \item Запишите линейное однородное дифференциальное уравнение второго порядка,
            описывающее свободные незатухающие колебания гармонического осциллятора,
        \item запишите общее решение этого уравнения,
        \item подпишите в выписанном решении фазу и амплитуду колебаний,
        \item запишите выражение для скорости,
        \item запишите выражение для ускорения.
    \end{itemize}
}
\answer{%
    \begin{align*}
    &\ddot x + \omega^2 x = 0 \Longleftrightarrow a_x + \omega^2 x = 0, \\
    &x = A \cos(\omega t + \varphi_0) \text{ или же } x = A \sin(\omega t + \varphi_0) \text{ или же } x = a \cos(\omega t) + b \sin(\omega t), \\
    &A \text{\, или \,} \sqrt{a^2 + b^2} \text{ --- это амплитуда}, \omega t + \varphi_0\text{ --- это фаза}, \\
    &v = \dot x = -\omega A \sin(\omega t + \varphi_0), \\
    &a = \dot v = \ddot x = -\omega^2 A \cos(\omega t + \varphi_0) = -\omega^2 x,
    \end{align*}
}
\solutionspace{135pt}

\tasknumber{2}%
\task{%
    Тело колеблется по гармоническому закону,
    амплитуда этих колебаний $18\,\text{см}$, период $4\,\text{c}$.
    Чему равно смещение тела относительно положения равновесия через $25\,\text{c}$
    после прохождения положения максимального отклонения?
}
\answer{%
    $x = A \cos \omega t = A \cos \cbr{ \frac {2\pi}T t } = A \cos \cbr{ 2\pi \frac tT } = 18\,\text{см} \cdot \cos \cbr{ 2\pi \cdot \frac {25\,\text{c}}{4\,\text{c}}}\approx 0\,\text{см}.$
}
\solutionspace{120pt}

\tasknumber{3}%
\task{%
    Тело совершает гармонические колебания с периодом $4\,\text{c}$.
    За какое время тело смещается от положения наибольшего отклонения до смещения в половину амплитуды?
}
\answer{%
    $t = \frac T{6} \approx 0{,}7\,\text{c}.$
}
\solutionspace{120pt}

\tasknumber{4}%
\task{%
    Запишите формулу для периода колебаний пружинного маятника и ...
    \begin{itemize}
        \item укажите названия всех физических величин в формуле,
        \item выразите из неё циклическую частоту колебаний
        \item выразите из неё массу груза.
    \end{itemize}
}
\answer{%
    \begin{align*}
    T &= 2\pi \sqrt{\frac lg} \implies \nu = \frac 1T = \frac 1{2\pi}\sqrt{\frac gl}, \omega = 2\pi\nu = \sqrt{\frac gl}, l = g\sqr{\frac T{2\pi}}, g = l\sqr{\frac {2\pi}T} \\
    T &= 2\pi \sqrt{\frac mk} \implies \nu = \frac 1T = \frac 1{2\pi}\sqrt{\frac km}, \omega = 2\pi\nu = \sqrt{\frac km}, m = k\sqr{\frac T{2\pi}}, k = m\sqr{\frac {2\pi}T}
    \end{align*}
}
\solutionspace{120pt}

\tasknumber{5}%
\task{%
    Частота колебаний математического маятника равна $8\,\text{Гц}$.
    Определите периоды колебаний
    \begin{itemize}
        \item кинетической энергии системы,
        \item ускорения груза,
        \item модуля ускорения груза.
    \end{itemize}
}
\answer{%
    $T = \frac 1\nu \approx 125{,}0\,\text{мc}, T_1 = \frac T2 \approx 62{,}5\,\text{мc}, T_2 = T \approx 125{,}0\,\text{мc}, T_3 = \frac T2 \approx 62{,}5\,\text{мc}.$
}
\solutionspace{80pt}

\tasknumber{6}%
\task{%
    Тело колеблется по гармоническому закону с амплитудой $6\,\text{см}$.
    Какой максимальный путь тело может пройти за четверть периода?
}
\answer{%
    $2 * A \frac{1}{\sqrt 2} = A\sqrt{2} \approx 8{,}5\,\text{см}.$
}
\solutionspace{80pt}

\tasknumber{7}%
\task{%
    Период колебаний математического маятника равен $3\,\text{с}$,
    а их амплитуда — $10\,\text{см}$.
    Определите амплитуду колебаний скорости маятника.
}
\answer{%
    $
        T = \frac{2\pi}{\omega}
        \implies \omega = \frac{2\pi}{T}
        \implies v_{\max} = \omega A = \frac{2\pi}{T}A
        \approx 20{,}9\,\frac{\text{см}}{\text{с}}.
    $
}
\solutionspace{100pt}

\tasknumber{8}%
\task{%
    Определите период колебаний груза массой $m$, подвешенного к пружине жёсткостью $k$.
    Ускорение свободного падения $g$.
}
\answer{%
    \begin{align*}
    &-kx_0 + mg = 0, \\
    F &= -k(x_0 + \Delta x), \\
    ma &= -k(x_0 + \Delta x) + mg, \\
    ma &= -kx_0 -k \Delta x + mg = -mg -k \Delta x + mg = -k \Delta x, \\
    a &+ \frac k m x = 0, \\
    \omega^2 &= \frac k m \implies T = \frac{2\pi}\omega = 2\pi\sqrt{\frac m k}.
    \end{align*}
}
\solutionspace{150pt}

\tasknumber{9}%
\task{%
    Куб со стороной $a$ и плотности $\rho$ плавает в жидкости плотностью $\rho_0$.
    Определите частоту колебаний куба, считая, что 4 грани куба всегда вертикальны.
}
\answer{%
    \begin{align*}
    \Delta F &= -\rho_0 g a^2 \Delta x,\Delta F = m\ddot x, m = \rho a^3\implies \rho a^3 \ddot x = -\rho_0 g a^2 \Delta x \implies \\
    \implies \ddot x &= -\frac{\rho_0}{\rho} \frac g a \Delta x\implies \omega^2 = \frac{\rho_0}{\rho} \cdot \frac g a\implies T = \frac{2 \pi}{\omega} = 2 \pi\sqrt{\frac{\rho}{\rho_0} \cdot \frac a g}.
    \end{align*}
}
\solutionspace{150pt}

\tasknumber{10}%
\task{%
    Математический маятник с нитью длиной $40\,\text{см}$ подвешен к потолку в лифте.
    За $20\,\text{с}$ маятник совершил $14$ колебаний.
    Определите модуль и направление ускорения лифта.
    Куда движется лифт?
}
\answer{%
    $
        T = 2\pi\sqrt{\frac\ell {a + g}}, T = \frac {t}{N}
        \implies a + g = \ell \cdot \frac{4 \pi ^ 2}{T^2},
        a = \ell \cdot \frac{4 \pi ^ 2}{T^2} - g = \ell \cdot \frac{4 \pi ^ 2 N^2}{t^2} - g \approx -2{,}260\,\frac{\text{м}}{\text{c}^{2}},
        \text{вверх}.
    $
}
\solutionspace{120pt}

\tasknumber{11}%
\task{%
    Масса груза в пружинном маятнике равна $600\,\text{г}$, при этом период его колебаний равен $1{,}4\,\text{с}$.
    Груз облегчают на $100\,\text{г}$.
    Определите новый период колебаний маятника.
}
\answer{%
    $
        T'
            = 2\pi\sqrt{\frac{M - m}{k}}
            = 2\pi\sqrt{\frac{M}{k} \cdot \frac{M - m}{M}}
            = T\sqrt{\frac{M - m}{M}} =  T\sqrt{1 - \frac{m}{M}} \approx 1{,}28\,\text{с}.
    $
}
\solutionspace{120pt}

\tasknumber{12}%
\task{%
    При какой длине нити математического маятника период колебаний груза массой $400\,\text{г}$
    окажется равен периоду колебаний этого же груза в пружинном маятнике с пружиной жёсткостью $50\,\frac{\text{Н}}{\text{м}}$?
}
\answer{%
    $
        2\pi \sqrt{\frac \ell g} = 2\pi \sqrt{\frac m k}
        \implies \frac \ell g = \frac m k
        \implies \ell = g \frac m k \approx 8{,}0\,\text{см}.
    $
}
\solutionspace{120pt}

\tasknumber{13}%
\task{%
    Груз подвесили к пружине, при этом удлинение пружины составило $45\,\text{мм}$.
    Определите частоту колебаний пружинного маятника, собранного из этой пружины и этого груза.
}
\answer{%
    $
        mg -k\Delta x = 0 \implies \frac m k = \frac{\Delta x} g
        \implies T = 2\pi \sqrt{\frac m k } = 2\pi \sqrt{\frac{\Delta x} g } \approx $0{,}42\,\text{с}$,
        \nu = \frac 1T \approx $2{,}37\,\text{Гц}$.
    $
}
\solutionspace{120pt}

\tasknumber{14}%
\task{%
    Определите период колебаний системы: математический маятник ограничен с одной стороны стенкой (см.
    рис.
    на доске).
    Удары маятника о стенку абсолютно упругие, $n = \frac{2}{\sqrt{3}}$.
    Длина маятника $\ell$, ускорение свободного падения $g$.
}
\answer{%
    $
        T = 2\pi\sqrt{\frac\ell g}, \qquad
        T' = 2 \cdot \frac T 4 + 2 \cdot \frac T{6} = \frac56T = \frac53\pi\sqrt{\frac\ell g}
    $
}

\variantsplitter

\addpersonalvariant{Леонид Никитин}

\tasknumber{1}%
\task{%
    \begin{itemize}
        \item Запишите линейное однородное дифференциальное уравнение второго порядка,
            описывающее свободные незатухающие колебания гармонического осциллятора,
        \item запишите общее решение этого уравнения,
        \item подпишите в выписанном решении фазу и амплитуду колебаний,
        \item запишите выражение для скорости,
        \item запишите выражение для ускорения.
    \end{itemize}
}
\answer{%
    \begin{align*}
    &\ddot x + \omega^2 x = 0 \Longleftrightarrow a_x + \omega^2 x = 0, \\
    &x = A \cos(\omega t + \varphi_0) \text{ или же } x = A \sin(\omega t + \varphi_0) \text{ или же } x = a \cos(\omega t) + b \sin(\omega t), \\
    &A \text{\, или \,} \sqrt{a^2 + b^2} \text{ --- это амплитуда}, \omega t + \varphi_0\text{ --- это фаза}, \\
    &v = \dot x = -\omega A \sin(\omega t + \varphi_0), \\
    &a = \dot v = \ddot x = -\omega^2 A \cos(\omega t + \varphi_0) = -\omega^2 x,
    \end{align*}
}
\solutionspace{135pt}

\tasknumber{2}%
\task{%
    Тело колеблется по гармоническому закону,
    амплитуда этих колебаний $16\,\text{см}$, период $4\,\text{c}$.
    Чему равно смещение тела относительно положения равновесия через $26\,\text{c}$
    после прохождения положения равновесия?
}
\answer{%
    $x = A \sin \omega t = A \sin \cbr{ \frac {2\pi}T t } = A \sin \cbr{ 2\pi \frac tT } = 16\,\text{см} \cdot \sin \cbr{ 2\pi \cdot \frac {26\,\text{c}}{4\,\text{c}}}\approx 0\,\text{см}.$
}
\solutionspace{120pt}

\tasknumber{3}%
\task{%
    Тело совершает гармонические колебания с периодом $6\,\text{c}$.
    За какое время тело смещается от положения равновесия до смещения в половину амплитуды?
}
\answer{%
    $t = \frac T{12} \approx 0{,}5\,\text{c}.$
}
\solutionspace{120pt}

\tasknumber{4}%
\task{%
    Запишите формулу для периода колебаний пружинного маятника и ...
    \begin{itemize}
        \item укажите названия всех физических величин в формуле,
        \item выразите из неё частоту колебаний
        \item выразите из неё жёсткость пружины.
    \end{itemize}
}
\answer{%
    \begin{align*}
    T &= 2\pi \sqrt{\frac lg} \implies \nu = \frac 1T = \frac 1{2\pi}\sqrt{\frac gl}, \omega = 2\pi\nu = \sqrt{\frac gl}, l = g\sqr{\frac T{2\pi}}, g = l\sqr{\frac {2\pi}T} \\
    T &= 2\pi \sqrt{\frac mk} \implies \nu = \frac 1T = \frac 1{2\pi}\sqrt{\frac km}, \omega = 2\pi\nu = \sqrt{\frac km}, m = k\sqr{\frac T{2\pi}}, k = m\sqr{\frac {2\pi}T}
    \end{align*}
}
\solutionspace{120pt}

\tasknumber{5}%
\task{%
    Частота колебаний математического маятника равна $12\,\text{Гц}$.
    Определите периоды колебаний
    \begin{itemize}
        \item кинетической энергии системы,
        \item скорости груза,
        \item модуля ускорения груза.
    \end{itemize}
}
\answer{%
    $T = \frac 1\nu \approx 83{,}3\,\text{мc}, T_1 = \frac T2 \approx 41{,}7\,\text{мc}, T_2 = T \approx 83{,}3\,\text{мc}, T_3 = \frac T2 \approx 41{,}7\,\text{мc}.$
}
\solutionspace{80pt}

\tasknumber{6}%
\task{%
    Тело колеблется по гармоническому закону с амплитудой $4\,\text{см}$.
    Какой максимальный путь тело может пройти за одну шестую долю периода?
}
\answer{%
    $2 * A \frac{1}{2} = A \approx 4{,}0\,\text{см}.$
}
\solutionspace{80pt}

\tasknumber{7}%
\task{%
    Период колебаний математического маятника равен $3\,\text{с}$,
    а их амплитуда — $15\,\text{см}$.
    Определите максимальную скорость маятника.
}
\answer{%
    $
        T = \frac{2\pi}{\omega}
        \implies \omega = \frac{2\pi}{T}
        \implies v_{\max} = \omega A = \frac{2\pi}{T}A
        \approx 31{,}4\,\frac{\text{см}}{\text{с}}.
    $
}
\solutionspace{100pt}

\tasknumber{8}%
\task{%
    Определите период колебаний груза массой $m$, подвешенного к пружине жёсткостью $k$.
    Ускорение свободного падения $g$.
}
\answer{%
    \begin{align*}
    &-kx_0 + mg = 0, \\
    F &= -k(x_0 + \Delta x), \\
    ma &= -k(x_0 + \Delta x) + mg, \\
    ma &= -kx_0 -k \Delta x + mg = -mg -k \Delta x + mg = -k \Delta x, \\
    a &+ \frac k m x = 0, \\
    \omega^2 &= \frac k m \implies T = \frac{2\pi}\omega = 2\pi\sqrt{\frac m k}.
    \end{align*}
}
\solutionspace{150pt}

\tasknumber{9}%
\task{%
    Куб со стороной $a$ и плотности $\rho$ плавает в жидкости плотностью $\rho_0$.
    Определите частоту колебаний куба, считая, что 4 грани куба всегда вертикальны.
}
\answer{%
    \begin{align*}
    \Delta F &= -\rho_0 g a^2 \Delta x,\Delta F = m\ddot x, m = \rho a^3\implies \rho a^3 \ddot x = -\rho_0 g a^2 \Delta x \implies \\
    \implies \ddot x &= -\frac{\rho_0}{\rho} \frac g a \Delta x\implies \omega^2 = \frac{\rho_0}{\rho} \cdot \frac g a\implies T = \frac{2 \pi}{\omega} = 2 \pi\sqrt{\frac{\rho}{\rho_0} \cdot \frac a g}.
    \end{align*}
}
\solutionspace{150pt}

\tasknumber{10}%
\task{%
    Математический маятник с нитью длиной $43\,\text{см}$ подвешен к потолку в лифте.
    За $30\,\text{с}$ маятник совершил $21$ колебаний.
    Определите модуль и направление ускорения лифта.
    Куда движется лифт?
}
\answer{%
    $
        T = 2\pi\sqrt{\frac\ell {a + g}}, T = \frac {t}{N}
        \implies a + g = \ell \cdot \frac{4 \pi ^ 2}{T^2},
        a = \ell \cdot \frac{4 \pi ^ 2}{T^2} - g = \ell \cdot \frac{4 \pi ^ 2 N^2}{t^2} - g \approx -1{,}6800\,\frac{\text{м}}{\text{c}^{2}},
        \text{вверх}.
    $
}
\solutionspace{120pt}

\tasknumber{11}%
\task{%
    Масса груза в пружинном маятнике равна $400\,\text{г}$, при этом период его колебаний равен $1{,}3\,\text{с}$.
    Груз облегчают на $150\,\text{г}$.
    Определите новый период колебаний маятника.
}
\answer{%
    $
        T'
            = 2\pi\sqrt{\frac{M - m}{k}}
            = 2\pi\sqrt{\frac{M}{k} \cdot \frac{M - m}{M}}
            = T\sqrt{\frac{M - m}{M}} =  T\sqrt{1 - \frac{m}{M}} \approx 1{,}03\,\text{с}.
    $
}
\solutionspace{120pt}

\tasknumber{12}%
\task{%
    При какой длине нити математического маятника период колебаний груза массой $300\,\text{г}$
    окажется равен периоду колебаний этого же груза в пружинном маятнике с пружиной жёсткостью $50\,\frac{\text{Н}}{\text{м}}$?
}
\answer{%
    $
        2\pi \sqrt{\frac \ell g} = 2\pi \sqrt{\frac m k}
        \implies \frac \ell g = \frac m k
        \implies \ell = g \frac m k \approx 6{,}0\,\text{см}.
    $
}
\solutionspace{120pt}

\tasknumber{13}%
\task{%
    Груз подвесили к пружине, при этом удлинение пружины составило $30\,\text{мм}$.
    Определите частоту колебаний пружинного маятника, собранного из этой пружины и этого груза.
}
\answer{%
    $
        mg -k\Delta x = 0 \implies \frac m k = \frac{\Delta x} g
        \implies T = 2\pi \sqrt{\frac m k } = 2\pi \sqrt{\frac{\Delta x} g } \approx $0{,}34\,\text{с}$,
        \nu = \frac 1T \approx $2{,}91\,\text{Гц}$.
    $
}
\solutionspace{120pt}

\tasknumber{14}%
\task{%
    Определите период колебаний системы: математический маятник ограничен с одной стороны стенкой (см.
    рис.
    на доске).
    Удары маятника о стенку абсолютно упругие, $n = \frac{2}{\sqrt{3}}$.
    Длина маятника $\ell$, ускорение свободного падения $g$.
}
\answer{%
    $
        T = 2\pi\sqrt{\frac\ell g}, \qquad
        T' = 2 \cdot \frac T 4 + 2 \cdot \frac T{6} = \frac56T = \frac53\pi\sqrt{\frac\ell g}
    $
}

\variantsplitter

\addpersonalvariant{Тимофей Полетаев}

\tasknumber{1}%
\task{%
    \begin{itemize}
        \item Запишите линейное однородное дифференциальное уравнение второго порядка,
            описывающее свободные незатухающие колебания гармонического осциллятора,
        \item запишите общее решение этого уравнения,
        \item подпишите в выписанном решении фазу и амплитуду колебаний,
        \item запишите выражение для скорости,
        \item запишите выражение для ускорения.
    \end{itemize}
}
\answer{%
    \begin{align*}
    &\ddot x + \omega^2 x = 0 \Longleftrightarrow a_x + \omega^2 x = 0, \\
    &x = A \cos(\omega t + \varphi_0) \text{ или же } x = A \sin(\omega t + \varphi_0) \text{ или же } x = a \cos(\omega t) + b \sin(\omega t), \\
    &A \text{\, или \,} \sqrt{a^2 + b^2} \text{ --- это амплитуда}, \omega t + \varphi_0\text{ --- это фаза}, \\
    &v = \dot x = -\omega A \sin(\omega t + \varphi_0), \\
    &a = \dot v = \ddot x = -\omega^2 A \cos(\omega t + \varphi_0) = -\omega^2 x,
    \end{align*}
}
\solutionspace{135pt}

\tasknumber{2}%
\task{%
    Тело колеблется по гармоническому закону,
    амплитуда этих колебаний $10\,\text{см}$, период $6\,\text{c}$.
    Чему равно смещение тела относительно положения равновесия через $26\,\text{c}$
    после прохождения положения максимального отклонения?
}
\answer{%
    $x = A \cos \omega t = A \cos \cbr{ \frac {2\pi}T t } = A \cos \cbr{ 2\pi \frac tT } = 10\,\text{см} \cdot \cos \cbr{ 2\pi \cdot \frac {26\,\text{c}}{6\,\text{c}}}\approx -5{,}00\,\text{см}.$
}
\solutionspace{120pt}

\tasknumber{3}%
\task{%
    Тело совершает гармонические колебания с периодом $5\,\text{c}$.
    За какое время тело смещается от положения наибольшего отклонения до смещения в половину амплитуды?
}
\answer{%
    $t = \frac T{6} \approx 0{,}8\,\text{c}.$
}
\solutionspace{120pt}

\tasknumber{4}%
\task{%
    Запишите формулу для периода колебаний пружинного маятника и ...
    \begin{itemize}
        \item укажите названия всех физических величин в формуле,
        \item выразите из неё частоту колебаний
        \item выразите из неё массу груза.
    \end{itemize}
}
\answer{%
    \begin{align*}
    T &= 2\pi \sqrt{\frac lg} \implies \nu = \frac 1T = \frac 1{2\pi}\sqrt{\frac gl}, \omega = 2\pi\nu = \sqrt{\frac gl}, l = g\sqr{\frac T{2\pi}}, g = l\sqr{\frac {2\pi}T} \\
    T &= 2\pi \sqrt{\frac mk} \implies \nu = \frac 1T = \frac 1{2\pi}\sqrt{\frac km}, \omega = 2\pi\nu = \sqrt{\frac km}, m = k\sqr{\frac T{2\pi}}, k = m\sqr{\frac {2\pi}T}
    \end{align*}
}
\solutionspace{120pt}

\tasknumber{5}%
\task{%
    Частота колебаний математического маятника равна $10\,\text{Гц}$.
    Определите периоды колебаний
    \begin{itemize}
        \item кинетической энергии системы,
        \item скорости груза,
        \item модуля ускорения груза.
    \end{itemize}
}
\answer{%
    $T = \frac 1\nu \approx 100{,}0\,\text{мc}, T_1 = \frac T2 \approx 50{,}0\,\text{мc}, T_2 = T \approx 100{,}0\,\text{мc}, T_3 = \frac T2 \approx 50{,}0\,\text{мc}.$
}
\solutionspace{80pt}

\tasknumber{6}%
\task{%
    Тело колеблется по гармоническому закону с амплитудой $6\,\text{см}$.
    Какой минимальный путь тело может пройти за четверть периода?
}
\answer{%
    $2 * A \cbr{1 - \frac 1{\sqrt 2}} = A\cbr{2 - \sqrt{2}} \approx 3{,}5\,\text{см}.$
}
\solutionspace{80pt}

\tasknumber{7}%
\task{%
    Период колебаний математического маятника равен $2\,\text{с}$,
    а их амплитуда — $20\,\text{см}$.
    Определите амплитуду колебаний скорости маятника.
}
\answer{%
    $
        T = \frac{2\pi}{\omega}
        \implies \omega = \frac{2\pi}{T}
        \implies v_{\max} = \omega A = \frac{2\pi}{T}A
        \approx 62{,}8\,\frac{\text{см}}{\text{с}}.
    $
}
\solutionspace{100pt}

\tasknumber{8}%
\task{%
    Определите период колебаний груза массой $m$, подвешенного к пружине жёсткостью $k$.
    Ускорение свободного падения $g$.
}
\answer{%
    \begin{align*}
    &-kx_0 + mg = 0, \\
    F &= -k(x_0 + \Delta x), \\
    ma &= -k(x_0 + \Delta x) + mg, \\
    ma &= -kx_0 -k \Delta x + mg = -mg -k \Delta x + mg = -k \Delta x, \\
    a &+ \frac k m x = 0, \\
    \omega^2 &= \frac k m \implies T = \frac{2\pi}\omega = 2\pi\sqrt{\frac m k}.
    \end{align*}
}
\solutionspace{150pt}

\tasknumber{9}%
\task{%
    Куб со стороной $a$ и плотности $\rho$ плавает в жидкости плотностью $\rho_0$.
    Определите частоту колебаний куба, считая, что 4 грани куба всегда вертикальны.
}
\answer{%
    \begin{align*}
    \Delta F &= -\rho_0 g a^2 \Delta x,\Delta F = m\ddot x, m = \rho a^3\implies \rho a^3 \ddot x = -\rho_0 g a^2 \Delta x \implies \\
    \implies \ddot x &= -\frac{\rho_0}{\rho} \frac g a \Delta x\implies \omega^2 = \frac{\rho_0}{\rho} \cdot \frac g a\implies T = \frac{2 \pi}{\omega} = 2 \pi\sqrt{\frac{\rho}{\rho_0} \cdot \frac a g}.
    \end{align*}
}
\solutionspace{150pt}

\tasknumber{10}%
\task{%
    Математический маятник с нитью длиной $37\,\text{см}$ подвешен к потолку в лифте.
    За $20\,\text{с}$ маятник совершил $17$ колебаний.
    Определите модуль и направление ускорения лифта.
    Куда движется лифт?
}
\answer{%
    $
        T = 2\pi\sqrt{\frac\ell {a + g}}, T = \frac {t}{N}
        \implies a + g = \ell \cdot \frac{4 \pi ^ 2}{T^2},
        a = \ell \cdot \frac{4 \pi ^ 2}{T^2} - g = \ell \cdot \frac{4 \pi ^ 2 N^2}{t^2} - g \approx 0{,}55\,\frac{\text{м}}{\text{c}^{2}},
        \text{вниз}.
    $
}
\solutionspace{120pt}

\tasknumber{11}%
\task{%
    Масса груза в пружинном маятнике равна $400\,\text{г}$, при этом период его колебаний равен $1{,}3\,\text{с}$.
    Груз утяжеляют на $150\,\text{г}$.
    Определите новый период колебаний маятника.
}
\answer{%
    $
        T'
            = 2\pi\sqrt{\frac{M + m}{k}}
            = 2\pi\sqrt{\frac{M}{k} \cdot \frac{M + m}{M}}
            = T\sqrt{\frac{M + m}{M}} =  T\sqrt{1 + \frac{m}{M}} \approx 1{,}52\,\text{с}.
    $
}
\solutionspace{120pt}

\tasknumber{12}%
\task{%
    При какой длине нити математического маятника период колебаний груза массой $400\,\text{г}$
    окажется равен периоду колебаний этого же груза в пружинном маятнике с пружиной жёсткостью $40\,\frac{\text{Н}}{\text{м}}$?
}
\answer{%
    $
        2\pi \sqrt{\frac \ell g} = 2\pi \sqrt{\frac m k}
        \implies \frac \ell g = \frac m k
        \implies \ell = g \frac m k \approx 10{,}0\,\text{см}.
    $
}
\solutionspace{120pt}

\tasknumber{13}%
\task{%
    Груз подвесили к пружине, при этом удлинение пружины составило $30\,\text{мм}$.
    Определите частоту колебаний пружинного маятника, собранного из этой пружины и этого груза.
}
\answer{%
    $
        mg -k\Delta x = 0 \implies \frac m k = \frac{\Delta x} g
        \implies T = 2\pi \sqrt{\frac m k } = 2\pi \sqrt{\frac{\Delta x} g } \approx $0{,}34\,\text{с}$,
        \nu = \frac 1T \approx $2{,}91\,\text{Гц}$.
    $
}
\solutionspace{120pt}

\tasknumber{14}%
\task{%
    Определите период колебаний системы: математический маятник ограничен с одной стороны стенкой (см.
    рис.
    на доске).
    Удары маятника о стенку абсолютно упругие, $n = \sqrt{2}$.
    Длина маятника $\ell$, ускорение свободного падения $g$.
}
\answer{%
    $
        T = 2\pi\sqrt{\frac\ell g}, \qquad
        T' = 2 \cdot \frac T 4 + 2 \cdot \frac T{8} = \frac34T = \frac32\pi\sqrt{\frac\ell g}
    $
}

\variantsplitter

\addpersonalvariant{Андрей Рожков}

\tasknumber{1}%
\task{%
    \begin{itemize}
        \item Запишите линейное однородное дифференциальное уравнение второго порядка,
            описывающее свободные незатухающие колебания гармонического осциллятора,
        \item запишите общее решение этого уравнения,
        \item подпишите в выписанном решении фазу и амплитуду колебаний,
        \item запишите выражение для скорости,
        \item запишите выражение для ускорения.
    \end{itemize}
}
\answer{%
    \begin{align*}
    &\ddot x + \omega^2 x = 0 \Longleftrightarrow a_x + \omega^2 x = 0, \\
    &x = A \cos(\omega t + \varphi_0) \text{ или же } x = A \sin(\omega t + \varphi_0) \text{ или же } x = a \cos(\omega t) + b \sin(\omega t), \\
    &A \text{\, или \,} \sqrt{a^2 + b^2} \text{ --- это амплитуда}, \omega t + \varphi_0\text{ --- это фаза}, \\
    &v = \dot x = -\omega A \sin(\omega t + \varphi_0), \\
    &a = \dot v = \ddot x = -\omega^2 A \cos(\omega t + \varphi_0) = -\omega^2 x,
    \end{align*}
}
\solutionspace{135pt}

\tasknumber{2}%
\task{%
    Тело колеблется по гармоническому закону,
    амплитуда этих колебаний $18\,\text{см}$, период $2\,\text{c}$.
    Чему равно смещение тела относительно положения равновесия через $23\,\text{c}$
    после прохождения положения равновесия?
}
\answer{%
    $x = A \sin \omega t = A \sin \cbr{ \frac {2\pi}T t } = A \sin \cbr{ 2\pi \frac tT } = 18\,\text{см} \cdot \sin \cbr{ 2\pi \cdot \frac {23\,\text{c}}{2\,\text{c}}}\approx 0\,\text{см}.$
}
\solutionspace{120pt}

\tasknumber{3}%
\task{%
    Тело совершает гармонические колебания с периодом $6\,\text{c}$.
    За какое время тело смещается от положения равновесия до смещения в половину амплитуды?
}
\answer{%
    $t = \frac T{12} \approx 0{,}5\,\text{c}.$
}
\solutionspace{120pt}

\tasknumber{4}%
\task{%
    Запишите формулу для периода колебаний математического маятника и ...
    \begin{itemize}
        \item укажите названия всех физических величин в формуле,
        \item выразите из неё частоту колебаний
        \item выразите из неё ускорение свободного падения.
    \end{itemize}
}
\answer{%
    \begin{align*}
    T &= 2\pi \sqrt{\frac lg} \implies \nu = \frac 1T = \frac 1{2\pi}\sqrt{\frac gl}, \omega = 2\pi\nu = \sqrt{\frac gl}, l = g\sqr{\frac T{2\pi}}, g = l\sqr{\frac {2\pi}T} \\
    T &= 2\pi \sqrt{\frac mk} \implies \nu = \frac 1T = \frac 1{2\pi}\sqrt{\frac km}, \omega = 2\pi\nu = \sqrt{\frac km}, m = k\sqr{\frac T{2\pi}}, k = m\sqr{\frac {2\pi}T}
    \end{align*}
}
\solutionspace{120pt}

\tasknumber{5}%
\task{%
    Частота колебаний пружинного маятника равна $12\,\text{Гц}$.
    Определите периоды колебаний
    \begin{itemize}
        \item кинетической энергии системы,
        \item скорости груза,
        \item модуля ускорения груза.
    \end{itemize}
}
\answer{%
    $T = \frac 1\nu \approx 83{,}3\,\text{мc}, T_1 = \frac T2 \approx 41{,}7\,\text{мc}, T_2 = T \approx 83{,}3\,\text{мc}, T_3 = \frac T2 \approx 41{,}7\,\text{мc}.$
}
\solutionspace{80pt}

\tasknumber{6}%
\task{%
    Тело колеблется по гармоническому закону с амплитудой $4\,\text{см}$.
    Какой минимальный путь тело может пройти за половину периода?
}
\answer{%
    $2A \approx 8{,}0\,\text{см}.$
}
\solutionspace{80pt}

\tasknumber{7}%
\task{%
    Период колебаний математического маятника равен $5\,\text{с}$,
    а их амплитуда — $15\,\text{см}$.
    Определите максимальную скорость маятника.
}
\answer{%
    $
        T = \frac{2\pi}{\omega}
        \implies \omega = \frac{2\pi}{T}
        \implies v_{\max} = \omega A = \frac{2\pi}{T}A
        \approx 18{,}8\,\frac{\text{см}}{\text{с}}.
    $
}
\solutionspace{100pt}

\tasknumber{8}%
\task{%
    Определите период колебаний груза массой $m$, подвешенного к пружине жёсткостью $k$.
    Ускорение свободного падения $g$.
}
\answer{%
    \begin{align*}
    &-kx_0 + mg = 0, \\
    F &= -k(x_0 + \Delta x), \\
    ma &= -k(x_0 + \Delta x) + mg, \\
    ma &= -kx_0 -k \Delta x + mg = -mg -k \Delta x + mg = -k \Delta x, \\
    a &+ \frac k m x = 0, \\
    \omega^2 &= \frac k m \implies T = \frac{2\pi}\omega = 2\pi\sqrt{\frac m k}.
    \end{align*}
}
\solutionspace{150pt}

\tasknumber{9}%
\task{%
    Куб со стороной $a$ и плотности $\rho$ плавает в жидкости плотностью $\rho_0$.
    Определите частоту колебаний куба, считая, что 4 грани куба всегда вертикальны.
}
\answer{%
    \begin{align*}
    \Delta F &= -\rho_0 g a^2 \Delta x,\Delta F = m\ddot x, m = \rho a^3\implies \rho a^3 \ddot x = -\rho_0 g a^2 \Delta x \implies \\
    \implies \ddot x &= -\frac{\rho_0}{\rho} \frac g a \Delta x\implies \omega^2 = \frac{\rho_0}{\rho} \cdot \frac g a\implies T = \frac{2 \pi}{\omega} = 2 \pi\sqrt{\frac{\rho}{\rho_0} \cdot \frac a g}.
    \end{align*}
}
\solutionspace{150pt}

\tasknumber{10}%
\task{%
    Математический маятник с нитью длиной $40\,\text{см}$ подвешен к потолку в лифте.
    За $20\,\text{с}$ маятник совершил $14$ колебаний.
    Определите модуль и направление ускорения лифта.
    Куда движется лифт?
}
\answer{%
    $
        T = 2\pi\sqrt{\frac\ell {a + g}}, T = \frac {t}{N}
        \implies a + g = \ell \cdot \frac{4 \pi ^ 2}{T^2},
        a = \ell \cdot \frac{4 \pi ^ 2}{T^2} - g = \ell \cdot \frac{4 \pi ^ 2 N^2}{t^2} - g \approx -2{,}260\,\frac{\text{м}}{\text{c}^{2}},
        \text{вверх}.
    $
}
\solutionspace{120pt}

\tasknumber{11}%
\task{%
    Масса груза в пружинном маятнике равна $500\,\text{г}$, при этом период его колебаний равен $1{,}3\,\text{с}$.
    Груз облегчают на $50\,\text{г}$.
    Определите новый период колебаний маятника.
}
\answer{%
    $
        T'
            = 2\pi\sqrt{\frac{M - m}{k}}
            = 2\pi\sqrt{\frac{M}{k} \cdot \frac{M - m}{M}}
            = T\sqrt{\frac{M - m}{M}} =  T\sqrt{1 - \frac{m}{M}} \approx 1{,}23\,\text{с}.
    $
}
\solutionspace{120pt}

\tasknumber{12}%
\task{%
    При какой длине нити математического маятника период колебаний груза массой $400\,\text{г}$
    окажется равен периоду колебаний этого же груза в пружинном маятнике с пружиной жёсткостью $60\,\frac{\text{Н}}{\text{м}}$?
}
\answer{%
    $
        2\pi \sqrt{\frac \ell g} = 2\pi \sqrt{\frac m k}
        \implies \frac \ell g = \frac m k
        \implies \ell = g \frac m k \approx 6{,}7\,\text{см}.
    $
}
\solutionspace{120pt}

\tasknumber{13}%
\task{%
    Груз подвесили к пружине, при этом удлинение пружины составило $60\,\text{мм}$.
    Определите частоту колебаний пружинного маятника, собранного из этой пружины и этого груза.
}
\answer{%
    $
        mg -k\Delta x = 0 \implies \frac m k = \frac{\Delta x} g
        \implies T = 2\pi \sqrt{\frac m k } = 2\pi \sqrt{\frac{\Delta x} g } \approx $0{,}49\,\text{с}$,
        \nu = \frac 1T \approx $2{,}05\,\text{Гц}$.
    $
}
\solutionspace{120pt}

\tasknumber{14}%
\task{%
    Определите период колебаний системы: математический маятник ограничен с одной стороны стенкой (см.
    рис.
    на доске).
    Удары маятника о стенку абсолютно упругие, $n = 2$.
    Длина маятника $\ell$, ускорение свободного падения $g$.
}
\answer{%
    $
        T = 2\pi\sqrt{\frac\ell g}, \qquad
        T' = 2 \cdot \frac T 4 + 2 \cdot \frac T{12} = \frac23T = \frac43\pi\sqrt{\frac\ell g}
    $
}

\variantsplitter

\addpersonalvariant{Рената Таржиманова}

\tasknumber{1}%
\task{%
    \begin{itemize}
        \item Запишите линейное однородное дифференциальное уравнение второго порядка,
            описывающее свободные незатухающие колебания гармонического осциллятора,
        \item запишите общее решение этого уравнения,
        \item подпишите в выписанном решении фазу и амплитуду колебаний,
        \item запишите выражение для скорости,
        \item запишите выражение для ускорения.
    \end{itemize}
}
\answer{%
    \begin{align*}
    &\ddot x + \omega^2 x = 0 \Longleftrightarrow a_x + \omega^2 x = 0, \\
    &x = A \cos(\omega t + \varphi_0) \text{ или же } x = A \sin(\omega t + \varphi_0) \text{ или же } x = a \cos(\omega t) + b \sin(\omega t), \\
    &A \text{\, или \,} \sqrt{a^2 + b^2} \text{ --- это амплитуда}, \omega t + \varphi_0\text{ --- это фаза}, \\
    &v = \dot x = -\omega A \sin(\omega t + \varphi_0), \\
    &a = \dot v = \ddot x = -\omega^2 A \cos(\omega t + \varphi_0) = -\omega^2 x,
    \end{align*}
}
\solutionspace{135pt}

\tasknumber{2}%
\task{%
    Тело колеблется по гармоническому закону,
    амплитуда этих колебаний $12\,\text{см}$, период $6\,\text{c}$.
    Чему равно смещение тела относительно положения равновесия через $24\,\text{c}$
    после прохождения положения равновесия?
}
\answer{%
    $x = A \sin \omega t = A \sin \cbr{ \frac {2\pi}T t } = A \sin \cbr{ 2\pi \frac tT } = 12\,\text{см} \cdot \sin \cbr{ 2\pi \cdot \frac {24\,\text{c}}{6\,\text{c}}}\approx 0\,\text{см}.$
}
\solutionspace{120pt}

\tasknumber{3}%
\task{%
    Тело совершает гармонические колебания с периодом $6\,\text{c}$.
    За какое время тело смещается от положения наибольшего отклонения до смещения в половину амплитуды?
}
\answer{%
    $t = \frac T{6} \approx 1{,}0\,\text{c}.$
}
\solutionspace{120pt}

\tasknumber{4}%
\task{%
    Запишите формулу для периода колебаний пружинного маятника и ...
    \begin{itemize}
        \item укажите названия всех физических величин в формуле,
        \item выразите из неё частоту колебаний
        \item выразите из неё жёсткость пружины.
    \end{itemize}
}
\answer{%
    \begin{align*}
    T &= 2\pi \sqrt{\frac lg} \implies \nu = \frac 1T = \frac 1{2\pi}\sqrt{\frac gl}, \omega = 2\pi\nu = \sqrt{\frac gl}, l = g\sqr{\frac T{2\pi}}, g = l\sqr{\frac {2\pi}T} \\
    T &= 2\pi \sqrt{\frac mk} \implies \nu = \frac 1T = \frac 1{2\pi}\sqrt{\frac km}, \omega = 2\pi\nu = \sqrt{\frac km}, m = k\sqr{\frac T{2\pi}}, k = m\sqr{\frac {2\pi}T}
    \end{align*}
}
\solutionspace{120pt}

\tasknumber{5}%
\task{%
    Частота колебаний математического маятника равна $8\,\text{Гц}$.
    Определите периоды колебаний
    \begin{itemize}
        \item потенциальной энергии системы,
        \item ускорения груза,
        \item модуля ускорения груза.
    \end{itemize}
}
\answer{%
    $T = \frac 1\nu \approx 125{,}0\,\text{мc}, T_1 = \frac T2 \approx 62{,}5\,\text{мc}, T_2 = T \approx 125{,}0\,\text{мc}, T_3 = \frac T2 \approx 62{,}5\,\text{мc}.$
}
\solutionspace{80pt}

\tasknumber{6}%
\task{%
    Тело колеблется по гармоническому закону с амплитудой $12\,\text{см}$.
    Какой минимальный путь тело может пройти за одну шестую долю периода?
}
\answer{%
    $2 * A \frac{1}{12} = \frac{A:L:s}3 \approx 2{,}0\,\text{см}.$
}
\solutionspace{80pt}

\tasknumber{7}%
\task{%
    Период колебаний математического маятника равен $4\,\text{с}$,
    а их амплитуда — $20\,\text{см}$.
    Определите амплитуду колебаний скорости маятника.
}
\answer{%
    $
        T = \frac{2\pi}{\omega}
        \implies \omega = \frac{2\pi}{T}
        \implies v_{\max} = \omega A = \frac{2\pi}{T}A
        \approx 31{,}4\,\frac{\text{см}}{\text{с}}.
    $
}
\solutionspace{100pt}

\tasknumber{8}%
\task{%
    Определите период колебаний груза массой $m$, подвешенного к пружине жёсткостью $k$.
    Ускорение свободного падения $g$.
}
\answer{%
    \begin{align*}
    &-kx_0 + mg = 0, \\
    F &= -k(x_0 + \Delta x), \\
    ma &= -k(x_0 + \Delta x) + mg, \\
    ma &= -kx_0 -k \Delta x + mg = -mg -k \Delta x + mg = -k \Delta x, \\
    a &+ \frac k m x = 0, \\
    \omega^2 &= \frac k m \implies T = \frac{2\pi}\omega = 2\pi\sqrt{\frac m k}.
    \end{align*}
}
\solutionspace{150pt}

\tasknumber{9}%
\task{%
    Куб со стороной $a$ и плотности $\rho$ плавает в жидкости плотностью $\rho_0$.
    Определите частоту колебаний куба, считая, что 4 грани куба всегда вертикальны.
}
\answer{%
    \begin{align*}
    \Delta F &= -\rho_0 g a^2 \Delta x,\Delta F = m\ddot x, m = \rho a^3\implies \rho a^3 \ddot x = -\rho_0 g a^2 \Delta x \implies \\
    \implies \ddot x &= -\frac{\rho_0}{\rho} \frac g a \Delta x\implies \omega^2 = \frac{\rho_0}{\rho} \cdot \frac g a\implies T = \frac{2 \pi}{\omega} = 2 \pi\sqrt{\frac{\rho}{\rho_0} \cdot \frac a g}.
    \end{align*}
}
\solutionspace{150pt}

\tasknumber{10}%
\task{%
    Математический маятник с нитью длиной $40\,\text{см}$ подвешен к потолку в лифте.
    За $25\,\text{с}$ маятник совершил $19$ колебаний.
    Определите модуль и направление ускорения лифта.
    Куда движется лифт?
}
\answer{%
    $
        T = 2\pi\sqrt{\frac\ell {a + g}}, T = \frac {t}{N}
        \implies a + g = \ell \cdot \frac{4 \pi ^ 2}{T^2},
        a = \ell \cdot \frac{4 \pi ^ 2}{T^2} - g = \ell \cdot \frac{4 \pi ^ 2 N^2}{t^2} - g \approx -0{,}8800\,\frac{\text{м}}{\text{c}^{2}},
        \text{вверх}.
    $
}
\solutionspace{120pt}

\tasknumber{11}%
\task{%
    Масса груза в пружинном маятнике равна $600\,\text{г}$, при этом период его колебаний равен $1{,}5\,\text{с}$.
    Груз утяжеляют на $100\,\text{г}$.
    Определите новый период колебаний маятника.
}
\answer{%
    $
        T'
            = 2\pi\sqrt{\frac{M + m}{k}}
            = 2\pi\sqrt{\frac{M}{k} \cdot \frac{M + m}{M}}
            = T\sqrt{\frac{M + m}{M}} =  T\sqrt{1 + \frac{m}{M}} \approx 1{,}62\,\text{с}.
    $
}
\solutionspace{120pt}

\tasknumber{12}%
\task{%
    При какой длине нити математического маятника период колебаний груза массой $300\,\text{г}$
    окажется равен периоду колебаний этого же груза в пружинном маятнике с пружиной жёсткостью $50\,\frac{\text{Н}}{\text{м}}$?
}
\answer{%
    $
        2\pi \sqrt{\frac \ell g} = 2\pi \sqrt{\frac m k}
        \implies \frac \ell g = \frac m k
        \implies \ell = g \frac m k \approx 6{,}0\,\text{см}.
    $
}
\solutionspace{120pt}

\tasknumber{13}%
\task{%
    Груз подвесили к пружине, при этом удлинение пружины составило $45\,\text{мм}$.
    Определите частоту колебаний пружинного маятника, собранного из этой пружины и этого груза.
}
\answer{%
    $
        mg -k\Delta x = 0 \implies \frac m k = \frac{\Delta x} g
        \implies T = 2\pi \sqrt{\frac m k } = 2\pi \sqrt{\frac{\Delta x} g } \approx $0{,}42\,\text{с}$,
        \nu = \frac 1T \approx $2{,}37\,\text{Гц}$.
    $
}
\solutionspace{120pt}

\tasknumber{14}%
\task{%
    Определите период колебаний системы: математический маятник ограничен с одной стороны стенкой (см.
    рис.
    на доске).
    Удары маятника о стенку абсолютно упругие, $n = \frac{2}{\sqrt{3}}$.
    Длина маятника $\ell$, ускорение свободного падения $g$.
}
\answer{%
    $
        T = 2\pi\sqrt{\frac\ell g}, \qquad
        T' = 2 \cdot \frac T 4 + 2 \cdot \frac T{6} = \frac56T = \frac53\pi\sqrt{\frac\ell g}
    $
}

\variantsplitter

\addpersonalvariant{Андрей Щербаков}

\tasknumber{1}%
\task{%
    \begin{itemize}
        \item Запишите линейное однородное дифференциальное уравнение второго порядка,
            описывающее свободные незатухающие колебания гармонического осциллятора,
        \item запишите общее решение этого уравнения,
        \item подпишите в выписанном решении фазу и амплитуду колебаний,
        \item запишите выражение для скорости,
        \item запишите выражение для ускорения.
    \end{itemize}
}
\answer{%
    \begin{align*}
    &\ddot x + \omega^2 x = 0 \Longleftrightarrow a_x + \omega^2 x = 0, \\
    &x = A \cos(\omega t + \varphi_0) \text{ или же } x = A \sin(\omega t + \varphi_0) \text{ или же } x = a \cos(\omega t) + b \sin(\omega t), \\
    &A \text{\, или \,} \sqrt{a^2 + b^2} \text{ --- это амплитуда}, \omega t + \varphi_0\text{ --- это фаза}, \\
    &v = \dot x = -\omega A \sin(\omega t + \varphi_0), \\
    &a = \dot v = \ddot x = -\omega^2 A \cos(\omega t + \varphi_0) = -\omega^2 x,
    \end{align*}
}
\solutionspace{135pt}

\tasknumber{2}%
\task{%
    Тело колеблется по гармоническому закону,
    амплитуда этих колебаний $14\,\text{см}$, период $2\,\text{c}$.
    Чему равно смещение тела относительно положения равновесия через $26\,\text{c}$
    после прохождения положения максимального отклонения?
}
\answer{%
    $x = A \cos \omega t = A \cos \cbr{ \frac {2\pi}T t } = A \cos \cbr{ 2\pi \frac tT } = 14\,\text{см} \cdot \cos \cbr{ 2\pi \cdot \frac {26\,\text{c}}{2\,\text{c}}}\approx 14{,}0\,\text{см}.$
}
\solutionspace{120pt}

\tasknumber{3}%
\task{%
    Тело совершает гармонические колебания с периодом $4\,\text{c}$.
    За какое время тело смещается от положения равновесия до смещения в половину амплитуды?
}
\answer{%
    $t = \frac T{12} \approx 0{,}3\,\text{c}.$
}
\solutionspace{120pt}

\tasknumber{4}%
\task{%
    Запишите формулу для периода колебаний пружинного маятника и ...
    \begin{itemize}
        \item укажите названия всех физических величин в формуле,
        \item выразите из неё циклическую частоту колебаний
        \item выразите из неё жёсткость пружины.
    \end{itemize}
}
\answer{%
    \begin{align*}
    T &= 2\pi \sqrt{\frac lg} \implies \nu = \frac 1T = \frac 1{2\pi}\sqrt{\frac gl}, \omega = 2\pi\nu = \sqrt{\frac gl}, l = g\sqr{\frac T{2\pi}}, g = l\sqr{\frac {2\pi}T} \\
    T &= 2\pi \sqrt{\frac mk} \implies \nu = \frac 1T = \frac 1{2\pi}\sqrt{\frac km}, \omega = 2\pi\nu = \sqrt{\frac km}, m = k\sqr{\frac T{2\pi}}, k = m\sqr{\frac {2\pi}T}
    \end{align*}
}
\solutionspace{120pt}

\tasknumber{5}%
\task{%
    Частота колебаний математического маятника равна $15\,\text{Гц}$.
    Определите периоды колебаний
    \begin{itemize}
        \item потенциальной энергии системы,
        \item скорости груза,
        \item модуля ускорения груза.
    \end{itemize}
}
\answer{%
    $T = \frac 1\nu \approx 66{,}7\,\text{мc}, T_1 = \frac T2 \approx 33{,}3\,\text{мc}, T_2 = T \approx 66{,}7\,\text{мc}, T_3 = \frac T2 \approx 33{,}3\,\text{мc}.$
}
\solutionspace{80pt}

\tasknumber{6}%
\task{%
    Тело колеблется по гармоническому закону с амплитудой $12\,\text{см}$.
    Какой максимальный путь тело может пройти за одну шестую долю периода?
}
\answer{%
    $2 * A \frac{1}{2} = A \approx 12{,}0\,\text{см}.$
}
\solutionspace{80pt}

\tasknumber{7}%
\task{%
    Период колебаний математического маятника равен $3\,\text{с}$,
    а их амплитуда — $10\,\text{см}$.
    Определите максимальную скорость маятника.
}
\answer{%
    $
        T = \frac{2\pi}{\omega}
        \implies \omega = \frac{2\pi}{T}
        \implies v_{\max} = \omega A = \frac{2\pi}{T}A
        \approx 20{,}9\,\frac{\text{см}}{\text{с}}.
    $
}
\solutionspace{100pt}

\tasknumber{8}%
\task{%
    Определите период колебаний груза массой $m$, подвешенного к пружине жёсткостью $k$.
    Ускорение свободного падения $g$.
}
\answer{%
    \begin{align*}
    &-kx_0 + mg = 0, \\
    F &= -k(x_0 + \Delta x), \\
    ma &= -k(x_0 + \Delta x) + mg, \\
    ma &= -kx_0 -k \Delta x + mg = -mg -k \Delta x + mg = -k \Delta x, \\
    a &+ \frac k m x = 0, \\
    \omega^2 &= \frac k m \implies T = \frac{2\pi}\omega = 2\pi\sqrt{\frac m k}.
    \end{align*}
}
\solutionspace{150pt}

\tasknumber{9}%
\task{%
    Куб со стороной $a$ и плотности $\rho$ плавает в жидкости плотностью $\rho_0$.
    Определите частоту колебаний куба, считая, что 4 грани куба всегда вертикальны.
}
\answer{%
    \begin{align*}
    \Delta F &= -\rho_0 g a^2 \Delta x,\Delta F = m\ddot x, m = \rho a^3\implies \rho a^3 \ddot x = -\rho_0 g a^2 \Delta x \implies \\
    \implies \ddot x &= -\frac{\rho_0}{\rho} \frac g a \Delta x\implies \omega^2 = \frac{\rho_0}{\rho} \cdot \frac g a\implies T = \frac{2 \pi}{\omega} = 2 \pi\sqrt{\frac{\rho}{\rho_0} \cdot \frac a g}.
    \end{align*}
}
\solutionspace{150pt}

\tasknumber{10}%
\task{%
    Математический маятник с нитью длиной $40\,\text{см}$ подвешен к потолку в лифте.
    За $25\,\text{с}$ маятник совершил $18$ колебаний.
    Определите модуль и направление ускорения лифта.
    Куда движется лифт?
}
\answer{%
    $
        T = 2\pi\sqrt{\frac\ell {a + g}}, T = \frac {t}{N}
        \implies a + g = \ell \cdot \frac{4 \pi ^ 2}{T^2},
        a = \ell \cdot \frac{4 \pi ^ 2}{T^2} - g = \ell \cdot \frac{4 \pi ^ 2 N^2}{t^2} - g \approx -1{,}8100\,\frac{\text{м}}{\text{c}^{2}},
        \text{вверх}.
    $
}
\solutionspace{120pt}

\tasknumber{11}%
\task{%
    Масса груза в пружинном маятнике равна $400\,\text{г}$, при этом период его колебаний равен $1{,}3\,\text{с}$.
    Груз утяжеляют на $50\,\text{г}$.
    Определите новый период колебаний маятника.
}
\answer{%
    $
        T'
            = 2\pi\sqrt{\frac{M + m}{k}}
            = 2\pi\sqrt{\frac{M}{k} \cdot \frac{M + m}{M}}
            = T\sqrt{\frac{M + m}{M}} =  T\sqrt{1 + \frac{m}{M}} \approx 1{,}38\,\text{с}.
    $
}
\solutionspace{120pt}

\tasknumber{12}%
\task{%
    При какой длине нити математического маятника период колебаний груза массой $300\,\text{г}$
    окажется равен периоду колебаний этого же груза в пружинном маятнике с пружиной жёсткостью $50\,\frac{\text{Н}}{\text{м}}$?
}
\answer{%
    $
        2\pi \sqrt{\frac \ell g} = 2\pi \sqrt{\frac m k}
        \implies \frac \ell g = \frac m k
        \implies \ell = g \frac m k \approx 6{,}0\,\text{см}.
    $
}
\solutionspace{120pt}

\tasknumber{13}%
\task{%
    Груз подвесили к пружине, при этом удлинение пружины составило $30\,\text{мм}$.
    Определите частоту колебаний пружинного маятника, собранного из этой пружины и этого груза.
}
\answer{%
    $
        mg -k\Delta x = 0 \implies \frac m k = \frac{\Delta x} g
        \implies T = 2\pi \sqrt{\frac m k } = 2\pi \sqrt{\frac{\Delta x} g } \approx $0{,}34\,\text{с}$,
        \nu = \frac 1T \approx $2{,}91\,\text{Гц}$.
    $
}
\solutionspace{120pt}

\tasknumber{14}%
\task{%
    Определите период колебаний системы: математический маятник ограничен с одной стороны стенкой (см.
    рис.
    на доске).
    Удары маятника о стенку абсолютно упругие, $n = \sqrt{2}$.
    Длина маятника $\ell$, ускорение свободного падения $g$.
}
\answer{%
    $
        T = 2\pi\sqrt{\frac\ell g}, \qquad
        T' = 2 \cdot \frac T 4 + 2 \cdot \frac T{8} = \frac34T = \frac32\pi\sqrt{\frac\ell g}
    $
}

\variantsplitter

\addpersonalvariant{Михаил Ярошевский}

\tasknumber{1}%
\task{%
    \begin{itemize}
        \item Запишите линейное однородное дифференциальное уравнение второго порядка,
            описывающее свободные незатухающие колебания гармонического осциллятора,
        \item запишите общее решение этого уравнения,
        \item подпишите в выписанном решении фазу и амплитуду колебаний,
        \item запишите выражение для скорости,
        \item запишите выражение для ускорения.
    \end{itemize}
}
\answer{%
    \begin{align*}
    &\ddot x + \omega^2 x = 0 \Longleftrightarrow a_x + \omega^2 x = 0, \\
    &x = A \cos(\omega t + \varphi_0) \text{ или же } x = A \sin(\omega t + \varphi_0) \text{ или же } x = a \cos(\omega t) + b \sin(\omega t), \\
    &A \text{\, или \,} \sqrt{a^2 + b^2} \text{ --- это амплитуда}, \omega t + \varphi_0\text{ --- это фаза}, \\
    &v = \dot x = -\omega A \sin(\omega t + \varphi_0), \\
    &a = \dot v = \ddot x = -\omega^2 A \cos(\omega t + \varphi_0) = -\omega^2 x,
    \end{align*}
}
\solutionspace{135pt}

\tasknumber{2}%
\task{%
    Тело колеблется по гармоническому закону,
    амплитуда этих колебаний $10\,\text{см}$, период $6\,\text{c}$.
    Чему равно смещение тела относительно положения равновесия через $25\,\text{c}$
    после прохождения положения максимального отклонения?
}
\answer{%
    $x = A \cos \omega t = A \cos \cbr{ \frac {2\pi}T t } = A \cos \cbr{ 2\pi \frac tT } = 10\,\text{см} \cdot \cos \cbr{ 2\pi \cdot \frac {25\,\text{c}}{6\,\text{c}}}\approx 5{,}0\,\text{см}.$
}
\solutionspace{120pt}

\tasknumber{3}%
\task{%
    Тело совершает гармонические колебания с периодом $5\,\text{c}$.
    За какое время тело смещается от положения наибольшего отклонения до смещения в половину амплитуды?
}
\answer{%
    $t = \frac T{6} \approx 0{,}8\,\text{c}.$
}
\solutionspace{120pt}

\tasknumber{4}%
\task{%
    Запишите формулу для периода колебаний математического маятника и ...
    \begin{itemize}
        \item укажите названия всех физических величин в формуле,
        \item выразите из неё циклическую частоту колебаний
        \item выразите из неё ускорение свободного падения.
    \end{itemize}
}
\answer{%
    \begin{align*}
    T &= 2\pi \sqrt{\frac lg} \implies \nu = \frac 1T = \frac 1{2\pi}\sqrt{\frac gl}, \omega = 2\pi\nu = \sqrt{\frac gl}, l = g\sqr{\frac T{2\pi}}, g = l\sqr{\frac {2\pi}T} \\
    T &= 2\pi \sqrt{\frac mk} \implies \nu = \frac 1T = \frac 1{2\pi}\sqrt{\frac km}, \omega = 2\pi\nu = \sqrt{\frac km}, m = k\sqr{\frac T{2\pi}}, k = m\sqr{\frac {2\pi}T}
    \end{align*}
}
\solutionspace{120pt}

\tasknumber{5}%
\task{%
    Частота колебаний пружинного маятника равна $15\,\text{Гц}$.
    Определите периоды колебаний
    \begin{itemize}
        \item потенциальной энергии системы,
        \item скорости груза,
        \item модуля скорости груза.
    \end{itemize}
}
\answer{%
    $T = \frac 1\nu \approx 66{,}7\,\text{мc}, T_1 = \frac T2 \approx 33{,}3\,\text{мc}, T_2 = T \approx 66{,}7\,\text{мc}, T_3 = \frac T2 \approx 33{,}3\,\text{мc}.$
}
\solutionspace{80pt}

\tasknumber{6}%
\task{%
    Тело колеблется по гармоническому закону с амплитудой $6\,\text{см}$.
    Какой минимальный путь тело может пройти за одну шестую долю периода?
}
\answer{%
    $2 * A \frac{1}{12} = \frac{A:L:s}3 \approx 1{,}0\,\text{см}.$
}
\solutionspace{80pt}

\tasknumber{7}%
\task{%
    Период колебаний математического маятника равен $4\,\text{с}$,
    а их амплитуда — $10\,\text{см}$.
    Определите максимальную скорость маятника.
}
\answer{%
    $
        T = \frac{2\pi}{\omega}
        \implies \omega = \frac{2\pi}{T}
        \implies v_{\max} = \omega A = \frac{2\pi}{T}A
        \approx 15{,}7\,\frac{\text{см}}{\text{с}}.
    $
}
\solutionspace{100pt}

\tasknumber{8}%
\task{%
    Определите период колебаний груза массой $m$, подвешенного к пружине жёсткостью $k$.
    Ускорение свободного падения $g$.
}
\answer{%
    \begin{align*}
    &-kx_0 + mg = 0, \\
    F &= -k(x_0 + \Delta x), \\
    ma &= -k(x_0 + \Delta x) + mg, \\
    ma &= -kx_0 -k \Delta x + mg = -mg -k \Delta x + mg = -k \Delta x, \\
    a &+ \frac k m x = 0, \\
    \omega^2 &= \frac k m \implies T = \frac{2\pi}\omega = 2\pi\sqrt{\frac m k}.
    \end{align*}
}
\solutionspace{150pt}

\tasknumber{9}%
\task{%
    Куб со стороной $a$ и плотности $\rho$ плавает в жидкости плотностью $\rho_0$.
    Определите частоту колебаний куба, считая, что 4 грани куба всегда вертикальны.
}
\answer{%
    \begin{align*}
    \Delta F &= -\rho_0 g a^2 \Delta x,\Delta F = m\ddot x, m = \rho a^3\implies \rho a^3 \ddot x = -\rho_0 g a^2 \Delta x \implies \\
    \implies \ddot x &= -\frac{\rho_0}{\rho} \frac g a \Delta x\implies \omega^2 = \frac{\rho_0}{\rho} \cdot \frac g a\implies T = \frac{2 \pi}{\omega} = 2 \pi\sqrt{\frac{\rho}{\rho_0} \cdot \frac a g}.
    \end{align*}
}
\solutionspace{150pt}

\tasknumber{10}%
\task{%
    Математический маятник с нитью длиной $40\,\text{см}$ подвешен к потолку в лифте.
    За $30\,\text{с}$ маятник совершил $21$ колебаний.
    Определите модуль и направление ускорения лифта.
    Куда движется лифт?
}
\answer{%
    $
        T = 2\pi\sqrt{\frac\ell {a + g}}, T = \frac {t}{N}
        \implies a + g = \ell \cdot \frac{4 \pi ^ 2}{T^2},
        a = \ell \cdot \frac{4 \pi ^ 2}{T^2} - g = \ell \cdot \frac{4 \pi ^ 2 N^2}{t^2} - g \approx -2{,}260\,\frac{\text{м}}{\text{c}^{2}},
        \text{вверх}.
    $
}
\solutionspace{120pt}

\tasknumber{11}%
\task{%
    Масса груза в пружинном маятнике равна $400\,\text{г}$, при этом период его колебаний равен $1{,}3\,\text{с}$.
    Груз утяжеляют на $100\,\text{г}$.
    Определите новый период колебаний маятника.
}
\answer{%
    $
        T'
            = 2\pi\sqrt{\frac{M + m}{k}}
            = 2\pi\sqrt{\frac{M}{k} \cdot \frac{M + m}{M}}
            = T\sqrt{\frac{M + m}{M}} =  T\sqrt{1 + \frac{m}{M}} \approx 1{,}45\,\text{с}.
    $
}
\solutionspace{120pt}

\tasknumber{12}%
\task{%
    При какой длине нити математического маятника период колебаний груза массой $400\,\text{г}$
    окажется равен периоду колебаний этого же груза в пружинном маятнике с пружиной жёсткостью $60\,\frac{\text{Н}}{\text{м}}$?
}
\answer{%
    $
        2\pi \sqrt{\frac \ell g} = 2\pi \sqrt{\frac m k}
        \implies \frac \ell g = \frac m k
        \implies \ell = g \frac m k \approx 6{,}7\,\text{см}.
    $
}
\solutionspace{120pt}

\tasknumber{13}%
\task{%
    Груз подвесили к пружине, при этом удлинение пружины составило $45\,\text{мм}$.
    Определите частоту колебаний пружинного маятника, собранного из этой пружины и этого груза.
}
\answer{%
    $
        mg -k\Delta x = 0 \implies \frac m k = \frac{\Delta x} g
        \implies T = 2\pi \sqrt{\frac m k } = 2\pi \sqrt{\frac{\Delta x} g } \approx $0{,}42\,\text{с}$,
        \nu = \frac 1T \approx $2{,}37\,\text{Гц}$.
    $
}
\solutionspace{120pt}

\tasknumber{14}%
\task{%
    Определите период колебаний системы: математический маятник ограничен с одной стороны стенкой (см.
    рис.
    на доске).
    Удары маятника о стенку абсолютно упругие, $n = \frac{2}{\sqrt{3}}$.
    Длина маятника $\ell$, ускорение свободного падения $g$.
}
\answer{%
    $
        T = 2\pi\sqrt{\frac\ell g}, \qquad
        T' = 2 \cdot \frac T 4 + 2 \cdot \frac T{6} = \frac56T = \frac53\pi\sqrt{\frac\ell g}
    $
}

\variantsplitter

\addpersonalvariant{Алексей Алимпиев}

\tasknumber{1}%
\task{%
    \begin{itemize}
        \item Запишите линейное однородное дифференциальное уравнение второго порядка,
            описывающее свободные незатухающие колебания гармонического осциллятора,
        \item запишите общее решение этого уравнения,
        \item подпишите в выписанном решении фазу и амплитуду колебаний,
        \item запишите выражение для скорости,
        \item запишите выражение для ускорения.
    \end{itemize}
}
\answer{%
    \begin{align*}
    &\ddot x + \omega^2 x = 0 \Longleftrightarrow a_x + \omega^2 x = 0, \\
    &x = A \cos(\omega t + \varphi_0) \text{ или же } x = A \sin(\omega t + \varphi_0) \text{ или же } x = a \cos(\omega t) + b \sin(\omega t), \\
    &A \text{\, или \,} \sqrt{a^2 + b^2} \text{ --- это амплитуда}, \omega t + \varphi_0\text{ --- это фаза}, \\
    &v = \dot x = -\omega A \sin(\omega t + \varphi_0), \\
    &a = \dot v = \ddot x = -\omega^2 A \cos(\omega t + \varphi_0) = -\omega^2 x,
    \end{align*}
}
\solutionspace{135pt}

\tasknumber{2}%
\task{%
    Тело колеблется по гармоническому закону,
    амплитуда этих колебаний $20\,\text{см}$, период $6\,\text{c}$.
    Чему равно смещение тела относительно положения равновесия через $21\,\text{c}$
    после прохождения положения максимального отклонения?
}
\answer{%
    $x = A \cos \omega t = A \cos \cbr{ \frac {2\pi}T t } = A \cos \cbr{ 2\pi \frac tT } = 20\,\text{см} \cdot \cos \cbr{ 2\pi \cdot \frac {21\,\text{c}}{6\,\text{c}}}\approx -20{,}00\,\text{см}.$
}
\solutionspace{120pt}

\tasknumber{3}%
\task{%
    Тело совершает гармонические колебания с периодом $6\,\text{c}$.
    За какое время тело смещается от положения равновесия до смещения в половину амплитуды?
}
\answer{%
    $t = \frac T{12} \approx 0{,}5\,\text{c}.$
}
\solutionspace{120pt}

\tasknumber{4}%
\task{%
    Запишите формулу для периода колебаний пружинного маятника и ...
    \begin{itemize}
        \item укажите названия всех физических величин в формуле,
        \item выразите из неё частоту колебаний
        \item выразите из неё массу груза.
    \end{itemize}
}
\answer{%
    \begin{align*}
    T &= 2\pi \sqrt{\frac lg} \implies \nu = \frac 1T = \frac 1{2\pi}\sqrt{\frac gl}, \omega = 2\pi\nu = \sqrt{\frac gl}, l = g\sqr{\frac T{2\pi}}, g = l\sqr{\frac {2\pi}T} \\
    T &= 2\pi \sqrt{\frac mk} \implies \nu = \frac 1T = \frac 1{2\pi}\sqrt{\frac km}, \omega = 2\pi\nu = \sqrt{\frac km}, m = k\sqr{\frac T{2\pi}}, k = m\sqr{\frac {2\pi}T}
    \end{align*}
}
\solutionspace{120pt}

\tasknumber{5}%
\task{%
    Частота колебаний пружинного маятника равна $15\,\text{Гц}$.
    Определите периоды колебаний
    \begin{itemize}
        \item кинетической энергии системы,
        \item скорости груза,
        \item модуля ускорения груза.
    \end{itemize}
}
\answer{%
    $T = \frac 1\nu \approx 66{,}7\,\text{мc}, T_1 = \frac T2 \approx 33{,}3\,\text{мc}, T_2 = T \approx 66{,}7\,\text{мc}, T_3 = \frac T2 \approx 33{,}3\,\text{мc}.$
}
\solutionspace{80pt}

\tasknumber{6}%
\task{%
    Тело колеблется по гармоническому закону с амплитудой $12\,\text{см}$.
    Какой максимальный путь тело может пройти за половину периода?
}
\answer{%
    $2A \approx 24{,}0\,\text{см}.$
}
\solutionspace{80pt}

\tasknumber{7}%
\task{%
    Период колебаний математического маятника равен $2\,\text{с}$,
    а их амплитуда — $10\,\text{см}$.
    Определите максимальную скорость маятника.
}
\answer{%
    $
        T = \frac{2\pi}{\omega}
        \implies \omega = \frac{2\pi}{T}
        \implies v_{\max} = \omega A = \frac{2\pi}{T}A
        \approx 31{,}4\,\frac{\text{см}}{\text{с}}.
    $
}
\solutionspace{100pt}

\tasknumber{8}%
\task{%
    Определите период колебаний груза массой $m$, подвешенного к пружине жёсткостью $k$.
    Ускорение свободного падения $g$.
}
\answer{%
    \begin{align*}
    &-kx_0 + mg = 0, \\
    F &= -k(x_0 + \Delta x), \\
    ma &= -k(x_0 + \Delta x) + mg, \\
    ma &= -kx_0 -k \Delta x + mg = -mg -k \Delta x + mg = -k \Delta x, \\
    a &+ \frac k m x = 0, \\
    \omega^2 &= \frac k m \implies T = \frac{2\pi}\omega = 2\pi\sqrt{\frac m k}.
    \end{align*}
}
\solutionspace{150pt}

\tasknumber{9}%
\task{%
    Куб со стороной $a$ и плотности $\rho$ плавает в жидкости плотностью $\rho_0$.
    Определите частоту колебаний куба, считая, что 4 грани куба всегда вертикальны.
}
\answer{%
    \begin{align*}
    \Delta F &= -\rho_0 g a^2 \Delta x,\Delta F = m\ddot x, m = \rho a^3\implies \rho a^3 \ddot x = -\rho_0 g a^2 \Delta x \implies \\
    \implies \ddot x &= -\frac{\rho_0}{\rho} \frac g a \Delta x\implies \omega^2 = \frac{\rho_0}{\rho} \cdot \frac g a\implies T = \frac{2 \pi}{\omega} = 2 \pi\sqrt{\frac{\rho}{\rho_0} \cdot \frac a g}.
    \end{align*}
}
\solutionspace{150pt}

\tasknumber{10}%
\task{%
    Математический маятник с нитью длиной $37\,\text{см}$ подвешен к потолку в лифте.
    За $30\,\text{с}$ маятник совершил $24$ колебаний.
    Определите модуль и направление ускорения лифта.
    Куда движется лифт?
}
\answer{%
    $
        T = 2\pi\sqrt{\frac\ell {a + g}}, T = \frac {t}{N}
        \implies a + g = \ell \cdot \frac{4 \pi ^ 2}{T^2},
        a = \ell \cdot \frac{4 \pi ^ 2}{T^2} - g = \ell \cdot \frac{4 \pi ^ 2 N^2}{t^2} - g \approx -0{,}6500\,\frac{\text{м}}{\text{c}^{2}},
        \text{вверх}.
    $
}
\solutionspace{120pt}

\tasknumber{11}%
\task{%
    Масса груза в пружинном маятнике равна $600\,\text{г}$, при этом период его колебаний равен $1{,}3\,\text{с}$.
    Груз утяжеляют на $50\,\text{г}$.
    Определите новый период колебаний маятника.
}
\answer{%
    $
        T'
            = 2\pi\sqrt{\frac{M + m}{k}}
            = 2\pi\sqrt{\frac{M}{k} \cdot \frac{M + m}{M}}
            = T\sqrt{\frac{M + m}{M}} =  T\sqrt{1 + \frac{m}{M}} \approx 1{,}35\,\text{с}.
    $
}
\solutionspace{120pt}

\tasknumber{12}%
\task{%
    При какой длине нити математического маятника период колебаний груза массой $200\,\text{г}$
    окажется равен периоду колебаний этого же груза в пружинном маятнике с пружиной жёсткостью $50\,\frac{\text{Н}}{\text{м}}$?
}
\answer{%
    $
        2\pi \sqrt{\frac \ell g} = 2\pi \sqrt{\frac m k}
        \implies \frac \ell g = \frac m k
        \implies \ell = g \frac m k \approx 4{,}0\,\text{см}.
    $
}
\solutionspace{120pt}

\tasknumber{13}%
\task{%
    Груз подвесили к пружине, при этом удлинение пружины составило $60\,\text{мм}$.
    Определите частоту колебаний пружинного маятника, собранного из этой пружины и этого груза.
}
\answer{%
    $
        mg -k\Delta x = 0 \implies \frac m k = \frac{\Delta x} g
        \implies T = 2\pi \sqrt{\frac m k } = 2\pi \sqrt{\frac{\Delta x} g } \approx $0{,}49\,\text{с}$,
        \nu = \frac 1T \approx $2{,}05\,\text{Гц}$.
    $
}
\solutionspace{120pt}

\tasknumber{14}%
\task{%
    Определите период колебаний системы: математический маятник ограничен с одной стороны стенкой (см.
    рис.
    на доске).
    Удары маятника о стенку абсолютно упругие, $n = \frac{2}{\sqrt{3}}$.
    Длина маятника $\ell$, ускорение свободного падения $g$.
}
\answer{%
    $
        T = 2\pi\sqrt{\frac\ell g}, \qquad
        T' = 2 \cdot \frac T 4 + 2 \cdot \frac T{6} = \frac56T = \frac53\pi\sqrt{\frac\ell g}
    $
}

\variantsplitter

\addpersonalvariant{Евгений Васин}

\tasknumber{1}%
\task{%
    \begin{itemize}
        \item Запишите линейное однородное дифференциальное уравнение второго порядка,
            описывающее свободные незатухающие колебания гармонического осциллятора,
        \item запишите общее решение этого уравнения,
        \item подпишите в выписанном решении фазу и амплитуду колебаний,
        \item запишите выражение для скорости,
        \item запишите выражение для ускорения.
    \end{itemize}
}
\answer{%
    \begin{align*}
    &\ddot x + \omega^2 x = 0 \Longleftrightarrow a_x + \omega^2 x = 0, \\
    &x = A \cos(\omega t + \varphi_0) \text{ или же } x = A \sin(\omega t + \varphi_0) \text{ или же } x = a \cos(\omega t) + b \sin(\omega t), \\
    &A \text{\, или \,} \sqrt{a^2 + b^2} \text{ --- это амплитуда}, \omega t + \varphi_0\text{ --- это фаза}, \\
    &v = \dot x = -\omega A \sin(\omega t + \varphi_0), \\
    &a = \dot v = \ddot x = -\omega^2 A \cos(\omega t + \varphi_0) = -\omega^2 x,
    \end{align*}
}
\solutionspace{135pt}

\tasknumber{2}%
\task{%
    Тело колеблется по гармоническому закону,
    амплитуда этих колебаний $10\,\text{см}$, период $2\,\text{c}$.
    Чему равно смещение тела относительно положения равновесия через $25\,\text{c}$
    после прохождения положения равновесия?
}
\answer{%
    $x = A \sin \omega t = A \sin \cbr{ \frac {2\pi}T t } = A \sin \cbr{ 2\pi \frac tT } = 10\,\text{см} \cdot \sin \cbr{ 2\pi \cdot \frac {25\,\text{c}}{2\,\text{c}}}\approx 0\,\text{см}.$
}
\solutionspace{120pt}

\tasknumber{3}%
\task{%
    Тело совершает гармонические колебания с периодом $5\,\text{c}$.
    За какое время тело смещается от положения наибольшего отклонения до смещения в половину амплитуды?
}
\answer{%
    $t = \frac T{6} \approx 0{,}8\,\text{c}.$
}
\solutionspace{120pt}

\tasknumber{4}%
\task{%
    Запишите формулу для периода колебаний пружинного маятника и ...
    \begin{itemize}
        \item укажите названия всех физических величин в формуле,
        \item выразите из неё циклическую частоту колебаний
        \item выразите из неё жёсткость пружины.
    \end{itemize}
}
\answer{%
    \begin{align*}
    T &= 2\pi \sqrt{\frac lg} \implies \nu = \frac 1T = \frac 1{2\pi}\sqrt{\frac gl}, \omega = 2\pi\nu = \sqrt{\frac gl}, l = g\sqr{\frac T{2\pi}}, g = l\sqr{\frac {2\pi}T} \\
    T &= 2\pi \sqrt{\frac mk} \implies \nu = \frac 1T = \frac 1{2\pi}\sqrt{\frac km}, \omega = 2\pi\nu = \sqrt{\frac km}, m = k\sqr{\frac T{2\pi}}, k = m\sqr{\frac {2\pi}T}
    \end{align*}
}
\solutionspace{120pt}

\tasknumber{5}%
\task{%
    Частота колебаний математического маятника равна $10\,\text{Гц}$.
    Определите периоды колебаний
    \begin{itemize}
        \item потенциальной энергии системы,
        \item скорости груза,
        \item модуля ускорения груза.
    \end{itemize}
}
\answer{%
    $T = \frac 1\nu \approx 100{,}0\,\text{мc}, T_1 = \frac T2 \approx 50{,}0\,\text{мc}, T_2 = T \approx 100{,}0\,\text{мc}, T_3 = \frac T2 \approx 50{,}0\,\text{мc}.$
}
\solutionspace{80pt}

\tasknumber{6}%
\task{%
    Тело колеблется по гармоническому закону с амплитудой $4\,\text{см}$.
    Какой минимальный путь тело может пройти за половину периода?
}
\answer{%
    $2A \approx 8{,}0\,\text{см}.$
}
\solutionspace{80pt}

\tasknumber{7}%
\task{%
    Период колебаний математического маятника равен $4\,\text{с}$,
    а их амплитуда — $20\,\text{см}$.
    Определите амплитуду колебаний скорости маятника.
}
\answer{%
    $
        T = \frac{2\pi}{\omega}
        \implies \omega = \frac{2\pi}{T}
        \implies v_{\max} = \omega A = \frac{2\pi}{T}A
        \approx 31{,}4\,\frac{\text{см}}{\text{с}}.
    $
}
\solutionspace{100pt}

\tasknumber{8}%
\task{%
    Определите период колебаний груза массой $m$, подвешенного к пружине жёсткостью $k$.
    Ускорение свободного падения $g$.
}
\answer{%
    \begin{align*}
    &-kx_0 + mg = 0, \\
    F &= -k(x_0 + \Delta x), \\
    ma &= -k(x_0 + \Delta x) + mg, \\
    ma &= -kx_0 -k \Delta x + mg = -mg -k \Delta x + mg = -k \Delta x, \\
    a &+ \frac k m x = 0, \\
    \omega^2 &= \frac k m \implies T = \frac{2\pi}\omega = 2\pi\sqrt{\frac m k}.
    \end{align*}
}
\solutionspace{150pt}

\tasknumber{9}%
\task{%
    Куб со стороной $a$ и плотности $\rho$ плавает в жидкости плотностью $\rho_0$.
    Определите частоту колебаний куба, считая, что 4 грани куба всегда вертикальны.
}
\answer{%
    \begin{align*}
    \Delta F &= -\rho_0 g a^2 \Delta x,\Delta F = m\ddot x, m = \rho a^3\implies \rho a^3 \ddot x = -\rho_0 g a^2 \Delta x \implies \\
    \implies \ddot x &= -\frac{\rho_0}{\rho} \frac g a \Delta x\implies \omega^2 = \frac{\rho_0}{\rho} \cdot \frac g a\implies T = \frac{2 \pi}{\omega} = 2 \pi\sqrt{\frac{\rho}{\rho_0} \cdot \frac a g}.
    \end{align*}
}
\solutionspace{150pt}

\tasknumber{10}%
\task{%
    Математический маятник с нитью длиной $40\,\text{см}$ подвешен к потолку в лифте.
    За $25\,\text{с}$ маятник совершил $22$ колебаний.
    Определите модуль и направление ускорения лифта.
    Куда движется лифт?
}
\answer{%
    $
        T = 2\pi\sqrt{\frac\ell {a + g}}, T = \frac {t}{N}
        \implies a + g = \ell \cdot \frac{4 \pi ^ 2}{T^2},
        a = \ell \cdot \frac{4 \pi ^ 2}{T^2} - g = \ell \cdot \frac{4 \pi ^ 2 N^2}{t^2} - g \approx 2{,}23\,\frac{\text{м}}{\text{c}^{2}},
        \text{вниз}.
    $
}
\solutionspace{120pt}

\tasknumber{11}%
\task{%
    Масса груза в пружинном маятнике равна $500\,\text{г}$, при этом период его колебаний равен $1{,}5\,\text{с}$.
    Груз утяжеляют на $50\,\text{г}$.
    Определите новый период колебаний маятника.
}
\answer{%
    $
        T'
            = 2\pi\sqrt{\frac{M + m}{k}}
            = 2\pi\sqrt{\frac{M}{k} \cdot \frac{M + m}{M}}
            = T\sqrt{\frac{M + m}{M}} =  T\sqrt{1 + \frac{m}{M}} \approx 1{,}57\,\text{с}.
    $
}
\solutionspace{120pt}

\tasknumber{12}%
\task{%
    При какой длине нити математического маятника период колебаний груза массой $300\,\text{г}$
    окажется равен периоду колебаний этого же груза в пружинном маятнике с пружиной жёсткостью $50\,\frac{\text{Н}}{\text{м}}$?
}
\answer{%
    $
        2\pi \sqrt{\frac \ell g} = 2\pi \sqrt{\frac m k}
        \implies \frac \ell g = \frac m k
        \implies \ell = g \frac m k \approx 6{,}0\,\text{см}.
    $
}
\solutionspace{120pt}

\tasknumber{13}%
\task{%
    Груз подвесили к пружине, при этом удлинение пружины составило $30\,\text{мм}$.
    Определите частоту колебаний пружинного маятника, собранного из этой пружины и этого груза.
}
\answer{%
    $
        mg -k\Delta x = 0 \implies \frac m k = \frac{\Delta x} g
        \implies T = 2\pi \sqrt{\frac m k } = 2\pi \sqrt{\frac{\Delta x} g } \approx $0{,}34\,\text{с}$,
        \nu = \frac 1T \approx $2{,}91\,\text{Гц}$.
    $
}
\solutionspace{120pt}

\tasknumber{14}%
\task{%
    Определите период колебаний системы: математический маятник ограничен с одной стороны стенкой (см.
    рис.
    на доске).
    Удары маятника о стенку абсолютно упругие, $n = 2$.
    Длина маятника $\ell$, ускорение свободного падения $g$.
}
\answer{%
    $
        T = 2\pi\sqrt{\frac\ell g}, \qquad
        T' = 2 \cdot \frac T 4 + 2 \cdot \frac T{12} = \frac23T = \frac43\pi\sqrt{\frac\ell g}
    $
}

\variantsplitter

\addpersonalvariant{Вячеслав Волохов}

\tasknumber{1}%
\task{%
    \begin{itemize}
        \item Запишите линейное однородное дифференциальное уравнение второго порядка,
            описывающее свободные незатухающие колебания гармонического осциллятора,
        \item запишите общее решение этого уравнения,
        \item подпишите в выписанном решении фазу и амплитуду колебаний,
        \item запишите выражение для скорости,
        \item запишите выражение для ускорения.
    \end{itemize}
}
\answer{%
    \begin{align*}
    &\ddot x + \omega^2 x = 0 \Longleftrightarrow a_x + \omega^2 x = 0, \\
    &x = A \cos(\omega t + \varphi_0) \text{ или же } x = A \sin(\omega t + \varphi_0) \text{ или же } x = a \cos(\omega t) + b \sin(\omega t), \\
    &A \text{\, или \,} \sqrt{a^2 + b^2} \text{ --- это амплитуда}, \omega t + \varphi_0\text{ --- это фаза}, \\
    &v = \dot x = -\omega A \sin(\omega t + \varphi_0), \\
    &a = \dot v = \ddot x = -\omega^2 A \cos(\omega t + \varphi_0) = -\omega^2 x,
    \end{align*}
}
\solutionspace{135pt}

\tasknumber{2}%
\task{%
    Тело колеблется по гармоническому закону,
    амплитуда этих колебаний $14\,\text{см}$, период $6\,\text{c}$.
    Чему равно смещение тела относительно положения равновесия через $22\,\text{c}$
    после прохождения положения максимального отклонения?
}
\answer{%
    $x = A \cos \omega t = A \cos \cbr{ \frac {2\pi}T t } = A \cos \cbr{ 2\pi \frac tT } = 14\,\text{см} \cdot \cos \cbr{ 2\pi \cdot \frac {22\,\text{c}}{6\,\text{c}}}\approx -7{,}00\,\text{см}.$
}
\solutionspace{120pt}

\tasknumber{3}%
\task{%
    Тело совершает гармонические колебания с периодом $5\,\text{c}$.
    За какое время тело смещается от положения равновесия до смещения в половину амплитуды?
}
\answer{%
    $t = \frac T{12} \approx 0{,}4\,\text{c}.$
}
\solutionspace{120pt}

\tasknumber{4}%
\task{%
    Запишите формулу для периода колебаний математического маятника и ...
    \begin{itemize}
        \item укажите названия всех физических величин в формуле,
        \item выразите из неё циклическую частоту колебаний
        \item выразите из неё ускорение свободного падения.
    \end{itemize}
}
\answer{%
    \begin{align*}
    T &= 2\pi \sqrt{\frac lg} \implies \nu = \frac 1T = \frac 1{2\pi}\sqrt{\frac gl}, \omega = 2\pi\nu = \sqrt{\frac gl}, l = g\sqr{\frac T{2\pi}}, g = l\sqr{\frac {2\pi}T} \\
    T &= 2\pi \sqrt{\frac mk} \implies \nu = \frac 1T = \frac 1{2\pi}\sqrt{\frac km}, \omega = 2\pi\nu = \sqrt{\frac km}, m = k\sqr{\frac T{2\pi}}, k = m\sqr{\frac {2\pi}T}
    \end{align*}
}
\solutionspace{120pt}

\tasknumber{5}%
\task{%
    Частота колебаний пружинного маятника равна $8\,\text{Гц}$.
    Определите периоды колебаний
    \begin{itemize}
        \item кинетической энергии системы,
        \item скорости груза,
        \item модуля скорости груза.
    \end{itemize}
}
\answer{%
    $T = \frac 1\nu \approx 125{,}0\,\text{мc}, T_1 = \frac T2 \approx 62{,}5\,\text{мc}, T_2 = T \approx 125{,}0\,\text{мc}, T_3 = \frac T2 \approx 62{,}5\,\text{мc}.$
}
\solutionspace{80pt}

\tasknumber{6}%
\task{%
    Тело колеблется по гармоническому закону с амплитудой $12\,\text{см}$.
    Какой максимальный путь тело может пройти за четверть периода?
}
\answer{%
    $2 * A \frac{1}{\sqrt 2} = A\sqrt{2} \approx 17{,}0\,\text{см}.$
}
\solutionspace{80pt}

\tasknumber{7}%
\task{%
    Период колебаний математического маятника равен $3\,\text{с}$,
    а их амплитуда — $20\,\text{см}$.
    Определите максимальную скорость маятника.
}
\answer{%
    $
        T = \frac{2\pi}{\omega}
        \implies \omega = \frac{2\pi}{T}
        \implies v_{\max} = \omega A = \frac{2\pi}{T}A
        \approx 41{,}9\,\frac{\text{см}}{\text{с}}.
    $
}
\solutionspace{100pt}

\tasknumber{8}%
\task{%
    Определите период колебаний груза массой $m$, подвешенного к пружине жёсткостью $k$.
    Ускорение свободного падения $g$.
}
\answer{%
    \begin{align*}
    &-kx_0 + mg = 0, \\
    F &= -k(x_0 + \Delta x), \\
    ma &= -k(x_0 + \Delta x) + mg, \\
    ma &= -kx_0 -k \Delta x + mg = -mg -k \Delta x + mg = -k \Delta x, \\
    a &+ \frac k m x = 0, \\
    \omega^2 &= \frac k m \implies T = \frac{2\pi}\omega = 2\pi\sqrt{\frac m k}.
    \end{align*}
}
\solutionspace{150pt}

\tasknumber{9}%
\task{%
    Куб со стороной $a$ и плотности $\rho$ плавает в жидкости плотностью $\rho_0$.
    Определите частоту колебаний куба, считая, что 4 грани куба всегда вертикальны.
}
\answer{%
    \begin{align*}
    \Delta F &= -\rho_0 g a^2 \Delta x,\Delta F = m\ddot x, m = \rho a^3\implies \rho a^3 \ddot x = -\rho_0 g a^2 \Delta x \implies \\
    \implies \ddot x &= -\frac{\rho_0}{\rho} \frac g a \Delta x\implies \omega^2 = \frac{\rho_0}{\rho} \cdot \frac g a\implies T = \frac{2 \pi}{\omega} = 2 \pi\sqrt{\frac{\rho}{\rho_0} \cdot \frac a g}.
    \end{align*}
}
\solutionspace{150pt}

\tasknumber{10}%
\task{%
    Математический маятник с нитью длиной $37\,\text{см}$ подвешен к потолку в лифте.
    За $25\,\text{с}$ маятник совершил $20$ колебаний.
    Определите модуль и направление ускорения лифта.
    Куда движется лифт?
}
\answer{%
    $
        T = 2\pi\sqrt{\frac\ell {a + g}}, T = \frac {t}{N}
        \implies a + g = \ell \cdot \frac{4 \pi ^ 2}{T^2},
        a = \ell \cdot \frac{4 \pi ^ 2}{T^2} - g = \ell \cdot \frac{4 \pi ^ 2 N^2}{t^2} - g \approx -0{,}6500\,\frac{\text{м}}{\text{c}^{2}},
        \text{вверх}.
    $
}
\solutionspace{120pt}

\tasknumber{11}%
\task{%
    Масса груза в пружинном маятнике равна $600\,\text{г}$, при этом период его колебаний равен $1{,}4\,\text{с}$.
    Груз утяжеляют на $150\,\text{г}$.
    Определите новый период колебаний маятника.
}
\answer{%
    $
        T'
            = 2\pi\sqrt{\frac{M + m}{k}}
            = 2\pi\sqrt{\frac{M}{k} \cdot \frac{M + m}{M}}
            = T\sqrt{\frac{M + m}{M}} =  T\sqrt{1 + \frac{m}{M}} \approx 1{,}57\,\text{с}.
    $
}
\solutionspace{120pt}

\tasknumber{12}%
\task{%
    При какой длине нити математического маятника период колебаний груза массой $200\,\text{г}$
    окажется равен периоду колебаний этого же груза в пружинном маятнике с пружиной жёсткостью $60\,\frac{\text{Н}}{\text{м}}$?
}
\answer{%
    $
        2\pi \sqrt{\frac \ell g} = 2\pi \sqrt{\frac m k}
        \implies \frac \ell g = \frac m k
        \implies \ell = g \frac m k \approx 3{,}3\,\text{см}.
    $
}
\solutionspace{120pt}

\tasknumber{13}%
\task{%
    Груз подвесили к пружине, при этом удлинение пружины составило $45\,\text{мм}$.
    Определите частоту колебаний пружинного маятника, собранного из этой пружины и этого груза.
}
\answer{%
    $
        mg -k\Delta x = 0 \implies \frac m k = \frac{\Delta x} g
        \implies T = 2\pi \sqrt{\frac m k } = 2\pi \sqrt{\frac{\Delta x} g } \approx $0{,}42\,\text{с}$,
        \nu = \frac 1T \approx $2{,}37\,\text{Гц}$.
    $
}
\solutionspace{120pt}

\tasknumber{14}%
\task{%
    Определите период колебаний системы: математический маятник ограничен с одной стороны стенкой (см.
    рис.
    на доске).
    Удары маятника о стенку абсолютно упругие, $n = \sqrt{2}$.
    Длина маятника $\ell$, ускорение свободного падения $g$.
}
\answer{%
    $
        T = 2\pi\sqrt{\frac\ell g}, \qquad
        T' = 2 \cdot \frac T 4 + 2 \cdot \frac T{8} = \frac34T = \frac32\pi\sqrt{\frac\ell g}
    $
}

\variantsplitter

\addpersonalvariant{Герман Говоров}

\tasknumber{1}%
\task{%
    \begin{itemize}
        \item Запишите линейное однородное дифференциальное уравнение второго порядка,
            описывающее свободные незатухающие колебания гармонического осциллятора,
        \item запишите общее решение этого уравнения,
        \item подпишите в выписанном решении фазу и амплитуду колебаний,
        \item запишите выражение для скорости,
        \item запишите выражение для ускорения.
    \end{itemize}
}
\answer{%
    \begin{align*}
    &\ddot x + \omega^2 x = 0 \Longleftrightarrow a_x + \omega^2 x = 0, \\
    &x = A \cos(\omega t + \varphi_0) \text{ или же } x = A \sin(\omega t + \varphi_0) \text{ или же } x = a \cos(\omega t) + b \sin(\omega t), \\
    &A \text{\, или \,} \sqrt{a^2 + b^2} \text{ --- это амплитуда}, \omega t + \varphi_0\text{ --- это фаза}, \\
    &v = \dot x = -\omega A \sin(\omega t + \varphi_0), \\
    &a = \dot v = \ddot x = -\omega^2 A \cos(\omega t + \varphi_0) = -\omega^2 x,
    \end{align*}
}
\solutionspace{135pt}

\tasknumber{2}%
\task{%
    Тело колеблется по гармоническому закону,
    амплитуда этих колебаний $16\,\text{см}$, период $4\,\text{c}$.
    Чему равно смещение тела относительно положения равновесия через $25\,\text{c}$
    после прохождения положения равновесия?
}
\answer{%
    $x = A \sin \omega t = A \sin \cbr{ \frac {2\pi}T t } = A \sin \cbr{ 2\pi \frac tT } = 16\,\text{см} \cdot \sin \cbr{ 2\pi \cdot \frac {25\,\text{c}}{4\,\text{c}}}\approx 16{,}0\,\text{см}.$
}
\solutionspace{120pt}

\tasknumber{3}%
\task{%
    Тело совершает гармонические колебания с периодом $4\,\text{c}$.
    За какое время тело смещается от положения наибольшего отклонения до смещения в половину амплитуды?
}
\answer{%
    $t = \frac T{6} \approx 0{,}7\,\text{c}.$
}
\solutionspace{120pt}

\tasknumber{4}%
\task{%
    Запишите формулу для периода колебаний математического маятника и ...
    \begin{itemize}
        \item укажите названия всех физических величин в формуле,
        \item выразите из неё циклическую частоту колебаний
        \item выразите из неё длину маятника.
    \end{itemize}
}
\answer{%
    \begin{align*}
    T &= 2\pi \sqrt{\frac lg} \implies \nu = \frac 1T = \frac 1{2\pi}\sqrt{\frac gl}, \omega = 2\pi\nu = \sqrt{\frac gl}, l = g\sqr{\frac T{2\pi}}, g = l\sqr{\frac {2\pi}T} \\
    T &= 2\pi \sqrt{\frac mk} \implies \nu = \frac 1T = \frac 1{2\pi}\sqrt{\frac km}, \omega = 2\pi\nu = \sqrt{\frac km}, m = k\sqr{\frac T{2\pi}}, k = m\sqr{\frac {2\pi}T}
    \end{align*}
}
\solutionspace{120pt}

\tasknumber{5}%
\task{%
    Частота колебаний математического маятника равна $8\,\text{Гц}$.
    Определите периоды колебаний
    \begin{itemize}
        \item кинетической энергии системы,
        \item ускорения груза,
        \item модуля ускорения груза.
    \end{itemize}
}
\answer{%
    $T = \frac 1\nu \approx 125{,}0\,\text{мc}, T_1 = \frac T2 \approx 62{,}5\,\text{мc}, T_2 = T \approx 125{,}0\,\text{мc}, T_3 = \frac T2 \approx 62{,}5\,\text{мc}.$
}
\solutionspace{80pt}

\tasknumber{6}%
\task{%
    Тело колеблется по гармоническому закону с амплитудой $4\,\text{см}$.
    Какой минимальный путь тело может пройти за одну шестую долю периода?
}
\answer{%
    $2 * A \frac{1}{12} = \frac{A:L:s}3 \approx 0{,}7\,\text{см}.$
}
\solutionspace{80pt}

\tasknumber{7}%
\task{%
    Период колебаний математического маятника равен $2\,\text{с}$,
    а их амплитуда — $20\,\text{см}$.
    Определите амплитуду колебаний скорости маятника.
}
\answer{%
    $
        T = \frac{2\pi}{\omega}
        \implies \omega = \frac{2\pi}{T}
        \implies v_{\max} = \omega A = \frac{2\pi}{T}A
        \approx 62{,}8\,\frac{\text{см}}{\text{с}}.
    $
}
\solutionspace{100pt}

\tasknumber{8}%
\task{%
    Определите период колебаний груза массой $m$, подвешенного к пружине жёсткостью $k$.
    Ускорение свободного падения $g$.
}
\answer{%
    \begin{align*}
    &-kx_0 + mg = 0, \\
    F &= -k(x_0 + \Delta x), \\
    ma &= -k(x_0 + \Delta x) + mg, \\
    ma &= -kx_0 -k \Delta x + mg = -mg -k \Delta x + mg = -k \Delta x, \\
    a &+ \frac k m x = 0, \\
    \omega^2 &= \frac k m \implies T = \frac{2\pi}\omega = 2\pi\sqrt{\frac m k}.
    \end{align*}
}
\solutionspace{150pt}

\tasknumber{9}%
\task{%
    Куб со стороной $a$ и плотности $\rho$ плавает в жидкости плотностью $\rho_0$.
    Определите частоту колебаний куба, считая, что 4 грани куба всегда вертикальны.
}
\answer{%
    \begin{align*}
    \Delta F &= -\rho_0 g a^2 \Delta x,\Delta F = m\ddot x, m = \rho a^3\implies \rho a^3 \ddot x = -\rho_0 g a^2 \Delta x \implies \\
    \implies \ddot x &= -\frac{\rho_0}{\rho} \frac g a \Delta x\implies \omega^2 = \frac{\rho_0}{\rho} \cdot \frac g a\implies T = \frac{2 \pi}{\omega} = 2 \pi\sqrt{\frac{\rho}{\rho_0} \cdot \frac a g}.
    \end{align*}
}
\solutionspace{150pt}

\tasknumber{10}%
\task{%
    Математический маятник с нитью длиной $43\,\text{см}$ подвешен к потолку в лифте.
    За $30\,\text{с}$ маятник совершил $24$ колебаний.
    Определите модуль и направление ускорения лифта.
    Куда движется лифт?
}
\answer{%
    $
        T = 2\pi\sqrt{\frac\ell {a + g}}, T = \frac {t}{N}
        \implies a + g = \ell \cdot \frac{4 \pi ^ 2}{T^2},
        a = \ell \cdot \frac{4 \pi ^ 2}{T^2} - g = \ell \cdot \frac{4 \pi ^ 2 N^2}{t^2} - g \approx 0{,}86\,\frac{\text{м}}{\text{c}^{2}},
        \text{вниз}.
    $
}
\solutionspace{120pt}

\tasknumber{11}%
\task{%
    Масса груза в пружинном маятнике равна $400\,\text{г}$, при этом период его колебаний равен $1{,}5\,\text{с}$.
    Груз облегчают на $100\,\text{г}$.
    Определите новый период колебаний маятника.
}
\answer{%
    $
        T'
            = 2\pi\sqrt{\frac{M - m}{k}}
            = 2\pi\sqrt{\frac{M}{k} \cdot \frac{M - m}{M}}
            = T\sqrt{\frac{M - m}{M}} =  T\sqrt{1 - \frac{m}{M}} \approx 1{,}30\,\text{с}.
    $
}
\solutionspace{120pt}

\tasknumber{12}%
\task{%
    При какой длине нити математического маятника период колебаний груза массой $200\,\text{г}$
    окажется равен периоду колебаний этого же груза в пружинном маятнике с пружиной жёсткостью $50\,\frac{\text{Н}}{\text{м}}$?
}
\answer{%
    $
        2\pi \sqrt{\frac \ell g} = 2\pi \sqrt{\frac m k}
        \implies \frac \ell g = \frac m k
        \implies \ell = g \frac m k \approx 4{,}0\,\text{см}.
    $
}
\solutionspace{120pt}

\tasknumber{13}%
\task{%
    Груз подвесили к пружине, при этом удлинение пружины составило $45\,\text{мм}$.
    Определите частоту колебаний пружинного маятника, собранного из этой пружины и этого груза.
}
\answer{%
    $
        mg -k\Delta x = 0 \implies \frac m k = \frac{\Delta x} g
        \implies T = 2\pi \sqrt{\frac m k } = 2\pi \sqrt{\frac{\Delta x} g } \approx $0{,}42\,\text{с}$,
        \nu = \frac 1T \approx $2{,}37\,\text{Гц}$.
    $
}
\solutionspace{120pt}

\tasknumber{14}%
\task{%
    Определите период колебаний системы: математический маятник ограничен с одной стороны стенкой (см.
    рис.
    на доске).
    Удары маятника о стенку абсолютно упругие, $n = 2$.
    Длина маятника $\ell$, ускорение свободного падения $g$.
}
\answer{%
    $
        T = 2\pi\sqrt{\frac\ell g}, \qquad
        T' = 2 \cdot \frac T 4 + 2 \cdot \frac T{12} = \frac23T = \frac43\pi\sqrt{\frac\ell g}
    $
}

\variantsplitter

\addpersonalvariant{София Журавлёва}

\tasknumber{1}%
\task{%
    \begin{itemize}
        \item Запишите линейное однородное дифференциальное уравнение второго порядка,
            описывающее свободные незатухающие колебания гармонического осциллятора,
        \item запишите общее решение этого уравнения,
        \item подпишите в выписанном решении фазу и амплитуду колебаний,
        \item запишите выражение для скорости,
        \item запишите выражение для ускорения.
    \end{itemize}
}
\answer{%
    \begin{align*}
    &\ddot x + \omega^2 x = 0 \Longleftrightarrow a_x + \omega^2 x = 0, \\
    &x = A \cos(\omega t + \varphi_0) \text{ или же } x = A \sin(\omega t + \varphi_0) \text{ или же } x = a \cos(\omega t) + b \sin(\omega t), \\
    &A \text{\, или \,} \sqrt{a^2 + b^2} \text{ --- это амплитуда}, \omega t + \varphi_0\text{ --- это фаза}, \\
    &v = \dot x = -\omega A \sin(\omega t + \varphi_0), \\
    &a = \dot v = \ddot x = -\omega^2 A \cos(\omega t + \varphi_0) = -\omega^2 x,
    \end{align*}
}
\solutionspace{135pt}

\tasknumber{2}%
\task{%
    Тело колеблется по гармоническому закону,
    амплитуда этих колебаний $18\,\text{см}$, период $4\,\text{c}$.
    Чему равно смещение тела относительно положения равновесия через $22\,\text{c}$
    после прохождения положения максимального отклонения?
}
\answer{%
    $x = A \cos \omega t = A \cos \cbr{ \frac {2\pi}T t } = A \cos \cbr{ 2\pi \frac tT } = 18\,\text{см} \cdot \cos \cbr{ 2\pi \cdot \frac {22\,\text{c}}{4\,\text{c}}}\approx -18{,}000\,\text{см}.$
}
\solutionspace{120pt}

\tasknumber{3}%
\task{%
    Тело совершает гармонические колебания с периодом $4\,\text{c}$.
    За какое время тело смещается от положения равновесия до смещения в половину амплитуды?
}
\answer{%
    $t = \frac T{12} \approx 0{,}3\,\text{c}.$
}
\solutionspace{120pt}

\tasknumber{4}%
\task{%
    Запишите формулу для периода колебаний математического маятника и ...
    \begin{itemize}
        \item укажите названия всех физических величин в формуле,
        \item выразите из неё циклическую частоту колебаний
        \item выразите из неё длину маятника.
    \end{itemize}
}
\answer{%
    \begin{align*}
    T &= 2\pi \sqrt{\frac lg} \implies \nu = \frac 1T = \frac 1{2\pi}\sqrt{\frac gl}, \omega = 2\pi\nu = \sqrt{\frac gl}, l = g\sqr{\frac T{2\pi}}, g = l\sqr{\frac {2\pi}T} \\
    T &= 2\pi \sqrt{\frac mk} \implies \nu = \frac 1T = \frac 1{2\pi}\sqrt{\frac km}, \omega = 2\pi\nu = \sqrt{\frac km}, m = k\sqr{\frac T{2\pi}}, k = m\sqr{\frac {2\pi}T}
    \end{align*}
}
\solutionspace{120pt}

\tasknumber{5}%
\task{%
    Частота колебаний математического маятника равна $15\,\text{Гц}$.
    Определите периоды колебаний
    \begin{itemize}
        \item потенциальной энергии системы,
        \item скорости груза,
        \item модуля скорости груза.
    \end{itemize}
}
\answer{%
    $T = \frac 1\nu \approx 66{,}7\,\text{мc}, T_1 = \frac T2 \approx 33{,}3\,\text{мc}, T_2 = T \approx 66{,}7\,\text{мc}, T_3 = \frac T2 \approx 33{,}3\,\text{мc}.$
}
\solutionspace{80pt}

\tasknumber{6}%
\task{%
    Тело колеблется по гармоническому закону с амплитудой $6\,\text{см}$.
    Какой максимальный путь тело может пройти за половину периода?
}
\answer{%
    $2A \approx 12{,}0\,\text{см}.$
}
\solutionspace{80pt}

\tasknumber{7}%
\task{%
    Период колебаний математического маятника равен $5\,\text{с}$,
    а их амплитуда — $20\,\text{см}$.
    Определите максимальную скорость маятника.
}
\answer{%
    $
        T = \frac{2\pi}{\omega}
        \implies \omega = \frac{2\pi}{T}
        \implies v_{\max} = \omega A = \frac{2\pi}{T}A
        \approx 25{,}1\,\frac{\text{см}}{\text{с}}.
    $
}
\solutionspace{100pt}

\tasknumber{8}%
\task{%
    Определите период колебаний груза массой $m$, подвешенного к пружине жёсткостью $k$.
    Ускорение свободного падения $g$.
}
\answer{%
    \begin{align*}
    &-kx_0 + mg = 0, \\
    F &= -k(x_0 + \Delta x), \\
    ma &= -k(x_0 + \Delta x) + mg, \\
    ma &= -kx_0 -k \Delta x + mg = -mg -k \Delta x + mg = -k \Delta x, \\
    a &+ \frac k m x = 0, \\
    \omega^2 &= \frac k m \implies T = \frac{2\pi}\omega = 2\pi\sqrt{\frac m k}.
    \end{align*}
}
\solutionspace{150pt}

\tasknumber{9}%
\task{%
    Куб со стороной $a$ и плотности $\rho$ плавает в жидкости плотностью $\rho_0$.
    Определите частоту колебаний куба, считая, что 4 грани куба всегда вертикальны.
}
\answer{%
    \begin{align*}
    \Delta F &= -\rho_0 g a^2 \Delta x,\Delta F = m\ddot x, m = \rho a^3\implies \rho a^3 \ddot x = -\rho_0 g a^2 \Delta x \implies \\
    \implies \ddot x &= -\frac{\rho_0}{\rho} \frac g a \Delta x\implies \omega^2 = \frac{\rho_0}{\rho} \cdot \frac g a\implies T = \frac{2 \pi}{\omega} = 2 \pi\sqrt{\frac{\rho}{\rho_0} \cdot \frac a g}.
    \end{align*}
}
\solutionspace{150pt}

\tasknumber{10}%
\task{%
    Математический маятник с нитью длиной $43\,\text{см}$ подвешен к потолку в лифте.
    За $20\,\text{с}$ маятник совершил $17$ колебаний.
    Определите модуль и направление ускорения лифта.
    Куда движется лифт?
}
\answer{%
    $
        T = 2\pi\sqrt{\frac\ell {a + g}}, T = \frac {t}{N}
        \implies a + g = \ell \cdot \frac{4 \pi ^ 2}{T^2},
        a = \ell \cdot \frac{4 \pi ^ 2}{T^2} - g = \ell \cdot \frac{4 \pi ^ 2 N^2}{t^2} - g \approx 2{,}26\,\frac{\text{м}}{\text{c}^{2}},
        \text{вниз}.
    $
}
\solutionspace{120pt}

\tasknumber{11}%
\task{%
    Масса груза в пружинном маятнике равна $500\,\text{г}$, при этом период его колебаний равен $1{,}4\,\text{с}$.
    Груз утяжеляют на $50\,\text{г}$.
    Определите новый период колебаний маятника.
}
\answer{%
    $
        T'
            = 2\pi\sqrt{\frac{M + m}{k}}
            = 2\pi\sqrt{\frac{M}{k} \cdot \frac{M + m}{M}}
            = T\sqrt{\frac{M + m}{M}} =  T\sqrt{1 + \frac{m}{M}} \approx 1{,}47\,\text{с}.
    $
}
\solutionspace{120pt}

\tasknumber{12}%
\task{%
    При какой длине нити математического маятника период колебаний груза массой $300\,\text{г}$
    окажется равен периоду колебаний этого же груза в пружинном маятнике с пружиной жёсткостью $50\,\frac{\text{Н}}{\text{м}}$?
}
\answer{%
    $
        2\pi \sqrt{\frac \ell g} = 2\pi \sqrt{\frac m k}
        \implies \frac \ell g = \frac m k
        \implies \ell = g \frac m k \approx 6{,}0\,\text{см}.
    $
}
\solutionspace{120pt}

\tasknumber{13}%
\task{%
    Груз подвесили к пружине, при этом удлинение пружины составило $45\,\text{мм}$.
    Определите частоту колебаний пружинного маятника, собранного из этой пружины и этого груза.
}
\answer{%
    $
        mg -k\Delta x = 0 \implies \frac m k = \frac{\Delta x} g
        \implies T = 2\pi \sqrt{\frac m k } = 2\pi \sqrt{\frac{\Delta x} g } \approx $0{,}42\,\text{с}$,
        \nu = \frac 1T \approx $2{,}37\,\text{Гц}$.
    $
}
\solutionspace{120pt}

\tasknumber{14}%
\task{%
    Определите период колебаний системы: математический маятник ограничен с одной стороны стенкой (см.
    рис.
    на доске).
    Удары маятника о стенку абсолютно упругие, $n = \frac{2}{\sqrt{3}}$.
    Длина маятника $\ell$, ускорение свободного падения $g$.
}
\answer{%
    $
        T = 2\pi\sqrt{\frac\ell g}, \qquad
        T' = 2 \cdot \frac T 4 + 2 \cdot \frac T{6} = \frac56T = \frac53\pi\sqrt{\frac\ell g}
    $
}

\variantsplitter

\addpersonalvariant{Константин Козлов}

\tasknumber{1}%
\task{%
    \begin{itemize}
        \item Запишите линейное однородное дифференциальное уравнение второго порядка,
            описывающее свободные незатухающие колебания гармонического осциллятора,
        \item запишите общее решение этого уравнения,
        \item подпишите в выписанном решении фазу и амплитуду колебаний,
        \item запишите выражение для скорости,
        \item запишите выражение для ускорения.
    \end{itemize}
}
\answer{%
    \begin{align*}
    &\ddot x + \omega^2 x = 0 \Longleftrightarrow a_x + \omega^2 x = 0, \\
    &x = A \cos(\omega t + \varphi_0) \text{ или же } x = A \sin(\omega t + \varphi_0) \text{ или же } x = a \cos(\omega t) + b \sin(\omega t), \\
    &A \text{\, или \,} \sqrt{a^2 + b^2} \text{ --- это амплитуда}, \omega t + \varphi_0\text{ --- это фаза}, \\
    &v = \dot x = -\omega A \sin(\omega t + \varphi_0), \\
    &a = \dot v = \ddot x = -\omega^2 A \cos(\omega t + \varphi_0) = -\omega^2 x,
    \end{align*}
}
\solutionspace{135pt}

\tasknumber{2}%
\task{%
    Тело колеблется по гармоническому закону,
    амплитуда этих колебаний $20\,\text{см}$, период $2\,\text{c}$.
    Чему равно смещение тела относительно положения равновесия через $22\,\text{c}$
    после прохождения положения равновесия?
}
\answer{%
    $x = A \sin \omega t = A \sin \cbr{ \frac {2\pi}T t } = A \sin \cbr{ 2\pi \frac tT } = 20\,\text{см} \cdot \sin \cbr{ 2\pi \cdot \frac {22\,\text{c}}{2\,\text{c}}}\approx 0\,\text{см}.$
}
\solutionspace{120pt}

\tasknumber{3}%
\task{%
    Тело совершает гармонические колебания с периодом $5\,\text{c}$.
    За какое время тело смещается от положения наибольшего отклонения до смещения в половину амплитуды?
}
\answer{%
    $t = \frac T{6} \approx 0{,}8\,\text{c}.$
}
\solutionspace{120pt}

\tasknumber{4}%
\task{%
    Запишите формулу для периода колебаний математического маятника и ...
    \begin{itemize}
        \item укажите названия всех физических величин в формуле,
        \item выразите из неё циклическую частоту колебаний
        \item выразите из неё длину маятника.
    \end{itemize}
}
\answer{%
    \begin{align*}
    T &= 2\pi \sqrt{\frac lg} \implies \nu = \frac 1T = \frac 1{2\pi}\sqrt{\frac gl}, \omega = 2\pi\nu = \sqrt{\frac gl}, l = g\sqr{\frac T{2\pi}}, g = l\sqr{\frac {2\pi}T} \\
    T &= 2\pi \sqrt{\frac mk} \implies \nu = \frac 1T = \frac 1{2\pi}\sqrt{\frac km}, \omega = 2\pi\nu = \sqrt{\frac km}, m = k\sqr{\frac T{2\pi}}, k = m\sqr{\frac {2\pi}T}
    \end{align*}
}
\solutionspace{120pt}

\tasknumber{5}%
\task{%
    Частота колебаний пружинного маятника равна $12\,\text{Гц}$.
    Определите периоды колебаний
    \begin{itemize}
        \item кинетической энергии системы,
        \item скорости груза,
        \item модуля скорости груза.
    \end{itemize}
}
\answer{%
    $T = \frac 1\nu \approx 83{,}3\,\text{мc}, T_1 = \frac T2 \approx 41{,}7\,\text{мc}, T_2 = T \approx 83{,}3\,\text{мc}, T_3 = \frac T2 \approx 41{,}7\,\text{мc}.$
}
\solutionspace{80pt}

\tasknumber{6}%
\task{%
    Тело колеблется по гармоническому закону с амплитудой $12\,\text{см}$.
    Какой максимальный путь тело может пройти за половину периода?
}
\answer{%
    $2A \approx 24{,}0\,\text{см}.$
}
\solutionspace{80pt}

\tasknumber{7}%
\task{%
    Период колебаний математического маятника равен $3\,\text{с}$,
    а их амплитуда — $10\,\text{см}$.
    Определите амплитуду колебаний скорости маятника.
}
\answer{%
    $
        T = \frac{2\pi}{\omega}
        \implies \omega = \frac{2\pi}{T}
        \implies v_{\max} = \omega A = \frac{2\pi}{T}A
        \approx 20{,}9\,\frac{\text{см}}{\text{с}}.
    $
}
\solutionspace{100pt}

\tasknumber{8}%
\task{%
    Определите период колебаний груза массой $m$, подвешенного к пружине жёсткостью $k$.
    Ускорение свободного падения $g$.
}
\answer{%
    \begin{align*}
    &-kx_0 + mg = 0, \\
    F &= -k(x_0 + \Delta x), \\
    ma &= -k(x_0 + \Delta x) + mg, \\
    ma &= -kx_0 -k \Delta x + mg = -mg -k \Delta x + mg = -k \Delta x, \\
    a &+ \frac k m x = 0, \\
    \omega^2 &= \frac k m \implies T = \frac{2\pi}\omega = 2\pi\sqrt{\frac m k}.
    \end{align*}
}
\solutionspace{150pt}

\tasknumber{9}%
\task{%
    Куб со стороной $a$ и плотности $\rho$ плавает в жидкости плотностью $\rho_0$.
    Определите частоту колебаний куба, считая, что 4 грани куба всегда вертикальны.
}
\answer{%
    \begin{align*}
    \Delta F &= -\rho_0 g a^2 \Delta x,\Delta F = m\ddot x, m = \rho a^3\implies \rho a^3 \ddot x = -\rho_0 g a^2 \Delta x \implies \\
    \implies \ddot x &= -\frac{\rho_0}{\rho} \frac g a \Delta x\implies \omega^2 = \frac{\rho_0}{\rho} \cdot \frac g a\implies T = \frac{2 \pi}{\omega} = 2 \pi\sqrt{\frac{\rho}{\rho_0} \cdot \frac a g}.
    \end{align*}
}
\solutionspace{150pt}

\tasknumber{10}%
\task{%
    Математический маятник с нитью длиной $40\,\text{см}$ подвешен к потолку в лифте.
    За $25\,\text{с}$ маятник совершил $22$ колебаний.
    Определите модуль и направление ускорения лифта.
    Куда движется лифт?
}
\answer{%
    $
        T = 2\pi\sqrt{\frac\ell {a + g}}, T = \frac {t}{N}
        \implies a + g = \ell \cdot \frac{4 \pi ^ 2}{T^2},
        a = \ell \cdot \frac{4 \pi ^ 2}{T^2} - g = \ell \cdot \frac{4 \pi ^ 2 N^2}{t^2} - g \approx 2{,}23\,\frac{\text{м}}{\text{c}^{2}},
        \text{вниз}.
    $
}
\solutionspace{120pt}

\tasknumber{11}%
\task{%
    Масса груза в пружинном маятнике равна $500\,\text{г}$, при этом период его колебаний равен $1{,}4\,\text{с}$.
    Груз утяжеляют на $50\,\text{г}$.
    Определите новый период колебаний маятника.
}
\answer{%
    $
        T'
            = 2\pi\sqrt{\frac{M + m}{k}}
            = 2\pi\sqrt{\frac{M}{k} \cdot \frac{M + m}{M}}
            = T\sqrt{\frac{M + m}{M}} =  T\sqrt{1 + \frac{m}{M}} \approx 1{,}47\,\text{с}.
    $
}
\solutionspace{120pt}

\tasknumber{12}%
\task{%
    При какой длине нити математического маятника период колебаний груза массой $200\,\text{г}$
    окажется равен периоду колебаний этого же груза в пружинном маятнике с пружиной жёсткостью $60\,\frac{\text{Н}}{\text{м}}$?
}
\answer{%
    $
        2\pi \sqrt{\frac \ell g} = 2\pi \sqrt{\frac m k}
        \implies \frac \ell g = \frac m k
        \implies \ell = g \frac m k \approx 3{,}3\,\text{см}.
    $
}
\solutionspace{120pt}

\tasknumber{13}%
\task{%
    Груз подвесили к пружине, при этом удлинение пружины составило $60\,\text{мм}$.
    Определите частоту колебаний пружинного маятника, собранного из этой пружины и этого груза.
}
\answer{%
    $
        mg -k\Delta x = 0 \implies \frac m k = \frac{\Delta x} g
        \implies T = 2\pi \sqrt{\frac m k } = 2\pi \sqrt{\frac{\Delta x} g } \approx $0{,}49\,\text{с}$,
        \nu = \frac 1T \approx $2{,}05\,\text{Гц}$.
    $
}
\solutionspace{120pt}

\tasknumber{14}%
\task{%
    Определите период колебаний системы: математический маятник ограничен с одной стороны стенкой (см.
    рис.
    на доске).
    Удары маятника о стенку абсолютно упругие, $n = \sqrt{2}$.
    Длина маятника $\ell$, ускорение свободного падения $g$.
}
\answer{%
    $
        T = 2\pi\sqrt{\frac\ell g}, \qquad
        T' = 2 \cdot \frac T 4 + 2 \cdot \frac T{8} = \frac34T = \frac32\pi\sqrt{\frac\ell g}
    $
}

\variantsplitter

\addpersonalvariant{Наталья Кравченко}

\tasknumber{1}%
\task{%
    \begin{itemize}
        \item Запишите линейное однородное дифференциальное уравнение второго порядка,
            описывающее свободные незатухающие колебания гармонического осциллятора,
        \item запишите общее решение этого уравнения,
        \item подпишите в выписанном решении фазу и амплитуду колебаний,
        \item запишите выражение для скорости,
        \item запишите выражение для ускорения.
    \end{itemize}
}
\answer{%
    \begin{align*}
    &\ddot x + \omega^2 x = 0 \Longleftrightarrow a_x + \omega^2 x = 0, \\
    &x = A \cos(\omega t + \varphi_0) \text{ или же } x = A \sin(\omega t + \varphi_0) \text{ или же } x = a \cos(\omega t) + b \sin(\omega t), \\
    &A \text{\, или \,} \sqrt{a^2 + b^2} \text{ --- это амплитуда}, \omega t + \varphi_0\text{ --- это фаза}, \\
    &v = \dot x = -\omega A \sin(\omega t + \varphi_0), \\
    &a = \dot v = \ddot x = -\omega^2 A \cos(\omega t + \varphi_0) = -\omega^2 x,
    \end{align*}
}
\solutionspace{135pt}

\tasknumber{2}%
\task{%
    Тело колеблется по гармоническому закону,
    амплитуда этих колебаний $12\,\text{см}$, период $2\,\text{c}$.
    Чему равно смещение тела относительно положения равновесия через $26\,\text{c}$
    после прохождения положения равновесия?
}
\answer{%
    $x = A \sin \omega t = A \sin \cbr{ \frac {2\pi}T t } = A \sin \cbr{ 2\pi \frac tT } = 12\,\text{см} \cdot \sin \cbr{ 2\pi \cdot \frac {26\,\text{c}}{2\,\text{c}}}\approx 0\,\text{см}.$
}
\solutionspace{120pt}

\tasknumber{3}%
\task{%
    Тело совершает гармонические колебания с периодом $6\,\text{c}$.
    За какое время тело смещается от положения равновесия до смещения в половину амплитуды?
}
\answer{%
    $t = \frac T{12} \approx 0{,}5\,\text{c}.$
}
\solutionspace{120pt}

\tasknumber{4}%
\task{%
    Запишите формулу для периода колебаний математического маятника и ...
    \begin{itemize}
        \item укажите названия всех физических величин в формуле,
        \item выразите из неё циклическую частоту колебаний
        \item выразите из неё ускорение свободного падения.
    \end{itemize}
}
\answer{%
    \begin{align*}
    T &= 2\pi \sqrt{\frac lg} \implies \nu = \frac 1T = \frac 1{2\pi}\sqrt{\frac gl}, \omega = 2\pi\nu = \sqrt{\frac gl}, l = g\sqr{\frac T{2\pi}}, g = l\sqr{\frac {2\pi}T} \\
    T &= 2\pi \sqrt{\frac mk} \implies \nu = \frac 1T = \frac 1{2\pi}\sqrt{\frac km}, \omega = 2\pi\nu = \sqrt{\frac km}, m = k\sqr{\frac T{2\pi}}, k = m\sqr{\frac {2\pi}T}
    \end{align*}
}
\solutionspace{120pt}

\tasknumber{5}%
\task{%
    Частота колебаний математического маятника равна $10\,\text{Гц}$.
    Определите периоды колебаний
    \begin{itemize}
        \item потенциальной энергии системы,
        \item ускорения груза,
        \item модуля скорости груза.
    \end{itemize}
}
\answer{%
    $T = \frac 1\nu \approx 100{,}0\,\text{мc}, T_1 = \frac T2 \approx 50{,}0\,\text{мc}, T_2 = T \approx 100{,}0\,\text{мc}, T_3 = \frac T2 \approx 50{,}0\,\text{мc}.$
}
\solutionspace{80pt}

\tasknumber{6}%
\task{%
    Тело колеблется по гармоническому закону с амплитудой $12\,\text{см}$.
    Какой максимальный путь тело может пройти за одну шестую долю периода?
}
\answer{%
    $2 * A \frac{1}{2} = A \approx 12{,}0\,\text{см}.$
}
\solutionspace{80pt}

\tasknumber{7}%
\task{%
    Период колебаний математического маятника равен $3\,\text{с}$,
    а их амплитуда — $20\,\text{см}$.
    Определите максимальную скорость маятника.
}
\answer{%
    $
        T = \frac{2\pi}{\omega}
        \implies \omega = \frac{2\pi}{T}
        \implies v_{\max} = \omega A = \frac{2\pi}{T}A
        \approx 41{,}9\,\frac{\text{см}}{\text{с}}.
    $
}
\solutionspace{100pt}

\tasknumber{8}%
\task{%
    Определите период колебаний груза массой $m$, подвешенного к пружине жёсткостью $k$.
    Ускорение свободного падения $g$.
}
\answer{%
    \begin{align*}
    &-kx_0 + mg = 0, \\
    F &= -k(x_0 + \Delta x), \\
    ma &= -k(x_0 + \Delta x) + mg, \\
    ma &= -kx_0 -k \Delta x + mg = -mg -k \Delta x + mg = -k \Delta x, \\
    a &+ \frac k m x = 0, \\
    \omega^2 &= \frac k m \implies T = \frac{2\pi}\omega = 2\pi\sqrt{\frac m k}.
    \end{align*}
}
\solutionspace{150pt}

\tasknumber{9}%
\task{%
    Куб со стороной $a$ и плотности $\rho$ плавает в жидкости плотностью $\rho_0$.
    Определите частоту колебаний куба, считая, что 4 грани куба всегда вертикальны.
}
\answer{%
    \begin{align*}
    \Delta F &= -\rho_0 g a^2 \Delta x,\Delta F = m\ddot x, m = \rho a^3\implies \rho a^3 \ddot x = -\rho_0 g a^2 \Delta x \implies \\
    \implies \ddot x &= -\frac{\rho_0}{\rho} \frac g a \Delta x\implies \omega^2 = \frac{\rho_0}{\rho} \cdot \frac g a\implies T = \frac{2 \pi}{\omega} = 2 \pi\sqrt{\frac{\rho}{\rho_0} \cdot \frac a g}.
    \end{align*}
}
\solutionspace{150pt}

\tasknumber{10}%
\task{%
    Математический маятник с нитью длиной $40\,\text{см}$ подвешен к потолку в лифте.
    За $20\,\text{с}$ маятник совершил $18$ колебаний.
    Определите модуль и направление ускорения лифта.
    Куда движется лифт?
}
\answer{%
    $
        T = 2\pi\sqrt{\frac\ell {a + g}}, T = \frac {t}{N}
        \implies a + g = \ell \cdot \frac{4 \pi ^ 2}{T^2},
        a = \ell \cdot \frac{4 \pi ^ 2}{T^2} - g = \ell \cdot \frac{4 \pi ^ 2 N^2}{t^2} - g \approx 2{,}79\,\frac{\text{м}}{\text{c}^{2}},
        \text{вниз}.
    $
}
\solutionspace{120pt}

\tasknumber{11}%
\task{%
    Масса груза в пружинном маятнике равна $500\,\text{г}$, при этом период его колебаний равен $1{,}3\,\text{с}$.
    Груз утяжеляют на $100\,\text{г}$.
    Определите новый период колебаний маятника.
}
\answer{%
    $
        T'
            = 2\pi\sqrt{\frac{M + m}{k}}
            = 2\pi\sqrt{\frac{M}{k} \cdot \frac{M + m}{M}}
            = T\sqrt{\frac{M + m}{M}} =  T\sqrt{1 + \frac{m}{M}} \approx 1{,}42\,\text{с}.
    $
}
\solutionspace{120pt}

\tasknumber{12}%
\task{%
    При какой длине нити математического маятника период колебаний груза массой $300\,\text{г}$
    окажется равен периоду колебаний этого же груза в пружинном маятнике с пружиной жёсткостью $50\,\frac{\text{Н}}{\text{м}}$?
}
\answer{%
    $
        2\pi \sqrt{\frac \ell g} = 2\pi \sqrt{\frac m k}
        \implies \frac \ell g = \frac m k
        \implies \ell = g \frac m k \approx 6{,}0\,\text{см}.
    $
}
\solutionspace{120pt}

\tasknumber{13}%
\task{%
    Груз подвесили к пружине, при этом удлинение пружины составило $30\,\text{мм}$.
    Определите частоту колебаний пружинного маятника, собранного из этой пружины и этого груза.
}
\answer{%
    $
        mg -k\Delta x = 0 \implies \frac m k = \frac{\Delta x} g
        \implies T = 2\pi \sqrt{\frac m k } = 2\pi \sqrt{\frac{\Delta x} g } \approx $0{,}34\,\text{с}$,
        \nu = \frac 1T \approx $2{,}91\,\text{Гц}$.
    $
}
\solutionspace{120pt}

\tasknumber{14}%
\task{%
    Определите период колебаний системы: математический маятник ограничен с одной стороны стенкой (см.
    рис.
    на доске).
    Удары маятника о стенку абсолютно упругие, $n = \sqrt{2}$.
    Длина маятника $\ell$, ускорение свободного падения $g$.
}
\answer{%
    $
        T = 2\pi\sqrt{\frac\ell g}, \qquad
        T' = 2 \cdot \frac T 4 + 2 \cdot \frac T{8} = \frac34T = \frac32\pi\sqrt{\frac\ell g}
    $
}

\variantsplitter

\addpersonalvariant{Сергей Малышев}

\tasknumber{1}%
\task{%
    \begin{itemize}
        \item Запишите линейное однородное дифференциальное уравнение второго порядка,
            описывающее свободные незатухающие колебания гармонического осциллятора,
        \item запишите общее решение этого уравнения,
        \item подпишите в выписанном решении фазу и амплитуду колебаний,
        \item запишите выражение для скорости,
        \item запишите выражение для ускорения.
    \end{itemize}
}
\answer{%
    \begin{align*}
    &\ddot x + \omega^2 x = 0 \Longleftrightarrow a_x + \omega^2 x = 0, \\
    &x = A \cos(\omega t + \varphi_0) \text{ или же } x = A \sin(\omega t + \varphi_0) \text{ или же } x = a \cos(\omega t) + b \sin(\omega t), \\
    &A \text{\, или \,} \sqrt{a^2 + b^2} \text{ --- это амплитуда}, \omega t + \varphi_0\text{ --- это фаза}, \\
    &v = \dot x = -\omega A \sin(\omega t + \varphi_0), \\
    &a = \dot v = \ddot x = -\omega^2 A \cos(\omega t + \varphi_0) = -\omega^2 x,
    \end{align*}
}
\solutionspace{135pt}

\tasknumber{2}%
\task{%
    Тело колеблется по гармоническому закону,
    амплитуда этих колебаний $12\,\text{см}$, период $6\,\text{c}$.
    Чему равно смещение тела относительно положения равновесия через $26\,\text{c}$
    после прохождения положения максимального отклонения?
}
\answer{%
    $x = A \cos \omega t = A \cos \cbr{ \frac {2\pi}T t } = A \cos \cbr{ 2\pi \frac tT } = 12\,\text{см} \cdot \cos \cbr{ 2\pi \cdot \frac {26\,\text{c}}{6\,\text{c}}}\approx -6{,}00\,\text{см}.$
}
\solutionspace{120pt}

\tasknumber{3}%
\task{%
    Тело совершает гармонические колебания с периодом $5\,\text{c}$.
    За какое время тело смещается от положения наибольшего отклонения до смещения в половину амплитуды?
}
\answer{%
    $t = \frac T{6} \approx 0{,}8\,\text{c}.$
}
\solutionspace{120pt}

\tasknumber{4}%
\task{%
    Запишите формулу для периода колебаний пружинного маятника и ...
    \begin{itemize}
        \item укажите названия всех физических величин в формуле,
        \item выразите из неё частоту колебаний
        \item выразите из неё жёсткость пружины.
    \end{itemize}
}
\answer{%
    \begin{align*}
    T &= 2\pi \sqrt{\frac lg} \implies \nu = \frac 1T = \frac 1{2\pi}\sqrt{\frac gl}, \omega = 2\pi\nu = \sqrt{\frac gl}, l = g\sqr{\frac T{2\pi}}, g = l\sqr{\frac {2\pi}T} \\
    T &= 2\pi \sqrt{\frac mk} \implies \nu = \frac 1T = \frac 1{2\pi}\sqrt{\frac km}, \omega = 2\pi\nu = \sqrt{\frac km}, m = k\sqr{\frac T{2\pi}}, k = m\sqr{\frac {2\pi}T}
    \end{align*}
}
\solutionspace{120pt}

\tasknumber{5}%
\task{%
    Частота колебаний пружинного маятника равна $15\,\text{Гц}$.
    Определите периоды колебаний
    \begin{itemize}
        \item кинетической энергии системы,
        \item ускорения груза,
        \item модуля скорости груза.
    \end{itemize}
}
\answer{%
    $T = \frac 1\nu \approx 66{,}7\,\text{мc}, T_1 = \frac T2 \approx 33{,}3\,\text{мc}, T_2 = T \approx 66{,}7\,\text{мc}, T_3 = \frac T2 \approx 33{,}3\,\text{мc}.$
}
\solutionspace{80pt}

\tasknumber{6}%
\task{%
    Тело колеблется по гармоническому закону с амплитудой $12\,\text{см}$.
    Какой минимальный путь тело может пройти за половину периода?
}
\answer{%
    $2A \approx 24{,}0\,\text{см}.$
}
\solutionspace{80pt}

\tasknumber{7}%
\task{%
    Период колебаний математического маятника равен $2\,\text{с}$,
    а их амплитуда — $10\,\text{см}$.
    Определите амплитуду колебаний скорости маятника.
}
\answer{%
    $
        T = \frac{2\pi}{\omega}
        \implies \omega = \frac{2\pi}{T}
        \implies v_{\max} = \omega A = \frac{2\pi}{T}A
        \approx 31{,}4\,\frac{\text{см}}{\text{с}}.
    $
}
\solutionspace{100pt}

\tasknumber{8}%
\task{%
    Определите период колебаний груза массой $m$, подвешенного к пружине жёсткостью $k$.
    Ускорение свободного падения $g$.
}
\answer{%
    \begin{align*}
    &-kx_0 + mg = 0, \\
    F &= -k(x_0 + \Delta x), \\
    ma &= -k(x_0 + \Delta x) + mg, \\
    ma &= -kx_0 -k \Delta x + mg = -mg -k \Delta x + mg = -k \Delta x, \\
    a &+ \frac k m x = 0, \\
    \omega^2 &= \frac k m \implies T = \frac{2\pi}\omega = 2\pi\sqrt{\frac m k}.
    \end{align*}
}
\solutionspace{150pt}

\tasknumber{9}%
\task{%
    Куб со стороной $a$ и плотности $\rho$ плавает в жидкости плотностью $\rho_0$.
    Определите частоту колебаний куба, считая, что 4 грани куба всегда вертикальны.
}
\answer{%
    \begin{align*}
    \Delta F &= -\rho_0 g a^2 \Delta x,\Delta F = m\ddot x, m = \rho a^3\implies \rho a^3 \ddot x = -\rho_0 g a^2 \Delta x \implies \\
    \implies \ddot x &= -\frac{\rho_0}{\rho} \frac g a \Delta x\implies \omega^2 = \frac{\rho_0}{\rho} \cdot \frac g a\implies T = \frac{2 \pi}{\omega} = 2 \pi\sqrt{\frac{\rho}{\rho_0} \cdot \frac a g}.
    \end{align*}
}
\solutionspace{150pt}

\tasknumber{10}%
\task{%
    Математический маятник с нитью длиной $37\,\text{см}$ подвешен к потолку в лифте.
    За $20\,\text{с}$ маятник совершил $18$ колебаний.
    Определите модуль и направление ускорения лифта.
    Куда движется лифт?
}
\answer{%
    $
        T = 2\pi\sqrt{\frac\ell {a + g}}, T = \frac {t}{N}
        \implies a + g = \ell \cdot \frac{4 \pi ^ 2}{T^2},
        a = \ell \cdot \frac{4 \pi ^ 2}{T^2} - g = \ell \cdot \frac{4 \pi ^ 2 N^2}{t^2} - g \approx 1{,}83\,\frac{\text{м}}{\text{c}^{2}},
        \text{вниз}.
    $
}
\solutionspace{120pt}

\tasknumber{11}%
\task{%
    Масса груза в пружинном маятнике равна $600\,\text{г}$, при этом период его колебаний равен $1{,}4\,\text{с}$.
    Груз утяжеляют на $50\,\text{г}$.
    Определите новый период колебаний маятника.
}
\answer{%
    $
        T'
            = 2\pi\sqrt{\frac{M + m}{k}}
            = 2\pi\sqrt{\frac{M}{k} \cdot \frac{M + m}{M}}
            = T\sqrt{\frac{M + m}{M}} =  T\sqrt{1 + \frac{m}{M}} \approx 1{,}46\,\text{с}.
    $
}
\solutionspace{120pt}

\tasknumber{12}%
\task{%
    При какой длине нити математического маятника период колебаний груза массой $300\,\text{г}$
    окажется равен периоду колебаний этого же груза в пружинном маятнике с пружиной жёсткостью $40\,\frac{\text{Н}}{\text{м}}$?
}
\answer{%
    $
        2\pi \sqrt{\frac \ell g} = 2\pi \sqrt{\frac m k}
        \implies \frac \ell g = \frac m k
        \implies \ell = g \frac m k \approx 7{,}5\,\text{см}.
    $
}
\solutionspace{120pt}

\tasknumber{13}%
\task{%
    Груз подвесили к пружине, при этом удлинение пружины составило $45\,\text{мм}$.
    Определите частоту колебаний пружинного маятника, собранного из этой пружины и этого груза.
}
\answer{%
    $
        mg -k\Delta x = 0 \implies \frac m k = \frac{\Delta x} g
        \implies T = 2\pi \sqrt{\frac m k } = 2\pi \sqrt{\frac{\Delta x} g } \approx $0{,}42\,\text{с}$,
        \nu = \frac 1T \approx $2{,}37\,\text{Гц}$.
    $
}
\solutionspace{120pt}

\tasknumber{14}%
\task{%
    Определите период колебаний системы: математический маятник ограничен с одной стороны стенкой (см.
    рис.
    на доске).
    Удары маятника о стенку абсолютно упругие, $n = \sqrt{2}$.
    Длина маятника $\ell$, ускорение свободного падения $g$.
}
\answer{%
    $
        T = 2\pi\sqrt{\frac\ell g}, \qquad
        T' = 2 \cdot \frac T 4 + 2 \cdot \frac T{8} = \frac34T = \frac32\pi\sqrt{\frac\ell g}
    $
}

\variantsplitter

\addpersonalvariant{Алина Полканова}

\tasknumber{1}%
\task{%
    \begin{itemize}
        \item Запишите линейное однородное дифференциальное уравнение второго порядка,
            описывающее свободные незатухающие колебания гармонического осциллятора,
        \item запишите общее решение этого уравнения,
        \item подпишите в выписанном решении фазу и амплитуду колебаний,
        \item запишите выражение для скорости,
        \item запишите выражение для ускорения.
    \end{itemize}
}
\answer{%
    \begin{align*}
    &\ddot x + \omega^2 x = 0 \Longleftrightarrow a_x + \omega^2 x = 0, \\
    &x = A \cos(\omega t + \varphi_0) \text{ или же } x = A \sin(\omega t + \varphi_0) \text{ или же } x = a \cos(\omega t) + b \sin(\omega t), \\
    &A \text{\, или \,} \sqrt{a^2 + b^2} \text{ --- это амплитуда}, \omega t + \varphi_0\text{ --- это фаза}, \\
    &v = \dot x = -\omega A \sin(\omega t + \varphi_0), \\
    &a = \dot v = \ddot x = -\omega^2 A \cos(\omega t + \varphi_0) = -\omega^2 x,
    \end{align*}
}
\solutionspace{135pt}

\tasknumber{2}%
\task{%
    Тело колеблется по гармоническому закону,
    амплитуда этих колебаний $16\,\text{см}$, период $2\,\text{c}$.
    Чему равно смещение тела относительно положения равновесия через $24\,\text{c}$
    после прохождения положения равновесия?
}
\answer{%
    $x = A \sin \omega t = A \sin \cbr{ \frac {2\pi}T t } = A \sin \cbr{ 2\pi \frac tT } = 16\,\text{см} \cdot \sin \cbr{ 2\pi \cdot \frac {24\,\text{c}}{2\,\text{c}}}\approx 0\,\text{см}.$
}
\solutionspace{120pt}

\tasknumber{3}%
\task{%
    Тело совершает гармонические колебания с периодом $5\,\text{c}$.
    За какое время тело смещается от положения равновесия до смещения в половину амплитуды?
}
\answer{%
    $t = \frac T{12} \approx 0{,}4\,\text{c}.$
}
\solutionspace{120pt}

\tasknumber{4}%
\task{%
    Запишите формулу для периода колебаний пружинного маятника и ...
    \begin{itemize}
        \item укажите названия всех физических величин в формуле,
        \item выразите из неё циклическую частоту колебаний
        \item выразите из неё жёсткость пружины.
    \end{itemize}
}
\answer{%
    \begin{align*}
    T &= 2\pi \sqrt{\frac lg} \implies \nu = \frac 1T = \frac 1{2\pi}\sqrt{\frac gl}, \omega = 2\pi\nu = \sqrt{\frac gl}, l = g\sqr{\frac T{2\pi}}, g = l\sqr{\frac {2\pi}T} \\
    T &= 2\pi \sqrt{\frac mk} \implies \nu = \frac 1T = \frac 1{2\pi}\sqrt{\frac km}, \omega = 2\pi\nu = \sqrt{\frac km}, m = k\sqr{\frac T{2\pi}}, k = m\sqr{\frac {2\pi}T}
    \end{align*}
}
\solutionspace{120pt}

\tasknumber{5}%
\task{%
    Частота колебаний пружинного маятника равна $15\,\text{Гц}$.
    Определите периоды колебаний
    \begin{itemize}
        \item потенциальной энергии системы,
        \item скорости груза,
        \item модуля скорости груза.
    \end{itemize}
}
\answer{%
    $T = \frac 1\nu \approx 66{,}7\,\text{мc}, T_1 = \frac T2 \approx 33{,}3\,\text{мc}, T_2 = T \approx 66{,}7\,\text{мc}, T_3 = \frac T2 \approx 33{,}3\,\text{мc}.$
}
\solutionspace{80pt}

\tasknumber{6}%
\task{%
    Тело колеблется по гармоническому закону с амплитудой $6\,\text{см}$.
    Какой минимальный путь тело может пройти за четверть периода?
}
\answer{%
    $2 * A \cbr{1 - \frac 1{\sqrt 2}} = A\cbr{2 - \sqrt{2}} \approx 3{,}5\,\text{см}.$
}
\solutionspace{80pt}

\tasknumber{7}%
\task{%
    Период колебаний математического маятника равен $5\,\text{с}$,
    а их амплитуда — $15\,\text{см}$.
    Определите максимальную скорость маятника.
}
\answer{%
    $
        T = \frac{2\pi}{\omega}
        \implies \omega = \frac{2\pi}{T}
        \implies v_{\max} = \omega A = \frac{2\pi}{T}A
        \approx 18{,}8\,\frac{\text{см}}{\text{с}}.
    $
}
\solutionspace{100pt}

\tasknumber{8}%
\task{%
    Определите период колебаний груза массой $m$, подвешенного к пружине жёсткостью $k$.
    Ускорение свободного падения $g$.
}
\answer{%
    \begin{align*}
    &-kx_0 + mg = 0, \\
    F &= -k(x_0 + \Delta x), \\
    ma &= -k(x_0 + \Delta x) + mg, \\
    ma &= -kx_0 -k \Delta x + mg = -mg -k \Delta x + mg = -k \Delta x, \\
    a &+ \frac k m x = 0, \\
    \omega^2 &= \frac k m \implies T = \frac{2\pi}\omega = 2\pi\sqrt{\frac m k}.
    \end{align*}
}
\solutionspace{150pt}

\tasknumber{9}%
\task{%
    Куб со стороной $a$ и плотности $\rho$ плавает в жидкости плотностью $\rho_0$.
    Определите частоту колебаний куба, считая, что 4 грани куба всегда вертикальны.
}
\answer{%
    \begin{align*}
    \Delta F &= -\rho_0 g a^2 \Delta x,\Delta F = m\ddot x, m = \rho a^3\implies \rho a^3 \ddot x = -\rho_0 g a^2 \Delta x \implies \\
    \implies \ddot x &= -\frac{\rho_0}{\rho} \frac g a \Delta x\implies \omega^2 = \frac{\rho_0}{\rho} \cdot \frac g a\implies T = \frac{2 \pi}{\omega} = 2 \pi\sqrt{\frac{\rho}{\rho_0} \cdot \frac a g}.
    \end{align*}
}
\solutionspace{150pt}

\tasknumber{10}%
\task{%
    Математический маятник с нитью длиной $40\,\text{см}$ подвешен к потолку в лифте.
    За $20\,\text{с}$ маятник совершил $15$ колебаний.
    Определите модуль и направление ускорения лифта.
    Куда движется лифт?
}
\answer{%
    $
        T = 2\pi\sqrt{\frac\ell {a + g}}, T = \frac {t}{N}
        \implies a + g = \ell \cdot \frac{4 \pi ^ 2}{T^2},
        a = \ell \cdot \frac{4 \pi ^ 2}{T^2} - g = \ell \cdot \frac{4 \pi ^ 2 N^2}{t^2} - g \approx -1{,}1200\,\frac{\text{м}}{\text{c}^{2}},
        \text{вверх}.
    $
}
\solutionspace{120pt}

\tasknumber{11}%
\task{%
    Масса груза в пружинном маятнике равна $600\,\text{г}$, при этом период его колебаний равен $1{,}5\,\text{с}$.
    Груз облегчают на $100\,\text{г}$.
    Определите новый период колебаний маятника.
}
\answer{%
    $
        T'
            = 2\pi\sqrt{\frac{M - m}{k}}
            = 2\pi\sqrt{\frac{M}{k} \cdot \frac{M - m}{M}}
            = T\sqrt{\frac{M - m}{M}} =  T\sqrt{1 - \frac{m}{M}} \approx 1{,}37\,\text{с}.
    $
}
\solutionspace{120pt}

\tasknumber{12}%
\task{%
    При какой длине нити математического маятника период колебаний груза массой $200\,\text{г}$
    окажется равен периоду колебаний этого же груза в пружинном маятнике с пружиной жёсткостью $60\,\frac{\text{Н}}{\text{м}}$?
}
\answer{%
    $
        2\pi \sqrt{\frac \ell g} = 2\pi \sqrt{\frac m k}
        \implies \frac \ell g = \frac m k
        \implies \ell = g \frac m k \approx 3{,}3\,\text{см}.
    $
}
\solutionspace{120pt}

\tasknumber{13}%
\task{%
    Груз подвесили к пружине, при этом удлинение пружины составило $45\,\text{мм}$.
    Определите частоту колебаний пружинного маятника, собранного из этой пружины и этого груза.
}
\answer{%
    $
        mg -k\Delta x = 0 \implies \frac m k = \frac{\Delta x} g
        \implies T = 2\pi \sqrt{\frac m k } = 2\pi \sqrt{\frac{\Delta x} g } \approx $0{,}42\,\text{с}$,
        \nu = \frac 1T \approx $2{,}37\,\text{Гц}$.
    $
}
\solutionspace{120pt}

\tasknumber{14}%
\task{%
    Определите период колебаний системы: математический маятник ограничен с одной стороны стенкой (см.
    рис.
    на доске).
    Удары маятника о стенку абсолютно упругие, $n = \sqrt{2}$.
    Длина маятника $\ell$, ускорение свободного падения $g$.
}
\answer{%
    $
        T = 2\pi\sqrt{\frac\ell g}, \qquad
        T' = 2 \cdot \frac T 4 + 2 \cdot \frac T{8} = \frac34T = \frac32\pi\sqrt{\frac\ell g}
    $
}

\variantsplitter

\addpersonalvariant{Сергей Пономарёв}

\tasknumber{1}%
\task{%
    \begin{itemize}
        \item Запишите линейное однородное дифференциальное уравнение второго порядка,
            описывающее свободные незатухающие колебания гармонического осциллятора,
        \item запишите общее решение этого уравнения,
        \item подпишите в выписанном решении фазу и амплитуду колебаний,
        \item запишите выражение для скорости,
        \item запишите выражение для ускорения.
    \end{itemize}
}
\answer{%
    \begin{align*}
    &\ddot x + \omega^2 x = 0 \Longleftrightarrow a_x + \omega^2 x = 0, \\
    &x = A \cos(\omega t + \varphi_0) \text{ или же } x = A \sin(\omega t + \varphi_0) \text{ или же } x = a \cos(\omega t) + b \sin(\omega t), \\
    &A \text{\, или \,} \sqrt{a^2 + b^2} \text{ --- это амплитуда}, \omega t + \varphi_0\text{ --- это фаза}, \\
    &v = \dot x = -\omega A \sin(\omega t + \varphi_0), \\
    &a = \dot v = \ddot x = -\omega^2 A \cos(\omega t + \varphi_0) = -\omega^2 x,
    \end{align*}
}
\solutionspace{135pt}

\tasknumber{2}%
\task{%
    Тело колеблется по гармоническому закону,
    амплитуда этих колебаний $12\,\text{см}$, период $2\,\text{c}$.
    Чему равно смещение тела относительно положения равновесия через $22\,\text{c}$
    после прохождения положения максимального отклонения?
}
\answer{%
    $x = A \cos \omega t = A \cos \cbr{ \frac {2\pi}T t } = A \cos \cbr{ 2\pi \frac tT } = 12\,\text{см} \cdot \cos \cbr{ 2\pi \cdot \frac {22\,\text{c}}{2\,\text{c}}}\approx 12{,}0\,\text{см}.$
}
\solutionspace{120pt}

\tasknumber{3}%
\task{%
    Тело совершает гармонические колебания с периодом $4\,\text{c}$.
    За какое время тело смещается от положения равновесия до смещения в половину амплитуды?
}
\answer{%
    $t = \frac T{12} \approx 0{,}3\,\text{c}.$
}
\solutionspace{120pt}

\tasknumber{4}%
\task{%
    Запишите формулу для периода колебаний математического маятника и ...
    \begin{itemize}
        \item укажите названия всех физических величин в формуле,
        \item выразите из неё циклическую частоту колебаний
        \item выразите из неё длину маятника.
    \end{itemize}
}
\answer{%
    \begin{align*}
    T &= 2\pi \sqrt{\frac lg} \implies \nu = \frac 1T = \frac 1{2\pi}\sqrt{\frac gl}, \omega = 2\pi\nu = \sqrt{\frac gl}, l = g\sqr{\frac T{2\pi}}, g = l\sqr{\frac {2\pi}T} \\
    T &= 2\pi \sqrt{\frac mk} \implies \nu = \frac 1T = \frac 1{2\pi}\sqrt{\frac km}, \omega = 2\pi\nu = \sqrt{\frac km}, m = k\sqr{\frac T{2\pi}}, k = m\sqr{\frac {2\pi}T}
    \end{align*}
}
\solutionspace{120pt}

\tasknumber{5}%
\task{%
    Частота колебаний математического маятника равна $12\,\text{Гц}$.
    Определите периоды колебаний
    \begin{itemize}
        \item потенциальной энергии системы,
        \item ускорения груза,
        \item модуля скорости груза.
    \end{itemize}
}
\answer{%
    $T = \frac 1\nu \approx 83{,}3\,\text{мc}, T_1 = \frac T2 \approx 41{,}7\,\text{мc}, T_2 = T \approx 83{,}3\,\text{мc}, T_3 = \frac T2 \approx 41{,}7\,\text{мc}.$
}
\solutionspace{80pt}

\tasknumber{6}%
\task{%
    Тело колеблется по гармоническому закону с амплитудой $6\,\text{см}$.
    Какой минимальный путь тело может пройти за одну шестую долю периода?
}
\answer{%
    $2 * A \frac{1}{12} = \frac{A:L:s}3 \approx 1{,}0\,\text{см}.$
}
\solutionspace{80pt}

\tasknumber{7}%
\task{%
    Период колебаний математического маятника равен $2\,\text{с}$,
    а их амплитуда — $10\,\text{см}$.
    Определите амплитуду колебаний скорости маятника.
}
\answer{%
    $
        T = \frac{2\pi}{\omega}
        \implies \omega = \frac{2\pi}{T}
        \implies v_{\max} = \omega A = \frac{2\pi}{T}A
        \approx 31{,}4\,\frac{\text{см}}{\text{с}}.
    $
}
\solutionspace{100pt}

\tasknumber{8}%
\task{%
    Определите период колебаний груза массой $m$, подвешенного к пружине жёсткостью $k$.
    Ускорение свободного падения $g$.
}
\answer{%
    \begin{align*}
    &-kx_0 + mg = 0, \\
    F &= -k(x_0 + \Delta x), \\
    ma &= -k(x_0 + \Delta x) + mg, \\
    ma &= -kx_0 -k \Delta x + mg = -mg -k \Delta x + mg = -k \Delta x, \\
    a &+ \frac k m x = 0, \\
    \omega^2 &= \frac k m \implies T = \frac{2\pi}\omega = 2\pi\sqrt{\frac m k}.
    \end{align*}
}
\solutionspace{150pt}

\tasknumber{9}%
\task{%
    Куб со стороной $a$ и плотности $\rho$ плавает в жидкости плотностью $\rho_0$.
    Определите частоту колебаний куба, считая, что 4 грани куба всегда вертикальны.
}
\answer{%
    \begin{align*}
    \Delta F &= -\rho_0 g a^2 \Delta x,\Delta F = m\ddot x, m = \rho a^3\implies \rho a^3 \ddot x = -\rho_0 g a^2 \Delta x \implies \\
    \implies \ddot x &= -\frac{\rho_0}{\rho} \frac g a \Delta x\implies \omega^2 = \frac{\rho_0}{\rho} \cdot \frac g a\implies T = \frac{2 \pi}{\omega} = 2 \pi\sqrt{\frac{\rho}{\rho_0} \cdot \frac a g}.
    \end{align*}
}
\solutionspace{150pt}

\tasknumber{10}%
\task{%
    Математический маятник с нитью длиной $37\,\text{см}$ подвешен к потолку в лифте.
    За $30\,\text{с}$ маятник совершил $23$ колебаний.
    Определите модуль и направление ускорения лифта.
    Куда движется лифт?
}
\answer{%
    $
        T = 2\pi\sqrt{\frac\ell {a + g}}, T = \frac {t}{N}
        \implies a + g = \ell \cdot \frac{4 \pi ^ 2}{T^2},
        a = \ell \cdot \frac{4 \pi ^ 2}{T^2} - g = \ell \cdot \frac{4 \pi ^ 2 N^2}{t^2} - g \approx -1{,}4100\,\frac{\text{м}}{\text{c}^{2}},
        \text{вверх}.
    $
}
\solutionspace{120pt}

\tasknumber{11}%
\task{%
    Масса груза в пружинном маятнике равна $400\,\text{г}$, при этом период его колебаний равен $1{,}2\,\text{с}$.
    Груз облегчают на $50\,\text{г}$.
    Определите новый период колебаний маятника.
}
\answer{%
    $
        T'
            = 2\pi\sqrt{\frac{M - m}{k}}
            = 2\pi\sqrt{\frac{M}{k} \cdot \frac{M - m}{M}}
            = T\sqrt{\frac{M - m}{M}} =  T\sqrt{1 - \frac{m}{M}} \approx 1{,}12\,\text{с}.
    $
}
\solutionspace{120pt}

\tasknumber{12}%
\task{%
    При какой длине нити математического маятника период колебаний груза массой $300\,\text{г}$
    окажется равен периоду колебаний этого же груза в пружинном маятнике с пружиной жёсткостью $40\,\frac{\text{Н}}{\text{м}}$?
}
\answer{%
    $
        2\pi \sqrt{\frac \ell g} = 2\pi \sqrt{\frac m k}
        \implies \frac \ell g = \frac m k
        \implies \ell = g \frac m k \approx 7{,}5\,\text{см}.
    $
}
\solutionspace{120pt}

\tasknumber{13}%
\task{%
    Груз подвесили к пружине, при этом удлинение пружины составило $30\,\text{мм}$.
    Определите частоту колебаний пружинного маятника, собранного из этой пружины и этого груза.
}
\answer{%
    $
        mg -k\Delta x = 0 \implies \frac m k = \frac{\Delta x} g
        \implies T = 2\pi \sqrt{\frac m k } = 2\pi \sqrt{\frac{\Delta x} g } \approx $0{,}34\,\text{с}$,
        \nu = \frac 1T \approx $2{,}91\,\text{Гц}$.
    $
}
\solutionspace{120pt}

\tasknumber{14}%
\task{%
    Определите период колебаний системы: математический маятник ограничен с одной стороны стенкой (см.
    рис.
    на доске).
    Удары маятника о стенку абсолютно упругие, $n = 2$.
    Длина маятника $\ell$, ускорение свободного падения $g$.
}
\answer{%
    $
        T = 2\pi\sqrt{\frac\ell g}, \qquad
        T' = 2 \cdot \frac T 4 + 2 \cdot \frac T{12} = \frac23T = \frac43\pi\sqrt{\frac\ell g}
    $
}

\variantsplitter

\addpersonalvariant{Егор Свистушкин}

\tasknumber{1}%
\task{%
    \begin{itemize}
        \item Запишите линейное однородное дифференциальное уравнение второго порядка,
            описывающее свободные незатухающие колебания гармонического осциллятора,
        \item запишите общее решение этого уравнения,
        \item подпишите в выписанном решении фазу и амплитуду колебаний,
        \item запишите выражение для скорости,
        \item запишите выражение для ускорения.
    \end{itemize}
}
\answer{%
    \begin{align*}
    &\ddot x + \omega^2 x = 0 \Longleftrightarrow a_x + \omega^2 x = 0, \\
    &x = A \cos(\omega t + \varphi_0) \text{ или же } x = A \sin(\omega t + \varphi_0) \text{ или же } x = a \cos(\omega t) + b \sin(\omega t), \\
    &A \text{\, или \,} \sqrt{a^2 + b^2} \text{ --- это амплитуда}, \omega t + \varphi_0\text{ --- это фаза}, \\
    &v = \dot x = -\omega A \sin(\omega t + \varphi_0), \\
    &a = \dot v = \ddot x = -\omega^2 A \cos(\omega t + \varphi_0) = -\omega^2 x,
    \end{align*}
}
\solutionspace{135pt}

\tasknumber{2}%
\task{%
    Тело колеблется по гармоническому закону,
    амплитуда этих колебаний $10\,\text{см}$, период $6\,\text{c}$.
    Чему равно смещение тела относительно положения равновесия через $25\,\text{c}$
    после прохождения положения максимального отклонения?
}
\answer{%
    $x = A \cos \omega t = A \cos \cbr{ \frac {2\pi}T t } = A \cos \cbr{ 2\pi \frac tT } = 10\,\text{см} \cdot \cos \cbr{ 2\pi \cdot \frac {25\,\text{c}}{6\,\text{c}}}\approx 5{,}0\,\text{см}.$
}
\solutionspace{120pt}

\tasknumber{3}%
\task{%
    Тело совершает гармонические колебания с периодом $4\,\text{c}$.
    За какое время тело смещается от положения равновесия до смещения в половину амплитуды?
}
\answer{%
    $t = \frac T{12} \approx 0{,}3\,\text{c}.$
}
\solutionspace{120pt}

\tasknumber{4}%
\task{%
    Запишите формулу для периода колебаний пружинного маятника и ...
    \begin{itemize}
        \item укажите названия всех физических величин в формуле,
        \item выразите из неё частоту колебаний
        \item выразите из неё массу груза.
    \end{itemize}
}
\answer{%
    \begin{align*}
    T &= 2\pi \sqrt{\frac lg} \implies \nu = \frac 1T = \frac 1{2\pi}\sqrt{\frac gl}, \omega = 2\pi\nu = \sqrt{\frac gl}, l = g\sqr{\frac T{2\pi}}, g = l\sqr{\frac {2\pi}T} \\
    T &= 2\pi \sqrt{\frac mk} \implies \nu = \frac 1T = \frac 1{2\pi}\sqrt{\frac km}, \omega = 2\pi\nu = \sqrt{\frac km}, m = k\sqr{\frac T{2\pi}}, k = m\sqr{\frac {2\pi}T}
    \end{align*}
}
\solutionspace{120pt}

\tasknumber{5}%
\task{%
    Частота колебаний пружинного маятника равна $8\,\text{Гц}$.
    Определите периоды колебаний
    \begin{itemize}
        \item кинетической энергии системы,
        \item скорости груза,
        \item модуля скорости груза.
    \end{itemize}
}
\answer{%
    $T = \frac 1\nu \approx 125{,}0\,\text{мc}, T_1 = \frac T2 \approx 62{,}5\,\text{мc}, T_2 = T \approx 125{,}0\,\text{мc}, T_3 = \frac T2 \approx 62{,}5\,\text{мc}.$
}
\solutionspace{80pt}

\tasknumber{6}%
\task{%
    Тело колеблется по гармоническому закону с амплитудой $6\,\text{см}$.
    Какой максимальный путь тело может пройти за половину периода?
}
\answer{%
    $2A \approx 12{,}0\,\text{см}.$
}
\solutionspace{80pt}

\tasknumber{7}%
\task{%
    Период колебаний математического маятника равен $2\,\text{с}$,
    а их амплитуда — $10\,\text{см}$.
    Определите амплитуду колебаний скорости маятника.
}
\answer{%
    $
        T = \frac{2\pi}{\omega}
        \implies \omega = \frac{2\pi}{T}
        \implies v_{\max} = \omega A = \frac{2\pi}{T}A
        \approx 31{,}4\,\frac{\text{см}}{\text{с}}.
    $
}
\solutionspace{100pt}

\tasknumber{8}%
\task{%
    Определите период колебаний груза массой $m$, подвешенного к пружине жёсткостью $k$.
    Ускорение свободного падения $g$.
}
\answer{%
    \begin{align*}
    &-kx_0 + mg = 0, \\
    F &= -k(x_0 + \Delta x), \\
    ma &= -k(x_0 + \Delta x) + mg, \\
    ma &= -kx_0 -k \Delta x + mg = -mg -k \Delta x + mg = -k \Delta x, \\
    a &+ \frac k m x = 0, \\
    \omega^2 &= \frac k m \implies T = \frac{2\pi}\omega = 2\pi\sqrt{\frac m k}.
    \end{align*}
}
\solutionspace{150pt}

\tasknumber{9}%
\task{%
    Куб со стороной $a$ и плотности $\rho$ плавает в жидкости плотностью $\rho_0$.
    Определите частоту колебаний куба, считая, что 4 грани куба всегда вертикальны.
}
\answer{%
    \begin{align*}
    \Delta F &= -\rho_0 g a^2 \Delta x,\Delta F = m\ddot x, m = \rho a^3\implies \rho a^3 \ddot x = -\rho_0 g a^2 \Delta x \implies \\
    \implies \ddot x &= -\frac{\rho_0}{\rho} \frac g a \Delta x\implies \omega^2 = \frac{\rho_0}{\rho} \cdot \frac g a\implies T = \frac{2 \pi}{\omega} = 2 \pi\sqrt{\frac{\rho}{\rho_0} \cdot \frac a g}.
    \end{align*}
}
\solutionspace{150pt}

\tasknumber{10}%
\task{%
    Математический маятник с нитью длиной $37\,\text{см}$ подвешен к потолку в лифте.
    За $30\,\text{с}$ маятник совершил $27$ колебаний.
    Определите модуль и направление ускорения лифта.
    Куда движется лифт?
}
\answer{%
    $
        T = 2\pi\sqrt{\frac\ell {a + g}}, T = \frac {t}{N}
        \implies a + g = \ell \cdot \frac{4 \pi ^ 2}{T^2},
        a = \ell \cdot \frac{4 \pi ^ 2}{T^2} - g = \ell \cdot \frac{4 \pi ^ 2 N^2}{t^2} - g \approx 1{,}83\,\frac{\text{м}}{\text{c}^{2}},
        \text{вниз}.
    $
}
\solutionspace{120pt}

\tasknumber{11}%
\task{%
    Масса груза в пружинном маятнике равна $400\,\text{г}$, при этом период его колебаний равен $1{,}5\,\text{с}$.
    Груз утяжеляют на $150\,\text{г}$.
    Определите новый период колебаний маятника.
}
\answer{%
    $
        T'
            = 2\pi\sqrt{\frac{M + m}{k}}
            = 2\pi\sqrt{\frac{M}{k} \cdot \frac{M + m}{M}}
            = T\sqrt{\frac{M + m}{M}} =  T\sqrt{1 + \frac{m}{M}} \approx 1{,}76\,\text{с}.
    $
}
\solutionspace{120pt}

\tasknumber{12}%
\task{%
    При какой длине нити математического маятника период колебаний груза массой $400\,\text{г}$
    окажется равен периоду колебаний этого же груза в пружинном маятнике с пружиной жёсткостью $60\,\frac{\text{Н}}{\text{м}}$?
}
\answer{%
    $
        2\pi \sqrt{\frac \ell g} = 2\pi \sqrt{\frac m k}
        \implies \frac \ell g = \frac m k
        \implies \ell = g \frac m k \approx 6{,}7\,\text{см}.
    $
}
\solutionspace{120pt}

\tasknumber{13}%
\task{%
    Груз подвесили к пружине, при этом удлинение пружины составило $45\,\text{мм}$.
    Определите частоту колебаний пружинного маятника, собранного из этой пружины и этого груза.
}
\answer{%
    $
        mg -k\Delta x = 0 \implies \frac m k = \frac{\Delta x} g
        \implies T = 2\pi \sqrt{\frac m k } = 2\pi \sqrt{\frac{\Delta x} g } \approx $0{,}42\,\text{с}$,
        \nu = \frac 1T \approx $2{,}37\,\text{Гц}$.
    $
}
\solutionspace{120pt}

\tasknumber{14}%
\task{%
    Определите период колебаний системы: математический маятник ограничен с одной стороны стенкой (см.
    рис.
    на доске).
    Удары маятника о стенку абсолютно упругие, $n = \sqrt{2}$.
    Длина маятника $\ell$, ускорение свободного падения $g$.
}
\answer{%
    $
        T = 2\pi\sqrt{\frac\ell g}, \qquad
        T' = 2 \cdot \frac T 4 + 2 \cdot \frac T{8} = \frac34T = \frac32\pi\sqrt{\frac\ell g}
    $
}

\variantsplitter

\addpersonalvariant{Дмитрий Соколов}

\tasknumber{1}%
\task{%
    \begin{itemize}
        \item Запишите линейное однородное дифференциальное уравнение второго порядка,
            описывающее свободные незатухающие колебания гармонического осциллятора,
        \item запишите общее решение этого уравнения,
        \item подпишите в выписанном решении фазу и амплитуду колебаний,
        \item запишите выражение для скорости,
        \item запишите выражение для ускорения.
    \end{itemize}
}
\answer{%
    \begin{align*}
    &\ddot x + \omega^2 x = 0 \Longleftrightarrow a_x + \omega^2 x = 0, \\
    &x = A \cos(\omega t + \varphi_0) \text{ или же } x = A \sin(\omega t + \varphi_0) \text{ или же } x = a \cos(\omega t) + b \sin(\omega t), \\
    &A \text{\, или \,} \sqrt{a^2 + b^2} \text{ --- это амплитуда}, \omega t + \varphi_0\text{ --- это фаза}, \\
    &v = \dot x = -\omega A \sin(\omega t + \varphi_0), \\
    &a = \dot v = \ddot x = -\omega^2 A \cos(\omega t + \varphi_0) = -\omega^2 x,
    \end{align*}
}
\solutionspace{135pt}

\tasknumber{2}%
\task{%
    Тело колеблется по гармоническому закону,
    амплитуда этих колебаний $20\,\text{см}$, период $2\,\text{c}$.
    Чему равно смещение тела относительно положения равновесия через $24\,\text{c}$
    после прохождения положения максимального отклонения?
}
\answer{%
    $x = A \cos \omega t = A \cos \cbr{ \frac {2\pi}T t } = A \cos \cbr{ 2\pi \frac tT } = 20\,\text{см} \cdot \cos \cbr{ 2\pi \cdot \frac {24\,\text{c}}{2\,\text{c}}}\approx 20{,}0\,\text{см}.$
}
\solutionspace{120pt}

\tasknumber{3}%
\task{%
    Тело совершает гармонические колебания с периодом $6\,\text{c}$.
    За какое время тело смещается от положения наибольшего отклонения до смещения в половину амплитуды?
}
\answer{%
    $t = \frac T{6} \approx 1{,}0\,\text{c}.$
}
\solutionspace{120pt}

\tasknumber{4}%
\task{%
    Запишите формулу для периода колебаний пружинного маятника и ...
    \begin{itemize}
        \item укажите названия всех физических величин в формуле,
        \item выразите из неё частоту колебаний
        \item выразите из неё жёсткость пружины.
    \end{itemize}
}
\answer{%
    \begin{align*}
    T &= 2\pi \sqrt{\frac lg} \implies \nu = \frac 1T = \frac 1{2\pi}\sqrt{\frac gl}, \omega = 2\pi\nu = \sqrt{\frac gl}, l = g\sqr{\frac T{2\pi}}, g = l\sqr{\frac {2\pi}T} \\
    T &= 2\pi \sqrt{\frac mk} \implies \nu = \frac 1T = \frac 1{2\pi}\sqrt{\frac km}, \omega = 2\pi\nu = \sqrt{\frac km}, m = k\sqr{\frac T{2\pi}}, k = m\sqr{\frac {2\pi}T}
    \end{align*}
}
\solutionspace{120pt}

\tasknumber{5}%
\task{%
    Частота колебаний пружинного маятника равна $10\,\text{Гц}$.
    Определите периоды колебаний
    \begin{itemize}
        \item кинетической энергии системы,
        \item скорости груза,
        \item модуля ускорения груза.
    \end{itemize}
}
\answer{%
    $T = \frac 1\nu \approx 100{,}0\,\text{мc}, T_1 = \frac T2 \approx 50{,}0\,\text{мc}, T_2 = T \approx 100{,}0\,\text{мc}, T_3 = \frac T2 \approx 50{,}0\,\text{мc}.$
}
\solutionspace{80pt}

\tasknumber{6}%
\task{%
    Тело колеблется по гармоническому закону с амплитудой $12\,\text{см}$.
    Какой минимальный путь тело может пройти за одну шестую долю периода?
}
\answer{%
    $2 * A \frac{1}{12} = \frac{A:L:s}3 \approx 2{,}0\,\text{см}.$
}
\solutionspace{80pt}

\tasknumber{7}%
\task{%
    Период колебаний математического маятника равен $4\,\text{с}$,
    а их амплитуда — $20\,\text{см}$.
    Определите максимальную скорость маятника.
}
\answer{%
    $
        T = \frac{2\pi}{\omega}
        \implies \omega = \frac{2\pi}{T}
        \implies v_{\max} = \omega A = \frac{2\pi}{T}A
        \approx 31{,}4\,\frac{\text{см}}{\text{с}}.
    $
}
\solutionspace{100pt}

\tasknumber{8}%
\task{%
    Определите период колебаний груза массой $m$, подвешенного к пружине жёсткостью $k$.
    Ускорение свободного падения $g$.
}
\answer{%
    \begin{align*}
    &-kx_0 + mg = 0, \\
    F &= -k(x_0 + \Delta x), \\
    ma &= -k(x_0 + \Delta x) + mg, \\
    ma &= -kx_0 -k \Delta x + mg = -mg -k \Delta x + mg = -k \Delta x, \\
    a &+ \frac k m x = 0, \\
    \omega^2 &= \frac k m \implies T = \frac{2\pi}\omega = 2\pi\sqrt{\frac m k}.
    \end{align*}
}
\solutionspace{150pt}

\tasknumber{9}%
\task{%
    Куб со стороной $a$ и плотности $\rho$ плавает в жидкости плотностью $\rho_0$.
    Определите частоту колебаний куба, считая, что 4 грани куба всегда вертикальны.
}
\answer{%
    \begin{align*}
    \Delta F &= -\rho_0 g a^2 \Delta x,\Delta F = m\ddot x, m = \rho a^3\implies \rho a^3 \ddot x = -\rho_0 g a^2 \Delta x \implies \\
    \implies \ddot x &= -\frac{\rho_0}{\rho} \frac g a \Delta x\implies \omega^2 = \frac{\rho_0}{\rho} \cdot \frac g a\implies T = \frac{2 \pi}{\omega} = 2 \pi\sqrt{\frac{\rho}{\rho_0} \cdot \frac a g}.
    \end{align*}
}
\solutionspace{150pt}

\tasknumber{10}%
\task{%
    Математический маятник с нитью длиной $43\,\text{см}$ подвешен к потолку в лифте.
    За $20\,\text{с}$ маятник совершил $17$ колебаний.
    Определите модуль и направление ускорения лифта.
    Куда движется лифт?
}
\answer{%
    $
        T = 2\pi\sqrt{\frac\ell {a + g}}, T = \frac {t}{N}
        \implies a + g = \ell \cdot \frac{4 \pi ^ 2}{T^2},
        a = \ell \cdot \frac{4 \pi ^ 2}{T^2} - g = \ell \cdot \frac{4 \pi ^ 2 N^2}{t^2} - g \approx 2{,}26\,\frac{\text{м}}{\text{c}^{2}},
        \text{вниз}.
    $
}
\solutionspace{120pt}

\tasknumber{11}%
\task{%
    Масса груза в пружинном маятнике равна $600\,\text{г}$, при этом период его колебаний равен $1{,}5\,\text{с}$.
    Груз утяжеляют на $100\,\text{г}$.
    Определите новый период колебаний маятника.
}
\answer{%
    $
        T'
            = 2\pi\sqrt{\frac{M + m}{k}}
            = 2\pi\sqrt{\frac{M}{k} \cdot \frac{M + m}{M}}
            = T\sqrt{\frac{M + m}{M}} =  T\sqrt{1 + \frac{m}{M}} \approx 1{,}62\,\text{с}.
    $
}
\solutionspace{120pt}

\tasknumber{12}%
\task{%
    При какой длине нити математического маятника период колебаний груза массой $300\,\text{г}$
    окажется равен периоду колебаний этого же груза в пружинном маятнике с пружиной жёсткостью $50\,\frac{\text{Н}}{\text{м}}$?
}
\answer{%
    $
        2\pi \sqrt{\frac \ell g} = 2\pi \sqrt{\frac m k}
        \implies \frac \ell g = \frac m k
        \implies \ell = g \frac m k \approx 6{,}0\,\text{см}.
    $
}
\solutionspace{120pt}

\tasknumber{13}%
\task{%
    Груз подвесили к пружине, при этом удлинение пружины составило $30\,\text{мм}$.
    Определите частоту колебаний пружинного маятника, собранного из этой пружины и этого груза.
}
\answer{%
    $
        mg -k\Delta x = 0 \implies \frac m k = \frac{\Delta x} g
        \implies T = 2\pi \sqrt{\frac m k } = 2\pi \sqrt{\frac{\Delta x} g } \approx $0{,}34\,\text{с}$,
        \nu = \frac 1T \approx $2{,}91\,\text{Гц}$.
    $
}
\solutionspace{120pt}

\tasknumber{14}%
\task{%
    Определите период колебаний системы: математический маятник ограничен с одной стороны стенкой (см.
    рис.
    на доске).
    Удары маятника о стенку абсолютно упругие, $n = \sqrt{2}$.
    Длина маятника $\ell$, ускорение свободного падения $g$.
}
\answer{%
    $
        T = 2\pi\sqrt{\frac\ell g}, \qquad
        T' = 2 \cdot \frac T 4 + 2 \cdot \frac T{8} = \frac34T = \frac32\pi\sqrt{\frac\ell g}
    $
}

\variantsplitter

\addpersonalvariant{Арсений Трофимов}

\tasknumber{1}%
\task{%
    \begin{itemize}
        \item Запишите линейное однородное дифференциальное уравнение второго порядка,
            описывающее свободные незатухающие колебания гармонического осциллятора,
        \item запишите общее решение этого уравнения,
        \item подпишите в выписанном решении фазу и амплитуду колебаний,
        \item запишите выражение для скорости,
        \item запишите выражение для ускорения.
    \end{itemize}
}
\answer{%
    \begin{align*}
    &\ddot x + \omega^2 x = 0 \Longleftrightarrow a_x + \omega^2 x = 0, \\
    &x = A \cos(\omega t + \varphi_0) \text{ или же } x = A \sin(\omega t + \varphi_0) \text{ или же } x = a \cos(\omega t) + b \sin(\omega t), \\
    &A \text{\, или \,} \sqrt{a^2 + b^2} \text{ --- это амплитуда}, \omega t + \varphi_0\text{ --- это фаза}, \\
    &v = \dot x = -\omega A \sin(\omega t + \varphi_0), \\
    &a = \dot v = \ddot x = -\omega^2 A \cos(\omega t + \varphi_0) = -\omega^2 x,
    \end{align*}
}
\solutionspace{135pt}

\tasknumber{2}%
\task{%
    Тело колеблется по гармоническому закону,
    амплитуда этих колебаний $20\,\text{см}$, период $2\,\text{c}$.
    Чему равно смещение тела относительно положения равновесия через $20\,\text{c}$
    после прохождения положения равновесия?
}
\answer{%
    $x = A \sin \omega t = A \sin \cbr{ \frac {2\pi}T t } = A \sin \cbr{ 2\pi \frac tT } = 20\,\text{см} \cdot \sin \cbr{ 2\pi \cdot \frac {20\,\text{c}}{2\,\text{c}}}\approx 0\,\text{см}.$
}
\solutionspace{120pt}

\tasknumber{3}%
\task{%
    Тело совершает гармонические колебания с периодом $5\,\text{c}$.
    За какое время тело смещается от положения равновесия до смещения в половину амплитуды?
}
\answer{%
    $t = \frac T{12} \approx 0{,}4\,\text{c}.$
}
\solutionspace{120pt}

\tasknumber{4}%
\task{%
    Запишите формулу для периода колебаний математического маятника и ...
    \begin{itemize}
        \item укажите названия всех физических величин в формуле,
        \item выразите из неё частоту колебаний
        \item выразите из неё ускорение свободного падения.
    \end{itemize}
}
\answer{%
    \begin{align*}
    T &= 2\pi \sqrt{\frac lg} \implies \nu = \frac 1T = \frac 1{2\pi}\sqrt{\frac gl}, \omega = 2\pi\nu = \sqrt{\frac gl}, l = g\sqr{\frac T{2\pi}}, g = l\sqr{\frac {2\pi}T} \\
    T &= 2\pi \sqrt{\frac mk} \implies \nu = \frac 1T = \frac 1{2\pi}\sqrt{\frac km}, \omega = 2\pi\nu = \sqrt{\frac km}, m = k\sqr{\frac T{2\pi}}, k = m\sqr{\frac {2\pi}T}
    \end{align*}
}
\solutionspace{120pt}

\tasknumber{5}%
\task{%
    Частота колебаний математического маятника равна $15\,\text{Гц}$.
    Определите периоды колебаний
    \begin{itemize}
        \item кинетической энергии системы,
        \item скорости груза,
        \item модуля скорости груза.
    \end{itemize}
}
\answer{%
    $T = \frac 1\nu \approx 66{,}7\,\text{мc}, T_1 = \frac T2 \approx 33{,}3\,\text{мc}, T_2 = T \approx 66{,}7\,\text{мc}, T_3 = \frac T2 \approx 33{,}3\,\text{мc}.$
}
\solutionspace{80pt}

\tasknumber{6}%
\task{%
    Тело колеблется по гармоническому закону с амплитудой $4\,\text{см}$.
    Какой максимальный путь тело может пройти за одну шестую долю периода?
}
\answer{%
    $2 * A \frac{1}{2} = A \approx 4{,}0\,\text{см}.$
}
\solutionspace{80pt}

\tasknumber{7}%
\task{%
    Период колебаний математического маятника равен $2\,\text{с}$,
    а их амплитуда — $20\,\text{см}$.
    Определите амплитуду колебаний скорости маятника.
}
\answer{%
    $
        T = \frac{2\pi}{\omega}
        \implies \omega = \frac{2\pi}{T}
        \implies v_{\max} = \omega A = \frac{2\pi}{T}A
        \approx 62{,}8\,\frac{\text{см}}{\text{с}}.
    $
}
\solutionspace{100pt}

\tasknumber{8}%
\task{%
    Определите период колебаний груза массой $m$, подвешенного к пружине жёсткостью $k$.
    Ускорение свободного падения $g$.
}
\answer{%
    \begin{align*}
    &-kx_0 + mg = 0, \\
    F &= -k(x_0 + \Delta x), \\
    ma &= -k(x_0 + \Delta x) + mg, \\
    ma &= -kx_0 -k \Delta x + mg = -mg -k \Delta x + mg = -k \Delta x, \\
    a &+ \frac k m x = 0, \\
    \omega^2 &= \frac k m \implies T = \frac{2\pi}\omega = 2\pi\sqrt{\frac m k}.
    \end{align*}
}
\solutionspace{150pt}

\tasknumber{9}%
\task{%
    Куб со стороной $a$ и плотности $\rho$ плавает в жидкости плотностью $\rho_0$.
    Определите частоту колебаний куба, считая, что 4 грани куба всегда вертикальны.
}
\answer{%
    \begin{align*}
    \Delta F &= -\rho_0 g a^2 \Delta x,\Delta F = m\ddot x, m = \rho a^3\implies \rho a^3 \ddot x = -\rho_0 g a^2 \Delta x \implies \\
    \implies \ddot x &= -\frac{\rho_0}{\rho} \frac g a \Delta x\implies \omega^2 = \frac{\rho_0}{\rho} \cdot \frac g a\implies T = \frac{2 \pi}{\omega} = 2 \pi\sqrt{\frac{\rho}{\rho_0} \cdot \frac a g}.
    \end{align*}
}
\solutionspace{150pt}

\tasknumber{10}%
\task{%
    Математический маятник с нитью длиной $37\,\text{см}$ подвешен к потолку в лифте.
    За $30\,\text{с}$ маятник совершил $27$ колебаний.
    Определите модуль и направление ускорения лифта.
    Куда движется лифт?
}
\answer{%
    $
        T = 2\pi\sqrt{\frac\ell {a + g}}, T = \frac {t}{N}
        \implies a + g = \ell \cdot \frac{4 \pi ^ 2}{T^2},
        a = \ell \cdot \frac{4 \pi ^ 2}{T^2} - g = \ell \cdot \frac{4 \pi ^ 2 N^2}{t^2} - g \approx 1{,}83\,\frac{\text{м}}{\text{c}^{2}},
        \text{вниз}.
    $
}
\solutionspace{120pt}

\tasknumber{11}%
\task{%
    Масса груза в пружинном маятнике равна $500\,\text{г}$, при этом период его колебаний равен $1{,}2\,\text{с}$.
    Груз утяжеляют на $100\,\text{г}$.
    Определите новый период колебаний маятника.
}
\answer{%
    $
        T'
            = 2\pi\sqrt{\frac{M + m}{k}}
            = 2\pi\sqrt{\frac{M}{k} \cdot \frac{M + m}{M}}
            = T\sqrt{\frac{M + m}{M}} =  T\sqrt{1 + \frac{m}{M}} \approx 1{,}31\,\text{с}.
    $
}
\solutionspace{120pt}

\tasknumber{12}%
\task{%
    При какой длине нити математического маятника период колебаний груза массой $300\,\text{г}$
    окажется равен периоду колебаний этого же груза в пружинном маятнике с пружиной жёсткостью $40\,\frac{\text{Н}}{\text{м}}$?
}
\answer{%
    $
        2\pi \sqrt{\frac \ell g} = 2\pi \sqrt{\frac m k}
        \implies \frac \ell g = \frac m k
        \implies \ell = g \frac m k \approx 7{,}5\,\text{см}.
    $
}
\solutionspace{120pt}

\tasknumber{13}%
\task{%
    Груз подвесили к пружине, при этом удлинение пружины составило $30\,\text{мм}$.
    Определите частоту колебаний пружинного маятника, собранного из этой пружины и этого груза.
}
\answer{%
    $
        mg -k\Delta x = 0 \implies \frac m k = \frac{\Delta x} g
        \implies T = 2\pi \sqrt{\frac m k } = 2\pi \sqrt{\frac{\Delta x} g } \approx $0{,}34\,\text{с}$,
        \nu = \frac 1T \approx $2{,}91\,\text{Гц}$.
    $
}
\solutionspace{120pt}

\tasknumber{14}%
\task{%
    Определите период колебаний системы: математический маятник ограничен с одной стороны стенкой (см.
    рис.
    на доске).
    Удары маятника о стенку абсолютно упругие, $n = 2$.
    Длина маятника $\ell$, ускорение свободного падения $g$.
}
\answer{%
    $
        T = 2\pi\sqrt{\frac\ell g}, \qquad
        T' = 2 \cdot \frac T 4 + 2 \cdot \frac T{12} = \frac23T = \frac43\pi\sqrt{\frac\ell g}
    $
}
% autogenerated
