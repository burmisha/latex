\setdate{30~сентября~2021}
\setclass{11«БА»}

\addpersonalvariant{Михаил Бурмистров}

\tasknumber{1}%
\task{%
    Определите энергию магнитного поля в катушке индуктивностью $600\,\text{мГн}$, если протекающий через неё ток равен $8\,\text{А}$.
}
\answer{%
    $W = \frac{L\eli^2}2 = \frac{600\,\text{мГн} \cdot \sqr{8\,\text{А}}}2 \approx 19{,}20\,\text{Дж}.$
}
\solutionspace{60pt}

\tasknumber{2}%
\task{%
    В одной катушке индуктивностью $277\,\text{мГн}$ протекает электрический ток силой $611\,\text{мА}$.
    А в другой — с индуктивностью в шесть раз больше — ток в три раза сильнее.
    Определите энергию магнитного поля первой катушки, индуктивность второй катушки
    и отношение энергий магнитного поля в двух этих катушках.
}
\answer{%
    $
        L_2 = 6L_1 = 1662{,}00\,\text{мГн}, \quad
        W_1 = \frac{L_1\eli_1^2}2 = \frac{277\,\text{мГн} \cdot \sqr{611\,\text{мА}}}2 \approx 0{,}052\,\text{Дж}, \quad
        \frac{W_2}{W_1} = \frac{\frac{L_2\eli_2^2}2}{\frac{L_1\eli_1^2}2} = 54.
    $
}
\solutionspace{90pt}

\tasknumber{3}%
\task{%
    Определите индуктивность катушки, если при пропускании тока силой $1{,}5\,\text{А}$
    в ней возникает магнитное поле индукцией $5\,\text{Тл}$.
    Катушка представляет собой цилиндр радиусом $4\,\text{см}$ и высотой $6\,\text{мм}$.
    Число витков в катушке 200.
}
\answer{%
    $
        \Phi = L\eli,
        \Phi = BSN,
        S=\pi R^2
        \implies L = \frac{\pi B R^2 N}{\eli} = \frac{\pi \cdot 5\,\text{Тл} \cdot \sqr{4\,\text{см}} \cdot 200}{1{,}5\,\text{А}}
        \approx 3{,}35\,\text{Гн}.
    $
}
\solutionspace{90pt}

\tasknumber{4}%
\task{%
    Тонкий прямой стержень длиной $40\,\text{см}$ вращается в горизонтальной плоскости
    вокруг одного из своих концов.
    Период обращения стержня $2\,\text{c}$.
    Однородное магнитное поле индукцией $200\,\text{мТл}$ направлено вертикально.
    Чему равна разность потенциалов на концах стержня? Ответ выразите в милливольтах.
}
\answer{%
    $
        \ele_i
            = \frac{\Delta \Phi}{\Delta t}
            = \frac{B\Delta S}{\Delta t}
            = \frac{B \frac 12 l^2 \Delta \alpha}{\Delta t}
            = \frac{B l^2 \omega}2, \quad
        \omega = \frac{2 \pi}T \implies
        \ele_i = \frac{\pi B l^2}T \approx 50{,}3\,\text{мВ}.
    $
}
\solutionspace{120pt}

\tasknumber{5}%
\task{%
    Проводник лежит на горизонтальных рельсах,
    замкнутых резистором сопротивлением $4\,\text{Ом}$ (см.
    рис.
    на доске).
    Расстояние между рельсами $50\,\text{см}$.
    Конструкция помещена в вертикальное однородное магнитное поле индукцией $300\,\text{мТл}$.
    Какую силу необходимо прикладывать к проводнику, чтобы двигать его вдоль рельс с постоянной скоростью $2\,\frac{\text{м}}{\text{c}}$?
    Трением пренебречь, сопротивления рельс и проводника малы по сравнению с сопротивлением резистора.
    Ответ выразите в миллиньютонах.
}
\answer{%
    $
        F
            = F_A
            = \eli B l
            = \frac{\ele}R \cdot B l
            = \frac{B v l}R \cdot B l
            = \frac{B^2 v l^2}R
            = \frac{\sqr{300\,\text{мТл}} \cdot 2\,\frac{\text{м}}{\text{c}} \cdot \sqr{50\,\text{см}}}{4\,\text{Ом}}
            \approx 11{,}25\,\text{мН}.
    $
}

\variantsplitter

\addpersonalvariant{Ирина Ан}

\tasknumber{1}%
\task{%
    Определите энергию магнитного поля в катушке индуктивностью $200\,\text{мГн}$, если протекающий через неё ток равен $8\,\text{А}$.
}
\answer{%
    $W = \frac{L\eli^2}2 = \frac{200\,\text{мГн} \cdot \sqr{8\,\text{А}}}2 \approx 6{,}40\,\text{Дж}.$
}
\solutionspace{60pt}

\tasknumber{2}%
\task{%
    В одной катушке индуктивностью $739\,\text{мГн}$ протекает электрический ток силой $795\,\text{мА}$.
    А в другой — с индуктивностью в три раза больше — ток в шесть раз сильнее.
    Определите энергию магнитного поля первой катушки, индуктивность второй катушки
    и отношение энергий магнитного поля в двух этих катушках.
}
\answer{%
    $
        L_2 = 3L_1 = 2217{,}00\,\text{мГн}, \quad
        W_1 = \frac{L_1\eli_1^2}2 = \frac{739\,\text{мГн} \cdot \sqr{795\,\text{мА}}}2 \approx 0{,}234\,\text{Дж}, \quad
        \frac{W_2}{W_1} = \frac{\frac{L_2\eli_2^2}2}{\frac{L_1\eli_1^2}2} = 108.
    $
}
\solutionspace{90pt}

\tasknumber{3}%
\task{%
    Определите индуктивность катушки, если при пропускании тока силой $1{,}5\,\text{А}$
    в ней возникает магнитное поле индукцией $5\,\text{Тл}$.
    Катушка представляет собой цилиндр радиусом $4\,\text{см}$ и высотой $8\,\text{мм}$.
    Число витков в катушке 200.
}
\answer{%
    $
        \Phi = L\eli,
        \Phi = BSN,
        S=\pi R^2
        \implies L = \frac{\pi B R^2 N}{\eli} = \frac{\pi \cdot 5\,\text{Тл} \cdot \sqr{4\,\text{см}} \cdot 200}{1{,}5\,\text{А}}
        \approx 3{,}35\,\text{Гн}.
    $
}
\solutionspace{90pt}

\tasknumber{4}%
\task{%
    Тонкий прямой стержень длиной $30\,\text{см}$ вращается в горизонтальной плоскости
    вокруг одного из своих концов.
    Период обращения стержня $4\,\text{c}$.
    Однородное магнитное поле индукцией $400\,\text{мТл}$ направлено вертикально.
    Чему равна разность потенциалов на концах стержня? Ответ выразите в милливольтах.
}
\answer{%
    $
        \ele_i
            = \frac{\Delta \Phi}{\Delta t}
            = \frac{B\Delta S}{\Delta t}
            = \frac{B \frac 12 l^2 \Delta \alpha}{\Delta t}
            = \frac{B l^2 \omega}2, \quad
        \omega = \frac{2 \pi}T \implies
        \ele_i = \frac{\pi B l^2}T \approx 28{,}3\,\text{мВ}.
    $
}
\solutionspace{120pt}

\tasknumber{5}%
\task{%
    Проводник лежит на горизонтальных рельсах,
    замкнутых резистором сопротивлением $4\,\text{Ом}$ (см.
    рис.
    на доске).
    Расстояние между рельсами $80\,\text{см}$.
    Конструкция помещена в вертикальное однородное магнитное поле индукцией $400\,\text{мТл}$.
    Какую силу необходимо прикладывать к проводнику, чтобы двигать его вдоль рельс с постоянной скоростью $4\,\frac{\text{м}}{\text{c}}$?
    Трением пренебречь, сопротивления рельс и проводника малы по сравнению с сопротивлением резистора.
    Ответ выразите в миллиньютонах.
}
\answer{%
    $
        F
            = F_A
            = \eli B l
            = \frac{\ele}R \cdot B l
            = \frac{B v l}R \cdot B l
            = \frac{B^2 v l^2}R
            = \frac{\sqr{400\,\text{мТл}} \cdot 4\,\frac{\text{м}}{\text{c}} \cdot \sqr{80\,\text{см}}}{4\,\text{Ом}}
            \approx 102{,}40\,\text{мН}.
    $
}

\variantsplitter

\addpersonalvariant{Софья Андрианова}

\tasknumber{1}%
\task{%
    Определите энергию магнитного поля в катушке индуктивностью $300\,\text{мГн}$, если протекающий через неё ток равен $5\,\text{А}$.
}
\answer{%
    $W = \frac{L\eli^2}2 = \frac{300\,\text{мГн} \cdot \sqr{5\,\text{А}}}2 \approx 3{,}75\,\text{Дж}.$
}
\solutionspace{60pt}

\tasknumber{2}%
\task{%
    В одной катушке индуктивностью $277\,\text{мГн}$ протекает электрический ток силой $841\,\text{мА}$.
    А в другой — с индуктивностью в пять раз меньше — ток в пять раз сильнее.
    Определите энергию магнитного поля первой катушки, индуктивность второй катушки
    и отношение энергий магнитного поля в двух этих катушках.
}
\answer{%
    $
        L_2 = \frac15L_1 = 55{,}40\,\text{мГн}, \quad
        W_1 = \frac{L_1\eli_1^2}2 = \frac{277\,\text{мГн} \cdot \sqr{841\,\text{мА}}}2 \approx 0{,}098\,\text{Дж}, \quad
        \frac{W_2}{W_1} = \frac{\frac{L_2\eli_2^2}2}{\frac{L_1\eli_1^2}2} = 5.
    $
}
\solutionspace{90pt}

\tasknumber{3}%
\task{%
    Определите индуктивность катушки, если при пропускании тока силой $3\,\text{А}$
    в ней возникает магнитное поле индукцией $5\,\text{Тл}$.
    Катушка представляет собой цилиндр радиусом $5\,\text{см}$ и высотой $6\,\text{мм}$.
    Число витков в катушке 500.
}
\answer{%
    $
        \Phi = L\eli,
        \Phi = BSN,
        S=\pi R^2
        \implies L = \frac{\pi B R^2 N}{\eli} = \frac{\pi \cdot 5\,\text{Тл} \cdot \sqr{5\,\text{см}} \cdot 500}{3\,\text{А}}
        \approx 6{,}54\,\text{Гн}.
    $
}
\solutionspace{90pt}

\tasknumber{4}%
\task{%
    Тонкий прямой стержень длиной $40\,\text{см}$ вращается в горизонтальной плоскости
    вокруг одного из своих концов.
    Период обращения стержня $2\,\text{c}$.
    Однородное магнитное поле индукцией $300\,\text{мТл}$ направлено вертикально.
    Чему равна разность потенциалов на концах стержня? Ответ выразите в милливольтах.
}
\answer{%
    $
        \ele_i
            = \frac{\Delta \Phi}{\Delta t}
            = \frac{B\Delta S}{\Delta t}
            = \frac{B \frac 12 l^2 \Delta \alpha}{\Delta t}
            = \frac{B l^2 \omega}2, \quad
        \omega = \frac{2 \pi}T \implies
        \ele_i = \frac{\pi B l^2}T \approx 75{,}4\,\text{мВ}.
    $
}
\solutionspace{120pt}

\tasknumber{5}%
\task{%
    Проводник лежит на горизонтальных рельсах,
    замкнутых резистором сопротивлением $4\,\text{Ом}$ (см.
    рис.
    на доске).
    Расстояние между рельсами $70\,\text{см}$.
    Конструкция помещена в вертикальное однородное магнитное поле индукцией $300\,\text{мТл}$.
    Какую силу необходимо прикладывать к проводнику, чтобы двигать его вдоль рельс с постоянной скоростью $2\,\frac{\text{м}}{\text{c}}$?
    Трением пренебречь, сопротивления рельс и проводника малы по сравнению с сопротивлением резистора.
    Ответ выразите в миллиньютонах.
}
\answer{%
    $
        F
            = F_A
            = \eli B l
            = \frac{\ele}R \cdot B l
            = \frac{B v l}R \cdot B l
            = \frac{B^2 v l^2}R
            = \frac{\sqr{300\,\text{мТл}} \cdot 2\,\frac{\text{м}}{\text{c}} \cdot \sqr{70\,\text{см}}}{4\,\text{Ом}}
            \approx 22{,}05\,\text{мН}.
    $
}

\variantsplitter

\addpersonalvariant{Владимир Артемчук}

\tasknumber{1}%
\task{%
    Определите энергию магнитного поля в катушке индуктивностью $600\,\text{мГн}$, если её собственный магнитный поток равен $7\,\text{Вб}$.
}
\answer{%
    $W = \frac{\Phi^2}{2L} = \frac{\sqr{7\,\text{Вб}}}{2 \cdot 600\,\text{мГн}} \approx 40{,}83\,\text{Дж}.$
}
\solutionspace{60pt}

\tasknumber{2}%
\task{%
    В одной катушке индуктивностью $541\,\text{мГн}$ протекает электрический ток силой $680\,\text{мА}$.
    А в другой — с индуктивностью в три раза больше — ток в пять раз сильнее.
    Определите энергию магнитного поля первой катушки, индуктивность второй катушки
    и отношение энергий магнитного поля в двух этих катушках.
}
\answer{%
    $
        L_2 = 3L_1 = 1623{,}00\,\text{мГн}, \quad
        W_1 = \frac{L_1\eli_1^2}2 = \frac{541\,\text{мГн} \cdot \sqr{680\,\text{мА}}}2 \approx 0{,}125\,\text{Дж}, \quad
        \frac{W_2}{W_1} = \frac{\frac{L_2\eli_2^2}2}{\frac{L_1\eli_1^2}2} = 75.
    $
}
\solutionspace{90pt}

\tasknumber{3}%
\task{%
    Определите индуктивность катушки, если при пропускании тока силой $3\,\text{А}$
    в ней возникает магнитное поле индукцией $5\,\text{Тл}$.
    Катушка представляет собой цилиндр радиусом $4\,\text{см}$ и высотой $6\,\text{мм}$.
    Число витков в катушке 200.
}
\answer{%
    $
        \Phi = L\eli,
        \Phi = BSN,
        S=\pi R^2
        \implies L = \frac{\pi B R^2 N}{\eli} = \frac{\pi \cdot 5\,\text{Тл} \cdot \sqr{4\,\text{см}} \cdot 200}{3\,\text{А}}
        \approx 1{,}68\,\text{Гн}.
    $
}
\solutionspace{90pt}

\tasknumber{4}%
\task{%
    Тонкий прямой стержень длиной $50\,\text{см}$ вращается в горизонтальной плоскости
    вокруг одного из своих концов.
    Период обращения стержня $3\,\text{c}$.
    Однородное магнитное поле индукцией $400\,\text{мТл}$ направлено вертикально.
    Чему равна разность потенциалов на концах стержня? Ответ выразите в милливольтах.
}
\answer{%
    $
        \ele_i
            = \frac{\Delta \Phi}{\Delta t}
            = \frac{B\Delta S}{\Delta t}
            = \frac{B \frac 12 l^2 \Delta \alpha}{\Delta t}
            = \frac{B l^2 \omega}2, \quad
        \omega = \frac{2 \pi}T \implies
        \ele_i = \frac{\pi B l^2}T \approx 104{,}7\,\text{мВ}.
    $
}
\solutionspace{120pt}

\tasknumber{5}%
\task{%
    Проводник лежит на горизонтальных рельсах,
    замкнутых резистором сопротивлением $2\,\text{Ом}$ (см.
    рис.
    на доске).
    Расстояние между рельсами $70\,\text{см}$.
    Конструкция помещена в вертикальное однородное магнитное поле индукцией $400\,\text{мТл}$.
    Какую силу необходимо прикладывать к проводнику, чтобы двигать его вдоль рельс с постоянной скоростью $5\,\frac{\text{м}}{\text{c}}$?
    Трением пренебречь, сопротивления рельс и проводника малы по сравнению с сопротивлением резистора.
    Ответ выразите в миллиньютонах.
}
\answer{%
    $
        F
            = F_A
            = \eli B l
            = \frac{\ele}R \cdot B l
            = \frac{B v l}R \cdot B l
            = \frac{B^2 v l^2}R
            = \frac{\sqr{400\,\text{мТл}} \cdot 5\,\frac{\text{м}}{\text{c}} \cdot \sqr{70\,\text{см}}}{2\,\text{Ом}}
            \approx 196{,}00\,\text{мН}.
    $
}

\variantsplitter

\addpersonalvariant{Софья Белянкина}

\tasknumber{1}%
\task{%
    Определите энергию магнитного поля в катушке индуктивностью $600\,\text{мГн}$, если её собственный магнитный поток равен $7\,\text{Вб}$.
}
\answer{%
    $W = \frac{\Phi^2}{2L} = \frac{\sqr{7\,\text{Вб}}}{2 \cdot 600\,\text{мГн}} \approx 40{,}83\,\text{Дж}.$
}
\solutionspace{60pt}

\tasknumber{2}%
\task{%
    В одной катушке индуктивностью $233\,\text{мГн}$ протекает электрический ток силой $703\,\text{мА}$.
    А в другой — с индуктивностью в три раза меньше — ток в три раза сильнее.
    Определите энергию магнитного поля первой катушки, индуктивность второй катушки
    и отношение энергий магнитного поля в двух этих катушках.
}
\answer{%
    $
        L_2 = \frac13L_1 = 77{,}67\,\text{мГн}, \quad
        W_1 = \frac{L_1\eli_1^2}2 = \frac{233\,\text{мГн} \cdot \sqr{703\,\text{мА}}}2 \approx 0{,}058\,\text{Дж}, \quad
        \frac{W_2}{W_1} = \frac{\frac{L_2\eli_2^2}2}{\frac{L_1\eli_1^2}2} = 3.
    $
}
\solutionspace{90pt}

\tasknumber{3}%
\task{%
    Определите индуктивность катушки, если при пропускании тока силой $1{,}5\,\text{А}$
    в ней возникает магнитное поле индукцией $2\,\text{Тл}$.
    Катушка представляет собой цилиндр радиусом $4\,\text{см}$ и высотой $8\,\text{мм}$.
    Число витков в катушке 400.
}
\answer{%
    $
        \Phi = L\eli,
        \Phi = BSN,
        S=\pi R^2
        \implies L = \frac{\pi B R^2 N}{\eli} = \frac{\pi \cdot 2\,\text{Тл} \cdot \sqr{4\,\text{см}} \cdot 400}{1{,}5\,\text{А}}
        \approx 2{,}68\,\text{Гн}.
    $
}
\solutionspace{90pt}

\tasknumber{4}%
\task{%
    Тонкий прямой стержень длиной $40\,\text{см}$ вращается в горизонтальной плоскости
    вокруг одного из своих концов.
    Период обращения стержня $4\,\text{c}$.
    Однородное магнитное поле индукцией $150\,\text{мТл}$ направлено вертикально.
    Чему равна разность потенциалов на концах стержня? Ответ выразите в милливольтах.
}
\answer{%
    $
        \ele_i
            = \frac{\Delta \Phi}{\Delta t}
            = \frac{B\Delta S}{\Delta t}
            = \frac{B \frac 12 l^2 \Delta \alpha}{\Delta t}
            = \frac{B l^2 \omega}2, \quad
        \omega = \frac{2 \pi}T \implies
        \ele_i = \frac{\pi B l^2}T \approx 18{,}8\,\text{мВ}.
    $
}
\solutionspace{120pt}

\tasknumber{5}%
\task{%
    Проводник лежит на горизонтальных рельсах,
    замкнутых резистором сопротивлением $3\,\text{Ом}$ (см.
    рис.
    на доске).
    Расстояние между рельсами $70\,\text{см}$.
    Конструкция помещена в вертикальное однородное магнитное поле индукцией $200\,\text{мТл}$.
    Какую силу необходимо прикладывать к проводнику, чтобы двигать его вдоль рельс с постоянной скоростью $4\,\frac{\text{м}}{\text{c}}$?
    Трением пренебречь, сопротивления рельс и проводника малы по сравнению с сопротивлением резистора.
    Ответ выразите в миллиньютонах.
}
\answer{%
    $
        F
            = F_A
            = \eli B l
            = \frac{\ele}R \cdot B l
            = \frac{B v l}R \cdot B l
            = \frac{B^2 v l^2}R
            = \frac{\sqr{200\,\text{мТл}} \cdot 4\,\frac{\text{м}}{\text{c}} \cdot \sqr{70\,\text{см}}}{3\,\text{Ом}}
            \approx 26{,}13\,\text{мН}.
    $
}

\variantsplitter

\addpersonalvariant{Варвара Егиазарян}

\tasknumber{1}%
\task{%
    Определите энергию магнитного поля в катушке индуктивностью $200\,\text{мГн}$, если протекающий через неё ток равен $6\,\text{А}$.
}
\answer{%
    $W = \frac{L\eli^2}2 = \frac{200\,\text{мГн} \cdot \sqr{6\,\text{А}}}2 \approx 3{,}60\,\text{Дж}.$
}
\solutionspace{60pt}

\tasknumber{2}%
\task{%
    В одной катушке индуктивностью $519\,\text{мГн}$ протекает электрический ток силой $887\,\text{мА}$.
    А в другой — с индуктивностью в два раза меньше — ток в четыре раза сильнее.
    Определите энергию магнитного поля первой катушки, индуктивность второй катушки
    и отношение энергий магнитного поля в двух этих катушках.
}
\answer{%
    $
        L_2 = \frac12L_1 = 259{,}50\,\text{мГн}, \quad
        W_1 = \frac{L_1\eli_1^2}2 = \frac{519\,\text{мГн} \cdot \sqr{887\,\text{мА}}}2 \approx 0{,}204\,\text{Дж}, \quad
        \frac{W_2}{W_1} = \frac{\frac{L_2\eli_2^2}2}{\frac{L_1\eli_1^2}2} = 8.
    $
}
\solutionspace{90pt}

\tasknumber{3}%
\task{%
    Определите индуктивность катушки, если при пропускании тока силой $2{,}5\,\text{А}$
    в ней возникает магнитное поле индукцией $2\,\text{Тл}$.
    Катушка представляет собой цилиндр радиусом $4\,\text{см}$ и высотой $7\,\text{мм}$.
    Число витков в катушке 400.
}
\answer{%
    $
        \Phi = L\eli,
        \Phi = BSN,
        S=\pi R^2
        \implies L = \frac{\pi B R^2 N}{\eli} = \frac{\pi \cdot 2\,\text{Тл} \cdot \sqr{4\,\text{см}} \cdot 400}{2{,}5\,\text{А}}
        \approx 1{,}61\,\text{Гн}.
    $
}
\solutionspace{90pt}

\tasknumber{4}%
\task{%
    Тонкий прямой стержень длиной $50\,\text{см}$ вращается в горизонтальной плоскости
    вокруг одного из своих концов.
    Период обращения стержня $2\,\text{c}$.
    Однородное магнитное поле индукцией $200\,\text{мТл}$ направлено вертикально.
    Чему равна разность потенциалов на концах стержня? Ответ выразите в милливольтах.
}
\answer{%
    $
        \ele_i
            = \frac{\Delta \Phi}{\Delta t}
            = \frac{B\Delta S}{\Delta t}
            = \frac{B \frac 12 l^2 \Delta \alpha}{\Delta t}
            = \frac{B l^2 \omega}2, \quad
        \omega = \frac{2 \pi}T \implies
        \ele_i = \frac{\pi B l^2}T \approx 78{,}5\,\text{мВ}.
    $
}
\solutionspace{120pt}

\tasknumber{5}%
\task{%
    Проводник лежит на горизонтальных рельсах,
    замкнутых резистором сопротивлением $4\,\text{Ом}$ (см.
    рис.
    на доске).
    Расстояние между рельсами $50\,\text{см}$.
    Конструкция помещена в вертикальное однородное магнитное поле индукцией $200\,\text{мТл}$.
    Какую силу необходимо прикладывать к проводнику, чтобы двигать его вдоль рельс с постоянной скоростью $2\,\frac{\text{м}}{\text{c}}$?
    Трением пренебречь, сопротивления рельс и проводника малы по сравнению с сопротивлением резистора.
    Ответ выразите в миллиньютонах.
}
\answer{%
    $
        F
            = F_A
            = \eli B l
            = \frac{\ele}R \cdot B l
            = \frac{B v l}R \cdot B l
            = \frac{B^2 v l^2}R
            = \frac{\sqr{200\,\text{мТл}} \cdot 2\,\frac{\text{м}}{\text{c}} \cdot \sqr{50\,\text{см}}}{4\,\text{Ом}}
            \approx 5{,}00\,\text{мН}.
    $
}

\variantsplitter

\addpersonalvariant{Владислав Емелин}

\tasknumber{1}%
\task{%
    Определите энергию магнитного поля в катушке индуктивностью $300\,\text{мГн}$, если её собственный магнитный поток равен $6\,\text{Вб}$.
}
\answer{%
    $W = \frac{\Phi^2}{2L} = \frac{\sqr{6\,\text{Вб}}}{2 \cdot 300\,\text{мГн}} \approx 60{,}00\,\text{Дж}.$
}
\solutionspace{60pt}

\tasknumber{2}%
\task{%
    В одной катушке индуктивностью $464\,\text{мГн}$ протекает электрический ток силой $864\,\text{мА}$.
    А в другой — с индуктивностью в три раза больше — ток в четыре раза сильнее.
    Определите энергию магнитного поля первой катушки, индуктивность второй катушки
    и отношение энергий магнитного поля в двух этих катушках.
}
\answer{%
    $
        L_2 = 3L_1 = 1392{,}00\,\text{мГн}, \quad
        W_1 = \frac{L_1\eli_1^2}2 = \frac{464\,\text{мГн} \cdot \sqr{864\,\text{мА}}}2 \approx 0{,}173\,\text{Дж}, \quad
        \frac{W_2}{W_1} = \frac{\frac{L_2\eli_2^2}2}{\frac{L_1\eli_1^2}2} = 48.
    $
}
\solutionspace{90pt}

\tasknumber{3}%
\task{%
    Определите индуктивность катушки, если при пропускании тока силой $3\,\text{А}$
    в ней возникает магнитное поле индукцией $5\,\text{Тл}$.
    Катушка представляет собой цилиндр радиусом $5\,\text{см}$ и высотой $6\,\text{мм}$.
    Число витков в катушке 400.
}
\answer{%
    $
        \Phi = L\eli,
        \Phi = BSN,
        S=\pi R^2
        \implies L = \frac{\pi B R^2 N}{\eli} = \frac{\pi \cdot 5\,\text{Тл} \cdot \sqr{5\,\text{см}} \cdot 400}{3\,\text{А}}
        \approx 5{,}24\,\text{Гн}.
    $
}
\solutionspace{90pt}

\tasknumber{4}%
\task{%
    Тонкий прямой стержень длиной $25\,\text{см}$ вращается в горизонтальной плоскости
    вокруг одного из своих концов.
    Период обращения стержня $2\,\text{c}$.
    Однородное магнитное поле индукцией $400\,\text{мТл}$ направлено вертикально.
    Чему равна разность потенциалов на концах стержня? Ответ выразите в милливольтах.
}
\answer{%
    $
        \ele_i
            = \frac{\Delta \Phi}{\Delta t}
            = \frac{B\Delta S}{\Delta t}
            = \frac{B \frac 12 l^2 \Delta \alpha}{\Delta t}
            = \frac{B l^2 \omega}2, \quad
        \omega = \frac{2 \pi}T \implies
        \ele_i = \frac{\pi B l^2}T \approx 39{,}3\,\text{мВ}.
    $
}
\solutionspace{120pt}

\tasknumber{5}%
\task{%
    Проводник лежит на горизонтальных рельсах,
    замкнутых резистором сопротивлением $2\,\text{Ом}$ (см.
    рис.
    на доске).
    Расстояние между рельсами $60\,\text{см}$.
    Конструкция помещена в вертикальное однородное магнитное поле индукцией $300\,\text{мТл}$.
    Какую силу необходимо прикладывать к проводнику, чтобы двигать его вдоль рельс с постоянной скоростью $3\,\frac{\text{м}}{\text{c}}$?
    Трением пренебречь, сопротивления рельс и проводника малы по сравнению с сопротивлением резистора.
    Ответ выразите в миллиньютонах.
}
\answer{%
    $
        F
            = F_A
            = \eli B l
            = \frac{\ele}R \cdot B l
            = \frac{B v l}R \cdot B l
            = \frac{B^2 v l^2}R
            = \frac{\sqr{300\,\text{мТл}} \cdot 3\,\frac{\text{м}}{\text{c}} \cdot \sqr{60\,\text{см}}}{2\,\text{Ом}}
            \approx 48{,}60\,\text{мН}.
    $
}

\variantsplitter

\addpersonalvariant{Артём Жичин}

\tasknumber{1}%
\task{%
    Определите энергию магнитного поля в катушке индуктивностью $600\,\text{мГн}$, если протекающий через неё ток равен $3\,\text{А}$.
}
\answer{%
    $W = \frac{L\eli^2}2 = \frac{600\,\text{мГн} \cdot \sqr{3\,\text{А}}}2 \approx 2{,}70\,\text{Дж}.$
}
\solutionspace{60pt}

\tasknumber{2}%
\task{%
    В одной катушке индуктивностью $695\,\text{мГн}$ протекает электрический ток силой $795\,\text{мА}$.
    А в другой — с индуктивностью в два раза меньше — ток в шесть раз сильнее.
    Определите энергию магнитного поля первой катушки, индуктивность второй катушки
    и отношение энергий магнитного поля в двух этих катушках.
}
\answer{%
    $
        L_2 = \frac12L_1 = 347{,}50\,\text{мГн}, \quad
        W_1 = \frac{L_1\eli_1^2}2 = \frac{695\,\text{мГн} \cdot \sqr{795\,\text{мА}}}2 \approx 0{,}220\,\text{Дж}, \quad
        \frac{W_2}{W_1} = \frac{\frac{L_2\eli_2^2}2}{\frac{L_1\eli_1^2}2} = 18.
    $
}
\solutionspace{90pt}

\tasknumber{3}%
\task{%
    Определите индуктивность катушки, если при пропускании тока силой $1{,}5\,\text{А}$
    в ней возникает магнитное поле индукцией $2\,\text{Тл}$.
    Катушка представляет собой цилиндр радиусом $4\,\text{см}$ и высотой $6\,\text{мм}$.
    Число витков в катушке 200.
}
\answer{%
    $
        \Phi = L\eli,
        \Phi = BSN,
        S=\pi R^2
        \implies L = \frac{\pi B R^2 N}{\eli} = \frac{\pi \cdot 2\,\text{Тл} \cdot \sqr{4\,\text{см}} \cdot 200}{1{,}5\,\text{А}}
        \approx 1{,}34\,\text{Гн}.
    $
}
\solutionspace{90pt}

\tasknumber{4}%
\task{%
    Тонкий прямой стержень длиной $50\,\text{см}$ вращается в горизонтальной плоскости
    вокруг одного из своих концов.
    Период обращения стержня $3\,\text{c}$.
    Однородное магнитное поле индукцией $200\,\text{мТл}$ направлено вертикально.
    Чему равна разность потенциалов на концах стержня? Ответ выразите в милливольтах.
}
\answer{%
    $
        \ele_i
            = \frac{\Delta \Phi}{\Delta t}
            = \frac{B\Delta S}{\Delta t}
            = \frac{B \frac 12 l^2 \Delta \alpha}{\Delta t}
            = \frac{B l^2 \omega}2, \quad
        \omega = \frac{2 \pi}T \implies
        \ele_i = \frac{\pi B l^2}T \approx 52{,}4\,\text{мВ}.
    $
}
\solutionspace{120pt}

\tasknumber{5}%
\task{%
    Проводник лежит на горизонтальных рельсах,
    замкнутых резистором сопротивлением $2\,\text{Ом}$ (см.
    рис.
    на доске).
    Расстояние между рельсами $70\,\text{см}$.
    Конструкция помещена в вертикальное однородное магнитное поле индукцией $400\,\text{мТл}$.
    Какую силу необходимо прикладывать к проводнику, чтобы двигать его вдоль рельс с постоянной скоростью $2\,\frac{\text{м}}{\text{c}}$?
    Трением пренебречь, сопротивления рельс и проводника малы по сравнению с сопротивлением резистора.
    Ответ выразите в миллиньютонах.
}
\answer{%
    $
        F
            = F_A
            = \eli B l
            = \frac{\ele}R \cdot B l
            = \frac{B v l}R \cdot B l
            = \frac{B^2 v l^2}R
            = \frac{\sqr{400\,\text{мТл}} \cdot 2\,\frac{\text{м}}{\text{c}} \cdot \sqr{70\,\text{см}}}{2\,\text{Ом}}
            \approx 78{,}40\,\text{мН}.
    $
}

\variantsplitter

\addpersonalvariant{Дарья Кошман}

\tasknumber{1}%
\task{%
    Определите энергию магнитного поля в катушке индуктивностью $400\,\text{мГн}$, если её собственный магнитный поток равен $3\,\text{Вб}$.
}
\answer{%
    $W = \frac{\Phi^2}{2L} = \frac{\sqr{3\,\text{Вб}}}{2 \cdot 400\,\text{мГн}} \approx 11{,}25\,\text{Дж}.$
}
\solutionspace{60pt}

\tasknumber{2}%
\task{%
    В одной катушке индуктивностью $156\,\text{мГн}$ протекает электрический ток силой $818\,\text{мА}$.
    А в другой — с индуктивностью в четыре раза меньше — ток в шесть раз сильнее.
    Определите энергию магнитного поля первой катушки, индуктивность второй катушки
    и отношение энергий магнитного поля в двух этих катушках.
}
\answer{%
    $
        L_2 = \frac14L_1 = 39{,}00\,\text{мГн}, \quad
        W_1 = \frac{L_1\eli_1^2}2 = \frac{156\,\text{мГн} \cdot \sqr{818\,\text{мА}}}2 \approx 0{,}052\,\text{Дж}, \quad
        \frac{W_2}{W_1} = \frac{\frac{L_2\eli_2^2}2}{\frac{L_1\eli_1^2}2} = 9.
    $
}
\solutionspace{90pt}

\tasknumber{3}%
\task{%
    Определите индуктивность катушки, если при пропускании тока силой $1{,}5\,\text{А}$
    в ней возникает магнитное поле индукцией $8\,\text{Тл}$.
    Катушка представляет собой цилиндр радиусом $5\,\text{см}$ и высотой $7\,\text{мм}$.
    Число витков в катушке 200.
}
\answer{%
    $
        \Phi = L\eli,
        \Phi = BSN,
        S=\pi R^2
        \implies L = \frac{\pi B R^2 N}{\eli} = \frac{\pi \cdot 8\,\text{Тл} \cdot \sqr{5\,\text{см}} \cdot 200}{1{,}5\,\text{А}}
        \approx 8{,}38\,\text{Гн}.
    $
}
\solutionspace{90pt}

\tasknumber{4}%
\task{%
    Тонкий прямой стержень длиной $30\,\text{см}$ вращается в горизонтальной плоскости
    вокруг одного из своих концов.
    Период обращения стержня $2\,\text{c}$.
    Однородное магнитное поле индукцией $300\,\text{мТл}$ направлено вертикально.
    Чему равна разность потенциалов на концах стержня? Ответ выразите в милливольтах.
}
\answer{%
    $
        \ele_i
            = \frac{\Delta \Phi}{\Delta t}
            = \frac{B\Delta S}{\Delta t}
            = \frac{B \frac 12 l^2 \Delta \alpha}{\Delta t}
            = \frac{B l^2 \omega}2, \quad
        \omega = \frac{2 \pi}T \implies
        \ele_i = \frac{\pi B l^2}T \approx 42{,}4\,\text{мВ}.
    $
}
\solutionspace{120pt}

\tasknumber{5}%
\task{%
    Проводник лежит на горизонтальных рельсах,
    замкнутых резистором сопротивлением $3\,\text{Ом}$ (см.
    рис.
    на доске).
    Расстояние между рельсами $80\,\text{см}$.
    Конструкция помещена в вертикальное однородное магнитное поле индукцией $150\,\text{мТл}$.
    Какую силу необходимо прикладывать к проводнику, чтобы двигать его вдоль рельс с постоянной скоростью $4\,\frac{\text{м}}{\text{c}}$?
    Трением пренебречь, сопротивления рельс и проводника малы по сравнению с сопротивлением резистора.
    Ответ выразите в миллиньютонах.
}
\answer{%
    $
        F
            = F_A
            = \eli B l
            = \frac{\ele}R \cdot B l
            = \frac{B v l}R \cdot B l
            = \frac{B^2 v l^2}R
            = \frac{\sqr{150\,\text{мТл}} \cdot 4\,\frac{\text{м}}{\text{c}} \cdot \sqr{80\,\text{см}}}{3\,\text{Ом}}
            \approx 19{,}20\,\text{мН}.
    $
}

\variantsplitter

\addpersonalvariant{Анна Кузьмичёва}

\tasknumber{1}%
\task{%
    Определите энергию магнитного поля в катушке индуктивностью $300\,\text{мГн}$, если её собственный магнитный поток равен $3\,\text{Вб}$.
}
\answer{%
    $W = \frac{\Phi^2}{2L} = \frac{\sqr{3\,\text{Вб}}}{2 \cdot 300\,\text{мГн}} \approx 15{,}00\,\text{Дж}.$
}
\solutionspace{60pt}

\tasknumber{2}%
\task{%
    В одной катушке индуктивностью $475\,\text{мГн}$ протекает электрический ток силой $887\,\text{мА}$.
    А в другой — с индуктивностью в два раза меньше — ток в шесть раз сильнее.
    Определите энергию магнитного поля первой катушки, индуктивность второй катушки
    и отношение энергий магнитного поля в двух этих катушках.
}
\answer{%
    $
        L_2 = \frac12L_1 = 237{,}50\,\text{мГн}, \quad
        W_1 = \frac{L_1\eli_1^2}2 = \frac{475\,\text{мГн} \cdot \sqr{887\,\text{мА}}}2 \approx 0{,}187\,\text{Дж}, \quad
        \frac{W_2}{W_1} = \frac{\frac{L_2\eli_2^2}2}{\frac{L_1\eli_1^2}2} = 18.
    $
}
\solutionspace{90pt}

\tasknumber{3}%
\task{%
    Определите индуктивность катушки, если при пропускании тока силой $3\,\text{А}$
    в ней возникает магнитное поле индукцией $8\,\text{Тл}$.
    Катушка представляет собой цилиндр радиусом $5\,\text{см}$ и высотой $6\,\text{мм}$.
    Число витков в катушке 400.
}
\answer{%
    $
        \Phi = L\eli,
        \Phi = BSN,
        S=\pi R^2
        \implies L = \frac{\pi B R^2 N}{\eli} = \frac{\pi \cdot 8\,\text{Тл} \cdot \sqr{5\,\text{см}} \cdot 400}{3\,\text{А}}
        \approx 8{,}38\,\text{Гн}.
    $
}
\solutionspace{90pt}

\tasknumber{4}%
\task{%
    Тонкий прямой стержень длиной $40\,\text{см}$ вращается в горизонтальной плоскости
    вокруг одного из своих концов.
    Период обращения стержня $3\,\text{c}$.
    Однородное магнитное поле индукцией $400\,\text{мТл}$ направлено вертикально.
    Чему равна разность потенциалов на концах стержня? Ответ выразите в милливольтах.
}
\answer{%
    $
        \ele_i
            = \frac{\Delta \Phi}{\Delta t}
            = \frac{B\Delta S}{\Delta t}
            = \frac{B \frac 12 l^2 \Delta \alpha}{\Delta t}
            = \frac{B l^2 \omega}2, \quad
        \omega = \frac{2 \pi}T \implies
        \ele_i = \frac{\pi B l^2}T \approx 67{,}0\,\text{мВ}.
    $
}
\solutionspace{120pt}

\tasknumber{5}%
\task{%
    Проводник лежит на горизонтальных рельсах,
    замкнутых резистором сопротивлением $4\,\text{Ом}$ (см.
    рис.
    на доске).
    Расстояние между рельсами $80\,\text{см}$.
    Конструкция помещена в вертикальное однородное магнитное поле индукцией $300\,\text{мТл}$.
    Какую силу необходимо прикладывать к проводнику, чтобы двигать его вдоль рельс с постоянной скоростью $4\,\frac{\text{м}}{\text{c}}$?
    Трением пренебречь, сопротивления рельс и проводника малы по сравнению с сопротивлением резистора.
    Ответ выразите в миллиньютонах.
}
\answer{%
    $
        F
            = F_A
            = \eli B l
            = \frac{\ele}R \cdot B l
            = \frac{B v l}R \cdot B l
            = \frac{B^2 v l^2}R
            = \frac{\sqr{300\,\text{мТл}} \cdot 4\,\frac{\text{м}}{\text{c}} \cdot \sqr{80\,\text{см}}}{4\,\text{Ом}}
            \approx 57{,}60\,\text{мН}.
    $
}

\variantsplitter

\addpersonalvariant{Алёна Куприянова}

\tasknumber{1}%
\task{%
    Определите энергию магнитного поля в катушке индуктивностью $600\,\text{мГн}$, если её собственный магнитный поток равен $6\,\text{Вб}$.
}
\answer{%
    $W = \frac{\Phi^2}{2L} = \frac{\sqr{6\,\text{Вб}}}{2 \cdot 600\,\text{мГн}} \approx 30{,}00\,\text{Дж}.$
}
\solutionspace{60pt}

\tasknumber{2}%
\task{%
    В одной катушке индуктивностью $387\,\text{мГн}$ протекает электрический ток силой $749\,\text{мА}$.
    А в другой — с индуктивностью в пять раз больше — ток в четыре раза сильнее.
    Определите энергию магнитного поля первой катушки, индуктивность второй катушки
    и отношение энергий магнитного поля в двух этих катушках.
}
\answer{%
    $
        L_2 = 5L_1 = 1935{,}00\,\text{мГн}, \quad
        W_1 = \frac{L_1\eli_1^2}2 = \frac{387\,\text{мГн} \cdot \sqr{749\,\text{мА}}}2 \approx 0{,}109\,\text{Дж}, \quad
        \frac{W_2}{W_1} = \frac{\frac{L_2\eli_2^2}2}{\frac{L_1\eli_1^2}2} = 80.
    $
}
\solutionspace{90pt}

\tasknumber{3}%
\task{%
    Определите индуктивность катушки, если при пропускании тока силой $1{,}5\,\text{А}$
    в ней возникает магнитное поле индукцией $2\,\text{Тл}$.
    Катушка представляет собой цилиндр радиусом $5\,\text{см}$ и высотой $7\,\text{мм}$.
    Число витков в катушке 500.
}
\answer{%
    $
        \Phi = L\eli,
        \Phi = BSN,
        S=\pi R^2
        \implies L = \frac{\pi B R^2 N}{\eli} = \frac{\pi \cdot 2\,\text{Тл} \cdot \sqr{5\,\text{см}} \cdot 500}{1{,}5\,\text{А}}
        \approx 5{,}24\,\text{Гн}.
    $
}
\solutionspace{90pt}

\tasknumber{4}%
\task{%
    Тонкий прямой стержень длиной $25\,\text{см}$ вращается в горизонтальной плоскости
    вокруг одного из своих концов.
    Период обращения стержня $4\,\text{c}$.
    Однородное магнитное поле индукцией $200\,\text{мТл}$ направлено вертикально.
    Чему равна разность потенциалов на концах стержня? Ответ выразите в милливольтах.
}
\answer{%
    $
        \ele_i
            = \frac{\Delta \Phi}{\Delta t}
            = \frac{B\Delta S}{\Delta t}
            = \frac{B \frac 12 l^2 \Delta \alpha}{\Delta t}
            = \frac{B l^2 \omega}2, \quad
        \omega = \frac{2 \pi}T \implies
        \ele_i = \frac{\pi B l^2}T \approx 9{,}8\,\text{мВ}.
    $
}
\solutionspace{120pt}

\tasknumber{5}%
\task{%
    Проводник лежит на горизонтальных рельсах,
    замкнутых резистором сопротивлением $3\,\text{Ом}$ (см.
    рис.
    на доске).
    Расстояние между рельсами $70\,\text{см}$.
    Конструкция помещена в вертикальное однородное магнитное поле индукцией $400\,\text{мТл}$.
    Какую силу необходимо прикладывать к проводнику, чтобы двигать его вдоль рельс с постоянной скоростью $5\,\frac{\text{м}}{\text{c}}$?
    Трением пренебречь, сопротивления рельс и проводника малы по сравнению с сопротивлением резистора.
    Ответ выразите в миллиньютонах.
}
\answer{%
    $
        F
            = F_A
            = \eli B l
            = \frac{\ele}R \cdot B l
            = \frac{B v l}R \cdot B l
            = \frac{B^2 v l^2}R
            = \frac{\sqr{400\,\text{мТл}} \cdot 5\,\frac{\text{м}}{\text{c}} \cdot \sqr{70\,\text{см}}}{3\,\text{Ом}}
            \approx 130{,}67\,\text{мН}.
    $
}

\variantsplitter

\addpersonalvariant{Ярослав Лавровский}

\tasknumber{1}%
\task{%
    Определите энергию магнитного поля в катушке индуктивностью $400\,\text{мГн}$, если её собственный магнитный поток равен $4\,\text{Вб}$.
}
\answer{%
    $W = \frac{\Phi^2}{2L} = \frac{\sqr{4\,\text{Вб}}}{2 \cdot 400\,\text{мГн}} \approx 20{,}00\,\text{Дж}.$
}
\solutionspace{60pt}

\tasknumber{2}%
\task{%
    В одной катушке индуктивностью $200\,\text{мГн}$ протекает электрический ток силой $542\,\text{мА}$.
    А в другой — с индуктивностью в два раза больше — ток в шесть раз сильнее.
    Определите энергию магнитного поля первой катушки, индуктивность второй катушки
    и отношение энергий магнитного поля в двух этих катушках.
}
\answer{%
    $
        L_2 = 2L_1 = 400{,}00\,\text{мГн}, \quad
        W_1 = \frac{L_1\eli_1^2}2 = \frac{200\,\text{мГн} \cdot \sqr{542\,\text{мА}}}2 \approx 0{,}029\,\text{Дж}, \quad
        \frac{W_2}{W_1} = \frac{\frac{L_2\eli_2^2}2}{\frac{L_1\eli_1^2}2} = 72.
    $
}
\solutionspace{90pt}

\tasknumber{3}%
\task{%
    Определите индуктивность катушки, если при пропускании тока силой $3\,\text{А}$
    в ней возникает магнитное поле индукцией $8\,\text{Тл}$.
    Катушка представляет собой цилиндр радиусом $4\,\text{см}$ и высотой $8\,\text{мм}$.
    Число витков в катушке 200.
}
\answer{%
    $
        \Phi = L\eli,
        \Phi = BSN,
        S=\pi R^2
        \implies L = \frac{\pi B R^2 N}{\eli} = \frac{\pi \cdot 8\,\text{Тл} \cdot \sqr{4\,\text{см}} \cdot 200}{3\,\text{А}}
        \approx 2{,}68\,\text{Гн}.
    $
}
\solutionspace{90pt}

\tasknumber{4}%
\task{%
    Тонкий прямой стержень длиной $40\,\text{см}$ вращается в горизонтальной плоскости
    вокруг одного из своих концов.
    Период обращения стержня $3\,\text{c}$.
    Однородное магнитное поле индукцией $300\,\text{мТл}$ направлено вертикально.
    Чему равна разность потенциалов на концах стержня? Ответ выразите в милливольтах.
}
\answer{%
    $
        \ele_i
            = \frac{\Delta \Phi}{\Delta t}
            = \frac{B\Delta S}{\Delta t}
            = \frac{B \frac 12 l^2 \Delta \alpha}{\Delta t}
            = \frac{B l^2 \omega}2, \quad
        \omega = \frac{2 \pi}T \implies
        \ele_i = \frac{\pi B l^2}T \approx 50{,}3\,\text{мВ}.
    $
}
\solutionspace{120pt}

\tasknumber{5}%
\task{%
    Проводник лежит на горизонтальных рельсах,
    замкнутых резистором сопротивлением $2\,\text{Ом}$ (см.
    рис.
    на доске).
    Расстояние между рельсами $70\,\text{см}$.
    Конструкция помещена в вертикальное однородное магнитное поле индукцией $400\,\text{мТл}$.
    Какую силу необходимо прикладывать к проводнику, чтобы двигать его вдоль рельс с постоянной скоростью $3\,\frac{\text{м}}{\text{c}}$?
    Трением пренебречь, сопротивления рельс и проводника малы по сравнению с сопротивлением резистора.
    Ответ выразите в миллиньютонах.
}
\answer{%
    $
        F
            = F_A
            = \eli B l
            = \frac{\ele}R \cdot B l
            = \frac{B v l}R \cdot B l
            = \frac{B^2 v l^2}R
            = \frac{\sqr{400\,\text{мТл}} \cdot 3\,\frac{\text{м}}{\text{c}} \cdot \sqr{70\,\text{см}}}{2\,\text{Ом}}
            \approx 117{,}60\,\text{мН}.
    $
}

\variantsplitter

\addpersonalvariant{Анастасия Ламанова}

\tasknumber{1}%
\task{%
    Определите энергию магнитного поля в катушке индуктивностью $600\,\text{мГн}$, если протекающий через неё ток равен $7\,\text{А}$.
}
\answer{%
    $W = \frac{L\eli^2}2 = \frac{600\,\text{мГн} \cdot \sqr{7\,\text{А}}}2 \approx 14{,}70\,\text{Дж}.$
}
\solutionspace{60pt}

\tasknumber{2}%
\task{%
    В одной катушке индуктивностью $277\,\text{мГн}$ протекает электрический ток силой $772\,\text{мА}$.
    А в другой — с индуктивностью в три раза больше — ток в пять раз сильнее.
    Определите энергию магнитного поля первой катушки, индуктивность второй катушки
    и отношение энергий магнитного поля в двух этих катушках.
}
\answer{%
    $
        L_2 = 3L_1 = 831{,}00\,\text{мГн}, \quad
        W_1 = \frac{L_1\eli_1^2}2 = \frac{277\,\text{мГн} \cdot \sqr{772\,\text{мА}}}2 \approx 0{,}083\,\text{Дж}, \quad
        \frac{W_2}{W_1} = \frac{\frac{L_2\eli_2^2}2}{\frac{L_1\eli_1^2}2} = 75.
    $
}
\solutionspace{90pt}

\tasknumber{3}%
\task{%
    Определите индуктивность катушки, если при пропускании тока силой $1{,}5\,\text{А}$
    в ней возникает магнитное поле индукцией $2\,\text{Тл}$.
    Катушка представляет собой цилиндр радиусом $3\,\text{см}$ и высотой $7\,\text{мм}$.
    Число витков в катушке 400.
}
\answer{%
    $
        \Phi = L\eli,
        \Phi = BSN,
        S=\pi R^2
        \implies L = \frac{\pi B R^2 N}{\eli} = \frac{\pi \cdot 2\,\text{Тл} \cdot \sqr{3\,\text{см}} \cdot 400}{1{,}5\,\text{А}}
        \approx 1{,}51\,\text{Гн}.
    $
}
\solutionspace{90pt}

\tasknumber{4}%
\task{%
    Тонкий прямой стержень длиной $20\,\text{см}$ вращается в горизонтальной плоскости
    вокруг одного из своих концов.
    Период обращения стержня $3\,\text{c}$.
    Однородное магнитное поле индукцией $400\,\text{мТл}$ направлено вертикально.
    Чему равна разность потенциалов на концах стержня? Ответ выразите в милливольтах.
}
\answer{%
    $
        \ele_i
            = \frac{\Delta \Phi}{\Delta t}
            = \frac{B\Delta S}{\Delta t}
            = \frac{B \frac 12 l^2 \Delta \alpha}{\Delta t}
            = \frac{B l^2 \omega}2, \quad
        \omega = \frac{2 \pi}T \implies
        \ele_i = \frac{\pi B l^2}T \approx 16{,}8\,\text{мВ}.
    $
}
\solutionspace{120pt}

\tasknumber{5}%
\task{%
    Проводник лежит на горизонтальных рельсах,
    замкнутых резистором сопротивлением $4\,\text{Ом}$ (см.
    рис.
    на доске).
    Расстояние между рельсами $50\,\text{см}$.
    Конструкция помещена в вертикальное однородное магнитное поле индукцией $200\,\text{мТл}$.
    Какую силу необходимо прикладывать к проводнику, чтобы двигать его вдоль рельс с постоянной скоростью $4\,\frac{\text{м}}{\text{c}}$?
    Трением пренебречь, сопротивления рельс и проводника малы по сравнению с сопротивлением резистора.
    Ответ выразите в миллиньютонах.
}
\answer{%
    $
        F
            = F_A
            = \eli B l
            = \frac{\ele}R \cdot B l
            = \frac{B v l}R \cdot B l
            = \frac{B^2 v l^2}R
            = \frac{\sqr{200\,\text{мТл}} \cdot 4\,\frac{\text{м}}{\text{c}} \cdot \sqr{50\,\text{см}}}{4\,\text{Ом}}
            \approx 10{,}00\,\text{мН}.
    $
}

\variantsplitter

\addpersonalvariant{Виктория Легонькова}

\tasknumber{1}%
\task{%
    Определите энергию магнитного поля в катушке индуктивностью $200\,\text{мГн}$, если протекающий через неё ток равен $5\,\text{А}$.
}
\answer{%
    $W = \frac{L\eli^2}2 = \frac{200\,\text{мГн} \cdot \sqr{5\,\text{А}}}2 \approx 2{,}50\,\text{Дж}.$
}
\solutionspace{60pt}

\tasknumber{2}%
\task{%
    В одной катушке индуктивностью $431\,\text{мГн}$ протекает электрический ток силой $749\,\text{мА}$.
    А в другой — с индуктивностью в восемь раз меньше — ток в три раза сильнее.
    Определите энергию магнитного поля первой катушки, индуктивность второй катушки
    и отношение энергий магнитного поля в двух этих катушках.
}
\answer{%
    $
        L_2 = \frac18L_1 = 53{,}88\,\text{мГн}, \quad
        W_1 = \frac{L_1\eli_1^2}2 = \frac{431\,\text{мГн} \cdot \sqr{749\,\text{мА}}}2 \approx 0{,}121\,\text{Дж}, \quad
        \frac{W_2}{W_1} = \frac{\frac{L_2\eli_2^2}2}{\frac{L_1\eli_1^2}2} = \frac98.
    $
}
\solutionspace{90pt}

\tasknumber{3}%
\task{%
    Определите индуктивность катушки, если при пропускании тока силой $1{,}5\,\text{А}$
    в ней возникает магнитное поле индукцией $5\,\text{Тл}$.
    Катушка представляет собой цилиндр радиусом $3\,\text{см}$ и высотой $8\,\text{мм}$.
    Число витков в катушке 200.
}
\answer{%
    $
        \Phi = L\eli,
        \Phi = BSN,
        S=\pi R^2
        \implies L = \frac{\pi B R^2 N}{\eli} = \frac{\pi \cdot 5\,\text{Тл} \cdot \sqr{3\,\text{см}} \cdot 200}{1{,}5\,\text{А}}
        \approx 1{,}88\,\text{Гн}.
    $
}
\solutionspace{90pt}

\tasknumber{4}%
\task{%
    Тонкий прямой стержень длиной $30\,\text{см}$ вращается в горизонтальной плоскости
    вокруг одного из своих концов.
    Период обращения стержня $4\,\text{c}$.
    Однородное магнитное поле индукцией $300\,\text{мТл}$ направлено вертикально.
    Чему равна разность потенциалов на концах стержня? Ответ выразите в милливольтах.
}
\answer{%
    $
        \ele_i
            = \frac{\Delta \Phi}{\Delta t}
            = \frac{B\Delta S}{\Delta t}
            = \frac{B \frac 12 l^2 \Delta \alpha}{\Delta t}
            = \frac{B l^2 \omega}2, \quad
        \omega = \frac{2 \pi}T \implies
        \ele_i = \frac{\pi B l^2}T \approx 21{,}2\,\text{мВ}.
    $
}
\solutionspace{120pt}

\tasknumber{5}%
\task{%
    Проводник лежит на горизонтальных рельсах,
    замкнутых резистором сопротивлением $4\,\text{Ом}$ (см.
    рис.
    на доске).
    Расстояние между рельсами $60\,\text{см}$.
    Конструкция помещена в вертикальное однородное магнитное поле индукцией $200\,\text{мТл}$.
    Какую силу необходимо прикладывать к проводнику, чтобы двигать его вдоль рельс с постоянной скоростью $3\,\frac{\text{м}}{\text{c}}$?
    Трением пренебречь, сопротивления рельс и проводника малы по сравнению с сопротивлением резистора.
    Ответ выразите в миллиньютонах.
}
\answer{%
    $
        F
            = F_A
            = \eli B l
            = \frac{\ele}R \cdot B l
            = \frac{B v l}R \cdot B l
            = \frac{B^2 v l^2}R
            = \frac{\sqr{200\,\text{мТл}} \cdot 3\,\frac{\text{м}}{\text{c}} \cdot \sqr{60\,\text{см}}}{4\,\text{Ом}}
            \approx 10{,}80\,\text{мН}.
    $
}

\variantsplitter

\addpersonalvariant{Семён Мартынов}

\tasknumber{1}%
\task{%
    Определите энергию магнитного поля в катушке индуктивностью $200\,\text{мГн}$, если её собственный магнитный поток равен $8\,\text{Вб}$.
}
\answer{%
    $W = \frac{\Phi^2}{2L} = \frac{\sqr{8\,\text{Вб}}}{2 \cdot 200\,\text{мГн}} \approx 160{,}00\,\text{Дж}.$
}
\solutionspace{60pt}

\tasknumber{2}%
\task{%
    В одной катушке индуктивностью $211\,\text{мГн}$ протекает электрический ток силой $887\,\text{мА}$.
    А в другой — с индуктивностью в два раза меньше — ток в три раза сильнее.
    Определите энергию магнитного поля первой катушки, индуктивность второй катушки
    и отношение энергий магнитного поля в двух этих катушках.
}
\answer{%
    $
        L_2 = \frac12L_1 = 105{,}50\,\text{мГн}, \quad
        W_1 = \frac{L_1\eli_1^2}2 = \frac{211\,\text{мГн} \cdot \sqr{887\,\text{мА}}}2 \approx 0{,}083\,\text{Дж}, \quad
        \frac{W_2}{W_1} = \frac{\frac{L_2\eli_2^2}2}{\frac{L_1\eli_1^2}2} = \frac92.
    $
}
\solutionspace{90pt}

\tasknumber{3}%
\task{%
    Определите индуктивность катушки, если при пропускании тока силой $1{,}5\,\text{А}$
    в ней возникает магнитное поле индукцией $8\,\text{Тл}$.
    Катушка представляет собой цилиндр радиусом $3\,\text{см}$ и высотой $7\,\text{мм}$.
    Число витков в катушке 400.
}
\answer{%
    $
        \Phi = L\eli,
        \Phi = BSN,
        S=\pi R^2
        \implies L = \frac{\pi B R^2 N}{\eli} = \frac{\pi \cdot 8\,\text{Тл} \cdot \sqr{3\,\text{см}} \cdot 400}{1{,}5\,\text{А}}
        \approx 6{,}03\,\text{Гн}.
    $
}
\solutionspace{90pt}

\tasknumber{4}%
\task{%
    Тонкий прямой стержень длиной $40\,\text{см}$ вращается в горизонтальной плоскости
    вокруг одного из своих концов.
    Период обращения стержня $5\,\text{c}$.
    Однородное магнитное поле индукцией $400\,\text{мТл}$ направлено вертикально.
    Чему равна разность потенциалов на концах стержня? Ответ выразите в милливольтах.
}
\answer{%
    $
        \ele_i
            = \frac{\Delta \Phi}{\Delta t}
            = \frac{B\Delta S}{\Delta t}
            = \frac{B \frac 12 l^2 \Delta \alpha}{\Delta t}
            = \frac{B l^2 \omega}2, \quad
        \omega = \frac{2 \pi}T \implies
        \ele_i = \frac{\pi B l^2}T \approx 40{,}2\,\text{мВ}.
    $
}
\solutionspace{120pt}

\tasknumber{5}%
\task{%
    Проводник лежит на горизонтальных рельсах,
    замкнутых резистором сопротивлением $2\,\text{Ом}$ (см.
    рис.
    на доске).
    Расстояние между рельсами $80\,\text{см}$.
    Конструкция помещена в вертикальное однородное магнитное поле индукцией $200\,\text{мТл}$.
    Какую силу необходимо прикладывать к проводнику, чтобы двигать его вдоль рельс с постоянной скоростью $5\,\frac{\text{м}}{\text{c}}$?
    Трением пренебречь, сопротивления рельс и проводника малы по сравнению с сопротивлением резистора.
    Ответ выразите в миллиньютонах.
}
\answer{%
    $
        F
            = F_A
            = \eli B l
            = \frac{\ele}R \cdot B l
            = \frac{B v l}R \cdot B l
            = \frac{B^2 v l^2}R
            = \frac{\sqr{200\,\text{мТл}} \cdot 5\,\frac{\text{м}}{\text{c}} \cdot \sqr{80\,\text{см}}}{2\,\text{Ом}}
            \approx 64{,}00\,\text{мН}.
    $
}

\variantsplitter

\addpersonalvariant{Варвара Минаева}

\tasknumber{1}%
\task{%
    Определите энергию магнитного поля в катушке индуктивностью $400\,\text{мГн}$, если её собственный магнитный поток равен $3\,\text{Вб}$.
}
\answer{%
    $W = \frac{\Phi^2}{2L} = \frac{\sqr{3\,\text{Вб}}}{2 \cdot 400\,\text{мГн}} \approx 11{,}25\,\text{Дж}.$
}
\solutionspace{60pt}

\tasknumber{2}%
\task{%
    В одной катушке индуктивностью $794\,\text{мГн}$ протекает электрический ток силой $611\,\text{мА}$.
    А в другой — с индуктивностью в шесть раз меньше — ток в три раза сильнее.
    Определите энергию магнитного поля первой катушки, индуктивность второй катушки
    и отношение энергий магнитного поля в двух этих катушках.
}
\answer{%
    $
        L_2 = \frac16L_1 = 132{,}33\,\text{мГн}, \quad
        W_1 = \frac{L_1\eli_1^2}2 = \frac{794\,\text{мГн} \cdot \sqr{611\,\text{мА}}}2 \approx 0{,}148\,\text{Дж}, \quad
        \frac{W_2}{W_1} = \frac{\frac{L_2\eli_2^2}2}{\frac{L_1\eli_1^2}2} = \frac32.
    $
}
\solutionspace{90pt}

\tasknumber{3}%
\task{%
    Определите индуктивность катушки, если при пропускании тока силой $1{,}5\,\text{А}$
    в ней возникает магнитное поле индукцией $8\,\text{Тл}$.
    Катушка представляет собой цилиндр радиусом $4\,\text{см}$ и высотой $6\,\text{мм}$.
    Число витков в катушке 200.
}
\answer{%
    $
        \Phi = L\eli,
        \Phi = BSN,
        S=\pi R^2
        \implies L = \frac{\pi B R^2 N}{\eli} = \frac{\pi \cdot 8\,\text{Тл} \cdot \sqr{4\,\text{см}} \cdot 200}{1{,}5\,\text{А}}
        \approx 5{,}36\,\text{Гн}.
    $
}
\solutionspace{90pt}

\tasknumber{4}%
\task{%
    Тонкий прямой стержень длиной $50\,\text{см}$ вращается в горизонтальной плоскости
    вокруг одного из своих концов.
    Период обращения стержня $2\,\text{c}$.
    Однородное магнитное поле индукцией $200\,\text{мТл}$ направлено вертикально.
    Чему равна разность потенциалов на концах стержня? Ответ выразите в милливольтах.
}
\answer{%
    $
        \ele_i
            = \frac{\Delta \Phi}{\Delta t}
            = \frac{B\Delta S}{\Delta t}
            = \frac{B \frac 12 l^2 \Delta \alpha}{\Delta t}
            = \frac{B l^2 \omega}2, \quad
        \omega = \frac{2 \pi}T \implies
        \ele_i = \frac{\pi B l^2}T \approx 78{,}5\,\text{мВ}.
    $
}
\solutionspace{120pt}

\tasknumber{5}%
\task{%
    Проводник лежит на горизонтальных рельсах,
    замкнутых резистором сопротивлением $4\,\text{Ом}$ (см.
    рис.
    на доске).
    Расстояние между рельсами $80\,\text{см}$.
    Конструкция помещена в вертикальное однородное магнитное поле индукцией $400\,\text{мТл}$.
    Какую силу необходимо прикладывать к проводнику, чтобы двигать его вдоль рельс с постоянной скоростью $2\,\frac{\text{м}}{\text{c}}$?
    Трением пренебречь, сопротивления рельс и проводника малы по сравнению с сопротивлением резистора.
    Ответ выразите в миллиньютонах.
}
\answer{%
    $
        F
            = F_A
            = \eli B l
            = \frac{\ele}R \cdot B l
            = \frac{B v l}R \cdot B l
            = \frac{B^2 v l^2}R
            = \frac{\sqr{400\,\text{мТл}} \cdot 2\,\frac{\text{м}}{\text{c}} \cdot \sqr{80\,\text{см}}}{4\,\text{Ом}}
            \approx 51{,}20\,\text{мН}.
    $
}

\variantsplitter

\addpersonalvariant{Леонид Никитин}

\tasknumber{1}%
\task{%
    Определите энергию магнитного поля в катушке индуктивностью $300\,\text{мГн}$, если её собственный магнитный поток равен $6\,\text{Вб}$.
}
\answer{%
    $W = \frac{\Phi^2}{2L} = \frac{\sqr{6\,\text{Вб}}}{2 \cdot 300\,\text{мГн}} \approx 60{,}00\,\text{Дж}.$
}
\solutionspace{60pt}

\tasknumber{2}%
\task{%
    В одной катушке индуктивностью $332\,\text{мГн}$ протекает электрический ток силой $519\,\text{мА}$.
    А в другой — с индуктивностью в пять раз меньше — ток в шесть раз сильнее.
    Определите энергию магнитного поля первой катушки, индуктивность второй катушки
    и отношение энергий магнитного поля в двух этих катушках.
}
\answer{%
    $
        L_2 = \frac15L_1 = 66{,}40\,\text{мГн}, \quad
        W_1 = \frac{L_1\eli_1^2}2 = \frac{332\,\text{мГн} \cdot \sqr{519\,\text{мА}}}2 \approx 0{,}045\,\text{Дж}, \quad
        \frac{W_2}{W_1} = \frac{\frac{L_2\eli_2^2}2}{\frac{L_1\eli_1^2}2} = \frac{36}5.
    $
}
\solutionspace{90pt}

\tasknumber{3}%
\task{%
    Определите индуктивность катушки, если при пропускании тока силой $2{,}5\,\text{А}$
    в ней возникает магнитное поле индукцией $5\,\text{Тл}$.
    Катушка представляет собой цилиндр радиусом $4\,\text{см}$ и высотой $7\,\text{мм}$.
    Число витков в катушке 200.
}
\answer{%
    $
        \Phi = L\eli,
        \Phi = BSN,
        S=\pi R^2
        \implies L = \frac{\pi B R^2 N}{\eli} = \frac{\pi \cdot 5\,\text{Тл} \cdot \sqr{4\,\text{см}} \cdot 200}{2{,}5\,\text{А}}
        \approx 2{,}01\,\text{Гн}.
    $
}
\solutionspace{90pt}

\tasknumber{4}%
\task{%
    Тонкий прямой стержень длиной $50\,\text{см}$ вращается в горизонтальной плоскости
    вокруг одного из своих концов.
    Период обращения стержня $5\,\text{c}$.
    Однородное магнитное поле индукцией $150\,\text{мТл}$ направлено вертикально.
    Чему равна разность потенциалов на концах стержня? Ответ выразите в милливольтах.
}
\answer{%
    $
        \ele_i
            = \frac{\Delta \Phi}{\Delta t}
            = \frac{B\Delta S}{\Delta t}
            = \frac{B \frac 12 l^2 \Delta \alpha}{\Delta t}
            = \frac{B l^2 \omega}2, \quad
        \omega = \frac{2 \pi}T \implies
        \ele_i = \frac{\pi B l^2}T \approx 23{,}6\,\text{мВ}.
    $
}
\solutionspace{120pt}

\tasknumber{5}%
\task{%
    Проводник лежит на горизонтальных рельсах,
    замкнутых резистором сопротивлением $2\,\text{Ом}$ (см.
    рис.
    на доске).
    Расстояние между рельсами $70\,\text{см}$.
    Конструкция помещена в вертикальное однородное магнитное поле индукцией $200\,\text{мТл}$.
    Какую силу необходимо прикладывать к проводнику, чтобы двигать его вдоль рельс с постоянной скоростью $4\,\frac{\text{м}}{\text{c}}$?
    Трением пренебречь, сопротивления рельс и проводника малы по сравнению с сопротивлением резистора.
    Ответ выразите в миллиньютонах.
}
\answer{%
    $
        F
            = F_A
            = \eli B l
            = \frac{\ele}R \cdot B l
            = \frac{B v l}R \cdot B l
            = \frac{B^2 v l^2}R
            = \frac{\sqr{200\,\text{мТл}} \cdot 4\,\frac{\text{м}}{\text{c}} \cdot \sqr{70\,\text{см}}}{2\,\text{Ом}}
            \approx 39{,}20\,\text{мН}.
    $
}

\variantsplitter

\addpersonalvariant{Тимофей Полетаев}

\tasknumber{1}%
\task{%
    Определите энергию магнитного поля в катушке индуктивностью $200\,\text{мГн}$, если протекающий через неё ток равен $4\,\text{А}$.
}
\answer{%
    $W = \frac{L\eli^2}2 = \frac{200\,\text{мГн} \cdot \sqr{4\,\text{А}}}2 \approx 1{,}60\,\text{Дж}.$
}
\solutionspace{60pt}

\tasknumber{2}%
\task{%
    В одной катушке индуктивностью $772\,\text{мГн}$ протекает электрический ток силой $680\,\text{мА}$.
    А в другой — с индуктивностью в два раза больше — ток в три раза сильнее.
    Определите энергию магнитного поля первой катушки, индуктивность второй катушки
    и отношение энергий магнитного поля в двух этих катушках.
}
\answer{%
    $
        L_2 = 2L_1 = 1544{,}00\,\text{мГн}, \quad
        W_1 = \frac{L_1\eli_1^2}2 = \frac{772\,\text{мГн} \cdot \sqr{680\,\text{мА}}}2 \approx 0{,}178\,\text{Дж}, \quad
        \frac{W_2}{W_1} = \frac{\frac{L_2\eli_2^2}2}{\frac{L_1\eli_1^2}2} = 18.
    $
}
\solutionspace{90pt}

\tasknumber{3}%
\task{%
    Определите индуктивность катушки, если при пропускании тока силой $1{,}5\,\text{А}$
    в ней возникает магнитное поле индукцией $5\,\text{Тл}$.
    Катушка представляет собой цилиндр радиусом $5\,\text{см}$ и высотой $7\,\text{мм}$.
    Число витков в катушке 200.
}
\answer{%
    $
        \Phi = L\eli,
        \Phi = BSN,
        S=\pi R^2
        \implies L = \frac{\pi B R^2 N}{\eli} = \frac{\pi \cdot 5\,\text{Тл} \cdot \sqr{5\,\text{см}} \cdot 200}{1{,}5\,\text{А}}
        \approx 5{,}24\,\text{Гн}.
    $
}
\solutionspace{90pt}

\tasknumber{4}%
\task{%
    Тонкий прямой стержень длиной $50\,\text{см}$ вращается в горизонтальной плоскости
    вокруг одного из своих концов.
    Период обращения стержня $2\,\text{c}$.
    Однородное магнитное поле индукцией $400\,\text{мТл}$ направлено вертикально.
    Чему равна разность потенциалов на концах стержня? Ответ выразите в милливольтах.
}
\answer{%
    $
        \ele_i
            = \frac{\Delta \Phi}{\Delta t}
            = \frac{B\Delta S}{\Delta t}
            = \frac{B \frac 12 l^2 \Delta \alpha}{\Delta t}
            = \frac{B l^2 \omega}2, \quad
        \omega = \frac{2 \pi}T \implies
        \ele_i = \frac{\pi B l^2}T \approx 157{,}1\,\text{мВ}.
    $
}
\solutionspace{120pt}

\tasknumber{5}%
\task{%
    Проводник лежит на горизонтальных рельсах,
    замкнутых резистором сопротивлением $3\,\text{Ом}$ (см.
    рис.
    на доске).
    Расстояние между рельсами $70\,\text{см}$.
    Конструкция помещена в вертикальное однородное магнитное поле индукцией $400\,\text{мТл}$.
    Какую силу необходимо прикладывать к проводнику, чтобы двигать его вдоль рельс с постоянной скоростью $5\,\frac{\text{м}}{\text{c}}$?
    Трением пренебречь, сопротивления рельс и проводника малы по сравнению с сопротивлением резистора.
    Ответ выразите в миллиньютонах.
}
\answer{%
    $
        F
            = F_A
            = \eli B l
            = \frac{\ele}R \cdot B l
            = \frac{B v l}R \cdot B l
            = \frac{B^2 v l^2}R
            = \frac{\sqr{400\,\text{мТл}} \cdot 5\,\frac{\text{м}}{\text{c}} \cdot \sqr{70\,\text{см}}}{3\,\text{Ом}}
            \approx 130{,}67\,\text{мН}.
    $
}

\variantsplitter

\addpersonalvariant{Андрей Рожков}

\tasknumber{1}%
\task{%
    Определите энергию магнитного поля в катушке индуктивностью $400\,\text{мГн}$, если протекающий через неё ток равен $5\,\text{А}$.
}
\answer{%
    $W = \frac{L\eli^2}2 = \frac{400\,\text{мГн} \cdot \sqr{5\,\text{А}}}2 \approx 5{,}00\,\text{Дж}.$
}
\solutionspace{60pt}

\tasknumber{2}%
\task{%
    В одной катушке индуктивностью $563\,\text{мГн}$ протекает электрический ток силой $818\,\text{мА}$.
    А в другой — с индуктивностью в два раза больше — ток в три раза сильнее.
    Определите энергию магнитного поля первой катушки, индуктивность второй катушки
    и отношение энергий магнитного поля в двух этих катушках.
}
\answer{%
    $
        L_2 = 2L_1 = 1126{,}00\,\text{мГн}, \quad
        W_1 = \frac{L_1\eli_1^2}2 = \frac{563\,\text{мГн} \cdot \sqr{818\,\text{мА}}}2 \approx 0{,}188\,\text{Дж}, \quad
        \frac{W_2}{W_1} = \frac{\frac{L_2\eli_2^2}2}{\frac{L_1\eli_1^2}2} = 18.
    $
}
\solutionspace{90pt}

\tasknumber{3}%
\task{%
    Определите индуктивность катушки, если при пропускании тока силой $3\,\text{А}$
    в ней возникает магнитное поле индукцией $2\,\text{Тл}$.
    Катушка представляет собой цилиндр радиусом $3\,\text{см}$ и высотой $7\,\text{мм}$.
    Число витков в катушке 200.
}
\answer{%
    $
        \Phi = L\eli,
        \Phi = BSN,
        S=\pi R^2
        \implies L = \frac{\pi B R^2 N}{\eli} = \frac{\pi \cdot 2\,\text{Тл} \cdot \sqr{3\,\text{см}} \cdot 200}{3\,\text{А}}
        \approx 0{,}38\,\text{Гн}.
    $
}
\solutionspace{90pt}

\tasknumber{4}%
\task{%
    Тонкий прямой стержень длиной $25\,\text{см}$ вращается в горизонтальной плоскости
    вокруг одного из своих концов.
    Период обращения стержня $3\,\text{c}$.
    Однородное магнитное поле индукцией $150\,\text{мТл}$ направлено вертикально.
    Чему равна разность потенциалов на концах стержня? Ответ выразите в милливольтах.
}
\answer{%
    $
        \ele_i
            = \frac{\Delta \Phi}{\Delta t}
            = \frac{B\Delta S}{\Delta t}
            = \frac{B \frac 12 l^2 \Delta \alpha}{\Delta t}
            = \frac{B l^2 \omega}2, \quad
        \omega = \frac{2 \pi}T \implies
        \ele_i = \frac{\pi B l^2}T \approx 9{,}8\,\text{мВ}.
    $
}
\solutionspace{120pt}

\tasknumber{5}%
\task{%
    Проводник лежит на горизонтальных рельсах,
    замкнутых резистором сопротивлением $4\,\text{Ом}$ (см.
    рис.
    на доске).
    Расстояние между рельсами $70\,\text{см}$.
    Конструкция помещена в вертикальное однородное магнитное поле индукцией $150\,\text{мТл}$.
    Какую силу необходимо прикладывать к проводнику, чтобы двигать его вдоль рельс с постоянной скоростью $2\,\frac{\text{м}}{\text{c}}$?
    Трением пренебречь, сопротивления рельс и проводника малы по сравнению с сопротивлением резистора.
    Ответ выразите в миллиньютонах.
}
\answer{%
    $
        F
            = F_A
            = \eli B l
            = \frac{\ele}R \cdot B l
            = \frac{B v l}R \cdot B l
            = \frac{B^2 v l^2}R
            = \frac{\sqr{150\,\text{мТл}} \cdot 2\,\frac{\text{м}}{\text{c}} \cdot \sqr{70\,\text{см}}}{4\,\text{Ом}}
            \approx 5{,}51\,\text{мН}.
    $
}

\variantsplitter

\addpersonalvariant{Рената Таржиманова}

\tasknumber{1}%
\task{%
    Определите энергию магнитного поля в катушке индуктивностью $400\,\text{мГн}$, если протекающий через неё ток равен $3\,\text{А}$.
}
\answer{%
    $W = \frac{L\eli^2}2 = \frac{400\,\text{мГн} \cdot \sqr{3\,\text{А}}}2 \approx 1{,}80\,\text{Дж}.$
}
\solutionspace{60pt}

\tasknumber{2}%
\task{%
    В одной катушке индуктивностью $585\,\text{мГн}$ протекает электрический ток силой $565\,\text{мА}$.
    А в другой — с индуктивностью в шесть раз меньше — ток в три раза сильнее.
    Определите энергию магнитного поля первой катушки, индуктивность второй катушки
    и отношение энергий магнитного поля в двух этих катушках.
}
\answer{%
    $
        L_2 = \frac16L_1 = 97{,}50\,\text{мГн}, \quad
        W_1 = \frac{L_1\eli_1^2}2 = \frac{585\,\text{мГн} \cdot \sqr{565\,\text{мА}}}2 \approx 0{,}093\,\text{Дж}, \quad
        \frac{W_2}{W_1} = \frac{\frac{L_2\eli_2^2}2}{\frac{L_1\eli_1^2}2} = \frac32.
    $
}
\solutionspace{90pt}

\tasknumber{3}%
\task{%
    Определите индуктивность катушки, если при пропускании тока силой $2{,}5\,\text{А}$
    в ней возникает магнитное поле индукцией $8\,\text{Тл}$.
    Катушка представляет собой цилиндр радиусом $3\,\text{см}$ и высотой $8\,\text{мм}$.
    Число витков в катушке 200.
}
\answer{%
    $
        \Phi = L\eli,
        \Phi = BSN,
        S=\pi R^2
        \implies L = \frac{\pi B R^2 N}{\eli} = \frac{\pi \cdot 8\,\text{Тл} \cdot \sqr{3\,\text{см}} \cdot 200}{2{,}5\,\text{А}}
        \approx 1{,}81\,\text{Гн}.
    $
}
\solutionspace{90pt}

\tasknumber{4}%
\task{%
    Тонкий прямой стержень длиной $25\,\text{см}$ вращается в горизонтальной плоскости
    вокруг одного из своих концов.
    Период обращения стержня $3\,\text{c}$.
    Однородное магнитное поле индукцией $400\,\text{мТл}$ направлено вертикально.
    Чему равна разность потенциалов на концах стержня? Ответ выразите в милливольтах.
}
\answer{%
    $
        \ele_i
            = \frac{\Delta \Phi}{\Delta t}
            = \frac{B\Delta S}{\Delta t}
            = \frac{B \frac 12 l^2 \Delta \alpha}{\Delta t}
            = \frac{B l^2 \omega}2, \quad
        \omega = \frac{2 \pi}T \implies
        \ele_i = \frac{\pi B l^2}T \approx 26{,}2\,\text{мВ}.
    $
}
\solutionspace{120pt}

\tasknumber{5}%
\task{%
    Проводник лежит на горизонтальных рельсах,
    замкнутых резистором сопротивлением $4\,\text{Ом}$ (см.
    рис.
    на доске).
    Расстояние между рельсами $50\,\text{см}$.
    Конструкция помещена в вертикальное однородное магнитное поле индукцией $200\,\text{мТл}$.
    Какую силу необходимо прикладывать к проводнику, чтобы двигать его вдоль рельс с постоянной скоростью $5\,\frac{\text{м}}{\text{c}}$?
    Трением пренебречь, сопротивления рельс и проводника малы по сравнению с сопротивлением резистора.
    Ответ выразите в миллиньютонах.
}
\answer{%
    $
        F
            = F_A
            = \eli B l
            = \frac{\ele}R \cdot B l
            = \frac{B v l}R \cdot B l
            = \frac{B^2 v l^2}R
            = \frac{\sqr{200\,\text{мТл}} \cdot 5\,\frac{\text{м}}{\text{c}} \cdot \sqr{50\,\text{см}}}{4\,\text{Ом}}
            \approx 12{,}50\,\text{мН}.
    $
}

\variantsplitter

\addpersonalvariant{Андрей Щербаков}

\tasknumber{1}%
\task{%
    Определите энергию магнитного поля в катушке индуктивностью $400\,\text{мГн}$, если её собственный магнитный поток равен $4\,\text{Вб}$.
}
\answer{%
    $W = \frac{\Phi^2}{2L} = \frac{\sqr{4\,\text{Вб}}}{2 \cdot 400\,\text{мГн}} \approx 20{,}00\,\text{Дж}.$
}
\solutionspace{60pt}

\tasknumber{2}%
\task{%
    В одной катушке индуктивностью $178\,\text{мГн}$ протекает электрический ток силой $864\,\text{мА}$.
    А в другой — с индуктивностью в восемь раз меньше — ток в три раза сильнее.
    Определите энергию магнитного поля первой катушки, индуктивность второй катушки
    и отношение энергий магнитного поля в двух этих катушках.
}
\answer{%
    $
        L_2 = \frac18L_1 = 22{,}25\,\text{мГн}, \quad
        W_1 = \frac{L_1\eli_1^2}2 = \frac{178\,\text{мГн} \cdot \sqr{864\,\text{мА}}}2 \approx 0{,}066\,\text{Дж}, \quad
        \frac{W_2}{W_1} = \frac{\frac{L_2\eli_2^2}2}{\frac{L_1\eli_1^2}2} = \frac98.
    $
}
\solutionspace{90pt}

\tasknumber{3}%
\task{%
    Определите индуктивность катушки, если при пропускании тока силой $1{,}5\,\text{А}$
    в ней возникает магнитное поле индукцией $5\,\text{Тл}$.
    Катушка представляет собой цилиндр радиусом $4\,\text{см}$ и высотой $8\,\text{мм}$.
    Число витков в катушке 200.
}
\answer{%
    $
        \Phi = L\eli,
        \Phi = BSN,
        S=\pi R^2
        \implies L = \frac{\pi B R^2 N}{\eli} = \frac{\pi \cdot 5\,\text{Тл} \cdot \sqr{4\,\text{см}} \cdot 200}{1{,}5\,\text{А}}
        \approx 3{,}35\,\text{Гн}.
    $
}
\solutionspace{90pt}

\tasknumber{4}%
\task{%
    Тонкий прямой стержень длиной $50\,\text{см}$ вращается в горизонтальной плоскости
    вокруг одного из своих концов.
    Период обращения стержня $4\,\text{c}$.
    Однородное магнитное поле индукцией $150\,\text{мТл}$ направлено вертикально.
    Чему равна разность потенциалов на концах стержня? Ответ выразите в милливольтах.
}
\answer{%
    $
        \ele_i
            = \frac{\Delta \Phi}{\Delta t}
            = \frac{B\Delta S}{\Delta t}
            = \frac{B \frac 12 l^2 \Delta \alpha}{\Delta t}
            = \frac{B l^2 \omega}2, \quad
        \omega = \frac{2 \pi}T \implies
        \ele_i = \frac{\pi B l^2}T \approx 29{,}5\,\text{мВ}.
    $
}
\solutionspace{120pt}

\tasknumber{5}%
\task{%
    Проводник лежит на горизонтальных рельсах,
    замкнутых резистором сопротивлением $3\,\text{Ом}$ (см.
    рис.
    на доске).
    Расстояние между рельсами $70\,\text{см}$.
    Конструкция помещена в вертикальное однородное магнитное поле индукцией $150\,\text{мТл}$.
    Какую силу необходимо прикладывать к проводнику, чтобы двигать его вдоль рельс с постоянной скоростью $5\,\frac{\text{м}}{\text{c}}$?
    Трением пренебречь, сопротивления рельс и проводника малы по сравнению с сопротивлением резистора.
    Ответ выразите в миллиньютонах.
}
\answer{%
    $
        F
            = F_A
            = \eli B l
            = \frac{\ele}R \cdot B l
            = \frac{B v l}R \cdot B l
            = \frac{B^2 v l^2}R
            = \frac{\sqr{150\,\text{мТл}} \cdot 5\,\frac{\text{м}}{\text{c}} \cdot \sqr{70\,\text{см}}}{3\,\text{Ом}}
            \approx 18{,}38\,\text{мН}.
    $
}

\variantsplitter

\addpersonalvariant{Михаил Ярошевский}

\tasknumber{1}%
\task{%
    Определите энергию магнитного поля в катушке индуктивностью $600\,\text{мГн}$, если её собственный магнитный поток равен $7\,\text{Вб}$.
}
\answer{%
    $W = \frac{\Phi^2}{2L} = \frac{\sqr{7\,\text{Вб}}}{2 \cdot 600\,\text{мГн}} \approx 40{,}83\,\text{Дж}.$
}
\solutionspace{60pt}

\tasknumber{2}%
\task{%
    В одной катушке индуктивностью $596\,\text{мГн}$ протекает электрический ток силой $588\,\text{мА}$.
    А в другой — с индуктивностью в три раза больше — ток в пять раз сильнее.
    Определите энергию магнитного поля первой катушки, индуктивность второй катушки
    и отношение энергий магнитного поля в двух этих катушках.
}
\answer{%
    $
        L_2 = 3L_1 = 1788{,}00\,\text{мГн}, \quad
        W_1 = \frac{L_1\eli_1^2}2 = \frac{596\,\text{мГн} \cdot \sqr{588\,\text{мА}}}2 \approx 0{,}103\,\text{Дж}, \quad
        \frac{W_2}{W_1} = \frac{\frac{L_2\eli_2^2}2}{\frac{L_1\eli_1^2}2} = 75.
    $
}
\solutionspace{90pt}

\tasknumber{3}%
\task{%
    Определите индуктивность катушки, если при пропускании тока силой $3\,\text{А}$
    в ней возникает магнитное поле индукцией $8\,\text{Тл}$.
    Катушка представляет собой цилиндр радиусом $5\,\text{см}$ и высотой $8\,\text{мм}$.
    Число витков в катушке 400.
}
\answer{%
    $
        \Phi = L\eli,
        \Phi = BSN,
        S=\pi R^2
        \implies L = \frac{\pi B R^2 N}{\eli} = \frac{\pi \cdot 8\,\text{Тл} \cdot \sqr{5\,\text{см}} \cdot 400}{3\,\text{А}}
        \approx 8{,}38\,\text{Гн}.
    $
}
\solutionspace{90pt}

\tasknumber{4}%
\task{%
    Тонкий прямой стержень длиной $50\,\text{см}$ вращается в горизонтальной плоскости
    вокруг одного из своих концов.
    Период обращения стержня $5\,\text{c}$.
    Однородное магнитное поле индукцией $300\,\text{мТл}$ направлено вертикально.
    Чему равна разность потенциалов на концах стержня? Ответ выразите в милливольтах.
}
\answer{%
    $
        \ele_i
            = \frac{\Delta \Phi}{\Delta t}
            = \frac{B\Delta S}{\Delta t}
            = \frac{B \frac 12 l^2 \Delta \alpha}{\Delta t}
            = \frac{B l^2 \omega}2, \quad
        \omega = \frac{2 \pi}T \implies
        \ele_i = \frac{\pi B l^2}T \approx 47{,}1\,\text{мВ}.
    $
}
\solutionspace{120pt}

\tasknumber{5}%
\task{%
    Проводник лежит на горизонтальных рельсах,
    замкнутых резистором сопротивлением $2\,\text{Ом}$ (см.
    рис.
    на доске).
    Расстояние между рельсами $80\,\text{см}$.
    Конструкция помещена в вертикальное однородное магнитное поле индукцией $400\,\text{мТл}$.
    Какую силу необходимо прикладывать к проводнику, чтобы двигать его вдоль рельс с постоянной скоростью $4\,\frac{\text{м}}{\text{c}}$?
    Трением пренебречь, сопротивления рельс и проводника малы по сравнению с сопротивлением резистора.
    Ответ выразите в миллиньютонах.
}
\answer{%
    $
        F
            = F_A
            = \eli B l
            = \frac{\ele}R \cdot B l
            = \frac{B v l}R \cdot B l
            = \frac{B^2 v l^2}R
            = \frac{\sqr{400\,\text{мТл}} \cdot 4\,\frac{\text{м}}{\text{c}} \cdot \sqr{80\,\text{см}}}{2\,\text{Ом}}
            \approx 204{,}80\,\text{мН}.
    $
}

\variantsplitter

\addpersonalvariant{Алексей Алимпиев}

\tasknumber{1}%
\task{%
    Определите энергию магнитного поля в катушке индуктивностью $600\,\text{мГн}$, если протекающий через неё ток равен $8\,\text{А}$.
}
\answer{%
    $W = \frac{L\eli^2}2 = \frac{600\,\text{мГн} \cdot \sqr{8\,\text{А}}}2 \approx 19{,}20\,\text{Дж}.$
}
\solutionspace{60pt}

\tasknumber{2}%
\task{%
    В одной катушке индуктивностью $453\,\text{мГн}$ протекает электрический ток силой $519\,\text{мА}$.
    А в другой — с индуктивностью в два раза больше — ток в шесть раз сильнее.
    Определите энергию магнитного поля первой катушки, индуктивность второй катушки
    и отношение энергий магнитного поля в двух этих катушках.
}
\answer{%
    $
        L_2 = 2L_1 = 906{,}00\,\text{мГн}, \quad
        W_1 = \frac{L_1\eli_1^2}2 = \frac{453\,\text{мГн} \cdot \sqr{519\,\text{мА}}}2 \approx 0{,}061\,\text{Дж}, \quad
        \frac{W_2}{W_1} = \frac{\frac{L_2\eli_2^2}2}{\frac{L_1\eli_1^2}2} = 72.
    $
}
\solutionspace{90pt}

\tasknumber{3}%
\task{%
    Определите индуктивность катушки, если при пропускании тока силой $3\,\text{А}$
    в ней возникает магнитное поле индукцией $5\,\text{Тл}$.
    Катушка представляет собой цилиндр радиусом $3\,\text{см}$ и высотой $7\,\text{мм}$.
    Число витков в катушке 200.
}
\answer{%
    $
        \Phi = L\eli,
        \Phi = BSN,
        S=\pi R^2
        \implies L = \frac{\pi B R^2 N}{\eli} = \frac{\pi \cdot 5\,\text{Тл} \cdot \sqr{3\,\text{см}} \cdot 200}{3\,\text{А}}
        \approx 0{,}94\,\text{Гн}.
    $
}
\solutionspace{90pt}

\tasknumber{4}%
\task{%
    Тонкий прямой стержень длиной $30\,\text{см}$ вращается в горизонтальной плоскости
    вокруг одного из своих концов.
    Период обращения стержня $3\,\text{c}$.
    Однородное магнитное поле индукцией $150\,\text{мТл}$ направлено вертикально.
    Чему равна разность потенциалов на концах стержня? Ответ выразите в милливольтах.
}
\answer{%
    $
        \ele_i
            = \frac{\Delta \Phi}{\Delta t}
            = \frac{B\Delta S}{\Delta t}
            = \frac{B \frac 12 l^2 \Delta \alpha}{\Delta t}
            = \frac{B l^2 \omega}2, \quad
        \omega = \frac{2 \pi}T \implies
        \ele_i = \frac{\pi B l^2}T \approx 14{,}1\,\text{мВ}.
    $
}
\solutionspace{120pt}

\tasknumber{5}%
\task{%
    Проводник лежит на горизонтальных рельсах,
    замкнутых резистором сопротивлением $2\,\text{Ом}$ (см.
    рис.
    на доске).
    Расстояние между рельсами $70\,\text{см}$.
    Конструкция помещена в вертикальное однородное магнитное поле индукцией $200\,\text{мТл}$.
    Какую силу необходимо прикладывать к проводнику, чтобы двигать его вдоль рельс с постоянной скоростью $2\,\frac{\text{м}}{\text{c}}$?
    Трением пренебречь, сопротивления рельс и проводника малы по сравнению с сопротивлением резистора.
    Ответ выразите в миллиньютонах.
}
\answer{%
    $
        F
            = F_A
            = \eli B l
            = \frac{\ele}R \cdot B l
            = \frac{B v l}R \cdot B l
            = \frac{B^2 v l^2}R
            = \frac{\sqr{200\,\text{мТл}} \cdot 2\,\frac{\text{м}}{\text{c}} \cdot \sqr{70\,\text{см}}}{2\,\text{Ом}}
            \approx 19{,}60\,\text{мН}.
    $
}

\variantsplitter

\addpersonalvariant{Евгений Васин}

\tasknumber{1}%
\task{%
    Определите энергию магнитного поля в катушке индуктивностью $200\,\text{мГн}$, если её собственный магнитный поток равен $5\,\text{Вб}$.
}
\answer{%
    $W = \frac{\Phi^2}{2L} = \frac{\sqr{5\,\text{Вб}}}{2 \cdot 200\,\text{мГн}} \approx 62{,}50\,\text{Дж}.$
}
\solutionspace{60pt}

\tasknumber{2}%
\task{%
    В одной катушке индуктивностью $585\,\text{мГн}$ протекает электрический ток силой $565\,\text{мА}$.
    А в другой — с индуктивностью в семь раз больше — ток в пять раз сильнее.
    Определите энергию магнитного поля первой катушки, индуктивность второй катушки
    и отношение энергий магнитного поля в двух этих катушках.
}
\answer{%
    $
        L_2 = 7L_1 = 4095{,}00\,\text{мГн}, \quad
        W_1 = \frac{L_1\eli_1^2}2 = \frac{585\,\text{мГн} \cdot \sqr{565\,\text{мА}}}2 \approx 0{,}093\,\text{Дж}, \quad
        \frac{W_2}{W_1} = \frac{\frac{L_2\eli_2^2}2}{\frac{L_1\eli_1^2}2} = 175.
    $
}
\solutionspace{90pt}

\tasknumber{3}%
\task{%
    Определите индуктивность катушки, если при пропускании тока силой $2{,}5\,\text{А}$
    в ней возникает магнитное поле индукцией $8\,\text{Тл}$.
    Катушка представляет собой цилиндр радиусом $3\,\text{см}$ и высотой $7\,\text{мм}$.
    Число витков в катушке 200.
}
\answer{%
    $
        \Phi = L\eli,
        \Phi = BSN,
        S=\pi R^2
        \implies L = \frac{\pi B R^2 N}{\eli} = \frac{\pi \cdot 8\,\text{Тл} \cdot \sqr{3\,\text{см}} \cdot 200}{2{,}5\,\text{А}}
        \approx 1{,}81\,\text{Гн}.
    $
}
\solutionspace{90pt}

\tasknumber{4}%
\task{%
    Тонкий прямой стержень длиной $40\,\text{см}$ вращается в горизонтальной плоскости
    вокруг одного из своих концов.
    Период обращения стержня $4\,\text{c}$.
    Однородное магнитное поле индукцией $400\,\text{мТл}$ направлено вертикально.
    Чему равна разность потенциалов на концах стержня? Ответ выразите в милливольтах.
}
\answer{%
    $
        \ele_i
            = \frac{\Delta \Phi}{\Delta t}
            = \frac{B\Delta S}{\Delta t}
            = \frac{B \frac 12 l^2 \Delta \alpha}{\Delta t}
            = \frac{B l^2 \omega}2, \quad
        \omega = \frac{2 \pi}T \implies
        \ele_i = \frac{\pi B l^2}T \approx 50{,}3\,\text{мВ}.
    $
}
\solutionspace{120pt}

\tasknumber{5}%
\task{%
    Проводник лежит на горизонтальных рельсах,
    замкнутых резистором сопротивлением $3\,\text{Ом}$ (см.
    рис.
    на доске).
    Расстояние между рельсами $80\,\text{см}$.
    Конструкция помещена в вертикальное однородное магнитное поле индукцией $200\,\text{мТл}$.
    Какую силу необходимо прикладывать к проводнику, чтобы двигать его вдоль рельс с постоянной скоростью $4\,\frac{\text{м}}{\text{c}}$?
    Трением пренебречь, сопротивления рельс и проводника малы по сравнению с сопротивлением резистора.
    Ответ выразите в миллиньютонах.
}
\answer{%
    $
        F
            = F_A
            = \eli B l
            = \frac{\ele}R \cdot B l
            = \frac{B v l}R \cdot B l
            = \frac{B^2 v l^2}R
            = \frac{\sqr{200\,\text{мТл}} \cdot 4\,\frac{\text{м}}{\text{c}} \cdot \sqr{80\,\text{см}}}{3\,\text{Ом}}
            \approx 34{,}13\,\text{мН}.
    $
}

\variantsplitter

\addpersonalvariant{Вячеслав Волохов}

\tasknumber{1}%
\task{%
    Определите энергию магнитного поля в катушке индуктивностью $600\,\text{мГн}$, если протекающий через неё ток равен $6\,\text{А}$.
}
\answer{%
    $W = \frac{L\eli^2}2 = \frac{600\,\text{мГн} \cdot \sqr{6\,\text{А}}}2 \approx 10{,}80\,\text{Дж}.$
}
\solutionspace{60pt}

\tasknumber{2}%
\task{%
    В одной катушке индуктивностью $266\,\text{мГн}$ протекает электрический ток силой $611\,\text{мА}$.
    А в другой — с индуктивностью в два раза меньше — ток в три раза сильнее.
    Определите энергию магнитного поля первой катушки, индуктивность второй катушки
    и отношение энергий магнитного поля в двух этих катушках.
}
\answer{%
    $
        L_2 = \frac12L_1 = 133{,}00\,\text{мГн}, \quad
        W_1 = \frac{L_1\eli_1^2}2 = \frac{266\,\text{мГн} \cdot \sqr{611\,\text{мА}}}2 \approx 0{,}050\,\text{Дж}, \quad
        \frac{W_2}{W_1} = \frac{\frac{L_2\eli_2^2}2}{\frac{L_1\eli_1^2}2} = \frac92.
    $
}
\solutionspace{90pt}

\tasknumber{3}%
\task{%
    Определите индуктивность катушки, если при пропускании тока силой $3\,\text{А}$
    в ней возникает магнитное поле индукцией $5\,\text{Тл}$.
    Катушка представляет собой цилиндр радиусом $3\,\text{см}$ и высотой $7\,\text{мм}$.
    Число витков в катушке 500.
}
\answer{%
    $
        \Phi = L\eli,
        \Phi = BSN,
        S=\pi R^2
        \implies L = \frac{\pi B R^2 N}{\eli} = \frac{\pi \cdot 5\,\text{Тл} \cdot \sqr{3\,\text{см}} \cdot 500}{3\,\text{А}}
        \approx 2{,}36\,\text{Гн}.
    $
}
\solutionspace{90pt}

\tasknumber{4}%
\task{%
    Тонкий прямой стержень длиной $50\,\text{см}$ вращается в горизонтальной плоскости
    вокруг одного из своих концов.
    Период обращения стержня $5\,\text{c}$.
    Однородное магнитное поле индукцией $200\,\text{мТл}$ направлено вертикально.
    Чему равна разность потенциалов на концах стержня? Ответ выразите в милливольтах.
}
\answer{%
    $
        \ele_i
            = \frac{\Delta \Phi}{\Delta t}
            = \frac{B\Delta S}{\Delta t}
            = \frac{B \frac 12 l^2 \Delta \alpha}{\Delta t}
            = \frac{B l^2 \omega}2, \quad
        \omega = \frac{2 \pi}T \implies
        \ele_i = \frac{\pi B l^2}T \approx 31{,}4\,\text{мВ}.
    $
}
\solutionspace{120pt}

\tasknumber{5}%
\task{%
    Проводник лежит на горизонтальных рельсах,
    замкнутых резистором сопротивлением $4\,\text{Ом}$ (см.
    рис.
    на доске).
    Расстояние между рельсами $50\,\text{см}$.
    Конструкция помещена в вертикальное однородное магнитное поле индукцией $400\,\text{мТл}$.
    Какую силу необходимо прикладывать к проводнику, чтобы двигать его вдоль рельс с постоянной скоростью $2\,\frac{\text{м}}{\text{c}}$?
    Трением пренебречь, сопротивления рельс и проводника малы по сравнению с сопротивлением резистора.
    Ответ выразите в миллиньютонах.
}
\answer{%
    $
        F
            = F_A
            = \eli B l
            = \frac{\ele}R \cdot B l
            = \frac{B v l}R \cdot B l
            = \frac{B^2 v l^2}R
            = \frac{\sqr{400\,\text{мТл}} \cdot 2\,\frac{\text{м}}{\text{c}} \cdot \sqr{50\,\text{см}}}{4\,\text{Ом}}
            \approx 20{,}00\,\text{мН}.
    $
}

\variantsplitter

\addpersonalvariant{Герман Говоров}

\tasknumber{1}%
\task{%
    Определите энергию магнитного поля в катушке индуктивностью $600\,\text{мГн}$, если её собственный магнитный поток равен $4\,\text{Вб}$.
}
\answer{%
    $W = \frac{\Phi^2}{2L} = \frac{\sqr{4\,\text{Вб}}}{2 \cdot 600\,\text{мГн}} \approx 13{,}33\,\text{Дж}.$
}
\solutionspace{60pt}

\tasknumber{2}%
\task{%
    В одной катушке индуктивностью $233\,\text{мГн}$ протекает электрический ток силой $887\,\text{мА}$.
    А в другой — с индуктивностью в пять раз больше — ток в четыре раза сильнее.
    Определите энергию магнитного поля первой катушки, индуктивность второй катушки
    и отношение энергий магнитного поля в двух этих катушках.
}
\answer{%
    $
        L_2 = 5L_1 = 1165{,}00\,\text{мГн}, \quad
        W_1 = \frac{L_1\eli_1^2}2 = \frac{233\,\text{мГн} \cdot \sqr{887\,\text{мА}}}2 \approx 0{,}092\,\text{Дж}, \quad
        \frac{W_2}{W_1} = \frac{\frac{L_2\eli_2^2}2}{\frac{L_1\eli_1^2}2} = 80.
    $
}
\solutionspace{90pt}

\tasknumber{3}%
\task{%
    Определите индуктивность катушки, если при пропускании тока силой $2{,}5\,\text{А}$
    в ней возникает магнитное поле индукцией $8\,\text{Тл}$.
    Катушка представляет собой цилиндр радиусом $3\,\text{см}$ и высотой $7\,\text{мм}$.
    Число витков в катушке 400.
}
\answer{%
    $
        \Phi = L\eli,
        \Phi = BSN,
        S=\pi R^2
        \implies L = \frac{\pi B R^2 N}{\eli} = \frac{\pi \cdot 8\,\text{Тл} \cdot \sqr{3\,\text{см}} \cdot 400}{2{,}5\,\text{А}}
        \approx 3{,}62\,\text{Гн}.
    $
}
\solutionspace{90pt}

\tasknumber{4}%
\task{%
    Тонкий прямой стержень длиной $50\,\text{см}$ вращается в горизонтальной плоскости
    вокруг одного из своих концов.
    Период обращения стержня $3\,\text{c}$.
    Однородное магнитное поле индукцией $200\,\text{мТл}$ направлено вертикально.
    Чему равна разность потенциалов на концах стержня? Ответ выразите в милливольтах.
}
\answer{%
    $
        \ele_i
            = \frac{\Delta \Phi}{\Delta t}
            = \frac{B\Delta S}{\Delta t}
            = \frac{B \frac 12 l^2 \Delta \alpha}{\Delta t}
            = \frac{B l^2 \omega}2, \quad
        \omega = \frac{2 \pi}T \implies
        \ele_i = \frac{\pi B l^2}T \approx 52{,}4\,\text{мВ}.
    $
}
\solutionspace{120pt}

\tasknumber{5}%
\task{%
    Проводник лежит на горизонтальных рельсах,
    замкнутых резистором сопротивлением $3\,\text{Ом}$ (см.
    рис.
    на доске).
    Расстояние между рельсами $50\,\text{см}$.
    Конструкция помещена в вертикальное однородное магнитное поле индукцией $300\,\text{мТл}$.
    Какую силу необходимо прикладывать к проводнику, чтобы двигать его вдоль рельс с постоянной скоростью $5\,\frac{\text{м}}{\text{c}}$?
    Трением пренебречь, сопротивления рельс и проводника малы по сравнению с сопротивлением резистора.
    Ответ выразите в миллиньютонах.
}
\answer{%
    $
        F
            = F_A
            = \eli B l
            = \frac{\ele}R \cdot B l
            = \frac{B v l}R \cdot B l
            = \frac{B^2 v l^2}R
            = \frac{\sqr{300\,\text{мТл}} \cdot 5\,\frac{\text{м}}{\text{c}} \cdot \sqr{50\,\text{см}}}{3\,\text{Ом}}
            \approx 37{,}50\,\text{мН}.
    $
}

\variantsplitter

\addpersonalvariant{София Журавлёва}

\tasknumber{1}%
\task{%
    Определите энергию магнитного поля в катушке индуктивностью $600\,\text{мГн}$, если её собственный магнитный поток равен $3\,\text{Вб}$.
}
\answer{%
    $W = \frac{\Phi^2}{2L} = \frac{\sqr{3\,\text{Вб}}}{2 \cdot 600\,\text{мГн}} \approx 7{,}50\,\text{Дж}.$
}
\solutionspace{60pt}

\tasknumber{2}%
\task{%
    В одной катушке индуктивностью $761\,\text{мГн}$ протекает электрический ток силой $703\,\text{мА}$.
    А в другой — с индуктивностью в два раза меньше — ток в шесть раз сильнее.
    Определите энергию магнитного поля первой катушки, индуктивность второй катушки
    и отношение энергий магнитного поля в двух этих катушках.
}
\answer{%
    $
        L_2 = \frac12L_1 = 380{,}50\,\text{мГн}, \quad
        W_1 = \frac{L_1\eli_1^2}2 = \frac{761\,\text{мГн} \cdot \sqr{703\,\text{мА}}}2 \approx 0{,}188\,\text{Дж}, \quad
        \frac{W_2}{W_1} = \frac{\frac{L_2\eli_2^2}2}{\frac{L_1\eli_1^2}2} = 18.
    $
}
\solutionspace{90pt}

\tasknumber{3}%
\task{%
    Определите индуктивность катушки, если при пропускании тока силой $3\,\text{А}$
    в ней возникает магнитное поле индукцией $2\,\text{Тл}$.
    Катушка представляет собой цилиндр радиусом $4\,\text{см}$ и высотой $6\,\text{мм}$.
    Число витков в катушке 200.
}
\answer{%
    $
        \Phi = L\eli,
        \Phi = BSN,
        S=\pi R^2
        \implies L = \frac{\pi B R^2 N}{\eli} = \frac{\pi \cdot 2\,\text{Тл} \cdot \sqr{4\,\text{см}} \cdot 200}{3\,\text{А}}
        \approx 0{,}67\,\text{Гн}.
    $
}
\solutionspace{90pt}

\tasknumber{4}%
\task{%
    Тонкий прямой стержень длиной $50\,\text{см}$ вращается в горизонтальной плоскости
    вокруг одного из своих концов.
    Период обращения стержня $5\,\text{c}$.
    Однородное магнитное поле индукцией $400\,\text{мТл}$ направлено вертикально.
    Чему равна разность потенциалов на концах стержня? Ответ выразите в милливольтах.
}
\answer{%
    $
        \ele_i
            = \frac{\Delta \Phi}{\Delta t}
            = \frac{B\Delta S}{\Delta t}
            = \frac{B \frac 12 l^2 \Delta \alpha}{\Delta t}
            = \frac{B l^2 \omega}2, \quad
        \omega = \frac{2 \pi}T \implies
        \ele_i = \frac{\pi B l^2}T \approx 62{,}8\,\text{мВ}.
    $
}
\solutionspace{120pt}

\tasknumber{5}%
\task{%
    Проводник лежит на горизонтальных рельсах,
    замкнутых резистором сопротивлением $4\,\text{Ом}$ (см.
    рис.
    на доске).
    Расстояние между рельсами $80\,\text{см}$.
    Конструкция помещена в вертикальное однородное магнитное поле индукцией $200\,\text{мТл}$.
    Какую силу необходимо прикладывать к проводнику, чтобы двигать его вдоль рельс с постоянной скоростью $3\,\frac{\text{м}}{\text{c}}$?
    Трением пренебречь, сопротивления рельс и проводника малы по сравнению с сопротивлением резистора.
    Ответ выразите в миллиньютонах.
}
\answer{%
    $
        F
            = F_A
            = \eli B l
            = \frac{\ele}R \cdot B l
            = \frac{B v l}R \cdot B l
            = \frac{B^2 v l^2}R
            = \frac{\sqr{200\,\text{мТл}} \cdot 3\,\frac{\text{м}}{\text{c}} \cdot \sqr{80\,\text{см}}}{4\,\text{Ом}}
            \approx 19{,}20\,\text{мН}.
    $
}

\variantsplitter

\addpersonalvariant{Константин Козлов}

\tasknumber{1}%
\task{%
    Определите энергию магнитного поля в катушке индуктивностью $200\,\text{мГн}$, если протекающий через неё ток равен $8\,\text{А}$.
}
\answer{%
    $W = \frac{L\eli^2}2 = \frac{200\,\text{мГн} \cdot \sqr{8\,\text{А}}}2 \approx 6{,}40\,\text{Дж}.$
}
\solutionspace{60pt}

\tasknumber{2}%
\task{%
    В одной катушке индуктивностью $673\,\text{мГн}$ протекает электрический ток силой $496\,\text{мА}$.
    А в другой — с индуктивностью в шесть раз больше — ток в четыре раза сильнее.
    Определите энергию магнитного поля первой катушки, индуктивность второй катушки
    и отношение энергий магнитного поля в двух этих катушках.
}
\answer{%
    $
        L_2 = 6L_1 = 4038{,}00\,\text{мГн}, \quad
        W_1 = \frac{L_1\eli_1^2}2 = \frac{673\,\text{мГн} \cdot \sqr{496\,\text{мА}}}2 \approx 0{,}083\,\text{Дж}, \quad
        \frac{W_2}{W_1} = \frac{\frac{L_2\eli_2^2}2}{\frac{L_1\eli_1^2}2} = 96.
    $
}
\solutionspace{90pt}

\tasknumber{3}%
\task{%
    Определите индуктивность катушки, если при пропускании тока силой $2{,}5\,\text{А}$
    в ней возникает магнитное поле индукцией $2\,\text{Тл}$.
    Катушка представляет собой цилиндр радиусом $4\,\text{см}$ и высотой $7\,\text{мм}$.
    Число витков в катушке 400.
}
\answer{%
    $
        \Phi = L\eli,
        \Phi = BSN,
        S=\pi R^2
        \implies L = \frac{\pi B R^2 N}{\eli} = \frac{\pi \cdot 2\,\text{Тл} \cdot \sqr{4\,\text{см}} \cdot 400}{2{,}5\,\text{А}}
        \approx 1{,}61\,\text{Гн}.
    $
}
\solutionspace{90pt}

\tasknumber{4}%
\task{%
    Тонкий прямой стержень длиной $20\,\text{см}$ вращается в горизонтальной плоскости
    вокруг одного из своих концов.
    Период обращения стержня $2\,\text{c}$.
    Однородное магнитное поле индукцией $200\,\text{мТл}$ направлено вертикально.
    Чему равна разность потенциалов на концах стержня? Ответ выразите в милливольтах.
}
\answer{%
    $
        \ele_i
            = \frac{\Delta \Phi}{\Delta t}
            = \frac{B\Delta S}{\Delta t}
            = \frac{B \frac 12 l^2 \Delta \alpha}{\Delta t}
            = \frac{B l^2 \omega}2, \quad
        \omega = \frac{2 \pi}T \implies
        \ele_i = \frac{\pi B l^2}T \approx 12{,}6\,\text{мВ}.
    $
}
\solutionspace{120pt}

\tasknumber{5}%
\task{%
    Проводник лежит на горизонтальных рельсах,
    замкнутых резистором сопротивлением $3\,\text{Ом}$ (см.
    рис.
    на доске).
    Расстояние между рельсами $70\,\text{см}$.
    Конструкция помещена в вертикальное однородное магнитное поле индукцией $200\,\text{мТл}$.
    Какую силу необходимо прикладывать к проводнику, чтобы двигать его вдоль рельс с постоянной скоростью $4\,\frac{\text{м}}{\text{c}}$?
    Трением пренебречь, сопротивления рельс и проводника малы по сравнению с сопротивлением резистора.
    Ответ выразите в миллиньютонах.
}
\answer{%
    $
        F
            = F_A
            = \eli B l
            = \frac{\ele}R \cdot B l
            = \frac{B v l}R \cdot B l
            = \frac{B^2 v l^2}R
            = \frac{\sqr{200\,\text{мТл}} \cdot 4\,\frac{\text{м}}{\text{c}} \cdot \sqr{70\,\text{см}}}{3\,\text{Ом}}
            \approx 26{,}13\,\text{мН}.
    $
}

\variantsplitter

\addpersonalvariant{Наталья Кравченко}

\tasknumber{1}%
\task{%
    Определите энергию магнитного поля в катушке индуктивностью $600\,\text{мГн}$, если протекающий через неё ток равен $7\,\text{А}$.
}
\answer{%
    $W = \frac{L\eli^2}2 = \frac{600\,\text{мГн} \cdot \sqr{7\,\text{А}}}2 \approx 14{,}70\,\text{Дж}.$
}
\solutionspace{60pt}

\tasknumber{2}%
\task{%
    В одной катушке индуктивностью $497\,\text{мГн}$ протекает электрический ток силой $519\,\text{мА}$.
    А в другой — с индуктивностью в пять раз больше — ток в пять раз сильнее.
    Определите энергию магнитного поля первой катушки, индуктивность второй катушки
    и отношение энергий магнитного поля в двух этих катушках.
}
\answer{%
    $
        L_2 = 5L_1 = 2485{,}00\,\text{мГн}, \quad
        W_1 = \frac{L_1\eli_1^2}2 = \frac{497\,\text{мГн} \cdot \sqr{519\,\text{мА}}}2 \approx 0{,}067\,\text{Дж}, \quad
        \frac{W_2}{W_1} = \frac{\frac{L_2\eli_2^2}2}{\frac{L_1\eli_1^2}2} = 125.
    $
}
\solutionspace{90pt}

\tasknumber{3}%
\task{%
    Определите индуктивность катушки, если при пропускании тока силой $3\,\text{А}$
    в ней возникает магнитное поле индукцией $8\,\text{Тл}$.
    Катушка представляет собой цилиндр радиусом $3\,\text{см}$ и высотой $8\,\text{мм}$.
    Число витков в катушке 200.
}
\answer{%
    $
        \Phi = L\eli,
        \Phi = BSN,
        S=\pi R^2
        \implies L = \frac{\pi B R^2 N}{\eli} = \frac{\pi \cdot 8\,\text{Тл} \cdot \sqr{3\,\text{см}} \cdot 200}{3\,\text{А}}
        \approx 1{,}51\,\text{Гн}.
    $
}
\solutionspace{90pt}

\tasknumber{4}%
\task{%
    Тонкий прямой стержень длиной $50\,\text{см}$ вращается в горизонтальной плоскости
    вокруг одного из своих концов.
    Период обращения стержня $4\,\text{c}$.
    Однородное магнитное поле индукцией $400\,\text{мТл}$ направлено вертикально.
    Чему равна разность потенциалов на концах стержня? Ответ выразите в милливольтах.
}
\answer{%
    $
        \ele_i
            = \frac{\Delta \Phi}{\Delta t}
            = \frac{B\Delta S}{\Delta t}
            = \frac{B \frac 12 l^2 \Delta \alpha}{\Delta t}
            = \frac{B l^2 \omega}2, \quad
        \omega = \frac{2 \pi}T \implies
        \ele_i = \frac{\pi B l^2}T \approx 78{,}5\,\text{мВ}.
    $
}
\solutionspace{120pt}

\tasknumber{5}%
\task{%
    Проводник лежит на горизонтальных рельсах,
    замкнутых резистором сопротивлением $3\,\text{Ом}$ (см.
    рис.
    на доске).
    Расстояние между рельсами $50\,\text{см}$.
    Конструкция помещена в вертикальное однородное магнитное поле индукцией $150\,\text{мТл}$.
    Какую силу необходимо прикладывать к проводнику, чтобы двигать его вдоль рельс с постоянной скоростью $2\,\frac{\text{м}}{\text{c}}$?
    Трением пренебречь, сопротивления рельс и проводника малы по сравнению с сопротивлением резистора.
    Ответ выразите в миллиньютонах.
}
\answer{%
    $
        F
            = F_A
            = \eli B l
            = \frac{\ele}R \cdot B l
            = \frac{B v l}R \cdot B l
            = \frac{B^2 v l^2}R
            = \frac{\sqr{150\,\text{мТл}} \cdot 2\,\frac{\text{м}}{\text{c}} \cdot \sqr{50\,\text{см}}}{3\,\text{Ом}}
            \approx 3{,}75\,\text{мН}.
    $
}

\variantsplitter

\addpersonalvariant{Сергей Малышев}

\tasknumber{1}%
\task{%
    Определите энергию магнитного поля в катушке индуктивностью $400\,\text{мГн}$, если протекающий через неё ток равен $8\,\text{А}$.
}
\answer{%
    $W = \frac{L\eli^2}2 = \frac{400\,\text{мГн} \cdot \sqr{8\,\text{А}}}2 \approx 12{,}80\,\text{Дж}.$
}
\solutionspace{60pt}

\tasknumber{2}%
\task{%
    В одной катушке индуктивностью $310\,\text{мГн}$ протекает электрический ток силой $795\,\text{мА}$.
    А в другой — с индуктивностью в три раза меньше — ток в три раза сильнее.
    Определите энергию магнитного поля первой катушки, индуктивность второй катушки
    и отношение энергий магнитного поля в двух этих катушках.
}
\answer{%
    $
        L_2 = \frac13L_1 = 103{,}33\,\text{мГн}, \quad
        W_1 = \frac{L_1\eli_1^2}2 = \frac{310\,\text{мГн} \cdot \sqr{795\,\text{мА}}}2 \approx 0{,}098\,\text{Дж}, \quad
        \frac{W_2}{W_1} = \frac{\frac{L_2\eli_2^2}2}{\frac{L_1\eli_1^2}2} = 3.
    $
}
\solutionspace{90pt}

\tasknumber{3}%
\task{%
    Определите индуктивность катушки, если при пропускании тока силой $2{,}5\,\text{А}$
    в ней возникает магнитное поле индукцией $5\,\text{Тл}$.
    Катушка представляет собой цилиндр радиусом $4\,\text{см}$ и высотой $6\,\text{мм}$.
    Число витков в катушке 200.
}
\answer{%
    $
        \Phi = L\eli,
        \Phi = BSN,
        S=\pi R^2
        \implies L = \frac{\pi B R^2 N}{\eli} = \frac{\pi \cdot 5\,\text{Тл} \cdot \sqr{4\,\text{см}} \cdot 200}{2{,}5\,\text{А}}
        \approx 2{,}01\,\text{Гн}.
    $
}
\solutionspace{90pt}

\tasknumber{4}%
\task{%
    Тонкий прямой стержень длиной $30\,\text{см}$ вращается в горизонтальной плоскости
    вокруг одного из своих концов.
    Период обращения стержня $2\,\text{c}$.
    Однородное магнитное поле индукцией $150\,\text{мТл}$ направлено вертикально.
    Чему равна разность потенциалов на концах стержня? Ответ выразите в милливольтах.
}
\answer{%
    $
        \ele_i
            = \frac{\Delta \Phi}{\Delta t}
            = \frac{B\Delta S}{\Delta t}
            = \frac{B \frac 12 l^2 \Delta \alpha}{\Delta t}
            = \frac{B l^2 \omega}2, \quad
        \omega = \frac{2 \pi}T \implies
        \ele_i = \frac{\pi B l^2}T \approx 21{,}2\,\text{мВ}.
    $
}
\solutionspace{120pt}

\tasknumber{5}%
\task{%
    Проводник лежит на горизонтальных рельсах,
    замкнутых резистором сопротивлением $3\,\text{Ом}$ (см.
    рис.
    на доске).
    Расстояние между рельсами $80\,\text{см}$.
    Конструкция помещена в вертикальное однородное магнитное поле индукцией $300\,\text{мТл}$.
    Какую силу необходимо прикладывать к проводнику, чтобы двигать его вдоль рельс с постоянной скоростью $5\,\frac{\text{м}}{\text{c}}$?
    Трением пренебречь, сопротивления рельс и проводника малы по сравнению с сопротивлением резистора.
    Ответ выразите в миллиньютонах.
}
\answer{%
    $
        F
            = F_A
            = \eli B l
            = \frac{\ele}R \cdot B l
            = \frac{B v l}R \cdot B l
            = \frac{B^2 v l^2}R
            = \frac{\sqr{300\,\text{мТл}} \cdot 5\,\frac{\text{м}}{\text{c}} \cdot \sqr{80\,\text{см}}}{3\,\text{Ом}}
            \approx 96{,}00\,\text{мН}.
    $
}

\variantsplitter

\addpersonalvariant{Алина Полканова}

\tasknumber{1}%
\task{%
    Определите энергию магнитного поля в катушке индуктивностью $200\,\text{мГн}$, если её собственный магнитный поток равен $8\,\text{Вб}$.
}
\answer{%
    $W = \frac{\Phi^2}{2L} = \frac{\sqr{8\,\text{Вб}}}{2 \cdot 200\,\text{мГн}} \approx 160{,}00\,\text{Дж}.$
}
\solutionspace{60pt}

\tasknumber{2}%
\task{%
    В одной катушке индуктивностью $717\,\text{мГн}$ протекает электрический ток силой $611\,\text{мА}$.
    А в другой — с индуктивностью в семь раз больше — ток в пять раз сильнее.
    Определите энергию магнитного поля первой катушки, индуктивность второй катушки
    и отношение энергий магнитного поля в двух этих катушках.
}
\answer{%
    $
        L_2 = 7L_1 = 5019{,}00\,\text{мГн}, \quad
        W_1 = \frac{L_1\eli_1^2}2 = \frac{717\,\text{мГн} \cdot \sqr{611\,\text{мА}}}2 \approx 0{,}134\,\text{Дж}, \quad
        \frac{W_2}{W_1} = \frac{\frac{L_2\eli_2^2}2}{\frac{L_1\eli_1^2}2} = 175.
    $
}
\solutionspace{90pt}

\tasknumber{3}%
\task{%
    Определите индуктивность катушки, если при пропускании тока силой $2{,}5\,\text{А}$
    в ней возникает магнитное поле индукцией $8\,\text{Тл}$.
    Катушка представляет собой цилиндр радиусом $5\,\text{см}$ и высотой $7\,\text{мм}$.
    Число витков в катушке 400.
}
\answer{%
    $
        \Phi = L\eli,
        \Phi = BSN,
        S=\pi R^2
        \implies L = \frac{\pi B R^2 N}{\eli} = \frac{\pi \cdot 8\,\text{Тл} \cdot \sqr{5\,\text{см}} \cdot 400}{2{,}5\,\text{А}}
        \approx 10{,}05\,\text{Гн}.
    $
}
\solutionspace{90pt}

\tasknumber{4}%
\task{%
    Тонкий прямой стержень длиной $20\,\text{см}$ вращается в горизонтальной плоскости
    вокруг одного из своих концов.
    Период обращения стержня $2\,\text{c}$.
    Однородное магнитное поле индукцией $200\,\text{мТл}$ направлено вертикально.
    Чему равна разность потенциалов на концах стержня? Ответ выразите в милливольтах.
}
\answer{%
    $
        \ele_i
            = \frac{\Delta \Phi}{\Delta t}
            = \frac{B\Delta S}{\Delta t}
            = \frac{B \frac 12 l^2 \Delta \alpha}{\Delta t}
            = \frac{B l^2 \omega}2, \quad
        \omega = \frac{2 \pi}T \implies
        \ele_i = \frac{\pi B l^2}T \approx 12{,}6\,\text{мВ}.
    $
}
\solutionspace{120pt}

\tasknumber{5}%
\task{%
    Проводник лежит на горизонтальных рельсах,
    замкнутых резистором сопротивлением $2\,\text{Ом}$ (см.
    рис.
    на доске).
    Расстояние между рельсами $50\,\text{см}$.
    Конструкция помещена в вертикальное однородное магнитное поле индукцией $400\,\text{мТл}$.
    Какую силу необходимо прикладывать к проводнику, чтобы двигать его вдоль рельс с постоянной скоростью $2\,\frac{\text{м}}{\text{c}}$?
    Трением пренебречь, сопротивления рельс и проводника малы по сравнению с сопротивлением резистора.
    Ответ выразите в миллиньютонах.
}
\answer{%
    $
        F
            = F_A
            = \eli B l
            = \frac{\ele}R \cdot B l
            = \frac{B v l}R \cdot B l
            = \frac{B^2 v l^2}R
            = \frac{\sqr{400\,\text{мТл}} \cdot 2\,\frac{\text{м}}{\text{c}} \cdot \sqr{50\,\text{см}}}{2\,\text{Ом}}
            \approx 40{,}00\,\text{мН}.
    $
}

\variantsplitter

\addpersonalvariant{Сергей Пономарёв}

\tasknumber{1}%
\task{%
    Определите энергию магнитного поля в катушке индуктивностью $300\,\text{мГн}$, если её собственный магнитный поток равен $6\,\text{Вб}$.
}
\answer{%
    $W = \frac{\Phi^2}{2L} = \frac{\sqr{6\,\text{Вб}}}{2 \cdot 300\,\text{мГн}} \approx 60{,}00\,\text{Дж}.$
}
\solutionspace{60pt}

\tasknumber{2}%
\task{%
    В одной катушке индуктивностью $134\,\text{мГн}$ протекает электрический ток силой $795\,\text{мА}$.
    А в другой — с индуктивностью в четыре раза больше — ток в пять раз сильнее.
    Определите энергию магнитного поля первой катушки, индуктивность второй катушки
    и отношение энергий магнитного поля в двух этих катушках.
}
\answer{%
    $
        L_2 = 4L_1 = 536{,}00\,\text{мГн}, \quad
        W_1 = \frac{L_1\eli_1^2}2 = \frac{134\,\text{мГн} \cdot \sqr{795\,\text{мА}}}2 \approx 0{,}042\,\text{Дж}, \quad
        \frac{W_2}{W_1} = \frac{\frac{L_2\eli_2^2}2}{\frac{L_1\eli_1^2}2} = 100.
    $
}
\solutionspace{90pt}

\tasknumber{3}%
\task{%
    Определите индуктивность катушки, если при пропускании тока силой $3\,\text{А}$
    в ней возникает магнитное поле индукцией $2\,\text{Тл}$.
    Катушка представляет собой цилиндр радиусом $3\,\text{см}$ и высотой $8\,\text{мм}$.
    Число витков в катушке 400.
}
\answer{%
    $
        \Phi = L\eli,
        \Phi = BSN,
        S=\pi R^2
        \implies L = \frac{\pi B R^2 N}{\eli} = \frac{\pi \cdot 2\,\text{Тл} \cdot \sqr{3\,\text{см}} \cdot 400}{3\,\text{А}}
        \approx 0{,}75\,\text{Гн}.
    $
}
\solutionspace{90pt}

\tasknumber{4}%
\task{%
    Тонкий прямой стержень длиной $25\,\text{см}$ вращается в горизонтальной плоскости
    вокруг одного из своих концов.
    Период обращения стержня $5\,\text{c}$.
    Однородное магнитное поле индукцией $150\,\text{мТл}$ направлено вертикально.
    Чему равна разность потенциалов на концах стержня? Ответ выразите в милливольтах.
}
\answer{%
    $
        \ele_i
            = \frac{\Delta \Phi}{\Delta t}
            = \frac{B\Delta S}{\Delta t}
            = \frac{B \frac 12 l^2 \Delta \alpha}{\Delta t}
            = \frac{B l^2 \omega}2, \quad
        \omega = \frac{2 \pi}T \implies
        \ele_i = \frac{\pi B l^2}T \approx 5{,}9\,\text{мВ}.
    $
}
\solutionspace{120pt}

\tasknumber{5}%
\task{%
    Проводник лежит на горизонтальных рельсах,
    замкнутых резистором сопротивлением $2\,\text{Ом}$ (см.
    рис.
    на доске).
    Расстояние между рельсами $70\,\text{см}$.
    Конструкция помещена в вертикальное однородное магнитное поле индукцией $200\,\text{мТл}$.
    Какую силу необходимо прикладывать к проводнику, чтобы двигать его вдоль рельс с постоянной скоростью $4\,\frac{\text{м}}{\text{c}}$?
    Трением пренебречь, сопротивления рельс и проводника малы по сравнению с сопротивлением резистора.
    Ответ выразите в миллиньютонах.
}
\answer{%
    $
        F
            = F_A
            = \eli B l
            = \frac{\ele}R \cdot B l
            = \frac{B v l}R \cdot B l
            = \frac{B^2 v l^2}R
            = \frac{\sqr{200\,\text{мТл}} \cdot 4\,\frac{\text{м}}{\text{c}} \cdot \sqr{70\,\text{см}}}{2\,\text{Ом}}
            \approx 39{,}20\,\text{мН}.
    $
}

\variantsplitter

\addpersonalvariant{Егор Свистушкин}

\tasknumber{1}%
\task{%
    Определите энергию магнитного поля в катушке индуктивностью $400\,\text{мГн}$, если протекающий через неё ток равен $4\,\text{А}$.
}
\answer{%
    $W = \frac{L\eli^2}2 = \frac{400\,\text{мГн} \cdot \sqr{4\,\text{А}}}2 \approx 3{,}20\,\text{Дж}.$
}
\solutionspace{60pt}

\tasknumber{2}%
\task{%
    В одной катушке индуктивностью $409\,\text{мГн}$ протекает электрический ток силой $634\,\text{мА}$.
    А в другой — с индуктивностью в три раза больше — ток в пять раз сильнее.
    Определите энергию магнитного поля первой катушки, индуктивность второй катушки
    и отношение энергий магнитного поля в двух этих катушках.
}
\answer{%
    $
        L_2 = 3L_1 = 1227{,}00\,\text{мГн}, \quad
        W_1 = \frac{L_1\eli_1^2}2 = \frac{409\,\text{мГн} \cdot \sqr{634\,\text{мА}}}2 \approx 0{,}082\,\text{Дж}, \quad
        \frac{W_2}{W_1} = \frac{\frac{L_2\eli_2^2}2}{\frac{L_1\eli_1^2}2} = 75.
    $
}
\solutionspace{90pt}

\tasknumber{3}%
\task{%
    Определите индуктивность катушки, если при пропускании тока силой $1{,}5\,\text{А}$
    в ней возникает магнитное поле индукцией $2\,\text{Тл}$.
    Катушка представляет собой цилиндр радиусом $5\,\text{см}$ и высотой $7\,\text{мм}$.
    Число витков в катушке 200.
}
\answer{%
    $
        \Phi = L\eli,
        \Phi = BSN,
        S=\pi R^2
        \implies L = \frac{\pi B R^2 N}{\eli} = \frac{\pi \cdot 2\,\text{Тл} \cdot \sqr{5\,\text{см}} \cdot 200}{1{,}5\,\text{А}}
        \approx 2{,}09\,\text{Гн}.
    $
}
\solutionspace{90pt}

\tasknumber{4}%
\task{%
    Тонкий прямой стержень длиной $20\,\text{см}$ вращается в горизонтальной плоскости
    вокруг одного из своих концов.
    Период обращения стержня $4\,\text{c}$.
    Однородное магнитное поле индукцией $400\,\text{мТл}$ направлено вертикально.
    Чему равна разность потенциалов на концах стержня? Ответ выразите в милливольтах.
}
\answer{%
    $
        \ele_i
            = \frac{\Delta \Phi}{\Delta t}
            = \frac{B\Delta S}{\Delta t}
            = \frac{B \frac 12 l^2 \Delta \alpha}{\Delta t}
            = \frac{B l^2 \omega}2, \quad
        \omega = \frac{2 \pi}T \implies
        \ele_i = \frac{\pi B l^2}T \approx 12{,}6\,\text{мВ}.
    $
}
\solutionspace{120pt}

\tasknumber{5}%
\task{%
    Проводник лежит на горизонтальных рельсах,
    замкнутых резистором сопротивлением $3\,\text{Ом}$ (см.
    рис.
    на доске).
    Расстояние между рельсами $60\,\text{см}$.
    Конструкция помещена в вертикальное однородное магнитное поле индукцией $150\,\text{мТл}$.
    Какую силу необходимо прикладывать к проводнику, чтобы двигать его вдоль рельс с постоянной скоростью $5\,\frac{\text{м}}{\text{c}}$?
    Трением пренебречь, сопротивления рельс и проводника малы по сравнению с сопротивлением резистора.
    Ответ выразите в миллиньютонах.
}
\answer{%
    $
        F
            = F_A
            = \eli B l
            = \frac{\ele}R \cdot B l
            = \frac{B v l}R \cdot B l
            = \frac{B^2 v l^2}R
            = \frac{\sqr{150\,\text{мТл}} \cdot 5\,\frac{\text{м}}{\text{c}} \cdot \sqr{60\,\text{см}}}{3\,\text{Ом}}
            \approx 13{,}50\,\text{мН}.
    $
}

\variantsplitter

\addpersonalvariant{Дмитрий Соколов}

\tasknumber{1}%
\task{%
    Определите энергию магнитного поля в катушке индуктивностью $400\,\text{мГн}$, если её собственный магнитный поток равен $3\,\text{Вб}$.
}
\answer{%
    $W = \frac{\Phi^2}{2L} = \frac{\sqr{3\,\text{Вб}}}{2 \cdot 400\,\text{мГн}} \approx 11{,}25\,\text{Дж}.$
}
\solutionspace{60pt}

\tasknumber{2}%
\task{%
    В одной катушке индуктивностью $277\,\text{мГн}$ протекает электрический ток силой $864\,\text{мА}$.
    А в другой — с индуктивностью в шесть раз больше — ток в шесть раз сильнее.
    Определите энергию магнитного поля первой катушки, индуктивность второй катушки
    и отношение энергий магнитного поля в двух этих катушках.
}
\answer{%
    $
        L_2 = 6L_1 = 1662{,}00\,\text{мГн}, \quad
        W_1 = \frac{L_1\eli_1^2}2 = \frac{277\,\text{мГн} \cdot \sqr{864\,\text{мА}}}2 \approx 0{,}103\,\text{Дж}, \quad
        \frac{W_2}{W_1} = \frac{\frac{L_2\eli_2^2}2}{\frac{L_1\eli_1^2}2} = 216.
    $
}
\solutionspace{90pt}

\tasknumber{3}%
\task{%
    Определите индуктивность катушки, если при пропускании тока силой $1{,}5\,\text{А}$
    в ней возникает магнитное поле индукцией $8\,\text{Тл}$.
    Катушка представляет собой цилиндр радиусом $5\,\text{см}$ и высотой $7\,\text{мм}$.
    Число витков в катушке 200.
}
\answer{%
    $
        \Phi = L\eli,
        \Phi = BSN,
        S=\pi R^2
        \implies L = \frac{\pi B R^2 N}{\eli} = \frac{\pi \cdot 8\,\text{Тл} \cdot \sqr{5\,\text{см}} \cdot 200}{1{,}5\,\text{А}}
        \approx 8{,}38\,\text{Гн}.
    $
}
\solutionspace{90pt}

\tasknumber{4}%
\task{%
    Тонкий прямой стержень длиной $50\,\text{см}$ вращается в горизонтальной плоскости
    вокруг одного из своих концов.
    Период обращения стержня $3\,\text{c}$.
    Однородное магнитное поле индукцией $300\,\text{мТл}$ направлено вертикально.
    Чему равна разность потенциалов на концах стержня? Ответ выразите в милливольтах.
}
\answer{%
    $
        \ele_i
            = \frac{\Delta \Phi}{\Delta t}
            = \frac{B\Delta S}{\Delta t}
            = \frac{B \frac 12 l^2 \Delta \alpha}{\Delta t}
            = \frac{B l^2 \omega}2, \quad
        \omega = \frac{2 \pi}T \implies
        \ele_i = \frac{\pi B l^2}T \approx 78{,}5\,\text{мВ}.
    $
}
\solutionspace{120pt}

\tasknumber{5}%
\task{%
    Проводник лежит на горизонтальных рельсах,
    замкнутых резистором сопротивлением $2\,\text{Ом}$ (см.
    рис.
    на доске).
    Расстояние между рельсами $60\,\text{см}$.
    Конструкция помещена в вертикальное однородное магнитное поле индукцией $200\,\text{мТл}$.
    Какую силу необходимо прикладывать к проводнику, чтобы двигать его вдоль рельс с постоянной скоростью $5\,\frac{\text{м}}{\text{c}}$?
    Трением пренебречь, сопротивления рельс и проводника малы по сравнению с сопротивлением резистора.
    Ответ выразите в миллиньютонах.
}
\answer{%
    $
        F
            = F_A
            = \eli B l
            = \frac{\ele}R \cdot B l
            = \frac{B v l}R \cdot B l
            = \frac{B^2 v l^2}R
            = \frac{\sqr{200\,\text{мТл}} \cdot 5\,\frac{\text{м}}{\text{c}} \cdot \sqr{60\,\text{см}}}{2\,\text{Ом}}
            \approx 36{,}00\,\text{мН}.
    $
}

\variantsplitter

\addpersonalvariant{Арсений Трофимов}

\tasknumber{1}%
\task{%
    Определите энергию магнитного поля в катушке индуктивностью $600\,\text{мГн}$, если протекающий через неё ток равен $7\,\text{А}$.
}
\answer{%
    $W = \frac{L\eli^2}2 = \frac{600\,\text{мГн} \cdot \sqr{7\,\text{А}}}2 \approx 14{,}70\,\text{Дж}.$
}
\solutionspace{60pt}

\tasknumber{2}%
\task{%
    В одной катушке индуктивностью $431\,\text{мГн}$ протекает электрический ток силой $565\,\text{мА}$.
    А в другой — с индуктивностью в три раза больше — ток в шесть раз сильнее.
    Определите энергию магнитного поля первой катушки, индуктивность второй катушки
    и отношение энергий магнитного поля в двух этих катушках.
}
\answer{%
    $
        L_2 = 3L_1 = 1293{,}00\,\text{мГн}, \quad
        W_1 = \frac{L_1\eli_1^2}2 = \frac{431\,\text{мГн} \cdot \sqr{565\,\text{мА}}}2 \approx 0{,}069\,\text{Дж}, \quad
        \frac{W_2}{W_1} = \frac{\frac{L_2\eli_2^2}2}{\frac{L_1\eli_1^2}2} = 108.
    $
}
\solutionspace{90pt}

\tasknumber{3}%
\task{%
    Определите индуктивность катушки, если при пропускании тока силой $3\,\text{А}$
    в ней возникает магнитное поле индукцией $2\,\text{Тл}$.
    Катушка представляет собой цилиндр радиусом $5\,\text{см}$ и высотой $6\,\text{мм}$.
    Число витков в катушке 500.
}
\answer{%
    $
        \Phi = L\eli,
        \Phi = BSN,
        S=\pi R^2
        \implies L = \frac{\pi B R^2 N}{\eli} = \frac{\pi \cdot 2\,\text{Тл} \cdot \sqr{5\,\text{см}} \cdot 500}{3\,\text{А}}
        \approx 2{,}62\,\text{Гн}.
    $
}
\solutionspace{90pt}

\tasknumber{4}%
\task{%
    Тонкий прямой стержень длиной $40\,\text{см}$ вращается в горизонтальной плоскости
    вокруг одного из своих концов.
    Период обращения стержня $3\,\text{c}$.
    Однородное магнитное поле индукцией $300\,\text{мТл}$ направлено вертикально.
    Чему равна разность потенциалов на концах стержня? Ответ выразите в милливольтах.
}
\answer{%
    $
        \ele_i
            = \frac{\Delta \Phi}{\Delta t}
            = \frac{B\Delta S}{\Delta t}
            = \frac{B \frac 12 l^2 \Delta \alpha}{\Delta t}
            = \frac{B l^2 \omega}2, \quad
        \omega = \frac{2 \pi}T \implies
        \ele_i = \frac{\pi B l^2}T \approx 50{,}3\,\text{мВ}.
    $
}
\solutionspace{120pt}

\tasknumber{5}%
\task{%
    Проводник лежит на горизонтальных рельсах,
    замкнутых резистором сопротивлением $4\,\text{Ом}$ (см.
    рис.
    на доске).
    Расстояние между рельсами $60\,\text{см}$.
    Конструкция помещена в вертикальное однородное магнитное поле индукцией $400\,\text{мТл}$.
    Какую силу необходимо прикладывать к проводнику, чтобы двигать его вдоль рельс с постоянной скоростью $5\,\frac{\text{м}}{\text{c}}$?
    Трением пренебречь, сопротивления рельс и проводника малы по сравнению с сопротивлением резистора.
    Ответ выразите в миллиньютонах.
}
\answer{%
    $
        F
            = F_A
            = \eli B l
            = \frac{\ele}R \cdot B l
            = \frac{B v l}R \cdot B l
            = \frac{B^2 v l^2}R
            = \frac{\sqr{400\,\text{мТл}} \cdot 5\,\frac{\text{м}}{\text{c}} \cdot \sqr{60\,\text{см}}}{4\,\text{Ом}}
            \approx 72{,}00\,\text{мН}.
    $
}
% autogenerated
