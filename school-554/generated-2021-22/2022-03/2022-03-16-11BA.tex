\setdate{16~марта~2022}
\setclass{11«БА»}

\addpersonalvariant{Михаил Бурмистров}

\tasknumber{1}%
\task{%
    При переходе электрона в атоме с одной стационарной орбиты на другую
    излучается фотон с энергией $4{,}04 \cdot 10^{-19}\,\text{Дж}$.
    Какова длина волны этой линии спектра?
    Постоянная Планка $h = 6{,}626 \cdot 10^{-34}\,\text{Дж}\cdot\text{с}$, скорость света $c = 3 \cdot 10^{8}\,\frac{\text{м}}{\text{с}}$.
}
\answer{%
    $
        E = h\nu = h \frac c\lambda
        \implies \lambda = \frac{hc}E
            = \frac{6{,}626 \cdot 10^{-34}\,\text{Дж}\cdot\text{с} \cdot {3 \cdot 10^{8}\,\frac{\text{м}}{\text{с}}}}{4{,}04 \cdot 10^{-19}\,\text{Дж}}
            = 492{,}03\,\text{нм}.
    $
}
\solutionspace{80pt}

\tasknumber{2}%
\task{%
    Излучение какой длины волны поглотил атом водорода, если полная энергия в атоме увеличилась на $2 \cdot 10^{-19}\,\text{Дж}$?
    Постоянная Планка $h = 6{,}626 \cdot 10^{-34}\,\text{Дж}\cdot\text{с}$, скорость света $c = 3 \cdot 10^{8}\,\frac{\text{м}}{\text{с}}$.
}
\answer{%
    $
        E = h\nu = h \frac c\lambda
        \implies \lambda = \frac{hc}E
            = \frac{6{,}626 \cdot 10^{-34}\,\text{Дж}\cdot\text{с} \cdot {3 \cdot 10^{8}\,\frac{\text{м}}{\text{с}}}}{2 \cdot 10^{-19}\,\text{Дж}}
            = 994\,\text{нм}.
    $
}
\solutionspace{80pt}

\tasknumber{3}%
\task{%
    Сделайте схематичный рисунок энергетических уровней атома водорода
    и отметьте на нём первый (основной) уровень и последующие.
    Сколько различных длин волн может испустить атом водорода,
    находящийся в 5-м возбуждённом состоянии (рассмотрите и сложные переходы)?
    Отметьте все соответствующие переходы на рисунке и укажите,
    при каком переходе (среди отмеченных) длина волны излучённого фотона максимальна.
}
\answer{%
    $N = 10, \text{ самая короткая линия }$
}
\solutionspace{150pt}

\tasknumber{4}%
\task{%
    Во сколько раз уменьшается радиус орбиты электрона в атоме водорода,
    если при переходе атома из одного стационарного состояния в другое
    кинетическая энергия электрона увеличивается в восемь раз?
}
\answer{%
    $m_e \frac{v^2}r = k\frac{e^2}{r^2} \implies mv^2 = k\frac{e^2} r \implies E_{\text{кин.}} = k\frac{e^2}{2r} \implies 8$
}
\solutionspace{80pt}

\tasknumber{5}%
\task{%
    Во сколько раз увеличилась кинетическая энергия электрона в атоме водорода при переходе
    из одного стационарного состояния в другое, если угловая скорость вращения по орбите увеличилась в десять раз?
    (Считая, что такие уровни существуют, что можно обсудить отдельно).
}
\answer{%
    $m_e \frac{v^2}r = k\frac{e^2}{r^2}, v = \omega r \implies m_e v^2 = k\frac{e^2}{r} = k\frac{e^2\omega}{ v } \implies v^3 =  k\frac{e^2}{ m_e } \omega \implies 4{,}64$
}
\solutionspace{80pt}

\tasknumber{6}%
\task{%
    Во сколько раз увеличивается угловая скорость вращения электрона в атоме водорода,
    если при переходе атома из одного стационарного состояния в другое радиус орбиты электрона уменьшается в семь раз?
    (Считая, что такие уровни существуют, что можно обсудить отдельно).
}
\answer{%
    $m_e \frac{v^2}r = k\frac{e^2}{r^2}, v = \omega r \implies m_e \omega^2 r = k\frac{e^2}{r^2} \implies \omega = \sqrt{ k\frac{e^2}{ m_e } } r^{-\frac 32} \implies 18{,}52$
}

\variantsplitter

\addpersonalvariant{Ирина Ан}

\tasknumber{1}%
\task{%
    При переходе электрона в атоме с одной стационарной орбиты на другую
    излучается фотон с энергией $4{,}04 \cdot 10^{-19}\,\text{Дж}$.
    Какова длина волны этой линии спектра?
    Постоянная Планка $h = 6{,}626 \cdot 10^{-34}\,\text{Дж}\cdot\text{с}$, скорость света $c = 3 \cdot 10^{8}\,\frac{\text{м}}{\text{с}}$.
}
\answer{%
    $
        E = h\nu = h \frac c\lambda
        \implies \lambda = \frac{hc}E
            = \frac{6{,}626 \cdot 10^{-34}\,\text{Дж}\cdot\text{с} \cdot {3 \cdot 10^{8}\,\frac{\text{м}}{\text{с}}}}{4{,}04 \cdot 10^{-19}\,\text{Дж}}
            = 492{,}03\,\text{нм}.
    $
}
\solutionspace{80pt}

\tasknumber{2}%
\task{%
    Излучение какой длины волны поглотил атом водорода, если полная энергия в атоме увеличилась на $4 \cdot 10^{-19}\,\text{Дж}$?
    Постоянная Планка $h = 6{,}626 \cdot 10^{-34}\,\text{Дж}\cdot\text{с}$, скорость света $c = 3 \cdot 10^{8}\,\frac{\text{м}}{\text{с}}$.
}
\answer{%
    $
        E = h\nu = h \frac c\lambda
        \implies \lambda = \frac{hc}E
            = \frac{6{,}626 \cdot 10^{-34}\,\text{Дж}\cdot\text{с} \cdot {3 \cdot 10^{8}\,\frac{\text{м}}{\text{с}}}}{4 \cdot 10^{-19}\,\text{Дж}}
            = 497\,\text{нм}.
    $
}
\solutionspace{80pt}

\tasknumber{3}%
\task{%
    Сделайте схематичный рисунок энергетических уровней атома водорода
    и отметьте на нём первый (основной) уровень и последующие.
    Сколько различных длин волн может испустить атом водорода,
    находящийся в 4-м возбуждённом состоянии (рассмотрите и сложные переходы)?
    Отметьте все соответствующие переходы на рисунке и укажите,
    при каком переходе (среди отмеченных) длина волны излучённого фотона максимальна.
}
\answer{%
    $N = 6, \text{ самая короткая линия }$
}
\solutionspace{150pt}

\tasknumber{4}%
\task{%
    Во сколько раз уменьшается радиус орбиты электрона в атоме водорода,
    если при переходе атома из одного стационарного состояния в другое
    кинетическая энергия электрона увеличивается в восемь раз?
}
\answer{%
    $m_e \frac{v^2}r = k\frac{e^2}{r^2} \implies mv^2 = k\frac{e^2} r \implies E_{\text{кин.}} = k\frac{e^2}{2r} \implies 8$
}
\solutionspace{80pt}

\tasknumber{5}%
\task{%
    Во сколько раз увеличилась кинетическая энергия электрона в атоме водорода при переходе
    из одного стационарного состояния в другое, если угловая скорость вращения по орбите увеличилась в восемь раз?
    (Считая, что такие уровни существуют, что можно обсудить отдельно).
}
\answer{%
    $m_e \frac{v^2}r = k\frac{e^2}{r^2}, v = \omega r \implies m_e v^2 = k\frac{e^2}{r} = k\frac{e^2\omega}{ v } \implies v^3 =  k\frac{e^2}{ m_e } \omega \implies 4{,}00$
}
\solutionspace{80pt}

\tasknumber{6}%
\task{%
    Во сколько раз увеличивается угловая скорость вращения электрона в атоме водорода,
    если при переходе атома из одного стационарного состояния в другое радиус орбиты электрона уменьшается в семь раз?
    (Считая, что такие уровни существуют, что можно обсудить отдельно).
}
\answer{%
    $m_e \frac{v^2}r = k\frac{e^2}{r^2}, v = \omega r \implies m_e \omega^2 r = k\frac{e^2}{r^2} \implies \omega = \sqrt{ k\frac{e^2}{ m_e } } r^{-\frac 32} \implies 18{,}52$
}

\variantsplitter

\addpersonalvariant{Софья Андрианова}

\tasknumber{1}%
\task{%
    При переходе электрона в атоме с одной стационарной орбиты на другую
    излучается фотон с энергией $4{,}04 \cdot 10^{-19}\,\text{Дж}$.
    Какова длина волны этой линии спектра?
    Постоянная Планка $h = 6{,}626 \cdot 10^{-34}\,\text{Дж}\cdot\text{с}$, скорость света $c = 3 \cdot 10^{8}\,\frac{\text{м}}{\text{с}}$.
}
\answer{%
    $
        E = h\nu = h \frac c\lambda
        \implies \lambda = \frac{hc}E
            = \frac{6{,}626 \cdot 10^{-34}\,\text{Дж}\cdot\text{с} \cdot {3 \cdot 10^{8}\,\frac{\text{м}}{\text{с}}}}{4{,}04 \cdot 10^{-19}\,\text{Дж}}
            = 492{,}03\,\text{нм}.
    $
}
\solutionspace{80pt}

\tasknumber{2}%
\task{%
    Излучение какой длины волны поглотил атом водорода, если полная энергия в атоме увеличилась на $4 \cdot 10^{-19}\,\text{Дж}$?
    Постоянная Планка $h = 6{,}626 \cdot 10^{-34}\,\text{Дж}\cdot\text{с}$, скорость света $c = 3 \cdot 10^{8}\,\frac{\text{м}}{\text{с}}$.
}
\answer{%
    $
        E = h\nu = h \frac c\lambda
        \implies \lambda = \frac{hc}E
            = \frac{6{,}626 \cdot 10^{-34}\,\text{Дж}\cdot\text{с} \cdot {3 \cdot 10^{8}\,\frac{\text{м}}{\text{с}}}}{4 \cdot 10^{-19}\,\text{Дж}}
            = 497\,\text{нм}.
    $
}
\solutionspace{80pt}

\tasknumber{3}%
\task{%
    Сделайте схематичный рисунок энергетических уровней атома водорода
    и отметьте на нём первый (основной) уровень и последующие.
    Сколько различных длин волн может испустить атом водорода,
    находящийся в 4-м возбуждённом состоянии (рассмотрите и сложные переходы)?
    Отметьте все соответствующие переходы на рисунке и укажите,
    при каком переходе (среди отмеченных) энергия излучённого фотона минимальна.
}
\answer{%
    $N = 6, \text{ самая короткая линия }$
}
\solutionspace{150pt}

\tasknumber{4}%
\task{%
    Во сколько раз уменьшается радиус орбиты электрона в атоме водорода,
    если при переходе атома из одного стационарного состояния в другое
    кинетическая энергия электрона увеличивается в три раза?
}
\answer{%
    $m_e \frac{v^2}r = k\frac{e^2}{r^2} \implies mv^2 = k\frac{e^2} r \implies E_{\text{кин.}} = k\frac{e^2}{2r} \implies 3$
}
\solutionspace{80pt}

\tasknumber{5}%
\task{%
    Во сколько раз увеличилась кинетическая энергия электрона в атоме водорода при переходе
    из одного стационарного состояния в другое, если угловая скорость вращения по орбите увеличилась в шесть раз?
    (Считая, что такие уровни существуют, что можно обсудить отдельно).
}
\answer{%
    $m_e \frac{v^2}r = k\frac{e^2}{r^2}, v = \omega r \implies m_e v^2 = k\frac{e^2}{r} = k\frac{e^2\omega}{ v } \implies v^3 =  k\frac{e^2}{ m_e } \omega \implies 3{,}30$
}
\solutionspace{80pt}

\tasknumber{6}%
\task{%
    Во сколько раз увеличивается угловая скорость вращения электрона в атоме водорода,
    если при переходе атома из одного стационарного состояния в другое радиус орбиты электрона уменьшается в два раза?
    (Считая, что такие уровни существуют, что можно обсудить отдельно).
}
\answer{%
    $m_e \frac{v^2}r = k\frac{e^2}{r^2}, v = \omega r \implies m_e \omega^2 r = k\frac{e^2}{r^2} \implies \omega = \sqrt{ k\frac{e^2}{ m_e } } r^{-\frac 32} \implies 2{,}83$
}

\variantsplitter

\addpersonalvariant{Владимир Артемчук}

\tasknumber{1}%
\task{%
    При переходе электрона в атоме с одной стационарной орбиты на другую
    излучается фотон с энергией $1{,}01 \cdot 10^{-19}\,\text{Дж}$.
    Какова длина волны этой линии спектра?
    Постоянная Планка $h = 6{,}626 \cdot 10^{-34}\,\text{Дж}\cdot\text{с}$, скорость света $c = 3 \cdot 10^{8}\,\frac{\text{м}}{\text{с}}$.
}
\answer{%
    $
        E = h\nu = h \frac c\lambda
        \implies \lambda = \frac{hc}E
            = \frac{6{,}626 \cdot 10^{-34}\,\text{Дж}\cdot\text{с} \cdot {3 \cdot 10^{8}\,\frac{\text{м}}{\text{с}}}}{1{,}01 \cdot 10^{-19}\,\text{Дж}}
            = 1968{,}1\,\text{нм}.
    $
}
\solutionspace{80pt}

\tasknumber{2}%
\task{%
    Излучение какой длины волны поглотил атом водорода, если полная энергия в атоме увеличилась на $4 \cdot 10^{-19}\,\text{Дж}$?
    Постоянная Планка $h = 6{,}626 \cdot 10^{-34}\,\text{Дж}\cdot\text{с}$, скорость света $c = 3 \cdot 10^{8}\,\frac{\text{м}}{\text{с}}$.
}
\answer{%
    $
        E = h\nu = h \frac c\lambda
        \implies \lambda = \frac{hc}E
            = \frac{6{,}626 \cdot 10^{-34}\,\text{Дж}\cdot\text{с} \cdot {3 \cdot 10^{8}\,\frac{\text{м}}{\text{с}}}}{4 \cdot 10^{-19}\,\text{Дж}}
            = 497\,\text{нм}.
    $
}
\solutionspace{80pt}

\tasknumber{3}%
\task{%
    Сделайте схематичный рисунок энергетических уровней атома водорода
    и отметьте на нём первый (основной) уровень и последующие.
    Сколько различных длин волн может испустить атом водорода,
    находящийся в 4-м возбуждённом состоянии (рассмотрите и сложные переходы)?
    Отметьте все соответствующие переходы на рисунке и укажите,
    при каком переходе (среди отмеченных) длина волны излучённого фотона минимальна.
}
\answer{%
    $N = 6, \text{ самая длинная линия }$
}
\solutionspace{150pt}

\tasknumber{4}%
\task{%
    Во сколько раз уменьшается радиус орбиты электрона в атоме водорода,
    если при переходе атома из одного стационарного состояния в другое
    кинетическая энергия электрона увеличивается в двенадцать раз?
}
\answer{%
    $m_e \frac{v^2}r = k\frac{e^2}{r^2} \implies mv^2 = k\frac{e^2} r \implies E_{\text{кин.}} = k\frac{e^2}{2r} \implies 12$
}
\solutionspace{80pt}

\tasknumber{5}%
\task{%
    Во сколько раз увеличилась кинетическая энергия электрона в атоме водорода при переходе
    из одного стационарного состояния в другое, если угловая скорость вращения по орбите увеличилась в пять раз?
    (Считая, что такие уровни существуют, что можно обсудить отдельно).
}
\answer{%
    $m_e \frac{v^2}r = k\frac{e^2}{r^2}, v = \omega r \implies m_e v^2 = k\frac{e^2}{r} = k\frac{e^2\omega}{ v } \implies v^3 =  k\frac{e^2}{ m_e } \omega \implies 2{,}92$
}
\solutionspace{80pt}

\tasknumber{6}%
\task{%
    Во сколько раз увеличивается угловая скорость вращения электрона в атоме водорода,
    если при переходе атома из одного стационарного состояния в другое радиус орбиты электрона уменьшается в восемь раз?
    (Считая, что такие уровни существуют, что можно обсудить отдельно).
}
\answer{%
    $m_e \frac{v^2}r = k\frac{e^2}{r^2}, v = \omega r \implies m_e \omega^2 r = k\frac{e^2}{r^2} \implies \omega = \sqrt{ k\frac{e^2}{ m_e } } r^{-\frac 32} \implies 22{,}63$
}

\variantsplitter

\addpersonalvariant{Софья Белянкина}

\tasknumber{1}%
\task{%
    При переходе электрона в атоме с одной стационарной орбиты на другую
    излучается фотон с энергией $4{,}04 \cdot 10^{-19}\,\text{Дж}$.
    Какова длина волны этой линии спектра?
    Постоянная Планка $h = 6{,}626 \cdot 10^{-34}\,\text{Дж}\cdot\text{с}$, скорость света $c = 3 \cdot 10^{8}\,\frac{\text{м}}{\text{с}}$.
}
\answer{%
    $
        E = h\nu = h \frac c\lambda
        \implies \lambda = \frac{hc}E
            = \frac{6{,}626 \cdot 10^{-34}\,\text{Дж}\cdot\text{с} \cdot {3 \cdot 10^{8}\,\frac{\text{м}}{\text{с}}}}{4{,}04 \cdot 10^{-19}\,\text{Дж}}
            = 492{,}03\,\text{нм}.
    $
}
\solutionspace{80pt}

\tasknumber{2}%
\task{%
    Излучение какой длины волны поглотил атом водорода, если полная энергия в атоме увеличилась на $4 \cdot 10^{-19}\,\text{Дж}$?
    Постоянная Планка $h = 6{,}626 \cdot 10^{-34}\,\text{Дж}\cdot\text{с}$, скорость света $c = 3 \cdot 10^{8}\,\frac{\text{м}}{\text{с}}$.
}
\answer{%
    $
        E = h\nu = h \frac c\lambda
        \implies \lambda = \frac{hc}E
            = \frac{6{,}626 \cdot 10^{-34}\,\text{Дж}\cdot\text{с} \cdot {3 \cdot 10^{8}\,\frac{\text{м}}{\text{с}}}}{4 \cdot 10^{-19}\,\text{Дж}}
            = 497\,\text{нм}.
    $
}
\solutionspace{80pt}

\tasknumber{3}%
\task{%
    Сделайте схематичный рисунок энергетических уровней атома водорода
    и отметьте на нём первый (основной) уровень и последующие.
    Сколько различных длин волн может испустить атом водорода,
    находящийся в 4-м возбуждённом состоянии (рассмотрите и сложные переходы)?
    Отметьте все соответствующие переходы на рисунке и укажите,
    при каком переходе (среди отмеченных) энергия излучённого фотона минимальна.
}
\answer{%
    $N = 6, \text{ самая короткая линия }$
}
\solutionspace{150pt}

\tasknumber{4}%
\task{%
    Во сколько раз уменьшается радиус орбиты электрона в атоме водорода,
    если при переходе атома из одного стационарного состояния в другое
    кинетическая энергия электрона увеличивается в двенадцать раз?
}
\answer{%
    $m_e \frac{v^2}r = k\frac{e^2}{r^2} \implies mv^2 = k\frac{e^2} r \implies E_{\text{кин.}} = k\frac{e^2}{2r} \implies 12$
}
\solutionspace{80pt}

\tasknumber{5}%
\task{%
    Во сколько раз увеличилась кинетическая энергия электрона в атоме водорода при переходе
    из одного стационарного состояния в другое, если угловая скорость вращения по орбите увеличилась в двенадцать раз?
    (Считая, что такие уровни существуют, что можно обсудить отдельно).
}
\answer{%
    $m_e \frac{v^2}r = k\frac{e^2}{r^2}, v = \omega r \implies m_e v^2 = k\frac{e^2}{r} = k\frac{e^2\omega}{ v } \implies v^3 =  k\frac{e^2}{ m_e } \omega \implies 5{,}24$
}
\solutionspace{80pt}

\tasknumber{6}%
\task{%
    Во сколько раз увеличивается угловая скорость вращения электрона в атоме водорода,
    если при переходе атома из одного стационарного состояния в другое радиус орбиты электрона уменьшается в семь раз?
    (Считая, что такие уровни существуют, что можно обсудить отдельно).
}
\answer{%
    $m_e \frac{v^2}r = k\frac{e^2}{r^2}, v = \omega r \implies m_e \omega^2 r = k\frac{e^2}{r^2} \implies \omega = \sqrt{ k\frac{e^2}{ m_e } } r^{-\frac 32} \implies 18{,}52$
}

\variantsplitter

\addpersonalvariant{Варвара Егиазарян}

\tasknumber{1}%
\task{%
    При переходе электрона в атоме с одной стационарной орбиты на другую
    излучается фотон с энергией $4{,}04 \cdot 10^{-19}\,\text{Дж}$.
    Какова длина волны этой линии спектра?
    Постоянная Планка $h = 6{,}626 \cdot 10^{-34}\,\text{Дж}\cdot\text{с}$, скорость света $c = 3 \cdot 10^{8}\,\frac{\text{м}}{\text{с}}$.
}
\answer{%
    $
        E = h\nu = h \frac c\lambda
        \implies \lambda = \frac{hc}E
            = \frac{6{,}626 \cdot 10^{-34}\,\text{Дж}\cdot\text{с} \cdot {3 \cdot 10^{8}\,\frac{\text{м}}{\text{с}}}}{4{,}04 \cdot 10^{-19}\,\text{Дж}}
            = 492{,}03\,\text{нм}.
    $
}
\solutionspace{80pt}

\tasknumber{2}%
\task{%
    Излучение какой длины волны поглотил атом водорода, если полная энергия в атоме увеличилась на $4 \cdot 10^{-19}\,\text{Дж}$?
    Постоянная Планка $h = 6{,}626 \cdot 10^{-34}\,\text{Дж}\cdot\text{с}$, скорость света $c = 3 \cdot 10^{8}\,\frac{\text{м}}{\text{с}}$.
}
\answer{%
    $
        E = h\nu = h \frac c\lambda
        \implies \lambda = \frac{hc}E
            = \frac{6{,}626 \cdot 10^{-34}\,\text{Дж}\cdot\text{с} \cdot {3 \cdot 10^{8}\,\frac{\text{м}}{\text{с}}}}{4 \cdot 10^{-19}\,\text{Дж}}
            = 497\,\text{нм}.
    $
}
\solutionspace{80pt}

\tasknumber{3}%
\task{%
    Сделайте схематичный рисунок энергетических уровней атома водорода
    и отметьте на нём первый (основной) уровень и последующие.
    Сколько различных длин волн может испустить атом водорода,
    находящийся в 5-м возбуждённом состоянии (рассмотрите и сложные переходы)?
    Отметьте все соответствующие переходы на рисунке и укажите,
    при каком переходе (среди отмеченных) энергия излучённого фотона минимальна.
}
\answer{%
    $N = 10, \text{ самая короткая линия }$
}
\solutionspace{150pt}

\tasknumber{4}%
\task{%
    Во сколько раз уменьшается радиус орбиты электрона в атоме водорода,
    если при переходе атома из одного стационарного состояния в другое
    кинетическая энергия электрона увеличивается в двенадцать раз?
}
\answer{%
    $m_e \frac{v^2}r = k\frac{e^2}{r^2} \implies mv^2 = k\frac{e^2} r \implies E_{\text{кин.}} = k\frac{e^2}{2r} \implies 12$
}
\solutionspace{80pt}

\tasknumber{5}%
\task{%
    Во сколько раз увеличилась кинетическая энергия электрона в атоме водорода при переходе
    из одного стационарного состояния в другое, если угловая скорость вращения по орбите увеличилась в восемь раз?
    (Считая, что такие уровни существуют, что можно обсудить отдельно).
}
\answer{%
    $m_e \frac{v^2}r = k\frac{e^2}{r^2}, v = \omega r \implies m_e v^2 = k\frac{e^2}{r} = k\frac{e^2\omega}{ v } \implies v^3 =  k\frac{e^2}{ m_e } \omega \implies 4{,}00$
}
\solutionspace{80pt}

\tasknumber{6}%
\task{%
    Во сколько раз увеличивается угловая скорость вращения электрона в атоме водорода,
    если при переходе атома из одного стационарного состояния в другое радиус орбиты электрона уменьшается в семь раз?
    (Считая, что такие уровни существуют, что можно обсудить отдельно).
}
\answer{%
    $m_e \frac{v^2}r = k\frac{e^2}{r^2}, v = \omega r \implies m_e \omega^2 r = k\frac{e^2}{r^2} \implies \omega = \sqrt{ k\frac{e^2}{ m_e } } r^{-\frac 32} \implies 18{,}52$
}

\variantsplitter

\addpersonalvariant{Владислав Емелин}

\tasknumber{1}%
\task{%
    При переходе электрона в атоме с одной стационарной орбиты на другую
    излучается фотон с энергией $5{,}05 \cdot 10^{-19}\,\text{Дж}$.
    Какова длина волны этой линии спектра?
    Постоянная Планка $h = 6{,}626 \cdot 10^{-34}\,\text{Дж}\cdot\text{с}$, скорость света $c = 3 \cdot 10^{8}\,\frac{\text{м}}{\text{с}}$.
}
\answer{%
    $
        E = h\nu = h \frac c\lambda
        \implies \lambda = \frac{hc}E
            = \frac{6{,}626 \cdot 10^{-34}\,\text{Дж}\cdot\text{с} \cdot {3 \cdot 10^{8}\,\frac{\text{м}}{\text{с}}}}{5{,}05 \cdot 10^{-19}\,\text{Дж}}
            = 393{,}62\,\text{нм}.
    $
}
\solutionspace{80pt}

\tasknumber{2}%
\task{%
    Излучение какой длины волны поглотил атом водорода, если полная энергия в атоме увеличилась на $3 \cdot 10^{-19}\,\text{Дж}$?
    Постоянная Планка $h = 6{,}626 \cdot 10^{-34}\,\text{Дж}\cdot\text{с}$, скорость света $c = 3 \cdot 10^{8}\,\frac{\text{м}}{\text{с}}$.
}
\answer{%
    $
        E = h\nu = h \frac c\lambda
        \implies \lambda = \frac{hc}E
            = \frac{6{,}626 \cdot 10^{-34}\,\text{Дж}\cdot\text{с} \cdot {3 \cdot 10^{8}\,\frac{\text{м}}{\text{с}}}}{3 \cdot 10^{-19}\,\text{Дж}}
            = 663\,\text{нм}.
    $
}
\solutionspace{80pt}

\tasknumber{3}%
\task{%
    Сделайте схематичный рисунок энергетических уровней атома водорода
    и отметьте на нём первый (основной) уровень и последующие.
    Сколько различных длин волн может испустить атом водорода,
    находящийся в 5-м возбуждённом состоянии (рассмотрите и сложные переходы)?
    Отметьте все соответствующие переходы на рисунке и укажите,
    при каком переходе (среди отмеченных) энергия излучённого фотона минимальна.
}
\answer{%
    $N = 10, \text{ самая короткая линия }$
}
\solutionspace{150pt}

\tasknumber{4}%
\task{%
    Во сколько раз уменьшается радиус орбиты электрона в атоме водорода,
    если при переходе атома из одного стационарного состояния в другое
    кинетическая энергия электрона увеличивается в двенадцать раз?
}
\answer{%
    $m_e \frac{v^2}r = k\frac{e^2}{r^2} \implies mv^2 = k\frac{e^2} r \implies E_{\text{кин.}} = k\frac{e^2}{2r} \implies 12$
}
\solutionspace{80pt}

\tasknumber{5}%
\task{%
    Во сколько раз увеличилась кинетическая энергия электрона в атоме водорода при переходе
    из одного стационарного состояния в другое, если угловая скорость вращения по орбите увеличилась в шесть раз?
    (Считая, что такие уровни существуют, что можно обсудить отдельно).
}
\answer{%
    $m_e \frac{v^2}r = k\frac{e^2}{r^2}, v = \omega r \implies m_e v^2 = k\frac{e^2}{r} = k\frac{e^2\omega}{ v } \implies v^3 =  k\frac{e^2}{ m_e } \omega \implies 3{,}30$
}
\solutionspace{80pt}

\tasknumber{6}%
\task{%
    Во сколько раз увеличивается угловая скорость вращения электрона в атоме водорода,
    если при переходе атома из одного стационарного состояния в другое радиус орбиты электрона уменьшается в два раза?
    (Считая, что такие уровни существуют, что можно обсудить отдельно).
}
\answer{%
    $m_e \frac{v^2}r = k\frac{e^2}{r^2}, v = \omega r \implies m_e \omega^2 r = k\frac{e^2}{r^2} \implies \omega = \sqrt{ k\frac{e^2}{ m_e } } r^{-\frac 32} \implies 2{,}83$
}

\variantsplitter

\addpersonalvariant{Артём Жичин}

\tasknumber{1}%
\task{%
    При переходе электрона в атоме с одной стационарной орбиты на другую
    излучается фотон с энергией $5{,}05 \cdot 10^{-19}\,\text{Дж}$.
    Какова длина волны этой линии спектра?
    Постоянная Планка $h = 6{,}626 \cdot 10^{-34}\,\text{Дж}\cdot\text{с}$, скорость света $c = 3 \cdot 10^{8}\,\frac{\text{м}}{\text{с}}$.
}
\answer{%
    $
        E = h\nu = h \frac c\lambda
        \implies \lambda = \frac{hc}E
            = \frac{6{,}626 \cdot 10^{-34}\,\text{Дж}\cdot\text{с} \cdot {3 \cdot 10^{8}\,\frac{\text{м}}{\text{с}}}}{5{,}05 \cdot 10^{-19}\,\text{Дж}}
            = 393{,}62\,\text{нм}.
    $
}
\solutionspace{80pt}

\tasknumber{2}%
\task{%
    Излучение какой длины волны поглотил атом водорода, если полная энергия в атоме увеличилась на $6 \cdot 10^{-19}\,\text{Дж}$?
    Постоянная Планка $h = 6{,}626 \cdot 10^{-34}\,\text{Дж}\cdot\text{с}$, скорость света $c = 3 \cdot 10^{8}\,\frac{\text{м}}{\text{с}}$.
}
\answer{%
    $
        E = h\nu = h \frac c\lambda
        \implies \lambda = \frac{hc}E
            = \frac{6{,}626 \cdot 10^{-34}\,\text{Дж}\cdot\text{с} \cdot {3 \cdot 10^{8}\,\frac{\text{м}}{\text{с}}}}{6 \cdot 10^{-19}\,\text{Дж}}
            = 331\,\text{нм}.
    $
}
\solutionspace{80pt}

\tasknumber{3}%
\task{%
    Сделайте схематичный рисунок энергетических уровней атома водорода
    и отметьте на нём первый (основной) уровень и последующие.
    Сколько различных длин волн может испустить атом водорода,
    находящийся в 5-м возбуждённом состоянии (рассмотрите и сложные переходы)?
    Отметьте все соответствующие переходы на рисунке и укажите,
    при каком переходе (среди отмеченных) длина волны излучённого фотона минимальна.
}
\answer{%
    $N = 10, \text{ самая длинная линия }$
}
\solutionspace{150pt}

\tasknumber{4}%
\task{%
    Во сколько раз уменьшается радиус орбиты электрона в атоме водорода,
    если при переходе атома из одного стационарного состояния в другое
    кинетическая энергия электрона увеличивается в три раза?
}
\answer{%
    $m_e \frac{v^2}r = k\frac{e^2}{r^2} \implies mv^2 = k\frac{e^2} r \implies E_{\text{кин.}} = k\frac{e^2}{2r} \implies 3$
}
\solutionspace{80pt}

\tasknumber{5}%
\task{%
    Во сколько раз увеличилась кинетическая энергия электрона в атоме водорода при переходе
    из одного стационарного состояния в другое, если угловая скорость вращения по орбите увеличилась в двенадцать раз?
    (Считая, что такие уровни существуют, что можно обсудить отдельно).
}
\answer{%
    $m_e \frac{v^2}r = k\frac{e^2}{r^2}, v = \omega r \implies m_e v^2 = k\frac{e^2}{r} = k\frac{e^2\omega}{ v } \implies v^3 =  k\frac{e^2}{ m_e } \omega \implies 5{,}24$
}
\solutionspace{80pt}

\tasknumber{6}%
\task{%
    Во сколько раз увеличивается угловая скорость вращения электрона в атоме водорода,
    если при переходе атома из одного стационарного состояния в другое радиус орбиты электрона уменьшается в пять раз?
    (Считая, что такие уровни существуют, что можно обсудить отдельно).
}
\answer{%
    $m_e \frac{v^2}r = k\frac{e^2}{r^2}, v = \omega r \implies m_e \omega^2 r = k\frac{e^2}{r^2} \implies \omega = \sqrt{ k\frac{e^2}{ m_e } } r^{-\frac 32} \implies 11{,}18$
}

\variantsplitter

\addpersonalvariant{Дарья Кошман}

\tasknumber{1}%
\task{%
    При переходе электрона в атоме с одной стационарной орбиты на другую
    излучается фотон с энергией $5{,}05 \cdot 10^{-19}\,\text{Дж}$.
    Какова длина волны этой линии спектра?
    Постоянная Планка $h = 6{,}626 \cdot 10^{-34}\,\text{Дж}\cdot\text{с}$, скорость света $c = 3 \cdot 10^{8}\,\frac{\text{м}}{\text{с}}$.
}
\answer{%
    $
        E = h\nu = h \frac c\lambda
        \implies \lambda = \frac{hc}E
            = \frac{6{,}626 \cdot 10^{-34}\,\text{Дж}\cdot\text{с} \cdot {3 \cdot 10^{8}\,\frac{\text{м}}{\text{с}}}}{5{,}05 \cdot 10^{-19}\,\text{Дж}}
            = 393{,}62\,\text{нм}.
    $
}
\solutionspace{80pt}

\tasknumber{2}%
\task{%
    Излучение какой длины волны поглотил атом водорода, если полная энергия в атоме увеличилась на $6 \cdot 10^{-19}\,\text{Дж}$?
    Постоянная Планка $h = 6{,}626 \cdot 10^{-34}\,\text{Дж}\cdot\text{с}$, скорость света $c = 3 \cdot 10^{8}\,\frac{\text{м}}{\text{с}}$.
}
\answer{%
    $
        E = h\nu = h \frac c\lambda
        \implies \lambda = \frac{hc}E
            = \frac{6{,}626 \cdot 10^{-34}\,\text{Дж}\cdot\text{с} \cdot {3 \cdot 10^{8}\,\frac{\text{м}}{\text{с}}}}{6 \cdot 10^{-19}\,\text{Дж}}
            = 331\,\text{нм}.
    $
}
\solutionspace{80pt}

\tasknumber{3}%
\task{%
    Сделайте схематичный рисунок энергетических уровней атома водорода
    и отметьте на нём первый (основной) уровень и последующие.
    Сколько различных длин волн может испустить атом водорода,
    находящийся в 4-м возбуждённом состоянии (рассмотрите и сложные переходы)?
    Отметьте все соответствующие переходы на рисунке и укажите,
    при каком переходе (среди отмеченных) частота излучённого фотона максимальна.
}
\answer{%
    $N = 6, \text{ самая длинная линия }$
}
\solutionspace{150pt}

\tasknumber{4}%
\task{%
    Во сколько раз уменьшается радиус орбиты электрона в атоме водорода,
    если при переходе атома из одного стационарного состояния в другое
    кинетическая энергия электрона увеличивается в четыре раза?
}
\answer{%
    $m_e \frac{v^2}r = k\frac{e^2}{r^2} \implies mv^2 = k\frac{e^2} r \implies E_{\text{кин.}} = k\frac{e^2}{2r} \implies 4$
}
\solutionspace{80pt}

\tasknumber{5}%
\task{%
    Во сколько раз увеличилась кинетическая энергия электрона в атоме водорода при переходе
    из одного стационарного состояния в другое, если угловая скорость вращения по орбите увеличилась в шесть раз?
    (Считая, что такие уровни существуют, что можно обсудить отдельно).
}
\answer{%
    $m_e \frac{v^2}r = k\frac{e^2}{r^2}, v = \omega r \implies m_e v^2 = k\frac{e^2}{r} = k\frac{e^2\omega}{ v } \implies v^3 =  k\frac{e^2}{ m_e } \omega \implies 3{,}30$
}
\solutionspace{80pt}

\tasknumber{6}%
\task{%
    Во сколько раз увеличивается угловая скорость вращения электрона в атоме водорода,
    если при переходе атома из одного стационарного состояния в другое радиус орбиты электрона уменьшается в шесть раз?
    (Считая, что такие уровни существуют, что можно обсудить отдельно).
}
\answer{%
    $m_e \frac{v^2}r = k\frac{e^2}{r^2}, v = \omega r \implies m_e \omega^2 r = k\frac{e^2}{r^2} \implies \omega = \sqrt{ k\frac{e^2}{ m_e } } r^{-\frac 32} \implies 14{,}70$
}

\variantsplitter

\addpersonalvariant{Анна Кузьмичёва}

\tasknumber{1}%
\task{%
    При переходе электрона в атоме с одной стационарной орбиты на другую
    излучается фотон с энергией $5{,}05 \cdot 10^{-19}\,\text{Дж}$.
    Какова длина волны этой линии спектра?
    Постоянная Планка $h = 6{,}626 \cdot 10^{-34}\,\text{Дж}\cdot\text{с}$, скорость света $c = 3 \cdot 10^{8}\,\frac{\text{м}}{\text{с}}$.
}
\answer{%
    $
        E = h\nu = h \frac c\lambda
        \implies \lambda = \frac{hc}E
            = \frac{6{,}626 \cdot 10^{-34}\,\text{Дж}\cdot\text{с} \cdot {3 \cdot 10^{8}\,\frac{\text{м}}{\text{с}}}}{5{,}05 \cdot 10^{-19}\,\text{Дж}}
            = 393{,}62\,\text{нм}.
    $
}
\solutionspace{80pt}

\tasknumber{2}%
\task{%
    Излучение какой длины волны поглотил атом водорода, если полная энергия в атоме увеличилась на $6 \cdot 10^{-19}\,\text{Дж}$?
    Постоянная Планка $h = 6{,}626 \cdot 10^{-34}\,\text{Дж}\cdot\text{с}$, скорость света $c = 3 \cdot 10^{8}\,\frac{\text{м}}{\text{с}}$.
}
\answer{%
    $
        E = h\nu = h \frac c\lambda
        \implies \lambda = \frac{hc}E
            = \frac{6{,}626 \cdot 10^{-34}\,\text{Дж}\cdot\text{с} \cdot {3 \cdot 10^{8}\,\frac{\text{м}}{\text{с}}}}{6 \cdot 10^{-19}\,\text{Дж}}
            = 331\,\text{нм}.
    $
}
\solutionspace{80pt}

\tasknumber{3}%
\task{%
    Сделайте схематичный рисунок энергетических уровней атома водорода
    и отметьте на нём первый (основной) уровень и последующие.
    Сколько различных длин волн может испустить атом водорода,
    находящийся в 4-м возбуждённом состоянии (рассмотрите и сложные переходы)?
    Отметьте все соответствующие переходы на рисунке и укажите,
    при каком переходе (среди отмеченных) частота излучённого фотона максимальна.
}
\answer{%
    $N = 6, \text{ самая длинная линия }$
}
\solutionspace{150pt}

\tasknumber{4}%
\task{%
    Во сколько раз уменьшается радиус орбиты электрона в атоме водорода,
    если при переходе атома из одного стационарного состояния в другое
    кинетическая энергия электрона увеличивается в восемь раз?
}
\answer{%
    $m_e \frac{v^2}r = k\frac{e^2}{r^2} \implies mv^2 = k\frac{e^2} r \implies E_{\text{кин.}} = k\frac{e^2}{2r} \implies 8$
}
\solutionspace{80pt}

\tasknumber{5}%
\task{%
    Во сколько раз увеличилась кинетическая энергия электрона в атоме водорода при переходе
    из одного стационарного состояния в другое, если угловая скорость вращения по орбите увеличилась в шесть раз?
    (Считая, что такие уровни существуют, что можно обсудить отдельно).
}
\answer{%
    $m_e \frac{v^2}r = k\frac{e^2}{r^2}, v = \omega r \implies m_e v^2 = k\frac{e^2}{r} = k\frac{e^2\omega}{ v } \implies v^3 =  k\frac{e^2}{ m_e } \omega \implies 3{,}30$
}
\solutionspace{80pt}

\tasknumber{6}%
\task{%
    Во сколько раз увеличивается угловая скорость вращения электрона в атоме водорода,
    если при переходе атома из одного стационарного состояния в другое радиус орбиты электрона уменьшается в семь раз?
    (Считая, что такие уровни существуют, что можно обсудить отдельно).
}
\answer{%
    $m_e \frac{v^2}r = k\frac{e^2}{r^2}, v = \omega r \implies m_e \omega^2 r = k\frac{e^2}{r^2} \implies \omega = \sqrt{ k\frac{e^2}{ m_e } } r^{-\frac 32} \implies 18{,}52$
}

\variantsplitter

\addpersonalvariant{Алёна Куприянова}

\tasknumber{1}%
\task{%
    При переходе электрона в атоме с одной стационарной орбиты на другую
    излучается фотон с энергией $5{,}05 \cdot 10^{-19}\,\text{Дж}$.
    Какова длина волны этой линии спектра?
    Постоянная Планка $h = 6{,}626 \cdot 10^{-34}\,\text{Дж}\cdot\text{с}$, скорость света $c = 3 \cdot 10^{8}\,\frac{\text{м}}{\text{с}}$.
}
\answer{%
    $
        E = h\nu = h \frac c\lambda
        \implies \lambda = \frac{hc}E
            = \frac{6{,}626 \cdot 10^{-34}\,\text{Дж}\cdot\text{с} \cdot {3 \cdot 10^{8}\,\frac{\text{м}}{\text{с}}}}{5{,}05 \cdot 10^{-19}\,\text{Дж}}
            = 393{,}62\,\text{нм}.
    $
}
\solutionspace{80pt}

\tasknumber{2}%
\task{%
    Излучение какой длины волны поглотил атом водорода, если полная энергия в атоме увеличилась на $6 \cdot 10^{-19}\,\text{Дж}$?
    Постоянная Планка $h = 6{,}626 \cdot 10^{-34}\,\text{Дж}\cdot\text{с}$, скорость света $c = 3 \cdot 10^{8}\,\frac{\text{м}}{\text{с}}$.
}
\answer{%
    $
        E = h\nu = h \frac c\lambda
        \implies \lambda = \frac{hc}E
            = \frac{6{,}626 \cdot 10^{-34}\,\text{Дж}\cdot\text{с} \cdot {3 \cdot 10^{8}\,\frac{\text{м}}{\text{с}}}}{6 \cdot 10^{-19}\,\text{Дж}}
            = 331\,\text{нм}.
    $
}
\solutionspace{80pt}

\tasknumber{3}%
\task{%
    Сделайте схематичный рисунок энергетических уровней атома водорода
    и отметьте на нём первый (основной) уровень и последующие.
    Сколько различных длин волн может испустить атом водорода,
    находящийся в 5-м возбуждённом состоянии (рассмотрите и сложные переходы)?
    Отметьте все соответствующие переходы на рисунке и укажите,
    при каком переходе (среди отмеченных) частота излучённого фотона минимальна.
}
\answer{%
    $N = 10, \text{ самая короткая линия }$
}
\solutionspace{150pt}

\tasknumber{4}%
\task{%
    Во сколько раз уменьшается радиус орбиты электрона в атоме водорода,
    если при переходе атома из одного стационарного состояния в другое
    кинетическая энергия электрона увеличивается в шестнадцать раз?
}
\answer{%
    $m_e \frac{v^2}r = k\frac{e^2}{r^2} \implies mv^2 = k\frac{e^2} r \implies E_{\text{кин.}} = k\frac{e^2}{2r} \implies 16$
}
\solutionspace{80pt}

\tasknumber{5}%
\task{%
    Во сколько раз увеличилась кинетическая энергия электрона в атоме водорода при переходе
    из одного стационарного состояния в другое, если угловая скорость вращения по орбите увеличилась в шесть раз?
    (Считая, что такие уровни существуют, что можно обсудить отдельно).
}
\answer{%
    $m_e \frac{v^2}r = k\frac{e^2}{r^2}, v = \omega r \implies m_e v^2 = k\frac{e^2}{r} = k\frac{e^2\omega}{ v } \implies v^3 =  k\frac{e^2}{ m_e } \omega \implies 3{,}30$
}
\solutionspace{80pt}

\tasknumber{6}%
\task{%
    Во сколько раз увеличивается угловая скорость вращения электрона в атоме водорода,
    если при переходе атома из одного стационарного состояния в другое радиус орбиты электрона уменьшается в шесть раз?
    (Считая, что такие уровни существуют, что можно обсудить отдельно).
}
\answer{%
    $m_e \frac{v^2}r = k\frac{e^2}{r^2}, v = \omega r \implies m_e \omega^2 r = k\frac{e^2}{r^2} \implies \omega = \sqrt{ k\frac{e^2}{ m_e } } r^{-\frac 32} \implies 14{,}70$
}

\variantsplitter

\addpersonalvariant{Ярослав Лавровский}

\tasknumber{1}%
\task{%
    При переходе электрона в атоме с одной стационарной орбиты на другую
    излучается фотон с энергией $7{,}07 \cdot 10^{-19}\,\text{Дж}$.
    Какова длина волны этой линии спектра?
    Постоянная Планка $h = 6{,}626 \cdot 10^{-34}\,\text{Дж}\cdot\text{с}$, скорость света $c = 3 \cdot 10^{8}\,\frac{\text{м}}{\text{с}}$.
}
\answer{%
    $
        E = h\nu = h \frac c\lambda
        \implies \lambda = \frac{hc}E
            = \frac{6{,}626 \cdot 10^{-34}\,\text{Дж}\cdot\text{с} \cdot {3 \cdot 10^{8}\,\frac{\text{м}}{\text{с}}}}{7{,}07 \cdot 10^{-19}\,\text{Дж}}
            = 281{,}16\,\text{нм}.
    $
}
\solutionspace{80pt}

\tasknumber{2}%
\task{%
    Излучение какой длины волны поглотил атом водорода, если полная энергия в атоме увеличилась на $3 \cdot 10^{-19}\,\text{Дж}$?
    Постоянная Планка $h = 6{,}626 \cdot 10^{-34}\,\text{Дж}\cdot\text{с}$, скорость света $c = 3 \cdot 10^{8}\,\frac{\text{м}}{\text{с}}$.
}
\answer{%
    $
        E = h\nu = h \frac c\lambda
        \implies \lambda = \frac{hc}E
            = \frac{6{,}626 \cdot 10^{-34}\,\text{Дж}\cdot\text{с} \cdot {3 \cdot 10^{8}\,\frac{\text{м}}{\text{с}}}}{3 \cdot 10^{-19}\,\text{Дж}}
            = 663\,\text{нм}.
    $
}
\solutionspace{80pt}

\tasknumber{3}%
\task{%
    Сделайте схематичный рисунок энергетических уровней атома водорода
    и отметьте на нём первый (основной) уровень и последующие.
    Сколько различных длин волн может испустить атом водорода,
    находящийся в 3-м возбуждённом состоянии (рассмотрите и сложные переходы)?
    Отметьте все соответствующие переходы на рисунке и укажите,
    при каком переходе (среди отмеченных) энергия излучённого фотона минимальна.
}
\answer{%
    $N = 3, \text{ самая короткая линия }$
}
\solutionspace{150pt}

\tasknumber{4}%
\task{%
    Во сколько раз уменьшается радиус орбиты электрона в атоме водорода,
    если при переходе атома из одного стационарного состояния в другое
    кинетическая энергия электрона увеличивается в двенадцать раз?
}
\answer{%
    $m_e \frac{v^2}r = k\frac{e^2}{r^2} \implies mv^2 = k\frac{e^2} r \implies E_{\text{кин.}} = k\frac{e^2}{2r} \implies 12$
}
\solutionspace{80pt}

\tasknumber{5}%
\task{%
    Во сколько раз увеличилась кинетическая энергия электрона в атоме водорода при переходе
    из одного стационарного состояния в другое, если угловая скорость вращения по орбите увеличилась в двенадцать раз?
    (Считая, что такие уровни существуют, что можно обсудить отдельно).
}
\answer{%
    $m_e \frac{v^2}r = k\frac{e^2}{r^2}, v = \omega r \implies m_e v^2 = k\frac{e^2}{r} = k\frac{e^2\omega}{ v } \implies v^3 =  k\frac{e^2}{ m_e } \omega \implies 5{,}24$
}
\solutionspace{80pt}

\tasknumber{6}%
\task{%
    Во сколько раз увеличивается угловая скорость вращения электрона в атоме водорода,
    если при переходе атома из одного стационарного состояния в другое радиус орбиты электрона уменьшается в четыре раза?
    (Считая, что такие уровни существуют, что можно обсудить отдельно).
}
\answer{%
    $m_e \frac{v^2}r = k\frac{e^2}{r^2}, v = \omega r \implies m_e \omega^2 r = k\frac{e^2}{r^2} \implies \omega = \sqrt{ k\frac{e^2}{ m_e } } r^{-\frac 32} \implies 8{,}00$
}

\variantsplitter

\addpersonalvariant{Анастасия Ламанова}

\tasknumber{1}%
\task{%
    При переходе электрона в атоме с одной стационарной орбиты на другую
    излучается фотон с энергией $1{,}01 \cdot 10^{-19}\,\text{Дж}$.
    Какова длина волны этой линии спектра?
    Постоянная Планка $h = 6{,}626 \cdot 10^{-34}\,\text{Дж}\cdot\text{с}$, скорость света $c = 3 \cdot 10^{8}\,\frac{\text{м}}{\text{с}}$.
}
\answer{%
    $
        E = h\nu = h \frac c\lambda
        \implies \lambda = \frac{hc}E
            = \frac{6{,}626 \cdot 10^{-34}\,\text{Дж}\cdot\text{с} \cdot {3 \cdot 10^{8}\,\frac{\text{м}}{\text{с}}}}{1{,}01 \cdot 10^{-19}\,\text{Дж}}
            = 1968{,}1\,\text{нм}.
    $
}
\solutionspace{80pt}

\tasknumber{2}%
\task{%
    Излучение какой длины волны поглотил атом водорода, если полная энергия в атоме увеличилась на $2 \cdot 10^{-19}\,\text{Дж}$?
    Постоянная Планка $h = 6{,}626 \cdot 10^{-34}\,\text{Дж}\cdot\text{с}$, скорость света $c = 3 \cdot 10^{8}\,\frac{\text{м}}{\text{с}}$.
}
\answer{%
    $
        E = h\nu = h \frac c\lambda
        \implies \lambda = \frac{hc}E
            = \frac{6{,}626 \cdot 10^{-34}\,\text{Дж}\cdot\text{с} \cdot {3 \cdot 10^{8}\,\frac{\text{м}}{\text{с}}}}{2 \cdot 10^{-19}\,\text{Дж}}
            = 994\,\text{нм}.
    $
}
\solutionspace{80pt}

\tasknumber{3}%
\task{%
    Сделайте схематичный рисунок энергетических уровней атома водорода
    и отметьте на нём первый (основной) уровень и последующие.
    Сколько различных длин волн может испустить атом водорода,
    находящийся в 3-м возбуждённом состоянии (рассмотрите и сложные переходы)?
    Отметьте все соответствующие переходы на рисунке и укажите,
    при каком переходе (среди отмеченных) энергия излучённого фотона максимальна.
}
\answer{%
    $N = 3, \text{ самая длинная линия }$
}
\solutionspace{150pt}

\tasknumber{4}%
\task{%
    Во сколько раз уменьшается радиус орбиты электрона в атоме водорода,
    если при переходе атома из одного стационарного состояния в другое
    кинетическая энергия электрона увеличивается в четыре раза?
}
\answer{%
    $m_e \frac{v^2}r = k\frac{e^2}{r^2} \implies mv^2 = k\frac{e^2} r \implies E_{\text{кин.}} = k\frac{e^2}{2r} \implies 4$
}
\solutionspace{80pt}

\tasknumber{5}%
\task{%
    Во сколько раз увеличилась кинетическая энергия электрона в атоме водорода при переходе
    из одного стационарного состояния в другое, если угловая скорость вращения по орбите увеличилась в восемь раз?
    (Считая, что такие уровни существуют, что можно обсудить отдельно).
}
\answer{%
    $m_e \frac{v^2}r = k\frac{e^2}{r^2}, v = \omega r \implies m_e v^2 = k\frac{e^2}{r} = k\frac{e^2\omega}{ v } \implies v^3 =  k\frac{e^2}{ m_e } \omega \implies 4{,}00$
}
\solutionspace{80pt}

\tasknumber{6}%
\task{%
    Во сколько раз увеличивается угловая скорость вращения электрона в атоме водорода,
    если при переходе атома из одного стационарного состояния в другое радиус орбиты электрона уменьшается в семь раз?
    (Считая, что такие уровни существуют, что можно обсудить отдельно).
}
\answer{%
    $m_e \frac{v^2}r = k\frac{e^2}{r^2}, v = \omega r \implies m_e \omega^2 r = k\frac{e^2}{r^2} \implies \omega = \sqrt{ k\frac{e^2}{ m_e } } r^{-\frac 32} \implies 18{,}52$
}

\variantsplitter

\addpersonalvariant{Виктория Легонькова}

\tasknumber{1}%
\task{%
    При переходе электрона в атоме с одной стационарной орбиты на другую
    излучается фотон с энергией $0{,}55 \cdot 10^{-19}\,\text{Дж}$.
    Какова длина волны этой линии спектра?
    Постоянная Планка $h = 6{,}626 \cdot 10^{-34}\,\text{Дж}\cdot\text{с}$, скорость света $c = 3 \cdot 10^{8}\,\frac{\text{м}}{\text{с}}$.
}
\answer{%
    $
        E = h\nu = h \frac c\lambda
        \implies \lambda = \frac{hc}E
            = \frac{6{,}626 \cdot 10^{-34}\,\text{Дж}\cdot\text{с} \cdot {3 \cdot 10^{8}\,\frac{\text{м}}{\text{с}}}}{0{,}55 \cdot 10^{-19}\,\text{Дж}}
            = 3614\,\text{нм}.
    $
}
\solutionspace{80pt}

\tasknumber{2}%
\task{%
    Излучение какой длины волны поглотил атом водорода, если полная энергия в атоме увеличилась на $6 \cdot 10^{-19}\,\text{Дж}$?
    Постоянная Планка $h = 6{,}626 \cdot 10^{-34}\,\text{Дж}\cdot\text{с}$, скорость света $c = 3 \cdot 10^{8}\,\frac{\text{м}}{\text{с}}$.
}
\answer{%
    $
        E = h\nu = h \frac c\lambda
        \implies \lambda = \frac{hc}E
            = \frac{6{,}626 \cdot 10^{-34}\,\text{Дж}\cdot\text{с} \cdot {3 \cdot 10^{8}\,\frac{\text{м}}{\text{с}}}}{6 \cdot 10^{-19}\,\text{Дж}}
            = 331\,\text{нм}.
    $
}
\solutionspace{80pt}

\tasknumber{3}%
\task{%
    Сделайте схематичный рисунок энергетических уровней атома водорода
    и отметьте на нём первый (основной) уровень и последующие.
    Сколько различных длин волн может испустить атом водорода,
    находящийся в 4-м возбуждённом состоянии (рассмотрите и сложные переходы)?
    Отметьте все соответствующие переходы на рисунке и укажите,
    при каком переходе (среди отмеченных) длина волны излучённого фотона минимальна.
}
\answer{%
    $N = 6, \text{ самая длинная линия }$
}
\solutionspace{150pt}

\tasknumber{4}%
\task{%
    Во сколько раз уменьшается радиус орбиты электрона в атоме водорода,
    если при переходе атома из одного стационарного состояния в другое
    кинетическая энергия электрона увеличивается в четыре раза?
}
\answer{%
    $m_e \frac{v^2}r = k\frac{e^2}{r^2} \implies mv^2 = k\frac{e^2} r \implies E_{\text{кин.}} = k\frac{e^2}{2r} \implies 4$
}
\solutionspace{80pt}

\tasknumber{5}%
\task{%
    Во сколько раз увеличилась кинетическая энергия электрона в атоме водорода при переходе
    из одного стационарного состояния в другое, если угловая скорость вращения по орбите увеличилась в восемь раз?
    (Считая, что такие уровни существуют, что можно обсудить отдельно).
}
\answer{%
    $m_e \frac{v^2}r = k\frac{e^2}{r^2}, v = \omega r \implies m_e v^2 = k\frac{e^2}{r} = k\frac{e^2\omega}{ v } \implies v^3 =  k\frac{e^2}{ m_e } \omega \implies 4{,}00$
}
\solutionspace{80pt}

\tasknumber{6}%
\task{%
    Во сколько раз увеличивается угловая скорость вращения электрона в атоме водорода,
    если при переходе атома из одного стационарного состояния в другое радиус орбиты электрона уменьшается в семь раз?
    (Считая, что такие уровни существуют, что можно обсудить отдельно).
}
\answer{%
    $m_e \frac{v^2}r = k\frac{e^2}{r^2}, v = \omega r \implies m_e \omega^2 r = k\frac{e^2}{r^2} \implies \omega = \sqrt{ k\frac{e^2}{ m_e } } r^{-\frac 32} \implies 18{,}52$
}

\variantsplitter

\addpersonalvariant{Семён Мартынов}

\tasknumber{1}%
\task{%
    При переходе электрона в атоме с одной стационарной орбиты на другую
    излучается фотон с энергией $5{,}05 \cdot 10^{-19}\,\text{Дж}$.
    Какова длина волны этой линии спектра?
    Постоянная Планка $h = 6{,}626 \cdot 10^{-34}\,\text{Дж}\cdot\text{с}$, скорость света $c = 3 \cdot 10^{8}\,\frac{\text{м}}{\text{с}}$.
}
\answer{%
    $
        E = h\nu = h \frac c\lambda
        \implies \lambda = \frac{hc}E
            = \frac{6{,}626 \cdot 10^{-34}\,\text{Дж}\cdot\text{с} \cdot {3 \cdot 10^{8}\,\frac{\text{м}}{\text{с}}}}{5{,}05 \cdot 10^{-19}\,\text{Дж}}
            = 393{,}62\,\text{нм}.
    $
}
\solutionspace{80pt}

\tasknumber{2}%
\task{%
    Излучение какой длины волны поглотил атом водорода, если полная энергия в атоме увеличилась на $3 \cdot 10^{-19}\,\text{Дж}$?
    Постоянная Планка $h = 6{,}626 \cdot 10^{-34}\,\text{Дж}\cdot\text{с}$, скорость света $c = 3 \cdot 10^{8}\,\frac{\text{м}}{\text{с}}$.
}
\answer{%
    $
        E = h\nu = h \frac c\lambda
        \implies \lambda = \frac{hc}E
            = \frac{6{,}626 \cdot 10^{-34}\,\text{Дж}\cdot\text{с} \cdot {3 \cdot 10^{8}\,\frac{\text{м}}{\text{с}}}}{3 \cdot 10^{-19}\,\text{Дж}}
            = 663\,\text{нм}.
    $
}
\solutionspace{80pt}

\tasknumber{3}%
\task{%
    Сделайте схематичный рисунок энергетических уровней атома водорода
    и отметьте на нём первый (основной) уровень и последующие.
    Сколько различных длин волн может испустить атом водорода,
    находящийся в 5-м возбуждённом состоянии (рассмотрите и сложные переходы)?
    Отметьте все соответствующие переходы на рисунке и укажите,
    при каком переходе (среди отмеченных) энергия излучённого фотона минимальна.
}
\answer{%
    $N = 10, \text{ самая короткая линия }$
}
\solutionspace{150pt}

\tasknumber{4}%
\task{%
    Во сколько раз уменьшается радиус орбиты электрона в атоме водорода,
    если при переходе атома из одного стационарного состояния в другое
    кинетическая энергия электрона увеличивается в двенадцать раз?
}
\answer{%
    $m_e \frac{v^2}r = k\frac{e^2}{r^2} \implies mv^2 = k\frac{e^2} r \implies E_{\text{кин.}} = k\frac{e^2}{2r} \implies 12$
}
\solutionspace{80pt}

\tasknumber{5}%
\task{%
    Во сколько раз увеличилась кинетическая энергия электрона в атоме водорода при переходе
    из одного стационарного состояния в другое, если угловая скорость вращения по орбите увеличилась в десять раз?
    (Считая, что такие уровни существуют, что можно обсудить отдельно).
}
\answer{%
    $m_e \frac{v^2}r = k\frac{e^2}{r^2}, v = \omega r \implies m_e v^2 = k\frac{e^2}{r} = k\frac{e^2\omega}{ v } \implies v^3 =  k\frac{e^2}{ m_e } \omega \implies 4{,}64$
}
\solutionspace{80pt}

\tasknumber{6}%
\task{%
    Во сколько раз увеличивается угловая скорость вращения электрона в атоме водорода,
    если при переходе атома из одного стационарного состояния в другое радиус орбиты электрона уменьшается в два раза?
    (Считая, что такие уровни существуют, что можно обсудить отдельно).
}
\answer{%
    $m_e \frac{v^2}r = k\frac{e^2}{r^2}, v = \omega r \implies m_e \omega^2 r = k\frac{e^2}{r^2} \implies \omega = \sqrt{ k\frac{e^2}{ m_e } } r^{-\frac 32} \implies 2{,}83$
}

\variantsplitter

\addpersonalvariant{Варвара Минаева}

\tasknumber{1}%
\task{%
    При переходе электрона в атоме с одной стационарной орбиты на другую
    излучается фотон с энергией $2{,}02 \cdot 10^{-19}\,\text{Дж}$.
    Какова длина волны этой линии спектра?
    Постоянная Планка $h = 6{,}626 \cdot 10^{-34}\,\text{Дж}\cdot\text{с}$, скорость света $c = 3 \cdot 10^{8}\,\frac{\text{м}}{\text{с}}$.
}
\answer{%
    $
        E = h\nu = h \frac c\lambda
        \implies \lambda = \frac{hc}E
            = \frac{6{,}626 \cdot 10^{-34}\,\text{Дж}\cdot\text{с} \cdot {3 \cdot 10^{8}\,\frac{\text{м}}{\text{с}}}}{2{,}02 \cdot 10^{-19}\,\text{Дж}}
            = 984{,}06\,\text{нм}.
    $
}
\solutionspace{80pt}

\tasknumber{2}%
\task{%
    Излучение какой длины волны поглотил атом водорода, если полная энергия в атоме увеличилась на $2 \cdot 10^{-19}\,\text{Дж}$?
    Постоянная Планка $h = 6{,}626 \cdot 10^{-34}\,\text{Дж}\cdot\text{с}$, скорость света $c = 3 \cdot 10^{8}\,\frac{\text{м}}{\text{с}}$.
}
\answer{%
    $
        E = h\nu = h \frac c\lambda
        \implies \lambda = \frac{hc}E
            = \frac{6{,}626 \cdot 10^{-34}\,\text{Дж}\cdot\text{с} \cdot {3 \cdot 10^{8}\,\frac{\text{м}}{\text{с}}}}{2 \cdot 10^{-19}\,\text{Дж}}
            = 994\,\text{нм}.
    $
}
\solutionspace{80pt}

\tasknumber{3}%
\task{%
    Сделайте схематичный рисунок энергетических уровней атома водорода
    и отметьте на нём первый (основной) уровень и последующие.
    Сколько различных длин волн может испустить атом водорода,
    находящийся в 5-м возбуждённом состоянии (рассмотрите и сложные переходы)?
    Отметьте все соответствующие переходы на рисунке и укажите,
    при каком переходе (среди отмеченных) длина волны излучённого фотона минимальна.
}
\answer{%
    $N = 10, \text{ самая длинная линия }$
}
\solutionspace{150pt}

\tasknumber{4}%
\task{%
    Во сколько раз уменьшается радиус орбиты электрона в атоме водорода,
    если при переходе атома из одного стационарного состояния в другое
    кинетическая энергия электрона увеличивается в шестнадцать раз?
}
\answer{%
    $m_e \frac{v^2}r = k\frac{e^2}{r^2} \implies mv^2 = k\frac{e^2} r \implies E_{\text{кин.}} = k\frac{e^2}{2r} \implies 16$
}
\solutionspace{80pt}

\tasknumber{5}%
\task{%
    Во сколько раз увеличилась кинетическая энергия электрона в атоме водорода при переходе
    из одного стационарного состояния в другое, если угловая скорость вращения по орбите увеличилась в двенадцать раз?
    (Считая, что такие уровни существуют, что можно обсудить отдельно).
}
\answer{%
    $m_e \frac{v^2}r = k\frac{e^2}{r^2}, v = \omega r \implies m_e v^2 = k\frac{e^2}{r} = k\frac{e^2\omega}{ v } \implies v^3 =  k\frac{e^2}{ m_e } \omega \implies 5{,}24$
}
\solutionspace{80pt}

\tasknumber{6}%
\task{%
    Во сколько раз увеличивается угловая скорость вращения электрона в атоме водорода,
    если при переходе атома из одного стационарного состояния в другое радиус орбиты электрона уменьшается в три раза?
    (Считая, что такие уровни существуют, что можно обсудить отдельно).
}
\answer{%
    $m_e \frac{v^2}r = k\frac{e^2}{r^2}, v = \omega r \implies m_e \omega^2 r = k\frac{e^2}{r^2} \implies \omega = \sqrt{ k\frac{e^2}{ m_e } } r^{-\frac 32} \implies 5{,}20$
}

\variantsplitter

\addpersonalvariant{Леонид Никитин}

\tasknumber{1}%
\task{%
    При переходе электрона в атоме с одной стационарной орбиты на другую
    излучается фотон с энергией $4{,}04 \cdot 10^{-19}\,\text{Дж}$.
    Какова длина волны этой линии спектра?
    Постоянная Планка $h = 6{,}626 \cdot 10^{-34}\,\text{Дж}\cdot\text{с}$, скорость света $c = 3 \cdot 10^{8}\,\frac{\text{м}}{\text{с}}$.
}
\answer{%
    $
        E = h\nu = h \frac c\lambda
        \implies \lambda = \frac{hc}E
            = \frac{6{,}626 \cdot 10^{-34}\,\text{Дж}\cdot\text{с} \cdot {3 \cdot 10^{8}\,\frac{\text{м}}{\text{с}}}}{4{,}04 \cdot 10^{-19}\,\text{Дж}}
            = 492{,}03\,\text{нм}.
    $
}
\solutionspace{80pt}

\tasknumber{2}%
\task{%
    Излучение какой длины волны поглотил атом водорода, если полная энергия в атоме увеличилась на $2 \cdot 10^{-19}\,\text{Дж}$?
    Постоянная Планка $h = 6{,}626 \cdot 10^{-34}\,\text{Дж}\cdot\text{с}$, скорость света $c = 3 \cdot 10^{8}\,\frac{\text{м}}{\text{с}}$.
}
\answer{%
    $
        E = h\nu = h \frac c\lambda
        \implies \lambda = \frac{hc}E
            = \frac{6{,}626 \cdot 10^{-34}\,\text{Дж}\cdot\text{с} \cdot {3 \cdot 10^{8}\,\frac{\text{м}}{\text{с}}}}{2 \cdot 10^{-19}\,\text{Дж}}
            = 994\,\text{нм}.
    $
}
\solutionspace{80pt}

\tasknumber{3}%
\task{%
    Сделайте схематичный рисунок энергетических уровней атома водорода
    и отметьте на нём первый (основной) уровень и последующие.
    Сколько различных длин волн может испустить атом водорода,
    находящийся в 4-м возбуждённом состоянии (рассмотрите и сложные переходы)?
    Отметьте все соответствующие переходы на рисунке и укажите,
    при каком переходе (среди отмеченных) длина волны излучённого фотона минимальна.
}
\answer{%
    $N = 6, \text{ самая длинная линия }$
}
\solutionspace{150pt}

\tasknumber{4}%
\task{%
    Во сколько раз уменьшается радиус орбиты электрона в атоме водорода,
    если при переходе атома из одного стационарного состояния в другое
    кинетическая энергия электрона увеличивается в три раза?
}
\answer{%
    $m_e \frac{v^2}r = k\frac{e^2}{r^2} \implies mv^2 = k\frac{e^2} r \implies E_{\text{кин.}} = k\frac{e^2}{2r} \implies 3$
}
\solutionspace{80pt}

\tasknumber{5}%
\task{%
    Во сколько раз увеличилась кинетическая энергия электрона в атоме водорода при переходе
    из одного стационарного состояния в другое, если угловая скорость вращения по орбите увеличилась в пять раз?
    (Считая, что такие уровни существуют, что можно обсудить отдельно).
}
\answer{%
    $m_e \frac{v^2}r = k\frac{e^2}{r^2}, v = \omega r \implies m_e v^2 = k\frac{e^2}{r} = k\frac{e^2\omega}{ v } \implies v^3 =  k\frac{e^2}{ m_e } \omega \implies 2{,}92$
}
\solutionspace{80pt}

\tasknumber{6}%
\task{%
    Во сколько раз увеличивается угловая скорость вращения электрона в атоме водорода,
    если при переходе атома из одного стационарного состояния в другое радиус орбиты электрона уменьшается в восемь раз?
    (Считая, что такие уровни существуют, что можно обсудить отдельно).
}
\answer{%
    $m_e \frac{v^2}r = k\frac{e^2}{r^2}, v = \omega r \implies m_e \omega^2 r = k\frac{e^2}{r^2} \implies \omega = \sqrt{ k\frac{e^2}{ m_e } } r^{-\frac 32} \implies 22{,}63$
}

\variantsplitter

\addpersonalvariant{Тимофей Полетаев}

\tasknumber{1}%
\task{%
    При переходе электрона в атоме с одной стационарной орбиты на другую
    излучается фотон с энергией $2{,}02 \cdot 10^{-19}\,\text{Дж}$.
    Какова длина волны этой линии спектра?
    Постоянная Планка $h = 6{,}626 \cdot 10^{-34}\,\text{Дж}\cdot\text{с}$, скорость света $c = 3 \cdot 10^{8}\,\frac{\text{м}}{\text{с}}$.
}
\answer{%
    $
        E = h\nu = h \frac c\lambda
        \implies \lambda = \frac{hc}E
            = \frac{6{,}626 \cdot 10^{-34}\,\text{Дж}\cdot\text{с} \cdot {3 \cdot 10^{8}\,\frac{\text{м}}{\text{с}}}}{2{,}02 \cdot 10^{-19}\,\text{Дж}}
            = 984{,}06\,\text{нм}.
    $
}
\solutionspace{80pt}

\tasknumber{2}%
\task{%
    Излучение какой длины волны поглотил атом водорода, если полная энергия в атоме увеличилась на $2 \cdot 10^{-19}\,\text{Дж}$?
    Постоянная Планка $h = 6{,}626 \cdot 10^{-34}\,\text{Дж}\cdot\text{с}$, скорость света $c = 3 \cdot 10^{8}\,\frac{\text{м}}{\text{с}}$.
}
\answer{%
    $
        E = h\nu = h \frac c\lambda
        \implies \lambda = \frac{hc}E
            = \frac{6{,}626 \cdot 10^{-34}\,\text{Дж}\cdot\text{с} \cdot {3 \cdot 10^{8}\,\frac{\text{м}}{\text{с}}}}{2 \cdot 10^{-19}\,\text{Дж}}
            = 994\,\text{нм}.
    $
}
\solutionspace{80pt}

\tasknumber{3}%
\task{%
    Сделайте схематичный рисунок энергетических уровней атома водорода
    и отметьте на нём первый (основной) уровень и последующие.
    Сколько различных длин волн может испустить атом водорода,
    находящийся в 5-м возбуждённом состоянии (рассмотрите и сложные переходы)?
    Отметьте все соответствующие переходы на рисунке и укажите,
    при каком переходе (среди отмеченных) длина волны излучённого фотона минимальна.
}
\answer{%
    $N = 10, \text{ самая длинная линия }$
}
\solutionspace{150pt}

\tasknumber{4}%
\task{%
    Во сколько раз уменьшается радиус орбиты электрона в атоме водорода,
    если при переходе атома из одного стационарного состояния в другое
    кинетическая энергия электрона увеличивается в четыре раза?
}
\answer{%
    $m_e \frac{v^2}r = k\frac{e^2}{r^2} \implies mv^2 = k\frac{e^2} r \implies E_{\text{кин.}} = k\frac{e^2}{2r} \implies 4$
}
\solutionspace{80pt}

\tasknumber{5}%
\task{%
    Во сколько раз увеличилась кинетическая энергия электрона в атоме водорода при переходе
    из одного стационарного состояния в другое, если угловая скорость вращения по орбите увеличилась в восемь раз?
    (Считая, что такие уровни существуют, что можно обсудить отдельно).
}
\answer{%
    $m_e \frac{v^2}r = k\frac{e^2}{r^2}, v = \omega r \implies m_e v^2 = k\frac{e^2}{r} = k\frac{e^2\omega}{ v } \implies v^3 =  k\frac{e^2}{ m_e } \omega \implies 4{,}00$
}
\solutionspace{80pt}

\tasknumber{6}%
\task{%
    Во сколько раз увеличивается угловая скорость вращения электрона в атоме водорода,
    если при переходе атома из одного стационарного состояния в другое радиус орбиты электрона уменьшается в восемь раз?
    (Считая, что такие уровни существуют, что можно обсудить отдельно).
}
\answer{%
    $m_e \frac{v^2}r = k\frac{e^2}{r^2}, v = \omega r \implies m_e \omega^2 r = k\frac{e^2}{r^2} \implies \omega = \sqrt{ k\frac{e^2}{ m_e } } r^{-\frac 32} \implies 22{,}63$
}

\variantsplitter

\addpersonalvariant{Андрей Рожков}

\tasknumber{1}%
\task{%
    При переходе электрона в атоме с одной стационарной орбиты на другую
    излучается фотон с энергией $5{,}05 \cdot 10^{-19}\,\text{Дж}$.
    Какова длина волны этой линии спектра?
    Постоянная Планка $h = 6{,}626 \cdot 10^{-34}\,\text{Дж}\cdot\text{с}$, скорость света $c = 3 \cdot 10^{8}\,\frac{\text{м}}{\text{с}}$.
}
\answer{%
    $
        E = h\nu = h \frac c\lambda
        \implies \lambda = \frac{hc}E
            = \frac{6{,}626 \cdot 10^{-34}\,\text{Дж}\cdot\text{с} \cdot {3 \cdot 10^{8}\,\frac{\text{м}}{\text{с}}}}{5{,}05 \cdot 10^{-19}\,\text{Дж}}
            = 393{,}62\,\text{нм}.
    $
}
\solutionspace{80pt}

\tasknumber{2}%
\task{%
    Излучение какой длины волны поглотил атом водорода, если полная энергия в атоме увеличилась на $6 \cdot 10^{-19}\,\text{Дж}$?
    Постоянная Планка $h = 6{,}626 \cdot 10^{-34}\,\text{Дж}\cdot\text{с}$, скорость света $c = 3 \cdot 10^{8}\,\frac{\text{м}}{\text{с}}$.
}
\answer{%
    $
        E = h\nu = h \frac c\lambda
        \implies \lambda = \frac{hc}E
            = \frac{6{,}626 \cdot 10^{-34}\,\text{Дж}\cdot\text{с} \cdot {3 \cdot 10^{8}\,\frac{\text{м}}{\text{с}}}}{6 \cdot 10^{-19}\,\text{Дж}}
            = 331\,\text{нм}.
    $
}
\solutionspace{80pt}

\tasknumber{3}%
\task{%
    Сделайте схематичный рисунок энергетических уровней атома водорода
    и отметьте на нём первый (основной) уровень и последующие.
    Сколько различных длин волн может испустить атом водорода,
    находящийся в 3-м возбуждённом состоянии (рассмотрите и сложные переходы)?
    Отметьте все соответствующие переходы на рисунке и укажите,
    при каком переходе (среди отмеченных) длина волны излучённого фотона минимальна.
}
\answer{%
    $N = 3, \text{ самая длинная линия }$
}
\solutionspace{150pt}

\tasknumber{4}%
\task{%
    Во сколько раз уменьшается радиус орбиты электрона в атоме водорода,
    если при переходе атома из одного стационарного состояния в другое
    кинетическая энергия электрона увеличивается в двенадцать раз?
}
\answer{%
    $m_e \frac{v^2}r = k\frac{e^2}{r^2} \implies mv^2 = k\frac{e^2} r \implies E_{\text{кин.}} = k\frac{e^2}{2r} \implies 12$
}
\solutionspace{80pt}

\tasknumber{5}%
\task{%
    Во сколько раз увеличилась кинетическая энергия электрона в атоме водорода при переходе
    из одного стационарного состояния в другое, если угловая скорость вращения по орбите увеличилась в шесть раз?
    (Считая, что такие уровни существуют, что можно обсудить отдельно).
}
\answer{%
    $m_e \frac{v^2}r = k\frac{e^2}{r^2}, v = \omega r \implies m_e v^2 = k\frac{e^2}{r} = k\frac{e^2\omega}{ v } \implies v^3 =  k\frac{e^2}{ m_e } \omega \implies 3{,}30$
}
\solutionspace{80pt}

\tasknumber{6}%
\task{%
    Во сколько раз увеличивается угловая скорость вращения электрона в атоме водорода,
    если при переходе атома из одного стационарного состояния в другое радиус орбиты электрона уменьшается в четыре раза?
    (Считая, что такие уровни существуют, что можно обсудить отдельно).
}
\answer{%
    $m_e \frac{v^2}r = k\frac{e^2}{r^2}, v = \omega r \implies m_e \omega^2 r = k\frac{e^2}{r^2} \implies \omega = \sqrt{ k\frac{e^2}{ m_e } } r^{-\frac 32} \implies 8{,}00$
}

\variantsplitter

\addpersonalvariant{Рената Таржиманова}

\tasknumber{1}%
\task{%
    При переходе электрона в атоме с одной стационарной орбиты на другую
    излучается фотон с энергией $1{,}01 \cdot 10^{-19}\,\text{Дж}$.
    Какова длина волны этой линии спектра?
    Постоянная Планка $h = 6{,}626 \cdot 10^{-34}\,\text{Дж}\cdot\text{с}$, скорость света $c = 3 \cdot 10^{8}\,\frac{\text{м}}{\text{с}}$.
}
\answer{%
    $
        E = h\nu = h \frac c\lambda
        \implies \lambda = \frac{hc}E
            = \frac{6{,}626 \cdot 10^{-34}\,\text{Дж}\cdot\text{с} \cdot {3 \cdot 10^{8}\,\frac{\text{м}}{\text{с}}}}{1{,}01 \cdot 10^{-19}\,\text{Дж}}
            = 1968{,}1\,\text{нм}.
    $
}
\solutionspace{80pt}

\tasknumber{2}%
\task{%
    Излучение какой длины волны поглотил атом водорода, если полная энергия в атоме увеличилась на $4 \cdot 10^{-19}\,\text{Дж}$?
    Постоянная Планка $h = 6{,}626 \cdot 10^{-34}\,\text{Дж}\cdot\text{с}$, скорость света $c = 3 \cdot 10^{8}\,\frac{\text{м}}{\text{с}}$.
}
\answer{%
    $
        E = h\nu = h \frac c\lambda
        \implies \lambda = \frac{hc}E
            = \frac{6{,}626 \cdot 10^{-34}\,\text{Дж}\cdot\text{с} \cdot {3 \cdot 10^{8}\,\frac{\text{м}}{\text{с}}}}{4 \cdot 10^{-19}\,\text{Дж}}
            = 497\,\text{нм}.
    $
}
\solutionspace{80pt}

\tasknumber{3}%
\task{%
    Сделайте схематичный рисунок энергетических уровней атома водорода
    и отметьте на нём первый (основной) уровень и последующие.
    Сколько различных длин волн может испустить атом водорода,
    находящийся в 5-м возбуждённом состоянии (рассмотрите и сложные переходы)?
    Отметьте все соответствующие переходы на рисунке и укажите,
    при каком переходе (среди отмеченных) энергия излучённого фотона максимальна.
}
\answer{%
    $N = 10, \text{ самая длинная линия }$
}
\solutionspace{150pt}

\tasknumber{4}%
\task{%
    Во сколько раз уменьшается радиус орбиты электрона в атоме водорода,
    если при переходе атома из одного стационарного состояния в другое
    кинетическая энергия электрона увеличивается в три раза?
}
\answer{%
    $m_e \frac{v^2}r = k\frac{e^2}{r^2} \implies mv^2 = k\frac{e^2} r \implies E_{\text{кин.}} = k\frac{e^2}{2r} \implies 3$
}
\solutionspace{80pt}

\tasknumber{5}%
\task{%
    Во сколько раз увеличилась кинетическая энергия электрона в атоме водорода при переходе
    из одного стационарного состояния в другое, если угловая скорость вращения по орбите увеличилась в двенадцать раз?
    (Считая, что такие уровни существуют, что можно обсудить отдельно).
}
\answer{%
    $m_e \frac{v^2}r = k\frac{e^2}{r^2}, v = \omega r \implies m_e v^2 = k\frac{e^2}{r} = k\frac{e^2\omega}{ v } \implies v^3 =  k\frac{e^2}{ m_e } \omega \implies 5{,}24$
}
\solutionspace{80pt}

\tasknumber{6}%
\task{%
    Во сколько раз увеличивается угловая скорость вращения электрона в атоме водорода,
    если при переходе атома из одного стационарного состояния в другое радиус орбиты электрона уменьшается в семь раз?
    (Считая, что такие уровни существуют, что можно обсудить отдельно).
}
\answer{%
    $m_e \frac{v^2}r = k\frac{e^2}{r^2}, v = \omega r \implies m_e \omega^2 r = k\frac{e^2}{r^2} \implies \omega = \sqrt{ k\frac{e^2}{ m_e } } r^{-\frac 32} \implies 18{,}52$
}

\variantsplitter

\addpersonalvariant{Андрей Щербаков}

\tasknumber{1}%
\task{%
    При переходе электрона в атоме с одной стационарной орбиты на другую
    излучается фотон с энергией $4{,}04 \cdot 10^{-19}\,\text{Дж}$.
    Какова длина волны этой линии спектра?
    Постоянная Планка $h = 6{,}626 \cdot 10^{-34}\,\text{Дж}\cdot\text{с}$, скорость света $c = 3 \cdot 10^{8}\,\frac{\text{м}}{\text{с}}$.
}
\answer{%
    $
        E = h\nu = h \frac c\lambda
        \implies \lambda = \frac{hc}E
            = \frac{6{,}626 \cdot 10^{-34}\,\text{Дж}\cdot\text{с} \cdot {3 \cdot 10^{8}\,\frac{\text{м}}{\text{с}}}}{4{,}04 \cdot 10^{-19}\,\text{Дж}}
            = 492{,}03\,\text{нм}.
    $
}
\solutionspace{80pt}

\tasknumber{2}%
\task{%
    Излучение какой длины волны поглотил атом водорода, если полная энергия в атоме увеличилась на $2 \cdot 10^{-19}\,\text{Дж}$?
    Постоянная Планка $h = 6{,}626 \cdot 10^{-34}\,\text{Дж}\cdot\text{с}$, скорость света $c = 3 \cdot 10^{8}\,\frac{\text{м}}{\text{с}}$.
}
\answer{%
    $
        E = h\nu = h \frac c\lambda
        \implies \lambda = \frac{hc}E
            = \frac{6{,}626 \cdot 10^{-34}\,\text{Дж}\cdot\text{с} \cdot {3 \cdot 10^{8}\,\frac{\text{м}}{\text{с}}}}{2 \cdot 10^{-19}\,\text{Дж}}
            = 994\,\text{нм}.
    $
}
\solutionspace{80pt}

\tasknumber{3}%
\task{%
    Сделайте схематичный рисунок энергетических уровней атома водорода
    и отметьте на нём первый (основной) уровень и последующие.
    Сколько различных длин волн может испустить атом водорода,
    находящийся в 4-м возбуждённом состоянии (рассмотрите и сложные переходы)?
    Отметьте все соответствующие переходы на рисунке и укажите,
    при каком переходе (среди отмеченных) частота излучённого фотона минимальна.
}
\answer{%
    $N = 6, \text{ самая короткая линия }$
}
\solutionspace{150pt}

\tasknumber{4}%
\task{%
    Во сколько раз уменьшается радиус орбиты электрона в атоме водорода,
    если при переходе атома из одного стационарного состояния в другое
    кинетическая энергия электрона увеличивается в девять раз?
}
\answer{%
    $m_e \frac{v^2}r = k\frac{e^2}{r^2} \implies mv^2 = k\frac{e^2} r \implies E_{\text{кин.}} = k\frac{e^2}{2r} \implies 9$
}
\solutionspace{80pt}

\tasknumber{5}%
\task{%
    Во сколько раз увеличилась кинетическая энергия электрона в атоме водорода при переходе
    из одного стационарного состояния в другое, если угловая скорость вращения по орбите увеличилась в восемь раз?
    (Считая, что такие уровни существуют, что можно обсудить отдельно).
}
\answer{%
    $m_e \frac{v^2}r = k\frac{e^2}{r^2}, v = \omega r \implies m_e v^2 = k\frac{e^2}{r} = k\frac{e^2\omega}{ v } \implies v^3 =  k\frac{e^2}{ m_e } \omega \implies 4{,}00$
}
\solutionspace{80pt}

\tasknumber{6}%
\task{%
    Во сколько раз увеличивается угловая скорость вращения электрона в атоме водорода,
    если при переходе атома из одного стационарного состояния в другое радиус орбиты электрона уменьшается в четыре раза?
    (Считая, что такие уровни существуют, что можно обсудить отдельно).
}
\answer{%
    $m_e \frac{v^2}r = k\frac{e^2}{r^2}, v = \omega r \implies m_e \omega^2 r = k\frac{e^2}{r^2} \implies \omega = \sqrt{ k\frac{e^2}{ m_e } } r^{-\frac 32} \implies 8{,}00$
}

\variantsplitter

\addpersonalvariant{Михаил Ярошевский}

\tasknumber{1}%
\task{%
    При переходе электрона в атоме с одной стационарной орбиты на другую
    излучается фотон с энергией $1{,}01 \cdot 10^{-19}\,\text{Дж}$.
    Какова длина волны этой линии спектра?
    Постоянная Планка $h = 6{,}626 \cdot 10^{-34}\,\text{Дж}\cdot\text{с}$, скорость света $c = 3 \cdot 10^{8}\,\frac{\text{м}}{\text{с}}$.
}
\answer{%
    $
        E = h\nu = h \frac c\lambda
        \implies \lambda = \frac{hc}E
            = \frac{6{,}626 \cdot 10^{-34}\,\text{Дж}\cdot\text{с} \cdot {3 \cdot 10^{8}\,\frac{\text{м}}{\text{с}}}}{1{,}01 \cdot 10^{-19}\,\text{Дж}}
            = 1968{,}1\,\text{нм}.
    $
}
\solutionspace{80pt}

\tasknumber{2}%
\task{%
    Излучение какой длины волны поглотил атом водорода, если полная энергия в атоме увеличилась на $4 \cdot 10^{-19}\,\text{Дж}$?
    Постоянная Планка $h = 6{,}626 \cdot 10^{-34}\,\text{Дж}\cdot\text{с}$, скорость света $c = 3 \cdot 10^{8}\,\frac{\text{м}}{\text{с}}$.
}
\answer{%
    $
        E = h\nu = h \frac c\lambda
        \implies \lambda = \frac{hc}E
            = \frac{6{,}626 \cdot 10^{-34}\,\text{Дж}\cdot\text{с} \cdot {3 \cdot 10^{8}\,\frac{\text{м}}{\text{с}}}}{4 \cdot 10^{-19}\,\text{Дж}}
            = 497\,\text{нм}.
    $
}
\solutionspace{80pt}

\tasknumber{3}%
\task{%
    Сделайте схематичный рисунок энергетических уровней атома водорода
    и отметьте на нём первый (основной) уровень и последующие.
    Сколько различных длин волн может испустить атом водорода,
    находящийся в 5-м возбуждённом состоянии (рассмотрите и сложные переходы)?
    Отметьте все соответствующие переходы на рисунке и укажите,
    при каком переходе (среди отмеченных) частота излучённого фотона максимальна.
}
\answer{%
    $N = 10, \text{ самая длинная линия }$
}
\solutionspace{150pt}

\tasknumber{4}%
\task{%
    Во сколько раз уменьшается радиус орбиты электрона в атоме водорода,
    если при переходе атома из одного стационарного состояния в другое
    кинетическая энергия электрона увеличивается в девять раз?
}
\answer{%
    $m_e \frac{v^2}r = k\frac{e^2}{r^2} \implies mv^2 = k\frac{e^2} r \implies E_{\text{кин.}} = k\frac{e^2}{2r} \implies 9$
}
\solutionspace{80pt}

\tasknumber{5}%
\task{%
    Во сколько раз увеличилась кинетическая энергия электрона в атоме водорода при переходе
    из одного стационарного состояния в другое, если угловая скорость вращения по орбите увеличилась в пять раз?
    (Считая, что такие уровни существуют, что можно обсудить отдельно).
}
\answer{%
    $m_e \frac{v^2}r = k\frac{e^2}{r^2}, v = \omega r \implies m_e v^2 = k\frac{e^2}{r} = k\frac{e^2\omega}{ v } \implies v^3 =  k\frac{e^2}{ m_e } \omega \implies 2{,}92$
}
\solutionspace{80pt}

\tasknumber{6}%
\task{%
    Во сколько раз увеличивается угловая скорость вращения электрона в атоме водорода,
    если при переходе атома из одного стационарного состояния в другое радиус орбиты электрона уменьшается в семь раз?
    (Считая, что такие уровни существуют, что можно обсудить отдельно).
}
\answer{%
    $m_e \frac{v^2}r = k\frac{e^2}{r^2}, v = \omega r \implies m_e \omega^2 r = k\frac{e^2}{r^2} \implies \omega = \sqrt{ k\frac{e^2}{ m_e } } r^{-\frac 32} \implies 18{,}52$
}

\variantsplitter

\addpersonalvariant{Алексей Алимпиев}

\tasknumber{1}%
\task{%
    При переходе электрона в атоме с одной стационарной орбиты на другую
    излучается фотон с энергией $5{,}05 \cdot 10^{-19}\,\text{Дж}$.
    Какова длина волны этой линии спектра?
    Постоянная Планка $h = 6{,}626 \cdot 10^{-34}\,\text{Дж}\cdot\text{с}$, скорость света $c = 3 \cdot 10^{8}\,\frac{\text{м}}{\text{с}}$.
}
\answer{%
    $
        E = h\nu = h \frac c\lambda
        \implies \lambda = \frac{hc}E
            = \frac{6{,}626 \cdot 10^{-34}\,\text{Дж}\cdot\text{с} \cdot {3 \cdot 10^{8}\,\frac{\text{м}}{\text{с}}}}{5{,}05 \cdot 10^{-19}\,\text{Дж}}
            = 393{,}62\,\text{нм}.
    $
}
\solutionspace{80pt}

\tasknumber{2}%
\task{%
    Излучение какой длины волны поглотил атом водорода, если полная энергия в атоме увеличилась на $3 \cdot 10^{-19}\,\text{Дж}$?
    Постоянная Планка $h = 6{,}626 \cdot 10^{-34}\,\text{Дж}\cdot\text{с}$, скорость света $c = 3 \cdot 10^{8}\,\frac{\text{м}}{\text{с}}$.
}
\answer{%
    $
        E = h\nu = h \frac c\lambda
        \implies \lambda = \frac{hc}E
            = \frac{6{,}626 \cdot 10^{-34}\,\text{Дж}\cdot\text{с} \cdot {3 \cdot 10^{8}\,\frac{\text{м}}{\text{с}}}}{3 \cdot 10^{-19}\,\text{Дж}}
            = 663\,\text{нм}.
    $
}
\solutionspace{80pt}

\tasknumber{3}%
\task{%
    Сделайте схематичный рисунок энергетических уровней атома водорода
    и отметьте на нём первый (основной) уровень и последующие.
    Сколько различных длин волн может испустить атом водорода,
    находящийся в 3-м возбуждённом состоянии (рассмотрите и сложные переходы)?
    Отметьте все соответствующие переходы на рисунке и укажите,
    при каком переходе (среди отмеченных) энергия излучённого фотона минимальна.
}
\answer{%
    $N = 3, \text{ самая короткая линия }$
}
\solutionspace{150pt}

\tasknumber{4}%
\task{%
    Во сколько раз уменьшается радиус орбиты электрона в атоме водорода,
    если при переходе атома из одного стационарного состояния в другое
    кинетическая энергия электрона увеличивается в шестнадцать раз?
}
\answer{%
    $m_e \frac{v^2}r = k\frac{e^2}{r^2} \implies mv^2 = k\frac{e^2} r \implies E_{\text{кин.}} = k\frac{e^2}{2r} \implies 16$
}
\solutionspace{80pt}

\tasknumber{5}%
\task{%
    Во сколько раз увеличилась кинетическая энергия электрона в атоме водорода при переходе
    из одного стационарного состояния в другое, если угловая скорость вращения по орбите увеличилась в пять раз?
    (Считая, что такие уровни существуют, что можно обсудить отдельно).
}
\answer{%
    $m_e \frac{v^2}r = k\frac{e^2}{r^2}, v = \omega r \implies m_e v^2 = k\frac{e^2}{r} = k\frac{e^2\omega}{ v } \implies v^3 =  k\frac{e^2}{ m_e } \omega \implies 2{,}92$
}
\solutionspace{80pt}

\tasknumber{6}%
\task{%
    Во сколько раз увеличивается угловая скорость вращения электрона в атоме водорода,
    если при переходе атома из одного стационарного состояния в другое радиус орбиты электрона уменьшается в пять раз?
    (Считая, что такие уровни существуют, что можно обсудить отдельно).
}
\answer{%
    $m_e \frac{v^2}r = k\frac{e^2}{r^2}, v = \omega r \implies m_e \omega^2 r = k\frac{e^2}{r^2} \implies \omega = \sqrt{ k\frac{e^2}{ m_e } } r^{-\frac 32} \implies 11{,}18$
}

\variantsplitter

\addpersonalvariant{Евгений Васин}

\tasknumber{1}%
\task{%
    При переходе электрона в атоме с одной стационарной орбиты на другую
    излучается фотон с энергией $7{,}07 \cdot 10^{-19}\,\text{Дж}$.
    Какова длина волны этой линии спектра?
    Постоянная Планка $h = 6{,}626 \cdot 10^{-34}\,\text{Дж}\cdot\text{с}$, скорость света $c = 3 \cdot 10^{8}\,\frac{\text{м}}{\text{с}}$.
}
\answer{%
    $
        E = h\nu = h \frac c\lambda
        \implies \lambda = \frac{hc}E
            = \frac{6{,}626 \cdot 10^{-34}\,\text{Дж}\cdot\text{с} \cdot {3 \cdot 10^{8}\,\frac{\text{м}}{\text{с}}}}{7{,}07 \cdot 10^{-19}\,\text{Дж}}
            = 281{,}16\,\text{нм}.
    $
}
\solutionspace{80pt}

\tasknumber{2}%
\task{%
    Излучение какой длины волны поглотил атом водорода, если полная энергия в атоме увеличилась на $6 \cdot 10^{-19}\,\text{Дж}$?
    Постоянная Планка $h = 6{,}626 \cdot 10^{-34}\,\text{Дж}\cdot\text{с}$, скорость света $c = 3 \cdot 10^{8}\,\frac{\text{м}}{\text{с}}$.
}
\answer{%
    $
        E = h\nu = h \frac c\lambda
        \implies \lambda = \frac{hc}E
            = \frac{6{,}626 \cdot 10^{-34}\,\text{Дж}\cdot\text{с} \cdot {3 \cdot 10^{8}\,\frac{\text{м}}{\text{с}}}}{6 \cdot 10^{-19}\,\text{Дж}}
            = 331\,\text{нм}.
    $
}
\solutionspace{80pt}

\tasknumber{3}%
\task{%
    Сделайте схематичный рисунок энергетических уровней атома водорода
    и отметьте на нём первый (основной) уровень и последующие.
    Сколько различных длин волн может испустить атом водорода,
    находящийся в 5-м возбуждённом состоянии (рассмотрите и сложные переходы)?
    Отметьте все соответствующие переходы на рисунке и укажите,
    при каком переходе (среди отмеченных) частота излучённого фотона максимальна.
}
\answer{%
    $N = 10, \text{ самая длинная линия }$
}
\solutionspace{150pt}

\tasknumber{4}%
\task{%
    Во сколько раз уменьшается радиус орбиты электрона в атоме водорода,
    если при переходе атома из одного стационарного состояния в другое
    кинетическая энергия электрона увеличивается в восемь раз?
}
\answer{%
    $m_e \frac{v^2}r = k\frac{e^2}{r^2} \implies mv^2 = k\frac{e^2} r \implies E_{\text{кин.}} = k\frac{e^2}{2r} \implies 8$
}
\solutionspace{80pt}

\tasknumber{5}%
\task{%
    Во сколько раз увеличилась кинетическая энергия электрона в атоме водорода при переходе
    из одного стационарного состояния в другое, если угловая скорость вращения по орбите увеличилась в двенадцать раз?
    (Считая, что такие уровни существуют, что можно обсудить отдельно).
}
\answer{%
    $m_e \frac{v^2}r = k\frac{e^2}{r^2}, v = \omega r \implies m_e v^2 = k\frac{e^2}{r} = k\frac{e^2\omega}{ v } \implies v^3 =  k\frac{e^2}{ m_e } \omega \implies 5{,}24$
}
\solutionspace{80pt}

\tasknumber{6}%
\task{%
    Во сколько раз увеличивается угловая скорость вращения электрона в атоме водорода,
    если при переходе атома из одного стационарного состояния в другое радиус орбиты электрона уменьшается в семь раз?
    (Считая, что такие уровни существуют, что можно обсудить отдельно).
}
\answer{%
    $m_e \frac{v^2}r = k\frac{e^2}{r^2}, v = \omega r \implies m_e \omega^2 r = k\frac{e^2}{r^2} \implies \omega = \sqrt{ k\frac{e^2}{ m_e } } r^{-\frac 32} \implies 18{,}52$
}

\variantsplitter

\addpersonalvariant{Вячеслав Волохов}

\tasknumber{1}%
\task{%
    При переходе электрона в атоме с одной стационарной орбиты на другую
    излучается фотон с энергией $7{,}07 \cdot 10^{-19}\,\text{Дж}$.
    Какова длина волны этой линии спектра?
    Постоянная Планка $h = 6{,}626 \cdot 10^{-34}\,\text{Дж}\cdot\text{с}$, скорость света $c = 3 \cdot 10^{8}\,\frac{\text{м}}{\text{с}}$.
}
\answer{%
    $
        E = h\nu = h \frac c\lambda
        \implies \lambda = \frac{hc}E
            = \frac{6{,}626 \cdot 10^{-34}\,\text{Дж}\cdot\text{с} \cdot {3 \cdot 10^{8}\,\frac{\text{м}}{\text{с}}}}{7{,}07 \cdot 10^{-19}\,\text{Дж}}
            = 281{,}16\,\text{нм}.
    $
}
\solutionspace{80pt}

\tasknumber{2}%
\task{%
    Излучение какой длины волны поглотил атом водорода, если полная энергия в атоме увеличилась на $6 \cdot 10^{-19}\,\text{Дж}$?
    Постоянная Планка $h = 6{,}626 \cdot 10^{-34}\,\text{Дж}\cdot\text{с}$, скорость света $c = 3 \cdot 10^{8}\,\frac{\text{м}}{\text{с}}$.
}
\answer{%
    $
        E = h\nu = h \frac c\lambda
        \implies \lambda = \frac{hc}E
            = \frac{6{,}626 \cdot 10^{-34}\,\text{Дж}\cdot\text{с} \cdot {3 \cdot 10^{8}\,\frac{\text{м}}{\text{с}}}}{6 \cdot 10^{-19}\,\text{Дж}}
            = 331\,\text{нм}.
    $
}
\solutionspace{80pt}

\tasknumber{3}%
\task{%
    Сделайте схематичный рисунок энергетических уровней атома водорода
    и отметьте на нём первый (основной) уровень и последующие.
    Сколько различных длин волн может испустить атом водорода,
    находящийся в 4-м возбуждённом состоянии (рассмотрите и сложные переходы)?
    Отметьте все соответствующие переходы на рисунке и укажите,
    при каком переходе (среди отмеченных) частота излучённого фотона максимальна.
}
\answer{%
    $N = 6, \text{ самая длинная линия }$
}
\solutionspace{150pt}

\tasknumber{4}%
\task{%
    Во сколько раз уменьшается радиус орбиты электрона в атоме водорода,
    если при переходе атома из одного стационарного состояния в другое
    кинетическая энергия электрона увеличивается в шестнадцать раз?
}
\answer{%
    $m_e \frac{v^2}r = k\frac{e^2}{r^2} \implies mv^2 = k\frac{e^2} r \implies E_{\text{кин.}} = k\frac{e^2}{2r} \implies 16$
}
\solutionspace{80pt}

\tasknumber{5}%
\task{%
    Во сколько раз увеличилась кинетическая энергия электрона в атоме водорода при переходе
    из одного стационарного состояния в другое, если угловая скорость вращения по орбите увеличилась в шесть раз?
    (Считая, что такие уровни существуют, что можно обсудить отдельно).
}
\answer{%
    $m_e \frac{v^2}r = k\frac{e^2}{r^2}, v = \omega r \implies m_e v^2 = k\frac{e^2}{r} = k\frac{e^2\omega}{ v } \implies v^3 =  k\frac{e^2}{ m_e } \omega \implies 3{,}30$
}
\solutionspace{80pt}

\tasknumber{6}%
\task{%
    Во сколько раз увеличивается угловая скорость вращения электрона в атоме водорода,
    если при переходе атома из одного стационарного состояния в другое радиус орбиты электрона уменьшается в два раза?
    (Считая, что такие уровни существуют, что можно обсудить отдельно).
}
\answer{%
    $m_e \frac{v^2}r = k\frac{e^2}{r^2}, v = \omega r \implies m_e \omega^2 r = k\frac{e^2}{r^2} \implies \omega = \sqrt{ k\frac{e^2}{ m_e } } r^{-\frac 32} \implies 2{,}83$
}

\variantsplitter

\addpersonalvariant{Герман Говоров}

\tasknumber{1}%
\task{%
    При переходе электрона в атоме с одной стационарной орбиты на другую
    излучается фотон с энергией $0{,}55 \cdot 10^{-19}\,\text{Дж}$.
    Какова длина волны этой линии спектра?
    Постоянная Планка $h = 6{,}626 \cdot 10^{-34}\,\text{Дж}\cdot\text{с}$, скорость света $c = 3 \cdot 10^{8}\,\frac{\text{м}}{\text{с}}$.
}
\answer{%
    $
        E = h\nu = h \frac c\lambda
        \implies \lambda = \frac{hc}E
            = \frac{6{,}626 \cdot 10^{-34}\,\text{Дж}\cdot\text{с} \cdot {3 \cdot 10^{8}\,\frac{\text{м}}{\text{с}}}}{0{,}55 \cdot 10^{-19}\,\text{Дж}}
            = 3614\,\text{нм}.
    $
}
\solutionspace{80pt}

\tasknumber{2}%
\task{%
    Излучение какой длины волны поглотил атом водорода, если полная энергия в атоме увеличилась на $6 \cdot 10^{-19}\,\text{Дж}$?
    Постоянная Планка $h = 6{,}626 \cdot 10^{-34}\,\text{Дж}\cdot\text{с}$, скорость света $c = 3 \cdot 10^{8}\,\frac{\text{м}}{\text{с}}$.
}
\answer{%
    $
        E = h\nu = h \frac c\lambda
        \implies \lambda = \frac{hc}E
            = \frac{6{,}626 \cdot 10^{-34}\,\text{Дж}\cdot\text{с} \cdot {3 \cdot 10^{8}\,\frac{\text{м}}{\text{с}}}}{6 \cdot 10^{-19}\,\text{Дж}}
            = 331\,\text{нм}.
    $
}
\solutionspace{80pt}

\tasknumber{3}%
\task{%
    Сделайте схематичный рисунок энергетических уровней атома водорода
    и отметьте на нём первый (основной) уровень и последующие.
    Сколько различных длин волн может испустить атом водорода,
    находящийся в 4-м возбуждённом состоянии (рассмотрите и сложные переходы)?
    Отметьте все соответствующие переходы на рисунке и укажите,
    при каком переходе (среди отмеченных) частота излучённого фотона минимальна.
}
\answer{%
    $N = 6, \text{ самая короткая линия }$
}
\solutionspace{150pt}

\tasknumber{4}%
\task{%
    Во сколько раз уменьшается радиус орбиты электрона в атоме водорода,
    если при переходе атома из одного стационарного состояния в другое
    кинетическая энергия электрона увеличивается в четыре раза?
}
\answer{%
    $m_e \frac{v^2}r = k\frac{e^2}{r^2} \implies mv^2 = k\frac{e^2} r \implies E_{\text{кин.}} = k\frac{e^2}{2r} \implies 4$
}
\solutionspace{80pt}

\tasknumber{5}%
\task{%
    Во сколько раз увеличилась кинетическая энергия электрона в атоме водорода при переходе
    из одного стационарного состояния в другое, если угловая скорость вращения по орбите увеличилась в пять раз?
    (Считая, что такие уровни существуют, что можно обсудить отдельно).
}
\answer{%
    $m_e \frac{v^2}r = k\frac{e^2}{r^2}, v = \omega r \implies m_e v^2 = k\frac{e^2}{r} = k\frac{e^2\omega}{ v } \implies v^3 =  k\frac{e^2}{ m_e } \omega \implies 2{,}92$
}
\solutionspace{80pt}

\tasknumber{6}%
\task{%
    Во сколько раз увеличивается угловая скорость вращения электрона в атоме водорода,
    если при переходе атома из одного стационарного состояния в другое радиус орбиты электрона уменьшается в восемь раз?
    (Считая, что такие уровни существуют, что можно обсудить отдельно).
}
\answer{%
    $m_e \frac{v^2}r = k\frac{e^2}{r^2}, v = \omega r \implies m_e \omega^2 r = k\frac{e^2}{r^2} \implies \omega = \sqrt{ k\frac{e^2}{ m_e } } r^{-\frac 32} \implies 22{,}63$
}

\variantsplitter

\addpersonalvariant{София Журавлёва}

\tasknumber{1}%
\task{%
    При переходе электрона в атоме с одной стационарной орбиты на другую
    излучается фотон с энергией $5{,}05 \cdot 10^{-19}\,\text{Дж}$.
    Какова длина волны этой линии спектра?
    Постоянная Планка $h = 6{,}626 \cdot 10^{-34}\,\text{Дж}\cdot\text{с}$, скорость света $c = 3 \cdot 10^{8}\,\frac{\text{м}}{\text{с}}$.
}
\answer{%
    $
        E = h\nu = h \frac c\lambda
        \implies \lambda = \frac{hc}E
            = \frac{6{,}626 \cdot 10^{-34}\,\text{Дж}\cdot\text{с} \cdot {3 \cdot 10^{8}\,\frac{\text{м}}{\text{с}}}}{5{,}05 \cdot 10^{-19}\,\text{Дж}}
            = 393{,}62\,\text{нм}.
    $
}
\solutionspace{80pt}

\tasknumber{2}%
\task{%
    Излучение какой длины волны поглотил атом водорода, если полная энергия в атоме увеличилась на $6 \cdot 10^{-19}\,\text{Дж}$?
    Постоянная Планка $h = 6{,}626 \cdot 10^{-34}\,\text{Дж}\cdot\text{с}$, скорость света $c = 3 \cdot 10^{8}\,\frac{\text{м}}{\text{с}}$.
}
\answer{%
    $
        E = h\nu = h \frac c\lambda
        \implies \lambda = \frac{hc}E
            = \frac{6{,}626 \cdot 10^{-34}\,\text{Дж}\cdot\text{с} \cdot {3 \cdot 10^{8}\,\frac{\text{м}}{\text{с}}}}{6 \cdot 10^{-19}\,\text{Дж}}
            = 331\,\text{нм}.
    $
}
\solutionspace{80pt}

\tasknumber{3}%
\task{%
    Сделайте схематичный рисунок энергетических уровней атома водорода
    и отметьте на нём первый (основной) уровень и последующие.
    Сколько различных длин волн может испустить атом водорода,
    находящийся в 5-м возбуждённом состоянии (рассмотрите и сложные переходы)?
    Отметьте все соответствующие переходы на рисунке и укажите,
    при каком переходе (среди отмеченных) частота излучённого фотона минимальна.
}
\answer{%
    $N = 10, \text{ самая короткая линия }$
}
\solutionspace{150pt}

\tasknumber{4}%
\task{%
    Во сколько раз уменьшается радиус орбиты электрона в атоме водорода,
    если при переходе атома из одного стационарного состояния в другое
    кинетическая энергия электрона увеличивается в восемь раз?
}
\answer{%
    $m_e \frac{v^2}r = k\frac{e^2}{r^2} \implies mv^2 = k\frac{e^2} r \implies E_{\text{кин.}} = k\frac{e^2}{2r} \implies 8$
}
\solutionspace{80pt}

\tasknumber{5}%
\task{%
    Во сколько раз увеличилась кинетическая энергия электрона в атоме водорода при переходе
    из одного стационарного состояния в другое, если угловая скорость вращения по орбите увеличилась в восемь раз?
    (Считая, что такие уровни существуют, что можно обсудить отдельно).
}
\answer{%
    $m_e \frac{v^2}r = k\frac{e^2}{r^2}, v = \omega r \implies m_e v^2 = k\frac{e^2}{r} = k\frac{e^2\omega}{ v } \implies v^3 =  k\frac{e^2}{ m_e } \omega \implies 4{,}00$
}
\solutionspace{80pt}

\tasknumber{6}%
\task{%
    Во сколько раз увеличивается угловая скорость вращения электрона в атоме водорода,
    если при переходе атома из одного стационарного состояния в другое радиус орбиты электрона уменьшается в восемь раз?
    (Считая, что такие уровни существуют, что можно обсудить отдельно).
}
\answer{%
    $m_e \frac{v^2}r = k\frac{e^2}{r^2}, v = \omega r \implies m_e \omega^2 r = k\frac{e^2}{r^2} \implies \omega = \sqrt{ k\frac{e^2}{ m_e } } r^{-\frac 32} \implies 22{,}63$
}

\variantsplitter

\addpersonalvariant{Константин Козлов}

\tasknumber{1}%
\task{%
    При переходе электрона в атоме с одной стационарной орбиты на другую
    излучается фотон с энергией $4{,}04 \cdot 10^{-19}\,\text{Дж}$.
    Какова длина волны этой линии спектра?
    Постоянная Планка $h = 6{,}626 \cdot 10^{-34}\,\text{Дж}\cdot\text{с}$, скорость света $c = 3 \cdot 10^{8}\,\frac{\text{м}}{\text{с}}$.
}
\answer{%
    $
        E = h\nu = h \frac c\lambda
        \implies \lambda = \frac{hc}E
            = \frac{6{,}626 \cdot 10^{-34}\,\text{Дж}\cdot\text{с} \cdot {3 \cdot 10^{8}\,\frac{\text{м}}{\text{с}}}}{4{,}04 \cdot 10^{-19}\,\text{Дж}}
            = 492{,}03\,\text{нм}.
    $
}
\solutionspace{80pt}

\tasknumber{2}%
\task{%
    Излучение какой длины волны поглотил атом водорода, если полная энергия в атоме увеличилась на $2 \cdot 10^{-19}\,\text{Дж}$?
    Постоянная Планка $h = 6{,}626 \cdot 10^{-34}\,\text{Дж}\cdot\text{с}$, скорость света $c = 3 \cdot 10^{8}\,\frac{\text{м}}{\text{с}}$.
}
\answer{%
    $
        E = h\nu = h \frac c\lambda
        \implies \lambda = \frac{hc}E
            = \frac{6{,}626 \cdot 10^{-34}\,\text{Дж}\cdot\text{с} \cdot {3 \cdot 10^{8}\,\frac{\text{м}}{\text{с}}}}{2 \cdot 10^{-19}\,\text{Дж}}
            = 994\,\text{нм}.
    $
}
\solutionspace{80pt}

\tasknumber{3}%
\task{%
    Сделайте схематичный рисунок энергетических уровней атома водорода
    и отметьте на нём первый (основной) уровень и последующие.
    Сколько различных длин волн может испустить атом водорода,
    находящийся в 4-м возбуждённом состоянии (рассмотрите и сложные переходы)?
    Отметьте все соответствующие переходы на рисунке и укажите,
    при каком переходе (среди отмеченных) частота излучённого фотона максимальна.
}
\answer{%
    $N = 6, \text{ самая длинная линия }$
}
\solutionspace{150pt}

\tasknumber{4}%
\task{%
    Во сколько раз уменьшается радиус орбиты электрона в атоме водорода,
    если при переходе атома из одного стационарного состояния в другое
    кинетическая энергия электрона увеличивается в четыре раза?
}
\answer{%
    $m_e \frac{v^2}r = k\frac{e^2}{r^2} \implies mv^2 = k\frac{e^2} r \implies E_{\text{кин.}} = k\frac{e^2}{2r} \implies 4$
}
\solutionspace{80pt}

\tasknumber{5}%
\task{%
    Во сколько раз увеличилась кинетическая энергия электрона в атоме водорода при переходе
    из одного стационарного состояния в другое, если угловая скорость вращения по орбите увеличилась в восемь раз?
    (Считая, что такие уровни существуют, что можно обсудить отдельно).
}
\answer{%
    $m_e \frac{v^2}r = k\frac{e^2}{r^2}, v = \omega r \implies m_e v^2 = k\frac{e^2}{r} = k\frac{e^2\omega}{ v } \implies v^3 =  k\frac{e^2}{ m_e } \omega \implies 4{,}00$
}
\solutionspace{80pt}

\tasknumber{6}%
\task{%
    Во сколько раз увеличивается угловая скорость вращения электрона в атоме водорода,
    если при переходе атома из одного стационарного состояния в другое радиус орбиты электрона уменьшается в три раза?
    (Считая, что такие уровни существуют, что можно обсудить отдельно).
}
\answer{%
    $m_e \frac{v^2}r = k\frac{e^2}{r^2}, v = \omega r \implies m_e \omega^2 r = k\frac{e^2}{r^2} \implies \omega = \sqrt{ k\frac{e^2}{ m_e } } r^{-\frac 32} \implies 5{,}20$
}

\variantsplitter

\addpersonalvariant{Наталья Кравченко}

\tasknumber{1}%
\task{%
    При переходе электрона в атоме с одной стационарной орбиты на другую
    излучается фотон с энергией $7{,}07 \cdot 10^{-19}\,\text{Дж}$.
    Какова длина волны этой линии спектра?
    Постоянная Планка $h = 6{,}626 \cdot 10^{-34}\,\text{Дж}\cdot\text{с}$, скорость света $c = 3 \cdot 10^{8}\,\frac{\text{м}}{\text{с}}$.
}
\answer{%
    $
        E = h\nu = h \frac c\lambda
        \implies \lambda = \frac{hc}E
            = \frac{6{,}626 \cdot 10^{-34}\,\text{Дж}\cdot\text{с} \cdot {3 \cdot 10^{8}\,\frac{\text{м}}{\text{с}}}}{7{,}07 \cdot 10^{-19}\,\text{Дж}}
            = 281{,}16\,\text{нм}.
    $
}
\solutionspace{80pt}

\tasknumber{2}%
\task{%
    Излучение какой длины волны поглотил атом водорода, если полная энергия в атоме увеличилась на $6 \cdot 10^{-19}\,\text{Дж}$?
    Постоянная Планка $h = 6{,}626 \cdot 10^{-34}\,\text{Дж}\cdot\text{с}$, скорость света $c = 3 \cdot 10^{8}\,\frac{\text{м}}{\text{с}}$.
}
\answer{%
    $
        E = h\nu = h \frac c\lambda
        \implies \lambda = \frac{hc}E
            = \frac{6{,}626 \cdot 10^{-34}\,\text{Дж}\cdot\text{с} \cdot {3 \cdot 10^{8}\,\frac{\text{м}}{\text{с}}}}{6 \cdot 10^{-19}\,\text{Дж}}
            = 331\,\text{нм}.
    $
}
\solutionspace{80pt}

\tasknumber{3}%
\task{%
    Сделайте схематичный рисунок энергетических уровней атома водорода
    и отметьте на нём первый (основной) уровень и последующие.
    Сколько различных длин волн может испустить атом водорода,
    находящийся в 5-м возбуждённом состоянии (рассмотрите и сложные переходы)?
    Отметьте все соответствующие переходы на рисунке и укажите,
    при каком переходе (среди отмеченных) частота излучённого фотона максимальна.
}
\answer{%
    $N = 10, \text{ самая длинная линия }$
}
\solutionspace{150pt}

\tasknumber{4}%
\task{%
    Во сколько раз уменьшается радиус орбиты электрона в атоме водорода,
    если при переходе атома из одного стационарного состояния в другое
    кинетическая энергия электрона увеличивается в три раза?
}
\answer{%
    $m_e \frac{v^2}r = k\frac{e^2}{r^2} \implies mv^2 = k\frac{e^2} r \implies E_{\text{кин.}} = k\frac{e^2}{2r} \implies 3$
}
\solutionspace{80pt}

\tasknumber{5}%
\task{%
    Во сколько раз увеличилась кинетическая энергия электрона в атоме водорода при переходе
    из одного стационарного состояния в другое, если угловая скорость вращения по орбите увеличилась в шесть раз?
    (Считая, что такие уровни существуют, что можно обсудить отдельно).
}
\answer{%
    $m_e \frac{v^2}r = k\frac{e^2}{r^2}, v = \omega r \implies m_e v^2 = k\frac{e^2}{r} = k\frac{e^2\omega}{ v } \implies v^3 =  k\frac{e^2}{ m_e } \omega \implies 3{,}30$
}
\solutionspace{80pt}

\tasknumber{6}%
\task{%
    Во сколько раз увеличивается угловая скорость вращения электрона в атоме водорода,
    если при переходе атома из одного стационарного состояния в другое радиус орбиты электрона уменьшается в три раза?
    (Считая, что такие уровни существуют, что можно обсудить отдельно).
}
\answer{%
    $m_e \frac{v^2}r = k\frac{e^2}{r^2}, v = \omega r \implies m_e \omega^2 r = k\frac{e^2}{r^2} \implies \omega = \sqrt{ k\frac{e^2}{ m_e } } r^{-\frac 32} \implies 5{,}20$
}

\variantsplitter

\addpersonalvariant{Матвей Кузьмин}

\tasknumber{1}%
\task{%
    При переходе электрона в атоме с одной стационарной орбиты на другую
    излучается фотон с энергией $7{,}07 \cdot 10^{-19}\,\text{Дж}$.
    Какова длина волны этой линии спектра?
    Постоянная Планка $h = 6{,}626 \cdot 10^{-34}\,\text{Дж}\cdot\text{с}$, скорость света $c = 3 \cdot 10^{8}\,\frac{\text{м}}{\text{с}}$.
}
\answer{%
    $
        E = h\nu = h \frac c\lambda
        \implies \lambda = \frac{hc}E
            = \frac{6{,}626 \cdot 10^{-34}\,\text{Дж}\cdot\text{с} \cdot {3 \cdot 10^{8}\,\frac{\text{м}}{\text{с}}}}{7{,}07 \cdot 10^{-19}\,\text{Дж}}
            = 281{,}16\,\text{нм}.
    $
}
\solutionspace{80pt}

\tasknumber{2}%
\task{%
    Излучение какой длины волны поглотил атом водорода, если полная энергия в атоме увеличилась на $6 \cdot 10^{-19}\,\text{Дж}$?
    Постоянная Планка $h = 6{,}626 \cdot 10^{-34}\,\text{Дж}\cdot\text{с}$, скорость света $c = 3 \cdot 10^{8}\,\frac{\text{м}}{\text{с}}$.
}
\answer{%
    $
        E = h\nu = h \frac c\lambda
        \implies \lambda = \frac{hc}E
            = \frac{6{,}626 \cdot 10^{-34}\,\text{Дж}\cdot\text{с} \cdot {3 \cdot 10^{8}\,\frac{\text{м}}{\text{с}}}}{6 \cdot 10^{-19}\,\text{Дж}}
            = 331\,\text{нм}.
    $
}
\solutionspace{80pt}

\tasknumber{3}%
\task{%
    Сделайте схематичный рисунок энергетических уровней атома водорода
    и отметьте на нём первый (основной) уровень и последующие.
    Сколько различных длин волн может испустить атом водорода,
    находящийся в 4-м возбуждённом состоянии (рассмотрите и сложные переходы)?
    Отметьте все соответствующие переходы на рисунке и укажите,
    при каком переходе (среди отмеченных) частота излучённого фотона минимальна.
}
\answer{%
    $N = 6, \text{ самая короткая линия }$
}
\solutionspace{150pt}

\tasknumber{4}%
\task{%
    Во сколько раз уменьшается радиус орбиты электрона в атоме водорода,
    если при переходе атома из одного стационарного состояния в другое
    кинетическая энергия электрона увеличивается в шестнадцать раз?
}
\answer{%
    $m_e \frac{v^2}r = k\frac{e^2}{r^2} \implies mv^2 = k\frac{e^2} r \implies E_{\text{кин.}} = k\frac{e^2}{2r} \implies 16$
}
\solutionspace{80pt}

\tasknumber{5}%
\task{%
    Во сколько раз увеличилась кинетическая энергия электрона в атоме водорода при переходе
    из одного стационарного состояния в другое, если угловая скорость вращения по орбите увеличилась в десять раз?
    (Считая, что такие уровни существуют, что можно обсудить отдельно).
}
\answer{%
    $m_e \frac{v^2}r = k\frac{e^2}{r^2}, v = \omega r \implies m_e v^2 = k\frac{e^2}{r} = k\frac{e^2\omega}{ v } \implies v^3 =  k\frac{e^2}{ m_e } \omega \implies 4{,}64$
}
\solutionspace{80pt}

\tasknumber{6}%
\task{%
    Во сколько раз увеличивается угловая скорость вращения электрона в атоме водорода,
    если при переходе атома из одного стационарного состояния в другое радиус орбиты электрона уменьшается в три раза?
    (Считая, что такие уровни существуют, что можно обсудить отдельно).
}
\answer{%
    $m_e \frac{v^2}r = k\frac{e^2}{r^2}, v = \omega r \implies m_e \omega^2 r = k\frac{e^2}{r^2} \implies \omega = \sqrt{ k\frac{e^2}{ m_e } } r^{-\frac 32} \implies 5{,}20$
}

\variantsplitter

\addpersonalvariant{Сергей Малышев}

\tasknumber{1}%
\task{%
    При переходе электрона в атоме с одной стационарной орбиты на другую
    излучается фотон с энергией $7{,}07 \cdot 10^{-19}\,\text{Дж}$.
    Какова длина волны этой линии спектра?
    Постоянная Планка $h = 6{,}626 \cdot 10^{-34}\,\text{Дж}\cdot\text{с}$, скорость света $c = 3 \cdot 10^{8}\,\frac{\text{м}}{\text{с}}$.
}
\answer{%
    $
        E = h\nu = h \frac c\lambda
        \implies \lambda = \frac{hc}E
            = \frac{6{,}626 \cdot 10^{-34}\,\text{Дж}\cdot\text{с} \cdot {3 \cdot 10^{8}\,\frac{\text{м}}{\text{с}}}}{7{,}07 \cdot 10^{-19}\,\text{Дж}}
            = 281{,}16\,\text{нм}.
    $
}
\solutionspace{80pt}

\tasknumber{2}%
\task{%
    Излучение какой длины волны поглотил атом водорода, если полная энергия в атоме увеличилась на $6 \cdot 10^{-19}\,\text{Дж}$?
    Постоянная Планка $h = 6{,}626 \cdot 10^{-34}\,\text{Дж}\cdot\text{с}$, скорость света $c = 3 \cdot 10^{8}\,\frac{\text{м}}{\text{с}}$.
}
\answer{%
    $
        E = h\nu = h \frac c\lambda
        \implies \lambda = \frac{hc}E
            = \frac{6{,}626 \cdot 10^{-34}\,\text{Дж}\cdot\text{с} \cdot {3 \cdot 10^{8}\,\frac{\text{м}}{\text{с}}}}{6 \cdot 10^{-19}\,\text{Дж}}
            = 331\,\text{нм}.
    $
}
\solutionspace{80pt}

\tasknumber{3}%
\task{%
    Сделайте схематичный рисунок энергетических уровней атома водорода
    и отметьте на нём первый (основной) уровень и последующие.
    Сколько различных длин волн может испустить атом водорода,
    находящийся в 5-м возбуждённом состоянии (рассмотрите и сложные переходы)?
    Отметьте все соответствующие переходы на рисунке и укажите,
    при каком переходе (среди отмеченных) энергия излучённого фотона максимальна.
}
\answer{%
    $N = 10, \text{ самая длинная линия }$
}
\solutionspace{150pt}

\tasknumber{4}%
\task{%
    Во сколько раз уменьшается радиус орбиты электрона в атоме водорода,
    если при переходе атома из одного стационарного состояния в другое
    кинетическая энергия электрона увеличивается в четыре раза?
}
\answer{%
    $m_e \frac{v^2}r = k\frac{e^2}{r^2} \implies mv^2 = k\frac{e^2} r \implies E_{\text{кин.}} = k\frac{e^2}{2r} \implies 4$
}
\solutionspace{80pt}

\tasknumber{5}%
\task{%
    Во сколько раз увеличилась кинетическая энергия электрона в атоме водорода при переходе
    из одного стационарного состояния в другое, если угловая скорость вращения по орбите увеличилась в шесть раз?
    (Считая, что такие уровни существуют, что можно обсудить отдельно).
}
\answer{%
    $m_e \frac{v^2}r = k\frac{e^2}{r^2}, v = \omega r \implies m_e v^2 = k\frac{e^2}{r} = k\frac{e^2\omega}{ v } \implies v^3 =  k\frac{e^2}{ m_e } \omega \implies 3{,}30$
}
\solutionspace{80pt}

\tasknumber{6}%
\task{%
    Во сколько раз увеличивается угловая скорость вращения электрона в атоме водорода,
    если при переходе атома из одного стационарного состояния в другое радиус орбиты электрона уменьшается в восемь раз?
    (Считая, что такие уровни существуют, что можно обсудить отдельно).
}
\answer{%
    $m_e \frac{v^2}r = k\frac{e^2}{r^2}, v = \omega r \implies m_e \omega^2 r = k\frac{e^2}{r^2} \implies \omega = \sqrt{ k\frac{e^2}{ m_e } } r^{-\frac 32} \implies 22{,}63$
}

\variantsplitter

\addpersonalvariant{Алина Полканова}

\tasknumber{1}%
\task{%
    При переходе электрона в атоме с одной стационарной орбиты на другую
    излучается фотон с энергией $1{,}01 \cdot 10^{-19}\,\text{Дж}$.
    Какова длина волны этой линии спектра?
    Постоянная Планка $h = 6{,}626 \cdot 10^{-34}\,\text{Дж}\cdot\text{с}$, скорость света $c = 3 \cdot 10^{8}\,\frac{\text{м}}{\text{с}}$.
}
\answer{%
    $
        E = h\nu = h \frac c\lambda
        \implies \lambda = \frac{hc}E
            = \frac{6{,}626 \cdot 10^{-34}\,\text{Дж}\cdot\text{с} \cdot {3 \cdot 10^{8}\,\frac{\text{м}}{\text{с}}}}{1{,}01 \cdot 10^{-19}\,\text{Дж}}
            = 1968{,}1\,\text{нм}.
    $
}
\solutionspace{80pt}

\tasknumber{2}%
\task{%
    Излучение какой длины волны поглотил атом водорода, если полная энергия в атоме увеличилась на $4 \cdot 10^{-19}\,\text{Дж}$?
    Постоянная Планка $h = 6{,}626 \cdot 10^{-34}\,\text{Дж}\cdot\text{с}$, скорость света $c = 3 \cdot 10^{8}\,\frac{\text{м}}{\text{с}}$.
}
\answer{%
    $
        E = h\nu = h \frac c\lambda
        \implies \lambda = \frac{hc}E
            = \frac{6{,}626 \cdot 10^{-34}\,\text{Дж}\cdot\text{с} \cdot {3 \cdot 10^{8}\,\frac{\text{м}}{\text{с}}}}{4 \cdot 10^{-19}\,\text{Дж}}
            = 497\,\text{нм}.
    $
}
\solutionspace{80pt}

\tasknumber{3}%
\task{%
    Сделайте схематичный рисунок энергетических уровней атома водорода
    и отметьте на нём первый (основной) уровень и последующие.
    Сколько различных длин волн может испустить атом водорода,
    находящийся в 3-м возбуждённом состоянии (рассмотрите и сложные переходы)?
    Отметьте все соответствующие переходы на рисунке и укажите,
    при каком переходе (среди отмеченных) энергия излучённого фотона минимальна.
}
\answer{%
    $N = 3, \text{ самая короткая линия }$
}
\solutionspace{150pt}

\tasknumber{4}%
\task{%
    Во сколько раз уменьшается радиус орбиты электрона в атоме водорода,
    если при переходе атома из одного стационарного состояния в другое
    кинетическая энергия электрона увеличивается в двенадцать раз?
}
\answer{%
    $m_e \frac{v^2}r = k\frac{e^2}{r^2} \implies mv^2 = k\frac{e^2} r \implies E_{\text{кин.}} = k\frac{e^2}{2r} \implies 12$
}
\solutionspace{80pt}

\tasknumber{5}%
\task{%
    Во сколько раз увеличилась кинетическая энергия электрона в атоме водорода при переходе
    из одного стационарного состояния в другое, если угловая скорость вращения по орбите увеличилась в шесть раз?
    (Считая, что такие уровни существуют, что можно обсудить отдельно).
}
\answer{%
    $m_e \frac{v^2}r = k\frac{e^2}{r^2}, v = \omega r \implies m_e v^2 = k\frac{e^2}{r} = k\frac{e^2\omega}{ v } \implies v^3 =  k\frac{e^2}{ m_e } \omega \implies 3{,}30$
}
\solutionspace{80pt}

\tasknumber{6}%
\task{%
    Во сколько раз увеличивается угловая скорость вращения электрона в атоме водорода,
    если при переходе атома из одного стационарного состояния в другое радиус орбиты электрона уменьшается в пять раз?
    (Считая, что такие уровни существуют, что можно обсудить отдельно).
}
\answer{%
    $m_e \frac{v^2}r = k\frac{e^2}{r^2}, v = \omega r \implies m_e \omega^2 r = k\frac{e^2}{r^2} \implies \omega = \sqrt{ k\frac{e^2}{ m_e } } r^{-\frac 32} \implies 11{,}18$
}

\variantsplitter

\addpersonalvariant{Сергей Пономарёв}

\tasknumber{1}%
\task{%
    При переходе электрона в атоме с одной стационарной орбиты на другую
    излучается фотон с энергией $7{,}07 \cdot 10^{-19}\,\text{Дж}$.
    Какова длина волны этой линии спектра?
    Постоянная Планка $h = 6{,}626 \cdot 10^{-34}\,\text{Дж}\cdot\text{с}$, скорость света $c = 3 \cdot 10^{8}\,\frac{\text{м}}{\text{с}}$.
}
\answer{%
    $
        E = h\nu = h \frac c\lambda
        \implies \lambda = \frac{hc}E
            = \frac{6{,}626 \cdot 10^{-34}\,\text{Дж}\cdot\text{с} \cdot {3 \cdot 10^{8}\,\frac{\text{м}}{\text{с}}}}{7{,}07 \cdot 10^{-19}\,\text{Дж}}
            = 281{,}16\,\text{нм}.
    $
}
\solutionspace{80pt}

\tasknumber{2}%
\task{%
    Излучение какой длины волны поглотил атом водорода, если полная энергия в атоме увеличилась на $6 \cdot 10^{-19}\,\text{Дж}$?
    Постоянная Планка $h = 6{,}626 \cdot 10^{-34}\,\text{Дж}\cdot\text{с}$, скорость света $c = 3 \cdot 10^{8}\,\frac{\text{м}}{\text{с}}$.
}
\answer{%
    $
        E = h\nu = h \frac c\lambda
        \implies \lambda = \frac{hc}E
            = \frac{6{,}626 \cdot 10^{-34}\,\text{Дж}\cdot\text{с} \cdot {3 \cdot 10^{8}\,\frac{\text{м}}{\text{с}}}}{6 \cdot 10^{-19}\,\text{Дж}}
            = 331\,\text{нм}.
    $
}
\solutionspace{80pt}

\tasknumber{3}%
\task{%
    Сделайте схематичный рисунок энергетических уровней атома водорода
    и отметьте на нём первый (основной) уровень и последующие.
    Сколько различных длин волн может испустить атом водорода,
    находящийся в 3-м возбуждённом состоянии (рассмотрите и сложные переходы)?
    Отметьте все соответствующие переходы на рисунке и укажите,
    при каком переходе (среди отмеченных) энергия излучённого фотона минимальна.
}
\answer{%
    $N = 3, \text{ самая короткая линия }$
}
\solutionspace{150pt}

\tasknumber{4}%
\task{%
    Во сколько раз уменьшается радиус орбиты электрона в атоме водорода,
    если при переходе атома из одного стационарного состояния в другое
    кинетическая энергия электрона увеличивается в восемь раз?
}
\answer{%
    $m_e \frac{v^2}r = k\frac{e^2}{r^2} \implies mv^2 = k\frac{e^2} r \implies E_{\text{кин.}} = k\frac{e^2}{2r} \implies 8$
}
\solutionspace{80pt}

\tasknumber{5}%
\task{%
    Во сколько раз увеличилась кинетическая энергия электрона в атоме водорода при переходе
    из одного стационарного состояния в другое, если угловая скорость вращения по орбите увеличилась в двенадцать раз?
    (Считая, что такие уровни существуют, что можно обсудить отдельно).
}
\answer{%
    $m_e \frac{v^2}r = k\frac{e^2}{r^2}, v = \omega r \implies m_e v^2 = k\frac{e^2}{r} = k\frac{e^2\omega}{ v } \implies v^3 =  k\frac{e^2}{ m_e } \omega \implies 5{,}24$
}
\solutionspace{80pt}

\tasknumber{6}%
\task{%
    Во сколько раз увеличивается угловая скорость вращения электрона в атоме водорода,
    если при переходе атома из одного стационарного состояния в другое радиус орбиты электрона уменьшается в три раза?
    (Считая, что такие уровни существуют, что можно обсудить отдельно).
}
\answer{%
    $m_e \frac{v^2}r = k\frac{e^2}{r^2}, v = \omega r \implies m_e \omega^2 r = k\frac{e^2}{r^2} \implies \omega = \sqrt{ k\frac{e^2}{ m_e } } r^{-\frac 32} \implies 5{,}20$
}

\variantsplitter

\addpersonalvariant{Егор Свистушкин}

\tasknumber{1}%
\task{%
    При переходе электрона в атоме с одной стационарной орбиты на другую
    излучается фотон с энергией $4{,}04 \cdot 10^{-19}\,\text{Дж}$.
    Какова длина волны этой линии спектра?
    Постоянная Планка $h = 6{,}626 \cdot 10^{-34}\,\text{Дж}\cdot\text{с}$, скорость света $c = 3 \cdot 10^{8}\,\frac{\text{м}}{\text{с}}$.
}
\answer{%
    $
        E = h\nu = h \frac c\lambda
        \implies \lambda = \frac{hc}E
            = \frac{6{,}626 \cdot 10^{-34}\,\text{Дж}\cdot\text{с} \cdot {3 \cdot 10^{8}\,\frac{\text{м}}{\text{с}}}}{4{,}04 \cdot 10^{-19}\,\text{Дж}}
            = 492{,}03\,\text{нм}.
    $
}
\solutionspace{80pt}

\tasknumber{2}%
\task{%
    Излучение какой длины волны поглотил атом водорода, если полная энергия в атоме увеличилась на $2 \cdot 10^{-19}\,\text{Дж}$?
    Постоянная Планка $h = 6{,}626 \cdot 10^{-34}\,\text{Дж}\cdot\text{с}$, скорость света $c = 3 \cdot 10^{8}\,\frac{\text{м}}{\text{с}}$.
}
\answer{%
    $
        E = h\nu = h \frac c\lambda
        \implies \lambda = \frac{hc}E
            = \frac{6{,}626 \cdot 10^{-34}\,\text{Дж}\cdot\text{с} \cdot {3 \cdot 10^{8}\,\frac{\text{м}}{\text{с}}}}{2 \cdot 10^{-19}\,\text{Дж}}
            = 994\,\text{нм}.
    $
}
\solutionspace{80pt}

\tasknumber{3}%
\task{%
    Сделайте схематичный рисунок энергетических уровней атома водорода
    и отметьте на нём первый (основной) уровень и последующие.
    Сколько различных длин волн может испустить атом водорода,
    находящийся в 3-м возбуждённом состоянии (рассмотрите и сложные переходы)?
    Отметьте все соответствующие переходы на рисунке и укажите,
    при каком переходе (среди отмеченных) частота излучённого фотона максимальна.
}
\answer{%
    $N = 3, \text{ самая длинная линия }$
}
\solutionspace{150pt}

\tasknumber{4}%
\task{%
    Во сколько раз уменьшается радиус орбиты электрона в атоме водорода,
    если при переходе атома из одного стационарного состояния в другое
    кинетическая энергия электрона увеличивается в двенадцать раз?
}
\answer{%
    $m_e \frac{v^2}r = k\frac{e^2}{r^2} \implies mv^2 = k\frac{e^2} r \implies E_{\text{кин.}} = k\frac{e^2}{2r} \implies 12$
}
\solutionspace{80pt}

\tasknumber{5}%
\task{%
    Во сколько раз увеличилась кинетическая энергия электрона в атоме водорода при переходе
    из одного стационарного состояния в другое, если угловая скорость вращения по орбите увеличилась в пять раз?
    (Считая, что такие уровни существуют, что можно обсудить отдельно).
}
\answer{%
    $m_e \frac{v^2}r = k\frac{e^2}{r^2}, v = \omega r \implies m_e v^2 = k\frac{e^2}{r} = k\frac{e^2\omega}{ v } \implies v^3 =  k\frac{e^2}{ m_e } \omega \implies 2{,}92$
}
\solutionspace{80pt}

\tasknumber{6}%
\task{%
    Во сколько раз увеличивается угловая скорость вращения электрона в атоме водорода,
    если при переходе атома из одного стационарного состояния в другое радиус орбиты электрона уменьшается в три раза?
    (Считая, что такие уровни существуют, что можно обсудить отдельно).
}
\answer{%
    $m_e \frac{v^2}r = k\frac{e^2}{r^2}, v = \omega r \implies m_e \omega^2 r = k\frac{e^2}{r^2} \implies \omega = \sqrt{ k\frac{e^2}{ m_e } } r^{-\frac 32} \implies 5{,}20$
}

\variantsplitter

\addpersonalvariant{Дмитрий Соколов}

\tasknumber{1}%
\task{%
    При переходе электрона в атоме с одной стационарной орбиты на другую
    излучается фотон с энергией $5{,}05 \cdot 10^{-19}\,\text{Дж}$.
    Какова длина волны этой линии спектра?
    Постоянная Планка $h = 6{,}626 \cdot 10^{-34}\,\text{Дж}\cdot\text{с}$, скорость света $c = 3 \cdot 10^{8}\,\frac{\text{м}}{\text{с}}$.
}
\answer{%
    $
        E = h\nu = h \frac c\lambda
        \implies \lambda = \frac{hc}E
            = \frac{6{,}626 \cdot 10^{-34}\,\text{Дж}\cdot\text{с} \cdot {3 \cdot 10^{8}\,\frac{\text{м}}{\text{с}}}}{5{,}05 \cdot 10^{-19}\,\text{Дж}}
            = 393{,}62\,\text{нм}.
    $
}
\solutionspace{80pt}

\tasknumber{2}%
\task{%
    Излучение какой длины волны поглотил атом водорода, если полная энергия в атоме увеличилась на $6 \cdot 10^{-19}\,\text{Дж}$?
    Постоянная Планка $h = 6{,}626 \cdot 10^{-34}\,\text{Дж}\cdot\text{с}$, скорость света $c = 3 \cdot 10^{8}\,\frac{\text{м}}{\text{с}}$.
}
\answer{%
    $
        E = h\nu = h \frac c\lambda
        \implies \lambda = \frac{hc}E
            = \frac{6{,}626 \cdot 10^{-34}\,\text{Дж}\cdot\text{с} \cdot {3 \cdot 10^{8}\,\frac{\text{м}}{\text{с}}}}{6 \cdot 10^{-19}\,\text{Дж}}
            = 331\,\text{нм}.
    $
}
\solutionspace{80pt}

\tasknumber{3}%
\task{%
    Сделайте схематичный рисунок энергетических уровней атома водорода
    и отметьте на нём первый (основной) уровень и последующие.
    Сколько различных длин волн может испустить атом водорода,
    находящийся в 5-м возбуждённом состоянии (рассмотрите и сложные переходы)?
    Отметьте все соответствующие переходы на рисунке и укажите,
    при каком переходе (среди отмеченных) длина волны излучённого фотона максимальна.
}
\answer{%
    $N = 10, \text{ самая короткая линия }$
}
\solutionspace{150pt}

\tasknumber{4}%
\task{%
    Во сколько раз уменьшается радиус орбиты электрона в атоме водорода,
    если при переходе атома из одного стационарного состояния в другое
    кинетическая энергия электрона увеличивается в девять раз?
}
\answer{%
    $m_e \frac{v^2}r = k\frac{e^2}{r^2} \implies mv^2 = k\frac{e^2} r \implies E_{\text{кин.}} = k\frac{e^2}{2r} \implies 9$
}
\solutionspace{80pt}

\tasknumber{5}%
\task{%
    Во сколько раз увеличилась кинетическая энергия электрона в атоме водорода при переходе
    из одного стационарного состояния в другое, если угловая скорость вращения по орбите увеличилась в шесть раз?
    (Считая, что такие уровни существуют, что можно обсудить отдельно).
}
\answer{%
    $m_e \frac{v^2}r = k\frac{e^2}{r^2}, v = \omega r \implies m_e v^2 = k\frac{e^2}{r} = k\frac{e^2\omega}{ v } \implies v^3 =  k\frac{e^2}{ m_e } \omega \implies 3{,}30$
}
\solutionspace{80pt}

\tasknumber{6}%
\task{%
    Во сколько раз увеличивается угловая скорость вращения электрона в атоме водорода,
    если при переходе атома из одного стационарного состояния в другое радиус орбиты электрона уменьшается в пять раз?
    (Считая, что такие уровни существуют, что можно обсудить отдельно).
}
\answer{%
    $m_e \frac{v^2}r = k\frac{e^2}{r^2}, v = \omega r \implies m_e \omega^2 r = k\frac{e^2}{r^2} \implies \omega = \sqrt{ k\frac{e^2}{ m_e } } r^{-\frac 32} \implies 11{,}18$
}

\variantsplitter

\addpersonalvariant{Арсений Трофимов}

\tasknumber{1}%
\task{%
    При переходе электрона в атоме с одной стационарной орбиты на другую
    излучается фотон с энергией $5{,}05 \cdot 10^{-19}\,\text{Дж}$.
    Какова длина волны этой линии спектра?
    Постоянная Планка $h = 6{,}626 \cdot 10^{-34}\,\text{Дж}\cdot\text{с}$, скорость света $c = 3 \cdot 10^{8}\,\frac{\text{м}}{\text{с}}$.
}
\answer{%
    $
        E = h\nu = h \frac c\lambda
        \implies \lambda = \frac{hc}E
            = \frac{6{,}626 \cdot 10^{-34}\,\text{Дж}\cdot\text{с} \cdot {3 \cdot 10^{8}\,\frac{\text{м}}{\text{с}}}}{5{,}05 \cdot 10^{-19}\,\text{Дж}}
            = 393{,}62\,\text{нм}.
    $
}
\solutionspace{80pt}

\tasknumber{2}%
\task{%
    Излучение какой длины волны поглотил атом водорода, если полная энергия в атоме увеличилась на $6 \cdot 10^{-19}\,\text{Дж}$?
    Постоянная Планка $h = 6{,}626 \cdot 10^{-34}\,\text{Дж}\cdot\text{с}$, скорость света $c = 3 \cdot 10^{8}\,\frac{\text{м}}{\text{с}}$.
}
\answer{%
    $
        E = h\nu = h \frac c\lambda
        \implies \lambda = \frac{hc}E
            = \frac{6{,}626 \cdot 10^{-34}\,\text{Дж}\cdot\text{с} \cdot {3 \cdot 10^{8}\,\frac{\text{м}}{\text{с}}}}{6 \cdot 10^{-19}\,\text{Дж}}
            = 331\,\text{нм}.
    $
}
\solutionspace{80pt}

\tasknumber{3}%
\task{%
    Сделайте схематичный рисунок энергетических уровней атома водорода
    и отметьте на нём первый (основной) уровень и последующие.
    Сколько различных длин волн может испустить атом водорода,
    находящийся в 4-м возбуждённом состоянии (рассмотрите и сложные переходы)?
    Отметьте все соответствующие переходы на рисунке и укажите,
    при каком переходе (среди отмеченных) частота излучённого фотона минимальна.
}
\answer{%
    $N = 6, \text{ самая короткая линия }$
}
\solutionspace{150pt}

\tasknumber{4}%
\task{%
    Во сколько раз уменьшается радиус орбиты электрона в атоме водорода,
    если при переходе атома из одного стационарного состояния в другое
    кинетическая энергия электрона увеличивается в двенадцать раз?
}
\answer{%
    $m_e \frac{v^2}r = k\frac{e^2}{r^2} \implies mv^2 = k\frac{e^2} r \implies E_{\text{кин.}} = k\frac{e^2}{2r} \implies 12$
}
\solutionspace{80pt}

\tasknumber{5}%
\task{%
    Во сколько раз увеличилась кинетическая энергия электрона в атоме водорода при переходе
    из одного стационарного состояния в другое, если угловая скорость вращения по орбите увеличилась в пять раз?
    (Считая, что такие уровни существуют, что можно обсудить отдельно).
}
\answer{%
    $m_e \frac{v^2}r = k\frac{e^2}{r^2}, v = \omega r \implies m_e v^2 = k\frac{e^2}{r} = k\frac{e^2\omega}{ v } \implies v^3 =  k\frac{e^2}{ m_e } \omega \implies 2{,}92$
}
\solutionspace{80pt}

\tasknumber{6}%
\task{%
    Во сколько раз увеличивается угловая скорость вращения электрона в атоме водорода,
    если при переходе атома из одного стационарного состояния в другое радиус орбиты электрона уменьшается в четыре раза?
    (Считая, что такие уровни существуют, что можно обсудить отдельно).
}
\answer{%
    $m_e \frac{v^2}r = k\frac{e^2}{r^2}, v = \omega r \implies m_e \omega^2 r = k\frac{e^2}{r^2} \implies \omega = \sqrt{ k\frac{e^2}{ m_e } } r^{-\frac 32} \implies 8{,}00$
}
% autogenerated
