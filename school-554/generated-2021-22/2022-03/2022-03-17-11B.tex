\setdate{17~марта~2022}
\setclass{11«Б»}

\addpersonalvariant{Михаил Бурмистров}

\tasknumber{1}%
\task{%
    Кратко опишите модель атома Бора и укажите проблемы этой теории (какой наблюдаемый эффект она не способна описать),
    сделайте необходимые рисунки.
}
\solutionspace{150pt}

\tasknumber{2}%
\task{%
    При переходе электрона в атоме с одной стационарной орбиты на другую
    излучается фотон с энергией $5{,}05 \cdot 10^{-19}\,\text{Дж}$.
    Какова длина волны этой линии спектра?
    Постоянная Планка $h = 6{,}626 \cdot 10^{-34}\,\text{Дж}\cdot\text{с}$, скорость света $c = 3 \cdot 10^{8}\,\frac{\text{м}}{\text{с}}$.
}
\answer{%
    $
        E = h\nu = h \frac c\lambda
        \implies \lambda = \frac{hc}E
            = \frac{6{,}626 \cdot 10^{-34}\,\text{Дж}\cdot\text{с} \cdot {3 \cdot 10^{8}\,\frac{\text{м}}{\text{с}}}}{5{,}05 \cdot 10^{-19}\,\text{Дж}}
            = 393{,}62\,\text{нм}.
    $
}
\solutionspace{80pt}

\tasknumber{3}%
\task{%
    Электрон в некотором атоме переходит из стационарного состояния с энергией $-3{,}55 \cdot 10^{-19}\,\text{Дж}$
    в другое стационарного состояния с энергией $-3{,}95 \cdot 10^{-19}\,\text{Дж}$.
    \begin{itemize}
        \item Определите, происходит при этом поглощение или излучение кванта света?
        \item Чему равна энергия этого фотона?
        \item Чему равна частота этого фотона?
        \item Чему равна длина волны этого кванта света?
    \end{itemize}
}
\solutionspace{80pt}

\tasknumber{4}%
\task{%
    В некотором атоме есть 3 энергетических уровня с энергиями $E_1 = -1{,}5 \cdot 10^{-19}\,\text{Дж}$, $E_2 = -2{,}7 \cdot 10^{-19}\,\text{Дж}$ и $E_3 = -3{,}4 \cdot 10^{-19}\,\text{Дж}$.
    Электрон находится на втором ($E_2$).
    Определите энергию фотона, который может быть поглощён таким атомом.
}
\solutionspace{80pt}

\tasknumber{5}%
\task{%
    Сделайте схематичный рисунок энергетических уровней атома водорода
    и отметьте на нём первый (основной) уровень и последующие.
    Сколько различных длин волн может испустить атом водорода,
    находящийся в 3-м возбуждённом состоянии (рассмотрите и сложные переходы)?
    Отметьте все соответствующие переходы на рисунке и укажите,
    при каком переходе (среди отмеченных) частота излучённого фотона максимальна.
}
\answer{%
    $N = 3, \text{ самая длинная линия }$
}

\variantsplitter

\addpersonalvariant{Снежана Авдошина}

\tasknumber{1}%
\task{%
    Кратко опишите планетарную модель атома и укажите проблемы этой теории (какой наблюдаемый эффект она не способна описать),
    сделайте необходимые рисунки.
}
\solutionspace{150pt}

\tasknumber{2}%
\task{%
    При переходе электрона в атоме с одной стационарной орбиты на другую
    излучается фотон с энергией $7{,}07 \cdot 10^{-19}\,\text{Дж}$.
    Какова длина волны этой линии спектра?
    Постоянная Планка $h = 6{,}626 \cdot 10^{-34}\,\text{Дж}\cdot\text{с}$, скорость света $c = 3 \cdot 10^{8}\,\frac{\text{м}}{\text{с}}$.
}
\answer{%
    $
        E = h\nu = h \frac c\lambda
        \implies \lambda = \frac{hc}E
            = \frac{6{,}626 \cdot 10^{-34}\,\text{Дж}\cdot\text{с} \cdot {3 \cdot 10^{8}\,\frac{\text{м}}{\text{с}}}}{7{,}07 \cdot 10^{-19}\,\text{Дж}}
            = 281{,}16\,\text{нм}.
    $
}
\solutionspace{80pt}

\tasknumber{3}%
\task{%
    Электрон в некотором атоме переходит из стационарного состояния с энергией $-1{,}15 \cdot 10^{-19}\,\text{Дж}$
    в другое стационарного состояния с энергией $-2{,}85 \cdot 10^{-19}\,\text{Дж}$.
    \begin{itemize}
        \item Определите, происходит при этом поглощение или излучение кванта света?
        \item Чему равна энергия этого фотона?
        \item Чему равна частота этого фотона?
        \item Чему равна длина волны этого кванта света?
    \end{itemize}
}
\solutionspace{80pt}

\tasknumber{4}%
\task{%
    В некотором атоме есть 3 энергетических уровня с энергиями $E_1 = -1{,}4 \cdot 10^{-19}\,\text{Дж}$, $E_2 = -2{,}7 \cdot 10^{-19}\,\text{Дж}$ и $E_3 = -3{,}7 \cdot 10^{-19}\,\text{Дж}$.
    Электрон находится на втором ($E_2$).
    Определите энергию фотона, который может быть испущен таким атомом.
}
\solutionspace{80pt}

\tasknumber{5}%
\task{%
    Сделайте схематичный рисунок энергетических уровней атома водорода
    и отметьте на нём первый (основной) уровень и последующие.
    Сколько различных длин волн может испустить атом водорода,
    находящийся в 4-м возбуждённом состоянии (рассмотрите и сложные переходы)?
    Отметьте все соответствующие переходы на рисунке и укажите,
    при каком переходе (среди отмеченных) частота излучённого фотона минимальна.
}
\answer{%
    $N = 6, \text{ самая короткая линия }$
}

\variantsplitter

\addpersonalvariant{Марьяна Аристова}

\tasknumber{1}%
\task{%
    Кратко опишите модель атома Резерфорда и укажите проблемы этой теории (какой наблюдаемый эффект она не способна описать),
    сделайте необходимые рисунки.
}
\solutionspace{150pt}

\tasknumber{2}%
\task{%
    При переходе электрона в атоме с одной стационарной орбиты на другую
    излучается фотон с энергией $5{,}05 \cdot 10^{-19}\,\text{Дж}$.
    Какова длина волны этой линии спектра?
    Постоянная Планка $h = 6{,}626 \cdot 10^{-34}\,\text{Дж}\cdot\text{с}$, скорость света $c = 3 \cdot 10^{8}\,\frac{\text{м}}{\text{с}}$.
}
\answer{%
    $
        E = h\nu = h \frac c\lambda
        \implies \lambda = \frac{hc}E
            = \frac{6{,}626 \cdot 10^{-34}\,\text{Дж}\cdot\text{с} \cdot {3 \cdot 10^{8}\,\frac{\text{м}}{\text{с}}}}{5{,}05 \cdot 10^{-19}\,\text{Дж}}
            = 393{,}62\,\text{нм}.
    $
}
\solutionspace{80pt}

\tasknumber{3}%
\task{%
    Электрон в некотором атоме переходит из стационарного состояния с энергией $-3{,}55 \cdot 10^{-19}\,\text{Дж}$
    в другое стационарного состояния с энергией $-3{,}95 \cdot 10^{-19}\,\text{Дж}$.
    \begin{itemize}
        \item Определите, происходит при этом поглощение или излучение кванта света?
        \item Чему равна энергия этого фотона?
        \item Чему равна частота этого фотона?
        \item Чему равна длина волны этого кванта света?
    \end{itemize}
}
\solutionspace{80pt}

\tasknumber{4}%
\task{%
    В некотором атоме есть 3 энергетических уровня с энергиями $E_1 = -1{,}3 \cdot 10^{-19}\,\text{Дж}$, $E_2 = -2{,}9 \cdot 10^{-19}\,\text{Дж}$ и $E_3 = -3{,}7 \cdot 10^{-19}\,\text{Дж}$.
    Электрон находится на втором ($E_2$).
    Определите энергию фотона, который может быть испущен таким атомом.
}
\solutionspace{80pt}

\tasknumber{5}%
\task{%
    Сделайте схематичный рисунок энергетических уровней атома водорода
    и отметьте на нём первый (основной) уровень и последующие.
    Сколько различных длин волн может испустить атом водорода,
    находящийся в 4-м возбуждённом состоянии (рассмотрите и сложные переходы)?
    Отметьте все соответствующие переходы на рисунке и укажите,
    при каком переходе (среди отмеченных) длина волны излучённого фотона минимальна.
}
\answer{%
    $N = 6, \text{ самая длинная линия }$
}

\variantsplitter

\addpersonalvariant{Никита Иванов}

\tasknumber{1}%
\task{%
    Кратко опишите модель атома Резерфорда и укажите проблемы этой теории (какой наблюдаемый эффект она не способна описать),
    сделайте необходимые рисунки.
}
\solutionspace{150pt}

\tasknumber{2}%
\task{%
    При переходе электрона в атоме с одной стационарной орбиты на другую
    излучается фотон с энергией $0{,}55 \cdot 10^{-19}\,\text{Дж}$.
    Какова длина волны этой линии спектра?
    Постоянная Планка $h = 6{,}626 \cdot 10^{-34}\,\text{Дж}\cdot\text{с}$, скорость света $c = 3 \cdot 10^{8}\,\frac{\text{м}}{\text{с}}$.
}
\answer{%
    $
        E = h\nu = h \frac c\lambda
        \implies \lambda = \frac{hc}E
            = \frac{6{,}626 \cdot 10^{-34}\,\text{Дж}\cdot\text{с} \cdot {3 \cdot 10^{8}\,\frac{\text{м}}{\text{с}}}}{0{,}55 \cdot 10^{-19}\,\text{Дж}}
            = 3614\,\text{нм}.
    $
}
\solutionspace{80pt}

\tasknumber{3}%
\task{%
    Электрон в некотором атоме переходит из стационарного состояния с энергией $-2{,}35 \cdot 10^{-19}\,\text{Дж}$
    в другое стационарного состояния с энергией $-1{,}75 \cdot 10^{-19}\,\text{Дж}$.
    \begin{itemize}
        \item Определите, происходит при этом поглощение или излучение кванта света?
        \item Чему равна энергия этого фотона?
        \item Чему равна частота этого фотона?
        \item Чему равна длина волны этого кванта света?
    \end{itemize}
}
\solutionspace{80pt}

\tasknumber{4}%
\task{%
    В некотором атоме есть 3 энергетических уровня с энергиями $E_1 = -1{,}1 \cdot 10^{-19}\,\text{Дж}$, $E_2 = -2{,}2 \cdot 10^{-19}\,\text{Дж}$ и $E_3 = -3{,}4 \cdot 10^{-19}\,\text{Дж}$.
    Электрон находится на втором ($E_2$).
    Определите энергию фотона, который может быть поглощён таким атомом.
}
\solutionspace{80pt}

\tasknumber{5}%
\task{%
    Сделайте схематичный рисунок энергетических уровней атома водорода
    и отметьте на нём первый (основной) уровень и последующие.
    Сколько различных длин волн может испустить атом водорода,
    находящийся в 3-м возбуждённом состоянии (рассмотрите и сложные переходы)?
    Отметьте все соответствующие переходы на рисунке и укажите,
    при каком переходе (среди отмеченных) частота излучённого фотона минимальна.
}
\answer{%
    $N = 3, \text{ самая короткая линия }$
}

\variantsplitter

\addpersonalvariant{Анастасия Князева}

\tasknumber{1}%
\task{%
    Кратко опишите модель атома Томсона и укажите проблемы этой теории (какой наблюдаемый эффект она не способна описать),
    сделайте необходимые рисунки.
}
\solutionspace{150pt}

\tasknumber{2}%
\task{%
    При переходе электрона в атоме с одной стационарной орбиты на другую
    излучается фотон с энергией $5{,}05 \cdot 10^{-19}\,\text{Дж}$.
    Какова длина волны этой линии спектра?
    Постоянная Планка $h = 6{,}626 \cdot 10^{-34}\,\text{Дж}\cdot\text{с}$, скорость света $c = 3 \cdot 10^{8}\,\frac{\text{м}}{\text{с}}$.
}
\answer{%
    $
        E = h\nu = h \frac c\lambda
        \implies \lambda = \frac{hc}E
            = \frac{6{,}626 \cdot 10^{-34}\,\text{Дж}\cdot\text{с} \cdot {3 \cdot 10^{8}\,\frac{\text{м}}{\text{с}}}}{5{,}05 \cdot 10^{-19}\,\text{Дж}}
            = 393{,}62\,\text{нм}.
    $
}
\solutionspace{80pt}

\tasknumber{3}%
\task{%
    Электрон в некотором атоме переходит из стационарного состояния с энергией $-1{,}15 \cdot 10^{-19}\,\text{Дж}$
    в другое стационарного состояния с энергией $-1{,}75 \cdot 10^{-19}\,\text{Дж}$.
    \begin{itemize}
        \item Определите, происходит при этом поглощение или излучение кванта света?
        \item Чему равна энергия этого фотона?
        \item Чему равна частота этого фотона?
        \item Чему равна длина волны этого кванта света?
    \end{itemize}
}
\solutionspace{80pt}

\tasknumber{4}%
\task{%
    В некотором атоме есть 3 энергетических уровня с энергиями $E_1 = -1{,}1 \cdot 10^{-19}\,\text{Дж}$, $E_2 = -2{,}5 \cdot 10^{-19}\,\text{Дж}$ и $E_3 = -3{,}4 \cdot 10^{-19}\,\text{Дж}$.
    Электрон находится на втором ($E_2$).
    Определите энергию фотона, который может быть поглощён таким атомом.
}
\solutionspace{80pt}

\tasknumber{5}%
\task{%
    Сделайте схематичный рисунок энергетических уровней атома водорода
    и отметьте на нём первый (основной) уровень и последующие.
    Сколько различных длин волн может испустить атом водорода,
    находящийся в 4-м возбуждённом состоянии (рассмотрите и сложные переходы)?
    Отметьте все соответствующие переходы на рисунке и укажите,
    при каком переходе (среди отмеченных) энергия излучённого фотона максимальна.
}
\answer{%
    $N = 6, \text{ самая длинная линия }$
}

\variantsplitter

\addpersonalvariant{Елизавета Кутумова}

\tasknumber{1}%
\task{%
    Кратко опишите модель атома Резерфорда и укажите проблемы этой теории (какой наблюдаемый эффект она не способна описать),
    сделайте необходимые рисунки.
}
\solutionspace{150pt}

\tasknumber{2}%
\task{%
    При переходе электрона в атоме с одной стационарной орбиты на другую
    излучается фотон с энергией $7{,}07 \cdot 10^{-19}\,\text{Дж}$.
    Какова длина волны этой линии спектра?
    Постоянная Планка $h = 6{,}626 \cdot 10^{-34}\,\text{Дж}\cdot\text{с}$, скорость света $c = 3 \cdot 10^{8}\,\frac{\text{м}}{\text{с}}$.
}
\answer{%
    $
        E = h\nu = h \frac c\lambda
        \implies \lambda = \frac{hc}E
            = \frac{6{,}626 \cdot 10^{-34}\,\text{Дж}\cdot\text{с} \cdot {3 \cdot 10^{8}\,\frac{\text{м}}{\text{с}}}}{7{,}07 \cdot 10^{-19}\,\text{Дж}}
            = 281{,}16\,\text{нм}.
    $
}
\solutionspace{80pt}

\tasknumber{3}%
\task{%
    Электрон в некотором атоме переходит из стационарного состояния с энергией $-1{,}15 \cdot 10^{-19}\,\text{Дж}$
    в другое стационарного состояния с энергией $-3{,}95 \cdot 10^{-19}\,\text{Дж}$.
    \begin{itemize}
        \item Определите, происходит при этом поглощение или излучение кванта света?
        \item Чему равна энергия этого фотона?
        \item Чему равна частота этого фотона?
        \item Чему равна длина волны этого кванта света?
    \end{itemize}
}
\solutionspace{80pt}

\tasknumber{4}%
\task{%
    В некотором атоме есть 3 энергетических уровня с энергиями $E_1 = -1{,}2 \cdot 10^{-19}\,\text{Дж}$, $E_2 = -2{,}2 \cdot 10^{-19}\,\text{Дж}$ и $E_3 = -3{,}1 \cdot 10^{-19}\,\text{Дж}$.
    Электрон находится на втором ($E_2$).
    Определите энергию фотона, который может быть поглощён таким атомом.
}
\solutionspace{80pt}

\tasknumber{5}%
\task{%
    Сделайте схематичный рисунок энергетических уровней атома водорода
    и отметьте на нём первый (основной) уровень и последующие.
    Сколько различных длин волн может испустить атом водорода,
    находящийся в 4-м возбуждённом состоянии (рассмотрите и сложные переходы)?
    Отметьте все соответствующие переходы на рисунке и укажите,
    при каком переходе (среди отмеченных) частота излучённого фотона максимальна.
}
\answer{%
    $N = 6, \text{ самая длинная линия }$
}

\variantsplitter

\addpersonalvariant{Роксана Мехтиева}

\tasknumber{1}%
\task{%
    Кратко опишите модель атома Резерфорда и укажите проблемы этой теории (какой наблюдаемый эффект она не способна описать),
    сделайте необходимые рисунки.
}
\solutionspace{150pt}

\tasknumber{2}%
\task{%
    При переходе электрона в атоме с одной стационарной орбиты на другую
    излучается фотон с энергией $1{,}01 \cdot 10^{-19}\,\text{Дж}$.
    Какова длина волны этой линии спектра?
    Постоянная Планка $h = 6{,}626 \cdot 10^{-34}\,\text{Дж}\cdot\text{с}$, скорость света $c = 3 \cdot 10^{8}\,\frac{\text{м}}{\text{с}}$.
}
\answer{%
    $
        E = h\nu = h \frac c\lambda
        \implies \lambda = \frac{hc}E
            = \frac{6{,}626 \cdot 10^{-34}\,\text{Дж}\cdot\text{с} \cdot {3 \cdot 10^{8}\,\frac{\text{м}}{\text{с}}}}{1{,}01 \cdot 10^{-19}\,\text{Дж}}
            = 1968{,}1\,\text{нм}.
    $
}
\solutionspace{80pt}

\tasknumber{3}%
\task{%
    Электрон в некотором атоме переходит из стационарного состояния с энергией $-1{,}15 \cdot 10^{-19}\,\text{Дж}$
    в другое стационарного состояния с энергией $-1{,}75 \cdot 10^{-19}\,\text{Дж}$.
    \begin{itemize}
        \item Определите, происходит при этом поглощение или излучение кванта света?
        \item Чему равна энергия этого фотона?
        \item Чему равна частота этого фотона?
        \item Чему равна длина волны этого кванта света?
    \end{itemize}
}
\solutionspace{80pt}

\tasknumber{4}%
\task{%
    В некотором атоме есть 3 энергетических уровня с энергиями $E_1 = -1{,}5 \cdot 10^{-19}\,\text{Дж}$, $E_2 = -2{,}2 \cdot 10^{-19}\,\text{Дж}$ и $E_3 = -3{,}4 \cdot 10^{-19}\,\text{Дж}$.
    Электрон находится на втором ($E_2$).
    Определите энергию фотона, который может быть испущен таким атомом.
}
\solutionspace{80pt}

\tasknumber{5}%
\task{%
    Сделайте схематичный рисунок энергетических уровней атома водорода
    и отметьте на нём первый (основной) уровень и последующие.
    Сколько различных длин волн может испустить атом водорода,
    находящийся в 3-м возбуждённом состоянии (рассмотрите и сложные переходы)?
    Отметьте все соответствующие переходы на рисунке и укажите,
    при каком переходе (среди отмеченных) энергия излучённого фотона максимальна.
}
\answer{%
    $N = 3, \text{ самая длинная линия }$
}

\variantsplitter

\addpersonalvariant{Дилноза Нодиршоева}

\tasknumber{1}%
\task{%
    Кратко опишите модель атома Томсона и укажите проблемы этой теории (какой наблюдаемый эффект она не способна описать),
    сделайте необходимые рисунки.
}
\solutionspace{150pt}

\tasknumber{2}%
\task{%
    При переходе электрона в атоме с одной стационарной орбиты на другую
    излучается фотон с энергией $1{,}01 \cdot 10^{-19}\,\text{Дж}$.
    Какова длина волны этой линии спектра?
    Постоянная Планка $h = 6{,}626 \cdot 10^{-34}\,\text{Дж}\cdot\text{с}$, скорость света $c = 3 \cdot 10^{8}\,\frac{\text{м}}{\text{с}}$.
}
\answer{%
    $
        E = h\nu = h \frac c\lambda
        \implies \lambda = \frac{hc}E
            = \frac{6{,}626 \cdot 10^{-34}\,\text{Дж}\cdot\text{с} \cdot {3 \cdot 10^{8}\,\frac{\text{м}}{\text{с}}}}{1{,}01 \cdot 10^{-19}\,\text{Дж}}
            = 1968{,}1\,\text{нм}.
    $
}
\solutionspace{80pt}

\tasknumber{3}%
\task{%
    Электрон в некотором атоме переходит из стационарного состояния с энергией $-2{,}35 \cdot 10^{-19}\,\text{Дж}$
    в другое стационарного состояния с энергией $-3{,}95 \cdot 10^{-19}\,\text{Дж}$.
    \begin{itemize}
        \item Определите, происходит при этом поглощение или излучение кванта света?
        \item Чему равна энергия этого фотона?
        \item Чему равна частота этого фотона?
        \item Чему равна длина волны этого кванта света?
    \end{itemize}
}
\solutionspace{80pt}

\tasknumber{4}%
\task{%
    В некотором атоме есть 3 энергетических уровня с энергиями $E_1 = -1{,}1 \cdot 10^{-19}\,\text{Дж}$, $E_2 = -2{,}5 \cdot 10^{-19}\,\text{Дж}$ и $E_3 = -3{,}4 \cdot 10^{-19}\,\text{Дж}$.
    Электрон находится на втором ($E_2$).
    Определите энергию фотона, который может быть поглощён таким атомом.
}
\solutionspace{80pt}

\tasknumber{5}%
\task{%
    Сделайте схематичный рисунок энергетических уровней атома водорода
    и отметьте на нём первый (основной) уровень и последующие.
    Сколько различных длин волн может испустить атом водорода,
    находящийся в 5-м возбуждённом состоянии (рассмотрите и сложные переходы)?
    Отметьте все соответствующие переходы на рисунке и укажите,
    при каком переходе (среди отмеченных) длина волны излучённого фотона минимальна.
}
\answer{%
    $N = 10, \text{ самая длинная линия }$
}

\variantsplitter

\addpersonalvariant{Жаклин Пантелеева}

\tasknumber{1}%
\task{%
    Кратко опишите модель атома Бора и укажите проблемы этой теории (какой наблюдаемый эффект она не способна описать),
    сделайте необходимые рисунки.
}
\solutionspace{150pt}

\tasknumber{2}%
\task{%
    При переходе электрона в атоме с одной стационарной орбиты на другую
    излучается фотон с энергией $0{,}55 \cdot 10^{-19}\,\text{Дж}$.
    Какова длина волны этой линии спектра?
    Постоянная Планка $h = 6{,}626 \cdot 10^{-34}\,\text{Дж}\cdot\text{с}$, скорость света $c = 3 \cdot 10^{8}\,\frac{\text{м}}{\text{с}}$.
}
\answer{%
    $
        E = h\nu = h \frac c\lambda
        \implies \lambda = \frac{hc}E
            = \frac{6{,}626 \cdot 10^{-34}\,\text{Дж}\cdot\text{с} \cdot {3 \cdot 10^{8}\,\frac{\text{м}}{\text{с}}}}{0{,}55 \cdot 10^{-19}\,\text{Дж}}
            = 3614\,\text{нм}.
    $
}
\solutionspace{80pt}

\tasknumber{3}%
\task{%
    Электрон в некотором атоме переходит из стационарного состояния с энергией $-1{,}15 \cdot 10^{-19}\,\text{Дж}$
    в другое стационарного состояния с энергией $-3{,}95 \cdot 10^{-19}\,\text{Дж}$.
    \begin{itemize}
        \item Определите, происходит при этом поглощение или излучение кванта света?
        \item Чему равна энергия этого фотона?
        \item Чему равна частота этого фотона?
        \item Чему равна длина волны этого кванта света?
    \end{itemize}
}
\solutionspace{80pt}

\tasknumber{4}%
\task{%
    В некотором атоме есть 3 энергетических уровня с энергиями $E_1 = -1{,}5 \cdot 10^{-19}\,\text{Дж}$, $E_2 = -2{,}2 \cdot 10^{-19}\,\text{Дж}$ и $E_3 = -3{,}4 \cdot 10^{-19}\,\text{Дж}$.
    Электрон находится на втором ($E_2$).
    Определите энергию фотона, который может быть поглощён таким атомом.
}
\solutionspace{80pt}

\tasknumber{5}%
\task{%
    Сделайте схематичный рисунок энергетических уровней атома водорода
    и отметьте на нём первый (основной) уровень и последующие.
    Сколько различных длин волн может испустить атом водорода,
    находящийся в 4-м возбуждённом состоянии (рассмотрите и сложные переходы)?
    Отметьте все соответствующие переходы на рисунке и укажите,
    при каком переходе (среди отмеченных) энергия излучённого фотона максимальна.
}
\answer{%
    $N = 6, \text{ самая длинная линия }$
}

\variantsplitter

\addpersonalvariant{Артём Переверзев}

\tasknumber{1}%
\task{%
    Кратко опишите модель атома Бора и укажите проблемы этой теории (какой наблюдаемый эффект она не способна описать),
    сделайте необходимые рисунки.
}
\solutionspace{150pt}

\tasknumber{2}%
\task{%
    При переходе электрона в атоме с одной стационарной орбиты на другую
    излучается фотон с энергией $0{,}55 \cdot 10^{-19}\,\text{Дж}$.
    Какова длина волны этой линии спектра?
    Постоянная Планка $h = 6{,}626 \cdot 10^{-34}\,\text{Дж}\cdot\text{с}$, скорость света $c = 3 \cdot 10^{8}\,\frac{\text{м}}{\text{с}}$.
}
\answer{%
    $
        E = h\nu = h \frac c\lambda
        \implies \lambda = \frac{hc}E
            = \frac{6{,}626 \cdot 10^{-34}\,\text{Дж}\cdot\text{с} \cdot {3 \cdot 10^{8}\,\frac{\text{м}}{\text{с}}}}{0{,}55 \cdot 10^{-19}\,\text{Дж}}
            = 3614\,\text{нм}.
    $
}
\solutionspace{80pt}

\tasknumber{3}%
\task{%
    Электрон в некотором атоме переходит из стационарного состояния с энергией $-3{,}55 \cdot 10^{-19}\,\text{Дж}$
    в другое стационарного состояния с энергией $-1{,}75 \cdot 10^{-19}\,\text{Дж}$.
    \begin{itemize}
        \item Определите, происходит при этом поглощение или излучение кванта света?
        \item Чему равна энергия этого фотона?
        \item Чему равна частота этого фотона?
        \item Чему равна длина волны этого кванта света?
    \end{itemize}
}
\solutionspace{80pt}

\tasknumber{4}%
\task{%
    В некотором атоме есть 3 энергетических уровня с энергиями $E_1 = -1{,}3 \cdot 10^{-19}\,\text{Дж}$, $E_2 = -2{,}2 \cdot 10^{-19}\,\text{Дж}$ и $E_3 = -3{,}1 \cdot 10^{-19}\,\text{Дж}$.
    Электрон находится на втором ($E_2$).
    Определите энергию фотона, который может быть поглощён таким атомом.
}
\solutionspace{80pt}

\tasknumber{5}%
\task{%
    Сделайте схематичный рисунок энергетических уровней атома водорода
    и отметьте на нём первый (основной) уровень и последующие.
    Сколько различных длин волн может испустить атом водорода,
    находящийся в 3-м возбуждённом состоянии (рассмотрите и сложные переходы)?
    Отметьте все соответствующие переходы на рисунке и укажите,
    при каком переходе (среди отмеченных) длина волны излучённого фотона максимальна.
}
\answer{%
    $N = 3, \text{ самая короткая линия }$
}

\variantsplitter

\addpersonalvariant{Варвара Пранова}

\tasknumber{1}%
\task{%
    Кратко опишите планетарную модель атома и укажите проблемы этой теории (какой наблюдаемый эффект она не способна описать),
    сделайте необходимые рисунки.
}
\solutionspace{150pt}

\tasknumber{2}%
\task{%
    При переходе электрона в атоме с одной стационарной орбиты на другую
    излучается фотон с энергией $1{,}01 \cdot 10^{-19}\,\text{Дж}$.
    Какова длина волны этой линии спектра?
    Постоянная Планка $h = 6{,}626 \cdot 10^{-34}\,\text{Дж}\cdot\text{с}$, скорость света $c = 3 \cdot 10^{8}\,\frac{\text{м}}{\text{с}}$.
}
\answer{%
    $
        E = h\nu = h \frac c\lambda
        \implies \lambda = \frac{hc}E
            = \frac{6{,}626 \cdot 10^{-34}\,\text{Дж}\cdot\text{с} \cdot {3 \cdot 10^{8}\,\frac{\text{м}}{\text{с}}}}{1{,}01 \cdot 10^{-19}\,\text{Дж}}
            = 1968{,}1\,\text{нм}.
    $
}
\solutionspace{80pt}

\tasknumber{3}%
\task{%
    Электрон в некотором атоме переходит из стационарного состояния с энергией $-1{,}15 \cdot 10^{-19}\,\text{Дж}$
    в другое стационарного состояния с энергией $-3{,}95 \cdot 10^{-19}\,\text{Дж}$.
    \begin{itemize}
        \item Определите, происходит при этом поглощение или излучение кванта света?
        \item Чему равна энергия этого фотона?
        \item Чему равна частота этого фотона?
        \item Чему равна длина волны этого кванта света?
    \end{itemize}
}
\solutionspace{80pt}

\tasknumber{4}%
\task{%
    В некотором атоме есть 3 энергетических уровня с энергиями $E_1 = -1{,}5 \cdot 10^{-19}\,\text{Дж}$, $E_2 = -2{,}9 \cdot 10^{-19}\,\text{Дж}$ и $E_3 = -3{,}1 \cdot 10^{-19}\,\text{Дж}$.
    Электрон находится на втором ($E_2$).
    Определите энергию фотона, который может быть поглощён таким атомом.
}
\solutionspace{80pt}

\tasknumber{5}%
\task{%
    Сделайте схематичный рисунок энергетических уровней атома водорода
    и отметьте на нём первый (основной) уровень и последующие.
    Сколько различных длин волн может испустить атом водорода,
    находящийся в 5-м возбуждённом состоянии (рассмотрите и сложные переходы)?
    Отметьте все соответствующие переходы на рисунке и укажите,
    при каком переходе (среди отмеченных) частота излучённого фотона минимальна.
}
\answer{%
    $N = 10, \text{ самая короткая линия }$
}

\variantsplitter

\addpersonalvariant{Марьям Салимова}

\tasknumber{1}%
\task{%
    Кратко опишите модель атома Резерфорда и укажите проблемы этой теории (какой наблюдаемый эффект она не способна описать),
    сделайте необходимые рисунки.
}
\solutionspace{150pt}

\tasknumber{2}%
\task{%
    При переходе электрона в атоме с одной стационарной орбиты на другую
    излучается фотон с энергией $4{,}04 \cdot 10^{-19}\,\text{Дж}$.
    Какова длина волны этой линии спектра?
    Постоянная Планка $h = 6{,}626 \cdot 10^{-34}\,\text{Дж}\cdot\text{с}$, скорость света $c = 3 \cdot 10^{8}\,\frac{\text{м}}{\text{с}}$.
}
\answer{%
    $
        E = h\nu = h \frac c\lambda
        \implies \lambda = \frac{hc}E
            = \frac{6{,}626 \cdot 10^{-34}\,\text{Дж}\cdot\text{с} \cdot {3 \cdot 10^{8}\,\frac{\text{м}}{\text{с}}}}{4{,}04 \cdot 10^{-19}\,\text{Дж}}
            = 492{,}03\,\text{нм}.
    $
}
\solutionspace{80pt}

\tasknumber{3}%
\task{%
    Электрон в некотором атоме переходит из стационарного состояния с энергией $-2{,}35 \cdot 10^{-19}\,\text{Дж}$
    в другое стационарного состояния с энергией $-2{,}85 \cdot 10^{-19}\,\text{Дж}$.
    \begin{itemize}
        \item Определите, происходит при этом поглощение или излучение кванта света?
        \item Чему равна энергия этого фотона?
        \item Чему равна частота этого фотона?
        \item Чему равна длина волны этого кванта света?
    \end{itemize}
}
\solutionspace{80pt}

\tasknumber{4}%
\task{%
    В некотором атоме есть 3 энергетических уровня с энергиями $E_1 = -1{,}2 \cdot 10^{-19}\,\text{Дж}$, $E_2 = -2{,}2 \cdot 10^{-19}\,\text{Дж}$ и $E_3 = -3{,}4 \cdot 10^{-19}\,\text{Дж}$.
    Электрон находится на втором ($E_2$).
    Определите энергию фотона, который может быть испущен таким атомом.
}
\solutionspace{80pt}

\tasknumber{5}%
\task{%
    Сделайте схематичный рисунок энергетических уровней атома водорода
    и отметьте на нём первый (основной) уровень и последующие.
    Сколько различных длин волн может испустить атом водорода,
    находящийся в 5-м возбуждённом состоянии (рассмотрите и сложные переходы)?
    Отметьте все соответствующие переходы на рисунке и укажите,
    при каком переходе (среди отмеченных) частота излучённого фотона минимальна.
}
\answer{%
    $N = 10, \text{ самая короткая линия }$
}

\variantsplitter

\addpersonalvariant{Юлия Шевченко}

\tasknumber{1}%
\task{%
    Кратко опишите модель атома Бора и укажите проблемы этой теории (какой наблюдаемый эффект она не способна описать),
    сделайте необходимые рисунки.
}
\solutionspace{150pt}

\tasknumber{2}%
\task{%
    При переходе электрона в атоме с одной стационарной орбиты на другую
    излучается фотон с энергией $4{,}04 \cdot 10^{-19}\,\text{Дж}$.
    Какова длина волны этой линии спектра?
    Постоянная Планка $h = 6{,}626 \cdot 10^{-34}\,\text{Дж}\cdot\text{с}$, скорость света $c = 3 \cdot 10^{8}\,\frac{\text{м}}{\text{с}}$.
}
\answer{%
    $
        E = h\nu = h \frac c\lambda
        \implies \lambda = \frac{hc}E
            = \frac{6{,}626 \cdot 10^{-34}\,\text{Дж}\cdot\text{с} \cdot {3 \cdot 10^{8}\,\frac{\text{м}}{\text{с}}}}{4{,}04 \cdot 10^{-19}\,\text{Дж}}
            = 492{,}03\,\text{нм}.
    $
}
\solutionspace{80pt}

\tasknumber{3}%
\task{%
    Электрон в некотором атоме переходит из стационарного состояния с энергией $-1{,}15 \cdot 10^{-19}\,\text{Дж}$
    в другое стационарного состояния с энергией $-2{,}85 \cdot 10^{-19}\,\text{Дж}$.
    \begin{itemize}
        \item Определите, происходит при этом поглощение или излучение кванта света?
        \item Чему равна энергия этого фотона?
        \item Чему равна частота этого фотона?
        \item Чему равна длина волны этого кванта света?
    \end{itemize}
}
\solutionspace{80pt}

\tasknumber{4}%
\task{%
    В некотором атоме есть 3 энергетических уровня с энергиями $E_1 = -1{,}3 \cdot 10^{-19}\,\text{Дж}$, $E_2 = -2{,}9 \cdot 10^{-19}\,\text{Дж}$ и $E_3 = -3{,}1 \cdot 10^{-19}\,\text{Дж}$.
    Электрон находится на втором ($E_2$).
    Определите энергию фотона, который может быть поглощён таким атомом.
}
\solutionspace{80pt}

\tasknumber{5}%
\task{%
    Сделайте схематичный рисунок энергетических уровней атома водорода
    и отметьте на нём первый (основной) уровень и последующие.
    Сколько различных длин волн может испустить атом водорода,
    находящийся в 3-м возбуждённом состоянии (рассмотрите и сложные переходы)?
    Отметьте все соответствующие переходы на рисунке и укажите,
    при каком переходе (среди отмеченных) частота излучённого фотона минимальна.
}
\answer{%
    $N = 3, \text{ самая короткая линия }$
}
% autogenerated
