\setdate{10~марта~2022}
\setclass{11«БА»}

\addpersonalvariant{Михаил Бурмистров}

\tasknumber{1}%
\task{%
    Определите длину волны (в нм) света, которым освещается поверхность металла,
    если фотоэлектроны имеют максимальную кинетическую энергию $5 \cdot 10^{-20}\,\text{Дж}$,
    а работа выхода электронов из этого металла $11 \cdot 10^{-19}\,\text{Дж}$.
    Постоянная Планка $h = 6{,}626 \cdot 10^{-34}\,\text{Дж}\cdot\text{с}$.
}
\answer{%
    $h \frac c\lambda = A_{\text{вых.}} + E_{\text{кин.}} \implies \lambda = \frac{h c}{A_{\text{вых.}} + E_{\text{кин.}}} = \frac{ 6{,}626 \cdot 10^{-34}\,\text{Дж}\cdot\text{с} \cdot {Const.c:V} }{11 \cdot 10^{-19}\,\text{Дж} + 5 \cdot 10^{-20}\,\text{Дж}} \approx 0{,}1729 \cdot 10^{-6}\,\text{м}.$
}
\solutionspace{80pt}

\tasknumber{2}%
\task{%
    Работа выхода электронов из некоторого металла $4{,}3\,\text{эВ}$.
    Найдите скорость электронов (в км/с),
    вылетающих с поверхности металла при освещении его светом с длиной волны $2{,}7 \cdot 10^{-5}\,\text{см}$.
    Масса электрона $m_{e} = 9{,}1 \cdot 10^{-31}\,\text{кг}$.
    Постоянная Планка $h = 6{,}626 \cdot 10^{-34}\,\text{Дж}\cdot\text{с}$, заряд электрона $e = 1{,}6 \cdot 10^{-19}\,\text{Кл}$.
}
\answer{%
    $h \frac c\lambda = A_{\text{вых.}} + \frac{ m_{e}v^2 }2 \implies v = \sqrt{ \frac 2{m_{e}}\cbr{ h \frac c\lambda - A_{\text{вых.}} } } \approx 325{,}6\,\frac{\text{км}}{\text{c}}.$
}
\solutionspace{80pt}

\tasknumber{3}%
\task{%
    Сколько фотонов испускает за $5\,\text{мин}$ лазер,
    если мощность его излучения $75\,\text{мВт}$?
    Длина волны излучения $600\,\text{нм}$.
    $h = 6{,}626 \cdot 10^{-34}\,\text{Дж}\cdot\text{с}$.
}
\answer{%
    $
        N
            = \frac{E_{\text{общая}}}{E_{\text{одного фотона}}}
            = \frac{Pt}{h\nu} = \frac{Pt}{h \frac c\lambda}
            = \frac{Pt\lambda}{hc}
            = \frac{75\,\text{мВт} \cdot 5\,\text{мин} \cdot 600\,\text{нм}}{6{,}626 \cdot 10^{-34}\,\text{Дж}\cdot\text{с} \cdot 3 \cdot 10^{8}\,\frac{\text{м}}{\text{с}}}
            \approx 67{,}9 \cdot 10^{18}\units{фотонов}
    $
}
\solutionspace{120pt}

\tasknumber{4}%
\task{%
    Определите энергию фотона излучения частотой $9 \cdot 10^{16}\,\text{Гц}$.
    Ответ получите в джоулях и в электронвольтах.
}
\answer{%
    $E = h \nu = 6{,}626 \cdot 10^{-34}\,\text{Дж}\cdot\text{с} \cdot 9 \cdot 10^{16}\,\text{Гц} \approx 60 \cdot 10^{-18}\,\text{Дж} \approx 370\,\text{эВ}$
}
\solutionspace{80pt}

\tasknumber{5}%
\task{%
    Определите энергию кванта света с длиной волны $700\,\text{нм}$.
    Ответ выразите в электронвольтах.
    Способен ли человеческий глаз увидеть один такой квант, а импульс таких квантов?'
}
\answer{%
    $E = h\nu = \frac{hc}{\lambda} = \frac{6{,}626 \cdot 10^{-34}\,\text{Дж}\cdot\text{с} \cdot 3 \cdot 10^{8}\,\frac{\text{м}}{\text{с}}}{700\,\text{нм}} \approx 0{,}284 \cdot 10^{-18}\,\text{Дж} \approx 1{,}775\,\text{эВ}$
}
\solutionspace{80pt}

\tasknumber{6}%
\task{%
    Определите длину волны лучей, фотоны которых имеют энергию
    равную кинетической энергии электрона, ускоренного напряжением $5\,\text{В}$.
}

\variantsplitter

\addpersonalvariant{Ирина Ан}

\tasknumber{1}%
\task{%
    Определите длину волны (в нм) света, которым освещается поверхность металла,
    если фотоэлектроны имеют максимальную кинетическую энергию $9 \cdot 10^{-20}\,\text{Дж}$,
    а работа выхода электронов из этого металла $11 \cdot 10^{-19}\,\text{Дж}$.
    Постоянная Планка $h = 6{,}626 \cdot 10^{-34}\,\text{Дж}\cdot\text{с}$.
}
\answer{%
    $h \frac c\lambda = A_{\text{вых.}} + E_{\text{кин.}} \implies \lambda = \frac{h c}{A_{\text{вых.}} + E_{\text{кин.}}} = \frac{ 6{,}626 \cdot 10^{-34}\,\text{Дж}\cdot\text{с} \cdot {Const.c:V} }{11 \cdot 10^{-19}\,\text{Дж} + 9 \cdot 10^{-20}\,\text{Дж}} \approx 0{,}1670 \cdot 10^{-6}\,\text{м}.$
}
\solutionspace{80pt}

\tasknumber{2}%
\task{%
    Работа выхода электронов из некоторого металла $2{,}1\,\text{эВ}$.
    Найдите скорость электронов (в км/с),
    вылетающих с поверхности металла при освещении его светом с длиной волны $1{,}7 \cdot 10^{-5}\,\text{см}$.
    Масса электрона $m_{e} = 9{,}1 \cdot 10^{-31}\,\text{кг}$.
    Постоянная Планка $h = 6{,}626 \cdot 10^{-34}\,\text{Дж}\cdot\text{с}$, заряд электрона $e = 1{,}6 \cdot 10^{-19}\,\text{Кл}$.
}
\answer{%
    $h \frac c\lambda = A_{\text{вых.}} + \frac{ m_{e}v^2 }2 \implies v = \sqrt{ \frac 2{m_{e}}\cbr{ h \frac c\lambda - A_{\text{вых.}} } } \approx 1353{,}3\,\frac{\text{км}}{\text{c}}.$
}
\solutionspace{80pt}

\tasknumber{3}%
\task{%
    Сколько фотонов испускает за $120\,\text{мин}$ лазер,
    если мощность его излучения $15\,\text{мВт}$?
    Длина волны излучения $500\,\text{нм}$.
    $h = 6{,}626 \cdot 10^{-34}\,\text{Дж}\cdot\text{с}$.
}
\answer{%
    $
        N
            = \frac{E_{\text{общая}}}{E_{\text{одного фотона}}}
            = \frac{Pt}{h\nu} = \frac{Pt}{h \frac c\lambda}
            = \frac{Pt\lambda}{hc}
            = \frac{15\,\text{мВт} \cdot 120\,\text{мин} \cdot 500\,\text{нм}}{6{,}626 \cdot 10^{-34}\,\text{Дж}\cdot\text{с} \cdot 3 \cdot 10^{8}\,\frac{\text{м}}{\text{с}}}
            \approx 272 \cdot 10^{18}\units{фотонов}
    $
}
\solutionspace{120pt}

\tasknumber{4}%
\task{%
    Определите энергию фотона излучения частотой $9 \cdot 10^{16}\,\text{Гц}$.
    Ответ получите в джоулях и в электронвольтах.
}
\answer{%
    $E = h \nu = 6{,}626 \cdot 10^{-34}\,\text{Дж}\cdot\text{с} \cdot 9 \cdot 10^{16}\,\text{Гц} \approx 60 \cdot 10^{-18}\,\text{Дж} \approx 370\,\text{эВ}$
}
\solutionspace{80pt}

\tasknumber{5}%
\task{%
    Определите энергию кванта света с длиной волны $150\,\text{нм}$.
    Ответ выразите в джоулях.
    Способен ли человеческий глаз увидеть один такой квант, а импульс таких квантов?'
}
\answer{%
    $E = h\nu = \frac{hc}{\lambda} = \frac{6{,}626 \cdot 10^{-34}\,\text{Дж}\cdot\text{с} \cdot 3 \cdot 10^{8}\,\frac{\text{м}}{\text{с}}}{150\,\text{нм}} \approx 1{,}33 \cdot 10^{-18}\,\text{Дж} \approx 8{,}3\,\text{эВ}$
}
\solutionspace{80pt}

\tasknumber{6}%
\task{%
    Определите длину волны лучей, фотоны которых имеют энергию
    равную кинетической энергии электрона, ускоренного напряжением $34\,\text{В}$.
}

\variantsplitter

\addpersonalvariant{Софья Андрианова}

\tasknumber{1}%
\task{%
    Определите длину волны (в нм) света, которым освещается поверхность металла,
    если фотоэлектроны имеют максимальную кинетическую энергию $9 \cdot 10^{-20}\,\text{Дж}$,
    а работа выхода электронов из этого металла $9 \cdot 10^{-19}\,\text{Дж}$.
    Постоянная Планка $h = 6{,}626 \cdot 10^{-34}\,\text{Дж}\cdot\text{с}$.
}
\answer{%
    $h \frac c\lambda = A_{\text{вых.}} + E_{\text{кин.}} \implies \lambda = \frac{h c}{A_{\text{вых.}} + E_{\text{кин.}}} = \frac{ 6{,}626 \cdot 10^{-34}\,\text{Дж}\cdot\text{с} \cdot {Const.c:V} }{9 \cdot 10^{-19}\,\text{Дж} + 9 \cdot 10^{-20}\,\text{Дж}} \approx 0{,}20 \cdot 10^{-6}\,\text{м}.$
}
\solutionspace{80pt}

\tasknumber{2}%
\task{%
    Работа выхода электронов из некоторого металла $4{,}3\,\text{эВ}$.
    Найдите скорость электронов (в км/с),
    вылетающих с поверхности металла при освещении его светом с длиной волны $1{,}7 \cdot 10^{-5}\,\text{см}$.
    Масса электрона $m_{e} = 9{,}1 \cdot 10^{-31}\,\text{кг}$.
    Постоянная Планка $h = 6{,}626 \cdot 10^{-34}\,\text{Дж}\cdot\text{с}$, заряд электрона $e = 1{,}6 \cdot 10^{-19}\,\text{Кл}$.
}
\answer{%
    $h \frac c\lambda = A_{\text{вых.}} + \frac{ m_{e}v^2 }2 \implies v = \sqrt{ \frac 2{m_{e}}\cbr{ h \frac c\lambda - A_{\text{вых.}} } } \approx 1028{,}5\,\frac{\text{км}}{\text{c}}.$
}
\solutionspace{80pt}

\tasknumber{3}%
\task{%
    Сколько фотонов испускает за $5\,\text{мин}$ лазер,
    если мощность его излучения $200\,\text{мВт}$?
    Длина волны излучения $600\,\text{нм}$.
    $h = 6{,}626 \cdot 10^{-34}\,\text{Дж}\cdot\text{с}$.
}
\answer{%
    $
        N
            = \frac{E_{\text{общая}}}{E_{\text{одного фотона}}}
            = \frac{Pt}{h\nu} = \frac{Pt}{h \frac c\lambda}
            = \frac{Pt\lambda}{hc}
            = \frac{200\,\text{мВт} \cdot 5\,\text{мин} \cdot 600\,\text{нм}}{6{,}626 \cdot 10^{-34}\,\text{Дж}\cdot\text{с} \cdot 3 \cdot 10^{8}\,\frac{\text{м}}{\text{с}}}
            \approx 181{,}1 \cdot 10^{18}\units{фотонов}
    $
}
\solutionspace{120pt}

\tasknumber{4}%
\task{%
    Определите энергию фотона излучения частотой $8 \cdot 10^{16}\,\text{Гц}$.
    Ответ получите в джоулях и в электронвольтах.
}
\answer{%
    $E = h \nu = 6{,}626 \cdot 10^{-34}\,\text{Дж}\cdot\text{с} \cdot 8 \cdot 10^{16}\,\text{Гц} \approx 53 \cdot 10^{-18}\,\text{Дж} \approx 330\,\text{эВ}$
}
\solutionspace{80pt}

\tasknumber{5}%
\task{%
    Определите энергию фотона с длиной волны $500\,\text{нм}$.
    Ответ выразите в джоулях.
    Способен ли человеческий глаз увидеть один такой квант, а импульс таких квантов?'
}
\answer{%
    $E = h\nu = \frac{hc}{\lambda} = \frac{6{,}626 \cdot 10^{-34}\,\text{Дж}\cdot\text{с} \cdot 3 \cdot 10^{8}\,\frac{\text{м}}{\text{с}}}{500\,\text{нм}} \approx 0{,}398 \cdot 10^{-18}\,\text{Дж} \approx 2{,}48\,\text{эВ}$
}
\solutionspace{80pt}

\tasknumber{6}%
\task{%
    Определите длину волны лучей, фотоны которых имеют энергию
    равную кинетической энергии электрона, ускоренного напряжением $3\,\text{В}$.
}

\variantsplitter

\addpersonalvariant{Владимир Артемчук}

\tasknumber{1}%
\task{%
    Определите длину волны (в нм) света, которым освещается поверхность металла,
    если фотоэлектроны имеют максимальную кинетическую энергию $3 \cdot 10^{-20}\,\text{Дж}$,
    а работа выхода электронов из этого металла $11 \cdot 10^{-19}\,\text{Дж}$.
    Постоянная Планка $h = 6{,}626 \cdot 10^{-34}\,\text{Дж}\cdot\text{с}$.
}
\answer{%
    $h \frac c\lambda = A_{\text{вых.}} + E_{\text{кин.}} \implies \lambda = \frac{h c}{A_{\text{вых.}} + E_{\text{кин.}}} = \frac{ 6{,}626 \cdot 10^{-34}\,\text{Дж}\cdot\text{с} \cdot {Const.c:V} }{11 \cdot 10^{-19}\,\text{Дж} + 3 \cdot 10^{-20}\,\text{Дж}} \approx 0{,}1759 \cdot 10^{-6}\,\text{м}.$
}
\solutionspace{80pt}

\tasknumber{2}%
\task{%
    Работа выхода электронов из некоторого металла $2{,}1\,\text{эВ}$.
    Найдите скорость электронов (в км/с),
    вылетающих с поверхности металла при освещении его светом с длиной волны $2{,}7 \cdot 10^{-5}\,\text{см}$.
    Масса электрона $m_{e} = 9{,}1 \cdot 10^{-31}\,\text{кг}$.
    Постоянная Планка $h = 6{,}626 \cdot 10^{-34}\,\text{Дж}\cdot\text{с}$, заряд электрона $e = 1{,}6 \cdot 10^{-19}\,\text{Кл}$.
}
\answer{%
    $h \frac c\lambda = A_{\text{вых.}} + \frac{ m_{e}v^2 }2 \implies v = \sqrt{ \frac 2{m_{e}}\cbr{ h \frac c\lambda - A_{\text{вых.}} } } \approx 937{,}9\,\frac{\text{км}}{\text{c}}.$
}
\solutionspace{80pt}

\tasknumber{3}%
\task{%
    Сколько фотонов испускает за $20\,\text{мин}$ лазер,
    если мощность его излучения $75\,\text{мВт}$?
    Длина волны излучения $750\,\text{нм}$.
    $h = 6{,}626 \cdot 10^{-34}\,\text{Дж}\cdot\text{с}$.
}
\answer{%
    $
        N
            = \frac{E_{\text{общая}}}{E_{\text{одного фотона}}}
            = \frac{Pt}{h\nu} = \frac{Pt}{h \frac c\lambda}
            = \frac{Pt\lambda}{hc}
            = \frac{75\,\text{мВт} \cdot 20\,\text{мин} \cdot 750\,\text{нм}}{6{,}626 \cdot 10^{-34}\,\text{Дж}\cdot\text{с} \cdot 3 \cdot 10^{8}\,\frac{\text{м}}{\text{с}}}
            \approx 339{,}6 \cdot 10^{18}\units{фотонов}
    $
}
\solutionspace{120pt}

\tasknumber{4}%
\task{%
    Определите энергию фотона излучения частотой $8 \cdot 10^{16}\,\text{Гц}$.
    Ответ получите в джоулях и в электронвольтах.
}
\answer{%
    $E = h \nu = 6{,}626 \cdot 10^{-34}\,\text{Дж}\cdot\text{с} \cdot 8 \cdot 10^{16}\,\text{Гц} \approx 53 \cdot 10^{-18}\,\text{Дж} \approx 330\,\text{эВ}$
}
\solutionspace{80pt}

\tasknumber{5}%
\task{%
    Определите энергию кванта света с длиной волны $700\,\text{нм}$.
    Ответ выразите в электронвольтах.
    Способен ли человеческий глаз увидеть один такой квант, а импульс таких квантов?'
}
\answer{%
    $E = h\nu = \frac{hc}{\lambda} = \frac{6{,}626 \cdot 10^{-34}\,\text{Дж}\cdot\text{с} \cdot 3 \cdot 10^{8}\,\frac{\text{м}}{\text{с}}}{700\,\text{нм}} \approx 0{,}284 \cdot 10^{-18}\,\text{Дж} \approx 1{,}775\,\text{эВ}$
}
\solutionspace{80pt}

\tasknumber{6}%
\task{%
    Определите длину волны лучей, фотоны которых имеют энергию
    равную кинетической энергии электрона, ускоренного напряжением $1\,\text{В}$.
}

\variantsplitter

\addpersonalvariant{Софья Белянкина}

\tasknumber{1}%
\task{%
    Определите длину волны (в нм) света, которым освещается поверхность металла,
    если фотоэлектроны имеют максимальную кинетическую энергию $9 \cdot 10^{-20}\,\text{Дж}$,
    а работа выхода электронов из этого металла $13 \cdot 10^{-19}\,\text{Дж}$.
    Постоянная Планка $h = 6{,}626 \cdot 10^{-34}\,\text{Дж}\cdot\text{с}$.
}
\answer{%
    $h \frac c\lambda = A_{\text{вых.}} + E_{\text{кин.}} \implies \lambda = \frac{h c}{A_{\text{вых.}} + E_{\text{кин.}}} = \frac{ 6{,}626 \cdot 10^{-34}\,\text{Дж}\cdot\text{с} \cdot {Const.c:V} }{13 \cdot 10^{-19}\,\text{Дж} + 9 \cdot 10^{-20}\,\text{Дж}} \approx 0{,}1430 \cdot 10^{-6}\,\text{м}.$
}
\solutionspace{80pt}

\tasknumber{2}%
\task{%
    Работа выхода электронов из некоторого металла $4{,}3\,\text{эВ}$.
    Найдите скорость электронов (в км/с),
    вылетающих с поверхности металла при освещении его светом с длиной волны $1{,}7 \cdot 10^{-5}\,\text{см}$.
    Масса электрона $m_{e} = 9{,}1 \cdot 10^{-31}\,\text{кг}$.
    Постоянная Планка $h = 6{,}626 \cdot 10^{-34}\,\text{Дж}\cdot\text{с}$, заряд электрона $e = 1{,}6 \cdot 10^{-19}\,\text{Кл}$.
}
\answer{%
    $h \frac c\lambda = A_{\text{вых.}} + \frac{ m_{e}v^2 }2 \implies v = \sqrt{ \frac 2{m_{e}}\cbr{ h \frac c\lambda - A_{\text{вых.}} } } \approx 1028{,}5\,\frac{\text{км}}{\text{c}}.$
}
\solutionspace{80pt}

\tasknumber{3}%
\task{%
    Сколько фотонов испускает за $20\,\text{мин}$ лазер,
    если мощность его излучения $75\,\text{мВт}$?
    Длина волны излучения $750\,\text{нм}$.
    $h = 6{,}626 \cdot 10^{-34}\,\text{Дж}\cdot\text{с}$.
}
\answer{%
    $
        N
            = \frac{E_{\text{общая}}}{E_{\text{одного фотона}}}
            = \frac{Pt}{h\nu} = \frac{Pt}{h \frac c\lambda}
            = \frac{Pt\lambda}{hc}
            = \frac{75\,\text{мВт} \cdot 20\,\text{мин} \cdot 750\,\text{нм}}{6{,}626 \cdot 10^{-34}\,\text{Дж}\cdot\text{с} \cdot 3 \cdot 10^{8}\,\frac{\text{м}}{\text{с}}}
            \approx 339{,}6 \cdot 10^{18}\units{фотонов}
    $
}
\solutionspace{120pt}

\tasknumber{4}%
\task{%
    Определите энергию фотона излучения частотой $4 \cdot 10^{16}\,\text{Гц}$.
    Ответ получите в джоулях и в электронвольтах.
}
\answer{%
    $E = h \nu = 6{,}626 \cdot 10^{-34}\,\text{Дж}\cdot\text{с} \cdot 4 \cdot 10^{16}\,\text{Гц} \approx 27 \cdot 10^{-18}\,\text{Дж} \approx 166\,\text{эВ}$
}
\solutionspace{80pt}

\tasknumber{5}%
\task{%
    Определите энергию кванта света с длиной волны $150\,\text{нм}$.
    Ответ выразите в электронвольтах.
    Способен ли человеческий глаз увидеть один такой квант, а импульс таких квантов?'
}
\answer{%
    $E = h\nu = \frac{hc}{\lambda} = \frac{6{,}626 \cdot 10^{-34}\,\text{Дж}\cdot\text{с} \cdot 3 \cdot 10^{8}\,\frac{\text{м}}{\text{с}}}{150\,\text{нм}} \approx 1{,}33 \cdot 10^{-18}\,\text{Дж} \approx 8{,}3\,\text{эВ}$
}
\solutionspace{80pt}

\tasknumber{6}%
\task{%
    Определите длину волны лучей, фотоны которых имеют энергию
    равную кинетической энергии электрона, ускоренного напряжением $144\,\text{В}$.
}

\variantsplitter

\addpersonalvariant{Варвара Егиазарян}

\tasknumber{1}%
\task{%
    Определите длину волны (в нм) света, которым освещается поверхность металла,
    если фотоэлектроны имеют максимальную кинетическую энергию $4 \cdot 10^{-20}\,\text{Дж}$,
    а работа выхода электронов из этого металла $11 \cdot 10^{-19}\,\text{Дж}$.
    Постоянная Планка $h = 6{,}626 \cdot 10^{-34}\,\text{Дж}\cdot\text{с}$.
}
\answer{%
    $h \frac c\lambda = A_{\text{вых.}} + E_{\text{кин.}} \implies \lambda = \frac{h c}{A_{\text{вых.}} + E_{\text{кин.}}} = \frac{ 6{,}626 \cdot 10^{-34}\,\text{Дж}\cdot\text{с} \cdot {Const.c:V} }{11 \cdot 10^{-19}\,\text{Дж} + 4 \cdot 10^{-20}\,\text{Дж}} \approx 0{,}1744 \cdot 10^{-6}\,\text{м}.$
}
\solutionspace{80pt}

\tasknumber{2}%
\task{%
    Работа выхода электронов из некоторого металла $4{,}3\,\text{эВ}$.
    Найдите скорость электронов (в км/с),
    вылетающих с поверхности металла при освещении его светом с длиной волны $1{,}7 \cdot 10^{-5}\,\text{см}$.
    Масса электрона $m_{e} = 9{,}1 \cdot 10^{-31}\,\text{кг}$.
    Постоянная Планка $h = 6{,}626 \cdot 10^{-34}\,\text{Дж}\cdot\text{с}$, заряд электрона $e = 1{,}6 \cdot 10^{-19}\,\text{Кл}$.
}
\answer{%
    $h \frac c\lambda = A_{\text{вых.}} + \frac{ m_{e}v^2 }2 \implies v = \sqrt{ \frac 2{m_{e}}\cbr{ h \frac c\lambda - A_{\text{вых.}} } } \approx 1028{,}5\,\frac{\text{км}}{\text{c}}.$
}
\solutionspace{80pt}

\tasknumber{3}%
\task{%
    Сколько фотонов испускает за $5\,\text{мин}$ лазер,
    если мощность его излучения $200\,\text{мВт}$?
    Длина волны излучения $600\,\text{нм}$.
    $h = 6{,}626 \cdot 10^{-34}\,\text{Дж}\cdot\text{с}$.
}
\answer{%
    $
        N
            = \frac{E_{\text{общая}}}{E_{\text{одного фотона}}}
            = \frac{Pt}{h\nu} = \frac{Pt}{h \frac c\lambda}
            = \frac{Pt\lambda}{hc}
            = \frac{200\,\text{мВт} \cdot 5\,\text{мин} \cdot 600\,\text{нм}}{6{,}626 \cdot 10^{-34}\,\text{Дж}\cdot\text{с} \cdot 3 \cdot 10^{8}\,\frac{\text{м}}{\text{с}}}
            \approx 181{,}1 \cdot 10^{18}\units{фотонов}
    $
}
\solutionspace{120pt}

\tasknumber{4}%
\task{%
    Определите энергию фотона излучения частотой $9 \cdot 10^{16}\,\text{Гц}$.
    Ответ получите в джоулях и в электронвольтах.
}
\answer{%
    $E = h \nu = 6{,}626 \cdot 10^{-34}\,\text{Дж}\cdot\text{с} \cdot 9 \cdot 10^{16}\,\text{Гц} \approx 60 \cdot 10^{-18}\,\text{Дж} \approx 370\,\text{эВ}$
}
\solutionspace{80pt}

\tasknumber{5}%
\task{%
    Определите энергию кванта света с длиной волны $150\,\text{нм}$.
    Ответ выразите в джоулях.
    Способен ли человеческий глаз увидеть один такой квант, а импульс таких квантов?'
}
\answer{%
    $E = h\nu = \frac{hc}{\lambda} = \frac{6{,}626 \cdot 10^{-34}\,\text{Дж}\cdot\text{с} \cdot 3 \cdot 10^{8}\,\frac{\text{м}}{\text{с}}}{150\,\text{нм}} \approx 1{,}33 \cdot 10^{-18}\,\text{Дж} \approx 8{,}3\,\text{эВ}$
}
\solutionspace{80pt}

\tasknumber{6}%
\task{%
    Определите длину волны лучей, фотоны которых имеют энергию
    равную кинетической энергии электрона, ускоренного напряжением $377\,\text{В}$.
}

\variantsplitter

\addpersonalvariant{Владислав Емелин}

\tasknumber{1}%
\task{%
    Определите длину волны (в нм) света, которым освещается поверхность металла,
    если фотоэлектроны имеют максимальную кинетическую энергию $9 \cdot 10^{-20}\,\text{Дж}$,
    а работа выхода электронов из этого металла $7 \cdot 10^{-19}\,\text{Дж}$.
    Постоянная Планка $h = 6{,}626 \cdot 10^{-34}\,\text{Дж}\cdot\text{с}$.
}
\answer{%
    $h \frac c\lambda = A_{\text{вых.}} + E_{\text{кин.}} \implies \lambda = \frac{h c}{A_{\text{вых.}} + E_{\text{кин.}}} = \frac{ 6{,}626 \cdot 10^{-34}\,\text{Дж}\cdot\text{с} \cdot {Const.c:V} }{7 \cdot 10^{-19}\,\text{Дж} + 9 \cdot 10^{-20}\,\text{Дж}} \approx 0{,}25 \cdot 10^{-6}\,\text{м}.$
}
\solutionspace{80pt}

\tasknumber{2}%
\task{%
    Работа выхода электронов из некоторого металла $4{,}3\,\text{эВ}$.
    Найдите скорость электронов (в км/с),
    вылетающих с поверхности металла при освещении его светом с длиной волны $2{,}2 \cdot 10^{-5}\,\text{см}$.
    Масса электрона $m_{e} = 9{,}1 \cdot 10^{-31}\,\text{кг}$.
    Постоянная Планка $h = 6{,}626 \cdot 10^{-34}\,\text{Дж}\cdot\text{с}$, заряд электрона $e = 1{,}6 \cdot 10^{-19}\,\text{Кл}$.
}
\answer{%
    $h \frac c\lambda = A_{\text{вых.}} + \frac{ m_{e}v^2 }2 \implies v = \sqrt{ \frac 2{m_{e}}\cbr{ h \frac c\lambda - A_{\text{вых.}} } } \approx 688{,}3\,\frac{\text{км}}{\text{c}}.$
}
\solutionspace{80pt}

\tasknumber{3}%
\task{%
    Сколько фотонов испускает за $10\,\text{мин}$ лазер,
    если мощность его излучения $15\,\text{мВт}$?
    Длина волны излучения $600\,\text{нм}$.
    $h = 6{,}626 \cdot 10^{-34}\,\text{Дж}\cdot\text{с}$.
}
\answer{%
    $
        N
            = \frac{E_{\text{общая}}}{E_{\text{одного фотона}}}
            = \frac{Pt}{h\nu} = \frac{Pt}{h \frac c\lambda}
            = \frac{Pt\lambda}{hc}
            = \frac{15\,\text{мВт} \cdot 10\,\text{мин} \cdot 600\,\text{нм}}{6{,}626 \cdot 10^{-34}\,\text{Дж}\cdot\text{с} \cdot 3 \cdot 10^{8}\,\frac{\text{м}}{\text{с}}}
            \approx 27{,}2 \cdot 10^{18}\units{фотонов}
    $
}
\solutionspace{120pt}

\tasknumber{4}%
\task{%
    Определите энергию фотона излучения частотой $8 \cdot 10^{16}\,\text{Гц}$.
    Ответ получите в джоулях и в электронвольтах.
}
\answer{%
    $E = h \nu = 6{,}626 \cdot 10^{-34}\,\text{Дж}\cdot\text{с} \cdot 8 \cdot 10^{16}\,\text{Гц} \approx 53 \cdot 10^{-18}\,\text{Дж} \approx 330\,\text{эВ}$
}
\solutionspace{80pt}

\tasknumber{5}%
\task{%
    Определите энергию фотона с длиной волны $700\,\text{нм}$.
    Ответ выразите в джоулях.
    Способен ли человеческий глаз увидеть один такой квант, а импульс таких квантов?'
}
\answer{%
    $E = h\nu = \frac{hc}{\lambda} = \frac{6{,}626 \cdot 10^{-34}\,\text{Дж}\cdot\text{с} \cdot 3 \cdot 10^{8}\,\frac{\text{м}}{\text{с}}}{700\,\text{нм}} \approx 0{,}284 \cdot 10^{-18}\,\text{Дж} \approx 1{,}775\,\text{эВ}$
}
\solutionspace{80pt}

\tasknumber{6}%
\task{%
    Определите длину волны лучей, фотоны которых имеют энергию
    равную кинетической энергии электрона, ускоренного напряжением $2\,\text{В}$.
}

\variantsplitter

\addpersonalvariant{Артём Жичин}

\tasknumber{1}%
\task{%
    Определите длину волны (в нм) света, которым освещается поверхность металла,
    если фотоэлектроны имеют максимальную кинетическую энергию $9 \cdot 10^{-20}\,\text{Дж}$,
    а работа выхода электронов из этого металла $11 \cdot 10^{-19}\,\text{Дж}$.
    Постоянная Планка $h = 6{,}626 \cdot 10^{-34}\,\text{Дж}\cdot\text{с}$.
}
\answer{%
    $h \frac c\lambda = A_{\text{вых.}} + E_{\text{кин.}} \implies \lambda = \frac{h c}{A_{\text{вых.}} + E_{\text{кин.}}} = \frac{ 6{,}626 \cdot 10^{-34}\,\text{Дж}\cdot\text{с} \cdot {Const.c:V} }{11 \cdot 10^{-19}\,\text{Дж} + 9 \cdot 10^{-20}\,\text{Дж}} \approx 0{,}1670 \cdot 10^{-6}\,\text{м}.$
}
\solutionspace{80pt}

\tasknumber{2}%
\task{%
    Работа выхода электронов из некоторого металла $2{,}1\,\text{эВ}$.
    Найдите скорость электронов (в км/с),
    вылетающих с поверхности металла при освещении его светом с длиной волны $2{,}2 \cdot 10^{-5}\,\text{см}$.
    Масса электрона $m_{e} = 9{,}1 \cdot 10^{-31}\,\text{кг}$.
    Постоянная Планка $h = 6{,}626 \cdot 10^{-34}\,\text{Дж}\cdot\text{с}$, заряд электрона $e = 1{,}6 \cdot 10^{-19}\,\text{Кл}$.
}
\answer{%
    $h \frac c\lambda = A_{\text{вых.}} + \frac{ m_{e}v^2 }2 \implies v = \sqrt{ \frac 2{m_{e}}\cbr{ h \frac c\lambda - A_{\text{вых.}} } } \approx 1116{,}8\,\frac{\text{км}}{\text{c}}.$
}
\solutionspace{80pt}

\tasknumber{3}%
\task{%
    Сколько фотонов испускает за $30\,\text{мин}$ лазер,
    если мощность его излучения $15\,\text{мВт}$?
    Длина волны излучения $750\,\text{нм}$.
    $h = 6{,}626 \cdot 10^{-34}\,\text{Дж}\cdot\text{с}$.
}
\answer{%
    $
        N
            = \frac{E_{\text{общая}}}{E_{\text{одного фотона}}}
            = \frac{Pt}{h\nu} = \frac{Pt}{h \frac c\lambda}
            = \frac{Pt\lambda}{hc}
            = \frac{15\,\text{мВт} \cdot 30\,\text{мин} \cdot 750\,\text{нм}}{6{,}626 \cdot 10^{-34}\,\text{Дж}\cdot\text{с} \cdot 3 \cdot 10^{8}\,\frac{\text{м}}{\text{с}}}
            \approx 101{,}9 \cdot 10^{18}\units{фотонов}
    $
}
\solutionspace{120pt}

\tasknumber{4}%
\task{%
    Определите энергию фотона излучения частотой $5 \cdot 10^{16}\,\text{Гц}$.
    Ответ получите в джоулях и в электронвольтах.
}
\answer{%
    $E = h \nu = 6{,}626 \cdot 10^{-34}\,\text{Дж}\cdot\text{с} \cdot 5 \cdot 10^{16}\,\text{Гц} \approx 33 \cdot 10^{-18}\,\text{Дж} \approx 210\,\text{эВ}$
}
\solutionspace{80pt}

\tasknumber{5}%
\task{%
    Определите энергию кванта света с длиной волны $400\,\text{нм}$.
    Ответ выразите в джоулях.
    Способен ли человеческий глаз увидеть один такой квант, а импульс таких квантов?'
}
\answer{%
    $E = h\nu = \frac{hc}{\lambda} = \frac{6{,}626 \cdot 10^{-34}\,\text{Дж}\cdot\text{с} \cdot 3 \cdot 10^{8}\,\frac{\text{м}}{\text{с}}}{400\,\text{нм}} \approx 0{,}497 \cdot 10^{-18}\,\text{Дж} \approx 3{,}11\,\text{эВ}$
}
\solutionspace{80pt}

\tasknumber{6}%
\task{%
    Определите длину волны лучей, фотоны которых имеют энергию
    равную кинетической энергии электрона, ускоренного напряжением $34\,\text{В}$.
}

\variantsplitter

\addpersonalvariant{Дарья Кошман}

\tasknumber{1}%
\task{%
    Определите длину волны (в нм) света, которым освещается поверхность металла,
    если фотоэлектроны имеют максимальную кинетическую энергию $3 \cdot 10^{-20}\,\text{Дж}$,
    а работа выхода электронов из этого металла $9 \cdot 10^{-19}\,\text{Дж}$.
    Постоянная Планка $h = 6{,}626 \cdot 10^{-34}\,\text{Дж}\cdot\text{с}$.
}
\answer{%
    $h \frac c\lambda = A_{\text{вых.}} + E_{\text{кин.}} \implies \lambda = \frac{h c}{A_{\text{вых.}} + E_{\text{кин.}}} = \frac{ 6{,}626 \cdot 10^{-34}\,\text{Дж}\cdot\text{с} \cdot {Const.c:V} }{9 \cdot 10^{-19}\,\text{Дж} + 3 \cdot 10^{-20}\,\text{Дж}} \approx 0{,}21 \cdot 10^{-6}\,\text{м}.$
}
\solutionspace{80pt}

\tasknumber{2}%
\task{%
    Работа выхода электронов из некоторого металла $2{,}1\,\text{эВ}$.
    Найдите скорость электронов (в км/с),
    вылетающих с поверхности металла при освещении его светом с длиной волны $2{,}7 \cdot 10^{-5}\,\text{см}$.
    Масса электрона $m_{e} = 9{,}1 \cdot 10^{-31}\,\text{кг}$.
    Постоянная Планка $h = 6{,}626 \cdot 10^{-34}\,\text{Дж}\cdot\text{с}$, заряд электрона $e = 1{,}6 \cdot 10^{-19}\,\text{Кл}$.
}
\answer{%
    $h \frac c\lambda = A_{\text{вых.}} + \frac{ m_{e}v^2 }2 \implies v = \sqrt{ \frac 2{m_{e}}\cbr{ h \frac c\lambda - A_{\text{вых.}} } } \approx 937{,}9\,\frac{\text{км}}{\text{c}}.$
}
\solutionspace{80pt}

\tasknumber{3}%
\task{%
    Сколько фотонов испускает за $40\,\text{мин}$ лазер,
    если мощность его излучения $75\,\text{мВт}$?
    Длина волны излучения $600\,\text{нм}$.
    $h = 6{,}626 \cdot 10^{-34}\,\text{Дж}\cdot\text{с}$.
}
\answer{%
    $
        N
            = \frac{E_{\text{общая}}}{E_{\text{одного фотона}}}
            = \frac{Pt}{h\nu} = \frac{Pt}{h \frac c\lambda}
            = \frac{Pt\lambda}{hc}
            = \frac{75\,\text{мВт} \cdot 40\,\text{мин} \cdot 600\,\text{нм}}{6{,}626 \cdot 10^{-34}\,\text{Дж}\cdot\text{с} \cdot 3 \cdot 10^{8}\,\frac{\text{м}}{\text{с}}}
            \approx 543{,}3 \cdot 10^{18}\units{фотонов}
    $
}
\solutionspace{120pt}

\tasknumber{4}%
\task{%
    Определите энергию фотона излучения частотой $8 \cdot 10^{16}\,\text{Гц}$.
    Ответ получите в джоулях и в электронвольтах.
}
\answer{%
    $E = h \nu = 6{,}626 \cdot 10^{-34}\,\text{Дж}\cdot\text{с} \cdot 8 \cdot 10^{16}\,\text{Гц} \approx 53 \cdot 10^{-18}\,\text{Дж} \approx 330\,\text{эВ}$
}
\solutionspace{80pt}

\tasknumber{5}%
\task{%
    Определите энергию фотона с длиной волны $150\,\text{нм}$.
    Ответ выразите в электронвольтах.
    Способен ли человеческий глаз увидеть один такой квант, а импульс таких квантов?'
}
\answer{%
    $E = h\nu = \frac{hc}{\lambda} = \frac{6{,}626 \cdot 10^{-34}\,\text{Дж}\cdot\text{с} \cdot 3 \cdot 10^{8}\,\frac{\text{м}}{\text{с}}}{150\,\text{нм}} \approx 1{,}33 \cdot 10^{-18}\,\text{Дж} \approx 8{,}3\,\text{эВ}$
}
\solutionspace{80pt}

\tasknumber{6}%
\task{%
    Определите длину волны лучей, фотоны которых имеют энергию
    равную кинетической энергии электрона, ускоренного напряжением $233\,\text{В}$.
}

\variantsplitter

\addpersonalvariant{Анна Кузьмичёва}

\tasknumber{1}%
\task{%
    Определите длину волны (в нм) света, которым освещается поверхность металла,
    если фотоэлектроны имеют максимальную кинетическую энергию $4 \cdot 10^{-20}\,\text{Дж}$,
    а работа выхода электронов из этого металла $7 \cdot 10^{-19}\,\text{Дж}$.
    Постоянная Планка $h = 6{,}626 \cdot 10^{-34}\,\text{Дж}\cdot\text{с}$.
}
\answer{%
    $h \frac c\lambda = A_{\text{вых.}} + E_{\text{кин.}} \implies \lambda = \frac{h c}{A_{\text{вых.}} + E_{\text{кин.}}} = \frac{ 6{,}626 \cdot 10^{-34}\,\text{Дж}\cdot\text{с} \cdot {Const.c:V} }{7 \cdot 10^{-19}\,\text{Дж} + 4 \cdot 10^{-20}\,\text{Дж}} \approx 0{,}27 \cdot 10^{-6}\,\text{м}.$
}
\solutionspace{80pt}

\tasknumber{2}%
\task{%
    Работа выхода электронов из некоторого металла $2{,}1\,\text{эВ}$.
    Найдите скорость электронов (в км/с),
    вылетающих с поверхности металла при освещении его светом с длиной волны $2{,}7 \cdot 10^{-5}\,\text{см}$.
    Масса электрона $m_{e} = 9{,}1 \cdot 10^{-31}\,\text{кг}$.
    Постоянная Планка $h = 6{,}626 \cdot 10^{-34}\,\text{Дж}\cdot\text{с}$, заряд электрона $e = 1{,}6 \cdot 10^{-19}\,\text{Кл}$.
}
\answer{%
    $h \frac c\lambda = A_{\text{вых.}} + \frac{ m_{e}v^2 }2 \implies v = \sqrt{ \frac 2{m_{e}}\cbr{ h \frac c\lambda - A_{\text{вых.}} } } \approx 937{,}9\,\frac{\text{км}}{\text{c}}.$
}
\solutionspace{80pt}

\tasknumber{3}%
\task{%
    Сколько фотонов испускает за $20\,\text{мин}$ лазер,
    если мощность его излучения $15\,\text{мВт}$?
    Длина волны излучения $500\,\text{нм}$.
    $h = 6{,}626 \cdot 10^{-34}\,\text{Дж}\cdot\text{с}$.
}
\answer{%
    $
        N
            = \frac{E_{\text{общая}}}{E_{\text{одного фотона}}}
            = \frac{Pt}{h\nu} = \frac{Pt}{h \frac c\lambda}
            = \frac{Pt\lambda}{hc}
            = \frac{15\,\text{мВт} \cdot 20\,\text{мин} \cdot 500\,\text{нм}}{6{,}626 \cdot 10^{-34}\,\text{Дж}\cdot\text{с} \cdot 3 \cdot 10^{8}\,\frac{\text{м}}{\text{с}}}
            \approx 45{,}3 \cdot 10^{18}\units{фотонов}
    $
}
\solutionspace{120pt}

\tasknumber{4}%
\task{%
    Определите энергию фотона излучения частотой $5 \cdot 10^{16}\,\text{Гц}$.
    Ответ получите в джоулях и в электронвольтах.
}
\answer{%
    $E = h \nu = 6{,}626 \cdot 10^{-34}\,\text{Дж}\cdot\text{с} \cdot 5 \cdot 10^{16}\,\text{Гц} \approx 33 \cdot 10^{-18}\,\text{Дж} \approx 210\,\text{эВ}$
}
\solutionspace{80pt}

\tasknumber{5}%
\task{%
    Определите энергию кванта света с длиной волны $200\,\text{нм}$.
    Ответ выразите в джоулях.
    Способен ли человеческий глаз увидеть один такой квант, а импульс таких квантов?'
}
\answer{%
    $E = h\nu = \frac{hc}{\lambda} = \frac{6{,}626 \cdot 10^{-34}\,\text{Дж}\cdot\text{с} \cdot 3 \cdot 10^{8}\,\frac{\text{м}}{\text{с}}}{200\,\text{нм}} \approx 0{,}994 \cdot 10^{-18}\,\text{Дж} \approx 6{,}21\,\text{эВ}$
}
\solutionspace{80pt}

\tasknumber{6}%
\task{%
    Определите длину волны лучей, фотоны которых имеют энергию
    равную кинетической энергии электрона, ускоренного напряжением $34\,\text{В}$.
}

\variantsplitter

\addpersonalvariant{Алёна Куприянова}

\tasknumber{1}%
\task{%
    Определите длину волны (в нм) света, которым освещается поверхность металла,
    если фотоэлектроны имеют максимальную кинетическую энергию $4 \cdot 10^{-20}\,\text{Дж}$,
    а работа выхода электронов из этого металла $13 \cdot 10^{-19}\,\text{Дж}$.
    Постоянная Планка $h = 6{,}626 \cdot 10^{-34}\,\text{Дж}\cdot\text{с}$.
}
\answer{%
    $h \frac c\lambda = A_{\text{вых.}} + E_{\text{кин.}} \implies \lambda = \frac{h c}{A_{\text{вых.}} + E_{\text{кин.}}} = \frac{ 6{,}626 \cdot 10^{-34}\,\text{Дж}\cdot\text{с} \cdot {Const.c:V} }{13 \cdot 10^{-19}\,\text{Дж} + 4 \cdot 10^{-20}\,\text{Дж}} \approx 0{,}1483 \cdot 10^{-6}\,\text{м}.$
}
\solutionspace{80pt}

\tasknumber{2}%
\task{%
    Работа выхода электронов из некоторого металла $4{,}3\,\text{эВ}$.
    Найдите скорость электронов (в км/с),
    вылетающих с поверхности металла при освещении его светом с длиной волны $2{,}2 \cdot 10^{-5}\,\text{см}$.
    Масса электрона $m_{e} = 9{,}1 \cdot 10^{-31}\,\text{кг}$.
    Постоянная Планка $h = 6{,}626 \cdot 10^{-34}\,\text{Дж}\cdot\text{с}$, заряд электрона $e = 1{,}6 \cdot 10^{-19}\,\text{Кл}$.
}
\answer{%
    $h \frac c\lambda = A_{\text{вых.}} + \frac{ m_{e}v^2 }2 \implies v = \sqrt{ \frac 2{m_{e}}\cbr{ h \frac c\lambda - A_{\text{вых.}} } } \approx 688{,}3\,\frac{\text{км}}{\text{c}}.$
}
\solutionspace{80pt}

\tasknumber{3}%
\task{%
    Сколько фотонов испускает за $5\,\text{мин}$ лазер,
    если мощность его излучения $75\,\text{мВт}$?
    Длина волны излучения $500\,\text{нм}$.
    $h = 6{,}626 \cdot 10^{-34}\,\text{Дж}\cdot\text{с}$.
}
\answer{%
    $
        N
            = \frac{E_{\text{общая}}}{E_{\text{одного фотона}}}
            = \frac{Pt}{h\nu} = \frac{Pt}{h \frac c\lambda}
            = \frac{Pt\lambda}{hc}
            = \frac{75\,\text{мВт} \cdot 5\,\text{мин} \cdot 500\,\text{нм}}{6{,}626 \cdot 10^{-34}\,\text{Дж}\cdot\text{с} \cdot 3 \cdot 10^{8}\,\frac{\text{м}}{\text{с}}}
            \approx 56{,}6 \cdot 10^{18}\units{фотонов}
    $
}
\solutionspace{120pt}

\tasknumber{4}%
\task{%
    Определите энергию фотона излучения частотой $6 \cdot 10^{16}\,\text{Гц}$.
    Ответ получите в джоулях и в электронвольтах.
}
\answer{%
    $E = h \nu = 6{,}626 \cdot 10^{-34}\,\text{Дж}\cdot\text{с} \cdot 6 \cdot 10^{16}\,\text{Гц} \approx 40 \cdot 10^{-18}\,\text{Дж} \approx 250\,\text{эВ}$
}
\solutionspace{80pt}

\tasknumber{5}%
\task{%
    Определите энергию фотона с длиной волны $500\,\text{нм}$.
    Ответ выразите в электронвольтах.
    Способен ли человеческий глаз увидеть один такой квант, а импульс таких квантов?'
}
\answer{%
    $E = h\nu = \frac{hc}{\lambda} = \frac{6{,}626 \cdot 10^{-34}\,\text{Дж}\cdot\text{с} \cdot 3 \cdot 10^{8}\,\frac{\text{м}}{\text{с}}}{500\,\text{нм}} \approx 0{,}398 \cdot 10^{-18}\,\text{Дж} \approx 2{,}48\,\text{эВ}$
}
\solutionspace{80pt}

\tasknumber{6}%
\task{%
    Определите длину волны лучей, фотоны которых имеют энергию
    равную кинетической энергии электрона, ускоренного напряжением $13\,\text{В}$.
}

\variantsplitter

\addpersonalvariant{Ярослав Лавровский}

\tasknumber{1}%
\task{%
    Определите длину волны (в нм) света, которым освещается поверхность металла,
    если фотоэлектроны имеют максимальную кинетическую энергию $5 \cdot 10^{-20}\,\text{Дж}$,
    а работа выхода электронов из этого металла $7 \cdot 10^{-19}\,\text{Дж}$.
    Постоянная Планка $h = 6{,}626 \cdot 10^{-34}\,\text{Дж}\cdot\text{с}$.
}
\answer{%
    $h \frac c\lambda = A_{\text{вых.}} + E_{\text{кин.}} \implies \lambda = \frac{h c}{A_{\text{вых.}} + E_{\text{кин.}}} = \frac{ 6{,}626 \cdot 10^{-34}\,\text{Дж}\cdot\text{с} \cdot {Const.c:V} }{7 \cdot 10^{-19}\,\text{Дж} + 5 \cdot 10^{-20}\,\text{Дж}} \approx 0{,}27 \cdot 10^{-6}\,\text{м}.$
}
\solutionspace{80pt}

\tasknumber{2}%
\task{%
    Работа выхода электронов из некоторого металла $2{,}1\,\text{эВ}$.
    Найдите скорость электронов (в км/с),
    вылетающих с поверхности металла при освещении его светом с длиной волны $1{,}7 \cdot 10^{-5}\,\text{см}$.
    Масса электрона $m_{e} = 9{,}1 \cdot 10^{-31}\,\text{кг}$.
    Постоянная Планка $h = 6{,}626 \cdot 10^{-34}\,\text{Дж}\cdot\text{с}$, заряд электрона $e = 1{,}6 \cdot 10^{-19}\,\text{Кл}$.
}
\answer{%
    $h \frac c\lambda = A_{\text{вых.}} + \frac{ m_{e}v^2 }2 \implies v = \sqrt{ \frac 2{m_{e}}\cbr{ h \frac c\lambda - A_{\text{вых.}} } } \approx 1353{,}3\,\frac{\text{км}}{\text{c}}.$
}
\solutionspace{80pt}

\tasknumber{3}%
\task{%
    Сколько фотонов испускает за $10\,\text{мин}$ лазер,
    если мощность его излучения $40\,\text{мВт}$?
    Длина волны излучения $750\,\text{нм}$.
    $h = 6{,}626 \cdot 10^{-34}\,\text{Дж}\cdot\text{с}$.
}
\answer{%
    $
        N
            = \frac{E_{\text{общая}}}{E_{\text{одного фотона}}}
            = \frac{Pt}{h\nu} = \frac{Pt}{h \frac c\lambda}
            = \frac{Pt\lambda}{hc}
            = \frac{40\,\text{мВт} \cdot 10\,\text{мин} \cdot 750\,\text{нм}}{6{,}626 \cdot 10^{-34}\,\text{Дж}\cdot\text{с} \cdot 3 \cdot 10^{8}\,\frac{\text{м}}{\text{с}}}
            \approx 90{,}6 \cdot 10^{18}\units{фотонов}
    $
}
\solutionspace{120pt}

\tasknumber{4}%
\task{%
    Определите энергию фотона излучения частотой $5 \cdot 10^{16}\,\text{Гц}$.
    Ответ получите в джоулях и в электронвольтах.
}
\answer{%
    $E = h \nu = 6{,}626 \cdot 10^{-34}\,\text{Дж}\cdot\text{с} \cdot 5 \cdot 10^{16}\,\text{Гц} \approx 33 \cdot 10^{-18}\,\text{Дж} \approx 210\,\text{эВ}$
}
\solutionspace{80pt}

\tasknumber{5}%
\task{%
    Определите энергию фотона с длиной волны $600\,\text{нм}$.
    Ответ выразите в джоулях.
    Способен ли человеческий глаз увидеть один такой квант, а импульс таких квантов?'
}
\answer{%
    $E = h\nu = \frac{hc}{\lambda} = \frac{6{,}626 \cdot 10^{-34}\,\text{Дж}\cdot\text{с} \cdot 3 \cdot 10^{8}\,\frac{\text{м}}{\text{с}}}{600\,\text{нм}} \approx 0{,}331 \cdot 10^{-18}\,\text{Дж} \approx 2{,}07\,\text{эВ}$
}
\solutionspace{80pt}

\tasknumber{6}%
\task{%
    Определите длину волны лучей, фотоны которых имеют энергию
    равную кинетической энергии электрона, ускоренного напряжением $3\,\text{В}$.
}

\variantsplitter

\addpersonalvariant{Анастасия Ламанова}

\tasknumber{1}%
\task{%
    Определите длину волны (в нм) света, которым освещается поверхность металла,
    если фотоэлектроны имеют максимальную кинетическую энергию $3 \cdot 10^{-20}\,\text{Дж}$,
    а работа выхода электронов из этого металла $13 \cdot 10^{-19}\,\text{Дж}$.
    Постоянная Планка $h = 6{,}626 \cdot 10^{-34}\,\text{Дж}\cdot\text{с}$.
}
\answer{%
    $h \frac c\lambda = A_{\text{вых.}} + E_{\text{кин.}} \implies \lambda = \frac{h c}{A_{\text{вых.}} + E_{\text{кин.}}} = \frac{ 6{,}626 \cdot 10^{-34}\,\text{Дж}\cdot\text{с} \cdot {Const.c:V} }{13 \cdot 10^{-19}\,\text{Дж} + 3 \cdot 10^{-20}\,\text{Дж}} \approx 0{,}1495 \cdot 10^{-6}\,\text{м}.$
}
\solutionspace{80pt}

\tasknumber{2}%
\task{%
    Работа выхода электронов из некоторого металла $4{,}3\,\text{эВ}$.
    Найдите скорость электронов (в км/с),
    вылетающих с поверхности металла при освещении его светом с длиной волны $2{,}2 \cdot 10^{-5}\,\text{см}$.
    Масса электрона $m_{e} = 9{,}1 \cdot 10^{-31}\,\text{кг}$.
    Постоянная Планка $h = 6{,}626 \cdot 10^{-34}\,\text{Дж}\cdot\text{с}$, заряд электрона $e = 1{,}6 \cdot 10^{-19}\,\text{Кл}$.
}
\answer{%
    $h \frac c\lambda = A_{\text{вых.}} + \frac{ m_{e}v^2 }2 \implies v = \sqrt{ \frac 2{m_{e}}\cbr{ h \frac c\lambda - A_{\text{вых.}} } } \approx 688{,}3\,\frac{\text{км}}{\text{c}}.$
}
\solutionspace{80pt}

\tasknumber{3}%
\task{%
    Сколько фотонов испускает за $40\,\text{мин}$ лазер,
    если мощность его излучения $75\,\text{мВт}$?
    Длина волны излучения $600\,\text{нм}$.
    $h = 6{,}626 \cdot 10^{-34}\,\text{Дж}\cdot\text{с}$.
}
\answer{%
    $
        N
            = \frac{E_{\text{общая}}}{E_{\text{одного фотона}}}
            = \frac{Pt}{h\nu} = \frac{Pt}{h \frac c\lambda}
            = \frac{Pt\lambda}{hc}
            = \frac{75\,\text{мВт} \cdot 40\,\text{мин} \cdot 600\,\text{нм}}{6{,}626 \cdot 10^{-34}\,\text{Дж}\cdot\text{с} \cdot 3 \cdot 10^{8}\,\frac{\text{м}}{\text{с}}}
            \approx 543{,}3 \cdot 10^{18}\units{фотонов}
    $
}
\solutionspace{120pt}

\tasknumber{4}%
\task{%
    Определите энергию фотона излучения частотой $8 \cdot 10^{16}\,\text{Гц}$.
    Ответ получите в джоулях и в электронвольтах.
}
\answer{%
    $E = h \nu = 6{,}626 \cdot 10^{-34}\,\text{Дж}\cdot\text{с} \cdot 8 \cdot 10^{16}\,\text{Гц} \approx 53 \cdot 10^{-18}\,\text{Дж} \approx 330\,\text{эВ}$
}
\solutionspace{80pt}

\tasknumber{5}%
\task{%
    Определите энергию кванта света с длиной волны $850\,\text{нм}$.
    Ответ выразите в джоулях.
    Способен ли человеческий глаз увидеть один такой квант, а импульс таких квантов?'
}
\answer{%
    $E = h\nu = \frac{hc}{\lambda} = \frac{6{,}626 \cdot 10^{-34}\,\text{Дж}\cdot\text{с} \cdot 3 \cdot 10^{8}\,\frac{\text{м}}{\text{с}}}{850\,\text{нм}} \approx 0{,}234 \cdot 10^{-18}\,\text{Дж} \approx 1{,}462\,\text{эВ}$
}
\solutionspace{80pt}

\tasknumber{6}%
\task{%
    Определите длину волны лучей, фотоны которых имеют энергию
    равную кинетической энергии электрона, ускоренного напряжением $144\,\text{В}$.
}

\variantsplitter

\addpersonalvariant{Виктория Легонькова}

\tasknumber{1}%
\task{%
    Определите длину волны (в нм) света, которым освещается поверхность металла,
    если фотоэлектроны имеют максимальную кинетическую энергию $9 \cdot 10^{-20}\,\text{Дж}$,
    а работа выхода электронов из этого металла $7 \cdot 10^{-19}\,\text{Дж}$.
    Постоянная Планка $h = 6{,}626 \cdot 10^{-34}\,\text{Дж}\cdot\text{с}$.
}
\answer{%
    $h \frac c\lambda = A_{\text{вых.}} + E_{\text{кин.}} \implies \lambda = \frac{h c}{A_{\text{вых.}} + E_{\text{кин.}}} = \frac{ 6{,}626 \cdot 10^{-34}\,\text{Дж}\cdot\text{с} \cdot {Const.c:V} }{7 \cdot 10^{-19}\,\text{Дж} + 9 \cdot 10^{-20}\,\text{Дж}} \approx 0{,}25 \cdot 10^{-6}\,\text{м}.$
}
\solutionspace{80pt}

\tasknumber{2}%
\task{%
    Работа выхода электронов из некоторого металла $4{,}3\,\text{эВ}$.
    Найдите скорость электронов (в км/с),
    вылетающих с поверхности металла при освещении его светом с длиной волны $1{,}7 \cdot 10^{-5}\,\text{см}$.
    Масса электрона $m_{e} = 9{,}1 \cdot 10^{-31}\,\text{кг}$.
    Постоянная Планка $h = 6{,}626 \cdot 10^{-34}\,\text{Дж}\cdot\text{с}$, заряд электрона $e = 1{,}6 \cdot 10^{-19}\,\text{Кл}$.
}
\answer{%
    $h \frac c\lambda = A_{\text{вых.}} + \frac{ m_{e}v^2 }2 \implies v = \sqrt{ \frac 2{m_{e}}\cbr{ h \frac c\lambda - A_{\text{вых.}} } } \approx 1028{,}5\,\frac{\text{км}}{\text{c}}.$
}
\solutionspace{80pt}

\tasknumber{3}%
\task{%
    Сколько фотонов испускает за $30\,\text{мин}$ лазер,
    если мощность его излучения $40\,\text{мВт}$?
    Длина волны излучения $500\,\text{нм}$.
    $h = 6{,}626 \cdot 10^{-34}\,\text{Дж}\cdot\text{с}$.
}
\answer{%
    $
        N
            = \frac{E_{\text{общая}}}{E_{\text{одного фотона}}}
            = \frac{Pt}{h\nu} = \frac{Pt}{h \frac c\lambda}
            = \frac{Pt\lambda}{hc}
            = \frac{40\,\text{мВт} \cdot 30\,\text{мин} \cdot 500\,\text{нм}}{6{,}626 \cdot 10^{-34}\,\text{Дж}\cdot\text{с} \cdot 3 \cdot 10^{8}\,\frac{\text{м}}{\text{с}}}
            \approx 181{,}10 \cdot 10^{18}\units{фотонов}
    $
}
\solutionspace{120pt}

\tasknumber{4}%
\task{%
    Определите энергию фотона излучения частотой $7 \cdot 10^{16}\,\text{Гц}$.
    Ответ получите в джоулях и в электронвольтах.
}
\answer{%
    $E = h \nu = 6{,}626 \cdot 10^{-34}\,\text{Дж}\cdot\text{с} \cdot 7 \cdot 10^{16}\,\text{Гц} \approx 46 \cdot 10^{-18}\,\text{Дж} \approx 290\,\text{эВ}$
}
\solutionspace{80pt}

\tasknumber{5}%
\task{%
    Определите энергию фотона с длиной волны $500\,\text{нм}$.
    Ответ выразите в джоулях.
    Способен ли человеческий глаз увидеть один такой квант, а импульс таких квантов?'
}
\answer{%
    $E = h\nu = \frac{hc}{\lambda} = \frac{6{,}626 \cdot 10^{-34}\,\text{Дж}\cdot\text{с} \cdot 3 \cdot 10^{8}\,\frac{\text{м}}{\text{с}}}{500\,\text{нм}} \approx 0{,}398 \cdot 10^{-18}\,\text{Дж} \approx 2{,}48\,\text{эВ}$
}
\solutionspace{80pt}

\tasknumber{6}%
\task{%
    Определите длину волны лучей, фотоны которых имеют энергию
    равную кинетической энергии электрона, ускоренного напряжением $5\,\text{В}$.
}

\variantsplitter

\addpersonalvariant{Семён Мартынов}

\tasknumber{1}%
\task{%
    Определите длину волны (в нм) света, которым освещается поверхность металла,
    если фотоэлектроны имеют максимальную кинетическую энергию $4 \cdot 10^{-20}\,\text{Дж}$,
    а работа выхода электронов из этого металла $9 \cdot 10^{-19}\,\text{Дж}$.
    Постоянная Планка $h = 6{,}626 \cdot 10^{-34}\,\text{Дж}\cdot\text{с}$.
}
\answer{%
    $h \frac c\lambda = A_{\text{вых.}} + E_{\text{кин.}} \implies \lambda = \frac{h c}{A_{\text{вых.}} + E_{\text{кин.}}} = \frac{ 6{,}626 \cdot 10^{-34}\,\text{Дж}\cdot\text{с} \cdot {Const.c:V} }{9 \cdot 10^{-19}\,\text{Дж} + 4 \cdot 10^{-20}\,\text{Дж}} \approx 0{,}21 \cdot 10^{-6}\,\text{м}.$
}
\solutionspace{80pt}

\tasknumber{2}%
\task{%
    Работа выхода электронов из некоторого металла $3{,}4\,\text{эВ}$.
    Найдите скорость электронов (в км/с),
    вылетающих с поверхности металла при освещении его светом с длиной волны $2{,}2 \cdot 10^{-5}\,\text{см}$.
    Масса электрона $m_{e} = 9{,}1 \cdot 10^{-31}\,\text{кг}$.
    Постоянная Планка $h = 6{,}626 \cdot 10^{-34}\,\text{Дж}\cdot\text{с}$, заряд электрона $e = 1{,}6 \cdot 10^{-19}\,\text{Кл}$.
}
\answer{%
    $h \frac c\lambda = A_{\text{вых.}} + \frac{ m_{e}v^2 }2 \implies v = \sqrt{ \frac 2{m_{e}}\cbr{ h \frac c\lambda - A_{\text{вых.}} } } \approx 888{,}9\,\frac{\text{км}}{\text{c}}.$
}
\solutionspace{80pt}

\tasknumber{3}%
\task{%
    Сколько фотонов испускает за $30\,\text{мин}$ лазер,
    если мощность его излучения $75\,\text{мВт}$?
    Длина волны излучения $600\,\text{нм}$.
    $h = 6{,}626 \cdot 10^{-34}\,\text{Дж}\cdot\text{с}$.
}
\answer{%
    $
        N
            = \frac{E_{\text{общая}}}{E_{\text{одного фотона}}}
            = \frac{Pt}{h\nu} = \frac{Pt}{h \frac c\lambda}
            = \frac{Pt\lambda}{hc}
            = \frac{75\,\text{мВт} \cdot 30\,\text{мин} \cdot 600\,\text{нм}}{6{,}626 \cdot 10^{-34}\,\text{Дж}\cdot\text{с} \cdot 3 \cdot 10^{8}\,\frac{\text{м}}{\text{с}}}
            \approx 407{,}5 \cdot 10^{18}\units{фотонов}
    $
}
\solutionspace{120pt}

\tasknumber{4}%
\task{%
    Определите энергию фотона излучения частотой $8 \cdot 10^{16}\,\text{Гц}$.
    Ответ получите в джоулях и в электронвольтах.
}
\answer{%
    $E = h \nu = 6{,}626 \cdot 10^{-34}\,\text{Дж}\cdot\text{с} \cdot 8 \cdot 10^{16}\,\text{Гц} \approx 53 \cdot 10^{-18}\,\text{Дж} \approx 330\,\text{эВ}$
}
\solutionspace{80pt}

\tasknumber{5}%
\task{%
    Определите энергию кванта света с длиной волны $600\,\text{нм}$.
    Ответ выразите в электронвольтах.
    Способен ли человеческий глаз увидеть один такой квант, а импульс таких квантов?'
}
\answer{%
    $E = h\nu = \frac{hc}{\lambda} = \frac{6{,}626 \cdot 10^{-34}\,\text{Дж}\cdot\text{с} \cdot 3 \cdot 10^{8}\,\frac{\text{м}}{\text{с}}}{600\,\text{нм}} \approx 0{,}331 \cdot 10^{-18}\,\text{Дж} \approx 2{,}07\,\text{эВ}$
}
\solutionspace{80pt}

\tasknumber{6}%
\task{%
    Определите длину волны лучей, фотоны которых имеют энергию
    равную кинетической энергии электрона, ускоренного напряжением $21\,\text{В}$.
}

\variantsplitter

\addpersonalvariant{Варвара Минаева}

\tasknumber{1}%
\task{%
    Определите длину волны (в нм) света, которым освещается поверхность металла,
    если фотоэлектроны имеют максимальную кинетическую энергию $4 \cdot 10^{-20}\,\text{Дж}$,
    а работа выхода электронов из этого металла $7 \cdot 10^{-19}\,\text{Дж}$.
    Постоянная Планка $h = 6{,}626 \cdot 10^{-34}\,\text{Дж}\cdot\text{с}$.
}
\answer{%
    $h \frac c\lambda = A_{\text{вых.}} + E_{\text{кин.}} \implies \lambda = \frac{h c}{A_{\text{вых.}} + E_{\text{кин.}}} = \frac{ 6{,}626 \cdot 10^{-34}\,\text{Дж}\cdot\text{с} \cdot {Const.c:V} }{7 \cdot 10^{-19}\,\text{Дж} + 4 \cdot 10^{-20}\,\text{Дж}} \approx 0{,}27 \cdot 10^{-6}\,\text{м}.$
}
\solutionspace{80pt}

\tasknumber{2}%
\task{%
    Работа выхода электронов из некоторого металла $2{,}1\,\text{эВ}$.
    Найдите скорость электронов (в км/с),
    вылетающих с поверхности металла при освещении его светом с длиной волны $1{,}7 \cdot 10^{-5}\,\text{см}$.
    Масса электрона $m_{e} = 9{,}1 \cdot 10^{-31}\,\text{кг}$.
    Постоянная Планка $h = 6{,}626 \cdot 10^{-34}\,\text{Дж}\cdot\text{с}$, заряд электрона $e = 1{,}6 \cdot 10^{-19}\,\text{Кл}$.
}
\answer{%
    $h \frac c\lambda = A_{\text{вых.}} + \frac{ m_{e}v^2 }2 \implies v = \sqrt{ \frac 2{m_{e}}\cbr{ h \frac c\lambda - A_{\text{вых.}} } } \approx 1353{,}3\,\frac{\text{км}}{\text{c}}.$
}
\solutionspace{80pt}

\tasknumber{3}%
\task{%
    Сколько фотонов испускает за $40\,\text{мин}$ лазер,
    если мощность его излучения $40\,\text{мВт}$?
    Длина волны излучения $500\,\text{нм}$.
    $h = 6{,}626 \cdot 10^{-34}\,\text{Дж}\cdot\text{с}$.
}
\answer{%
    $
        N
            = \frac{E_{\text{общая}}}{E_{\text{одного фотона}}}
            = \frac{Pt}{h\nu} = \frac{Pt}{h \frac c\lambda}
            = \frac{Pt\lambda}{hc}
            = \frac{40\,\text{мВт} \cdot 40\,\text{мин} \cdot 500\,\text{нм}}{6{,}626 \cdot 10^{-34}\,\text{Дж}\cdot\text{с} \cdot 3 \cdot 10^{8}\,\frac{\text{м}}{\text{с}}}
            \approx 241{,}5 \cdot 10^{18}\units{фотонов}
    $
}
\solutionspace{120pt}

\tasknumber{4}%
\task{%
    Определите энергию фотона излучения частотой $4 \cdot 10^{16}\,\text{Гц}$.
    Ответ получите в джоулях и в электронвольтах.
}
\answer{%
    $E = h \nu = 6{,}626 \cdot 10^{-34}\,\text{Дж}\cdot\text{с} \cdot 4 \cdot 10^{16}\,\text{Гц} \approx 27 \cdot 10^{-18}\,\text{Дж} \approx 166\,\text{эВ}$
}
\solutionspace{80pt}

\tasknumber{5}%
\task{%
    Определите энергию кванта света с длиной волны $900\,\text{нм}$.
    Ответ выразите в электронвольтах.
    Способен ли человеческий глаз увидеть один такой квант, а импульс таких квантов?'
}
\answer{%
    $E = h\nu = \frac{hc}{\lambda} = \frac{6{,}626 \cdot 10^{-34}\,\text{Дж}\cdot\text{с} \cdot 3 \cdot 10^{8}\,\frac{\text{м}}{\text{с}}}{900\,\text{нм}} \approx 0{,}221 \cdot 10^{-18}\,\text{Дж} \approx 1{,}380\,\text{эВ}$
}
\solutionspace{80pt}

\tasknumber{6}%
\task{%
    Определите длину волны лучей, фотоны которых имеют энергию
    равную кинетической энергии электрона, ускоренного напряжением $89\,\text{В}$.
}

\variantsplitter

\addpersonalvariant{Леонид Никитин}

\tasknumber{1}%
\task{%
    Определите длину волны (в нм) света, которым освещается поверхность металла,
    если фотоэлектроны имеют максимальную кинетическую энергию $3 \cdot 10^{-20}\,\text{Дж}$,
    а работа выхода электронов из этого металла $7 \cdot 10^{-19}\,\text{Дж}$.
    Постоянная Планка $h = 6{,}626 \cdot 10^{-34}\,\text{Дж}\cdot\text{с}$.
}
\answer{%
    $h \frac c\lambda = A_{\text{вых.}} + E_{\text{кин.}} \implies \lambda = \frac{h c}{A_{\text{вых.}} + E_{\text{кин.}}} = \frac{ 6{,}626 \cdot 10^{-34}\,\text{Дж}\cdot\text{с} \cdot {Const.c:V} }{7 \cdot 10^{-19}\,\text{Дж} + 3 \cdot 10^{-20}\,\text{Дж}} \approx 0{,}27 \cdot 10^{-6}\,\text{м}.$
}
\solutionspace{80pt}

\tasknumber{2}%
\task{%
    Работа выхода электронов из некоторого металла $4{,}3\,\text{эВ}$.
    Найдите скорость электронов (в км/с),
    вылетающих с поверхности металла при освещении его светом с длиной волны $2{,}7 \cdot 10^{-5}\,\text{см}$.
    Масса электрона $m_{e} = 9{,}1 \cdot 10^{-31}\,\text{кг}$.
    Постоянная Планка $h = 6{,}626 \cdot 10^{-34}\,\text{Дж}\cdot\text{с}$, заряд электрона $e = 1{,}6 \cdot 10^{-19}\,\text{Кл}$.
}
\answer{%
    $h \frac c\lambda = A_{\text{вых.}} + \frac{ m_{e}v^2 }2 \implies v = \sqrt{ \frac 2{m_{e}}\cbr{ h \frac c\lambda - A_{\text{вых.}} } } \approx 325{,}6\,\frac{\text{км}}{\text{c}}.$
}
\solutionspace{80pt}

\tasknumber{3}%
\task{%
    Сколько фотонов испускает за $5\,\text{мин}$ лазер,
    если мощность его излучения $75\,\text{мВт}$?
    Длина волны излучения $750\,\text{нм}$.
    $h = 6{,}626 \cdot 10^{-34}\,\text{Дж}\cdot\text{с}$.
}
\answer{%
    $
        N
            = \frac{E_{\text{общая}}}{E_{\text{одного фотона}}}
            = \frac{Pt}{h\nu} = \frac{Pt}{h \frac c\lambda}
            = \frac{Pt\lambda}{hc}
            = \frac{75\,\text{мВт} \cdot 5\,\text{мин} \cdot 750\,\text{нм}}{6{,}626 \cdot 10^{-34}\,\text{Дж}\cdot\text{с} \cdot 3 \cdot 10^{8}\,\frac{\text{м}}{\text{с}}}
            \approx 84{,}9 \cdot 10^{18}\units{фотонов}
    $
}
\solutionspace{120pt}

\tasknumber{4}%
\task{%
    Определите энергию фотона излучения частотой $4 \cdot 10^{16}\,\text{Гц}$.
    Ответ получите в джоулях и в электронвольтах.
}
\answer{%
    $E = h \nu = 6{,}626 \cdot 10^{-34}\,\text{Дж}\cdot\text{с} \cdot 4 \cdot 10^{16}\,\text{Гц} \approx 27 \cdot 10^{-18}\,\text{Дж} \approx 166\,\text{эВ}$
}
\solutionspace{80pt}

\tasknumber{5}%
\task{%
    Определите энергию кванта света с длиной волны $700\,\text{нм}$.
    Ответ выразите в электронвольтах.
    Способен ли человеческий глаз увидеть один такой квант, а импульс таких квантов?'
}
\answer{%
    $E = h\nu = \frac{hc}{\lambda} = \frac{6{,}626 \cdot 10^{-34}\,\text{Дж}\cdot\text{с} \cdot 3 \cdot 10^{8}\,\frac{\text{м}}{\text{с}}}{700\,\text{нм}} \approx 0{,}284 \cdot 10^{-18}\,\text{Дж} \approx 1{,}775\,\text{эВ}$
}
\solutionspace{80pt}

\tasknumber{6}%
\task{%
    Определите длину волны лучей, фотоны которых имеют энергию
    равную кинетической энергии электрона, ускоренного напряжением $8\,\text{В}$.
}

\variantsplitter

\addpersonalvariant{Тимофей Полетаев}

\tasknumber{1}%
\task{%
    Определите длину волны (в нм) света, которым освещается поверхность металла,
    если фотоэлектроны имеют максимальную кинетическую энергию $9 \cdot 10^{-20}\,\text{Дж}$,
    а работа выхода электронов из этого металла $13 \cdot 10^{-19}\,\text{Дж}$.
    Постоянная Планка $h = 6{,}626 \cdot 10^{-34}\,\text{Дж}\cdot\text{с}$.
}
\answer{%
    $h \frac c\lambda = A_{\text{вых.}} + E_{\text{кин.}} \implies \lambda = \frac{h c}{A_{\text{вых.}} + E_{\text{кин.}}} = \frac{ 6{,}626 \cdot 10^{-34}\,\text{Дж}\cdot\text{с} \cdot {Const.c:V} }{13 \cdot 10^{-19}\,\text{Дж} + 9 \cdot 10^{-20}\,\text{Дж}} \approx 0{,}1430 \cdot 10^{-6}\,\text{м}.$
}
\solutionspace{80pt}

\tasknumber{2}%
\task{%
    Работа выхода электронов из некоторого металла $2{,}1\,\text{эВ}$.
    Найдите скорость электронов (в км/с),
    вылетающих с поверхности металла при освещении его светом с длиной волны $2{,}7 \cdot 10^{-5}\,\text{см}$.
    Масса электрона $m_{e} = 9{,}1 \cdot 10^{-31}\,\text{кг}$.
    Постоянная Планка $h = 6{,}626 \cdot 10^{-34}\,\text{Дж}\cdot\text{с}$, заряд электрона $e = 1{,}6 \cdot 10^{-19}\,\text{Кл}$.
}
\answer{%
    $h \frac c\lambda = A_{\text{вых.}} + \frac{ m_{e}v^2 }2 \implies v = \sqrt{ \frac 2{m_{e}}\cbr{ h \frac c\lambda - A_{\text{вых.}} } } \approx 937{,}9\,\frac{\text{км}}{\text{c}}.$
}
\solutionspace{80pt}

\tasknumber{3}%
\task{%
    Сколько фотонов испускает за $120\,\text{мин}$ лазер,
    если мощность его излучения $75\,\text{мВт}$?
    Длина волны излучения $500\,\text{нм}$.
    $h = 6{,}626 \cdot 10^{-34}\,\text{Дж}\cdot\text{с}$.
}
\answer{%
    $
        N
            = \frac{E_{\text{общая}}}{E_{\text{одного фотона}}}
            = \frac{Pt}{h\nu} = \frac{Pt}{h \frac c\lambda}
            = \frac{Pt\lambda}{hc}
            = \frac{75\,\text{мВт} \cdot 120\,\text{мин} \cdot 500\,\text{нм}}{6{,}626 \cdot 10^{-34}\,\text{Дж}\cdot\text{с} \cdot 3 \cdot 10^{8}\,\frac{\text{м}}{\text{с}}}
            \approx 1{,}3583 \cdot 10^{21}\units{фотонов}
    $
}
\solutionspace{120pt}

\tasknumber{4}%
\task{%
    Определите энергию фотона излучения частотой $7 \cdot 10^{16}\,\text{Гц}$.
    Ответ получите в джоулях и в электронвольтах.
}
\answer{%
    $E = h \nu = 6{,}626 \cdot 10^{-34}\,\text{Дж}\cdot\text{с} \cdot 7 \cdot 10^{16}\,\text{Гц} \approx 46 \cdot 10^{-18}\,\text{Дж} \approx 290\,\text{эВ}$
}
\solutionspace{80pt}

\tasknumber{5}%
\task{%
    Определите энергию фотона с длиной волны $850\,\text{нм}$.
    Ответ выразите в электронвольтах.
    Способен ли человеческий глаз увидеть один такой квант, а импульс таких квантов?'
}
\answer{%
    $E = h\nu = \frac{hc}{\lambda} = \frac{6{,}626 \cdot 10^{-34}\,\text{Дж}\cdot\text{с} \cdot 3 \cdot 10^{8}\,\frac{\text{м}}{\text{с}}}{850\,\text{нм}} \approx 0{,}234 \cdot 10^{-18}\,\text{Дж} \approx 1{,}462\,\text{эВ}$
}
\solutionspace{80pt}

\tasknumber{6}%
\task{%
    Определите длину волны лучей, фотоны которых имеют энергию
    равную кинетической энергии электрона, ускоренного напряжением $2\,\text{В}$.
}

\variantsplitter

\addpersonalvariant{Андрей Рожков}

\tasknumber{1}%
\task{%
    Определите длину волны (в нм) света, которым освещается поверхность металла,
    если фотоэлектроны имеют максимальную кинетическую энергию $5 \cdot 10^{-20}\,\text{Дж}$,
    а работа выхода электронов из этого металла $11 \cdot 10^{-19}\,\text{Дж}$.
    Постоянная Планка $h = 6{,}626 \cdot 10^{-34}\,\text{Дж}\cdot\text{с}$.
}
\answer{%
    $h \frac c\lambda = A_{\text{вых.}} + E_{\text{кин.}} \implies \lambda = \frac{h c}{A_{\text{вых.}} + E_{\text{кин.}}} = \frac{ 6{,}626 \cdot 10^{-34}\,\text{Дж}\cdot\text{с} \cdot {Const.c:V} }{11 \cdot 10^{-19}\,\text{Дж} + 5 \cdot 10^{-20}\,\text{Дж}} \approx 0{,}1729 \cdot 10^{-6}\,\text{м}.$
}
\solutionspace{80pt}

\tasknumber{2}%
\task{%
    Работа выхода электронов из некоторого металла $3{,}4\,\text{эВ}$.
    Найдите скорость электронов (в км/с),
    вылетающих с поверхности металла при освещении его светом с длиной волны $1{,}7 \cdot 10^{-5}\,\text{см}$.
    Масса электрона $m_{e} = 9{,}1 \cdot 10^{-31}\,\text{кг}$.
    Постоянная Планка $h = 6{,}626 \cdot 10^{-34}\,\text{Дж}\cdot\text{с}$, заряд электрона $e = 1{,}6 \cdot 10^{-19}\,\text{Кл}$.
}
\answer{%
    $h \frac c\lambda = A_{\text{вых.}} + \frac{ m_{e}v^2 }2 \implies v = \sqrt{ \frac 2{m_{e}}\cbr{ h \frac c\lambda - A_{\text{вых.}} } } \approx 1172{,}3\,\frac{\text{км}}{\text{c}}.$
}
\solutionspace{80pt}

\tasknumber{3}%
\task{%
    Сколько фотонов испускает за $30\,\text{мин}$ лазер,
    если мощность его излучения $40\,\text{мВт}$?
    Длина волны излучения $600\,\text{нм}$.
    $h = 6{,}626 \cdot 10^{-34}\,\text{Дж}\cdot\text{с}$.
}
\answer{%
    $
        N
            = \frac{E_{\text{общая}}}{E_{\text{одного фотона}}}
            = \frac{Pt}{h\nu} = \frac{Pt}{h \frac c\lambda}
            = \frac{Pt\lambda}{hc}
            = \frac{40\,\text{мВт} \cdot 30\,\text{мин} \cdot 600\,\text{нм}}{6{,}626 \cdot 10^{-34}\,\text{Дж}\cdot\text{с} \cdot 3 \cdot 10^{8}\,\frac{\text{м}}{\text{с}}}
            \approx 217{,}3 \cdot 10^{18}\units{фотонов}
    $
}
\solutionspace{120pt}

\tasknumber{4}%
\task{%
    Определите энергию фотона излучения частотой $4 \cdot 10^{16}\,\text{Гц}$.
    Ответ получите в джоулях и в электронвольтах.
}
\answer{%
    $E = h \nu = 6{,}626 \cdot 10^{-34}\,\text{Дж}\cdot\text{с} \cdot 4 \cdot 10^{16}\,\text{Гц} \approx 27 \cdot 10^{-18}\,\text{Дж} \approx 166\,\text{эВ}$
}
\solutionspace{80pt}

\tasknumber{5}%
\task{%
    Определите энергию кванта света с длиной волны $150\,\text{нм}$.
    Ответ выразите в джоулях.
    Способен ли человеческий глаз увидеть один такой квант, а импульс таких квантов?'
}
\answer{%
    $E = h\nu = \frac{hc}{\lambda} = \frac{6{,}626 \cdot 10^{-34}\,\text{Дж}\cdot\text{с} \cdot 3 \cdot 10^{8}\,\frac{\text{м}}{\text{с}}}{150\,\text{нм}} \approx 1{,}33 \cdot 10^{-18}\,\text{Дж} \approx 8{,}3\,\text{эВ}$
}
\solutionspace{80pt}

\tasknumber{6}%
\task{%
    Определите длину волны лучей, фотоны которых имеют энергию
    равную кинетической энергии электрона, ускоренного напряжением $21\,\text{В}$.
}

\variantsplitter

\addpersonalvariant{Рената Таржиманова}

\tasknumber{1}%
\task{%
    Определите длину волны (в нм) света, которым освещается поверхность металла,
    если фотоэлектроны имеют максимальную кинетическую энергию $9 \cdot 10^{-20}\,\text{Дж}$,
    а работа выхода электронов из этого металла $11 \cdot 10^{-19}\,\text{Дж}$.
    Постоянная Планка $h = 6{,}626 \cdot 10^{-34}\,\text{Дж}\cdot\text{с}$.
}
\answer{%
    $h \frac c\lambda = A_{\text{вых.}} + E_{\text{кин.}} \implies \lambda = \frac{h c}{A_{\text{вых.}} + E_{\text{кин.}}} = \frac{ 6{,}626 \cdot 10^{-34}\,\text{Дж}\cdot\text{с} \cdot {Const.c:V} }{11 \cdot 10^{-19}\,\text{Дж} + 9 \cdot 10^{-20}\,\text{Дж}} \approx 0{,}1670 \cdot 10^{-6}\,\text{м}.$
}
\solutionspace{80pt}

\tasknumber{2}%
\task{%
    Работа выхода электронов из некоторого металла $4{,}3\,\text{эВ}$.
    Найдите скорость электронов (в км/с),
    вылетающих с поверхности металла при освещении его светом с длиной волны $2{,}2 \cdot 10^{-5}\,\text{см}$.
    Масса электрона $m_{e} = 9{,}1 \cdot 10^{-31}\,\text{кг}$.
    Постоянная Планка $h = 6{,}626 \cdot 10^{-34}\,\text{Дж}\cdot\text{с}$, заряд электрона $e = 1{,}6 \cdot 10^{-19}\,\text{Кл}$.
}
\answer{%
    $h \frac c\lambda = A_{\text{вых.}} + \frac{ m_{e}v^2 }2 \implies v = \sqrt{ \frac 2{m_{e}}\cbr{ h \frac c\lambda - A_{\text{вых.}} } } \approx 688{,}3\,\frac{\text{км}}{\text{c}}.$
}
\solutionspace{80pt}

\tasknumber{3}%
\task{%
    Сколько фотонов испускает за $10\,\text{мин}$ лазер,
    если мощность его излучения $40\,\text{мВт}$?
    Длина волны излучения $500\,\text{нм}$.
    $h = 6{,}626 \cdot 10^{-34}\,\text{Дж}\cdot\text{с}$.
}
\answer{%
    $
        N
            = \frac{E_{\text{общая}}}{E_{\text{одного фотона}}}
            = \frac{Pt}{h\nu} = \frac{Pt}{h \frac c\lambda}
            = \frac{Pt\lambda}{hc}
            = \frac{40\,\text{мВт} \cdot 10\,\text{мин} \cdot 500\,\text{нм}}{6{,}626 \cdot 10^{-34}\,\text{Дж}\cdot\text{с} \cdot 3 \cdot 10^{8}\,\frac{\text{м}}{\text{с}}}
            \approx 60{,}4 \cdot 10^{18}\units{фотонов}
    $
}
\solutionspace{120pt}

\tasknumber{4}%
\task{%
    Определите энергию фотона излучения частотой $9 \cdot 10^{16}\,\text{Гц}$.
    Ответ получите в джоулях и в электронвольтах.
}
\answer{%
    $E = h \nu = 6{,}626 \cdot 10^{-34}\,\text{Дж}\cdot\text{с} \cdot 9 \cdot 10^{16}\,\text{Гц} \approx 60 \cdot 10^{-18}\,\text{Дж} \approx 370\,\text{эВ}$
}
\solutionspace{80pt}

\tasknumber{5}%
\task{%
    Определите энергию фотона с длиной волны $900\,\text{нм}$.
    Ответ выразите в джоулях.
    Способен ли человеческий глаз увидеть один такой квант, а импульс таких квантов?'
}
\answer{%
    $E = h\nu = \frac{hc}{\lambda} = \frac{6{,}626 \cdot 10^{-34}\,\text{Дж}\cdot\text{с} \cdot 3 \cdot 10^{8}\,\frac{\text{м}}{\text{с}}}{900\,\text{нм}} \approx 0{,}221 \cdot 10^{-18}\,\text{Дж} \approx 1{,}380\,\text{эВ}$
}
\solutionspace{80pt}

\tasknumber{6}%
\task{%
    Определите длину волны лучей, фотоны которых имеют энергию
    равную кинетической энергии электрона, ускоренного напряжением $34\,\text{В}$.
}

\variantsplitter

\addpersonalvariant{Андрей Щербаков}

\tasknumber{1}%
\task{%
    Определите длину волны (в нм) света, которым освещается поверхность металла,
    если фотоэлектроны имеют максимальную кинетическую энергию $5 \cdot 10^{-20}\,\text{Дж}$,
    а работа выхода электронов из этого металла $7 \cdot 10^{-19}\,\text{Дж}$.
    Постоянная Планка $h = 6{,}626 \cdot 10^{-34}\,\text{Дж}\cdot\text{с}$.
}
\answer{%
    $h \frac c\lambda = A_{\text{вых.}} + E_{\text{кин.}} \implies \lambda = \frac{h c}{A_{\text{вых.}} + E_{\text{кин.}}} = \frac{ 6{,}626 \cdot 10^{-34}\,\text{Дж}\cdot\text{с} \cdot {Const.c:V} }{7 \cdot 10^{-19}\,\text{Дж} + 5 \cdot 10^{-20}\,\text{Дж}} \approx 0{,}27 \cdot 10^{-6}\,\text{м}.$
}
\solutionspace{80pt}

\tasknumber{2}%
\task{%
    Работа выхода электронов из некоторого металла $2{,}1\,\text{эВ}$.
    Найдите скорость электронов (в км/с),
    вылетающих с поверхности металла при освещении его светом с длиной волны $2{,}7 \cdot 10^{-5}\,\text{см}$.
    Масса электрона $m_{e} = 9{,}1 \cdot 10^{-31}\,\text{кг}$.
    Постоянная Планка $h = 6{,}626 \cdot 10^{-34}\,\text{Дж}\cdot\text{с}$, заряд электрона $e = 1{,}6 \cdot 10^{-19}\,\text{Кл}$.
}
\answer{%
    $h \frac c\lambda = A_{\text{вых.}} + \frac{ m_{e}v^2 }2 \implies v = \sqrt{ \frac 2{m_{e}}\cbr{ h \frac c\lambda - A_{\text{вых.}} } } \approx 937{,}9\,\frac{\text{км}}{\text{c}}.$
}
\solutionspace{80pt}

\tasknumber{3}%
\task{%
    Сколько фотонов испускает за $120\,\text{мин}$ лазер,
    если мощность его излучения $75\,\text{мВт}$?
    Длина волны излучения $750\,\text{нм}$.
    $h = 6{,}626 \cdot 10^{-34}\,\text{Дж}\cdot\text{с}$.
}
\answer{%
    $
        N
            = \frac{E_{\text{общая}}}{E_{\text{одного фотона}}}
            = \frac{Pt}{h\nu} = \frac{Pt}{h \frac c\lambda}
            = \frac{Pt\lambda}{hc}
            = \frac{75\,\text{мВт} \cdot 120\,\text{мин} \cdot 750\,\text{нм}}{6{,}626 \cdot 10^{-34}\,\text{Дж}\cdot\text{с} \cdot 3 \cdot 10^{8}\,\frac{\text{м}}{\text{с}}}
            \approx 2{,}037 \cdot 10^{21}\units{фотонов}
    $
}
\solutionspace{120pt}

\tasknumber{4}%
\task{%
    Определите энергию фотона излучения частотой $7 \cdot 10^{16}\,\text{Гц}$.
    Ответ получите в джоулях и в электронвольтах.
}
\answer{%
    $E = h \nu = 6{,}626 \cdot 10^{-34}\,\text{Дж}\cdot\text{с} \cdot 7 \cdot 10^{16}\,\text{Гц} \approx 46 \cdot 10^{-18}\,\text{Дж} \approx 290\,\text{эВ}$
}
\solutionspace{80pt}

\tasknumber{5}%
\task{%
    Определите энергию кванта света с длиной волны $150\,\text{нм}$.
    Ответ выразите в электронвольтах.
    Способен ли человеческий глаз увидеть один такой квант, а импульс таких квантов?'
}
\answer{%
    $E = h\nu = \frac{hc}{\lambda} = \frac{6{,}626 \cdot 10^{-34}\,\text{Дж}\cdot\text{с} \cdot 3 \cdot 10^{8}\,\frac{\text{м}}{\text{с}}}{150\,\text{нм}} \approx 1{,}33 \cdot 10^{-18}\,\text{Дж} \approx 8{,}3\,\text{эВ}$
}
\solutionspace{80pt}

\tasknumber{6}%
\task{%
    Определите длину волны лучей, фотоны которых имеют энергию
    равную кинетической энергии электрона, ускоренного напряжением $89\,\text{В}$.
}

\variantsplitter

\addpersonalvariant{Михаил Ярошевский}

\tasknumber{1}%
\task{%
    Определите длину волны (в нм) света, которым освещается поверхность металла,
    если фотоэлектроны имеют максимальную кинетическую энергию $9 \cdot 10^{-20}\,\text{Дж}$,
    а работа выхода электронов из этого металла $13 \cdot 10^{-19}\,\text{Дж}$.
    Постоянная Планка $h = 6{,}626 \cdot 10^{-34}\,\text{Дж}\cdot\text{с}$.
}
\answer{%
    $h \frac c\lambda = A_{\text{вых.}} + E_{\text{кин.}} \implies \lambda = \frac{h c}{A_{\text{вых.}} + E_{\text{кин.}}} = \frac{ 6{,}626 \cdot 10^{-34}\,\text{Дж}\cdot\text{с} \cdot {Const.c:V} }{13 \cdot 10^{-19}\,\text{Дж} + 9 \cdot 10^{-20}\,\text{Дж}} \approx 0{,}1430 \cdot 10^{-6}\,\text{м}.$
}
\solutionspace{80pt}

\tasknumber{2}%
\task{%
    Работа выхода электронов из некоторого металла $4{,}3\,\text{эВ}$.
    Найдите скорость электронов (в км/с),
    вылетающих с поверхности металла при освещении его светом с длиной волны $1{,}7 \cdot 10^{-5}\,\text{см}$.
    Масса электрона $m_{e} = 9{,}1 \cdot 10^{-31}\,\text{кг}$.
    Постоянная Планка $h = 6{,}626 \cdot 10^{-34}\,\text{Дж}\cdot\text{с}$, заряд электрона $e = 1{,}6 \cdot 10^{-19}\,\text{Кл}$.
}
\answer{%
    $h \frac c\lambda = A_{\text{вых.}} + \frac{ m_{e}v^2 }2 \implies v = \sqrt{ \frac 2{m_{e}}\cbr{ h \frac c\lambda - A_{\text{вых.}} } } \approx 1028{,}5\,\frac{\text{км}}{\text{c}}.$
}
\solutionspace{80pt}

\tasknumber{3}%
\task{%
    Сколько фотонов испускает за $30\,\text{мин}$ лазер,
    если мощность его излучения $15\,\text{мВт}$?
    Длина волны излучения $750\,\text{нм}$.
    $h = 6{,}626 \cdot 10^{-34}\,\text{Дж}\cdot\text{с}$.
}
\answer{%
    $
        N
            = \frac{E_{\text{общая}}}{E_{\text{одного фотона}}}
            = \frac{Pt}{h\nu} = \frac{Pt}{h \frac c\lambda}
            = \frac{Pt\lambda}{hc}
            = \frac{15\,\text{мВт} \cdot 30\,\text{мин} \cdot 750\,\text{нм}}{6{,}626 \cdot 10^{-34}\,\text{Дж}\cdot\text{с} \cdot 3 \cdot 10^{8}\,\frac{\text{м}}{\text{с}}}
            \approx 101{,}9 \cdot 10^{18}\units{фотонов}
    $
}
\solutionspace{120pt}

\tasknumber{4}%
\task{%
    Определите энергию фотона излучения частотой $5 \cdot 10^{16}\,\text{Гц}$.
    Ответ получите в джоулях и в электронвольтах.
}
\answer{%
    $E = h \nu = 6{,}626 \cdot 10^{-34}\,\text{Дж}\cdot\text{с} \cdot 5 \cdot 10^{16}\,\text{Гц} \approx 33 \cdot 10^{-18}\,\text{Дж} \approx 210\,\text{эВ}$
}
\solutionspace{80pt}

\tasknumber{5}%
\task{%
    Определите энергию фотона с длиной волны $700\,\text{нм}$.
    Ответ выразите в электронвольтах.
    Способен ли человеческий глаз увидеть один такой квант, а импульс таких квантов?'
}
\answer{%
    $E = h\nu = \frac{hc}{\lambda} = \frac{6{,}626 \cdot 10^{-34}\,\text{Дж}\cdot\text{с} \cdot 3 \cdot 10^{8}\,\frac{\text{м}}{\text{с}}}{700\,\text{нм}} \approx 0{,}284 \cdot 10^{-18}\,\text{Дж} \approx 1{,}775\,\text{эВ}$
}
\solutionspace{80pt}

\tasknumber{6}%
\task{%
    Определите длину волны лучей, фотоны которых имеют энергию
    равную кинетической энергии электрона, ускоренного напряжением $2\,\text{В}$.
}

\variantsplitter

\addpersonalvariant{Алексей Алимпиев}

\tasknumber{1}%
\task{%
    Определите длину волны (в нм) света, которым освещается поверхность металла,
    если фотоэлектроны имеют максимальную кинетическую энергию $4 \cdot 10^{-20}\,\text{Дж}$,
    а работа выхода электронов из этого металла $9 \cdot 10^{-19}\,\text{Дж}$.
    Постоянная Планка $h = 6{,}626 \cdot 10^{-34}\,\text{Дж}\cdot\text{с}$.
}
\answer{%
    $h \frac c\lambda = A_{\text{вых.}} + E_{\text{кин.}} \implies \lambda = \frac{h c}{A_{\text{вых.}} + E_{\text{кин.}}} = \frac{ 6{,}626 \cdot 10^{-34}\,\text{Дж}\cdot\text{с} \cdot {Const.c:V} }{9 \cdot 10^{-19}\,\text{Дж} + 4 \cdot 10^{-20}\,\text{Дж}} \approx 0{,}21 \cdot 10^{-6}\,\text{м}.$
}
\solutionspace{80pt}

\tasknumber{2}%
\task{%
    Работа выхода электронов из некоторого металла $2{,}1\,\text{эВ}$.
    Найдите скорость электронов (в км/с),
    вылетающих с поверхности металла при освещении его светом с длиной волны $2{,}2 \cdot 10^{-5}\,\text{см}$.
    Масса электрона $m_{e} = 9{,}1 \cdot 10^{-31}\,\text{кг}$.
    Постоянная Планка $h = 6{,}626 \cdot 10^{-34}\,\text{Дж}\cdot\text{с}$, заряд электрона $e = 1{,}6 \cdot 10^{-19}\,\text{Кл}$.
}
\answer{%
    $h \frac c\lambda = A_{\text{вых.}} + \frac{ m_{e}v^2 }2 \implies v = \sqrt{ \frac 2{m_{e}}\cbr{ h \frac c\lambda - A_{\text{вых.}} } } \approx 1116{,}8\,\frac{\text{км}}{\text{c}}.$
}
\solutionspace{80pt}

\tasknumber{3}%
\task{%
    Сколько фотонов испускает за $60\,\text{мин}$ лазер,
    если мощность его излучения $75\,\text{мВт}$?
    Длина волны излучения $600\,\text{нм}$.
    $h = 6{,}626 \cdot 10^{-34}\,\text{Дж}\cdot\text{с}$.
}
\answer{%
    $
        N
            = \frac{E_{\text{общая}}}{E_{\text{одного фотона}}}
            = \frac{Pt}{h\nu} = \frac{Pt}{h \frac c\lambda}
            = \frac{Pt\lambda}{hc}
            = \frac{75\,\text{мВт} \cdot 60\,\text{мин} \cdot 600\,\text{нм}}{6{,}626 \cdot 10^{-34}\,\text{Дж}\cdot\text{с} \cdot 3 \cdot 10^{8}\,\frac{\text{м}}{\text{с}}}
            \approx 815{,}0 \cdot 10^{18}\units{фотонов}
    $
}
\solutionspace{120pt}

\tasknumber{4}%
\task{%
    Определите энергию фотона излучения частотой $6 \cdot 10^{16}\,\text{Гц}$.
    Ответ получите в джоулях и в электронвольтах.
}
\answer{%
    $E = h \nu = 6{,}626 \cdot 10^{-34}\,\text{Дж}\cdot\text{с} \cdot 6 \cdot 10^{16}\,\text{Гц} \approx 40 \cdot 10^{-18}\,\text{Дж} \approx 250\,\text{эВ}$
}
\solutionspace{80pt}

\tasknumber{5}%
\task{%
    Определите энергию фотона с длиной волны $200\,\text{нм}$.
    Ответ выразите в джоулях.
    Способен ли человеческий глаз увидеть один такой квант, а импульс таких квантов?'
}
\answer{%
    $E = h\nu = \frac{hc}{\lambda} = \frac{6{,}626 \cdot 10^{-34}\,\text{Дж}\cdot\text{с} \cdot 3 \cdot 10^{8}\,\frac{\text{м}}{\text{с}}}{200\,\text{нм}} \approx 0{,}994 \cdot 10^{-18}\,\text{Дж} \approx 6{,}21\,\text{эВ}$
}
\solutionspace{80pt}

\tasknumber{6}%
\task{%
    Определите длину волны лучей, фотоны которых имеют энергию
    равную кинетической энергии электрона, ускоренного напряжением $21\,\text{В}$.
}

\variantsplitter

\addpersonalvariant{Евгений Васин}

\tasknumber{1}%
\task{%
    Определите длину волны (в нм) света, которым освещается поверхность металла,
    если фотоэлектроны имеют максимальную кинетическую энергию $4 \cdot 10^{-20}\,\text{Дж}$,
    а работа выхода электронов из этого металла $7 \cdot 10^{-19}\,\text{Дж}$.
    Постоянная Планка $h = 6{,}626 \cdot 10^{-34}\,\text{Дж}\cdot\text{с}$.
}
\answer{%
    $h \frac c\lambda = A_{\text{вых.}} + E_{\text{кин.}} \implies \lambda = \frac{h c}{A_{\text{вых.}} + E_{\text{кин.}}} = \frac{ 6{,}626 \cdot 10^{-34}\,\text{Дж}\cdot\text{с} \cdot {Const.c:V} }{7 \cdot 10^{-19}\,\text{Дж} + 4 \cdot 10^{-20}\,\text{Дж}} \approx 0{,}27 \cdot 10^{-6}\,\text{м}.$
}
\solutionspace{80pt}

\tasknumber{2}%
\task{%
    Работа выхода электронов из некоторого металла $4{,}3\,\text{эВ}$.
    Найдите скорость электронов (в км/с),
    вылетающих с поверхности металла при освещении его светом с длиной волны $2{,}7 \cdot 10^{-5}\,\text{см}$.
    Масса электрона $m_{e} = 9{,}1 \cdot 10^{-31}\,\text{кг}$.
    Постоянная Планка $h = 6{,}626 \cdot 10^{-34}\,\text{Дж}\cdot\text{с}$, заряд электрона $e = 1{,}6 \cdot 10^{-19}\,\text{Кл}$.
}
\answer{%
    $h \frac c\lambda = A_{\text{вых.}} + \frac{ m_{e}v^2 }2 \implies v = \sqrt{ \frac 2{m_{e}}\cbr{ h \frac c\lambda - A_{\text{вых.}} } } \approx 325{,}6\,\frac{\text{км}}{\text{c}}.$
}
\solutionspace{80pt}

\tasknumber{3}%
\task{%
    Сколько фотонов испускает за $30\,\text{мин}$ лазер,
    если мощность его излучения $15\,\text{мВт}$?
    Длина волны излучения $750\,\text{нм}$.
    $h = 6{,}626 \cdot 10^{-34}\,\text{Дж}\cdot\text{с}$.
}
\answer{%
    $
        N
            = \frac{E_{\text{общая}}}{E_{\text{одного фотона}}}
            = \frac{Pt}{h\nu} = \frac{Pt}{h \frac c\lambda}
            = \frac{Pt\lambda}{hc}
            = \frac{15\,\text{мВт} \cdot 30\,\text{мин} \cdot 750\,\text{нм}}{6{,}626 \cdot 10^{-34}\,\text{Дж}\cdot\text{с} \cdot 3 \cdot 10^{8}\,\frac{\text{м}}{\text{с}}}
            \approx 101{,}9 \cdot 10^{18}\units{фотонов}
    $
}
\solutionspace{120pt}

\tasknumber{4}%
\task{%
    Определите энергию фотона излучения частотой $6 \cdot 10^{16}\,\text{Гц}$.
    Ответ получите в джоулях и в электронвольтах.
}
\answer{%
    $E = h \nu = 6{,}626 \cdot 10^{-34}\,\text{Дж}\cdot\text{с} \cdot 6 \cdot 10^{16}\,\text{Гц} \approx 40 \cdot 10^{-18}\,\text{Дж} \approx 250\,\text{эВ}$
}
\solutionspace{80pt}

\tasknumber{5}%
\task{%
    Определите энергию кванта света с длиной волны $900\,\text{нм}$.
    Ответ выразите в джоулях.
    Способен ли человеческий глаз увидеть один такой квант, а импульс таких квантов?'
}
\answer{%
    $E = h\nu = \frac{hc}{\lambda} = \frac{6{,}626 \cdot 10^{-34}\,\text{Дж}\cdot\text{с} \cdot 3 \cdot 10^{8}\,\frac{\text{м}}{\text{с}}}{900\,\text{нм}} \approx 0{,}221 \cdot 10^{-18}\,\text{Дж} \approx 1{,}380\,\text{эВ}$
}
\solutionspace{80pt}

\tasknumber{6}%
\task{%
    Определите длину волны лучей, фотоны которых имеют энергию
    равную кинетической энергии электрона, ускоренного напряжением $3\,\text{В}$.
}

\variantsplitter

\addpersonalvariant{Вячеслав Волохов}

\tasknumber{1}%
\task{%
    Определите длину волны (в нм) света, которым освещается поверхность металла,
    если фотоэлектроны имеют максимальную кинетическую энергию $5 \cdot 10^{-20}\,\text{Дж}$,
    а работа выхода электронов из этого металла $11 \cdot 10^{-19}\,\text{Дж}$.
    Постоянная Планка $h = 6{,}626 \cdot 10^{-34}\,\text{Дж}\cdot\text{с}$.
}
\answer{%
    $h \frac c\lambda = A_{\text{вых.}} + E_{\text{кин.}} \implies \lambda = \frac{h c}{A_{\text{вых.}} + E_{\text{кин.}}} = \frac{ 6{,}626 \cdot 10^{-34}\,\text{Дж}\cdot\text{с} \cdot {Const.c:V} }{11 \cdot 10^{-19}\,\text{Дж} + 5 \cdot 10^{-20}\,\text{Дж}} \approx 0{,}1729 \cdot 10^{-6}\,\text{м}.$
}
\solutionspace{80pt}

\tasknumber{2}%
\task{%
    Работа выхода электронов из некоторого металла $4{,}3\,\text{эВ}$.
    Найдите скорость электронов (в км/с),
    вылетающих с поверхности металла при освещении его светом с длиной волны $1{,}7 \cdot 10^{-5}\,\text{см}$.
    Масса электрона $m_{e} = 9{,}1 \cdot 10^{-31}\,\text{кг}$.
    Постоянная Планка $h = 6{,}626 \cdot 10^{-34}\,\text{Дж}\cdot\text{с}$, заряд электрона $e = 1{,}6 \cdot 10^{-19}\,\text{Кл}$.
}
\answer{%
    $h \frac c\lambda = A_{\text{вых.}} + \frac{ m_{e}v^2 }2 \implies v = \sqrt{ \frac 2{m_{e}}\cbr{ h \frac c\lambda - A_{\text{вых.}} } } \approx 1028{,}5\,\frac{\text{км}}{\text{c}}.$
}
\solutionspace{80pt}

\tasknumber{3}%
\task{%
    Сколько фотонов испускает за $5\,\text{мин}$ лазер,
    если мощность его излучения $15\,\text{мВт}$?
    Длина волны излучения $500\,\text{нм}$.
    $h = 6{,}626 \cdot 10^{-34}\,\text{Дж}\cdot\text{с}$.
}
\answer{%
    $
        N
            = \frac{E_{\text{общая}}}{E_{\text{одного фотона}}}
            = \frac{Pt}{h\nu} = \frac{Pt}{h \frac c\lambda}
            = \frac{Pt\lambda}{hc}
            = \frac{15\,\text{мВт} \cdot 5\,\text{мин} \cdot 500\,\text{нм}}{6{,}626 \cdot 10^{-34}\,\text{Дж}\cdot\text{с} \cdot 3 \cdot 10^{8}\,\frac{\text{м}}{\text{с}}}
            \approx 11{,}32 \cdot 10^{18}\units{фотонов}
    $
}
\solutionspace{120pt}

\tasknumber{4}%
\task{%
    Определите энергию фотона излучения частотой $5 \cdot 10^{16}\,\text{Гц}$.
    Ответ получите в джоулях и в электронвольтах.
}
\answer{%
    $E = h \nu = 6{,}626 \cdot 10^{-34}\,\text{Дж}\cdot\text{с} \cdot 5 \cdot 10^{16}\,\text{Гц} \approx 33 \cdot 10^{-18}\,\text{Дж} \approx 210\,\text{эВ}$
}
\solutionspace{80pt}

\tasknumber{5}%
\task{%
    Определите энергию кванта света с длиной волны $700\,\text{нм}$.
    Ответ выразите в электронвольтах.
    Способен ли человеческий глаз увидеть один такой квант, а импульс таких квантов?'
}
\answer{%
    $E = h\nu = \frac{hc}{\lambda} = \frac{6{,}626 \cdot 10^{-34}\,\text{Дж}\cdot\text{с} \cdot 3 \cdot 10^{8}\,\frac{\text{м}}{\text{с}}}{700\,\text{нм}} \approx 0{,}284 \cdot 10^{-18}\,\text{Дж} \approx 1{,}775\,\text{эВ}$
}
\solutionspace{80pt}

\tasknumber{6}%
\task{%
    Определите длину волны лучей, фотоны которых имеют энергию
    равную кинетической энергии электрона, ускоренного напряжением $377\,\text{В}$.
}

\variantsplitter

\addpersonalvariant{Герман Говоров}

\tasknumber{1}%
\task{%
    Определите длину волны (в нм) света, которым освещается поверхность металла,
    если фотоэлектроны имеют максимальную кинетическую энергию $5 \cdot 10^{-20}\,\text{Дж}$,
    а работа выхода электронов из этого металла $13 \cdot 10^{-19}\,\text{Дж}$.
    Постоянная Планка $h = 6{,}626 \cdot 10^{-34}\,\text{Дж}\cdot\text{с}$.
}
\answer{%
    $h \frac c\lambda = A_{\text{вых.}} + E_{\text{кин.}} \implies \lambda = \frac{h c}{A_{\text{вых.}} + E_{\text{кин.}}} = \frac{ 6{,}626 \cdot 10^{-34}\,\text{Дж}\cdot\text{с} \cdot {Const.c:V} }{13 \cdot 10^{-19}\,\text{Дж} + 5 \cdot 10^{-20}\,\text{Дж}} \approx 0{,}1472 \cdot 10^{-6}\,\text{м}.$
}
\solutionspace{80pt}

\tasknumber{2}%
\task{%
    Работа выхода электронов из некоторого металла $4{,}3\,\text{эВ}$.
    Найдите скорость электронов (в км/с),
    вылетающих с поверхности металла при освещении его светом с длиной волны $2{,}7 \cdot 10^{-5}\,\text{см}$.
    Масса электрона $m_{e} = 9{,}1 \cdot 10^{-31}\,\text{кг}$.
    Постоянная Планка $h = 6{,}626 \cdot 10^{-34}\,\text{Дж}\cdot\text{с}$, заряд электрона $e = 1{,}6 \cdot 10^{-19}\,\text{Кл}$.
}
\answer{%
    $h \frac c\lambda = A_{\text{вых.}} + \frac{ m_{e}v^2 }2 \implies v = \sqrt{ \frac 2{m_{e}}\cbr{ h \frac c\lambda - A_{\text{вых.}} } } \approx 325{,}6\,\frac{\text{км}}{\text{c}}.$
}
\solutionspace{80pt}

\tasknumber{3}%
\task{%
    Сколько фотонов испускает за $60\,\text{мин}$ лазер,
    если мощность его излучения $75\,\text{мВт}$?
    Длина волны излучения $500\,\text{нм}$.
    $h = 6{,}626 \cdot 10^{-34}\,\text{Дж}\cdot\text{с}$.
}
\answer{%
    $
        N
            = \frac{E_{\text{общая}}}{E_{\text{одного фотона}}}
            = \frac{Pt}{h\nu} = \frac{Pt}{h \frac c\lambda}
            = \frac{Pt\lambda}{hc}
            = \frac{75\,\text{мВт} \cdot 60\,\text{мин} \cdot 500\,\text{нм}}{6{,}626 \cdot 10^{-34}\,\text{Дж}\cdot\text{с} \cdot 3 \cdot 10^{8}\,\frac{\text{м}}{\text{с}}}
            \approx 679{,}1 \cdot 10^{18}\units{фотонов}
    $
}
\solutionspace{120pt}

\tasknumber{4}%
\task{%
    Определите энергию фотона излучения частотой $8 \cdot 10^{16}\,\text{Гц}$.
    Ответ получите в джоулях и в электронвольтах.
}
\answer{%
    $E = h \nu = 6{,}626 \cdot 10^{-34}\,\text{Дж}\cdot\text{с} \cdot 8 \cdot 10^{16}\,\text{Гц} \approx 53 \cdot 10^{-18}\,\text{Дж} \approx 330\,\text{эВ}$
}
\solutionspace{80pt}

\tasknumber{5}%
\task{%
    Определите энергию кванта света с длиной волны $900\,\text{нм}$.
    Ответ выразите в джоулях.
    Способен ли человеческий глаз увидеть один такой квант, а импульс таких квантов?'
}
\answer{%
    $E = h\nu = \frac{hc}{\lambda} = \frac{6{,}626 \cdot 10^{-34}\,\text{Дж}\cdot\text{с} \cdot 3 \cdot 10^{8}\,\frac{\text{м}}{\text{с}}}{900\,\text{нм}} \approx 0{,}221 \cdot 10^{-18}\,\text{Дж} \approx 1{,}380\,\text{эВ}$
}
\solutionspace{80pt}

\tasknumber{6}%
\task{%
    Определите длину волны лучей, фотоны которых имеют энергию
    равную кинетической энергии электрона, ускоренного напряжением $89\,\text{В}$.
}

\variantsplitter

\addpersonalvariant{София Журавлёва}

\tasknumber{1}%
\task{%
    Определите длину волны (в нм) света, которым освещается поверхность металла,
    если фотоэлектроны имеют максимальную кинетическую энергию $9 \cdot 10^{-20}\,\text{Дж}$,
    а работа выхода электронов из этого металла $13 \cdot 10^{-19}\,\text{Дж}$.
    Постоянная Планка $h = 6{,}626 \cdot 10^{-34}\,\text{Дж}\cdot\text{с}$.
}
\answer{%
    $h \frac c\lambda = A_{\text{вых.}} + E_{\text{кин.}} \implies \lambda = \frac{h c}{A_{\text{вых.}} + E_{\text{кин.}}} = \frac{ 6{,}626 \cdot 10^{-34}\,\text{Дж}\cdot\text{с} \cdot {Const.c:V} }{13 \cdot 10^{-19}\,\text{Дж} + 9 \cdot 10^{-20}\,\text{Дж}} \approx 0{,}1430 \cdot 10^{-6}\,\text{м}.$
}
\solutionspace{80pt}

\tasknumber{2}%
\task{%
    Работа выхода электронов из некоторого металла $4{,}3\,\text{эВ}$.
    Найдите скорость электронов (в км/с),
    вылетающих с поверхности металла при освещении его светом с длиной волны $1{,}7 \cdot 10^{-5}\,\text{см}$.
    Масса электрона $m_{e} = 9{,}1 \cdot 10^{-31}\,\text{кг}$.
    Постоянная Планка $h = 6{,}626 \cdot 10^{-34}\,\text{Дж}\cdot\text{с}$, заряд электрона $e = 1{,}6 \cdot 10^{-19}\,\text{Кл}$.
}
\answer{%
    $h \frac c\lambda = A_{\text{вых.}} + \frac{ m_{e}v^2 }2 \implies v = \sqrt{ \frac 2{m_{e}}\cbr{ h \frac c\lambda - A_{\text{вых.}} } } \approx 1028{,}5\,\frac{\text{км}}{\text{c}}.$
}
\solutionspace{80pt}

\tasknumber{3}%
\task{%
    Сколько фотонов испускает за $40\,\text{мин}$ лазер,
    если мощность его излучения $200\,\text{мВт}$?
    Длина волны излучения $500\,\text{нм}$.
    $h = 6{,}626 \cdot 10^{-34}\,\text{Дж}\cdot\text{с}$.
}
\answer{%
    $
        N
            = \frac{E_{\text{общая}}}{E_{\text{одного фотона}}}
            = \frac{Pt}{h\nu} = \frac{Pt}{h \frac c\lambda}
            = \frac{Pt\lambda}{hc}
            = \frac{200\,\text{мВт} \cdot 40\,\text{мин} \cdot 500\,\text{нм}}{6{,}626 \cdot 10^{-34}\,\text{Дж}\cdot\text{с} \cdot 3 \cdot 10^{8}\,\frac{\text{м}}{\text{с}}}
            \approx 1{,}2074 \cdot 10^{21}\units{фотонов}
    $
}
\solutionspace{120pt}

\tasknumber{4}%
\task{%
    Определите энергию фотона излучения частотой $8 \cdot 10^{16}\,\text{Гц}$.
    Ответ получите в джоулях и в электронвольтах.
}
\answer{%
    $E = h \nu = 6{,}626 \cdot 10^{-34}\,\text{Дж}\cdot\text{с} \cdot 8 \cdot 10^{16}\,\text{Гц} \approx 53 \cdot 10^{-18}\,\text{Дж} \approx 330\,\text{эВ}$
}
\solutionspace{80pt}

\tasknumber{5}%
\task{%
    Определите энергию кванта света с длиной волны $500\,\text{нм}$.
    Ответ выразите в электронвольтах.
    Способен ли человеческий глаз увидеть один такой квант, а импульс таких квантов?'
}
\answer{%
    $E = h\nu = \frac{hc}{\lambda} = \frac{6{,}626 \cdot 10^{-34}\,\text{Дж}\cdot\text{с} \cdot 3 \cdot 10^{8}\,\frac{\text{м}}{\text{с}}}{500\,\text{нм}} \approx 0{,}398 \cdot 10^{-18}\,\text{Дж} \approx 2{,}48\,\text{эВ}$
}
\solutionspace{80pt}

\tasknumber{6}%
\task{%
    Определите длину волны лучей, фотоны которых имеют энергию
    равную кинетической энергии электрона, ускоренного напряжением $89\,\text{В}$.
}

\variantsplitter

\addpersonalvariant{Константин Козлов}

\tasknumber{1}%
\task{%
    Определите длину волны (в нм) света, которым освещается поверхность металла,
    если фотоэлектроны имеют максимальную кинетическую энергию $9 \cdot 10^{-20}\,\text{Дж}$,
    а работа выхода электронов из этого металла $9 \cdot 10^{-19}\,\text{Дж}$.
    Постоянная Планка $h = 6{,}626 \cdot 10^{-34}\,\text{Дж}\cdot\text{с}$.
}
\answer{%
    $h \frac c\lambda = A_{\text{вых.}} + E_{\text{кин.}} \implies \lambda = \frac{h c}{A_{\text{вых.}} + E_{\text{кин.}}} = \frac{ 6{,}626 \cdot 10^{-34}\,\text{Дж}\cdot\text{с} \cdot {Const.c:V} }{9 \cdot 10^{-19}\,\text{Дж} + 9 \cdot 10^{-20}\,\text{Дж}} \approx 0{,}20 \cdot 10^{-6}\,\text{м}.$
}
\solutionspace{80pt}

\tasknumber{2}%
\task{%
    Работа выхода электронов из некоторого металла $3{,}4\,\text{эВ}$.
    Найдите скорость электронов (в км/с),
    вылетающих с поверхности металла при освещении его светом с длиной волны $2{,}7 \cdot 10^{-5}\,\text{см}$.
    Масса электрона $m_{e} = 9{,}1 \cdot 10^{-31}\,\text{кг}$.
    Постоянная Планка $h = 6{,}626 \cdot 10^{-34}\,\text{Дж}\cdot\text{с}$, заряд электрона $e = 1{,}6 \cdot 10^{-19}\,\text{Кл}$.
}
\answer{%
    $h \frac c\lambda = A_{\text{вых.}} + \frac{ m_{e}v^2 }2 \implies v = \sqrt{ \frac 2{m_{e}}\cbr{ h \frac c\lambda - A_{\text{вых.}} } } \approx 650\,\frac{\text{км}}{\text{c}}.$
}
\solutionspace{80pt}

\tasknumber{3}%
\task{%
    Сколько фотонов испускает за $60\,\text{мин}$ лазер,
    если мощность его излучения $40\,\text{мВт}$?
    Длина волны излучения $750\,\text{нм}$.
    $h = 6{,}626 \cdot 10^{-34}\,\text{Дж}\cdot\text{с}$.
}
\answer{%
    $
        N
            = \frac{E_{\text{общая}}}{E_{\text{одного фотона}}}
            = \frac{Pt}{h\nu} = \frac{Pt}{h \frac c\lambda}
            = \frac{Pt\lambda}{hc}
            = \frac{40\,\text{мВт} \cdot 60\,\text{мин} \cdot 750\,\text{нм}}{6{,}626 \cdot 10^{-34}\,\text{Дж}\cdot\text{с} \cdot 3 \cdot 10^{8}\,\frac{\text{м}}{\text{с}}}
            \approx 543{,}3 \cdot 10^{18}\units{фотонов}
    $
}
\solutionspace{120pt}

\tasknumber{4}%
\task{%
    Определите энергию фотона излучения частотой $4 \cdot 10^{16}\,\text{Гц}$.
    Ответ получите в джоулях и в электронвольтах.
}
\answer{%
    $E = h \nu = 6{,}626 \cdot 10^{-34}\,\text{Дж}\cdot\text{с} \cdot 4 \cdot 10^{16}\,\text{Гц} \approx 27 \cdot 10^{-18}\,\text{Дж} \approx 166\,\text{эВ}$
}
\solutionspace{80pt}

\tasknumber{5}%
\task{%
    Определите энергию кванта света с длиной волны $200\,\text{нм}$.
    Ответ выразите в джоулях.
    Способен ли человеческий глаз увидеть один такой квант, а импульс таких квантов?'
}
\answer{%
    $E = h\nu = \frac{hc}{\lambda} = \frac{6{,}626 \cdot 10^{-34}\,\text{Дж}\cdot\text{с} \cdot 3 \cdot 10^{8}\,\frac{\text{м}}{\text{с}}}{200\,\text{нм}} \approx 0{,}994 \cdot 10^{-18}\,\text{Дж} \approx 6{,}21\,\text{эВ}$
}
\solutionspace{80pt}

\tasknumber{6}%
\task{%
    Определите длину волны лучей, фотоны которых имеют энергию
    равную кинетической энергии электрона, ускоренного напряжением $144\,\text{В}$.
}

\variantsplitter

\addpersonalvariant{Наталья Кравченко}

\tasknumber{1}%
\task{%
    Определите длину волны (в нм) света, которым освещается поверхность металла,
    если фотоэлектроны имеют максимальную кинетическую энергию $4 \cdot 10^{-20}\,\text{Дж}$,
    а работа выхода электронов из этого металла $11 \cdot 10^{-19}\,\text{Дж}$.
    Постоянная Планка $h = 6{,}626 \cdot 10^{-34}\,\text{Дж}\cdot\text{с}$.
}
\answer{%
    $h \frac c\lambda = A_{\text{вых.}} + E_{\text{кин.}} \implies \lambda = \frac{h c}{A_{\text{вых.}} + E_{\text{кин.}}} = \frac{ 6{,}626 \cdot 10^{-34}\,\text{Дж}\cdot\text{с} \cdot {Const.c:V} }{11 \cdot 10^{-19}\,\text{Дж} + 4 \cdot 10^{-20}\,\text{Дж}} \approx 0{,}1744 \cdot 10^{-6}\,\text{м}.$
}
\solutionspace{80pt}

\tasknumber{2}%
\task{%
    Работа выхода электронов из некоторого металла $3{,}4\,\text{эВ}$.
    Найдите скорость электронов (в км/с),
    вылетающих с поверхности металла при освещении его светом с длиной волны $1{,}7 \cdot 10^{-5}\,\text{см}$.
    Масса электрона $m_{e} = 9{,}1 \cdot 10^{-31}\,\text{кг}$.
    Постоянная Планка $h = 6{,}626 \cdot 10^{-34}\,\text{Дж}\cdot\text{с}$, заряд электрона $e = 1{,}6 \cdot 10^{-19}\,\text{Кл}$.
}
\answer{%
    $h \frac c\lambda = A_{\text{вых.}} + \frac{ m_{e}v^2 }2 \implies v = \sqrt{ \frac 2{m_{e}}\cbr{ h \frac c\lambda - A_{\text{вых.}} } } \approx 1172{,}3\,\frac{\text{км}}{\text{c}}.$
}
\solutionspace{80pt}

\tasknumber{3}%
\task{%
    Сколько фотонов испускает за $30\,\text{мин}$ лазер,
    если мощность его излучения $75\,\text{мВт}$?
    Длина волны излучения $600\,\text{нм}$.
    $h = 6{,}626 \cdot 10^{-34}\,\text{Дж}\cdot\text{с}$.
}
\answer{%
    $
        N
            = \frac{E_{\text{общая}}}{E_{\text{одного фотона}}}
            = \frac{Pt}{h\nu} = \frac{Pt}{h \frac c\lambda}
            = \frac{Pt\lambda}{hc}
            = \frac{75\,\text{мВт} \cdot 30\,\text{мин} \cdot 600\,\text{нм}}{6{,}626 \cdot 10^{-34}\,\text{Дж}\cdot\text{с} \cdot 3 \cdot 10^{8}\,\frac{\text{м}}{\text{с}}}
            \approx 407{,}5 \cdot 10^{18}\units{фотонов}
    $
}
\solutionspace{120pt}

\tasknumber{4}%
\task{%
    Определите энергию фотона излучения частотой $4 \cdot 10^{16}\,\text{Гц}$.
    Ответ получите в джоулях и в электронвольтах.
}
\answer{%
    $E = h \nu = 6{,}626 \cdot 10^{-34}\,\text{Дж}\cdot\text{с} \cdot 4 \cdot 10^{16}\,\text{Гц} \approx 27 \cdot 10^{-18}\,\text{Дж} \approx 166\,\text{эВ}$
}
\solutionspace{80pt}

\tasknumber{5}%
\task{%
    Определите энергию фотона с длиной волны $600\,\text{нм}$.
    Ответ выразите в электронвольтах.
    Способен ли человеческий глаз увидеть один такой квант, а импульс таких квантов?'
}
\answer{%
    $E = h\nu = \frac{hc}{\lambda} = \frac{6{,}626 \cdot 10^{-34}\,\text{Дж}\cdot\text{с} \cdot 3 \cdot 10^{8}\,\frac{\text{м}}{\text{с}}}{600\,\text{нм}} \approx 0{,}331 \cdot 10^{-18}\,\text{Дж} \approx 2{,}07\,\text{эВ}$
}
\solutionspace{80pt}

\tasknumber{6}%
\task{%
    Определите длину волны лучей, фотоны которых имеют энергию
    равную кинетической энергии электрона, ускоренного напряжением $144\,\text{В}$.
}

\variantsplitter

\addpersonalvariant{Матвей Кузьмин}

\tasknumber{1}%
\task{%
    Определите длину волны (в нм) света, которым освещается поверхность металла,
    если фотоэлектроны имеют максимальную кинетическую энергию $9 \cdot 10^{-20}\,\text{Дж}$,
    а работа выхода электронов из этого металла $13 \cdot 10^{-19}\,\text{Дж}$.
    Постоянная Планка $h = 6{,}626 \cdot 10^{-34}\,\text{Дж}\cdot\text{с}$.
}
\answer{%
    $h \frac c\lambda = A_{\text{вых.}} + E_{\text{кин.}} \implies \lambda = \frac{h c}{A_{\text{вых.}} + E_{\text{кин.}}} = \frac{ 6{,}626 \cdot 10^{-34}\,\text{Дж}\cdot\text{с} \cdot {Const.c:V} }{13 \cdot 10^{-19}\,\text{Дж} + 9 \cdot 10^{-20}\,\text{Дж}} \approx 0{,}1430 \cdot 10^{-6}\,\text{м}.$
}
\solutionspace{80pt}

\tasknumber{2}%
\task{%
    Работа выхода электронов из некоторого металла $4{,}3\,\text{эВ}$.
    Найдите скорость электронов (в км/с),
    вылетающих с поверхности металла при освещении его светом с длиной волны $2{,}2 \cdot 10^{-5}\,\text{см}$.
    Масса электрона $m_{e} = 9{,}1 \cdot 10^{-31}\,\text{кг}$.
    Постоянная Планка $h = 6{,}626 \cdot 10^{-34}\,\text{Дж}\cdot\text{с}$, заряд электрона $e = 1{,}6 \cdot 10^{-19}\,\text{Кл}$.
}
\answer{%
    $h \frac c\lambda = A_{\text{вых.}} + \frac{ m_{e}v^2 }2 \implies v = \sqrt{ \frac 2{m_{e}}\cbr{ h \frac c\lambda - A_{\text{вых.}} } } \approx 688{,}3\,\frac{\text{км}}{\text{c}}.$
}
\solutionspace{80pt}

\tasknumber{3}%
\task{%
    Сколько фотонов испускает за $30\,\text{мин}$ лазер,
    если мощность его излучения $15\,\text{мВт}$?
    Длина волны излучения $500\,\text{нм}$.
    $h = 6{,}626 \cdot 10^{-34}\,\text{Дж}\cdot\text{с}$.
}
\answer{%
    $
        N
            = \frac{E_{\text{общая}}}{E_{\text{одного фотона}}}
            = \frac{Pt}{h\nu} = \frac{Pt}{h \frac c\lambda}
            = \frac{Pt\lambda}{hc}
            = \frac{15\,\text{мВт} \cdot 30\,\text{мин} \cdot 500\,\text{нм}}{6{,}626 \cdot 10^{-34}\,\text{Дж}\cdot\text{с} \cdot 3 \cdot 10^{8}\,\frac{\text{м}}{\text{с}}}
            \approx 67{,}9 \cdot 10^{18}\units{фотонов}
    $
}
\solutionspace{120pt}

\tasknumber{4}%
\task{%
    Определите энергию фотона излучения частотой $4 \cdot 10^{16}\,\text{Гц}$.
    Ответ получите в джоулях и в электронвольтах.
}
\answer{%
    $E = h \nu = 6{,}626 \cdot 10^{-34}\,\text{Дж}\cdot\text{с} \cdot 4 \cdot 10^{16}\,\text{Гц} \approx 27 \cdot 10^{-18}\,\text{Дж} \approx 166\,\text{эВ}$
}
\solutionspace{80pt}

\tasknumber{5}%
\task{%
    Определите энергию фотона с длиной волны $850\,\text{нм}$.
    Ответ выразите в электронвольтах.
    Способен ли человеческий глаз увидеть один такой квант, а импульс таких квантов?'
}
\answer{%
    $E = h\nu = \frac{hc}{\lambda} = \frac{6{,}626 \cdot 10^{-34}\,\text{Дж}\cdot\text{с} \cdot 3 \cdot 10^{8}\,\frac{\text{м}}{\text{с}}}{850\,\text{нм}} \approx 0{,}234 \cdot 10^{-18}\,\text{Дж} \approx 1{,}462\,\text{эВ}$
}
\solutionspace{80pt}

\tasknumber{6}%
\task{%
    Определите длину волны лучей, фотоны которых имеют энергию
    равную кинетической энергии электрона, ускоренного напряжением $8\,\text{В}$.
}

\variantsplitter

\addpersonalvariant{Сергей Малышев}

\tasknumber{1}%
\task{%
    Определите длину волны (в нм) света, которым освещается поверхность металла,
    если фотоэлектроны имеют максимальную кинетическую энергию $4 \cdot 10^{-20}\,\text{Дж}$,
    а работа выхода электронов из этого металла $7 \cdot 10^{-19}\,\text{Дж}$.
    Постоянная Планка $h = 6{,}626 \cdot 10^{-34}\,\text{Дж}\cdot\text{с}$.
}
\answer{%
    $h \frac c\lambda = A_{\text{вых.}} + E_{\text{кин.}} \implies \lambda = \frac{h c}{A_{\text{вых.}} + E_{\text{кин.}}} = \frac{ 6{,}626 \cdot 10^{-34}\,\text{Дж}\cdot\text{с} \cdot {Const.c:V} }{7 \cdot 10^{-19}\,\text{Дж} + 4 \cdot 10^{-20}\,\text{Дж}} \approx 0{,}27 \cdot 10^{-6}\,\text{м}.$
}
\solutionspace{80pt}

\tasknumber{2}%
\task{%
    Работа выхода электронов из некоторого металла $4{,}3\,\text{эВ}$.
    Найдите скорость электронов (в км/с),
    вылетающих с поверхности металла при освещении его светом с длиной волны $2{,}7 \cdot 10^{-5}\,\text{см}$.
    Масса электрона $m_{e} = 9{,}1 \cdot 10^{-31}\,\text{кг}$.
    Постоянная Планка $h = 6{,}626 \cdot 10^{-34}\,\text{Дж}\cdot\text{с}$, заряд электрона $e = 1{,}6 \cdot 10^{-19}\,\text{Кл}$.
}
\answer{%
    $h \frac c\lambda = A_{\text{вых.}} + \frac{ m_{e}v^2 }2 \implies v = \sqrt{ \frac 2{m_{e}}\cbr{ h \frac c\lambda - A_{\text{вых.}} } } \approx 325{,}6\,\frac{\text{км}}{\text{c}}.$
}
\solutionspace{80pt}

\tasknumber{3}%
\task{%
    Сколько фотонов испускает за $20\,\text{мин}$ лазер,
    если мощность его излучения $40\,\text{мВт}$?
    Длина волны излучения $750\,\text{нм}$.
    $h = 6{,}626 \cdot 10^{-34}\,\text{Дж}\cdot\text{с}$.
}
\answer{%
    $
        N
            = \frac{E_{\text{общая}}}{E_{\text{одного фотона}}}
            = \frac{Pt}{h\nu} = \frac{Pt}{h \frac c\lambda}
            = \frac{Pt\lambda}{hc}
            = \frac{40\,\text{мВт} \cdot 20\,\text{мин} \cdot 750\,\text{нм}}{6{,}626 \cdot 10^{-34}\,\text{Дж}\cdot\text{с} \cdot 3 \cdot 10^{8}\,\frac{\text{м}}{\text{с}}}
            \approx 181{,}10 \cdot 10^{18}\units{фотонов}
    $
}
\solutionspace{120pt}

\tasknumber{4}%
\task{%
    Определите энергию фотона излучения частотой $7 \cdot 10^{16}\,\text{Гц}$.
    Ответ получите в джоулях и в электронвольтах.
}
\answer{%
    $E = h \nu = 6{,}626 \cdot 10^{-34}\,\text{Дж}\cdot\text{с} \cdot 7 \cdot 10^{16}\,\text{Гц} \approx 46 \cdot 10^{-18}\,\text{Дж} \approx 290\,\text{эВ}$
}
\solutionspace{80pt}

\tasknumber{5}%
\task{%
    Определите энергию кванта света с длиной волны $150\,\text{нм}$.
    Ответ выразите в джоулях.
    Способен ли человеческий глаз увидеть один такой квант, а импульс таких квантов?'
}
\answer{%
    $E = h\nu = \frac{hc}{\lambda} = \frac{6{,}626 \cdot 10^{-34}\,\text{Дж}\cdot\text{с} \cdot 3 \cdot 10^{8}\,\frac{\text{м}}{\text{с}}}{150\,\text{нм}} \approx 1{,}33 \cdot 10^{-18}\,\text{Дж} \approx 8{,}3\,\text{эВ}$
}
\solutionspace{80pt}

\tasknumber{6}%
\task{%
    Определите длину волны лучей, фотоны которых имеют энергию
    равную кинетической энергии электрона, ускоренного напряжением $8\,\text{В}$.
}

\variantsplitter

\addpersonalvariant{Алина Полканова}

\tasknumber{1}%
\task{%
    Определите длину волны (в нм) света, которым освещается поверхность металла,
    если фотоэлектроны имеют максимальную кинетическую энергию $9 \cdot 10^{-20}\,\text{Дж}$,
    а работа выхода электронов из этого металла $7 \cdot 10^{-19}\,\text{Дж}$.
    Постоянная Планка $h = 6{,}626 \cdot 10^{-34}\,\text{Дж}\cdot\text{с}$.
}
\answer{%
    $h \frac c\lambda = A_{\text{вых.}} + E_{\text{кин.}} \implies \lambda = \frac{h c}{A_{\text{вых.}} + E_{\text{кин.}}} = \frac{ 6{,}626 \cdot 10^{-34}\,\text{Дж}\cdot\text{с} \cdot {Const.c:V} }{7 \cdot 10^{-19}\,\text{Дж} + 9 \cdot 10^{-20}\,\text{Дж}} \approx 0{,}25 \cdot 10^{-6}\,\text{м}.$
}
\solutionspace{80pt}

\tasknumber{2}%
\task{%
    Работа выхода электронов из некоторого металла $2{,}1\,\text{эВ}$.
    Найдите скорость электронов (в км/с),
    вылетающих с поверхности металла при освещении его светом с длиной волны $2{,}7 \cdot 10^{-5}\,\text{см}$.
    Масса электрона $m_{e} = 9{,}1 \cdot 10^{-31}\,\text{кг}$.
    Постоянная Планка $h = 6{,}626 \cdot 10^{-34}\,\text{Дж}\cdot\text{с}$, заряд электрона $e = 1{,}6 \cdot 10^{-19}\,\text{Кл}$.
}
\answer{%
    $h \frac c\lambda = A_{\text{вых.}} + \frac{ m_{e}v^2 }2 \implies v = \sqrt{ \frac 2{m_{e}}\cbr{ h \frac c\lambda - A_{\text{вых.}} } } \approx 937{,}9\,\frac{\text{км}}{\text{c}}.$
}
\solutionspace{80pt}

\tasknumber{3}%
\task{%
    Сколько фотонов испускает за $5\,\text{мин}$ лазер,
    если мощность его излучения $15\,\text{мВт}$?
    Длина волны излучения $600\,\text{нм}$.
    $h = 6{,}626 \cdot 10^{-34}\,\text{Дж}\cdot\text{с}$.
}
\answer{%
    $
        N
            = \frac{E_{\text{общая}}}{E_{\text{одного фотона}}}
            = \frac{Pt}{h\nu} = \frac{Pt}{h \frac c\lambda}
            = \frac{Pt\lambda}{hc}
            = \frac{15\,\text{мВт} \cdot 5\,\text{мин} \cdot 600\,\text{нм}}{6{,}626 \cdot 10^{-34}\,\text{Дж}\cdot\text{с} \cdot 3 \cdot 10^{8}\,\frac{\text{м}}{\text{с}}}
            \approx 13{,}58 \cdot 10^{18}\units{фотонов}
    $
}
\solutionspace{120pt}

\tasknumber{4}%
\task{%
    Определите энергию фотона излучения частотой $6 \cdot 10^{16}\,\text{Гц}$.
    Ответ получите в джоулях и в электронвольтах.
}
\answer{%
    $E = h \nu = 6{,}626 \cdot 10^{-34}\,\text{Дж}\cdot\text{с} \cdot 6 \cdot 10^{16}\,\text{Гц} \approx 40 \cdot 10^{-18}\,\text{Дж} \approx 250\,\text{эВ}$
}
\solutionspace{80pt}

\tasknumber{5}%
\task{%
    Определите энергию кванта света с длиной волны $900\,\text{нм}$.
    Ответ выразите в электронвольтах.
    Способен ли человеческий глаз увидеть один такой квант, а импульс таких квантов?'
}
\answer{%
    $E = h\nu = \frac{hc}{\lambda} = \frac{6{,}626 \cdot 10^{-34}\,\text{Дж}\cdot\text{с} \cdot 3 \cdot 10^{8}\,\frac{\text{м}}{\text{с}}}{900\,\text{нм}} \approx 0{,}221 \cdot 10^{-18}\,\text{Дж} \approx 1{,}380\,\text{эВ}$
}
\solutionspace{80pt}

\tasknumber{6}%
\task{%
    Определите длину волны лучей, фотоны которых имеют энергию
    равную кинетической энергии электрона, ускоренного напряжением $5\,\text{В}$.
}

\variantsplitter

\addpersonalvariant{Сергей Пономарёв}

\tasknumber{1}%
\task{%
    Определите длину волны (в нм) света, которым освещается поверхность металла,
    если фотоэлектроны имеют максимальную кинетическую энергию $9 \cdot 10^{-20}\,\text{Дж}$,
    а работа выхода электронов из этого металла $13 \cdot 10^{-19}\,\text{Дж}$.
    Постоянная Планка $h = 6{,}626 \cdot 10^{-34}\,\text{Дж}\cdot\text{с}$.
}
\answer{%
    $h \frac c\lambda = A_{\text{вых.}} + E_{\text{кин.}} \implies \lambda = \frac{h c}{A_{\text{вых.}} + E_{\text{кин.}}} = \frac{ 6{,}626 \cdot 10^{-34}\,\text{Дж}\cdot\text{с} \cdot {Const.c:V} }{13 \cdot 10^{-19}\,\text{Дж} + 9 \cdot 10^{-20}\,\text{Дж}} \approx 0{,}1430 \cdot 10^{-6}\,\text{м}.$
}
\solutionspace{80pt}

\tasknumber{2}%
\task{%
    Работа выхода электронов из некоторого металла $2{,}1\,\text{эВ}$.
    Найдите скорость электронов (в км/с),
    вылетающих с поверхности металла при освещении его светом с длиной волны $2{,}7 \cdot 10^{-5}\,\text{см}$.
    Масса электрона $m_{e} = 9{,}1 \cdot 10^{-31}\,\text{кг}$.
    Постоянная Планка $h = 6{,}626 \cdot 10^{-34}\,\text{Дж}\cdot\text{с}$, заряд электрона $e = 1{,}6 \cdot 10^{-19}\,\text{Кл}$.
}
\answer{%
    $h \frac c\lambda = A_{\text{вых.}} + \frac{ m_{e}v^2 }2 \implies v = \sqrt{ \frac 2{m_{e}}\cbr{ h \frac c\lambda - A_{\text{вых.}} } } \approx 937{,}9\,\frac{\text{км}}{\text{c}}.$
}
\solutionspace{80pt}

\tasknumber{3}%
\task{%
    Сколько фотонов испускает за $40\,\text{мин}$ лазер,
    если мощность его излучения $40\,\text{мВт}$?
    Длина волны излучения $600\,\text{нм}$.
    $h = 6{,}626 \cdot 10^{-34}\,\text{Дж}\cdot\text{с}$.
}
\answer{%
    $
        N
            = \frac{E_{\text{общая}}}{E_{\text{одного фотона}}}
            = \frac{Pt}{h\nu} = \frac{Pt}{h \frac c\lambda}
            = \frac{Pt\lambda}{hc}
            = \frac{40\,\text{мВт} \cdot 40\,\text{мин} \cdot 600\,\text{нм}}{6{,}626 \cdot 10^{-34}\,\text{Дж}\cdot\text{с} \cdot 3 \cdot 10^{8}\,\frac{\text{м}}{\text{с}}}
            \approx 289{,}8 \cdot 10^{18}\units{фотонов}
    $
}
\solutionspace{120pt}

\tasknumber{4}%
\task{%
    Определите энергию фотона излучения частотой $9 \cdot 10^{16}\,\text{Гц}$.
    Ответ получите в джоулях и в электронвольтах.
}
\answer{%
    $E = h \nu = 6{,}626 \cdot 10^{-34}\,\text{Дж}\cdot\text{с} \cdot 9 \cdot 10^{16}\,\text{Гц} \approx 60 \cdot 10^{-18}\,\text{Дж} \approx 370\,\text{эВ}$
}
\solutionspace{80pt}

\tasknumber{5}%
\task{%
    Определите энергию кванта света с длиной волны $850\,\text{нм}$.
    Ответ выразите в джоулях.
    Способен ли человеческий глаз увидеть один такой квант, а импульс таких квантов?'
}
\answer{%
    $E = h\nu = \frac{hc}{\lambda} = \frac{6{,}626 \cdot 10^{-34}\,\text{Дж}\cdot\text{с} \cdot 3 \cdot 10^{8}\,\frac{\text{м}}{\text{с}}}{850\,\text{нм}} \approx 0{,}234 \cdot 10^{-18}\,\text{Дж} \approx 1{,}462\,\text{эВ}$
}
\solutionspace{80pt}

\tasknumber{6}%
\task{%
    Определите длину волны лучей, фотоны которых имеют энергию
    равную кинетической энергии электрона, ускоренного напряжением $610\,\text{В}$.
}

\variantsplitter

\addpersonalvariant{Егор Свистушкин}

\tasknumber{1}%
\task{%
    Определите длину волны (в нм) света, которым освещается поверхность металла,
    если фотоэлектроны имеют максимальную кинетическую энергию $9 \cdot 10^{-20}\,\text{Дж}$,
    а работа выхода электронов из этого металла $11 \cdot 10^{-19}\,\text{Дж}$.
    Постоянная Планка $h = 6{,}626 \cdot 10^{-34}\,\text{Дж}\cdot\text{с}$.
}
\answer{%
    $h \frac c\lambda = A_{\text{вых.}} + E_{\text{кин.}} \implies \lambda = \frac{h c}{A_{\text{вых.}} + E_{\text{кин.}}} = \frac{ 6{,}626 \cdot 10^{-34}\,\text{Дж}\cdot\text{с} \cdot {Const.c:V} }{11 \cdot 10^{-19}\,\text{Дж} + 9 \cdot 10^{-20}\,\text{Дж}} \approx 0{,}1670 \cdot 10^{-6}\,\text{м}.$
}
\solutionspace{80pt}

\tasknumber{2}%
\task{%
    Работа выхода электронов из некоторого металла $3{,}4\,\text{эВ}$.
    Найдите скорость электронов (в км/с),
    вылетающих с поверхности металла при освещении его светом с длиной волны $2{,}7 \cdot 10^{-5}\,\text{см}$.
    Масса электрона $m_{e} = 9{,}1 \cdot 10^{-31}\,\text{кг}$.
    Постоянная Планка $h = 6{,}626 \cdot 10^{-34}\,\text{Дж}\cdot\text{с}$, заряд электрона $e = 1{,}6 \cdot 10^{-19}\,\text{Кл}$.
}
\answer{%
    $h \frac c\lambda = A_{\text{вых.}} + \frac{ m_{e}v^2 }2 \implies v = \sqrt{ \frac 2{m_{e}}\cbr{ h \frac c\lambda - A_{\text{вых.}} } } \approx 650\,\frac{\text{км}}{\text{c}}.$
}
\solutionspace{80pt}

\tasknumber{3}%
\task{%
    Сколько фотонов испускает за $120\,\text{мин}$ лазер,
    если мощность его излучения $75\,\text{мВт}$?
    Длина волны излучения $750\,\text{нм}$.
    $h = 6{,}626 \cdot 10^{-34}\,\text{Дж}\cdot\text{с}$.
}
\answer{%
    $
        N
            = \frac{E_{\text{общая}}}{E_{\text{одного фотона}}}
            = \frac{Pt}{h\nu} = \frac{Pt}{h \frac c\lambda}
            = \frac{Pt\lambda}{hc}
            = \frac{75\,\text{мВт} \cdot 120\,\text{мин} \cdot 750\,\text{нм}}{6{,}626 \cdot 10^{-34}\,\text{Дж}\cdot\text{с} \cdot 3 \cdot 10^{8}\,\frac{\text{м}}{\text{с}}}
            \approx 2{,}037 \cdot 10^{21}\units{фотонов}
    $
}
\solutionspace{120pt}

\tasknumber{4}%
\task{%
    Определите энергию фотона излучения частотой $8 \cdot 10^{16}\,\text{Гц}$.
    Ответ получите в джоулях и в электронвольтах.
}
\answer{%
    $E = h \nu = 6{,}626 \cdot 10^{-34}\,\text{Дж}\cdot\text{с} \cdot 8 \cdot 10^{16}\,\text{Гц} \approx 53 \cdot 10^{-18}\,\text{Дж} \approx 330\,\text{эВ}$
}
\solutionspace{80pt}

\tasknumber{5}%
\task{%
    Определите энергию кванта света с длиной волны $900\,\text{нм}$.
    Ответ выразите в электронвольтах.
    Способен ли человеческий глаз увидеть один такой квант, а импульс таких квантов?'
}
\answer{%
    $E = h\nu = \frac{hc}{\lambda} = \frac{6{,}626 \cdot 10^{-34}\,\text{Дж}\cdot\text{с} \cdot 3 \cdot 10^{8}\,\frac{\text{м}}{\text{с}}}{900\,\text{нм}} \approx 0{,}221 \cdot 10^{-18}\,\text{Дж} \approx 1{,}380\,\text{эВ}$
}
\solutionspace{80pt}

\tasknumber{6}%
\task{%
    Определите длину волны лучей, фотоны которых имеют энергию
    равную кинетической энергии электрона, ускоренного напряжением $3\,\text{В}$.
}

\variantsplitter

\addpersonalvariant{Дмитрий Соколов}

\tasknumber{1}%
\task{%
    Определите длину волны (в нм) света, которым освещается поверхность металла,
    если фотоэлектроны имеют максимальную кинетическую энергию $4 \cdot 10^{-20}\,\text{Дж}$,
    а работа выхода электронов из этого металла $9 \cdot 10^{-19}\,\text{Дж}$.
    Постоянная Планка $h = 6{,}626 \cdot 10^{-34}\,\text{Дж}\cdot\text{с}$.
}
\answer{%
    $h \frac c\lambda = A_{\text{вых.}} + E_{\text{кин.}} \implies \lambda = \frac{h c}{A_{\text{вых.}} + E_{\text{кин.}}} = \frac{ 6{,}626 \cdot 10^{-34}\,\text{Дж}\cdot\text{с} \cdot {Const.c:V} }{9 \cdot 10^{-19}\,\text{Дж} + 4 \cdot 10^{-20}\,\text{Дж}} \approx 0{,}21 \cdot 10^{-6}\,\text{м}.$
}
\solutionspace{80pt}

\tasknumber{2}%
\task{%
    Работа выхода электронов из некоторого металла $3{,}4\,\text{эВ}$.
    Найдите скорость электронов (в км/с),
    вылетающих с поверхности металла при освещении его светом с длиной волны $1{,}7 \cdot 10^{-5}\,\text{см}$.
    Масса электрона $m_{e} = 9{,}1 \cdot 10^{-31}\,\text{кг}$.
    Постоянная Планка $h = 6{,}626 \cdot 10^{-34}\,\text{Дж}\cdot\text{с}$, заряд электрона $e = 1{,}6 \cdot 10^{-19}\,\text{Кл}$.
}
\answer{%
    $h \frac c\lambda = A_{\text{вых.}} + \frac{ m_{e}v^2 }2 \implies v = \sqrt{ \frac 2{m_{e}}\cbr{ h \frac c\lambda - A_{\text{вых.}} } } \approx 1172{,}3\,\frac{\text{км}}{\text{c}}.$
}
\solutionspace{80pt}

\tasknumber{3}%
\task{%
    Сколько фотонов испускает за $30\,\text{мин}$ лазер,
    если мощность его излучения $40\,\text{мВт}$?
    Длина волны излучения $750\,\text{нм}$.
    $h = 6{,}626 \cdot 10^{-34}\,\text{Дж}\cdot\text{с}$.
}
\answer{%
    $
        N
            = \frac{E_{\text{общая}}}{E_{\text{одного фотона}}}
            = \frac{Pt}{h\nu} = \frac{Pt}{h \frac c\lambda}
            = \frac{Pt\lambda}{hc}
            = \frac{40\,\text{мВт} \cdot 30\,\text{мин} \cdot 750\,\text{нм}}{6{,}626 \cdot 10^{-34}\,\text{Дж}\cdot\text{с} \cdot 3 \cdot 10^{8}\,\frac{\text{м}}{\text{с}}}
            \approx 271{,}7 \cdot 10^{18}\units{фотонов}
    $
}
\solutionspace{120pt}

\tasknumber{4}%
\task{%
    Определите энергию фотона излучения частотой $6 \cdot 10^{16}\,\text{Гц}$.
    Ответ получите в джоулях и в электронвольтах.
}
\answer{%
    $E = h \nu = 6{,}626 \cdot 10^{-34}\,\text{Дж}\cdot\text{с} \cdot 6 \cdot 10^{16}\,\text{Гц} \approx 40 \cdot 10^{-18}\,\text{Дж} \approx 250\,\text{эВ}$
}
\solutionspace{80pt}

\tasknumber{5}%
\task{%
    Определите энергию кванта света с длиной волны $900\,\text{нм}$.
    Ответ выразите в джоулях.
    Способен ли человеческий глаз увидеть один такой квант, а импульс таких квантов?'
}
\answer{%
    $E = h\nu = \frac{hc}{\lambda} = \frac{6{,}626 \cdot 10^{-34}\,\text{Дж}\cdot\text{с} \cdot 3 \cdot 10^{8}\,\frac{\text{м}}{\text{с}}}{900\,\text{нм}} \approx 0{,}221 \cdot 10^{-18}\,\text{Дж} \approx 1{,}380\,\text{эВ}$
}
\solutionspace{80pt}

\tasknumber{6}%
\task{%
    Определите длину волны лучей, фотоны которых имеют энергию
    равную кинетической энергии электрона, ускоренного напряжением $1\,\text{В}$.
}

\variantsplitter

\addpersonalvariant{Арсений Трофимов}

\tasknumber{1}%
\task{%
    Определите длину волны (в нм) света, которым освещается поверхность металла,
    если фотоэлектроны имеют максимальную кинетическую энергию $4 \cdot 10^{-20}\,\text{Дж}$,
    а работа выхода электронов из этого металла $7 \cdot 10^{-19}\,\text{Дж}$.
    Постоянная Планка $h = 6{,}626 \cdot 10^{-34}\,\text{Дж}\cdot\text{с}$.
}
\answer{%
    $h \frac c\lambda = A_{\text{вых.}} + E_{\text{кин.}} \implies \lambda = \frac{h c}{A_{\text{вых.}} + E_{\text{кин.}}} = \frac{ 6{,}626 \cdot 10^{-34}\,\text{Дж}\cdot\text{с} \cdot {Const.c:V} }{7 \cdot 10^{-19}\,\text{Дж} + 4 \cdot 10^{-20}\,\text{Дж}} \approx 0{,}27 \cdot 10^{-6}\,\text{м}.$
}
\solutionspace{80pt}

\tasknumber{2}%
\task{%
    Работа выхода электронов из некоторого металла $2{,}1\,\text{эВ}$.
    Найдите скорость электронов (в км/с),
    вылетающих с поверхности металла при освещении его светом с длиной волны $2{,}2 \cdot 10^{-5}\,\text{см}$.
    Масса электрона $m_{e} = 9{,}1 \cdot 10^{-31}\,\text{кг}$.
    Постоянная Планка $h = 6{,}626 \cdot 10^{-34}\,\text{Дж}\cdot\text{с}$, заряд электрона $e = 1{,}6 \cdot 10^{-19}\,\text{Кл}$.
}
\answer{%
    $h \frac c\lambda = A_{\text{вых.}} + \frac{ m_{e}v^2 }2 \implies v = \sqrt{ \frac 2{m_{e}}\cbr{ h \frac c\lambda - A_{\text{вых.}} } } \approx 1116{,}8\,\frac{\text{км}}{\text{c}}.$
}
\solutionspace{80pt}

\tasknumber{3}%
\task{%
    Сколько фотонов испускает за $10\,\text{мин}$ лазер,
    если мощность его излучения $40\,\text{мВт}$?
    Длина волны излучения $750\,\text{нм}$.
    $h = 6{,}626 \cdot 10^{-34}\,\text{Дж}\cdot\text{с}$.
}
\answer{%
    $
        N
            = \frac{E_{\text{общая}}}{E_{\text{одного фотона}}}
            = \frac{Pt}{h\nu} = \frac{Pt}{h \frac c\lambda}
            = \frac{Pt\lambda}{hc}
            = \frac{40\,\text{мВт} \cdot 10\,\text{мин} \cdot 750\,\text{нм}}{6{,}626 \cdot 10^{-34}\,\text{Дж}\cdot\text{с} \cdot 3 \cdot 10^{8}\,\frac{\text{м}}{\text{с}}}
            \approx 90{,}6 \cdot 10^{18}\units{фотонов}
    $
}
\solutionspace{120pt}

\tasknumber{4}%
\task{%
    Определите энергию фотона излучения частотой $6 \cdot 10^{16}\,\text{Гц}$.
    Ответ получите в джоулях и в электронвольтах.
}
\answer{%
    $E = h \nu = 6{,}626 \cdot 10^{-34}\,\text{Дж}\cdot\text{с} \cdot 6 \cdot 10^{16}\,\text{Гц} \approx 40 \cdot 10^{-18}\,\text{Дж} \approx 250\,\text{эВ}$
}
\solutionspace{80pt}

\tasknumber{5}%
\task{%
    Определите энергию кванта света с длиной волны $500\,\text{нм}$.
    Ответ выразите в электронвольтах.
    Способен ли человеческий глаз увидеть один такой квант, а импульс таких квантов?'
}
\answer{%
    $E = h\nu = \frac{hc}{\lambda} = \frac{6{,}626 \cdot 10^{-34}\,\text{Дж}\cdot\text{с} \cdot 3 \cdot 10^{8}\,\frac{\text{м}}{\text{с}}}{500\,\text{нм}} \approx 0{,}398 \cdot 10^{-18}\,\text{Дж} \approx 2{,}48\,\text{эВ}$
}
\solutionspace{80pt}

\tasknumber{6}%
\task{%
    Определите длину волны лучей, фотоны которых имеют энергию
    равную кинетической энергии электрона, ускоренного напряжением $21\,\text{В}$.
}
% autogenerated
