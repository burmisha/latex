\setdate{10~марта~2022}
\setclass{11«Б»}

\addpersonalvariant{Михаил Бурмистров}

\tasknumber{1}%
\task{%
    Красная граница фотоэффекта для некоторого металла соответствует длине волны $6{,}2 \cdot 10^{-7}\,\text{м}$.
    Чему равно напряжение, полностью задерживающее фотоэлектроны, вырываемые из этого металла излучением
    с длиной волны $1{,}7 \cdot 10^{-5}\,\text{см}$? Постоянная Планка $h = 6{,}626 \cdot 10^{-34}\,\text{Дж}\cdot\text{с}$, заряд электрона $e = 1{,}6 \cdot 10^{-19}\,\text{Кл}$.
}
\answer{%
    $eU = K = E - A = \frac{hc}{\lambda} - A, \qquad 0 = \frac{hc}{\lambda_0} - A \implies U = \frac{ \frac{hc}{\lambda} - \frac{hc}{\lambda_0} }{ e } = \frac{hc}{ e }\cbr{ \frac 1{\lambda} - \frac 1{\lambda_0}}  \approx 5\,\text{В}$
}
\solutionspace{80pt}

\tasknumber{2}%
\task{%
    Сколько фотонов испускает за $10\,\text{мин}$ лазер,
    если мощность его излучения $75\,\text{мВт}$?
    Длина волны излучения $500\,\text{нм}$.
    $h = 6{,}626 \cdot 10^{-34}\,\text{Дж}\cdot\text{с}$.
}
\answer{%
    $
        N
            = \frac{E_{\text{общая}}}{E_{\text{одного фотона}}}
            = \frac{Pt}{h\nu} = \frac{Pt}{h \frac c\lambda}
            = \frac{Pt\lambda}{hc}
            = \frac{75\,\text{мВт} \cdot 10\,\text{мин} \cdot 500\,\text{нм}}{6{,}626 \cdot 10^{-34}\,\text{Дж}\cdot\text{с} \cdot 3 \cdot 10^{8}\,\frac{\text{м}}{\text{с}}}
            \approx 113{,}2 \cdot 10^{18}\units{фотонов}
    $
}
\solutionspace{120pt}

\tasknumber{3}%
\task{%
    Определите энергию фотона излучения частотой $7 \cdot 10^{16}\,\text{Гц}$.
    Ответ получите в джоулях и в электронвольтах.
}
\answer{%
    $E = h \nu = 6{,}626 \cdot 10^{-34}\,\text{Дж}\cdot\text{с} \cdot 7 \cdot 10^{16}\,\text{Гц} \approx 46 \cdot 10^{-18}\,\text{Дж} \approx 290\,\text{эВ}$
}
\solutionspace{80pt}

\tasknumber{4}%
\task{%
    Определите энергию фотона с длиной волны $500\,\text{нм}$.
    Ответ выразите в джоулях.
    Способен ли человеческий глаз увидеть один такой квант, а импульс таких квантов?'
}
\answer{%
    $E = h\nu = \frac{hc}{\lambda} = \frac{6{,}626 \cdot 10^{-34}\,\text{Дж}\cdot\text{с} \cdot 3 \cdot 10^{8}\,\frac{\text{м}}{\text{с}}}{500\,\text{нм}} \approx 0{,}398 \cdot 10^{-18}\,\text{Дж} \approx 2{,}48\,\text{эВ}$
}
\solutionspace{80pt}

\tasknumber{5}%
\task{%
    Из формулы Планка выразите (нужен вывод, не только ответ)...
    \begin{enumerate}
        \item длину соответствующей электромагнитной волны,
        \item период колебаний индукции магнитного поля в соответствующей электромагнитной волне.
    \end{enumerate}
}
\solutionspace{40pt}

\tasknumber{6}%
\task{%
    Определите длину волны лучей, фотоны которых имеют энергию
    равную кинетической энергии электрона, ускоренного напряжением $377\,\text{В}$.
}
\answer{%
    $E = h\frac c\lambda = e U \implies \lambda = \frac{hc}{eU} = \frac{6{,}626 \cdot 10^{-34}\,\text{Дж}\cdot\text{с} \cdot 3 \cdot 10^{8}\,\frac{\text{м}}{\text{с}}}{1{,}6 \cdot 10^{-19}\,\text{Кл} \cdot 377\,\text{В}} \approx 3{,}30\,\text{нм}.$
}

\variantsplitter

\addpersonalvariant{Снежана Авдошина}

\tasknumber{1}%
\task{%
    Красная граница фотоэффекта для некоторого металла соответствует длине волны $5{,}7 \cdot 10^{-7}\,\text{м}$.
    Чему равно напряжение, полностью задерживающее фотоэлектроны, вырываемые из этого металла излучением
    с длиной волны $2{,}2 \cdot 10^{-5}\,\text{см}$? Постоянная Планка $h = 6{,}626 \cdot 10^{-34}\,\text{Дж}\cdot\text{с}$, заряд электрона $e = 1{,}6 \cdot 10^{-19}\,\text{Кл}$.
}
\answer{%
    $eU = K = E - A = \frac{hc}{\lambda} - A, \qquad 0 = \frac{hc}{\lambda_0} - A \implies U = \frac{ \frac{hc}{\lambda} - \frac{hc}{\lambda_0} }{ e } = \frac{hc}{ e }\cbr{ \frac 1{\lambda} - \frac 1{\lambda_0}}  \approx 3{,}5\,\text{В}$
}
\solutionspace{80pt}

\tasknumber{2}%
\task{%
    Сколько фотонов испускает за $20\,\text{мин}$ лазер,
    если мощность его излучения $75\,\text{мВт}$?
    Длина волны излучения $500\,\text{нм}$.
    $h = 6{,}626 \cdot 10^{-34}\,\text{Дж}\cdot\text{с}$.
}
\answer{%
    $
        N
            = \frac{E_{\text{общая}}}{E_{\text{одного фотона}}}
            = \frac{Pt}{h\nu} = \frac{Pt}{h \frac c\lambda}
            = \frac{Pt\lambda}{hc}
            = \frac{75\,\text{мВт} \cdot 20\,\text{мин} \cdot 500\,\text{нм}}{6{,}626 \cdot 10^{-34}\,\text{Дж}\cdot\text{с} \cdot 3 \cdot 10^{8}\,\frac{\text{м}}{\text{с}}}
            \approx 226{,}4 \cdot 10^{18}\units{фотонов}
    $
}
\solutionspace{120pt}

\tasknumber{3}%
\task{%
    Определите энергию фотона излучения частотой $6 \cdot 10^{16}\,\text{Гц}$.
    Ответ получите в джоулях и в электронвольтах.
}
\answer{%
    $E = h \nu = 6{,}626 \cdot 10^{-34}\,\text{Дж}\cdot\text{с} \cdot 6 \cdot 10^{16}\,\text{Гц} \approx 40 \cdot 10^{-18}\,\text{Дж} \approx 250\,\text{эВ}$
}
\solutionspace{80pt}

\tasknumber{4}%
\task{%
    Определите энергию фотона с длиной волны $400\,\text{нм}$.
    Ответ выразите в джоулях.
    Способен ли человеческий глаз увидеть один такой квант, а импульс таких квантов?'
}
\answer{%
    $E = h\nu = \frac{hc}{\lambda} = \frac{6{,}626 \cdot 10^{-34}\,\text{Дж}\cdot\text{с} \cdot 3 \cdot 10^{8}\,\frac{\text{м}}{\text{с}}}{400\,\text{нм}} \approx 0{,}497 \cdot 10^{-18}\,\text{Дж} \approx 3{,}11\,\text{эВ}$
}
\solutionspace{80pt}

\tasknumber{5}%
\task{%
    Из формулы Планка выразите (нужен вывод, не только ответ)...
    \begin{enumerate}
        \item длину соответствующей электромагнитной волны,
        \item период колебаний индукции магнитного поля в соответствующей электромагнитной волне.
    \end{enumerate}
}
\solutionspace{40pt}

\tasknumber{6}%
\task{%
    Определите длину волны лучей, фотоны которых имеют энергию
    равную кинетической энергии электрона, ускоренного напряжением $3\,\text{В}$.
}
\answer{%
    $E = h\frac c\lambda = e U \implies \lambda = \frac{hc}{eU} = \frac{6{,}626 \cdot 10^{-34}\,\text{Дж}\cdot\text{с} \cdot 3 \cdot 10^{8}\,\frac{\text{м}}{\text{с}}}{1{,}6 \cdot 10^{-19}\,\text{Кл} \cdot 3\,\text{В}} \approx 410\,\text{нм}.$
}

\variantsplitter

\addpersonalvariant{Марьяна Аристова}

\tasknumber{1}%
\task{%
    Красная граница фотоэффекта для некоторого металла соответствует длине волны $5{,}7 \cdot 10^{-7}\,\text{м}$.
    Чему равно напряжение, полностью задерживающее фотоэлектроны, вырываемые из этого металла излучением
    с длиной волны $3{,}2 \cdot 10^{-5}\,\text{см}$? Постоянная Планка $h = 6{,}626 \cdot 10^{-34}\,\text{Дж}\cdot\text{с}$, заряд электрона $e = 1{,}6 \cdot 10^{-19}\,\text{Кл}$.
}
\answer{%
    $eU = K = E - A = \frac{hc}{\lambda} - A, \qquad 0 = \frac{hc}{\lambda_0} - A \implies U = \frac{ \frac{hc}{\lambda} - \frac{hc}{\lambda_0} }{ e } = \frac{hc}{ e }\cbr{ \frac 1{\lambda} - \frac 1{\lambda_0}}  \approx 1{,}70\,\text{В}$
}
\solutionspace{80pt}

\tasknumber{2}%
\task{%
    Сколько фотонов испускает за $5\,\text{мин}$ лазер,
    если мощность его излучения $75\,\text{мВт}$?
    Длина волны излучения $500\,\text{нм}$.
    $h = 6{,}626 \cdot 10^{-34}\,\text{Дж}\cdot\text{с}$.
}
\answer{%
    $
        N
            = \frac{E_{\text{общая}}}{E_{\text{одного фотона}}}
            = \frac{Pt}{h\nu} = \frac{Pt}{h \frac c\lambda}
            = \frac{Pt\lambda}{hc}
            = \frac{75\,\text{мВт} \cdot 5\,\text{мин} \cdot 500\,\text{нм}}{6{,}626 \cdot 10^{-34}\,\text{Дж}\cdot\text{с} \cdot 3 \cdot 10^{8}\,\frac{\text{м}}{\text{с}}}
            \approx 56{,}6 \cdot 10^{18}\units{фотонов}
    $
}
\solutionspace{120pt}

\tasknumber{3}%
\task{%
    Определите энергию фотона излучения частотой $9 \cdot 10^{16}\,\text{Гц}$.
    Ответ получите в джоулях и в электронвольтах.
}
\answer{%
    $E = h \nu = 6{,}626 \cdot 10^{-34}\,\text{Дж}\cdot\text{с} \cdot 9 \cdot 10^{16}\,\text{Гц} \approx 60 \cdot 10^{-18}\,\text{Дж} \approx 370\,\text{эВ}$
}
\solutionspace{80pt}

\tasknumber{4}%
\task{%
    Определите энергию кванта света с длиной волны $850\,\text{нм}$.
    Ответ выразите в электронвольтах.
    Способен ли человеческий глаз увидеть один такой квант, а импульс таких квантов?'
}
\answer{%
    $E = h\nu = \frac{hc}{\lambda} = \frac{6{,}626 \cdot 10^{-34}\,\text{Дж}\cdot\text{с} \cdot 3 \cdot 10^{8}\,\frac{\text{м}}{\text{с}}}{850\,\text{нм}} \approx 0{,}234 \cdot 10^{-18}\,\text{Дж} \approx 1{,}462\,\text{эВ}$
}
\solutionspace{80pt}

\tasknumber{5}%
\task{%
    Из формулы Планка выразите (нужен вывод, не только ответ)...
    \begin{enumerate}
        \item длину соответствующей электромагнитной волны,
        \item период колебаний электрического поля в соответствующей электромагнитной волне.
    \end{enumerate}
}
\solutionspace{40pt}

\tasknumber{6}%
\task{%
    Определите длину волны лучей, фотоны которых имеют энергию
    равную кинетической энергии электрона, ускоренного напряжением $2\,\text{В}$.
}
\answer{%
    $E = h\frac c\lambda = e U \implies \lambda = \frac{hc}{eU} = \frac{6{,}626 \cdot 10^{-34}\,\text{Дж}\cdot\text{с} \cdot 3 \cdot 10^{8}\,\frac{\text{м}}{\text{с}}}{1{,}6 \cdot 10^{-19}\,\text{Кл} \cdot 2\,\text{В}} \approx 620\,\text{нм}.$
}

\variantsplitter

\addpersonalvariant{Никита Иванов}

\tasknumber{1}%
\task{%
    Красная граница фотоэффекта для некоторого металла соответствует длине волны $5{,}7 \cdot 10^{-7}\,\text{м}$.
    Чему равно напряжение, полностью задерживающее фотоэлектроны, вырываемые из этого металла излучением
    с длиной волны $3{,}2 \cdot 10^{-5}\,\text{см}$? Постоянная Планка $h = 6{,}626 \cdot 10^{-34}\,\text{Дж}\cdot\text{с}$, заряд электрона $e = 1{,}6 \cdot 10^{-19}\,\text{Кл}$.
}
\answer{%
    $eU = K = E - A = \frac{hc}{\lambda} - A, \qquad 0 = \frac{hc}{\lambda_0} - A \implies U = \frac{ \frac{hc}{\lambda} - \frac{hc}{\lambda_0} }{ e } = \frac{hc}{ e }\cbr{ \frac 1{\lambda} - \frac 1{\lambda_0}}  \approx 1{,}70\,\text{В}$
}
\solutionspace{80pt}

\tasknumber{2}%
\task{%
    Сколько фотонов испускает за $40\,\text{мин}$ лазер,
    если мощность его излучения $40\,\text{мВт}$?
    Длина волны излучения $750\,\text{нм}$.
    $h = 6{,}626 \cdot 10^{-34}\,\text{Дж}\cdot\text{с}$.
}
\answer{%
    $
        N
            = \frac{E_{\text{общая}}}{E_{\text{одного фотона}}}
            = \frac{Pt}{h\nu} = \frac{Pt}{h \frac c\lambda}
            = \frac{Pt\lambda}{hc}
            = \frac{40\,\text{мВт} \cdot 40\,\text{мин} \cdot 750\,\text{нм}}{6{,}626 \cdot 10^{-34}\,\text{Дж}\cdot\text{с} \cdot 3 \cdot 10^{8}\,\frac{\text{м}}{\text{с}}}
            \approx 362{,}2 \cdot 10^{18}\units{фотонов}
    $
}
\solutionspace{120pt}

\tasknumber{3}%
\task{%
    Определите энергию фотона излучения частотой $6 \cdot 10^{16}\,\text{Гц}$.
    Ответ получите в джоулях и в электронвольтах.
}
\answer{%
    $E = h \nu = 6{,}626 \cdot 10^{-34}\,\text{Дж}\cdot\text{с} \cdot 6 \cdot 10^{16}\,\text{Гц} \approx 40 \cdot 10^{-18}\,\text{Дж} \approx 250\,\text{эВ}$
}
\solutionspace{80pt}

\tasknumber{4}%
\task{%
    Определите энергию кванта света с длиной волны $850\,\text{нм}$.
    Ответ выразите в джоулях.
    Способен ли человеческий глаз увидеть один такой квант, а импульс таких квантов?'
}
\answer{%
    $E = h\nu = \frac{hc}{\lambda} = \frac{6{,}626 \cdot 10^{-34}\,\text{Дж}\cdot\text{с} \cdot 3 \cdot 10^{8}\,\frac{\text{м}}{\text{с}}}{850\,\text{нм}} \approx 0{,}234 \cdot 10^{-18}\,\text{Дж} \approx 1{,}462\,\text{эВ}$
}
\solutionspace{80pt}

\tasknumber{5}%
\task{%
    Из формулы Планка выразите (нужен вывод, не только ответ)...
    \begin{enumerate}
        \item длину соответствующей электромагнитной волны,
        \item период колебаний индукции магнитного поля в соответствующей электромагнитной волне.
    \end{enumerate}
}
\solutionspace{40pt}

\tasknumber{6}%
\task{%
    Определите длину волны лучей, фотоны которых имеют энергию
    равную кинетической энергии электрона, ускоренного напряжением $610\,\text{В}$.
}
\answer{%
    $E = h\frac c\lambda = e U \implies \lambda = \frac{hc}{eU} = \frac{6{,}626 \cdot 10^{-34}\,\text{Дж}\cdot\text{с} \cdot 3 \cdot 10^{8}\,\frac{\text{м}}{\text{с}}}{1{,}6 \cdot 10^{-19}\,\text{Кл} \cdot 610\,\text{В}} \approx 2{,}04\,\text{нм}.$
}

\variantsplitter

\addpersonalvariant{Анастасия Князева}

\tasknumber{1}%
\task{%
    Красная граница фотоэффекта для некоторого металла соответствует длине волны $6{,}2 \cdot 10^{-7}\,\text{м}$.
    Чему равно напряжение, полностью задерживающее фотоэлектроны, вырываемые из этого металла излучением
    с длиной волны $2{,}2 \cdot 10^{-5}\,\text{см}$? Постоянная Планка $h = 6{,}626 \cdot 10^{-34}\,\text{Дж}\cdot\text{с}$, заряд электрона $e = 1{,}6 \cdot 10^{-19}\,\text{Кл}$.
}
\answer{%
    $eU = K = E - A = \frac{hc}{\lambda} - A, \qquad 0 = \frac{hc}{\lambda_0} - A \implies U = \frac{ \frac{hc}{\lambda} - \frac{hc}{\lambda_0} }{ e } = \frac{hc}{ e }\cbr{ \frac 1{\lambda} - \frac 1{\lambda_0}}  \approx 3{,}6\,\text{В}$
}
\solutionspace{80pt}

\tasknumber{2}%
\task{%
    Сколько фотонов испускает за $60\,\text{мин}$ лазер,
    если мощность его излучения $15\,\text{мВт}$?
    Длина волны излучения $600\,\text{нм}$.
    $h = 6{,}626 \cdot 10^{-34}\,\text{Дж}\cdot\text{с}$.
}
\answer{%
    $
        N
            = \frac{E_{\text{общая}}}{E_{\text{одного фотона}}}
            = \frac{Pt}{h\nu} = \frac{Pt}{h \frac c\lambda}
            = \frac{Pt\lambda}{hc}
            = \frac{15\,\text{мВт} \cdot 60\,\text{мин} \cdot 600\,\text{нм}}{6{,}626 \cdot 10^{-34}\,\text{Дж}\cdot\text{с} \cdot 3 \cdot 10^{8}\,\frac{\text{м}}{\text{с}}}
            \approx 163{,}0 \cdot 10^{18}\units{фотонов}
    $
}
\solutionspace{120pt}

\tasknumber{3}%
\task{%
    Определите энергию фотона излучения частотой $8 \cdot 10^{16}\,\text{Гц}$.
    Ответ получите в джоулях и в электронвольтах.
}
\answer{%
    $E = h \nu = 6{,}626 \cdot 10^{-34}\,\text{Дж}\cdot\text{с} \cdot 8 \cdot 10^{16}\,\text{Гц} \approx 53 \cdot 10^{-18}\,\text{Дж} \approx 330\,\text{эВ}$
}
\solutionspace{80pt}

\tasknumber{4}%
\task{%
    Определите энергию фотона с длиной волны $200\,\text{нм}$.
    Ответ выразите в джоулях.
    Способен ли человеческий глаз увидеть один такой квант, а импульс таких квантов?'
}
\answer{%
    $E = h\nu = \frac{hc}{\lambda} = \frac{6{,}626 \cdot 10^{-34}\,\text{Дж}\cdot\text{с} \cdot 3 \cdot 10^{8}\,\frac{\text{м}}{\text{с}}}{200\,\text{нм}} \approx 0{,}994 \cdot 10^{-18}\,\text{Дж} \approx 6{,}21\,\text{эВ}$
}
\solutionspace{80pt}

\tasknumber{5}%
\task{%
    Из формулы Планка выразите (нужен вывод, не только ответ)...
    \begin{enumerate}
        \item длину соответствующей электромагнитной волны,
        \item период колебаний электрического поля в соответствующей электромагнитной волне.
    \end{enumerate}
}
\solutionspace{40pt}

\tasknumber{6}%
\task{%
    Определите длину волны лучей, фотоны которых имеют энергию
    равную кинетической энергии электрона, ускоренного напряжением $144\,\text{В}$.
}
\answer{%
    $E = h\frac c\lambda = e U \implies \lambda = \frac{hc}{eU} = \frac{6{,}626 \cdot 10^{-34}\,\text{Дж}\cdot\text{с} \cdot 3 \cdot 10^{8}\,\frac{\text{м}}{\text{с}}}{1{,}6 \cdot 10^{-19}\,\text{Кл} \cdot 144\,\text{В}} \approx 8{,}63\,\text{нм}.$
}

\variantsplitter

\addpersonalvariant{Елизавета Кутумова}

\tasknumber{1}%
\task{%
    Красная граница фотоэффекта для некоторого металла соответствует длине волны $6{,}6 \cdot 10^{-7}\,\text{м}$.
    Чему равно напряжение, полностью задерживающее фотоэлектроны, вырываемые из этого металла излучением
    с длиной волны $3{,}2 \cdot 10^{-5}\,\text{см}$? Постоянная Планка $h = 6{,}626 \cdot 10^{-34}\,\text{Дж}\cdot\text{с}$, заряд электрона $e = 1{,}6 \cdot 10^{-19}\,\text{Кл}$.
}
\answer{%
    $eU = K = E - A = \frac{hc}{\lambda} - A, \qquad 0 = \frac{hc}{\lambda_0} - A \implies U = \frac{ \frac{hc}{\lambda} - \frac{hc}{\lambda_0} }{ e } = \frac{hc}{ e }\cbr{ \frac 1{\lambda} - \frac 1{\lambda_0}}  \approx 2{,}0\,\text{В}$
}
\solutionspace{80pt}

\tasknumber{2}%
\task{%
    Сколько фотонов испускает за $10\,\text{мин}$ лазер,
    если мощность его излучения $15\,\text{мВт}$?
    Длина волны излучения $500\,\text{нм}$.
    $h = 6{,}626 \cdot 10^{-34}\,\text{Дж}\cdot\text{с}$.
}
\answer{%
    $
        N
            = \frac{E_{\text{общая}}}{E_{\text{одного фотона}}}
            = \frac{Pt}{h\nu} = \frac{Pt}{h \frac c\lambda}
            = \frac{Pt\lambda}{hc}
            = \frac{15\,\text{мВт} \cdot 10\,\text{мин} \cdot 500\,\text{нм}}{6{,}626 \cdot 10^{-34}\,\text{Дж}\cdot\text{с} \cdot 3 \cdot 10^{8}\,\frac{\text{м}}{\text{с}}}
            \approx 22{,}6 \cdot 10^{18}\units{фотонов}
    $
}
\solutionspace{120pt}

\tasknumber{3}%
\task{%
    Определите энергию фотона излучения частотой $4 \cdot 10^{16}\,\text{Гц}$.
    Ответ получите в джоулях и в электронвольтах.
}
\answer{%
    $E = h \nu = 6{,}626 \cdot 10^{-34}\,\text{Дж}\cdot\text{с} \cdot 4 \cdot 10^{16}\,\text{Гц} \approx 27 \cdot 10^{-18}\,\text{Дж} \approx 166\,\text{эВ}$
}
\solutionspace{80pt}

\tasknumber{4}%
\task{%
    Определите энергию фотона с длиной волны $600\,\text{нм}$.
    Ответ выразите в джоулях.
    Способен ли человеческий глаз увидеть один такой квант, а импульс таких квантов?'
}
\answer{%
    $E = h\nu = \frac{hc}{\lambda} = \frac{6{,}626 \cdot 10^{-34}\,\text{Дж}\cdot\text{с} \cdot 3 \cdot 10^{8}\,\frac{\text{м}}{\text{с}}}{600\,\text{нм}} \approx 0{,}331 \cdot 10^{-18}\,\text{Дж} \approx 2{,}07\,\text{эВ}$
}
\solutionspace{80pt}

\tasknumber{5}%
\task{%
    Из формулы Планка выразите (нужен вывод, не только ответ)...
    \begin{enumerate}
        \item длину соответствующей электромагнитной волны,
        \item период колебаний индукции магнитного поля в соответствующей электромагнитной волне.
    \end{enumerate}
}
\solutionspace{40pt}

\tasknumber{6}%
\task{%
    Определите длину волны лучей, фотоны которых имеют энергию
    равную кинетической энергии электрона, ускоренного напряжением $377\,\text{В}$.
}
\answer{%
    $E = h\frac c\lambda = e U \implies \lambda = \frac{hc}{eU} = \frac{6{,}626 \cdot 10^{-34}\,\text{Дж}\cdot\text{с} \cdot 3 \cdot 10^{8}\,\frac{\text{м}}{\text{с}}}{1{,}6 \cdot 10^{-19}\,\text{Кл} \cdot 377\,\text{В}} \approx 3{,}30\,\text{нм}.$
}

\variantsplitter

\addpersonalvariant{Роксана Мехтиева}

\tasknumber{1}%
\task{%
    Красная граница фотоэффекта для некоторого металла соответствует длине волны $6{,}6 \cdot 10^{-7}\,\text{м}$.
    Чему равно напряжение, полностью задерживающее фотоэлектроны, вырываемые из этого металла излучением
    с длиной волны $1{,}7 \cdot 10^{-5}\,\text{см}$? Постоянная Планка $h = 6{,}626 \cdot 10^{-34}\,\text{Дж}\cdot\text{с}$, заряд электрона $e = 1{,}6 \cdot 10^{-19}\,\text{Кл}$.
}
\answer{%
    $eU = K = E - A = \frac{hc}{\lambda} - A, \qquad 0 = \frac{hc}{\lambda_0} - A \implies U = \frac{ \frac{hc}{\lambda} - \frac{hc}{\lambda_0} }{ e } = \frac{hc}{ e }\cbr{ \frac 1{\lambda} - \frac 1{\lambda_0}}  \approx 5\,\text{В}$
}
\solutionspace{80pt}

\tasknumber{2}%
\task{%
    Сколько фотонов испускает за $10\,\text{мин}$ лазер,
    если мощность его излучения $15\,\text{мВт}$?
    Длина волны излучения $750\,\text{нм}$.
    $h = 6{,}626 \cdot 10^{-34}\,\text{Дж}\cdot\text{с}$.
}
\answer{%
    $
        N
            = \frac{E_{\text{общая}}}{E_{\text{одного фотона}}}
            = \frac{Pt}{h\nu} = \frac{Pt}{h \frac c\lambda}
            = \frac{Pt\lambda}{hc}
            = \frac{15\,\text{мВт} \cdot 10\,\text{мин} \cdot 750\,\text{нм}}{6{,}626 \cdot 10^{-34}\,\text{Дж}\cdot\text{с} \cdot 3 \cdot 10^{8}\,\frac{\text{м}}{\text{с}}}
            \approx 34{,}0 \cdot 10^{18}\units{фотонов}
    $
}
\solutionspace{120pt}

\tasknumber{3}%
\task{%
    Определите энергию фотона излучения частотой $6 \cdot 10^{16}\,\text{Гц}$.
    Ответ получите в джоулях и в электронвольтах.
}
\answer{%
    $E = h \nu = 6{,}626 \cdot 10^{-34}\,\text{Дж}\cdot\text{с} \cdot 6 \cdot 10^{16}\,\text{Гц} \approx 40 \cdot 10^{-18}\,\text{Дж} \approx 250\,\text{эВ}$
}
\solutionspace{80pt}

\tasknumber{4}%
\task{%
    Определите энергию кванта света с длиной волны $850\,\text{нм}$.
    Ответ выразите в джоулях.
    Способен ли человеческий глаз увидеть один такой квант, а импульс таких квантов?'
}
\answer{%
    $E = h\nu = \frac{hc}{\lambda} = \frac{6{,}626 \cdot 10^{-34}\,\text{Дж}\cdot\text{с} \cdot 3 \cdot 10^{8}\,\frac{\text{м}}{\text{с}}}{850\,\text{нм}} \approx 0{,}234 \cdot 10^{-18}\,\text{Дж} \approx 1{,}462\,\text{эВ}$
}
\solutionspace{80pt}

\tasknumber{5}%
\task{%
    Из формулы Планка выразите (нужен вывод, не только ответ)...
    \begin{enumerate}
        \item длину соответствующей электромагнитной волны,
        \item период колебаний индукции магнитного поля в соответствующей электромагнитной волне.
    \end{enumerate}
}
\solutionspace{40pt}

\tasknumber{6}%
\task{%
    Определите длину волны лучей, фотоны которых имеют энергию
    равную кинетической энергии электрона, ускоренного напряжением $34\,\text{В}$.
}
\answer{%
    $E = h\frac c\lambda = e U \implies \lambda = \frac{hc}{eU} = \frac{6{,}626 \cdot 10^{-34}\,\text{Дж}\cdot\text{с} \cdot 3 \cdot 10^{8}\,\frac{\text{м}}{\text{с}}}{1{,}6 \cdot 10^{-19}\,\text{Кл} \cdot 34\,\text{В}} \approx 36{,}5\,\text{нм}.$
}

\variantsplitter

\addpersonalvariant{Дилноза Нодиршоева}

\tasknumber{1}%
\task{%
    Красная граница фотоэффекта для некоторого металла соответствует длине волны $5{,}3 \cdot 10^{-7}\,\text{м}$.
    Чему равно напряжение, полностью задерживающее фотоэлектроны, вырываемые из этого металла излучением
    с длиной волны $1{,}7 \cdot 10^{-5}\,\text{см}$? Постоянная Планка $h = 6{,}626 \cdot 10^{-34}\,\text{Дж}\cdot\text{с}$, заряд электрона $e = 1{,}6 \cdot 10^{-19}\,\text{Кл}$.
}
\answer{%
    $eU = K = E - A = \frac{hc}{\lambda} - A, \qquad 0 = \frac{hc}{\lambda_0} - A \implies U = \frac{ \frac{hc}{\lambda} - \frac{hc}{\lambda_0} }{ e } = \frac{hc}{ e }\cbr{ \frac 1{\lambda} - \frac 1{\lambda_0}}  \approx 5\,\text{В}$
}
\solutionspace{80pt}

\tasknumber{2}%
\task{%
    Сколько фотонов испускает за $5\,\text{мин}$ лазер,
    если мощность его излучения $200\,\text{мВт}$?
    Длина волны излучения $500\,\text{нм}$.
    $h = 6{,}626 \cdot 10^{-34}\,\text{Дж}\cdot\text{с}$.
}
\answer{%
    $
        N
            = \frac{E_{\text{общая}}}{E_{\text{одного фотона}}}
            = \frac{Pt}{h\nu} = \frac{Pt}{h \frac c\lambda}
            = \frac{Pt\lambda}{hc}
            = \frac{200\,\text{мВт} \cdot 5\,\text{мин} \cdot 500\,\text{нм}}{6{,}626 \cdot 10^{-34}\,\text{Дж}\cdot\text{с} \cdot 3 \cdot 10^{8}\,\frac{\text{м}}{\text{с}}}
            \approx 150{,}9 \cdot 10^{18}\units{фотонов}
    $
}
\solutionspace{120pt}

\tasknumber{3}%
\task{%
    Определите энергию фотона излучения частотой $4 \cdot 10^{16}\,\text{Гц}$.
    Ответ получите в джоулях и в электронвольтах.
}
\answer{%
    $E = h \nu = 6{,}626 \cdot 10^{-34}\,\text{Дж}\cdot\text{с} \cdot 4 \cdot 10^{16}\,\text{Гц} \approx 27 \cdot 10^{-18}\,\text{Дж} \approx 166\,\text{эВ}$
}
\solutionspace{80pt}

\tasknumber{4}%
\task{%
    Определите энергию фотона с длиной волны $150\,\text{нм}$.
    Ответ выразите в джоулях.
    Способен ли человеческий глаз увидеть один такой квант, а импульс таких квантов?'
}
\answer{%
    $E = h\nu = \frac{hc}{\lambda} = \frac{6{,}626 \cdot 10^{-34}\,\text{Дж}\cdot\text{с} \cdot 3 \cdot 10^{8}\,\frac{\text{м}}{\text{с}}}{150\,\text{нм}} \approx 1{,}33 \cdot 10^{-18}\,\text{Дж} \approx 8{,}3\,\text{эВ}$
}
\solutionspace{80pt}

\tasknumber{5}%
\task{%
    Из формулы Планка выразите (нужен вывод, не только ответ)...
    \begin{enumerate}
        \item длину соответствующей электромагнитной волны,
        \item период колебаний электрического поля в соответствующей электромагнитной волне.
    \end{enumerate}
}
\solutionspace{40pt}

\tasknumber{6}%
\task{%
    Определите длину волны лучей, фотоны которых имеют энергию
    равную кинетической энергии электрона, ускоренного напряжением $13\,\text{В}$.
}
\answer{%
    $E = h\frac c\lambda = e U \implies \lambda = \frac{hc}{eU} = \frac{6{,}626 \cdot 10^{-34}\,\text{Дж}\cdot\text{с} \cdot 3 \cdot 10^{8}\,\frac{\text{м}}{\text{с}}}{1{,}6 \cdot 10^{-19}\,\text{Кл} \cdot 13\,\text{В}} \approx 96\,\text{нм}.$
}

\variantsplitter

\addpersonalvariant{Жаклин Пантелеева}

\tasknumber{1}%
\task{%
    Красная граница фотоэффекта для некоторого металла соответствует длине волны $5{,}7 \cdot 10^{-7}\,\text{м}$.
    Чему равно напряжение, полностью задерживающее фотоэлектроны, вырываемые из этого металла излучением
    с длиной волны $3{,}2 \cdot 10^{-5}\,\text{см}$? Постоянная Планка $h = 6{,}626 \cdot 10^{-34}\,\text{Дж}\cdot\text{с}$, заряд электрона $e = 1{,}6 \cdot 10^{-19}\,\text{Кл}$.
}
\answer{%
    $eU = K = E - A = \frac{hc}{\lambda} - A, \qquad 0 = \frac{hc}{\lambda_0} - A \implies U = \frac{ \frac{hc}{\lambda} - \frac{hc}{\lambda_0} }{ e } = \frac{hc}{ e }\cbr{ \frac 1{\lambda} - \frac 1{\lambda_0}}  \approx 1{,}70\,\text{В}$
}
\solutionspace{80pt}

\tasknumber{2}%
\task{%
    Сколько фотонов испускает за $10\,\text{мин}$ лазер,
    если мощность его излучения $75\,\text{мВт}$?
    Длина волны излучения $750\,\text{нм}$.
    $h = 6{,}626 \cdot 10^{-34}\,\text{Дж}\cdot\text{с}$.
}
\answer{%
    $
        N
            = \frac{E_{\text{общая}}}{E_{\text{одного фотона}}}
            = \frac{Pt}{h\nu} = \frac{Pt}{h \frac c\lambda}
            = \frac{Pt\lambda}{hc}
            = \frac{75\,\text{мВт} \cdot 10\,\text{мин} \cdot 750\,\text{нм}}{6{,}626 \cdot 10^{-34}\,\text{Дж}\cdot\text{с} \cdot 3 \cdot 10^{8}\,\frac{\text{м}}{\text{с}}}
            \approx 169{,}8 \cdot 10^{18}\units{фотонов}
    $
}
\solutionspace{120pt}

\tasknumber{3}%
\task{%
    Определите энергию фотона излучения частотой $5 \cdot 10^{16}\,\text{Гц}$.
    Ответ получите в джоулях и в электронвольтах.
}
\answer{%
    $E = h \nu = 6{,}626 \cdot 10^{-34}\,\text{Дж}\cdot\text{с} \cdot 5 \cdot 10^{16}\,\text{Гц} \approx 33 \cdot 10^{-18}\,\text{Дж} \approx 210\,\text{эВ}$
}
\solutionspace{80pt}

\tasknumber{4}%
\task{%
    Определите энергию фотона с длиной волны $900\,\text{нм}$.
    Ответ выразите в джоулях.
    Способен ли человеческий глаз увидеть один такой квант, а импульс таких квантов?'
}
\answer{%
    $E = h\nu = \frac{hc}{\lambda} = \frac{6{,}626 \cdot 10^{-34}\,\text{Дж}\cdot\text{с} \cdot 3 \cdot 10^{8}\,\frac{\text{м}}{\text{с}}}{900\,\text{нм}} \approx 0{,}221 \cdot 10^{-18}\,\text{Дж} \approx 1{,}380\,\text{эВ}$
}
\solutionspace{80pt}

\tasknumber{5}%
\task{%
    Из формулы Планка выразите (нужен вывод, не только ответ)...
    \begin{enumerate}
        \item длину соответствующей электромагнитной волны,
        \item период колебаний электрического поля в соответствующей электромагнитной волне.
    \end{enumerate}
}
\solutionspace{40pt}

\tasknumber{6}%
\task{%
    Определите длину волны лучей, фотоны которых имеют энергию
    равную кинетической энергии электрона, ускоренного напряжением $2\,\text{В}$.
}
\answer{%
    $E = h\frac c\lambda = e U \implies \lambda = \frac{hc}{eU} = \frac{6{,}626 \cdot 10^{-34}\,\text{Дж}\cdot\text{с} \cdot 3 \cdot 10^{8}\,\frac{\text{м}}{\text{с}}}{1{,}6 \cdot 10^{-19}\,\text{Кл} \cdot 2\,\text{В}} \approx 620\,\text{нм}.$
}

\variantsplitter

\addpersonalvariant{Артём Переверзев}

\tasknumber{1}%
\task{%
    Красная граница фотоэффекта для некоторого металла соответствует длине волны $6{,}2 \cdot 10^{-7}\,\text{м}$.
    Чему равно напряжение, полностью задерживающее фотоэлектроны, вырываемые из этого металла излучением
    с длиной волны $1{,}7 \cdot 10^{-5}\,\text{см}$? Постоянная Планка $h = 6{,}626 \cdot 10^{-34}\,\text{Дж}\cdot\text{с}$, заряд электрона $e = 1{,}6 \cdot 10^{-19}\,\text{Кл}$.
}
\answer{%
    $eU = K = E - A = \frac{hc}{\lambda} - A, \qquad 0 = \frac{hc}{\lambda_0} - A \implies U = \frac{ \frac{hc}{\lambda} - \frac{hc}{\lambda_0} }{ e } = \frac{hc}{ e }\cbr{ \frac 1{\lambda} - \frac 1{\lambda_0}}  \approx 5\,\text{В}$
}
\solutionspace{80pt}

\tasknumber{2}%
\task{%
    Сколько фотонов испускает за $120\,\text{мин}$ лазер,
    если мощность его излучения $15\,\text{мВт}$?
    Длина волны излучения $750\,\text{нм}$.
    $h = 6{,}626 \cdot 10^{-34}\,\text{Дж}\cdot\text{с}$.
}
\answer{%
    $
        N
            = \frac{E_{\text{общая}}}{E_{\text{одного фотона}}}
            = \frac{Pt}{h\nu} = \frac{Pt}{h \frac c\lambda}
            = \frac{Pt\lambda}{hc}
            = \frac{15\,\text{мВт} \cdot 120\,\text{мин} \cdot 750\,\text{нм}}{6{,}626 \cdot 10^{-34}\,\text{Дж}\cdot\text{с} \cdot 3 \cdot 10^{8}\,\frac{\text{м}}{\text{с}}}
            \approx 407 \cdot 10^{18}\units{фотонов}
    $
}
\solutionspace{120pt}

\tasknumber{3}%
\task{%
    Определите энергию фотона излучения частотой $5 \cdot 10^{16}\,\text{Гц}$.
    Ответ получите в джоулях и в электронвольтах.
}
\answer{%
    $E = h \nu = 6{,}626 \cdot 10^{-34}\,\text{Дж}\cdot\text{с} \cdot 5 \cdot 10^{16}\,\text{Гц} \approx 33 \cdot 10^{-18}\,\text{Дж} \approx 210\,\text{эВ}$
}
\solutionspace{80pt}

\tasknumber{4}%
\task{%
    Определите энергию фотона с длиной волны $400\,\text{нм}$.
    Ответ выразите в электронвольтах.
    Способен ли человеческий глаз увидеть один такой квант, а импульс таких квантов?'
}
\answer{%
    $E = h\nu = \frac{hc}{\lambda} = \frac{6{,}626 \cdot 10^{-34}\,\text{Дж}\cdot\text{с} \cdot 3 \cdot 10^{8}\,\frac{\text{м}}{\text{с}}}{400\,\text{нм}} \approx 0{,}497 \cdot 10^{-18}\,\text{Дж} \approx 3{,}11\,\text{эВ}$
}
\solutionspace{80pt}

\tasknumber{5}%
\task{%
    Из формулы Планка выразите (нужен вывод, не только ответ)...
    \begin{enumerate}
        \item длину соответствующей электромагнитной волны,
        \item период колебаний индукции магнитного поля в соответствующей электромагнитной волне.
    \end{enumerate}
}
\solutionspace{40pt}

\tasknumber{6}%
\task{%
    Определите длину волны лучей, фотоны которых имеют энергию
    равную кинетической энергии электрона, ускоренного напряжением $610\,\text{В}$.
}
\answer{%
    $E = h\frac c\lambda = e U \implies \lambda = \frac{hc}{eU} = \frac{6{,}626 \cdot 10^{-34}\,\text{Дж}\cdot\text{с} \cdot 3 \cdot 10^{8}\,\frac{\text{м}}{\text{с}}}{1{,}6 \cdot 10^{-19}\,\text{Кл} \cdot 610\,\text{В}} \approx 2{,}04\,\text{нм}.$
}

\variantsplitter

\addpersonalvariant{Варвара Пранова}

\tasknumber{1}%
\task{%
    Красная граница фотоэффекта для некоторого металла соответствует длине волны $5{,}7 \cdot 10^{-7}\,\text{м}$.
    Чему равно напряжение, полностью задерживающее фотоэлектроны, вырываемые из этого металла излучением
    с длиной волны $3{,}2 \cdot 10^{-5}\,\text{см}$? Постоянная Планка $h = 6{,}626 \cdot 10^{-34}\,\text{Дж}\cdot\text{с}$, заряд электрона $e = 1{,}6 \cdot 10^{-19}\,\text{Кл}$.
}
\answer{%
    $eU = K = E - A = \frac{hc}{\lambda} - A, \qquad 0 = \frac{hc}{\lambda_0} - A \implies U = \frac{ \frac{hc}{\lambda} - \frac{hc}{\lambda_0} }{ e } = \frac{hc}{ e }\cbr{ \frac 1{\lambda} - \frac 1{\lambda_0}}  \approx 1{,}70\,\text{В}$
}
\solutionspace{80pt}

\tasknumber{2}%
\task{%
    Сколько фотонов испускает за $30\,\text{мин}$ лазер,
    если мощность его излучения $75\,\text{мВт}$?
    Длина волны излучения $500\,\text{нм}$.
    $h = 6{,}626 \cdot 10^{-34}\,\text{Дж}\cdot\text{с}$.
}
\answer{%
    $
        N
            = \frac{E_{\text{общая}}}{E_{\text{одного фотона}}}
            = \frac{Pt}{h\nu} = \frac{Pt}{h \frac c\lambda}
            = \frac{Pt\lambda}{hc}
            = \frac{75\,\text{мВт} \cdot 30\,\text{мин} \cdot 500\,\text{нм}}{6{,}626 \cdot 10^{-34}\,\text{Дж}\cdot\text{с} \cdot 3 \cdot 10^{8}\,\frac{\text{м}}{\text{с}}}
            \approx 339{,}6 \cdot 10^{18}\units{фотонов}
    $
}
\solutionspace{120pt}

\tasknumber{3}%
\task{%
    Определите энергию фотона излучения частотой $9 \cdot 10^{16}\,\text{Гц}$.
    Ответ получите в джоулях и в электронвольтах.
}
\answer{%
    $E = h \nu = 6{,}626 \cdot 10^{-34}\,\text{Дж}\cdot\text{с} \cdot 9 \cdot 10^{16}\,\text{Гц} \approx 60 \cdot 10^{-18}\,\text{Дж} \approx 370\,\text{эВ}$
}
\solutionspace{80pt}

\tasknumber{4}%
\task{%
    Определите энергию кванта света с длиной волны $500\,\text{нм}$.
    Ответ выразите в электронвольтах.
    Способен ли человеческий глаз увидеть один такой квант, а импульс таких квантов?'
}
\answer{%
    $E = h\nu = \frac{hc}{\lambda} = \frac{6{,}626 \cdot 10^{-34}\,\text{Дж}\cdot\text{с} \cdot 3 \cdot 10^{8}\,\frac{\text{м}}{\text{с}}}{500\,\text{нм}} \approx 0{,}398 \cdot 10^{-18}\,\text{Дж} \approx 2{,}48\,\text{эВ}$
}
\solutionspace{80pt}

\tasknumber{5}%
\task{%
    Из формулы Планка выразите (нужен вывод, не только ответ)...
    \begin{enumerate}
        \item длину соответствующей электромагнитной волны,
        \item период колебаний электрического поля в соответствующей электромагнитной волне.
    \end{enumerate}
}
\solutionspace{40pt}

\tasknumber{6}%
\task{%
    Определите длину волны лучей, фотоны которых имеют энергию
    равную кинетической энергии электрона, ускоренного напряжением $1\,\text{В}$.
}
\answer{%
    $E = h\frac c\lambda = e U \implies \lambda = \frac{hc}{eU} = \frac{6{,}626 \cdot 10^{-34}\,\text{Дж}\cdot\text{с} \cdot 3 \cdot 10^{8}\,\frac{\text{м}}{\text{с}}}{1{,}6 \cdot 10^{-19}\,\text{Кл} \cdot 1\,\text{В}} \approx 1240\,\text{нм}.$
}

\variantsplitter

\addpersonalvariant{Марьям Салимова}

\tasknumber{1}%
\task{%
    Красная граница фотоэффекта для некоторого металла соответствует длине волны $6{,}6 \cdot 10^{-7}\,\text{м}$.
    Чему равно напряжение, полностью задерживающее фотоэлектроны, вырываемые из этого металла излучением
    с длиной волны $1{,}7 \cdot 10^{-5}\,\text{см}$? Постоянная Планка $h = 6{,}626 \cdot 10^{-34}\,\text{Дж}\cdot\text{с}$, заряд электрона $e = 1{,}6 \cdot 10^{-19}\,\text{Кл}$.
}
\answer{%
    $eU = K = E - A = \frac{hc}{\lambda} - A, \qquad 0 = \frac{hc}{\lambda_0} - A \implies U = \frac{ \frac{hc}{\lambda} - \frac{hc}{\lambda_0} }{ e } = \frac{hc}{ e }\cbr{ \frac 1{\lambda} - \frac 1{\lambda_0}}  \approx 5\,\text{В}$
}
\solutionspace{80pt}

\tasknumber{2}%
\task{%
    Сколько фотонов испускает за $5\,\text{мин}$ лазер,
    если мощность его излучения $200\,\text{мВт}$?
    Длина волны излучения $500\,\text{нм}$.
    $h = 6{,}626 \cdot 10^{-34}\,\text{Дж}\cdot\text{с}$.
}
\answer{%
    $
        N
            = \frac{E_{\text{общая}}}{E_{\text{одного фотона}}}
            = \frac{Pt}{h\nu} = \frac{Pt}{h \frac c\lambda}
            = \frac{Pt\lambda}{hc}
            = \frac{200\,\text{мВт} \cdot 5\,\text{мин} \cdot 500\,\text{нм}}{6{,}626 \cdot 10^{-34}\,\text{Дж}\cdot\text{с} \cdot 3 \cdot 10^{8}\,\frac{\text{м}}{\text{с}}}
            \approx 150{,}9 \cdot 10^{18}\units{фотонов}
    $
}
\solutionspace{120pt}

\tasknumber{3}%
\task{%
    Определите энергию фотона излучения частотой $4 \cdot 10^{16}\,\text{Гц}$.
    Ответ получите в джоулях и в электронвольтах.
}
\answer{%
    $E = h \nu = 6{,}626 \cdot 10^{-34}\,\text{Дж}\cdot\text{с} \cdot 4 \cdot 10^{16}\,\text{Гц} \approx 27 \cdot 10^{-18}\,\text{Дж} \approx 166\,\text{эВ}$
}
\solutionspace{80pt}

\tasknumber{4}%
\task{%
    Определите энергию фотона с длиной волны $400\,\text{нм}$.
    Ответ выразите в электронвольтах.
    Способен ли человеческий глаз увидеть один такой квант, а импульс таких квантов?'
}
\answer{%
    $E = h\nu = \frac{hc}{\lambda} = \frac{6{,}626 \cdot 10^{-34}\,\text{Дж}\cdot\text{с} \cdot 3 \cdot 10^{8}\,\frac{\text{м}}{\text{с}}}{400\,\text{нм}} \approx 0{,}497 \cdot 10^{-18}\,\text{Дж} \approx 3{,}11\,\text{эВ}$
}
\solutionspace{80pt}

\tasknumber{5}%
\task{%
    Из формулы Планка выразите (нужен вывод, не только ответ)...
    \begin{enumerate}
        \item длину соответствующей электромагнитной волны,
        \item период колебаний электрического поля в соответствующей электромагнитной волне.
    \end{enumerate}
}
\solutionspace{40pt}

\tasknumber{6}%
\task{%
    Определите длину волны лучей, фотоны которых имеют энергию
    равную кинетической энергии электрона, ускоренного напряжением $610\,\text{В}$.
}
\answer{%
    $E = h\frac c\lambda = e U \implies \lambda = \frac{hc}{eU} = \frac{6{,}626 \cdot 10^{-34}\,\text{Дж}\cdot\text{с} \cdot 3 \cdot 10^{8}\,\frac{\text{м}}{\text{с}}}{1{,}6 \cdot 10^{-19}\,\text{Кл} \cdot 610\,\text{В}} \approx 2{,}04\,\text{нм}.$
}

\variantsplitter

\addpersonalvariant{Юлия Шевченко}

\tasknumber{1}%
\task{%
    Красная граница фотоэффекта для некоторого металла соответствует длине волны $5{,}7 \cdot 10^{-7}\,\text{м}$.
    Чему равно напряжение, полностью задерживающее фотоэлектроны, вырываемые из этого металла излучением
    с длиной волны $2{,}6 \cdot 10^{-5}\,\text{см}$? Постоянная Планка $h = 6{,}626 \cdot 10^{-34}\,\text{Дж}\cdot\text{с}$, заряд электрона $e = 1{,}6 \cdot 10^{-19}\,\text{Кл}$.
}
\answer{%
    $eU = K = E - A = \frac{hc}{\lambda} - A, \qquad 0 = \frac{hc}{\lambda_0} - A \implies U = \frac{ \frac{hc}{\lambda} - \frac{hc}{\lambda_0} }{ e } = \frac{hc}{ e }\cbr{ \frac 1{\lambda} - \frac 1{\lambda_0}}  \approx 2{,}6\,\text{В}$
}
\solutionspace{80pt}

\tasknumber{2}%
\task{%
    Сколько фотонов испускает за $20\,\text{мин}$ лазер,
    если мощность его излучения $15\,\text{мВт}$?
    Длина волны излучения $500\,\text{нм}$.
    $h = 6{,}626 \cdot 10^{-34}\,\text{Дж}\cdot\text{с}$.
}
\answer{%
    $
        N
            = \frac{E_{\text{общая}}}{E_{\text{одного фотона}}}
            = \frac{Pt}{h\nu} = \frac{Pt}{h \frac c\lambda}
            = \frac{Pt\lambda}{hc}
            = \frac{15\,\text{мВт} \cdot 20\,\text{мин} \cdot 500\,\text{нм}}{6{,}626 \cdot 10^{-34}\,\text{Дж}\cdot\text{с} \cdot 3 \cdot 10^{8}\,\frac{\text{м}}{\text{с}}}
            \approx 45{,}3 \cdot 10^{18}\units{фотонов}
    $
}
\solutionspace{120pt}

\tasknumber{3}%
\task{%
    Определите энергию фотона излучения частотой $7 \cdot 10^{16}\,\text{Гц}$.
    Ответ получите в джоулях и в электронвольтах.
}
\answer{%
    $E = h \nu = 6{,}626 \cdot 10^{-34}\,\text{Дж}\cdot\text{с} \cdot 7 \cdot 10^{16}\,\text{Гц} \approx 46 \cdot 10^{-18}\,\text{Дж} \approx 290\,\text{эВ}$
}
\solutionspace{80pt}

\tasknumber{4}%
\task{%
    Определите энергию кванта света с длиной волны $150\,\text{нм}$.
    Ответ выразите в электронвольтах.
    Способен ли человеческий глаз увидеть один такой квант, а импульс таких квантов?'
}
\answer{%
    $E = h\nu = \frac{hc}{\lambda} = \frac{6{,}626 \cdot 10^{-34}\,\text{Дж}\cdot\text{с} \cdot 3 \cdot 10^{8}\,\frac{\text{м}}{\text{с}}}{150\,\text{нм}} \approx 1{,}33 \cdot 10^{-18}\,\text{Дж} \approx 8{,}3\,\text{эВ}$
}
\solutionspace{80pt}

\tasknumber{5}%
\task{%
    Из формулы Планка выразите (нужен вывод, не только ответ)...
    \begin{enumerate}
        \item длину соответствующей электромагнитной волны,
        \item период колебаний индукции магнитного поля в соответствующей электромагнитной волне.
    \end{enumerate}
}
\solutionspace{40pt}

\tasknumber{6}%
\task{%
    Определите длину волны лучей, фотоны которых имеют энергию
    равную кинетической энергии электрона, ускоренного напряжением $144\,\text{В}$.
}
\answer{%
    $E = h\frac c\lambda = e U \implies \lambda = \frac{hc}{eU} = \frac{6{,}626 \cdot 10^{-34}\,\text{Дж}\cdot\text{с} \cdot 3 \cdot 10^{8}\,\frac{\text{м}}{\text{с}}}{1{,}6 \cdot 10^{-19}\,\text{Кл} \cdot 144\,\text{В}} \approx 8{,}63\,\text{нм}.$
}
% autogenerated
