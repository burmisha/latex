\setdate{24~марта~2022}
\setclass{11«Б»}

\addpersonalvariant{Михаил Бурмистров}

\tasknumber{1}%
\task{%
    Определите число электронов в атоме $\ce{^{1}_{1}{H}}$.
}
\answer{%
    $Z = 1$ протонов и столько же электронов $A = 1$ нуклонов, $A - Z = 0$ нейтронов.
    Ответ: 1
}

\tasknumber{2}%
\task{%
    Определите число нейтронов в атоме $\ce{^{7}_{3}{Li}}$.
}
\answer{%
    $Z = 3$ протонов и столько же электронов $A = 7$ нуклонов, $A - Z = 4$ нейтронов.
    Ответ: 4
}

\tasknumber{3}%
\task{%
    Определите число протонов в атоме $\ce{^{9}_{4}{Be}}$.
}
\answer{%
    $Z = 4$ протонов и столько же электронов $A = 9$ нуклонов, $A - Z = 5$ нейтронов.
    Ответ: 4
}

\tasknumber{4}%
\task{%
    Определите число электронов в атоме $\ce{^{14}_{7}{N}}$.
}
\answer{%
    $Z = 7$ протонов и столько же электронов $A = 14$ нуклонов, $A - Z = 7$ нейтронов.
    Ответ: 7
}

\tasknumber{5}%
\task{%
    Определите число электронов в атоме $\ce{^{19}_{9}{F}}$.
}
\answer{%
    $Z = 9$ протонов и столько же электронов $A = 19$ нуклонов, $A - Z = 10$ нейтронов.
    Ответ: 9
}

\tasknumber{6}%
\task{%
    Определите число нейтронов в атоме $\ce{^{26}_{12}{Mg}}$.
}
\answer{%
    $Z = 12$ протонов и столько же электронов $A = 26$ нуклонов, $A - Z = 14$ нейтронов.
    Ответ: 14
}

\tasknumber{7}%
\task{%
    Определите число электронов в атоме $\text{кремний-32}$.
}
\answer{%
    $Z = 14$ протонов и столько же электронов, $A = 32$ нуклонов, $A - Z = 18$ нейтронов.
    Ответ: 14
}

\tasknumber{8}%
\task{%
    Определите число протонов в атоме $\text{аргон-38}$.
}
\answer{%
    $Z = 18$ протонов и столько же электронов, $A = 38$ нуклонов, $A - Z = 20$ нейтронов.
    Ответ: 18
}

\tasknumber{9}%
\task{%
    Определите число протонов в атоме $\text{кальций-45}$.
}
\answer{%
    $Z = 20$ протонов и столько же электронов, $A = 45$ нуклонов, $A - Z = 25$ нейтронов.
    Ответ: 20
}

\tasknumber{10}%
\task{%
    Определите число нейтронов в атоме $\text{титан-47}$.
}
\answer{%
    $Z = 22$ протонов и столько же электронов, $A = 47$ нуклонов, $A - Z = 25$ нейтронов.
    Ответ: 25
}

\tasknumber{11}%
\task{%
    Определите число нуклонов в атоме $\text{марганец-55}$.
}
\answer{%
    $Z = 25$ протонов и столько же электронов, $A = 55$ нуклонов, $A - Z = 30$ нейтронов.
    Ответ: 55
}

\tasknumber{12}%
\task{%
    Определите число нуклонов в атоме $\text{железо-55}$.
}
\answer{%
    $Z = 26$ протонов и столько же электронов, $A = 55$ нуклонов, $A - Z = 29$ нейтронов.
    Ответ: 55
}

\tasknumber{13}%
\task{%
    В какое ядро превращается исходное в результате ядерного распада?
    Запишите уравнение реакции и явно укажите число протонов и нейтронов в получившемся ядре.
    \begin{itemize}
        \item ядро тория $\ce{^{234}_{90}{Th}}$, $\beta^-$-распад,
        \item ядро радона $\ce{^{222}_{86}{Rn}}$, $\alpha$-распад,
        \item ядро свинца $\ce{^{210}_{82}{Pb}}$, $\beta$-распад,
        \item ядро свинца $\ce{^{209}_{82}{Pb}}$, $\beta$-распад.
    \end{itemize}
}
\answer{%
    \begin{align*}
    &\ce{^{234}_{90}{Th}} \to \ce{^{234}_{91}{Pa}} + e^- + \tilde\nu_e: \qquad \text{ядро протактиния $\ce{^{234}_{91}{Pa}}$}: 91\,p^+, 143\,n^0, \\
    &\ce{^{222}_{86}{Rn}} \to \ce{^{218}_{84}{Po}} + \ce{^4_2{He}}: \qquad \text{ядро полония $\ce{^{218}_{84}{Po}}$}: 84\,p^+, 134\,n^0, \\
    &\ce{^{210}_{82}{Pb}} \to \ce{^{210}_{83}{Bi}} + e^- + \tilde\nu_e: \qquad \text{ядро висмута $\ce{^{210}_{83}{Bi}}$}: 83\,p^+, 127\,n^0, \\
    &\ce{^{209}_{82}{Pb}} \to \ce{^{209}_{83}{Bi}} + e^- + \tilde\nu_e: \qquad \text{ядро висмута $\ce{^{209}_{83}{Bi}}$}: 83\,p^+, 126\,n^0.
    \end{align*}
}
\solutionspace{80pt}

\tasknumber{14}%
\task{%
    Какая доля (от начального количества) радиоактивных ядер распадётся через время,
    равное двум периодам полураспада? Ответ выразить в процентах.
}
\answer{%
    \begin{align*}
    N &= N_0 \cdot 2^{- \frac t{T_{1/2}}} \implies
        \frac N{N_0} = 2^{- \frac t{T_{1/2}}}
        = 2^{-2} \approx 0{,}25 \approx 25\% \\
    N_\text{расп.} &= N_0 - N = N_0 - N_0 \cdot 2^{-\frac t{T_{1/2}}}
        = N_0\cbr{1 - 2^{-\frac t{T_{1/2}}}} \implies
        \frac{N_\text{расп.}}{N_0} = 1 - 2^{-\frac t{T_{1/2}}}
        = 1 - 2^{-2} \approx 0{,}75 \approx 75\%
    \end{align*}
}
\solutionspace{90pt}

\tasknumber{15}%
\task{%
    Сколько процентов ядер радиоактивного железа $\ce{^{59}Fe}$
    останется через $91{,}2\,\text{суток}$, если период его полураспада составляет $45{,}6\,\text{суток}$?
}
\answer{%
    \begin{align*}
    N &= N_0 \cdot 2^{-\frac t{T_{1/2}}}
        = 2^{-\frac{91{,}2\,\text{суток}}{45{,}6\,\text{суток}}}
        \approx 0{,}2500 = 25{,}00\%
    \end{align*}
}
\solutionspace{90pt}

\tasknumber{16}%
\task{%
    За $3\,\text{суток}$ от начального количества ядер радиоизотопа осталась одна шестнадцатая.
    Каков период полураспада этого изотопа (ответ приведите в сутках)?
    Какая ещё доля (также от начального количества) распадётся, если подождать ещё столько же?
}
\answer{%
    \begin{align*}
            N &= N_0 \cdot 2^{-\frac t{T_{1/2}}}
            \implies \frac N{N_0} = 2^{-\frac t{T_{1/2}}}
            \implies \frac 1{16} = 2^{-\frac {3\,\text{суток}}{T_{1/2}}}
            \implies 4 = \frac {3\,\text{суток}}{T_{1/2}}
            \implies T_{1/2} = \frac {3\,\text{суток}}4 \approx 0{,}75\,\text{суток}.
         \\
            \delta &= \frac{N(t)}{N_0} - \frac{N(2t)}{N_0}
            = 2^{-\frac t{T_{1/2}}} - 2^{-\frac {2t}{T_{1/2}}}
            = 2^{-\frac t{T_{1/2}}}\cbr{1 - 2^{-\frac t{T_{1/2}}}}
            = \frac 1{16} \cdot \cbr{1-\frac 1{16}} \approx 0{,}059
    \end{align*}
}

\variantsplitter

\addpersonalvariant{Снежана Авдошина}

\tasknumber{1}%
\task{%
    Определите число нуклонов в атоме $\ce{^{3}_{1}{T}}$.
}
\answer{%
    $Z = 1$ протонов и столько же электронов $A = 3$ нуклонов, $A - Z = 2$ нейтронов.
    Ответ: 3
}

\tasknumber{2}%
\task{%
    Определите число нуклонов в атоме $\ce{^{6}_{3}{Li}}$.
}
\answer{%
    $Z = 3$ протонов и столько же электронов $A = 6$ нуклонов, $A - Z = 3$ нейтронов.
    Ответ: 6
}

\tasknumber{3}%
\task{%
    Определите число нуклонов в атоме $\ce{^{11}_{5}{B}}$.
}
\answer{%
    $Z = 5$ протонов и столько же электронов $A = 11$ нуклонов, $A - Z = 6$ нейтронов.
    Ответ: 11
}

\tasknumber{4}%
\task{%
    Определите число протонов в атоме $\ce{^{12}_{6}{C}}$.
}
\answer{%
    $Z = 6$ протонов и столько же электронов $A = 12$ нуклонов, $A - Z = 6$ нейтронов.
    Ответ: 6
}

\tasknumber{5}%
\task{%
    Определите число нуклонов в атоме $\ce{^{22}_{11}{Na}}$.
}
\answer{%
    $Z = 11$ протонов и столько же электронов $A = 22$ нуклонов, $A - Z = 11$ нейтронов.
    Ответ: 22
}

\tasknumber{6}%
\task{%
    Определите число протонов в атоме $\ce{^{26}_{13}{Al}}$.
}
\answer{%
    $Z = 13$ протонов и столько же электронов $A = 26$ нуклонов, $A - Z = 13$ нейтронов.
    Ответ: 13
}

\tasknumber{7}%
\task{%
    Определите число электронов в атоме $\text{сера-35}$.
}
\answer{%
    $Z = 16$ протонов и столько же электронов, $A = 35$ нуклонов, $A - Z = 19$ нейтронов.
    Ответ: 16
}

\tasknumber{8}%
\task{%
    Определите число нуклонов в атоме $\text{аргон-36}$.
}
\answer{%
    $Z = 18$ протонов и столько же электронов, $A = 36$ нуклонов, $A - Z = 18$ нейтронов.
    Ответ: 36
}

\tasknumber{9}%
\task{%
    Определите число электронов в атоме $\text{кальций-45}$.
}
\answer{%
    $Z = 20$ протонов и столько же электронов, $A = 45$ нуклонов, $A - Z = 25$ нейтронов.
    Ответ: 20
}

\tasknumber{10}%
\task{%
    Определите число электронов в атоме $\text{титан-46}$.
}
\answer{%
    $Z = 22$ протонов и столько же электронов, $A = 46$ нуклонов, $A - Z = 24$ нейтронов.
    Ответ: 22
}

\tasknumber{11}%
\task{%
    Определите число нуклонов в атоме $\text{марганец-52}$.
}
\answer{%
    $Z = 25$ протонов и столько же электронов, $A = 52$ нуклонов, $A - Z = 27$ нейтронов.
    Ответ: 52
}

\tasknumber{12}%
\task{%
    Определите число нуклонов в атоме $\text{железо-56}$.
}
\answer{%
    $Z = 26$ протонов и столько же электронов, $A = 56$ нуклонов, $A - Z = 30$ нейтронов.
    Ответ: 56
}

\tasknumber{13}%
\task{%
    В какое ядро превращается исходное в результате ядерного распада?
    Запишите уравнение реакции и явно укажите число протонов и нейтронов в получившемся ядре.
    \begin{itemize}
        \item ядро тория $\ce{^{234}_{90}{Th}}$, $\beta$-распад,
        \item ядро радия $\ce{^{226}_{88}{Ra}}$, $\alpha$-распад,
        \item ядро свинца $\ce{^{210}_{82}{Pb}}$, $\beta$-распад,
        \item ядро свинца $\ce{^{209}_{82}{Pb}}$, $\beta$-распад.
    \end{itemize}
}
\answer{%
    \begin{align*}
    &\ce{^{234}_{90}{Th}} \to \ce{^{234}_{91}{Pa}} + e^- + \tilde\nu_e: \qquad \text{ядро протактиния $\ce{^{234}_{91}{Pa}}$}: 91\,p^+, 143\,n^0, \\
    &\ce{^{226}_{88}{Ra}} \to \ce{^{222}_{86}{Rn}} + \ce{^4_2{He}}: \qquad \text{ядро радона $\ce{^{222}_{86}{Rn}}$}: 86\,p^+, 136\,n^0, \\
    &\ce{^{210}_{82}{Pb}} \to \ce{^{210}_{83}{Bi}} + e^- + \tilde\nu_e: \qquad \text{ядро висмута $\ce{^{210}_{83}{Bi}}$}: 83\,p^+, 127\,n^0, \\
    &\ce{^{209}_{82}{Pb}} \to \ce{^{209}_{83}{Bi}} + e^- + \tilde\nu_e: \qquad \text{ядро висмута $\ce{^{209}_{83}{Bi}}$}: 83\,p^+, 126\,n^0.
    \end{align*}
}
\solutionspace{80pt}

\tasknumber{14}%
\task{%
    Какая доля (от начального количества) радиоактивных ядер останется через время,
    равное четырём периодам полураспада? Ответ выразить в процентах.
}
\answer{%
    \begin{align*}
    N &= N_0 \cdot 2^{- \frac t{T_{1/2}}} \implies
        \frac N{N_0} = 2^{- \frac t{T_{1/2}}}
        = 2^{-4} \approx 0{,}06 \approx 6\% \\
    N_\text{расп.} &= N_0 - N = N_0 - N_0 \cdot 2^{-\frac t{T_{1/2}}}
        = N_0\cbr{1 - 2^{-\frac t{T_{1/2}}}} \implies
        \frac{N_\text{расп.}}{N_0} = 1 - 2^{-\frac t{T_{1/2}}}
        = 1 - 2^{-4} \approx 0{,}94 \approx 94\%
    \end{align*}
}
\solutionspace{90pt}

\tasknumber{15}%
\task{%
    Сколько процентов ядер радиоактивного железа $\ce{^{59}Fe}$
    останется через $182{,}4\,\text{суток}$, если период его полураспада составляет $45{,}6\,\text{суток}$?
}
\answer{%
    \begin{align*}
    N &= N_0 \cdot 2^{-\frac t{T_{1/2}}}
        = 2^{-\frac{182{,}4\,\text{суток}}{45{,}6\,\text{суток}}}
        \approx 0{,}0625 = 6{,}25\%
    \end{align*}
}
\solutionspace{90pt}

\tasknumber{16}%
\task{%
    За $2\,\text{суток}$ от начального количества ядер радиоизотопа осталась четверть.
    Каков период полураспада этого изотопа (ответ приведите в сутках)?
    Какая ещё доля (также от начального количества) распадётся, если подождать ещё столько же?
}
\answer{%
    \begin{align*}
            N &= N_0 \cdot 2^{-\frac t{T_{1/2}}}
            \implies \frac N{N_0} = 2^{-\frac t{T_{1/2}}}
            \implies \frac 1{4} = 2^{-\frac {2\,\text{суток}}{T_{1/2}}}
            \implies 2 = \frac {2\,\text{суток}}{T_{1/2}}
            \implies T_{1/2} = \frac {2\,\text{суток}}2 \approx 1\,\text{суток}.
         \\
            \delta &= \frac{N(t)}{N_0} - \frac{N(2t)}{N_0}
            = 2^{-\frac t{T_{1/2}}} - 2^{-\frac {2t}{T_{1/2}}}
            = 2^{-\frac t{T_{1/2}}}\cbr{1 - 2^{-\frac t{T_{1/2}}}}
            = \frac 1{4} \cdot \cbr{1-\frac 1{4}} \approx 0{,}188
    \end{align*}
}

\variantsplitter

\addpersonalvariant{Марьяна Аристова}

\tasknumber{1}%
\task{%
    Определите число электронов в атоме $\ce{^{1}_{1}{H}}$.
}
\answer{%
    $Z = 1$ протонов и столько же электронов $A = 1$ нуклонов, $A - Z = 0$ нейтронов.
    Ответ: 1
}

\tasknumber{2}%
\task{%
    Определите число нейтронов в атоме $\ce{^{3}_{2}{He}}$.
}
\answer{%
    $Z = 2$ протонов и столько же электронов $A = 3$ нуклонов, $A - Z = 1$ нейтронов.
    Ответ: 1
}

\tasknumber{3}%
\task{%
    Определите число электронов в атоме $\ce{^{11}_{5}{B}}$.
}
\answer{%
    $Z = 5$ протонов и столько же электронов $A = 11$ нуклонов, $A - Z = 6$ нейтронов.
    Ответ: 5
}

\tasknumber{4}%
\task{%
    Определите число электронов в атоме $\ce{^{14}_{6}{C}}$.
}
\answer{%
    $Z = 6$ протонов и столько же электронов $A = 14$ нуклонов, $A - Z = 8$ нейтронов.
    Ответ: 6
}

\tasknumber{5}%
\task{%
    Определите число электронов в атоме $\ce{^{22}_{11}{Na}}$.
}
\answer{%
    $Z = 11$ протонов и столько же электронов $A = 22$ нуклонов, $A - Z = 11$ нейтронов.
    Ответ: 11
}

\tasknumber{6}%
\task{%
    Определите число электронов в атоме $\ce{^{26}_{13}{Al}}$.
}
\answer{%
    $Z = 13$ протонов и столько же электронов $A = 26$ нуклонов, $A - Z = 13$ нейтронов.
    Ответ: 13
}

\tasknumber{7}%
\task{%
    Определите число электронов в атоме $\text{кремний-32}$.
}
\answer{%
    $Z = 14$ протонов и столько же электронов, $A = 32$ нуклонов, $A - Z = 18$ нейтронов.
    Ответ: 14
}

\tasknumber{8}%
\task{%
    Определите число протонов в атоме $\text{хлор-37}$.
}
\answer{%
    $Z = 17$ протонов и столько же электронов, $A = 37$ нуклонов, $A - Z = 20$ нейтронов.
    Ответ: 17
}

\tasknumber{9}%
\task{%
    Определите число нуклонов в атоме $\text{кальций-40}$.
}
\answer{%
    $Z = 20$ протонов и столько же электронов, $A = 40$ нуклонов, $A - Z = 20$ нейтронов.
    Ответ: 40
}

\tasknumber{10}%
\task{%
    Определите число протонов в атоме $\text{титан-46}$.
}
\answer{%
    $Z = 22$ протонов и столько же электронов, $A = 46$ нуклонов, $A - Z = 24$ нейтронов.
    Ответ: 22
}

\tasknumber{11}%
\task{%
    Определите число протонов в атоме $\text{ванадий-51}$.
}
\answer{%
    $Z = 23$ протонов и столько же электронов, $A = 51$ нуклонов, $A - Z = 28$ нейтронов.
    Ответ: 23
}

\tasknumber{12}%
\task{%
    Определите число электронов в атоме $\text{железо-56}$.
}
\answer{%
    $Z = 26$ протонов и столько же электронов, $A = 56$ нуклонов, $A - Z = 30$ нейтронов.
    Ответ: 26
}

\tasknumber{13}%
\task{%
    В какое ядро превращается исходное в результате ядерного распада?
    Запишите уравнение реакции и явно укажите число протонов и нейтронов в получившемся ядре.
    \begin{itemize}
        \item ядро тория $\ce{^{234}_{90}{Th}}$, $\beta^-$-распад,
        \item ядро полония $\ce{^{210}_{84}{Po}}$, $\alpha$-распад,
        \item ядро свинца $\ce{^{210}_{82}{Pb}}$, $\beta$-распад,
        \item ядро свинца $\ce{^{209}_{82}{Pb}}$, $\beta$-распад.
    \end{itemize}
}
\answer{%
    \begin{align*}
    &\ce{^{234}_{90}{Th}} \to \ce{^{234}_{91}{Pa}} + e^- + \tilde\nu_e: \qquad \text{ядро протактиния $\ce{^{234}_{91}{Pa}}$}: 91\,p^+, 143\,n^0, \\
    &\ce{^{210}_{84}{Po}} \to \ce{^{206}_{82}{Pb}} + \ce{^4_2{He}}: \qquad \text{ядро свинца $\ce{^{206}_{82}{Pb}}$}: 82\,p^+, 124\,n^0, \\
    &\ce{^{210}_{82}{Pb}} \to \ce{^{210}_{83}{Bi}} + e^- + \tilde\nu_e: \qquad \text{ядро висмута $\ce{^{210}_{83}{Bi}}$}: 83\,p^+, 127\,n^0, \\
    &\ce{^{209}_{82}{Pb}} \to \ce{^{209}_{83}{Bi}} + e^- + \tilde\nu_e: \qquad \text{ядро висмута $\ce{^{209}_{83}{Bi}}$}: 83\,p^+, 126\,n^0.
    \end{align*}
}
\solutionspace{80pt}

\tasknumber{14}%
\task{%
    Какая доля (от начального количества) радиоактивных ядер останется через время,
    равное четырём периодам полураспада? Ответ выразить в процентах.
}
\answer{%
    \begin{align*}
    N &= N_0 \cdot 2^{- \frac t{T_{1/2}}} \implies
        \frac N{N_0} = 2^{- \frac t{T_{1/2}}}
        = 2^{-4} \approx 0{,}06 \approx 6\% \\
    N_\text{расп.} &= N_0 - N = N_0 - N_0 \cdot 2^{-\frac t{T_{1/2}}}
        = N_0\cbr{1 - 2^{-\frac t{T_{1/2}}}} \implies
        \frac{N_\text{расп.}}{N_0} = 1 - 2^{-\frac t{T_{1/2}}}
        = 1 - 2^{-4} \approx 0{,}94 \approx 94\%
    \end{align*}
}
\solutionspace{90pt}

\tasknumber{15}%
\task{%
    Сколько процентов ядер радиоактивного железа $\ce{^{59}Fe}$
    останется через $91{,}2\,\text{суток}$, если период его полураспада составляет $45{,}6\,\text{суток}$?
}
\answer{%
    \begin{align*}
    N &= N_0 \cdot 2^{-\frac t{T_{1/2}}}
        = 2^{-\frac{91{,}2\,\text{суток}}{45{,}6\,\text{суток}}}
        \approx 0{,}2500 = 25{,}00\%
    \end{align*}
}
\solutionspace{90pt}

\tasknumber{16}%
\task{%
    За $3\,\text{суток}$ от начального количества ядер радиоизотопа осталась половина.
    Каков период полураспада этого изотопа (ответ приведите в сутках)?
    Какая ещё доля (также от начального количества) распадётся, если подождать ещё столько же?
}
\answer{%
    \begin{align*}
            N &= N_0 \cdot 2^{-\frac t{T_{1/2}}}
            \implies \frac N{N_0} = 2^{-\frac t{T_{1/2}}}
            \implies \frac 1{2} = 2^{-\frac {3\,\text{суток}}{T_{1/2}}}
            \implies 1 = \frac {3\,\text{суток}}{T_{1/2}}
            \implies T_{1/2} = \frac {3\,\text{суток}}1 \approx 3\,\text{суток}.
         \\
            \delta &= \frac{N(t)}{N_0} - \frac{N(2t)}{N_0}
            = 2^{-\frac t{T_{1/2}}} - 2^{-\frac {2t}{T_{1/2}}}
            = 2^{-\frac t{T_{1/2}}}\cbr{1 - 2^{-\frac t{T_{1/2}}}}
            = \frac 1{2} \cdot \cbr{1-\frac 1{2}} \approx 0{,}250
    \end{align*}
}

\variantsplitter

\addpersonalvariant{Никита Иванов}

\tasknumber{1}%
\task{%
    Определите число электронов в атоме $\ce{^{1}_{1}{H}}$.
}
\answer{%
    $Z = 1$ протонов и столько же электронов $A = 1$ нуклонов, $A - Z = 0$ нейтронов.
    Ответ: 1
}

\tasknumber{2}%
\task{%
    Определите число нуклонов в атоме $\ce{^{7}_{3}{Li}}$.
}
\answer{%
    $Z = 3$ протонов и столько же электронов $A = 7$ нуклонов, $A - Z = 4$ нейтронов.
    Ответ: 7
}

\tasknumber{3}%
\task{%
    Определите число протонов в атоме $\ce{^{11}_{5}{B}}$.
}
\answer{%
    $Z = 5$ протонов и столько же электронов $A = 11$ нуклонов, $A - Z = 6$ нейтронов.
    Ответ: 5
}

\tasknumber{4}%
\task{%
    Определите число протонов в атоме $\ce{^{14}_{7}{N}}$.
}
\answer{%
    $Z = 7$ протонов и столько же электронов $A = 14$ нуклонов, $A - Z = 7$ нейтронов.
    Ответ: 7
}

\tasknumber{5}%
\task{%
    Определите число протонов в атоме $\ce{^{17}_{8}{O}}$.
}
\answer{%
    $Z = 8$ протонов и столько же электронов $A = 17$ нуклонов, $A - Z = 9$ нейтронов.
    Ответ: 8
}

\tasknumber{6}%
\task{%
    Определите число электронов в атоме $\ce{^{26}_{13}{Al}}$.
}
\answer{%
    $Z = 13$ протонов и столько же электронов $A = 26$ нуклонов, $A - Z = 13$ нейтронов.
    Ответ: 13
}

\tasknumber{7}%
\task{%
    Определите число электронов в атоме $\text{сера-34}$.
}
\answer{%
    $Z = 16$ протонов и столько же электронов, $A = 34$ нуклонов, $A - Z = 18$ нейтронов.
    Ответ: 16
}

\tasknumber{8}%
\task{%
    Определите число нейтронов в атоме $\text{хлор-35}$.
}
\answer{%
    $Z = 17$ протонов и столько же электронов, $A = 35$ нуклонов, $A - Z = 18$ нейтронов.
    Ответ: 18
}

\tasknumber{9}%
\task{%
    Определите число нейтронов в атоме $\text{калий-41}$.
}
\answer{%
    $Z = 19$ протонов и столько же электронов, $A = 41$ нуклонов, $A - Z = 22$ нейтронов.
    Ответ: 22
}

\tasknumber{10}%
\task{%
    Определите число протонов в атоме $\text{скандий-45}$.
}
\answer{%
    $Z = 21$ протонов и столько же электронов, $A = 45$ нуклонов, $A - Z = 24$ нейтронов.
    Ответ: 21
}

\tasknumber{11}%
\task{%
    Определите число протонов в атоме $\text{ванадий-50}$.
}
\answer{%
    $Z = 23$ протонов и столько же электронов, $A = 50$ нуклонов, $A - Z = 27$ нейтронов.
    Ответ: 23
}

\tasknumber{12}%
\task{%
    Определите число электронов в атоме $\text{кобальт-60}$.
}
\answer{%
    $Z = 27$ протонов и столько же электронов, $A = 60$ нуклонов, $A - Z = 33$ нейтронов.
    Ответ: 27
}

\tasknumber{13}%
\task{%
    В какое ядро превращается исходное в результате ядерного распада?
    Запишите уравнение реакции и явно укажите число протонов и нейтронов в получившемся ядре.
    \begin{itemize}
        \item ядро фосфора $\ce{^{30}_{15}{P}}$, $\beta^+$-распад,
        \item ядро урана $\ce{^{234}_{92}{U}}$, $\alpha$-распад,
        \item ядро свинца $\ce{^{214}_{82}{Pb}}$, $\alpha$-распад,
        \item ядро свинца $\ce{^{209}_{82}{Pb}}$, $\beta$-распад.
    \end{itemize}
}
\answer{%
    \begin{align*}
    &\ce{^{30}_{15}{P}} \to \ce{^{30}_{14}{Si}} + e^+ + \nu_e: \qquad \text{ядро кремния $\ce{^{30}_{14}{Si}}$}: 14\,p^+, 16\,n^0, \\
    &\ce{^{234}_{92}{U}} \to \ce{^{230}_{90}{Th}} + \ce{^4_2{He}}: \qquad \text{ядро тория $\ce{^{230}_{90}{Th}}$}: 90\,p^+, 140\,n^0, \\
    &\ce{^{214}_{82}{Pb}} \to \ce{^{210}_{80}{Hg}} + \ce{^4_2{He}}: \qquad \text{ядро ртути $\ce{^{210}_{80}{Hg}}$}: 80\,p^+, 130\,n^0, \\
    &\ce{^{209}_{82}{Pb}} \to \ce{^{209}_{83}{Bi}} + e^- + \tilde\nu_e: \qquad \text{ядро висмута $\ce{^{209}_{83}{Bi}}$}: 83\,p^+, 126\,n^0.
    \end{align*}
}
\solutionspace{80pt}

\tasknumber{14}%
\task{%
    Какая доля (от начального количества) радиоактивных ядер распадётся через время,
    равное четырём периодам полураспада? Ответ выразить в процентах.
}
\answer{%
    \begin{align*}
    N &= N_0 \cdot 2^{- \frac t{T_{1/2}}} \implies
        \frac N{N_0} = 2^{- \frac t{T_{1/2}}}
        = 2^{-4} \approx 0{,}06 \approx 6\% \\
    N_\text{расп.} &= N_0 - N = N_0 - N_0 \cdot 2^{-\frac t{T_{1/2}}}
        = N_0\cbr{1 - 2^{-\frac t{T_{1/2}}}} \implies
        \frac{N_\text{расп.}}{N_0} = 1 - 2^{-\frac t{T_{1/2}}}
        = 1 - 2^{-4} \approx 0{,}94 \approx 94\%
    \end{align*}
}
\solutionspace{90pt}

\tasknumber{15}%
\task{%
    Сколько процентов ядер радиоактивного железа $\ce{^{59}Fe}$
    останется через $91{,}2\,\text{суток}$, если период его полураспада составляет $45{,}6\,\text{суток}$?
}
\answer{%
    \begin{align*}
    N &= N_0 \cdot 2^{-\frac t{T_{1/2}}}
        = 2^{-\frac{91{,}2\,\text{суток}}{45{,}6\,\text{суток}}}
        \approx 0{,}2500 = 25{,}00\%
    \end{align*}
}
\solutionspace{90pt}

\tasknumber{16}%
\task{%
    За $5\,\text{суток}$ от начального количества ядер радиоизотопа осталась одна шестнадцатая.
    Каков период полураспада этого изотопа (ответ приведите в сутках)?
    Какая ещё доля (также от начального количества) распадётся, если подождать ещё столько же?
}
\answer{%
    \begin{align*}
            N &= N_0 \cdot 2^{-\frac t{T_{1/2}}}
            \implies \frac N{N_0} = 2^{-\frac t{T_{1/2}}}
            \implies \frac 1{16} = 2^{-\frac {5\,\text{суток}}{T_{1/2}}}
            \implies 4 = \frac {5\,\text{суток}}{T_{1/2}}
            \implies T_{1/2} = \frac {5\,\text{суток}}4 \approx 1{,}25\,\text{суток}.
         \\
            \delta &= \frac{N(t)}{N_0} - \frac{N(2t)}{N_0}
            = 2^{-\frac t{T_{1/2}}} - 2^{-\frac {2t}{T_{1/2}}}
            = 2^{-\frac t{T_{1/2}}}\cbr{1 - 2^{-\frac t{T_{1/2}}}}
            = \frac 1{16} \cdot \cbr{1-\frac 1{16}} \approx 0{,}059
    \end{align*}
}

\variantsplitter

\addpersonalvariant{Анастасия Князева}

\tasknumber{1}%
\task{%
    Определите число протонов в атоме $\ce{^{1}_{1}{H}}$.
}
\answer{%
    $Z = 1$ протонов и столько же электронов $A = 1$ нуклонов, $A - Z = 0$ нейтронов.
    Ответ: 1
}

\tasknumber{2}%
\task{%
    Определите число нуклонов в атоме $\ce{^{4}_{2}{He}}$.
}
\answer{%
    $Z = 2$ протонов и столько же электронов $A = 4$ нуклонов, $A - Z = 2$ нейтронов.
    Ответ: 4
}

\tasknumber{3}%
\task{%
    Определите число протонов в атоме $\ce{^{10}_{4}{Be}}$.
}
\answer{%
    $Z = 4$ протонов и столько же электронов $A = 10$ нуклонов, $A - Z = 6$ нейтронов.
    Ответ: 4
}

\tasknumber{4}%
\task{%
    Определите число нейтронов в атоме $\ce{^{16}_{8}{O}}$.
}
\answer{%
    $Z = 8$ протонов и столько же электронов $A = 16$ нуклонов, $A - Z = 8$ нейтронов.
    Ответ: 8
}

\tasknumber{5}%
\task{%
    Определите число электронов в атоме $\ce{^{19}_{9}{F}}$.
}
\answer{%
    $Z = 9$ протонов и столько же электронов $A = 19$ нуклонов, $A - Z = 10$ нейтронов.
    Ответ: 9
}

\tasknumber{6}%
\task{%
    Определите число нуклонов в атоме $\ce{^{24}_{12}{Mg}}$.
}
\answer{%
    $Z = 12$ протонов и столько же электронов $A = 24$ нуклонов, $A - Z = 12$ нейтронов.
    Ответ: 24
}

\tasknumber{7}%
\task{%
    Определите число протонов в атоме $\text{сера-34}$.
}
\answer{%
    $Z = 16$ протонов и столько же электронов, $A = 34$ нуклонов, $A - Z = 18$ нейтронов.
    Ответ: 16
}

\tasknumber{8}%
\task{%
    Определите число нейтронов в атоме $\text{аргон-42}$.
}
\answer{%
    $Z = 18$ протонов и столько же электронов, $A = 42$ нуклонов, $A - Z = 24$ нейтронов.
    Ответ: 24
}

\tasknumber{9}%
\task{%
    Определите число электронов в атоме $\text{кальций-47}$.
}
\answer{%
    $Z = 20$ протонов и столько же электронов, $A = 47$ нуклонов, $A - Z = 27$ нейтронов.
    Ответ: 20
}

\tasknumber{10}%
\task{%
    Определите число нуклонов в атоме $\text{скандий-44}$.
}
\answer{%
    $Z = 21$ протонов и столько же электронов, $A = 44$ нуклонов, $A - Z = 23$ нейтронов.
    Ответ: 44
}

\tasknumber{11}%
\task{%
    Определите число протонов в атоме $\text{хром-50}$.
}
\answer{%
    $Z = 24$ протонов и столько же электронов, $A = 50$ нуклонов, $A - Z = 26$ нейтронов.
    Ответ: 24
}

\tasknumber{12}%
\task{%
    Определите число нуклонов в атоме $\text{никель-58}$.
}
\answer{%
    $Z = 28$ протонов и столько же электронов, $A = 58$ нуклонов, $A - Z = 30$ нейтронов.
    Ответ: 58
}

\tasknumber{13}%
\task{%
    В какое ядро превращается исходное в результате ядерного распада?
    Запишите уравнение реакции и явно укажите число протонов и нейтронов в получившемся ядре.
    \begin{itemize}
        \item ядро фосфора $\ce{^{30}_{15}{P}}$, $\beta^+$-распад,
        \item ядро урана $\ce{^{238}_{92}{U}}$, $\alpha$-распад,
        \item ядро висмута $\ce{^{214}_{83}{Bi}}$, $\beta$-распад,
        \item ядро свинца $\ce{^{209}_{82}{Pb}}$, $\beta$-распад.
    \end{itemize}
}
\answer{%
    \begin{align*}
    &\ce{^{30}_{15}{P}} \to \ce{^{30}_{14}{Si}} + e^+ + \nu_e: \qquad \text{ядро кремния $\ce{^{30}_{14}{Si}}$}: 14\,p^+, 16\,n^0, \\
    &\ce{^{238}_{92}{U}} \to \ce{^{234}_{90}{Th}} + \ce{^4_2{He}}: \qquad \text{ядро тория $\ce{^{234}_{90}{Th}}$}: 90\,p^+, 144\,n^0, \\
    &\ce{^{214}_{83}{Bi}} \to \ce{^{214}_{84}{Po}} + e^- + \tilde\nu_e: \qquad \text{ядро полония $\ce{^{214}_{84}{Po}}$}: 84\,p^+, 130\,n^0, \\
    &\ce{^{209}_{82}{Pb}} \to \ce{^{209}_{83}{Bi}} + e^- + \tilde\nu_e: \qquad \text{ядро висмута $\ce{^{209}_{83}{Bi}}$}: 83\,p^+, 126\,n^0.
    \end{align*}
}
\solutionspace{80pt}

\tasknumber{14}%
\task{%
    Какая доля (от начального количества) радиоактивных ядер распадётся через время,
    равное четырём периодам полураспада? Ответ выразить в процентах.
}
\answer{%
    \begin{align*}
    N &= N_0 \cdot 2^{- \frac t{T_{1/2}}} \implies
        \frac N{N_0} = 2^{- \frac t{T_{1/2}}}
        = 2^{-4} \approx 0{,}06 \approx 6\% \\
    N_\text{расп.} &= N_0 - N = N_0 - N_0 \cdot 2^{-\frac t{T_{1/2}}}
        = N_0\cbr{1 - 2^{-\frac t{T_{1/2}}}} \implies
        \frac{N_\text{расп.}}{N_0} = 1 - 2^{-\frac t{T_{1/2}}}
        = 1 - 2^{-4} \approx 0{,}94 \approx 94\%
    \end{align*}
}
\solutionspace{90pt}

\tasknumber{15}%
\task{%
    Сколько процентов ядер радиоактивного железа $\ce{^{59}Fe}$
    останется через $136{,}8\,\text{суток}$, если период его полураспада составляет $45{,}6\,\text{суток}$?
}
\answer{%
    \begin{align*}
    N &= N_0 \cdot 2^{-\frac t{T_{1/2}}}
        = 2^{-\frac{136{,}8\,\text{суток}}{45{,}6\,\text{суток}}}
        \approx 0{,}1250 = 12{,}50\%
    \end{align*}
}
\solutionspace{90pt}

\tasknumber{16}%
\task{%
    За $3\,\text{суток}$ от начального количества ядер радиоизотопа осталась одна восьмая.
    Каков период полураспада этого изотопа (ответ приведите в сутках)?
    Какая ещё доля (также от начального количества) распадётся, если подождать ещё столько же?
}
\answer{%
    \begin{align*}
            N &= N_0 \cdot 2^{-\frac t{T_{1/2}}}
            \implies \frac N{N_0} = 2^{-\frac t{T_{1/2}}}
            \implies \frac 1{8} = 2^{-\frac {3\,\text{суток}}{T_{1/2}}}
            \implies 3 = \frac {3\,\text{суток}}{T_{1/2}}
            \implies T_{1/2} = \frac {3\,\text{суток}}3 \approx 1\,\text{суток}.
         \\
            \delta &= \frac{N(t)}{N_0} - \frac{N(2t)}{N_0}
            = 2^{-\frac t{T_{1/2}}} - 2^{-\frac {2t}{T_{1/2}}}
            = 2^{-\frac t{T_{1/2}}}\cbr{1 - 2^{-\frac t{T_{1/2}}}}
            = \frac 1{8} \cdot \cbr{1-\frac 1{8}} \approx 0{,}109
    \end{align*}
}

\variantsplitter

\addpersonalvariant{Елизавета Кутумова}

\tasknumber{1}%
\task{%
    Определите число нейтронов в атоме $\ce{^{2}_{1}{D}}$.
}
\answer{%
    $Z = 1$ протонов и столько же электронов $A = 2$ нуклонов, $A - Z = 1$ нейтронов.
    Ответ: 1
}

\tasknumber{2}%
\task{%
    Определите число протонов в атоме $\ce{^{4}_{2}{He}}$.
}
\answer{%
    $Z = 2$ протонов и столько же электронов $A = 4$ нуклонов, $A - Z = 2$ нейтронов.
    Ответ: 2
}

\tasknumber{3}%
\task{%
    Определите число нуклонов в атоме $\ce{^{10}_{5}{B}}$.
}
\answer{%
    $Z = 5$ протонов и столько же электронов $A = 10$ нуклонов, $A - Z = 5$ нейтронов.
    Ответ: 10
}

\tasknumber{4}%
\task{%
    Определите число нуклонов в атоме $\ce{^{16}_{8}{O}}$.
}
\answer{%
    $Z = 8$ протонов и столько же электронов $A = 16$ нуклонов, $A - Z = 8$ нейтронов.
    Ответ: 16
}

\tasknumber{5}%
\task{%
    Определите число нейтронов в атоме $\ce{^{17}_{8}{O}}$.
}
\answer{%
    $Z = 8$ протонов и столько же электронов $A = 17$ нуклонов, $A - Z = 9$ нейтронов.
    Ответ: 9
}

\tasknumber{6}%
\task{%
    Определите число нуклонов в атоме $\ce{^{25}_{12}{Mg}}$.
}
\answer{%
    $Z = 12$ протонов и столько же электронов $A = 25$ нуклонов, $A - Z = 13$ нейтронов.
    Ответ: 25
}

\tasknumber{7}%
\task{%
    Определите число протонов в атоме $\text{сера-35}$.
}
\answer{%
    $Z = 16$ протонов и столько же электронов, $A = 35$ нуклонов, $A - Z = 19$ нейтронов.
    Ответ: 16
}

\tasknumber{8}%
\task{%
    Определите число нуклонов в атоме $\text{аргон-38}$.
}
\answer{%
    $Z = 18$ протонов и столько же электронов, $A = 38$ нуклонов, $A - Z = 20$ нейтронов.
    Ответ: 38
}

\tasknumber{9}%
\task{%
    Определите число нейтронов в атоме $\text{кальций-43}$.
}
\answer{%
    $Z = 20$ протонов и столько же электронов, $A = 43$ нуклонов, $A - Z = 23$ нейтронов.
    Ответ: 23
}

\tasknumber{10}%
\task{%
    Определите число нуклонов в атоме $\text{ванадий-48}$.
}
\answer{%
    $Z = 23$ протонов и столько же электронов, $A = 48$ нуклонов, $A - Z = 25$ нейтронов.
    Ответ: 48
}

\tasknumber{11}%
\task{%
    Определите число протонов в атоме $\text{марганец-53}$.
}
\answer{%
    $Z = 25$ протонов и столько же электронов, $A = 53$ нуклонов, $A - Z = 28$ нейтронов.
    Ответ: 25
}

\tasknumber{12}%
\task{%
    Определите число нуклонов в атоме $\text{кобальт-56}$.
}
\answer{%
    $Z = 27$ протонов и столько же электронов, $A = 56$ нуклонов, $A - Z = 29$ нейтронов.
    Ответ: 56
}

\tasknumber{13}%
\task{%
    В какое ядро превращается исходное в результате ядерного распада?
    Запишите уравнение реакции и явно укажите число протонов и нейтронов в получившемся ядре.
    \begin{itemize}
        \item ядро тория $\ce{^{234}_{90}{Th}}$, $\beta$-распад,
        \item ядро радона $\ce{^{222}_{86}{Rn}}$, $\alpha$-распад,
        \item ядро полония $\ce{^{210}_{84}{Po}}$, $\beta$-распад,
        \item ядро плутония $\ce{^{239}_{94}{Pu}}$, $\alpha$-распад.
    \end{itemize}
}
\answer{%
    \begin{align*}
    &\ce{^{234}_{90}{Th}} \to \ce{^{234}_{91}{Pa}} + e^- + \tilde\nu_e: \qquad \text{ядро протактиния $\ce{^{234}_{91}{Pa}}$}: 91\,p^+, 143\,n^0, \\
    &\ce{^{222}_{86}{Rn}} \to \ce{^{218}_{84}{Po}} + \ce{^4_2{He}}: \qquad \text{ядро полония $\ce{^{218}_{84}{Po}}$}: 84\,p^+, 134\,n^0, \\
    &\ce{^{210}_{84}{Po}} \to \ce{^{210}_{85}{At}} + e^- + \tilde\nu_e: \qquad \text{ядро астата $\ce{^{210}_{85}{At}}$}: 85\,p^+, 125\,n^0, \\
    &\ce{^{239}_{94}{Pu}} \to \ce{^{235}_{92}{U}} + \ce{^4_2{He}}: \qquad \text{ядро урана $\ce{^{235}_{92}{U}}$}: 92\,p^+, 143\,n^0.
    \end{align*}
}
\solutionspace{80pt}

\tasknumber{14}%
\task{%
    Какая доля (от начального количества) радиоактивных ядер распадётся через время,
    равное трём периодам полураспада? Ответ выразить в процентах.
}
\answer{%
    \begin{align*}
    N &= N_0 \cdot 2^{- \frac t{T_{1/2}}} \implies
        \frac N{N_0} = 2^{- \frac t{T_{1/2}}}
        = 2^{-3} \approx 0{,}12 \approx 12\% \\
    N_\text{расп.} &= N_0 - N = N_0 - N_0 \cdot 2^{-\frac t{T_{1/2}}}
        = N_0\cbr{1 - 2^{-\frac t{T_{1/2}}}} \implies
        \frac{N_\text{расп.}}{N_0} = 1 - 2^{-\frac t{T_{1/2}}}
        = 1 - 2^{-3} \approx 0{,}88 \approx 88\%
    \end{align*}
}
\solutionspace{90pt}

\tasknumber{15}%
\task{%
    Сколько процентов ядер радиоактивного железа $\ce{^{59}Fe}$
    останется через $182{,}4\,\text{суток}$, если период его полураспада составляет $45{,}6\,\text{суток}$?
}
\answer{%
    \begin{align*}
    N &= N_0 \cdot 2^{-\frac t{T_{1/2}}}
        = 2^{-\frac{182{,}4\,\text{суток}}{45{,}6\,\text{суток}}}
        \approx 0{,}0625 = 6{,}25\%
    \end{align*}
}
\solutionspace{90pt}

\tasknumber{16}%
\task{%
    За $2\,\text{суток}$ от начального количества ядер радиоизотопа осталась одна восьмая.
    Каков период полураспада этого изотопа (ответ приведите в сутках)?
    Какая ещё доля (также от начального количества) распадётся, если подождать ещё столько же?
}
\answer{%
    \begin{align*}
            N &= N_0 \cdot 2^{-\frac t{T_{1/2}}}
            \implies \frac N{N_0} = 2^{-\frac t{T_{1/2}}}
            \implies \frac 1{8} = 2^{-\frac {2\,\text{суток}}{T_{1/2}}}
            \implies 3 = \frac {2\,\text{суток}}{T_{1/2}}
            \implies T_{1/2} = \frac {2\,\text{суток}}3 \approx 0{,}67\,\text{суток}.
         \\
            \delta &= \frac{N(t)}{N_0} - \frac{N(2t)}{N_0}
            = 2^{-\frac t{T_{1/2}}} - 2^{-\frac {2t}{T_{1/2}}}
            = 2^{-\frac t{T_{1/2}}}\cbr{1 - 2^{-\frac t{T_{1/2}}}}
            = \frac 1{8} \cdot \cbr{1-\frac 1{8}} \approx 0{,}109
    \end{align*}
}

\variantsplitter

\addpersonalvariant{Роксана Мехтиева}

\tasknumber{1}%
\task{%
    Определите число нуклонов в атоме $\ce{^{1}_{1}{H}}$.
}
\answer{%
    $Z = 1$ протонов и столько же электронов $A = 1$ нуклонов, $A - Z = 0$ нейтронов.
    Ответ: 1
}

\tasknumber{2}%
\task{%
    Определите число протонов в атоме $\ce{^{6}_{3}{Li}}$.
}
\answer{%
    $Z = 3$ протонов и столько же электронов $A = 6$ нуклонов, $A - Z = 3$ нейтронов.
    Ответ: 3
}

\tasknumber{3}%
\task{%
    Определите число нейтронов в атоме $\ce{^{10}_{5}{B}}$.
}
\answer{%
    $Z = 5$ протонов и столько же электронов $A = 10$ нуклонов, $A - Z = 5$ нейтронов.
    Ответ: 5
}

\tasknumber{4}%
\task{%
    Определите число протонов в атоме $\ce{^{12}_{6}{C}}$.
}
\answer{%
    $Z = 6$ протонов и столько же электронов $A = 12$ нуклонов, $A - Z = 6$ нейтронов.
    Ответ: 6
}

\tasknumber{5}%
\task{%
    Определите число нейтронов в атоме $\ce{^{22}_{11}{Na}}$.
}
\answer{%
    $Z = 11$ протонов и столько же электронов $A = 22$ нуклонов, $A - Z = 11$ нейтронов.
    Ответ: 11
}

\tasknumber{6}%
\task{%
    Определите число электронов в атоме $\ce{^{28}_{14}{Si}}$.
}
\answer{%
    $Z = 14$ протонов и столько же электронов $A = 28$ нуклонов, $A - Z = 14$ нейтронов.
    Ответ: 14
}

\tasknumber{7}%
\task{%
    Определите число электронов в атоме $\text{фосфор-31}$.
}
\answer{%
    $Z = 15$ протонов и столько же электронов, $A = 31$ нуклонов, $A - Z = 16$ нейтронов.
    Ответ: 15
}

\tasknumber{8}%
\task{%
    Определите число электронов в атоме $\text{аргон-39}$.
}
\answer{%
    $Z = 18$ протонов и столько же электронов, $A = 39$ нуклонов, $A - Z = 21$ нейтронов.
    Ответ: 18
}

\tasknumber{9}%
\task{%
    Определите число электронов в атоме $\text{кальций-44}$.
}
\answer{%
    $Z = 20$ протонов и столько же электронов, $A = 44$ нуклонов, $A - Z = 24$ нейтронов.
    Ответ: 20
}

\tasknumber{10}%
\task{%
    Определите число нуклонов в атоме $\text{титан-47}$.
}
\answer{%
    $Z = 22$ протонов и столько же электронов, $A = 47$ нуклонов, $A - Z = 25$ нейтронов.
    Ответ: 47
}

\tasknumber{11}%
\task{%
    Определите число нейтронов в атоме $\text{железо-54}$.
}
\answer{%
    $Z = 26$ протонов и столько же электронов, $A = 54$ нуклонов, $A - Z = 28$ нейтронов.
    Ответ: 28
}

\tasknumber{12}%
\task{%
    Определите число нуклонов в атоме $\text{никель-58}$.
}
\answer{%
    $Z = 28$ протонов и столько же электронов, $A = 58$ нуклонов, $A - Z = 30$ нейтронов.
    Ответ: 58
}

\tasknumber{13}%
\task{%
    В какое ядро превращается исходное в результате ядерного распада?
    Запишите уравнение реакции и явно укажите число протонов и нейтронов в получившемся ядре.
    \begin{itemize}
        \item ядро фосфора $\ce{^{30}_{15}{P}}$, $\beta^+$-распад,
        \item ядро урана $\ce{^{234}_{92}{U}}$, $\alpha$-распад,
        \item ядро полония $\ce{^{210}_{84}{Po}}$, $\beta$-распад,
        \item ядро свинца $\ce{^{209}_{82}{Pb}}$, $\beta$-распад.
    \end{itemize}
}
\answer{%
    \begin{align*}
    &\ce{^{30}_{15}{P}} \to \ce{^{30}_{14}{Si}} + e^+ + \nu_e: \qquad \text{ядро кремния $\ce{^{30}_{14}{Si}}$}: 14\,p^+, 16\,n^0, \\
    &\ce{^{234}_{92}{U}} \to \ce{^{230}_{90}{Th}} + \ce{^4_2{He}}: \qquad \text{ядро тория $\ce{^{230}_{90}{Th}}$}: 90\,p^+, 140\,n^0, \\
    &\ce{^{210}_{84}{Po}} \to \ce{^{210}_{85}{At}} + e^- + \tilde\nu_e: \qquad \text{ядро астата $\ce{^{210}_{85}{At}}$}: 85\,p^+, 125\,n^0, \\
    &\ce{^{209}_{82}{Pb}} \to \ce{^{209}_{83}{Bi}} + e^- + \tilde\nu_e: \qquad \text{ядро висмута $\ce{^{209}_{83}{Bi}}$}: 83\,p^+, 126\,n^0.
    \end{align*}
}
\solutionspace{80pt}

\tasknumber{14}%
\task{%
    Какая доля (от начального количества) радиоактивных ядер распадётся через время,
    равное трём периодам полураспада? Ответ выразить в процентах.
}
\answer{%
    \begin{align*}
    N &= N_0 \cdot 2^{- \frac t{T_{1/2}}} \implies
        \frac N{N_0} = 2^{- \frac t{T_{1/2}}}
        = 2^{-3} \approx 0{,}12 \approx 12\% \\
    N_\text{расп.} &= N_0 - N = N_0 - N_0 \cdot 2^{-\frac t{T_{1/2}}}
        = N_0\cbr{1 - 2^{-\frac t{T_{1/2}}}} \implies
        \frac{N_\text{расп.}}{N_0} = 1 - 2^{-\frac t{T_{1/2}}}
        = 1 - 2^{-3} \approx 0{,}88 \approx 88\%
    \end{align*}
}
\solutionspace{90pt}

\tasknumber{15}%
\task{%
    Сколько процентов ядер радиоактивного железа $\ce{^{59}Fe}$
    останется через $91{,}2\,\text{суток}$, если период его полураспада составляет $45{,}6\,\text{суток}$?
}
\answer{%
    \begin{align*}
    N &= N_0 \cdot 2^{-\frac t{T_{1/2}}}
        = 2^{-\frac{91{,}2\,\text{суток}}{45{,}6\,\text{суток}}}
        \approx 0{,}2500 = 25{,}00\%
    \end{align*}
}
\solutionspace{90pt}

\tasknumber{16}%
\task{%
    За $4\,\text{суток}$ от начального количества ядер радиоизотопа осталась половина.
    Каков период полураспада этого изотопа (ответ приведите в сутках)?
    Какая ещё доля (также от начального количества) распадётся, если подождать ещё столько же?
}
\answer{%
    \begin{align*}
            N &= N_0 \cdot 2^{-\frac t{T_{1/2}}}
            \implies \frac N{N_0} = 2^{-\frac t{T_{1/2}}}
            \implies \frac 1{2} = 2^{-\frac {4\,\text{суток}}{T_{1/2}}}
            \implies 1 = \frac {4\,\text{суток}}{T_{1/2}}
            \implies T_{1/2} = \frac {4\,\text{суток}}1 \approx 4\,\text{суток}.
         \\
            \delta &= \frac{N(t)}{N_0} - \frac{N(2t)}{N_0}
            = 2^{-\frac t{T_{1/2}}} - 2^{-\frac {2t}{T_{1/2}}}
            = 2^{-\frac t{T_{1/2}}}\cbr{1 - 2^{-\frac t{T_{1/2}}}}
            = \frac 1{2} \cdot \cbr{1-\frac 1{2}} \approx 0{,}250
    \end{align*}
}

\variantsplitter

\addpersonalvariant{Дилноза Нодиршоева}

\tasknumber{1}%
\task{%
    Определите число нуклонов в атоме $\ce{^{3}_{1}{T}}$.
}
\answer{%
    $Z = 1$ протонов и столько же электронов $A = 3$ нуклонов, $A - Z = 2$ нейтронов.
    Ответ: 3
}

\tasknumber{2}%
\task{%
    Определите число нейтронов в атоме $\ce{^{3}_{2}{He}}$.
}
\answer{%
    $Z = 2$ протонов и столько же электронов $A = 3$ нуклонов, $A - Z = 1$ нейтронов.
    Ответ: 1
}

\tasknumber{3}%
\task{%
    Определите число протонов в атоме $\ce{^{10}_{5}{B}}$.
}
\answer{%
    $Z = 5$ протонов и столько же электронов $A = 10$ нуклонов, $A - Z = 5$ нейтронов.
    Ответ: 5
}

\tasknumber{4}%
\task{%
    Определите число протонов в атоме $\ce{^{12}_{6}{C}}$.
}
\answer{%
    $Z = 6$ протонов и столько же электронов $A = 12$ нуклонов, $A - Z = 6$ нейтронов.
    Ответ: 6
}

\tasknumber{5}%
\task{%
    Определите число электронов в атоме $\ce{^{22}_{10}{Ne}}$.
}
\answer{%
    $Z = 10$ протонов и столько же электронов $A = 22$ нуклонов, $A - Z = 12$ нейтронов.
    Ответ: 10
}

\tasknumber{6}%
\task{%
    Определите число нуклонов в атоме $\ce{^{23}_{11}{Na}}$.
}
\answer{%
    $Z = 11$ протонов и столько же электронов $A = 23$ нуклонов, $A - Z = 12$ нейтронов.
    Ответ: 23
}

\tasknumber{7}%
\task{%
    Определите число протонов в атоме $\text{сера-32}$.
}
\answer{%
    $Z = 16$ протонов и столько же электронов, $A = 32$ нуклонов, $A - Z = 16$ нейтронов.
    Ответ: 16
}

\tasknumber{8}%
\task{%
    Определите число протонов в атоме $\text{калий-39}$.
}
\answer{%
    $Z = 19$ протонов и столько же электронов, $A = 39$ нуклонов, $A - Z = 20$ нейтронов.
    Ответ: 19
}

\tasknumber{9}%
\task{%
    Определите число нейтронов в атоме $\text{кальций-48}$.
}
\answer{%
    $Z = 20$ протонов и столько же электронов, $A = 48$ нуклонов, $A - Z = 28$ нейтронов.
    Ответ: 28
}

\tasknumber{10}%
\task{%
    Определите число электронов в атоме $\text{ванадий-48}$.
}
\answer{%
    $Z = 23$ протонов и столько же электронов, $A = 48$ нуклонов, $A - Z = 25$ нейтронов.
    Ответ: 23
}

\tasknumber{11}%
\task{%
    Определите число нейтронов в атоме $\text{марганец-52}$.
}
\answer{%
    $Z = 25$ протонов и столько же электронов, $A = 52$ нуклонов, $A - Z = 27$ нейтронов.
    Ответ: 27
}

\tasknumber{12}%
\task{%
    Определите число нуклонов в атоме $\text{железо-59}$.
}
\answer{%
    $Z = 26$ протонов и столько же электронов, $A = 59$ нуклонов, $A - Z = 33$ нейтронов.
    Ответ: 59
}

\tasknumber{13}%
\task{%
    В какое ядро превращается исходное в результате ядерного распада?
    Запишите уравнение реакции и явно укажите число протонов и нейтронов в получившемся ядре.
    \begin{itemize}
        \item ядро тория $\ce{^{234}_{90}{Th}}$, $\beta$-распад,
        \item ядро урана $\ce{^{238}_{92}{U}}$, $\alpha$-распад,
        \item ядро свинца $\ce{^{210}_{82}{Pb}}$, $\beta$-распад,
        \item ядро свинца $\ce{^{209}_{82}{Pb}}$, $\beta$-распад.
    \end{itemize}
}
\answer{%
    \begin{align*}
    &\ce{^{234}_{90}{Th}} \to \ce{^{234}_{91}{Pa}} + e^- + \tilde\nu_e: \qquad \text{ядро протактиния $\ce{^{234}_{91}{Pa}}$}: 91\,p^+, 143\,n^0, \\
    &\ce{^{238}_{92}{U}} \to \ce{^{234}_{90}{Th}} + \ce{^4_2{He}}: \qquad \text{ядро тория $\ce{^{234}_{90}{Th}}$}: 90\,p^+, 144\,n^0, \\
    &\ce{^{210}_{82}{Pb}} \to \ce{^{210}_{83}{Bi}} + e^- + \tilde\nu_e: \qquad \text{ядро висмута $\ce{^{210}_{83}{Bi}}$}: 83\,p^+, 127\,n^0, \\
    &\ce{^{209}_{82}{Pb}} \to \ce{^{209}_{83}{Bi}} + e^- + \tilde\nu_e: \qquad \text{ядро висмута $\ce{^{209}_{83}{Bi}}$}: 83\,p^+, 126\,n^0.
    \end{align*}
}
\solutionspace{80pt}

\tasknumber{14}%
\task{%
    Какая доля (от начального количества) радиоактивных ядер останется через время,
    равное двум периодам полураспада? Ответ выразить в процентах.
}
\answer{%
    \begin{align*}
    N &= N_0 \cdot 2^{- \frac t{T_{1/2}}} \implies
        \frac N{N_0} = 2^{- \frac t{T_{1/2}}}
        = 2^{-2} \approx 0{,}25 \approx 25\% \\
    N_\text{расп.} &= N_0 - N = N_0 - N_0 \cdot 2^{-\frac t{T_{1/2}}}
        = N_0\cbr{1 - 2^{-\frac t{T_{1/2}}}} \implies
        \frac{N_\text{расп.}}{N_0} = 1 - 2^{-\frac t{T_{1/2}}}
        = 1 - 2^{-2} \approx 0{,}75 \approx 75\%
    \end{align*}
}
\solutionspace{90pt}

\tasknumber{15}%
\task{%
    Сколько процентов ядер радиоактивного железа $\ce{^{59}Fe}$
    останется через $91{,}2\,\text{суток}$, если период его полураспада составляет $45{,}6\,\text{суток}$?
}
\answer{%
    \begin{align*}
    N &= N_0 \cdot 2^{-\frac t{T_{1/2}}}
        = 2^{-\frac{91{,}2\,\text{суток}}{45{,}6\,\text{суток}}}
        \approx 0{,}2500 = 25{,}00\%
    \end{align*}
}
\solutionspace{90pt}

\tasknumber{16}%
\task{%
    За $3\,\text{суток}$ от начального количества ядер радиоизотопа осталась четверть.
    Каков период полураспада этого изотопа (ответ приведите в сутках)?
    Какая ещё доля (также от начального количества) распадётся, если подождать ещё столько же?
}
\answer{%
    \begin{align*}
            N &= N_0 \cdot 2^{-\frac t{T_{1/2}}}
            \implies \frac N{N_0} = 2^{-\frac t{T_{1/2}}}
            \implies \frac 1{4} = 2^{-\frac {3\,\text{суток}}{T_{1/2}}}
            \implies 2 = \frac {3\,\text{суток}}{T_{1/2}}
            \implies T_{1/2} = \frac {3\,\text{суток}}2 \approx 1{,}50\,\text{суток}.
         \\
            \delta &= \frac{N(t)}{N_0} - \frac{N(2t)}{N_0}
            = 2^{-\frac t{T_{1/2}}} - 2^{-\frac {2t}{T_{1/2}}}
            = 2^{-\frac t{T_{1/2}}}\cbr{1 - 2^{-\frac t{T_{1/2}}}}
            = \frac 1{4} \cdot \cbr{1-\frac 1{4}} \approx 0{,}188
    \end{align*}
}

\variantsplitter

\addpersonalvariant{Жаклин Пантелеева}

\tasknumber{1}%
\task{%
    Определите число электронов в атоме $\ce{^{1}_{1}{H}}$.
}
\answer{%
    $Z = 1$ протонов и столько же электронов $A = 1$ нуклонов, $A - Z = 0$ нейтронов.
    Ответ: 1
}

\tasknumber{2}%
\task{%
    Определите число протонов в атоме $\ce{^{6}_{3}{Li}}$.
}
\answer{%
    $Z = 3$ протонов и столько же электронов $A = 6$ нуклонов, $A - Z = 3$ нейтронов.
    Ответ: 3
}

\tasknumber{3}%
\task{%
    Определите число электронов в атоме $\ce{^{10}_{4}{Be}}$.
}
\answer{%
    $Z = 4$ протонов и столько же электронов $A = 10$ нуклонов, $A - Z = 6$ нейтронов.
    Ответ: 4
}

\tasknumber{4}%
\task{%
    Определите число нейтронов в атоме $\ce{^{14}_{7}{N}}$.
}
\answer{%
    $Z = 7$ протонов и столько же электронов $A = 14$ нуклонов, $A - Z = 7$ нейтронов.
    Ответ: 7
}

\tasknumber{5}%
\task{%
    Определите число нейтронов в атоме $\ce{^{19}_{9}{F}}$.
}
\answer{%
    $Z = 9$ протонов и столько же электронов $A = 19$ нуклонов, $A - Z = 10$ нейтронов.
    Ответ: 10
}

\tasknumber{6}%
\task{%
    Определите число протонов в атоме $\ce{^{29}_{14}{Si}}$.
}
\answer{%
    $Z = 14$ протонов и столько же электронов $A = 29$ нуклонов, $A - Z = 15$ нейтронов.
    Ответ: 14
}

\tasknumber{7}%
\task{%
    Определите число электронов в атоме $\text{сера-33}$.
}
\answer{%
    $Z = 16$ протонов и столько же электронов, $A = 33$ нуклонов, $A - Z = 17$ нейтронов.
    Ответ: 16
}

\tasknumber{8}%
\task{%
    Определите число нейтронов в атоме $\text{калий-39}$.
}
\answer{%
    $Z = 19$ протонов и столько же электронов, $A = 39$ нуклонов, $A - Z = 20$ нейтронов.
    Ответ: 20
}

\tasknumber{9}%
\task{%
    Определите число электронов в атоме $\text{кальций-48}$.
}
\answer{%
    $Z = 20$ протонов и столько же электронов, $A = 48$ нуклонов, $A - Z = 28$ нейтронов.
    Ответ: 20
}

\tasknumber{10}%
\task{%
    Определите число нуклонов в атоме $\text{титан-47}$.
}
\answer{%
    $Z = 22$ протонов и столько же электронов, $A = 47$ нуклонов, $A - Z = 25$ нейтронов.
    Ответ: 47
}

\tasknumber{11}%
\task{%
    Определите число электронов в атоме $\text{ванадий-49}$.
}
\answer{%
    $Z = 23$ протонов и столько же электронов, $A = 49$ нуклонов, $A - Z = 26$ нейтронов.
    Ответ: 23
}

\tasknumber{12}%
\task{%
    Определите число протонов в атоме $\text{никель-57}$.
}
\answer{%
    $Z = 28$ протонов и столько же электронов, $A = 57$ нуклонов, $A - Z = 29$ нейтронов.
    Ответ: 28
}

\tasknumber{13}%
\task{%
    В какое ядро превращается исходное в результате ядерного распада?
    Запишите уравнение реакции и явно укажите число протонов и нейтронов в получившемся ядре.
    \begin{itemize}
        \item ядро тория $\ce{^{234}_{90}{Th}}$, $\beta^-$-распад,
        \item ядро полония $\ce{^{210}_{84}{Po}}$, $\alpha$-распад,
        \item ядро полония $\ce{^{214}_{84}{Po}}$, $\alpha$-распад,
        \item ядро плутония $\ce{^{239}_{94}{Pu}}$, $\alpha$-распад.
    \end{itemize}
}
\answer{%
    \begin{align*}
    &\ce{^{234}_{90}{Th}} \to \ce{^{234}_{91}{Pa}} + e^- + \tilde\nu_e: \qquad \text{ядро протактиния $\ce{^{234}_{91}{Pa}}$}: 91\,p^+, 143\,n^0, \\
    &\ce{^{210}_{84}{Po}} \to \ce{^{206}_{82}{Pb}} + \ce{^4_2{He}}: \qquad \text{ядро свинца $\ce{^{206}_{82}{Pb}}$}: 82\,p^+, 124\,n^0, \\
    &\ce{^{214}_{84}{Po}} \to \ce{^{210}_{82}{Pb}} + \ce{^4_2{He}}: \qquad \text{ядро свинца $\ce{^{210}_{82}{Pb}}$}: 82\,p^+, 128\,n^0, \\
    &\ce{^{239}_{94}{Pu}} \to \ce{^{235}_{92}{U}} + \ce{^4_2{He}}: \qquad \text{ядро урана $\ce{^{235}_{92}{U}}$}: 92\,p^+, 143\,n^0.
    \end{align*}
}
\solutionspace{80pt}

\tasknumber{14}%
\task{%
    Какая доля (от начального количества) радиоактивных ядер распадётся через время,
    равное двум периодам полураспада? Ответ выразить в процентах.
}
\answer{%
    \begin{align*}
    N &= N_0 \cdot 2^{- \frac t{T_{1/2}}} \implies
        \frac N{N_0} = 2^{- \frac t{T_{1/2}}}
        = 2^{-2} \approx 0{,}25 \approx 25\% \\
    N_\text{расп.} &= N_0 - N = N_0 - N_0 \cdot 2^{-\frac t{T_{1/2}}}
        = N_0\cbr{1 - 2^{-\frac t{T_{1/2}}}} \implies
        \frac{N_\text{расп.}}{N_0} = 1 - 2^{-\frac t{T_{1/2}}}
        = 1 - 2^{-2} \approx 0{,}75 \approx 75\%
    \end{align*}
}
\solutionspace{90pt}

\tasknumber{15}%
\task{%
    Сколько процентов ядер радиоактивного железа $\ce{^{59}Fe}$
    останется через $91{,}2\,\text{суток}$, если период его полураспада составляет $45{,}6\,\text{суток}$?
}
\answer{%
    \begin{align*}
    N &= N_0 \cdot 2^{-\frac t{T_{1/2}}}
        = 2^{-\frac{91{,}2\,\text{суток}}{45{,}6\,\text{суток}}}
        \approx 0{,}2500 = 25{,}00\%
    \end{align*}
}
\solutionspace{90pt}

\tasknumber{16}%
\task{%
    За $4\,\text{суток}$ от начального количества ядер радиоизотопа осталась четверть.
    Каков период полураспада этого изотопа (ответ приведите в сутках)?
    Какая ещё доля (также от начального количества) распадётся, если подождать ещё столько же?
}
\answer{%
    \begin{align*}
            N &= N_0 \cdot 2^{-\frac t{T_{1/2}}}
            \implies \frac N{N_0} = 2^{-\frac t{T_{1/2}}}
            \implies \frac 1{4} = 2^{-\frac {4\,\text{суток}}{T_{1/2}}}
            \implies 2 = \frac {4\,\text{суток}}{T_{1/2}}
            \implies T_{1/2} = \frac {4\,\text{суток}}2 \approx 2\,\text{суток}.
         \\
            \delta &= \frac{N(t)}{N_0} - \frac{N(2t)}{N_0}
            = 2^{-\frac t{T_{1/2}}} - 2^{-\frac {2t}{T_{1/2}}}
            = 2^{-\frac t{T_{1/2}}}\cbr{1 - 2^{-\frac t{T_{1/2}}}}
            = \frac 1{4} \cdot \cbr{1-\frac 1{4}} \approx 0{,}188
    \end{align*}
}

\variantsplitter

\addpersonalvariant{Артём Переверзев}

\tasknumber{1}%
\task{%
    Определите число протонов в атоме $\ce{^{2}_{1}{D}}$.
}
\answer{%
    $Z = 1$ протонов и столько же электронов $A = 2$ нуклонов, $A - Z = 1$ нейтронов.
    Ответ: 1
}

\tasknumber{2}%
\task{%
    Определите число нейтронов в атоме $\ce{^{4}_{2}{He}}$.
}
\answer{%
    $Z = 2$ протонов и столько же электронов $A = 4$ нуклонов, $A - Z = 2$ нейтронов.
    Ответ: 2
}

\tasknumber{3}%
\task{%
    Определите число протонов в атоме $\ce{^{10}_{5}{B}}$.
}
\answer{%
    $Z = 5$ протонов и столько же электронов $A = 10$ нуклонов, $A - Z = 5$ нейтронов.
    Ответ: 5
}

\tasknumber{4}%
\task{%
    Определите число нуклонов в атоме $\ce{^{14}_{7}{N}}$.
}
\answer{%
    $Z = 7$ протонов и столько же электронов $A = 14$ нуклонов, $A - Z = 7$ нейтронов.
    Ответ: 14
}

\tasknumber{5}%
\task{%
    Определите число электронов в атоме $\ce{^{18}_{8}{O}}$.
}
\answer{%
    $Z = 8$ протонов и столько же электронов $A = 18$ нуклонов, $A - Z = 10$ нейтронов.
    Ответ: 8
}

\tasknumber{6}%
\task{%
    Определите число протонов в атоме $\ce{^{23}_{11}{Na}}$.
}
\answer{%
    $Z = 11$ протонов и столько же электронов $A = 23$ нуклонов, $A - Z = 12$ нейтронов.
    Ответ: 11
}

\tasknumber{7}%
\task{%
    Определите число нуклонов в атоме $\text{сера-35}$.
}
\answer{%
    $Z = 16$ протонов и столько же электронов, $A = 35$ нуклонов, $A - Z = 19$ нейтронов.
    Ответ: 35
}

\tasknumber{8}%
\task{%
    Определите число электронов в атоме $\text{калий-39}$.
}
\answer{%
    $Z = 19$ протонов и столько же электронов, $A = 39$ нуклонов, $A - Z = 20$ нейтронов.
    Ответ: 19
}

\tasknumber{9}%
\task{%
    Определите число нуклонов в атоме $\text{калий-41}$.
}
\answer{%
    $Z = 19$ протонов и столько же электронов, $A = 41$ нуклонов, $A - Z = 22$ нейтронов.
    Ответ: 41
}

\tasknumber{10}%
\task{%
    Определите число нуклонов в атоме $\text{титан-49}$.
}
\answer{%
    $Z = 22$ протонов и столько же электронов, $A = 49$ нуклонов, $A - Z = 27$ нейтронов.
    Ответ: 49
}

\tasknumber{11}%
\task{%
    Определите число нейтронов в атоме $\text{хром-54}$.
}
\answer{%
    $Z = 24$ протонов и столько же электронов, $A = 54$ нуклонов, $A - Z = 30$ нейтронов.
    Ответ: 30
}

\tasknumber{12}%
\task{%
    Определите число нейтронов в атоме $\text{кобальт-56}$.
}
\answer{%
    $Z = 27$ протонов и столько же электронов, $A = 56$ нуклонов, $A - Z = 29$ нейтронов.
    Ответ: 29
}

\tasknumber{13}%
\task{%
    В какое ядро превращается исходное в результате ядерного распада?
    Запишите уравнение реакции и явно укажите число протонов и нейтронов в получившемся ядре.
    \begin{itemize}
        \item ядро протактиния $\ce{^{234}_{91}{Pa}}$, $\beta$-распад,
        \item ядро тория $\ce{^{230}_{90}{Th}}$, $\alpha$-распад,
        \item ядро полония $\ce{^{210}_{84}{Po}}$, $\beta$-распад,
        \item ядро свинца $\ce{^{209}_{82}{Pb}}$, $\beta$-распад.
    \end{itemize}
}
\answer{%
    \begin{align*}
    &\ce{^{234}_{91}{Pa}} \to \ce{^{234}_{92}{U}} + e^- + \tilde\nu_e: \qquad \text{ядро урана $\ce{^{234}_{92}{U}}$}: 92\,p^+, 142\,n^0, \\
    &\ce{^{230}_{90}{Th}} \to \ce{^{226}_{88}{Ra}} + \ce{^4_2{He}}: \qquad \text{ядро радия $\ce{^{226}_{88}{Ra}}$}: 88\,p^+, 138\,n^0, \\
    &\ce{^{210}_{84}{Po}} \to \ce{^{210}_{85}{At}} + e^- + \tilde\nu_e: \qquad \text{ядро астата $\ce{^{210}_{85}{At}}$}: 85\,p^+, 125\,n^0, \\
    &\ce{^{209}_{82}{Pb}} \to \ce{^{209}_{83}{Bi}} + e^- + \tilde\nu_e: \qquad \text{ядро висмута $\ce{^{209}_{83}{Bi}}$}: 83\,p^+, 126\,n^0.
    \end{align*}
}
\solutionspace{80pt}

\tasknumber{14}%
\task{%
    Какая доля (от начального количества) радиоактивных ядер останется через время,
    равное двум периодам полураспада? Ответ выразить в процентах.
}
\answer{%
    \begin{align*}
    N &= N_0 \cdot 2^{- \frac t{T_{1/2}}} \implies
        \frac N{N_0} = 2^{- \frac t{T_{1/2}}}
        = 2^{-2} \approx 0{,}25 \approx 25\% \\
    N_\text{расп.} &= N_0 - N = N_0 - N_0 \cdot 2^{-\frac t{T_{1/2}}}
        = N_0\cbr{1 - 2^{-\frac t{T_{1/2}}}} \implies
        \frac{N_\text{расп.}}{N_0} = 1 - 2^{-\frac t{T_{1/2}}}
        = 1 - 2^{-2} \approx 0{,}75 \approx 75\%
    \end{align*}
}
\solutionspace{90pt}

\tasknumber{15}%
\task{%
    Сколько процентов ядер радиоактивного железа $\ce{^{59}Fe}$
    останется через $91{,}2\,\text{суток}$, если период его полураспада составляет $45{,}6\,\text{суток}$?
}
\answer{%
    \begin{align*}
    N &= N_0 \cdot 2^{-\frac t{T_{1/2}}}
        = 2^{-\frac{91{,}2\,\text{суток}}{45{,}6\,\text{суток}}}
        \approx 0{,}2500 = 25{,}00\%
    \end{align*}
}
\solutionspace{90pt}

\tasknumber{16}%
\task{%
    За $2\,\text{суток}$ от начального количества ядер радиоизотопа осталась одна шестнадцатая.
    Каков период полураспада этого изотопа (ответ приведите в сутках)?
    Какая ещё доля (также от начального количества) распадётся, если подождать ещё столько же?
}
\answer{%
    \begin{align*}
            N &= N_0 \cdot 2^{-\frac t{T_{1/2}}}
            \implies \frac N{N_0} = 2^{-\frac t{T_{1/2}}}
            \implies \frac 1{16} = 2^{-\frac {2\,\text{суток}}{T_{1/2}}}
            \implies 4 = \frac {2\,\text{суток}}{T_{1/2}}
            \implies T_{1/2} = \frac {2\,\text{суток}}4 \approx 0{,}50\,\text{суток}.
         \\
            \delta &= \frac{N(t)}{N_0} - \frac{N(2t)}{N_0}
            = 2^{-\frac t{T_{1/2}}} - 2^{-\frac {2t}{T_{1/2}}}
            = 2^{-\frac t{T_{1/2}}}\cbr{1 - 2^{-\frac t{T_{1/2}}}}
            = \frac 1{16} \cdot \cbr{1-\frac 1{16}} \approx 0{,}059
    \end{align*}
}

\variantsplitter

\addpersonalvariant{Варвара Пранова}

\tasknumber{1}%
\task{%
    Определите число электронов в атоме $\ce{^{3}_{1}{T}}$.
}
\answer{%
    $Z = 1$ протонов и столько же электронов $A = 3$ нуклонов, $A - Z = 2$ нейтронов.
    Ответ: 1
}

\tasknumber{2}%
\task{%
    Определите число электронов в атоме $\ce{^{6}_{3}{Li}}$.
}
\answer{%
    $Z = 3$ протонов и столько же электронов $A = 6$ нуклонов, $A - Z = 3$ нейтронов.
    Ответ: 3
}

\tasknumber{3}%
\task{%
    Определите число нуклонов в атоме $\ce{^{10}_{4}{Be}}$.
}
\answer{%
    $Z = 4$ протонов и столько же электронов $A = 10$ нуклонов, $A - Z = 6$ нейтронов.
    Ответ: 10
}

\tasknumber{4}%
\task{%
    Определите число нуклонов в атоме $\ce{^{14}_{7}{N}}$.
}
\answer{%
    $Z = 7$ протонов и столько же электронов $A = 14$ нуклонов, $A - Z = 7$ нейтронов.
    Ответ: 14
}

\tasknumber{5}%
\task{%
    Определите число нейтронов в атоме $\ce{^{21}_{10}{Ne}}$.
}
\answer{%
    $Z = 10$ протонов и столько же электронов $A = 21$ нуклонов, $A - Z = 11$ нейтронов.
    Ответ: 11
}

\tasknumber{6}%
\task{%
    Определите число электронов в атоме $\ce{^{27}_{13}{Al}}$.
}
\answer{%
    $Z = 13$ протонов и столько же электронов $A = 27$ нуклонов, $A - Z = 14$ нейтронов.
    Ответ: 13
}

\tasknumber{7}%
\task{%
    Определите число электронов в атоме $\text{сера-35}$.
}
\answer{%
    $Z = 16$ протонов и столько же электронов, $A = 35$ нуклонов, $A - Z = 19$ нейтронов.
    Ответ: 16
}

\tasknumber{8}%
\task{%
    Определите число нейтронов в атоме $\text{хлор-35}$.
}
\answer{%
    $Z = 17$ протонов и столько же электронов, $A = 35$ нуклонов, $A - Z = 18$ нейтронов.
    Ответ: 18
}

\tasknumber{9}%
\task{%
    Определите число нейтронов в атоме $\text{кальций-40}$.
}
\answer{%
    $Z = 20$ протонов и столько же электронов, $A = 40$ нуклонов, $A - Z = 20$ нейтронов.
    Ответ: 20
}

\tasknumber{10}%
\task{%
    Определите число нуклонов в атоме $\text{скандий-47}$.
}
\answer{%
    $Z = 21$ протонов и столько же электронов, $A = 47$ нуклонов, $A - Z = 26$ нейтронов.
    Ответ: 47
}

\tasknumber{11}%
\task{%
    Определите число нуклонов в атоме $\text{хром-50}$.
}
\answer{%
    $Z = 24$ протонов и столько же электронов, $A = 50$ нуклонов, $A - Z = 26$ нейтронов.
    Ответ: 50
}

\tasknumber{12}%
\task{%
    Определите число нуклонов в атоме $\text{кобальт-56}$.
}
\answer{%
    $Z = 27$ протонов и столько же электронов, $A = 56$ нуклонов, $A - Z = 29$ нейтронов.
    Ответ: 56
}

\tasknumber{13}%
\task{%
    В какое ядро превращается исходное в результате ядерного распада?
    Запишите уравнение реакции и явно укажите число протонов и нейтронов в получившемся ядре.
    \begin{itemize}
        \item ядро тория $\ce{^{234}_{90}{Th}}$, $\beta^-$-распад,
        \item ядро урана $\ce{^{238}_{92}{U}}$, $\alpha$-распад,
        \item ядро свинца $\ce{^{210}_{82}{Pb}}$, $\beta$-распад,
        \item ядро плутония $\ce{^{239}_{94}{Pu}}$, $\alpha$-распад.
    \end{itemize}
}
\answer{%
    \begin{align*}
    &\ce{^{234}_{90}{Th}} \to \ce{^{234}_{91}{Pa}} + e^- + \tilde\nu_e: \qquad \text{ядро протактиния $\ce{^{234}_{91}{Pa}}$}: 91\,p^+, 143\,n^0, \\
    &\ce{^{238}_{92}{U}} \to \ce{^{234}_{90}{Th}} + \ce{^4_2{He}}: \qquad \text{ядро тория $\ce{^{234}_{90}{Th}}$}: 90\,p^+, 144\,n^0, \\
    &\ce{^{210}_{82}{Pb}} \to \ce{^{210}_{83}{Bi}} + e^- + \tilde\nu_e: \qquad \text{ядро висмута $\ce{^{210}_{83}{Bi}}$}: 83\,p^+, 127\,n^0, \\
    &\ce{^{239}_{94}{Pu}} \to \ce{^{235}_{92}{U}} + \ce{^4_2{He}}: \qquad \text{ядро урана $\ce{^{235}_{92}{U}}$}: 92\,p^+, 143\,n^0.
    \end{align*}
}
\solutionspace{80pt}

\tasknumber{14}%
\task{%
    Какая доля (от начального количества) радиоактивных ядер распадётся через время,
    равное двум периодам полураспада? Ответ выразить в процентах.
}
\answer{%
    \begin{align*}
    N &= N_0 \cdot 2^{- \frac t{T_{1/2}}} \implies
        \frac N{N_0} = 2^{- \frac t{T_{1/2}}}
        = 2^{-2} \approx 0{,}25 \approx 25\% \\
    N_\text{расп.} &= N_0 - N = N_0 - N_0 \cdot 2^{-\frac t{T_{1/2}}}
        = N_0\cbr{1 - 2^{-\frac t{T_{1/2}}}} \implies
        \frac{N_\text{расп.}}{N_0} = 1 - 2^{-\frac t{T_{1/2}}}
        = 1 - 2^{-2} \approx 0{,}75 \approx 75\%
    \end{align*}
}
\solutionspace{90pt}

\tasknumber{15}%
\task{%
    Сколько процентов ядер радиоактивного железа $\ce{^{59}Fe}$
    останется через $136{,}8\,\text{суток}$, если период его полураспада составляет $45{,}6\,\text{суток}$?
}
\answer{%
    \begin{align*}
    N &= N_0 \cdot 2^{-\frac t{T_{1/2}}}
        = 2^{-\frac{136{,}8\,\text{суток}}{45{,}6\,\text{суток}}}
        \approx 0{,}1250 = 12{,}50\%
    \end{align*}
}
\solutionspace{90pt}

\tasknumber{16}%
\task{%
    За $5\,\text{суток}$ от начального количества ядер радиоизотопа осталась половина.
    Каков период полураспада этого изотопа (ответ приведите в сутках)?
    Какая ещё доля (также от начального количества) распадётся, если подождать ещё столько же?
}
\answer{%
    \begin{align*}
            N &= N_0 \cdot 2^{-\frac t{T_{1/2}}}
            \implies \frac N{N_0} = 2^{-\frac t{T_{1/2}}}
            \implies \frac 1{2} = 2^{-\frac {5\,\text{суток}}{T_{1/2}}}
            \implies 1 = \frac {5\,\text{суток}}{T_{1/2}}
            \implies T_{1/2} = \frac {5\,\text{суток}}1 \approx 5\,\text{суток}.
         \\
            \delta &= \frac{N(t)}{N_0} - \frac{N(2t)}{N_0}
            = 2^{-\frac t{T_{1/2}}} - 2^{-\frac {2t}{T_{1/2}}}
            = 2^{-\frac t{T_{1/2}}}\cbr{1 - 2^{-\frac t{T_{1/2}}}}
            = \frac 1{2} \cdot \cbr{1-\frac 1{2}} \approx 0{,}250
    \end{align*}
}

\variantsplitter

\addpersonalvariant{Марьям Салимова}

\tasknumber{1}%
\task{%
    Определите число нейтронов в атоме $\ce{^{2}_{1}{D}}$.
}
\answer{%
    $Z = 1$ протонов и столько же электронов $A = 2$ нуклонов, $A - Z = 1$ нейтронов.
    Ответ: 1
}

\tasknumber{2}%
\task{%
    Определите число электронов в атоме $\ce{^{4}_{2}{He}}$.
}
\answer{%
    $Z = 2$ протонов и столько же электронов $A = 4$ нуклонов, $A - Z = 2$ нейтронов.
    Ответ: 2
}

\tasknumber{3}%
\task{%
    Определите число нуклонов в атоме $\ce{^{7}_{4}{Be}}$.
}
\answer{%
    $Z = 4$ протонов и столько же электронов $A = 7$ нуклонов, $A - Z = 3$ нейтронов.
    Ответ: 7
}

\tasknumber{4}%
\task{%
    Определите число нуклонов в атоме $\ce{^{15}_{7}{N}}$.
}
\answer{%
    $Z = 7$ протонов и столько же электронов $A = 15$ нуклонов, $A - Z = 8$ нейтронов.
    Ответ: 15
}

\tasknumber{5}%
\task{%
    Определите число нейтронов в атоме $\ce{^{19}_{9}{F}}$.
}
\answer{%
    $Z = 9$ протонов и столько же электронов $A = 19$ нуклонов, $A - Z = 10$ нейтронов.
    Ответ: 10
}

\tasknumber{6}%
\task{%
    Определите число электронов в атоме $\ce{^{26}_{12}{Mg}}$.
}
\answer{%
    $Z = 12$ протонов и столько же электронов $A = 26$ нуклонов, $A - Z = 14$ нейтронов.
    Ответ: 12
}

\tasknumber{7}%
\task{%
    Определите число электронов в атоме $\text{сера-33}$.
}
\answer{%
    $Z = 16$ протонов и столько же электронов, $A = 33$ нуклонов, $A - Z = 17$ нейтронов.
    Ответ: 16
}

\tasknumber{8}%
\task{%
    Определите число нейтронов в атоме $\text{аргон-37}$.
}
\answer{%
    $Z = 18$ протонов и столько же электронов, $A = 37$ нуклонов, $A - Z = 19$ нейтронов.
    Ответ: 19
}

\tasknumber{9}%
\task{%
    Определите число нейтронов в атоме $\text{калий-41}$.
}
\answer{%
    $Z = 19$ протонов и столько же электронов, $A = 41$ нуклонов, $A - Z = 22$ нейтронов.
    Ответ: 22
}

\tasknumber{10}%
\task{%
    Определите число протонов в атоме $\text{титан-50}$.
}
\answer{%
    $Z = 22$ протонов и столько же электронов, $A = 50$ нуклонов, $A - Z = 28$ нейтронов.
    Ответ: 22
}

\tasknumber{11}%
\task{%
    Определите число протонов в атоме $\text{марганец-53}$.
}
\answer{%
    $Z = 25$ протонов и столько же электронов, $A = 53$ нуклонов, $A - Z = 28$ нейтронов.
    Ответ: 25
}

\tasknumber{12}%
\task{%
    Определите число нейтронов в атоме $\text{железо-56}$.
}
\answer{%
    $Z = 26$ протонов и столько же электронов, $A = 56$ нуклонов, $A - Z = 30$ нейтронов.
    Ответ: 30
}

\tasknumber{13}%
\task{%
    В какое ядро превращается исходное в результате ядерного распада?
    Запишите уравнение реакции и явно укажите число протонов и нейтронов в получившемся ядре.
    \begin{itemize}
        \item ядро тория $\ce{^{234}_{90}{Th}}$, $\beta^-$-распад,
        \item ядро радия $\ce{^{226}_{88}{Ra}}$, $\alpha$-распад,
        \item ядро свинца $\ce{^{210}_{82}{Pb}}$, $\beta$-распад,
        \item ядро плутония $\ce{^{239}_{94}{Pu}}$, $\alpha$-распад.
    \end{itemize}
}
\answer{%
    \begin{align*}
    &\ce{^{234}_{90}{Th}} \to \ce{^{234}_{91}{Pa}} + e^- + \tilde\nu_e: \qquad \text{ядро протактиния $\ce{^{234}_{91}{Pa}}$}: 91\,p^+, 143\,n^0, \\
    &\ce{^{226}_{88}{Ra}} \to \ce{^{222}_{86}{Rn}} + \ce{^4_2{He}}: \qquad \text{ядро радона $\ce{^{222}_{86}{Rn}}$}: 86\,p^+, 136\,n^0, \\
    &\ce{^{210}_{82}{Pb}} \to \ce{^{210}_{83}{Bi}} + e^- + \tilde\nu_e: \qquad \text{ядро висмута $\ce{^{210}_{83}{Bi}}$}: 83\,p^+, 127\,n^0, \\
    &\ce{^{239}_{94}{Pu}} \to \ce{^{235}_{92}{U}} + \ce{^4_2{He}}: \qquad \text{ядро урана $\ce{^{235}_{92}{U}}$}: 92\,p^+, 143\,n^0.
    \end{align*}
}
\solutionspace{80pt}

\tasknumber{14}%
\task{%
    Какая доля (от начального количества) радиоактивных ядер останется через время,
    равное двум периодам полураспада? Ответ выразить в процентах.
}
\answer{%
    \begin{align*}
    N &= N_0 \cdot 2^{- \frac t{T_{1/2}}} \implies
        \frac N{N_0} = 2^{- \frac t{T_{1/2}}}
        = 2^{-2} \approx 0{,}25 \approx 25\% \\
    N_\text{расп.} &= N_0 - N = N_0 - N_0 \cdot 2^{-\frac t{T_{1/2}}}
        = N_0\cbr{1 - 2^{-\frac t{T_{1/2}}}} \implies
        \frac{N_\text{расп.}}{N_0} = 1 - 2^{-\frac t{T_{1/2}}}
        = 1 - 2^{-2} \approx 0{,}75 \approx 75\%
    \end{align*}
}
\solutionspace{90pt}

\tasknumber{15}%
\task{%
    Сколько процентов ядер радиоактивного железа $\ce{^{59}Fe}$
    останется через $182{,}4\,\text{суток}$, если период его полураспада составляет $45{,}6\,\text{суток}$?
}
\answer{%
    \begin{align*}
    N &= N_0 \cdot 2^{-\frac t{T_{1/2}}}
        = 2^{-\frac{182{,}4\,\text{суток}}{45{,}6\,\text{суток}}}
        \approx 0{,}0625 = 6{,}25\%
    \end{align*}
}
\solutionspace{90pt}

\tasknumber{16}%
\task{%
    За $2\,\text{суток}$ от начального количества ядер радиоизотопа осталась половина.
    Каков период полураспада этого изотопа (ответ приведите в сутках)?
    Какая ещё доля (также от начального количества) распадётся, если подождать ещё столько же?
}
\answer{%
    \begin{align*}
            N &= N_0 \cdot 2^{-\frac t{T_{1/2}}}
            \implies \frac N{N_0} = 2^{-\frac t{T_{1/2}}}
            \implies \frac 1{2} = 2^{-\frac {2\,\text{суток}}{T_{1/2}}}
            \implies 1 = \frac {2\,\text{суток}}{T_{1/2}}
            \implies T_{1/2} = \frac {2\,\text{суток}}1 \approx 2\,\text{суток}.
         \\
            \delta &= \frac{N(t)}{N_0} - \frac{N(2t)}{N_0}
            = 2^{-\frac t{T_{1/2}}} - 2^{-\frac {2t}{T_{1/2}}}
            = 2^{-\frac t{T_{1/2}}}\cbr{1 - 2^{-\frac t{T_{1/2}}}}
            = \frac 1{2} \cdot \cbr{1-\frac 1{2}} \approx 0{,}250
    \end{align*}
}

\variantsplitter

\addpersonalvariant{Юлия Шевченко}

\tasknumber{1}%
\task{%
    Определите число нейтронов в атоме $\ce{^{1}_{1}{H}}$.
}
\answer{%
    $Z = 1$ протонов и столько же электронов $A = 1$ нуклонов, $A - Z = 0$ нейтронов.
    Ответ: 0
}

\tasknumber{2}%
\task{%
    Определите число протонов в атоме $\ce{^{4}_{2}{He}}$.
}
\answer{%
    $Z = 2$ протонов и столько же электронов $A = 4$ нуклонов, $A - Z = 2$ нейтронов.
    Ответ: 2
}

\tasknumber{3}%
\task{%
    Определите число электронов в атоме $\ce{^{10}_{5}{B}}$.
}
\answer{%
    $Z = 5$ протонов и столько же электронов $A = 10$ нуклонов, $A - Z = 5$ нейтронов.
    Ответ: 5
}

\tasknumber{4}%
\task{%
    Определите число электронов в атоме $\ce{^{14}_{6}{C}}$.
}
\answer{%
    $Z = 6$ протонов и столько же электронов $A = 14$ нуклонов, $A - Z = 8$ нейтронов.
    Ответ: 6
}

\tasknumber{5}%
\task{%
    Определите число нуклонов в атоме $\ce{^{17}_{8}{O}}$.
}
\answer{%
    $Z = 8$ протонов и столько же электронов $A = 17$ нуклонов, $A - Z = 9$ нейтронов.
    Ответ: 17
}

\tasknumber{6}%
\task{%
    Определите число нуклонов в атоме $\ce{^{29}_{14}{Si}}$.
}
\answer{%
    $Z = 14$ протонов и столько же электронов $A = 29$ нуклонов, $A - Z = 15$ нейтронов.
    Ответ: 29
}

\tasknumber{7}%
\task{%
    Определите число нейтронов в атоме $\text{сера-34}$.
}
\answer{%
    $Z = 16$ протонов и столько же электронов, $A = 34$ нуклонов, $A - Z = 18$ нейтронов.
    Ответ: 18
}

\tasknumber{8}%
\task{%
    Определите число нейтронов в атоме $\text{хлор-35}$.
}
\answer{%
    $Z = 17$ протонов и столько же электронов, $A = 35$ нуклонов, $A - Z = 18$ нейтронов.
    Ответ: 18
}

\tasknumber{9}%
\task{%
    Определите число протонов в атоме $\text{кальций-46}$.
}
\answer{%
    $Z = 20$ протонов и столько же электронов, $A = 46$ нуклонов, $A - Z = 26$ нейтронов.
    Ответ: 20
}

\tasknumber{10}%
\task{%
    Определите число нуклонов в атоме $\text{титан-48}$.
}
\answer{%
    $Z = 22$ протонов и столько же электронов, $A = 48$ нуклонов, $A - Z = 26$ нейтронов.
    Ответ: 48
}

\tasknumber{11}%
\task{%
    Определите число протонов в атоме $\text{железо-54}$.
}
\answer{%
    $Z = 26$ протонов и столько же электронов, $A = 54$ нуклонов, $A - Z = 28$ нейтронов.
    Ответ: 26
}

\tasknumber{12}%
\task{%
    Определите число протонов в атоме $\text{никель-57}$.
}
\answer{%
    $Z = 28$ протонов и столько же электронов, $A = 57$ нуклонов, $A - Z = 29$ нейтронов.
    Ответ: 28
}

\tasknumber{13}%
\task{%
    В какое ядро превращается исходное в результате ядерного распада?
    Запишите уравнение реакции и явно укажите число протонов и нейтронов в получившемся ядре.
    \begin{itemize}
        \item ядро тория $\ce{^{234}_{90}{Th}}$, $\beta$-распад,
        \item ядро урана $\ce{^{234}_{92}{U}}$, $\alpha$-распад,
        \item ядро полония $\ce{^{214}_{84}{Po}}$, $\alpha$-распад,
        \item ядро свинца $\ce{^{209}_{82}{Pb}}$, $\beta$-распад.
    \end{itemize}
}
\answer{%
    \begin{align*}
    &\ce{^{234}_{90}{Th}} \to \ce{^{234}_{91}{Pa}} + e^- + \tilde\nu_e: \qquad \text{ядро протактиния $\ce{^{234}_{91}{Pa}}$}: 91\,p^+, 143\,n^0, \\
    &\ce{^{234}_{92}{U}} \to \ce{^{230}_{90}{Th}} + \ce{^4_2{He}}: \qquad \text{ядро тория $\ce{^{230}_{90}{Th}}$}: 90\,p^+, 140\,n^0, \\
    &\ce{^{214}_{84}{Po}} \to \ce{^{210}_{82}{Pb}} + \ce{^4_2{He}}: \qquad \text{ядро свинца $\ce{^{210}_{82}{Pb}}$}: 82\,p^+, 128\,n^0, \\
    &\ce{^{209}_{82}{Pb}} \to \ce{^{209}_{83}{Bi}} + e^- + \tilde\nu_e: \qquad \text{ядро висмута $\ce{^{209}_{83}{Bi}}$}: 83\,p^+, 126\,n^0.
    \end{align*}
}
\solutionspace{80pt}

\tasknumber{14}%
\task{%
    Какая доля (от начального количества) радиоактивных ядер останется через время,
    равное четырём периодам полураспада? Ответ выразить в процентах.
}
\answer{%
    \begin{align*}
    N &= N_0 \cdot 2^{- \frac t{T_{1/2}}} \implies
        \frac N{N_0} = 2^{- \frac t{T_{1/2}}}
        = 2^{-4} \approx 0{,}06 \approx 6\% \\
    N_\text{расп.} &= N_0 - N = N_0 - N_0 \cdot 2^{-\frac t{T_{1/2}}}
        = N_0\cbr{1 - 2^{-\frac t{T_{1/2}}}} \implies
        \frac{N_\text{расп.}}{N_0} = 1 - 2^{-\frac t{T_{1/2}}}
        = 1 - 2^{-4} \approx 0{,}94 \approx 94\%
    \end{align*}
}
\solutionspace{90pt}

\tasknumber{15}%
\task{%
    Сколько процентов ядер радиоактивного железа $\ce{^{59}Fe}$
    останется через $91{,}2\,\text{суток}$, если период его полураспада составляет $45{,}6\,\text{суток}$?
}
\answer{%
    \begin{align*}
    N &= N_0 \cdot 2^{-\frac t{T_{1/2}}}
        = 2^{-\frac{91{,}2\,\text{суток}}{45{,}6\,\text{суток}}}
        \approx 0{,}2500 = 25{,}00\%
    \end{align*}
}
\solutionspace{90pt}

\tasknumber{16}%
\task{%
    За $2\,\text{суток}$ от начального количества ядер радиоизотопа осталась половина.
    Каков период полураспада этого изотопа (ответ приведите в сутках)?
    Какая ещё доля (также от начального количества) распадётся, если подождать ещё столько же?
}
\answer{%
    \begin{align*}
            N &= N_0 \cdot 2^{-\frac t{T_{1/2}}}
            \implies \frac N{N_0} = 2^{-\frac t{T_{1/2}}}
            \implies \frac 1{2} = 2^{-\frac {2\,\text{суток}}{T_{1/2}}}
            \implies 1 = \frac {2\,\text{суток}}{T_{1/2}}
            \implies T_{1/2} = \frac {2\,\text{суток}}1 \approx 2\,\text{суток}.
         \\
            \delta &= \frac{N(t)}{N_0} - \frac{N(2t)}{N_0}
            = 2^{-\frac t{T_{1/2}}} - 2^{-\frac {2t}{T_{1/2}}}
            = 2^{-\frac t{T_{1/2}}}\cbr{1 - 2^{-\frac t{T_{1/2}}}}
            = \frac 1{2} \cdot \cbr{1-\frac 1{2}} \approx 0{,}250
    \end{align*}
}
% autogenerated
