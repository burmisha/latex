\setdate{3~марта~2022}
\setclass{11«Б»}

\addpersonalvariant{Михаил Бурмистров}

\tasknumber{1}%
\task{%
    Переведите в джоули:
    \begin{itemize}
        \item $1\,\text{эВ}$,
        \item $3{,}2\,\text{эВ}$,
        \item $13{,}5\,\text{МэВ}$,
        \item $7{,}3\,\text{кэВ}$,
        \item $1{,}6 \cdot 10^{-3}\,\text{эВ}$,
        \item $3{,}3 \cdot 10^{7}\,\text{эВ}$,
    \end{itemize}
}
\answer{%
    \begin{align*}
    1\,\text{эВ} &\approx 0{,}160 \cdot 10^{-18}\,\text{Дж} \\
    3{,}2\,\text{эВ} &\approx 0{,}512 \cdot 10^{-18}\,\text{Дж} \\
    13{,}5\,\text{МэВ} &\approx 2{,}16 \cdot 10^{-12}\,\text{Дж} \\
    7{,}3\,\text{кэВ} &\approx 1{,}168 \cdot 10^{-15}\,\text{Дж} \\
    1{,}6 \cdot 10^{-3}\,\text{эВ} &\approx 0{,}26 \cdot 10^{-21}\,\text{Дж} \\
    3{,}3 \cdot 10^{7}\,\text{эВ} &\approx 5{,}28 \cdot 10^{-12}\,\text{Дж}
    \end{align*}
}
\solutionspace{40pt}

\tasknumber{2}%
\task{%
    Переведите в электронвольты:
    \begin{itemize}
        \item $1\,\text{Дж}$,
        \item $3{,}2\,\text{Дж}$,
        \item $13{,}5\,\text{мкДж}$,
        \item $7{,}3\,\text{кДж}$,
        \item $1{,}6 \cdot 10^{-17}\,\text{Дж}$,
        \item $3{,}3 \cdot 10^{-21}\,\text{Дж}$,
    \end{itemize}
}
\answer{%
    \begin{align*}
    1\,\text{Дж} &\approx 6{,}3 \cdot 10^{18}\,\text{эВ} \\
    3{,}2\,\text{Дж} &\approx 20 \cdot 10^{18}\,\text{эВ} \\
    13{,}5\,\text{мкДж} &\approx 84{,}4 \cdot 10^{12}\,\text{эВ} \\
    7{,}3\,\text{кДж} &\approx 45{,}6 \cdot 10^{21}\,\text{эВ} \\
    1{,}6 \cdot 10^{-17}\,\text{Дж} &\approx 100\,\text{эВ} \\
    3{,}3 \cdot 10^{-21}\,\text{Дж} &\approx 0{,}0206\,\text{эВ}
    \end{align*}
}
\solutionspace{40pt}

\tasknumber{3}%
\task{%
    Свет с энергией кванта $4{,}1\,\text{эВ}$ вырывает из металлической пластинки электроны,
    имеющие максимальную кинетическую энергию $1{,}5\,\text{эВ}$.
    Найдите работу выхода (в эВ) электрона из этого металла.
}
\answer{%
    $A = E - K \approx 2{,}6\,\text{эВ}$
}
\solutionspace{80pt}

\tasknumber{4}%
\task{%
    Какой максимальной кинетической энергией (в эВ) обладают электроны,
    вырванные из металла при действии на него ультрафиолетового излучения с длиной волны $0{,}40\,\text{мкм}$,
    если работа выхода электрона $3{,}4 \cdot 10^{-19}\,\text{Дж}$? Постоянная Планка $h = 6{,}626 \cdot 10^{-34}\,\text{Дж}\cdot\text{с}$, заряд электрона $e = 1{,}6 \cdot 10^{-19}\,\text{Кл}$.
}
\answer{%
    $K = \frac{hc}{\lambda} - A \approx 0{,}98\,\text{эВ}$
}
\solutionspace{80pt}

\tasknumber{5}%
\task{%
    Чему равно задерживающее напряжение для фотоэлектронов, вырываемых с поверхности металла светом
    с энергией фотонов $6{,}3 \cdot 10^{-19}\,\text{Дж}$, если работа выхода из этого металла $5 \cdot 10^{-19}\,\text{Дж}$? Заряд электрона $e = 1{,}6 \cdot 10^{-19}\,\text{Кл}$.
}
\answer{%
    $eU = K = E - A \implies U = \frac{E - A}{ e } \approx 0{,}81\,\text{В}$
}
\solutionspace{80pt}

\tasknumber{6}%
\task{%
    Красная граница фотоэффекта для некоторого металла соответствует длине волны $5{,}7 \cdot 10^{-7}\,\text{м}$.
    Чему равно напряжение, полностью задерживающее фотоэлектроны, вырываемые из этого металла излучением
    с длиной волны $2{,}2 \cdot 10^{-5}\,\text{см}$? Постоянная Планка $h = 6{,}626 \cdot 10^{-34}\,\text{Дж}\cdot\text{с}$, заряд электрона $e = 1{,}6 \cdot 10^{-19}\,\text{Кл}$.
}
\answer{%
    $eU = K = E - A = \frac{hc}{\lambda} - A, \qquad 0 = \frac{hc}{\lambda_0} - A \implies U = \frac{ \frac{hc}{\lambda} - \frac{hc}{\lambda_0} }{ e } = \frac{hc}{ e }\cbr{ \frac 1{\lambda} - \frac 1{\lambda_0}}  \approx 3{,}5\,\text{В}$
}
\solutionspace{80pt}

\tasknumber{7}%
\task{%
    Определите длину волны (в нм) света, которым освещается поверхность металла,
    если фотоэлектроны имеют максимальную кинетическую энергию $4 \cdot 10^{-20}\,\text{Дж}$,
    а работа выхода электронов из этого металла $7 \cdot 10^{-19}\,\text{Дж}$.
    Постоянная Планка $h = 6{,}626 \cdot 10^{-34}\,\text{Дж}\cdot\text{с}$.
}
\answer{%
    $h \frac c\lambda = A_{\text{вых.}} + E_{\text{кин.}} \implies \lambda = \frac{h c}{A_{\text{вых.}} + E_{\text{кин.}}} = \frac{ 6{,}626 \cdot 10^{-34}\,\text{Дж}\cdot\text{с} \cdot {Const.c:V} }{7 \cdot 10^{-19}\,\text{Дж} + 4 \cdot 10^{-20}\,\text{Дж}} \approx 0{,}27 \cdot 10^{-6}\,\text{м}.$
}
\solutionspace{80pt}

\tasknumber{8}%
\task{%
    Работа выхода электронов из некоторого металла $4{,}3\,\text{эВ}$.
    Найдите скорость электронов (в км/с),
    вылетающих с поверхности металла при освещении его светом с длиной волны $2{,}7 \cdot 10^{-5}\,\text{см}$.
    Масса электрона $m_{e} = 9{,}1 \cdot 10^{-31}\,\text{кг}$.
    Постоянная Планка $h = 6{,}626 \cdot 10^{-34}\,\text{Дж}\cdot\text{с}$, заряд электрона $e = 1{,}6 \cdot 10^{-19}\,\text{Кл}$.
}
\answer{%
    $h \frac c\lambda = A_{\text{вых.}} + \frac{ m_{e}v^2 }2 \implies v = \sqrt{ \frac 2{m_{e}}\cbr{ h \frac c\lambda - A_{\text{вых.}} } } \approx 325{,}6\,\frac{\text{км}}{\text{c}}.$
}
\solutionspace{80pt}

\tasknumber{9}%
\task{%
    Работа выхода электронов из некоторого металла $4{,}30\,\text{эВ}$.
    На металл падают фотоны с импульсом $3 \cdot 10^{-27}\,\frac{\text{кг}\cdot\text{м}}{\text{с}}$.
    Во сколько раз максимальный импульс электронов, вылетающих с поверхности металла при фотоэффекте,
    больше импульса падающих фотонов? Масса электрона $m_{e} = 9{,}1 \cdot 10^{-31}\,\text{кг}$.
}
\answer{%
    $h \frac c\lambda = A_{\text{вых.}} + \frac{p_e^2}{2m}, p=\frac h\lambda\implies p_e = \sqrt{2m\cbr{pc - A_{\text{вых.}}}} \implies \frac{p_e}p = \sqrt{\frac{2m}p \cbr{c - \frac{A_{\text{вых.}}}p } } \approx 207{,}05$
}

\variantsplitter

\addpersonalvariant{Снежана Авдошина}

\tasknumber{1}%
\task{%
    Переведите в джоули:
    \begin{itemize}
        \item $1\,\text{эВ}$,
        \item $4{,}3\,\text{эВ}$,
        \item $1{,}4\,\text{МэВ}$,
        \item $6{,}2\,\text{кэВ}$,
        \item $8{,}3 \cdot 10^{-3}\,\text{эВ}$,
        \item $1{,}7 \cdot 10^{7}\,\text{эВ}$,
    \end{itemize}
}
\answer{%
    \begin{align*}
    1\,\text{эВ} &\approx 0{,}160 \cdot 10^{-18}\,\text{Дж} \\
    4{,}3\,\text{эВ} &\approx 0{,}688 \cdot 10^{-18}\,\text{Дж} \\
    1{,}4\,\text{МэВ} &\approx 0{,}22 \cdot 10^{-12}\,\text{Дж} \\
    6{,}2\,\text{кэВ} &\approx 0{,}992 \cdot 10^{-15}\,\text{Дж} \\
    8{,}3 \cdot 10^{-3}\,\text{эВ} &\approx 1{,}328 \cdot 10^{-21}\,\text{Дж} \\
    1{,}7 \cdot 10^{7}\,\text{эВ} &\approx 2{,}7 \cdot 10^{-12}\,\text{Дж}
    \end{align*}
}
\solutionspace{40pt}

\tasknumber{2}%
\task{%
    Переведите в электронвольты:
    \begin{itemize}
        \item $1\,\text{Дж}$,
        \item $4{,}3\,\text{Дж}$,
        \item $1{,}4\,\text{мкДж}$,
        \item $6{,}2\,\text{кДж}$,
        \item $8{,}3 \cdot 10^{-17}\,\text{Дж}$,
        \item $1{,}7 \cdot 10^{-21}\,\text{Дж}$,
    \end{itemize}
}
\answer{%
    \begin{align*}
    1\,\text{Дж} &\approx 6{,}3 \cdot 10^{18}\,\text{эВ} \\
    4{,}3\,\text{Дж} &\approx 26{,}9 \cdot 10^{18}\,\text{эВ} \\
    1{,}4\,\text{мкДж} &\approx 8{,}8 \cdot 10^{12}\,\text{эВ} \\
    6{,}2\,\text{кДж} &\approx 38{,}8 \cdot 10^{21}\,\text{эВ} \\
    8{,}3 \cdot 10^{-17}\,\text{Дж} &\approx 519\,\text{эВ} \\
    1{,}7 \cdot 10^{-21}\,\text{Дж} &\approx 0{,}0106\,\text{эВ}
    \end{align*}
}
\solutionspace{40pt}

\tasknumber{3}%
\task{%
    Свет с энергией кванта $4{,}1\,\text{эВ}$ вырывает из металлической пластинки электроны,
    имеющие максимальную кинетическую энергию $1{,}7\,\text{эВ}$.
    Найдите работу выхода (в эВ) электрона из этого металла.
}
\answer{%
    $A = E - K \approx 2{,}4\,\text{эВ}$
}
\solutionspace{80pt}

\tasknumber{4}%
\task{%
    Какой максимальной кинетической энергией (в эВ) обладают электроны,
    вырванные из металла при действии на него ультрафиолетового излучения с длиной волны $0{,}33\,\text{мкм}$,
    если работа выхода электрона $3{,}2 \cdot 10^{-19}\,\text{Дж}$? Постоянная Планка $h = 6{,}626 \cdot 10^{-34}\,\text{Дж}\cdot\text{с}$, заряд электрона $e = 1{,}6 \cdot 10^{-19}\,\text{Кл}$.
}
\answer{%
    $K = \frac{hc}{\lambda} - A \approx 1{,}76\,\text{эВ}$
}
\solutionspace{80pt}

\tasknumber{5}%
\task{%
    Чему равно задерживающее напряжение для фотоэлектронов, вырываемых с поверхности металла светом
    с энергией фотонов $8{,}5 \cdot 10^{-19}\,\text{Дж}$, если работа выхода из этого металла $3 \cdot 10^{-19}\,\text{Дж}$? Заряд электрона $e = 1{,}6 \cdot 10^{-19}\,\text{Кл}$.
}
\answer{%
    $eU = K = E - A \implies U = \frac{E - A}{ e } \approx 3{,}4\,\text{В}$
}
\solutionspace{80pt}

\tasknumber{6}%
\task{%
    Красная граница фотоэффекта для некоторого металла соответствует длине волны $5{,}3 \cdot 10^{-7}\,\text{м}$.
    Чему равно напряжение, полностью задерживающее фотоэлектроны, вырываемые из этого металла излучением
    с длиной волны $2{,}6 \cdot 10^{-5}\,\text{см}$? Постоянная Планка $h = 6{,}626 \cdot 10^{-34}\,\text{Дж}\cdot\text{с}$, заряд электрона $e = 1{,}6 \cdot 10^{-19}\,\text{Кл}$.
}
\answer{%
    $eU = K = E - A = \frac{hc}{\lambda} - A, \qquad 0 = \frac{hc}{\lambda_0} - A \implies U = \frac{ \frac{hc}{\lambda} - \frac{hc}{\lambda_0} }{ e } = \frac{hc}{ e }\cbr{ \frac 1{\lambda} - \frac 1{\lambda_0}}  \approx 2{,}4\,\text{В}$
}
\solutionspace{80pt}

\tasknumber{7}%
\task{%
    Определите длину волны (в нм) света, которым освещается поверхность металла,
    если фотоэлектроны имеют максимальную кинетическую энергию $4 \cdot 10^{-20}\,\text{Дж}$,
    а работа выхода электронов из этого металла $13 \cdot 10^{-19}\,\text{Дж}$.
    Постоянная Планка $h = 6{,}626 \cdot 10^{-34}\,\text{Дж}\cdot\text{с}$.
}
\answer{%
    $h \frac c\lambda = A_{\text{вых.}} + E_{\text{кин.}} \implies \lambda = \frac{h c}{A_{\text{вых.}} + E_{\text{кин.}}} = \frac{ 6{,}626 \cdot 10^{-34}\,\text{Дж}\cdot\text{с} \cdot {Const.c:V} }{13 \cdot 10^{-19}\,\text{Дж} + 4 \cdot 10^{-20}\,\text{Дж}} \approx 0{,}1483 \cdot 10^{-6}\,\text{м}.$
}
\solutionspace{80pt}

\tasknumber{8}%
\task{%
    Работа выхода электронов из некоторого металла $3{,}4\,\text{эВ}$.
    Найдите скорость электронов (в км/с),
    вылетающих с поверхности металла при освещении его светом с длиной волны $2{,}2 \cdot 10^{-5}\,\text{см}$.
    Масса электрона $m_{e} = 9{,}1 \cdot 10^{-31}\,\text{кг}$.
    Постоянная Планка $h = 6{,}626 \cdot 10^{-34}\,\text{Дж}\cdot\text{с}$, заряд электрона $e = 1{,}6 \cdot 10^{-19}\,\text{Кл}$.
}
\answer{%
    $h \frac c\lambda = A_{\text{вых.}} + \frac{ m_{e}v^2 }2 \implies v = \sqrt{ \frac 2{m_{e}}\cbr{ h \frac c\lambda - A_{\text{вых.}} } } \approx 888{,}9\,\frac{\text{км}}{\text{c}}.$
}
\solutionspace{80pt}

\tasknumber{9}%
\task{%
    Работа выхода электронов из некоторого металла $4{,}30\,\text{эВ}$.
    На металл падают фотоны с импульсом $3 \cdot 10^{-27}\,\frac{\text{кг}\cdot\text{м}}{\text{с}}$.
    Во сколько раз максимальный импульс электронов, вылетающих с поверхности металла при фотоэффекте,
    больше импульса падающих фотонов? Масса электрона $m_{e} = 9{,}1 \cdot 10^{-31}\,\text{кг}$.
}
\answer{%
    $h \frac c\lambda = A_{\text{вых.}} + \frac{p_e^2}{2m}, p=\frac h\lambda\implies p_e = \sqrt{2m\cbr{pc - A_{\text{вых.}}}} \implies \frac{p_e}p = \sqrt{\frac{2m}p \cbr{c - \frac{A_{\text{вых.}}}p } } \approx 207{,}05$
}

\variantsplitter

\addpersonalvariant{Марьяна Аристова}

\tasknumber{1}%
\task{%
    Переведите в джоули:
    \begin{itemize}
        \item $1\,\text{эВ}$,
        \item $3{,}2\,\text{эВ}$,
        \item $13{,}5\,\text{МэВ}$,
        \item $7{,}3\,\text{кэВ}$,
        \item $1{,}6 \cdot 10^{-3}\,\text{эВ}$,
        \item $3{,}3 \cdot 10^{7}\,\text{эВ}$,
    \end{itemize}
}
\answer{%
    \begin{align*}
    1\,\text{эВ} &\approx 0{,}160 \cdot 10^{-18}\,\text{Дж} \\
    3{,}2\,\text{эВ} &\approx 0{,}512 \cdot 10^{-18}\,\text{Дж} \\
    13{,}5\,\text{МэВ} &\approx 2{,}16 \cdot 10^{-12}\,\text{Дж} \\
    7{,}3\,\text{кэВ} &\approx 1{,}168 \cdot 10^{-15}\,\text{Дж} \\
    1{,}6 \cdot 10^{-3}\,\text{эВ} &\approx 0{,}26 \cdot 10^{-21}\,\text{Дж} \\
    3{,}3 \cdot 10^{7}\,\text{эВ} &\approx 5{,}28 \cdot 10^{-12}\,\text{Дж}
    \end{align*}
}
\solutionspace{40pt}

\tasknumber{2}%
\task{%
    Переведите в электронвольты:
    \begin{itemize}
        \item $1\,\text{Дж}$,
        \item $3{,}2\,\text{Дж}$,
        \item $13{,}5\,\text{мкДж}$,
        \item $7{,}3\,\text{кДж}$,
        \item $1{,}6 \cdot 10^{-17}\,\text{Дж}$,
        \item $3{,}3 \cdot 10^{-21}\,\text{Дж}$,
    \end{itemize}
}
\answer{%
    \begin{align*}
    1\,\text{Дж} &\approx 6{,}3 \cdot 10^{18}\,\text{эВ} \\
    3{,}2\,\text{Дж} &\approx 20 \cdot 10^{18}\,\text{эВ} \\
    13{,}5\,\text{мкДж} &\approx 84{,}4 \cdot 10^{12}\,\text{эВ} \\
    7{,}3\,\text{кДж} &\approx 45{,}6 \cdot 10^{21}\,\text{эВ} \\
    1{,}6 \cdot 10^{-17}\,\text{Дж} &\approx 100\,\text{эВ} \\
    3{,}3 \cdot 10^{-21}\,\text{Дж} &\approx 0{,}0206\,\text{эВ}
    \end{align*}
}
\solutionspace{40pt}

\tasknumber{3}%
\task{%
    Свет с энергией кванта $3{,}8\,\text{эВ}$ вырывает из металлической пластинки электроны,
    имеющие максимальную кинетическую энергию $1{,}8\,\text{эВ}$.
    Найдите работу выхода (в эВ) электрона из этого металла.
}
\answer{%
    $A = E - K \approx 2\,\text{эВ}$
}
\solutionspace{80pt}

\tasknumber{4}%
\task{%
    Какой максимальной кинетической энергией (в эВ) обладают электроны,
    вырванные из металла при действии на него ультрафиолетового излучения с длиной волны $0{,}33\,\text{мкм}$,
    если работа выхода электрона $2{,}8 \cdot 10^{-19}\,\text{Дж}$? Постоянная Планка $h = 6{,}626 \cdot 10^{-34}\,\text{Дж}\cdot\text{с}$, заряд электрона $e = 1{,}6 \cdot 10^{-19}\,\text{Кл}$.
}
\answer{%
    $K = \frac{hc}{\lambda} - A \approx 2{,}0\,\text{эВ}$
}
\solutionspace{80pt}

\tasknumber{5}%
\task{%
    Чему равно задерживающее напряжение для фотоэлектронов, вырываемых с поверхности металла светом
    с энергией фотонов $8{,}5 \cdot 10^{-19}\,\text{Дж}$, если работа выхода из этого металла $3 \cdot 10^{-19}\,\text{Дж}$? Заряд электрона $e = 1{,}6 \cdot 10^{-19}\,\text{Кл}$.
}
\answer{%
    $eU = K = E - A \implies U = \frac{E - A}{ e } \approx 3{,}4\,\text{В}$
}
\solutionspace{80pt}

\tasknumber{6}%
\task{%
    Красная граница фотоэффекта для некоторого металла соответствует длине волны $6{,}6 \cdot 10^{-7}\,\text{м}$.
    Чему равно напряжение, полностью задерживающее фотоэлектроны, вырываемые из этого металла излучением
    с длиной волны $3{,}2 \cdot 10^{-5}\,\text{см}$? Постоянная Планка $h = 6{,}626 \cdot 10^{-34}\,\text{Дж}\cdot\text{с}$, заряд электрона $e = 1{,}6 \cdot 10^{-19}\,\text{Кл}$.
}
\answer{%
    $eU = K = E - A = \frac{hc}{\lambda} - A, \qquad 0 = \frac{hc}{\lambda_0} - A \implies U = \frac{ \frac{hc}{\lambda} - \frac{hc}{\lambda_0} }{ e } = \frac{hc}{ e }\cbr{ \frac 1{\lambda} - \frac 1{\lambda_0}}  \approx 2{,}0\,\text{В}$
}
\solutionspace{80pt}

\tasknumber{7}%
\task{%
    Определите длину волны (в нм) света, которым освещается поверхность металла,
    если фотоэлектроны имеют максимальную кинетическую энергию $3 \cdot 10^{-20}\,\text{Дж}$,
    а работа выхода электронов из этого металла $11 \cdot 10^{-19}\,\text{Дж}$.
    Постоянная Планка $h = 6{,}626 \cdot 10^{-34}\,\text{Дж}\cdot\text{с}$.
}
\answer{%
    $h \frac c\lambda = A_{\text{вых.}} + E_{\text{кин.}} \implies \lambda = \frac{h c}{A_{\text{вых.}} + E_{\text{кин.}}} = \frac{ 6{,}626 \cdot 10^{-34}\,\text{Дж}\cdot\text{с} \cdot {Const.c:V} }{11 \cdot 10^{-19}\,\text{Дж} + 3 \cdot 10^{-20}\,\text{Дж}} \approx 0{,}1759 \cdot 10^{-6}\,\text{м}.$
}
\solutionspace{80pt}

\tasknumber{8}%
\task{%
    Работа выхода электронов из некоторого металла $2{,}1\,\text{эВ}$.
    Найдите скорость электронов (в км/с),
    вылетающих с поверхности металла при освещении его светом с длиной волны $2{,}2 \cdot 10^{-5}\,\text{см}$.
    Масса электрона $m_{e} = 9{,}1 \cdot 10^{-31}\,\text{кг}$.
    Постоянная Планка $h = 6{,}626 \cdot 10^{-34}\,\text{Дж}\cdot\text{с}$, заряд электрона $e = 1{,}6 \cdot 10^{-19}\,\text{Кл}$.
}
\answer{%
    $h \frac c\lambda = A_{\text{вых.}} + \frac{ m_{e}v^2 }2 \implies v = \sqrt{ \frac 2{m_{e}}\cbr{ h \frac c\lambda - A_{\text{вых.}} } } \approx 1116{,}8\,\frac{\text{км}}{\text{c}}.$
}
\solutionspace{80pt}

\tasknumber{9}%
\task{%
    Работа выхода электронов из некоторого металла $3{,}40\,\text{эВ}$.
    На металл падают фотоны с импульсом $2{,}40 \cdot 10^{-27}\,\frac{\text{кг}\cdot\text{м}}{\text{с}}$.
    Во сколько раз максимальный импульс электронов, вылетающих с поверхности металла при фотоэффекте,
    больше импульса падающих фотонов? Масса электрона $m_{e} = 9{,}1 \cdot 10^{-31}\,\text{кг}$.
}
\answer{%
    $h \frac c\lambda = A_{\text{вых.}} + \frac{p_e^2}{2m}, p=\frac h\lambda\implies p_e = \sqrt{2m\cbr{pc - A_{\text{вых.}}}} \implies \frac{p_e}p = \sqrt{\frac{2m}p \cbr{c - \frac{A_{\text{вых.}}}p } } \approx 235{,}82$
}

\variantsplitter

\addpersonalvariant{Никита Иванов}

\tasknumber{1}%
\task{%
    Переведите в джоули:
    \begin{itemize}
        \item $1\,\text{эВ}$,
        \item $4{,}3\,\text{эВ}$,
        \item $1{,}4\,\text{МэВ}$,
        \item $7{,}3\,\text{кэВ}$,
        \item $9{,}4 \cdot 10^{-3}\,\text{эВ}$,
        \item $3{,}3 \cdot 10^{7}\,\text{эВ}$,
    \end{itemize}
}
\answer{%
    \begin{align*}
    1\,\text{эВ} &\approx 0{,}160 \cdot 10^{-18}\,\text{Дж} \\
    4{,}3\,\text{эВ} &\approx 0{,}688 \cdot 10^{-18}\,\text{Дж} \\
    1{,}4\,\text{МэВ} &\approx 0{,}22 \cdot 10^{-12}\,\text{Дж} \\
    7{,}3\,\text{кэВ} &\approx 1{,}168 \cdot 10^{-15}\,\text{Дж} \\
    9{,}4 \cdot 10^{-3}\,\text{эВ} &\approx 1{,}504 \cdot 10^{-21}\,\text{Дж} \\
    3{,}3 \cdot 10^{7}\,\text{эВ} &\approx 5{,}28 \cdot 10^{-12}\,\text{Дж}
    \end{align*}
}
\solutionspace{40pt}

\tasknumber{2}%
\task{%
    Переведите в электронвольты:
    \begin{itemize}
        \item $1\,\text{Дж}$,
        \item $4{,}3\,\text{Дж}$,
        \item $1{,}4\,\text{мкДж}$,
        \item $7{,}3\,\text{кДж}$,
        \item $9{,}4 \cdot 10^{-17}\,\text{Дж}$,
        \item $3{,}3 \cdot 10^{-21}\,\text{Дж}$,
    \end{itemize}
}
\answer{%
    \begin{align*}
    1\,\text{Дж} &\approx 6{,}3 \cdot 10^{18}\,\text{эВ} \\
    4{,}3\,\text{Дж} &\approx 26{,}9 \cdot 10^{18}\,\text{эВ} \\
    1{,}4\,\text{мкДж} &\approx 8{,}8 \cdot 10^{12}\,\text{эВ} \\
    7{,}3\,\text{кДж} &\approx 45{,}6 \cdot 10^{21}\,\text{эВ} \\
    9{,}4 \cdot 10^{-17}\,\text{Дж} &\approx 588\,\text{эВ} \\
    3{,}3 \cdot 10^{-21}\,\text{Дж} &\approx 0{,}0206\,\text{эВ}
    \end{align*}
}
\solutionspace{40pt}

\tasknumber{3}%
\task{%
    Свет с энергией кванта $3{,}8\,\text{эВ}$ вырывает из металлической пластинки электроны,
    имеющие максимальную кинетическую энергию $1{,}8\,\text{эВ}$.
    Найдите работу выхода (в эВ) электрона из этого металла.
}
\answer{%
    $A = E - K \approx 2\,\text{эВ}$
}
\solutionspace{80pt}

\tasknumber{4}%
\task{%
    Какой максимальной кинетической энергией (в эВ) обладают электроны,
    вырванные из металла при действии на него ультрафиолетового излучения с длиной волны $0{,}40\,\text{мкм}$,
    если работа выхода электрона $2{,}8 \cdot 10^{-19}\,\text{Дж}$? Постоянная Планка $h = 6{,}626 \cdot 10^{-34}\,\text{Дж}\cdot\text{с}$, заряд электрона $e = 1{,}6 \cdot 10^{-19}\,\text{Кл}$.
}
\answer{%
    $K = \frac{hc}{\lambda} - A \approx 1{,}36\,\text{эВ}$
}
\solutionspace{80pt}

\tasknumber{5}%
\task{%
    Чему равно задерживающее напряжение для фотоэлектронов, вырываемых с поверхности металла светом
    с энергией фотонов $6{,}3 \cdot 10^{-19}\,\text{Дж}$, если работа выхода из этого металла $4 \cdot 10^{-19}\,\text{Дж}$? Заряд электрона $e = 1{,}6 \cdot 10^{-19}\,\text{Кл}$.
}
\answer{%
    $eU = K = E - A \implies U = \frac{E - A}{ e } \approx 1{,}44\,\text{В}$
}
\solutionspace{80pt}

\tasknumber{6}%
\task{%
    Красная граница фотоэффекта для некоторого металла соответствует длине волны $6{,}2 \cdot 10^{-7}\,\text{м}$.
    Чему равно напряжение, полностью задерживающее фотоэлектроны, вырываемые из этого металла излучением
    с длиной волны $2{,}2 \cdot 10^{-5}\,\text{см}$? Постоянная Планка $h = 6{,}626 \cdot 10^{-34}\,\text{Дж}\cdot\text{с}$, заряд электрона $e = 1{,}6 \cdot 10^{-19}\,\text{Кл}$.
}
\answer{%
    $eU = K = E - A = \frac{hc}{\lambda} - A, \qquad 0 = \frac{hc}{\lambda_0} - A \implies U = \frac{ \frac{hc}{\lambda} - \frac{hc}{\lambda_0} }{ e } = \frac{hc}{ e }\cbr{ \frac 1{\lambda} - \frac 1{\lambda_0}}  \approx 3{,}6\,\text{В}$
}
\solutionspace{80pt}

\tasknumber{7}%
\task{%
    Определите длину волны (в нм) света, которым освещается поверхность металла,
    если фотоэлектроны имеют максимальную кинетическую энергию $4 \cdot 10^{-20}\,\text{Дж}$,
    а работа выхода электронов из этого металла $13 \cdot 10^{-19}\,\text{Дж}$.
    Постоянная Планка $h = 6{,}626 \cdot 10^{-34}\,\text{Дж}\cdot\text{с}$.
}
\answer{%
    $h \frac c\lambda = A_{\text{вых.}} + E_{\text{кин.}} \implies \lambda = \frac{h c}{A_{\text{вых.}} + E_{\text{кин.}}} = \frac{ 6{,}626 \cdot 10^{-34}\,\text{Дж}\cdot\text{с} \cdot {Const.c:V} }{13 \cdot 10^{-19}\,\text{Дж} + 4 \cdot 10^{-20}\,\text{Дж}} \approx 0{,}1483 \cdot 10^{-6}\,\text{м}.$
}
\solutionspace{80pt}

\tasknumber{8}%
\task{%
    Работа выхода электронов из некоторого металла $2{,}1\,\text{эВ}$.
    Найдите скорость электронов (в км/с),
    вылетающих с поверхности металла при освещении его светом с длиной волны $2{,}7 \cdot 10^{-5}\,\text{см}$.
    Масса электрона $m_{e} = 9{,}1 \cdot 10^{-31}\,\text{кг}$.
    Постоянная Планка $h = 6{,}626 \cdot 10^{-34}\,\text{Дж}\cdot\text{с}$, заряд электрона $e = 1{,}6 \cdot 10^{-19}\,\text{Кл}$.
}
\answer{%
    $h \frac c\lambda = A_{\text{вых.}} + \frac{ m_{e}v^2 }2 \implies v = \sqrt{ \frac 2{m_{e}}\cbr{ h \frac c\lambda - A_{\text{вых.}} } } \approx 937{,}9\,\frac{\text{км}}{\text{c}}.$
}
\solutionspace{80pt}

\tasknumber{9}%
\task{%
    Работа выхода электронов из некоторого металла $3{,}40\,\text{эВ}$.
    На металл падают фотоны с импульсом $2{,}40 \cdot 10^{-27}\,\frac{\text{кг}\cdot\text{м}}{\text{с}}$.
    Во сколько раз максимальный импульс электронов, вылетающих с поверхности металла при фотоэффекте,
    больше импульса падающих фотонов? Масса электрона $m_{e} = 9{,}1 \cdot 10^{-31}\,\text{кг}$.
}
\answer{%
    $h \frac c\lambda = A_{\text{вых.}} + \frac{p_e^2}{2m}, p=\frac h\lambda\implies p_e = \sqrt{2m\cbr{pc - A_{\text{вых.}}}} \implies \frac{p_e}p = \sqrt{\frac{2m}p \cbr{c - \frac{A_{\text{вых.}}}p } } \approx 235{,}82$
}

\variantsplitter

\addpersonalvariant{Анастасия Князева}

\tasknumber{1}%
\task{%
    Переведите в джоули:
    \begin{itemize}
        \item $1\,\text{эВ}$,
        \item $2{,}1\,\text{эВ}$,
        \item $4{,}6\,\text{МэВ}$,
        \item $1{,}5\,\text{кэВ}$,
        \item $9{,}4 \cdot 10^{-3}\,\text{эВ}$,
        \item $5{,}4 \cdot 10^{7}\,\text{эВ}$,
    \end{itemize}
}
\answer{%
    \begin{align*}
    1\,\text{эВ} &\approx 0{,}160 \cdot 10^{-18}\,\text{Дж} \\
    2{,}1\,\text{эВ} &\approx 0{,}336 \cdot 10^{-18}\,\text{Дж} \\
    4{,}6\,\text{МэВ} &\approx 0{,}736 \cdot 10^{-12}\,\text{Дж} \\
    1{,}5\,\text{кэВ} &\approx 0{,}24 \cdot 10^{-15}\,\text{Дж} \\
    9{,}4 \cdot 10^{-3}\,\text{эВ} &\approx 1{,}504 \cdot 10^{-21}\,\text{Дж} \\
    5{,}4 \cdot 10^{7}\,\text{эВ} &\approx 8{,}64 \cdot 10^{-12}\,\text{Дж}
    \end{align*}
}
\solutionspace{40pt}

\tasknumber{2}%
\task{%
    Переведите в электронвольты:
    \begin{itemize}
        \item $1\,\text{Дж}$,
        \item $2{,}1\,\text{Дж}$,
        \item $4{,}6\,\text{мкДж}$,
        \item $1{,}5\,\text{кДж}$,
        \item $9{,}4 \cdot 10^{-17}\,\text{Дж}$,
        \item $5{,}4 \cdot 10^{-21}\,\text{Дж}$,
    \end{itemize}
}
\answer{%
    \begin{align*}
    1\,\text{Дж} &\approx 6{,}3 \cdot 10^{18}\,\text{эВ} \\
    2{,}1\,\text{Дж} &\approx 13{,}13 \cdot 10^{18}\,\text{эВ} \\
    4{,}6\,\text{мкДж} &\approx 28{,}8 \cdot 10^{12}\,\text{эВ} \\
    1{,}5\,\text{кДж} &\approx 9{,}4 \cdot 10^{21}\,\text{эВ} \\
    9{,}4 \cdot 10^{-17}\,\text{Дж} &\approx 588\,\text{эВ} \\
    5{,}4 \cdot 10^{-21}\,\text{Дж} &\approx 0{,}0338\,\text{эВ}
    \end{align*}
}
\solutionspace{40pt}

\tasknumber{3}%
\task{%
    Свет с энергией кванта $4{,}1\,\text{эВ}$ вырывает из металлической пластинки электроны,
    имеющие максимальную кинетическую энергию $1{,}7\,\text{эВ}$.
    Найдите работу выхода (в эВ) электрона из этого металла.
}
\answer{%
    $A = E - K \approx 2{,}4\,\text{эВ}$
}
\solutionspace{80pt}

\tasknumber{4}%
\task{%
    Какой максимальной кинетической энергией (в эВ) обладают электроны,
    вырванные из металла при действии на него ультрафиолетового излучения с длиной волны $0{,}33\,\text{мкм}$,
    если работа выхода электрона $2{,}8 \cdot 10^{-19}\,\text{Дж}$? Постоянная Планка $h = 6{,}626 \cdot 10^{-34}\,\text{Дж}\cdot\text{с}$, заряд электрона $e = 1{,}6 \cdot 10^{-19}\,\text{Кл}$.
}
\answer{%
    $K = \frac{hc}{\lambda} - A \approx 2{,}0\,\text{эВ}$
}
\solutionspace{80pt}

\tasknumber{5}%
\task{%
    Чему равно задерживающее напряжение для фотоэлектронов, вырываемых с поверхности металла светом
    с энергией фотонов $7{,}8 \cdot 10^{-19}\,\text{Дж}$, если работа выхода из этого металла $5 \cdot 10^{-19}\,\text{Дж}$? Заряд электрона $e = 1{,}6 \cdot 10^{-19}\,\text{Кл}$.
}
\answer{%
    $eU = K = E - A \implies U = \frac{E - A}{ e } \approx 1{,}75\,\text{В}$
}
\solutionspace{80pt}

\tasknumber{6}%
\task{%
    Красная граница фотоэффекта для некоторого металла соответствует длине волны $5{,}7 \cdot 10^{-7}\,\text{м}$.
    Чему равно напряжение, полностью задерживающее фотоэлектроны, вырываемые из этого металла излучением
    с длиной волны $2{,}6 \cdot 10^{-5}\,\text{см}$? Постоянная Планка $h = 6{,}626 \cdot 10^{-34}\,\text{Дж}\cdot\text{с}$, заряд электрона $e = 1{,}6 \cdot 10^{-19}\,\text{Кл}$.
}
\answer{%
    $eU = K = E - A = \frac{hc}{\lambda} - A, \qquad 0 = \frac{hc}{\lambda_0} - A \implies U = \frac{ \frac{hc}{\lambda} - \frac{hc}{\lambda_0} }{ e } = \frac{hc}{ e }\cbr{ \frac 1{\lambda} - \frac 1{\lambda_0}}  \approx 2{,}6\,\text{В}$
}
\solutionspace{80pt}

\tasknumber{7}%
\task{%
    Определите длину волны (в нм) света, которым освещается поверхность металла,
    если фотоэлектроны имеют максимальную кинетическую энергию $4 \cdot 10^{-20}\,\text{Дж}$,
    а работа выхода электронов из этого металла $11 \cdot 10^{-19}\,\text{Дж}$.
    Постоянная Планка $h = 6{,}626 \cdot 10^{-34}\,\text{Дж}\cdot\text{с}$.
}
\answer{%
    $h \frac c\lambda = A_{\text{вых.}} + E_{\text{кин.}} \implies \lambda = \frac{h c}{A_{\text{вых.}} + E_{\text{кин.}}} = \frac{ 6{,}626 \cdot 10^{-34}\,\text{Дж}\cdot\text{с} \cdot {Const.c:V} }{11 \cdot 10^{-19}\,\text{Дж} + 4 \cdot 10^{-20}\,\text{Дж}} \approx 0{,}1744 \cdot 10^{-6}\,\text{м}.$
}
\solutionspace{80pt}

\tasknumber{8}%
\task{%
    Работа выхода электронов из некоторого металла $2{,}1\,\text{эВ}$.
    Найдите скорость электронов (в км/с),
    вылетающих с поверхности металла при освещении его светом с длиной волны $2{,}2 \cdot 10^{-5}\,\text{см}$.
    Масса электрона $m_{e} = 9{,}1 \cdot 10^{-31}\,\text{кг}$.
    Постоянная Планка $h = 6{,}626 \cdot 10^{-34}\,\text{Дж}\cdot\text{с}$, заряд электрона $e = 1{,}6 \cdot 10^{-19}\,\text{Кл}$.
}
\answer{%
    $h \frac c\lambda = A_{\text{вых.}} + \frac{ m_{e}v^2 }2 \implies v = \sqrt{ \frac 2{m_{e}}\cbr{ h \frac c\lambda - A_{\text{вых.}} } } \approx 1116{,}8\,\frac{\text{км}}{\text{c}}.$
}
\solutionspace{80pt}

\tasknumber{9}%
\task{%
    Работа выхода электронов из некоторого металла $3{,}40\,\text{эВ}$.
    На металл падают фотоны с импульсом $2{,}40 \cdot 10^{-27}\,\frac{\text{кг}\cdot\text{м}}{\text{с}}$.
    Во сколько раз максимальный импульс электронов, вылетающих с поверхности металла при фотоэффекте,
    больше импульса падающих фотонов? Масса электрона $m_{e} = 9{,}1 \cdot 10^{-31}\,\text{кг}$.
}
\answer{%
    $h \frac c\lambda = A_{\text{вых.}} + \frac{p_e^2}{2m}, p=\frac h\lambda\implies p_e = \sqrt{2m\cbr{pc - A_{\text{вых.}}}} \implies \frac{p_e}p = \sqrt{\frac{2m}p \cbr{c - \frac{A_{\text{вых.}}}p } } \approx 235{,}82$
}

\variantsplitter

\addpersonalvariant{Елизавета Кутумова}

\tasknumber{1}%
\task{%
    Переведите в джоули:
    \begin{itemize}
        \item $1\,\text{эВ}$,
        \item $1{,}3\,\text{эВ}$,
        \item $4{,}6\,\text{МэВ}$,
        \item $7{,}3\,\text{кэВ}$,
        \item $7{,}2 \cdot 10^{-3}\,\text{эВ}$,
        \item $5{,}4 \cdot 10^{7}\,\text{эВ}$,
    \end{itemize}
}
\answer{%
    \begin{align*}
    1\,\text{эВ} &\approx 0{,}160 \cdot 10^{-18}\,\text{Дж} \\
    1{,}3\,\text{эВ} &\approx 0{,}21 \cdot 10^{-18}\,\text{Дж} \\
    4{,}6\,\text{МэВ} &\approx 0{,}736 \cdot 10^{-12}\,\text{Дж} \\
    7{,}3\,\text{кэВ} &\approx 1{,}168 \cdot 10^{-15}\,\text{Дж} \\
    7{,}2 \cdot 10^{-3}\,\text{эВ} &\approx 1{,}152 \cdot 10^{-21}\,\text{Дж} \\
    5{,}4 \cdot 10^{7}\,\text{эВ} &\approx 8{,}64 \cdot 10^{-12}\,\text{Дж}
    \end{align*}
}
\solutionspace{40pt}

\tasknumber{2}%
\task{%
    Переведите в электронвольты:
    \begin{itemize}
        \item $1\,\text{Дж}$,
        \item $1{,}3\,\text{Дж}$,
        \item $4{,}6\,\text{мкДж}$,
        \item $7{,}3\,\text{кДж}$,
        \item $7{,}2 \cdot 10^{-17}\,\text{Дж}$,
        \item $5{,}4 \cdot 10^{-21}\,\text{Дж}$,
    \end{itemize}
}
\answer{%
    \begin{align*}
    1\,\text{Дж} &\approx 6{,}3 \cdot 10^{18}\,\text{эВ} \\
    1{,}3\,\text{Дж} &\approx 8{,}1 \cdot 10^{18}\,\text{эВ} \\
    4{,}6\,\text{мкДж} &\approx 28{,}8 \cdot 10^{12}\,\text{эВ} \\
    7{,}3\,\text{кДж} &\approx 45{,}6 \cdot 10^{21}\,\text{эВ} \\
    7{,}2 \cdot 10^{-17}\,\text{Дж} &\approx 450\,\text{эВ} \\
    5{,}4 \cdot 10^{-21}\,\text{Дж} &\approx 0{,}0338\,\text{эВ}
    \end{align*}
}
\solutionspace{40pt}

\tasknumber{3}%
\task{%
    Свет с энергией кванта $4{,}1\,\text{эВ}$ вырывает из металлической пластинки электроны,
    имеющие максимальную кинетическую энергию $1{,}5\,\text{эВ}$.
    Найдите работу выхода (в эВ) электрона из этого металла.
}
\answer{%
    $A = E - K \approx 2{,}6\,\text{эВ}$
}
\solutionspace{80pt}

\tasknumber{4}%
\task{%
    Какой максимальной кинетической энергией (в эВ) обладают электроны,
    вырванные из металла при действии на него ультрафиолетового излучения с длиной волны $0{,}33\,\text{мкм}$,
    если работа выхода электрона $2{,}8 \cdot 10^{-19}\,\text{Дж}$? Постоянная Планка $h = 6{,}626 \cdot 10^{-34}\,\text{Дж}\cdot\text{с}$, заряд электрона $e = 1{,}6 \cdot 10^{-19}\,\text{Кл}$.
}
\answer{%
    $K = \frac{hc}{\lambda} - A \approx 2{,}0\,\text{эВ}$
}
\solutionspace{80pt}

\tasknumber{5}%
\task{%
    Чему равно задерживающее напряжение для фотоэлектронов, вырываемых с поверхности металла светом
    с энергией фотонов $8{,}5 \cdot 10^{-19}\,\text{Дж}$, если работа выхода из этого металла $5 \cdot 10^{-19}\,\text{Дж}$? Заряд электрона $e = 1{,}6 \cdot 10^{-19}\,\text{Кл}$.
}
\answer{%
    $eU = K = E - A \implies U = \frac{E - A}{ e } \approx 2{,}2\,\text{В}$
}
\solutionspace{80pt}

\tasknumber{6}%
\task{%
    Красная граница фотоэффекта для некоторого металла соответствует длине волны $5{,}7 \cdot 10^{-7}\,\text{м}$.
    Чему равно напряжение, полностью задерживающее фотоэлектроны, вырываемые из этого металла излучением
    с длиной волны $1{,}7 \cdot 10^{-5}\,\text{см}$? Постоянная Планка $h = 6{,}626 \cdot 10^{-34}\,\text{Дж}\cdot\text{с}$, заряд электрона $e = 1{,}6 \cdot 10^{-19}\,\text{Кл}$.
}
\answer{%
    $eU = K = E - A = \frac{hc}{\lambda} - A, \qquad 0 = \frac{hc}{\lambda_0} - A \implies U = \frac{ \frac{hc}{\lambda} - \frac{hc}{\lambda_0} }{ e } = \frac{hc}{ e }\cbr{ \frac 1{\lambda} - \frac 1{\lambda_0}}  \approx 5\,\text{В}$
}
\solutionspace{80pt}

\tasknumber{7}%
\task{%
    Определите длину волны (в нм) света, которым освещается поверхность металла,
    если фотоэлектроны имеют максимальную кинетическую энергию $5 \cdot 10^{-20}\,\text{Дж}$,
    а работа выхода электронов из этого металла $7 \cdot 10^{-19}\,\text{Дж}$.
    Постоянная Планка $h = 6{,}626 \cdot 10^{-34}\,\text{Дж}\cdot\text{с}$.
}
\answer{%
    $h \frac c\lambda = A_{\text{вых.}} + E_{\text{кин.}} \implies \lambda = \frac{h c}{A_{\text{вых.}} + E_{\text{кин.}}} = \frac{ 6{,}626 \cdot 10^{-34}\,\text{Дж}\cdot\text{с} \cdot {Const.c:V} }{7 \cdot 10^{-19}\,\text{Дж} + 5 \cdot 10^{-20}\,\text{Дж}} \approx 0{,}27 \cdot 10^{-6}\,\text{м}.$
}
\solutionspace{80pt}

\tasknumber{8}%
\task{%
    Работа выхода электронов из некоторого металла $2{,}1\,\text{эВ}$.
    Найдите скорость электронов (в км/с),
    вылетающих с поверхности металла при освещении его светом с длиной волны $2{,}2 \cdot 10^{-5}\,\text{см}$.
    Масса электрона $m_{e} = 9{,}1 \cdot 10^{-31}\,\text{кг}$.
    Постоянная Планка $h = 6{,}626 \cdot 10^{-34}\,\text{Дж}\cdot\text{с}$, заряд электрона $e = 1{,}6 \cdot 10^{-19}\,\text{Кл}$.
}
\answer{%
    $h \frac c\lambda = A_{\text{вых.}} + \frac{ m_{e}v^2 }2 \implies v = \sqrt{ \frac 2{m_{e}}\cbr{ h \frac c\lambda - A_{\text{вых.}} } } \approx 1116{,}8\,\frac{\text{км}}{\text{c}}.$
}
\solutionspace{80pt}

\tasknumber{9}%
\task{%
    Работа выхода электронов из некоторого металла $4{,}30\,\text{эВ}$.
    На металл падают фотоны с импульсом $2{,}40 \cdot 10^{-27}\,\frac{\text{кг}\cdot\text{м}}{\text{с}}$.
    Во сколько раз максимальный импульс электронов, вылетающих с поверхности металла при фотоэффекте,
    больше импульса падающих фотонов? Масса электрона $m_{e} = 9{,}1 \cdot 10^{-31}\,\text{кг}$.
}
\answer{%
    $h \frac c\lambda = A_{\text{вых.}} + \frac{p_e^2}{2m}, p=\frac h\lambda\implies p_e = \sqrt{2m\cbr{pc - A_{\text{вых.}}}} \implies \frac{p_e}p = \sqrt{\frac{2m}p \cbr{c - \frac{A_{\text{вых.}}}p } } \approx 100{,}55$
}

\variantsplitter

\addpersonalvariant{Роксана Мехтиева}

\tasknumber{1}%
\task{%
    Переведите в джоули:
    \begin{itemize}
        \item $1\,\text{эВ}$,
        \item $4{,}3\,\text{эВ}$,
        \item $1{,}4\,\text{МэВ}$,
        \item $7{,}3\,\text{кэВ}$,
        \item $9{,}4 \cdot 10^{-3}\,\text{эВ}$,
        \item $5{,}4 \cdot 10^{7}\,\text{эВ}$,
    \end{itemize}
}
\answer{%
    \begin{align*}
    1\,\text{эВ} &\approx 0{,}160 \cdot 10^{-18}\,\text{Дж} \\
    4{,}3\,\text{эВ} &\approx 0{,}688 \cdot 10^{-18}\,\text{Дж} \\
    1{,}4\,\text{МэВ} &\approx 0{,}22 \cdot 10^{-12}\,\text{Дж} \\
    7{,}3\,\text{кэВ} &\approx 1{,}168 \cdot 10^{-15}\,\text{Дж} \\
    9{,}4 \cdot 10^{-3}\,\text{эВ} &\approx 1{,}504 \cdot 10^{-21}\,\text{Дж} \\
    5{,}4 \cdot 10^{7}\,\text{эВ} &\approx 8{,}64 \cdot 10^{-12}\,\text{Дж}
    \end{align*}
}
\solutionspace{40pt}

\tasknumber{2}%
\task{%
    Переведите в электронвольты:
    \begin{itemize}
        \item $1\,\text{Дж}$,
        \item $4{,}3\,\text{Дж}$,
        \item $1{,}4\,\text{мкДж}$,
        \item $7{,}3\,\text{кДж}$,
        \item $9{,}4 \cdot 10^{-17}\,\text{Дж}$,
        \item $5{,}4 \cdot 10^{-21}\,\text{Дж}$,
    \end{itemize}
}
\answer{%
    \begin{align*}
    1\,\text{Дж} &\approx 6{,}3 \cdot 10^{18}\,\text{эВ} \\
    4{,}3\,\text{Дж} &\approx 26{,}9 \cdot 10^{18}\,\text{эВ} \\
    1{,}4\,\text{мкДж} &\approx 8{,}8 \cdot 10^{12}\,\text{эВ} \\
    7{,}3\,\text{кДж} &\approx 45{,}6 \cdot 10^{21}\,\text{эВ} \\
    9{,}4 \cdot 10^{-17}\,\text{Дж} &\approx 588\,\text{эВ} \\
    5{,}4 \cdot 10^{-21}\,\text{Дж} &\approx 0{,}0338\,\text{эВ}
    \end{align*}
}
\solutionspace{40pt}

\tasknumber{3}%
\task{%
    Свет с энергией кванта $3{,}5\,\text{эВ}$ вырывает из металлической пластинки электроны,
    имеющие максимальную кинетическую энергию $1{,}7\,\text{эВ}$.
    Найдите работу выхода (в эВ) электрона из этого металла.
}
\answer{%
    $A = E - K \approx 1{,}80\,\text{эВ}$
}
\solutionspace{80pt}

\tasknumber{4}%
\task{%
    Какой максимальной кинетической энергией (в эВ) обладают электроны,
    вырванные из металла при действии на него ультрафиолетового излучения с длиной волны $0{,}25\,\text{мкм}$,
    если работа выхода электрона $3{,}4 \cdot 10^{-19}\,\text{Дж}$? Постоянная Планка $h = 6{,}626 \cdot 10^{-34}\,\text{Дж}\cdot\text{с}$, заряд электрона $e = 1{,}6 \cdot 10^{-19}\,\text{Кл}$.
}
\answer{%
    $K = \frac{hc}{\lambda} - A \approx 2{,}8\,\text{эВ}$
}
\solutionspace{80pt}

\tasknumber{5}%
\task{%
    Чему равно задерживающее напряжение для фотоэлектронов, вырываемых с поверхности металла светом
    с энергией фотонов $8{,}5 \cdot 10^{-19}\,\text{Дж}$, если работа выхода из этого металла $3 \cdot 10^{-19}\,\text{Дж}$? Заряд электрона $e = 1{,}6 \cdot 10^{-19}\,\text{Кл}$.
}
\answer{%
    $eU = K = E - A \implies U = \frac{E - A}{ e } \approx 3{,}4\,\text{В}$
}
\solutionspace{80pt}

\tasknumber{6}%
\task{%
    Красная граница фотоэффекта для некоторого металла соответствует длине волны $6{,}2 \cdot 10^{-7}\,\text{м}$.
    Чему равно напряжение, полностью задерживающее фотоэлектроны, вырываемые из этого металла излучением
    с длиной волны $2{,}2 \cdot 10^{-5}\,\text{см}$? Постоянная Планка $h = 6{,}626 \cdot 10^{-34}\,\text{Дж}\cdot\text{с}$, заряд электрона $e = 1{,}6 \cdot 10^{-19}\,\text{Кл}$.
}
\answer{%
    $eU = K = E - A = \frac{hc}{\lambda} - A, \qquad 0 = \frac{hc}{\lambda_0} - A \implies U = \frac{ \frac{hc}{\lambda} - \frac{hc}{\lambda_0} }{ e } = \frac{hc}{ e }\cbr{ \frac 1{\lambda} - \frac 1{\lambda_0}}  \approx 3{,}6\,\text{В}$
}
\solutionspace{80pt}

\tasknumber{7}%
\task{%
    Определите длину волны (в нм) света, которым освещается поверхность металла,
    если фотоэлектроны имеют максимальную кинетическую энергию $3 \cdot 10^{-20}\,\text{Дж}$,
    а работа выхода электронов из этого металла $13 \cdot 10^{-19}\,\text{Дж}$.
    Постоянная Планка $h = 6{,}626 \cdot 10^{-34}\,\text{Дж}\cdot\text{с}$.
}
\answer{%
    $h \frac c\lambda = A_{\text{вых.}} + E_{\text{кин.}} \implies \lambda = \frac{h c}{A_{\text{вых.}} + E_{\text{кин.}}} = \frac{ 6{,}626 \cdot 10^{-34}\,\text{Дж}\cdot\text{с} \cdot {Const.c:V} }{13 \cdot 10^{-19}\,\text{Дж} + 3 \cdot 10^{-20}\,\text{Дж}} \approx 0{,}1495 \cdot 10^{-6}\,\text{м}.$
}
\solutionspace{80pt}

\tasknumber{8}%
\task{%
    Работа выхода электронов из некоторого металла $4{,}3\,\text{эВ}$.
    Найдите скорость электронов (в км/с),
    вылетающих с поверхности металла при освещении его светом с длиной волны $1{,}7 \cdot 10^{-5}\,\text{см}$.
    Масса электрона $m_{e} = 9{,}1 \cdot 10^{-31}\,\text{кг}$.
    Постоянная Планка $h = 6{,}626 \cdot 10^{-34}\,\text{Дж}\cdot\text{с}$, заряд электрона $e = 1{,}6 \cdot 10^{-19}\,\text{Кл}$.
}
\answer{%
    $h \frac c\lambda = A_{\text{вых.}} + \frac{ m_{e}v^2 }2 \implies v = \sqrt{ \frac 2{m_{e}}\cbr{ h \frac c\lambda - A_{\text{вых.}} } } \approx 1028{,}5\,\frac{\text{км}}{\text{c}}.$
}
\solutionspace{80pt}

\tasknumber{9}%
\task{%
    Работа выхода электронов из некоторого металла $3{,}40\,\text{эВ}$.
    На металл падают фотоны с импульсом $3 \cdot 10^{-27}\,\frac{\text{кг}\cdot\text{м}}{\text{с}}$.
    Во сколько раз максимальный импульс электронов, вылетающих с поверхности металла при фотоэффекте,
    больше импульса падающих фотонов? Масса электрона $m_{e} = 9{,}1 \cdot 10^{-31}\,\text{кг}$.
}
\answer{%
    $h \frac c\lambda = A_{\text{вых.}} + \frac{p_e^2}{2m}, p=\frac h\lambda\implies p_e = \sqrt{2m\cbr{pc - A_{\text{вых.}}}} \implies \frac{p_e}p = \sqrt{\frac{2m}p \cbr{c - \frac{A_{\text{вых.}}}p } } \approx 268{,}31$
}

\variantsplitter

\addpersonalvariant{Дилноза Нодиршоева}

\tasknumber{1}%
\task{%
    Переведите в джоули:
    \begin{itemize}
        \item $1\,\text{эВ}$,
        \item $3{,}2\,\text{эВ}$,
        \item $13{,}5\,\text{МэВ}$,
        \item $6{,}2\,\text{кэВ}$,
        \item $8{,}3 \cdot 10^{-3}\,\text{эВ}$,
        \item $3{,}3 \cdot 10^{7}\,\text{эВ}$,
    \end{itemize}
}
\answer{%
    \begin{align*}
    1\,\text{эВ} &\approx 0{,}160 \cdot 10^{-18}\,\text{Дж} \\
    3{,}2\,\text{эВ} &\approx 0{,}512 \cdot 10^{-18}\,\text{Дж} \\
    13{,}5\,\text{МэВ} &\approx 2{,}16 \cdot 10^{-12}\,\text{Дж} \\
    6{,}2\,\text{кэВ} &\approx 0{,}992 \cdot 10^{-15}\,\text{Дж} \\
    8{,}3 \cdot 10^{-3}\,\text{эВ} &\approx 1{,}328 \cdot 10^{-21}\,\text{Дж} \\
    3{,}3 \cdot 10^{7}\,\text{эВ} &\approx 5{,}28 \cdot 10^{-12}\,\text{Дж}
    \end{align*}
}
\solutionspace{40pt}

\tasknumber{2}%
\task{%
    Переведите в электронвольты:
    \begin{itemize}
        \item $1\,\text{Дж}$,
        \item $3{,}2\,\text{Дж}$,
        \item $13{,}5\,\text{мкДж}$,
        \item $6{,}2\,\text{кДж}$,
        \item $8{,}3 \cdot 10^{-17}\,\text{Дж}$,
        \item $3{,}3 \cdot 10^{-21}\,\text{Дж}$,
    \end{itemize}
}
\answer{%
    \begin{align*}
    1\,\text{Дж} &\approx 6{,}3 \cdot 10^{18}\,\text{эВ} \\
    3{,}2\,\text{Дж} &\approx 20 \cdot 10^{18}\,\text{эВ} \\
    13{,}5\,\text{мкДж} &\approx 84{,}4 \cdot 10^{12}\,\text{эВ} \\
    6{,}2\,\text{кДж} &\approx 38{,}8 \cdot 10^{21}\,\text{эВ} \\
    8{,}3 \cdot 10^{-17}\,\text{Дж} &\approx 519\,\text{эВ} \\
    3{,}3 \cdot 10^{-21}\,\text{Дж} &\approx 0{,}0206\,\text{эВ}
    \end{align*}
}
\solutionspace{40pt}

\tasknumber{3}%
\task{%
    Свет с энергией кванта $4{,}1\,\text{эВ}$ вырывает из металлической пластинки электроны,
    имеющие максимальную кинетическую энергию $1{,}8\,\text{эВ}$.
    Найдите работу выхода (в эВ) электрона из этого металла.
}
\answer{%
    $A = E - K \approx 2{,}3\,\text{эВ}$
}
\solutionspace{80pt}

\tasknumber{4}%
\task{%
    Какой максимальной кинетической энергией (в эВ) обладают электроны,
    вырванные из металла при действии на него ультрафиолетового излучения с длиной волны $0{,}25\,\text{мкм}$,
    если работа выхода электрона $3{,}2 \cdot 10^{-19}\,\text{Дж}$? Постоянная Планка $h = 6{,}626 \cdot 10^{-34}\,\text{Дж}\cdot\text{с}$, заряд электрона $e = 1{,}6 \cdot 10^{-19}\,\text{Кл}$.
}
\answer{%
    $K = \frac{hc}{\lambda} - A \approx 3{,}0\,\text{эВ}$
}
\solutionspace{80pt}

\tasknumber{5}%
\task{%
    Чему равно задерживающее напряжение для фотоэлектронов, вырываемых с поверхности металла светом
    с энергией фотонов $6{,}3 \cdot 10^{-19}\,\text{Дж}$, если работа выхода из этого металла $5 \cdot 10^{-19}\,\text{Дж}$? Заряд электрона $e = 1{,}6 \cdot 10^{-19}\,\text{Кл}$.
}
\answer{%
    $eU = K = E - A \implies U = \frac{E - A}{ e } \approx 0{,}81\,\text{В}$
}
\solutionspace{80pt}

\tasknumber{6}%
\task{%
    Красная граница фотоэффекта для некоторого металла соответствует длине волны $5{,}3 \cdot 10^{-7}\,\text{м}$.
    Чему равно напряжение, полностью задерживающее фотоэлектроны, вырываемые из этого металла излучением
    с длиной волны $1{,}7 \cdot 10^{-5}\,\text{см}$? Постоянная Планка $h = 6{,}626 \cdot 10^{-34}\,\text{Дж}\cdot\text{с}$, заряд электрона $e = 1{,}6 \cdot 10^{-19}\,\text{Кл}$.
}
\answer{%
    $eU = K = E - A = \frac{hc}{\lambda} - A, \qquad 0 = \frac{hc}{\lambda_0} - A \implies U = \frac{ \frac{hc}{\lambda} - \frac{hc}{\lambda_0} }{ e } = \frac{hc}{ e }\cbr{ \frac 1{\lambda} - \frac 1{\lambda_0}}  \approx 5\,\text{В}$
}
\solutionspace{80pt}

\tasknumber{7}%
\task{%
    Определите длину волны (в нм) света, которым освещается поверхность металла,
    если фотоэлектроны имеют максимальную кинетическую энергию $9 \cdot 10^{-20}\,\text{Дж}$,
    а работа выхода электронов из этого металла $13 \cdot 10^{-19}\,\text{Дж}$.
    Постоянная Планка $h = 6{,}626 \cdot 10^{-34}\,\text{Дж}\cdot\text{с}$.
}
\answer{%
    $h \frac c\lambda = A_{\text{вых.}} + E_{\text{кин.}} \implies \lambda = \frac{h c}{A_{\text{вых.}} + E_{\text{кин.}}} = \frac{ 6{,}626 \cdot 10^{-34}\,\text{Дж}\cdot\text{с} \cdot {Const.c:V} }{13 \cdot 10^{-19}\,\text{Дж} + 9 \cdot 10^{-20}\,\text{Дж}} \approx 0{,}1430 \cdot 10^{-6}\,\text{м}.$
}
\solutionspace{80pt}

\tasknumber{8}%
\task{%
    Работа выхода электронов из некоторого металла $3{,}4\,\text{эВ}$.
    Найдите скорость электронов (в км/с),
    вылетающих с поверхности металла при освещении его светом с длиной волны $1{,}7 \cdot 10^{-5}\,\text{см}$.
    Масса электрона $m_{e} = 9{,}1 \cdot 10^{-31}\,\text{кг}$.
    Постоянная Планка $h = 6{,}626 \cdot 10^{-34}\,\text{Дж}\cdot\text{с}$, заряд электрона $e = 1{,}6 \cdot 10^{-19}\,\text{Кл}$.
}
\answer{%
    $h \frac c\lambda = A_{\text{вых.}} + \frac{ m_{e}v^2 }2 \implies v = \sqrt{ \frac 2{m_{e}}\cbr{ h \frac c\lambda - A_{\text{вых.}} } } \approx 1172{,}3\,\frac{\text{км}}{\text{c}}.$
}
\solutionspace{80pt}

\tasknumber{9}%
\task{%
    Работа выхода электронов из некоторого металла $4{,}30\,\text{эВ}$.
    На металл падают фотоны с импульсом $2{,}40 \cdot 10^{-27}\,\frac{\text{кг}\cdot\text{м}}{\text{с}}$.
    Во сколько раз максимальный импульс электронов, вылетающих с поверхности металла при фотоэффекте,
    больше импульса падающих фотонов? Масса электрона $m_{e} = 9{,}1 \cdot 10^{-31}\,\text{кг}$.
}
\answer{%
    $h \frac c\lambda = A_{\text{вых.}} + \frac{p_e^2}{2m}, p=\frac h\lambda\implies p_e = \sqrt{2m\cbr{pc - A_{\text{вых.}}}} \implies \frac{p_e}p = \sqrt{\frac{2m}p \cbr{c - \frac{A_{\text{вых.}}}p } } \approx 100{,}55$
}

\variantsplitter

\addpersonalvariant{Жаклин Пантелеева}

\tasknumber{1}%
\task{%
    Переведите в джоули:
    \begin{itemize}
        \item $1\,\text{эВ}$,
        \item $1{,}3\,\text{эВ}$,
        \item $5{,}7\,\text{МэВ}$,
        \item $7{,}3\,\text{кэВ}$,
        \item $9{,}4 \cdot 10^{-3}\,\text{эВ}$,
        \item $5{,}4 \cdot 10^{7}\,\text{эВ}$,
    \end{itemize}
}
\answer{%
    \begin{align*}
    1\,\text{эВ} &\approx 0{,}160 \cdot 10^{-18}\,\text{Дж} \\
    1{,}3\,\text{эВ} &\approx 0{,}21 \cdot 10^{-18}\,\text{Дж} \\
    5{,}7\,\text{МэВ} &\approx 0{,}912 \cdot 10^{-12}\,\text{Дж} \\
    7{,}3\,\text{кэВ} &\approx 1{,}168 \cdot 10^{-15}\,\text{Дж} \\
    9{,}4 \cdot 10^{-3}\,\text{эВ} &\approx 1{,}504 \cdot 10^{-21}\,\text{Дж} \\
    5{,}4 \cdot 10^{7}\,\text{эВ} &\approx 8{,}64 \cdot 10^{-12}\,\text{Дж}
    \end{align*}
}
\solutionspace{40pt}

\tasknumber{2}%
\task{%
    Переведите в электронвольты:
    \begin{itemize}
        \item $1\,\text{Дж}$,
        \item $1{,}3\,\text{Дж}$,
        \item $5{,}7\,\text{мкДж}$,
        \item $7{,}3\,\text{кДж}$,
        \item $9{,}4 \cdot 10^{-17}\,\text{Дж}$,
        \item $5{,}4 \cdot 10^{-21}\,\text{Дж}$,
    \end{itemize}
}
\answer{%
    \begin{align*}
    1\,\text{Дж} &\approx 6{,}3 \cdot 10^{18}\,\text{эВ} \\
    1{,}3\,\text{Дж} &\approx 8{,}1 \cdot 10^{18}\,\text{эВ} \\
    5{,}7\,\text{мкДж} &\approx 35{,}6 \cdot 10^{12}\,\text{эВ} \\
    7{,}3\,\text{кДж} &\approx 45{,}6 \cdot 10^{21}\,\text{эВ} \\
    9{,}4 \cdot 10^{-17}\,\text{Дж} &\approx 588\,\text{эВ} \\
    5{,}4 \cdot 10^{-21}\,\text{Дж} &\approx 0{,}0338\,\text{эВ}
    \end{align*}
}
\solutionspace{40pt}

\tasknumber{3}%
\task{%
    Свет с энергией кванта $3{,}5\,\text{эВ}$ вырывает из металлической пластинки электроны,
    имеющие максимальную кинетическую энергию $1{,}8\,\text{эВ}$.
    Найдите работу выхода (в эВ) электрона из этого металла.
}
\answer{%
    $A = E - K \approx 1{,}70\,\text{эВ}$
}
\solutionspace{80pt}

\tasknumber{4}%
\task{%
    Какой максимальной кинетической энергией (в эВ) обладают электроны,
    вырванные из металла при действии на него ультрафиолетового излучения с длиной волны $0{,}40\,\text{мкм}$,
    если работа выхода электрона $3{,}2 \cdot 10^{-19}\,\text{Дж}$? Постоянная Планка $h = 6{,}626 \cdot 10^{-34}\,\text{Дж}\cdot\text{с}$, заряд электрона $e = 1{,}6 \cdot 10^{-19}\,\text{Кл}$.
}
\answer{%
    $K = \frac{hc}{\lambda} - A \approx 1{,}11\,\text{эВ}$
}
\solutionspace{80pt}

\tasknumber{5}%
\task{%
    Чему равно задерживающее напряжение для фотоэлектронов, вырываемых с поверхности металла светом
    с энергией фотонов $6{,}3 \cdot 10^{-19}\,\text{Дж}$, если работа выхода из этого металла $3 \cdot 10^{-19}\,\text{Дж}$? Заряд электрона $e = 1{,}6 \cdot 10^{-19}\,\text{Кл}$.
}
\answer{%
    $eU = K = E - A \implies U = \frac{E - A}{ e } \approx 2{,}1\,\text{В}$
}
\solutionspace{80pt}

\tasknumber{6}%
\task{%
    Красная граница фотоэффекта для некоторого металла соответствует длине волны $5{,}7 \cdot 10^{-7}\,\text{м}$.
    Чему равно напряжение, полностью задерживающее фотоэлектроны, вырываемые из этого металла излучением
    с длиной волны $1{,}7 \cdot 10^{-5}\,\text{см}$? Постоянная Планка $h = 6{,}626 \cdot 10^{-34}\,\text{Дж}\cdot\text{с}$, заряд электрона $e = 1{,}6 \cdot 10^{-19}\,\text{Кл}$.
}
\answer{%
    $eU = K = E - A = \frac{hc}{\lambda} - A, \qquad 0 = \frac{hc}{\lambda_0} - A \implies U = \frac{ \frac{hc}{\lambda} - \frac{hc}{\lambda_0} }{ e } = \frac{hc}{ e }\cbr{ \frac 1{\lambda} - \frac 1{\lambda_0}}  \approx 5\,\text{В}$
}
\solutionspace{80pt}

\tasknumber{7}%
\task{%
    Определите длину волны (в нм) света, которым освещается поверхность металла,
    если фотоэлектроны имеют максимальную кинетическую энергию $3 \cdot 10^{-20}\,\text{Дж}$,
    а работа выхода электронов из этого металла $7 \cdot 10^{-19}\,\text{Дж}$.
    Постоянная Планка $h = 6{,}626 \cdot 10^{-34}\,\text{Дж}\cdot\text{с}$.
}
\answer{%
    $h \frac c\lambda = A_{\text{вых.}} + E_{\text{кин.}} \implies \lambda = \frac{h c}{A_{\text{вых.}} + E_{\text{кин.}}} = \frac{ 6{,}626 \cdot 10^{-34}\,\text{Дж}\cdot\text{с} \cdot {Const.c:V} }{7 \cdot 10^{-19}\,\text{Дж} + 3 \cdot 10^{-20}\,\text{Дж}} \approx 0{,}27 \cdot 10^{-6}\,\text{м}.$
}
\solutionspace{80pt}

\tasknumber{8}%
\task{%
    Работа выхода электронов из некоторого металла $3{,}4\,\text{эВ}$.
    Найдите скорость электронов (в км/с),
    вылетающих с поверхности металла при освещении его светом с длиной волны $2{,}7 \cdot 10^{-5}\,\text{см}$.
    Масса электрона $m_{e} = 9{,}1 \cdot 10^{-31}\,\text{кг}$.
    Постоянная Планка $h = 6{,}626 \cdot 10^{-34}\,\text{Дж}\cdot\text{с}$, заряд электрона $e = 1{,}6 \cdot 10^{-19}\,\text{Кл}$.
}
\answer{%
    $h \frac c\lambda = A_{\text{вых.}} + \frac{ m_{e}v^2 }2 \implies v = \sqrt{ \frac 2{m_{e}}\cbr{ h \frac c\lambda - A_{\text{вых.}} } } \approx 650\,\frac{\text{км}}{\text{c}}.$
}
\solutionspace{80pt}

\tasknumber{9}%
\task{%
    Работа выхода электронов из некоторого металла $3{,}40\,\text{эВ}$.
    На металл падают фотоны с импульсом $2{,}70 \cdot 10^{-27}\,\frac{\text{кг}\cdot\text{м}}{\text{с}}$.
    Во сколько раз максимальный импульс электронов, вылетающих с поверхности металла при фотоэффекте,
    больше импульса падающих фотонов? Масса электрона $m_{e} = 9{,}1 \cdot 10^{-31}\,\text{кг}$.
}
\answer{%
    $h \frac c\lambda = A_{\text{вых.}} + \frac{p_e^2}{2m}, p=\frac h\lambda\implies p_e = \sqrt{2m\cbr{pc - A_{\text{вых.}}}} \implies \frac{p_e}p = \sqrt{\frac{2m}p \cbr{c - \frac{A_{\text{вых.}}}p } } \approx 257{,}70$
}

\variantsplitter

\addpersonalvariant{Артём Переверзев}

\tasknumber{1}%
\task{%
    Переведите в джоули:
    \begin{itemize}
        \item $1\,\text{эВ}$,
        \item $1{,}3\,\text{эВ}$,
        \item $1{,}4\,\text{МэВ}$,
        \item $6{,}2\,\text{кэВ}$,
        \item $1{,}6 \cdot 10^{-3}\,\text{эВ}$,
        \item $5{,}4 \cdot 10^{7}\,\text{эВ}$,
    \end{itemize}
}
\answer{%
    \begin{align*}
    1\,\text{эВ} &\approx 0{,}160 \cdot 10^{-18}\,\text{Дж} \\
    1{,}3\,\text{эВ} &\approx 0{,}21 \cdot 10^{-18}\,\text{Дж} \\
    1{,}4\,\text{МэВ} &\approx 0{,}22 \cdot 10^{-12}\,\text{Дж} \\
    6{,}2\,\text{кэВ} &\approx 0{,}992 \cdot 10^{-15}\,\text{Дж} \\
    1{,}6 \cdot 10^{-3}\,\text{эВ} &\approx 0{,}26 \cdot 10^{-21}\,\text{Дж} \\
    5{,}4 \cdot 10^{7}\,\text{эВ} &\approx 8{,}64 \cdot 10^{-12}\,\text{Дж}
    \end{align*}
}
\solutionspace{40pt}

\tasknumber{2}%
\task{%
    Переведите в электронвольты:
    \begin{itemize}
        \item $1\,\text{Дж}$,
        \item $1{,}3\,\text{Дж}$,
        \item $1{,}4\,\text{мкДж}$,
        \item $6{,}2\,\text{кДж}$,
        \item $1{,}6 \cdot 10^{-17}\,\text{Дж}$,
        \item $5{,}4 \cdot 10^{-21}\,\text{Дж}$,
    \end{itemize}
}
\answer{%
    \begin{align*}
    1\,\text{Дж} &\approx 6{,}3 \cdot 10^{18}\,\text{эВ} \\
    1{,}3\,\text{Дж} &\approx 8{,}1 \cdot 10^{18}\,\text{эВ} \\
    1{,}4\,\text{мкДж} &\approx 8{,}8 \cdot 10^{12}\,\text{эВ} \\
    6{,}2\,\text{кДж} &\approx 38{,}8 \cdot 10^{21}\,\text{эВ} \\
    1{,}6 \cdot 10^{-17}\,\text{Дж} &\approx 100\,\text{эВ} \\
    5{,}4 \cdot 10^{-21}\,\text{Дж} &\approx 0{,}0338\,\text{эВ}
    \end{align*}
}
\solutionspace{40pt}

\tasknumber{3}%
\task{%
    Свет с энергией кванта $3{,}5\,\text{эВ}$ вырывает из металлической пластинки электроны,
    имеющие максимальную кинетическую энергию $1{,}8\,\text{эВ}$.
    Найдите работу выхода (в эВ) электрона из этого металла.
}
\answer{%
    $A = E - K \approx 1{,}70\,\text{эВ}$
}
\solutionspace{80pt}

\tasknumber{4}%
\task{%
    Какой максимальной кинетической энергией (в эВ) обладают электроны,
    вырванные из металла при действии на него ультрафиолетового излучения с длиной волны $0{,}33\,\text{мкм}$,
    если работа выхода электрона $3{,}4 \cdot 10^{-19}\,\text{Дж}$? Постоянная Планка $h = 6{,}626 \cdot 10^{-34}\,\text{Дж}\cdot\text{с}$, заряд электрона $e = 1{,}6 \cdot 10^{-19}\,\text{Кл}$.
}
\answer{%
    $K = \frac{hc}{\lambda} - A \approx 1{,}64\,\text{эВ}$
}
\solutionspace{80pt}

\tasknumber{5}%
\task{%
    Чему равно задерживающее напряжение для фотоэлектронов, вырываемых с поверхности металла светом
    с энергией фотонов $8{,}5 \cdot 10^{-19}\,\text{Дж}$, если работа выхода из этого металла $4 \cdot 10^{-19}\,\text{Дж}$? Заряд электрона $e = 1{,}6 \cdot 10^{-19}\,\text{Кл}$.
}
\answer{%
    $eU = K = E - A \implies U = \frac{E - A}{ e } \approx 2{,}8\,\text{В}$
}
\solutionspace{80pt}

\tasknumber{6}%
\task{%
    Красная граница фотоэффекта для некоторого металла соответствует длине волны $5{,}3 \cdot 10^{-7}\,\text{м}$.
    Чему равно напряжение, полностью задерживающее фотоэлектроны, вырываемые из этого металла излучением
    с длиной волны $1{,}7 \cdot 10^{-5}\,\text{см}$? Постоянная Планка $h = 6{,}626 \cdot 10^{-34}\,\text{Дж}\cdot\text{с}$, заряд электрона $e = 1{,}6 \cdot 10^{-19}\,\text{Кл}$.
}
\answer{%
    $eU = K = E - A = \frac{hc}{\lambda} - A, \qquad 0 = \frac{hc}{\lambda_0} - A \implies U = \frac{ \frac{hc}{\lambda} - \frac{hc}{\lambda_0} }{ e } = \frac{hc}{ e }\cbr{ \frac 1{\lambda} - \frac 1{\lambda_0}}  \approx 5\,\text{В}$
}
\solutionspace{80pt}

\tasknumber{7}%
\task{%
    Определите длину волны (в нм) света, которым освещается поверхность металла,
    если фотоэлектроны имеют максимальную кинетическую энергию $4 \cdot 10^{-20}\,\text{Дж}$,
    а работа выхода электронов из этого металла $9 \cdot 10^{-19}\,\text{Дж}$.
    Постоянная Планка $h = 6{,}626 \cdot 10^{-34}\,\text{Дж}\cdot\text{с}$.
}
\answer{%
    $h \frac c\lambda = A_{\text{вых.}} + E_{\text{кин.}} \implies \lambda = \frac{h c}{A_{\text{вых.}} + E_{\text{кин.}}} = \frac{ 6{,}626 \cdot 10^{-34}\,\text{Дж}\cdot\text{с} \cdot {Const.c:V} }{9 \cdot 10^{-19}\,\text{Дж} + 4 \cdot 10^{-20}\,\text{Дж}} \approx 0{,}21 \cdot 10^{-6}\,\text{м}.$
}
\solutionspace{80pt}

\tasknumber{8}%
\task{%
    Работа выхода электронов из некоторого металла $4{,}3\,\text{эВ}$.
    Найдите скорость электронов (в км/с),
    вылетающих с поверхности металла при освещении его светом с длиной волны $2{,}2 \cdot 10^{-5}\,\text{см}$.
    Масса электрона $m_{e} = 9{,}1 \cdot 10^{-31}\,\text{кг}$.
    Постоянная Планка $h = 6{,}626 \cdot 10^{-34}\,\text{Дж}\cdot\text{с}$, заряд электрона $e = 1{,}6 \cdot 10^{-19}\,\text{Кл}$.
}
\answer{%
    $h \frac c\lambda = A_{\text{вых.}} + \frac{ m_{e}v^2 }2 \implies v = \sqrt{ \frac 2{m_{e}}\cbr{ h \frac c\lambda - A_{\text{вых.}} } } \approx 688{,}3\,\frac{\text{км}}{\text{c}}.$
}
\solutionspace{80pt}

\tasknumber{9}%
\task{%
    Работа выхода электронов из некоторого металла $4{,}30\,\text{эВ}$.
    На металл падают фотоны с импульсом $3 \cdot 10^{-27}\,\frac{\text{кг}\cdot\text{м}}{\text{с}}$.
    Во сколько раз максимальный импульс электронов, вылетающих с поверхности металла при фотоэффекте,
    больше импульса падающих фотонов? Масса электрона $m_{e} = 9{,}1 \cdot 10^{-31}\,\text{кг}$.
}
\answer{%
    $h \frac c\lambda = A_{\text{вых.}} + \frac{p_e^2}{2m}, p=\frac h\lambda\implies p_e = \sqrt{2m\cbr{pc - A_{\text{вых.}}}} \implies \frac{p_e}p = \sqrt{\frac{2m}p \cbr{c - \frac{A_{\text{вых.}}}p } } \approx 207{,}05$
}

\variantsplitter

\addpersonalvariant{Варвара Пранова}

\tasknumber{1}%
\task{%
    Переведите в джоули:
    \begin{itemize}
        \item $1\,\text{эВ}$,
        \item $3{,}2\,\text{эВ}$,
        \item $4{,}6\,\text{МэВ}$,
        \item $5{,}1\,\text{кэВ}$,
        \item $1{,}6 \cdot 10^{-3}\,\text{эВ}$,
        \item $5{,}4 \cdot 10^{7}\,\text{эВ}$,
    \end{itemize}
}
\answer{%
    \begin{align*}
    1\,\text{эВ} &\approx 0{,}160 \cdot 10^{-18}\,\text{Дж} \\
    3{,}2\,\text{эВ} &\approx 0{,}512 \cdot 10^{-18}\,\text{Дж} \\
    4{,}6\,\text{МэВ} &\approx 0{,}736 \cdot 10^{-12}\,\text{Дж} \\
    5{,}1\,\text{кэВ} &\approx 0{,}816 \cdot 10^{-15}\,\text{Дж} \\
    1{,}6 \cdot 10^{-3}\,\text{эВ} &\approx 0{,}26 \cdot 10^{-21}\,\text{Дж} \\
    5{,}4 \cdot 10^{7}\,\text{эВ} &\approx 8{,}64 \cdot 10^{-12}\,\text{Дж}
    \end{align*}
}
\solutionspace{40pt}

\tasknumber{2}%
\task{%
    Переведите в электронвольты:
    \begin{itemize}
        \item $1\,\text{Дж}$,
        \item $3{,}2\,\text{Дж}$,
        \item $4{,}6\,\text{мкДж}$,
        \item $5{,}1\,\text{кДж}$,
        \item $1{,}6 \cdot 10^{-17}\,\text{Дж}$,
        \item $5{,}4 \cdot 10^{-21}\,\text{Дж}$,
    \end{itemize}
}
\answer{%
    \begin{align*}
    1\,\text{Дж} &\approx 6{,}3 \cdot 10^{18}\,\text{эВ} \\
    3{,}2\,\text{Дж} &\approx 20 \cdot 10^{18}\,\text{эВ} \\
    4{,}6\,\text{мкДж} &\approx 28{,}8 \cdot 10^{12}\,\text{эВ} \\
    5{,}1\,\text{кДж} &\approx 31{,}9 \cdot 10^{21}\,\text{эВ} \\
    1{,}6 \cdot 10^{-17}\,\text{Дж} &\approx 100\,\text{эВ} \\
    5{,}4 \cdot 10^{-21}\,\text{Дж} &\approx 0{,}0338\,\text{эВ}
    \end{align*}
}
\solutionspace{40pt}

\tasknumber{3}%
\task{%
    Свет с энергией кванта $4{,}1\,\text{эВ}$ вырывает из металлической пластинки электроны,
    имеющие максимальную кинетическую энергию $1{,}8\,\text{эВ}$.
    Найдите работу выхода (в эВ) электрона из этого металла.
}
\answer{%
    $A = E - K \approx 2{,}3\,\text{эВ}$
}
\solutionspace{80pt}

\tasknumber{4}%
\task{%
    Какой максимальной кинетической энергией (в эВ) обладают электроны,
    вырванные из металла при действии на него ультрафиолетового излучения с длиной волны $0{,}40\,\text{мкм}$,
    если работа выхода электрона $2{,}8 \cdot 10^{-19}\,\text{Дж}$? Постоянная Планка $h = 6{,}626 \cdot 10^{-34}\,\text{Дж}\cdot\text{с}$, заряд электрона $e = 1{,}6 \cdot 10^{-19}\,\text{Кл}$.
}
\answer{%
    $K = \frac{hc}{\lambda} - A \approx 1{,}36\,\text{эВ}$
}
\solutionspace{80pt}

\tasknumber{5}%
\task{%
    Чему равно задерживающее напряжение для фотоэлектронов, вырываемых с поверхности металла светом
    с энергией фотонов $7{,}8 \cdot 10^{-19}\,\text{Дж}$, если работа выхода из этого металла $4 \cdot 10^{-19}\,\text{Дж}$? Заряд электрона $e = 1{,}6 \cdot 10^{-19}\,\text{Кл}$.
}
\answer{%
    $eU = K = E - A \implies U = \frac{E - A}{ e } \approx 2{,}4\,\text{В}$
}
\solutionspace{80pt}

\tasknumber{6}%
\task{%
    Красная граница фотоэффекта для некоторого металла соответствует длине волны $5{,}7 \cdot 10^{-7}\,\text{м}$.
    Чему равно напряжение, полностью задерживающее фотоэлектроны, вырываемые из этого металла излучением
    с длиной волны $2{,}6 \cdot 10^{-5}\,\text{см}$? Постоянная Планка $h = 6{,}626 \cdot 10^{-34}\,\text{Дж}\cdot\text{с}$, заряд электрона $e = 1{,}6 \cdot 10^{-19}\,\text{Кл}$.
}
\answer{%
    $eU = K = E - A = \frac{hc}{\lambda} - A, \qquad 0 = \frac{hc}{\lambda_0} - A \implies U = \frac{ \frac{hc}{\lambda} - \frac{hc}{\lambda_0} }{ e } = \frac{hc}{ e }\cbr{ \frac 1{\lambda} - \frac 1{\lambda_0}}  \approx 2{,}6\,\text{В}$
}
\solutionspace{80pt}

\tasknumber{7}%
\task{%
    Определите длину волны (в нм) света, которым освещается поверхность металла,
    если фотоэлектроны имеют максимальную кинетическую энергию $4 \cdot 10^{-20}\,\text{Дж}$,
    а работа выхода электронов из этого металла $11 \cdot 10^{-19}\,\text{Дж}$.
    Постоянная Планка $h = 6{,}626 \cdot 10^{-34}\,\text{Дж}\cdot\text{с}$.
}
\answer{%
    $h \frac c\lambda = A_{\text{вых.}} + E_{\text{кин.}} \implies \lambda = \frac{h c}{A_{\text{вых.}} + E_{\text{кин.}}} = \frac{ 6{,}626 \cdot 10^{-34}\,\text{Дж}\cdot\text{с} \cdot {Const.c:V} }{11 \cdot 10^{-19}\,\text{Дж} + 4 \cdot 10^{-20}\,\text{Дж}} \approx 0{,}1744 \cdot 10^{-6}\,\text{м}.$
}
\solutionspace{80pt}

\tasknumber{8}%
\task{%
    Работа выхода электронов из некоторого металла $2{,}1\,\text{эВ}$.
    Найдите скорость электронов (в км/с),
    вылетающих с поверхности металла при освещении его светом с длиной волны $2{,}7 \cdot 10^{-5}\,\text{см}$.
    Масса электрона $m_{e} = 9{,}1 \cdot 10^{-31}\,\text{кг}$.
    Постоянная Планка $h = 6{,}626 \cdot 10^{-34}\,\text{Дж}\cdot\text{с}$, заряд электрона $e = 1{,}6 \cdot 10^{-19}\,\text{Кл}$.
}
\answer{%
    $h \frac c\lambda = A_{\text{вых.}} + \frac{ m_{e}v^2 }2 \implies v = \sqrt{ \frac 2{m_{e}}\cbr{ h \frac c\lambda - A_{\text{вых.}} } } \approx 937{,}9\,\frac{\text{км}}{\text{c}}.$
}
\solutionspace{80pt}

\tasknumber{9}%
\task{%
    Работа выхода электронов из некоторого металла $4{,}30\,\text{эВ}$.
    На металл падают фотоны с импульсом $3 \cdot 10^{-27}\,\frac{\text{кг}\cdot\text{м}}{\text{с}}$.
    Во сколько раз максимальный импульс электронов, вылетающих с поверхности металла при фотоэффекте,
    больше импульса падающих фотонов? Масса электрона $m_{e} = 9{,}1 \cdot 10^{-31}\,\text{кг}$.
}
\answer{%
    $h \frac c\lambda = A_{\text{вых.}} + \frac{p_e^2}{2m}, p=\frac h\lambda\implies p_e = \sqrt{2m\cbr{pc - A_{\text{вых.}}}} \implies \frac{p_e}p = \sqrt{\frac{2m}p \cbr{c - \frac{A_{\text{вых.}}}p } } \approx 207{,}05$
}

\variantsplitter

\addpersonalvariant{Марьям Салимова}

\tasknumber{1}%
\task{%
    Переведите в джоули:
    \begin{itemize}
        \item $1\,\text{эВ}$,
        \item $4{,}3\,\text{эВ}$,
        \item $5{,}7\,\text{МэВ}$,
        \item $7{,}3\,\text{кэВ}$,
        \item $8{,}3 \cdot 10^{-3}\,\text{эВ}$,
        \item $5{,}4 \cdot 10^{7}\,\text{эВ}$,
    \end{itemize}
}
\answer{%
    \begin{align*}
    1\,\text{эВ} &\approx 0{,}160 \cdot 10^{-18}\,\text{Дж} \\
    4{,}3\,\text{эВ} &\approx 0{,}688 \cdot 10^{-18}\,\text{Дж} \\
    5{,}7\,\text{МэВ} &\approx 0{,}912 \cdot 10^{-12}\,\text{Дж} \\
    7{,}3\,\text{кэВ} &\approx 1{,}168 \cdot 10^{-15}\,\text{Дж} \\
    8{,}3 \cdot 10^{-3}\,\text{эВ} &\approx 1{,}328 \cdot 10^{-21}\,\text{Дж} \\
    5{,}4 \cdot 10^{7}\,\text{эВ} &\approx 8{,}64 \cdot 10^{-12}\,\text{Дж}
    \end{align*}
}
\solutionspace{40pt}

\tasknumber{2}%
\task{%
    Переведите в электронвольты:
    \begin{itemize}
        \item $1\,\text{Дж}$,
        \item $4{,}3\,\text{Дж}$,
        \item $5{,}7\,\text{мкДж}$,
        \item $7{,}3\,\text{кДж}$,
        \item $8{,}3 \cdot 10^{-17}\,\text{Дж}$,
        \item $5{,}4 \cdot 10^{-21}\,\text{Дж}$,
    \end{itemize}
}
\answer{%
    \begin{align*}
    1\,\text{Дж} &\approx 6{,}3 \cdot 10^{18}\,\text{эВ} \\
    4{,}3\,\text{Дж} &\approx 26{,}9 \cdot 10^{18}\,\text{эВ} \\
    5{,}7\,\text{мкДж} &\approx 35{,}6 \cdot 10^{12}\,\text{эВ} \\
    7{,}3\,\text{кДж} &\approx 45{,}6 \cdot 10^{21}\,\text{эВ} \\
    8{,}3 \cdot 10^{-17}\,\text{Дж} &\approx 519\,\text{эВ} \\
    5{,}4 \cdot 10^{-21}\,\text{Дж} &\approx 0{,}0338\,\text{эВ}
    \end{align*}
}
\solutionspace{40pt}

\tasknumber{3}%
\task{%
    Свет с энергией кванта $4{,}1\,\text{эВ}$ вырывает из металлической пластинки электроны,
    имеющие максимальную кинетическую энергию $1{,}7\,\text{эВ}$.
    Найдите работу выхода (в эВ) электрона из этого металла.
}
\answer{%
    $A = E - K \approx 2{,}4\,\text{эВ}$
}
\solutionspace{80pt}

\tasknumber{4}%
\task{%
    Какой максимальной кинетической энергией (в эВ) обладают электроны,
    вырванные из металла при действии на него ультрафиолетового излучения с длиной волны $0{,}40\,\text{мкм}$,
    если работа выхода электрона $2{,}8 \cdot 10^{-19}\,\text{Дж}$? Постоянная Планка $h = 6{,}626 \cdot 10^{-34}\,\text{Дж}\cdot\text{с}$, заряд электрона $e = 1{,}6 \cdot 10^{-19}\,\text{Кл}$.
}
\answer{%
    $K = \frac{hc}{\lambda} - A \approx 1{,}36\,\text{эВ}$
}
\solutionspace{80pt}

\tasknumber{5}%
\task{%
    Чему равно задерживающее напряжение для фотоэлектронов, вырываемых с поверхности металла светом
    с энергией фотонов $9{,}2 \cdot 10^{-19}\,\text{Дж}$, если работа выхода из этого металла $3 \cdot 10^{-19}\,\text{Дж}$? Заряд электрона $e = 1{,}6 \cdot 10^{-19}\,\text{Кл}$.
}
\answer{%
    $eU = K = E - A \implies U = \frac{E - A}{ e } \approx 3{,}9\,\text{В}$
}
\solutionspace{80pt}

\tasknumber{6}%
\task{%
    Красная граница фотоэффекта для некоторого металла соответствует длине волны $6{,}2 \cdot 10^{-7}\,\text{м}$.
    Чему равно напряжение, полностью задерживающее фотоэлектроны, вырываемые из этого металла излучением
    с длиной волны $2{,}6 \cdot 10^{-5}\,\text{см}$? Постоянная Планка $h = 6{,}626 \cdot 10^{-34}\,\text{Дж}\cdot\text{с}$, заряд электрона $e = 1{,}6 \cdot 10^{-19}\,\text{Кл}$.
}
\answer{%
    $eU = K = E - A = \frac{hc}{\lambda} - A, \qquad 0 = \frac{hc}{\lambda_0} - A \implies U = \frac{ \frac{hc}{\lambda} - \frac{hc}{\lambda_0} }{ e } = \frac{hc}{ e }\cbr{ \frac 1{\lambda} - \frac 1{\lambda_0}}  \approx 2{,}8\,\text{В}$
}
\solutionspace{80pt}

\tasknumber{7}%
\task{%
    Определите длину волны (в нм) света, которым освещается поверхность металла,
    если фотоэлектроны имеют максимальную кинетическую энергию $3 \cdot 10^{-20}\,\text{Дж}$,
    а работа выхода электронов из этого металла $13 \cdot 10^{-19}\,\text{Дж}$.
    Постоянная Планка $h = 6{,}626 \cdot 10^{-34}\,\text{Дж}\cdot\text{с}$.
}
\answer{%
    $h \frac c\lambda = A_{\text{вых.}} + E_{\text{кин.}} \implies \lambda = \frac{h c}{A_{\text{вых.}} + E_{\text{кин.}}} = \frac{ 6{,}626 \cdot 10^{-34}\,\text{Дж}\cdot\text{с} \cdot {Const.c:V} }{13 \cdot 10^{-19}\,\text{Дж} + 3 \cdot 10^{-20}\,\text{Дж}} \approx 0{,}1495 \cdot 10^{-6}\,\text{м}.$
}
\solutionspace{80pt}

\tasknumber{8}%
\task{%
    Работа выхода электронов из некоторого металла $2{,}1\,\text{эВ}$.
    Найдите скорость электронов (в км/с),
    вылетающих с поверхности металла при освещении его светом с длиной волны $2{,}7 \cdot 10^{-5}\,\text{см}$.
    Масса электрона $m_{e} = 9{,}1 \cdot 10^{-31}\,\text{кг}$.
    Постоянная Планка $h = 6{,}626 \cdot 10^{-34}\,\text{Дж}\cdot\text{с}$, заряд электрона $e = 1{,}6 \cdot 10^{-19}\,\text{Кл}$.
}
\answer{%
    $h \frac c\lambda = A_{\text{вых.}} + \frac{ m_{e}v^2 }2 \implies v = \sqrt{ \frac 2{m_{e}}\cbr{ h \frac c\lambda - A_{\text{вых.}} } } \approx 937{,}9\,\frac{\text{км}}{\text{c}}.$
}
\solutionspace{80pt}

\tasknumber{9}%
\task{%
    Работа выхода электронов из некоторого металла $4{,}30\,\text{эВ}$.
    На металл падают фотоны с импульсом $2{,}40 \cdot 10^{-27}\,\frac{\text{кг}\cdot\text{м}}{\text{с}}$.
    Во сколько раз максимальный импульс электронов, вылетающих с поверхности металла при фотоэффекте,
    больше импульса падающих фотонов? Масса электрона $m_{e} = 9{,}1 \cdot 10^{-31}\,\text{кг}$.
}
\answer{%
    $h \frac c\lambda = A_{\text{вых.}} + \frac{p_e^2}{2m}, p=\frac h\lambda\implies p_e = \sqrt{2m\cbr{pc - A_{\text{вых.}}}} \implies \frac{p_e}p = \sqrt{\frac{2m}p \cbr{c - \frac{A_{\text{вых.}}}p } } \approx 100{,}55$
}

\variantsplitter

\addpersonalvariant{Юлия Шевченко}

\tasknumber{1}%
\task{%
    Переведите в джоули:
    \begin{itemize}
        \item $1\,\text{эВ}$,
        \item $1{,}3\,\text{эВ}$,
        \item $1{,}4\,\text{МэВ}$,
        \item $6{,}2\,\text{кэВ}$,
        \item $8{,}3 \cdot 10^{-3}\,\text{эВ}$,
        \item $3{,}3 \cdot 10^{7}\,\text{эВ}$,
    \end{itemize}
}
\answer{%
    \begin{align*}
    1\,\text{эВ} &\approx 0{,}160 \cdot 10^{-18}\,\text{Дж} \\
    1{,}3\,\text{эВ} &\approx 0{,}21 \cdot 10^{-18}\,\text{Дж} \\
    1{,}4\,\text{МэВ} &\approx 0{,}22 \cdot 10^{-12}\,\text{Дж} \\
    6{,}2\,\text{кэВ} &\approx 0{,}992 \cdot 10^{-15}\,\text{Дж} \\
    8{,}3 \cdot 10^{-3}\,\text{эВ} &\approx 1{,}328 \cdot 10^{-21}\,\text{Дж} \\
    3{,}3 \cdot 10^{7}\,\text{эВ} &\approx 5{,}28 \cdot 10^{-12}\,\text{Дж}
    \end{align*}
}
\solutionspace{40pt}

\tasknumber{2}%
\task{%
    Переведите в электронвольты:
    \begin{itemize}
        \item $1\,\text{Дж}$,
        \item $1{,}3\,\text{Дж}$,
        \item $1{,}4\,\text{мкДж}$,
        \item $6{,}2\,\text{кДж}$,
        \item $8{,}3 \cdot 10^{-17}\,\text{Дж}$,
        \item $3{,}3 \cdot 10^{-21}\,\text{Дж}$,
    \end{itemize}
}
\answer{%
    \begin{align*}
    1\,\text{Дж} &\approx 6{,}3 \cdot 10^{18}\,\text{эВ} \\
    1{,}3\,\text{Дж} &\approx 8{,}1 \cdot 10^{18}\,\text{эВ} \\
    1{,}4\,\text{мкДж} &\approx 8{,}8 \cdot 10^{12}\,\text{эВ} \\
    6{,}2\,\text{кДж} &\approx 38{,}8 \cdot 10^{21}\,\text{эВ} \\
    8{,}3 \cdot 10^{-17}\,\text{Дж} &\approx 519\,\text{эВ} \\
    3{,}3 \cdot 10^{-21}\,\text{Дж} &\approx 0{,}0206\,\text{эВ}
    \end{align*}
}
\solutionspace{40pt}

\tasknumber{3}%
\task{%
    Свет с энергией кванта $4{,}1\,\text{эВ}$ вырывает из металлической пластинки электроны,
    имеющие максимальную кинетическую энергию $1{,}7\,\text{эВ}$.
    Найдите работу выхода (в эВ) электрона из этого металла.
}
\answer{%
    $A = E - K \approx 2{,}4\,\text{эВ}$
}
\solutionspace{80pt}

\tasknumber{4}%
\task{%
    Какой максимальной кинетической энергией (в эВ) обладают электроны,
    вырванные из металла при действии на него ультрафиолетового излучения с длиной волны $0{,}40\,\text{мкм}$,
    если работа выхода электрона $3{,}2 \cdot 10^{-19}\,\text{Дж}$? Постоянная Планка $h = 6{,}626 \cdot 10^{-34}\,\text{Дж}\cdot\text{с}$, заряд электрона $e = 1{,}6 \cdot 10^{-19}\,\text{Кл}$.
}
\answer{%
    $K = \frac{hc}{\lambda} - A \approx 1{,}11\,\text{эВ}$
}
\solutionspace{80pt}

\tasknumber{5}%
\task{%
    Чему равно задерживающее напряжение для фотоэлектронов, вырываемых с поверхности металла светом
    с энергией фотонов $8{,}5 \cdot 10^{-19}\,\text{Дж}$, если работа выхода из этого металла $5 \cdot 10^{-19}\,\text{Дж}$? Заряд электрона $e = 1{,}6 \cdot 10^{-19}\,\text{Кл}$.
}
\answer{%
    $eU = K = E - A \implies U = \frac{E - A}{ e } \approx 2{,}2\,\text{В}$
}
\solutionspace{80pt}

\tasknumber{6}%
\task{%
    Красная граница фотоэффекта для некоторого металла соответствует длине волны $5{,}7 \cdot 10^{-7}\,\text{м}$.
    Чему равно напряжение, полностью задерживающее фотоэлектроны, вырываемые из этого металла излучением
    с длиной волны $2{,}2 \cdot 10^{-5}\,\text{см}$? Постоянная Планка $h = 6{,}626 \cdot 10^{-34}\,\text{Дж}\cdot\text{с}$, заряд электрона $e = 1{,}6 \cdot 10^{-19}\,\text{Кл}$.
}
\answer{%
    $eU = K = E - A = \frac{hc}{\lambda} - A, \qquad 0 = \frac{hc}{\lambda_0} - A \implies U = \frac{ \frac{hc}{\lambda} - \frac{hc}{\lambda_0} }{ e } = \frac{hc}{ e }\cbr{ \frac 1{\lambda} - \frac 1{\lambda_0}}  \approx 3{,}5\,\text{В}$
}
\solutionspace{80pt}

\tasknumber{7}%
\task{%
    Определите длину волны (в нм) света, которым освещается поверхность металла,
    если фотоэлектроны имеют максимальную кинетическую энергию $3 \cdot 10^{-20}\,\text{Дж}$,
    а работа выхода электронов из этого металла $11 \cdot 10^{-19}\,\text{Дж}$.
    Постоянная Планка $h = 6{,}626 \cdot 10^{-34}\,\text{Дж}\cdot\text{с}$.
}
\answer{%
    $h \frac c\lambda = A_{\text{вых.}} + E_{\text{кин.}} \implies \lambda = \frac{h c}{A_{\text{вых.}} + E_{\text{кин.}}} = \frac{ 6{,}626 \cdot 10^{-34}\,\text{Дж}\cdot\text{с} \cdot {Const.c:V} }{11 \cdot 10^{-19}\,\text{Дж} + 3 \cdot 10^{-20}\,\text{Дж}} \approx 0{,}1759 \cdot 10^{-6}\,\text{м}.$
}
\solutionspace{80pt}

\tasknumber{8}%
\task{%
    Работа выхода электронов из некоторого металла $3{,}4\,\text{эВ}$.
    Найдите скорость электронов (в км/с),
    вылетающих с поверхности металла при освещении его светом с длиной волны $2{,}7 \cdot 10^{-5}\,\text{см}$.
    Масса электрона $m_{e} = 9{,}1 \cdot 10^{-31}\,\text{кг}$.
    Постоянная Планка $h = 6{,}626 \cdot 10^{-34}\,\text{Дж}\cdot\text{с}$, заряд электрона $e = 1{,}6 \cdot 10^{-19}\,\text{Кл}$.
}
\answer{%
    $h \frac c\lambda = A_{\text{вых.}} + \frac{ m_{e}v^2 }2 \implies v = \sqrt{ \frac 2{m_{e}}\cbr{ h \frac c\lambda - A_{\text{вых.}} } } \approx 650\,\frac{\text{км}}{\text{c}}.$
}
\solutionspace{80pt}

\tasknumber{9}%
\task{%
    Работа выхода электронов из некоторого металла $4{,}30\,\text{эВ}$.
    На металл падают фотоны с импульсом $2{,}40 \cdot 10^{-27}\,\frac{\text{кг}\cdot\text{м}}{\text{с}}$.
    Во сколько раз максимальный импульс электронов, вылетающих с поверхности металла при фотоэффекте,
    больше импульса падающих фотонов? Масса электрона $m_{e} = 9{,}1 \cdot 10^{-31}\,\text{кг}$.
}
\answer{%
    $h \frac c\lambda = A_{\text{вых.}} + \frac{p_e^2}{2m}, p=\frac h\lambda\implies p_e = \sqrt{2m\cbr{pc - A_{\text{вых.}}}} \implies \frac{p_e}p = \sqrt{\frac{2m}p \cbr{c - \frac{A_{\text{вых.}}}p } } \approx 100{,}55$
}
% autogenerated
