\setdate{28~сентября~2021}
\setclass{11«Б»}

\addpersonalvariant{Михаил Бурмистров}

\tasknumber{1}%
\task{%
    Установите каждой букве в соответствие ровно одну цифру и запишите в ответ только цифры (без других символов).

    А) $\ele$, Б) $\Delta \Phi$, В) $\Delta t$.

    1) $t_2 - t_1$, 2) $-\frac{\Delta \Phi}{\Delta t}$, 3) $\Phi_2 - \Phi_1$, 4) $\Phi_1 - \Phi_2$, 5) $t_1 - t_2$.
}
\answer{%
    $231$
}
\solutionspace{20pt}

\tasknumber{2}%
\task{%
    Установите каждой букве в соответствие ровно одну цифру и запишите ответ (только цифры, без других символов).

    А) $\Delta \Phi$, Б) $\Delta \eli$, В) $\Phi$.

    1) $\frac{L}{\eli}$, 2) $L\eli$, 3) $\Phi_2 - \Phi_1$, 4) $\eli_2 - \eli_1$, 5) $\eli_1 - \eli_2$, 6) $\Phi_1 - \Phi_2$.
}
\answer{%
    $342$
}

\tasknumber{3}%
\task{%
    Установите каждой букве в соответствие ровно одну цифру и запишите ответ (только цифры, без других символов).

    А) поток магнитной индукции, Б) электрический ток, В) индукция магнитного поля.

    1) $\vec B$, 2) $\Phi$, 3) $\eli$, 4) $q$, 5) $L$.
}
\answer{%
    $231$
}

\tasknumber{4}%
\task{%
    Установите каждой букве в соответствие ровно одну цифру и запишите ответ (только цифры, без других символов).

    А) время, Б) индукция магнитного поля, В) поток магнитной индукции.

    1) Вб, 2) с, 3) Тл, 4) А, 5) Гн.
}
\answer{%
    $231$
}

\tasknumber{5}%
\task{%
    	В катушке, индуктивность которой равна $6\,\text{мГн}$, сила тока равномерно уменьшается
    	с $2\,\text{А}$ до $9\,\text{А}$ за $0{,}2\,\text{c}$.
    Опредилите ЭДС самоиндукции, ответ выразите в мВ и округлите до целых.
}
\answer{%
    $
        \ele
        = L\frac{\abs{\Delta \eli}}{\Delta t}
        = L\frac{\abs{\eli_2 - \eli_1}}{\Delta t}
        = 6\,\text{мГн} \cdot \frac{\abs{9\,\text{А} - 2\,\text{А}}}{0{,}2\,\text{c}}
        \approx 210{,}000\,\text{мВ} \to 210
    $
}
\solutionspace{60pt}

\tasknumber{6}%
\task{%
    	В катушке, индуктивность которой равна $60\,\text{мГн}$, течёт электрический ток силой $6\,\text{А}$.
    	Число витков в катушке: {N}.
    Определите магнитный поток, пронизывающий 1 виток катушки.
    	Ответ выразите в милливеберах и округлите до целых.
}
\answer{%
    $
        \Phi_\text{1 виток}
        = \frac{\Phi}{N}
        = \frac{L\eli}{N}
        = \frac{60\,\text{мГн} \cdot 6\,\text{А}}{40}
        \approx 9{,}000\,\text{мВб}
        \to 9
    $
}

\variantsplitter

\addpersonalvariant{Снежана Авдошина}

\tasknumber{1}%
\task{%
    Установите каждой букве в соответствие ровно одну цифру и запишите в ответ только цифры (без других символов).

    А) $\ele$, Б) $\Delta t$, В) $\Delta \Phi$.

    1) $\Phi_1 - \Phi_2$, 2) $t_2 - t_1$, 3) $t_1 - t_2$, 4) $\Phi_2 - \Phi_1$, 5) $-\frac{\Delta \Phi}{\Delta t}$.
}
\answer{%
    $524$
}
\solutionspace{20pt}

\tasknumber{2}%
\task{%
    Установите каждой букве в соответствие ровно одну цифру и запишите ответ (только цифры, без других символов).

    А) $\Delta \eli$, Б) $\Phi$, В) $\Delta \Phi$.

    1) $\eli_2 - \eli_1$, 2) $\Phi_1 - \Phi_2$, 3) $\Phi_2 - \Phi_1$, 4) $\eli_1 - \eli_2$, 5) $\frac{\eli}{L}$, 6) $L\eli$.
}
\answer{%
    $163$
}

\tasknumber{3}%
\task{%
    Установите каждой букве в соответствие ровно одну цифру и запишите ответ (только цифры, без других символов).

    А) индуктивность, Б) электрический заряд, В) поток магнитной индукции.

    1) $\eli$, 2) $q$, 3) $R$, 4) $\Phi$, 5) $L$.
}
\answer{%
    $524$
}

\tasknumber{4}%
\task{%
    Установите каждой букве в соответствие ровно одну цифру и запишите ответ (только цифры, без других символов).

    А) поток магнитной индукции, Б) индуктивность, В) время.

    1) А, 2) Гн, 3) Кл, 4) с, 5) Вб.
}
\answer{%
    $524$
}

\tasknumber{5}%
\task{%
    	В катушке, индуктивность которой равна $5\,\text{мГн}$, сила тока равномерно уменьшается
    	с $2\,\text{А}$ до $9\,\text{А}$ за $0{,}4\,\text{c}$.
    Опредилите ЭДС самоиндукции, ответ выразите в мВ и округлите до целых.
}
\answer{%
    $
        \ele
        = L\frac{\abs{\Delta \eli}}{\Delta t}
        = L\frac{\abs{\eli_2 - \eli_1}}{\Delta t}
        = 5\,\text{мГн} \cdot \frac{\abs{9\,\text{А} - 2\,\text{А}}}{0{,}4\,\text{c}}
        \approx 87{,}500\,\text{мВ} \to 88
    $
}
\solutionspace{60pt}

\tasknumber{6}%
\task{%
    	В катушке, индуктивность которой равна $50\,\text{мГн}$, течёт электрический ток силой $5\,\text{А}$.
    	Число витков в катушке: {N}.
    Определите магнитный поток, пронизывающий 1 виток катушки.
    	Ответ выразите в милливеберах и округлите до целых.
}
\answer{%
    $
        \Phi_\text{1 виток}
        = \frac{\Phi}{N}
        = \frac{L\eli}{N}
        = \frac{50\,\text{мГн} \cdot 5\,\text{А}}{30}
        \approx 8{,}333\,\text{мВб}
        \to 8
    $
}

\variantsplitter

\addpersonalvariant{Марьяна Аристова}

\tasknumber{1}%
\task{%
    Установите каждой букве в соответствие ровно одну цифру и запишите в ответ только цифры (без других символов).

    А) $\Delta t$, Б) $\ele$, В) $\Delta \Phi$.

    1) $t_2 - t_1$, 2) $\Phi_2 - \Phi_1$, 3) $\Phi_1 - \Phi_2$, 4) $-\frac{\Delta \Phi}{\Delta t}$, 5) $t_1 - t_2$.
}
\answer{%
    $142$
}
\solutionspace{20pt}

\tasknumber{2}%
\task{%
    Установите каждой букве в соответствие ровно одну цифру и запишите ответ (только цифры, без других символов).

    А) $\Delta \eli$, Б) $\Delta \Phi$, В) $\Phi$.

    1) $\eli_1 - \eli_2$, 2) $\eli_2 - \eli_1$, 3) $L\eli$, 4) $\frac{L}{\eli}$, 5) $\Phi_2 - \Phi_1$, 6) $\Phi_1 - \Phi_2$.
}
\answer{%
    $253$
}

\tasknumber{3}%
\task{%
    Установите каждой букве в соответствие ровно одну цифру и запишите ответ (только цифры, без других символов).

    А) поток магнитной индукции, Б) индукция магнитного поля, В) электрический заряд.

    1) $\Phi$, 2) $q$, 3) $L$, 4) $\vec B$, 5) $R$.
}
\answer{%
    $142$
}

\tasknumber{4}%
\task{%
    Установите каждой букве в соответствие ровно одну цифру и запишите ответ (только цифры, без других символов).

    А) время, Б) индукция магнитного поля, В) индуктивность.

    1) с, 2) Гн, 3) Вб, 4) Тл, 5) Кл.
}
\answer{%
    $142$
}

\tasknumber{5}%
\task{%
    	В катушке, индуктивность которой равна $4\,\text{мГн}$, сила тока равномерно уменьшается
    	с $2\,\text{А}$ до $7\,\text{А}$ за $0{,}5\,\text{c}$.
    Опредилите ЭДС самоиндукции, ответ выразите в мВ и округлите до целых.
}
\answer{%
    $
        \ele
        = L\frac{\abs{\Delta \eli}}{\Delta t}
        = L\frac{\abs{\eli_2 - \eli_1}}{\Delta t}
        = 4\,\text{мГн} \cdot \frac{\abs{7\,\text{А} - 2\,\text{А}}}{0{,}5\,\text{c}}
        \approx 40{,}000\,\text{мВ} \to 40
    $
}
\solutionspace{60pt}

\tasknumber{6}%
\task{%
    	В катушке, индуктивность которой равна $80\,\text{мГн}$, течёт электрический ток силой $5\,\text{А}$.
    	Число витков в катушке: {N}.
    Определите магнитный поток, пронизывающий 1 виток катушки.
    	Ответ выразите в милливеберах и округлите до целых.
}
\answer{%
    $
        \Phi_\text{1 виток}
        = \frac{\Phi}{N}
        = \frac{L\eli}{N}
        = \frac{80\,\text{мГн} \cdot 5\,\text{А}}{20}
        \approx 20{,}000\,\text{мВб}
        \to 20
    $
}

\variantsplitter

\addpersonalvariant{Никита Иванов}

\tasknumber{1}%
\task{%
    Установите каждой букве в соответствие ровно одну цифру и запишите в ответ только цифры (без других символов).

    А) $\ele$, Б) $\Delta t$, В) $\Delta \Phi$.

    1) $t_1 - t_2$, 2) $\Phi_2 - \Phi_1$, 3) $\Phi_1 - \Phi_2$, 4) $-\frac{\Delta \Phi}{\Delta t}$, 5) $t_2 - t_1$.
}
\answer{%
    $452$
}
\solutionspace{20pt}

\tasknumber{2}%
\task{%
    Установите каждой букве в соответствие ровно одну цифру и запишите ответ (только цифры, без других символов).

    А) $\Delta \Phi$, Б) $\Delta \eli$, В) $\Phi$.

    1) $\Phi_2 - \Phi_1$, 2) $\eli_1 - \eli_2$, 3) $\Phi_1 - \Phi_2$, 4) $\frac{L}{\eli}$, 5) $\eli_2 - \eli_1$, 6) $L\eli$.
}
\answer{%
    $156$
}

\tasknumber{3}%
\task{%
    Установите каждой букве в соответствие ровно одну цифру и запишите ответ (только цифры, без других символов).

    А) электрический заряд, Б) поток магнитной индукции, В) электрический ток.

    1) $L$, 2) $\eli$, 3) $\phi$, 4) $q$, 5) $\Phi$.
}
\answer{%
    $452$
}

\tasknumber{4}%
\task{%
    Установите каждой букве в соответствие ровно одну цифру и запишите ответ (только цифры, без других символов).

    А) индуктивность, Б) индукция магнитного поля, В) время.

    1) Вб, 2) с, 3) А, 4) Гн, 5) Тл.
}
\answer{%
    $452$
}

\tasknumber{5}%
\task{%
    	В катушке, индуктивность которой равна $5\,\text{мГн}$, сила тока равномерно уменьшается
    	с $3\,\text{А}$ до $9\,\text{А}$ за $0{,}3\,\text{c}$.
    Опредилите ЭДС самоиндукции, ответ выразите в мВ и округлите до целых.
}
\answer{%
    $
        \ele
        = L\frac{\abs{\Delta \eli}}{\Delta t}
        = L\frac{\abs{\eli_2 - \eli_1}}{\Delta t}
        = 5\,\text{мГн} \cdot \frac{\abs{9\,\text{А} - 3\,\text{А}}}{0{,}3\,\text{c}}
        \approx 100{,}000\,\text{мВ} \to 100
    $
}
\solutionspace{60pt}

\tasknumber{6}%
\task{%
    	В катушке, индуктивность которой равна $90\,\text{мГн}$, течёт электрический ток силой $7\,\text{А}$.
    	Число витков в катушке: {N}.
    Определите магнитный поток, пронизывающий 1 виток катушки.
    	Ответ выразите в милливеберах и округлите до целых.
}
\answer{%
    $
        \Phi_\text{1 виток}
        = \frac{\Phi}{N}
        = \frac{L\eli}{N}
        = \frac{90\,\text{мГн} \cdot 7\,\text{А}}{30}
        \approx 21{,}000\,\text{мВб}
        \to 21
    $
}

\variantsplitter

\addpersonalvariant{Анастасия Князева}

\tasknumber{1}%
\task{%
    Установите каждой букве в соответствие ровно одну цифру и запишите в ответ только цифры (без других символов).

    А) $\Delta t$, Б) $\ele$, В) $\Delta \Phi$.

    1) $\Phi_2 - \Phi_1$, 2) $t_2 - t_1$, 3) $\Phi_1 - \Phi_2$, 4) $-\frac{\Delta \Phi}{\Delta t}$, 5) $t_1 - t_2$.
}
\answer{%
    $241$
}
\solutionspace{20pt}

\tasknumber{2}%
\task{%
    Установите каждой букве в соответствие ровно одну цифру и запишите ответ (только цифры, без других символов).

    А) $\Phi$, Б) $\Delta \eli$, В) $\Delta \Phi$.

    1) $\frac{L}{\eli}$, 2) $\Phi_2 - \Phi_1$, 3) $L\eli$, 4) $\Phi_1 - \Phi_2$, 5) $\eli_2 - \eli_1$, 6) $\eli_1 - \eli_2$.
}
\answer{%
    $352$
}

\tasknumber{3}%
\task{%
    Установите каждой букве в соответствие ровно одну цифру и запишите ответ (только цифры, без других символов).

    А) индукция магнитного поля, Б) электрический заряд, В) электрический ток.

    1) $\eli$, 2) $\vec B$, 3) $R$, 4) $q$, 5) $\phi$.
}
\answer{%
    $241$
}

\tasknumber{4}%
\task{%
    Установите каждой букве в соответствие ровно одну цифру и запишите ответ (только цифры, без других символов).

    А) индукция магнитного поля, Б) индуктивность, В) время.

    1) с, 2) Тл, 3) м, 4) Гн, 5) Вб.
}
\answer{%
    $241$
}

\tasknumber{5}%
\task{%
    	В катушке, индуктивность которой равна $4\,\text{мГн}$, сила тока равномерно уменьшается
    	с $1\,\text{А}$ до $7\,\text{А}$ за $0{,}4\,\text{c}$.
    Опредилите ЭДС самоиндукции, ответ выразите в мВ и округлите до целых.
}
\answer{%
    $
        \ele
        = L\frac{\abs{\Delta \eli}}{\Delta t}
        = L\frac{\abs{\eli_2 - \eli_1}}{\Delta t}
        = 4\,\text{мГн} \cdot \frac{\abs{7\,\text{А} - 1\,\text{А}}}{0{,}4\,\text{c}}
        \approx 60{,}000\,\text{мВ} \to 60
    $
}
\solutionspace{60pt}

\tasknumber{6}%
\task{%
    	В катушке, индуктивность которой равна $50\,\text{мГн}$, течёт электрический ток силой $6\,\text{А}$.
    	Число витков в катушке: {N}.
    Определите магнитный поток, пронизывающий 1 виток катушки.
    	Ответ выразите в милливеберах и округлите до целых.
}
\answer{%
    $
        \Phi_\text{1 виток}
        = \frac{\Phi}{N}
        = \frac{L\eli}{N}
        = \frac{50\,\text{мГн} \cdot 6\,\text{А}}{40}
        \approx 7{,}500\,\text{мВб}
        \to 8
    $
}

\variantsplitter

\addpersonalvariant{Матвей Кузьмин}

\tasknumber{1}%
\task{%
    Установите каждой букве в соответствие ровно одну цифру и запишите в ответ только цифры (без других символов).

    А) $\Delta \Phi$, Б) $\ele$, В) $\Delta t$.

    1) $\Phi_2 - \Phi_1$, 2) $t_1 - t_2$, 3) $t_2 - t_1$, 4) $-\frac{\Delta \Phi}{\Delta t}$, 5) $\Phi_1 - \Phi_2$.
}
\answer{%
    $143$
}
\solutionspace{20pt}

\tasknumber{2}%
\task{%
    Установите каждой букве в соответствие ровно одну цифру и запишите ответ (только цифры, без других символов).

    А) $\Delta \eli$, Б) $\Phi$, В) $\Delta \Phi$.

    1) $\Phi_2 - \Phi_1$, 2) $\eli_2 - \eli_1$, 3) $\frac{L}{\eli}$, 4) $\frac{\eli}{L}$, 5) $L\eli$, 6) $\eli_1 - \eli_2$.
}
\answer{%
    $251$
}

\tasknumber{3}%
\task{%
    Установите каждой букве в соответствие ровно одну цифру и запишите ответ (только цифры, без других символов).

    А) электрический ток, Б) индуктивность, В) индукция магнитного поля.

    1) $\eli$, 2) $R$, 3) $\vec B$, 4) $L$, 5) $\phi$.
}
\answer{%
    $143$
}

\tasknumber{4}%
\task{%
    Установите каждой букве в соответствие ровно одну цифру и запишите ответ (только цифры, без других символов).

    А) поток магнитной индукции, Б) время, В) индукция магнитного поля.

    1) Вб, 2) м, 3) Тл, 4) с, 5) Кл.
}
\answer{%
    $143$
}

\tasknumber{5}%
\task{%
    	В катушке, индуктивность которой равна $6\,\text{мГн}$, сила тока равномерно уменьшается
    	с $4\,\text{А}$ до $8\,\text{А}$ за $0{,}2\,\text{c}$.
    Опредилите ЭДС самоиндукции, ответ выразите в мВ и округлите до целых.
}
\answer{%
    $
        \ele
        = L\frac{\abs{\Delta \eli}}{\Delta t}
        = L\frac{\abs{\eli_2 - \eli_1}}{\Delta t}
        = 6\,\text{мГн} \cdot \frac{\abs{8\,\text{А} - 4\,\text{А}}}{0{,}2\,\text{c}}
        \approx 120{,}000\,\text{мВ} \to 120
    $
}
\solutionspace{60pt}

\tasknumber{6}%
\task{%
    	В катушке, индуктивность которой равна $80\,\text{мГн}$, течёт электрический ток силой $6\,\text{А}$.
    	Число витков в катушке: {N}.
    Определите магнитный поток, пронизывающий 1 виток катушки.
    	Ответ выразите в милливеберах и округлите до целых.
}
\answer{%
    $
        \Phi_\text{1 виток}
        = \frac{\Phi}{N}
        = \frac{L\eli}{N}
        = \frac{80\,\text{мГн} \cdot 6\,\text{А}}{20}
        \approx 24{,}000\,\text{мВб}
        \to 24
    $
}

\variantsplitter

\addpersonalvariant{Елизавета Кутумова}

\tasknumber{1}%
\task{%
    Установите каждой букве в соответствие ровно одну цифру и запишите в ответ только цифры (без других символов).

    А) $\ele$, Б) $\Delta t$, В) $\Delta \Phi$.

    1) $t_2 - t_1$, 2) $\Phi_1 - \Phi_2$, 3) $-\frac{\Delta \Phi}{\Delta t}$, 4) $\Phi_2 - \Phi_1$, 5) $t_1 - t_2$.
}
\answer{%
    $314$
}
\solutionspace{20pt}

\tasknumber{2}%
\task{%
    Установите каждой букве в соответствие ровно одну цифру и запишите ответ (только цифры, без других символов).

    А) $\Delta \eli$, Б) $\Phi$, В) $\Delta \Phi$.

    1) $\eli_2 - \eli_1$, 2) $\Phi_2 - \Phi_1$, 3) $\Phi_1 - \Phi_2$, 4) $L\eli$, 5) $\eli_1 - \eli_2$, 6) $\frac{L}{\eli}$.
}
\answer{%
    $142$
}

\tasknumber{3}%
\task{%
    Установите каждой букве в соответствие ровно одну цифру и запишите ответ (только цифры, без других символов).

    А) поток магнитной индукции, Б) электрический заряд, В) электрический ток.

    1) $q$, 2) $\phi$, 3) $\Phi$, 4) $\eli$, 5) $L$.
}
\answer{%
    $314$
}

\tasknumber{4}%
\task{%
    Установите каждой букве в соответствие ровно одну цифру и запишите ответ (только цифры, без других символов).

    А) индукция магнитного поля, Б) индуктивность, В) время.

    1) Гн, 2) А, 3) Тл, 4) с, 5) Вб.
}
\answer{%
    $314$
}

\tasknumber{5}%
\task{%
    	В катушке, индуктивность которой равна $5\,\text{мГн}$, сила тока равномерно уменьшается
    	с $4\,\text{А}$ до $9\,\text{А}$ за $0{,}5\,\text{c}$.
    Опредилите ЭДС самоиндукции, ответ выразите в мВ и округлите до целых.
}
\answer{%
    $
        \ele
        = L\frac{\abs{\Delta \eli}}{\Delta t}
        = L\frac{\abs{\eli_2 - \eli_1}}{\Delta t}
        = 5\,\text{мГн} \cdot \frac{\abs{9\,\text{А} - 4\,\text{А}}}{0{,}5\,\text{c}}
        \approx 50{,}000\,\text{мВ} \to 50
    $
}
\solutionspace{60pt}

\tasknumber{6}%
\task{%
    	В катушке, индуктивность которой равна $70\,\text{мГн}$, течёт электрический ток силой $7\,\text{А}$.
    	Число витков в катушке: {N}.
    Определите магнитный поток, пронизывающий 1 виток катушки.
    	Ответ выразите в милливеберах и округлите до целых.
}
\answer{%
    $
        \Phi_\text{1 виток}
        = \frac{\Phi}{N}
        = \frac{L\eli}{N}
        = \frac{70\,\text{мГн} \cdot 7\,\text{А}}{40}
        \approx 12{,}250\,\text{мВб}
        \to 12
    $
}

\variantsplitter

\addpersonalvariant{Роксана Мехтиева}

\tasknumber{1}%
\task{%
    Установите каждой букве в соответствие ровно одну цифру и запишите в ответ только цифры (без других символов).

    А) $\Delta \Phi$, Б) $\ele$, В) $\Delta t$.

    1) $t_1 - t_2$, 2) $\Phi_2 - \Phi_1$, 3) $-\frac{\Delta \Phi}{\Delta t}$, 4) $t_2 - t_1$, 5) $\Phi_1 - \Phi_2$.
}
\answer{%
    $234$
}
\solutionspace{20pt}

\tasknumber{2}%
\task{%
    Установите каждой букве в соответствие ровно одну цифру и запишите ответ (только цифры, без других символов).

    А) $\Delta \Phi$, Б) $\Delta \eli$, В) $\Phi$.

    1) $\eli_1 - \eli_2$, 2) $\Phi_1 - \Phi_2$, 3) $\Phi_2 - \Phi_1$, 4) $\eli_2 - \eli_1$, 5) $L\eli$, 6) $\frac{\eli}{L}$.
}
\answer{%
    $345$
}

\tasknumber{3}%
\task{%
    Установите каждой букве в соответствие ровно одну цифру и запишите ответ (только цифры, без других символов).

    А) индукция магнитного поля, Б) поток магнитной индукции, В) электрический заряд.

    1) $R$, 2) $\vec B$, 3) $\Phi$, 4) $q$, 5) $\eli$.
}
\answer{%
    $234$
}

\tasknumber{4}%
\task{%
    Установите каждой букве в соответствие ровно одну цифру и запишите ответ (только цифры, без других символов).

    А) индуктивность, Б) поток магнитной индукции, В) время.

    1) м, 2) Гн, 3) Вб, 4) с, 5) А.
}
\answer{%
    $234$
}

\tasknumber{5}%
\task{%
    	В катушке, индуктивность которой равна $7\,\text{мГн}$, сила тока равномерно уменьшается
    	с $3\,\text{А}$ до $9\,\text{А}$ за $0{,}4\,\text{c}$.
    Опредилите ЭДС самоиндукции, ответ выразите в мВ и округлите до целых.
}
\answer{%
    $
        \ele
        = L\frac{\abs{\Delta \eli}}{\Delta t}
        = L\frac{\abs{\eli_2 - \eli_1}}{\Delta t}
        = 7\,\text{мГн} \cdot \frac{\abs{9\,\text{А} - 3\,\text{А}}}{0{,}4\,\text{c}}
        \approx 105{,}000\,\text{мВ} \to 105
    $
}
\solutionspace{60pt}

\tasknumber{6}%
\task{%
    	В катушке, индуктивность которой равна $50\,\text{мГн}$, течёт электрический ток силой $6\,\text{А}$.
    	Число витков в катушке: {N}.
    Определите магнитный поток, пронизывающий 1 виток катушки.
    	Ответ выразите в милливеберах и округлите до целых.
}
\answer{%
    $
        \Phi_\text{1 виток}
        = \frac{\Phi}{N}
        = \frac{L\eli}{N}
        = \frac{50\,\text{мГн} \cdot 6\,\text{А}}{30}
        \approx 10{,}000\,\text{мВб}
        \to 10
    $
}

\variantsplitter

\addpersonalvariant{Дилноза Нодиршоева}

\tasknumber{1}%
\task{%
    Установите каждой букве в соответствие ровно одну цифру и запишите в ответ только цифры (без других символов).

    А) $\ele$, Б) $\Delta \Phi$, В) $\Delta t$.

    1) $t_1 - t_2$, 2) $-\frac{\Delta \Phi}{\Delta t}$, 3) $\Phi_2 - \Phi_1$, 4) $t_2 - t_1$, 5) $\Phi_1 - \Phi_2$.
}
\answer{%
    $234$
}
\solutionspace{20pt}

\tasknumber{2}%
\task{%
    Установите каждой букве в соответствие ровно одну цифру и запишите ответ (только цифры, без других символов).

    А) $\Delta \eli$, Б) $\Phi$, В) $\Delta \Phi$.

    1) $\frac{\eli}{L}$, 2) $\eli_1 - \eli_2$, 3) $\eli_2 - \eli_1$, 4) $L\eli$, 5) $\Phi_2 - \Phi_1$, 6) $\Phi_1 - \Phi_2$.
}
\answer{%
    $345$
}

\tasknumber{3}%
\task{%
    Установите каждой букве в соответствие ровно одну цифру и запишите ответ (только цифры, без других символов).

    А) электрический заряд, Б) индуктивность, В) индукция магнитного поля.

    1) $R$, 2) $q$, 3) $L$, 4) $\vec B$, 5) $g$.
}
\answer{%
    $234$
}

\tasknumber{4}%
\task{%
    Установите каждой букве в соответствие ровно одну цифру и запишите ответ (только цифры, без других символов).

    А) поток магнитной индукции, Б) время, В) индукция магнитного поля.

    1) Гн, 2) Вб, 3) с, 4) Тл, 5) А.
}
\answer{%
    $234$
}

\tasknumber{5}%
\task{%
    	В катушке, индуктивность которой равна $5\,\text{мГн}$, сила тока равномерно уменьшается
    	с $2\,\text{А}$ до $7\,\text{А}$ за $0{,}2\,\text{c}$.
    Опредилите ЭДС самоиндукции, ответ выразите в мВ и округлите до целых.
}
\answer{%
    $
        \ele
        = L\frac{\abs{\Delta \eli}}{\Delta t}
        = L\frac{\abs{\eli_2 - \eli_1}}{\Delta t}
        = 5\,\text{мГн} \cdot \frac{\abs{7\,\text{А} - 2\,\text{А}}}{0{,}2\,\text{c}}
        \approx 125{,}000\,\text{мВ} \to 125
    $
}
\solutionspace{60pt}

\tasknumber{6}%
\task{%
    	В катушке, индуктивность которой равна $90\,\text{мГн}$, течёт электрический ток силой $7\,\text{А}$.
    	Число витков в катушке: {N}.
    Определите магнитный поток, пронизывающий 1 виток катушки.
    	Ответ выразите в милливеберах и округлите до целых.
}
\answer{%
    $
        \Phi_\text{1 виток}
        = \frac{\Phi}{N}
        = \frac{L\eli}{N}
        = \frac{90\,\text{мГн} \cdot 7\,\text{А}}{30}
        \approx 21{,}000\,\text{мВб}
        \to 21
    $
}

\variantsplitter

\addpersonalvariant{Артём Переверзев}

\tasknumber{1}%
\task{%
    Установите каждой букве в соответствие ровно одну цифру и запишите в ответ только цифры (без других символов).

    А) $\Delta t$, Б) $\Delta \Phi$, В) $\ele$.

    1) $t_2 - t_1$, 2) $\Phi_1 - \Phi_2$, 3) $t_1 - t_2$, 4) $-\frac{\Delta \Phi}{\Delta t}$, 5) $\Phi_2 - \Phi_1$.
}
\answer{%
    $154$
}
\solutionspace{20pt}

\tasknumber{2}%
\task{%
    Установите каждой букве в соответствие ровно одну цифру и запишите ответ (только цифры, без других символов).

    А) $\Phi$, Б) $\Delta \Phi$, В) $\Delta \eli$.

    1) $\Phi_2 - \Phi_1$, 2) $L\eli$, 3) $\frac{\eli}{L}$, 4) $\frac{L}{\eli}$, 5) $\Phi_1 - \Phi_2$, 6) $\eli_2 - \eli_1$.
}
\answer{%
    $216$
}

\tasknumber{3}%
\task{%
    Установите каждой букве в соответствие ровно одну цифру и запишите ответ (только цифры, без других символов).

    А) индуктивность, Б) индукция магнитного поля, В) поток магнитной индукции.

    1) $L$, 2) $g$, 3) $q$, 4) $\Phi$, 5) $\vec B$.
}
\answer{%
    $154$
}

\tasknumber{4}%
\task{%
    Установите каждой букве в соответствие ровно одну цифру и запишите ответ (только цифры, без других символов).

    А) индуктивность, Б) индукция магнитного поля, В) время.

    1) Гн, 2) Вб, 3) Кл, 4) с, 5) Тл.
}
\answer{%
    $154$
}

\tasknumber{5}%
\task{%
    	В катушке, индуктивность которой равна $4\,\text{мГн}$, сила тока равномерно уменьшается
    	с $4\,\text{А}$ до $7\,\text{А}$ за $0{,}2\,\text{c}$.
    Опредилите ЭДС самоиндукции, ответ выразите в мВ и округлите до целых.
}
\answer{%
    $
        \ele
        = L\frac{\abs{\Delta \eli}}{\Delta t}
        = L\frac{\abs{\eli_2 - \eli_1}}{\Delta t}
        = 4\,\text{мГн} \cdot \frac{\abs{7\,\text{А} - 4\,\text{А}}}{0{,}2\,\text{c}}
        \approx 60{,}000\,\text{мВ} \to 60
    $
}
\solutionspace{60pt}

\tasknumber{6}%
\task{%
    	В катушке, индуктивность которой равна $90\,\text{мГн}$, течёт электрический ток силой $5\,\text{А}$.
    	Число витков в катушке: {N}.
    Определите магнитный поток, пронизывающий 1 виток катушки.
    	Ответ выразите в милливеберах и округлите до целых.
}
\answer{%
    $
        \Phi_\text{1 виток}
        = \frac{\Phi}{N}
        = \frac{L\eli}{N}
        = \frac{90\,\text{мГн} \cdot 5\,\text{А}}{30}
        \approx 15{,}000\,\text{мВб}
        \to 15
    $
}

\variantsplitter

\addpersonalvariant{Варвара Пранова}

\tasknumber{1}%
\task{%
    Установите каждой букве в соответствие ровно одну цифру и запишите в ответ только цифры (без других символов).

    А) $\Delta \Phi$, Б) $\ele$, В) $\Delta t$.

    1) $t_2 - t_1$, 2) $\Phi_1 - \Phi_2$, 3) $\Phi_2 - \Phi_1$, 4) $-\frac{\Delta \Phi}{\Delta t}$, 5) $t_1 - t_2$.
}
\answer{%
    $341$
}
\solutionspace{20pt}

\tasknumber{2}%
\task{%
    Установите каждой букве в соответствие ровно одну цифру и запишите ответ (только цифры, без других символов).

    А) $\Phi$, Б) $\Delta \Phi$, В) $\Delta \eli$.

    1) $\Phi_1 - \Phi_2$, 2) $\eli_2 - \eli_1$, 3) $\frac{\eli}{L}$, 4) $L\eli$, 5) $\Phi_2 - \Phi_1$, 6) $\frac{L}{\eli}$.
}
\answer{%
    $452$
}

\tasknumber{3}%
\task{%
    Установите каждой букве в соответствие ровно одну цифру и запишите ответ (только цифры, без других символов).

    А) индуктивность, Б) поток магнитной индукции, В) электрический заряд.

    1) $q$, 2) $\eli$, 3) $L$, 4) $\Phi$, 5) $g$.
}
\answer{%
    $341$
}

\tasknumber{4}%
\task{%
    Установите каждой букве в соответствие ровно одну цифру и запишите ответ (только цифры, без других символов).

    А) индуктивность, Б) индукция магнитного поля, В) время.

    1) с, 2) А, 3) Гн, 4) Тл, 5) м.
}
\answer{%
    $341$
}

\tasknumber{5}%
\task{%
    	В катушке, индуктивность которой равна $5\,\text{мГн}$, сила тока равномерно уменьшается
    	с $4\,\text{А}$ до $8\,\text{А}$ за $0{,}3\,\text{c}$.
    Опредилите ЭДС самоиндукции, ответ выразите в мВ и округлите до целых.
}
\answer{%
    $
        \ele
        = L\frac{\abs{\Delta \eli}}{\Delta t}
        = L\frac{\abs{\eli_2 - \eli_1}}{\Delta t}
        = 5\,\text{мГн} \cdot \frac{\abs{8\,\text{А} - 4\,\text{А}}}{0{,}3\,\text{c}}
        \approx 66{,}667\,\text{мВ} \to 67
    $
}
\solutionspace{60pt}

\tasknumber{6}%
\task{%
    	В катушке, индуктивность которой равна $90\,\text{мГн}$, течёт электрический ток силой $7\,\text{А}$.
    	Число витков в катушке: {N}.
    Определите магнитный поток, пронизывающий 1 виток катушки.
    	Ответ выразите в милливеберах и округлите до целых.
}
\answer{%
    $
        \Phi_\text{1 виток}
        = \frac{\Phi}{N}
        = \frac{L\eli}{N}
        = \frac{90\,\text{мГн} \cdot 7\,\text{А}}{20}
        \approx 31{,}500\,\text{мВб}
        \to 32
    $
}

\variantsplitter

\addpersonalvariant{Марьям Салимова}

\tasknumber{1}%
\task{%
    Установите каждой букве в соответствие ровно одну цифру и запишите в ответ только цифры (без других символов).

    А) $\Delta \Phi$, Б) $\ele$, В) $\Delta t$.

    1) $t_2 - t_1$, 2) $\Phi_1 - \Phi_2$, 3) $t_1 - t_2$, 4) $\Phi_2 - \Phi_1$, 5) $-\frac{\Delta \Phi}{\Delta t}$.
}
\answer{%
    $451$
}
\solutionspace{20pt}

\tasknumber{2}%
\task{%
    Установите каждой букве в соответствие ровно одну цифру и запишите ответ (только цифры, без других символов).

    А) $\Delta \Phi$, Б) $\Delta \eli$, В) $\Phi$.

    1) $\eli_1 - \eli_2$, 2) $L\eli$, 3) $\frac{\eli}{L}$, 4) $\Phi_1 - \Phi_2$, 5) $\Phi_2 - \Phi_1$, 6) $\eli_2 - \eli_1$.
}
\answer{%
    $562$
}

\tasknumber{3}%
\task{%
    Установите каждой букве в соответствие ровно одну цифру и запишите ответ (только цифры, без других символов).

    А) поток магнитной индукции, Б) электрический заряд, В) индукция магнитного поля.

    1) $\vec B$, 2) $R$, 3) $\eli$, 4) $\Phi$, 5) $q$.
}
\answer{%
    $451$
}

\tasknumber{4}%
\task{%
    Установите каждой букве в соответствие ровно одну цифру и запишите ответ (только цифры, без других символов).

    А) индуктивность, Б) поток магнитной индукции, В) время.

    1) с, 2) А, 3) м, 4) Гн, 5) Вб.
}
\answer{%
    $451$
}

\tasknumber{5}%
\task{%
    	В катушке, индуктивность которой равна $7\,\text{мГн}$, сила тока равномерно уменьшается
    	с $2\,\text{А}$ до $9\,\text{А}$ за $0{,}3\,\text{c}$.
    Опредилите ЭДС самоиндукции, ответ выразите в мВ и округлите до целых.
}
\answer{%
    $
        \ele
        = L\frac{\abs{\Delta \eli}}{\Delta t}
        = L\frac{\abs{\eli_2 - \eli_1}}{\Delta t}
        = 7\,\text{мГн} \cdot \frac{\abs{9\,\text{А} - 2\,\text{А}}}{0{,}3\,\text{c}}
        \approx 163{,}333\,\text{мВ} \to 163
    $
}
\solutionspace{60pt}

\tasknumber{6}%
\task{%
    	В катушке, индуктивность которой равна $50\,\text{мГн}$, течёт электрический ток силой $6\,\text{А}$.
    	Число витков в катушке: {N}.
    Определите магнитный поток, пронизывающий 1 виток катушки.
    	Ответ выразите в милливеберах и округлите до целых.
}
\answer{%
    $
        \Phi_\text{1 виток}
        = \frac{\Phi}{N}
        = \frac{L\eli}{N}
        = \frac{50\,\text{мГн} \cdot 6\,\text{А}}{20}
        \approx 15{,}000\,\text{мВб}
        \to 15
    $
}

\variantsplitter

\addpersonalvariant{Юлия Шевченко}

\tasknumber{1}%
\task{%
    Установите каждой букве в соответствие ровно одну цифру и запишите в ответ только цифры (без других символов).

    А) $\Delta \Phi$, Б) $\ele$, В) $\Delta t$.

    1) $\Phi_2 - \Phi_1$, 2) $\Phi_1 - \Phi_2$, 3) $-\frac{\Delta \Phi}{\Delta t}$, 4) $t_2 - t_1$, 5) $t_1 - t_2$.
}
\answer{%
    $134$
}
\solutionspace{20pt}

\tasknumber{2}%
\task{%
    Установите каждой букве в соответствие ровно одну цифру и запишите ответ (только цифры, без других символов).

    А) $\Phi$, Б) $\Delta \Phi$, В) $\Delta \eli$.

    1) $\eli_2 - \eli_1$, 2) $L\eli$, 3) $\eli_1 - \eli_2$, 4) $\Phi_2 - \Phi_1$, 5) $\frac{\eli}{L}$, 6) $\frac{L}{\eli}$.
}
\answer{%
    $241$
}

\tasknumber{3}%
\task{%
    Установите каждой букве в соответствие ровно одну цифру и запишите ответ (только цифры, без других символов).

    А) поток магнитной индукции, Б) электрический ток, В) индукция магнитного поля.

    1) $\Phi$, 2) $\phi$, 3) $\eli$, 4) $\vec B$, 5) $L$.
}
\answer{%
    $134$
}

\tasknumber{4}%
\task{%
    Установите каждой букве в соответствие ровно одну цифру и запишите ответ (только цифры, без других символов).

    А) время, Б) индуктивность, В) индукция магнитного поля.

    1) с, 2) Кл, 3) Гн, 4) Тл, 5) м.
}
\answer{%
    $134$
}

\tasknumber{5}%
\task{%
    	В катушке, индуктивность которой равна $6\,\text{мГн}$, сила тока равномерно уменьшается
    	с $2\,\text{А}$ до $8\,\text{А}$ за $0{,}2\,\text{c}$.
    Опредилите ЭДС самоиндукции, ответ выразите в мВ и округлите до целых.
}
\answer{%
    $
        \ele
        = L\frac{\abs{\Delta \eli}}{\Delta t}
        = L\frac{\abs{\eli_2 - \eli_1}}{\Delta t}
        = 6\,\text{мГн} \cdot \frac{\abs{8\,\text{А} - 2\,\text{А}}}{0{,}2\,\text{c}}
        \approx 180{,}000\,\text{мВ} \to 180
    $
}
\solutionspace{60pt}

\tasknumber{6}%
\task{%
    	В катушке, индуктивность которой равна $50\,\text{мГн}$, течёт электрический ток силой $6\,\text{А}$.
    	Число витков в катушке: {N}.
    Определите магнитный поток, пронизывающий 1 виток катушки.
    	Ответ выразите в милливеберах и округлите до целых.
}
\answer{%
    $
        \Phi_\text{1 виток}
        = \frac{\Phi}{N}
        = \frac{L\eli}{N}
        = \frac{50\,\text{мГн} \cdot 6\,\text{А}}{40}
        \approx 7{,}500\,\text{мВб}
        \to 8
    $
}
% autogenerated
