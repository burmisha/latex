\setdate{28~сентября~2021}
\setclass{11«БА»}

\addpersonalvariant{Михаил Бурмистров}

\tasknumber{1}%
\task{%
    Установите каждой букве в соответствие ровно одну цифру и запишите в ответ только цифры (без других символов).

    А) $\Delta t$, Б) $\Delta \Phi$, В) $\ele$.

    1) $\Phi_2 - \Phi_1$, 2) $\Phi_1 - \Phi_2$, 3) $t_2 - t_1$, 4) $-\frac{\Delta \Phi}{\Delta t}$, 5) $t_1 - t_2$.
}
\answer{%
    $314$
}
\solutionspace{20pt}

\tasknumber{2}%
\task{%
    Установите каждой букве в соответствие ровно одну цифру и запишите ответ (только цифры, без других символов).

    А) $\Delta \eli$, Б) $\Delta \Phi$, В) $\Phi$.

    1) $\eli_2 - \eli_1$, 2) $L\eli$, 3) $\eli_1 - \eli_2$, 4) $\Phi_2 - \Phi_1$, 5) $\Phi_1 - \Phi_2$, 6) $\frac{\eli}{L}$.
}
\answer{%
    $142$
}
\solutionspace{10pt}

\tasknumber{3}%
\task{%
    Установите каждой букве в соответствие ровно одну цифру и запишите ответ (только цифры, без других символов).

    А) поток магнитной индукции, Б) индукция магнитного поля, В) индуктивность.

    1) $\vec B$, 2) $\ele$, 3) $\Phi$, 4) $L$, 5) $\eli$.
}
\answer{%
    $314$
}
\solutionspace{10pt}

\tasknumber{4}%
\task{%
    Установите каждой букве в соответствие ровно одну цифру и запишите ответ (только цифры, без других символов).

    А) индуктивность, Б) поток магнитной индукции.

    1) Кл, 2) Гн, 3) Вб, 4) Вт.
}
\answer{%
    $23$
}
\solutionspace{10pt}

\tasknumber{5}%
\task{%
    	В катушке, индуктивность которой равна $7\,\text{мГн}$, сила тока равномерно уменьшается
    	с $9\,\text{А}$ до $2\,\text{А}$ за $0{,}5\,\text{c}$.
    Определите ЭДС самоиндукции, ответ выразите в мВ и округлите до целых.
}
\answer{%
    $
        \ele
        = L\frac{\abs{\Delta \eli}}{\Delta t}
        = L\frac{\abs{\eli_2 - \eli_1}}{\Delta t}
        = 7\,\text{мГн} \cdot \frac{\abs{2\,\text{А} - 9\,\text{А}}}{0{,}5\,\text{c}}
        \approx 98{,}000\,\text{мВ} \to 98
    $
}
\solutionspace{60pt}

\tasknumber{6}%
\task{%
    	В катушке, индуктивность которой равна $70\,\text{мГн}$, течёт электрический ток силой $6\,\text{А}$.
    	Число витков в катушке: 30.
    Определите магнитный поток, пронизывающий 1 виток катушки.
    	Ответ выразите в милливеберах и округлите до целых.
}
\answer{%
    $
        \Phi_\text{1 виток}
        = \frac{\Phi}{N}
        = \frac{L\eli}{N}
        = \frac{70\,\text{мГн} \cdot 6\,\text{А}}{30}
        \approx 14{,}000\,\text{мВб}
        \to 14
    $
}

\variantsplitter

\addpersonalvariant{Ирина Ан}

\tasknumber{1}%
\task{%
    Установите каждой букве в соответствие ровно одну цифру и запишите в ответ только цифры (без других символов).

    А) $\ele$, Б) $\Delta t$, В) $\Delta \Phi$.

    1) $\Phi_1 - \Phi_2$, 2) $t_2 - t_1$, 3) $\Phi_2 - \Phi_1$, 4) $-\frac{\Delta \Phi}{\Delta t}$, 5) $t_1 - t_2$.
}
\answer{%
    $423$
}
\solutionspace{20pt}

\tasknumber{2}%
\task{%
    Установите каждой букве в соответствие ровно одну цифру и запишите ответ (только цифры, без других символов).

    А) $\Delta \Phi$, Б) $\Delta \eli$, В) $\Phi$.

    1) $\eli_2 - \eli_1$, 2) $\Phi_1 - \Phi_2$, 3) $L\eli$, 4) $\frac{L}{\eli}$, 5) $\Phi_2 - \Phi_1$, 6) $\frac{\eli}{L}$.
}
\answer{%
    $513$
}
\solutionspace{10pt}

\tasknumber{3}%
\task{%
    Установите каждой букве в соответствие ровно одну цифру и запишите ответ (только цифры, без других символов).

    А) электрический ток, Б) индуктивность, В) электрический заряд.

    1) $\ele$, 2) $L$, 3) $q$, 4) $\eli$, 5) $R$.
}
\answer{%
    $423$
}
\solutionspace{10pt}

\tasknumber{4}%
\task{%
    Установите каждой букве в соответствие ровно одну цифру и запишите ответ (только цифры, без других символов).

    А) поток магнитной индукции, Б) индуктивность.

    1) Гн, 2) с, 3) Вб, 4) Тл.
}
\answer{%
    $31$
}
\solutionspace{10pt}

\tasknumber{5}%
\task{%
    	В катушке, индуктивность которой равна $4\,\text{мГн}$, сила тока равномерно уменьшается
    	с $9\,\text{А}$ до $3\,\text{А}$ за $0{,}3\,\text{c}$.
    Определите ЭДС самоиндукции, ответ выразите в мВ и округлите до целых.
}
\answer{%
    $
        \ele
        = L\frac{\abs{\Delta \eli}}{\Delta t}
        = L\frac{\abs{\eli_2 - \eli_1}}{\Delta t}
        = 4\,\text{мГн} \cdot \frac{\abs{3\,\text{А} - 9\,\text{А}}}{0{,}3\,\text{c}}
        \approx 80{,}000\,\text{мВ} \to 80
    $
}
\solutionspace{60pt}

\tasknumber{6}%
\task{%
    	В катушке, индуктивность которой равна $60\,\text{мГн}$, течёт электрический ток силой $7\,\text{А}$.
    	Число витков в катушке: 20.
    Определите магнитный поток, пронизывающий 1 виток катушки.
    	Ответ выразите в милливеберах и округлите до целых.
}
\answer{%
    $
        \Phi_\text{1 виток}
        = \frac{\Phi}{N}
        = \frac{L\eli}{N}
        = \frac{60\,\text{мГн} \cdot 7\,\text{А}}{20}
        \approx 21{,}000\,\text{мВб}
        \to 21
    $
}

\variantsplitter

\addpersonalvariant{Софья Андрианова}

\tasknumber{1}%
\task{%
    Установите каждой букве в соответствие ровно одну цифру и запишите в ответ только цифры (без других символов).

    А) $\Delta t$, Б) $\ele$, В) $\Delta \Phi$.

    1) $t_1 - t_2$, 2) $\Phi_1 - \Phi_2$, 3) $t_2 - t_1$, 4) $\Phi_2 - \Phi_1$, 5) $-\frac{\Delta \Phi}{\Delta t}$.
}
\answer{%
    $354$
}
\solutionspace{20pt}

\tasknumber{2}%
\task{%
    Установите каждой букве в соответствие ровно одну цифру и запишите ответ (только цифры, без других символов).

    А) $\Delta \Phi$, Б) $\Delta \eli$, В) $\Phi$.

    1) $\frac{L}{\eli}$, 2) $\Phi_1 - \Phi_2$, 3) $\frac{\eli}{L}$, 4) $\Phi_2 - \Phi_1$, 5) $L\eli$, 6) $\eli_2 - \eli_1$.
}
\answer{%
    $465$
}
\solutionspace{10pt}

\tasknumber{3}%
\task{%
    Установите каждой букве в соответствие ровно одну цифру и запишите ответ (только цифры, без других символов).

    А) электрический заряд, Б) индуктивность, В) индукция магнитного поля.

    1) $\eli$, 2) $g$, 3) $q$, 4) $\vec B$, 5) $L$.
}
\answer{%
    $354$
}
\solutionspace{10pt}

\tasknumber{4}%
\task{%
    Установите каждой букве в соответствие ровно одну цифру и запишите ответ (только цифры, без других символов).

    А) поток магнитной индукции, Б) индукция магнитного поля.

    1) А, 2) Вб, 3) с, 4) Тл.
}
\answer{%
    $24$
}
\solutionspace{10pt}

\tasknumber{5}%
\task{%
    	В катушке, индуктивность которой равна $7\,\text{мГн}$, сила тока равномерно уменьшается
    	с $7\,\text{А}$ до $1\,\text{А}$ за $0{,}2\,\text{c}$.
    Определите ЭДС самоиндукции, ответ выразите в мВ и округлите до целых.
}
\answer{%
    $
        \ele
        = L\frac{\abs{\Delta \eli}}{\Delta t}
        = L\frac{\abs{\eli_2 - \eli_1}}{\Delta t}
        = 7\,\text{мГн} \cdot \frac{\abs{1\,\text{А} - 7\,\text{А}}}{0{,}2\,\text{c}}
        \approx 210{,}000\,\text{мВ} \to 210
    $
}
\solutionspace{60pt}

\tasknumber{6}%
\task{%
    	В катушке, индуктивность которой равна $70\,\text{мГн}$, течёт электрический ток силой $6\,\text{А}$.
    	Число витков в катушке: 30.
    Определите магнитный поток, пронизывающий 1 виток катушки.
    	Ответ выразите в милливеберах и округлите до целых.
}
\answer{%
    $
        \Phi_\text{1 виток}
        = \frac{\Phi}{N}
        = \frac{L\eli}{N}
        = \frac{70\,\text{мГн} \cdot 6\,\text{А}}{30}
        \approx 14{,}000\,\text{мВб}
        \to 14
    $
}

\variantsplitter

\addpersonalvariant{Владимир Артемчук}

\tasknumber{1}%
\task{%
    Установите каждой букве в соответствие ровно одну цифру и запишите в ответ только цифры (без других символов).

    А) $\Delta \Phi$, Б) $\Delta t$, В) $\ele$.

    1) $\Phi_2 - \Phi_1$, 2) $t_2 - t_1$, 3) $\Phi_1 - \Phi_2$, 4) $-\frac{\Delta \Phi}{\Delta t}$, 5) $t_1 - t_2$.
}
\answer{%
    $124$
}
\solutionspace{20pt}

\tasknumber{2}%
\task{%
    Установите каждой букве в соответствие ровно одну цифру и запишите ответ (только цифры, без других символов).

    А) $\Delta \Phi$, Б) $\Delta \eli$, В) $\Phi$.

    1) $\Phi_1 - \Phi_2$, 2) $\Phi_2 - \Phi_1$, 3) $\eli_2 - \eli_1$, 4) $\eli_1 - \eli_2$, 5) $L\eli$, 6) $\frac{\eli}{L}$.
}
\answer{%
    $235$
}
\solutionspace{10pt}

\tasknumber{3}%
\task{%
    Установите каждой букве в соответствие ровно одну цифру и запишите ответ (только цифры, без других символов).

    А) электрический заряд, Б) поток магнитной индукции, В) индукция магнитного поля.

    1) $q$, 2) $\Phi$, 3) $g$, 4) $\vec B$, 5) $\eli$.
}
\answer{%
    $124$
}
\solutionspace{10pt}

\tasknumber{4}%
\task{%
    Установите каждой букве в соответствие ровно одну цифру и запишите ответ (только цифры, без других символов).

    А) длина проводника, Б) индуктивность.

    1) м, 2) А, 3) Гн, 4) Кл.
}
\answer{%
    $13$
}
\solutionspace{10pt}

\tasknumber{5}%
\task{%
    	В катушке, индуктивность которой равна $5\,\text{мГн}$, сила тока равномерно уменьшается
    	с $7\,\text{А}$ до $2\,\text{А}$ за $0{,}5\,\text{c}$.
    Определите ЭДС самоиндукции, ответ выразите в мВ и округлите до целых.
}
\answer{%
    $
        \ele
        = L\frac{\abs{\Delta \eli}}{\Delta t}
        = L\frac{\abs{\eli_2 - \eli_1}}{\Delta t}
        = 5\,\text{мГн} \cdot \frac{\abs{2\,\text{А} - 7\,\text{А}}}{0{,}5\,\text{c}}
        \approx 50{,}000\,\text{мВ} \to 50
    $
}
\solutionspace{60pt}

\tasknumber{6}%
\task{%
    	В катушке, индуктивность которой равна $70\,\text{мГн}$, течёт электрический ток силой $5\,\text{А}$.
    	Число витков в катушке: 40.
    Определите магнитный поток, пронизывающий 1 виток катушки.
    	Ответ выразите в милливеберах и округлите до целых.
}
\answer{%
    $
        \Phi_\text{1 виток}
        = \frac{\Phi}{N}
        = \frac{L\eli}{N}
        = \frac{70\,\text{мГн} \cdot 5\,\text{А}}{40}
        \approx 8{,}750\,\text{мВб}
        \to 9
    $
}

\variantsplitter

\addpersonalvariant{Софья Белянкина}

\tasknumber{1}%
\task{%
    Установите каждой букве в соответствие ровно одну цифру и запишите в ответ только цифры (без других символов).

    А) $\Delta \Phi$, Б) $\ele$, В) $\Delta t$.

    1) $t_1 - t_2$, 2) $-\frac{\Delta \Phi}{\Delta t}$, 3) $\Phi_2 - \Phi_1$, 4) $t_2 - t_1$, 5) $\Phi_1 - \Phi_2$.
}
\answer{%
    $324$
}
\solutionspace{20pt}

\tasknumber{2}%
\task{%
    Установите каждой букве в соответствие ровно одну цифру и запишите ответ (только цифры, без других символов).

    А) $\Delta \eli$, Б) $\Phi$, В) $\Delta \Phi$.

    1) $\Phi_1 - \Phi_2$, 2) $\frac{L}{\eli}$, 3) $L\eli$, 4) $\eli_2 - \eli_1$, 5) $\Phi_2 - \Phi_1$, 6) $\frac{\eli}{L}$.
}
\answer{%
    $435$
}
\solutionspace{10pt}

\tasknumber{3}%
\task{%
    Установите каждой букве в соответствие ровно одну цифру и запишите ответ (только цифры, без других символов).

    А) электрический заряд, Б) поток магнитной индукции, В) индукция магнитного поля.

    1) $L$, 2) $\Phi$, 3) $q$, 4) $\vec B$, 5) $\varphi$.
}
\answer{%
    $324$
}
\solutionspace{10pt}

\tasknumber{4}%
\task{%
    Установите каждой букве в соответствие ровно одну цифру и запишите ответ (только цифры, без других символов).

    А) индуктивность, Б) время.

    1) с, 2) Гн, 3) Вб, 4) А.
}
\answer{%
    $21$
}
\solutionspace{10pt}

\tasknumber{5}%
\task{%
    	В катушке, индуктивность которой равна $5\,\text{мГн}$, сила тока равномерно уменьшается
    	с $8\,\text{А}$ до $4\,\text{А}$ за $0{,}2\,\text{c}$.
    Определите ЭДС самоиндукции, ответ выразите в мВ и округлите до целых.
}
\answer{%
    $
        \ele
        = L\frac{\abs{\Delta \eli}}{\Delta t}
        = L\frac{\abs{\eli_2 - \eli_1}}{\Delta t}
        = 5\,\text{мГн} \cdot \frac{\abs{4\,\text{А} - 8\,\text{А}}}{0{,}2\,\text{c}}
        \approx 100{,}000\,\text{мВ} \to 100
    $
}
\solutionspace{60pt}

\tasknumber{6}%
\task{%
    	В катушке, индуктивность которой равна $50\,\text{мГн}$, течёт электрический ток силой $6\,\text{А}$.
    	Число витков в катушке: 20.
    Определите магнитный поток, пронизывающий 1 виток катушки.
    	Ответ выразите в милливеберах и округлите до целых.
}
\answer{%
    $
        \Phi_\text{1 виток}
        = \frac{\Phi}{N}
        = \frac{L\eli}{N}
        = \frac{50\,\text{мГн} \cdot 6\,\text{А}}{20}
        \approx 15{,}000\,\text{мВб}
        \to 15
    $
}

\variantsplitter

\addpersonalvariant{Варвара Егиазарян}

\tasknumber{1}%
\task{%
    Установите каждой букве в соответствие ровно одну цифру и запишите в ответ только цифры (без других символов).

    А) $\Delta t$, Б) $\Delta \Phi$, В) $\ele$.

    1) $\Phi_1 - \Phi_2$, 2) $\Phi_2 - \Phi_1$, 3) $-\frac{\Delta \Phi}{\Delta t}$, 4) $t_2 - t_1$, 5) $t_1 - t_2$.
}
\answer{%
    $423$
}
\solutionspace{20pt}

\tasknumber{2}%
\task{%
    Установите каждой букве в соответствие ровно одну цифру и запишите ответ (только цифры, без других символов).

    А) $\Phi$, Б) $\Delta \Phi$, В) $\Delta \eli$.

    1) $\Phi_2 - \Phi_1$, 2) $\eli_1 - \eli_2$, 3) $\eli_2 - \eli_1$, 4) $\frac{\eli}{L}$, 5) $L\eli$, 6) $\frac{L}{\eli}$.
}
\answer{%
    $513$
}
\solutionspace{10pt}

\tasknumber{3}%
\task{%
    Установите каждой букве в соответствие ровно одну цифру и запишите ответ (только цифры, без других символов).

    А) электрический ток, Б) индуктивность, В) поток магнитной индукции.

    1) $\varphi$, 2) $L$, 3) $\Phi$, 4) $\eli$, 5) $R$.
}
\answer{%
    $423$
}
\solutionspace{10pt}

\tasknumber{4}%
\task{%
    Установите каждой букве в соответствие ровно одну цифру и запишите ответ (только цифры, без других символов).

    А) длина проводника, Б) поток магнитной индукции.

    1) Вб, 2) м / с, 3) м, 4) Кл.
}
\answer{%
    $31$
}
\solutionspace{10pt}

\tasknumber{5}%
\task{%
    	В катушке, индуктивность которой равна $6\,\text{мГн}$, сила тока равномерно уменьшается
    	с $9\,\text{А}$ до $4\,\text{А}$ за $0{,}5\,\text{c}$.
    Определите ЭДС самоиндукции, ответ выразите в мВ и округлите до целых.
}
\answer{%
    $
        \ele
        = L\frac{\abs{\Delta \eli}}{\Delta t}
        = L\frac{\abs{\eli_2 - \eli_1}}{\Delta t}
        = 6\,\text{мГн} \cdot \frac{\abs{4\,\text{А} - 9\,\text{А}}}{0{,}5\,\text{c}}
        \approx 60{,}000\,\text{мВ} \to 60
    $
}
\solutionspace{60pt}

\tasknumber{6}%
\task{%
    	В катушке, индуктивность которой равна $80\,\text{мГн}$, течёт электрический ток силой $7\,\text{А}$.
    	Число витков в катушке: 30.
    Определите магнитный поток, пронизывающий 1 виток катушки.
    	Ответ выразите в милливеберах и округлите до целых.
}
\answer{%
    $
        \Phi_\text{1 виток}
        = \frac{\Phi}{N}
        = \frac{L\eli}{N}
        = \frac{80\,\text{мГн} \cdot 7\,\text{А}}{30}
        \approx 18{,}667\,\text{мВб}
        \to 19
    $
}

\variantsplitter

\addpersonalvariant{Владислав Емелин}

\tasknumber{1}%
\task{%
    Установите каждой букве в соответствие ровно одну цифру и запишите в ответ только цифры (без других символов).

    А) $\Delta \Phi$, Б) $\Delta t$, В) $\ele$.

    1) $\Phi_2 - \Phi_1$, 2) $t_1 - t_2$, 3) $\Phi_1 - \Phi_2$, 4) $t_2 - t_1$, 5) $-\frac{\Delta \Phi}{\Delta t}$.
}
\answer{%
    $145$
}
\solutionspace{20pt}

\tasknumber{2}%
\task{%
    Установите каждой букве в соответствие ровно одну цифру и запишите ответ (только цифры, без других символов).

    А) $\Phi$, Б) $\Delta \Phi$, В) $\Delta \eli$.

    1) $\frac{L}{\eli}$, 2) $L\eli$, 3) $\eli_1 - \eli_2$, 4) $\frac{\eli}{L}$, 5) $\Phi_2 - \Phi_1$, 6) $\eli_2 - \eli_1$.
}
\answer{%
    $256$
}
\solutionspace{10pt}

\tasknumber{3}%
\task{%
    Установите каждой букве в соответствие ровно одну цифру и запишите ответ (только цифры, без других символов).

    А) электрический заряд, Б) индуктивность, В) поток магнитной индукции.

    1) $q$, 2) $g$, 3) $R$, 4) $L$, 5) $\Phi$.
}
\answer{%
    $145$
}
\solutionspace{10pt}

\tasknumber{4}%
\task{%
    Установите каждой букве в соответствие ровно одну цифру и запишите ответ (только цифры, без других символов).

    А) длина проводника, Б) индуктивность.

    1) Кл, 2) А, 3) м, 4) Гн.
}
\answer{%
    $34$
}
\solutionspace{10pt}

\tasknumber{5}%
\task{%
    	В катушке, индуктивность которой равна $4\,\text{мГн}$, сила тока равномерно уменьшается
    	с $7\,\text{А}$ до $4\,\text{А}$ за $0{,}3\,\text{c}$.
    Определите ЭДС самоиндукции, ответ выразите в мВ и округлите до целых.
}
\answer{%
    $
        \ele
        = L\frac{\abs{\Delta \eli}}{\Delta t}
        = L\frac{\abs{\eli_2 - \eli_1}}{\Delta t}
        = 4\,\text{мГн} \cdot \frac{\abs{4\,\text{А} - 7\,\text{А}}}{0{,}3\,\text{c}}
        \approx 40{,}000\,\text{мВ} \to 40
    $
}
\solutionspace{60pt}

\tasknumber{6}%
\task{%
    	В катушке, индуктивность которой равна $70\,\text{мГн}$, течёт электрический ток силой $6\,\text{А}$.
    	Число витков в катушке: 20.
    Определите магнитный поток, пронизывающий 1 виток катушки.
    	Ответ выразите в милливеберах и округлите до целых.
}
\answer{%
    $
        \Phi_\text{1 виток}
        = \frac{\Phi}{N}
        = \frac{L\eli}{N}
        = \frac{70\,\text{мГн} \cdot 6\,\text{А}}{20}
        \approx 21{,}000\,\text{мВб}
        \to 21
    $
}

\variantsplitter

\addpersonalvariant{Артём Жичин}

\tasknumber{1}%
\task{%
    Установите каждой букве в соответствие ровно одну цифру и запишите в ответ только цифры (без других символов).

    А) $\Delta t$, Б) $\Delta \Phi$, В) $\ele$.

    1) $-\frac{\Delta \Phi}{\Delta t}$, 2) $t_1 - t_2$, 3) $\Phi_2 - \Phi_1$, 4) $t_2 - t_1$, 5) $\Phi_1 - \Phi_2$.
}
\answer{%
    $431$
}
\solutionspace{20pt}

\tasknumber{2}%
\task{%
    Установите каждой букве в соответствие ровно одну цифру и запишите ответ (только цифры, без других символов).

    А) $\Delta \eli$, Б) $\Delta \Phi$, В) $\Phi$.

    1) $\Phi_2 - \Phi_1$, 2) $\eli_1 - \eli_2$, 3) $\Phi_1 - \Phi_2$, 4) $L\eli$, 5) $\eli_2 - \eli_1$, 6) $\frac{\eli}{L}$.
}
\answer{%
    $514$
}
\solutionspace{10pt}

\tasknumber{3}%
\task{%
    Установите каждой букве в соответствие ровно одну цифру и запишите ответ (только цифры, без других символов).

    А) электрический ток, Б) электрический заряд, В) индуктивность.

    1) $L$, 2) $R$, 3) $q$, 4) $\eli$, 5) $\ele$.
}
\answer{%
    $431$
}
\solutionspace{10pt}

\tasknumber{4}%
\task{%
    Установите каждой букве в соответствие ровно одну цифру и запишите ответ (только цифры, без других символов).

    А) время, Б) индукция магнитного поля.

    1) Кл, 2) Тл, 3) с, 4) Вб.
}
\answer{%
    $32$
}
\solutionspace{10pt}

\tasknumber{5}%
\task{%
    	В катушке, индуктивность которой равна $6\,\text{мГн}$, сила тока равномерно уменьшается
    	с $7\,\text{А}$ до $4\,\text{А}$ за $0{,}4\,\text{c}$.
    Определите ЭДС самоиндукции, ответ выразите в мВ и округлите до целых.
}
\answer{%
    $
        \ele
        = L\frac{\abs{\Delta \eli}}{\Delta t}
        = L\frac{\abs{\eli_2 - \eli_1}}{\Delta t}
        = 6\,\text{мГн} \cdot \frac{\abs{4\,\text{А} - 7\,\text{А}}}{0{,}4\,\text{c}}
        \approx 45{,}000\,\text{мВ} \to 45
    $
}
\solutionspace{60pt}

\tasknumber{6}%
\task{%
    	В катушке, индуктивность которой равна $70\,\text{мГн}$, течёт электрический ток силой $5\,\text{А}$.
    	Число витков в катушке: 40.
    Определите магнитный поток, пронизывающий 1 виток катушки.
    	Ответ выразите в милливеберах и округлите до целых.
}
\answer{%
    $
        \Phi_\text{1 виток}
        = \frac{\Phi}{N}
        = \frac{L\eli}{N}
        = \frac{70\,\text{мГн} \cdot 5\,\text{А}}{40}
        \approx 8{,}750\,\text{мВб}
        \to 9
    $
}

\variantsplitter

\addpersonalvariant{Дарья Кошман}

\tasknumber{1}%
\task{%
    Установите каждой букве в соответствие ровно одну цифру и запишите в ответ только цифры (без других символов).

    А) $\Delta t$, Б) $\ele$, В) $\Delta \Phi$.

    1) $\Phi_2 - \Phi_1$, 2) $t_2 - t_1$, 3) $t_1 - t_2$, 4) $-\frac{\Delta \Phi}{\Delta t}$, 5) $\Phi_1 - \Phi_2$.
}
\answer{%
    $241$
}
\solutionspace{20pt}

\tasknumber{2}%
\task{%
    Установите каждой букве в соответствие ровно одну цифру и запишите ответ (только цифры, без других символов).

    А) $\Delta \Phi$, Б) $\Phi$, В) $\Delta \eli$.

    1) $\eli_1 - \eli_2$, 2) $\eli_2 - \eli_1$, 3) $\Phi_2 - \Phi_1$, 4) $\frac{\eli}{L}$, 5) $L\eli$, 6) $\Phi_1 - \Phi_2$.
}
\answer{%
    $352$
}
\solutionspace{10pt}

\tasknumber{3}%
\task{%
    Установите каждой букве в соответствие ровно одну цифру и запишите ответ (только цифры, без других символов).

    А) поток магнитной индукции, Б) электрический заряд, В) индуктивность.

    1) $L$, 2) $\Phi$, 3) $g$, 4) $q$, 5) $\vec B$.
}
\answer{%
    $241$
}
\solutionspace{10pt}

\tasknumber{4}%
\task{%
    Установите каждой букве в соответствие ровно одну цифру и запишите ответ (только цифры, без других символов).

    А) поток магнитной индукции, Б) индукция магнитного поля.

    1) Вб, 2) с, 3) Тл, 4) Кл.
}
\answer{%
    $13$
}
\solutionspace{10pt}

\tasknumber{5}%
\task{%
    	В катушке, индуктивность которой равна $6\,\text{мГн}$, сила тока равномерно уменьшается
    	с $8\,\text{А}$ до $1\,\text{А}$ за $0{,}4\,\text{c}$.
    Определите ЭДС самоиндукции, ответ выразите в мВ и округлите до целых.
}
\answer{%
    $
        \ele
        = L\frac{\abs{\Delta \eli}}{\Delta t}
        = L\frac{\abs{\eli_2 - \eli_1}}{\Delta t}
        = 6\,\text{мГн} \cdot \frac{\abs{1\,\text{А} - 8\,\text{А}}}{0{,}4\,\text{c}}
        \approx 105{,}000\,\text{мВ} \to 105
    $
}
\solutionspace{60pt}

\tasknumber{6}%
\task{%
    	В катушке, индуктивность которой равна $90\,\text{мГн}$, течёт электрический ток силой $6\,\text{А}$.
    	Число витков в катушке: 20.
    Определите магнитный поток, пронизывающий 1 виток катушки.
    	Ответ выразите в милливеберах и округлите до целых.
}
\answer{%
    $
        \Phi_\text{1 виток}
        = \frac{\Phi}{N}
        = \frac{L\eli}{N}
        = \frac{90\,\text{мГн} \cdot 6\,\text{А}}{20}
        \approx 27{,}000\,\text{мВб}
        \to 27
    $
}

\variantsplitter

\addpersonalvariant{Анна Кузьмичёва}

\tasknumber{1}%
\task{%
    Установите каждой букве в соответствие ровно одну цифру и запишите в ответ только цифры (без других символов).

    А) $\Delta \Phi$, Б) $\Delta t$, В) $\ele$.

    1) $\Phi_2 - \Phi_1$, 2) $-\frac{\Delta \Phi}{\Delta t}$, 3) $t_2 - t_1$, 4) $t_1 - t_2$, 5) $\Phi_1 - \Phi_2$.
}
\answer{%
    $132$
}
\solutionspace{20pt}

\tasknumber{2}%
\task{%
    Установите каждой букве в соответствие ровно одну цифру и запишите ответ (только цифры, без других символов).

    А) $\Delta \eli$, Б) $\Phi$, В) $\Delta \Phi$.

    1) $\frac{\eli}{L}$, 2) $\eli_2 - \eli_1$, 3) $\Phi_2 - \Phi_1$, 4) $L\eli$, 5) $\eli_1 - \eli_2$, 6) $\frac{L}{\eli}$.
}
\answer{%
    $243$
}
\solutionspace{10pt}

\tasknumber{3}%
\task{%
    Установите каждой букве в соответствие ровно одну цифру и запишите ответ (только цифры, без других символов).

    А) электрический заряд, Б) индуктивность, В) электрический ток.

    1) $q$, 2) $\eli$, 3) $L$, 4) $R$, 5) $\ele$.
}
\answer{%
    $132$
}
\solutionspace{10pt}

\tasknumber{4}%
\task{%
    Установите каждой букве в соответствие ровно одну цифру и запишите ответ (только цифры, без других символов).

    А) время, Б) индуктивность.

    1) Гн, 2) с, 3) Кл, 4) Вт.
}
\answer{%
    $21$
}
\solutionspace{10pt}

\tasknumber{5}%
\task{%
    	В катушке, индуктивность которой равна $6\,\text{мГн}$, сила тока равномерно уменьшается
    	с $8\,\text{А}$ до $4\,\text{А}$ за $0{,}4\,\text{c}$.
    Определите ЭДС самоиндукции, ответ выразите в мВ и округлите до целых.
}
\answer{%
    $
        \ele
        = L\frac{\abs{\Delta \eli}}{\Delta t}
        = L\frac{\abs{\eli_2 - \eli_1}}{\Delta t}
        = 6\,\text{мГн} \cdot \frac{\abs{4\,\text{А} - 8\,\text{А}}}{0{,}4\,\text{c}}
        \approx 60{,}000\,\text{мВ} \to 60
    $
}
\solutionspace{60pt}

\tasknumber{6}%
\task{%
    	В катушке, индуктивность которой равна $50\,\text{мГн}$, течёт электрический ток силой $5\,\text{А}$.
    	Число витков в катушке: 30.
    Определите магнитный поток, пронизывающий 1 виток катушки.
    	Ответ выразите в милливеберах и округлите до целых.
}
\answer{%
    $
        \Phi_\text{1 виток}
        = \frac{\Phi}{N}
        = \frac{L\eli}{N}
        = \frac{50\,\text{мГн} \cdot 5\,\text{А}}{30}
        \approx 8{,}333\,\text{мВб}
        \to 8
    $
}

\variantsplitter

\addpersonalvariant{Алёна Куприянова}

\tasknumber{1}%
\task{%
    Установите каждой букве в соответствие ровно одну цифру и запишите в ответ только цифры (без других символов).

    А) $\ele$, Б) $\Delta \Phi$, В) $\Delta t$.

    1) $\Phi_1 - \Phi_2$, 2) $t_2 - t_1$, 3) $\Phi_2 - \Phi_1$, 4) $-\frac{\Delta \Phi}{\Delta t}$, 5) $t_1 - t_2$.
}
\answer{%
    $432$
}
\solutionspace{20pt}

\tasknumber{2}%
\task{%
    Установите каждой букве в соответствие ровно одну цифру и запишите ответ (только цифры, без других символов).

    А) $\Delta \eli$, Б) $\Delta \Phi$, В) $\Phi$.

    1) $L\eli$, 2) $\frac{\eli}{L}$, 3) $\eli_1 - \eli_2$, 4) $\Phi_2 - \Phi_1$, 5) $\eli_2 - \eli_1$, 6) $\Phi_1 - \Phi_2$.
}
\answer{%
    $541$
}
\solutionspace{10pt}

\tasknumber{3}%
\task{%
    Установите каждой букве в соответствие ровно одну цифру и запишите ответ (только цифры, без других символов).

    А) индукция магнитного поля, Б) электрический ток, В) поток магнитной индукции.

    1) $q$, 2) $\Phi$, 3) $\eli$, 4) $\vec B$, 5) $R$.
}
\answer{%
    $432$
}
\solutionspace{10pt}

\tasknumber{4}%
\task{%
    Установите каждой букве в соответствие ровно одну цифру и запишите ответ (только цифры, без других символов).

    А) поток магнитной индукции, Б) время.

    1) Кл, 2) с, 3) Вб, 4) Вт.
}
\answer{%
    $32$
}
\solutionspace{10pt}

\tasknumber{5}%
\task{%
    	В катушке, индуктивность которой равна $6\,\text{мГн}$, сила тока равномерно уменьшается
    	с $8\,\text{А}$ до $3\,\text{А}$ за $0{,}4\,\text{c}$.
    Определите ЭДС самоиндукции, ответ выразите в мВ и округлите до целых.
}
\answer{%
    $
        \ele
        = L\frac{\abs{\Delta \eli}}{\Delta t}
        = L\frac{\abs{\eli_2 - \eli_1}}{\Delta t}
        = 6\,\text{мГн} \cdot \frac{\abs{3\,\text{А} - 8\,\text{А}}}{0{,}4\,\text{c}}
        \approx 75{,}000\,\text{мВ} \to 75
    $
}
\solutionspace{60pt}

\tasknumber{6}%
\task{%
    	В катушке, индуктивность которой равна $50\,\text{мГн}$, течёт электрический ток силой $5\,\text{А}$.
    	Число витков в катушке: 40.
    Определите магнитный поток, пронизывающий 1 виток катушки.
    	Ответ выразите в милливеберах и округлите до целых.
}
\answer{%
    $
        \Phi_\text{1 виток}
        = \frac{\Phi}{N}
        = \frac{L\eli}{N}
        = \frac{50\,\text{мГн} \cdot 5\,\text{А}}{40}
        \approx 6{,}250\,\text{мВб}
        \to 6
    $
}

\variantsplitter

\addpersonalvariant{Ярослав Лавровский}

\tasknumber{1}%
\task{%
    Установите каждой букве в соответствие ровно одну цифру и запишите в ответ только цифры (без других символов).

    А) $\ele$, Б) $\Delta t$, В) $\Delta \Phi$.

    1) $\Phi_1 - \Phi_2$, 2) $t_2 - t_1$, 3) $t_1 - t_2$, 4) $\Phi_2 - \Phi_1$, 5) $-\frac{\Delta \Phi}{\Delta t}$.
}
\answer{%
    $524$
}
\solutionspace{20pt}

\tasknumber{2}%
\task{%
    Установите каждой букве в соответствие ровно одну цифру и запишите ответ (только цифры, без других символов).

    А) $\Delta \eli$, Б) $\Delta \Phi$, В) $\Phi$.

    1) $\Phi_2 - \Phi_1$, 2) $\Phi_1 - \Phi_2$, 3) $L\eli$, 4) $\frac{L}{\eli}$, 5) $\eli_1 - \eli_2$, 6) $\eli_2 - \eli_1$.
}
\answer{%
    $613$
}
\solutionspace{10pt}

\tasknumber{3}%
\task{%
    Установите каждой букве в соответствие ровно одну цифру и запишите ответ (только цифры, без других символов).

    А) электрический заряд, Б) индуктивность, В) индукция магнитного поля.

    1) $\varphi$, 2) $L$, 3) $\ele$, 4) $\vec B$, 5) $q$.
}
\answer{%
    $524$
}
\solutionspace{10pt}

\tasknumber{4}%
\task{%
    Установите каждой букве в соответствие ровно одну цифру и запишите ответ (только цифры, без других символов).

    А) индукция магнитного поля, Б) поток магнитной индукции.

    1) Вб, 2) м, 3) Вт, 4) Тл.
}
\answer{%
    $41$
}
\solutionspace{10pt}

\tasknumber{5}%
\task{%
    	В катушке, индуктивность которой равна $7\,\text{мГн}$, сила тока равномерно уменьшается
    	с $7\,\text{А}$ до $1\,\text{А}$ за $0{,}3\,\text{c}$.
    Определите ЭДС самоиндукции, ответ выразите в мВ и округлите до целых.
}
\answer{%
    $
        \ele
        = L\frac{\abs{\Delta \eli}}{\Delta t}
        = L\frac{\abs{\eli_2 - \eli_1}}{\Delta t}
        = 7\,\text{мГн} \cdot \frac{\abs{1\,\text{А} - 7\,\text{А}}}{0{,}3\,\text{c}}
        \approx 140{,}000\,\text{мВ} \to 140
    $
}
\solutionspace{60pt}

\tasknumber{6}%
\task{%
    	В катушке, индуктивность которой равна $70\,\text{мГн}$, течёт электрический ток силой $6\,\text{А}$.
    	Число витков в катушке: 30.
    Определите магнитный поток, пронизывающий 1 виток катушки.
    	Ответ выразите в милливеберах и округлите до целых.
}
\answer{%
    $
        \Phi_\text{1 виток}
        = \frac{\Phi}{N}
        = \frac{L\eli}{N}
        = \frac{70\,\text{мГн} \cdot 6\,\text{А}}{30}
        \approx 14{,}000\,\text{мВб}
        \to 14
    $
}

\variantsplitter

\addpersonalvariant{Анастасия Ламанова}

\tasknumber{1}%
\task{%
    Установите каждой букве в соответствие ровно одну цифру и запишите в ответ только цифры (без других символов).

    А) $\Delta t$, Б) $\ele$, В) $\Delta \Phi$.

    1) $\Phi_1 - \Phi_2$, 2) $-\frac{\Delta \Phi}{\Delta t}$, 3) $t_1 - t_2$, 4) $\Phi_2 - \Phi_1$, 5) $t_2 - t_1$.
}
\answer{%
    $524$
}
\solutionspace{20pt}

\tasknumber{2}%
\task{%
    Установите каждой букве в соответствие ровно одну цифру и запишите ответ (только цифры, без других символов).

    А) $\Delta \Phi$, Б) $\Delta \eli$, В) $\Phi$.

    1) $\Phi_1 - \Phi_2$, 2) $\frac{L}{\eli}$, 3) $\eli_2 - \eli_1$, 4) $\eli_1 - \eli_2$, 5) $L\eli$, 6) $\Phi_2 - \Phi_1$.
}
\answer{%
    $635$
}
\solutionspace{10pt}

\tasknumber{3}%
\task{%
    Установите каждой букве в соответствие ровно одну цифру и запишите ответ (только цифры, без других символов).

    А) индукция магнитного поля, Б) поток магнитной индукции, В) индуктивность.

    1) $R$, 2) $\Phi$, 3) $q$, 4) $L$, 5) $\vec B$.
}
\answer{%
    $524$
}
\solutionspace{10pt}

\tasknumber{4}%
\task{%
    Установите каждой букве в соответствие ровно одну цифру и запишите ответ (только цифры, без других символов).

    А) поток магнитной индукции, Б) длина проводника.

    1) м, 2) с, 3) Вт, 4) Вб.
}
\answer{%
    $41$
}
\solutionspace{10pt}

\tasknumber{5}%
\task{%
    	В катушке, индуктивность которой равна $4\,\text{мГн}$, сила тока равномерно уменьшается
    	с $7\,\text{А}$ до $3\,\text{А}$ за $0{,}2\,\text{c}$.
    Определите ЭДС самоиндукции, ответ выразите в мВ и округлите до целых.
}
\answer{%
    $
        \ele
        = L\frac{\abs{\Delta \eli}}{\Delta t}
        = L\frac{\abs{\eli_2 - \eli_1}}{\Delta t}
        = 4\,\text{мГн} \cdot \frac{\abs{3\,\text{А} - 7\,\text{А}}}{0{,}2\,\text{c}}
        \approx 80{,}000\,\text{мВ} \to 80
    $
}
\solutionspace{60pt}

\tasknumber{6}%
\task{%
    	В катушке, индуктивность которой равна $90\,\text{мГн}$, течёт электрический ток силой $5\,\text{А}$.
    	Число витков в катушке: 40.
    Определите магнитный поток, пронизывающий 1 виток катушки.
    	Ответ выразите в милливеберах и округлите до целых.
}
\answer{%
    $
        \Phi_\text{1 виток}
        = \frac{\Phi}{N}
        = \frac{L\eli}{N}
        = \frac{90\,\text{мГн} \cdot 5\,\text{А}}{40}
        \approx 11{,}250\,\text{мВб}
        \to 11
    $
}

\variantsplitter

\addpersonalvariant{Виктория Легонькова}

\tasknumber{1}%
\task{%
    Установите каждой букве в соответствие ровно одну цифру и запишите в ответ только цифры (без других символов).

    А) $\Delta \Phi$, Б) $\Delta t$, В) $\ele$.

    1) $-\frac{\Delta \Phi}{\Delta t}$, 2) $\Phi_1 - \Phi_2$, 3) $\Phi_2 - \Phi_1$, 4) $t_1 - t_2$, 5) $t_2 - t_1$.
}
\answer{%
    $351$
}
\solutionspace{20pt}

\tasknumber{2}%
\task{%
    Установите каждой букве в соответствие ровно одну цифру и запишите ответ (только цифры, без других символов).

    А) $\Delta \Phi$, Б) $\Delta \eli$, В) $\Phi$.

    1) $\frac{\eli}{L}$, 2) $L\eli$, 3) $\eli_1 - \eli_2$, 4) $\Phi_2 - \Phi_1$, 5) $\frac{L}{\eli}$, 6) $\eli_2 - \eli_1$.
}
\answer{%
    $462$
}
\solutionspace{10pt}

\tasknumber{3}%
\task{%
    Установите каждой букве в соответствие ровно одну цифру и запишите ответ (только цифры, без других символов).

    А) поток магнитной индукции, Б) индуктивность, В) электрический заряд.

    1) $q$, 2) $\eli$, 3) $\Phi$, 4) $\varphi$, 5) $L$.
}
\answer{%
    $351$
}
\solutionspace{10pt}

\tasknumber{4}%
\task{%
    Установите каждой букве в соответствие ровно одну цифру и запишите ответ (только цифры, без других символов).

    А) время, Б) поток магнитной индукции.

    1) Тл, 2) с, 3) А, 4) Вб.
}
\answer{%
    $24$
}
\solutionspace{10pt}

\tasknumber{5}%
\task{%
    	В катушке, индуктивность которой равна $5\,\text{мГн}$, сила тока равномерно уменьшается
    	с $7\,\text{А}$ до $2\,\text{А}$ за $0{,}4\,\text{c}$.
    Определите ЭДС самоиндукции, ответ выразите в мВ и округлите до целых.
}
\answer{%
    $
        \ele
        = L\frac{\abs{\Delta \eli}}{\Delta t}
        = L\frac{\abs{\eli_2 - \eli_1}}{\Delta t}
        = 5\,\text{мГн} \cdot \frac{\abs{2\,\text{А} - 7\,\text{А}}}{0{,}4\,\text{c}}
        \approx 62{,}500\,\text{мВ} \to 63
    $
}
\solutionspace{60pt}

\tasknumber{6}%
\task{%
    	В катушке, индуктивность которой равна $80\,\text{мГн}$, течёт электрический ток силой $6\,\text{А}$.
    	Число витков в катушке: 40.
    Определите магнитный поток, пронизывающий 1 виток катушки.
    	Ответ выразите в милливеберах и округлите до целых.
}
\answer{%
    $
        \Phi_\text{1 виток}
        = \frac{\Phi}{N}
        = \frac{L\eli}{N}
        = \frac{80\,\text{мГн} \cdot 6\,\text{А}}{40}
        \approx 12{,}000\,\text{мВб}
        \to 12
    $
}

\variantsplitter

\addpersonalvariant{Семён Мартынов}

\tasknumber{1}%
\task{%
    Установите каждой букве в соответствие ровно одну цифру и запишите в ответ только цифры (без других символов).

    А) $\ele$, Б) $\Delta \Phi$, В) $\Delta t$.

    1) $\Phi_2 - \Phi_1$, 2) $\Phi_1 - \Phi_2$, 3) $-\frac{\Delta \Phi}{\Delta t}$, 4) $t_2 - t_1$, 5) $t_1 - t_2$.
}
\answer{%
    $314$
}
\solutionspace{20pt}

\tasknumber{2}%
\task{%
    Установите каждой букве в соответствие ровно одну цифру и запишите ответ (только цифры, без других символов).

    А) $\Delta \eli$, Б) $\Phi$, В) $\Delta \Phi$.

    1) $\Phi_2 - \Phi_1$, 2) $L\eli$, 3) $\eli_1 - \eli_2$, 4) $\eli_2 - \eli_1$, 5) $\frac{\eli}{L}$, 6) $\Phi_1 - \Phi_2$.
}
\answer{%
    $421$
}
\solutionspace{10pt}

\tasknumber{3}%
\task{%
    Установите каждой букве в соответствие ровно одну цифру и запишите ответ (только цифры, без других символов).

    А) индуктивность, Б) электрический ток, В) электрический заряд.

    1) $\eli$, 2) $g$, 3) $L$, 4) $q$, 5) $\Phi$.
}
\answer{%
    $314$
}
\solutionspace{10pt}

\tasknumber{4}%
\task{%
    Установите каждой букве в соответствие ровно одну цифру и запишите ответ (только цифры, без других символов).

    А) длина проводника, Б) индуктивность.

    1) Кл, 2) м, 3) Гн, 4) с.
}
\answer{%
    $23$
}
\solutionspace{10pt}

\tasknumber{5}%
\task{%
    	В катушке, индуктивность которой равна $6\,\text{мГн}$, сила тока равномерно уменьшается
    	с $8\,\text{А}$ до $1\,\text{А}$ за $0{,}2\,\text{c}$.
    Определите ЭДС самоиндукции, ответ выразите в мВ и округлите до целых.
}
\answer{%
    $
        \ele
        = L\frac{\abs{\Delta \eli}}{\Delta t}
        = L\frac{\abs{\eli_2 - \eli_1}}{\Delta t}
        = 6\,\text{мГн} \cdot \frac{\abs{1\,\text{А} - 8\,\text{А}}}{0{,}2\,\text{c}}
        \approx 210{,}000\,\text{мВ} \to 210
    $
}
\solutionspace{60pt}

\tasknumber{6}%
\task{%
    	В катушке, индуктивность которой равна $60\,\text{мГн}$, течёт электрический ток силой $7\,\text{А}$.
    	Число витков в катушке: 40.
    Определите магнитный поток, пронизывающий 1 виток катушки.
    	Ответ выразите в милливеберах и округлите до целых.
}
\answer{%
    $
        \Phi_\text{1 виток}
        = \frac{\Phi}{N}
        = \frac{L\eli}{N}
        = \frac{60\,\text{мГн} \cdot 7\,\text{А}}{40}
        \approx 10{,}500\,\text{мВб}
        \to 11
    $
}

\variantsplitter

\addpersonalvariant{Варвара Минаева}

\tasknumber{1}%
\task{%
    Установите каждой букве в соответствие ровно одну цифру и запишите в ответ только цифры (без других символов).

    А) $\ele$, Б) $\Delta \Phi$, В) $\Delta t$.

    1) $t_2 - t_1$, 2) $\Phi_1 - \Phi_2$, 3) $\Phi_2 - \Phi_1$, 4) $-\frac{\Delta \Phi}{\Delta t}$, 5) $t_1 - t_2$.
}
\answer{%
    $431$
}
\solutionspace{20pt}

\tasknumber{2}%
\task{%
    Установите каждой букве в соответствие ровно одну цифру и запишите ответ (только цифры, без других символов).

    А) $\Delta \eli$, Б) $\Delta \Phi$, В) $\Phi$.

    1) $L\eli$, 2) $\frac{\eli}{L}$, 3) $\frac{L}{\eli}$, 4) $\Phi_2 - \Phi_1$, 5) $\eli_2 - \eli_1$, 6) $\Phi_1 - \Phi_2$.
}
\answer{%
    $541$
}
\solutionspace{10pt}

\tasknumber{3}%
\task{%
    Установите каждой букве в соответствие ровно одну цифру и запишите ответ (только цифры, без других символов).

    А) индукция магнитного поля, Б) индуктивность, В) электрический ток.

    1) $\eli$, 2) $\ele$, 3) $L$, 4) $\vec B$, 5) $q$.
}
\answer{%
    $431$
}
\solutionspace{10pt}

\tasknumber{4}%
\task{%
    Установите каждой букве в соответствие ровно одну цифру и запишите ответ (только цифры, без других символов).

    А) поток магнитной индукции, Б) время.

    1) Кл, 2) с, 3) Вб, 4) А.
}
\answer{%
    $32$
}
\solutionspace{10pt}

\tasknumber{5}%
\task{%
    	В катушке, индуктивность которой равна $4\,\text{мГн}$, сила тока равномерно уменьшается
    	с $8\,\text{А}$ до $3\,\text{А}$ за $0{,}5\,\text{c}$.
    Определите ЭДС самоиндукции, ответ выразите в мВ и округлите до целых.
}
\answer{%
    $
        \ele
        = L\frac{\abs{\Delta \eli}}{\Delta t}
        = L\frac{\abs{\eli_2 - \eli_1}}{\Delta t}
        = 4\,\text{мГн} \cdot \frac{\abs{3\,\text{А} - 8\,\text{А}}}{0{,}5\,\text{c}}
        \approx 40{,}000\,\text{мВ} \to 40
    $
}
\solutionspace{60pt}

\tasknumber{6}%
\task{%
    	В катушке, индуктивность которой равна $90\,\text{мГн}$, течёт электрический ток силой $5\,\text{А}$.
    	Число витков в катушке: 40.
    Определите магнитный поток, пронизывающий 1 виток катушки.
    	Ответ выразите в милливеберах и округлите до целых.
}
\answer{%
    $
        \Phi_\text{1 виток}
        = \frac{\Phi}{N}
        = \frac{L\eli}{N}
        = \frac{90\,\text{мГн} \cdot 5\,\text{А}}{40}
        \approx 11{,}250\,\text{мВб}
        \to 11
    $
}

\variantsplitter

\addpersonalvariant{Леонид Никитин}

\tasknumber{1}%
\task{%
    Установите каждой букве в соответствие ровно одну цифру и запишите в ответ только цифры (без других символов).

    А) $\ele$, Б) $\Delta \Phi$, В) $\Delta t$.

    1) $\Phi_1 - \Phi_2$, 2) $\Phi_2 - \Phi_1$, 3) $t_1 - t_2$, 4) $t_2 - t_1$, 5) $-\frac{\Delta \Phi}{\Delta t}$.
}
\answer{%
    $524$
}
\solutionspace{20pt}

\tasknumber{2}%
\task{%
    Установите каждой букве в соответствие ровно одну цифру и запишите ответ (только цифры, без других символов).

    А) $\Delta \eli$, Б) $\Delta \Phi$, В) $\Phi$.

    1) $\eli_1 - \eli_2$, 2) $\Phi_1 - \Phi_2$, 3) $\Phi_2 - \Phi_1$, 4) $\frac{\eli}{L}$, 5) $L\eli$, 6) $\eli_2 - \eli_1$.
}
\answer{%
    $635$
}
\solutionspace{10pt}

\tasknumber{3}%
\task{%
    Установите каждой букве в соответствие ровно одну цифру и запишите ответ (только цифры, без других символов).

    А) поток магнитной индукции, Б) электрический заряд, В) электрический ток.

    1) $R$, 2) $q$, 3) $\vec B$, 4) $\eli$, 5) $\Phi$.
}
\answer{%
    $524$
}
\solutionspace{10pt}

\tasknumber{4}%
\task{%
    Установите каждой букве в соответствие ровно одну цифру и запишите ответ (только цифры, без других символов).

    А) индукция магнитного поля, Б) индуктивность.

    1) Гн, 2) Вб, 3) с, 4) Тл.
}
\answer{%
    $41$
}
\solutionspace{10pt}

\tasknumber{5}%
\task{%
    	В катушке, индуктивность которой равна $4\,\text{мГн}$, сила тока равномерно уменьшается
    	с $9\,\text{А}$ до $3\,\text{А}$ за $0{,}4\,\text{c}$.
    Определите ЭДС самоиндукции, ответ выразите в мВ и округлите до целых.
}
\answer{%
    $
        \ele
        = L\frac{\abs{\Delta \eli}}{\Delta t}
        = L\frac{\abs{\eli_2 - \eli_1}}{\Delta t}
        = 4\,\text{мГн} \cdot \frac{\abs{3\,\text{А} - 9\,\text{А}}}{0{,}4\,\text{c}}
        \approx 60{,}000\,\text{мВ} \to 60
    $
}
\solutionspace{60pt}

\tasknumber{6}%
\task{%
    	В катушке, индуктивность которой равна $70\,\text{мГн}$, течёт электрический ток силой $6\,\text{А}$.
    	Число витков в катушке: 20.
    Определите магнитный поток, пронизывающий 1 виток катушки.
    	Ответ выразите в милливеберах и округлите до целых.
}
\answer{%
    $
        \Phi_\text{1 виток}
        = \frac{\Phi}{N}
        = \frac{L\eli}{N}
        = \frac{70\,\text{мГн} \cdot 6\,\text{А}}{20}
        \approx 21{,}000\,\text{мВб}
        \to 21
    $
}

\variantsplitter

\addpersonalvariant{Тимофей Полетаев}

\tasknumber{1}%
\task{%
    Установите каждой букве в соответствие ровно одну цифру и запишите в ответ только цифры (без других символов).

    А) $\ele$, Б) $\Delta \Phi$, В) $\Delta t$.

    1) $t_1 - t_2$, 2) $-\frac{\Delta \Phi}{\Delta t}$, 3) $\Phi_2 - \Phi_1$, 4) $t_2 - t_1$, 5) $\Phi_1 - \Phi_2$.
}
\answer{%
    $234$
}
\solutionspace{20pt}

\tasknumber{2}%
\task{%
    Установите каждой букве в соответствие ровно одну цифру и запишите ответ (только цифры, без других символов).

    А) $\Delta \eli$, Б) $\Phi$, В) $\Delta \Phi$.

    1) $\Phi_1 - \Phi_2$, 2) $\frac{L}{\eli}$, 3) $\eli_2 - \eli_1$, 4) $L\eli$, 5) $\Phi_2 - \Phi_1$, 6) $\eli_1 - \eli_2$.
}
\answer{%
    $345$
}
\solutionspace{10pt}

\tasknumber{3}%
\task{%
    Установите каждой букве в соответствие ровно одну цифру и запишите ответ (только цифры, без других символов).

    А) электрический заряд, Б) индуктивность, В) поток магнитной индукции.

    1) $\vec B$, 2) $q$, 3) $L$, 4) $\Phi$, 5) $\ele$.
}
\answer{%
    $234$
}
\solutionspace{10pt}

\tasknumber{4}%
\task{%
    Установите каждой букве в соответствие ровно одну цифру и запишите ответ (только цифры, без других символов).

    А) длина проводника, Б) поток магнитной индукции.

    1) м, 2) Вб, 3) Гн, 4) А.
}
\answer{%
    $12$
}
\solutionspace{10pt}

\tasknumber{5}%
\task{%
    	В катушке, индуктивность которой равна $5\,\text{мГн}$, сила тока равномерно уменьшается
    	с $9\,\text{А}$ до $1\,\text{А}$ за $0{,}5\,\text{c}$.
    Определите ЭДС самоиндукции, ответ выразите в мВ и округлите до целых.
}
\answer{%
    $
        \ele
        = L\frac{\abs{\Delta \eli}}{\Delta t}
        = L\frac{\abs{\eli_2 - \eli_1}}{\Delta t}
        = 5\,\text{мГн} \cdot \frac{\abs{1\,\text{А} - 9\,\text{А}}}{0{,}5\,\text{c}}
        \approx 80{,}000\,\text{мВ} \to 80
    $
}
\solutionspace{60pt}

\tasknumber{6}%
\task{%
    	В катушке, индуктивность которой равна $80\,\text{мГн}$, течёт электрический ток силой $5\,\text{А}$.
    	Число витков в катушке: 20.
    Определите магнитный поток, пронизывающий 1 виток катушки.
    	Ответ выразите в милливеберах и округлите до целых.
}
\answer{%
    $
        \Phi_\text{1 виток}
        = \frac{\Phi}{N}
        = \frac{L\eli}{N}
        = \frac{80\,\text{мГн} \cdot 5\,\text{А}}{20}
        \approx 20{,}000\,\text{мВб}
        \to 20
    $
}

\variantsplitter

\addpersonalvariant{Андрей Рожков}

\tasknumber{1}%
\task{%
    Установите каждой букве в соответствие ровно одну цифру и запишите в ответ только цифры (без других символов).

    А) $\Delta t$, Б) $\Delta \Phi$, В) $\ele$.

    1) $t_2 - t_1$, 2) $-\frac{\Delta \Phi}{\Delta t}$, 3) $t_1 - t_2$, 4) $\Phi_1 - \Phi_2$, 5) $\Phi_2 - \Phi_1$.
}
\answer{%
    $152$
}
\solutionspace{20pt}

\tasknumber{2}%
\task{%
    Установите каждой букве в соответствие ровно одну цифру и запишите ответ (только цифры, без других символов).

    А) $\Phi$, Б) $\Delta \eli$, В) $\Delta \Phi$.

    1) $L\eli$, 2) $\eli_2 - \eli_1$, 3) $\Phi_1 - \Phi_2$, 4) $\frac{\eli}{L}$, 5) $\eli_1 - \eli_2$, 6) $\Phi_2 - \Phi_1$.
}
\answer{%
    $126$
}
\solutionspace{10pt}

\tasknumber{3}%
\task{%
    Установите каждой букве в соответствие ровно одну цифру и запишите ответ (только цифры, без других символов).

    А) электрический ток, Б) индуктивность, В) поток магнитной индукции.

    1) $\eli$, 2) $\Phi$, 3) $\vec B$, 4) $\varphi$, 5) $L$.
}
\answer{%
    $152$
}
\solutionspace{10pt}

\tasknumber{4}%
\task{%
    Установите каждой букве в соответствие ровно одну цифру и запишите ответ (только цифры, без других символов).

    А) индукция магнитного поля, Б) поток магнитной индукции.

    1) Вб, 2) А, 3) м / с, 4) Тл.
}
\answer{%
    $41$
}
\solutionspace{10pt}

\tasknumber{5}%
\task{%
    	В катушке, индуктивность которой равна $6\,\text{мГн}$, сила тока равномерно уменьшается
    	с $8\,\text{А}$ до $4\,\text{А}$ за $0{,}3\,\text{c}$.
    Определите ЭДС самоиндукции, ответ выразите в мВ и округлите до целых.
}
\answer{%
    $
        \ele
        = L\frac{\abs{\Delta \eli}}{\Delta t}
        = L\frac{\abs{\eli_2 - \eli_1}}{\Delta t}
        = 6\,\text{мГн} \cdot \frac{\abs{4\,\text{А} - 8\,\text{А}}}{0{,}3\,\text{c}}
        \approx 80{,}000\,\text{мВ} \to 80
    $
}
\solutionspace{60pt}

\tasknumber{6}%
\task{%
    	В катушке, индуктивность которой равна $80\,\text{мГн}$, течёт электрический ток силой $7\,\text{А}$.
    	Число витков в катушке: 20.
    Определите магнитный поток, пронизывающий 1 виток катушки.
    	Ответ выразите в милливеберах и округлите до целых.
}
\answer{%
    $
        \Phi_\text{1 виток}
        = \frac{\Phi}{N}
        = \frac{L\eli}{N}
        = \frac{80\,\text{мГн} \cdot 7\,\text{А}}{20}
        \approx 28{,}000\,\text{мВб}
        \to 28
    $
}

\variantsplitter

\addpersonalvariant{Рената Таржиманова}

\tasknumber{1}%
\task{%
    Установите каждой букве в соответствие ровно одну цифру и запишите в ответ только цифры (без других символов).

    А) $\Delta \Phi$, Б) $\ele$, В) $\Delta t$.

    1) $-\frac{\Delta \Phi}{\Delta t}$, 2) $t_2 - t_1$, 3) $t_1 - t_2$, 4) $\Phi_2 - \Phi_1$, 5) $\Phi_1 - \Phi_2$.
}
\answer{%
    $412$
}
\solutionspace{20pt}

\tasknumber{2}%
\task{%
    Установите каждой букве в соответствие ровно одну цифру и запишите ответ (только цифры, без других символов).

    А) $\Delta \eli$, Б) $\Phi$, В) $\Delta \Phi$.

    1) $\Phi_1 - \Phi_2$, 2) $L\eli$, 3) $\Phi_2 - \Phi_1$, 4) $\frac{L}{\eli}$, 5) $\eli_2 - \eli_1$, 6) $\frac{\eli}{L}$.
}
\answer{%
    $523$
}
\solutionspace{10pt}

\tasknumber{3}%
\task{%
    Установите каждой букве в соответствие ровно одну цифру и запишите ответ (только цифры, без других символов).

    А) электрический заряд, Б) поток магнитной индукции, В) индукция магнитного поля.

    1) $\Phi$, 2) $\vec B$, 3) $L$, 4) $q$, 5) $\varphi$.
}
\answer{%
    $412$
}
\solutionspace{10pt}

\tasknumber{4}%
\task{%
    Установите каждой букве в соответствие ровно одну цифру и запишите ответ (только цифры, без других символов).

    А) индуктивность, Б) индукция магнитного поля.

    1) Тл, 2) Вб, 3) Гн, 4) м / с.
}
\answer{%
    $31$
}
\solutionspace{10pt}

\tasknumber{5}%
\task{%
    	В катушке, индуктивность которой равна $7\,\text{мГн}$, сила тока равномерно уменьшается
    	с $9\,\text{А}$ до $1\,\text{А}$ за $0{,}4\,\text{c}$.
    Определите ЭДС самоиндукции, ответ выразите в мВ и округлите до целых.
}
\answer{%
    $
        \ele
        = L\frac{\abs{\Delta \eli}}{\Delta t}
        = L\frac{\abs{\eli_2 - \eli_1}}{\Delta t}
        = 7\,\text{мГн} \cdot \frac{\abs{1\,\text{А} - 9\,\text{А}}}{0{,}4\,\text{c}}
        \approx 140{,}000\,\text{мВ} \to 140
    $
}
\solutionspace{60pt}

\tasknumber{6}%
\task{%
    	В катушке, индуктивность которой равна $60\,\text{мГн}$, течёт электрический ток силой $6\,\text{А}$.
    	Число витков в катушке: 40.
    Определите магнитный поток, пронизывающий 1 виток катушки.
    	Ответ выразите в милливеберах и округлите до целых.
}
\answer{%
    $
        \Phi_\text{1 виток}
        = \frac{\Phi}{N}
        = \frac{L\eli}{N}
        = \frac{60\,\text{мГн} \cdot 6\,\text{А}}{40}
        \approx 9{,}000\,\text{мВб}
        \to 9
    $
}

\variantsplitter

\addpersonalvariant{Андрей Щербаков}

\tasknumber{1}%
\task{%
    Установите каждой букве в соответствие ровно одну цифру и запишите в ответ только цифры (без других символов).

    А) $\ele$, Б) $\Delta t$, В) $\Delta \Phi$.

    1) $\Phi_2 - \Phi_1$, 2) $t_1 - t_2$, 3) $\Phi_1 - \Phi_2$, 4) $t_2 - t_1$, 5) $-\frac{\Delta \Phi}{\Delta t}$.
}
\answer{%
    $541$
}
\solutionspace{20pt}

\tasknumber{2}%
\task{%
    Установите каждой букве в соответствие ровно одну цифру и запишите ответ (только цифры, без других символов).

    А) $\Delta \eli$, Б) $\Delta \Phi$, В) $\Phi$.

    1) $\eli_2 - \eli_1$, 2) $\frac{L}{\eli}$, 3) $\eli_1 - \eli_2$, 4) $\Phi_1 - \Phi_2$, 5) $L\eli$, 6) $\Phi_2 - \Phi_1$.
}
\answer{%
    $165$
}
\solutionspace{10pt}

\tasknumber{3}%
\task{%
    Установите каждой букве в соответствие ровно одну цифру и запишите ответ (только цифры, без других символов).

    А) электрический заряд, Б) индукция магнитного поля, В) электрический ток.

    1) $\eli$, 2) $\ele$, 3) $\Phi$, 4) $\vec B$, 5) $q$.
}
\answer{%
    $541$
}
\solutionspace{10pt}

\tasknumber{4}%
\task{%
    Установите каждой букве в соответствие ровно одну цифру и запишите ответ (только цифры, без других символов).

    А) индукция магнитного поля, Б) длина проводника.

    1) Вт, 2) с, 3) м, 4) Тл.
}
\answer{%
    $43$
}
\solutionspace{10pt}

\tasknumber{5}%
\task{%
    	В катушке, индуктивность которой равна $4\,\text{мГн}$, сила тока равномерно уменьшается
    	с $7\,\text{А}$ до $1\,\text{А}$ за $0{,}5\,\text{c}$.
    Определите ЭДС самоиндукции, ответ выразите в мВ и округлите до целых.
}
\answer{%
    $
        \ele
        = L\frac{\abs{\Delta \eli}}{\Delta t}
        = L\frac{\abs{\eli_2 - \eli_1}}{\Delta t}
        = 4\,\text{мГн} \cdot \frac{\abs{1\,\text{А} - 7\,\text{А}}}{0{,}5\,\text{c}}
        \approx 48{,}000\,\text{мВ} \to 48
    $
}
\solutionspace{60pt}

\tasknumber{6}%
\task{%
    	В катушке, индуктивность которой равна $80\,\text{мГн}$, течёт электрический ток силой $6\,\text{А}$.
    	Число витков в катушке: 40.
    Определите магнитный поток, пронизывающий 1 виток катушки.
    	Ответ выразите в милливеберах и округлите до целых.
}
\answer{%
    $
        \Phi_\text{1 виток}
        = \frac{\Phi}{N}
        = \frac{L\eli}{N}
        = \frac{80\,\text{мГн} \cdot 6\,\text{А}}{40}
        \approx 12{,}000\,\text{мВб}
        \to 12
    $
}

\variantsplitter

\addpersonalvariant{Михаил Ярошевский}

\tasknumber{1}%
\task{%
    Установите каждой букве в соответствие ровно одну цифру и запишите в ответ только цифры (без других символов).

    А) $\ele$, Б) $\Delta \Phi$, В) $\Delta t$.

    1) $t_1 - t_2$, 2) $\Phi_2 - \Phi_1$, 3) $t_2 - t_1$, 4) $\Phi_1 - \Phi_2$, 5) $-\frac{\Delta \Phi}{\Delta t}$.
}
\answer{%
    $523$
}
\solutionspace{20pt}

\tasknumber{2}%
\task{%
    Установите каждой букве в соответствие ровно одну цифру и запишите ответ (только цифры, без других символов).

    А) $\Delta \Phi$, Б) $\Phi$, В) $\Delta \eli$.

    1) $\eli_1 - \eli_2$, 2) $\frac{L}{\eli}$, 3) $L\eli$, 4) $\eli_2 - \eli_1$, 5) $\Phi_1 - \Phi_2$, 6) $\Phi_2 - \Phi_1$.
}
\answer{%
    $634$
}
\solutionspace{10pt}

\tasknumber{3}%
\task{%
    Установите каждой букве в соответствие ровно одну цифру и запишите ответ (только цифры, без других символов).

    А) электрический ток, Б) поток магнитной индукции, В) индуктивность.

    1) $R$, 2) $\Phi$, 3) $L$, 4) $g$, 5) $\eli$.
}
\answer{%
    $523$
}
\solutionspace{10pt}

\tasknumber{4}%
\task{%
    Установите каждой букве в соответствие ровно одну цифру и запишите ответ (только цифры, без других символов).

    А) индукция магнитного поля, Б) время.

    1) с, 2) Вт, 3) А, 4) Тл.
}
\answer{%
    $41$
}
\solutionspace{10pt}

\tasknumber{5}%
\task{%
    	В катушке, индуктивность которой равна $4\,\text{мГн}$, сила тока равномерно уменьшается
    	с $9\,\text{А}$ до $3\,\text{А}$ за $0{,}4\,\text{c}$.
    Определите ЭДС самоиндукции, ответ выразите в мВ и округлите до целых.
}
\answer{%
    $
        \ele
        = L\frac{\abs{\Delta \eli}}{\Delta t}
        = L\frac{\abs{\eli_2 - \eli_1}}{\Delta t}
        = 4\,\text{мГн} \cdot \frac{\abs{3\,\text{А} - 9\,\text{А}}}{0{,}4\,\text{c}}
        \approx 60{,}000\,\text{мВ} \to 60
    $
}
\solutionspace{60pt}

\tasknumber{6}%
\task{%
    	В катушке, индуктивность которой равна $60\,\text{мГн}$, течёт электрический ток силой $6\,\text{А}$.
    	Число витков в катушке: 30.
    Определите магнитный поток, пронизывающий 1 виток катушки.
    	Ответ выразите в милливеберах и округлите до целых.
}
\answer{%
    $
        \Phi_\text{1 виток}
        = \frac{\Phi}{N}
        = \frac{L\eli}{N}
        = \frac{60\,\text{мГн} \cdot 6\,\text{А}}{30}
        \approx 12{,}000\,\text{мВб}
        \to 12
    $
}

\variantsplitter

\addpersonalvariant{Алексей Алимпиев}

\tasknumber{1}%
\task{%
    Установите каждой букве в соответствие ровно одну цифру и запишите в ответ только цифры (без других символов).

    А) $\Delta t$, Б) $\Delta \Phi$, В) $\ele$.

    1) $t_2 - t_1$, 2) $\Phi_2 - \Phi_1$, 3) $-\frac{\Delta \Phi}{\Delta t}$, 4) $\Phi_1 - \Phi_2$, 5) $t_1 - t_2$.
}
\answer{%
    $123$
}
\solutionspace{20pt}

\tasknumber{2}%
\task{%
    Установите каждой букве в соответствие ровно одну цифру и запишите ответ (только цифры, без других символов).

    А) $\Delta \Phi$, Б) $\Delta \eli$, В) $\Phi$.

    1) $\eli_2 - \eli_1$, 2) $\Phi_2 - \Phi_1$, 3) $L\eli$, 4) $\eli_1 - \eli_2$, 5) $\frac{L}{\eli}$, 6) $\frac{\eli}{L}$.
}
\answer{%
    $213$
}
\solutionspace{10pt}

\tasknumber{3}%
\task{%
    Установите каждой букве в соответствие ровно одну цифру и запишите ответ (только цифры, без других символов).

    А) индукция магнитного поля, Б) поток магнитной индукции, В) электрический заряд.

    1) $\vec B$, 2) $\Phi$, 3) $q$, 4) $\eli$, 5) $R$.
}
\answer{%
    $123$
}
\solutionspace{10pt}

\tasknumber{4}%
\task{%
    Установите каждой букве в соответствие ровно одну цифру и запишите ответ (только цифры, без других символов).

    А) поток магнитной индукции, Б) время.

    1) Вб, 2) с, 3) м, 4) м / с.
}
\answer{%
    $12$
}
\solutionspace{10pt}

\tasknumber{5}%
\task{%
    	В катушке, индуктивность которой равна $7\,\text{мГн}$, сила тока равномерно уменьшается
    	с $9\,\text{А}$ до $4\,\text{А}$ за $0{,}5\,\text{c}$.
    Определите ЭДС самоиндукции, ответ выразите в мВ и округлите до целых.
}
\answer{%
    $
        \ele
        = L\frac{\abs{\Delta \eli}}{\Delta t}
        = L\frac{\abs{\eli_2 - \eli_1}}{\Delta t}
        = 7\,\text{мГн} \cdot \frac{\abs{4\,\text{А} - 9\,\text{А}}}{0{,}5\,\text{c}}
        \approx 70{,}000\,\text{мВ} \to 70
    $
}
\solutionspace{60pt}

\tasknumber{6}%
\task{%
    	В катушке, индуктивность которой равна $80\,\text{мГн}$, течёт электрический ток силой $5\,\text{А}$.
    	Число витков в катушке: 20.
    Определите магнитный поток, пронизывающий 1 виток катушки.
    	Ответ выразите в милливеберах и округлите до целых.
}
\answer{%
    $
        \Phi_\text{1 виток}
        = \frac{\Phi}{N}
        = \frac{L\eli}{N}
        = \frac{80\,\text{мГн} \cdot 5\,\text{А}}{20}
        \approx 20{,}000\,\text{мВб}
        \to 20
    $
}

\variantsplitter

\addpersonalvariant{Евгений Васин}

\tasknumber{1}%
\task{%
    Установите каждой букве в соответствие ровно одну цифру и запишите в ответ только цифры (без других символов).

    А) $\Delta \Phi$, Б) $\Delta t$, В) $\ele$.

    1) $t_2 - t_1$, 2) $t_1 - t_2$, 3) $\Phi_2 - \Phi_1$, 4) $-\frac{\Delta \Phi}{\Delta t}$, 5) $\Phi_1 - \Phi_2$.
}
\answer{%
    $314$
}
\solutionspace{20pt}

\tasknumber{2}%
\task{%
    Установите каждой букве в соответствие ровно одну цифру и запишите ответ (только цифры, без других символов).

    А) $\Delta \Phi$, Б) $\Phi$, В) $\Delta \eli$.

    1) $\frac{\eli}{L}$, 2) $L\eli$, 3) $\frac{L}{\eli}$, 4) $\Phi_2 - \Phi_1$, 5) $\eli_2 - \eli_1$, 6) $\Phi_1 - \Phi_2$.
}
\answer{%
    $425$
}
\solutionspace{10pt}

\tasknumber{3}%
\task{%
    Установите каждой букве в соответствие ровно одну цифру и запишите ответ (только цифры, без других символов).

    А) электрический заряд, Б) поток магнитной индукции, В) электрический ток.

    1) $\Phi$, 2) $\varphi$, 3) $q$, 4) $\eli$, 5) $g$.
}
\answer{%
    $314$
}
\solutionspace{10pt}

\tasknumber{4}%
\task{%
    Установите каждой букве в соответствие ровно одну цифру и запишите ответ (только цифры, без других символов).

    А) время, Б) индуктивность.

    1) Тл, 2) с, 3) Гн, 4) Вт.
}
\answer{%
    $23$
}
\solutionspace{10pt}

\tasknumber{5}%
\task{%
    	В катушке, индуктивность которой равна $5\,\text{мГн}$, сила тока равномерно уменьшается
    	с $7\,\text{А}$ до $3\,\text{А}$ за $0{,}2\,\text{c}$.
    Определите ЭДС самоиндукции, ответ выразите в мВ и округлите до целых.
}
\answer{%
    $
        \ele
        = L\frac{\abs{\Delta \eli}}{\Delta t}
        = L\frac{\abs{\eli_2 - \eli_1}}{\Delta t}
        = 5\,\text{мГн} \cdot \frac{\abs{3\,\text{А} - 7\,\text{А}}}{0{,}2\,\text{c}}
        \approx 100{,}000\,\text{мВ} \to 100
    $
}
\solutionspace{60pt}

\tasknumber{6}%
\task{%
    	В катушке, индуктивность которой равна $70\,\text{мГн}$, течёт электрический ток силой $6\,\text{А}$.
    	Число витков в катушке: 30.
    Определите магнитный поток, пронизывающий 1 виток катушки.
    	Ответ выразите в милливеберах и округлите до целых.
}
\answer{%
    $
        \Phi_\text{1 виток}
        = \frac{\Phi}{N}
        = \frac{L\eli}{N}
        = \frac{70\,\text{мГн} \cdot 6\,\text{А}}{30}
        \approx 14{,}000\,\text{мВб}
        \to 14
    $
}

\variantsplitter

\addpersonalvariant{Вячеслав Волохов}

\tasknumber{1}%
\task{%
    Установите каждой букве в соответствие ровно одну цифру и запишите в ответ только цифры (без других символов).

    А) $\ele$, Б) $\Delta \Phi$, В) $\Delta t$.

    1) $t_2 - t_1$, 2) $\Phi_2 - \Phi_1$, 3) $t_1 - t_2$, 4) $-\frac{\Delta \Phi}{\Delta t}$, 5) $\Phi_1 - \Phi_2$.
}
\answer{%
    $421$
}
\solutionspace{20pt}

\tasknumber{2}%
\task{%
    Установите каждой букве в соответствие ровно одну цифру и запишите ответ (только цифры, без других символов).

    А) $\Delta \eli$, Б) $\Phi$, В) $\Delta \Phi$.

    1) $\Phi_2 - \Phi_1$, 2) $\frac{L}{\eli}$, 3) $L\eli$, 4) $\Phi_1 - \Phi_2$, 5) $\eli_2 - \eli_1$, 6) $\eli_1 - \eli_2$.
}
\answer{%
    $531$
}
\solutionspace{10pt}

\tasknumber{3}%
\task{%
    Установите каждой букве в соответствие ровно одну цифру и запишите ответ (только цифры, без других символов).

    А) электрический заряд, Б) электрический ток, В) поток магнитной индукции.

    1) $\Phi$, 2) $\eli$, 3) $\varphi$, 4) $q$, 5) $L$.
}
\answer{%
    $421$
}
\solutionspace{10pt}

\tasknumber{4}%
\task{%
    Установите каждой букве в соответствие ровно одну цифру и запишите ответ (только цифры, без других символов).

    А) индукция магнитного поля, Б) индуктивность.

    1) Гн, 2) А, 3) Тл, 4) м / с.
}
\answer{%
    $31$
}
\solutionspace{10pt}

\tasknumber{5}%
\task{%
    	В катушке, индуктивность которой равна $5\,\text{мГн}$, сила тока равномерно уменьшается
    	с $9\,\text{А}$ до $2\,\text{А}$ за $0{,}4\,\text{c}$.
    Определите ЭДС самоиндукции, ответ выразите в мВ и округлите до целых.
}
\answer{%
    $
        \ele
        = L\frac{\abs{\Delta \eli}}{\Delta t}
        = L\frac{\abs{\eli_2 - \eli_1}}{\Delta t}
        = 5\,\text{мГн} \cdot \frac{\abs{2\,\text{А} - 9\,\text{А}}}{0{,}4\,\text{c}}
        \approx 87{,}500\,\text{мВ} \to 88
    $
}
\solutionspace{60pt}

\tasknumber{6}%
\task{%
    	В катушке, индуктивность которой равна $70\,\text{мГн}$, течёт электрический ток силой $6\,\text{А}$.
    	Число витков в катушке: 30.
    Определите магнитный поток, пронизывающий 1 виток катушки.
    	Ответ выразите в милливеберах и округлите до целых.
}
\answer{%
    $
        \Phi_\text{1 виток}
        = \frac{\Phi}{N}
        = \frac{L\eli}{N}
        = \frac{70\,\text{мГн} \cdot 6\,\text{А}}{30}
        \approx 14{,}000\,\text{мВб}
        \to 14
    $
}

\variantsplitter

\addpersonalvariant{Герман Говоров}

\tasknumber{1}%
\task{%
    Установите каждой букве в соответствие ровно одну цифру и запишите в ответ только цифры (без других символов).

    А) $\Delta t$, Б) $\ele$, В) $\Delta \Phi$.

    1) $t_1 - t_2$, 2) $-\frac{\Delta \Phi}{\Delta t}$, 3) $\Phi_2 - \Phi_1$, 4) $t_2 - t_1$, 5) $\Phi_1 - \Phi_2$.
}
\answer{%
    $423$
}
\solutionspace{20pt}

\tasknumber{2}%
\task{%
    Установите каждой букве в соответствие ровно одну цифру и запишите ответ (только цифры, без других символов).

    А) $\Delta \eli$, Б) $\Phi$, В) $\Delta \Phi$.

    1) $\Phi_1 - \Phi_2$, 2) $\frac{L}{\eli}$, 3) $L\eli$, 4) $\Phi_2 - \Phi_1$, 5) $\eli_2 - \eli_1$, 6) $\eli_1 - \eli_2$.
}
\answer{%
    $534$
}
\solutionspace{10pt}

\tasknumber{3}%
\task{%
    Установите каждой букве в соответствие ровно одну цифру и запишите ответ (только цифры, без других символов).

    А) электрический ток, Б) индукция магнитного поля, В) поток магнитной индукции.

    1) $\varphi$, 2) $\vec B$, 3) $\Phi$, 4) $\eli$, 5) $\ele$.
}
\answer{%
    $423$
}
\solutionspace{10pt}

\tasknumber{4}%
\task{%
    Установите каждой букве в соответствие ровно одну цифру и запишите ответ (только цифры, без других символов).

    А) длина проводника, Б) поток магнитной индукции.

    1) Вб, 2) Кл, 3) м, 4) м / с.
}
\answer{%
    $31$
}
\solutionspace{10pt}

\tasknumber{5}%
\task{%
    	В катушке, индуктивность которой равна $4\,\text{мГн}$, сила тока равномерно уменьшается
    	с $7\,\text{А}$ до $4\,\text{А}$ за $0{,}4\,\text{c}$.
    Определите ЭДС самоиндукции, ответ выразите в мВ и округлите до целых.
}
\answer{%
    $
        \ele
        = L\frac{\abs{\Delta \eli}}{\Delta t}
        = L\frac{\abs{\eli_2 - \eli_1}}{\Delta t}
        = 4\,\text{мГн} \cdot \frac{\abs{4\,\text{А} - 7\,\text{А}}}{0{,}4\,\text{c}}
        \approx 30{,}000\,\text{мВ} \to 30
    $
}
\solutionspace{60pt}

\tasknumber{6}%
\task{%
    	В катушке, индуктивность которой равна $70\,\text{мГн}$, течёт электрический ток силой $6\,\text{А}$.
    	Число витков в катушке: 20.
    Определите магнитный поток, пронизывающий 1 виток катушки.
    	Ответ выразите в милливеберах и округлите до целых.
}
\answer{%
    $
        \Phi_\text{1 виток}
        = \frac{\Phi}{N}
        = \frac{L\eli}{N}
        = \frac{70\,\text{мГн} \cdot 6\,\text{А}}{20}
        \approx 21{,}000\,\text{мВб}
        \to 21
    $
}

\variantsplitter

\addpersonalvariant{София Журавлёва}

\tasknumber{1}%
\task{%
    Установите каждой букве в соответствие ровно одну цифру и запишите в ответ только цифры (без других символов).

    А) $\Delta t$, Б) $\Delta \Phi$, В) $\ele$.

    1) $t_2 - t_1$, 2) $t_1 - t_2$, 3) $\Phi_2 - \Phi_1$, 4) $-\frac{\Delta \Phi}{\Delta t}$, 5) $\Phi_1 - \Phi_2$.
}
\answer{%
    $134$
}
\solutionspace{20pt}

\tasknumber{2}%
\task{%
    Установите каждой букве в соответствие ровно одну цифру и запишите ответ (только цифры, без других символов).

    А) $\Delta \Phi$, Б) $\Phi$, В) $\Delta \eli$.

    1) $\Phi_2 - \Phi_1$, 2) $L\eli$, 3) $\frac{L}{\eli}$, 4) $\eli_2 - \eli_1$, 5) $\frac{\eli}{L}$, 6) $\eli_1 - \eli_2$.
}
\answer{%
    $124$
}
\solutionspace{10pt}

\tasknumber{3}%
\task{%
    Установите каждой букве в соответствие ровно одну цифру и запишите ответ (только цифры, без других символов).

    А) электрический ток, Б) электрический заряд, В) индуктивность.

    1) $\eli$, 2) $\varphi$, 3) $q$, 4) $L$, 5) $\ele$.
}
\answer{%
    $134$
}
\solutionspace{10pt}

\tasknumber{4}%
\task{%
    Установите каждой букве в соответствие ровно одну цифру и запишите ответ (только цифры, без других символов).

    А) индуктивность, Б) поток магнитной индукции.

    1) А, 2) Гн, 3) Вб, 4) Кл.
}
\answer{%
    $23$
}
\solutionspace{10pt}

\tasknumber{5}%
\task{%
    	В катушке, индуктивность которой равна $6\,\text{мГн}$, сила тока равномерно уменьшается
    	с $7\,\text{А}$ до $3\,\text{А}$ за $0{,}5\,\text{c}$.
    Определите ЭДС самоиндукции, ответ выразите в мВ и округлите до целых.
}
\answer{%
    $
        \ele
        = L\frac{\abs{\Delta \eli}}{\Delta t}
        = L\frac{\abs{\eli_2 - \eli_1}}{\Delta t}
        = 6\,\text{мГн} \cdot \frac{\abs{3\,\text{А} - 7\,\text{А}}}{0{,}5\,\text{c}}
        \approx 48{,}000\,\text{мВ} \to 48
    $
}
\solutionspace{60pt}

\tasknumber{6}%
\task{%
    	В катушке, индуктивность которой равна $50\,\text{мГн}$, течёт электрический ток силой $6\,\text{А}$.
    	Число витков в катушке: 40.
    Определите магнитный поток, пронизывающий 1 виток катушки.
    	Ответ выразите в милливеберах и округлите до целых.
}
\answer{%
    $
        \Phi_\text{1 виток}
        = \frac{\Phi}{N}
        = \frac{L\eli}{N}
        = \frac{50\,\text{мГн} \cdot 6\,\text{А}}{40}
        \approx 7{,}500\,\text{мВб}
        \to 8
    $
}

\variantsplitter

\addpersonalvariant{Константин Козлов}

\tasknumber{1}%
\task{%
    Установите каждой букве в соответствие ровно одну цифру и запишите в ответ только цифры (без других символов).

    А) $\Delta t$, Б) $\ele$, В) $\Delta \Phi$.

    1) $t_1 - t_2$, 2) $\Phi_2 - \Phi_1$, 3) $-\frac{\Delta \Phi}{\Delta t}$, 4) $\Phi_1 - \Phi_2$, 5) $t_2 - t_1$.
}
\answer{%
    $532$
}
\solutionspace{20pt}

\tasknumber{2}%
\task{%
    Установите каждой букве в соответствие ровно одну цифру и запишите ответ (только цифры, без других символов).

    А) $\Phi$, Б) $\Delta \Phi$, В) $\Delta \eli$.

    1) $\frac{\eli}{L}$, 2) $\eli_1 - \eli_2$, 3) $\eli_2 - \eli_1$, 4) $\Phi_2 - \Phi_1$, 5) $\Phi_1 - \Phi_2$, 6) $L\eli$.
}
\answer{%
    $643$
}
\solutionspace{10pt}

\tasknumber{3}%
\task{%
    Установите каждой букве в соответствие ровно одну цифру и запишите ответ (только цифры, без других символов).

    А) индукция магнитного поля, Б) индуктивность, В) поток магнитной индукции.

    1) $R$, 2) $\Phi$, 3) $L$, 4) $\eli$, 5) $\vec B$.
}
\answer{%
    $532$
}
\solutionspace{10pt}

\tasknumber{4}%
\task{%
    Установите каждой букве в соответствие ровно одну цифру и запишите ответ (только цифры, без других символов).

    А) длина проводника, Б) индукция магнитного поля.

    1) А, 2) Тл, 3) Вб, 4) м.
}
\answer{%
    $42$
}
\solutionspace{10pt}

\tasknumber{5}%
\task{%
    	В катушке, индуктивность которой равна $7\,\text{мГн}$, сила тока равномерно уменьшается
    	с $8\,\text{А}$ до $3\,\text{А}$ за $0{,}4\,\text{c}$.
    Определите ЭДС самоиндукции, ответ выразите в мВ и округлите до целых.
}
\answer{%
    $
        \ele
        = L\frac{\abs{\Delta \eli}}{\Delta t}
        = L\frac{\abs{\eli_2 - \eli_1}}{\Delta t}
        = 7\,\text{мГн} \cdot \frac{\abs{3\,\text{А} - 8\,\text{А}}}{0{,}4\,\text{c}}
        \approx 87{,}500\,\text{мВ} \to 88
    $
}
\solutionspace{60pt}

\tasknumber{6}%
\task{%
    	В катушке, индуктивность которой равна $50\,\text{мГн}$, течёт электрический ток силой $5\,\text{А}$.
    	Число витков в катушке: 30.
    Определите магнитный поток, пронизывающий 1 виток катушки.
    	Ответ выразите в милливеберах и округлите до целых.
}
\answer{%
    $
        \Phi_\text{1 виток}
        = \frac{\Phi}{N}
        = \frac{L\eli}{N}
        = \frac{50\,\text{мГн} \cdot 5\,\text{А}}{30}
        \approx 8{,}333\,\text{мВб}
        \to 8
    $
}

\variantsplitter

\addpersonalvariant{Наталья Кравченко}

\tasknumber{1}%
\task{%
    Установите каждой букве в соответствие ровно одну цифру и запишите в ответ только цифры (без других символов).

    А) $\ele$, Б) $\Delta \Phi$, В) $\Delta t$.

    1) $\Phi_2 - \Phi_1$, 2) $-\frac{\Delta \Phi}{\Delta t}$, 3) $t_2 - t_1$, 4) $t_1 - t_2$, 5) $\Phi_1 - \Phi_2$.
}
\answer{%
    $213$
}
\solutionspace{20pt}

\tasknumber{2}%
\task{%
    Установите каждой букве в соответствие ровно одну цифру и запишите ответ (только цифры, без других символов).

    А) $\Delta \eli$, Б) $\Phi$, В) $\Delta \Phi$.

    1) $\frac{L}{\eli}$, 2) $L\eli$, 3) $\eli_2 - \eli_1$, 4) $\Phi_2 - \Phi_1$, 5) $\frac{\eli}{L}$, 6) $\Phi_1 - \Phi_2$.
}
\answer{%
    $324$
}
\solutionspace{10pt}

\tasknumber{3}%
\task{%
    Установите каждой букве в соответствие ровно одну цифру и запишите ответ (только цифры, без других символов).

    А) электрический заряд, Б) поток магнитной индукции, В) индукция магнитного поля.

    1) $\Phi$, 2) $q$, 3) $\vec B$, 4) $R$, 5) $g$.
}
\answer{%
    $213$
}
\solutionspace{10pt}

\tasknumber{4}%
\task{%
    Установите каждой букве в соответствие ровно одну цифру и запишите ответ (только цифры, без других символов).

    А) длина проводника, Б) индуктивность.

    1) м, 2) Гн, 3) Тл, 4) м / с.
}
\answer{%
    $12$
}
\solutionspace{10pt}

\tasknumber{5}%
\task{%
    	В катушке, индуктивность которой равна $7\,\text{мГн}$, сила тока равномерно уменьшается
    	с $9\,\text{А}$ до $2\,\text{А}$ за $0{,}2\,\text{c}$.
    Определите ЭДС самоиндукции, ответ выразите в мВ и округлите до целых.
}
\answer{%
    $
        \ele
        = L\frac{\abs{\Delta \eli}}{\Delta t}
        = L\frac{\abs{\eli_2 - \eli_1}}{\Delta t}
        = 7\,\text{мГн} \cdot \frac{\abs{2\,\text{А} - 9\,\text{А}}}{0{,}2\,\text{c}}
        \approx 245{,}000\,\text{мВ} \to 245
    $
}
\solutionspace{60pt}

\tasknumber{6}%
\task{%
    	В катушке, индуктивность которой равна $50\,\text{мГн}$, течёт электрический ток силой $7\,\text{А}$.
    	Число витков в катушке: 30.
    Определите магнитный поток, пронизывающий 1 виток катушки.
    	Ответ выразите в милливеберах и округлите до целых.
}
\answer{%
    $
        \Phi_\text{1 виток}
        = \frac{\Phi}{N}
        = \frac{L\eli}{N}
        = \frac{50\,\text{мГн} \cdot 7\,\text{А}}{30}
        \approx 11{,}667\,\text{мВб}
        \to 12
    $
}

\variantsplitter

\addpersonalvariant{Сергей Малышев}

\tasknumber{1}%
\task{%
    Установите каждой букве в соответствие ровно одну цифру и запишите в ответ только цифры (без других символов).

    А) $\Delta t$, Б) $\Delta \Phi$, В) $\ele$.

    1) $\Phi_2 - \Phi_1$, 2) $-\frac{\Delta \Phi}{\Delta t}$, 3) $t_2 - t_1$, 4) $t_1 - t_2$, 5) $\Phi_1 - \Phi_2$.
}
\answer{%
    $312$
}
\solutionspace{20pt}

\tasknumber{2}%
\task{%
    Установите каждой букве в соответствие ровно одну цифру и запишите ответ (только цифры, без других символов).

    А) $\Phi$, Б) $\Delta \Phi$, В) $\Delta \eli$.

    1) $L\eli$, 2) $\eli_2 - \eli_1$, 3) $\frac{\eli}{L}$, 4) $\Phi_2 - \Phi_1$, 5) $\eli_1 - \eli_2$, 6) $\frac{L}{\eli}$.
}
\answer{%
    $142$
}
\solutionspace{10pt}

\tasknumber{3}%
\task{%
    Установите каждой букве в соответствие ровно одну цифру и запишите ответ (только цифры, без других символов).

    А) индукция магнитного поля, Б) индуктивность, В) электрический ток.

    1) $L$, 2) $\eli$, 3) $\vec B$, 4) $g$, 5) $\varphi$.
}
\answer{%
    $312$
}
\solutionspace{10pt}

\tasknumber{4}%
\task{%
    Установите каждой букве в соответствие ровно одну цифру и запишите ответ (только цифры, без других символов).

    А) время, Б) индукция магнитного поля.

    1) Тл, 2) с, 3) Вб, 4) Гн.
}
\answer{%
    $21$
}
\solutionspace{10pt}

\tasknumber{5}%
\task{%
    	В катушке, индуктивность которой равна $5\,\text{мГн}$, сила тока равномерно уменьшается
    	с $7\,\text{А}$ до $2\,\text{А}$ за $0{,}4\,\text{c}$.
    Определите ЭДС самоиндукции, ответ выразите в мВ и округлите до целых.
}
\answer{%
    $
        \ele
        = L\frac{\abs{\Delta \eli}}{\Delta t}
        = L\frac{\abs{\eli_2 - \eli_1}}{\Delta t}
        = 5\,\text{мГн} \cdot \frac{\abs{2\,\text{А} - 7\,\text{А}}}{0{,}4\,\text{c}}
        \approx 62{,}500\,\text{мВ} \to 63
    $
}
\solutionspace{60pt}

\tasknumber{6}%
\task{%
    	В катушке, индуктивность которой равна $90\,\text{мГн}$, течёт электрический ток силой $5\,\text{А}$.
    	Число витков в катушке: 20.
    Определите магнитный поток, пронизывающий 1 виток катушки.
    	Ответ выразите в милливеберах и округлите до целых.
}
\answer{%
    $
        \Phi_\text{1 виток}
        = \frac{\Phi}{N}
        = \frac{L\eli}{N}
        = \frac{90\,\text{мГн} \cdot 5\,\text{А}}{20}
        \approx 22{,}500\,\text{мВб}
        \to 23
    $
}

\variantsplitter

\addpersonalvariant{Алина Полканова}

\tasknumber{1}%
\task{%
    Установите каждой букве в соответствие ровно одну цифру и запишите в ответ только цифры (без других символов).

    А) $\Delta \Phi$, Б) $\Delta t$, В) $\ele$.

    1) $\Phi_2 - \Phi_1$, 2) $t_2 - t_1$, 3) $-\frac{\Delta \Phi}{\Delta t}$, 4) $\Phi_1 - \Phi_2$, 5) $t_1 - t_2$.
}
\answer{%
    $123$
}
\solutionspace{20pt}

\tasknumber{2}%
\task{%
    Установите каждой букве в соответствие ровно одну цифру и запишите ответ (только цифры, без других символов).

    А) $\Phi$, Б) $\Delta \eli$, В) $\Delta \Phi$.

    1) $\frac{L}{\eli}$, 2) $L\eli$, 3) $\eli_2 - \eli_1$, 4) $\Phi_2 - \Phi_1$, 5) $\Phi_1 - \Phi_2$, 6) $\frac{\eli}{L}$.
}
\answer{%
    $234$
}
\solutionspace{10pt}

\tasknumber{3}%
\task{%
    Установите каждой букве в соответствие ровно одну цифру и запишите ответ (только цифры, без других символов).

    А) индуктивность, Б) электрический ток, В) индукция магнитного поля.

    1) $L$, 2) $\eli$, 3) $\vec B$, 4) $\Phi$, 5) $\ele$.
}
\answer{%
    $123$
}
\solutionspace{10pt}

\tasknumber{4}%
\task{%
    Установите каждой букве в соответствие ровно одну цифру и запишите ответ (только цифры, без других символов).

    А) время, Б) поток магнитной индукции.

    1) с, 2) Вб, 3) Гн, 4) А.
}
\answer{%
    $12$
}
\solutionspace{10pt}

\tasknumber{5}%
\task{%
    	В катушке, индуктивность которой равна $5\,\text{мГн}$, сила тока равномерно уменьшается
    	с $8\,\text{А}$ до $2\,\text{А}$ за $0{,}5\,\text{c}$.
    Определите ЭДС самоиндукции, ответ выразите в мВ и округлите до целых.
}
\answer{%
    $
        \ele
        = L\frac{\abs{\Delta \eli}}{\Delta t}
        = L\frac{\abs{\eli_2 - \eli_1}}{\Delta t}
        = 5\,\text{мГн} \cdot \frac{\abs{2\,\text{А} - 8\,\text{А}}}{0{,}5\,\text{c}}
        \approx 60{,}000\,\text{мВ} \to 60
    $
}
\solutionspace{60pt}

\tasknumber{6}%
\task{%
    	В катушке, индуктивность которой равна $60\,\text{мГн}$, течёт электрический ток силой $5\,\text{А}$.
    	Число витков в катушке: 40.
    Определите магнитный поток, пронизывающий 1 виток катушки.
    	Ответ выразите в милливеберах и округлите до целых.
}
\answer{%
    $
        \Phi_\text{1 виток}
        = \frac{\Phi}{N}
        = \frac{L\eli}{N}
        = \frac{60\,\text{мГн} \cdot 5\,\text{А}}{40}
        \approx 7{,}500\,\text{мВб}
        \to 8
    $
}

\variantsplitter

\addpersonalvariant{Сергей Пономарёв}

\tasknumber{1}%
\task{%
    Установите каждой букве в соответствие ровно одну цифру и запишите в ответ только цифры (без других символов).

    А) $\ele$, Б) $\Delta \Phi$, В) $\Delta t$.

    1) $-\frac{\Delta \Phi}{\Delta t}$, 2) $t_1 - t_2$, 3) $t_2 - t_1$, 4) $\Phi_2 - \Phi_1$, 5) $\Phi_1 - \Phi_2$.
}
\answer{%
    $143$
}
\solutionspace{20pt}

\tasknumber{2}%
\task{%
    Установите каждой букве в соответствие ровно одну цифру и запишите ответ (только цифры, без других символов).

    А) $\Delta \Phi$, Б) $\Delta \eli$, В) $\Phi$.

    1) $L\eli$, 2) $\Phi_2 - \Phi_1$, 3) $\eli_1 - \eli_2$, 4) $\frac{L}{\eli}$, 5) $\eli_2 - \eli_1$, 6) $\Phi_1 - \Phi_2$.
}
\answer{%
    $251$
}
\solutionspace{10pt}

\tasknumber{3}%
\task{%
    Установите каждой букве в соответствие ровно одну цифру и запишите ответ (только цифры, без других символов).

    А) электрический ток, Б) поток магнитной индукции, В) индуктивность.

    1) $\eli$, 2) $\ele$, 3) $L$, 4) $\Phi$, 5) $q$.
}
\answer{%
    $143$
}
\solutionspace{10pt}

\tasknumber{4}%
\task{%
    Установите каждой букве в соответствие ровно одну цифру и запишите ответ (только цифры, без других символов).

    А) поток магнитной индукции, Б) индуктивность.

    1) м, 2) Гн, 3) Вб, 4) с.
}
\answer{%
    $32$
}
\solutionspace{10pt}

\tasknumber{5}%
\task{%
    	В катушке, индуктивность которой равна $6\,\text{мГн}$, сила тока равномерно уменьшается
    	с $7\,\text{А}$ до $3\,\text{А}$ за $0{,}2\,\text{c}$.
    Определите ЭДС самоиндукции, ответ выразите в мВ и округлите до целых.
}
\answer{%
    $
        \ele
        = L\frac{\abs{\Delta \eli}}{\Delta t}
        = L\frac{\abs{\eli_2 - \eli_1}}{\Delta t}
        = 6\,\text{мГн} \cdot \frac{\abs{3\,\text{А} - 7\,\text{А}}}{0{,}2\,\text{c}}
        \approx 120{,}000\,\text{мВ} \to 120
    $
}
\solutionspace{60pt}

\tasknumber{6}%
\task{%
    	В катушке, индуктивность которой равна $90\,\text{мГн}$, течёт электрический ток силой $5\,\text{А}$.
    	Число витков в катушке: 30.
    Определите магнитный поток, пронизывающий 1 виток катушки.
    	Ответ выразите в милливеберах и округлите до целых.
}
\answer{%
    $
        \Phi_\text{1 виток}
        = \frac{\Phi}{N}
        = \frac{L\eli}{N}
        = \frac{90\,\text{мГн} \cdot 5\,\text{А}}{30}
        \approx 15{,}000\,\text{мВб}
        \to 15
    $
}

\variantsplitter

\addpersonalvariant{Егор Свистушкин}

\tasknumber{1}%
\task{%
    Установите каждой букве в соответствие ровно одну цифру и запишите в ответ только цифры (без других символов).

    А) $\ele$, Б) $\Delta t$, В) $\Delta \Phi$.

    1) $t_2 - t_1$, 2) $-\frac{\Delta \Phi}{\Delta t}$, 3) $\Phi_1 - \Phi_2$, 4) $\Phi_2 - \Phi_1$, 5) $t_1 - t_2$.
}
\answer{%
    $214$
}
\solutionspace{20pt}

\tasknumber{2}%
\task{%
    Установите каждой букве в соответствие ровно одну цифру и запишите ответ (только цифры, без других символов).

    А) $\Phi$, Б) $\Delta \Phi$, В) $\Delta \eli$.

    1) $L\eli$, 2) $\eli_2 - \eli_1$, 3) $\Phi_2 - \Phi_1$, 4) $\Phi_1 - \Phi_2$, 5) $\eli_1 - \eli_2$, 6) $\frac{L}{\eli}$.
}
\answer{%
    $132$
}
\solutionspace{10pt}

\tasknumber{3}%
\task{%
    Установите каждой букве в соответствие ровно одну цифру и запишите ответ (только цифры, без других символов).

    А) электрический ток, Б) индуктивность, В) электрический заряд.

    1) $L$, 2) $\eli$, 3) $\ele$, 4) $q$, 5) $\varphi$.
}
\answer{%
    $214$
}
\solutionspace{10pt}

\tasknumber{4}%
\task{%
    Установите каждой букве в соответствие ровно одну цифру и запишите ответ (только цифры, без других символов).

    А) поток магнитной индукции, Б) индукция магнитного поля.

    1) Вб, 2) Гн, 3) Тл, 4) Вт.
}
\answer{%
    $13$
}
\solutionspace{10pt}

\tasknumber{5}%
\task{%
    	В катушке, индуктивность которой равна $5\,\text{мГн}$, сила тока равномерно уменьшается
    	с $8\,\text{А}$ до $4\,\text{А}$ за $0{,}5\,\text{c}$.
    Определите ЭДС самоиндукции, ответ выразите в мВ и округлите до целых.
}
\answer{%
    $
        \ele
        = L\frac{\abs{\Delta \eli}}{\Delta t}
        = L\frac{\abs{\eli_2 - \eli_1}}{\Delta t}
        = 5\,\text{мГн} \cdot \frac{\abs{4\,\text{А} - 8\,\text{А}}}{0{,}5\,\text{c}}
        \approx 40{,}000\,\text{мВ} \to 40
    $
}
\solutionspace{60pt}

\tasknumber{6}%
\task{%
    	В катушке, индуктивность которой равна $90\,\text{мГн}$, течёт электрический ток силой $6\,\text{А}$.
    	Число витков в катушке: 40.
    Определите магнитный поток, пронизывающий 1 виток катушки.
    	Ответ выразите в милливеберах и округлите до целых.
}
\answer{%
    $
        \Phi_\text{1 виток}
        = \frac{\Phi}{N}
        = \frac{L\eli}{N}
        = \frac{90\,\text{мГн} \cdot 6\,\text{А}}{40}
        \approx 13{,}500\,\text{мВб}
        \to 14
    $
}

\variantsplitter

\addpersonalvariant{Дмитрий Соколов}

\tasknumber{1}%
\task{%
    Установите каждой букве в соответствие ровно одну цифру и запишите в ответ только цифры (без других символов).

    А) $\Delta \Phi$, Б) $\Delta t$, В) $\ele$.

    1) $-\frac{\Delta \Phi}{\Delta t}$, 2) $t_1 - t_2$, 3) $t_2 - t_1$, 4) $\Phi_1 - \Phi_2$, 5) $\Phi_2 - \Phi_1$.
}
\answer{%
    $531$
}
\solutionspace{20pt}

\tasknumber{2}%
\task{%
    Установите каждой букве в соответствие ровно одну цифру и запишите ответ (только цифры, без других символов).

    А) $\Delta \eli$, Б) $\Delta \Phi$, В) $\Phi$.

    1) $\eli_1 - \eli_2$, 2) $L\eli$, 3) $\frac{\eli}{L}$, 4) $\Phi_2 - \Phi_1$, 5) $\Phi_1 - \Phi_2$, 6) $\eli_2 - \eli_1$.
}
\answer{%
    $642$
}
\solutionspace{10pt}

\tasknumber{3}%
\task{%
    Установите каждой букве в соответствие ровно одну цифру и запишите ответ (только цифры, без других символов).

    А) электрический ток, Б) электрический заряд, В) индукция магнитного поля.

    1) $\vec B$, 2) $\ele$, 3) $q$, 4) $g$, 5) $\eli$.
}
\answer{%
    $531$
}
\solutionspace{10pt}

\tasknumber{4}%
\task{%
    Установите каждой букве в соответствие ровно одну цифру и запишите ответ (только цифры, без других символов).

    А) поток магнитной индукции, Б) индукция магнитного поля.

    1) м / с, 2) Тл, 3) А, 4) Вб.
}
\answer{%
    $42$
}
\solutionspace{10pt}

\tasknumber{5}%
\task{%
    	В катушке, индуктивность которой равна $4\,\text{мГн}$, сила тока равномерно уменьшается
    	с $7\,\text{А}$ до $4\,\text{А}$ за $0{,}4\,\text{c}$.
    Определите ЭДС самоиндукции, ответ выразите в мВ и округлите до целых.
}
\answer{%
    $
        \ele
        = L\frac{\abs{\Delta \eli}}{\Delta t}
        = L\frac{\abs{\eli_2 - \eli_1}}{\Delta t}
        = 4\,\text{мГн} \cdot \frac{\abs{4\,\text{А} - 7\,\text{А}}}{0{,}4\,\text{c}}
        \approx 30{,}000\,\text{мВ} \to 30
    $
}
\solutionspace{60pt}

\tasknumber{6}%
\task{%
    	В катушке, индуктивность которой равна $70\,\text{мГн}$, течёт электрический ток силой $5\,\text{А}$.
    	Число витков в катушке: 20.
    Определите магнитный поток, пронизывающий 1 виток катушки.
    	Ответ выразите в милливеберах и округлите до целых.
}
\answer{%
    $
        \Phi_\text{1 виток}
        = \frac{\Phi}{N}
        = \frac{L\eli}{N}
        = \frac{70\,\text{мГн} \cdot 5\,\text{А}}{20}
        \approx 17{,}500\,\text{мВб}
        \to 18
    $
}

\variantsplitter

\addpersonalvariant{Арсений Трофимов}

\tasknumber{1}%
\task{%
    Установите каждой букве в соответствие ровно одну цифру и запишите в ответ только цифры (без других символов).

    А) $\ele$, Б) $\Delta \Phi$, В) $\Delta t$.

    1) $\Phi_1 - \Phi_2$, 2) $t_2 - t_1$, 3) $\Phi_2 - \Phi_1$, 4) $t_1 - t_2$, 5) $-\frac{\Delta \Phi}{\Delta t}$.
}
\answer{%
    $532$
}
\solutionspace{20pt}

\tasknumber{2}%
\task{%
    Установите каждой букве в соответствие ровно одну цифру и запишите ответ (только цифры, без других символов).

    А) $\Delta \eli$, Б) $\Phi$, В) $\Delta \Phi$.

    1) $\Phi_2 - \Phi_1$, 2) $\Phi_1 - \Phi_2$, 3) $\frac{\eli}{L}$, 4) $L\eli$, 5) $\eli_1 - \eli_2$, 6) $\eli_2 - \eli_1$.
}
\answer{%
    $641$
}
\solutionspace{10pt}

\tasknumber{3}%
\task{%
    Установите каждой букве в соответствие ровно одну цифру и запишите ответ (только цифры, без других символов).

    А) индуктивность, Б) индукция магнитного поля, В) электрический ток.

    1) $\Phi$, 2) $\eli$, 3) $\vec B$, 4) $g$, 5) $L$.
}
\answer{%
    $532$
}
\solutionspace{10pt}

\tasknumber{4}%
\task{%
    Установите каждой букве в соответствие ровно одну цифру и запишите ответ (только цифры, без других символов).

    А) поток магнитной индукции, Б) индукция магнитного поля.

    1) м, 2) Тл, 3) Вт, 4) Вб.
}
\answer{%
    $42$
}
\solutionspace{10pt}

\tasknumber{5}%
\task{%
    	В катушке, индуктивность которой равна $4\,\text{мГн}$, сила тока равномерно уменьшается
    	с $7\,\text{А}$ до $3\,\text{А}$ за $0{,}2\,\text{c}$.
    Определите ЭДС самоиндукции, ответ выразите в мВ и округлите до целых.
}
\answer{%
    $
        \ele
        = L\frac{\abs{\Delta \eli}}{\Delta t}
        = L\frac{\abs{\eli_2 - \eli_1}}{\Delta t}
        = 4\,\text{мГн} \cdot \frac{\abs{3\,\text{А} - 7\,\text{А}}}{0{,}2\,\text{c}}
        \approx 80{,}000\,\text{мВ} \to 80
    $
}
\solutionspace{60pt}

\tasknumber{6}%
\task{%
    	В катушке, индуктивность которой равна $90\,\text{мГн}$, течёт электрический ток силой $7\,\text{А}$.
    	Число витков в катушке: 20.
    Определите магнитный поток, пронизывающий 1 виток катушки.
    	Ответ выразите в милливеберах и округлите до целых.
}
\answer{%
    $
        \Phi_\text{1 виток}
        = \frac{\Phi}{N}
        = \frac{L\eli}{N}
        = \frac{90\,\text{мГн} \cdot 7\,\text{А}}{20}
        \approx 31{,}500\,\text{мВб}
        \to 32
    $
}
% autogenerated
