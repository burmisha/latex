\setdate{30~сентября~2021}
\setclass{11«Б»}

\addpersonalvariant{Михаил Бурмистров}

\tasknumber{1}%
\task{%
    Установите каждой букве в соответствие ровно одну цифру и запишите в ответ только цифры (без других символов).

    А) $\ele$, Б) $\Delta \Phi$, В) $\Delta t$.

    1) $t_1 - t_2$, 2) $\Phi_2 - \Phi_1$, 3) $-\frac{\Delta \Phi}{\Delta t}$, 4) $t_2 - t_1$, 5) $\Phi_1 - \Phi_2$.
}
\answer{%
    $324$
}
\solutionspace{20pt}

\tasknumber{2}%
\task{%
    Установите каждой букве в соответствие ровно одну цифру и запишите ответ (только цифры, без других символов).

    А) $\Delta \eli$, Б) $\Delta \Phi$, В) $\Phi$.

    1) $L\eli$, 2) $\eli_1 - \eli_2$, 3) $\Phi_2 - \Phi_1$, 4) $\eli_2 - \eli_1$, 5) $\Phi_1 - \Phi_2$, 6) $\frac{\eli}{L}$.
}
\answer{%
    $431$
}
\solutionspace{10pt}

\tasknumber{3}%
\task{%
    Установите каждой букве в соответствие ровно одну цифру и запишите ответ (только цифры, без других символов).

    А) электрический заряд, Б) электрический ток, В) поток магнитной индукции.

    1) $g$, 2) $\eli$, 3) $q$, 4) $\Phi$, 5) $\ele$.
}
\answer{%
    $324$
}
\solutionspace{10pt}

\tasknumber{4}%
\task{%
    Установите каждой букве в соответствие ровно одну цифру и запишите ответ (только цифры, без других символов).

    А) индуктивность, Б) индукция магнитного поля.

    1) Тл, 2) Гн, 3) с, 4) Вт.
}
\answer{%
    $21$
}
\solutionspace{10pt}

\tasknumber{5}%
\task{%
    В катушке, индуктивность которой равна $5\,\text{мГн}$, сила тока равномерно уменьшается
    с $8\,\text{А}$ до $4\,\text{А}$ за $0{,}5\,\text{c}$.
    Определите ЭДС самоиндукции, ответ выразите в мВ и округлите до целых.
}
\answer{%
    $
        \ele
        = L\frac{\abs{\Delta \eli}}{\Delta t}
        = L\frac{\abs{\eli_2 - \eli_1}}{\Delta t}
        = 5\,\text{мГн} \cdot \frac{\abs{4\,\text{А} - 8\,\text{А}}}{0{,}5\,\text{c}}
        \approx 40\,\text{мВ} \to 40
    $
}
\solutionspace{60pt}

\tasknumber{6}%
\task{%
    В катушке, индуктивность которой равна $70\,\text{мГн}$, течёт электрический ток силой $7\,\text{А}$.
    Число витков в катушке: 40.
    Определите магнитный поток, пронизывающий 1 виток катушки.
    Ответ выразите в милливеберах и округлите до целых.
}
\answer{%
    $
        \Phi_\text{1 виток}
        = \frac{\Phi}{N}
        = \frac{L\eli}{N}
        = \frac{70\,\text{мГн} \cdot 7\,\text{А}}{40}
        \approx 12{,}3\,\text{мВб}
        \to 12
    $
}
\solutionspace{60pt}

\tasknumber{7}%
\task{%
    Определите энергию магнитного поля в катушке индуктивностью $200\,\text{мГн}$, если её собственный магнитный поток равен $4\,\text{Вб}$.
}
\answer{%
    $W = \frac{\Phi^2}{2L} = \frac{\sqr{4\,\text{Вб}}}{2 \cdot 200\,\text{мГн}} \approx 40\,\text{Дж}.$
}
\solutionspace{60pt}

\tasknumber{8}%
\task{%
    В одной катушке индуктивностью $607\,\text{мГн}$ протекает электрический ток силой $864\,\text{мА}$.
    А в другой — с индуктивностью в шесть раз меньше — ток в шесть раз сильнее.
    Определите энергию магнитного поля первой катушки, индуктивность второй катушки
    и отношение энергий магнитного поля в двух этих катушках.
}
\answer{%
    $
        L_2 = \frac16L_1 = 101{,}17\,\text{мГн}, \quad
        W_1 = \frac{L_1\eli_1^2}2 = \frac{607\,\text{мГн} \cdot \sqr{864\,\text{мА}}}2 \approx 0{,}227\,\text{Дж}, \quad
        \frac{W_2}{W_1} = \frac{\frac{L_2\eli_2^2}2}{\frac{L_1\eli_1^2}2} = 6.
    $
}

\variantsplitter

\addpersonalvariant{Снежана Авдошина}

\tasknumber{1}%
\task{%
    Установите каждой букве в соответствие ровно одну цифру и запишите в ответ только цифры (без других символов).

    А) $\Delta \Phi$, Б) $\Delta t$, В) $\ele$.

    1) $t_2 - t_1$, 2) $\Phi_2 - \Phi_1$, 3) $-\frac{\Delta \Phi}{\Delta t}$, 4) $t_1 - t_2$, 5) $\Phi_1 - \Phi_2$.
}
\answer{%
    $213$
}
\solutionspace{20pt}

\tasknumber{2}%
\task{%
    Установите каждой букве в соответствие ровно одну цифру и запишите ответ (только цифры, без других символов).

    А) $\Delta \Phi$, Б) $\Delta \eli$, В) $\Phi$.

    1) $\Phi_1 - \Phi_2$, 2) $\eli_2 - \eli_1$, 3) $\Phi_2 - \Phi_1$, 4) $L\eli$, 5) $\frac{\eli}{L}$, 6) $\frac{L}{\eli}$.
}
\answer{%
    $324$
}
\solutionspace{10pt}

\tasknumber{3}%
\task{%
    Установите каждой букве в соответствие ровно одну цифру и запишите ответ (только цифры, без других символов).

    А) электрический ток, Б) электрический заряд, В) индукция магнитного поля.

    1) $q$, 2) $\eli$, 3) $\vec B$, 4) $\ele$, 5) $g$.
}
\answer{%
    $213$
}
\solutionspace{10pt}

\tasknumber{4}%
\task{%
    Установите каждой букве в соответствие ровно одну цифру и запишите ответ (только цифры, без других символов).

    А) индукция магнитного поля, Б) индуктивность.

    1) Тл, 2) Гн, 3) м, 4) Вб.
}
\answer{%
    $12$
}
\solutionspace{10pt}

\tasknumber{5}%
\task{%
    В катушке, индуктивность которой равна $5\,\text{мГн}$, сила тока равномерно уменьшается
    с $8\,\text{А}$ до $1\,\text{А}$ за $0{,}3\,\text{c}$.
    Определите ЭДС самоиндукции, ответ выразите в мВ и округлите до целых.
}
\answer{%
    $
        \ele
        = L\frac{\abs{\Delta \eli}}{\Delta t}
        = L\frac{\abs{\eli_2 - \eli_1}}{\Delta t}
        = 5\,\text{мГн} \cdot \frac{\abs{1\,\text{А} - 8\,\text{А}}}{0{,}3\,\text{c}}
        \approx 116{,}667\,\text{мВ} \to 117
    $
}
\solutionspace{60pt}

\tasknumber{6}%
\task{%
    В катушке, индуктивность которой равна $50\,\text{мГн}$, течёт электрический ток силой $6\,\text{А}$.
    Число витков в катушке: 40.
    Определите магнитный поток, пронизывающий 1 виток катушки.
    Ответ выразите в милливеберах и округлите до целых.
}
\answer{%
    $
        \Phi_\text{1 виток}
        = \frac{\Phi}{N}
        = \frac{L\eli}{N}
        = \frac{50\,\text{мГн} \cdot 6\,\text{А}}{40}
        \approx 7{,}5\,\text{мВб}
        \to 8
    $
}
\solutionspace{60pt}

\tasknumber{7}%
\task{%
    Определите энергию магнитного поля в катушке индуктивностью $300\,\text{мГн}$, если протекающий через неё ток равен $3\,\text{А}$.
}
\answer{%
    $W = \frac{L\eli^2}2 = \frac{300\,\text{мГн} \cdot \sqr{3\,\text{А}}}2 \approx 1{,}35\,\text{Дж}.$
}
\solutionspace{60pt}

\tasknumber{8}%
\task{%
    В одной катушке индуктивностью $596\,\text{мГн}$ протекает электрический ток силой $795\,\text{мА}$.
    А в другой — с индуктивностью в четыре раза меньше — ток в шесть раз сильнее.
    Определите энергию магнитного поля первой катушки, индуктивность второй катушки
    и отношение энергий магнитного поля в двух этих катушках.
}
\answer{%
    $
        L_2 = \frac14L_1 = 149\,\text{мГн}, \quad
        W_1 = \frac{L_1\eli_1^2}2 = \frac{596\,\text{мГн} \cdot \sqr{795\,\text{мА}}}2 \approx 0{,}188\,\text{Дж}, \quad
        \frac{W_2}{W_1} = \frac{\frac{L_2\eli_2^2}2}{\frac{L_1\eli_1^2}2} = 9.
    $
}

\variantsplitter

\addpersonalvariant{Марьяна Аристова}

\tasknumber{1}%
\task{%
    Установите каждой букве в соответствие ровно одну цифру и запишите в ответ только цифры (без других символов).

    А) $\ele$, Б) $\Delta t$, В) $\Delta \Phi$.

    1) $t_2 - t_1$, 2) $t_1 - t_2$, 3) $\Phi_2 - \Phi_1$, 4) $\Phi_1 - \Phi_2$, 5) $-\frac{\Delta \Phi}{\Delta t}$.
}
\answer{%
    $513$
}
\solutionspace{20pt}

\tasknumber{2}%
\task{%
    Установите каждой букве в соответствие ровно одну цифру и запишите ответ (только цифры, без других символов).

    А) $\Delta \eli$, Б) $\Phi$, В) $\Delta \Phi$.

    1) $L\eli$, 2) $\Phi_2 - \Phi_1$, 3) $\eli_1 - \eli_2$, 4) $\frac{L}{\eli}$, 5) $\Phi_1 - \Phi_2$, 6) $\eli_2 - \eli_1$.
}
\answer{%
    $612$
}
\solutionspace{10pt}

\tasknumber{3}%
\task{%
    Установите каждой букве в соответствие ровно одну цифру и запишите ответ (только цифры, без других символов).

    А) индукция магнитного поля, Б) поток магнитной индукции, В) индуктивность.

    1) $\Phi$, 2) $\eli$, 3) $L$, 4) $g$, 5) $\vec B$.
}
\answer{%
    $513$
}
\solutionspace{10pt}

\tasknumber{4}%
\task{%
    Установите каждой букве в соответствие ровно одну цифру и запишите ответ (только цифры, без других символов).

    А) поток магнитной индукции, Б) индукция магнитного поля.

    1) Кл, 2) Тл, 3) А, 4) Вб.
}
\answer{%
    $42$
}
\solutionspace{10pt}

\tasknumber{5}%
\task{%
    В катушке, индуктивность которой равна $5\,\text{мГн}$, сила тока равномерно уменьшается
    с $8\,\text{А}$ до $4\,\text{А}$ за $0{,}3\,\text{c}$.
    Определите ЭДС самоиндукции, ответ выразите в мВ и округлите до целых.
}
\answer{%
    $
        \ele
        = L\frac{\abs{\Delta \eli}}{\Delta t}
        = L\frac{\abs{\eli_2 - \eli_1}}{\Delta t}
        = 5\,\text{мГн} \cdot \frac{\abs{4\,\text{А} - 8\,\text{А}}}{0{,}3\,\text{c}}
        \approx 66{,}667\,\text{мВ} \to 67
    $
}
\solutionspace{60pt}

\tasknumber{6}%
\task{%
    В катушке, индуктивность которой равна $70\,\text{мГн}$, течёт электрический ток силой $7\,\text{А}$.
    Число витков в катушке: 40.
    Определите магнитный поток, пронизывающий 1 виток катушки.
    Ответ выразите в милливеберах и округлите до целых.
}
\answer{%
    $
        \Phi_\text{1 виток}
        = \frac{\Phi}{N}
        = \frac{L\eli}{N}
        = \frac{70\,\text{мГн} \cdot 7\,\text{А}}{40}
        \approx 12{,}3\,\text{мВб}
        \to 12
    $
}
\solutionspace{60pt}

\tasknumber{7}%
\task{%
    Определите энергию магнитного поля в катушке индуктивностью $600\,\text{мГн}$, если её собственный магнитный поток равен $7\,\text{Вб}$.
}
\answer{%
    $W = \frac{\Phi^2}{2L} = \frac{\sqr{7\,\text{Вб}}}{2 \cdot 600\,\text{мГн}} \approx 40{,}83\,\text{Дж}.$
}
\solutionspace{60pt}

\tasknumber{8}%
\task{%
    В одной катушке индуктивностью $728\,\text{мГн}$ протекает электрический ток силой $611\,\text{мА}$.
    А в другой — с индуктивностью в четыре раза больше — ток в три раза сильнее.
    Определите энергию магнитного поля первой катушки, индуктивность второй катушки
    и отношение энергий магнитного поля в двух этих катушках.
}
\answer{%
    $
        L_2 = 4L_1 = 2912\,\text{мГн}, \quad
        W_1 = \frac{L_1\eli_1^2}2 = \frac{728\,\text{мГн} \cdot \sqr{611\,\text{мА}}}2 \approx 0{,}136\,\text{Дж}, \quad
        \frac{W_2}{W_1} = \frac{\frac{L_2\eli_2^2}2}{\frac{L_1\eli_1^2}2} = 36.
    $
}

\variantsplitter

\addpersonalvariant{Никита Иванов}

\tasknumber{1}%
\task{%
    Установите каждой букве в соответствие ровно одну цифру и запишите в ответ только цифры (без других символов).

    А) $\Delta t$, Б) $\ele$, В) $\Delta \Phi$.

    1) $\Phi_2 - \Phi_1$, 2) $\Phi_1 - \Phi_2$, 3) $t_1 - t_2$, 4) $t_2 - t_1$, 5) $-\frac{\Delta \Phi}{\Delta t}$.
}
\answer{%
    $451$
}
\solutionspace{20pt}

\tasknumber{2}%
\task{%
    Установите каждой букве в соответствие ровно одну цифру и запишите ответ (только цифры, без других символов).

    А) $\Delta \Phi$, Б) $\Phi$, В) $\Delta \eli$.

    1) $\eli_1 - \eli_2$, 2) $\eli_2 - \eli_1$, 3) $\Phi_1 - \Phi_2$, 4) $\frac{\eli}{L}$, 5) $\Phi_2 - \Phi_1$, 6) $L\eli$.
}
\answer{%
    $562$
}
\solutionspace{10pt}

\tasknumber{3}%
\task{%
    Установите каждой букве в соответствие ровно одну цифру и запишите ответ (только цифры, без других символов).

    А) индукция магнитного поля, Б) электрический заряд, В) индуктивность.

    1) $L$, 2) $\varphi$, 3) $g$, 4) $\vec B$, 5) $q$.
}
\answer{%
    $451$
}
\solutionspace{10pt}

\tasknumber{4}%
\task{%
    Установите каждой букве в соответствие ровно одну цифру и запишите ответ (только цифры, без других символов).

    А) индукция магнитного поля, Б) длина проводника.

    1) Кл, 2) А, 3) Тл, 4) м.
}
\answer{%
    $34$
}
\solutionspace{10pt}

\tasknumber{5}%
\task{%
    В катушке, индуктивность которой равна $7\,\text{мГн}$, сила тока равномерно уменьшается
    с $7\,\text{А}$ до $3\,\text{А}$ за $0{,}2\,\text{c}$.
    Определите ЭДС самоиндукции, ответ выразите в мВ и округлите до целых.
}
\answer{%
    $
        \ele
        = L\frac{\abs{\Delta \eli}}{\Delta t}
        = L\frac{\abs{\eli_2 - \eli_1}}{\Delta t}
        = 7\,\text{мГн} \cdot \frac{\abs{3\,\text{А} - 7\,\text{А}}}{0{,}2\,\text{c}}
        \approx 140\,\text{мВ} \to 140
    $
}
\solutionspace{60pt}

\tasknumber{6}%
\task{%
    В катушке, индуктивность которой равна $60\,\text{мГн}$, течёт электрический ток силой $7\,\text{А}$.
    Число витков в катушке: 30.
    Определите магнитный поток, пронизывающий 1 виток катушки.
    Ответ выразите в милливеберах и округлите до целых.
}
\answer{%
    $
        \Phi_\text{1 виток}
        = \frac{\Phi}{N}
        = \frac{L\eli}{N}
        = \frac{60\,\text{мГн} \cdot 7\,\text{А}}{30}
        \approx 14\,\text{мВб}
        \to 14
    $
}
\solutionspace{60pt}

\tasknumber{7}%
\task{%
    Определите энергию магнитного поля в катушке индуктивностью $300\,\text{мГн}$, если её собственный магнитный поток равен $3\,\text{Вб}$.
}
\answer{%
    $W = \frac{\Phi^2}{2L} = \frac{\sqr{3\,\text{Вб}}}{2 \cdot 300\,\text{мГн}} \approx 15\,\text{Дж}.$
}
\solutionspace{60pt}

\tasknumber{8}%
\task{%
    В одной катушке индуктивностью $376\,\text{мГн}$ протекает электрический ток силой $887\,\text{мА}$.
    А в другой — с индуктивностью в два раза больше — ток в четыре раза сильнее.
    Определите энергию магнитного поля первой катушки, индуктивность второй катушки
    и отношение энергий магнитного поля в двух этих катушках.
}
\answer{%
    $
        L_2 = 2L_1 = 752\,\text{мГн}, \quad
        W_1 = \frac{L_1\eli_1^2}2 = \frac{376\,\text{мГн} \cdot \sqr{887\,\text{мА}}}2 \approx 0{,}148\,\text{Дж}, \quad
        \frac{W_2}{W_1} = \frac{\frac{L_2\eli_2^2}2}{\frac{L_1\eli_1^2}2} = 32.
    $
}

\variantsplitter

\addpersonalvariant{Анастасия Князева}

\tasknumber{1}%
\task{%
    Установите каждой букве в соответствие ровно одну цифру и запишите в ответ только цифры (без других символов).

    А) $\Delta \Phi$, Б) $\Delta t$, В) $\ele$.

    1) $-\frac{\Delta \Phi}{\Delta t}$, 2) $t_1 - t_2$, 3) $\Phi_1 - \Phi_2$, 4) $\Phi_2 - \Phi_1$, 5) $t_2 - t_1$.
}
\answer{%
    $451$
}
\solutionspace{20pt}

\tasknumber{2}%
\task{%
    Установите каждой букве в соответствие ровно одну цифру и запишите ответ (только цифры, без других символов).

    А) $\Delta \Phi$, Б) $\Phi$, В) $\Delta \eli$.

    1) $\frac{\eli}{L}$, 2) $\eli_2 - \eli_1$, 3) $\frac{L}{\eli}$, 4) $\eli_1 - \eli_2$, 5) $\Phi_2 - \Phi_1$, 6) $L\eli$.
}
\answer{%
    $562$
}
\solutionspace{10pt}

\tasknumber{3}%
\task{%
    Установите каждой букве в соответствие ровно одну цифру и запишите ответ (только цифры, без других символов).

    А) индуктивность, Б) электрический заряд, В) электрический ток.

    1) $\eli$, 2) $\varphi$, 3) $g$, 4) $L$, 5) $q$.
}
\answer{%
    $451$
}
\solutionspace{10pt}

\tasknumber{4}%
\task{%
    Установите каждой букве в соответствие ровно одну цифру и запишите ответ (только цифры, без других символов).

    А) длина проводника, Б) индуктивность.

    1) м / с, 2) А, 3) м, 4) Гн.
}
\answer{%
    $34$
}
\solutionspace{10pt}

\tasknumber{5}%
\task{%
    В катушке, индуктивность которой равна $5\,\text{мГн}$, сила тока равномерно уменьшается
    с $9\,\text{А}$ до $4\,\text{А}$ за $0{,}5\,\text{c}$.
    Определите ЭДС самоиндукции, ответ выразите в мВ и округлите до целых.
}
\answer{%
    $
        \ele
        = L\frac{\abs{\Delta \eli}}{\Delta t}
        = L\frac{\abs{\eli_2 - \eli_1}}{\Delta t}
        = 5\,\text{мГн} \cdot \frac{\abs{4\,\text{А} - 9\,\text{А}}}{0{,}5\,\text{c}}
        \approx 50\,\text{мВ} \to 50
    $
}
\solutionspace{60pt}

\tasknumber{6}%
\task{%
    В катушке, индуктивность которой равна $50\,\text{мГн}$, течёт электрический ток силой $7\,\text{А}$.
    Число витков в катушке: 40.
    Определите магнитный поток, пронизывающий 1 виток катушки.
    Ответ выразите в милливеберах и округлите до целых.
}
\answer{%
    $
        \Phi_\text{1 виток}
        = \frac{\Phi}{N}
        = \frac{L\eli}{N}
        = \frac{50\,\text{мГн} \cdot 7\,\text{А}}{40}
        \approx 8{,}8\,\text{мВб}
        \to 9
    $
}
\solutionspace{60pt}

\tasknumber{7}%
\task{%
    Определите энергию магнитного поля в катушке индуктивностью $200\,\text{мГн}$, если протекающий через неё ток равен $7\,\text{А}$.
}
\answer{%
    $W = \frac{L\eli^2}2 = \frac{200\,\text{мГн} \cdot \sqr{7\,\text{А}}}2 \approx 4{,}90\,\text{Дж}.$
}
\solutionspace{60pt}

\tasknumber{8}%
\task{%
    В одной катушке индуктивностью $299\,\text{мГн}$ протекает электрический ток силой $818\,\text{мА}$.
    А в другой — с индуктивностью в три раза больше — ток в шесть раз сильнее.
    Определите энергию магнитного поля первой катушки, индуктивность второй катушки
    и отношение энергий магнитного поля в двух этих катушках.
}
\answer{%
    $
        L_2 = 3L_1 = 897\,\text{мГн}, \quad
        W_1 = \frac{L_1\eli_1^2}2 = \frac{299\,\text{мГн} \cdot \sqr{818\,\text{мА}}}2 \approx 0{,}100\,\text{Дж}, \quad
        \frac{W_2}{W_1} = \frac{\frac{L_2\eli_2^2}2}{\frac{L_1\eli_1^2}2} = 108.
    $
}

\variantsplitter

\addpersonalvariant{Елизавета Кутумова}

\tasknumber{1}%
\task{%
    Установите каждой букве в соответствие ровно одну цифру и запишите в ответ только цифры (без других символов).

    А) $\ele$, Б) $\Delta t$, В) $\Delta \Phi$.

    1) $\Phi_1 - \Phi_2$, 2) $t_2 - t_1$, 3) $\Phi_2 - \Phi_1$, 4) $-\frac{\Delta \Phi}{\Delta t}$, 5) $t_1 - t_2$.
}
\answer{%
    $423$
}
\solutionspace{20pt}

\tasknumber{2}%
\task{%
    Установите каждой букве в соответствие ровно одну цифру и запишите ответ (только цифры, без других символов).

    А) $\Delta \eli$, Б) $\Phi$, В) $\Delta \Phi$.

    1) $\eli_2 - \eli_1$, 2) $\eli_1 - \eli_2$, 3) $\Phi_2 - \Phi_1$, 4) $\Phi_1 - \Phi_2$, 5) $L\eli$, 6) $\frac{L}{\eli}$.
}
\answer{%
    $153$
}
\solutionspace{10pt}

\tasknumber{3}%
\task{%
    Установите каждой букве в соответствие ровно одну цифру и запишите ответ (только цифры, без других символов).

    А) электрический ток, Б) индукция магнитного поля, В) поток магнитной индукции.

    1) $L$, 2) $\vec B$, 3) $\Phi$, 4) $\eli$, 5) $R$.
}
\answer{%
    $423$
}
\solutionspace{10pt}

\tasknumber{4}%
\task{%
    Установите каждой букве в соответствие ровно одну цифру и запишите ответ (только цифры, без других символов).

    А) поток магнитной индукции, Б) индуктивность.

    1) Гн, 2) м / с, 3) Вб, 4) Тл.
}
\answer{%
    $31$
}
\solutionspace{10pt}

\tasknumber{5}%
\task{%
    В катушке, индуктивность которой равна $6\,\text{мГн}$, сила тока равномерно уменьшается
    с $8\,\text{А}$ до $2\,\text{А}$ за $0{,}4\,\text{c}$.
    Определите ЭДС самоиндукции, ответ выразите в мВ и округлите до целых.
}
\answer{%
    $
        \ele
        = L\frac{\abs{\Delta \eli}}{\Delta t}
        = L\frac{\abs{\eli_2 - \eli_1}}{\Delta t}
        = 6\,\text{мГн} \cdot \frac{\abs{2\,\text{А} - 8\,\text{А}}}{0{,}4\,\text{c}}
        \approx 90\,\text{мВ} \to 90
    $
}
\solutionspace{60pt}

\tasknumber{6}%
\task{%
    В катушке, индуктивность которой равна $90\,\text{мГн}$, течёт электрический ток силой $6\,\text{А}$.
    Число витков в катушке: 40.
    Определите магнитный поток, пронизывающий 1 виток катушки.
    Ответ выразите в милливеберах и округлите до целых.
}
\answer{%
    $
        \Phi_\text{1 виток}
        = \frac{\Phi}{N}
        = \frac{L\eli}{N}
        = \frac{90\,\text{мГн} \cdot 6\,\text{А}}{40}
        \approx 13{,}5\,\text{мВб}
        \to 14
    $
}
\solutionspace{60pt}

\tasknumber{7}%
\task{%
    Определите энергию магнитного поля в катушке индуктивностью $300\,\text{мГн}$, если её собственный магнитный поток равен $7\,\text{Вб}$.
}
\answer{%
    $W = \frac{\Phi^2}{2L} = \frac{\sqr{7\,\text{Вб}}}{2 \cdot 300\,\text{мГн}} \approx 81{,}67\,\text{Дж}.$
}
\solutionspace{60pt}

\tasknumber{8}%
\task{%
    В одной катушке индуктивностью $761\,\text{мГн}$ протекает электрический ток силой $657\,\text{мА}$.
    А в другой — с индуктивностью в шесть раз больше — ток в четыре раза сильнее.
    Определите энергию магнитного поля первой катушки, индуктивность второй катушки
    и отношение энергий магнитного поля в двух этих катушках.
}
\answer{%
    $
        L_2 = 6L_1 = 4566\,\text{мГн}, \quad
        W_1 = \frac{L_1\eli_1^2}2 = \frac{761\,\text{мГн} \cdot \sqr{657\,\text{мА}}}2 \approx 0{,}164\,\text{Дж}, \quad
        \frac{W_2}{W_1} = \frac{\frac{L_2\eli_2^2}2}{\frac{L_1\eli_1^2}2} = 96.
    $
}

\variantsplitter

\addpersonalvariant{Роксана Мехтиева}

\tasknumber{1}%
\task{%
    Установите каждой букве в соответствие ровно одну цифру и запишите в ответ только цифры (без других символов).

    А) $\Delta t$, Б) $\Delta \Phi$, В) $\ele$.

    1) $-\frac{\Delta \Phi}{\Delta t}$, 2) $\Phi_2 - \Phi_1$, 3) $\Phi_1 - \Phi_2$, 4) $t_2 - t_1$, 5) $t_1 - t_2$.
}
\answer{%
    $421$
}
\solutionspace{20pt}

\tasknumber{2}%
\task{%
    Установите каждой букве в соответствие ровно одну цифру и запишите ответ (только цифры, без других символов).

    А) $\Phi$, Б) $\Delta \eli$, В) $\Delta \Phi$.

    1) $L\eli$, 2) $\Phi_1 - \Phi_2$, 3) $\Phi_2 - \Phi_1$, 4) $\frac{\eli}{L}$, 5) $\eli_2 - \eli_1$, 6) $\frac{L}{\eli}$.
}
\answer{%
    $153$
}
\solutionspace{10pt}

\tasknumber{3}%
\task{%
    Установите каждой букве в соответствие ровно одну цифру и запишите ответ (только цифры, без других символов).

    А) индуктивность, Б) электрический заряд, В) поток магнитной индукции.

    1) $\Phi$, 2) $q$, 3) $\varphi$, 4) $L$, 5) $\vec B$.
}
\answer{%
    $421$
}
\solutionspace{10pt}

\tasknumber{4}%
\task{%
    Установите каждой букве в соответствие ровно одну цифру и запишите ответ (только цифры, без других символов).

    А) длина проводника, Б) индукция магнитного поля.

    1) Тл, 2) Вб, 3) м, 4) Гн.
}
\answer{%
    $31$
}
\solutionspace{10pt}

\tasknumber{5}%
\task{%
    В катушке, индуктивность которой равна $7\,\text{мГн}$, сила тока равномерно уменьшается
    с $9\,\text{А}$ до $4\,\text{А}$ за $0{,}4\,\text{c}$.
    Определите ЭДС самоиндукции, ответ выразите в мВ и округлите до целых.
}
\answer{%
    $
        \ele
        = L\frac{\abs{\Delta \eli}}{\Delta t}
        = L\frac{\abs{\eli_2 - \eli_1}}{\Delta t}
        = 7\,\text{мГн} \cdot \frac{\abs{4\,\text{А} - 9\,\text{А}}}{0{,}4\,\text{c}}
        \approx 87{,}500\,\text{мВ} \to 88
    $
}
\solutionspace{60pt}

\tasknumber{6}%
\task{%
    В катушке, индуктивность которой равна $60\,\text{мГн}$, течёт электрический ток силой $6\,\text{А}$.
    Число витков в катушке: 40.
    Определите магнитный поток, пронизывающий 1 виток катушки.
    Ответ выразите в милливеберах и округлите до целых.
}
\answer{%
    $
        \Phi_\text{1 виток}
        = \frac{\Phi}{N}
        = \frac{L\eli}{N}
        = \frac{60\,\text{мГн} \cdot 6\,\text{А}}{40}
        \approx 9\,\text{мВб}
        \to 9
    $
}
\solutionspace{60pt}

\tasknumber{7}%
\task{%
    Определите энергию магнитного поля в катушке индуктивностью $300\,\text{мГн}$, если протекающий через неё ток равен $5\,\text{А}$.
}
\answer{%
    $W = \frac{L\eli^2}2 = \frac{300\,\text{мГн} \cdot \sqr{5\,\text{А}}}2 \approx 3{,}75\,\text{Дж}.$
}
\solutionspace{60pt}

\tasknumber{8}%
\task{%
    В одной катушке индуктивностью $574\,\text{мГн}$ протекает электрический ток силой $726\,\text{мА}$.
    А в другой — с индуктивностью в пять раз меньше — ток в три раза сильнее.
    Определите энергию магнитного поля первой катушки, индуктивность второй катушки
    и отношение энергий магнитного поля в двух этих катушках.
}
\answer{%
    $
        L_2 = \frac15L_1 = 114{,}80\,\text{мГн}, \quad
        W_1 = \frac{L_1\eli_1^2}2 = \frac{574\,\text{мГн} \cdot \sqr{726\,\text{мА}}}2 \approx 0{,}151\,\text{Дж}, \quad
        \frac{W_2}{W_1} = \frac{\frac{L_2\eli_2^2}2}{\frac{L_1\eli_1^2}2} = \frac95.
    $
}

\variantsplitter

\addpersonalvariant{Дилноза Нодиршоева}

\tasknumber{1}%
\task{%
    Установите каждой букве в соответствие ровно одну цифру и запишите в ответ только цифры (без других символов).

    А) $\Delta t$, Б) $\ele$, В) $\Delta \Phi$.

    1) $-\frac{\Delta \Phi}{\Delta t}$, 2) $t_2 - t_1$, 3) $t_1 - t_2$, 4) $\Phi_2 - \Phi_1$, 5) $\Phi_1 - \Phi_2$.
}
\answer{%
    $214$
}
\solutionspace{20pt}

\tasknumber{2}%
\task{%
    Установите каждой букве в соответствие ровно одну цифру и запишите ответ (только цифры, без других символов).

    А) $\Phi$, Б) $\Delta \eli$, В) $\Delta \Phi$.

    1) $\frac{\eli}{L}$, 2) $\eli_2 - \eli_1$, 3) $L\eli$, 4) $\Phi_1 - \Phi_2$, 5) $\Phi_2 - \Phi_1$, 6) $\frac{L}{\eli}$.
}
\answer{%
    $325$
}
\solutionspace{10pt}

\tasknumber{3}%
\task{%
    Установите каждой букве в соответствие ровно одну цифру и запишите ответ (только цифры, без других символов).

    А) электрический ток, Б) поток магнитной индукции, В) электрический заряд.

    1) $\Phi$, 2) $\eli$, 3) $\ele$, 4) $q$, 5) $\vec B$.
}
\answer{%
    $214$
}
\solutionspace{10pt}

\tasknumber{4}%
\task{%
    Установите каждой букве в соответствие ровно одну цифру и запишите ответ (только цифры, без других символов).

    А) поток магнитной индукции, Б) длина проводника.

    1) Вб, 2) Тл, 3) м, 4) Вт.
}
\answer{%
    $13$
}
\solutionspace{10pt}

\tasknumber{5}%
\task{%
    В катушке, индуктивность которой равна $5\,\text{мГн}$, сила тока равномерно уменьшается
    с $9\,\text{А}$ до $1\,\text{А}$ за $0{,}2\,\text{c}$.
    Определите ЭДС самоиндукции, ответ выразите в мВ и округлите до целых.
}
\answer{%
    $
        \ele
        = L\frac{\abs{\Delta \eli}}{\Delta t}
        = L\frac{\abs{\eli_2 - \eli_1}}{\Delta t}
        = 5\,\text{мГн} \cdot \frac{\abs{1\,\text{А} - 9\,\text{А}}}{0{,}2\,\text{c}}
        \approx 200\,\text{мВ} \to 200
    $
}
\solutionspace{60pt}

\tasknumber{6}%
\task{%
    В катушке, индуктивность которой равна $60\,\text{мГн}$, течёт электрический ток силой $5\,\text{А}$.
    Число витков в катушке: 20.
    Определите магнитный поток, пронизывающий 1 виток катушки.
    Ответ выразите в милливеберах и округлите до целых.
}
\answer{%
    $
        \Phi_\text{1 виток}
        = \frac{\Phi}{N}
        = \frac{L\eli}{N}
        = \frac{60\,\text{мГн} \cdot 5\,\text{А}}{20}
        \approx 15\,\text{мВб}
        \to 15
    $
}
\solutionspace{60pt}

\tasknumber{7}%
\task{%
    Определите энергию магнитного поля в катушке индуктивностью $400\,\text{мГн}$, если протекающий через неё ток равен $6\,\text{А}$.
}
\answer{%
    $W = \frac{L\eli^2}2 = \frac{400\,\text{мГн} \cdot \sqr{6\,\text{А}}}2 \approx 7{,}20\,\text{Дж}.$
}
\solutionspace{60pt}

\tasknumber{8}%
\task{%
    В одной катушке индуктивностью $211\,\text{мГн}$ протекает электрический ток силой $588\,\text{мА}$.
    А в другой — с индуктивностью в пять раз меньше — ток в три раза сильнее.
    Определите энергию магнитного поля первой катушки, индуктивность второй катушки
    и отношение энергий магнитного поля в двух этих катушках.
}
\answer{%
    $
        L_2 = \frac15L_1 = 42{,}20\,\text{мГн}, \quad
        W_1 = \frac{L_1\eli_1^2}2 = \frac{211\,\text{мГн} \cdot \sqr{588\,\text{мА}}}2 \approx 0{,}036\,\text{Дж}, \quad
        \frac{W_2}{W_1} = \frac{\frac{L_2\eli_2^2}2}{\frac{L_1\eli_1^2}2} = \frac95.
    $
}

\variantsplitter

\addpersonalvariant{Жаклин Пантелеева}

\tasknumber{1}%
\task{%
    Установите каждой букве в соответствие ровно одну цифру и запишите в ответ только цифры (без других символов).

    А) $\ele$, Б) $\Delta \Phi$, В) $\Delta t$.

    1) $\Phi_1 - \Phi_2$, 2) $t_1 - t_2$, 3) $\Phi_2 - \Phi_1$, 4) $t_2 - t_1$, 5) $-\frac{\Delta \Phi}{\Delta t}$.
}
\answer{%
    $534$
}
\solutionspace{20pt}

\tasknumber{2}%
\task{%
    Установите каждой букве в соответствие ровно одну цифру и запишите ответ (только цифры, без других символов).

    А) $\Delta \Phi$, Б) $\Delta \eli$, В) $\Phi$.

    1) $L\eli$, 2) $\frac{L}{\eli}$, 3) $\Phi_1 - \Phi_2$, 4) $\eli_2 - \eli_1$, 5) $\frac{\eli}{L}$, 6) $\Phi_2 - \Phi_1$.
}
\answer{%
    $641$
}
\solutionspace{10pt}

\tasknumber{3}%
\task{%
    Установите каждой букве в соответствие ровно одну цифру и запишите ответ (только цифры, без других символов).

    А) индуктивность, Б) электрический заряд, В) электрический ток.

    1) $\vec B$, 2) $R$, 3) $q$, 4) $\eli$, 5) $L$.
}
\answer{%
    $534$
}
\solutionspace{10pt}

\tasknumber{4}%
\task{%
    Установите каждой букве в соответствие ровно одну цифру и запишите ответ (только цифры, без других символов).

    А) время, Б) индукция магнитного поля.

    1) Вт, 2) Тл, 3) А, 4) с.
}
\answer{%
    $42$
}
\solutionspace{10pt}

\tasknumber{5}%
\task{%
    В катушке, индуктивность которой равна $4\,\text{мГн}$, сила тока равномерно уменьшается
    с $7\,\text{А}$ до $2\,\text{А}$ за $0{,}3\,\text{c}$.
    Определите ЭДС самоиндукции, ответ выразите в мВ и округлите до целых.
}
\answer{%
    $
        \ele
        = L\frac{\abs{\Delta \eli}}{\Delta t}
        = L\frac{\abs{\eli_2 - \eli_1}}{\Delta t}
        = 4\,\text{мГн} \cdot \frac{\abs{2\,\text{А} - 7\,\text{А}}}{0{,}3\,\text{c}}
        \approx 66{,}667\,\text{мВ} \to 67
    $
}
\solutionspace{60pt}

\tasknumber{6}%
\task{%
    В катушке, индуктивность которой равна $90\,\text{мГн}$, течёт электрический ток силой $5\,\text{А}$.
    Число витков в катушке: 30.
    Определите магнитный поток, пронизывающий 1 виток катушки.
    Ответ выразите в милливеберах и округлите до целых.
}
\answer{%
    $
        \Phi_\text{1 виток}
        = \frac{\Phi}{N}
        = \frac{L\eli}{N}
        = \frac{90\,\text{мГн} \cdot 5\,\text{А}}{30}
        \approx 15\,\text{мВб}
        \to 15
    $
}
\solutionspace{60pt}

\tasknumber{7}%
\task{%
    Определите энергию магнитного поля в катушке индуктивностью $600\,\text{мГн}$, если её собственный магнитный поток равен $3\,\text{Вб}$.
}
\answer{%
    $W = \frac{\Phi^2}{2L} = \frac{\sqr{3\,\text{Вб}}}{2 \cdot 600\,\text{мГн}} \approx 7{,}50\,\text{Дж}.$
}
\solutionspace{60pt}

\tasknumber{8}%
\task{%
    В одной катушке индуктивностью $365\,\text{мГн}$ протекает электрический ток силой $450\,\text{мА}$.
    А в другой — с индуктивностью в три раза больше — ток в три раза сильнее.
    Определите энергию магнитного поля первой катушки, индуктивность второй катушки
    и отношение энергий магнитного поля в двух этих катушках.
}
\answer{%
    $
        L_2 = 3L_1 = 1095\,\text{мГн}, \quad
        W_1 = \frac{L_1\eli_1^2}2 = \frac{365\,\text{мГн} \cdot \sqr{450\,\text{мА}}}2 \approx 0{,}037\,\text{Дж}, \quad
        \frac{W_2}{W_1} = \frac{\frac{L_2\eli_2^2}2}{\frac{L_1\eli_1^2}2} = 27.
    $
}

\variantsplitter

\addpersonalvariant{Артём Переверзев}

\tasknumber{1}%
\task{%
    Установите каждой букве в соответствие ровно одну цифру и запишите в ответ только цифры (без других символов).

    А) $\Delta t$, Б) $\ele$, В) $\Delta \Phi$.

    1) $t_2 - t_1$, 2) $-\frac{\Delta \Phi}{\Delta t}$, 3) $\Phi_2 - \Phi_1$, 4) $t_1 - t_2$, 5) $\Phi_1 - \Phi_2$.
}
\answer{%
    $123$
}
\solutionspace{20pt}

\tasknumber{2}%
\task{%
    Установите каждой букве в соответствие ровно одну цифру и запишите ответ (только цифры, без других символов).

    А) $\Phi$, Б) $\Delta \eli$, В) $\Delta \Phi$.

    1) $\Phi_1 - \Phi_2$, 2) $L\eli$, 3) $\eli_2 - \eli_1$, 4) $\Phi_2 - \Phi_1$, 5) $\eli_1 - \eli_2$, 6) $\frac{\eli}{L}$.
}
\answer{%
    $234$
}
\solutionspace{10pt}

\tasknumber{3}%
\task{%
    Установите каждой букве в соответствие ровно одну цифру и запишите ответ (только цифры, без других символов).

    А) поток магнитной индукции, Б) индукция магнитного поля, В) электрический ток.

    1) $\Phi$, 2) $\vec B$, 3) $\eli$, 4) $L$, 5) $\ele$.
}
\answer{%
    $123$
}
\solutionspace{10pt}

\tasknumber{4}%
\task{%
    Установите каждой букве в соответствие ровно одну цифру и запишите ответ (только цифры, без других символов).

    А) время, Б) индукция магнитного поля.

    1) с, 2) Тл, 3) Гн, 4) Вб.
}
\answer{%
    $12$
}
\solutionspace{10pt}

\tasknumber{5}%
\task{%
    В катушке, индуктивность которой равна $7\,\text{мГн}$, сила тока равномерно уменьшается
    с $8\,\text{А}$ до $4\,\text{А}$ за $0{,}2\,\text{c}$.
    Определите ЭДС самоиндукции, ответ выразите в мВ и округлите до целых.
}
\answer{%
    $
        \ele
        = L\frac{\abs{\Delta \eli}}{\Delta t}
        = L\frac{\abs{\eli_2 - \eli_1}}{\Delta t}
        = 7\,\text{мГн} \cdot \frac{\abs{4\,\text{А} - 8\,\text{А}}}{0{,}2\,\text{c}}
        \approx 140\,\text{мВ} \to 140
    $
}
\solutionspace{60pt}

\tasknumber{6}%
\task{%
    В катушке, индуктивность которой равна $70\,\text{мГн}$, течёт электрический ток силой $6\,\text{А}$.
    Число витков в катушке: 30.
    Определите магнитный поток, пронизывающий 1 виток катушки.
    Ответ выразите в милливеберах и округлите до целых.
}
\answer{%
    $
        \Phi_\text{1 виток}
        = \frac{\Phi}{N}
        = \frac{L\eli}{N}
        = \frac{70\,\text{мГн} \cdot 6\,\text{А}}{30}
        \approx 14\,\text{мВб}
        \to 14
    $
}
\solutionspace{60pt}

\tasknumber{7}%
\task{%
    Определите энергию магнитного поля в катушке индуктивностью $300\,\text{мГн}$, если её собственный магнитный поток равен $8\,\text{Вб}$.
}
\answer{%
    $W = \frac{\Phi^2}{2L} = \frac{\sqr{8\,\text{Вб}}}{2 \cdot 300\,\text{мГн}} \approx 106{,}67\,\text{Дж}.$
}
\solutionspace{60pt}

\tasknumber{8}%
\task{%
    В одной катушке индуктивностью $585\,\text{мГн}$ протекает электрический ток силой $565\,\text{мА}$.
    А в другой — с индуктивностью в шесть раз меньше — ток в пять раз сильнее.
    Определите энергию магнитного поля первой катушки, индуктивность второй катушки
    и отношение энергий магнитного поля в двух этих катушках.
}
\answer{%
    $
        L_2 = \frac16L_1 = 97{,}50\,\text{мГн}, \quad
        W_1 = \frac{L_1\eli_1^2}2 = \frac{585\,\text{мГн} \cdot \sqr{565\,\text{мА}}}2 \approx 0{,}093\,\text{Дж}, \quad
        \frac{W_2}{W_1} = \frac{\frac{L_2\eli_2^2}2}{\frac{L_1\eli_1^2}2} = \frac{25}6.
    $
}

\variantsplitter

\addpersonalvariant{Варвара Пранова}

\tasknumber{1}%
\task{%
    Установите каждой букве в соответствие ровно одну цифру и запишите в ответ только цифры (без других символов).

    А) $\ele$, Б) $\Delta \Phi$, В) $\Delta t$.

    1) $\Phi_1 - \Phi_2$, 2) $-\frac{\Delta \Phi}{\Delta t}$, 3) $\Phi_2 - \Phi_1$, 4) $t_1 - t_2$, 5) $t_2 - t_1$.
}
\answer{%
    $235$
}
\solutionspace{20pt}

\tasknumber{2}%
\task{%
    Установите каждой букве в соответствие ровно одну цифру и запишите ответ (только цифры, без других символов).

    А) $\Delta \eli$, Б) $\Delta \Phi$, В) $\Phi$.

    1) $L\eli$, 2) $\Phi_1 - \Phi_2$, 3) $\eli_2 - \eli_1$, 4) $\Phi_2 - \Phi_1$, 5) $\eli_1 - \eli_2$, 6) $\frac{\eli}{L}$.
}
\answer{%
    $341$
}
\solutionspace{10pt}

\tasknumber{3}%
\task{%
    Установите каждой букве в соответствие ровно одну цифру и запишите ответ (только цифры, без других символов).

    А) электрический заряд, Б) индуктивность, В) индукция магнитного поля.

    1) $R$, 2) $q$, 3) $L$, 4) $\eli$, 5) $\vec B$.
}
\answer{%
    $235$
}
\solutionspace{10pt}

\tasknumber{4}%
\task{%
    Установите каждой букве в соответствие ровно одну цифру и запишите ответ (только цифры, без других символов).

    А) длина проводника, Б) поток магнитной индукции.

    1) м, 2) Вб, 3) Гн, 4) Вт.
}
\answer{%
    $12$
}
\solutionspace{10pt}

\tasknumber{5}%
\task{%
    В катушке, индуктивность которой равна $4\,\text{мГн}$, сила тока равномерно уменьшается
    с $9\,\text{А}$ до $4\,\text{А}$ за $0{,}3\,\text{c}$.
    Определите ЭДС самоиндукции, ответ выразите в мВ и округлите до целых.
}
\answer{%
    $
        \ele
        = L\frac{\abs{\Delta \eli}}{\Delta t}
        = L\frac{\abs{\eli_2 - \eli_1}}{\Delta t}
        = 4\,\text{мГн} \cdot \frac{\abs{4\,\text{А} - 9\,\text{А}}}{0{,}3\,\text{c}}
        \approx 66{,}667\,\text{мВ} \to 67
    $
}
\solutionspace{60pt}

\tasknumber{6}%
\task{%
    В катушке, индуктивность которой равна $80\,\text{мГн}$, течёт электрический ток силой $7\,\text{А}$.
    Число витков в катушке: 30.
    Определите магнитный поток, пронизывающий 1 виток катушки.
    Ответ выразите в милливеберах и округлите до целых.
}
\answer{%
    $
        \Phi_\text{1 виток}
        = \frac{\Phi}{N}
        = \frac{L\eli}{N}
        = \frac{80\,\text{мГн} \cdot 7\,\text{А}}{30}
        \approx 18{,}7\,\text{мВб}
        \to 19
    $
}
\solutionspace{60pt}

\tasknumber{7}%
\task{%
    Определите энергию магнитного поля в катушке индуктивностью $400\,\text{мГн}$, если её собственный магнитный поток равен $4\,\text{Вб}$.
}
\answer{%
    $W = \frac{\Phi^2}{2L} = \frac{\sqr{4\,\text{Вб}}}{2 \cdot 400\,\text{мГн}} \approx 20\,\text{Дж}.$
}
\solutionspace{60pt}

\tasknumber{8}%
\task{%
    В одной катушке индуктивностью $783\,\text{мГн}$ протекает электрический ток силой $496\,\text{мА}$.
    А в другой — с индуктивностью в шесть раз меньше — ток в три раза сильнее.
    Определите энергию магнитного поля первой катушки, индуктивность второй катушки
    и отношение энергий магнитного поля в двух этих катушках.
}
\answer{%
    $
        L_2 = \frac16L_1 = 130{,}50\,\text{мГн}, \quad
        W_1 = \frac{L_1\eli_1^2}2 = \frac{783\,\text{мГн} \cdot \sqr{496\,\text{мА}}}2 \approx 0{,}096\,\text{Дж}, \quad
        \frac{W_2}{W_1} = \frac{\frac{L_2\eli_2^2}2}{\frac{L_1\eli_1^2}2} = \frac32.
    $
}

\variantsplitter

\addpersonalvariant{Марьям Салимова}

\tasknumber{1}%
\task{%
    Установите каждой букве в соответствие ровно одну цифру и запишите в ответ только цифры (без других символов).

    А) $\Delta t$, Б) $\Delta \Phi$, В) $\ele$.

    1) $\Phi_1 - \Phi_2$, 2) $\Phi_2 - \Phi_1$, 3) $t_2 - t_1$, 4) $-\frac{\Delta \Phi}{\Delta t}$, 5) $t_1 - t_2$.
}
\answer{%
    $324$
}
\solutionspace{20pt}

\tasknumber{2}%
\task{%
    Установите каждой букве в соответствие ровно одну цифру и запишите ответ (только цифры, без других символов).

    А) $\Delta \Phi$, Б) $\Delta \eli$, В) $\Phi$.

    1) $\eli_2 - \eli_1$, 2) $\eli_1 - \eli_2$, 3) $L\eli$, 4) $\Phi_2 - \Phi_1$, 5) $\frac{\eli}{L}$, 6) $\frac{L}{\eli}$.
}
\answer{%
    $413$
}
\solutionspace{10pt}

\tasknumber{3}%
\task{%
    Установите каждой букве в соответствие ровно одну цифру и запишите ответ (только цифры, без других символов).

    А) индуктивность, Б) поток магнитной индукции, В) электрический ток.

    1) $q$, 2) $\Phi$, 3) $L$, 4) $\eli$, 5) $\vec B$.
}
\answer{%
    $324$
}
\solutionspace{10pt}

\tasknumber{4}%
\task{%
    Установите каждой букве в соответствие ровно одну цифру и запишите ответ (только цифры, без других символов).

    А) длина проводника, Б) поток магнитной индукции.

    1) Вб, 2) м, 3) м / с, 4) Кл.
}
\answer{%
    $21$
}
\solutionspace{10pt}

\tasknumber{5}%
\task{%
    В катушке, индуктивность которой равна $4\,\text{мГн}$, сила тока равномерно уменьшается
    с $7\,\text{А}$ до $1\,\text{А}$ за $0{,}2\,\text{c}$.
    Определите ЭДС самоиндукции, ответ выразите в мВ и округлите до целых.
}
\answer{%
    $
        \ele
        = L\frac{\abs{\Delta \eli}}{\Delta t}
        = L\frac{\abs{\eli_2 - \eli_1}}{\Delta t}
        = 4\,\text{мГн} \cdot \frac{\abs{1\,\text{А} - 7\,\text{А}}}{0{,}2\,\text{c}}
        \approx 120\,\text{мВ} \to 120
    $
}
\solutionspace{60pt}

\tasknumber{6}%
\task{%
    В катушке, индуктивность которой равна $80\,\text{мГн}$, течёт электрический ток силой $6\,\text{А}$.
    Число витков в катушке: 20.
    Определите магнитный поток, пронизывающий 1 виток катушки.
    Ответ выразите в милливеберах и округлите до целых.
}
\answer{%
    $
        \Phi_\text{1 виток}
        = \frac{\Phi}{N}
        = \frac{L\eli}{N}
        = \frac{80\,\text{мГн} \cdot 6\,\text{А}}{20}
        \approx 24\,\text{мВб}
        \to 24
    $
}
\solutionspace{60pt}

\tasknumber{7}%
\task{%
    Определите энергию магнитного поля в катушке индуктивностью $200\,\text{мГн}$, если её собственный магнитный поток равен $3\,\text{Вб}$.
}
\answer{%
    $W = \frac{\Phi^2}{2L} = \frac{\sqr{3\,\text{Вб}}}{2 \cdot 200\,\text{мГн}} \approx 22{,}50\,\text{Дж}.$
}
\solutionspace{60pt}

\tasknumber{8}%
\task{%
    В одной катушке индуктивностью $486\,\text{мГн}$ протекает электрический ток силой $680\,\text{мА}$.
    А в другой — с индуктивностью в три раза меньше — ток в три раза сильнее.
    Определите энергию магнитного поля первой катушки, индуктивность второй катушки
    и отношение энергий магнитного поля в двух этих катушках.
}
\answer{%
    $
        L_2 = \frac13L_1 = 162\,\text{мГн}, \quad
        W_1 = \frac{L_1\eli_1^2}2 = \frac{486\,\text{мГн} \cdot \sqr{680\,\text{мА}}}2 \approx 0{,}112\,\text{Дж}, \quad
        \frac{W_2}{W_1} = \frac{\frac{L_2\eli_2^2}2}{\frac{L_1\eli_1^2}2} = 3.
    $
}

\variantsplitter

\addpersonalvariant{Юлия Шевченко}

\tasknumber{1}%
\task{%
    Установите каждой букве в соответствие ровно одну цифру и запишите в ответ только цифры (без других символов).

    А) $\ele$, Б) $\Delta \Phi$, В) $\Delta t$.

    1) $-\frac{\Delta \Phi}{\Delta t}$, 2) $\Phi_2 - \Phi_1$, 3) $\Phi_1 - \Phi_2$, 4) $t_2 - t_1$, 5) $t_1 - t_2$.
}
\answer{%
    $124$
}
\solutionspace{20pt}

\tasknumber{2}%
\task{%
    Установите каждой букве в соответствие ровно одну цифру и запишите ответ (только цифры, без других символов).

    А) $\Delta \eli$, Б) $\Phi$, В) $\Delta \Phi$.

    1) $\Phi_1 - \Phi_2$, 2) $\eli_2 - \eli_1$, 3) $L\eli$, 4) $\eli_1 - \eli_2$, 5) $\Phi_2 - \Phi_1$, 6) $\frac{\eli}{L}$.
}
\answer{%
    $235$
}
\solutionspace{10pt}

\tasknumber{3}%
\task{%
    Установите каждой букве в соответствие ровно одну цифру и запишите ответ (только цифры, без других символов).

    А) электрический заряд, Б) поток магнитной индукции, В) индукция магнитного поля.

    1) $q$, 2) $\Phi$, 3) $g$, 4) $\vec B$, 5) $R$.
}
\answer{%
    $124$
}
\solutionspace{10pt}

\tasknumber{4}%
\task{%
    Установите каждой букве в соответствие ровно одну цифру и запишите ответ (только цифры, без других символов).

    А) поток магнитной индукции, Б) длина проводника.

    1) Вб, 2) Тл, 3) м, 4) А.
}
\answer{%
    $13$
}
\solutionspace{10pt}

\tasknumber{5}%
\task{%
    В катушке, индуктивность которой равна $5\,\text{мГн}$, сила тока равномерно уменьшается
    с $8\,\text{А}$ до $2\,\text{А}$ за $0{,}5\,\text{c}$.
    Определите ЭДС самоиндукции, ответ выразите в мВ и округлите до целых.
}
\answer{%
    $
        \ele
        = L\frac{\abs{\Delta \eli}}{\Delta t}
        = L\frac{\abs{\eli_2 - \eli_1}}{\Delta t}
        = 5\,\text{мГн} \cdot \frac{\abs{2\,\text{А} - 8\,\text{А}}}{0{,}5\,\text{c}}
        \approx 60\,\text{мВ} \to 60
    $
}
\solutionspace{60pt}

\tasknumber{6}%
\task{%
    В катушке, индуктивность которой равна $70\,\text{мГн}$, течёт электрический ток силой $5\,\text{А}$.
    Число витков в катушке: 40.
    Определите магнитный поток, пронизывающий 1 виток катушки.
    Ответ выразите в милливеберах и округлите до целых.
}
\answer{%
    $
        \Phi_\text{1 виток}
        = \frac{\Phi}{N}
        = \frac{L\eli}{N}
        = \frac{70\,\text{мГн} \cdot 5\,\text{А}}{40}
        \approx 8{,}8\,\text{мВб}
        \to 9
    $
}
\solutionspace{60pt}

\tasknumber{7}%
\task{%
    Определите энергию магнитного поля в катушке индуктивностью $600\,\text{мГн}$, если её собственный магнитный поток равен $6\,\text{Вб}$.
}
\answer{%
    $W = \frac{\Phi^2}{2L} = \frac{\sqr{6\,\text{Вб}}}{2 \cdot 600\,\text{мГн}} \approx 30\,\text{Дж}.$
}
\solutionspace{60pt}

\tasknumber{8}%
\task{%
    В одной катушке индуктивностью $739\,\text{мГн}$ протекает электрический ток силой $749\,\text{мА}$.
    А в другой — с индуктивностью в пять раз больше — ток в шесть раз сильнее.
    Определите энергию магнитного поля первой катушки, индуктивность второй катушки
    и отношение энергий магнитного поля в двух этих катушках.
}
\answer{%
    $
        L_2 = 5L_1 = 3695\,\text{мГн}, \quad
        W_1 = \frac{L_1\eli_1^2}2 = \frac{739\,\text{мГн} \cdot \sqr{749\,\text{мА}}}2 \approx 0{,}207\,\text{Дж}, \quad
        \frac{W_2}{W_1} = \frac{\frac{L_2\eli_2^2}2}{\frac{L_1\eli_1^2}2} = 180.
    $
}
% autogenerated
