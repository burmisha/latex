\setdate{15~сентября~2021}
\setclass{11«БА»}

\addpersonalvariant{Михаил Бурмистров}

\tasknumber{1}%
\task{%
    Определите работу, которую совершает сила Ампера при перемещении проводника длиной $l = 30\,\text{см}$
    с током силой $\eli = 10\,\text{А}$ в однородном магнитном поле индукцией $B = 0{,}2\,\text{Тл}$ на расстояние $d = 80\,\text{см}$.
    Проводник перпендикулярен линиям поля и движется в направлении силы Ампера.
}
\answer{%
    $
        A   = F \cdot d = B \eli l \cdot d
            = 0{,}2\,\text{Тл} \cdot 10\,\text{А} \cdot 30\,\text{см} \cdot 80\,\text{см}
            = 0{,}480\,\text{Дж}.
    $
}
\solutionspace{80pt}

\tasknumber{2}%
\task{%
    В однородном горизонтальном магнитном поле с индукцией $B = 20\,\text{мТл}$ находится проводник,
    расположенный также горизонтально и перпендикулярно полю.
    Какой ток необходимо пустить по проводнику, чтобы он завис?
    Масса единицы длины проводника $\rho = 5\,\frac{\text{г}}{\text{м}}$, $g = 10\,\frac{\text{м}}{\text{с}^{2}}$.
}
\answer{%
    $
            mg = B\eli l, m=\rho l
            \implies \eli
                = \frac{g\rho}B
                = \frac{10\,\frac{\text{м}}{\text{с}^{2}} \cdot 5\,\frac{\text{г}}{\text{м}}}{20\,\text{мТл}}
                = 2{,}5\,\text{А}.
    $
}
\solutionspace{120pt}

\tasknumber{3}%
\task{%
    Частица, обладающая массой $m$ и положительным зарядом $q$, движется со скоростью $v$
    в магнитном поле перпендикулярно линиям его индукции.
    Индукция магнитного поля равна $B$.
    Выведите из базовых физических законов формулы для радиуса траектории частицы: сделайте рисунок, укажите вид движения и названия физических законов.
}
\answer{%
    $F = ma, F = qvB, a = v^2 / R \implies R = \frac{mv}{qB}.$
}
\solutionspace{100pt}

\tasknumber{4}%
\task{%
    Узкий пучок протонов, нейтронов и электронов влетает в однородное магнитное поле перпендикулярно его линиям (см.
    рис).
    Определите по трекам частиц 1 и 5 отношение радиусов их траекторий и их импульсов.
}
\solutionspace{100pt}

\tasknumber{5}%
\task{%
    Протон, прошедший через ускоряющую разность потенциалов, оказывается в магнитном поле индукцией $20\,\text{мТл}$
    и движется по окружности диаметром $4\,\text{мм}$.
    Сделайте рисунок, определите значение разности потенциалов
    и укажите, в какой области потенциал больше, а где меньше.
}

\variantsplitter

\addpersonalvariant{Ирина Ан}

\tasknumber{1}%
\task{%
    Определите работу, которую совершает сила Ампера при перемещении проводника длиной $l = 30\,\text{см}$
    с током силой $\eli = 20\,\text{А}$ в однородном магнитном поле индукцией $B = 0{,}2\,\text{Тл}$ на расстояние $d = 20\,\text{см}$.
    Проводник перпендикулярен линиям поля и движется в направлении силы Ампера.
}
\answer{%
    $
        A   = F \cdot d = B \eli l \cdot d
            = 0{,}2\,\text{Тл} \cdot 20\,\text{А} \cdot 30\,\text{см} \cdot 20\,\text{см}
            = 0{,}240\,\text{Дж}.
    $
}
\solutionspace{80pt}

\tasknumber{2}%
\task{%
    В однородном горизонтальном магнитном поле с индукцией $B = 100\,\text{мТл}$ находится проводник,
    расположенный также горизонтально и перпендикулярно полю.
    Какой ток необходимо пустить по проводнику, чтобы он завис?
    Масса единицы длины проводника $\rho = 20\,\frac{\text{г}}{\text{м}}$, $g = 10\,\frac{\text{м}}{\text{с}^{2}}$.
}
\answer{%
    $
            mg = B\eli l, m=\rho l
            \implies \eli
                = \frac{g\rho}B
                = \frac{10\,\frac{\text{м}}{\text{с}^{2}} \cdot 20\,\frac{\text{г}}{\text{м}}}{100\,\text{мТл}}
                = 2\,\text{А}.
    $
}
\solutionspace{120pt}

\tasknumber{3}%
\task{%
    Частица, обладающая массой $m$ и положительным зарядом $q$, движется со скоростью $v$
    в магнитном поле перпендикулярно линиям его индукции.
    Индукция магнитного поля равна $B$.
    Выведите из базовых физических законов формулы для радиуса траектории частицы: сделайте рисунок, укажите вид движения и названия физических законов.
}
\answer{%
    $F = ma, F = qvB, a = v^2 / R \implies R = \frac{mv}{qB}.$
}
\solutionspace{100pt}

\tasknumber{4}%
\task{%
    Узкий пучок протонов, нейтронов и электронов влетает в однородное магнитное поле перпендикулярно его линиям (см.
    рис).
    Определите по трекам частиц 1 и 5 отношение радиусов их траекторий и их кинетических энергий.
}
\solutionspace{100pt}

\tasknumber{5}%
\task{%
    Позитрон, прошедший через ускоряющую разность потенциалов, оказывается в магнитном поле индукцией $40\,\text{мТл}$
    и движется по окружности диаметром $6\,\text{мм}$.
    Сделайте рисунок, определите значение разности потенциалов
    и укажите, в какой области потенциал больше, а где меньше.
}

\variantsplitter

\addpersonalvariant{Софья Андрианова}

\tasknumber{1}%
\task{%
    Определите работу, которую совершает сила Ампера при перемещении проводника длиной $l = 50\,\text{см}$
    с током силой $\eli = 20\,\text{А}$ в однородном магнитном поле индукцией $B = 0{,}10\,\text{Тл}$ на расстояние $d = 50\,\text{см}$.
    Проводник перпендикулярен линиям поля и движется в направлении силы Ампера.
}
\answer{%
    $
        A   = F \cdot d = B \eli l \cdot d
            = 0{,}10\,\text{Тл} \cdot 20\,\text{А} \cdot 50\,\text{см} \cdot 50\,\text{см}
            = 0{,}500\,\text{Дж}.
    $
}
\solutionspace{80pt}

\tasknumber{2}%
\task{%
    В однородном горизонтальном магнитном поле с индукцией $B = 20\,\text{мТл}$ находится проводник,
    расположенный также горизонтально и перпендикулярно полю.
    Какой ток необходимо пустить по проводнику, чтобы он завис?
    Масса единицы длины проводника $\rho = 20\,\frac{\text{г}}{\text{м}}$, $g = 10\,\frac{\text{м}}{\text{с}^{2}}$.
}
\answer{%
    $
            mg = B\eli l, m=\rho l
            \implies \eli
                = \frac{g\rho}B
                = \frac{10\,\frac{\text{м}}{\text{с}^{2}} \cdot 20\,\frac{\text{г}}{\text{м}}}{20\,\text{мТл}}
                = 10\,\text{А}.
    $
}
\solutionspace{120pt}

\tasknumber{3}%
\task{%
    Частица, обладающая массой $m$ и положительным зарядом $q$, движется со скоростью $v$
    в магнитном поле перпендикулярно линиям его индукции.
    Индукция магнитного поля равна $B$.
    Выведите из базовых физических законов формулы для радиуса траектории частицы: сделайте рисунок, укажите вид движения и названия физических законов.
}
\answer{%
    $F = ma, F = qvB, a = v^2 / R \implies R = \frac{mv}{qB}.$
}
\solutionspace{100pt}

\tasknumber{4}%
\task{%
    Узкий пучок протонов, нейтронов и электронов влетает в однородное магнитное поле перпендикулярно его линиям (см.
    рис).
    Определите по трекам частиц 2 и 5 отношение радиусов их траекторий и их скоростей.
}
\solutionspace{100pt}

\tasknumber{5}%
\task{%
    Позитрон, прошедший через ускоряющую разность потенциалов, оказывается в магнитном поле индукцией $50\,\text{мТл}$
    и движется по окружности диаметром $8\,\text{мм}$.
    Сделайте рисунок, определите значение разности потенциалов
    и укажите, в какой области потенциал больше, а где меньше.
}

\variantsplitter

\addpersonalvariant{Владимир Артемчук}

\tasknumber{1}%
\task{%
    Определите работу, которую совершает сила Ампера при перемещении проводника длиной $l = 50\,\text{см}$
    с током силой $\eli = 10\,\text{А}$ в однородном магнитном поле индукцией $B = 0{,}2\,\text{Тл}$ на расстояние $d = 80\,\text{см}$.
    Проводник перпендикулярен линиям поля и движется в направлении силы Ампера.
}
\answer{%
    $
        A   = F \cdot d = B \eli l \cdot d
            = 0{,}2\,\text{Тл} \cdot 10\,\text{А} \cdot 50\,\text{см} \cdot 80\,\text{см}
            = 0{,}800\,\text{Дж}.
    $
}
\solutionspace{80pt}

\tasknumber{2}%
\task{%
    В однородном горизонтальном магнитном поле с индукцией $B = 20\,\text{мТл}$ находится проводник,
    расположенный также горизонтально и перпендикулярно полю.
    Какой ток необходимо пустить по проводнику, чтобы он завис?
    Масса единицы длины проводника $\rho = 40\,\frac{\text{г}}{\text{м}}$, $g = 10\,\frac{\text{м}}{\text{с}^{2}}$.
}
\answer{%
    $
            mg = B\eli l, m=\rho l
            \implies \eli
                = \frac{g\rho}B
                = \frac{10\,\frac{\text{м}}{\text{с}^{2}} \cdot 40\,\frac{\text{г}}{\text{м}}}{20\,\text{мТл}}
                = 20\,\text{А}.
    $
}
\solutionspace{120pt}

\tasknumber{3}%
\task{%
    Частица, обладающая массой $m$ и положительным зарядом $q$, движется со скоростью $v$
    в магнитном поле перпендикулярно линиям его индукции.
    Индукция магнитного поля равна $B$.
    Выведите из базовых физических законов формулы для радиуса траектории частицы: сделайте рисунок, укажите вид движения и названия физических законов.
}
\answer{%
    $F = ma, F = qvB, a = v^2 / R \implies R = \frac{mv}{qB}.$
}
\solutionspace{100pt}

\tasknumber{4}%
\task{%
    Узкий пучок протонов, нейтронов и электронов влетает в однородное магнитное поле перпендикулярно его линиям (см.
    рис).
    Определите по трекам частиц 3 и 7 отношение радиусов их траекторий и их кинетических энергий.
}
\solutionspace{100pt}

\tasknumber{5}%
\task{%
    Протон, прошедший через ускоряющую разность потенциалов, оказывается в магнитном поле индукцией $20\,\text{мТл}$
    и движется по окружности диаметром $8\,\text{мм}$.
    Сделайте рисунок, определите значение разности потенциалов
    и укажите, в какой области потенциал больше, а где меньше.
}

\variantsplitter

\addpersonalvariant{Софья Белянкина}

\tasknumber{1}%
\task{%
    Определите работу, которую совершает сила Ампера при перемещении проводника длиной $l = 40\,\text{см}$
    с током силой $\eli = 20\,\text{А}$ в однородном магнитном поле индукцией $B = 0{,}5\,\text{Тл}$ на расстояние $d = 80\,\text{см}$.
    Проводник перпендикулярен линиям поля и движется в направлении силы Ампера.
}
\answer{%
    $
        A   = F \cdot d = B \eli l \cdot d
            = 0{,}5\,\text{Тл} \cdot 20\,\text{А} \cdot 40\,\text{см} \cdot 80\,\text{см}
            = 3{,}200\,\text{Дж}.
    $
}
\solutionspace{80pt}

\tasknumber{2}%
\task{%
    В однородном горизонтальном магнитном поле с индукцией $B = 50\,\text{мТл}$ находится проводник,
    расположенный также горизонтально и перпендикулярно полю.
    Какой ток необходимо пустить по проводнику, чтобы он завис?
    Масса единицы длины проводника $\rho = 40\,\frac{\text{г}}{\text{м}}$, $g = 10\,\frac{\text{м}}{\text{с}^{2}}$.
}
\answer{%
    $
            mg = B\eli l, m=\rho l
            \implies \eli
                = \frac{g\rho}B
                = \frac{10\,\frac{\text{м}}{\text{с}^{2}} \cdot 40\,\frac{\text{г}}{\text{м}}}{50\,\text{мТл}}
                = 8\,\text{А}.
    $
}
\solutionspace{120pt}

\tasknumber{3}%
\task{%
    Частица, обладающая массой $m$ и положительным зарядом $q$, движется со скоростью $v$
    в магнитном поле перпендикулярно линиям его индукции.
    Индукция магнитного поля равна $B$.
    Выведите из базовых физических законов формулы для радиуса траектории частицы: сделайте рисунок, укажите вид движения и названия физических законов.
}
\answer{%
    $F = ma, F = qvB, a = v^2 / R \implies R = \frac{mv}{qB}.$
}
\solutionspace{100pt}

\tasknumber{4}%
\task{%
    Узкий пучок протонов, нейтронов и электронов влетает в однородное магнитное поле перпендикулярно его линиям (см.
    рис).
    Определите по трекам частиц 3 и 5 отношение радиусов их траекторий и их кинетических энергий.
}
\solutionspace{100pt}

\tasknumber{5}%
\task{%
    Электрон, прошедший через ускоряющую разность потенциалов, оказывается в магнитном поле индукцией $20\,\text{мТл}$
    и движется по окружности диаметром $8\,\text{мм}$.
    Сделайте рисунок, определите значение разности потенциалов
    и укажите, в какой области потенциал больше, а где меньше.
}

\variantsplitter

\addpersonalvariant{Варвара Егиазарян}

\tasknumber{1}%
\task{%
    Определите работу, которую совершает сила Ампера при перемещении проводника длиной $l = 40\,\text{см}$
    с током силой $\eli = 10\,\text{А}$ в однородном магнитном поле индукцией $B = 0{,}10\,\text{Тл}$ на расстояние $d = 20\,\text{см}$.
    Проводник перпендикулярен линиям поля и движется в направлении силы Ампера.
}
\answer{%
    $
        A   = F \cdot d = B \eli l \cdot d
            = 0{,}10\,\text{Тл} \cdot 10\,\text{А} \cdot 40\,\text{см} \cdot 20\,\text{см}
            = 0{,}080\,\text{Дж}.
    $
}
\solutionspace{80pt}

\tasknumber{2}%
\task{%
    В однородном горизонтальном магнитном поле с индукцией $B = 100\,\text{мТл}$ находится проводник,
    расположенный также горизонтально и перпендикулярно полю.
    Какой ток необходимо пустить по проводнику, чтобы он завис?
    Масса единицы длины проводника $\rho = 40\,\frac{\text{г}}{\text{м}}$, $g = 10\,\frac{\text{м}}{\text{с}^{2}}$.
}
\answer{%
    $
            mg = B\eli l, m=\rho l
            \implies \eli
                = \frac{g\rho}B
                = \frac{10\,\frac{\text{м}}{\text{с}^{2}} \cdot 40\,\frac{\text{г}}{\text{м}}}{100\,\text{мТл}}
                = 4\,\text{А}.
    $
}
\solutionspace{120pt}

\tasknumber{3}%
\task{%
    Частица, обладающая массой $m$ и положительным зарядом $q$, движется со скоростью $v$
    в магнитном поле перпендикулярно линиям его индукции.
    Индукция магнитного поля равна $B$.
    Выведите из базовых физических законов формулы для радиуса траектории частицы: сделайте рисунок, укажите вид движения и названия физических законов.
}
\answer{%
    $F = ma, F = qvB, a = v^2 / R \implies R = \frac{mv}{qB}.$
}
\solutionspace{100pt}

\tasknumber{4}%
\task{%
    Узкий пучок протонов, нейтронов и электронов влетает в однородное магнитное поле перпендикулярно его линиям (см.
    рис).
    Определите по трекам частиц 1 и 5 отношение радиусов их траекторий и их импульсов.
}
\solutionspace{100pt}

\tasknumber{5}%
\task{%
    Электрон, прошедший через ускоряющую разность потенциалов, оказывается в магнитном поле индукцией $40\,\text{мТл}$
    и движется по окружности диаметром $4\,\text{мм}$.
    Сделайте рисунок, определите значение разности потенциалов
    и укажите, в какой области потенциал больше, а где меньше.
}

\variantsplitter

\addpersonalvariant{Владислав Емелин}

\tasknumber{1}%
\task{%
    Определите работу, которую совершает сила Ампера при перемещении проводника длиной $l = 20\,\text{см}$
    с током силой $\eli = 20\,\text{А}$ в однородном магнитном поле индукцией $B = 0{,}5\,\text{Тл}$ на расстояние $d = 50\,\text{см}$.
    Проводник перпендикулярен линиям поля и движется в направлении силы Ампера.
}
\answer{%
    $
        A   = F \cdot d = B \eli l \cdot d
            = 0{,}5\,\text{Тл} \cdot 20\,\text{А} \cdot 20\,\text{см} \cdot 50\,\text{см}
            = 1\,\text{Дж}.
    $
}
\solutionspace{80pt}

\tasknumber{2}%
\task{%
    В однородном горизонтальном магнитном поле с индукцией $B = 20\,\text{мТл}$ находится проводник,
    расположенный также горизонтально и перпендикулярно полю.
    Какой ток необходимо пустить по проводнику, чтобы он завис?
    Масса единицы длины проводника $\rho = 5\,\frac{\text{г}}{\text{м}}$, $g = 10\,\frac{\text{м}}{\text{с}^{2}}$.
}
\answer{%
    $
            mg = B\eli l, m=\rho l
            \implies \eli
                = \frac{g\rho}B
                = \frac{10\,\frac{\text{м}}{\text{с}^{2}} \cdot 5\,\frac{\text{г}}{\text{м}}}{20\,\text{мТл}}
                = 2{,}5\,\text{А}.
    $
}
\solutionspace{120pt}

\tasknumber{3}%
\task{%
    Частица, обладающая массой $m$ и положительным зарядом $q$, движется со скоростью $v$
    в магнитном поле перпендикулярно линиям его индукции.
    Индукция магнитного поля равна $B$.
    Выведите из базовых физических законов формулы для радиуса траектории частицы: сделайте рисунок, укажите вид движения и названия физических законов.
}
\answer{%
    $F = ma, F = qvB, a = v^2 / R \implies R = \frac{mv}{qB}.$
}
\solutionspace{100pt}

\tasknumber{4}%
\task{%
    Узкий пучок протонов, нейтронов и электронов влетает в однородное магнитное поле перпендикулярно его линиям (см.
    рис).
    Определите по трекам частиц 2 и 7 отношение радиусов их траекторий и их импульсов.
}
\solutionspace{100pt}

\tasknumber{5}%
\task{%
    Позитрон, прошедший через ускоряющую разность потенциалов, оказывается в магнитном поле индукцией $50\,\text{мТл}$
    и движется по окружности диаметром $4\,\text{мм}$.
    Сделайте рисунок, определите значение разности потенциалов
    и укажите, в какой области потенциал больше, а где меньше.
}

\variantsplitter

\addpersonalvariant{Артём Жичин}

\tasknumber{1}%
\task{%
    Определите работу, которую совершает сила Ампера при перемещении проводника длиной $l = 30\,\text{см}$
    с током силой $\eli = 5\,\text{А}$ в однородном магнитном поле индукцией $B = 0{,}5\,\text{Тл}$ на расстояние $d = 50\,\text{см}$.
    Проводник перпендикулярен линиям поля и движется в направлении силы Ампера.
}
\answer{%
    $
        A   = F \cdot d = B \eli l \cdot d
            = 0{,}5\,\text{Тл} \cdot 5\,\text{А} \cdot 30\,\text{см} \cdot 50\,\text{см}
            = 0{,}375\,\text{Дж}.
    $
}
\solutionspace{80pt}

\tasknumber{2}%
\task{%
    В однородном горизонтальном магнитном поле с индукцией $B = 10\,\text{мТл}$ находится проводник,
    расположенный также горизонтально и перпендикулярно полю.
    Какой ток необходимо пустить по проводнику, чтобы он завис?
    Масса единицы длины проводника $\rho = 10\,\frac{\text{г}}{\text{м}}$, $g = 10\,\frac{\text{м}}{\text{с}^{2}}$.
}
\answer{%
    $
            mg = B\eli l, m=\rho l
            \implies \eli
                = \frac{g\rho}B
                = \frac{10\,\frac{\text{м}}{\text{с}^{2}} \cdot 10\,\frac{\text{г}}{\text{м}}}{10\,\text{мТл}}
                = 10\,\text{А}.
    $
}
\solutionspace{120pt}

\tasknumber{3}%
\task{%
    Частица, обладающая массой $m$ и положительным зарядом $q$, движется со скоростью $v$
    в магнитном поле перпендикулярно линиям его индукции.
    Индукция магнитного поля равна $B$.
    Выведите из базовых физических законов формулы для радиуса траектории частицы: сделайте рисунок, укажите вид движения и названия физических законов.
}
\answer{%
    $F = ma, F = qvB, a = v^2 / R \implies R = \frac{mv}{qB}.$
}
\solutionspace{100pt}

\tasknumber{4}%
\task{%
    Узкий пучок протонов, нейтронов и электронов влетает в однородное магнитное поле перпендикулярно его линиям (см.
    рис).
    Определите по трекам частиц 3 и 5 отношение радиусов их траекторий и их кинетических энергий.
}
\solutionspace{100pt}

\tasknumber{5}%
\task{%
    Протон, прошедший через ускоряющую разность потенциалов, оказывается в магнитном поле индукцией $40\,\text{мТл}$
    и движется по окружности диаметром $4\,\text{мм}$.
    Сделайте рисунок, определите значение разности потенциалов
    и укажите, в какой области потенциал больше, а где меньше.
}

\variantsplitter

\addpersonalvariant{Дарья Кошман}

\tasknumber{1}%
\task{%
    Определите работу, которую совершает сила Ампера при перемещении проводника длиной $l = 40\,\text{см}$
    с током силой $\eli = 20\,\text{А}$ в однородном магнитном поле индукцией $B = 0{,}5\,\text{Тл}$ на расстояние $d = 20\,\text{см}$.
    Проводник перпендикулярен линиям поля и движется в направлении силы Ампера.
}
\answer{%
    $
        A   = F \cdot d = B \eli l \cdot d
            = 0{,}5\,\text{Тл} \cdot 20\,\text{А} \cdot 40\,\text{см} \cdot 20\,\text{см}
            = 0{,}800\,\text{Дж}.
    $
}
\solutionspace{80pt}

\tasknumber{2}%
\task{%
    В однородном горизонтальном магнитном поле с индукцией $B = 10\,\text{мТл}$ находится проводник,
    расположенный также горизонтально и перпендикулярно полю.
    Какой ток необходимо пустить по проводнику, чтобы он завис?
    Масса единицы длины проводника $\rho = 5\,\frac{\text{г}}{\text{м}}$, $g = 10\,\frac{\text{м}}{\text{с}^{2}}$.
}
\answer{%
    $
            mg = B\eli l, m=\rho l
            \implies \eli
                = \frac{g\rho}B
                = \frac{10\,\frac{\text{м}}{\text{с}^{2}} \cdot 5\,\frac{\text{г}}{\text{м}}}{10\,\text{мТл}}
                = 5\,\text{А}.
    $
}
\solutionspace{120pt}

\tasknumber{3}%
\task{%
    Частица, обладающая массой $m$ и положительным зарядом $q$, движется со скоростью $v$
    в магнитном поле перпендикулярно линиям его индукции.
    Индукция магнитного поля равна $B$.
    Выведите из базовых физических законов формулы для радиуса траектории частицы: сделайте рисунок, укажите вид движения и названия физических законов.
}
\answer{%
    $F = ma, F = qvB, a = v^2 / R \implies R = \frac{mv}{qB}.$
}
\solutionspace{100pt}

\tasknumber{4}%
\task{%
    Узкий пучок протонов, нейтронов и электронов влетает в однородное магнитное поле перпендикулярно его линиям (см.
    рис).
    Определите по трекам частиц 3 и 6 отношение радиусов их траекторий и их скоростей.
}
\solutionspace{100pt}

\tasknumber{5}%
\task{%
    Позитрон, прошедший через ускоряющую разность потенциалов, оказывается в магнитном поле индукцией $20\,\text{мТл}$
    и движется по окружности диаметром $4\,\text{мм}$.
    Сделайте рисунок, определите значение разности потенциалов
    и укажите, в какой области потенциал больше, а где меньше.
}

\variantsplitter

\addpersonalvariant{Анна Кузьмичёва}

\tasknumber{1}%
\task{%
    Определите работу, которую совершает сила Ампера при перемещении проводника длиной $l = 40\,\text{см}$
    с током силой $\eli = 5\,\text{А}$ в однородном магнитном поле индукцией $B = 0{,}10\,\text{Тл}$ на расстояние $d = 50\,\text{см}$.
    Проводник перпендикулярен линиям поля и движется в направлении силы Ампера.
}
\answer{%
    $
        A   = F \cdot d = B \eli l \cdot d
            = 0{,}10\,\text{Тл} \cdot 5\,\text{А} \cdot 40\,\text{см} \cdot 50\,\text{см}
            = 0{,}100\,\text{Дж}.
    $
}
\solutionspace{80pt}

\tasknumber{2}%
\task{%
    В однородном горизонтальном магнитном поле с индукцией $B = 100\,\text{мТл}$ находится проводник,
    расположенный также горизонтально и перпендикулярно полю.
    Какой ток необходимо пустить по проводнику, чтобы он завис?
    Масса единицы длины проводника $\rho = 20\,\frac{\text{г}}{\text{м}}$, $g = 10\,\frac{\text{м}}{\text{с}^{2}}$.
}
\answer{%
    $
            mg = B\eli l, m=\rho l
            \implies \eli
                = \frac{g\rho}B
                = \frac{10\,\frac{\text{м}}{\text{с}^{2}} \cdot 20\,\frac{\text{г}}{\text{м}}}{100\,\text{мТл}}
                = 2\,\text{А}.
    $
}
\solutionspace{120pt}

\tasknumber{3}%
\task{%
    Частица, обладающая массой $m$ и положительным зарядом $q$, движется со скоростью $v$
    в магнитном поле перпендикулярно линиям его индукции.
    Индукция магнитного поля равна $B$.
    Выведите из базовых физических законов формулы для радиуса траектории частицы: сделайте рисунок, укажите вид движения и названия физических законов.
}
\answer{%
    $F = ma, F = qvB, a = v^2 / R \implies R = \frac{mv}{qB}.$
}
\solutionspace{100pt}

\tasknumber{4}%
\task{%
    Узкий пучок протонов, нейтронов и электронов влетает в однородное магнитное поле перпендикулярно его линиям (см.
    рис).
    Определите по трекам частиц 1 и 7 отношение радиусов их траекторий и их импульсов.
}
\solutionspace{100pt}

\tasknumber{5}%
\task{%
    Электрон, прошедший через ускоряющую разность потенциалов, оказывается в магнитном поле индукцией $20\,\text{мТл}$
    и движется по окружности диаметром $4\,\text{мм}$.
    Сделайте рисунок, определите значение разности потенциалов
    и укажите, в какой области потенциал больше, а где меньше.
}

\variantsplitter

\addpersonalvariant{Алёна Куприянова}

\tasknumber{1}%
\task{%
    Определите работу, которую совершает сила Ампера при перемещении проводника длиной $l = 50\,\text{см}$
    с током силой $\eli = 5\,\text{А}$ в однородном магнитном поле индукцией $B = 0{,}5\,\text{Тл}$ на расстояние $d = 50\,\text{см}$.
    Проводник перпендикулярен линиям поля и движется в направлении силы Ампера.
}
\answer{%
    $
        A   = F \cdot d = B \eli l \cdot d
            = 0{,}5\,\text{Тл} \cdot 5\,\text{А} \cdot 50\,\text{см} \cdot 50\,\text{см}
            = 0{,}625\,\text{Дж}.
    $
}
\solutionspace{80pt}

\tasknumber{2}%
\task{%
    В однородном горизонтальном магнитном поле с индукцией $B = 20\,\text{мТл}$ находится проводник,
    расположенный также горизонтально и перпендикулярно полю.
    Какой ток необходимо пустить по проводнику, чтобы он завис?
    Масса единицы длины проводника $\rho = 5\,\frac{\text{г}}{\text{м}}$, $g = 10\,\frac{\text{м}}{\text{с}^{2}}$.
}
\answer{%
    $
            mg = B\eli l, m=\rho l
            \implies \eli
                = \frac{g\rho}B
                = \frac{10\,\frac{\text{м}}{\text{с}^{2}} \cdot 5\,\frac{\text{г}}{\text{м}}}{20\,\text{мТл}}
                = 2{,}5\,\text{А}.
    $
}
\solutionspace{120pt}

\tasknumber{3}%
\task{%
    Частица, обладающая массой $m$ и положительным зарядом $q$, движется со скоростью $v$
    в магнитном поле перпендикулярно линиям его индукции.
    Индукция магнитного поля равна $B$.
    Выведите из базовых физических законов формулы для радиуса траектории частицы: сделайте рисунок, укажите вид движения и названия физических законов.
}
\answer{%
    $F = ma, F = qvB, a = v^2 / R \implies R = \frac{mv}{qB}.$
}
\solutionspace{100pt}

\tasknumber{4}%
\task{%
    Узкий пучок протонов, нейтронов и электронов влетает в однородное магнитное поле перпендикулярно его линиям (см.
    рис).
    Определите по трекам частиц 1 и 6 отношение радиусов их траекторий и их скоростей.
}
\solutionspace{100pt}

\tasknumber{5}%
\task{%
    Позитрон, прошедший через ускоряющую разность потенциалов, оказывается в магнитном поле индукцией $20\,\text{мТл}$
    и движется по окружности диаметром $4\,\text{мм}$.
    Сделайте рисунок, определите значение разности потенциалов
    и укажите, в какой области потенциал больше, а где меньше.
}

\variantsplitter

\addpersonalvariant{Ярослав Лавровский}

\tasknumber{1}%
\task{%
    Определите работу, которую совершает сила Ампера при перемещении проводника длиной $l = 50\,\text{см}$
    с током силой $\eli = 10\,\text{А}$ в однородном магнитном поле индукцией $B = 0{,}2\,\text{Тл}$ на расстояние $d = 80\,\text{см}$.
    Проводник перпендикулярен линиям поля и движется в направлении силы Ампера.
}
\answer{%
    $
        A   = F \cdot d = B \eli l \cdot d
            = 0{,}2\,\text{Тл} \cdot 10\,\text{А} \cdot 50\,\text{см} \cdot 80\,\text{см}
            = 0{,}800\,\text{Дж}.
    $
}
\solutionspace{80pt}

\tasknumber{2}%
\task{%
    В однородном горизонтальном магнитном поле с индукцией $B = 50\,\text{мТл}$ находится проводник,
    расположенный также горизонтально и перпендикулярно полю.
    Какой ток необходимо пустить по проводнику, чтобы он завис?
    Масса единицы длины проводника $\rho = 5\,\frac{\text{г}}{\text{м}}$, $g = 10\,\frac{\text{м}}{\text{с}^{2}}$.
}
\answer{%
    $
            mg = B\eli l, m=\rho l
            \implies \eli
                = \frac{g\rho}B
                = \frac{10\,\frac{\text{м}}{\text{с}^{2}} \cdot 5\,\frac{\text{г}}{\text{м}}}{50\,\text{мТл}}
                = 1\,\text{А}.
    $
}
\solutionspace{120pt}

\tasknumber{3}%
\task{%
    Частица, обладающая массой $m$ и положительным зарядом $q$, движется со скоростью $v$
    в магнитном поле перпендикулярно линиям его индукции.
    Индукция магнитного поля равна $B$.
    Выведите из базовых физических законов формулы для радиуса траектории частицы: сделайте рисунок, укажите вид движения и названия физических законов.
}
\answer{%
    $F = ma, F = qvB, a = v^2 / R \implies R = \frac{mv}{qB}.$
}
\solutionspace{100pt}

\tasknumber{4}%
\task{%
    Узкий пучок протонов, нейтронов и электронов влетает в однородное магнитное поле перпендикулярно его линиям (см.
    рис).
    Определите по трекам частиц 1 и 5 отношение радиусов их траекторий и их импульсов.
}
\solutionspace{100pt}

\tasknumber{5}%
\task{%
    Позитрон, прошедший через ускоряющую разность потенциалов, оказывается в магнитном поле индукцией $20\,\text{мТл}$
    и движется по окружности диаметром $4\,\text{мм}$.
    Сделайте рисунок, определите значение разности потенциалов
    и укажите, в какой области потенциал больше, а где меньше.
}

\variantsplitter

\addpersonalvariant{Анастасия Ламанова}

\tasknumber{1}%
\task{%
    Определите работу, которую совершает сила Ампера при перемещении проводника длиной $l = 50\,\text{см}$
    с током силой $\eli = 5\,\text{А}$ в однородном магнитном поле индукцией $B = 0{,}5\,\text{Тл}$ на расстояние $d = 80\,\text{см}$.
    Проводник перпендикулярен линиям поля и движется в направлении силы Ампера.
}
\answer{%
    $
        A   = F \cdot d = B \eli l \cdot d
            = 0{,}5\,\text{Тл} \cdot 5\,\text{А} \cdot 50\,\text{см} \cdot 80\,\text{см}
            = 1\,\text{Дж}.
    $
}
\solutionspace{80pt}

\tasknumber{2}%
\task{%
    В однородном горизонтальном магнитном поле с индукцией $B = 20\,\text{мТл}$ находится проводник,
    расположенный также горизонтально и перпендикулярно полю.
    Какой ток необходимо пустить по проводнику, чтобы он завис?
    Масса единицы длины проводника $\rho = 5\,\frac{\text{г}}{\text{м}}$, $g = 10\,\frac{\text{м}}{\text{с}^{2}}$.
}
\answer{%
    $
            mg = B\eli l, m=\rho l
            \implies \eli
                = \frac{g\rho}B
                = \frac{10\,\frac{\text{м}}{\text{с}^{2}} \cdot 5\,\frac{\text{г}}{\text{м}}}{20\,\text{мТл}}
                = 2{,}5\,\text{А}.
    $
}
\solutionspace{120pt}

\tasknumber{3}%
\task{%
    Частица, обладающая массой $m$ и положительным зарядом $q$, движется со скоростью $v$
    в магнитном поле перпендикулярно линиям его индукции.
    Индукция магнитного поля равна $B$.
    Выведите из базовых физических законов формулы для радиуса траектории частицы: сделайте рисунок, укажите вид движения и названия физических законов.
}
\answer{%
    $F = ma, F = qvB, a = v^2 / R \implies R = \frac{mv}{qB}.$
}
\solutionspace{100pt}

\tasknumber{4}%
\task{%
    Узкий пучок протонов, нейтронов и электронов влетает в однородное магнитное поле перпендикулярно его линиям (см.
    рис).
    Определите по трекам частиц 1 и 7 отношение радиусов их траекторий и их кинетических энергий.
}
\solutionspace{100pt}

\tasknumber{5}%
\task{%
    Электрон, прошедший через ускоряющую разность потенциалов, оказывается в магнитном поле индукцией $40\,\text{мТл}$
    и движется по окружности диаметром $6\,\text{мм}$.
    Сделайте рисунок, определите значение разности потенциалов
    и укажите, в какой области потенциал больше, а где меньше.
}

\variantsplitter

\addpersonalvariant{Виктория Легонькова}

\tasknumber{1}%
\task{%
    Определите работу, которую совершает сила Ампера при перемещении проводника длиной $l = 40\,\text{см}$
    с током силой $\eli = 20\,\text{А}$ в однородном магнитном поле индукцией $B = 0{,}5\,\text{Тл}$ на расстояние $d = 20\,\text{см}$.
    Проводник перпендикулярен линиям поля и движется в направлении силы Ампера.
}
\answer{%
    $
        A   = F \cdot d = B \eli l \cdot d
            = 0{,}5\,\text{Тл} \cdot 20\,\text{А} \cdot 40\,\text{см} \cdot 20\,\text{см}
            = 0{,}800\,\text{Дж}.
    $
}
\solutionspace{80pt}

\tasknumber{2}%
\task{%
    В однородном горизонтальном магнитном поле с индукцией $B = 50\,\text{мТл}$ находится проводник,
    расположенный также горизонтально и перпендикулярно полю.
    Какой ток необходимо пустить по проводнику, чтобы он завис?
    Масса единицы длины проводника $\rho = 40\,\frac{\text{г}}{\text{м}}$, $g = 10\,\frac{\text{м}}{\text{с}^{2}}$.
}
\answer{%
    $
            mg = B\eli l, m=\rho l
            \implies \eli
                = \frac{g\rho}B
                = \frac{10\,\frac{\text{м}}{\text{с}^{2}} \cdot 40\,\frac{\text{г}}{\text{м}}}{50\,\text{мТл}}
                = 8\,\text{А}.
    $
}
\solutionspace{120pt}

\tasknumber{3}%
\task{%
    Частица, обладающая массой $m$ и положительным зарядом $q$, движется со скоростью $v$
    в магнитном поле перпендикулярно линиям его индукции.
    Индукция магнитного поля равна $B$.
    Выведите из базовых физических законов формулы для радиуса траектории частицы: сделайте рисунок, укажите вид движения и названия физических законов.
}
\answer{%
    $F = ma, F = qvB, a = v^2 / R \implies R = \frac{mv}{qB}.$
}
\solutionspace{100pt}

\tasknumber{4}%
\task{%
    Узкий пучок протонов, нейтронов и электронов влетает в однородное магнитное поле перпендикулярно его линиям (см.
    рис).
    Определите по трекам частиц 1 и 7 отношение радиусов их траекторий и их кинетических энергий.
}
\solutionspace{100pt}

\tasknumber{5}%
\task{%
    Позитрон, прошедший через ускоряющую разность потенциалов, оказывается в магнитном поле индукцией $20\,\text{мТл}$
    и движется по окружности диаметром $8\,\text{мм}$.
    Сделайте рисунок, определите значение разности потенциалов
    и укажите, в какой области потенциал больше, а где меньше.
}

\variantsplitter

\addpersonalvariant{Семён Мартынов}

\tasknumber{1}%
\task{%
    Определите работу, которую совершает сила Ампера при перемещении проводника длиной $l = 20\,\text{см}$
    с током силой $\eli = 20\,\text{А}$ в однородном магнитном поле индукцией $B = 0{,}10\,\text{Тл}$ на расстояние $d = 80\,\text{см}$.
    Проводник перпендикулярен линиям поля и движется в направлении силы Ампера.
}
\answer{%
    $
        A   = F \cdot d = B \eli l \cdot d
            = 0{,}10\,\text{Тл} \cdot 20\,\text{А} \cdot 20\,\text{см} \cdot 80\,\text{см}
            = 0{,}320\,\text{Дж}.
    $
}
\solutionspace{80pt}

\tasknumber{2}%
\task{%
    В однородном горизонтальном магнитном поле с индукцией $B = 100\,\text{мТл}$ находится проводник,
    расположенный также горизонтально и перпендикулярно полю.
    Какой ток необходимо пустить по проводнику, чтобы он завис?
    Масса единицы длины проводника $\rho = 20\,\frac{\text{г}}{\text{м}}$, $g = 10\,\frac{\text{м}}{\text{с}^{2}}$.
}
\answer{%
    $
            mg = B\eli l, m=\rho l
            \implies \eli
                = \frac{g\rho}B
                = \frac{10\,\frac{\text{м}}{\text{с}^{2}} \cdot 20\,\frac{\text{г}}{\text{м}}}{100\,\text{мТл}}
                = 2\,\text{А}.
    $
}
\solutionspace{120pt}

\tasknumber{3}%
\task{%
    Частица, обладающая массой $m$ и положительным зарядом $q$, движется со скоростью $v$
    в магнитном поле перпендикулярно линиям его индукции.
    Индукция магнитного поля равна $B$.
    Выведите из базовых физических законов формулы для радиуса траектории частицы: сделайте рисунок, укажите вид движения и названия физических законов.
}
\answer{%
    $F = ma, F = qvB, a = v^2 / R \implies R = \frac{mv}{qB}.$
}
\solutionspace{100pt}

\tasknumber{4}%
\task{%
    Узкий пучок протонов, нейтронов и электронов влетает в однородное магнитное поле перпендикулярно его линиям (см.
    рис).
    Определите по трекам частиц 1 и 5 отношение радиусов их траекторий и их импульсов.
}
\solutionspace{100pt}

\tasknumber{5}%
\task{%
    Протон, прошедший через ускоряющую разность потенциалов, оказывается в магнитном поле индукцией $20\,\text{мТл}$
    и движется по окружности диаметром $6\,\text{мм}$.
    Сделайте рисунок, определите значение разности потенциалов
    и укажите, в какой области потенциал больше, а где меньше.
}

\variantsplitter

\addpersonalvariant{Варвара Минаева}

\tasknumber{1}%
\task{%
    Определите работу, которую совершает сила Ампера при перемещении проводника длиной $l = 30\,\text{см}$
    с током силой $\eli = 5\,\text{А}$ в однородном магнитном поле индукцией $B = 0{,}2\,\text{Тл}$ на расстояние $d = 20\,\text{см}$.
    Проводник перпендикулярен линиям поля и движется в направлении силы Ампера.
}
\answer{%
    $
        A   = F \cdot d = B \eli l \cdot d
            = 0{,}2\,\text{Тл} \cdot 5\,\text{А} \cdot 30\,\text{см} \cdot 20\,\text{см}
            = 0{,}060\,\text{Дж}.
    $
}
\solutionspace{80pt}

\tasknumber{2}%
\task{%
    В однородном горизонтальном магнитном поле с индукцией $B = 20\,\text{мТл}$ находится проводник,
    расположенный также горизонтально и перпендикулярно полю.
    Какой ток необходимо пустить по проводнику, чтобы он завис?
    Масса единицы длины проводника $\rho = 5\,\frac{\text{г}}{\text{м}}$, $g = 10\,\frac{\text{м}}{\text{с}^{2}}$.
}
\answer{%
    $
            mg = B\eli l, m=\rho l
            \implies \eli
                = \frac{g\rho}B
                = \frac{10\,\frac{\text{м}}{\text{с}^{2}} \cdot 5\,\frac{\text{г}}{\text{м}}}{20\,\text{мТл}}
                = 2{,}5\,\text{А}.
    $
}
\solutionspace{120pt}

\tasknumber{3}%
\task{%
    Частица, обладающая массой $m$ и положительным зарядом $q$, движется со скоростью $v$
    в магнитном поле перпендикулярно линиям его индукции.
    Индукция магнитного поля равна $B$.
    Выведите из базовых физических законов формулы для радиуса траектории частицы: сделайте рисунок, укажите вид движения и названия физических законов.
}
\answer{%
    $F = ma, F = qvB, a = v^2 / R \implies R = \frac{mv}{qB}.$
}
\solutionspace{100pt}

\tasknumber{4}%
\task{%
    Узкий пучок протонов, нейтронов и электронов влетает в однородное магнитное поле перпендикулярно его линиям (см.
    рис).
    Определите по трекам частиц 2 и 5 отношение радиусов их траекторий и их кинетических энергий.
}
\solutionspace{100pt}

\tasknumber{5}%
\task{%
    Электрон, прошедший через ускоряющую разность потенциалов, оказывается в магнитном поле индукцией $40\,\text{мТл}$
    и движется по окружности диаметром $4\,\text{мм}$.
    Сделайте рисунок, определите значение разности потенциалов
    и укажите, в какой области потенциал больше, а где меньше.
}

\variantsplitter

\addpersonalvariant{Леонид Никитин}

\tasknumber{1}%
\task{%
    Определите работу, которую совершает сила Ампера при перемещении проводника длиной $l = 20\,\text{см}$
    с током силой $\eli = 10\,\text{А}$ в однородном магнитном поле индукцией $B = 0{,}2\,\text{Тл}$ на расстояние $d = 20\,\text{см}$.
    Проводник перпендикулярен линиям поля и движется в направлении силы Ампера.
}
\answer{%
    $
        A   = F \cdot d = B \eli l \cdot d
            = 0{,}2\,\text{Тл} \cdot 10\,\text{А} \cdot 20\,\text{см} \cdot 20\,\text{см}
            = 0{,}080\,\text{Дж}.
    $
}
\solutionspace{80pt}

\tasknumber{2}%
\task{%
    В однородном горизонтальном магнитном поле с индукцией $B = 20\,\text{мТл}$ находится проводник,
    расположенный также горизонтально и перпендикулярно полю.
    Какой ток необходимо пустить по проводнику, чтобы он завис?
    Масса единицы длины проводника $\rho = 100\,\frac{\text{г}}{\text{м}}$, $g = 10\,\frac{\text{м}}{\text{с}^{2}}$.
}
\answer{%
    $
            mg = B\eli l, m=\rho l
            \implies \eli
                = \frac{g\rho}B
                = \frac{10\,\frac{\text{м}}{\text{с}^{2}} \cdot 100\,\frac{\text{г}}{\text{м}}}{20\,\text{мТл}}
                = 50\,\text{А}.
    $
}
\solutionspace{120pt}

\tasknumber{3}%
\task{%
    Частица, обладающая массой $m$ и положительным зарядом $q$, движется со скоростью $v$
    в магнитном поле перпендикулярно линиям его индукции.
    Индукция магнитного поля равна $B$.
    Выведите из базовых физических законов формулы для радиуса траектории частицы: сделайте рисунок, укажите вид движения и названия физических законов.
}
\answer{%
    $F = ma, F = qvB, a = v^2 / R \implies R = \frac{mv}{qB}.$
}
\solutionspace{100pt}

\tasknumber{4}%
\task{%
    Узкий пучок протонов, нейтронов и электронов влетает в однородное магнитное поле перпендикулярно его линиям (см.
    рис).
    Определите по трекам частиц 1 и 5 отношение радиусов их траекторий и их скоростей.
}
\solutionspace{100pt}

\tasknumber{5}%
\task{%
    Протон, прошедший через ускоряющую разность потенциалов, оказывается в магнитном поле индукцией $40\,\text{мТл}$
    и движется по окружности диаметром $8\,\text{мм}$.
    Сделайте рисунок, определите значение разности потенциалов
    и укажите, в какой области потенциал больше, а где меньше.
}

\variantsplitter

\addpersonalvariant{Тимофей Полетаев}

\tasknumber{1}%
\task{%
    Определите работу, которую совершает сила Ампера при перемещении проводника длиной $l = 20\,\text{см}$
    с током силой $\eli = 20\,\text{А}$ в однородном магнитном поле индукцией $B = 0{,}10\,\text{Тл}$ на расстояние $d = 20\,\text{см}$.
    Проводник перпендикулярен линиям поля и движется в направлении силы Ампера.
}
\answer{%
    $
        A   = F \cdot d = B \eli l \cdot d
            = 0{,}10\,\text{Тл} \cdot 20\,\text{А} \cdot 20\,\text{см} \cdot 20\,\text{см}
            = 0{,}080\,\text{Дж}.
    $
}
\solutionspace{80pt}

\tasknumber{2}%
\task{%
    В однородном горизонтальном магнитном поле с индукцией $B = 50\,\text{мТл}$ находится проводник,
    расположенный также горизонтально и перпендикулярно полю.
    Какой ток необходимо пустить по проводнику, чтобы он завис?
    Масса единицы длины проводника $\rho = 40\,\frac{\text{г}}{\text{м}}$, $g = 10\,\frac{\text{м}}{\text{с}^{2}}$.
}
\answer{%
    $
            mg = B\eli l, m=\rho l
            \implies \eli
                = \frac{g\rho}B
                = \frac{10\,\frac{\text{м}}{\text{с}^{2}} \cdot 40\,\frac{\text{г}}{\text{м}}}{50\,\text{мТл}}
                = 8\,\text{А}.
    $
}
\solutionspace{120pt}

\tasknumber{3}%
\task{%
    Частица, обладающая массой $m$ и положительным зарядом $q$, движется со скоростью $v$
    в магнитном поле перпендикулярно линиям его индукции.
    Индукция магнитного поля равна $B$.
    Выведите из базовых физических законов формулы для радиуса траектории частицы: сделайте рисунок, укажите вид движения и названия физических законов.
}
\answer{%
    $F = ma, F = qvB, a = v^2 / R \implies R = \frac{mv}{qB}.$
}
\solutionspace{100pt}

\tasknumber{4}%
\task{%
    Узкий пучок протонов, нейтронов и электронов влетает в однородное магнитное поле перпендикулярно его линиям (см.
    рис).
    Определите по трекам частиц 2 и 6 отношение радиусов их траекторий и их импульсов.
}
\solutionspace{100pt}

\tasknumber{5}%
\task{%
    Электрон, прошедший через ускоряющую разность потенциалов, оказывается в магнитном поле индукцией $50\,\text{мТл}$
    и движется по окружности диаметром $8\,\text{мм}$.
    Сделайте рисунок, определите значение разности потенциалов
    и укажите, в какой области потенциал больше, а где меньше.
}

\variantsplitter

\addpersonalvariant{Андрей Рожков}

\tasknumber{1}%
\task{%
    Определите работу, которую совершает сила Ампера при перемещении проводника длиной $l = 20\,\text{см}$
    с током силой $\eli = 5\,\text{А}$ в однородном магнитном поле индукцией $B = 0{,}10\,\text{Тл}$ на расстояние $d = 20\,\text{см}$.
    Проводник перпендикулярен линиям поля и движется в направлении силы Ампера.
}
\answer{%
    $
        A   = F \cdot d = B \eli l \cdot d
            = 0{,}10\,\text{Тл} \cdot 5\,\text{А} \cdot 20\,\text{см} \cdot 20\,\text{см}
            = 0{,}020\,\text{Дж}.
    $
}
\solutionspace{80pt}

\tasknumber{2}%
\task{%
    В однородном горизонтальном магнитном поле с индукцией $B = 20\,\text{мТл}$ находится проводник,
    расположенный также горизонтально и перпендикулярно полю.
    Какой ток необходимо пустить по проводнику, чтобы он завис?
    Масса единицы длины проводника $\rho = 100\,\frac{\text{г}}{\text{м}}$, $g = 10\,\frac{\text{м}}{\text{с}^{2}}$.
}
\answer{%
    $
            mg = B\eli l, m=\rho l
            \implies \eli
                = \frac{g\rho}B
                = \frac{10\,\frac{\text{м}}{\text{с}^{2}} \cdot 100\,\frac{\text{г}}{\text{м}}}{20\,\text{мТл}}
                = 50\,\text{А}.
    $
}
\solutionspace{120pt}

\tasknumber{3}%
\task{%
    Частица, обладающая массой $m$ и положительным зарядом $q$, движется со скоростью $v$
    в магнитном поле перпендикулярно линиям его индукции.
    Индукция магнитного поля равна $B$.
    Выведите из базовых физических законов формулы для радиуса траектории частицы: сделайте рисунок, укажите вид движения и названия физических законов.
}
\answer{%
    $F = ma, F = qvB, a = v^2 / R \implies R = \frac{mv}{qB}.$
}
\solutionspace{100pt}

\tasknumber{4}%
\task{%
    Узкий пучок протонов, нейтронов и электронов влетает в однородное магнитное поле перпендикулярно его линиям (см.
    рис).
    Определите по трекам частиц 3 и 7 отношение радиусов их траекторий и их импульсов.
}
\solutionspace{100pt}

\tasknumber{5}%
\task{%
    Позитрон, прошедший через ускоряющую разность потенциалов, оказывается в магнитном поле индукцией $20\,\text{мТл}$
    и движется по окружности диаметром $4\,\text{мм}$.
    Сделайте рисунок, определите значение разности потенциалов
    и укажите, в какой области потенциал больше, а где меньше.
}

\variantsplitter

\addpersonalvariant{Рената Таржиманова}

\tasknumber{1}%
\task{%
    Определите работу, которую совершает сила Ампера при перемещении проводника длиной $l = 50\,\text{см}$
    с током силой $\eli = 20\,\text{А}$ в однородном магнитном поле индукцией $B = 0{,}2\,\text{Тл}$ на расстояние $d = 20\,\text{см}$.
    Проводник перпендикулярен линиям поля и движется в направлении силы Ампера.
}
\answer{%
    $
        A   = F \cdot d = B \eli l \cdot d
            = 0{,}2\,\text{Тл} \cdot 20\,\text{А} \cdot 50\,\text{см} \cdot 20\,\text{см}
            = 0{,}400\,\text{Дж}.
    $
}
\solutionspace{80pt}

\tasknumber{2}%
\task{%
    В однородном горизонтальном магнитном поле с индукцией $B = 50\,\text{мТл}$ находится проводник,
    расположенный также горизонтально и перпендикулярно полю.
    Какой ток необходимо пустить по проводнику, чтобы он завис?
    Масса единицы длины проводника $\rho = 5\,\frac{\text{г}}{\text{м}}$, $g = 10\,\frac{\text{м}}{\text{с}^{2}}$.
}
\answer{%
    $
            mg = B\eli l, m=\rho l
            \implies \eli
                = \frac{g\rho}B
                = \frac{10\,\frac{\text{м}}{\text{с}^{2}} \cdot 5\,\frac{\text{г}}{\text{м}}}{50\,\text{мТл}}
                = 1\,\text{А}.
    $
}
\solutionspace{120pt}

\tasknumber{3}%
\task{%
    Частица, обладающая массой $m$ и положительным зарядом $q$, движется со скоростью $v$
    в магнитном поле перпендикулярно линиям его индукции.
    Индукция магнитного поля равна $B$.
    Выведите из базовых физических законов формулы для радиуса траектории частицы: сделайте рисунок, укажите вид движения и названия физических законов.
}
\answer{%
    $F = ma, F = qvB, a = v^2 / R \implies R = \frac{mv}{qB}.$
}
\solutionspace{100pt}

\tasknumber{4}%
\task{%
    Узкий пучок протонов, нейтронов и электронов влетает в однородное магнитное поле перпендикулярно его линиям (см.
    рис).
    Определите по трекам частиц 1 и 5 отношение радиусов их траекторий и их импульсов.
}
\solutionspace{100pt}

\tasknumber{5}%
\task{%
    Позитрон, прошедший через ускоряющую разность потенциалов, оказывается в магнитном поле индукцией $40\,\text{мТл}$
    и движется по окружности диаметром $6\,\text{мм}$.
    Сделайте рисунок, определите значение разности потенциалов
    и укажите, в какой области потенциал больше, а где меньше.
}

\variantsplitter

\addpersonalvariant{Андрей Щербаков}

\tasknumber{1}%
\task{%
    Определите работу, которую совершает сила Ампера при перемещении проводника длиной $l = 50\,\text{см}$
    с током силой $\eli = 20\,\text{А}$ в однородном магнитном поле индукцией $B = 0{,}5\,\text{Тл}$ на расстояние $d = 50\,\text{см}$.
    Проводник перпендикулярен линиям поля и движется в направлении силы Ампера.
}
\answer{%
    $
        A   = F \cdot d = B \eli l \cdot d
            = 0{,}5\,\text{Тл} \cdot 20\,\text{А} \cdot 50\,\text{см} \cdot 50\,\text{см}
            = 2{,}500\,\text{Дж}.
    $
}
\solutionspace{80pt}

\tasknumber{2}%
\task{%
    В однородном горизонтальном магнитном поле с индукцией $B = 100\,\text{мТл}$ находится проводник,
    расположенный также горизонтально и перпендикулярно полю.
    Какой ток необходимо пустить по проводнику, чтобы он завис?
    Масса единицы длины проводника $\rho = 5\,\frac{\text{г}}{\text{м}}$, $g = 10\,\frac{\text{м}}{\text{с}^{2}}$.
}
\answer{%
    $
            mg = B\eli l, m=\rho l
            \implies \eli
                = \frac{g\rho}B
                = \frac{10\,\frac{\text{м}}{\text{с}^{2}} \cdot 5\,\frac{\text{г}}{\text{м}}}{100\,\text{мТл}}
                = 0{,}5\,\text{А}.
    $
}
\solutionspace{120pt}

\tasknumber{3}%
\task{%
    Частица, обладающая массой $m$ и положительным зарядом $q$, движется со скоростью $v$
    в магнитном поле перпендикулярно линиям его индукции.
    Индукция магнитного поля равна $B$.
    Выведите из базовых физических законов формулы для радиуса траектории частицы: сделайте рисунок, укажите вид движения и названия физических законов.
}
\answer{%
    $F = ma, F = qvB, a = v^2 / R \implies R = \frac{mv}{qB}.$
}
\solutionspace{100pt}

\tasknumber{4}%
\task{%
    Узкий пучок протонов, нейтронов и электронов влетает в однородное магнитное поле перпендикулярно его линиям (см.
    рис).
    Определите по трекам частиц 3 и 5 отношение радиусов их траекторий и их импульсов.
}
\solutionspace{100pt}

\tasknumber{5}%
\task{%
    Электрон, прошедший через ускоряющую разность потенциалов, оказывается в магнитном поле индукцией $40\,\text{мТл}$
    и движется по окружности диаметром $4\,\text{мм}$.
    Сделайте рисунок, определите значение разности потенциалов
    и укажите, в какой области потенциал больше, а где меньше.
}

\variantsplitter

\addpersonalvariant{Михаил Ярошевский}

\tasknumber{1}%
\task{%
    Определите работу, которую совершает сила Ампера при перемещении проводника длиной $l = 30\,\text{см}$
    с током силой $\eli = 20\,\text{А}$ в однородном магнитном поле индукцией $B = 0{,}2\,\text{Тл}$ на расстояние $d = 50\,\text{см}$.
    Проводник перпендикулярен линиям поля и движется в направлении силы Ампера.
}
\answer{%
    $
        A   = F \cdot d = B \eli l \cdot d
            = 0{,}2\,\text{Тл} \cdot 20\,\text{А} \cdot 30\,\text{см} \cdot 50\,\text{см}
            = 0{,}600\,\text{Дж}.
    $
}
\solutionspace{80pt}

\tasknumber{2}%
\task{%
    В однородном горизонтальном магнитном поле с индукцией $B = 100\,\text{мТл}$ находится проводник,
    расположенный также горизонтально и перпендикулярно полю.
    Какой ток необходимо пустить по проводнику, чтобы он завис?
    Масса единицы длины проводника $\rho = 100\,\frac{\text{г}}{\text{м}}$, $g = 10\,\frac{\text{м}}{\text{с}^{2}}$.
}
\answer{%
    $
            mg = B\eli l, m=\rho l
            \implies \eli
                = \frac{g\rho}B
                = \frac{10\,\frac{\text{м}}{\text{с}^{2}} \cdot 100\,\frac{\text{г}}{\text{м}}}{100\,\text{мТл}}
                = 10\,\text{А}.
    $
}
\solutionspace{120pt}

\tasknumber{3}%
\task{%
    Частица, обладающая массой $m$ и положительным зарядом $q$, движется со скоростью $v$
    в магнитном поле перпендикулярно линиям его индукции.
    Индукция магнитного поля равна $B$.
    Выведите из базовых физических законов формулы для радиуса траектории частицы: сделайте рисунок, укажите вид движения и названия физических законов.
}
\answer{%
    $F = ma, F = qvB, a = v^2 / R \implies R = \frac{mv}{qB}.$
}
\solutionspace{100pt}

\tasknumber{4}%
\task{%
    Узкий пучок протонов, нейтронов и электронов влетает в однородное магнитное поле перпендикулярно его линиям (см.
    рис).
    Определите по трекам частиц 3 и 7 отношение радиусов их траекторий и их скоростей.
}
\solutionspace{100pt}

\tasknumber{5}%
\task{%
    Протон, прошедший через ускоряющую разность потенциалов, оказывается в магнитном поле индукцией $20\,\text{мТл}$
    и движется по окружности диаметром $4\,\text{мм}$.
    Сделайте рисунок, определите значение разности потенциалов
    и укажите, в какой области потенциал больше, а где меньше.
}

\variantsplitter

\addpersonalvariant{Алексей Алимпиев}

\tasknumber{1}%
\task{%
    Определите работу, которую совершает сила Ампера при перемещении проводника длиной $l = 30\,\text{см}$
    с током силой $\eli = 20\,\text{А}$ в однородном магнитном поле индукцией $B = 0{,}2\,\text{Тл}$ на расстояние $d = 50\,\text{см}$.
    Проводник перпендикулярен линиям поля и движется в направлении силы Ампера.
}
\answer{%
    $
        A   = F \cdot d = B \eli l \cdot d
            = 0{,}2\,\text{Тл} \cdot 20\,\text{А} \cdot 30\,\text{см} \cdot 50\,\text{см}
            = 0{,}600\,\text{Дж}.
    $
}
\solutionspace{80pt}

\tasknumber{2}%
\task{%
    В однородном горизонтальном магнитном поле с индукцией $B = 50\,\text{мТл}$ находится проводник,
    расположенный также горизонтально и перпендикулярно полю.
    Какой ток необходимо пустить по проводнику, чтобы он завис?
    Масса единицы длины проводника $\rho = 5\,\frac{\text{г}}{\text{м}}$, $g = 10\,\frac{\text{м}}{\text{с}^{2}}$.
}
\answer{%
    $
            mg = B\eli l, m=\rho l
            \implies \eli
                = \frac{g\rho}B
                = \frac{10\,\frac{\text{м}}{\text{с}^{2}} \cdot 5\,\frac{\text{г}}{\text{м}}}{50\,\text{мТл}}
                = 1\,\text{А}.
    $
}
\solutionspace{120pt}

\tasknumber{3}%
\task{%
    Частица, обладающая массой $m$ и положительным зарядом $q$, движется со скоростью $v$
    в магнитном поле перпендикулярно линиям его индукции.
    Индукция магнитного поля равна $B$.
    Выведите из базовых физических законов формулы для радиуса траектории частицы: сделайте рисунок, укажите вид движения и названия физических законов.
}
\answer{%
    $F = ma, F = qvB, a = v^2 / R \implies R = \frac{mv}{qB}.$
}
\solutionspace{100pt}

\tasknumber{4}%
\task{%
    Узкий пучок протонов, нейтронов и электронов влетает в однородное магнитное поле перпендикулярно его линиям (см.
    рис).
    Определите по трекам частиц 2 и 6 отношение радиусов их траекторий и их кинетических энергий.
}
\solutionspace{100pt}

\tasknumber{5}%
\task{%
    Электрон, прошедший через ускоряющую разность потенциалов, оказывается в магнитном поле индукцией $50\,\text{мТл}$
    и движется по окружности диаметром $8\,\text{мм}$.
    Сделайте рисунок, определите значение разности потенциалов
    и укажите, в какой области потенциал больше, а где меньше.
}

\variantsplitter

\addpersonalvariant{Евгений Васин}

\tasknumber{1}%
\task{%
    Определите работу, которую совершает сила Ампера при перемещении проводника длиной $l = 20\,\text{см}$
    с током силой $\eli = 10\,\text{А}$ в однородном магнитном поле индукцией $B = 0{,}2\,\text{Тл}$ на расстояние $d = 50\,\text{см}$.
    Проводник перпендикулярен линиям поля и движется в направлении силы Ампера.
}
\answer{%
    $
        A   = F \cdot d = B \eli l \cdot d
            = 0{,}2\,\text{Тл} \cdot 10\,\text{А} \cdot 20\,\text{см} \cdot 50\,\text{см}
            = 0{,}200\,\text{Дж}.
    $
}
\solutionspace{80pt}

\tasknumber{2}%
\task{%
    В однородном горизонтальном магнитном поле с индукцией $B = 50\,\text{мТл}$ находится проводник,
    расположенный также горизонтально и перпендикулярно полю.
    Какой ток необходимо пустить по проводнику, чтобы он завис?
    Масса единицы длины проводника $\rho = 100\,\frac{\text{г}}{\text{м}}$, $g = 10\,\frac{\text{м}}{\text{с}^{2}}$.
}
\answer{%
    $
            mg = B\eli l, m=\rho l
            \implies \eli
                = \frac{g\rho}B
                = \frac{10\,\frac{\text{м}}{\text{с}^{2}} \cdot 100\,\frac{\text{г}}{\text{м}}}{50\,\text{мТл}}
                = 20\,\text{А}.
    $
}
\solutionspace{120pt}

\tasknumber{3}%
\task{%
    Частица, обладающая массой $m$ и положительным зарядом $q$, движется со скоростью $v$
    в магнитном поле перпендикулярно линиям его индукции.
    Индукция магнитного поля равна $B$.
    Выведите из базовых физических законов формулы для радиуса траектории частицы: сделайте рисунок, укажите вид движения и названия физических законов.
}
\answer{%
    $F = ma, F = qvB, a = v^2 / R \implies R = \frac{mv}{qB}.$
}
\solutionspace{100pt}

\tasknumber{4}%
\task{%
    Узкий пучок протонов, нейтронов и электронов влетает в однородное магнитное поле перпендикулярно его линиям (см.
    рис).
    Определите по трекам частиц 2 и 5 отношение радиусов их траекторий и их кинетических энергий.
}
\solutionspace{100pt}

\tasknumber{5}%
\task{%
    Электрон, прошедший через ускоряющую разность потенциалов, оказывается в магнитном поле индукцией $20\,\text{мТл}$
    и движется по окружности диаметром $8\,\text{мм}$.
    Сделайте рисунок, определите значение разности потенциалов
    и укажите, в какой области потенциал больше, а где меньше.
}

\variantsplitter

\addpersonalvariant{Вячеслав Волохов}

\tasknumber{1}%
\task{%
    Определите работу, которую совершает сила Ампера при перемещении проводника длиной $l = 40\,\text{см}$
    с током силой $\eli = 10\,\text{А}$ в однородном магнитном поле индукцией $B = 0{,}10\,\text{Тл}$ на расстояние $d = 80\,\text{см}$.
    Проводник перпендикулярен линиям поля и движется в направлении силы Ампера.
}
\answer{%
    $
        A   = F \cdot d = B \eli l \cdot d
            = 0{,}10\,\text{Тл} \cdot 10\,\text{А} \cdot 40\,\text{см} \cdot 80\,\text{см}
            = 0{,}320\,\text{Дж}.
    $
}
\solutionspace{80pt}

\tasknumber{2}%
\task{%
    В однородном горизонтальном магнитном поле с индукцией $B = 10\,\text{мТл}$ находится проводник,
    расположенный также горизонтально и перпендикулярно полю.
    Какой ток необходимо пустить по проводнику, чтобы он завис?
    Масса единицы длины проводника $\rho = 100\,\frac{\text{г}}{\text{м}}$, $g = 10\,\frac{\text{м}}{\text{с}^{2}}$.
}
\answer{%
    $
            mg = B\eli l, m=\rho l
            \implies \eli
                = \frac{g\rho}B
                = \frac{10\,\frac{\text{м}}{\text{с}^{2}} \cdot 100\,\frac{\text{г}}{\text{м}}}{10\,\text{мТл}}
                = 100\,\text{А}.
    $
}
\solutionspace{120pt}

\tasknumber{3}%
\task{%
    Частица, обладающая массой $m$ и положительным зарядом $q$, движется со скоростью $v$
    в магнитном поле перпендикулярно линиям его индукции.
    Индукция магнитного поля равна $B$.
    Выведите из базовых физических законов формулы для радиуса траектории частицы: сделайте рисунок, укажите вид движения и названия физических законов.
}
\answer{%
    $F = ma, F = qvB, a = v^2 / R \implies R = \frac{mv}{qB}.$
}
\solutionspace{100pt}

\tasknumber{4}%
\task{%
    Узкий пучок протонов, нейтронов и электронов влетает в однородное магнитное поле перпендикулярно его линиям (см.
    рис).
    Определите по трекам частиц 1 и 7 отношение радиусов их траекторий и их кинетических энергий.
}
\solutionspace{100pt}

\tasknumber{5}%
\task{%
    Протон, прошедший через ускоряющую разность потенциалов, оказывается в магнитном поле индукцией $40\,\text{мТл}$
    и движется по окружности диаметром $6\,\text{мм}$.
    Сделайте рисунок, определите значение разности потенциалов
    и укажите, в какой области потенциал больше, а где меньше.
}

\variantsplitter

\addpersonalvariant{Герман Говоров}

\tasknumber{1}%
\task{%
    Определите работу, которую совершает сила Ампера при перемещении проводника длиной $l = 40\,\text{см}$
    с током силой $\eli = 5\,\text{А}$ в однородном магнитном поле индукцией $B = 0{,}5\,\text{Тл}$ на расстояние $d = 50\,\text{см}$.
    Проводник перпендикулярен линиям поля и движется в направлении силы Ампера.
}
\answer{%
    $
        A   = F \cdot d = B \eli l \cdot d
            = 0{,}5\,\text{Тл} \cdot 5\,\text{А} \cdot 40\,\text{см} \cdot 50\,\text{см}
            = 0{,}500\,\text{Дж}.
    $
}
\solutionspace{80pt}

\tasknumber{2}%
\task{%
    В однородном горизонтальном магнитном поле с индукцией $B = 10\,\text{мТл}$ находится проводник,
    расположенный также горизонтально и перпендикулярно полю.
    Какой ток необходимо пустить по проводнику, чтобы он завис?
    Масса единицы длины проводника $\rho = 20\,\frac{\text{г}}{\text{м}}$, $g = 10\,\frac{\text{м}}{\text{с}^{2}}$.
}
\answer{%
    $
            mg = B\eli l, m=\rho l
            \implies \eli
                = \frac{g\rho}B
                = \frac{10\,\frac{\text{м}}{\text{с}^{2}} \cdot 20\,\frac{\text{г}}{\text{м}}}{10\,\text{мТл}}
                = 20\,\text{А}.
    $
}
\solutionspace{120pt}

\tasknumber{3}%
\task{%
    Частица, обладающая массой $m$ и положительным зарядом $q$, движется со скоростью $v$
    в магнитном поле перпендикулярно линиям его индукции.
    Индукция магнитного поля равна $B$.
    Выведите из базовых физических законов формулы для радиуса траектории частицы: сделайте рисунок, укажите вид движения и названия физических законов.
}
\answer{%
    $F = ma, F = qvB, a = v^2 / R \implies R = \frac{mv}{qB}.$
}
\solutionspace{100pt}

\tasknumber{4}%
\task{%
    Узкий пучок протонов, нейтронов и электронов влетает в однородное магнитное поле перпендикулярно его линиям (см.
    рис).
    Определите по трекам частиц 1 и 6 отношение радиусов их траекторий и их кинетических энергий.
}
\solutionspace{100pt}

\tasknumber{5}%
\task{%
    Протон, прошедший через ускоряющую разность потенциалов, оказывается в магнитном поле индукцией $50\,\text{мТл}$
    и движется по окружности диаметром $4\,\text{мм}$.
    Сделайте рисунок, определите значение разности потенциалов
    и укажите, в какой области потенциал больше, а где меньше.
}

\variantsplitter

\addpersonalvariant{София Журавлёва}

\tasknumber{1}%
\task{%
    Определите работу, которую совершает сила Ампера при перемещении проводника длиной $l = 20\,\text{см}$
    с током силой $\eli = 10\,\text{А}$ в однородном магнитном поле индукцией $B = 0{,}10\,\text{Тл}$ на расстояние $d = 80\,\text{см}$.
    Проводник перпендикулярен линиям поля и движется в направлении силы Ампера.
}
\answer{%
    $
        A   = F \cdot d = B \eli l \cdot d
            = 0{,}10\,\text{Тл} \cdot 10\,\text{А} \cdot 20\,\text{см} \cdot 80\,\text{см}
            = 0{,}160\,\text{Дж}.
    $
}
\solutionspace{80pt}

\tasknumber{2}%
\task{%
    В однородном горизонтальном магнитном поле с индукцией $B = 100\,\text{мТл}$ находится проводник,
    расположенный также горизонтально и перпендикулярно полю.
    Какой ток необходимо пустить по проводнику, чтобы он завис?
    Масса единицы длины проводника $\rho = 40\,\frac{\text{г}}{\text{м}}$, $g = 10\,\frac{\text{м}}{\text{с}^{2}}$.
}
\answer{%
    $
            mg = B\eli l, m=\rho l
            \implies \eli
                = \frac{g\rho}B
                = \frac{10\,\frac{\text{м}}{\text{с}^{2}} \cdot 40\,\frac{\text{г}}{\text{м}}}{100\,\text{мТл}}
                = 4\,\text{А}.
    $
}
\solutionspace{120pt}

\tasknumber{3}%
\task{%
    Частица, обладающая массой $m$ и положительным зарядом $q$, движется со скоростью $v$
    в магнитном поле перпендикулярно линиям его индукции.
    Индукция магнитного поля равна $B$.
    Выведите из базовых физических законов формулы для радиуса траектории частицы: сделайте рисунок, укажите вид движения и названия физических законов.
}
\answer{%
    $F = ma, F = qvB, a = v^2 / R \implies R = \frac{mv}{qB}.$
}
\solutionspace{100pt}

\tasknumber{4}%
\task{%
    Узкий пучок протонов, нейтронов и электронов влетает в однородное магнитное поле перпендикулярно его линиям (см.
    рис).
    Определите по трекам частиц 1 и 7 отношение радиусов их траекторий и их скоростей.
}
\solutionspace{100pt}

\tasknumber{5}%
\task{%
    Электрон, прошедший через ускоряющую разность потенциалов, оказывается в магнитном поле индукцией $40\,\text{мТл}$
    и движется по окружности диаметром $6\,\text{мм}$.
    Сделайте рисунок, определите значение разности потенциалов
    и укажите, в какой области потенциал больше, а где меньше.
}

\variantsplitter

\addpersonalvariant{Константин Козлов}

\tasknumber{1}%
\task{%
    Определите работу, которую совершает сила Ампера при перемещении проводника длиной $l = 50\,\text{см}$
    с током силой $\eli = 10\,\text{А}$ в однородном магнитном поле индукцией $B = 0{,}5\,\text{Тл}$ на расстояние $d = 50\,\text{см}$.
    Проводник перпендикулярен линиям поля и движется в направлении силы Ампера.
}
\answer{%
    $
        A   = F \cdot d = B \eli l \cdot d
            = 0{,}5\,\text{Тл} \cdot 10\,\text{А} \cdot 50\,\text{см} \cdot 50\,\text{см}
            = 1{,}250\,\text{Дж}.
    $
}
\solutionspace{80pt}

\tasknumber{2}%
\task{%
    В однородном горизонтальном магнитном поле с индукцией $B = 10\,\text{мТл}$ находится проводник,
    расположенный также горизонтально и перпендикулярно полю.
    Какой ток необходимо пустить по проводнику, чтобы он завис?
    Масса единицы длины проводника $\rho = 100\,\frac{\text{г}}{\text{м}}$, $g = 10\,\frac{\text{м}}{\text{с}^{2}}$.
}
\answer{%
    $
            mg = B\eli l, m=\rho l
            \implies \eli
                = \frac{g\rho}B
                = \frac{10\,\frac{\text{м}}{\text{с}^{2}} \cdot 100\,\frac{\text{г}}{\text{м}}}{10\,\text{мТл}}
                = 100\,\text{А}.
    $
}
\solutionspace{120pt}

\tasknumber{3}%
\task{%
    Частица, обладающая массой $m$ и положительным зарядом $q$, движется со скоростью $v$
    в магнитном поле перпендикулярно линиям его индукции.
    Индукция магнитного поля равна $B$.
    Выведите из базовых физических законов формулы для радиуса траектории частицы: сделайте рисунок, укажите вид движения и названия физических законов.
}
\answer{%
    $F = ma, F = qvB, a = v^2 / R \implies R = \frac{mv}{qB}.$
}
\solutionspace{100pt}

\tasknumber{4}%
\task{%
    Узкий пучок протонов, нейтронов и электронов влетает в однородное магнитное поле перпендикулярно его линиям (см.
    рис).
    Определите по трекам частиц 3 и 5 отношение радиусов их траекторий и их кинетических энергий.
}
\solutionspace{100pt}

\tasknumber{5}%
\task{%
    Электрон, прошедший через ускоряющую разность потенциалов, оказывается в магнитном поле индукцией $50\,\text{мТл}$
    и движется по окружности диаметром $8\,\text{мм}$.
    Сделайте рисунок, определите значение разности потенциалов
    и укажите, в какой области потенциал больше, а где меньше.
}

\variantsplitter

\addpersonalvariant{Наталья Кравченко}

\tasknumber{1}%
\task{%
    Определите работу, которую совершает сила Ампера при перемещении проводника длиной $l = 20\,\text{см}$
    с током силой $\eli = 10\,\text{А}$ в однородном магнитном поле индукцией $B = 0{,}5\,\text{Тл}$ на расстояние $d = 50\,\text{см}$.
    Проводник перпендикулярен линиям поля и движется в направлении силы Ампера.
}
\answer{%
    $
        A   = F \cdot d = B \eli l \cdot d
            = 0{,}5\,\text{Тл} \cdot 10\,\text{А} \cdot 20\,\text{см} \cdot 50\,\text{см}
            = 0{,}500\,\text{Дж}.
    $
}
\solutionspace{80pt}

\tasknumber{2}%
\task{%
    В однородном горизонтальном магнитном поле с индукцией $B = 100\,\text{мТл}$ находится проводник,
    расположенный также горизонтально и перпендикулярно полю.
    Какой ток необходимо пустить по проводнику, чтобы он завис?
    Масса единицы длины проводника $\rho = 10\,\frac{\text{г}}{\text{м}}$, $g = 10\,\frac{\text{м}}{\text{с}^{2}}$.
}
\answer{%
    $
            mg = B\eli l, m=\rho l
            \implies \eli
                = \frac{g\rho}B
                = \frac{10\,\frac{\text{м}}{\text{с}^{2}} \cdot 10\,\frac{\text{г}}{\text{м}}}{100\,\text{мТл}}
                = 1\,\text{А}.
    $
}
\solutionspace{120pt}

\tasknumber{3}%
\task{%
    Частица, обладающая массой $m$ и положительным зарядом $q$, движется со скоростью $v$
    в магнитном поле перпендикулярно линиям его индукции.
    Индукция магнитного поля равна $B$.
    Выведите из базовых физических законов формулы для радиуса траектории частицы: сделайте рисунок, укажите вид движения и названия физических законов.
}
\answer{%
    $F = ma, F = qvB, a = v^2 / R \implies R = \frac{mv}{qB}.$
}
\solutionspace{100pt}

\tasknumber{4}%
\task{%
    Узкий пучок протонов, нейтронов и электронов влетает в однородное магнитное поле перпендикулярно его линиям (см.
    рис).
    Определите по трекам частиц 2 и 5 отношение радиусов их траекторий и их кинетических энергий.
}
\solutionspace{100pt}

\tasknumber{5}%
\task{%
    Протон, прошедший через ускоряющую разность потенциалов, оказывается в магнитном поле индукцией $50\,\text{мТл}$
    и движется по окружности диаметром $6\,\text{мм}$.
    Сделайте рисунок, определите значение разности потенциалов
    и укажите, в какой области потенциал больше, а где меньше.
}

\variantsplitter

\addpersonalvariant{Матвей Кузьмин}

\tasknumber{1}%
\task{%
    Определите работу, которую совершает сила Ампера при перемещении проводника длиной $l = 50\,\text{см}$
    с током силой $\eli = 20\,\text{А}$ в однородном магнитном поле индукцией $B = 0{,}2\,\text{Тл}$ на расстояние $d = 50\,\text{см}$.
    Проводник перпендикулярен линиям поля и движется в направлении силы Ампера.
}
\answer{%
    $
        A   = F \cdot d = B \eli l \cdot d
            = 0{,}2\,\text{Тл} \cdot 20\,\text{А} \cdot 50\,\text{см} \cdot 50\,\text{см}
            = 1\,\text{Дж}.
    $
}
\solutionspace{80pt}

\tasknumber{2}%
\task{%
    В однородном горизонтальном магнитном поле с индукцией $B = 50\,\text{мТл}$ находится проводник,
    расположенный также горизонтально и перпендикулярно полю.
    Какой ток необходимо пустить по проводнику, чтобы он завис?
    Масса единицы длины проводника $\rho = 100\,\frac{\text{г}}{\text{м}}$, $g = 10\,\frac{\text{м}}{\text{с}^{2}}$.
}
\answer{%
    $
            mg = B\eli l, m=\rho l
            \implies \eli
                = \frac{g\rho}B
                = \frac{10\,\frac{\text{м}}{\text{с}^{2}} \cdot 100\,\frac{\text{г}}{\text{м}}}{50\,\text{мТл}}
                = 20\,\text{А}.
    $
}
\solutionspace{120pt}

\tasknumber{3}%
\task{%
    Частица, обладающая массой $m$ и положительным зарядом $q$, движется со скоростью $v$
    в магнитном поле перпендикулярно линиям его индукции.
    Индукция магнитного поля равна $B$.
    Выведите из базовых физических законов формулы для радиуса траектории частицы: сделайте рисунок, укажите вид движения и названия физических законов.
}
\answer{%
    $F = ma, F = qvB, a = v^2 / R \implies R = \frac{mv}{qB}.$
}
\solutionspace{100pt}

\tasknumber{4}%
\task{%
    Узкий пучок протонов, нейтронов и электронов влетает в однородное магнитное поле перпендикулярно его линиям (см.
    рис).
    Определите по трекам частиц 3 и 6 отношение радиусов их траекторий и их кинетических энергий.
}
\solutionspace{100pt}

\tasknumber{5}%
\task{%
    Электрон, прошедший через ускоряющую разность потенциалов, оказывается в магнитном поле индукцией $50\,\text{мТл}$
    и движется по окружности диаметром $4\,\text{мм}$.
    Сделайте рисунок, определите значение разности потенциалов
    и укажите, в какой области потенциал больше, а где меньше.
}

\variantsplitter

\addpersonalvariant{Сергей Малышев}

\tasknumber{1}%
\task{%
    Определите работу, которую совершает сила Ампера при перемещении проводника длиной $l = 50\,\text{см}$
    с током силой $\eli = 10\,\text{А}$ в однородном магнитном поле индукцией $B = 0{,}10\,\text{Тл}$ на расстояние $d = 50\,\text{см}$.
    Проводник перпендикулярен линиям поля и движется в направлении силы Ампера.
}
\answer{%
    $
        A   = F \cdot d = B \eli l \cdot d
            = 0{,}10\,\text{Тл} \cdot 10\,\text{А} \cdot 50\,\text{см} \cdot 50\,\text{см}
            = 0{,}250\,\text{Дж}.
    $
}
\solutionspace{80pt}

\tasknumber{2}%
\task{%
    В однородном горизонтальном магнитном поле с индукцией $B = 20\,\text{мТл}$ находится проводник,
    расположенный также горизонтально и перпендикулярно полю.
    Какой ток необходимо пустить по проводнику, чтобы он завис?
    Масса единицы длины проводника $\rho = 10\,\frac{\text{г}}{\text{м}}$, $g = 10\,\frac{\text{м}}{\text{с}^{2}}$.
}
\answer{%
    $
            mg = B\eli l, m=\rho l
            \implies \eli
                = \frac{g\rho}B
                = \frac{10\,\frac{\text{м}}{\text{с}^{2}} \cdot 10\,\frac{\text{г}}{\text{м}}}{20\,\text{мТл}}
                = 5\,\text{А}.
    $
}
\solutionspace{120pt}

\tasknumber{3}%
\task{%
    Частица, обладающая массой $m$ и положительным зарядом $q$, движется со скоростью $v$
    в магнитном поле перпендикулярно линиям его индукции.
    Индукция магнитного поля равна $B$.
    Выведите из базовых физических законов формулы для радиуса траектории частицы: сделайте рисунок, укажите вид движения и названия физических законов.
}
\answer{%
    $F = ma, F = qvB, a = v^2 / R \implies R = \frac{mv}{qB}.$
}
\solutionspace{100pt}

\tasknumber{4}%
\task{%
    Узкий пучок протонов, нейтронов и электронов влетает в однородное магнитное поле перпендикулярно его линиям (см.
    рис).
    Определите по трекам частиц 1 и 5 отношение радиусов их траекторий и их импульсов.
}
\solutionspace{100pt}

\tasknumber{5}%
\task{%
    Позитрон, прошедший через ускоряющую разность потенциалов, оказывается в магнитном поле индукцией $20\,\text{мТл}$
    и движется по окружности диаметром $4\,\text{мм}$.
    Сделайте рисунок, определите значение разности потенциалов
    и укажите, в какой области потенциал больше, а где меньше.
}

\variantsplitter

\addpersonalvariant{Алина Полканова}

\tasknumber{1}%
\task{%
    Определите работу, которую совершает сила Ампера при перемещении проводника длиной $l = 50\,\text{см}$
    с током силой $\eli = 20\,\text{А}$ в однородном магнитном поле индукцией $B = 0{,}2\,\text{Тл}$ на расстояние $d = 20\,\text{см}$.
    Проводник перпендикулярен линиям поля и движется в направлении силы Ампера.
}
\answer{%
    $
        A   = F \cdot d = B \eli l \cdot d
            = 0{,}2\,\text{Тл} \cdot 20\,\text{А} \cdot 50\,\text{см} \cdot 20\,\text{см}
            = 0{,}400\,\text{Дж}.
    $
}
\solutionspace{80pt}

\tasknumber{2}%
\task{%
    В однородном горизонтальном магнитном поле с индукцией $B = 50\,\text{мТл}$ находится проводник,
    расположенный также горизонтально и перпендикулярно полю.
    Какой ток необходимо пустить по проводнику, чтобы он завис?
    Масса единицы длины проводника $\rho = 40\,\frac{\text{г}}{\text{м}}$, $g = 10\,\frac{\text{м}}{\text{с}^{2}}$.
}
\answer{%
    $
            mg = B\eli l, m=\rho l
            \implies \eli
                = \frac{g\rho}B
                = \frac{10\,\frac{\text{м}}{\text{с}^{2}} \cdot 40\,\frac{\text{г}}{\text{м}}}{50\,\text{мТл}}
                = 8\,\text{А}.
    $
}
\solutionspace{120pt}

\tasknumber{3}%
\task{%
    Частица, обладающая массой $m$ и положительным зарядом $q$, движется со скоростью $v$
    в магнитном поле перпендикулярно линиям его индукции.
    Индукция магнитного поля равна $B$.
    Выведите из базовых физических законов формулы для радиуса траектории частицы: сделайте рисунок, укажите вид движения и названия физических законов.
}
\answer{%
    $F = ma, F = qvB, a = v^2 / R \implies R = \frac{mv}{qB}.$
}
\solutionspace{100pt}

\tasknumber{4}%
\task{%
    Узкий пучок протонов, нейтронов и электронов влетает в однородное магнитное поле перпендикулярно его линиям (см.
    рис).
    Определите по трекам частиц 3 и 7 отношение радиусов их траекторий и их скоростей.
}
\solutionspace{100pt}

\tasknumber{5}%
\task{%
    Электрон, прошедший через ускоряющую разность потенциалов, оказывается в магнитном поле индукцией $50\,\text{мТл}$
    и движется по окружности диаметром $4\,\text{мм}$.
    Сделайте рисунок, определите значение разности потенциалов
    и укажите, в какой области потенциал больше, а где меньше.
}

\variantsplitter

\addpersonalvariant{Сергей Пономарёв}

\tasknumber{1}%
\task{%
    Определите работу, которую совершает сила Ампера при перемещении проводника длиной $l = 50\,\text{см}$
    с током силой $\eli = 10\,\text{А}$ в однородном магнитном поле индукцией $B = 0{,}10\,\text{Тл}$ на расстояние $d = 80\,\text{см}$.
    Проводник перпендикулярен линиям поля и движется в направлении силы Ампера.
}
\answer{%
    $
        A   = F \cdot d = B \eli l \cdot d
            = 0{,}10\,\text{Тл} \cdot 10\,\text{А} \cdot 50\,\text{см} \cdot 80\,\text{см}
            = 0{,}400\,\text{Дж}.
    $
}
\solutionspace{80pt}

\tasknumber{2}%
\task{%
    В однородном горизонтальном магнитном поле с индукцией $B = 50\,\text{мТл}$ находится проводник,
    расположенный также горизонтально и перпендикулярно полю.
    Какой ток необходимо пустить по проводнику, чтобы он завис?
    Масса единицы длины проводника $\rho = 40\,\frac{\text{г}}{\text{м}}$, $g = 10\,\frac{\text{м}}{\text{с}^{2}}$.
}
\answer{%
    $
            mg = B\eli l, m=\rho l
            \implies \eli
                = \frac{g\rho}B
                = \frac{10\,\frac{\text{м}}{\text{с}^{2}} \cdot 40\,\frac{\text{г}}{\text{м}}}{50\,\text{мТл}}
                = 8\,\text{А}.
    $
}
\solutionspace{120pt}

\tasknumber{3}%
\task{%
    Частица, обладающая массой $m$ и положительным зарядом $q$, движется со скоростью $v$
    в магнитном поле перпендикулярно линиям его индукции.
    Индукция магнитного поля равна $B$.
    Выведите из базовых физических законов формулы для радиуса траектории частицы: сделайте рисунок, укажите вид движения и названия физических законов.
}
\answer{%
    $F = ma, F = qvB, a = v^2 / R \implies R = \frac{mv}{qB}.$
}
\solutionspace{100pt}

\tasknumber{4}%
\task{%
    Узкий пучок протонов, нейтронов и электронов влетает в однородное магнитное поле перпендикулярно его линиям (см.
    рис).
    Определите по трекам частиц 2 и 6 отношение радиусов их траекторий и их импульсов.
}
\solutionspace{100pt}

\tasknumber{5}%
\task{%
    Электрон, прошедший через ускоряющую разность потенциалов, оказывается в магнитном поле индукцией $40\,\text{мТл}$
    и движется по окружности диаметром $4\,\text{мм}$.
    Сделайте рисунок, определите значение разности потенциалов
    и укажите, в какой области потенциал больше, а где меньше.
}

\variantsplitter

\addpersonalvariant{Егор Свистушкин}

\tasknumber{1}%
\task{%
    Определите работу, которую совершает сила Ампера при перемещении проводника длиной $l = 40\,\text{см}$
    с током силой $\eli = 5\,\text{А}$ в однородном магнитном поле индукцией $B = 0{,}5\,\text{Тл}$ на расстояние $d = 80\,\text{см}$.
    Проводник перпендикулярен линиям поля и движется в направлении силы Ампера.
}
\answer{%
    $
        A   = F \cdot d = B \eli l \cdot d
            = 0{,}5\,\text{Тл} \cdot 5\,\text{А} \cdot 40\,\text{см} \cdot 80\,\text{см}
            = 0{,}800\,\text{Дж}.
    $
}
\solutionspace{80pt}

\tasknumber{2}%
\task{%
    В однородном горизонтальном магнитном поле с индукцией $B = 20\,\text{мТл}$ находится проводник,
    расположенный также горизонтально и перпендикулярно полю.
    Какой ток необходимо пустить по проводнику, чтобы он завис?
    Масса единицы длины проводника $\rho = 100\,\frac{\text{г}}{\text{м}}$, $g = 10\,\frac{\text{м}}{\text{с}^{2}}$.
}
\answer{%
    $
            mg = B\eli l, m=\rho l
            \implies \eli
                = \frac{g\rho}B
                = \frac{10\,\frac{\text{м}}{\text{с}^{2}} \cdot 100\,\frac{\text{г}}{\text{м}}}{20\,\text{мТл}}
                = 50\,\text{А}.
    $
}
\solutionspace{120pt}

\tasknumber{3}%
\task{%
    Частица, обладающая массой $m$ и положительным зарядом $q$, движется со скоростью $v$
    в магнитном поле перпендикулярно линиям его индукции.
    Индукция магнитного поля равна $B$.
    Выведите из базовых физических законов формулы для радиуса траектории частицы: сделайте рисунок, укажите вид движения и названия физических законов.
}
\answer{%
    $F = ma, F = qvB, a = v^2 / R \implies R = \frac{mv}{qB}.$
}
\solutionspace{100pt}

\tasknumber{4}%
\task{%
    Узкий пучок протонов, нейтронов и электронов влетает в однородное магнитное поле перпендикулярно его линиям (см.
    рис).
    Определите по трекам частиц 1 и 5 отношение радиусов их траекторий и их скоростей.
}
\solutionspace{100pt}

\tasknumber{5}%
\task{%
    Электрон, прошедший через ускоряющую разность потенциалов, оказывается в магнитном поле индукцией $20\,\text{мТл}$
    и движется по окружности диаметром $6\,\text{мм}$.
    Сделайте рисунок, определите значение разности потенциалов
    и укажите, в какой области потенциал больше, а где меньше.
}

\variantsplitter

\addpersonalvariant{Дмитрий Соколов}

\tasknumber{1}%
\task{%
    Определите работу, которую совершает сила Ампера при перемещении проводника длиной $l = 30\,\text{см}$
    с током силой $\eli = 10\,\text{А}$ в однородном магнитном поле индукцией $B = 0{,}10\,\text{Тл}$ на расстояние $d = 50\,\text{см}$.
    Проводник перпендикулярен линиям поля и движется в направлении силы Ампера.
}
\answer{%
    $
        A   = F \cdot d = B \eli l \cdot d
            = 0{,}10\,\text{Тл} \cdot 10\,\text{А} \cdot 30\,\text{см} \cdot 50\,\text{см}
            = 0{,}150\,\text{Дж}.
    $
}
\solutionspace{80pt}

\tasknumber{2}%
\task{%
    В однородном горизонтальном магнитном поле с индукцией $B = 20\,\text{мТл}$ находится проводник,
    расположенный также горизонтально и перпендикулярно полю.
    Какой ток необходимо пустить по проводнику, чтобы он завис?
    Масса единицы длины проводника $\rho = 10\,\frac{\text{г}}{\text{м}}$, $g = 10\,\frac{\text{м}}{\text{с}^{2}}$.
}
\answer{%
    $
            mg = B\eli l, m=\rho l
            \implies \eli
                = \frac{g\rho}B
                = \frac{10\,\frac{\text{м}}{\text{с}^{2}} \cdot 10\,\frac{\text{г}}{\text{м}}}{20\,\text{мТл}}
                = 5\,\text{А}.
    $
}
\solutionspace{120pt}

\tasknumber{3}%
\task{%
    Частица, обладающая массой $m$ и положительным зарядом $q$, движется со скоростью $v$
    в магнитном поле перпендикулярно линиям его индукции.
    Индукция магнитного поля равна $B$.
    Выведите из базовых физических законов формулы для радиуса траектории частицы: сделайте рисунок, укажите вид движения и названия физических законов.
}
\answer{%
    $F = ma, F = qvB, a = v^2 / R \implies R = \frac{mv}{qB}.$
}
\solutionspace{100pt}

\tasknumber{4}%
\task{%
    Узкий пучок протонов, нейтронов и электронов влетает в однородное магнитное поле перпендикулярно его линиям (см.
    рис).
    Определите по трекам частиц 3 и 7 отношение радиусов их траекторий и их кинетических энергий.
}
\solutionspace{100pt}

\tasknumber{5}%
\task{%
    Позитрон, прошедший через ускоряющую разность потенциалов, оказывается в магнитном поле индукцией $50\,\text{мТл}$
    и движется по окружности диаметром $8\,\text{мм}$.
    Сделайте рисунок, определите значение разности потенциалов
    и укажите, в какой области потенциал больше, а где меньше.
}

\variantsplitter

\addpersonalvariant{Арсений Трофимов}

\tasknumber{1}%
\task{%
    Определите работу, которую совершает сила Ампера при перемещении проводника длиной $l = 30\,\text{см}$
    с током силой $\eli = 10\,\text{А}$ в однородном магнитном поле индукцией $B = 0{,}10\,\text{Тл}$ на расстояние $d = 20\,\text{см}$.
    Проводник перпендикулярен линиям поля и движется в направлении силы Ампера.
}
\answer{%
    $
        A   = F \cdot d = B \eli l \cdot d
            = 0{,}10\,\text{Тл} \cdot 10\,\text{А} \cdot 30\,\text{см} \cdot 20\,\text{см}
            = 0{,}060\,\text{Дж}.
    $
}
\solutionspace{80pt}

\tasknumber{2}%
\task{%
    В однородном горизонтальном магнитном поле с индукцией $B = 100\,\text{мТл}$ находится проводник,
    расположенный также горизонтально и перпендикулярно полю.
    Какой ток необходимо пустить по проводнику, чтобы он завис?
    Масса единицы длины проводника $\rho = 10\,\frac{\text{г}}{\text{м}}$, $g = 10\,\frac{\text{м}}{\text{с}^{2}}$.
}
\answer{%
    $
            mg = B\eli l, m=\rho l
            \implies \eli
                = \frac{g\rho}B
                = \frac{10\,\frac{\text{м}}{\text{с}^{2}} \cdot 10\,\frac{\text{г}}{\text{м}}}{100\,\text{мТл}}
                = 1\,\text{А}.
    $
}
\solutionspace{120pt}

\tasknumber{3}%
\task{%
    Частица, обладающая массой $m$ и положительным зарядом $q$, движется со скоростью $v$
    в магнитном поле перпендикулярно линиям его индукции.
    Индукция магнитного поля равна $B$.
    Выведите из базовых физических законов формулы для радиуса траектории частицы: сделайте рисунок, укажите вид движения и названия физических законов.
}
\answer{%
    $F = ma, F = qvB, a = v^2 / R \implies R = \frac{mv}{qB}.$
}
\solutionspace{100pt}

\tasknumber{4}%
\task{%
    Узкий пучок протонов, нейтронов и электронов влетает в однородное магнитное поле перпендикулярно его линиям (см.
    рис).
    Определите по трекам частиц 1 и 7 отношение радиусов их траекторий и их кинетических энергий.
}
\solutionspace{100pt}

\tasknumber{5}%
\task{%
    Протон, прошедший через ускоряющую разность потенциалов, оказывается в магнитном поле индукцией $20\,\text{мТл}$
    и движется по окружности диаметром $6\,\text{мм}$.
    Сделайте рисунок, определите значение разности потенциалов
    и укажите, в какой области потенциал больше, а где меньше.
}
% autogenerated
