\setdate{30~сентября~2019}
\setclass{11«Т»}

\addpersonalvariant{Михаил Бурмистров}

\tasknumber{1}%
\task{%
    Проводник длиной $l = 17\,\text{см}$ согнули под прямым углом так, что одна сторона угла оказалась равной $a = 12\,\text{см}$,
    и поместили в однородное магнитное поле с индукцией $B = 10\,\text{мТл}$ обеими сторонами перпендикулярно линиям индукции.
    Какая сила будет действовать на этот проводник при пропусканиии по нему тока $\mathcal{I} = 50\,\text{А}$?
}
\answer{%
    \begin{align*}
    F &= \sqrt{F_a^2 + F_b^2} = \sqrt{\sqr{\mathcal{I}Ba} + \sqr{\mathcal{I}Bb}}
                = \mathcal{I}B\sqrt{a^2 + b^2} = \mathcal{I}B\sqrt{a^2 + (l - a)^2} =  \\
    &= 50\,\text{А} \cdot 10\,\text{мТл} \cdot \sqrt{\sqr{ 12\,\text{см} } + \sqr{17\,\text{см} - 12\,\text{см}}} = 65{,}00\,\text{мН}.
    \end{align*}
}
\solutionspace{120pt}

\tasknumber{2}%
\task{%
    В однородном горизонтальном магнитном поле с индукцией $B = 50\,\text{мТл}$ находится проводник,
    расположенный также горизонтально и перпендикулярно полю.
    Какой ток необходимо пустить по проводнику, чтобы он завис?
    Масса единицы длины проводника $\rho = 40\,\frac{\text{кг}}{\text{м}}$, $g = 10\,\frac{\text{м}}{\text{с}^{2}}$.
}
\answer{%
    $
            mg = B\mathcal{I} l, m=\rho l
            \implies \mathcal{I}
                = \frac{g\rho} {B}
                = \frac{10\,\frac{\text{м}}{\text{с}^{2}} \cdot 40\,\frac{\text{кг}}{\text{м}}}{ 50\,\text{мТл} }
                = 8{,}000000\,\text{кА}.
    $
}
\solutionspace{120pt}

\tasknumber{3}%
\task{%
    Определите работу, которую совершает сила Ампера при перемещении проводника длиной $l = 20\,\text{см}$
    с током силой $\mathcal{I} = 5\,\text{А}$ в однородном магнитном поле индукцией $B = 0{,}500000\,\text{Тл}$ на расстояние $d = 50\,\text{см}$.
    Проводник перпендикулярен линиям поля и движется в направлении силы Ампера.
}
\answer{%
    $
        A   = F \cdot d = B\mathcal{I} l  \cdot d
            = 0{,}500000\,\text{Тл}  \cdot 5\,\text{А}  \cdot 20\,\text{см}  \cdot 50\,\text{см}
            = 0{,}25000\,\text{Дж}.
    $
}

\variantsplitter

\addpersonalvariant{Ирен Аракелян}

\tasknumber{1}%
\task{%
    Проводник длиной $l = 170\,\text{см}$ согнули под прямым углом так, что одна сторона угла оказалась равной $a = 50\,\text{см}$,
    и поместили в однородное магнитное поле с индукцией $B = 2\,\text{мТл}$ обеими сторонами перпендикулярно линиям индукции.
    Какая сила будет действовать на этот проводник при пропусканиии по нему тока $\mathcal{I} = 40\,\text{А}$?
}
\answer{%
    \begin{align*}
    F &= \sqrt{F_a^2 + F_b^2} = \sqrt{\sqr{\mathcal{I}Ba} + \sqr{\mathcal{I}Bb}}
                = \mathcal{I}B\sqrt{a^2 + b^2} = \mathcal{I}B\sqrt{a^2 + (l - a)^2} =  \\
    &= 40\,\text{А} \cdot 2\,\text{мТл} \cdot \sqrt{\sqr{ 50\,\text{см} } + \sqr{170\,\text{см} - 50\,\text{см}}} = 104{,}00\,\text{мН}.
    \end{align*}
}
\solutionspace{120pt}

\tasknumber{2}%
\task{%
    В однородном горизонтальном магнитном поле с индукцией $B = 10\,\text{мТл}$ находится проводник,
    расположенный также горизонтально и перпендикулярно полю.
    Какой ток необходимо пустить по проводнику, чтобы он завис?
    Масса единицы длины проводника $\rho = 20\,\frac{\text{кг}}{\text{м}}$, $g = 10\,\frac{\text{м}}{\text{с}^{2}}$.
}
\answer{%
    $
            mg = B\mathcal{I} l, m=\rho l
            \implies \mathcal{I}
                = \frac{g\rho} {B}
                = \frac{10\,\frac{\text{м}}{\text{с}^{2}} \cdot 20\,\frac{\text{кг}}{\text{м}}}{ 10\,\text{мТл} }
                = 20{,}000000\,\text{кА}.
    $
}
\solutionspace{120pt}

\tasknumber{3}%
\task{%
    Определите работу, которую совершает сила Ампера при перемещении проводника длиной $l = 40\,\text{см}$
    с током силой $\mathcal{I} = 5\,\text{А}$ в однородном магнитном поле индукцией $B = 0{,}100000\,\text{Тл}$ на расстояние $d = 80\,\text{см}$.
    Проводник перпендикулярен линиям поля и движется в направлении силы Ампера.
}
\answer{%
    $
        A   = F \cdot d = B\mathcal{I} l  \cdot d
            = 0{,}100000\,\text{Тл}  \cdot 5\,\text{А}  \cdot 40\,\text{см}  \cdot 80\,\text{см}
            = 0{,}16000\,\text{Дж}.
    $
}

\variantsplitter

\addpersonalvariant{Юлия Буянова}

\tasknumber{1}%
\task{%
    Проводник длиной $l = 170\,\text{см}$ согнули под прямым углом так, что одна сторона угла оказалась равной $a = 120\,\text{см}$,
    и поместили в однородное магнитное поле с индукцией $B = 10\,\text{мТл}$ обеими сторонами перпендикулярно линиям индукции.
    Какая сила будет действовать на этот проводник при пропусканиии по нему тока $\mathcal{I} = 40\,\text{А}$?
}
\answer{%
    \begin{align*}
    F &= \sqrt{F_a^2 + F_b^2} = \sqrt{\sqr{\mathcal{I}Ba} + \sqr{\mathcal{I}Bb}}
                = \mathcal{I}B\sqrt{a^2 + b^2} = \mathcal{I}B\sqrt{a^2 + (l - a)^2} =  \\
    &= 40\,\text{А} \cdot 10\,\text{мТл} \cdot \sqrt{\sqr{ 120\,\text{см} } + \sqr{170\,\text{см} - 120\,\text{см}}} = 520{,}00\,\text{мН}.
    \end{align*}
}
\solutionspace{120pt}

\tasknumber{2}%
\task{%
    В однородном горизонтальном магнитном поле с индукцией $B = 20\,\text{мТл}$ находится проводник,
    расположенный также горизонтально и перпендикулярно полю.
    Какой ток необходимо пустить по проводнику, чтобы он завис?
    Масса единицы длины проводника $\rho = 10\,\frac{\text{кг}}{\text{м}}$, $g = 10\,\frac{\text{м}}{\text{с}^{2}}$.
}
\answer{%
    $
            mg = B\mathcal{I} l, m=\rho l
            \implies \mathcal{I}
                = \frac{g\rho} {B}
                = \frac{10\,\frac{\text{м}}{\text{с}^{2}} \cdot 10\,\frac{\text{кг}}{\text{м}}}{ 20\,\text{мТл} }
                = 5{,}000000\,\text{кА}.
    $
}
\solutionspace{120pt}

\tasknumber{3}%
\task{%
    Определите работу, которую совершает сила Ампера при перемещении проводника длиной $l = 20\,\text{см}$
    с током силой $\mathcal{I} = 20\,\text{А}$ в однородном магнитном поле индукцией $B = 0{,}200000\,\text{Тл}$ на расстояние $d = 50\,\text{см}$.
    Проводник перпендикулярен линиям поля и движется в направлении силы Ампера.
}
\answer{%
    $
        A   = F \cdot d = B\mathcal{I} l  \cdot d
            = 0{,}200000\,\text{Тл}  \cdot 20\,\text{А}  \cdot 20\,\text{см}  \cdot 50\,\text{см}
            = 0{,}40000\,\text{Дж}.
    $
}

\variantsplitter

\addpersonalvariant{Леонид Викторов}

\tasknumber{1}%
\task{%
    Проводник длиной $l = 7\,\text{см}$ согнули под прямым углом так, что одна сторона угла оказалась равной $a = 4\,\text{см}$,
    и поместили в однородное магнитное поле с индукцией $B = 10\,\text{мТл}$ обеими сторонами перпендикулярно линиям индукции.
    Какая сила будет действовать на этот проводник при пропусканиии по нему тока $\mathcal{I} = 10\,\text{А}$?
}
\answer{%
    \begin{align*}
    F &= \sqrt{F_a^2 + F_b^2} = \sqrt{\sqr{\mathcal{I}Ba} + \sqr{\mathcal{I}Bb}}
                = \mathcal{I}B\sqrt{a^2 + b^2} = \mathcal{I}B\sqrt{a^2 + (l - a)^2} =  \\
    &= 10\,\text{А} \cdot 10\,\text{мТл} \cdot \sqrt{\sqr{ 4\,\text{см} } + \sqr{7\,\text{см} - 4\,\text{см}}} = 5{,}00\,\text{мН}.
    \end{align*}
}
\solutionspace{120pt}

\tasknumber{2}%
\task{%
    В однородном горизонтальном магнитном поле с индукцией $B = 20\,\text{мТл}$ находится проводник,
    расположенный также горизонтально и перпендикулярно полю.
    Какой ток необходимо пустить по проводнику, чтобы он завис?
    Масса единицы длины проводника $\rho = 40\,\frac{\text{кг}}{\text{м}}$, $g = 10\,\frac{\text{м}}{\text{с}^{2}}$.
}
\answer{%
    $
            mg = B\mathcal{I} l, m=\rho l
            \implies \mathcal{I}
                = \frac{g\rho} {B}
                = \frac{10\,\frac{\text{м}}{\text{с}^{2}} \cdot 40\,\frac{\text{кг}}{\text{м}}}{ 20\,\text{мТл} }
                = 20{,}000000\,\text{кА}.
    $
}
\solutionspace{120pt}

\tasknumber{3}%
\task{%
    Определите работу, которую совершает сила Ампера при перемещении проводника длиной $l = 40\,\text{см}$
    с током силой $\mathcal{I} = 10\,\text{А}$ в однородном магнитном поле индукцией $B = 0{,}500000\,\text{Тл}$ на расстояние $d = 20\,\text{см}$.
    Проводник перпендикулярен линиям поля и движется в направлении силы Ампера.
}
\answer{%
    $
        A   = F \cdot d = B\mathcal{I} l  \cdot d
            = 0{,}500000\,\text{Тл}  \cdot 10\,\text{А}  \cdot 40\,\text{см}  \cdot 20\,\text{см}
            = 0{,}40000\,\text{Дж}.
    $
}

\variantsplitter

\addpersonalvariant{Фёдор Гнутов}

\tasknumber{1}%
\task{%
    Проводник длиной $l = 310\,\text{см}$ согнули под прямым углом так, что одна сторона угла оказалась равной $a = 240\,\text{см}$,
    и поместили в однородное магнитное поле с индукцией $B = 2\,\text{мТл}$ обеими сторонами перпендикулярно линиям индукции.
    Какая сила будет действовать на этот проводник при пропусканиии по нему тока $\mathcal{I} = 40\,\text{А}$?
}
\answer{%
    \begin{align*}
    F &= \sqrt{F_a^2 + F_b^2} = \sqrt{\sqr{\mathcal{I}Ba} + \sqr{\mathcal{I}Bb}}
                = \mathcal{I}B\sqrt{a^2 + b^2} = \mathcal{I}B\sqrt{a^2 + (l - a)^2} =  \\
    &= 40\,\text{А} \cdot 2\,\text{мТл} \cdot \sqrt{\sqr{ 240\,\text{см} } + \sqr{310\,\text{см} - 240\,\text{см}}} = 200{,}00\,\text{мН}.
    \end{align*}
}
\solutionspace{120pt}

\tasknumber{2}%
\task{%
    В однородном горизонтальном магнитном поле с индукцией $B = 100\,\text{мТл}$ находится проводник,
    расположенный также горизонтально и перпендикулярно полю.
    Какой ток необходимо пустить по проводнику, чтобы он завис?
    Масса единицы длины проводника $\rho = 5\,\frac{\text{кг}}{\text{м}}$, $g = 10\,\frac{\text{м}}{\text{с}^{2}}$.
}
\answer{%
    $
            mg = B\mathcal{I} l, m=\rho l
            \implies \mathcal{I}
                = \frac{g\rho} {B}
                = \frac{10\,\frac{\text{м}}{\text{с}^{2}} \cdot 5\,\frac{\text{кг}}{\text{м}}}{ 100\,\text{мТл} }
                = 0{,}500000\,\text{кА}.
    $
}
\solutionspace{120pt}

\tasknumber{3}%
\task{%
    Определите работу, которую совершает сила Ампера при перемещении проводника длиной $l = 50\,\text{см}$
    с током силой $\mathcal{I} = 20\,\text{А}$ в однородном магнитном поле индукцией $B = 0{,}500000\,\text{Тл}$ на расстояние $d = 80\,\text{см}$.
    Проводник перпендикулярен линиям поля и движется в направлении силы Ампера.
}
\answer{%
    $
        A   = F \cdot d = B\mathcal{I} l  \cdot d
            = 0{,}500000\,\text{Тл}  \cdot 20\,\text{А}  \cdot 50\,\text{см}  \cdot 80\,\text{см}
            = 4{,}00000\,\text{Дж}.
    $
}

\variantsplitter

\addpersonalvariant{Виктор Жилин}

\tasknumber{1}%
\task{%
    Проводник длиной $l = 17\,\text{см}$ согнули под прямым углом так, что одна сторона угла оказалась равной $a = 5\,\text{см}$,
    и поместили в однородное магнитное поле с индукцией $B = 10\,\text{мТл}$ обеими сторонами перпендикулярно линиям индукции.
    Какая сила будет действовать на этот проводник при пропусканиии по нему тока $\mathcal{I} = 20\,\text{А}$?
}
\answer{%
    \begin{align*}
    F &= \sqrt{F_a^2 + F_b^2} = \sqrt{\sqr{\mathcal{I}Ba} + \sqr{\mathcal{I}Bb}}
                = \mathcal{I}B\sqrt{a^2 + b^2} = \mathcal{I}B\sqrt{a^2 + (l - a)^2} =  \\
    &= 20\,\text{А} \cdot 10\,\text{мТл} \cdot \sqrt{\sqr{ 5\,\text{см} } + \sqr{17\,\text{см} - 5\,\text{см}}} = 26{,}00\,\text{мН}.
    \end{align*}
}
\solutionspace{120pt}

\tasknumber{2}%
\task{%
    В однородном горизонтальном магнитном поле с индукцией $B = 20\,\text{мТл}$ находится проводник,
    расположенный также горизонтально и перпендикулярно полю.
    Какой ток необходимо пустить по проводнику, чтобы он завис?
    Масса единицы длины проводника $\rho = 100\,\frac{\text{кг}}{\text{м}}$, $g = 10\,\frac{\text{м}}{\text{с}^{2}}$.
}
\answer{%
    $
            mg = B\mathcal{I} l, m=\rho l
            \implies \mathcal{I}
                = \frac{g\rho} {B}
                = \frac{10\,\frac{\text{м}}{\text{с}^{2}} \cdot 100\,\frac{\text{кг}}{\text{м}}}{ 20\,\text{мТл} }
                = 50{,}000000\,\text{кА}.
    $
}
\solutionspace{120pt}

\tasknumber{3}%
\task{%
    Определите работу, которую совершает сила Ампера при перемещении проводника длиной $l = 50\,\text{см}$
    с током силой $\mathcal{I} = 5\,\text{А}$ в однородном магнитном поле индукцией $B = 0{,}100000\,\text{Тл}$ на расстояние $d = 50\,\text{см}$.
    Проводник перпендикулярен линиям поля и движется в направлении силы Ампера.
}
\answer{%
    $
        A   = F \cdot d = B\mathcal{I} l  \cdot d
            = 0{,}100000\,\text{Тл}  \cdot 5\,\text{А}  \cdot 50\,\text{см}  \cdot 50\,\text{см}
            = 0{,}12500\,\text{Дж}.
    $
}

\variantsplitter

\addpersonalvariant{Дмитрий Иванов}

\tasknumber{1}%
\task{%
    Проводник длиной $l = 310\,\text{см}$ согнули под прямым углом так, что одна сторона угла оказалась равной $a = 240\,\text{см}$,
    и поместили в однородное магнитное поле с индукцией $B = 2\,\text{мТл}$ обеими сторонами перпендикулярно линиям индукции.
    Какая сила будет действовать на этот проводник при пропусканиии по нему тока $\mathcal{I} = 20\,\text{А}$?
}
\answer{%
    \begin{align*}
    F &= \sqrt{F_a^2 + F_b^2} = \sqrt{\sqr{\mathcal{I}Ba} + \sqr{\mathcal{I}Bb}}
                = \mathcal{I}B\sqrt{a^2 + b^2} = \mathcal{I}B\sqrt{a^2 + (l - a)^2} =  \\
    &= 20\,\text{А} \cdot 2\,\text{мТл} \cdot \sqrt{\sqr{ 240\,\text{см} } + \sqr{310\,\text{см} - 240\,\text{см}}} = 100{,}00\,\text{мН}.
    \end{align*}
}
\solutionspace{120pt}

\tasknumber{2}%
\task{%
    В однородном горизонтальном магнитном поле с индукцией $B = 50\,\text{мТл}$ находится проводник,
    расположенный также горизонтально и перпендикулярно полю.
    Какой ток необходимо пустить по проводнику, чтобы он завис?
    Масса единицы длины проводника $\rho = 20\,\frac{\text{кг}}{\text{м}}$, $g = 10\,\frac{\text{м}}{\text{с}^{2}}$.
}
\answer{%
    $
            mg = B\mathcal{I} l, m=\rho l
            \implies \mathcal{I}
                = \frac{g\rho} {B}
                = \frac{10\,\frac{\text{м}}{\text{с}^{2}} \cdot 20\,\frac{\text{кг}}{\text{м}}}{ 50\,\text{мТл} }
                = 4{,}000000\,\text{кА}.
    $
}
\solutionspace{120pt}

\tasknumber{3}%
\task{%
    Определите работу, которую совершает сила Ампера при перемещении проводника длиной $l = 40\,\text{см}$
    с током силой $\mathcal{I} = 20\,\text{А}$ в однородном магнитном поле индукцией $B = 0{,}200000\,\text{Тл}$ на расстояние $d = 80\,\text{см}$.
    Проводник перпендикулярен линиям поля и движется в направлении силы Ампера.
}
\answer{%
    $
        A   = F \cdot d = B\mathcal{I} l  \cdot d
            = 0{,}200000\,\text{Тл}  \cdot 20\,\text{А}  \cdot 40\,\text{см}  \cdot 80\,\text{см}
            = 1{,}28000\,\text{Дж}.
    $
}

\variantsplitter

\addpersonalvariant{Олег Климов}

\tasknumber{1}%
\task{%
    Проводник длиной $l = 310\,\text{см}$ согнули под прямым углом так, что одна сторона угла оказалась равной $a = 240\,\text{см}$,
    и поместили в однородное магнитное поле с индукцией $B = 5\,\text{мТл}$ обеими сторонами перпендикулярно линиям индукции.
    Какая сила будет действовать на этот проводник при пропусканиии по нему тока $\mathcal{I} = 50\,\text{А}$?
}
\answer{%
    \begin{align*}
    F &= \sqrt{F_a^2 + F_b^2} = \sqrt{\sqr{\mathcal{I}Ba} + \sqr{\mathcal{I}Bb}}
                = \mathcal{I}B\sqrt{a^2 + b^2} = \mathcal{I}B\sqrt{a^2 + (l - a)^2} =  \\
    &= 50\,\text{А} \cdot 5\,\text{мТл} \cdot \sqrt{\sqr{ 240\,\text{см} } + \sqr{310\,\text{см} - 240\,\text{см}}} = 625{,}00\,\text{мН}.
    \end{align*}
}
\solutionspace{120pt}

\tasknumber{2}%
\task{%
    В однородном горизонтальном магнитном поле с индукцией $B = 20\,\text{мТл}$ находится проводник,
    расположенный также горизонтально и перпендикулярно полю.
    Какой ток необходимо пустить по проводнику, чтобы он завис?
    Масса единицы длины проводника $\rho = 40\,\frac{\text{кг}}{\text{м}}$, $g = 10\,\frac{\text{м}}{\text{с}^{2}}$.
}
\answer{%
    $
            mg = B\mathcal{I} l, m=\rho l
            \implies \mathcal{I}
                = \frac{g\rho} {B}
                = \frac{10\,\frac{\text{м}}{\text{с}^{2}} \cdot 40\,\frac{\text{кг}}{\text{м}}}{ 20\,\text{мТл} }
                = 20{,}000000\,\text{кА}.
    $
}
\solutionspace{120pt}

\tasknumber{3}%
\task{%
    Определите работу, которую совершает сила Ампера при перемещении проводника длиной $l = 40\,\text{см}$
    с током силой $\mathcal{I} = 20\,\text{А}$ в однородном магнитном поле индукцией $B = 0{,}100000\,\text{Тл}$ на расстояние $d = 20\,\text{см}$.
    Проводник перпендикулярен линиям поля и движется в направлении силы Ампера.
}
\answer{%
    $
        A   = F \cdot d = B\mathcal{I} l  \cdot d
            = 0{,}100000\,\text{Тл}  \cdot 20\,\text{А}  \cdot 40\,\text{см}  \cdot 20\,\text{см}
            = 0{,}16000\,\text{Дж}.
    $
}

\variantsplitter

\addpersonalvariant{Алина Леоничева}

\tasknumber{1}%
\task{%
    Проводник длиной $l = 7\,\text{см}$ согнули под прямым углом так, что одна сторона угла оказалась равной $a = 4\,\text{см}$,
    и поместили в однородное магнитное поле с индукцией $B = 5\,\text{мТл}$ обеими сторонами перпендикулярно линиям индукции.
    Какая сила будет действовать на этот проводник при пропусканиии по нему тока $\mathcal{I} = 40\,\text{А}$?
}
\answer{%
    \begin{align*}
    F &= \sqrt{F_a^2 + F_b^2} = \sqrt{\sqr{\mathcal{I}Ba} + \sqr{\mathcal{I}Bb}}
                = \mathcal{I}B\sqrt{a^2 + b^2} = \mathcal{I}B\sqrt{a^2 + (l - a)^2} =  \\
    &= 40\,\text{А} \cdot 5\,\text{мТл} \cdot \sqrt{\sqr{ 4\,\text{см} } + \sqr{7\,\text{см} - 4\,\text{см}}} = 10{,}00\,\text{мН}.
    \end{align*}
}
\solutionspace{120pt}

\tasknumber{2}%
\task{%
    В однородном горизонтальном магнитном поле с индукцией $B = 100\,\text{мТл}$ находится проводник,
    расположенный также горизонтально и перпендикулярно полю.
    Какой ток необходимо пустить по проводнику, чтобы он завис?
    Масса единицы длины проводника $\rho = 5\,\frac{\text{кг}}{\text{м}}$, $g = 10\,\frac{\text{м}}{\text{с}^{2}}$.
}
\answer{%
    $
            mg = B\mathcal{I} l, m=\rho l
            \implies \mathcal{I}
                = \frac{g\rho} {B}
                = \frac{10\,\frac{\text{м}}{\text{с}^{2}} \cdot 5\,\frac{\text{кг}}{\text{м}}}{ 100\,\text{мТл} }
                = 0{,}500000\,\text{кА}.
    $
}
\solutionspace{120pt}

\tasknumber{3}%
\task{%
    Определите работу, которую совершает сила Ампера при перемещении проводника длиной $l = 20\,\text{см}$
    с током силой $\mathcal{I} = 5\,\text{А}$ в однородном магнитном поле индукцией $B = 0{,}100000\,\text{Тл}$ на расстояние $d = 50\,\text{см}$.
    Проводник перпендикулярен линиям поля и движется в направлении силы Ампера.
}
\answer{%
    $
        A   = F \cdot d = B\mathcal{I} l  \cdot d
            = 0{,}100000\,\text{Тл}  \cdot 5\,\text{А}  \cdot 20\,\text{см}  \cdot 50\,\text{см}
            = 0{,}05000\,\text{Дж}.
    $
}

\variantsplitter

\addpersonalvariant{Ирина Лин}

\tasknumber{1}%
\task{%
    Проводник длиной $l = 7\,\text{см}$ согнули под прямым углом так, что одна сторона угла оказалась равной $a = 4\,\text{см}$,
    и поместили в однородное магнитное поле с индукцией $B = 10\,\text{мТл}$ обеими сторонами перпендикулярно линиям индукции.
    Какая сила будет действовать на этот проводник при пропусканиии по нему тока $\mathcal{I} = 10\,\text{А}$?
}
\answer{%
    \begin{align*}
    F &= \sqrt{F_a^2 + F_b^2} = \sqrt{\sqr{\mathcal{I}Ba} + \sqr{\mathcal{I}Bb}}
                = \mathcal{I}B\sqrt{a^2 + b^2} = \mathcal{I}B\sqrt{a^2 + (l - a)^2} =  \\
    &= 10\,\text{А} \cdot 10\,\text{мТл} \cdot \sqrt{\sqr{ 4\,\text{см} } + \sqr{7\,\text{см} - 4\,\text{см}}} = 5{,}00\,\text{мН}.
    \end{align*}
}
\solutionspace{120pt}

\tasknumber{2}%
\task{%
    В однородном горизонтальном магнитном поле с индукцией $B = 50\,\text{мТл}$ находится проводник,
    расположенный также горизонтально и перпендикулярно полю.
    Какой ток необходимо пустить по проводнику, чтобы он завис?
    Масса единицы длины проводника $\rho = 20\,\frac{\text{кг}}{\text{м}}$, $g = 10\,\frac{\text{м}}{\text{с}^{2}}$.
}
\answer{%
    $
            mg = B\mathcal{I} l, m=\rho l
            \implies \mathcal{I}
                = \frac{g\rho} {B}
                = \frac{10\,\frac{\text{м}}{\text{с}^{2}} \cdot 20\,\frac{\text{кг}}{\text{м}}}{ 50\,\text{мТл} }
                = 4{,}000000\,\text{кА}.
    $
}
\solutionspace{120pt}

\tasknumber{3}%
\task{%
    Определите работу, которую совершает сила Ампера при перемещении проводника длиной $l = 40\,\text{см}$
    с током силой $\mathcal{I} = 20\,\text{А}$ в однородном магнитном поле индукцией $B = 0{,}200000\,\text{Тл}$ на расстояние $d = 80\,\text{см}$.
    Проводник перпендикулярен линиям поля и движется в направлении силы Ампера.
}
\answer{%
    $
        A   = F \cdot d = B\mathcal{I} l  \cdot d
            = 0{,}200000\,\text{Тл}  \cdot 20\,\text{А}  \cdot 40\,\text{см}  \cdot 80\,\text{см}
            = 1{,}28000\,\text{Дж}.
    $
}

\variantsplitter

\addpersonalvariant{Александр Наумов}

\tasknumber{1}%
\task{%
    Проводник длиной $l = 7\,\text{см}$ согнули под прямым углом так, что одна сторона угла оказалась равной $a = 4\,\text{см}$,
    и поместили в однородное магнитное поле с индукцией $B = 10\,\text{мТл}$ обеими сторонами перпендикулярно линиям индукции.
    Какая сила будет действовать на этот проводник при пропусканиии по нему тока $\mathcal{I} = 50\,\text{А}$?
}
\answer{%
    \begin{align*}
    F &= \sqrt{F_a^2 + F_b^2} = \sqrt{\sqr{\mathcal{I}Ba} + \sqr{\mathcal{I}Bb}}
                = \mathcal{I}B\sqrt{a^2 + b^2} = \mathcal{I}B\sqrt{a^2 + (l - a)^2} =  \\
    &= 50\,\text{А} \cdot 10\,\text{мТл} \cdot \sqrt{\sqr{ 4\,\text{см} } + \sqr{7\,\text{см} - 4\,\text{см}}} = 25{,}00\,\text{мН}.
    \end{align*}
}
\solutionspace{120pt}

\tasknumber{2}%
\task{%
    В однородном горизонтальном магнитном поле с индукцией $B = 50\,\text{мТл}$ находится проводник,
    расположенный также горизонтально и перпендикулярно полю.
    Какой ток необходимо пустить по проводнику, чтобы он завис?
    Масса единицы длины проводника $\rho = 40\,\frac{\text{кг}}{\text{м}}$, $g = 10\,\frac{\text{м}}{\text{с}^{2}}$.
}
\answer{%
    $
            mg = B\mathcal{I} l, m=\rho l
            \implies \mathcal{I}
                = \frac{g\rho} {B}
                = \frac{10\,\frac{\text{м}}{\text{с}^{2}} \cdot 40\,\frac{\text{кг}}{\text{м}}}{ 50\,\text{мТл} }
                = 8{,}000000\,\text{кА}.
    $
}
\solutionspace{120pt}

\tasknumber{3}%
\task{%
    Определите работу, которую совершает сила Ампера при перемещении проводника длиной $l = 20\,\text{см}$
    с током силой $\mathcal{I} = 5\,\text{А}$ в однородном магнитном поле индукцией $B = 0{,}200000\,\text{Тл}$ на расстояние $d = 50\,\text{см}$.
    Проводник перпендикулярен линиям поля и движется в направлении силы Ампера.
}
\answer{%
    $
        A   = F \cdot d = B\mathcal{I} l  \cdot d
            = 0{,}200000\,\text{Тл}  \cdot 5\,\text{А}  \cdot 20\,\text{см}  \cdot 50\,\text{см}
            = 0{,}10000\,\text{Дж}.
    $
}

\variantsplitter

\addpersonalvariant{Георгий Новиков}

\tasknumber{1}%
\task{%
    Проводник длиной $l = 17\,\text{см}$ согнули под прямым углом так, что одна сторона угла оказалась равной $a = 5\,\text{см}$,
    и поместили в однородное магнитное поле с индукцией $B = 5\,\text{мТл}$ обеими сторонами перпендикулярно линиям индукции.
    Какая сила будет действовать на этот проводник при пропусканиии по нему тока $\mathcal{I} = 50\,\text{А}$?
}
\answer{%
    \begin{align*}
    F &= \sqrt{F_a^2 + F_b^2} = \sqrt{\sqr{\mathcal{I}Ba} + \sqr{\mathcal{I}Bb}}
                = \mathcal{I}B\sqrt{a^2 + b^2} = \mathcal{I}B\sqrt{a^2 + (l - a)^2} =  \\
    &= 50\,\text{А} \cdot 5\,\text{мТл} \cdot \sqrt{\sqr{ 5\,\text{см} } + \sqr{17\,\text{см} - 5\,\text{см}}} = 32{,}50\,\text{мН}.
    \end{align*}
}
\solutionspace{120pt}

\tasknumber{2}%
\task{%
    В однородном горизонтальном магнитном поле с индукцией $B = 20\,\text{мТл}$ находится проводник,
    расположенный также горизонтально и перпендикулярно полю.
    Какой ток необходимо пустить по проводнику, чтобы он завис?
    Масса единицы длины проводника $\rho = 20\,\frac{\text{кг}}{\text{м}}$, $g = 10\,\frac{\text{м}}{\text{с}^{2}}$.
}
\answer{%
    $
            mg = B\mathcal{I} l, m=\rho l
            \implies \mathcal{I}
                = \frac{g\rho} {B}
                = \frac{10\,\frac{\text{м}}{\text{с}^{2}} \cdot 20\,\frac{\text{кг}}{\text{м}}}{ 20\,\text{мТл} }
                = 10{,}000000\,\text{кА}.
    $
}
\solutionspace{120pt}

\tasknumber{3}%
\task{%
    Определите работу, которую совершает сила Ампера при перемещении проводника длиной $l = 40\,\text{см}$
    с током силой $\mathcal{I} = 10\,\text{А}$ в однородном магнитном поле индукцией $B = 0{,}100000\,\text{Тл}$ на расстояние $d = 80\,\text{см}$.
    Проводник перпендикулярен линиям поля и движется в направлении силы Ампера.
}
\answer{%
    $
        A   = F \cdot d = B\mathcal{I} l  \cdot d
            = 0{,}100000\,\text{Тл}  \cdot 10\,\text{А}  \cdot 40\,\text{см}  \cdot 80\,\text{см}
            = 0{,}32000\,\text{Дж}.
    $
}

\variantsplitter

\addpersonalvariant{Егор Осипов}

\tasknumber{1}%
\task{%
    Проводник длиной $l = 310\,\text{см}$ согнули под прямым углом так, что одна сторона угла оказалась равной $a = 240\,\text{см}$,
    и поместили в однородное магнитное поле с индукцией $B = 10\,\text{мТл}$ обеими сторонами перпендикулярно линиям индукции.
    Какая сила будет действовать на этот проводник при пропусканиии по нему тока $\mathcal{I} = 40\,\text{А}$?
}
\answer{%
    \begin{align*}
    F &= \sqrt{F_a^2 + F_b^2} = \sqrt{\sqr{\mathcal{I}Ba} + \sqr{\mathcal{I}Bb}}
                = \mathcal{I}B\sqrt{a^2 + b^2} = \mathcal{I}B\sqrt{a^2 + (l - a)^2} =  \\
    &= 40\,\text{А} \cdot 10\,\text{мТл} \cdot \sqrt{\sqr{ 240\,\text{см} } + \sqr{310\,\text{см} - 240\,\text{см}}} = 1000{,}00\,\text{мН}.
    \end{align*}
}
\solutionspace{120pt}

\tasknumber{2}%
\task{%
    В однородном горизонтальном магнитном поле с индукцией $B = 10\,\text{мТл}$ находится проводник,
    расположенный также горизонтально и перпендикулярно полю.
    Какой ток необходимо пустить по проводнику, чтобы он завис?
    Масса единицы длины проводника $\rho = 40\,\frac{\text{кг}}{\text{м}}$, $g = 10\,\frac{\text{м}}{\text{с}^{2}}$.
}
\answer{%
    $
            mg = B\mathcal{I} l, m=\rho l
            \implies \mathcal{I}
                = \frac{g\rho} {B}
                = \frac{10\,\frac{\text{м}}{\text{с}^{2}} \cdot 40\,\frac{\text{кг}}{\text{м}}}{ 10\,\text{мТл} }
                = 40{,}000000\,\text{кА}.
    $
}
\solutionspace{120pt}

\tasknumber{3}%
\task{%
    Определите работу, которую совершает сила Ампера при перемещении проводника длиной $l = 20\,\text{см}$
    с током силой $\mathcal{I} = 20\,\text{А}$ в однородном магнитном поле индукцией $B = 0{,}200000\,\text{Тл}$ на расстояние $d = 20\,\text{см}$.
    Проводник перпендикулярен линиям поля и движется в направлении силы Ампера.
}
\answer{%
    $
        A   = F \cdot d = B\mathcal{I} l  \cdot d
            = 0{,}200000\,\text{Тл}  \cdot 20\,\text{А}  \cdot 20\,\text{см}  \cdot 20\,\text{см}
            = 0{,}16000\,\text{Дж}.
    $
}

\variantsplitter

\addpersonalvariant{Руслан Перепелица}

\tasknumber{1}%
\task{%
    Проводник длиной $l = 310\,\text{см}$ согнули под прямым углом так, что одна сторона угла оказалась равной $a = 240\,\text{см}$,
    и поместили в однородное магнитное поле с индукцией $B = 2\,\text{мТл}$ обеими сторонами перпендикулярно линиям индукции.
    Какая сила будет действовать на этот проводник при пропусканиии по нему тока $\mathcal{I} = 20\,\text{А}$?
}
\answer{%
    \begin{align*}
    F &= \sqrt{F_a^2 + F_b^2} = \sqrt{\sqr{\mathcal{I}Ba} + \sqr{\mathcal{I}Bb}}
                = \mathcal{I}B\sqrt{a^2 + b^2} = \mathcal{I}B\sqrt{a^2 + (l - a)^2} =  \\
    &= 20\,\text{А} \cdot 2\,\text{мТл} \cdot \sqrt{\sqr{ 240\,\text{см} } + \sqr{310\,\text{см} - 240\,\text{см}}} = 100{,}00\,\text{мН}.
    \end{align*}
}
\solutionspace{120pt}

\tasknumber{2}%
\task{%
    В однородном горизонтальном магнитном поле с индукцией $B = 10\,\text{мТл}$ находится проводник,
    расположенный также горизонтально и перпендикулярно полю.
    Какой ток необходимо пустить по проводнику, чтобы он завис?
    Масса единицы длины проводника $\rho = 5\,\frac{\text{кг}}{\text{м}}$, $g = 10\,\frac{\text{м}}{\text{с}^{2}}$.
}
\answer{%
    $
            mg = B\mathcal{I} l, m=\rho l
            \implies \mathcal{I}
                = \frac{g\rho} {B}
                = \frac{10\,\frac{\text{м}}{\text{с}^{2}} \cdot 5\,\frac{\text{кг}}{\text{м}}}{ 10\,\text{мТл} }
                = 5{,}000000\,\text{кА}.
    $
}
\solutionspace{120pt}

\tasknumber{3}%
\task{%
    Определите работу, которую совершает сила Ампера при перемещении проводника длиной $l = 50\,\text{см}$
    с током силой $\mathcal{I} = 20\,\text{А}$ в однородном магнитном поле индукцией $B = 0{,}200000\,\text{Тл}$ на расстояние $d = 20\,\text{см}$.
    Проводник перпендикулярен линиям поля и движется в направлении силы Ампера.
}
\answer{%
    $
        A   = F \cdot d = B\mathcal{I} l  \cdot d
            = 0{,}200000\,\text{Тл}  \cdot 20\,\text{А}  \cdot 50\,\text{см}  \cdot 20\,\text{см}
            = 0{,}40000\,\text{Дж}.
    $
}

\variantsplitter

\addpersonalvariant{Егор Подуровский}

\tasknumber{1}%
\task{%
    Проводник длиной $l = 70\,\text{см}$ согнули под прямым углом так, что одна сторона угла оказалась равной $a = 40\,\text{см}$,
    и поместили в однородное магнитное поле с индукцией $B = 2\,\text{мТл}$ обеими сторонами перпендикулярно линиям индукции.
    Какая сила будет действовать на этот проводник при пропусканиии по нему тока $\mathcal{I} = 10\,\text{А}$?
}
\answer{%
    \begin{align*}
    F &= \sqrt{F_a^2 + F_b^2} = \sqrt{\sqr{\mathcal{I}Ba} + \sqr{\mathcal{I}Bb}}
                = \mathcal{I}B\sqrt{a^2 + b^2} = \mathcal{I}B\sqrt{a^2 + (l - a)^2} =  \\
    &= 10\,\text{А} \cdot 2\,\text{мТл} \cdot \sqrt{\sqr{ 40\,\text{см} } + \sqr{70\,\text{см} - 40\,\text{см}}} = 10{,}00\,\text{мН}.
    \end{align*}
}
\solutionspace{120pt}

\tasknumber{2}%
\task{%
    В однородном горизонтальном магнитном поле с индукцией $B = 50\,\text{мТл}$ находится проводник,
    расположенный также горизонтально и перпендикулярно полю.
    Какой ток необходимо пустить по проводнику, чтобы он завис?
    Масса единицы длины проводника $\rho = 5\,\frac{\text{кг}}{\text{м}}$, $g = 10\,\frac{\text{м}}{\text{с}^{2}}$.
}
\answer{%
    $
            mg = B\mathcal{I} l, m=\rho l
            \implies \mathcal{I}
                = \frac{g\rho} {B}
                = \frac{10\,\frac{\text{м}}{\text{с}^{2}} \cdot 5\,\frac{\text{кг}}{\text{м}}}{ 50\,\text{мТл} }
                = 1{,}000000\,\text{кА}.
    $
}
\solutionspace{120pt}

\tasknumber{3}%
\task{%
    Определите работу, которую совершает сила Ампера при перемещении проводника длиной $l = 50\,\text{см}$
    с током силой $\mathcal{I} = 10\,\text{А}$ в однородном магнитном поле индукцией $B = 0{,}200000\,\text{Тл}$ на расстояние $d = 20\,\text{см}$.
    Проводник перпендикулярен линиям поля и движется в направлении силы Ампера.
}
\answer{%
    $
        A   = F \cdot d = B\mathcal{I} l  \cdot d
            = 0{,}200000\,\text{Тл}  \cdot 10\,\text{А}  \cdot 50\,\text{см}  \cdot 20\,\text{см}
            = 0{,}20000\,\text{Дж}.
    $
}

\variantsplitter

\addpersonalvariant{Александр Селехметьев}

\tasknumber{1}%
\task{%
    Проводник длиной $l = 17\,\text{см}$ согнули под прямым углом так, что одна сторона угла оказалась равной $a = 12\,\text{см}$,
    и поместили в однородное магнитное поле с индукцией $B = 10\,\text{мТл}$ обеими сторонами перпендикулярно линиям индукции.
    Какая сила будет действовать на этот проводник при пропусканиии по нему тока $\mathcal{I} = 10\,\text{А}$?
}
\answer{%
    \begin{align*}
    F &= \sqrt{F_a^2 + F_b^2} = \sqrt{\sqr{\mathcal{I}Ba} + \sqr{\mathcal{I}Bb}}
                = \mathcal{I}B\sqrt{a^2 + b^2} = \mathcal{I}B\sqrt{a^2 + (l - a)^2} =  \\
    &= 10\,\text{А} \cdot 10\,\text{мТл} \cdot \sqrt{\sqr{ 12\,\text{см} } + \sqr{17\,\text{см} - 12\,\text{см}}} = 13{,}00\,\text{мН}.
    \end{align*}
}
\solutionspace{120pt}

\tasknumber{2}%
\task{%
    В однородном горизонтальном магнитном поле с индукцией $B = 50\,\text{мТл}$ находится проводник,
    расположенный также горизонтально и перпендикулярно полю.
    Какой ток необходимо пустить по проводнику, чтобы он завис?
    Масса единицы длины проводника $\rho = 20\,\frac{\text{кг}}{\text{м}}$, $g = 10\,\frac{\text{м}}{\text{с}^{2}}$.
}
\answer{%
    $
            mg = B\mathcal{I} l, m=\rho l
            \implies \mathcal{I}
                = \frac{g\rho} {B}
                = \frac{10\,\frac{\text{м}}{\text{с}^{2}} \cdot 20\,\frac{\text{кг}}{\text{м}}}{ 50\,\text{мТл} }
                = 4{,}000000\,\text{кА}.
    $
}
\solutionspace{120pt}

\tasknumber{3}%
\task{%
    Определите работу, которую совершает сила Ампера при перемещении проводника длиной $l = 20\,\text{см}$
    с током силой $\mathcal{I} = 5\,\text{А}$ в однородном магнитном поле индукцией $B = 0{,}200000\,\text{Тл}$ на расстояние $d = 80\,\text{см}$.
    Проводник перпендикулярен линиям поля и движется в направлении силы Ампера.
}
\answer{%
    $
        A   = F \cdot d = B\mathcal{I} l  \cdot d
            = 0{,}200000\,\text{Тл}  \cdot 5\,\text{А}  \cdot 20\,\text{см}  \cdot 80\,\text{см}
            = 0{,}16000\,\text{Дж}.
    $
}

\variantsplitter

\addpersonalvariant{Алина Филиппова}

\tasknumber{1}%
\task{%
    Проводник длиной $l = 170\,\text{см}$ согнули под прямым углом так, что одна сторона угла оказалась равной $a = 120\,\text{см}$,
    и поместили в однородное магнитное поле с индукцией $B = 2\,\text{мТл}$ обеими сторонами перпендикулярно линиям индукции.
    Какая сила будет действовать на этот проводник при пропусканиии по нему тока $\mathcal{I} = 20\,\text{А}$?
}
\answer{%
    \begin{align*}
    F &= \sqrt{F_a^2 + F_b^2} = \sqrt{\sqr{\mathcal{I}Ba} + \sqr{\mathcal{I}Bb}}
                = \mathcal{I}B\sqrt{a^2 + b^2} = \mathcal{I}B\sqrt{a^2 + (l - a)^2} =  \\
    &= 20\,\text{А} \cdot 2\,\text{мТл} \cdot \sqrt{\sqr{ 120\,\text{см} } + \sqr{170\,\text{см} - 120\,\text{см}}} = 52{,}00\,\text{мН}.
    \end{align*}
}
\solutionspace{120pt}

\tasknumber{2}%
\task{%
    В однородном горизонтальном магнитном поле с индукцией $B = 50\,\text{мТл}$ находится проводник,
    расположенный также горизонтально и перпендикулярно полю.
    Какой ток необходимо пустить по проводнику, чтобы он завис?
    Масса единицы длины проводника $\rho = 5\,\frac{\text{кг}}{\text{м}}$, $g = 10\,\frac{\text{м}}{\text{с}^{2}}$.
}
\answer{%
    $
            mg = B\mathcal{I} l, m=\rho l
            \implies \mathcal{I}
                = \frac{g\rho} {B}
                = \frac{10\,\frac{\text{м}}{\text{с}^{2}} \cdot 5\,\frac{\text{кг}}{\text{м}}}{ 50\,\text{мТл} }
                = 1{,}000000\,\text{кА}.
    $
}
\solutionspace{120pt}

\tasknumber{3}%
\task{%
    Определите работу, которую совершает сила Ампера при перемещении проводника длиной $l = 30\,\text{см}$
    с током силой $\mathcal{I} = 5\,\text{А}$ в однородном магнитном поле индукцией $B = 0{,}500000\,\text{Тл}$ на расстояние $d = 50\,\text{см}$.
    Проводник перпендикулярен линиям поля и движется в направлении силы Ампера.
}
\answer{%
    $
        A   = F \cdot d = B\mathcal{I} l  \cdot d
            = 0{,}500000\,\text{Тл}  \cdot 5\,\text{А}  \cdot 30\,\text{см}  \cdot 50\,\text{см}
            = 0{,}37500\,\text{Дж}.
    $
}

\variantsplitter

\addpersonalvariant{Алина Яшина}

\tasknumber{1}%
\task{%
    Проводник длиной $l = 7\,\text{см}$ согнули под прямым углом так, что одна сторона угла оказалась равной $a = 3\,\text{см}$,
    и поместили в однородное магнитное поле с индукцией $B = 2\,\text{мТл}$ обеими сторонами перпендикулярно линиям индукции.
    Какая сила будет действовать на этот проводник при пропусканиии по нему тока $\mathcal{I} = 20\,\text{А}$?
}
\answer{%
    \begin{align*}
    F &= \sqrt{F_a^2 + F_b^2} = \sqrt{\sqr{\mathcal{I}Ba} + \sqr{\mathcal{I}Bb}}
                = \mathcal{I}B\sqrt{a^2 + b^2} = \mathcal{I}B\sqrt{a^2 + (l - a)^2} =  \\
    &= 20\,\text{А} \cdot 2\,\text{мТл} \cdot \sqrt{\sqr{ 3\,\text{см} } + \sqr{7\,\text{см} - 3\,\text{см}}} = 2{,}00\,\text{мН}.
    \end{align*}
}
\solutionspace{120pt}

\tasknumber{2}%
\task{%
    В однородном горизонтальном магнитном поле с индукцией $B = 50\,\text{мТл}$ находится проводник,
    расположенный также горизонтально и перпендикулярно полю.
    Какой ток необходимо пустить по проводнику, чтобы он завис?
    Масса единицы длины проводника $\rho = 40\,\frac{\text{кг}}{\text{м}}$, $g = 10\,\frac{\text{м}}{\text{с}^{2}}$.
}
\answer{%
    $
            mg = B\mathcal{I} l, m=\rho l
            \implies \mathcal{I}
                = \frac{g\rho} {B}
                = \frac{10\,\frac{\text{м}}{\text{с}^{2}} \cdot 40\,\frac{\text{кг}}{\text{м}}}{ 50\,\text{мТл} }
                = 8{,}000000\,\text{кА}.
    $
}
\solutionspace{120pt}

\tasknumber{3}%
\task{%
    Определите работу, которую совершает сила Ампера при перемещении проводника длиной $l = 40\,\text{см}$
    с током силой $\mathcal{I} = 10\,\text{А}$ в однородном магнитном поле индукцией $B = 0{,}500000\,\text{Тл}$ на расстояние $d = 80\,\text{см}$.
    Проводник перпендикулярен линиям поля и движется в направлении силы Ампера.
}
\answer{%
    $
        A   = F \cdot d = B\mathcal{I} l  \cdot d
            = 0{,}500000\,\text{Тл}  \cdot 10\,\text{А}  \cdot 40\,\text{см}  \cdot 80\,\text{см}
            = 1{,}60000\,\text{Дж}.
    $
}
% autogenerated
