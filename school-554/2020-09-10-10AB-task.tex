\setdate{10~сентября~2020}
\setclass{10«АБ»}

\addpersonalvariant{Михаил Бурмистров}

\tasknumber{1}%
\task{%
    Запишите определения, формулы и физические законы (можно сокращать, но не упустите ключевое):
    \begin{enumerate}
        \item поступательное движение,
        \item равномерное прямолинейное движение,
        \item перемещение при равномерном прямолинейном движении (векторно),
        \item перемещение при равномерном прямолинейном движении (в проекциях).
    \end{enumerate}
}
\solutionspace{120pt}

\tasknumber{2}%
\task{%
    Положив $\vec a = 2\vec i -2 \vec j, \vec b = -4\vec i -2 \vec j$,
    \begin{enumerate}
        \item найдите сумму векторов $\vec a + \vec b$,
        \item постройте сумму векторов $\vec a + \vec b$ на чертеже,
        \item определите модуль суммы векторов $\modul{\vec a + \vec b}$,
        \item вычислите разность векторов $\vec a - \vec b.$
    \end{enumerate}
}
\answer{%
    $\vec a + \vec b = -2\vec i -4\vec i, \vec a - \vec b = 6\vec i + 0\vec i, \modul{\vec a + \vec b} = \sqrt{\sqr-2 + \sqr-4} \approx 4{,}47.$
}
\solutionspace{120pt}

\tasknumber{3}%
\task{%
    Небольшой лёгкий самолёт взлетел из аэропорта, пролетел $7\,\text{км}$ строго на север, потом повернул и пролетел $24\,\text{км}$ на запад,
    а после по прямой вернулся обратно в аэропорт.
    Определите путь и модуль перемещения самолёта, считая Землю плоской.
}
\solutionspace{120pt}

\tasknumber{4}%
\task{%
    Женя и Валя едут на лошадях: Женя движется на юг со скоростью $5\,\frac{\text{км}}{\text{ч}}$, Валя — на запад со скоростью $12\,\frac{\text{км}}{\text{ч}}$.
    Определите скорость Вали относительно Жени.
    Сделайте рисунок («вид сверху»), подпишите кто где, укажите скорости (в т.ч.
    направление).
}

\variantsplitter

\addpersonalvariant{Ирина Ан}

\tasknumber{1}%
\task{%
    Запишите определения, формулы и физические законы (можно сокращать, но не упустите ключевое):
    \begin{enumerate}
        \item система отсчёта,
        \item перемещение,
        \item положение тела при равномерном прямолинейном движении (векторно),
        \item положение тела при равномерном прямолинейном движении (в проекциях).
    \end{enumerate}
}
\solutionspace{120pt}

\tasknumber{2}%
\task{%
    Положив $\vec a = 2\vec i + 2 \vec j, \vec b = -4\vec i + 2 \vec j$,
    \begin{enumerate}
        \item найдите сумму векторов $\vec a + \vec b$,
        \item постройте сумму векторов $\vec a + \vec b$ на чертеже,
        \item определите модуль суммы векторов $\modul{\vec a + \vec b}$,
        \item вычислите разность векторов $\vec a - \vec b.$
    \end{enumerate}
}
\answer{%
    $\vec a + \vec b = -2\vec i + 4\vec i, \vec a - \vec b = 6\vec i + 0\vec i, \modul{\vec a + \vec b} = \sqrt{\sqr-2 + \sqr4} \approx 4{,}47.$
}
\solutionspace{120pt}

\tasknumber{3}%
\task{%
    Небольшой лёгкий самолёт взлетел из аэропорта, пролетел $24\,\text{км}$ строго на север, потом повернул и пролетел $7\,\text{км}$ на восток,
    а после по прямой вернулся обратно в аэропорт.
    Определите путь и модуль перемещения самолёта, считая Землю плоской.
}
\solutionspace{120pt}

\tasknumber{4}%
\task{%
    Женя и Валя едут на лошадях: Женя движется на восток со скоростью $12\,\frac{\text{м}}{\text{с}}$, Валя — на север со скоростью $5\,\frac{\text{м}}{\text{с}}$.
    Определите скорость Вали относительно Жени.
    Сделайте рисунок («вид сверху»), подпишите кто где, укажите скорости (в т.ч.
    направление).
}

\variantsplitter

\addpersonalvariant{Софья Андрианова}

\tasknumber{1}%
\task{%
    Запишите определения, формулы и физические законы (можно сокращать, но не упустите ключевое):
    \begin{enumerate}
        \item механическое движение,
        \item путь,
        \item перемещение при равномерном прямолинейном движении (векторно),
        \item перемещение при равномерном прямолинейном движении (в проекциях).
    \end{enumerate}
}
\solutionspace{120pt}

\tasknumber{2}%
\task{%
    Положив $\vec a = 3\vec i + 2 \vec j, \vec b = 3\vec i + 3 \vec j$,
    \begin{enumerate}
        \item найдите сумму векторов $\vec a + \vec b$,
        \item постройте сумму векторов $\vec a + \vec b$ на чертеже,
        \item определите модуль суммы векторов $\modul{\vec a + \vec b}$,
        \item вычислите разность векторов $\vec a - \vec b.$
    \end{enumerate}
}
\answer{%
    $\vec a + \vec b = 6\vec i + 5\vec i, \vec a - \vec b = 0\vec i -1\vec i, \modul{\vec a + \vec b} = \sqrt{\sqr6 + \sqr5} \approx 7{,}81.$
}
\solutionspace{120pt}

\tasknumber{3}%
\task{%
    Небольшой лёгкий самолёт взлетел из аэропорта, пролетел $40\,\text{км}$ строго на север, потом повернул и пролетел $30\,\text{км}$ на запад,
    а после по прямой вернулся обратно в аэропорт.
    Определите путь и модуль перемещения самолёта, считая Землю плоской.
}
\solutionspace{120pt}

\tasknumber{4}%
\task{%
    Женя и Валя едут на велосипедах: Женя движется на восток со скоростью $4\,\frac{\text{км}}{\text{ч}}$, Валя — на восток со скоростью $3\,\frac{\text{км}}{\text{ч}}$.
    Определите скорость Вали относительно Жени.
    Сделайте рисунок («вид сверху»), подпишите кто где, укажите скорости (в т.ч.
    направление).
}

\variantsplitter

\addpersonalvariant{Владимир Артемчук}

\tasknumber{1}%
\task{%
    Запишите определения, формулы и физические законы (можно сокращать, но не упустите ключевое):
    \begin{enumerate}
        \item механическое движение,
        \item перемещение,
        \item перемещение при равномерном прямолинейном движении (векторно),
        \item перемещение при равномерном прямолинейном движении (в проекциях).
    \end{enumerate}
}
\solutionspace{120pt}

\tasknumber{2}%
\task{%
    Положив $\vec a = 3\vec i -4 \vec j, \vec b = -4\vec i + 2 \vec j$,
    \begin{enumerate}
        \item найдите сумму векторов $\vec a + \vec b$,
        \item постройте сумму векторов $\vec a + \vec b$ на чертеже,
        \item определите модуль суммы векторов $\modul{\vec a + \vec b}$,
        \item вычислите разность векторов $\vec a - \vec b.$
    \end{enumerate}
}
\answer{%
    $\vec a + \vec b = -1\vec i -2\vec i, \vec a - \vec b = 7\vec i -6\vec i, \modul{\vec a + \vec b} = \sqrt{\sqr-1 + \sqr-2} \approx 2{,}24.$
}
\solutionspace{120pt}

\tasknumber{3}%
\task{%
    Небольшой лёгкий самолёт взлетел из аэропорта, пролетел $30\,\text{км}$ строго на север, потом повернул и пролетел $40\,\text{км}$ на запад,
    а после по прямой вернулся обратно в аэропорт.
    Определите путь и модуль перемещения самолёта, считая Землю плоской.
}
\solutionspace{120pt}

\tasknumber{4}%
\task{%
    Женя и Валя едут на мотоциклах: Женя движется на юг со скоростью $5\,\frac{\text{км}}{\text{ч}}$, Валя — на юг со скоростью $12\,\frac{\text{км}}{\text{ч}}$.
    Определите скорость Вали относительно Жени.
    Сделайте рисунок («вид сверху»), подпишите кто где, укажите скорости (в т.ч.
    направление).
}

\variantsplitter

\addpersonalvariant{Софья Белянкина}

\tasknumber{1}%
\task{%
    Запишите определения, формулы и физические законы (можно сокращать, но не упустите ключевое):
    \begin{enumerate}
        \item система отсчёта,
        \item равномерное прямолинейное движение,
        \item перемещение при равномерном прямолинейном движении (векторно),
        \item перемещение при равномерном прямолинейном движении (в проекциях).
    \end{enumerate}
}
\solutionspace{120pt}

\tasknumber{2}%
\task{%
    Положив $\vec a = -2\vec i + 4 \vec j, \vec b = -4\vec i -2 \vec j$,
    \begin{enumerate}
        \item найдите сумму векторов $\vec a + \vec b$,
        \item постройте сумму векторов $\vec a + \vec b$ на чертеже,
        \item определите модуль суммы векторов $\modul{\vec a + \vec b}$,
        \item вычислите разность векторов $\vec a - \vec b.$
    \end{enumerate}
}
\answer{%
    $\vec a + \vec b = -6\vec i + 2\vec i, \vec a - \vec b = 2\vec i + 6\vec i, \modul{\vec a + \vec b} = \sqrt{\sqr-6 + \sqr2} \approx 6{,}32.$
}
\solutionspace{120pt}

\tasknumber{3}%
\task{%
    Небольшой лёгкий самолёт взлетел из аэропорта, пролетел $12\,\text{км}$ строго на юг, потом повернул и пролетел $5\,\text{км}$ на запад,
    а после по прямой вернулся обратно в аэропорт.
    Определите путь и модуль перемещения самолёта, считая Землю плоской.
}
\solutionspace{120pt}

\tasknumber{4}%
\task{%
    Женя и Валя едут на мотоциклах: Женя движется на восток со скоростью $5\,\frac{\text{км}}{\text{ч}}$, Валя — на запад со скоростью $12\,\frac{\text{км}}{\text{ч}}$.
    Определите скорость Вали относительно Жени.
    Сделайте рисунок («вид сверху»), подпишите кто где, укажите скорости (в т.ч.
    направление).
}

\variantsplitter

\addpersonalvariant{Варвара Егиазарян}

\tasknumber{1}%
\task{%
    Запишите определения, формулы и физические законы (можно сокращать, но не упустите ключевое):
    \begin{enumerate}
        \item поступательное движение,
        \item траектория,
        \item положение тела при равномерном прямолинейном движении (векторно),
        \item положение тела при равномерном прямолинейном движении (в проекциях).
    \end{enumerate}
}
\solutionspace{120pt}

\tasknumber{2}%
\task{%
    Положив $\vec a = 3\vec i -2 \vec j, \vec b = -4\vec i + 3 \vec j$,
    \begin{enumerate}
        \item найдите сумму векторов $\vec a + \vec b$,
        \item постройте сумму векторов $\vec a + \vec b$ на чертеже,
        \item определите модуль суммы векторов $\modul{\vec a + \vec b}$,
        \item вычислите разность векторов $\vec a - \vec b.$
    \end{enumerate}
}
\answer{%
    $\vec a + \vec b = -1\vec i + 1\vec i, \vec a - \vec b = 7\vec i -5\vec i, \modul{\vec a + \vec b} = \sqrt{\sqr-1 + \sqr1} \approx 1{,}41.$
}
\solutionspace{120pt}

\tasknumber{3}%
\task{%
    Небольшой лёгкий самолёт взлетел из аэропорта, пролетел $7\,\text{км}$ строго на юг, потом повернул и пролетел $24\,\text{км}$ на восток,
    а после по прямой вернулся обратно в аэропорт.
    Определите путь и модуль перемещения самолёта, считая Землю плоской.
}
\solutionspace{120pt}

\tasknumber{4}%
\task{%
    Женя и Валя едут на лошадях: Женя движется на север со скоростью $4\,\frac{\text{км}}{\text{ч}}$, Валя — на восток со скоростью $3\,\frac{\text{км}}{\text{ч}}$.
    Определите скорость Вали относительно Жени.
    Сделайте рисунок («вид сверху»), подпишите кто где, укажите скорости (в т.ч.
    направление).
}

\variantsplitter

\addpersonalvariant{Владислав Емелин}

\tasknumber{1}%
\task{%
    Запишите определения, формулы и физические законы (можно сокращать, но не упустите ключевое):
    \begin{enumerate}
        \item основная задача механики,
        \item траектория,
        \item перемещение при равномерном прямолинейном движении (векторно),
        \item положение тела при равномерном прямолинейном движении (в проекциях).
    \end{enumerate}
}
\solutionspace{120pt}

\tasknumber{2}%
\task{%
    Положив $\vec a = -3\vec i + 4 \vec j, \vec b = 4\vec i -3 \vec j$,
    \begin{enumerate}
        \item найдите сумму векторов $\vec a + \vec b$,
        \item постройте сумму векторов $\vec a + \vec b$ на чертеже,
        \item определите модуль суммы векторов $\modul{\vec a + \vec b}$,
        \item вычислите разность векторов $\vec a - \vec b.$
    \end{enumerate}
}
\answer{%
    $\vec a + \vec b = 1\vec i + 1\vec i, \vec a - \vec b = -7\vec i + 7\vec i, \modul{\vec a + \vec b} = \sqrt{\sqr1 + \sqr1} \approx 1{,}41.$
}
\solutionspace{120pt}

\tasknumber{3}%
\task{%
    Небольшой лёгкий самолёт взлетел из аэропорта, пролетел $40\,\text{км}$ строго на юг, потом повернул и пролетел $30\,\text{км}$ на запад,
    а после по прямой вернулся обратно в аэропорт.
    Определите путь и модуль перемещения самолёта, считая Землю плоской.
}
\solutionspace{120pt}

\tasknumber{4}%
\task{%
    Женя и Валя едут на лошадях: Женя движется на север со скоростью $3\,\frac{\text{м}}{\text{с}}$, Валя — на юг со скоростью $4\,\frac{\text{м}}{\text{с}}$.
    Определите скорость Вали относительно Жени.
    Сделайте рисунок («вид сверху»), подпишите кто где, укажите скорости (в т.ч.
    направление).
}

\variantsplitter

\addpersonalvariant{Артём Жичин}

\tasknumber{1}%
\task{%
    Запишите определения, формулы и физические законы (можно сокращать, но не упустите ключевое):
    \begin{enumerate}
        \item поступательное движение,
        \item перемещение,
        \item положение тела при равномерном прямолинейном движении (векторно),
        \item положение тела при равномерном прямолинейном движении (в проекциях).
    \end{enumerate}
}
\solutionspace{120pt}

\tasknumber{2}%
\task{%
    Положив $\vec a = 3\vec i -4 \vec j, \vec b = 3\vec i + 2 \vec j$,
    \begin{enumerate}
        \item найдите сумму векторов $\vec a + \vec b$,
        \item постройте сумму векторов $\vec a + \vec b$ на чертеже,
        \item определите модуль суммы векторов $\modul{\vec a + \vec b}$,
        \item вычислите разность векторов $\vec a - \vec b.$
    \end{enumerate}
}
\answer{%
    $\vec a + \vec b = 6\vec i -2\vec i, \vec a - \vec b = 0\vec i -6\vec i, \modul{\vec a + \vec b} = \sqrt{\sqr6 + \sqr-2} \approx 6{,}32.$
}
\solutionspace{120pt}

\tasknumber{3}%
\task{%
    Небольшой лёгкий самолёт взлетел из аэропорта, пролетел $24\,\text{км}$ строго на юг, потом повернул и пролетел $7\,\text{км}$ на восток,
    а после по прямой вернулся обратно в аэропорт.
    Определите путь и модуль перемещения самолёта, считая Землю плоской.
}
\solutionspace{120pt}

\tasknumber{4}%
\task{%
    Женя и Валя едут на велосипедах: Женя движется на восток со скоростью $4\,\frac{\text{км}}{\text{ч}}$, Валя — на юг со скоростью $3\,\frac{\text{км}}{\text{ч}}$.
    Определите скорость Вали относительно Жени.
    Сделайте рисунок («вид сверху»), подпишите кто где, укажите скорости (в т.ч.
    направление).
}

\variantsplitter

\addpersonalvariant{Дарья Кошман}

\tasknumber{1}%
\task{%
    Запишите определения, формулы и физические законы (можно сокращать, но не упустите ключевое):
    \begin{enumerate}
        \item основная задача механики,
        \item траектория,
        \item перемещение при равномерном прямолинейном движении (векторно),
        \item перемещение при равномерном прямолинейном движении (в проекциях).
    \end{enumerate}
}
\solutionspace{120pt}

\tasknumber{2}%
\task{%
    Положив $\vec a = -2\vec i -2 \vec j, \vec b = 3\vec i -2 \vec j$,
    \begin{enumerate}
        \item найдите сумму векторов $\vec a + \vec b$,
        \item постройте сумму векторов $\vec a + \vec b$ на чертеже,
        \item определите модуль суммы векторов $\modul{\vec a + \vec b}$,
        \item вычислите разность векторов $\vec a - \vec b.$
    \end{enumerate}
}
\answer{%
    $\vec a + \vec b = 1\vec i -4\vec i, \vec a - \vec b = -5\vec i + 0\vec i, \modul{\vec a + \vec b} = \sqrt{\sqr1 + \sqr-4} \approx 4{,}12.$
}
\solutionspace{120pt}

\tasknumber{3}%
\task{%
    Небольшой лёгкий самолёт взлетел из аэропорта, пролетел $7\,\text{км}$ строго на север, потом повернул и пролетел $24\,\text{км}$ на запад,
    а после по прямой вернулся обратно в аэропорт.
    Определите путь и модуль перемещения самолёта, считая Землю плоской.
}
\solutionspace{120pt}

\tasknumber{4}%
\task{%
    Женя и Валя едут на лошадях: Женя движется на восток со скоростью $3\,\frac{\text{м}}{\text{с}}$, Валя — на восток со скоростью $4\,\frac{\text{м}}{\text{с}}$.
    Определите скорость Вали относительно Жени.
    Сделайте рисунок («вид сверху»), подпишите кто где, укажите скорости (в т.ч.
    направление).
}

\variantsplitter

\addpersonalvariant{Анна Кузьмичёва}

\tasknumber{1}%
\task{%
    Запишите определения, формулы и физические законы (можно сокращать, но не упустите ключевое):
    \begin{enumerate}
        \item система отсчёта,
        \item перемещение,
        \item перемещение при равномерном прямолинейном движении (векторно),
        \item положение тела при равномерном прямолинейном движении (в проекциях).
    \end{enumerate}
}
\solutionspace{120pt}

\tasknumber{2}%
\task{%
    Положив $\vec a = 2\vec i + 2 \vec j, \vec b = 4\vec i + 3 \vec j$,
    \begin{enumerate}
        \item найдите сумму векторов $\vec a + \vec b$,
        \item постройте сумму векторов $\vec a + \vec b$ на чертеже,
        \item определите модуль суммы векторов $\modul{\vec a + \vec b}$,
        \item вычислите разность векторов $\vec a - \vec b.$
    \end{enumerate}
}
\answer{%
    $\vec a + \vec b = 6\vec i + 5\vec i, \vec a - \vec b = -2\vec i -1\vec i, \modul{\vec a + \vec b} = \sqrt{\sqr6 + \sqr5} \approx 7{,}81.$
}
\solutionspace{120pt}

\tasknumber{3}%
\task{%
    Небольшой лёгкий самолёт взлетел из аэропорта, пролетел $30\,\text{км}$ строго на юг, потом повернул и пролетел $40\,\text{км}$ на восток,
    а после по прямой вернулся обратно в аэропорт.
    Определите путь и модуль перемещения самолёта, считая Землю плоской.
}
\solutionspace{120pt}

\tasknumber{4}%
\task{%
    Женя и Валя едут на велосипедах: Женя движется на юг со скоростью $12\,\frac{\text{м}}{\text{с}}$, Валя — на юг со скоростью $5\,\frac{\text{м}}{\text{с}}$.
    Определите скорость Вали относительно Жени.
    Сделайте рисунок («вид сверху»), подпишите кто где, укажите скорости (в т.ч.
    направление).
}

\variantsplitter

\addpersonalvariant{Алёна Куприянова}

\tasknumber{1}%
\task{%
    Запишите определения, формулы и физические законы (можно сокращать, но не упустите ключевое):
    \begin{enumerate}
        \item материальная точка,
        \item траектория,
        \item положение тела при равномерном прямолинейном движении (векторно),
        \item перемещение при равномерном прямолинейном движении (в проекциях).
    \end{enumerate}
}
\solutionspace{120pt}

\tasknumber{2}%
\task{%
    Положив $\vec a = 2\vec i + 4 \vec j, \vec b = 4\vec i + 3 \vec j$,
    \begin{enumerate}
        \item найдите сумму векторов $\vec a + \vec b$,
        \item постройте сумму векторов $\vec a + \vec b$ на чертеже,
        \item определите модуль суммы векторов $\modul{\vec a + \vec b}$,
        \item вычислите разность векторов $\vec a - \vec b.$
    \end{enumerate}
}
\answer{%
    $\vec a + \vec b = 6\vec i + 7\vec i, \vec a - \vec b = -2\vec i + 1\vec i, \modul{\vec a + \vec b} = \sqrt{\sqr6 + \sqr7} \approx 9{,}22.$
}
\solutionspace{120pt}

\tasknumber{3}%
\task{%
    Небольшой лёгкий самолёт взлетел из аэропорта, пролетел $24\,\text{км}$ строго на север, потом повернул и пролетел $7\,\text{км}$ на запад,
    а после по прямой вернулся обратно в аэропорт.
    Определите путь и модуль перемещения самолёта, считая Землю плоской.
}
\solutionspace{120pt}

\tasknumber{4}%
\task{%
    Женя и Валя едут на лошадях: Женя движется на восток со скоростью $4\,\frac{\text{км}}{\text{ч}}$, Валя — на север со скоростью $3\,\frac{\text{км}}{\text{ч}}$.
    Определите скорость Вали относительно Жени.
    Сделайте рисунок («вид сверху»), подпишите кто где, укажите скорости (в т.ч.
    направление).
}

\variantsplitter

\addpersonalvariant{Ярослав Лавровский}

\tasknumber{1}%
\task{%
    Запишите определения, формулы и физические законы (можно сокращать, но не упустите ключевое):
    \begin{enumerate}
        \item механическое движение,
        \item путь,
        \item положение тела при равномерном прямолинейном движении (векторно),
        \item перемещение при равномерном прямолинейном движении (в проекциях).
    \end{enumerate}
}
\solutionspace{120pt}

\tasknumber{2}%
\task{%
    Положив $\vec a = -2\vec i -4 \vec j, \vec b = 3\vec i + 2 \vec j$,
    \begin{enumerate}
        \item найдите сумму векторов $\vec a + \vec b$,
        \item постройте сумму векторов $\vec a + \vec b$ на чертеже,
        \item определите модуль суммы векторов $\modul{\vec a + \vec b}$,
        \item вычислите разность векторов $\vec a - \vec b.$
    \end{enumerate}
}
\answer{%
    $\vec a + \vec b = 1\vec i -2\vec i, \vec a - \vec b = -5\vec i -6\vec i, \modul{\vec a + \vec b} = \sqrt{\sqr1 + \sqr-2} \approx 2{,}24.$
}
\solutionspace{120pt}

\tasknumber{3}%
\task{%
    Небольшой лёгкий самолёт взлетел из аэропорта, пролетел $40\,\text{км}$ строго на юг, потом повернул и пролетел $30\,\text{км}$ на запад,
    а после по прямой вернулся обратно в аэропорт.
    Определите путь и модуль перемещения самолёта, считая Землю плоской.
}
\solutionspace{120pt}

\tasknumber{4}%
\task{%
    Женя и Валя едут на мотоциклах: Женя движется на север со скоростью $3\,\frac{\text{м}}{\text{с}}$, Валя — на юг со скоростью $4\,\frac{\text{м}}{\text{с}}$.
    Определите скорость Вали относительно Жени.
    Сделайте рисунок («вид сверху»), подпишите кто где, укажите скорости (в т.ч.
    направление).
}

\variantsplitter

\addpersonalvariant{Анастасия Ламанова}

\tasknumber{1}%
\task{%
    Запишите определения, формулы и физические законы (можно сокращать, но не упустите ключевое):
    \begin{enumerate}
        \item основная задача механики,
        \item перемещение,
        \item перемещение при равномерном прямолинейном движении (векторно),
        \item положение тела при равномерном прямолинейном движении (в проекциях).
    \end{enumerate}
}
\solutionspace{120pt}

\tasknumber{2}%
\task{%
    Положив $\vec a = -3\vec i -4 \vec j, \vec b = -3\vec i -3 \vec j$,
    \begin{enumerate}
        \item найдите сумму векторов $\vec a + \vec b$,
        \item постройте сумму векторов $\vec a + \vec b$ на чертеже,
        \item определите модуль суммы векторов $\modul{\vec a + \vec b}$,
        \item вычислите разность векторов $\vec a - \vec b.$
    \end{enumerate}
}
\answer{%
    $\vec a + \vec b = -6\vec i -7\vec i, \vec a - \vec b = 0\vec i -1\vec i, \modul{\vec a + \vec b} = \sqrt{\sqr-6 + \sqr-7} \approx 9{,}22.$
}
\solutionspace{120pt}

\tasknumber{3}%
\task{%
    Небольшой лёгкий самолёт взлетел из аэропорта, пролетел $12\,\text{км}$ строго на север, потом повернул и пролетел $5\,\text{км}$ на восток,
    а после по прямой вернулся обратно в аэропорт.
    Определите путь и модуль перемещения самолёта, считая Землю плоской.
}
\solutionspace{120pt}

\tasknumber{4}%
\task{%
    Женя и Валя едут на мотоциклах: Женя движется на запад со скоростью $4\,\frac{\text{км}}{\text{ч}}$, Валя — на юг со скоростью $3\,\frac{\text{км}}{\text{ч}}$.
    Определите скорость Вали относительно Жени.
    Сделайте рисунок («вид сверху»), подпишите кто где, укажите скорости (в т.ч.
    направление).
}

\variantsplitter

\addpersonalvariant{Виктория Легонькова}

\tasknumber{1}%
\task{%
    Запишите определения, формулы и физические законы (можно сокращать, но не упустите ключевое):
    \begin{enumerate}
        \item поступательное движение,
        \item путь,
        \item перемещение при равномерном прямолинейном движении (векторно),
        \item перемещение при равномерном прямолинейном движении (в проекциях).
    \end{enumerate}
}
\solutionspace{120pt}

\tasknumber{2}%
\task{%
    Положив $\vec a = 3\vec i + 2 \vec j, \vec b = -3\vec i -3 \vec j$,
    \begin{enumerate}
        \item найдите сумму векторов $\vec a + \vec b$,
        \item постройте сумму векторов $\vec a + \vec b$ на чертеже,
        \item определите модуль суммы векторов $\modul{\vec a + \vec b}$,
        \item вычислите разность векторов $\vec a - \vec b.$
    \end{enumerate}
}
\answer{%
    $\vec a + \vec b = 0\vec i -1\vec i, \vec a - \vec b = 6\vec i + 5\vec i, \modul{\vec a + \vec b} = \sqrt{\sqr0 + \sqr-1} \approx 1{,}00.$
}
\solutionspace{120pt}

\tasknumber{3}%
\task{%
    Небольшой лёгкий самолёт взлетел из аэропорта, пролетел $24\,\text{км}$ строго на юг, потом повернул и пролетел $7\,\text{км}$ на восток,
    а после по прямой вернулся обратно в аэропорт.
    Определите путь и модуль перемещения самолёта, считая Землю плоской.
}
\solutionspace{120pt}

\tasknumber{4}%
\task{%
    Женя и Валя едут на лошадях: Женя движется на восток со скоростью $3\,\frac{\text{м}}{\text{с}}$, Валя — на север со скоростью $4\,\frac{\text{м}}{\text{с}}$.
    Определите скорость Вали относительно Жени.
    Сделайте рисунок («вид сверху»), подпишите кто где, укажите скорости (в т.ч.
    направление).
}

\variantsplitter

\addpersonalvariant{Семён Мартынов}

\tasknumber{1}%
\task{%
    Запишите определения, формулы и физические законы (можно сокращать, но не упустите ключевое):
    \begin{enumerate}
        \item основная задача механики,
        \item перемещение,
        \item перемещение при равномерном прямолинейном движении (векторно),
        \item положение тела при равномерном прямолинейном движении (в проекциях).
    \end{enumerate}
}
\solutionspace{120pt}

\tasknumber{2}%
\task{%
    Положив $\vec a = 3\vec i + 4 \vec j, \vec b = 3\vec i -2 \vec j$,
    \begin{enumerate}
        \item найдите сумму векторов $\vec a + \vec b$,
        \item постройте сумму векторов $\vec a + \vec b$ на чертеже,
        \item определите модуль суммы векторов $\modul{\vec a + \vec b}$,
        \item вычислите разность векторов $\vec a - \vec b.$
    \end{enumerate}
}
\answer{%
    $\vec a + \vec b = 6\vec i + 2\vec i, \vec a - \vec b = 0\vec i + 6\vec i, \modul{\vec a + \vec b} = \sqrt{\sqr6 + \sqr2} \approx 6{,}32.$
}
\solutionspace{120pt}

\tasknumber{3}%
\task{%
    Небольшой лёгкий самолёт взлетел из аэропорта, пролетел $5\,\text{км}$ строго на юг, потом повернул и пролетел $12\,\text{км}$ на запад,
    а после по прямой вернулся обратно в аэропорт.
    Определите путь и модуль перемещения самолёта, считая Землю плоской.
}
\solutionspace{120pt}

\tasknumber{4}%
\task{%
    Женя и Валя едут на лошадях: Женя движется на запад со скоростью $12\,\frac{\text{м}}{\text{с}}$, Валя — на юг со скоростью $5\,\frac{\text{м}}{\text{с}}$.
    Определите скорость Вали относительно Жени.
    Сделайте рисунок («вид сверху»), подпишите кто где, укажите скорости (в т.ч.
    направление).
}

\variantsplitter

\addpersonalvariant{Варвара Минаева}

\tasknumber{1}%
\task{%
    Запишите определения, формулы и физические законы (можно сокращать, но не упустите ключевое):
    \begin{enumerate}
        \item материальная точка,
        \item перемещение,
        \item положение тела при равномерном прямолинейном движении (векторно),
        \item положение тела при равномерном прямолинейном движении (в проекциях).
    \end{enumerate}
}
\solutionspace{120pt}

\tasknumber{2}%
\task{%
    Положив $\vec a = -2\vec i -2 \vec j, \vec b = -3\vec i + 3 \vec j$,
    \begin{enumerate}
        \item найдите сумму векторов $\vec a + \vec b$,
        \item постройте сумму векторов $\vec a + \vec b$ на чертеже,
        \item определите модуль суммы векторов $\modul{\vec a + \vec b}$,
        \item вычислите разность векторов $\vec a - \vec b.$
    \end{enumerate}
}
\answer{%
    $\vec a + \vec b = -5\vec i + 1\vec i, \vec a - \vec b = 1\vec i -5\vec i, \modul{\vec a + \vec b} = \sqrt{\sqr-5 + \sqr1} \approx 5{,}10.$
}
\solutionspace{120pt}

\tasknumber{3}%
\task{%
    Небольшой лёгкий самолёт взлетел из аэропорта, пролетел $5\,\text{км}$ строго на север, потом повернул и пролетел $12\,\text{км}$ на запад,
    а после по прямой вернулся обратно в аэропорт.
    Определите путь и модуль перемещения самолёта, считая Землю плоской.
}
\solutionspace{120pt}

\tasknumber{4}%
\task{%
    Женя и Валя едут на мотоциклах: Женя движется на запад со скоростью $5\,\frac{\text{км}}{\text{ч}}$, Валя — на восток со скоростью $12\,\frac{\text{км}}{\text{ч}}$.
    Определите скорость Вали относительно Жени.
    Сделайте рисунок («вид сверху»), подпишите кто где, укажите скорости (в т.ч.
    направление).
}

\variantsplitter

\addpersonalvariant{Леонид Никитин}

\tasknumber{1}%
\task{%
    Запишите определения, формулы и физические законы (можно сокращать, но не упустите ключевое):
    \begin{enumerate}
        \item основная задача механики,
        \item путь,
        \item положение тела при равномерном прямолинейном движении (векторно),
        \item перемещение при равномерном прямолинейном движении (в проекциях).
    \end{enumerate}
}
\solutionspace{120pt}

\tasknumber{2}%
\task{%
    Положив $\vec a = 3\vec i + 2 \vec j, \vec b = -3\vec i -3 \vec j$,
    \begin{enumerate}
        \item найдите сумму векторов $\vec a + \vec b$,
        \item постройте сумму векторов $\vec a + \vec b$ на чертеже,
        \item определите модуль суммы векторов $\modul{\vec a + \vec b}$,
        \item вычислите разность векторов $\vec a - \vec b.$
    \end{enumerate}
}
\answer{%
    $\vec a + \vec b = 0\vec i -1\vec i, \vec a - \vec b = 6\vec i + 5\vec i, \modul{\vec a + \vec b} = \sqrt{\sqr0 + \sqr-1} \approx 1{,}00.$
}
\solutionspace{120pt}

\tasknumber{3}%
\task{%
    Небольшой лёгкий самолёт взлетел из аэропорта, пролетел $24\,\text{км}$ строго на юг, потом повернул и пролетел $7\,\text{км}$ на восток,
    а после по прямой вернулся обратно в аэропорт.
    Определите путь и модуль перемещения самолёта, считая Землю плоской.
}
\solutionspace{120pt}

\tasknumber{4}%
\task{%
    Женя и Валя едут на мотоциклах: Женя движется на север со скоростью $12\,\frac{\text{м}}{\text{с}}$, Валя — на запад со скоростью $5\,\frac{\text{м}}{\text{с}}$.
    Определите скорость Вали относительно Жени.
    Сделайте рисунок («вид сверху»), подпишите кто где, укажите скорости (в т.ч.
    направление).
}

\variantsplitter

\addpersonalvariant{Тимофей Полетаев}

\tasknumber{1}%
\task{%
    Запишите определения, формулы и физические законы (можно сокращать, но не упустите ключевое):
    \begin{enumerate}
        \item основная задача механики,
        \item путь,
        \item перемещение при равномерном прямолинейном движении (векторно),
        \item положение тела при равномерном прямолинейном движении (в проекциях).
    \end{enumerate}
}
\solutionspace{120pt}

\tasknumber{2}%
\task{%
    Положив $\vec a = 2\vec i + 2 \vec j, \vec b = -3\vec i + 3 \vec j$,
    \begin{enumerate}
        \item найдите сумму векторов $\vec a + \vec b$,
        \item постройте сумму векторов $\vec a + \vec b$ на чертеже,
        \item определите модуль суммы векторов $\modul{\vec a + \vec b}$,
        \item вычислите разность векторов $\vec a - \vec b.$
    \end{enumerate}
}
\answer{%
    $\vec a + \vec b = -1\vec i + 5\vec i, \vec a - \vec b = 5\vec i -1\vec i, \modul{\vec a + \vec b} = \sqrt{\sqr-1 + \sqr5} \approx 5{,}10.$
}
\solutionspace{120pt}

\tasknumber{3}%
\task{%
    Небольшой лёгкий самолёт взлетел из аэропорта, пролетел $5\,\text{км}$ строго на север, потом повернул и пролетел $12\,\text{км}$ на запад,
    а после по прямой вернулся обратно в аэропорт.
    Определите путь и модуль перемещения самолёта, считая Землю плоской.
}
\solutionspace{120pt}

\tasknumber{4}%
\task{%
    Женя и Валя едут на мотоциклах: Женя движется на юг со скоростью $5\,\frac{\text{км}}{\text{ч}}$, Валя — на север со скоростью $12\,\frac{\text{км}}{\text{ч}}$.
    Определите скорость Вали относительно Жени.
    Сделайте рисунок («вид сверху»), подпишите кто где, укажите скорости (в т.ч.
    направление).
}

\variantsplitter

\addpersonalvariant{Андрей Рожков}

\tasknumber{1}%
\task{%
    Запишите определения, формулы и физические законы (можно сокращать, но не упустите ключевое):
    \begin{enumerate}
        \item система отсчёта,
        \item траектория,
        \item перемещение при равномерном прямолинейном движении (векторно),
        \item перемещение при равномерном прямолинейном движении (в проекциях).
    \end{enumerate}
}
\solutionspace{120pt}

\tasknumber{2}%
\task{%
    Положив $\vec a = -3\vec i -2 \vec j, \vec b = 3\vec i -2 \vec j$,
    \begin{enumerate}
        \item найдите сумму векторов $\vec a + \vec b$,
        \item постройте сумму векторов $\vec a + \vec b$ на чертеже,
        \item определите модуль суммы векторов $\modul{\vec a + \vec b}$,
        \item вычислите разность векторов $\vec a - \vec b.$
    \end{enumerate}
}
\answer{%
    $\vec a + \vec b = 0\vec i -4\vec i, \vec a - \vec b = -6\vec i + 0\vec i, \modul{\vec a + \vec b} = \sqrt{\sqr0 + \sqr-4} \approx 4{,}00.$
}
\solutionspace{120pt}

\tasknumber{3}%
\task{%
    Небольшой лёгкий самолёт взлетел из аэропорта, пролетел $7\,\text{км}$ строго на юг, потом повернул и пролетел $24\,\text{км}$ на запад,
    а после по прямой вернулся обратно в аэропорт.
    Определите путь и модуль перемещения самолёта, считая Землю плоской.
}
\solutionspace{120pt}

\tasknumber{4}%
\task{%
    Женя и Валя едут на велосипедах: Женя движется на юг со скоростью $3\,\frac{\text{м}}{\text{с}}$, Валя — на юг со скоростью $4\,\frac{\text{м}}{\text{с}}$.
    Определите скорость Вали относительно Жени.
    Сделайте рисунок («вид сверху»), подпишите кто где, укажите скорости (в т.ч.
    направление).
}

\variantsplitter

\addpersonalvariant{Рената Таржиманова}

\tasknumber{1}%
\task{%
    Запишите определения, формулы и физические законы (можно сокращать, но не упустите ключевое):
    \begin{enumerate}
        \item механическое движение,
        \item путь,
        \item перемещение при равномерном прямолинейном движении (векторно),
        \item перемещение при равномерном прямолинейном движении (в проекциях).
    \end{enumerate}
}
\solutionspace{120pt}

\tasknumber{2}%
\task{%
    Положив $\vec a = 2\vec i + 4 \vec j, \vec b = -3\vec i -2 \vec j$,
    \begin{enumerate}
        \item найдите сумму векторов $\vec a + \vec b$,
        \item постройте сумму векторов $\vec a + \vec b$ на чертеже,
        \item определите модуль суммы векторов $\modul{\vec a + \vec b}$,
        \item вычислите разность векторов $\vec a - \vec b.$
    \end{enumerate}
}
\answer{%
    $\vec a + \vec b = -1\vec i + 2\vec i, \vec a - \vec b = 5\vec i + 6\vec i, \modul{\vec a + \vec b} = \sqrt{\sqr-1 + \sqr2} \approx 2{,}24.$
}
\solutionspace{120pt}

\tasknumber{3}%
\task{%
    Небольшой лёгкий самолёт взлетел из аэропорта, пролетел $30\,\text{км}$ строго на юг, потом повернул и пролетел $40\,\text{км}$ на запад,
    а после по прямой вернулся обратно в аэропорт.
    Определите путь и модуль перемещения самолёта, считая Землю плоской.
}
\solutionspace{120pt}

\tasknumber{4}%
\task{%
    Женя и Валя едут на мотоциклах: Женя движется на запад со скоростью $4\,\frac{\text{км}}{\text{ч}}$, Валя — на запад со скоростью $3\,\frac{\text{км}}{\text{ч}}$.
    Определите скорость Вали относительно Жени.
    Сделайте рисунок («вид сверху»), подпишите кто где, укажите скорости (в т.ч.
    направление).
}

\variantsplitter

\addpersonalvariant{Арсений Трофимов}

\tasknumber{1}%
\task{%
    Запишите определения, формулы и физические законы (можно сокращать, но не упустите ключевое):
    \begin{enumerate}
        \item система отсчёта,
        \item траектория,
        \item перемещение при равномерном прямолинейном движении (векторно),
        \item перемещение при равномерном прямолинейном движении (в проекциях).
    \end{enumerate}
}
\solutionspace{120pt}

\tasknumber{2}%
\task{%
    Положив $\vec a = -2\vec i -2 \vec j, \vec b = 3\vec i -3 \vec j$,
    \begin{enumerate}
        \item найдите сумму векторов $\vec a + \vec b$,
        \item постройте сумму векторов $\vec a + \vec b$ на чертеже,
        \item определите модуль суммы векторов $\modul{\vec a + \vec b}$,
        \item вычислите разность векторов $\vec a - \vec b.$
    \end{enumerate}
}
\answer{%
    $\vec a + \vec b = 1\vec i -5\vec i, \vec a - \vec b = -5\vec i + 1\vec i, \modul{\vec a + \vec b} = \sqrt{\sqr1 + \sqr-5} \approx 5{,}10.$
}
\solutionspace{120pt}

\tasknumber{3}%
\task{%
    Небольшой лёгкий самолёт взлетел из аэропорта, пролетел $30\,\text{км}$ строго на юг, потом повернул и пролетел $40\,\text{км}$ на восток,
    а после по прямой вернулся обратно в аэропорт.
    Определите путь и модуль перемещения самолёта, считая Землю плоской.
}
\solutionspace{120pt}

\tasknumber{4}%
\task{%
    Женя и Валя едут на мотоциклах: Женя движется на юг со скоростью $12\,\frac{\text{м}}{\text{с}}$, Валя — на восток со скоростью $5\,\frac{\text{м}}{\text{с}}$.
    Определите скорость Вали относительно Жени.
    Сделайте рисунок («вид сверху»), подпишите кто где, укажите скорости (в т.ч.
    направление).
}

\variantsplitter

\addpersonalvariant{Андрей Щербаков}

\tasknumber{1}%
\task{%
    Запишите определения, формулы и физические законы (можно сокращать, но не упустите ключевое):
    \begin{enumerate}
        \item материальная точка,
        \item путь,
        \item положение тела при равномерном прямолинейном движении (векторно),
        \item перемещение при равномерном прямолинейном движении (в проекциях).
    \end{enumerate}
}
\solutionspace{120pt}

\tasknumber{2}%
\task{%
    Положив $\vec a = -3\vec i + 4 \vec j, \vec b = -3\vec i + 3 \vec j$,
    \begin{enumerate}
        \item найдите сумму векторов $\vec a + \vec b$,
        \item постройте сумму векторов $\vec a + \vec b$ на чертеже,
        \item определите модуль суммы векторов $\modul{\vec a + \vec b}$,
        \item вычислите разность векторов $\vec a - \vec b.$
    \end{enumerate}
}
\answer{%
    $\vec a + \vec b = -6\vec i + 7\vec i, \vec a - \vec b = 0\vec i + 1\vec i, \modul{\vec a + \vec b} = \sqrt{\sqr-6 + \sqr7} \approx 9{,}22.$
}
\solutionspace{120pt}

\tasknumber{3}%
\task{%
    Небольшой лёгкий самолёт взлетел из аэропорта, пролетел $5\,\text{км}$ строго на север, потом повернул и пролетел $12\,\text{км}$ на запад,
    а после по прямой вернулся обратно в аэропорт.
    Определите путь и модуль перемещения самолёта, считая Землю плоской.
}
\solutionspace{120pt}

\tasknumber{4}%
\task{%
    Женя и Валя едут на мотоциклах: Женя движется на юг со скоростью $12\,\frac{\text{м}}{\text{с}}$, Валя — на восток со скоростью $5\,\frac{\text{м}}{\text{с}}$.
    Определите скорость Вали относительно Жени.
    Сделайте рисунок («вид сверху»), подпишите кто где, укажите скорости (в т.ч.
    направление).
}

\variantsplitter

\addpersonalvariant{Михаил Ярошевский}

\tasknumber{1}%
\task{%
    Запишите определения, формулы и физические законы (можно сокращать, но не упустите ключевое):
    \begin{enumerate}
        \item механическое движение,
        \item путь,
        \item положение тела при равномерном прямолинейном движении (векторно),
        \item перемещение при равномерном прямолинейном движении (в проекциях).
    \end{enumerate}
}
\solutionspace{120pt}

\tasknumber{2}%
\task{%
    Положив $\vec a = 2\vec i -4 \vec j, \vec b = -4\vec i + 2 \vec j$,
    \begin{enumerate}
        \item найдите сумму векторов $\vec a + \vec b$,
        \item постройте сумму векторов $\vec a + \vec b$ на чертеже,
        \item определите модуль суммы векторов $\modul{\vec a + \vec b}$,
        \item вычислите разность векторов $\vec a - \vec b.$
    \end{enumerate}
}
\answer{%
    $\vec a + \vec b = -2\vec i -2\vec i, \vec a - \vec b = 6\vec i -6\vec i, \modul{\vec a + \vec b} = \sqrt{\sqr-2 + \sqr-2} \approx 2{,}83.$
}
\solutionspace{120pt}

\tasknumber{3}%
\task{%
    Небольшой лёгкий самолёт взлетел из аэропорта, пролетел $12\,\text{км}$ строго на юг, потом повернул и пролетел $5\,\text{км}$ на запад,
    а после по прямой вернулся обратно в аэропорт.
    Определите путь и модуль перемещения самолёта, считая Землю плоской.
}
\solutionspace{120pt}

\tasknumber{4}%
\task{%
    Женя и Валя едут на мотоциклах: Женя движется на запад со скоростью $4\,\frac{\text{км}}{\text{ч}}$, Валя — на восток со скоростью $3\,\frac{\text{км}}{\text{ч}}$.
    Определите скорость Вали относительно Жени.
    Сделайте рисунок («вид сверху»), подпишите кто где, укажите скорости (в т.ч.
    направление).
}

\variantsplitter

\addpersonalvariant{Алексей Алимпиев}

\tasknumber{1}%
\task{%
    Запишите определения, формулы и физические законы (можно сокращать, но не упустите ключевое):
    \begin{enumerate}
        \item система отсчёта,
        \item перемещение,
        \item перемещение при равномерном прямолинейном движении (векторно),
        \item положение тела при равномерном прямолинейном движении (в проекциях).
    \end{enumerate}
}
\solutionspace{120pt}

\tasknumber{2}%
\task{%
    Положив $\vec a = -2\vec i -4 \vec j, \vec b = 4\vec i + 3 \vec j$,
    \begin{enumerate}
        \item найдите сумму векторов $\vec a + \vec b$,
        \item постройте сумму векторов $\vec a + \vec b$ на чертеже,
        \item определите модуль суммы векторов $\modul{\vec a + \vec b}$,
        \item вычислите разность векторов $\vec a - \vec b.$
    \end{enumerate}
}
\answer{%
    $\vec a + \vec b = 2\vec i -1\vec i, \vec a - \vec b = -6\vec i -7\vec i, \modul{\vec a + \vec b} = \sqrt{\sqr2 + \sqr-1} \approx 2{,}24.$
}
\solutionspace{120pt}

\tasknumber{3}%
\task{%
    Небольшой лёгкий самолёт взлетел из аэропорта, пролетел $12\,\text{км}$ строго на север, потом повернул и пролетел $5\,\text{км}$ на запад,
    а после по прямой вернулся обратно в аэропорт.
    Определите путь и модуль перемещения самолёта, считая Землю плоской.
}
\solutionspace{120pt}

\tasknumber{4}%
\task{%
    Женя и Валя едут на мотоциклах: Женя движется на север со скоростью $12\,\frac{\text{м}}{\text{с}}$, Валя — на север со скоростью $5\,\frac{\text{м}}{\text{с}}$.
    Определите скорость Вали относительно Жени.
    Сделайте рисунок («вид сверху»), подпишите кто где, укажите скорости (в т.ч.
    направление).
}

\variantsplitter

\addpersonalvariant{Евгений Васин}

\tasknumber{1}%
\task{%
    Запишите определения, формулы и физические законы (можно сокращать, но не упустите ключевое):
    \begin{enumerate}
        \item механическое движение,
        \item равномерное прямолинейное движение,
        \item перемещение при равномерном прямолинейном движении (векторно),
        \item положение тела при равномерном прямолинейном движении (в проекциях).
    \end{enumerate}
}
\solutionspace{120pt}

\tasknumber{2}%
\task{%
    Положив $\vec a = 2\vec i + 2 \vec j, \vec b = 3\vec i -3 \vec j$,
    \begin{enumerate}
        \item найдите сумму векторов $\vec a + \vec b$,
        \item постройте сумму векторов $\vec a + \vec b$ на чертеже,
        \item определите модуль суммы векторов $\modul{\vec a + \vec b}$,
        \item вычислите разность векторов $\vec a - \vec b.$
    \end{enumerate}
}
\answer{%
    $\vec a + \vec b = 5\vec i -1\vec i, \vec a - \vec b = -1\vec i + 5\vec i, \modul{\vec a + \vec b} = \sqrt{\sqr5 + \sqr-1} \approx 5{,}10.$
}
\solutionspace{120pt}

\tasknumber{3}%
\task{%
    Небольшой лёгкий самолёт взлетел из аэропорта, пролетел $24\,\text{км}$ строго на север, потом повернул и пролетел $7\,\text{км}$ на запад,
    а после по прямой вернулся обратно в аэропорт.
    Определите путь и модуль перемещения самолёта, считая Землю плоской.
}
\solutionspace{120pt}

\tasknumber{4}%
\task{%
    Женя и Валя едут на велосипедах: Женя движется на юг со скоростью $5\,\frac{\text{км}}{\text{ч}}$, Валя — на юг со скоростью $12\,\frac{\text{км}}{\text{ч}}$.
    Определите скорость Вали относительно Жени.
    Сделайте рисунок («вид сверху»), подпишите кто где, укажите скорости (в т.ч.
    направление).
}

\variantsplitter

\addpersonalvariant{Волохов Вячеслав}

\tasknumber{1}%
\task{%
    Запишите определения, формулы и физические законы (можно сокращать, но не упустите ключевое):
    \begin{enumerate}
        \item система отсчёта,
        \item путь,
        \item перемещение при равномерном прямолинейном движении (векторно),
        \item перемещение при равномерном прямолинейном движении (в проекциях).
    \end{enumerate}
}
\solutionspace{120pt}

\tasknumber{2}%
\task{%
    Положив $\vec a = 2\vec i + 4 \vec j, \vec b = -4\vec i -2 \vec j$,
    \begin{enumerate}
        \item найдите сумму векторов $\vec a + \vec b$,
        \item постройте сумму векторов $\vec a + \vec b$ на чертеже,
        \item определите модуль суммы векторов $\modul{\vec a + \vec b}$,
        \item вычислите разность векторов $\vec a - \vec b.$
    \end{enumerate}
}
\answer{%
    $\vec a + \vec b = -2\vec i + 2\vec i, \vec a - \vec b = 6\vec i + 6\vec i, \modul{\vec a + \vec b} = \sqrt{\sqr-2 + \sqr2} \approx 2{,}83.$
}
\solutionspace{120pt}

\tasknumber{3}%
\task{%
    Небольшой лёгкий самолёт взлетел из аэропорта, пролетел $24\,\text{км}$ строго на юг, потом повернул и пролетел $7\,\text{км}$ на восток,
    а после по прямой вернулся обратно в аэропорт.
    Определите путь и модуль перемещения самолёта, считая Землю плоской.
}
\solutionspace{120pt}

\tasknumber{4}%
\task{%
    Женя и Валя едут на велосипедах: Женя движется на восток со скоростью $3\,\frac{\text{м}}{\text{с}}$, Валя — на юг со скоростью $4\,\frac{\text{м}}{\text{с}}$.
    Определите скорость Вали относительно Жени.
    Сделайте рисунок («вид сверху»), подпишите кто где, укажите скорости (в т.ч.
    направление).
}

\variantsplitter

\addpersonalvariant{Герман Говоров}

\tasknumber{1}%
\task{%
    Запишите определения, формулы и физические законы (можно сокращать, но не упустите ключевое):
    \begin{enumerate}
        \item механическое движение,
        \item равномерное прямолинейное движение,
        \item перемещение при равномерном прямолинейном движении (векторно),
        \item перемещение при равномерном прямолинейном движении (в проекциях).
    \end{enumerate}
}
\solutionspace{120pt}

\tasknumber{2}%
\task{%
    Положив $\vec a = 3\vec i + 2 \vec j, \vec b = 3\vec i + 2 \vec j$,
    \begin{enumerate}
        \item найдите сумму векторов $\vec a + \vec b$,
        \item постройте сумму векторов $\vec a + \vec b$ на чертеже,
        \item определите модуль суммы векторов $\modul{\vec a + \vec b}$,
        \item вычислите разность векторов $\vec a - \vec b.$
    \end{enumerate}
}
\answer{%
    $\vec a + \vec b = 6\vec i + 4\vec i, \vec a - \vec b = 0\vec i + 0\vec i, \modul{\vec a + \vec b} = \sqrt{\sqr6 + \sqr4} \approx 7{,}21.$
}
\solutionspace{120pt}

\tasknumber{3}%
\task{%
    Небольшой лёгкий самолёт взлетел из аэропорта, пролетел $5\,\text{км}$ строго на юг, потом повернул и пролетел $12\,\text{км}$ на восток,
    а после по прямой вернулся обратно в аэропорт.
    Определите путь и модуль перемещения самолёта, считая Землю плоской.
}
\solutionspace{120pt}

\tasknumber{4}%
\task{%
    Женя и Валя едут на велосипедах: Женя движется на восток со скоростью $5\,\frac{\text{км}}{\text{ч}}$, Валя — на север со скоростью $12\,\frac{\text{км}}{\text{ч}}$.
    Определите скорость Вали относительно Жени.
    Сделайте рисунок («вид сверху»), подпишите кто где, укажите скорости (в т.ч.
    направление).
}

\variantsplitter

\addpersonalvariant{София Журавлева}

\tasknumber{1}%
\task{%
    Запишите определения, формулы и физические законы (можно сокращать, но не упустите ключевое):
    \begin{enumerate}
        \item система отсчёта,
        \item путь,
        \item положение тела при равномерном прямолинейном движении (векторно),
        \item положение тела при равномерном прямолинейном движении (в проекциях).
    \end{enumerate}
}
\solutionspace{120pt}

\tasknumber{2}%
\task{%
    Положив $\vec a = -2\vec i -2 \vec j, \vec b = -4\vec i -3 \vec j$,
    \begin{enumerate}
        \item найдите сумму векторов $\vec a + \vec b$,
        \item постройте сумму векторов $\vec a + \vec b$ на чертеже,
        \item определите модуль суммы векторов $\modul{\vec a + \vec b}$,
        \item вычислите разность векторов $\vec a - \vec b.$
    \end{enumerate}
}
\answer{%
    $\vec a + \vec b = -6\vec i -5\vec i, \vec a - \vec b = 2\vec i + 1\vec i, \modul{\vec a + \vec b} = \sqrt{\sqr-6 + \sqr-5} \approx 7{,}81.$
}
\solutionspace{120pt}

\tasknumber{3}%
\task{%
    Небольшой лёгкий самолёт взлетел из аэропорта, пролетел $5\,\text{км}$ строго на юг, потом повернул и пролетел $12\,\text{км}$ на восток,
    а после по прямой вернулся обратно в аэропорт.
    Определите путь и модуль перемещения самолёта, считая Землю плоской.
}
\solutionspace{120pt}

\tasknumber{4}%
\task{%
    Женя и Валя едут на велосипедах: Женя движется на восток со скоростью $12\,\frac{\text{м}}{\text{с}}$, Валя — на север со скоростью $5\,\frac{\text{м}}{\text{с}}$.
    Определите скорость Вали относительно Жени.
    Сделайте рисунок («вид сверху»), подпишите кто где, укажите скорости (в т.ч.
    направление).
}

\variantsplitter

\addpersonalvariant{Константин Козлов}

\tasknumber{1}%
\task{%
    Запишите определения, формулы и физические законы (можно сокращать, но не упустите ключевое):
    \begin{enumerate}
        \item материальная точка,
        \item перемещение,
        \item положение тела при равномерном прямолинейном движении (векторно),
        \item перемещение при равномерном прямолинейном движении (в проекциях).
    \end{enumerate}
}
\solutionspace{120pt}

\tasknumber{2}%
\task{%
    Положив $\vec a = 3\vec i -2 \vec j, \vec b = 4\vec i + 2 \vec j$,
    \begin{enumerate}
        \item найдите сумму векторов $\vec a + \vec b$,
        \item постройте сумму векторов $\vec a + \vec b$ на чертеже,
        \item определите модуль суммы векторов $\modul{\vec a + \vec b}$,
        \item вычислите разность векторов $\vec a - \vec b.$
    \end{enumerate}
}
\answer{%
    $\vec a + \vec b = 7\vec i + 0\vec i, \vec a - \vec b = -1\vec i -4\vec i, \modul{\vec a + \vec b} = \sqrt{\sqr7 + \sqr0} \approx 7{,}00.$
}
\solutionspace{120pt}

\tasknumber{3}%
\task{%
    Небольшой лёгкий самолёт взлетел из аэропорта, пролетел $7\,\text{км}$ строго на юг, потом повернул и пролетел $24\,\text{км}$ на восток,
    а после по прямой вернулся обратно в аэропорт.
    Определите путь и модуль перемещения самолёта, считая Землю плоской.
}
\solutionspace{120pt}

\tasknumber{4}%
\task{%
    Женя и Валя едут на велосипедах: Женя движется на юг со скоростью $4\,\frac{\text{км}}{\text{ч}}$, Валя — на запад со скоростью $3\,\frac{\text{км}}{\text{ч}}$.
    Определите скорость Вали относительно Жени.
    Сделайте рисунок («вид сверху»), подпишите кто где, укажите скорости (в т.ч.
    направление).
}

\variantsplitter

\addpersonalvariant{Наталья Кравченко}

\tasknumber{1}%
\task{%
    Запишите определения, формулы и физические законы (можно сокращать, но не упустите ключевое):
    \begin{enumerate}
        \item система отсчёта,
        \item путь,
        \item положение тела при равномерном прямолинейном движении (векторно),
        \item перемещение при равномерном прямолинейном движении (в проекциях).
    \end{enumerate}
}
\solutionspace{120pt}

\tasknumber{2}%
\task{%
    Положив $\vec a = 3\vec i -4 \vec j, \vec b = -3\vec i -2 \vec j$,
    \begin{enumerate}
        \item найдите сумму векторов $\vec a + \vec b$,
        \item постройте сумму векторов $\vec a + \vec b$ на чертеже,
        \item определите модуль суммы векторов $\modul{\vec a + \vec b}$,
        \item вычислите разность векторов $\vec a - \vec b.$
    \end{enumerate}
}
\answer{%
    $\vec a + \vec b = 0\vec i -6\vec i, \vec a - \vec b = 6\vec i -2\vec i, \modul{\vec a + \vec b} = \sqrt{\sqr0 + \sqr-6} \approx 6{,}00.$
}
\solutionspace{120pt}

\tasknumber{3}%
\task{%
    Небольшой лёгкий самолёт взлетел из аэропорта, пролетел $5\,\text{км}$ строго на юг, потом повернул и пролетел $12\,\text{км}$ на запад,
    а после по прямой вернулся обратно в аэропорт.
    Определите путь и модуль перемещения самолёта, считая Землю плоской.
}
\solutionspace{120pt}

\tasknumber{4}%
\task{%
    Женя и Валя едут на велосипедах: Женя движется на запад со скоростью $3\,\frac{\text{м}}{\text{с}}$, Валя — на запад со скоростью $4\,\frac{\text{м}}{\text{с}}$.
    Определите скорость Вали относительно Жени.
    Сделайте рисунок («вид сверху»), подпишите кто где, укажите скорости (в т.ч.
    направление).
}

\variantsplitter

\addpersonalvariant{Матвей Кузьмин}

\tasknumber{1}%
\task{%
    Запишите определения, формулы и физические законы (можно сокращать, но не упустите ключевое):
    \begin{enumerate}
        \item материальная точка,
        \item перемещение,
        \item положение тела при равномерном прямолинейном движении (векторно),
        \item перемещение при равномерном прямолинейном движении (в проекциях).
    \end{enumerate}
}
\solutionspace{120pt}

\tasknumber{2}%
\task{%
    Положив $\vec a = 2\vec i + 4 \vec j, \vec b = 3\vec i + 2 \vec j$,
    \begin{enumerate}
        \item найдите сумму векторов $\vec a + \vec b$,
        \item постройте сумму векторов $\vec a + \vec b$ на чертеже,
        \item определите модуль суммы векторов $\modul{\vec a + \vec b}$,
        \item вычислите разность векторов $\vec a - \vec b.$
    \end{enumerate}
}
\answer{%
    $\vec a + \vec b = 5\vec i + 6\vec i, \vec a - \vec b = -1\vec i + 2\vec i, \modul{\vec a + \vec b} = \sqrt{\sqr5 + \sqr6} \approx 7{,}81.$
}
\solutionspace{120pt}

\tasknumber{3}%
\task{%
    Небольшой лёгкий самолёт взлетел из аэропорта, пролетел $40\,\text{км}$ строго на север, потом повернул и пролетел $30\,\text{км}$ на восток,
    а после по прямой вернулся обратно в аэропорт.
    Определите путь и модуль перемещения самолёта, считая Землю плоской.
}
\solutionspace{120pt}

\tasknumber{4}%
\task{%
    Женя и Валя едут на мотоциклах: Женя движется на восток со скоростью $5\,\frac{\text{км}}{\text{ч}}$, Валя — на юг со скоростью $12\,\frac{\text{км}}{\text{ч}}$.
    Определите скорость Вали относительно Жени.
    Сделайте рисунок («вид сверху»), подпишите кто где, укажите скорости (в т.ч.
    направление).
}

\variantsplitter

\addpersonalvariant{Сергей Малышев}

\tasknumber{1}%
\task{%
    Запишите определения, формулы и физические законы (можно сокращать, но не упустите ключевое):
    \begin{enumerate}
        \item материальная точка,
        \item путь,
        \item перемещение при равномерном прямолинейном движении (векторно),
        \item перемещение при равномерном прямолинейном движении (в проекциях).
    \end{enumerate}
}
\solutionspace{120pt}

\tasknumber{2}%
\task{%
    Положив $\vec a = -2\vec i + 2 \vec j, \vec b = 3\vec i + 3 \vec j$,
    \begin{enumerate}
        \item найдите сумму векторов $\vec a + \vec b$,
        \item постройте сумму векторов $\vec a + \vec b$ на чертеже,
        \item определите модуль суммы векторов $\modul{\vec a + \vec b}$,
        \item вычислите разность векторов $\vec a - \vec b.$
    \end{enumerate}
}
\answer{%
    $\vec a + \vec b = 1\vec i + 5\vec i, \vec a - \vec b = -5\vec i -1\vec i, \modul{\vec a + \vec b} = \sqrt{\sqr1 + \sqr5} \approx 5{,}10.$
}
\solutionspace{120pt}

\tasknumber{3}%
\task{%
    Небольшой лёгкий самолёт взлетел из аэропорта, пролетел $30\,\text{км}$ строго на север, потом повернул и пролетел $40\,\text{км}$ на запад,
    а после по прямой вернулся обратно в аэропорт.
    Определите путь и модуль перемещения самолёта, считая Землю плоской.
}
\solutionspace{120pt}

\tasknumber{4}%
\task{%
    Женя и Валя едут на лошадях: Женя движется на юг со скоростью $3\,\frac{\text{м}}{\text{с}}$, Валя — на север со скоростью $4\,\frac{\text{м}}{\text{с}}$.
    Определите скорость Вали относительно Жени.
    Сделайте рисунок («вид сверху»), подпишите кто где, укажите скорости (в т.ч.
    направление).
}

\variantsplitter

\addpersonalvariant{Алина Полканова}

\tasknumber{1}%
\task{%
    Запишите определения, формулы и физические законы (можно сокращать, но не упустите ключевое):
    \begin{enumerate}
        \item материальная точка,
        \item путь,
        \item положение тела при равномерном прямолинейном движении (векторно),
        \item перемещение при равномерном прямолинейном движении (в проекциях).
    \end{enumerate}
}
\solutionspace{120pt}

\tasknumber{2}%
\task{%
    Положив $\vec a = 2\vec i + 2 \vec j, \vec b = -3\vec i + 3 \vec j$,
    \begin{enumerate}
        \item найдите сумму векторов $\vec a + \vec b$,
        \item постройте сумму векторов $\vec a + \vec b$ на чертеже,
        \item определите модуль суммы векторов $\modul{\vec a + \vec b}$,
        \item вычислите разность векторов $\vec a - \vec b.$
    \end{enumerate}
}
\answer{%
    $\vec a + \vec b = -1\vec i + 5\vec i, \vec a - \vec b = 5\vec i -1\vec i, \modul{\vec a + \vec b} = \sqrt{\sqr-1 + \sqr5} \approx 5{,}10.$
}
\solutionspace{120pt}

\tasknumber{3}%
\task{%
    Небольшой лёгкий самолёт взлетел из аэропорта, пролетел $24\,\text{км}$ строго на юг, потом повернул и пролетел $7\,\text{км}$ на запад,
    а после по прямой вернулся обратно в аэропорт.
    Определите путь и модуль перемещения самолёта, считая Землю плоской.
}
\solutionspace{120pt}

\tasknumber{4}%
\task{%
    Женя и Валя едут на велосипедах: Женя движется на север со скоростью $4\,\frac{\text{км}}{\text{ч}}$, Валя — на север со скоростью $3\,\frac{\text{км}}{\text{ч}}$.
    Определите скорость Вали относительно Жени.
    Сделайте рисунок («вид сверху»), подпишите кто где, укажите скорости (в т.ч.
    направление).
}

\variantsplitter

\addpersonalvariant{Сергей Пономарёв}

\tasknumber{1}%
\task{%
    Запишите определения, формулы и физические законы (можно сокращать, но не упустите ключевое):
    \begin{enumerate}
        \item механическое движение,
        \item перемещение,
        \item положение тела при равномерном прямолинейном движении (векторно),
        \item положение тела при равномерном прямолинейном движении (в проекциях).
    \end{enumerate}
}
\solutionspace{120pt}

\tasknumber{2}%
\task{%
    Положив $\vec a = 3\vec i + 4 \vec j, \vec b = -4\vec i + 2 \vec j$,
    \begin{enumerate}
        \item найдите сумму векторов $\vec a + \vec b$,
        \item постройте сумму векторов $\vec a + \vec b$ на чертеже,
        \item определите модуль суммы векторов $\modul{\vec a + \vec b}$,
        \item вычислите разность векторов $\vec a - \vec b.$
    \end{enumerate}
}
\answer{%
    $\vec a + \vec b = -1\vec i + 6\vec i, \vec a - \vec b = 7\vec i + 2\vec i, \modul{\vec a + \vec b} = \sqrt{\sqr-1 + \sqr6} \approx 6{,}08.$
}
\solutionspace{120pt}

\tasknumber{3}%
\task{%
    Небольшой лёгкий самолёт взлетел из аэропорта, пролетел $12\,\text{км}$ строго на север, потом повернул и пролетел $5\,\text{км}$ на запад,
    а после по прямой вернулся обратно в аэропорт.
    Определите путь и модуль перемещения самолёта, считая Землю плоской.
}
\solutionspace{120pt}

\tasknumber{4}%
\task{%
    Женя и Валя едут на велосипедах: Женя движется на север со скоростью $12\,\frac{\text{м}}{\text{с}}$, Валя — на юг со скоростью $5\,\frac{\text{м}}{\text{с}}$.
    Определите скорость Вали относительно Жени.
    Сделайте рисунок («вид сверху»), подпишите кто где, укажите скорости (в т.ч.
    направление).
}

\variantsplitter

\addpersonalvariant{Егор Свистушкин}

\tasknumber{1}%
\task{%
    Запишите определения, формулы и физические законы (можно сокращать, но не упустите ключевое):
    \begin{enumerate}
        \item материальная точка,
        \item перемещение,
        \item перемещение при равномерном прямолинейном движении (векторно),
        \item положение тела при равномерном прямолинейном движении (в проекциях).
    \end{enumerate}
}
\solutionspace{120pt}

\tasknumber{2}%
\task{%
    Положив $\vec a = -3\vec i + 2 \vec j, \vec b = 3\vec i -2 \vec j$,
    \begin{enumerate}
        \item найдите сумму векторов $\vec a + \vec b$,
        \item постройте сумму векторов $\vec a + \vec b$ на чертеже,
        \item определите модуль суммы векторов $\modul{\vec a + \vec b}$,
        \item вычислите разность векторов $\vec a - \vec b.$
    \end{enumerate}
}
\answer{%
    $\vec a + \vec b = 0\vec i + 0\vec i, \vec a - \vec b = -6\vec i + 4\vec i, \modul{\vec a + \vec b} = \sqrt{\sqr0 + \sqr0} \approx 0{,}00.$
}
\solutionspace{120pt}

\tasknumber{3}%
\task{%
    Небольшой лёгкий самолёт взлетел из аэропорта, пролетел $5\,\text{км}$ строго на юг, потом повернул и пролетел $12\,\text{км}$ на восток,
    а после по прямой вернулся обратно в аэропорт.
    Определите путь и модуль перемещения самолёта, считая Землю плоской.
}
\solutionspace{120pt}

\tasknumber{4}%
\task{%
    Женя и Валя едут на велосипедах: Женя движется на восток со скоростью $5\,\frac{\text{км}}{\text{ч}}$, Валя — на восток со скоростью $12\,\frac{\text{км}}{\text{ч}}$.
    Определите скорость Вали относительно Жени.
    Сделайте рисунок («вид сверху»), подпишите кто где, укажите скорости (в т.ч.
    направление).
}

\variantsplitter

\addpersonalvariant{Дмитрий Соколов}

\tasknumber{1}%
\task{%
    Запишите определения, формулы и физические законы (можно сокращать, но не упустите ключевое):
    \begin{enumerate}
        \item механическое движение,
        \item траектория,
        \item перемещение при равномерном прямолинейном движении (векторно),
        \item перемещение при равномерном прямолинейном движении (в проекциях).
    \end{enumerate}
}
\solutionspace{120pt}

\tasknumber{2}%
\task{%
    Положив $\vec a = -3\vec i -2 \vec j, \vec b = -3\vec i -2 \vec j$,
    \begin{enumerate}
        \item найдите сумму векторов $\vec a + \vec b$,
        \item постройте сумму векторов $\vec a + \vec b$ на чертеже,
        \item определите модуль суммы векторов $\modul{\vec a + \vec b}$,
        \item вычислите разность векторов $\vec a - \vec b.$
    \end{enumerate}
}
\answer{%
    $\vec a + \vec b = -6\vec i -4\vec i, \vec a - \vec b = 0\vec i + 0\vec i, \modul{\vec a + \vec b} = \sqrt{\sqr-6 + \sqr-4} \approx 7{,}21.$
}
\solutionspace{120pt}

\tasknumber{3}%
\task{%
    Небольшой лёгкий самолёт взлетел из аэропорта, пролетел $30\,\text{км}$ строго на север, потом повернул и пролетел $40\,\text{км}$ на запад,
    а после по прямой вернулся обратно в аэропорт.
    Определите путь и модуль перемещения самолёта, считая Землю плоской.
}
\solutionspace{120pt}

\tasknumber{4}%
\task{%
    Женя и Валя едут на велосипедах: Женя движется на запад со скоростью $12\,\frac{\text{м}}{\text{с}}$, Валя — на восток со скоростью $5\,\frac{\text{м}}{\text{с}}$.
    Определите скорость Вали относительно Жени.
    Сделайте рисунок («вид сверху»), подпишите кто где, укажите скорости (в т.ч.
    направление).
}
% autogenerated
