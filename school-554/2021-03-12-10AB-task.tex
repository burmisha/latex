\documentclass[12pt,a4paper]{amsart}%DVI-mode.
\usepackage{graphics,graphicx,epsfig}%DVI-mode.
% \documentclass[pdftex,12pt]{amsart} %PDF-mode.
% \usepackage[pdftex]{graphicx}       %PDF-mode.
% \usepackage[babel=true]{microtype}
% \usepackage[T1]{fontenc}
% \usepackage{lmodern}

\usepackage{cmap}
%\usepackage{a4wide}                 % Fit the text to A4 page tightly.
% \usepackage[utf8]{inputenc}
\usepackage[T2A]{fontenc}
\usepackage[english,russian]{babel} % Download Russian fonts.
\usepackage{amsmath,amsfonts,amssymb,amsthm,amscd,mathrsfs} % Use AMS symbols.
\usepackage{tikz}
\usetikzlibrary{circuits.ee.IEC}
\usetikzlibrary{shapes.geometric}
\usetikzlibrary{decorations.markings}
%\usetikzlibrary{dashs}
%\usetikzlibrary{info}


\textheight=28cm % высота текста
\textwidth=18cm % ширина текста
\topmargin=-2.5cm % отступ от верхнего края
\parskip=2pt % интервал между абзацами
\oddsidemargin=-1.5cm
\evensidemargin=-1.5cm 

\parindent=0pt % абзацный отступ
\tolerance=500 % терпимость к "жидким" строкам
\binoppenalty=10000 % штраф за перенос формул - 10000 - абсолютный запрет
\relpenalty=10000
\flushbottom % выравнивание высоты страниц
\pagenumbering{gobble}

\newcommand\bivec[2]{\begin{pmatrix} #1 \\ #2 \end{pmatrix}}

\newcommand\ol[1]{\overline{#1}}

\newcommand\p[1]{\Prob\!\left(#1\right)}
\newcommand\e[1]{\mathsf{E}\!\left(#1\right)}
\newcommand\disp[1]{\mathsf{D}\!\left(#1\right)}
%\newcommand\norm[2]{\mathcal{N}\!\cbr{#1,#2}}
\newcommand\sign{\text{ sign }}

\newcommand\al[1]{\begin{align*} #1 \end{align*}}
\newcommand\begcas[1]{\begin{cases}#1\end{cases}}
\newcommand\tab[2]{	\vspace{-#1pt}
						\begin{tabbing} 
						#2
						\end{tabbing}
					\vspace{-#1pt}
					}

\newcommand\maintext[1]{{\bfseries\sffamily{#1}}}
\newcommand\skipped[1]{ {\ensuremath{\text{\small{\sffamily{Пропущено:} #1} } } } }
\newcommand\simpletitle[1]{\begin{center} \maintext{#1} \end{center}}

\def\le{\leqslant}
\def\ge{\geqslant}
\def\Ell{\mathcal{L}}
\def\eps{{\varepsilon}}
\def\Rn{\mathbb{R}^n}
\def\RSS{\mathsf{RSS}}

\newcommand\foral[1]{\forall\,#1\:}
\newcommand\exist[1]{\exists\,#1\:\colon}

\newcommand\cbr[1]{\left(#1\right)} %circled brackets
\newcommand\fbr[1]{\left\{#1\right\}} %figure brackets
\newcommand\sbr[1]{\left[#1\right]} %square brackets
\newcommand\modul[1]{\left|#1\right|}

\newcommand\sqr[1]{\cbr{#1}^2}
\newcommand\inv[1]{\cbr{#1}^{-1}}

\newcommand\cdf[2]{\cdot\frac{#1}{#2}}
\newcommand\dd[2]{\frac{\partial#1}{\partial#2}}

\newcommand\integr[2]{\int\limits_{#1}^{#2}}
\newcommand\suml[2]{\sum\limits_{#1}^{#2}}
\newcommand\isum[2]{\sum\limits_{#1=#2}^{+\infty}}
\newcommand\idots[3]{#1_{#2},\ldots,#1_{#3}}
\newcommand\fdots[5]{#4{#1_{#2}}#5\ldots#5#4{#1_{#3}}}

\newcommand\obol[1]{O\!\cbr{#1}}
\newcommand\omal[1]{o\!\cbr{#1}}

\newcommand\addeps[2]{
	\begin{figure} [!ht] %lrp
		\centering
		\includegraphics[height=320px]{#1.eps}
		\vspace{-10pt}
		\caption{#2}
		\label{eps:#1}
	\end{figure}
}

\newcommand\addepssize[3]{
	\begin{figure} [!ht] %lrp hp
		\centering
		\includegraphics[height=#3px]{#1.eps}
		\vspace{-10pt}
		\caption{#2}
		\label{eps:#1}
	\end{figure}
}


\newcommand\norm[1]{\ensuremath{\left\|{#1}\right\|}}
\newcommand\ort{\bot}
\newcommand\theorem[1]{{\sffamily Теорема #1\ }}
\newcommand\lemma[1]{{\sffamily Лемма #1\ }}
\newcommand\difflim[2]{\frac{#1\cbr{#2 + \Delta#2} - #1\cbr{#2}}{\Delta #2}}
\renewcommand\proof[1]{\par\noindent$\square$ #1 \hfill$\blacksquare$\par}
\newcommand\defenition[1]{{\sffamilyОпределение #1\ }}

% \begin{document}
% %\raggedright
% \addclassdate{7}{20 апреля 2018}

\task 1
Площадь большого поршня гидравлического домкрата $S_1 = 20\units{см}^2$, а малого $S_2 = 0{,}5\units{см}^2.$ Груз какой максимальной массы можно поднять этим домкратом, если на малый поршень давить с силой не более $F=200\units{Н}?$ Силой трения от стенки цилиндров пренебречь.

\task 2
В сосуд налита вода. Расстояние от поверхности воды до дна $H = 0{,}5\units{м},$ площадь дна $S = 0{,}1\units{м}^2.$ Найти гидростатическое давление $P_1$ и полное давление $P_2$ вблизи дна. Найти силу давления воды на дно. Плотность воды \rhowater

\task 3
На лёгкий поршень площадью $S=900\units{см}^2,$ касающийся поверхности воды, поставили гирю массы $m=3\units{кг}$. Высота слоя воды в сосуде с вертикальными стенками $H = 20\units{см}$. Определить давление жидкости вблизи дна, если плотность воды \rhowater

\task 4
Давление газов в конце сгорания в цилиндре дизельного двигателя трактора $P = 9\units{МПа}.$ Диаметр цилиндра $d = 130\units{мм}.$ С какой силой газы давят на поршень в цилиндре? Площадь круга диаметром $D$ равна $S = \cfrac{\pi D^2}4.$

\task 5
Площадь малого поршня гидравлического подъёмника $S_1 = 0{,}8\units{см}^2$, а большого $S_2 = 40\units{см}^2.$ Какую силу $F$ надо приложить к малому поршню, чтобы поднять груз весом $P = 8\units{кН}?$

\task 6
Герметичный сосуд полностью заполнен водой и стоит на столе. На небольшой поршень площадью $S$ давят рукой с силой $F$. Поршень находится ниже крышки сосуда на $H_1$, выше дна на $H_2$ и может свободно перемещаться. Плотность воды $\rho$, атмосферное давление $P_A$. Найти давления $P_1$ и $P_2$ в воде вблизи крышки и дна сосуда.
\\ \\
\addclassdate{7}{20 апреля 2018}

\task 1
Площадь большого поршня гидравлического домкрата $S_1 = 20\units{см}^2$, а малого $S_2 = 0{,}5\units{см}^2.$ Груз какой максимальной массы можно поднять этим домкратом, если на малый поршень давить с силой не более $F=200\units{Н}?$ Силой трения от стенки цилиндров пренебречь.

\task 2
В сосуд налита вода. Расстояние от поверхности воды до дна $H = 0{,}5\units{м},$ площадь дна $S = 0{,}1\units{м}^2.$ Найти гидростатическое давление $P_1$ и полное давление $P_2$ вблизи дна. Найти силу давления воды на дно. Плотность воды \rhowater

\task 3
На лёгкий поршень площадью $S=900\units{см}^2,$ касающийся поверхности воды, поставили гирю массы $m=3\units{кг}$. Высота слоя воды в сосуде с вертикальными стенками $H = 20\units{см}$. Определить давление жидкости вблизи дна, если плотность воды \rhowater

\task 4
Давление газов в конце сгорания в цилиндре дизельного двигателя трактора $P = 9\units{МПа}.$ Диаметр цилиндра $d = 130\units{мм}.$ С какой силой газы давят на поршень в цилиндре? Площадь круга диаметром $D$ равна $S = \cfrac{\pi D^2}4.$

\task 5
Площадь малого поршня гидравлического подъёмника $S_1 = 0{,}8\units{см}^2$, а большого $S_2 = 40\units{см}^2.$ Какую силу $F$ надо приложить к малому поршню, чтобы поднять груз весом $P = 8\units{кН}?$

\task 6
Герметичный сосуд полностью заполнен водой и стоит на столе. На небольшой поршень площадью $S$ давят рукой с силой $F$. Поршень находится ниже крышки сосуда на $H_1$, выше дна на $H_2$ и может свободно перемещаться. Плотность воды $\rho$, атмосферное давление $P_A$. Найти давления $P_1$ и $P_2$ в воде вблизи крышки и дна сосуда.

\newpage

\adddate{8 класс. 20 апреля 2018}

\task 1
Между точками $A$ и $B$ электрической цепи подключены последовательно резисторы $R_1 = 10\units{Ом}$ и $R_2 = 20\units{Ом}$ и параллельно им $R_3 = 30\units{Ом}.$ Найдите эквивалентное сопротивление $R_{AB}$ этого участка цепи.

\task 2
Электрическая цепь состоит из последовательности $N$ одинаковых звеньев, в которых каждый резистор имеет сопротивление $r$. Последнее звено замкнуто резистором сопротивлением $R$. При каком соотношении $\cfrac{R}{r}$ сопротивление цепи не зависит от числа звеньев?

\task 3
Для измерения сопротивления $R$ проводника собрана электрическая цепь. Вольтметр $V$ показывает напряжение $U_V = 5\units{В},$ показание амперметра $A$ равно $I_A = 25\units{мА}.$ Найдите величину $R$ сопротивления проводника. Внутреннее сопротивление вольтметра $R_V = 1{,}0\units{кОм},$ внутреннее сопротивление амперметра $R_A = 2{,}0\units{Ом}.$

\task 4
Шкала гальванометра имеет $N=100$ делений, цена деления $\delta = 1\units{мкА}$. Внутреннее сопротивление гальванометра $R_G = 1{,}0\units{кОм}.$ Как из этого прибора сделать вольтметр для измерения напряжений до $U = 100\units{В}$ или амперметр для измерения токов силой до $I = 1\units{А}?$

\\ \\ \\ \\ \\ \\ \\ \\
\adddate{8 класс. 20 апреля 2018}

\task 1
Между точками $A$ и $B$ электрической цепи подключены последовательно резисторы $R_1 = 10\units{Ом}$ и $R_2 = 20\units{Ом}$ и параллельно им $R_3 = 30\units{Ом}.$ Найдите эквивалентное сопротивление $R_{AB}$ этого участка цепи.

\task 2
Электрическая цепь состоит из последовательности $N$ одинаковых звеньев, в которых каждый резистор имеет сопротивление $r$. Последнее звено замкнуто резистором сопротивлением $R$. При каком соотношении $\cfrac{R}{r}$ сопротивление цепи не зависит от числа звеньев?

\task 3
Для измерения сопротивления $R$ проводника собрана электрическая цепь. Вольтметр $V$ показывает напряжение $U_V = 5\units{В},$ показание амперметра $A$ равно $I_A = 25\units{мА}.$ Найдите величину $R$ сопротивления проводника. Внутреннее сопротивление вольтметра $R_V = 1{,}0\units{кОм},$ внутреннее сопротивление амперметра $R_A = 2{,}0\units{Ом}.$

\task 4
Шкала гальванометра имеет $N=100$ делений, цена деления $\delta = 1\units{мкА}$. Внутреннее сопротивление гальванометра $R_G = 1{,}0\units{кОм}.$ Как из этого прибора сделать вольтметр для измерения напряжений до $U = 100\units{В}$ или амперметр для измерения токов силой до $I = 1\units{А}?$


% % \begin{flushright}
\textsc{ГБОУ школа №554, 20 ноября 2018\,г.}
\end{flushright}

\begin{center}
\LARGE \textsc{Математический бой, 8 класс}
\end{center}

\problem{1} Есть тридцать карточек, на каждой написано по одному числу: на десяти карточках~–~$a$,  на десяти других~–~$b$ и на десяти оставшихся~–~$c$ (числа  различны). Известно, что к любым пяти карточкам можно подобрать ещё пять так, что сумма чисел на этих десяти карточках будет равна нулю. Докажите, что~одно из~чисел~$a, b, c$ равно нулю.

\problem{2} Вокруг стола стола пустили пакет с орешками. Первый взял один орешек, второй — 2, третий — 3 и так далее: каждый следующий брал на 1 орешек больше. Известно, что на втором круге было взято в сумме на 100 орешков больше, чем на первом. Сколько человек сидело за столом?

% \problem{2} Натуральное число разрешено увеличить на любое целое число процентов от 1 до 100, если при этом получаем натуральное число. Найдите наименьшее натуральное число, которое нельзя при помощи таких операций получить из~числа 1.

% \problem{3} Найти сумму $1^2 - 2^2 + 3^2 - 4^2 + 5^2 + \ldots - 2018^2$.

\problem{3} В кружке рукоделия, где занимается Валя, более 93\% участников~—~девочки. Какое наименьшее число участников может быть в таком кружке?

\problem{4} Произведение 2018 целых чисел равно 1. Может ли их сумма оказаться равной~0?

% \problem{4} Можно ли все натуральные числа от~1 до~9 записать в~клетки таблицы~$3\times3$ так, чтобы сумма в~любых двух соседних (по~вертикали или горизонтали) клетках равнялось простому числу?

\problem{5} На доске написано 2018 нулей и 2019 единиц. Женя стирает 2 числа и, если они были одинаковы, дописывает к оставшимся один ноль, а~если разные — единицу. Потом Женя повторяет эту операцию снова, потом ещё и~так далее. В~результате на~доске останется только одно число. Что это за~число?

\problem{6} Докажите, что в~любой компании людей найдутся 2~человека, имеющие равное число знакомых в этой компании (если $A$~знаком с~$B$, то~и $B$~знаком с~$A$).

\problem{7} Три колокола начинают бить одновременно. Интервалы между ударами колоколов соответственно составляют $\cfrac43$~секунды, $\cfrac53$~секунды и $2$~секунды. Совпавшие по времени удары воспринимаются за~один. Сколько ударов будет услышано за 1~минуту, включая первый и последний удары?

\problem{8} Восемь одинаковых момент расположены по кругу. Известно, что три из~них~— фальшивые, и они расположены рядом друг с~другом. Вес фальшивой монеты отличается от~веса настоящей. Все фальшивые монеты весят одинаково, но неизвестно, тяжелее или легче фальшивая монета настоящей. Покажите, что за~3~взвешивания на~чашечных весах без~гирь можно определить все фальшивые монеты.

% \end{document}

\begin{document}
\noanswers

\setdate{12~марта~2021}
\setclass{10«АБ»}

\addpersonalvariant{Михаил Бурмистров}

\tasknumber{1}%
\task{%
    Определите КПД цикла 12341, рабочим телом которого является идеальный одноатомный газ, если
    12 — изобарическое расширение газа в шесть раз,
    23 — изохорическое охлаждение газа, при котором температура уменьшается в три раза,
    34 — изобара, 41 — изохора.
    % Для этого:
    % \begin{enumerate}
    %     \item сделайте рисунок в PV-координатах,
    %     \item выберите удобные обозначения, чтобы не запутаться в множестве температур, давлений и объёмов,
    %     \item вычислите необходимые соотнощения между температурами, давлениями и объёмами
    %     (некоторые сразу видны по рисунку, некоторые — надо считать),
    %     \item определите для каждого участка поглощается или отдаётся тепло (и сколько именно:
    %     потребуется первое начало термодинамики, отдельный расчёт работ на участках через площади фигур и изменений внутренней энергии),
    %     \item вычислите полную работу газа в цикле,
    %     \item подставьте всё в формулу для КПД, упростите и доведите до ответа.
    % \end{enumerate}
    Определите КПД цикла Карно, температура нагревателя которого равна максимальной температуре в цикле 12341, а холодильника — минимальной.
    Ответы в обоих случаях оставьте точными в виде нескоратимой дроби, никаких округлений.
}
\answer{%
    \begin{align*}
    A_{12} &> 0, \Delta U_{12} > 0, \implies Q_{12} = A_{12} + \Delta U_{12} > 0, \\
    A_{23} &= 0, \Delta U_{23} < 0, \implies Q_{23} = A_{23} + \Delta U_{23} < 0, \\
    A_{34} &< 0, \Delta U_{34} < 0, \implies Q_{34} = A_{34} + \Delta U_{34} < 0, \\
    A_{41} &= 0, \Delta U_{41} > 0, \implies Q_{41} = A_{41} + \Delta U_{41} > 0.
    \\
    P_1V_1 &= \nu R T_1, P_2V_2 = \nu R T_2, P_3V_3 = \nu R T_3, P_4V_4 = \nu R T_4 \text{ — уравнения состояния идеального газа}, \\
    &\text{Пусть $P_0$, $V_0$, $T_0$ — давление, объём и температура в точке 4 (минимальные во всём цикле):} \\
    P_1 &= P_2, P_3 = P_4 = P_0, V_1 = V_4 = V_0, V_2 = V_3 = 6 V_1 = 6 V_0,, \text{остальные соотношения между объёмами и давлениями не даны, нужно считать} \\
    T_3 &= \frac{T_2}3 \text{(по условию)} \implies \frac{P_2}{P_3} = \frac{P_2 V_2}{P_3 V_3}= \frac{\nu R T_2}{\nu R T_3} = \frac{T_2}{T_3} = 3 \implies P_1 = P_2 = 3 P_0 \\
    A_\text{цикл} &= (3P_0 - P_0)(6V_0 - V_0) = 10P_0V_0, \\
    A_{12} &= 3P_0 \cdot (6V_0 - V_0) = 15P_0V_0, \\
    \Delta U_{12} &= \frac 32 \nu R T_2 - \frac 32 \nu R T_1 = \frac 32 P_2 V_2 - \frac 32 P_1 V_1 = \frac 32 \cdot 3 P_0 \cdot 6 V_0 -  \frac 32 \cdot 3 P_0 \cdot V_0 = \frac 32 \cdot 15 \cdot P_0V_0, \\
    \Delta U_{41} &= \frac 32 \nu R T_1 - \frac 32 \nu R T_4 = \frac 32 P_1 V_1 - \frac 32 P_4 V_4 = \frac 32 \cdot 3 P_0 V_0 - \frac 32 P_0 V_0 = \frac 32 \cdot 2 \cdot P_0V_0.
    \\
    \eta &= \frac{A_\text{цикл}}{Q_+} = \frac{A_\text{цикл}}{Q_{12} + Q_{41}}  = \frac{A_\text{цикл}}{A_{12} + \Delta U_{12} + A_{41} + \Delta U_{41}} =  \\
     &= \frac{10P_0V_0}{15P_0V_0 + \frac 32 \cdot 15 \cdot P_0V_0 + 0 + \frac 32 \cdot 2 \cdot P_0V_0} = \frac{10}{15 + \frac 32 \cdot 15 + \frac 32 \cdot 2} = \frac{20}{81} \approx 0{,}247.
     \\
    \eta_\text{Карно} &= 1 - \frac{T_\text{х}}{T_\text{н}} = 1 - \frac{T_\text{4}}{T_\text{2}} = 1 - \frac{\frac{P_4V_4}{\nu R}}{\frac{P_2V_2}{\nu R}} = 1 - \frac{P_4V_4}{P_2V_2} = 1 - \frac{P_0V_0}{3P_0 \cdot 6V_0} = 1 - \frac 1{3 \cdot 6}  = \frac{17}{18} \approx 0{,}944.
    \end{align*}
}
\solutionspace{360pt}

\tasknumber{2}%
\task{%
    Порция идеального одноатомного газа перешла из состояния 1 в состояние 2: $P_1 = 4\,\text{МПа}$, $V_1 = 3\,\text{л}$, $P_2 = 2{,}5\,\text{МПа}$, $V_2 = 6\,\text{л}$.
    Определите, какую работу при этом совершил газ, чему равно изменение внутренней энергии газа, сколько теплоты подвели к нему в этом процессе?
    При решении обратите внимание на знаки искомых величин.
    Известно, что в PV-координатах график процесса 12 представляет собой отрезок прямой.
}
\answer{%
    \begin{align*}
    P_1V_1 &= \nu R T_1, P_2V_2 = \nu R T_2, \\
    \Delta U &= U_2-U_1 = \frac 32 \nu R T_2- \frac 32 \nu R T_1 = \frac 32 P_2 V_2 - \frac 32 P_1 V_1= \frac 32 \cdot \cbr{2{,}5\,\text{МПа} \cdot6\,\text{л} - 4\,\text{МПа} \cdot3\,\text{л}} = 4{,}50\,\text{кДж}.
    \\
    A_\text{газа} &= \frac{P_2 + P_1} 2 \cdot (V_2 - V_1) = \frac{2{,}5\,\text{МПа} + 4\,\text{МПа}} 2 \cdot (6\,\text{л} - 3\,\text{л}) = 9{,}75\,\text{кДж}, \\
    Q &= A_\text{газа} + \Delta U = \frac 32 (P_2 V_2 - P_1 V_1) + \frac{P_2 + P_1} 2 \cdot (V_2 - V_1) = 4{,}50\,\text{кДж} + 9{,}75\,\text{кДж} = 14{,}25\,\text{кДж}.
    \end{align*}
}
\solutionspace{150pt}

\tasknumber{3}%
\task{%
    Запишите формулы и рядом с каждой физичической величиной укажите её название и единицы изменения в СИ:
    \begin{enumerate}
        \item первое начало термодинамики,
        \item внутренняя энергия идеального одноатомного газа.
    \end{enumerate}
}

\variantsplitter

\addpersonalvariant{Ирина Ан}

\tasknumber{1}%
\task{%
    Определите КПД цикла 12341, рабочим телом которого является идеальный одноатомный газ, если
    12 — изобарическое расширение газа в пять раз,
    23 — изохорическое охлаждение газа, при котором температура уменьшается в два раза,
    34 — изобара, 41 — изохора.
    % Для этого:
    % \begin{enumerate}
    %     \item сделайте рисунок в PV-координатах,
    %     \item выберите удобные обозначения, чтобы не запутаться в множестве температур, давлений и объёмов,
    %     \item вычислите необходимые соотнощения между температурами, давлениями и объёмами
    %     (некоторые сразу видны по рисунку, некоторые — надо считать),
    %     \item определите для каждого участка поглощается или отдаётся тепло (и сколько именно:
    %     потребуется первое начало термодинамики, отдельный расчёт работ на участках через площади фигур и изменений внутренней энергии),
    %     \item вычислите полную работу газа в цикле,
    %     \item подставьте всё в формулу для КПД, упростите и доведите до ответа.
    % \end{enumerate}
    Определите КПД цикла Карно, температура нагревателя которого равна максимальной температуре в цикле 12341, а холодильника — минимальной.
    Ответы в обоих случаях оставьте точными в виде нескоратимой дроби, никаких округлений.
}
\answer{%
    \begin{align*}
    A_{12} &> 0, \Delta U_{12} > 0, \implies Q_{12} = A_{12} + \Delta U_{12} > 0, \\
    A_{23} &= 0, \Delta U_{23} < 0, \implies Q_{23} = A_{23} + \Delta U_{23} < 0, \\
    A_{34} &< 0, \Delta U_{34} < 0, \implies Q_{34} = A_{34} + \Delta U_{34} < 0, \\
    A_{41} &= 0, \Delta U_{41} > 0, \implies Q_{41} = A_{41} + \Delta U_{41} > 0.
    \\
    P_1V_1 &= \nu R T_1, P_2V_2 = \nu R T_2, P_3V_3 = \nu R T_3, P_4V_4 = \nu R T_4 \text{ — уравнения состояния идеального газа}, \\
    &\text{Пусть $P_0$, $V_0$, $T_0$ — давление, объём и температура в точке 4 (минимальные во всём цикле):} \\
    P_1 &= P_2, P_3 = P_4 = P_0, V_1 = V_4 = V_0, V_2 = V_3 = 5 V_1 = 5 V_0,, \text{остальные соотношения между объёмами и давлениями не даны, нужно считать} \\
    T_3 &= \frac{T_2}2 \text{(по условию)} \implies \frac{P_2}{P_3} = \frac{P_2 V_2}{P_3 V_3}= \frac{\nu R T_2}{\nu R T_3} = \frac{T_2}{T_3} = 2 \implies P_1 = P_2 = 2 P_0 \\
    A_\text{цикл} &= (2P_0 - P_0)(5V_0 - V_0) = 4P_0V_0, \\
    A_{12} &= 2P_0 \cdot (5V_0 - V_0) = 8P_0V_0, \\
    \Delta U_{12} &= \frac 32 \nu R T_2 - \frac 32 \nu R T_1 = \frac 32 P_2 V_2 - \frac 32 P_1 V_1 = \frac 32 \cdot 2 P_0 \cdot 5 V_0 -  \frac 32 \cdot 2 P_0 \cdot V_0 = \frac 32 \cdot 8 \cdot P_0V_0, \\
    \Delta U_{41} &= \frac 32 \nu R T_1 - \frac 32 \nu R T_4 = \frac 32 P_1 V_1 - \frac 32 P_4 V_4 = \frac 32 \cdot 2 P_0 V_0 - \frac 32 P_0 V_0 = \frac 32 \cdot 1 \cdot P_0V_0.
    \\
    \eta &= \frac{A_\text{цикл}}{Q_+} = \frac{A_\text{цикл}}{Q_{12} + Q_{41}}  = \frac{A_\text{цикл}}{A_{12} + \Delta U_{12} + A_{41} + \Delta U_{41}} =  \\
     &= \frac{4P_0V_0}{8P_0V_0 + \frac 32 \cdot 8 \cdot P_0V_0 + 0 + \frac 32 \cdot 1 \cdot P_0V_0} = \frac{4}{8 + \frac 32 \cdot 8 + \frac 32 \cdot 1} = \frac{8}{43} \approx 0{,}186.
     \\
    \eta_\text{Карно} &= 1 - \frac{T_\text{х}}{T_\text{н}} = 1 - \frac{T_\text{4}}{T_\text{2}} = 1 - \frac{\frac{P_4V_4}{\nu R}}{\frac{P_2V_2}{\nu R}} = 1 - \frac{P_4V_4}{P_2V_2} = 1 - \frac{P_0V_0}{2P_0 \cdot 5V_0} = 1 - \frac 1{2 \cdot 5}  = \frac{9}{10} \approx 0{,}900.
    \end{align*}
}
\solutionspace{360pt}

\tasknumber{2}%
\task{%
    Порция идеального одноатомного газа перешла из состояния 1 в состояние 2: $P_1 = 3\,\text{МПа}$, $V_1 = 5\,\text{л}$, $P_2 = 4{,}5\,\text{МПа}$, $V_2 = 2\,\text{л}$.
    Определите, какую работу при этом совершил газ, чему равно изменение внутренней энергии газа, сколько теплоты подвели к нему в этом процессе?
    При решении обратите внимание на знаки искомых величин.
    Известно, что в PV-координатах график процесса 12 представляет собой отрезок прямой.
}
\answer{%
    \begin{align*}
    P_1V_1 &= \nu R T_1, P_2V_2 = \nu R T_2, \\
    \Delta U &= U_2-U_1 = \frac 32 \nu R T_2- \frac 32 \nu R T_1 = \frac 32 P_2 V_2 - \frac 32 P_1 V_1= \frac 32 \cdot \cbr{4{,}5\,\text{МПа} \cdot2\,\text{л} - 3\,\text{МПа} \cdot5\,\text{л}} = -9{,}000\,\text{кДж}.
    \\
    A_\text{газа} &= \frac{P_2 + P_1} 2 \cdot (V_2 - V_1) = \frac{4{,}5\,\text{МПа} + 3\,\text{МПа}} 2 \cdot (2\,\text{л} - 5\,\text{л}) = -11{,}2500\,\text{кДж}, \\
    Q &= A_\text{газа} + \Delta U = \frac 32 (P_2 V_2 - P_1 V_1) + \frac{P_2 + P_1} 2 \cdot (V_2 - V_1) = -9{,}000\,\text{кДж} -11{,}2500\,\text{кДж} = -20{,}250\,\text{кДж}.
    \end{align*}
}
\solutionspace{150pt}

\tasknumber{3}%
\task{%
    Запишите формулы и рядом с каждой физичической величиной укажите её название и единицы изменения в СИ:
    \begin{enumerate}
        \item первое начало термодинамики,
        \item внутренняя энергия идеального одноатомного газа.
    \end{enumerate}
}

\variantsplitter

\addpersonalvariant{Софья Андрианова}

\tasknumber{1}%
\task{%
    Определите КПД цикла 12341, рабочим телом которого является идеальный одноатомный газ, если
    12 — изобарическое расширение газа в шесть раз,
    23 — изохорическое охлаждение газа, при котором температура уменьшается в два раза,
    34 — изобара, 41 — изохора.
    % Для этого:
    % \begin{enumerate}
    %     \item сделайте рисунок в PV-координатах,
    %     \item выберите удобные обозначения, чтобы не запутаться в множестве температур, давлений и объёмов,
    %     \item вычислите необходимые соотнощения между температурами, давлениями и объёмами
    %     (некоторые сразу видны по рисунку, некоторые — надо считать),
    %     \item определите для каждого участка поглощается или отдаётся тепло (и сколько именно:
    %     потребуется первое начало термодинамики, отдельный расчёт работ на участках через площади фигур и изменений внутренней энергии),
    %     \item вычислите полную работу газа в цикле,
    %     \item подставьте всё в формулу для КПД, упростите и доведите до ответа.
    % \end{enumerate}
    Определите КПД цикла Карно, температура нагревателя которого равна максимальной температуре в цикле 12341, а холодильника — минимальной.
    Ответы в обоих случаях оставьте точными в виде нескоратимой дроби, никаких округлений.
}
\answer{%
    \begin{align*}
    A_{12} &> 0, \Delta U_{12} > 0, \implies Q_{12} = A_{12} + \Delta U_{12} > 0, \\
    A_{23} &= 0, \Delta U_{23} < 0, \implies Q_{23} = A_{23} + \Delta U_{23} < 0, \\
    A_{34} &< 0, \Delta U_{34} < 0, \implies Q_{34} = A_{34} + \Delta U_{34} < 0, \\
    A_{41} &= 0, \Delta U_{41} > 0, \implies Q_{41} = A_{41} + \Delta U_{41} > 0.
    \\
    P_1V_1 &= \nu R T_1, P_2V_2 = \nu R T_2, P_3V_3 = \nu R T_3, P_4V_4 = \nu R T_4 \text{ — уравнения состояния идеального газа}, \\
    &\text{Пусть $P_0$, $V_0$, $T_0$ — давление, объём и температура в точке 4 (минимальные во всём цикле):} \\
    P_1 &= P_2, P_3 = P_4 = P_0, V_1 = V_4 = V_0, V_2 = V_3 = 6 V_1 = 6 V_0,, \text{остальные соотношения между объёмами и давлениями не даны, нужно считать} \\
    T_3 &= \frac{T_2}2 \text{(по условию)} \implies \frac{P_2}{P_3} = \frac{P_2 V_2}{P_3 V_3}= \frac{\nu R T_2}{\nu R T_3} = \frac{T_2}{T_3} = 2 \implies P_1 = P_2 = 2 P_0 \\
    A_\text{цикл} &= (2P_0 - P_0)(6V_0 - V_0) = 5P_0V_0, \\
    A_{12} &= 2P_0 \cdot (6V_0 - V_0) = 10P_0V_0, \\
    \Delta U_{12} &= \frac 32 \nu R T_2 - \frac 32 \nu R T_1 = \frac 32 P_2 V_2 - \frac 32 P_1 V_1 = \frac 32 \cdot 2 P_0 \cdot 6 V_0 -  \frac 32 \cdot 2 P_0 \cdot V_0 = \frac 32 \cdot 10 \cdot P_0V_0, \\
    \Delta U_{41} &= \frac 32 \nu R T_1 - \frac 32 \nu R T_4 = \frac 32 P_1 V_1 - \frac 32 P_4 V_4 = \frac 32 \cdot 2 P_0 V_0 - \frac 32 P_0 V_0 = \frac 32 \cdot 1 \cdot P_0V_0.
    \\
    \eta &= \frac{A_\text{цикл}}{Q_+} = \frac{A_\text{цикл}}{Q_{12} + Q_{41}}  = \frac{A_\text{цикл}}{A_{12} + \Delta U_{12} + A_{41} + \Delta U_{41}} =  \\
     &= \frac{5P_0V_0}{10P_0V_0 + \frac 32 \cdot 10 \cdot P_0V_0 + 0 + \frac 32 \cdot 1 \cdot P_0V_0} = \frac{5}{10 + \frac 32 \cdot 10 + \frac 32 \cdot 1} = \frac{10}{53} \approx 0{,}189.
     \\
    \eta_\text{Карно} &= 1 - \frac{T_\text{х}}{T_\text{н}} = 1 - \frac{T_\text{4}}{T_\text{2}} = 1 - \frac{\frac{P_4V_4}{\nu R}}{\frac{P_2V_2}{\nu R}} = 1 - \frac{P_4V_4}{P_2V_2} = 1 - \frac{P_0V_0}{2P_0 \cdot 6V_0} = 1 - \frac 1{2 \cdot 6}  = \frac{11}{12} \approx 0{,}917.
    \end{align*}
}
\solutionspace{360pt}

\tasknumber{2}%
\task{%
    Порция идеального одноатомного газа перешла из состояния 1 в состояние 2: $P_1 = 4\,\text{МПа}$, $V_1 = 3\,\text{л}$, $P_2 = 1{,}5\,\text{МПа}$, $V_2 = 4\,\text{л}$.
    Определите, какую работу при этом совершил газ, чему равно изменение внутренней энергии газа, сколько теплоты подвели к нему в этом процессе?
    При решении обратите внимание на знаки искомых величин.
    Известно, что в PV-координатах график процесса 12 представляет собой отрезок прямой.
}
\answer{%
    \begin{align*}
    P_1V_1 &= \nu R T_1, P_2V_2 = \nu R T_2, \\
    \Delta U &= U_2-U_1 = \frac 32 \nu R T_2- \frac 32 \nu R T_1 = \frac 32 P_2 V_2 - \frac 32 P_1 V_1= \frac 32 \cdot \cbr{1{,}5\,\text{МПа} \cdot4\,\text{л} - 4\,\text{МПа} \cdot3\,\text{л}} = -9{,}000\,\text{кДж}.
    \\
    A_\text{газа} &= \frac{P_2 + P_1} 2 \cdot (V_2 - V_1) = \frac{1{,}5\,\text{МПа} + 4\,\text{МПа}} 2 \cdot (4\,\text{л} - 3\,\text{л}) = 2{,}75\,\text{кДж}, \\
    Q &= A_\text{газа} + \Delta U = \frac 32 (P_2 V_2 - P_1 V_1) + \frac{P_2 + P_1} 2 \cdot (V_2 - V_1) = -9{,}000\,\text{кДж} + 2{,}75\,\text{кДж} = -6{,}250\,\text{кДж}.
    \end{align*}
}
\solutionspace{150pt}

\tasknumber{3}%
\task{%
    Запишите формулы и рядом с каждой физичической величиной укажите её название и единицы изменения в СИ:
    \begin{enumerate}
        \item первое начало термодинамики,
        \item внутренняя энергия идеального одноатомного газа.
    \end{enumerate}
}

\variantsplitter

\addpersonalvariant{Владимир Артемчук}

\tasknumber{1}%
\task{%
    Определите КПД цикла 12341, рабочим телом которого является идеальный одноатомный газ, если
    12 — изобарическое расширение газа в шесть раз,
    23 — изохорическое охлаждение газа, при котором температура уменьшается в пять раз,
    34 — изобара, 41 — изохора.
    % Для этого:
    % \begin{enumerate}
    %     \item сделайте рисунок в PV-координатах,
    %     \item выберите удобные обозначения, чтобы не запутаться в множестве температур, давлений и объёмов,
    %     \item вычислите необходимые соотнощения между температурами, давлениями и объёмами
    %     (некоторые сразу видны по рисунку, некоторые — надо считать),
    %     \item определите для каждого участка поглощается или отдаётся тепло (и сколько именно:
    %     потребуется первое начало термодинамики, отдельный расчёт работ на участках через площади фигур и изменений внутренней энергии),
    %     \item вычислите полную работу газа в цикле,
    %     \item подставьте всё в формулу для КПД, упростите и доведите до ответа.
    % \end{enumerate}
    Определите КПД цикла Карно, температура нагревателя которого равна максимальной температуре в цикле 12341, а холодильника — минимальной.
    Ответы в обоих случаях оставьте точными в виде нескоратимой дроби, никаких округлений.
}
\answer{%
    \begin{align*}
    A_{12} &> 0, \Delta U_{12} > 0, \implies Q_{12} = A_{12} + \Delta U_{12} > 0, \\
    A_{23} &= 0, \Delta U_{23} < 0, \implies Q_{23} = A_{23} + \Delta U_{23} < 0, \\
    A_{34} &< 0, \Delta U_{34} < 0, \implies Q_{34} = A_{34} + \Delta U_{34} < 0, \\
    A_{41} &= 0, \Delta U_{41} > 0, \implies Q_{41} = A_{41} + \Delta U_{41} > 0.
    \\
    P_1V_1 &= \nu R T_1, P_2V_2 = \nu R T_2, P_3V_3 = \nu R T_3, P_4V_4 = \nu R T_4 \text{ — уравнения состояния идеального газа}, \\
    &\text{Пусть $P_0$, $V_0$, $T_0$ — давление, объём и температура в точке 4 (минимальные во всём цикле):} \\
    P_1 &= P_2, P_3 = P_4 = P_0, V_1 = V_4 = V_0, V_2 = V_3 = 6 V_1 = 6 V_0,, \text{остальные соотношения между объёмами и давлениями не даны, нужно считать} \\
    T_3 &= \frac{T_2}5 \text{(по условию)} \implies \frac{P_2}{P_3} = \frac{P_2 V_2}{P_3 V_3}= \frac{\nu R T_2}{\nu R T_3} = \frac{T_2}{T_3} = 5 \implies P_1 = P_2 = 5 P_0 \\
    A_\text{цикл} &= (5P_0 - P_0)(6V_0 - V_0) = 20P_0V_0, \\
    A_{12} &= 5P_0 \cdot (6V_0 - V_0) = 25P_0V_0, \\
    \Delta U_{12} &= \frac 32 \nu R T_2 - \frac 32 \nu R T_1 = \frac 32 P_2 V_2 - \frac 32 P_1 V_1 = \frac 32 \cdot 5 P_0 \cdot 6 V_0 -  \frac 32 \cdot 5 P_0 \cdot V_0 = \frac 32 \cdot 25 \cdot P_0V_0, \\
    \Delta U_{41} &= \frac 32 \nu R T_1 - \frac 32 \nu R T_4 = \frac 32 P_1 V_1 - \frac 32 P_4 V_4 = \frac 32 \cdot 5 P_0 V_0 - \frac 32 P_0 V_0 = \frac 32 \cdot 4 \cdot P_0V_0.
    \\
    \eta &= \frac{A_\text{цикл}}{Q_+} = \frac{A_\text{цикл}}{Q_{12} + Q_{41}}  = \frac{A_\text{цикл}}{A_{12} + \Delta U_{12} + A_{41} + \Delta U_{41}} =  \\
     &= \frac{20P_0V_0}{25P_0V_0 + \frac 32 \cdot 25 \cdot P_0V_0 + 0 + \frac 32 \cdot 4 \cdot P_0V_0} = \frac{20}{25 + \frac 32 \cdot 25 + \frac 32 \cdot 4} = \frac{40}{137} \approx 0{,}292.
     \\
    \eta_\text{Карно} &= 1 - \frac{T_\text{х}}{T_\text{н}} = 1 - \frac{T_\text{4}}{T_\text{2}} = 1 - \frac{\frac{P_4V_4}{\nu R}}{\frac{P_2V_2}{\nu R}} = 1 - \frac{P_4V_4}{P_2V_2} = 1 - \frac{P_0V_0}{5P_0 \cdot 6V_0} = 1 - \frac 1{5 \cdot 6}  = \frac{29}{30} \approx 0{,}967.
    \end{align*}
}
\solutionspace{360pt}

\tasknumber{2}%
\task{%
    Порция идеального одноатомного газа перешла из состояния 1 в состояние 2: $P_1 = 2\,\text{МПа}$, $V_1 = 5\,\text{л}$, $P_2 = 2{,}5\,\text{МПа}$, $V_2 = 8\,\text{л}$.
    Определите, какую работу при этом совершил газ, чему равно изменение внутренней энергии газа, сколько теплоты подвели к нему в этом процессе?
    При решении обратите внимание на знаки искомых величин.
    Известно, что в PV-координатах график процесса 12 представляет собой отрезок прямой.
}
\answer{%
    \begin{align*}
    P_1V_1 &= \nu R T_1, P_2V_2 = \nu R T_2, \\
    \Delta U &= U_2-U_1 = \frac 32 \nu R T_2- \frac 32 \nu R T_1 = \frac 32 P_2 V_2 - \frac 32 P_1 V_1= \frac 32 \cdot \cbr{2{,}5\,\text{МПа} \cdot8\,\text{л} - 2\,\text{МПа} \cdot5\,\text{л}} = 15{,}00\,\text{кДж}.
    \\
    A_\text{газа} &= \frac{P_2 + P_1} 2 \cdot (V_2 - V_1) = \frac{2{,}5\,\text{МПа} + 2\,\text{МПа}} 2 \cdot (8\,\text{л} - 5\,\text{л}) = 6{,}75\,\text{кДж}, \\
    Q &= A_\text{газа} + \Delta U = \frac 32 (P_2 V_2 - P_1 V_1) + \frac{P_2 + P_1} 2 \cdot (V_2 - V_1) = 15{,}00\,\text{кДж} + 6{,}75\,\text{кДж} = 21{,}75\,\text{кДж}.
    \end{align*}
}
\solutionspace{150pt}

\tasknumber{3}%
\task{%
    Запишите формулы и рядом с каждой физичической величиной укажите её название и единицы изменения в СИ:
    \begin{enumerate}
        \item первое начало термодинамики,
        \item внутренняя энергия идеального одноатомного газа.
    \end{enumerate}
}

\variantsplitter

\addpersonalvariant{Софья Белянкина}

\tasknumber{1}%
\task{%
    Определите КПД цикла 12341, рабочим телом которого является идеальный одноатомный газ, если
    12 — изобарическое расширение газа в четыре раза,
    23 — изохорическое охлаждение газа, при котором температура уменьшается в три раза,
    34 — изобара, 41 — изохора.
    % Для этого:
    % \begin{enumerate}
    %     \item сделайте рисунок в PV-координатах,
    %     \item выберите удобные обозначения, чтобы не запутаться в множестве температур, давлений и объёмов,
    %     \item вычислите необходимые соотнощения между температурами, давлениями и объёмами
    %     (некоторые сразу видны по рисунку, некоторые — надо считать),
    %     \item определите для каждого участка поглощается или отдаётся тепло (и сколько именно:
    %     потребуется первое начало термодинамики, отдельный расчёт работ на участках через площади фигур и изменений внутренней энергии),
    %     \item вычислите полную работу газа в цикле,
    %     \item подставьте всё в формулу для КПД, упростите и доведите до ответа.
    % \end{enumerate}
    Определите КПД цикла Карно, температура нагревателя которого равна максимальной температуре в цикле 12341, а холодильника — минимальной.
    Ответы в обоих случаях оставьте точными в виде нескоратимой дроби, никаких округлений.
}
\answer{%
    \begin{align*}
    A_{12} &> 0, \Delta U_{12} > 0, \implies Q_{12} = A_{12} + \Delta U_{12} > 0, \\
    A_{23} &= 0, \Delta U_{23} < 0, \implies Q_{23} = A_{23} + \Delta U_{23} < 0, \\
    A_{34} &< 0, \Delta U_{34} < 0, \implies Q_{34} = A_{34} + \Delta U_{34} < 0, \\
    A_{41} &= 0, \Delta U_{41} > 0, \implies Q_{41} = A_{41} + \Delta U_{41} > 0.
    \\
    P_1V_1 &= \nu R T_1, P_2V_2 = \nu R T_2, P_3V_3 = \nu R T_3, P_4V_4 = \nu R T_4 \text{ — уравнения состояния идеального газа}, \\
    &\text{Пусть $P_0$, $V_0$, $T_0$ — давление, объём и температура в точке 4 (минимальные во всём цикле):} \\
    P_1 &= P_2, P_3 = P_4 = P_0, V_1 = V_4 = V_0, V_2 = V_3 = 4 V_1 = 4 V_0,, \text{остальные соотношения между объёмами и давлениями не даны, нужно считать} \\
    T_3 &= \frac{T_2}3 \text{(по условию)} \implies \frac{P_2}{P_3} = \frac{P_2 V_2}{P_3 V_3}= \frac{\nu R T_2}{\nu R T_3} = \frac{T_2}{T_3} = 3 \implies P_1 = P_2 = 3 P_0 \\
    A_\text{цикл} &= (3P_0 - P_0)(4V_0 - V_0) = 6P_0V_0, \\
    A_{12} &= 3P_0 \cdot (4V_0 - V_0) = 9P_0V_0, \\
    \Delta U_{12} &= \frac 32 \nu R T_2 - \frac 32 \nu R T_1 = \frac 32 P_2 V_2 - \frac 32 P_1 V_1 = \frac 32 \cdot 3 P_0 \cdot 4 V_0 -  \frac 32 \cdot 3 P_0 \cdot V_0 = \frac 32 \cdot 9 \cdot P_0V_0, \\
    \Delta U_{41} &= \frac 32 \nu R T_1 - \frac 32 \nu R T_4 = \frac 32 P_1 V_1 - \frac 32 P_4 V_4 = \frac 32 \cdot 3 P_0 V_0 - \frac 32 P_0 V_0 = \frac 32 \cdot 2 \cdot P_0V_0.
    \\
    \eta &= \frac{A_\text{цикл}}{Q_+} = \frac{A_\text{цикл}}{Q_{12} + Q_{41}}  = \frac{A_\text{цикл}}{A_{12} + \Delta U_{12} + A_{41} + \Delta U_{41}} =  \\
     &= \frac{6P_0V_0}{9P_0V_0 + \frac 32 \cdot 9 \cdot P_0V_0 + 0 + \frac 32 \cdot 2 \cdot P_0V_0} = \frac{6}{9 + \frac 32 \cdot 9 + \frac 32 \cdot 2} = \frac{4}{17} \approx 0{,}235.
     \\
    \eta_\text{Карно} &= 1 - \frac{T_\text{х}}{T_\text{н}} = 1 - \frac{T_\text{4}}{T_\text{2}} = 1 - \frac{\frac{P_4V_4}{\nu R}}{\frac{P_2V_2}{\nu R}} = 1 - \frac{P_4V_4}{P_2V_2} = 1 - \frac{P_0V_0}{3P_0 \cdot 4V_0} = 1 - \frac 1{3 \cdot 4}  = \frac{11}{12} \approx 0{,}917.
    \end{align*}
}
\solutionspace{360pt}

\tasknumber{2}%
\task{%
    Порция идеального одноатомного газа перешла из состояния 1 в состояние 2: $P_1 = 4\,\text{МПа}$, $V_1 = 7\,\text{л}$, $P_2 = 1{,}5\,\text{МПа}$, $V_2 = 8\,\text{л}$.
    Определите, какую работу при этом совершил газ, чему равно изменение внутренней энергии газа, сколько теплоты подвели к нему в этом процессе?
    При решении обратите внимание на знаки искомых величин.
    Известно, что в PV-координатах график процесса 12 представляет собой отрезок прямой.
}
\answer{%
    \begin{align*}
    P_1V_1 &= \nu R T_1, P_2V_2 = \nu R T_2, \\
    \Delta U &= U_2-U_1 = \frac 32 \nu R T_2- \frac 32 \nu R T_1 = \frac 32 P_2 V_2 - \frac 32 P_1 V_1= \frac 32 \cdot \cbr{1{,}5\,\text{МПа} \cdot8\,\text{л} - 4\,\text{МПа} \cdot7\,\text{л}} = -24{,}000\,\text{кДж}.
    \\
    A_\text{газа} &= \frac{P_2 + P_1} 2 \cdot (V_2 - V_1) = \frac{1{,}5\,\text{МПа} + 4\,\text{МПа}} 2 \cdot (8\,\text{л} - 7\,\text{л}) = 2{,}75\,\text{кДж}, \\
    Q &= A_\text{газа} + \Delta U = \frac 32 (P_2 V_2 - P_1 V_1) + \frac{P_2 + P_1} 2 \cdot (V_2 - V_1) = -24{,}000\,\text{кДж} + 2{,}75\,\text{кДж} = -21{,}250\,\text{кДж}.
    \end{align*}
}
\solutionspace{150pt}

\tasknumber{3}%
\task{%
    Запишите формулы и рядом с каждой физичической величиной укажите её название и единицы изменения в СИ:
    \begin{enumerate}
        \item первое начало термодинамики,
        \item внутренняя энергия идеального одноатомного газа.
    \end{enumerate}
}

\variantsplitter

\addpersonalvariant{Варвара Егиазарян}

\tasknumber{1}%
\task{%
    Определите КПД цикла 12341, рабочим телом которого является идеальный одноатомный газ, если
    12 — изобарическое расширение газа в пять раз,
    23 — изохорическое охлаждение газа, при котором температура уменьшается в два раза,
    34 — изобара, 41 — изохора.
    % Для этого:
    % \begin{enumerate}
    %     \item сделайте рисунок в PV-координатах,
    %     \item выберите удобные обозначения, чтобы не запутаться в множестве температур, давлений и объёмов,
    %     \item вычислите необходимые соотнощения между температурами, давлениями и объёмами
    %     (некоторые сразу видны по рисунку, некоторые — надо считать),
    %     \item определите для каждого участка поглощается или отдаётся тепло (и сколько именно:
    %     потребуется первое начало термодинамики, отдельный расчёт работ на участках через площади фигур и изменений внутренней энергии),
    %     \item вычислите полную работу газа в цикле,
    %     \item подставьте всё в формулу для КПД, упростите и доведите до ответа.
    % \end{enumerate}
    Определите КПД цикла Карно, температура нагревателя которого равна максимальной температуре в цикле 12341, а холодильника — минимальной.
    Ответы в обоих случаях оставьте точными в виде нескоратимой дроби, никаких округлений.
}
\answer{%
    \begin{align*}
    A_{12} &> 0, \Delta U_{12} > 0, \implies Q_{12} = A_{12} + \Delta U_{12} > 0, \\
    A_{23} &= 0, \Delta U_{23} < 0, \implies Q_{23} = A_{23} + \Delta U_{23} < 0, \\
    A_{34} &< 0, \Delta U_{34} < 0, \implies Q_{34} = A_{34} + \Delta U_{34} < 0, \\
    A_{41} &= 0, \Delta U_{41} > 0, \implies Q_{41} = A_{41} + \Delta U_{41} > 0.
    \\
    P_1V_1 &= \nu R T_1, P_2V_2 = \nu R T_2, P_3V_3 = \nu R T_3, P_4V_4 = \nu R T_4 \text{ — уравнения состояния идеального газа}, \\
    &\text{Пусть $P_0$, $V_0$, $T_0$ — давление, объём и температура в точке 4 (минимальные во всём цикле):} \\
    P_1 &= P_2, P_3 = P_4 = P_0, V_1 = V_4 = V_0, V_2 = V_3 = 5 V_1 = 5 V_0,, \text{остальные соотношения между объёмами и давлениями не даны, нужно считать} \\
    T_3 &= \frac{T_2}2 \text{(по условию)} \implies \frac{P_2}{P_3} = \frac{P_2 V_2}{P_3 V_3}= \frac{\nu R T_2}{\nu R T_3} = \frac{T_2}{T_3} = 2 \implies P_1 = P_2 = 2 P_0 \\
    A_\text{цикл} &= (2P_0 - P_0)(5V_0 - V_0) = 4P_0V_0, \\
    A_{12} &= 2P_0 \cdot (5V_0 - V_0) = 8P_0V_0, \\
    \Delta U_{12} &= \frac 32 \nu R T_2 - \frac 32 \nu R T_1 = \frac 32 P_2 V_2 - \frac 32 P_1 V_1 = \frac 32 \cdot 2 P_0 \cdot 5 V_0 -  \frac 32 \cdot 2 P_0 \cdot V_0 = \frac 32 \cdot 8 \cdot P_0V_0, \\
    \Delta U_{41} &= \frac 32 \nu R T_1 - \frac 32 \nu R T_4 = \frac 32 P_1 V_1 - \frac 32 P_4 V_4 = \frac 32 \cdot 2 P_0 V_0 - \frac 32 P_0 V_0 = \frac 32 \cdot 1 \cdot P_0V_0.
    \\
    \eta &= \frac{A_\text{цикл}}{Q_+} = \frac{A_\text{цикл}}{Q_{12} + Q_{41}}  = \frac{A_\text{цикл}}{A_{12} + \Delta U_{12} + A_{41} + \Delta U_{41}} =  \\
     &= \frac{4P_0V_0}{8P_0V_0 + \frac 32 \cdot 8 \cdot P_0V_0 + 0 + \frac 32 \cdot 1 \cdot P_0V_0} = \frac{4}{8 + \frac 32 \cdot 8 + \frac 32 \cdot 1} = \frac{8}{43} \approx 0{,}186.
     \\
    \eta_\text{Карно} &= 1 - \frac{T_\text{х}}{T_\text{н}} = 1 - \frac{T_\text{4}}{T_\text{2}} = 1 - \frac{\frac{P_4V_4}{\nu R}}{\frac{P_2V_2}{\nu R}} = 1 - \frac{P_4V_4}{P_2V_2} = 1 - \frac{P_0V_0}{2P_0 \cdot 5V_0} = 1 - \frac 1{2 \cdot 5}  = \frac{9}{10} \approx 0{,}900.
    \end{align*}
}
\solutionspace{360pt}

\tasknumber{2}%
\task{%
    Порция идеального одноатомного газа перешла из состояния 1 в состояние 2: $P_1 = 4\,\text{МПа}$, $V_1 = 5\,\text{л}$, $P_2 = 4{,}5\,\text{МПа}$, $V_2 = 6\,\text{л}$.
    Определите, какую работу при этом совершил газ, чему равно изменение внутренней энергии газа, сколько теплоты подвели к нему в этом процессе?
    При решении обратите внимание на знаки искомых величин.
    Известно, что в PV-координатах график процесса 12 представляет собой отрезок прямой.
}
\answer{%
    \begin{align*}
    P_1V_1 &= \nu R T_1, P_2V_2 = \nu R T_2, \\
    \Delta U &= U_2-U_1 = \frac 32 \nu R T_2- \frac 32 \nu R T_1 = \frac 32 P_2 V_2 - \frac 32 P_1 V_1= \frac 32 \cdot \cbr{4{,}5\,\text{МПа} \cdot6\,\text{л} - 4\,\text{МПа} \cdot5\,\text{л}} = 10{,}50\,\text{кДж}.
    \\
    A_\text{газа} &= \frac{P_2 + P_1} 2 \cdot (V_2 - V_1) = \frac{4{,}5\,\text{МПа} + 4\,\text{МПа}} 2 \cdot (6\,\text{л} - 5\,\text{л}) = 4{,}25\,\text{кДж}, \\
    Q &= A_\text{газа} + \Delta U = \frac 32 (P_2 V_2 - P_1 V_1) + \frac{P_2 + P_1} 2 \cdot (V_2 - V_1) = 10{,}50\,\text{кДж} + 4{,}25\,\text{кДж} = 14{,}75\,\text{кДж}.
    \end{align*}
}
\solutionspace{150pt}

\tasknumber{3}%
\task{%
    Запишите формулы и рядом с каждой физичической величиной укажите её название и единицы изменения в СИ:
    \begin{enumerate}
        \item первое начало термодинамики,
        \item внутренняя энергия идеального одноатомного газа.
    \end{enumerate}
}

\variantsplitter

\addpersonalvariant{Владислав Емелин}

\tasknumber{1}%
\task{%
    Определите КПД цикла 12341, рабочим телом которого является идеальный одноатомный газ, если
    12 — изобарическое расширение газа в четыре раза,
    23 — изохорическое охлаждение газа, при котором температура уменьшается в два раза,
    34 — изобара, 41 — изохора.
    % Для этого:
    % \begin{enumerate}
    %     \item сделайте рисунок в PV-координатах,
    %     \item выберите удобные обозначения, чтобы не запутаться в множестве температур, давлений и объёмов,
    %     \item вычислите необходимые соотнощения между температурами, давлениями и объёмами
    %     (некоторые сразу видны по рисунку, некоторые — надо считать),
    %     \item определите для каждого участка поглощается или отдаётся тепло (и сколько именно:
    %     потребуется первое начало термодинамики, отдельный расчёт работ на участках через площади фигур и изменений внутренней энергии),
    %     \item вычислите полную работу газа в цикле,
    %     \item подставьте всё в формулу для КПД, упростите и доведите до ответа.
    % \end{enumerate}
    Определите КПД цикла Карно, температура нагревателя которого равна максимальной температуре в цикле 12341, а холодильника — минимальной.
    Ответы в обоих случаях оставьте точными в виде нескоратимой дроби, никаких округлений.
}
\answer{%
    \begin{align*}
    A_{12} &> 0, \Delta U_{12} > 0, \implies Q_{12} = A_{12} + \Delta U_{12} > 0, \\
    A_{23} &= 0, \Delta U_{23} < 0, \implies Q_{23} = A_{23} + \Delta U_{23} < 0, \\
    A_{34} &< 0, \Delta U_{34} < 0, \implies Q_{34} = A_{34} + \Delta U_{34} < 0, \\
    A_{41} &= 0, \Delta U_{41} > 0, \implies Q_{41} = A_{41} + \Delta U_{41} > 0.
    \\
    P_1V_1 &= \nu R T_1, P_2V_2 = \nu R T_2, P_3V_3 = \nu R T_3, P_4V_4 = \nu R T_4 \text{ — уравнения состояния идеального газа}, \\
    &\text{Пусть $P_0$, $V_0$, $T_0$ — давление, объём и температура в точке 4 (минимальные во всём цикле):} \\
    P_1 &= P_2, P_3 = P_4 = P_0, V_1 = V_4 = V_0, V_2 = V_3 = 4 V_1 = 4 V_0,, \text{остальные соотношения между объёмами и давлениями не даны, нужно считать} \\
    T_3 &= \frac{T_2}2 \text{(по условию)} \implies \frac{P_2}{P_3} = \frac{P_2 V_2}{P_3 V_3}= \frac{\nu R T_2}{\nu R T_3} = \frac{T_2}{T_3} = 2 \implies P_1 = P_2 = 2 P_0 \\
    A_\text{цикл} &= (2P_0 - P_0)(4V_0 - V_0) = 3P_0V_0, \\
    A_{12} &= 2P_0 \cdot (4V_0 - V_0) = 6P_0V_0, \\
    \Delta U_{12} &= \frac 32 \nu R T_2 - \frac 32 \nu R T_1 = \frac 32 P_2 V_2 - \frac 32 P_1 V_1 = \frac 32 \cdot 2 P_0 \cdot 4 V_0 -  \frac 32 \cdot 2 P_0 \cdot V_0 = \frac 32 \cdot 6 \cdot P_0V_0, \\
    \Delta U_{41} &= \frac 32 \nu R T_1 - \frac 32 \nu R T_4 = \frac 32 P_1 V_1 - \frac 32 P_4 V_4 = \frac 32 \cdot 2 P_0 V_0 - \frac 32 P_0 V_0 = \frac 32 \cdot 1 \cdot P_0V_0.
    \\
    \eta &= \frac{A_\text{цикл}}{Q_+} = \frac{A_\text{цикл}}{Q_{12} + Q_{41}}  = \frac{A_\text{цикл}}{A_{12} + \Delta U_{12} + A_{41} + \Delta U_{41}} =  \\
     &= \frac{3P_0V_0}{6P_0V_0 + \frac 32 \cdot 6 \cdot P_0V_0 + 0 + \frac 32 \cdot 1 \cdot P_0V_0} = \frac{3}{6 + \frac 32 \cdot 6 + \frac 32 \cdot 1} = \frac{2}{11} \approx 0{,}182.
     \\
    \eta_\text{Карно} &= 1 - \frac{T_\text{х}}{T_\text{н}} = 1 - \frac{T_\text{4}}{T_\text{2}} = 1 - \frac{\frac{P_4V_4}{\nu R}}{\frac{P_2V_2}{\nu R}} = 1 - \frac{P_4V_4}{P_2V_2} = 1 - \frac{P_0V_0}{2P_0 \cdot 4V_0} = 1 - \frac 1{2 \cdot 4}  = \frac{7}{8} \approx 0{,}875.
    \end{align*}
}
\solutionspace{360pt}

\tasknumber{2}%
\task{%
    Порция идеального одноатомного газа перешла из состояния 1 в состояние 2: $P_1 = 2\,\text{МПа}$, $V_1 = 5\,\text{л}$, $P_2 = 4{,}5\,\text{МПа}$, $V_2 = 2\,\text{л}$.
    Определите, какую работу при этом совершил газ, чему равно изменение внутренней энергии газа, сколько теплоты подвели к нему в этом процессе?
    При решении обратите внимание на знаки искомых величин.
    Известно, что в PV-координатах график процесса 12 представляет собой отрезок прямой.
}
\answer{%
    \begin{align*}
    P_1V_1 &= \nu R T_1, P_2V_2 = \nu R T_2, \\
    \Delta U &= U_2-U_1 = \frac 32 \nu R T_2- \frac 32 \nu R T_1 = \frac 32 P_2 V_2 - \frac 32 P_1 V_1= \frac 32 \cdot \cbr{4{,}5\,\text{МПа} \cdot2\,\text{л} - 2\,\text{МПа} \cdot5\,\text{л}} = -1{,}5000\,\text{кДж}.
    \\
    A_\text{газа} &= \frac{P_2 + P_1} 2 \cdot (V_2 - V_1) = \frac{4{,}5\,\text{МПа} + 2\,\text{МПа}} 2 \cdot (2\,\text{л} - 5\,\text{л}) = -9{,}750\,\text{кДж}, \\
    Q &= A_\text{газа} + \Delta U = \frac 32 (P_2 V_2 - P_1 V_1) + \frac{P_2 + P_1} 2 \cdot (V_2 - V_1) = -1{,}5000\,\text{кДж} -9{,}750\,\text{кДж} = -11{,}2500\,\text{кДж}.
    \end{align*}
}
\solutionspace{150pt}

\tasknumber{3}%
\task{%
    Запишите формулы и рядом с каждой физичической величиной укажите её название и единицы изменения в СИ:
    \begin{enumerate}
        \item первое начало термодинамики,
        \item внутренняя энергия идеального одноатомного газа.
    \end{enumerate}
}

\variantsplitter

\addpersonalvariant{Артём Жичин}

\tasknumber{1}%
\task{%
    Определите КПД цикла 12341, рабочим телом которого является идеальный одноатомный газ, если
    12 — изобарическое расширение газа в шесть раз,
    23 — изохорическое охлаждение газа, при котором температура уменьшается в шесть раз,
    34 — изобара, 41 — изохора.
    % Для этого:
    % \begin{enumerate}
    %     \item сделайте рисунок в PV-координатах,
    %     \item выберите удобные обозначения, чтобы не запутаться в множестве температур, давлений и объёмов,
    %     \item вычислите необходимые соотнощения между температурами, давлениями и объёмами
    %     (некоторые сразу видны по рисунку, некоторые — надо считать),
    %     \item определите для каждого участка поглощается или отдаётся тепло (и сколько именно:
    %     потребуется первое начало термодинамики, отдельный расчёт работ на участках через площади фигур и изменений внутренней энергии),
    %     \item вычислите полную работу газа в цикле,
    %     \item подставьте всё в формулу для КПД, упростите и доведите до ответа.
    % \end{enumerate}
    Определите КПД цикла Карно, температура нагревателя которого равна максимальной температуре в цикле 12341, а холодильника — минимальной.
    Ответы в обоих случаях оставьте точными в виде нескоратимой дроби, никаких округлений.
}
\answer{%
    \begin{align*}
    A_{12} &> 0, \Delta U_{12} > 0, \implies Q_{12} = A_{12} + \Delta U_{12} > 0, \\
    A_{23} &= 0, \Delta U_{23} < 0, \implies Q_{23} = A_{23} + \Delta U_{23} < 0, \\
    A_{34} &< 0, \Delta U_{34} < 0, \implies Q_{34} = A_{34} + \Delta U_{34} < 0, \\
    A_{41} &= 0, \Delta U_{41} > 0, \implies Q_{41} = A_{41} + \Delta U_{41} > 0.
    \\
    P_1V_1 &= \nu R T_1, P_2V_2 = \nu R T_2, P_3V_3 = \nu R T_3, P_4V_4 = \nu R T_4 \text{ — уравнения состояния идеального газа}, \\
    &\text{Пусть $P_0$, $V_0$, $T_0$ — давление, объём и температура в точке 4 (минимальные во всём цикле):} \\
    P_1 &= P_2, P_3 = P_4 = P_0, V_1 = V_4 = V_0, V_2 = V_3 = 6 V_1 = 6 V_0,, \text{остальные соотношения между объёмами и давлениями не даны, нужно считать} \\
    T_3 &= \frac{T_2}6 \text{(по условию)} \implies \frac{P_2}{P_3} = \frac{P_2 V_2}{P_3 V_3}= \frac{\nu R T_2}{\nu R T_3} = \frac{T_2}{T_3} = 6 \implies P_1 = P_2 = 6 P_0 \\
    A_\text{цикл} &= (6P_0 - P_0)(6V_0 - V_0) = 25P_0V_0, \\
    A_{12} &= 6P_0 \cdot (6V_0 - V_0) = 30P_0V_0, \\
    \Delta U_{12} &= \frac 32 \nu R T_2 - \frac 32 \nu R T_1 = \frac 32 P_2 V_2 - \frac 32 P_1 V_1 = \frac 32 \cdot 6 P_0 \cdot 6 V_0 -  \frac 32 \cdot 6 P_0 \cdot V_0 = \frac 32 \cdot 30 \cdot P_0V_0, \\
    \Delta U_{41} &= \frac 32 \nu R T_1 - \frac 32 \nu R T_4 = \frac 32 P_1 V_1 - \frac 32 P_4 V_4 = \frac 32 \cdot 6 P_0 V_0 - \frac 32 P_0 V_0 = \frac 32 \cdot 5 \cdot P_0V_0.
    \\
    \eta &= \frac{A_\text{цикл}}{Q_+} = \frac{A_\text{цикл}}{Q_{12} + Q_{41}}  = \frac{A_\text{цикл}}{A_{12} + \Delta U_{12} + A_{41} + \Delta U_{41}} =  \\
     &= \frac{25P_0V_0}{30P_0V_0 + \frac 32 \cdot 30 \cdot P_0V_0 + 0 + \frac 32 \cdot 5 \cdot P_0V_0} = \frac{25}{30 + \frac 32 \cdot 30 + \frac 32 \cdot 5} = \frac{10}{33} \approx 0{,}303.
     \\
    \eta_\text{Карно} &= 1 - \frac{T_\text{х}}{T_\text{н}} = 1 - \frac{T_\text{4}}{T_\text{2}} = 1 - \frac{\frac{P_4V_4}{\nu R}}{\frac{P_2V_2}{\nu R}} = 1 - \frac{P_4V_4}{P_2V_2} = 1 - \frac{P_0V_0}{6P_0 \cdot 6V_0} = 1 - \frac 1{6 \cdot 6}  = \frac{35}{36} \approx 0{,}972.
    \end{align*}
}
\solutionspace{360pt}

\tasknumber{2}%
\task{%
    Порция идеального одноатомного газа перешла из состояния 1 в состояние 2: $P_1 = 3\,\text{МПа}$, $V_1 = 5\,\text{л}$, $P_2 = 2{,}5\,\text{МПа}$, $V_2 = 2\,\text{л}$.
    Определите, какую работу при этом совершил газ, чему равно изменение внутренней энергии газа, сколько теплоты подвели к нему в этом процессе?
    При решении обратите внимание на знаки искомых величин.
    Известно, что в PV-координатах график процесса 12 представляет собой отрезок прямой.
}
\answer{%
    \begin{align*}
    P_1V_1 &= \nu R T_1, P_2V_2 = \nu R T_2, \\
    \Delta U &= U_2-U_1 = \frac 32 \nu R T_2- \frac 32 \nu R T_1 = \frac 32 P_2 V_2 - \frac 32 P_1 V_1= \frac 32 \cdot \cbr{2{,}5\,\text{МПа} \cdot2\,\text{л} - 3\,\text{МПа} \cdot5\,\text{л}} = -15{,}0000\,\text{кДж}.
    \\
    A_\text{газа} &= \frac{P_2 + P_1} 2 \cdot (V_2 - V_1) = \frac{2{,}5\,\text{МПа} + 3\,\text{МПа}} 2 \cdot (2\,\text{л} - 5\,\text{л}) = -8{,}250\,\text{кДж}, \\
    Q &= A_\text{газа} + \Delta U = \frac 32 (P_2 V_2 - P_1 V_1) + \frac{P_2 + P_1} 2 \cdot (V_2 - V_1) = -15{,}0000\,\text{кДж} -8{,}250\,\text{кДж} = -23{,}250\,\text{кДж}.
    \end{align*}
}
\solutionspace{150pt}

\tasknumber{3}%
\task{%
    Запишите формулы и рядом с каждой физичической величиной укажите её название и единицы изменения в СИ:
    \begin{enumerate}
        \item первое начало термодинамики,
        \item внутренняя энергия идеального одноатомного газа.
    \end{enumerate}
}

\variantsplitter

\addpersonalvariant{Дарья Кошман}

\tasknumber{1}%
\task{%
    Определите КПД цикла 12341, рабочим телом которого является идеальный одноатомный газ, если
    12 — изобарическое расширение газа в два раза,
    23 — изохорическое охлаждение газа, при котором температура уменьшается в два раза,
    34 — изобара, 41 — изохора.
    % Для этого:
    % \begin{enumerate}
    %     \item сделайте рисунок в PV-координатах,
    %     \item выберите удобные обозначения, чтобы не запутаться в множестве температур, давлений и объёмов,
    %     \item вычислите необходимые соотнощения между температурами, давлениями и объёмами
    %     (некоторые сразу видны по рисунку, некоторые — надо считать),
    %     \item определите для каждого участка поглощается или отдаётся тепло (и сколько именно:
    %     потребуется первое начало термодинамики, отдельный расчёт работ на участках через площади фигур и изменений внутренней энергии),
    %     \item вычислите полную работу газа в цикле,
    %     \item подставьте всё в формулу для КПД, упростите и доведите до ответа.
    % \end{enumerate}
    Определите КПД цикла Карно, температура нагревателя которого равна максимальной температуре в цикле 12341, а холодильника — минимальной.
    Ответы в обоих случаях оставьте точными в виде нескоратимой дроби, никаких округлений.
}
\answer{%
    \begin{align*}
    A_{12} &> 0, \Delta U_{12} > 0, \implies Q_{12} = A_{12} + \Delta U_{12} > 0, \\
    A_{23} &= 0, \Delta U_{23} < 0, \implies Q_{23} = A_{23} + \Delta U_{23} < 0, \\
    A_{34} &< 0, \Delta U_{34} < 0, \implies Q_{34} = A_{34} + \Delta U_{34} < 0, \\
    A_{41} &= 0, \Delta U_{41} > 0, \implies Q_{41} = A_{41} + \Delta U_{41} > 0.
    \\
    P_1V_1 &= \nu R T_1, P_2V_2 = \nu R T_2, P_3V_3 = \nu R T_3, P_4V_4 = \nu R T_4 \text{ — уравнения состояния идеального газа}, \\
    &\text{Пусть $P_0$, $V_0$, $T_0$ — давление, объём и температура в точке 4 (минимальные во всём цикле):} \\
    P_1 &= P_2, P_3 = P_4 = P_0, V_1 = V_4 = V_0, V_2 = V_3 = 2 V_1 = 2 V_0,, \text{остальные соотношения между объёмами и давлениями не даны, нужно считать} \\
    T_3 &= \frac{T_2}2 \text{(по условию)} \implies \frac{P_2}{P_3} = \frac{P_2 V_2}{P_3 V_3}= \frac{\nu R T_2}{\nu R T_3} = \frac{T_2}{T_3} = 2 \implies P_1 = P_2 = 2 P_0 \\
    A_\text{цикл} &= (2P_0 - P_0)(2V_0 - V_0) = 1P_0V_0, \\
    A_{12} &= 2P_0 \cdot (2V_0 - V_0) = 2P_0V_0, \\
    \Delta U_{12} &= \frac 32 \nu R T_2 - \frac 32 \nu R T_1 = \frac 32 P_2 V_2 - \frac 32 P_1 V_1 = \frac 32 \cdot 2 P_0 \cdot 2 V_0 -  \frac 32 \cdot 2 P_0 \cdot V_0 = \frac 32 \cdot 2 \cdot P_0V_0, \\
    \Delta U_{41} &= \frac 32 \nu R T_1 - \frac 32 \nu R T_4 = \frac 32 P_1 V_1 - \frac 32 P_4 V_4 = \frac 32 \cdot 2 P_0 V_0 - \frac 32 P_0 V_0 = \frac 32 \cdot 1 \cdot P_0V_0.
    \\
    \eta &= \frac{A_\text{цикл}}{Q_+} = \frac{A_\text{цикл}}{Q_{12} + Q_{41}}  = \frac{A_\text{цикл}}{A_{12} + \Delta U_{12} + A_{41} + \Delta U_{41}} =  \\
     &= \frac{1P_0V_0}{2P_0V_0 + \frac 32 \cdot 2 \cdot P_0V_0 + 0 + \frac 32 \cdot 1 \cdot P_0V_0} = \frac{1}{2 + \frac 32 \cdot 2 + \frac 32 \cdot 1} = \frac{2}{13} \approx 0{,}154.
     \\
    \eta_\text{Карно} &= 1 - \frac{T_\text{х}}{T_\text{н}} = 1 - \frac{T_\text{4}}{T_\text{2}} = 1 - \frac{\frac{P_4V_4}{\nu R}}{\frac{P_2V_2}{\nu R}} = 1 - \frac{P_4V_4}{P_2V_2} = 1 - \frac{P_0V_0}{2P_0 \cdot 2V_0} = 1 - \frac 1{2 \cdot 2}  = \frac{3}{4} \approx 0{,}750.
    \end{align*}
}
\solutionspace{360pt}

\tasknumber{2}%
\task{%
    Порция идеального одноатомного газа перешла из состояния 1 в состояние 2: $P_1 = 4\,\text{МПа}$, $V_1 = 5\,\text{л}$, $P_2 = 4{,}5\,\text{МПа}$, $V_2 = 2\,\text{л}$.
    Определите, какую работу при этом совершил газ, чему равно изменение внутренней энергии газа, сколько теплоты подвели к нему в этом процессе?
    При решении обратите внимание на знаки искомых величин.
    Известно, что в PV-координатах график процесса 12 представляет собой отрезок прямой.
}
\answer{%
    \begin{align*}
    P_1V_1 &= \nu R T_1, P_2V_2 = \nu R T_2, \\
    \Delta U &= U_2-U_1 = \frac 32 \nu R T_2- \frac 32 \nu R T_1 = \frac 32 P_2 V_2 - \frac 32 P_1 V_1= \frac 32 \cdot \cbr{4{,}5\,\text{МПа} \cdot2\,\text{л} - 4\,\text{МПа} \cdot5\,\text{л}} = -16{,}5000\,\text{кДж}.
    \\
    A_\text{газа} &= \frac{P_2 + P_1} 2 \cdot (V_2 - V_1) = \frac{4{,}5\,\text{МПа} + 4\,\text{МПа}} 2 \cdot (2\,\text{л} - 5\,\text{л}) = -12{,}7500\,\text{кДж}, \\
    Q &= A_\text{газа} + \Delta U = \frac 32 (P_2 V_2 - P_1 V_1) + \frac{P_2 + P_1} 2 \cdot (V_2 - V_1) = -16{,}5000\,\text{кДж} -12{,}7500\,\text{кДж} = -29{,}250\,\text{кДж}.
    \end{align*}
}
\solutionspace{150pt}

\tasknumber{3}%
\task{%
    Запишите формулы и рядом с каждой физичической величиной укажите её название и единицы изменения в СИ:
    \begin{enumerate}
        \item первое начало термодинамики,
        \item внутренняя энергия идеального одноатомного газа.
    \end{enumerate}
}

\variantsplitter

\addpersonalvariant{Анна Кузьмичёва}

\tasknumber{1}%
\task{%
    Определите КПД цикла 12341, рабочим телом которого является идеальный одноатомный газ, если
    12 — изобарическое расширение газа в четыре раза,
    23 — изохорическое охлаждение газа, при котором температура уменьшается в пять раз,
    34 — изобара, 41 — изохора.
    % Для этого:
    % \begin{enumerate}
    %     \item сделайте рисунок в PV-координатах,
    %     \item выберите удобные обозначения, чтобы не запутаться в множестве температур, давлений и объёмов,
    %     \item вычислите необходимые соотнощения между температурами, давлениями и объёмами
    %     (некоторые сразу видны по рисунку, некоторые — надо считать),
    %     \item определите для каждого участка поглощается или отдаётся тепло (и сколько именно:
    %     потребуется первое начало термодинамики, отдельный расчёт работ на участках через площади фигур и изменений внутренней энергии),
    %     \item вычислите полную работу газа в цикле,
    %     \item подставьте всё в формулу для КПД, упростите и доведите до ответа.
    % \end{enumerate}
    Определите КПД цикла Карно, температура нагревателя которого равна максимальной температуре в цикле 12341, а холодильника — минимальной.
    Ответы в обоих случаях оставьте точными в виде нескоратимой дроби, никаких округлений.
}
\answer{%
    \begin{align*}
    A_{12} &> 0, \Delta U_{12} > 0, \implies Q_{12} = A_{12} + \Delta U_{12} > 0, \\
    A_{23} &= 0, \Delta U_{23} < 0, \implies Q_{23} = A_{23} + \Delta U_{23} < 0, \\
    A_{34} &< 0, \Delta U_{34} < 0, \implies Q_{34} = A_{34} + \Delta U_{34} < 0, \\
    A_{41} &= 0, \Delta U_{41} > 0, \implies Q_{41} = A_{41} + \Delta U_{41} > 0.
    \\
    P_1V_1 &= \nu R T_1, P_2V_2 = \nu R T_2, P_3V_3 = \nu R T_3, P_4V_4 = \nu R T_4 \text{ — уравнения состояния идеального газа}, \\
    &\text{Пусть $P_0$, $V_0$, $T_0$ — давление, объём и температура в точке 4 (минимальные во всём цикле):} \\
    P_1 &= P_2, P_3 = P_4 = P_0, V_1 = V_4 = V_0, V_2 = V_3 = 4 V_1 = 4 V_0,, \text{остальные соотношения между объёмами и давлениями не даны, нужно считать} \\
    T_3 &= \frac{T_2}5 \text{(по условию)} \implies \frac{P_2}{P_3} = \frac{P_2 V_2}{P_3 V_3}= \frac{\nu R T_2}{\nu R T_3} = \frac{T_2}{T_3} = 5 \implies P_1 = P_2 = 5 P_0 \\
    A_\text{цикл} &= (5P_0 - P_0)(4V_0 - V_0) = 12P_0V_0, \\
    A_{12} &= 5P_0 \cdot (4V_0 - V_0) = 15P_0V_0, \\
    \Delta U_{12} &= \frac 32 \nu R T_2 - \frac 32 \nu R T_1 = \frac 32 P_2 V_2 - \frac 32 P_1 V_1 = \frac 32 \cdot 5 P_0 \cdot 4 V_0 -  \frac 32 \cdot 5 P_0 \cdot V_0 = \frac 32 \cdot 15 \cdot P_0V_0, \\
    \Delta U_{41} &= \frac 32 \nu R T_1 - \frac 32 \nu R T_4 = \frac 32 P_1 V_1 - \frac 32 P_4 V_4 = \frac 32 \cdot 5 P_0 V_0 - \frac 32 P_0 V_0 = \frac 32 \cdot 4 \cdot P_0V_0.
    \\
    \eta &= \frac{A_\text{цикл}}{Q_+} = \frac{A_\text{цикл}}{Q_{12} + Q_{41}}  = \frac{A_\text{цикл}}{A_{12} + \Delta U_{12} + A_{41} + \Delta U_{41}} =  \\
     &= \frac{12P_0V_0}{15P_0V_0 + \frac 32 \cdot 15 \cdot P_0V_0 + 0 + \frac 32 \cdot 4 \cdot P_0V_0} = \frac{12}{15 + \frac 32 \cdot 15 + \frac 32 \cdot 4} = \frac{8}{29} \approx 0{,}276.
     \\
    \eta_\text{Карно} &= 1 - \frac{T_\text{х}}{T_\text{н}} = 1 - \frac{T_\text{4}}{T_\text{2}} = 1 - \frac{\frac{P_4V_4}{\nu R}}{\frac{P_2V_2}{\nu R}} = 1 - \frac{P_4V_4}{P_2V_2} = 1 - \frac{P_0V_0}{5P_0 \cdot 4V_0} = 1 - \frac 1{5 \cdot 4}  = \frac{19}{20} \approx 0{,}950.
    \end{align*}
}
\solutionspace{360pt}

\tasknumber{2}%
\task{%
    Порция идеального одноатомного газа перешла из состояния 1 в состояние 2: $P_1 = 2\,\text{МПа}$, $V_1 = 5\,\text{л}$, $P_2 = 2{,}5\,\text{МПа}$, $V_2 = 8\,\text{л}$.
    Определите, какую работу при этом совершил газ, чему равно изменение внутренней энергии газа, сколько теплоты подвели к нему в этом процессе?
    При решении обратите внимание на знаки искомых величин.
    Известно, что в PV-координатах график процесса 12 представляет собой отрезок прямой.
}
\answer{%
    \begin{align*}
    P_1V_1 &= \nu R T_1, P_2V_2 = \nu R T_2, \\
    \Delta U &= U_2-U_1 = \frac 32 \nu R T_2- \frac 32 \nu R T_1 = \frac 32 P_2 V_2 - \frac 32 P_1 V_1= \frac 32 \cdot \cbr{2{,}5\,\text{МПа} \cdot8\,\text{л} - 2\,\text{МПа} \cdot5\,\text{л}} = 15{,}00\,\text{кДж}.
    \\
    A_\text{газа} &= \frac{P_2 + P_1} 2 \cdot (V_2 - V_1) = \frac{2{,}5\,\text{МПа} + 2\,\text{МПа}} 2 \cdot (8\,\text{л} - 5\,\text{л}) = 6{,}75\,\text{кДж}, \\
    Q &= A_\text{газа} + \Delta U = \frac 32 (P_2 V_2 - P_1 V_1) + \frac{P_2 + P_1} 2 \cdot (V_2 - V_1) = 15{,}00\,\text{кДж} + 6{,}75\,\text{кДж} = 21{,}75\,\text{кДж}.
    \end{align*}
}
\solutionspace{150pt}

\tasknumber{3}%
\task{%
    Запишите формулы и рядом с каждой физичической величиной укажите её название и единицы изменения в СИ:
    \begin{enumerate}
        \item первое начало термодинамики,
        \item внутренняя энергия идеального одноатомного газа.
    \end{enumerate}
}

\variantsplitter

\addpersonalvariant{Алёна Куприянова}

\tasknumber{1}%
\task{%
    Определите КПД цикла 12341, рабочим телом которого является идеальный одноатомный газ, если
    12 — изобарическое расширение газа в четыре раза,
    23 — изохорическое охлаждение газа, при котором температура уменьшается в пять раз,
    34 — изобара, 41 — изохора.
    % Для этого:
    % \begin{enumerate}
    %     \item сделайте рисунок в PV-координатах,
    %     \item выберите удобные обозначения, чтобы не запутаться в множестве температур, давлений и объёмов,
    %     \item вычислите необходимые соотнощения между температурами, давлениями и объёмами
    %     (некоторые сразу видны по рисунку, некоторые — надо считать),
    %     \item определите для каждого участка поглощается или отдаётся тепло (и сколько именно:
    %     потребуется первое начало термодинамики, отдельный расчёт работ на участках через площади фигур и изменений внутренней энергии),
    %     \item вычислите полную работу газа в цикле,
    %     \item подставьте всё в формулу для КПД, упростите и доведите до ответа.
    % \end{enumerate}
    Определите КПД цикла Карно, температура нагревателя которого равна максимальной температуре в цикле 12341, а холодильника — минимальной.
    Ответы в обоих случаях оставьте точными в виде нескоратимой дроби, никаких округлений.
}
\answer{%
    \begin{align*}
    A_{12} &> 0, \Delta U_{12} > 0, \implies Q_{12} = A_{12} + \Delta U_{12} > 0, \\
    A_{23} &= 0, \Delta U_{23} < 0, \implies Q_{23} = A_{23} + \Delta U_{23} < 0, \\
    A_{34} &< 0, \Delta U_{34} < 0, \implies Q_{34} = A_{34} + \Delta U_{34} < 0, \\
    A_{41} &= 0, \Delta U_{41} > 0, \implies Q_{41} = A_{41} + \Delta U_{41} > 0.
    \\
    P_1V_1 &= \nu R T_1, P_2V_2 = \nu R T_2, P_3V_3 = \nu R T_3, P_4V_4 = \nu R T_4 \text{ — уравнения состояния идеального газа}, \\
    &\text{Пусть $P_0$, $V_0$, $T_0$ — давление, объём и температура в точке 4 (минимальные во всём цикле):} \\
    P_1 &= P_2, P_3 = P_4 = P_0, V_1 = V_4 = V_0, V_2 = V_3 = 4 V_1 = 4 V_0,, \text{остальные соотношения между объёмами и давлениями не даны, нужно считать} \\
    T_3 &= \frac{T_2}5 \text{(по условию)} \implies \frac{P_2}{P_3} = \frac{P_2 V_2}{P_3 V_3}= \frac{\nu R T_2}{\nu R T_3} = \frac{T_2}{T_3} = 5 \implies P_1 = P_2 = 5 P_0 \\
    A_\text{цикл} &= (5P_0 - P_0)(4V_0 - V_0) = 12P_0V_0, \\
    A_{12} &= 5P_0 \cdot (4V_0 - V_0) = 15P_0V_0, \\
    \Delta U_{12} &= \frac 32 \nu R T_2 - \frac 32 \nu R T_1 = \frac 32 P_2 V_2 - \frac 32 P_1 V_1 = \frac 32 \cdot 5 P_0 \cdot 4 V_0 -  \frac 32 \cdot 5 P_0 \cdot V_0 = \frac 32 \cdot 15 \cdot P_0V_0, \\
    \Delta U_{41} &= \frac 32 \nu R T_1 - \frac 32 \nu R T_4 = \frac 32 P_1 V_1 - \frac 32 P_4 V_4 = \frac 32 \cdot 5 P_0 V_0 - \frac 32 P_0 V_0 = \frac 32 \cdot 4 \cdot P_0V_0.
    \\
    \eta &= \frac{A_\text{цикл}}{Q_+} = \frac{A_\text{цикл}}{Q_{12} + Q_{41}}  = \frac{A_\text{цикл}}{A_{12} + \Delta U_{12} + A_{41} + \Delta U_{41}} =  \\
     &= \frac{12P_0V_0}{15P_0V_0 + \frac 32 \cdot 15 \cdot P_0V_0 + 0 + \frac 32 \cdot 4 \cdot P_0V_0} = \frac{12}{15 + \frac 32 \cdot 15 + \frac 32 \cdot 4} = \frac{8}{29} \approx 0{,}276.
     \\
    \eta_\text{Карно} &= 1 - \frac{T_\text{х}}{T_\text{н}} = 1 - \frac{T_\text{4}}{T_\text{2}} = 1 - \frac{\frac{P_4V_4}{\nu R}}{\frac{P_2V_2}{\nu R}} = 1 - \frac{P_4V_4}{P_2V_2} = 1 - \frac{P_0V_0}{5P_0 \cdot 4V_0} = 1 - \frac 1{5 \cdot 4}  = \frac{19}{20} \approx 0{,}950.
    \end{align*}
}
\solutionspace{360pt}

\tasknumber{2}%
\task{%
    Порция идеального одноатомного газа перешла из состояния 1 в состояние 2: $P_1 = 2\,\text{МПа}$, $V_1 = 7\,\text{л}$, $P_2 = 3{,}5\,\text{МПа}$, $V_2 = 4\,\text{л}$.
    Определите, какую работу при этом совершил газ, чему равно изменение внутренней энергии газа, сколько теплоты подвели к нему в этом процессе?
    При решении обратите внимание на знаки искомых величин.
    Известно, что в PV-координатах график процесса 12 представляет собой отрезок прямой.
}
\answer{%
    \begin{align*}
    P_1V_1 &= \nu R T_1, P_2V_2 = \nu R T_2, \\
    \Delta U &= U_2-U_1 = \frac 32 \nu R T_2- \frac 32 \nu R T_1 = \frac 32 P_2 V_2 - \frac 32 P_1 V_1= \frac 32 \cdot \cbr{3{,}5\,\text{МПа} \cdot4\,\text{л} - 2\,\text{МПа} \cdot7\,\text{л}} = 0\,\text{кДж}.
    \\
    A_\text{газа} &= \frac{P_2 + P_1} 2 \cdot (V_2 - V_1) = \frac{3{,}5\,\text{МПа} + 2\,\text{МПа}} 2 \cdot (4\,\text{л} - 7\,\text{л}) = -8{,}250\,\text{кДж}, \\
    Q &= A_\text{газа} + \Delta U = \frac 32 (P_2 V_2 - P_1 V_1) + \frac{P_2 + P_1} 2 \cdot (V_2 - V_1) = 0\,\text{кДж} -8{,}250\,\text{кДж} = -8{,}250\,\text{кДж}.
    \end{align*}
}
\solutionspace{150pt}

\tasknumber{3}%
\task{%
    Запишите формулы и рядом с каждой физичической величиной укажите её название и единицы изменения в СИ:
    \begin{enumerate}
        \item первое начало термодинамики,
        \item внутренняя энергия идеального одноатомного газа.
    \end{enumerate}
}

\variantsplitter

\addpersonalvariant{Ярослав Лавровский}

\tasknumber{1}%
\task{%
    Определите КПД цикла 12341, рабочим телом которого является идеальный одноатомный газ, если
    12 — изобарическое расширение газа в пять раз,
    23 — изохорическое охлаждение газа, при котором температура уменьшается в четыре раза,
    34 — изобара, 41 — изохора.
    % Для этого:
    % \begin{enumerate}
    %     \item сделайте рисунок в PV-координатах,
    %     \item выберите удобные обозначения, чтобы не запутаться в множестве температур, давлений и объёмов,
    %     \item вычислите необходимые соотнощения между температурами, давлениями и объёмами
    %     (некоторые сразу видны по рисунку, некоторые — надо считать),
    %     \item определите для каждого участка поглощается или отдаётся тепло (и сколько именно:
    %     потребуется первое начало термодинамики, отдельный расчёт работ на участках через площади фигур и изменений внутренней энергии),
    %     \item вычислите полную работу газа в цикле,
    %     \item подставьте всё в формулу для КПД, упростите и доведите до ответа.
    % \end{enumerate}
    Определите КПД цикла Карно, температура нагревателя которого равна максимальной температуре в цикле 12341, а холодильника — минимальной.
    Ответы в обоих случаях оставьте точными в виде нескоратимой дроби, никаких округлений.
}
\answer{%
    \begin{align*}
    A_{12} &> 0, \Delta U_{12} > 0, \implies Q_{12} = A_{12} + \Delta U_{12} > 0, \\
    A_{23} &= 0, \Delta U_{23} < 0, \implies Q_{23} = A_{23} + \Delta U_{23} < 0, \\
    A_{34} &< 0, \Delta U_{34} < 0, \implies Q_{34} = A_{34} + \Delta U_{34} < 0, \\
    A_{41} &= 0, \Delta U_{41} > 0, \implies Q_{41} = A_{41} + \Delta U_{41} > 0.
    \\
    P_1V_1 &= \nu R T_1, P_2V_2 = \nu R T_2, P_3V_3 = \nu R T_3, P_4V_4 = \nu R T_4 \text{ — уравнения состояния идеального газа}, \\
    &\text{Пусть $P_0$, $V_0$, $T_0$ — давление, объём и температура в точке 4 (минимальные во всём цикле):} \\
    P_1 &= P_2, P_3 = P_4 = P_0, V_1 = V_4 = V_0, V_2 = V_3 = 5 V_1 = 5 V_0,, \text{остальные соотношения между объёмами и давлениями не даны, нужно считать} \\
    T_3 &= \frac{T_2}4 \text{(по условию)} \implies \frac{P_2}{P_3} = \frac{P_2 V_2}{P_3 V_3}= \frac{\nu R T_2}{\nu R T_3} = \frac{T_2}{T_3} = 4 \implies P_1 = P_2 = 4 P_0 \\
    A_\text{цикл} &= (4P_0 - P_0)(5V_0 - V_0) = 12P_0V_0, \\
    A_{12} &= 4P_0 \cdot (5V_0 - V_0) = 16P_0V_0, \\
    \Delta U_{12} &= \frac 32 \nu R T_2 - \frac 32 \nu R T_1 = \frac 32 P_2 V_2 - \frac 32 P_1 V_1 = \frac 32 \cdot 4 P_0 \cdot 5 V_0 -  \frac 32 \cdot 4 P_0 \cdot V_0 = \frac 32 \cdot 16 \cdot P_0V_0, \\
    \Delta U_{41} &= \frac 32 \nu R T_1 - \frac 32 \nu R T_4 = \frac 32 P_1 V_1 - \frac 32 P_4 V_4 = \frac 32 \cdot 4 P_0 V_0 - \frac 32 P_0 V_0 = \frac 32 \cdot 3 \cdot P_0V_0.
    \\
    \eta &= \frac{A_\text{цикл}}{Q_+} = \frac{A_\text{цикл}}{Q_{12} + Q_{41}}  = \frac{A_\text{цикл}}{A_{12} + \Delta U_{12} + A_{41} + \Delta U_{41}} =  \\
     &= \frac{12P_0V_0}{16P_0V_0 + \frac 32 \cdot 16 \cdot P_0V_0 + 0 + \frac 32 \cdot 3 \cdot P_0V_0} = \frac{12}{16 + \frac 32 \cdot 16 + \frac 32 \cdot 3} = \frac{24}{89} \approx 0{,}270.
     \\
    \eta_\text{Карно} &= 1 - \frac{T_\text{х}}{T_\text{н}} = 1 - \frac{T_\text{4}}{T_\text{2}} = 1 - \frac{\frac{P_4V_4}{\nu R}}{\frac{P_2V_2}{\nu R}} = 1 - \frac{P_4V_4}{P_2V_2} = 1 - \frac{P_0V_0}{4P_0 \cdot 5V_0} = 1 - \frac 1{4 \cdot 5}  = \frac{19}{20} \approx 0{,}950.
    \end{align*}
}
\solutionspace{360pt}

\tasknumber{2}%
\task{%
    Порция идеального одноатомного газа перешла из состояния 1 в состояние 2: $P_1 = 2\,\text{МПа}$, $V_1 = 5\,\text{л}$, $P_2 = 4{,}5\,\text{МПа}$, $V_2 = 2\,\text{л}$.
    Определите, какую работу при этом совершил газ, чему равно изменение внутренней энергии газа, сколько теплоты подвели к нему в этом процессе?
    При решении обратите внимание на знаки искомых величин.
    Известно, что в PV-координатах график процесса 12 представляет собой отрезок прямой.
}
\answer{%
    \begin{align*}
    P_1V_1 &= \nu R T_1, P_2V_2 = \nu R T_2, \\
    \Delta U &= U_2-U_1 = \frac 32 \nu R T_2- \frac 32 \nu R T_1 = \frac 32 P_2 V_2 - \frac 32 P_1 V_1= \frac 32 \cdot \cbr{4{,}5\,\text{МПа} \cdot2\,\text{л} - 2\,\text{МПа} \cdot5\,\text{л}} = -1{,}5000\,\text{кДж}.
    \\
    A_\text{газа} &= \frac{P_2 + P_1} 2 \cdot (V_2 - V_1) = \frac{4{,}5\,\text{МПа} + 2\,\text{МПа}} 2 \cdot (2\,\text{л} - 5\,\text{л}) = -9{,}750\,\text{кДж}, \\
    Q &= A_\text{газа} + \Delta U = \frac 32 (P_2 V_2 - P_1 V_1) + \frac{P_2 + P_1} 2 \cdot (V_2 - V_1) = -1{,}5000\,\text{кДж} -9{,}750\,\text{кДж} = -11{,}2500\,\text{кДж}.
    \end{align*}
}
\solutionspace{150pt}

\tasknumber{3}%
\task{%
    Запишите формулы и рядом с каждой физичической величиной укажите её название и единицы изменения в СИ:
    \begin{enumerate}
        \item первое начало термодинамики,
        \item внутренняя энергия идеального одноатомного газа.
    \end{enumerate}
}

\variantsplitter

\addpersonalvariant{Анастасия Ламанова}

\tasknumber{1}%
\task{%
    Определите КПД цикла 12341, рабочим телом которого является идеальный одноатомный газ, если
    12 — изобарическое расширение газа в шесть раз,
    23 — изохорическое охлаждение газа, при котором температура уменьшается в два раза,
    34 — изобара, 41 — изохора.
    % Для этого:
    % \begin{enumerate}
    %     \item сделайте рисунок в PV-координатах,
    %     \item выберите удобные обозначения, чтобы не запутаться в множестве температур, давлений и объёмов,
    %     \item вычислите необходимые соотнощения между температурами, давлениями и объёмами
    %     (некоторые сразу видны по рисунку, некоторые — надо считать),
    %     \item определите для каждого участка поглощается или отдаётся тепло (и сколько именно:
    %     потребуется первое начало термодинамики, отдельный расчёт работ на участках через площади фигур и изменений внутренней энергии),
    %     \item вычислите полную работу газа в цикле,
    %     \item подставьте всё в формулу для КПД, упростите и доведите до ответа.
    % \end{enumerate}
    Определите КПД цикла Карно, температура нагревателя которого равна максимальной температуре в цикле 12341, а холодильника — минимальной.
    Ответы в обоих случаях оставьте точными в виде нескоратимой дроби, никаких округлений.
}
\answer{%
    \begin{align*}
    A_{12} &> 0, \Delta U_{12} > 0, \implies Q_{12} = A_{12} + \Delta U_{12} > 0, \\
    A_{23} &= 0, \Delta U_{23} < 0, \implies Q_{23} = A_{23} + \Delta U_{23} < 0, \\
    A_{34} &< 0, \Delta U_{34} < 0, \implies Q_{34} = A_{34} + \Delta U_{34} < 0, \\
    A_{41} &= 0, \Delta U_{41} > 0, \implies Q_{41} = A_{41} + \Delta U_{41} > 0.
    \\
    P_1V_1 &= \nu R T_1, P_2V_2 = \nu R T_2, P_3V_3 = \nu R T_3, P_4V_4 = \nu R T_4 \text{ — уравнения состояния идеального газа}, \\
    &\text{Пусть $P_0$, $V_0$, $T_0$ — давление, объём и температура в точке 4 (минимальные во всём цикле):} \\
    P_1 &= P_2, P_3 = P_4 = P_0, V_1 = V_4 = V_0, V_2 = V_3 = 6 V_1 = 6 V_0,, \text{остальные соотношения между объёмами и давлениями не даны, нужно считать} \\
    T_3 &= \frac{T_2}2 \text{(по условию)} \implies \frac{P_2}{P_3} = \frac{P_2 V_2}{P_3 V_3}= \frac{\nu R T_2}{\nu R T_3} = \frac{T_2}{T_3} = 2 \implies P_1 = P_2 = 2 P_0 \\
    A_\text{цикл} &= (2P_0 - P_0)(6V_0 - V_0) = 5P_0V_0, \\
    A_{12} &= 2P_0 \cdot (6V_0 - V_0) = 10P_0V_0, \\
    \Delta U_{12} &= \frac 32 \nu R T_2 - \frac 32 \nu R T_1 = \frac 32 P_2 V_2 - \frac 32 P_1 V_1 = \frac 32 \cdot 2 P_0 \cdot 6 V_0 -  \frac 32 \cdot 2 P_0 \cdot V_0 = \frac 32 \cdot 10 \cdot P_0V_0, \\
    \Delta U_{41} &= \frac 32 \nu R T_1 - \frac 32 \nu R T_4 = \frac 32 P_1 V_1 - \frac 32 P_4 V_4 = \frac 32 \cdot 2 P_0 V_0 - \frac 32 P_0 V_0 = \frac 32 \cdot 1 \cdot P_0V_0.
    \\
    \eta &= \frac{A_\text{цикл}}{Q_+} = \frac{A_\text{цикл}}{Q_{12} + Q_{41}}  = \frac{A_\text{цикл}}{A_{12} + \Delta U_{12} + A_{41} + \Delta U_{41}} =  \\
     &= \frac{5P_0V_0}{10P_0V_0 + \frac 32 \cdot 10 \cdot P_0V_0 + 0 + \frac 32 \cdot 1 \cdot P_0V_0} = \frac{5}{10 + \frac 32 \cdot 10 + \frac 32 \cdot 1} = \frac{10}{53} \approx 0{,}189.
     \\
    \eta_\text{Карно} &= 1 - \frac{T_\text{х}}{T_\text{н}} = 1 - \frac{T_\text{4}}{T_\text{2}} = 1 - \frac{\frac{P_4V_4}{\nu R}}{\frac{P_2V_2}{\nu R}} = 1 - \frac{P_4V_4}{P_2V_2} = 1 - \frac{P_0V_0}{2P_0 \cdot 6V_0} = 1 - \frac 1{2 \cdot 6}  = \frac{11}{12} \approx 0{,}917.
    \end{align*}
}
\solutionspace{360pt}

\tasknumber{2}%
\task{%
    Порция идеального одноатомного газа перешла из состояния 1 в состояние 2: $P_1 = 3\,\text{МПа}$, $V_1 = 5\,\text{л}$, $P_2 = 3{,}5\,\text{МПа}$, $V_2 = 6\,\text{л}$.
    Определите, какую работу при этом совершил газ, чему равно изменение внутренней энергии газа, сколько теплоты подвели к нему в этом процессе?
    При решении обратите внимание на знаки искомых величин.
    Известно, что в PV-координатах график процесса 12 представляет собой отрезок прямой.
}
\answer{%
    \begin{align*}
    P_1V_1 &= \nu R T_1, P_2V_2 = \nu R T_2, \\
    \Delta U &= U_2-U_1 = \frac 32 \nu R T_2- \frac 32 \nu R T_1 = \frac 32 P_2 V_2 - \frac 32 P_1 V_1= \frac 32 \cdot \cbr{3{,}5\,\text{МПа} \cdot6\,\text{л} - 3\,\text{МПа} \cdot5\,\text{л}} = 9{,}00\,\text{кДж}.
    \\
    A_\text{газа} &= \frac{P_2 + P_1} 2 \cdot (V_2 - V_1) = \frac{3{,}5\,\text{МПа} + 3\,\text{МПа}} 2 \cdot (6\,\text{л} - 5\,\text{л}) = 3{,}25\,\text{кДж}, \\
    Q &= A_\text{газа} + \Delta U = \frac 32 (P_2 V_2 - P_1 V_1) + \frac{P_2 + P_1} 2 \cdot (V_2 - V_1) = 9{,}00\,\text{кДж} + 3{,}25\,\text{кДж} = 12{,}25\,\text{кДж}.
    \end{align*}
}
\solutionspace{150pt}

\tasknumber{3}%
\task{%
    Запишите формулы и рядом с каждой физичической величиной укажите её название и единицы изменения в СИ:
    \begin{enumerate}
        \item первое начало термодинамики,
        \item внутренняя энергия идеального одноатомного газа.
    \end{enumerate}
}

\variantsplitter

\addpersonalvariant{Виктория Легонькова}

\tasknumber{1}%
\task{%
    Определите КПД цикла 12341, рабочим телом которого является идеальный одноатомный газ, если
    12 — изобарическое расширение газа в три раза,
    23 — изохорическое охлаждение газа, при котором температура уменьшается в четыре раза,
    34 — изобара, 41 — изохора.
    % Для этого:
    % \begin{enumerate}
    %     \item сделайте рисунок в PV-координатах,
    %     \item выберите удобные обозначения, чтобы не запутаться в множестве температур, давлений и объёмов,
    %     \item вычислите необходимые соотнощения между температурами, давлениями и объёмами
    %     (некоторые сразу видны по рисунку, некоторые — надо считать),
    %     \item определите для каждого участка поглощается или отдаётся тепло (и сколько именно:
    %     потребуется первое начало термодинамики, отдельный расчёт работ на участках через площади фигур и изменений внутренней энергии),
    %     \item вычислите полную работу газа в цикле,
    %     \item подставьте всё в формулу для КПД, упростите и доведите до ответа.
    % \end{enumerate}
    Определите КПД цикла Карно, температура нагревателя которого равна максимальной температуре в цикле 12341, а холодильника — минимальной.
    Ответы в обоих случаях оставьте точными в виде нескоратимой дроби, никаких округлений.
}
\answer{%
    \begin{align*}
    A_{12} &> 0, \Delta U_{12} > 0, \implies Q_{12} = A_{12} + \Delta U_{12} > 0, \\
    A_{23} &= 0, \Delta U_{23} < 0, \implies Q_{23} = A_{23} + \Delta U_{23} < 0, \\
    A_{34} &< 0, \Delta U_{34} < 0, \implies Q_{34} = A_{34} + \Delta U_{34} < 0, \\
    A_{41} &= 0, \Delta U_{41} > 0, \implies Q_{41} = A_{41} + \Delta U_{41} > 0.
    \\
    P_1V_1 &= \nu R T_1, P_2V_2 = \nu R T_2, P_3V_3 = \nu R T_3, P_4V_4 = \nu R T_4 \text{ — уравнения состояния идеального газа}, \\
    &\text{Пусть $P_0$, $V_0$, $T_0$ — давление, объём и температура в точке 4 (минимальные во всём цикле):} \\
    P_1 &= P_2, P_3 = P_4 = P_0, V_1 = V_4 = V_0, V_2 = V_3 = 3 V_1 = 3 V_0,, \text{остальные соотношения между объёмами и давлениями не даны, нужно считать} \\
    T_3 &= \frac{T_2}4 \text{(по условию)} \implies \frac{P_2}{P_3} = \frac{P_2 V_2}{P_3 V_3}= \frac{\nu R T_2}{\nu R T_3} = \frac{T_2}{T_3} = 4 \implies P_1 = P_2 = 4 P_0 \\
    A_\text{цикл} &= (4P_0 - P_0)(3V_0 - V_0) = 6P_0V_0, \\
    A_{12} &= 4P_0 \cdot (3V_0 - V_0) = 8P_0V_0, \\
    \Delta U_{12} &= \frac 32 \nu R T_2 - \frac 32 \nu R T_1 = \frac 32 P_2 V_2 - \frac 32 P_1 V_1 = \frac 32 \cdot 4 P_0 \cdot 3 V_0 -  \frac 32 \cdot 4 P_0 \cdot V_0 = \frac 32 \cdot 8 \cdot P_0V_0, \\
    \Delta U_{41} &= \frac 32 \nu R T_1 - \frac 32 \nu R T_4 = \frac 32 P_1 V_1 - \frac 32 P_4 V_4 = \frac 32 \cdot 4 P_0 V_0 - \frac 32 P_0 V_0 = \frac 32 \cdot 3 \cdot P_0V_0.
    \\
    \eta &= \frac{A_\text{цикл}}{Q_+} = \frac{A_\text{цикл}}{Q_{12} + Q_{41}}  = \frac{A_\text{цикл}}{A_{12} + \Delta U_{12} + A_{41} + \Delta U_{41}} =  \\
     &= \frac{6P_0V_0}{8P_0V_0 + \frac 32 \cdot 8 \cdot P_0V_0 + 0 + \frac 32 \cdot 3 \cdot P_0V_0} = \frac{6}{8 + \frac 32 \cdot 8 + \frac 32 \cdot 3} = \frac{12}{49} \approx 0{,}245.
     \\
    \eta_\text{Карно} &= 1 - \frac{T_\text{х}}{T_\text{н}} = 1 - \frac{T_\text{4}}{T_\text{2}} = 1 - \frac{\frac{P_4V_4}{\nu R}}{\frac{P_2V_2}{\nu R}} = 1 - \frac{P_4V_4}{P_2V_2} = 1 - \frac{P_0V_0}{4P_0 \cdot 3V_0} = 1 - \frac 1{4 \cdot 3}  = \frac{11}{12} \approx 0{,}917.
    \end{align*}
}
\solutionspace{360pt}

\tasknumber{2}%
\task{%
    Порция идеального одноатомного газа перешла из состояния 1 в состояние 2: $P_1 = 2\,\text{МПа}$, $V_1 = 3\,\text{л}$, $P_2 = 3{,}5\,\text{МПа}$, $V_2 = 2\,\text{л}$.
    Определите, какую работу при этом совершил газ, чему равно изменение внутренней энергии газа, сколько теплоты подвели к нему в этом процессе?
    При решении обратите внимание на знаки искомых величин.
    Известно, что в PV-координатах график процесса 12 представляет собой отрезок прямой.
}
\answer{%
    \begin{align*}
    P_1V_1 &= \nu R T_1, P_2V_2 = \nu R T_2, \\
    \Delta U &= U_2-U_1 = \frac 32 \nu R T_2- \frac 32 \nu R T_1 = \frac 32 P_2 V_2 - \frac 32 P_1 V_1= \frac 32 \cdot \cbr{3{,}5\,\text{МПа} \cdot2\,\text{л} - 2\,\text{МПа} \cdot3\,\text{л}} = 1{,}50\,\text{кДж}.
    \\
    A_\text{газа} &= \frac{P_2 + P_1} 2 \cdot (V_2 - V_1) = \frac{3{,}5\,\text{МПа} + 2\,\text{МПа}} 2 \cdot (2\,\text{л} - 3\,\text{л}) = -2{,}750\,\text{кДж}, \\
    Q &= A_\text{газа} + \Delta U = \frac 32 (P_2 V_2 - P_1 V_1) + \frac{P_2 + P_1} 2 \cdot (V_2 - V_1) = 1{,}50\,\text{кДж} -2{,}750\,\text{кДж} = -1{,}2500\,\text{кДж}.
    \end{align*}
}
\solutionspace{150pt}

\tasknumber{3}%
\task{%
    Запишите формулы и рядом с каждой физичической величиной укажите её название и единицы изменения в СИ:
    \begin{enumerate}
        \item первое начало термодинамики,
        \item внутренняя энергия идеального одноатомного газа.
    \end{enumerate}
}

\variantsplitter

\addpersonalvariant{Семён Мартынов}

\tasknumber{1}%
\task{%
    Определите КПД цикла 12341, рабочим телом которого является идеальный одноатомный газ, если
    12 — изобарическое расширение газа в два раза,
    23 — изохорическое охлаждение газа, при котором температура уменьшается в два раза,
    34 — изобара, 41 — изохора.
    % Для этого:
    % \begin{enumerate}
    %     \item сделайте рисунок в PV-координатах,
    %     \item выберите удобные обозначения, чтобы не запутаться в множестве температур, давлений и объёмов,
    %     \item вычислите необходимые соотнощения между температурами, давлениями и объёмами
    %     (некоторые сразу видны по рисунку, некоторые — надо считать),
    %     \item определите для каждого участка поглощается или отдаётся тепло (и сколько именно:
    %     потребуется первое начало термодинамики, отдельный расчёт работ на участках через площади фигур и изменений внутренней энергии),
    %     \item вычислите полную работу газа в цикле,
    %     \item подставьте всё в формулу для КПД, упростите и доведите до ответа.
    % \end{enumerate}
    Определите КПД цикла Карно, температура нагревателя которого равна максимальной температуре в цикле 12341, а холодильника — минимальной.
    Ответы в обоих случаях оставьте точными в виде нескоратимой дроби, никаких округлений.
}
\answer{%
    \begin{align*}
    A_{12} &> 0, \Delta U_{12} > 0, \implies Q_{12} = A_{12} + \Delta U_{12} > 0, \\
    A_{23} &= 0, \Delta U_{23} < 0, \implies Q_{23} = A_{23} + \Delta U_{23} < 0, \\
    A_{34} &< 0, \Delta U_{34} < 0, \implies Q_{34} = A_{34} + \Delta U_{34} < 0, \\
    A_{41} &= 0, \Delta U_{41} > 0, \implies Q_{41} = A_{41} + \Delta U_{41} > 0.
    \\
    P_1V_1 &= \nu R T_1, P_2V_2 = \nu R T_2, P_3V_3 = \nu R T_3, P_4V_4 = \nu R T_4 \text{ — уравнения состояния идеального газа}, \\
    &\text{Пусть $P_0$, $V_0$, $T_0$ — давление, объём и температура в точке 4 (минимальные во всём цикле):} \\
    P_1 &= P_2, P_3 = P_4 = P_0, V_1 = V_4 = V_0, V_2 = V_3 = 2 V_1 = 2 V_0,, \text{остальные соотношения между объёмами и давлениями не даны, нужно считать} \\
    T_3 &= \frac{T_2}2 \text{(по условию)} \implies \frac{P_2}{P_3} = \frac{P_2 V_2}{P_3 V_3}= \frac{\nu R T_2}{\nu R T_3} = \frac{T_2}{T_3} = 2 \implies P_1 = P_2 = 2 P_0 \\
    A_\text{цикл} &= (2P_0 - P_0)(2V_0 - V_0) = 1P_0V_0, \\
    A_{12} &= 2P_0 \cdot (2V_0 - V_0) = 2P_0V_0, \\
    \Delta U_{12} &= \frac 32 \nu R T_2 - \frac 32 \nu R T_1 = \frac 32 P_2 V_2 - \frac 32 P_1 V_1 = \frac 32 \cdot 2 P_0 \cdot 2 V_0 -  \frac 32 \cdot 2 P_0 \cdot V_0 = \frac 32 \cdot 2 \cdot P_0V_0, \\
    \Delta U_{41} &= \frac 32 \nu R T_1 - \frac 32 \nu R T_4 = \frac 32 P_1 V_1 - \frac 32 P_4 V_4 = \frac 32 \cdot 2 P_0 V_0 - \frac 32 P_0 V_0 = \frac 32 \cdot 1 \cdot P_0V_0.
    \\
    \eta &= \frac{A_\text{цикл}}{Q_+} = \frac{A_\text{цикл}}{Q_{12} + Q_{41}}  = \frac{A_\text{цикл}}{A_{12} + \Delta U_{12} + A_{41} + \Delta U_{41}} =  \\
     &= \frac{1P_0V_0}{2P_0V_0 + \frac 32 \cdot 2 \cdot P_0V_0 + 0 + \frac 32 \cdot 1 \cdot P_0V_0} = \frac{1}{2 + \frac 32 \cdot 2 + \frac 32 \cdot 1} = \frac{2}{13} \approx 0{,}154.
     \\
    \eta_\text{Карно} &= 1 - \frac{T_\text{х}}{T_\text{н}} = 1 - \frac{T_\text{4}}{T_\text{2}} = 1 - \frac{\frac{P_4V_4}{\nu R}}{\frac{P_2V_2}{\nu R}} = 1 - \frac{P_4V_4}{P_2V_2} = 1 - \frac{P_0V_0}{2P_0 \cdot 2V_0} = 1 - \frac 1{2 \cdot 2}  = \frac{3}{4} \approx 0{,}750.
    \end{align*}
}
\solutionspace{360pt}

\tasknumber{2}%
\task{%
    Порция идеального одноатомного газа перешла из состояния 1 в состояние 2: $P_1 = 2\,\text{МПа}$, $V_1 = 3\,\text{л}$, $P_2 = 4{,}5\,\text{МПа}$, $V_2 = 8\,\text{л}$.
    Определите, какую работу при этом совершил газ, чему равно изменение внутренней энергии газа, сколько теплоты подвели к нему в этом процессе?
    При решении обратите внимание на знаки искомых величин.
    Известно, что в PV-координатах график процесса 12 представляет собой отрезок прямой.
}
\answer{%
    \begin{align*}
    P_1V_1 &= \nu R T_1, P_2V_2 = \nu R T_2, \\
    \Delta U &= U_2-U_1 = \frac 32 \nu R T_2- \frac 32 \nu R T_1 = \frac 32 P_2 V_2 - \frac 32 P_1 V_1= \frac 32 \cdot \cbr{4{,}5\,\text{МПа} \cdot8\,\text{л} - 2\,\text{МПа} \cdot3\,\text{л}} = 45{,}00\,\text{кДж}.
    \\
    A_\text{газа} &= \frac{P_2 + P_1} 2 \cdot (V_2 - V_1) = \frac{4{,}5\,\text{МПа} + 2\,\text{МПа}} 2 \cdot (8\,\text{л} - 3\,\text{л}) = 16{,}25\,\text{кДж}, \\
    Q &= A_\text{газа} + \Delta U = \frac 32 (P_2 V_2 - P_1 V_1) + \frac{P_2 + P_1} 2 \cdot (V_2 - V_1) = 45{,}00\,\text{кДж} + 16{,}25\,\text{кДж} = 61{,}25\,\text{кДж}.
    \end{align*}
}
\solutionspace{150pt}

\tasknumber{3}%
\task{%
    Запишите формулы и рядом с каждой физичической величиной укажите её название и единицы изменения в СИ:
    \begin{enumerate}
        \item первое начало термодинамики,
        \item внутренняя энергия идеального одноатомного газа.
    \end{enumerate}
}

\variantsplitter

\addpersonalvariant{Варвара Минаева}

\tasknumber{1}%
\task{%
    Определите КПД цикла 12341, рабочим телом которого является идеальный одноатомный газ, если
    12 — изобарическое расширение газа в три раза,
    23 — изохорическое охлаждение газа, при котором температура уменьшается в пять раз,
    34 — изобара, 41 — изохора.
    % Для этого:
    % \begin{enumerate}
    %     \item сделайте рисунок в PV-координатах,
    %     \item выберите удобные обозначения, чтобы не запутаться в множестве температур, давлений и объёмов,
    %     \item вычислите необходимые соотнощения между температурами, давлениями и объёмами
    %     (некоторые сразу видны по рисунку, некоторые — надо считать),
    %     \item определите для каждого участка поглощается или отдаётся тепло (и сколько именно:
    %     потребуется первое начало термодинамики, отдельный расчёт работ на участках через площади фигур и изменений внутренней энергии),
    %     \item вычислите полную работу газа в цикле,
    %     \item подставьте всё в формулу для КПД, упростите и доведите до ответа.
    % \end{enumerate}
    Определите КПД цикла Карно, температура нагревателя которого равна максимальной температуре в цикле 12341, а холодильника — минимальной.
    Ответы в обоих случаях оставьте точными в виде нескоратимой дроби, никаких округлений.
}
\answer{%
    \begin{align*}
    A_{12} &> 0, \Delta U_{12} > 0, \implies Q_{12} = A_{12} + \Delta U_{12} > 0, \\
    A_{23} &= 0, \Delta U_{23} < 0, \implies Q_{23} = A_{23} + \Delta U_{23} < 0, \\
    A_{34} &< 0, \Delta U_{34} < 0, \implies Q_{34} = A_{34} + \Delta U_{34} < 0, \\
    A_{41} &= 0, \Delta U_{41} > 0, \implies Q_{41} = A_{41} + \Delta U_{41} > 0.
    \\
    P_1V_1 &= \nu R T_1, P_2V_2 = \nu R T_2, P_3V_3 = \nu R T_3, P_4V_4 = \nu R T_4 \text{ — уравнения состояния идеального газа}, \\
    &\text{Пусть $P_0$, $V_0$, $T_0$ — давление, объём и температура в точке 4 (минимальные во всём цикле):} \\
    P_1 &= P_2, P_3 = P_4 = P_0, V_1 = V_4 = V_0, V_2 = V_3 = 3 V_1 = 3 V_0,, \text{остальные соотношения между объёмами и давлениями не даны, нужно считать} \\
    T_3 &= \frac{T_2}5 \text{(по условию)} \implies \frac{P_2}{P_3} = \frac{P_2 V_2}{P_3 V_3}= \frac{\nu R T_2}{\nu R T_3} = \frac{T_2}{T_3} = 5 \implies P_1 = P_2 = 5 P_0 \\
    A_\text{цикл} &= (5P_0 - P_0)(3V_0 - V_0) = 8P_0V_0, \\
    A_{12} &= 5P_0 \cdot (3V_0 - V_0) = 10P_0V_0, \\
    \Delta U_{12} &= \frac 32 \nu R T_2 - \frac 32 \nu R T_1 = \frac 32 P_2 V_2 - \frac 32 P_1 V_1 = \frac 32 \cdot 5 P_0 \cdot 3 V_0 -  \frac 32 \cdot 5 P_0 \cdot V_0 = \frac 32 \cdot 10 \cdot P_0V_0, \\
    \Delta U_{41} &= \frac 32 \nu R T_1 - \frac 32 \nu R T_4 = \frac 32 P_1 V_1 - \frac 32 P_4 V_4 = \frac 32 \cdot 5 P_0 V_0 - \frac 32 P_0 V_0 = \frac 32 \cdot 4 \cdot P_0V_0.
    \\
    \eta &= \frac{A_\text{цикл}}{Q_+} = \frac{A_\text{цикл}}{Q_{12} + Q_{41}}  = \frac{A_\text{цикл}}{A_{12} + \Delta U_{12} + A_{41} + \Delta U_{41}} =  \\
     &= \frac{8P_0V_0}{10P_0V_0 + \frac 32 \cdot 10 \cdot P_0V_0 + 0 + \frac 32 \cdot 4 \cdot P_0V_0} = \frac{8}{10 + \frac 32 \cdot 10 + \frac 32 \cdot 4} = \frac{8}{31} \approx 0{,}258.
     \\
    \eta_\text{Карно} &= 1 - \frac{T_\text{х}}{T_\text{н}} = 1 - \frac{T_\text{4}}{T_\text{2}} = 1 - \frac{\frac{P_4V_4}{\nu R}}{\frac{P_2V_2}{\nu R}} = 1 - \frac{P_4V_4}{P_2V_2} = 1 - \frac{P_0V_0}{5P_0 \cdot 3V_0} = 1 - \frac 1{5 \cdot 3}  = \frac{14}{15} \approx 0{,}933.
    \end{align*}
}
\solutionspace{360pt}

\tasknumber{2}%
\task{%
    Порция идеального одноатомного газа перешла из состояния 1 в состояние 2: $P_1 = 3\,\text{МПа}$, $V_1 = 7\,\text{л}$, $P_2 = 2{,}5\,\text{МПа}$, $V_2 = 8\,\text{л}$.
    Определите, какую работу при этом совершил газ, чему равно изменение внутренней энергии газа, сколько теплоты подвели к нему в этом процессе?
    При решении обратите внимание на знаки искомых величин.
    Известно, что в PV-координатах график процесса 12 представляет собой отрезок прямой.
}
\answer{%
    \begin{align*}
    P_1V_1 &= \nu R T_1, P_2V_2 = \nu R T_2, \\
    \Delta U &= U_2-U_1 = \frac 32 \nu R T_2- \frac 32 \nu R T_1 = \frac 32 P_2 V_2 - \frac 32 P_1 V_1= \frac 32 \cdot \cbr{2{,}5\,\text{МПа} \cdot8\,\text{л} - 3\,\text{МПа} \cdot7\,\text{л}} = -1{,}5000\,\text{кДж}.
    \\
    A_\text{газа} &= \frac{P_2 + P_1} 2 \cdot (V_2 - V_1) = \frac{2{,}5\,\text{МПа} + 3\,\text{МПа}} 2 \cdot (8\,\text{л} - 7\,\text{л}) = 2{,}75\,\text{кДж}, \\
    Q &= A_\text{газа} + \Delta U = \frac 32 (P_2 V_2 - P_1 V_1) + \frac{P_2 + P_1} 2 \cdot (V_2 - V_1) = -1{,}5000\,\text{кДж} + 2{,}75\,\text{кДж} = 1{,}25\,\text{кДж}.
    \end{align*}
}
\solutionspace{150pt}

\tasknumber{3}%
\task{%
    Запишите формулы и рядом с каждой физичической величиной укажите её название и единицы изменения в СИ:
    \begin{enumerate}
        \item первое начало термодинамики,
        \item внутренняя энергия идеального одноатомного газа.
    \end{enumerate}
}

\variantsplitter

\addpersonalvariant{Леонид Никитин}

\tasknumber{1}%
\task{%
    Определите КПД цикла 12341, рабочим телом которого является идеальный одноатомный газ, если
    12 — изобарическое расширение газа в пять раз,
    23 — изохорическое охлаждение газа, при котором температура уменьшается в шесть раз,
    34 — изобара, 41 — изохора.
    % Для этого:
    % \begin{enumerate}
    %     \item сделайте рисунок в PV-координатах,
    %     \item выберите удобные обозначения, чтобы не запутаться в множестве температур, давлений и объёмов,
    %     \item вычислите необходимые соотнощения между температурами, давлениями и объёмами
    %     (некоторые сразу видны по рисунку, некоторые — надо считать),
    %     \item определите для каждого участка поглощается или отдаётся тепло (и сколько именно:
    %     потребуется первое начало термодинамики, отдельный расчёт работ на участках через площади фигур и изменений внутренней энергии),
    %     \item вычислите полную работу газа в цикле,
    %     \item подставьте всё в формулу для КПД, упростите и доведите до ответа.
    % \end{enumerate}
    Определите КПД цикла Карно, температура нагревателя которого равна максимальной температуре в цикле 12341, а холодильника — минимальной.
    Ответы в обоих случаях оставьте точными в виде нескоратимой дроби, никаких округлений.
}
\answer{%
    \begin{align*}
    A_{12} &> 0, \Delta U_{12} > 0, \implies Q_{12} = A_{12} + \Delta U_{12} > 0, \\
    A_{23} &= 0, \Delta U_{23} < 0, \implies Q_{23} = A_{23} + \Delta U_{23} < 0, \\
    A_{34} &< 0, \Delta U_{34} < 0, \implies Q_{34} = A_{34} + \Delta U_{34} < 0, \\
    A_{41} &= 0, \Delta U_{41} > 0, \implies Q_{41} = A_{41} + \Delta U_{41} > 0.
    \\
    P_1V_1 &= \nu R T_1, P_2V_2 = \nu R T_2, P_3V_3 = \nu R T_3, P_4V_4 = \nu R T_4 \text{ — уравнения состояния идеального газа}, \\
    &\text{Пусть $P_0$, $V_0$, $T_0$ — давление, объём и температура в точке 4 (минимальные во всём цикле):} \\
    P_1 &= P_2, P_3 = P_4 = P_0, V_1 = V_4 = V_0, V_2 = V_3 = 5 V_1 = 5 V_0,, \text{остальные соотношения между объёмами и давлениями не даны, нужно считать} \\
    T_3 &= \frac{T_2}6 \text{(по условию)} \implies \frac{P_2}{P_3} = \frac{P_2 V_2}{P_3 V_3}= \frac{\nu R T_2}{\nu R T_3} = \frac{T_2}{T_3} = 6 \implies P_1 = P_2 = 6 P_0 \\
    A_\text{цикл} &= (6P_0 - P_0)(5V_0 - V_0) = 20P_0V_0, \\
    A_{12} &= 6P_0 \cdot (5V_0 - V_0) = 24P_0V_0, \\
    \Delta U_{12} &= \frac 32 \nu R T_2 - \frac 32 \nu R T_1 = \frac 32 P_2 V_2 - \frac 32 P_1 V_1 = \frac 32 \cdot 6 P_0 \cdot 5 V_0 -  \frac 32 \cdot 6 P_0 \cdot V_0 = \frac 32 \cdot 24 \cdot P_0V_0, \\
    \Delta U_{41} &= \frac 32 \nu R T_1 - \frac 32 \nu R T_4 = \frac 32 P_1 V_1 - \frac 32 P_4 V_4 = \frac 32 \cdot 6 P_0 V_0 - \frac 32 P_0 V_0 = \frac 32 \cdot 5 \cdot P_0V_0.
    \\
    \eta &= \frac{A_\text{цикл}}{Q_+} = \frac{A_\text{цикл}}{Q_{12} + Q_{41}}  = \frac{A_\text{цикл}}{A_{12} + \Delta U_{12} + A_{41} + \Delta U_{41}} =  \\
     &= \frac{20P_0V_0}{24P_0V_0 + \frac 32 \cdot 24 \cdot P_0V_0 + 0 + \frac 32 \cdot 5 \cdot P_0V_0} = \frac{20}{24 + \frac 32 \cdot 24 + \frac 32 \cdot 5} = \frac{8}{27} \approx 0{,}296.
     \\
    \eta_\text{Карно} &= 1 - \frac{T_\text{х}}{T_\text{н}} = 1 - \frac{T_\text{4}}{T_\text{2}} = 1 - \frac{\frac{P_4V_4}{\nu R}}{\frac{P_2V_2}{\nu R}} = 1 - \frac{P_4V_4}{P_2V_2} = 1 - \frac{P_0V_0}{6P_0 \cdot 5V_0} = 1 - \frac 1{6 \cdot 5}  = \frac{29}{30} \approx 0{,}967.
    \end{align*}
}
\solutionspace{360pt}

\tasknumber{2}%
\task{%
    Порция идеального одноатомного газа перешла из состояния 1 в состояние 2: $P_1 = 3\,\text{МПа}$, $V_1 = 5\,\text{л}$, $P_2 = 2{,}5\,\text{МПа}$, $V_2 = 6\,\text{л}$.
    Определите, какую работу при этом совершил газ, чему равно изменение внутренней энергии газа, сколько теплоты подвели к нему в этом процессе?
    При решении обратите внимание на знаки искомых величин.
    Известно, что в PV-координатах график процесса 12 представляет собой отрезок прямой.
}
\answer{%
    \begin{align*}
    P_1V_1 &= \nu R T_1, P_2V_2 = \nu R T_2, \\
    \Delta U &= U_2-U_1 = \frac 32 \nu R T_2- \frac 32 \nu R T_1 = \frac 32 P_2 V_2 - \frac 32 P_1 V_1= \frac 32 \cdot \cbr{2{,}5\,\text{МПа} \cdot6\,\text{л} - 3\,\text{МПа} \cdot5\,\text{л}} = 0\,\text{кДж}.
    \\
    A_\text{газа} &= \frac{P_2 + P_1} 2 \cdot (V_2 - V_1) = \frac{2{,}5\,\text{МПа} + 3\,\text{МПа}} 2 \cdot (6\,\text{л} - 5\,\text{л}) = 2{,}75\,\text{кДж}, \\
    Q &= A_\text{газа} + \Delta U = \frac 32 (P_2 V_2 - P_1 V_1) + \frac{P_2 + P_1} 2 \cdot (V_2 - V_1) = 0\,\text{кДж} + 2{,}75\,\text{кДж} = 2{,}75\,\text{кДж}.
    \end{align*}
}
\solutionspace{150pt}

\tasknumber{3}%
\task{%
    Запишите формулы и рядом с каждой физичической величиной укажите её название и единицы изменения в СИ:
    \begin{enumerate}
        \item первое начало термодинамики,
        \item внутренняя энергия идеального одноатомного газа.
    \end{enumerate}
}

\variantsplitter

\addpersonalvariant{Тимофей Полетаев}

\tasknumber{1}%
\task{%
    Определите КПД цикла 12341, рабочим телом которого является идеальный одноатомный газ, если
    12 — изобарическое расширение газа в пять раз,
    23 — изохорическое охлаждение газа, при котором температура уменьшается в четыре раза,
    34 — изобара, 41 — изохора.
    % Для этого:
    % \begin{enumerate}
    %     \item сделайте рисунок в PV-координатах,
    %     \item выберите удобные обозначения, чтобы не запутаться в множестве температур, давлений и объёмов,
    %     \item вычислите необходимые соотнощения между температурами, давлениями и объёмами
    %     (некоторые сразу видны по рисунку, некоторые — надо считать),
    %     \item определите для каждого участка поглощается или отдаётся тепло (и сколько именно:
    %     потребуется первое начало термодинамики, отдельный расчёт работ на участках через площади фигур и изменений внутренней энергии),
    %     \item вычислите полную работу газа в цикле,
    %     \item подставьте всё в формулу для КПД, упростите и доведите до ответа.
    % \end{enumerate}
    Определите КПД цикла Карно, температура нагревателя которого равна максимальной температуре в цикле 12341, а холодильника — минимальной.
    Ответы в обоих случаях оставьте точными в виде нескоратимой дроби, никаких округлений.
}
\answer{%
    \begin{align*}
    A_{12} &> 0, \Delta U_{12} > 0, \implies Q_{12} = A_{12} + \Delta U_{12} > 0, \\
    A_{23} &= 0, \Delta U_{23} < 0, \implies Q_{23} = A_{23} + \Delta U_{23} < 0, \\
    A_{34} &< 0, \Delta U_{34} < 0, \implies Q_{34} = A_{34} + \Delta U_{34} < 0, \\
    A_{41} &= 0, \Delta U_{41} > 0, \implies Q_{41} = A_{41} + \Delta U_{41} > 0.
    \\
    P_1V_1 &= \nu R T_1, P_2V_2 = \nu R T_2, P_3V_3 = \nu R T_3, P_4V_4 = \nu R T_4 \text{ — уравнения состояния идеального газа}, \\
    &\text{Пусть $P_0$, $V_0$, $T_0$ — давление, объём и температура в точке 4 (минимальные во всём цикле):} \\
    P_1 &= P_2, P_3 = P_4 = P_0, V_1 = V_4 = V_0, V_2 = V_3 = 5 V_1 = 5 V_0,, \text{остальные соотношения между объёмами и давлениями не даны, нужно считать} \\
    T_3 &= \frac{T_2}4 \text{(по условию)} \implies \frac{P_2}{P_3} = \frac{P_2 V_2}{P_3 V_3}= \frac{\nu R T_2}{\nu R T_3} = \frac{T_2}{T_3} = 4 \implies P_1 = P_2 = 4 P_0 \\
    A_\text{цикл} &= (4P_0 - P_0)(5V_0 - V_0) = 12P_0V_0, \\
    A_{12} &= 4P_0 \cdot (5V_0 - V_0) = 16P_0V_0, \\
    \Delta U_{12} &= \frac 32 \nu R T_2 - \frac 32 \nu R T_1 = \frac 32 P_2 V_2 - \frac 32 P_1 V_1 = \frac 32 \cdot 4 P_0 \cdot 5 V_0 -  \frac 32 \cdot 4 P_0 \cdot V_0 = \frac 32 \cdot 16 \cdot P_0V_0, \\
    \Delta U_{41} &= \frac 32 \nu R T_1 - \frac 32 \nu R T_4 = \frac 32 P_1 V_1 - \frac 32 P_4 V_4 = \frac 32 \cdot 4 P_0 V_0 - \frac 32 P_0 V_0 = \frac 32 \cdot 3 \cdot P_0V_0.
    \\
    \eta &= \frac{A_\text{цикл}}{Q_+} = \frac{A_\text{цикл}}{Q_{12} + Q_{41}}  = \frac{A_\text{цикл}}{A_{12} + \Delta U_{12} + A_{41} + \Delta U_{41}} =  \\
     &= \frac{12P_0V_0}{16P_0V_0 + \frac 32 \cdot 16 \cdot P_0V_0 + 0 + \frac 32 \cdot 3 \cdot P_0V_0} = \frac{12}{16 + \frac 32 \cdot 16 + \frac 32 \cdot 3} = \frac{24}{89} \approx 0{,}270.
     \\
    \eta_\text{Карно} &= 1 - \frac{T_\text{х}}{T_\text{н}} = 1 - \frac{T_\text{4}}{T_\text{2}} = 1 - \frac{\frac{P_4V_4}{\nu R}}{\frac{P_2V_2}{\nu R}} = 1 - \frac{P_4V_4}{P_2V_2} = 1 - \frac{P_0V_0}{4P_0 \cdot 5V_0} = 1 - \frac 1{4 \cdot 5}  = \frac{19}{20} \approx 0{,}950.
    \end{align*}
}
\solutionspace{360pt}

\tasknumber{2}%
\task{%
    Порция идеального одноатомного газа перешла из состояния 1 в состояние 2: $P_1 = 2\,\text{МПа}$, $V_1 = 5\,\text{л}$, $P_2 = 4{,}5\,\text{МПа}$, $V_2 = 8\,\text{л}$.
    Определите, какую работу при этом совершил газ, чему равно изменение внутренней энергии газа, сколько теплоты подвели к нему в этом процессе?
    При решении обратите внимание на знаки искомых величин.
    Известно, что в PV-координатах график процесса 12 представляет собой отрезок прямой.
}
\answer{%
    \begin{align*}
    P_1V_1 &= \nu R T_1, P_2V_2 = \nu R T_2, \\
    \Delta U &= U_2-U_1 = \frac 32 \nu R T_2- \frac 32 \nu R T_1 = \frac 32 P_2 V_2 - \frac 32 P_1 V_1= \frac 32 \cdot \cbr{4{,}5\,\text{МПа} \cdot8\,\text{л} - 2\,\text{МПа} \cdot5\,\text{л}} = 39{,}00\,\text{кДж}.
    \\
    A_\text{газа} &= \frac{P_2 + P_1} 2 \cdot (V_2 - V_1) = \frac{4{,}5\,\text{МПа} + 2\,\text{МПа}} 2 \cdot (8\,\text{л} - 5\,\text{л}) = 9{,}75\,\text{кДж}, \\
    Q &= A_\text{газа} + \Delta U = \frac 32 (P_2 V_2 - P_1 V_1) + \frac{P_2 + P_1} 2 \cdot (V_2 - V_1) = 39{,}00\,\text{кДж} + 9{,}75\,\text{кДж} = 48{,}75\,\text{кДж}.
    \end{align*}
}
\solutionspace{150pt}

\tasknumber{3}%
\task{%
    Запишите формулы и рядом с каждой физичической величиной укажите её название и единицы изменения в СИ:
    \begin{enumerate}
        \item первое начало термодинамики,
        \item внутренняя энергия идеального одноатомного газа.
    \end{enumerate}
}

\variantsplitter

\addpersonalvariant{Андрей Рожков}

\tasknumber{1}%
\task{%
    Определите КПД цикла 12341, рабочим телом которого является идеальный одноатомный газ, если
    12 — изобарическое расширение газа в три раза,
    23 — изохорическое охлаждение газа, при котором температура уменьшается в пять раз,
    34 — изобара, 41 — изохора.
    % Для этого:
    % \begin{enumerate}
    %     \item сделайте рисунок в PV-координатах,
    %     \item выберите удобные обозначения, чтобы не запутаться в множестве температур, давлений и объёмов,
    %     \item вычислите необходимые соотнощения между температурами, давлениями и объёмами
    %     (некоторые сразу видны по рисунку, некоторые — надо считать),
    %     \item определите для каждого участка поглощается или отдаётся тепло (и сколько именно:
    %     потребуется первое начало термодинамики, отдельный расчёт работ на участках через площади фигур и изменений внутренней энергии),
    %     \item вычислите полную работу газа в цикле,
    %     \item подставьте всё в формулу для КПД, упростите и доведите до ответа.
    % \end{enumerate}
    Определите КПД цикла Карно, температура нагревателя которого равна максимальной температуре в цикле 12341, а холодильника — минимальной.
    Ответы в обоих случаях оставьте точными в виде нескоратимой дроби, никаких округлений.
}
\answer{%
    \begin{align*}
    A_{12} &> 0, \Delta U_{12} > 0, \implies Q_{12} = A_{12} + \Delta U_{12} > 0, \\
    A_{23} &= 0, \Delta U_{23} < 0, \implies Q_{23} = A_{23} + \Delta U_{23} < 0, \\
    A_{34} &< 0, \Delta U_{34} < 0, \implies Q_{34} = A_{34} + \Delta U_{34} < 0, \\
    A_{41} &= 0, \Delta U_{41} > 0, \implies Q_{41} = A_{41} + \Delta U_{41} > 0.
    \\
    P_1V_1 &= \nu R T_1, P_2V_2 = \nu R T_2, P_3V_3 = \nu R T_3, P_4V_4 = \nu R T_4 \text{ — уравнения состояния идеального газа}, \\
    &\text{Пусть $P_0$, $V_0$, $T_0$ — давление, объём и температура в точке 4 (минимальные во всём цикле):} \\
    P_1 &= P_2, P_3 = P_4 = P_0, V_1 = V_4 = V_0, V_2 = V_3 = 3 V_1 = 3 V_0,, \text{остальные соотношения между объёмами и давлениями не даны, нужно считать} \\
    T_3 &= \frac{T_2}5 \text{(по условию)} \implies \frac{P_2}{P_3} = \frac{P_2 V_2}{P_3 V_3}= \frac{\nu R T_2}{\nu R T_3} = \frac{T_2}{T_3} = 5 \implies P_1 = P_2 = 5 P_0 \\
    A_\text{цикл} &= (5P_0 - P_0)(3V_0 - V_0) = 8P_0V_0, \\
    A_{12} &= 5P_0 \cdot (3V_0 - V_0) = 10P_0V_0, \\
    \Delta U_{12} &= \frac 32 \nu R T_2 - \frac 32 \nu R T_1 = \frac 32 P_2 V_2 - \frac 32 P_1 V_1 = \frac 32 \cdot 5 P_0 \cdot 3 V_0 -  \frac 32 \cdot 5 P_0 \cdot V_0 = \frac 32 \cdot 10 \cdot P_0V_0, \\
    \Delta U_{41} &= \frac 32 \nu R T_1 - \frac 32 \nu R T_4 = \frac 32 P_1 V_1 - \frac 32 P_4 V_4 = \frac 32 \cdot 5 P_0 V_0 - \frac 32 P_0 V_0 = \frac 32 \cdot 4 \cdot P_0V_0.
    \\
    \eta &= \frac{A_\text{цикл}}{Q_+} = \frac{A_\text{цикл}}{Q_{12} + Q_{41}}  = \frac{A_\text{цикл}}{A_{12} + \Delta U_{12} + A_{41} + \Delta U_{41}} =  \\
     &= \frac{8P_0V_0}{10P_0V_0 + \frac 32 \cdot 10 \cdot P_0V_0 + 0 + \frac 32 \cdot 4 \cdot P_0V_0} = \frac{8}{10 + \frac 32 \cdot 10 + \frac 32 \cdot 4} = \frac{8}{31} \approx 0{,}258.
     \\
    \eta_\text{Карно} &= 1 - \frac{T_\text{х}}{T_\text{н}} = 1 - \frac{T_\text{4}}{T_\text{2}} = 1 - \frac{\frac{P_4V_4}{\nu R}}{\frac{P_2V_2}{\nu R}} = 1 - \frac{P_4V_4}{P_2V_2} = 1 - \frac{P_0V_0}{5P_0 \cdot 3V_0} = 1 - \frac 1{5 \cdot 3}  = \frac{14}{15} \approx 0{,}933.
    \end{align*}
}
\solutionspace{360pt}

\tasknumber{2}%
\task{%
    Порция идеального одноатомного газа перешла из состояния 1 в состояние 2: $P_1 = 3\,\text{МПа}$, $V_1 = 5\,\text{л}$, $P_2 = 4{,}5\,\text{МПа}$, $V_2 = 4\,\text{л}$.
    Определите, какую работу при этом совершил газ, чему равно изменение внутренней энергии газа, сколько теплоты подвели к нему в этом процессе?
    При решении обратите внимание на знаки искомых величин.
    Известно, что в PV-координатах график процесса 12 представляет собой отрезок прямой.
}
\answer{%
    \begin{align*}
    P_1V_1 &= \nu R T_1, P_2V_2 = \nu R T_2, \\
    \Delta U &= U_2-U_1 = \frac 32 \nu R T_2- \frac 32 \nu R T_1 = \frac 32 P_2 V_2 - \frac 32 P_1 V_1= \frac 32 \cdot \cbr{4{,}5\,\text{МПа} \cdot4\,\text{л} - 3\,\text{МПа} \cdot5\,\text{л}} = 4{,}50\,\text{кДж}.
    \\
    A_\text{газа} &= \frac{P_2 + P_1} 2 \cdot (V_2 - V_1) = \frac{4{,}5\,\text{МПа} + 3\,\text{МПа}} 2 \cdot (4\,\text{л} - 5\,\text{л}) = -3{,}750\,\text{кДж}, \\
    Q &= A_\text{газа} + \Delta U = \frac 32 (P_2 V_2 - P_1 V_1) + \frac{P_2 + P_1} 2 \cdot (V_2 - V_1) = 4{,}50\,\text{кДж} -3{,}750\,\text{кДж} = 0{,}75\,\text{кДж}.
    \end{align*}
}
\solutionspace{150pt}

\tasknumber{3}%
\task{%
    Запишите формулы и рядом с каждой физичической величиной укажите её название и единицы изменения в СИ:
    \begin{enumerate}
        \item первое начало термодинамики,
        \item внутренняя энергия идеального одноатомного газа.
    \end{enumerate}
}

\variantsplitter

\addpersonalvariant{Рената Таржиманова}

\tasknumber{1}%
\task{%
    Определите КПД цикла 12341, рабочим телом которого является идеальный одноатомный газ, если
    12 — изобарическое расширение газа в два раза,
    23 — изохорическое охлаждение газа, при котором температура уменьшается в пять раз,
    34 — изобара, 41 — изохора.
    % Для этого:
    % \begin{enumerate}
    %     \item сделайте рисунок в PV-координатах,
    %     \item выберите удобные обозначения, чтобы не запутаться в множестве температур, давлений и объёмов,
    %     \item вычислите необходимые соотнощения между температурами, давлениями и объёмами
    %     (некоторые сразу видны по рисунку, некоторые — надо считать),
    %     \item определите для каждого участка поглощается или отдаётся тепло (и сколько именно:
    %     потребуется первое начало термодинамики, отдельный расчёт работ на участках через площади фигур и изменений внутренней энергии),
    %     \item вычислите полную работу газа в цикле,
    %     \item подставьте всё в формулу для КПД, упростите и доведите до ответа.
    % \end{enumerate}
    Определите КПД цикла Карно, температура нагревателя которого равна максимальной температуре в цикле 12341, а холодильника — минимальной.
    Ответы в обоих случаях оставьте точными в виде нескоратимой дроби, никаких округлений.
}
\answer{%
    \begin{align*}
    A_{12} &> 0, \Delta U_{12} > 0, \implies Q_{12} = A_{12} + \Delta U_{12} > 0, \\
    A_{23} &= 0, \Delta U_{23} < 0, \implies Q_{23} = A_{23} + \Delta U_{23} < 0, \\
    A_{34} &< 0, \Delta U_{34} < 0, \implies Q_{34} = A_{34} + \Delta U_{34} < 0, \\
    A_{41} &= 0, \Delta U_{41} > 0, \implies Q_{41} = A_{41} + \Delta U_{41} > 0.
    \\
    P_1V_1 &= \nu R T_1, P_2V_2 = \nu R T_2, P_3V_3 = \nu R T_3, P_4V_4 = \nu R T_4 \text{ — уравнения состояния идеального газа}, \\
    &\text{Пусть $P_0$, $V_0$, $T_0$ — давление, объём и температура в точке 4 (минимальные во всём цикле):} \\
    P_1 &= P_2, P_3 = P_4 = P_0, V_1 = V_4 = V_0, V_2 = V_3 = 2 V_1 = 2 V_0,, \text{остальные соотношения между объёмами и давлениями не даны, нужно считать} \\
    T_3 &= \frac{T_2}5 \text{(по условию)} \implies \frac{P_2}{P_3} = \frac{P_2 V_2}{P_3 V_3}= \frac{\nu R T_2}{\nu R T_3} = \frac{T_2}{T_3} = 5 \implies P_1 = P_2 = 5 P_0 \\
    A_\text{цикл} &= (5P_0 - P_0)(2V_0 - V_0) = 4P_0V_0, \\
    A_{12} &= 5P_0 \cdot (2V_0 - V_0) = 5P_0V_0, \\
    \Delta U_{12} &= \frac 32 \nu R T_2 - \frac 32 \nu R T_1 = \frac 32 P_2 V_2 - \frac 32 P_1 V_1 = \frac 32 \cdot 5 P_0 \cdot 2 V_0 -  \frac 32 \cdot 5 P_0 \cdot V_0 = \frac 32 \cdot 5 \cdot P_0V_0, \\
    \Delta U_{41} &= \frac 32 \nu R T_1 - \frac 32 \nu R T_4 = \frac 32 P_1 V_1 - \frac 32 P_4 V_4 = \frac 32 \cdot 5 P_0 V_0 - \frac 32 P_0 V_0 = \frac 32 \cdot 4 \cdot P_0V_0.
    \\
    \eta &= \frac{A_\text{цикл}}{Q_+} = \frac{A_\text{цикл}}{Q_{12} + Q_{41}}  = \frac{A_\text{цикл}}{A_{12} + \Delta U_{12} + A_{41} + \Delta U_{41}} =  \\
     &= \frac{4P_0V_0}{5P_0V_0 + \frac 32 \cdot 5 \cdot P_0V_0 + 0 + \frac 32 \cdot 4 \cdot P_0V_0} = \frac{4}{5 + \frac 32 \cdot 5 + \frac 32 \cdot 4} = \frac{8}{37} \approx 0{,}216.
     \\
    \eta_\text{Карно} &= 1 - \frac{T_\text{х}}{T_\text{н}} = 1 - \frac{T_\text{4}}{T_\text{2}} = 1 - \frac{\frac{P_4V_4}{\nu R}}{\frac{P_2V_2}{\nu R}} = 1 - \frac{P_4V_4}{P_2V_2} = 1 - \frac{P_0V_0}{5P_0 \cdot 2V_0} = 1 - \frac 1{5 \cdot 2}  = \frac{9}{10} \approx 0{,}900.
    \end{align*}
}
\solutionspace{360pt}

\tasknumber{2}%
\task{%
    Порция идеального одноатомного газа перешла из состояния 1 в состояние 2: $P_1 = 3\,\text{МПа}$, $V_1 = 3\,\text{л}$, $P_2 = 2{,}5\,\text{МПа}$, $V_2 = 4\,\text{л}$.
    Определите, какую работу при этом совершил газ, чему равно изменение внутренней энергии газа, сколько теплоты подвели к нему в этом процессе?
    При решении обратите внимание на знаки искомых величин.
    Известно, что в PV-координатах график процесса 12 представляет собой отрезок прямой.
}
\answer{%
    \begin{align*}
    P_1V_1 &= \nu R T_1, P_2V_2 = \nu R T_2, \\
    \Delta U &= U_2-U_1 = \frac 32 \nu R T_2- \frac 32 \nu R T_1 = \frac 32 P_2 V_2 - \frac 32 P_1 V_1= \frac 32 \cdot \cbr{2{,}5\,\text{МПа} \cdot4\,\text{л} - 3\,\text{МПа} \cdot3\,\text{л}} = 1{,}50\,\text{кДж}.
    \\
    A_\text{газа} &= \frac{P_2 + P_1} 2 \cdot (V_2 - V_1) = \frac{2{,}5\,\text{МПа} + 3\,\text{МПа}} 2 \cdot (4\,\text{л} - 3\,\text{л}) = 2{,}75\,\text{кДж}, \\
    Q &= A_\text{газа} + \Delta U = \frac 32 (P_2 V_2 - P_1 V_1) + \frac{P_2 + P_1} 2 \cdot (V_2 - V_1) = 1{,}50\,\text{кДж} + 2{,}75\,\text{кДж} = 4{,}25\,\text{кДж}.
    \end{align*}
}
\solutionspace{150pt}

\tasknumber{3}%
\task{%
    Запишите формулы и рядом с каждой физичической величиной укажите её название и единицы изменения в СИ:
    \begin{enumerate}
        \item первое начало термодинамики,
        \item внутренняя энергия идеального одноатомного газа.
    \end{enumerate}
}

\variantsplitter

\addpersonalvariant{Арсений Трофимов}

\tasknumber{1}%
\task{%
    Определите КПД цикла 12341, рабочим телом которого является идеальный одноатомный газ, если
    12 — изобарическое расширение газа в два раза,
    23 — изохорическое охлаждение газа, при котором температура уменьшается в четыре раза,
    34 — изобара, 41 — изохора.
    % Для этого:
    % \begin{enumerate}
    %     \item сделайте рисунок в PV-координатах,
    %     \item выберите удобные обозначения, чтобы не запутаться в множестве температур, давлений и объёмов,
    %     \item вычислите необходимые соотнощения между температурами, давлениями и объёмами
    %     (некоторые сразу видны по рисунку, некоторые — надо считать),
    %     \item определите для каждого участка поглощается или отдаётся тепло (и сколько именно:
    %     потребуется первое начало термодинамики, отдельный расчёт работ на участках через площади фигур и изменений внутренней энергии),
    %     \item вычислите полную работу газа в цикле,
    %     \item подставьте всё в формулу для КПД, упростите и доведите до ответа.
    % \end{enumerate}
    Определите КПД цикла Карно, температура нагревателя которого равна максимальной температуре в цикле 12341, а холодильника — минимальной.
    Ответы в обоих случаях оставьте точными в виде нескоратимой дроби, никаких округлений.
}
\answer{%
    \begin{align*}
    A_{12} &> 0, \Delta U_{12} > 0, \implies Q_{12} = A_{12} + \Delta U_{12} > 0, \\
    A_{23} &= 0, \Delta U_{23} < 0, \implies Q_{23} = A_{23} + \Delta U_{23} < 0, \\
    A_{34} &< 0, \Delta U_{34} < 0, \implies Q_{34} = A_{34} + \Delta U_{34} < 0, \\
    A_{41} &= 0, \Delta U_{41} > 0, \implies Q_{41} = A_{41} + \Delta U_{41} > 0.
    \\
    P_1V_1 &= \nu R T_1, P_2V_2 = \nu R T_2, P_3V_3 = \nu R T_3, P_4V_4 = \nu R T_4 \text{ — уравнения состояния идеального газа}, \\
    &\text{Пусть $P_0$, $V_0$, $T_0$ — давление, объём и температура в точке 4 (минимальные во всём цикле):} \\
    P_1 &= P_2, P_3 = P_4 = P_0, V_1 = V_4 = V_0, V_2 = V_3 = 2 V_1 = 2 V_0,, \text{остальные соотношения между объёмами и давлениями не даны, нужно считать} \\
    T_3 &= \frac{T_2}4 \text{(по условию)} \implies \frac{P_2}{P_3} = \frac{P_2 V_2}{P_3 V_3}= \frac{\nu R T_2}{\nu R T_3} = \frac{T_2}{T_3} = 4 \implies P_1 = P_2 = 4 P_0 \\
    A_\text{цикл} &= (4P_0 - P_0)(2V_0 - V_0) = 3P_0V_0, \\
    A_{12} &= 4P_0 \cdot (2V_0 - V_0) = 4P_0V_0, \\
    \Delta U_{12} &= \frac 32 \nu R T_2 - \frac 32 \nu R T_1 = \frac 32 P_2 V_2 - \frac 32 P_1 V_1 = \frac 32 \cdot 4 P_0 \cdot 2 V_0 -  \frac 32 \cdot 4 P_0 \cdot V_0 = \frac 32 \cdot 4 \cdot P_0V_0, \\
    \Delta U_{41} &= \frac 32 \nu R T_1 - \frac 32 \nu R T_4 = \frac 32 P_1 V_1 - \frac 32 P_4 V_4 = \frac 32 \cdot 4 P_0 V_0 - \frac 32 P_0 V_0 = \frac 32 \cdot 3 \cdot P_0V_0.
    \\
    \eta &= \frac{A_\text{цикл}}{Q_+} = \frac{A_\text{цикл}}{Q_{12} + Q_{41}}  = \frac{A_\text{цикл}}{A_{12} + \Delta U_{12} + A_{41} + \Delta U_{41}} =  \\
     &= \frac{3P_0V_0}{4P_0V_0 + \frac 32 \cdot 4 \cdot P_0V_0 + 0 + \frac 32 \cdot 3 \cdot P_0V_0} = \frac{3}{4 + \frac 32 \cdot 4 + \frac 32 \cdot 3} = \frac{6}{29} \approx 0{,}207.
     \\
    \eta_\text{Карно} &= 1 - \frac{T_\text{х}}{T_\text{н}} = 1 - \frac{T_\text{4}}{T_\text{2}} = 1 - \frac{\frac{P_4V_4}{\nu R}}{\frac{P_2V_2}{\nu R}} = 1 - \frac{P_4V_4}{P_2V_2} = 1 - \frac{P_0V_0}{4P_0 \cdot 2V_0} = 1 - \frac 1{4 \cdot 2}  = \frac{7}{8} \approx 0{,}875.
    \end{align*}
}
\solutionspace{360pt}

\tasknumber{2}%
\task{%
    Порция идеального одноатомного газа перешла из состояния 1 в состояние 2: $P_1 = 3\,\text{МПа}$, $V_1 = 7\,\text{л}$, $P_2 = 2{,}5\,\text{МПа}$, $V_2 = 6\,\text{л}$.
    Определите, какую работу при этом совершил газ, чему равно изменение внутренней энергии газа, сколько теплоты подвели к нему в этом процессе?
    При решении обратите внимание на знаки искомых величин.
    Известно, что в PV-координатах график процесса 12 представляет собой отрезок прямой.
}
\answer{%
    \begin{align*}
    P_1V_1 &= \nu R T_1, P_2V_2 = \nu R T_2, \\
    \Delta U &= U_2-U_1 = \frac 32 \nu R T_2- \frac 32 \nu R T_1 = \frac 32 P_2 V_2 - \frac 32 P_1 V_1= \frac 32 \cdot \cbr{2{,}5\,\text{МПа} \cdot6\,\text{л} - 3\,\text{МПа} \cdot7\,\text{л}} = -9{,}000\,\text{кДж}.
    \\
    A_\text{газа} &= \frac{P_2 + P_1} 2 \cdot (V_2 - V_1) = \frac{2{,}5\,\text{МПа} + 3\,\text{МПа}} 2 \cdot (6\,\text{л} - 7\,\text{л}) = -2{,}750\,\text{кДж}, \\
    Q &= A_\text{газа} + \Delta U = \frac 32 (P_2 V_2 - P_1 V_1) + \frac{P_2 + P_1} 2 \cdot (V_2 - V_1) = -9{,}000\,\text{кДж} -2{,}750\,\text{кДж} = -11{,}7500\,\text{кДж}.
    \end{align*}
}
\solutionspace{150pt}

\tasknumber{3}%
\task{%
    Запишите формулы и рядом с каждой физичической величиной укажите её название и единицы изменения в СИ:
    \begin{enumerate}
        \item первое начало термодинамики,
        \item внутренняя энергия идеального одноатомного газа.
    \end{enumerate}
}

\variantsplitter

\addpersonalvariant{Андрей Щербаков}

\tasknumber{1}%
\task{%
    Определите КПД цикла 12341, рабочим телом которого является идеальный одноатомный газ, если
    12 — изобарическое расширение газа в шесть раз,
    23 — изохорическое охлаждение газа, при котором температура уменьшается в два раза,
    34 — изобара, 41 — изохора.
    % Для этого:
    % \begin{enumerate}
    %     \item сделайте рисунок в PV-координатах,
    %     \item выберите удобные обозначения, чтобы не запутаться в множестве температур, давлений и объёмов,
    %     \item вычислите необходимые соотнощения между температурами, давлениями и объёмами
    %     (некоторые сразу видны по рисунку, некоторые — надо считать),
    %     \item определите для каждого участка поглощается или отдаётся тепло (и сколько именно:
    %     потребуется первое начало термодинамики, отдельный расчёт работ на участках через площади фигур и изменений внутренней энергии),
    %     \item вычислите полную работу газа в цикле,
    %     \item подставьте всё в формулу для КПД, упростите и доведите до ответа.
    % \end{enumerate}
    Определите КПД цикла Карно, температура нагревателя которого равна максимальной температуре в цикле 12341, а холодильника — минимальной.
    Ответы в обоих случаях оставьте точными в виде нескоратимой дроби, никаких округлений.
}
\answer{%
    \begin{align*}
    A_{12} &> 0, \Delta U_{12} > 0, \implies Q_{12} = A_{12} + \Delta U_{12} > 0, \\
    A_{23} &= 0, \Delta U_{23} < 0, \implies Q_{23} = A_{23} + \Delta U_{23} < 0, \\
    A_{34} &< 0, \Delta U_{34} < 0, \implies Q_{34} = A_{34} + \Delta U_{34} < 0, \\
    A_{41} &= 0, \Delta U_{41} > 0, \implies Q_{41} = A_{41} + \Delta U_{41} > 0.
    \\
    P_1V_1 &= \nu R T_1, P_2V_2 = \nu R T_2, P_3V_3 = \nu R T_3, P_4V_4 = \nu R T_4 \text{ — уравнения состояния идеального газа}, \\
    &\text{Пусть $P_0$, $V_0$, $T_0$ — давление, объём и температура в точке 4 (минимальные во всём цикле):} \\
    P_1 &= P_2, P_3 = P_4 = P_0, V_1 = V_4 = V_0, V_2 = V_3 = 6 V_1 = 6 V_0,, \text{остальные соотношения между объёмами и давлениями не даны, нужно считать} \\
    T_3 &= \frac{T_2}2 \text{(по условию)} \implies \frac{P_2}{P_3} = \frac{P_2 V_2}{P_3 V_3}= \frac{\nu R T_2}{\nu R T_3} = \frac{T_2}{T_3} = 2 \implies P_1 = P_2 = 2 P_0 \\
    A_\text{цикл} &= (2P_0 - P_0)(6V_0 - V_0) = 5P_0V_0, \\
    A_{12} &= 2P_0 \cdot (6V_0 - V_0) = 10P_0V_0, \\
    \Delta U_{12} &= \frac 32 \nu R T_2 - \frac 32 \nu R T_1 = \frac 32 P_2 V_2 - \frac 32 P_1 V_1 = \frac 32 \cdot 2 P_0 \cdot 6 V_0 -  \frac 32 \cdot 2 P_0 \cdot V_0 = \frac 32 \cdot 10 \cdot P_0V_0, \\
    \Delta U_{41} &= \frac 32 \nu R T_1 - \frac 32 \nu R T_4 = \frac 32 P_1 V_1 - \frac 32 P_4 V_4 = \frac 32 \cdot 2 P_0 V_0 - \frac 32 P_0 V_0 = \frac 32 \cdot 1 \cdot P_0V_0.
    \\
    \eta &= \frac{A_\text{цикл}}{Q_+} = \frac{A_\text{цикл}}{Q_{12} + Q_{41}}  = \frac{A_\text{цикл}}{A_{12} + \Delta U_{12} + A_{41} + \Delta U_{41}} =  \\
     &= \frac{5P_0V_0}{10P_0V_0 + \frac 32 \cdot 10 \cdot P_0V_0 + 0 + \frac 32 \cdot 1 \cdot P_0V_0} = \frac{5}{10 + \frac 32 \cdot 10 + \frac 32 \cdot 1} = \frac{10}{53} \approx 0{,}189.
     \\
    \eta_\text{Карно} &= 1 - \frac{T_\text{х}}{T_\text{н}} = 1 - \frac{T_\text{4}}{T_\text{2}} = 1 - \frac{\frac{P_4V_4}{\nu R}}{\frac{P_2V_2}{\nu R}} = 1 - \frac{P_4V_4}{P_2V_2} = 1 - \frac{P_0V_0}{2P_0 \cdot 6V_0} = 1 - \frac 1{2 \cdot 6}  = \frac{11}{12} \approx 0{,}917.
    \end{align*}
}
\solutionspace{360pt}

\tasknumber{2}%
\task{%
    Порция идеального одноатомного газа перешла из состояния 1 в состояние 2: $P_1 = 4\,\text{МПа}$, $V_1 = 7\,\text{л}$, $P_2 = 4{,}5\,\text{МПа}$, $V_2 = 6\,\text{л}$.
    Определите, какую работу при этом совершил газ, чему равно изменение внутренней энергии газа, сколько теплоты подвели к нему в этом процессе?
    При решении обратите внимание на знаки искомых величин.
    Известно, что в PV-координатах график процесса 12 представляет собой отрезок прямой.
}
\answer{%
    \begin{align*}
    P_1V_1 &= \nu R T_1, P_2V_2 = \nu R T_2, \\
    \Delta U &= U_2-U_1 = \frac 32 \nu R T_2- \frac 32 \nu R T_1 = \frac 32 P_2 V_2 - \frac 32 P_1 V_1= \frac 32 \cdot \cbr{4{,}5\,\text{МПа} \cdot6\,\text{л} - 4\,\text{МПа} \cdot7\,\text{л}} = -1{,}5000\,\text{кДж}.
    \\
    A_\text{газа} &= \frac{P_2 + P_1} 2 \cdot (V_2 - V_1) = \frac{4{,}5\,\text{МПа} + 4\,\text{МПа}} 2 \cdot (6\,\text{л} - 7\,\text{л}) = -4{,}250\,\text{кДж}, \\
    Q &= A_\text{газа} + \Delta U = \frac 32 (P_2 V_2 - P_1 V_1) + \frac{P_2 + P_1} 2 \cdot (V_2 - V_1) = -1{,}5000\,\text{кДж} -4{,}250\,\text{кДж} = -5{,}750\,\text{кДж}.
    \end{align*}
}
\solutionspace{150pt}

\tasknumber{3}%
\task{%
    Запишите формулы и рядом с каждой физичической величиной укажите её название и единицы изменения в СИ:
    \begin{enumerate}
        \item первое начало термодинамики,
        \item внутренняя энергия идеального одноатомного газа.
    \end{enumerate}
}

\variantsplitter

\addpersonalvariant{Михаил Ярошевский}

\tasknumber{1}%
\task{%
    Определите КПД цикла 12341, рабочим телом которого является идеальный одноатомный газ, если
    12 — изобарическое расширение газа в шесть раз,
    23 — изохорическое охлаждение газа, при котором температура уменьшается в три раза,
    34 — изобара, 41 — изохора.
    % Для этого:
    % \begin{enumerate}
    %     \item сделайте рисунок в PV-координатах,
    %     \item выберите удобные обозначения, чтобы не запутаться в множестве температур, давлений и объёмов,
    %     \item вычислите необходимые соотнощения между температурами, давлениями и объёмами
    %     (некоторые сразу видны по рисунку, некоторые — надо считать),
    %     \item определите для каждого участка поглощается или отдаётся тепло (и сколько именно:
    %     потребуется первое начало термодинамики, отдельный расчёт работ на участках через площади фигур и изменений внутренней энергии),
    %     \item вычислите полную работу газа в цикле,
    %     \item подставьте всё в формулу для КПД, упростите и доведите до ответа.
    % \end{enumerate}
    Определите КПД цикла Карно, температура нагревателя которого равна максимальной температуре в цикле 12341, а холодильника — минимальной.
    Ответы в обоих случаях оставьте точными в виде нескоратимой дроби, никаких округлений.
}
\answer{%
    \begin{align*}
    A_{12} &> 0, \Delta U_{12} > 0, \implies Q_{12} = A_{12} + \Delta U_{12} > 0, \\
    A_{23} &= 0, \Delta U_{23} < 0, \implies Q_{23} = A_{23} + \Delta U_{23} < 0, \\
    A_{34} &< 0, \Delta U_{34} < 0, \implies Q_{34} = A_{34} + \Delta U_{34} < 0, \\
    A_{41} &= 0, \Delta U_{41} > 0, \implies Q_{41} = A_{41} + \Delta U_{41} > 0.
    \\
    P_1V_1 &= \nu R T_1, P_2V_2 = \nu R T_2, P_3V_3 = \nu R T_3, P_4V_4 = \nu R T_4 \text{ — уравнения состояния идеального газа}, \\
    &\text{Пусть $P_0$, $V_0$, $T_0$ — давление, объём и температура в точке 4 (минимальные во всём цикле):} \\
    P_1 &= P_2, P_3 = P_4 = P_0, V_1 = V_4 = V_0, V_2 = V_3 = 6 V_1 = 6 V_0,, \text{остальные соотношения между объёмами и давлениями не даны, нужно считать} \\
    T_3 &= \frac{T_2}3 \text{(по условию)} \implies \frac{P_2}{P_3} = \frac{P_2 V_2}{P_3 V_3}= \frac{\nu R T_2}{\nu R T_3} = \frac{T_2}{T_3} = 3 \implies P_1 = P_2 = 3 P_0 \\
    A_\text{цикл} &= (3P_0 - P_0)(6V_0 - V_0) = 10P_0V_0, \\
    A_{12} &= 3P_0 \cdot (6V_0 - V_0) = 15P_0V_0, \\
    \Delta U_{12} &= \frac 32 \nu R T_2 - \frac 32 \nu R T_1 = \frac 32 P_2 V_2 - \frac 32 P_1 V_1 = \frac 32 \cdot 3 P_0 \cdot 6 V_0 -  \frac 32 \cdot 3 P_0 \cdot V_0 = \frac 32 \cdot 15 \cdot P_0V_0, \\
    \Delta U_{41} &= \frac 32 \nu R T_1 - \frac 32 \nu R T_4 = \frac 32 P_1 V_1 - \frac 32 P_4 V_4 = \frac 32 \cdot 3 P_0 V_0 - \frac 32 P_0 V_0 = \frac 32 \cdot 2 \cdot P_0V_0.
    \\
    \eta &= \frac{A_\text{цикл}}{Q_+} = \frac{A_\text{цикл}}{Q_{12} + Q_{41}}  = \frac{A_\text{цикл}}{A_{12} + \Delta U_{12} + A_{41} + \Delta U_{41}} =  \\
     &= \frac{10P_0V_0}{15P_0V_0 + \frac 32 \cdot 15 \cdot P_0V_0 + 0 + \frac 32 \cdot 2 \cdot P_0V_0} = \frac{10}{15 + \frac 32 \cdot 15 + \frac 32 \cdot 2} = \frac{20}{81} \approx 0{,}247.
     \\
    \eta_\text{Карно} &= 1 - \frac{T_\text{х}}{T_\text{н}} = 1 - \frac{T_\text{4}}{T_\text{2}} = 1 - \frac{\frac{P_4V_4}{\nu R}}{\frac{P_2V_2}{\nu R}} = 1 - \frac{P_4V_4}{P_2V_2} = 1 - \frac{P_0V_0}{3P_0 \cdot 6V_0} = 1 - \frac 1{3 \cdot 6}  = \frac{17}{18} \approx 0{,}944.
    \end{align*}
}
\solutionspace{360pt}

\tasknumber{2}%
\task{%
    Порция идеального одноатомного газа перешла из состояния 1 в состояние 2: $P_1 = 2\,\text{МПа}$, $V_1 = 3\,\text{л}$, $P_2 = 3{,}5\,\text{МПа}$, $V_2 = 4\,\text{л}$.
    Определите, какую работу при этом совершил газ, чему равно изменение внутренней энергии газа, сколько теплоты подвели к нему в этом процессе?
    При решении обратите внимание на знаки искомых величин.
    Известно, что в PV-координатах график процесса 12 представляет собой отрезок прямой.
}
\answer{%
    \begin{align*}
    P_1V_1 &= \nu R T_1, P_2V_2 = \nu R T_2, \\
    \Delta U &= U_2-U_1 = \frac 32 \nu R T_2- \frac 32 \nu R T_1 = \frac 32 P_2 V_2 - \frac 32 P_1 V_1= \frac 32 \cdot \cbr{3{,}5\,\text{МПа} \cdot4\,\text{л} - 2\,\text{МПа} \cdot3\,\text{л}} = 12{,}00\,\text{кДж}.
    \\
    A_\text{газа} &= \frac{P_2 + P_1} 2 \cdot (V_2 - V_1) = \frac{3{,}5\,\text{МПа} + 2\,\text{МПа}} 2 \cdot (4\,\text{л} - 3\,\text{л}) = 2{,}75\,\text{кДж}, \\
    Q &= A_\text{газа} + \Delta U = \frac 32 (P_2 V_2 - P_1 V_1) + \frac{P_2 + P_1} 2 \cdot (V_2 - V_1) = 12{,}00\,\text{кДж} + 2{,}75\,\text{кДж} = 14{,}75\,\text{кДж}.
    \end{align*}
}
\solutionspace{150pt}

\tasknumber{3}%
\task{%
    Запишите формулы и рядом с каждой физичической величиной укажите её название и единицы изменения в СИ:
    \begin{enumerate}
        \item первое начало термодинамики,
        \item внутренняя энергия идеального одноатомного газа.
    \end{enumerate}
}

\variantsplitter

\addpersonalvariant{Алексей Алимпиев}

\tasknumber{1}%
\task{%
    Определите КПД цикла 12341, рабочим телом которого является идеальный одноатомный газ, если
    12 — изобарическое расширение газа в три раза,
    23 — изохорическое охлаждение газа, при котором температура уменьшается в шесть раз,
    34 — изобара, 41 — изохора.
    % Для этого:
    % \begin{enumerate}
    %     \item сделайте рисунок в PV-координатах,
    %     \item выберите удобные обозначения, чтобы не запутаться в множестве температур, давлений и объёмов,
    %     \item вычислите необходимые соотнощения между температурами, давлениями и объёмами
    %     (некоторые сразу видны по рисунку, некоторые — надо считать),
    %     \item определите для каждого участка поглощается или отдаётся тепло (и сколько именно:
    %     потребуется первое начало термодинамики, отдельный расчёт работ на участках через площади фигур и изменений внутренней энергии),
    %     \item вычислите полную работу газа в цикле,
    %     \item подставьте всё в формулу для КПД, упростите и доведите до ответа.
    % \end{enumerate}
    Определите КПД цикла Карно, температура нагревателя которого равна максимальной температуре в цикле 12341, а холодильника — минимальной.
    Ответы в обоих случаях оставьте точными в виде нескоратимой дроби, никаких округлений.
}
\answer{%
    \begin{align*}
    A_{12} &> 0, \Delta U_{12} > 0, \implies Q_{12} = A_{12} + \Delta U_{12} > 0, \\
    A_{23} &= 0, \Delta U_{23} < 0, \implies Q_{23} = A_{23} + \Delta U_{23} < 0, \\
    A_{34} &< 0, \Delta U_{34} < 0, \implies Q_{34} = A_{34} + \Delta U_{34} < 0, \\
    A_{41} &= 0, \Delta U_{41} > 0, \implies Q_{41} = A_{41} + \Delta U_{41} > 0.
    \\
    P_1V_1 &= \nu R T_1, P_2V_2 = \nu R T_2, P_3V_3 = \nu R T_3, P_4V_4 = \nu R T_4 \text{ — уравнения состояния идеального газа}, \\
    &\text{Пусть $P_0$, $V_0$, $T_0$ — давление, объём и температура в точке 4 (минимальные во всём цикле):} \\
    P_1 &= P_2, P_3 = P_4 = P_0, V_1 = V_4 = V_0, V_2 = V_3 = 3 V_1 = 3 V_0,, \text{остальные соотношения между объёмами и давлениями не даны, нужно считать} \\
    T_3 &= \frac{T_2}6 \text{(по условию)} \implies \frac{P_2}{P_3} = \frac{P_2 V_2}{P_3 V_3}= \frac{\nu R T_2}{\nu R T_3} = \frac{T_2}{T_3} = 6 \implies P_1 = P_2 = 6 P_0 \\
    A_\text{цикл} &= (6P_0 - P_0)(3V_0 - V_0) = 10P_0V_0, \\
    A_{12} &= 6P_0 \cdot (3V_0 - V_0) = 12P_0V_0, \\
    \Delta U_{12} &= \frac 32 \nu R T_2 - \frac 32 \nu R T_1 = \frac 32 P_2 V_2 - \frac 32 P_1 V_1 = \frac 32 \cdot 6 P_0 \cdot 3 V_0 -  \frac 32 \cdot 6 P_0 \cdot V_0 = \frac 32 \cdot 12 \cdot P_0V_0, \\
    \Delta U_{41} &= \frac 32 \nu R T_1 - \frac 32 \nu R T_4 = \frac 32 P_1 V_1 - \frac 32 P_4 V_4 = \frac 32 \cdot 6 P_0 V_0 - \frac 32 P_0 V_0 = \frac 32 \cdot 5 \cdot P_0V_0.
    \\
    \eta &= \frac{A_\text{цикл}}{Q_+} = \frac{A_\text{цикл}}{Q_{12} + Q_{41}}  = \frac{A_\text{цикл}}{A_{12} + \Delta U_{12} + A_{41} + \Delta U_{41}} =  \\
     &= \frac{10P_0V_0}{12P_0V_0 + \frac 32 \cdot 12 \cdot P_0V_0 + 0 + \frac 32 \cdot 5 \cdot P_0V_0} = \frac{10}{12 + \frac 32 \cdot 12 + \frac 32 \cdot 5} = \frac{4}{15} \approx 0{,}267.
     \\
    \eta_\text{Карно} &= 1 - \frac{T_\text{х}}{T_\text{н}} = 1 - \frac{T_\text{4}}{T_\text{2}} = 1 - \frac{\frac{P_4V_4}{\nu R}}{\frac{P_2V_2}{\nu R}} = 1 - \frac{P_4V_4}{P_2V_2} = 1 - \frac{P_0V_0}{6P_0 \cdot 3V_0} = 1 - \frac 1{6 \cdot 3}  = \frac{17}{18} \approx 0{,}944.
    \end{align*}
}
\solutionspace{360pt}

\tasknumber{2}%
\task{%
    Порция идеального одноатомного газа перешла из состояния 1 в состояние 2: $P_1 = 2\,\text{МПа}$, $V_1 = 3\,\text{л}$, $P_2 = 1{,}5\,\text{МПа}$, $V_2 = 2\,\text{л}$.
    Определите, какую работу при этом совершил газ, чему равно изменение внутренней энергии газа, сколько теплоты подвели к нему в этом процессе?
    При решении обратите внимание на знаки искомых величин.
    Известно, что в PV-координатах график процесса 12 представляет собой отрезок прямой.
}
\answer{%
    \begin{align*}
    P_1V_1 &= \nu R T_1, P_2V_2 = \nu R T_2, \\
    \Delta U &= U_2-U_1 = \frac 32 \nu R T_2- \frac 32 \nu R T_1 = \frac 32 P_2 V_2 - \frac 32 P_1 V_1= \frac 32 \cdot \cbr{1{,}5\,\text{МПа} \cdot2\,\text{л} - 2\,\text{МПа} \cdot3\,\text{л}} = -4{,}500\,\text{кДж}.
    \\
    A_\text{газа} &= \frac{P_2 + P_1} 2 \cdot (V_2 - V_1) = \frac{1{,}5\,\text{МПа} + 2\,\text{МПа}} 2 \cdot (2\,\text{л} - 3\,\text{л}) = -1{,}7500\,\text{кДж}, \\
    Q &= A_\text{газа} + \Delta U = \frac 32 (P_2 V_2 - P_1 V_1) + \frac{P_2 + P_1} 2 \cdot (V_2 - V_1) = -4{,}500\,\text{кДж} -1{,}7500\,\text{кДж} = -6{,}250\,\text{кДж}.
    \end{align*}
}
\solutionspace{150pt}

\tasknumber{3}%
\task{%
    Запишите формулы и рядом с каждой физичической величиной укажите её название и единицы изменения в СИ:
    \begin{enumerate}
        \item первое начало термодинамики,
        \item внутренняя энергия идеального одноатомного газа.
    \end{enumerate}
}

\variantsplitter

\addpersonalvariant{Евгений Васин}

\tasknumber{1}%
\task{%
    Определите КПД цикла 12341, рабочим телом которого является идеальный одноатомный газ, если
    12 — изобарическое расширение газа в четыре раза,
    23 — изохорическое охлаждение газа, при котором температура уменьшается в пять раз,
    34 — изобара, 41 — изохора.
    % Для этого:
    % \begin{enumerate}
    %     \item сделайте рисунок в PV-координатах,
    %     \item выберите удобные обозначения, чтобы не запутаться в множестве температур, давлений и объёмов,
    %     \item вычислите необходимые соотнощения между температурами, давлениями и объёмами
    %     (некоторые сразу видны по рисунку, некоторые — надо считать),
    %     \item определите для каждого участка поглощается или отдаётся тепло (и сколько именно:
    %     потребуется первое начало термодинамики, отдельный расчёт работ на участках через площади фигур и изменений внутренней энергии),
    %     \item вычислите полную работу газа в цикле,
    %     \item подставьте всё в формулу для КПД, упростите и доведите до ответа.
    % \end{enumerate}
    Определите КПД цикла Карно, температура нагревателя которого равна максимальной температуре в цикле 12341, а холодильника — минимальной.
    Ответы в обоих случаях оставьте точными в виде нескоратимой дроби, никаких округлений.
}
\answer{%
    \begin{align*}
    A_{12} &> 0, \Delta U_{12} > 0, \implies Q_{12} = A_{12} + \Delta U_{12} > 0, \\
    A_{23} &= 0, \Delta U_{23} < 0, \implies Q_{23} = A_{23} + \Delta U_{23} < 0, \\
    A_{34} &< 0, \Delta U_{34} < 0, \implies Q_{34} = A_{34} + \Delta U_{34} < 0, \\
    A_{41} &= 0, \Delta U_{41} > 0, \implies Q_{41} = A_{41} + \Delta U_{41} > 0.
    \\
    P_1V_1 &= \nu R T_1, P_2V_2 = \nu R T_2, P_3V_3 = \nu R T_3, P_4V_4 = \nu R T_4 \text{ — уравнения состояния идеального газа}, \\
    &\text{Пусть $P_0$, $V_0$, $T_0$ — давление, объём и температура в точке 4 (минимальные во всём цикле):} \\
    P_1 &= P_2, P_3 = P_4 = P_0, V_1 = V_4 = V_0, V_2 = V_3 = 4 V_1 = 4 V_0,, \text{остальные соотношения между объёмами и давлениями не даны, нужно считать} \\
    T_3 &= \frac{T_2}5 \text{(по условию)} \implies \frac{P_2}{P_3} = \frac{P_2 V_2}{P_3 V_3}= \frac{\nu R T_2}{\nu R T_3} = \frac{T_2}{T_3} = 5 \implies P_1 = P_2 = 5 P_0 \\
    A_\text{цикл} &= (5P_0 - P_0)(4V_0 - V_0) = 12P_0V_0, \\
    A_{12} &= 5P_0 \cdot (4V_0 - V_0) = 15P_0V_0, \\
    \Delta U_{12} &= \frac 32 \nu R T_2 - \frac 32 \nu R T_1 = \frac 32 P_2 V_2 - \frac 32 P_1 V_1 = \frac 32 \cdot 5 P_0 \cdot 4 V_0 -  \frac 32 \cdot 5 P_0 \cdot V_0 = \frac 32 \cdot 15 \cdot P_0V_0, \\
    \Delta U_{41} &= \frac 32 \nu R T_1 - \frac 32 \nu R T_4 = \frac 32 P_1 V_1 - \frac 32 P_4 V_4 = \frac 32 \cdot 5 P_0 V_0 - \frac 32 P_0 V_0 = \frac 32 \cdot 4 \cdot P_0V_0.
    \\
    \eta &= \frac{A_\text{цикл}}{Q_+} = \frac{A_\text{цикл}}{Q_{12} + Q_{41}}  = \frac{A_\text{цикл}}{A_{12} + \Delta U_{12} + A_{41} + \Delta U_{41}} =  \\
     &= \frac{12P_0V_0}{15P_0V_0 + \frac 32 \cdot 15 \cdot P_0V_0 + 0 + \frac 32 \cdot 4 \cdot P_0V_0} = \frac{12}{15 + \frac 32 \cdot 15 + \frac 32 \cdot 4} = \frac{8}{29} \approx 0{,}276.
     \\
    \eta_\text{Карно} &= 1 - \frac{T_\text{х}}{T_\text{н}} = 1 - \frac{T_\text{4}}{T_\text{2}} = 1 - \frac{\frac{P_4V_4}{\nu R}}{\frac{P_2V_2}{\nu R}} = 1 - \frac{P_4V_4}{P_2V_2} = 1 - \frac{P_0V_0}{5P_0 \cdot 4V_0} = 1 - \frac 1{5 \cdot 4}  = \frac{19}{20} \approx 0{,}950.
    \end{align*}
}
\solutionspace{360pt}

\tasknumber{2}%
\task{%
    Порция идеального одноатомного газа перешла из состояния 1 в состояние 2: $P_1 = 2\,\text{МПа}$, $V_1 = 7\,\text{л}$, $P_2 = 3{,}5\,\text{МПа}$, $V_2 = 4\,\text{л}$.
    Определите, какую работу при этом совершил газ, чему равно изменение внутренней энергии газа, сколько теплоты подвели к нему в этом процессе?
    При решении обратите внимание на знаки искомых величин.
    Известно, что в PV-координатах график процесса 12 представляет собой отрезок прямой.
}
\answer{%
    \begin{align*}
    P_1V_1 &= \nu R T_1, P_2V_2 = \nu R T_2, \\
    \Delta U &= U_2-U_1 = \frac 32 \nu R T_2- \frac 32 \nu R T_1 = \frac 32 P_2 V_2 - \frac 32 P_1 V_1= \frac 32 \cdot \cbr{3{,}5\,\text{МПа} \cdot4\,\text{л} - 2\,\text{МПа} \cdot7\,\text{л}} = 0\,\text{кДж}.
    \\
    A_\text{газа} &= \frac{P_2 + P_1} 2 \cdot (V_2 - V_1) = \frac{3{,}5\,\text{МПа} + 2\,\text{МПа}} 2 \cdot (4\,\text{л} - 7\,\text{л}) = -8{,}250\,\text{кДж}, \\
    Q &= A_\text{газа} + \Delta U = \frac 32 (P_2 V_2 - P_1 V_1) + \frac{P_2 + P_1} 2 \cdot (V_2 - V_1) = 0\,\text{кДж} -8{,}250\,\text{кДж} = -8{,}250\,\text{кДж}.
    \end{align*}
}
\solutionspace{150pt}

\tasknumber{3}%
\task{%
    Запишите формулы и рядом с каждой физичической величиной укажите её название и единицы изменения в СИ:
    \begin{enumerate}
        \item первое начало термодинамики,
        \item внутренняя энергия идеального одноатомного газа.
    \end{enumerate}
}

\variantsplitter

\addpersonalvariant{Волохов Вячеслав}

\tasknumber{1}%
\task{%
    Определите КПД цикла 12341, рабочим телом которого является идеальный одноатомный газ, если
    12 — изобарическое расширение газа в шесть раз,
    23 — изохорическое охлаждение газа, при котором температура уменьшается в четыре раза,
    34 — изобара, 41 — изохора.
    % Для этого:
    % \begin{enumerate}
    %     \item сделайте рисунок в PV-координатах,
    %     \item выберите удобные обозначения, чтобы не запутаться в множестве температур, давлений и объёмов,
    %     \item вычислите необходимые соотнощения между температурами, давлениями и объёмами
    %     (некоторые сразу видны по рисунку, некоторые — надо считать),
    %     \item определите для каждого участка поглощается или отдаётся тепло (и сколько именно:
    %     потребуется первое начало термодинамики, отдельный расчёт работ на участках через площади фигур и изменений внутренней энергии),
    %     \item вычислите полную работу газа в цикле,
    %     \item подставьте всё в формулу для КПД, упростите и доведите до ответа.
    % \end{enumerate}
    Определите КПД цикла Карно, температура нагревателя которого равна максимальной температуре в цикле 12341, а холодильника — минимальной.
    Ответы в обоих случаях оставьте точными в виде нескоратимой дроби, никаких округлений.
}
\answer{%
    \begin{align*}
    A_{12} &> 0, \Delta U_{12} > 0, \implies Q_{12} = A_{12} + \Delta U_{12} > 0, \\
    A_{23} &= 0, \Delta U_{23} < 0, \implies Q_{23} = A_{23} + \Delta U_{23} < 0, \\
    A_{34} &< 0, \Delta U_{34} < 0, \implies Q_{34} = A_{34} + \Delta U_{34} < 0, \\
    A_{41} &= 0, \Delta U_{41} > 0, \implies Q_{41} = A_{41} + \Delta U_{41} > 0.
    \\
    P_1V_1 &= \nu R T_1, P_2V_2 = \nu R T_2, P_3V_3 = \nu R T_3, P_4V_4 = \nu R T_4 \text{ — уравнения состояния идеального газа}, \\
    &\text{Пусть $P_0$, $V_0$, $T_0$ — давление, объём и температура в точке 4 (минимальные во всём цикле):} \\
    P_1 &= P_2, P_3 = P_4 = P_0, V_1 = V_4 = V_0, V_2 = V_3 = 6 V_1 = 6 V_0,, \text{остальные соотношения между объёмами и давлениями не даны, нужно считать} \\
    T_3 &= \frac{T_2}4 \text{(по условию)} \implies \frac{P_2}{P_3} = \frac{P_2 V_2}{P_3 V_3}= \frac{\nu R T_2}{\nu R T_3} = \frac{T_2}{T_3} = 4 \implies P_1 = P_2 = 4 P_0 \\
    A_\text{цикл} &= (4P_0 - P_0)(6V_0 - V_0) = 15P_0V_0, \\
    A_{12} &= 4P_0 \cdot (6V_0 - V_0) = 20P_0V_0, \\
    \Delta U_{12} &= \frac 32 \nu R T_2 - \frac 32 \nu R T_1 = \frac 32 P_2 V_2 - \frac 32 P_1 V_1 = \frac 32 \cdot 4 P_0 \cdot 6 V_0 -  \frac 32 \cdot 4 P_0 \cdot V_0 = \frac 32 \cdot 20 \cdot P_0V_0, \\
    \Delta U_{41} &= \frac 32 \nu R T_1 - \frac 32 \nu R T_4 = \frac 32 P_1 V_1 - \frac 32 P_4 V_4 = \frac 32 \cdot 4 P_0 V_0 - \frac 32 P_0 V_0 = \frac 32 \cdot 3 \cdot P_0V_0.
    \\
    \eta &= \frac{A_\text{цикл}}{Q_+} = \frac{A_\text{цикл}}{Q_{12} + Q_{41}}  = \frac{A_\text{цикл}}{A_{12} + \Delta U_{12} + A_{41} + \Delta U_{41}} =  \\
     &= \frac{15P_0V_0}{20P_0V_0 + \frac 32 \cdot 20 \cdot P_0V_0 + 0 + \frac 32 \cdot 3 \cdot P_0V_0} = \frac{15}{20 + \frac 32 \cdot 20 + \frac 32 \cdot 3} = \frac{30}{109} \approx 0{,}275.
     \\
    \eta_\text{Карно} &= 1 - \frac{T_\text{х}}{T_\text{н}} = 1 - \frac{T_\text{4}}{T_\text{2}} = 1 - \frac{\frac{P_4V_4}{\nu R}}{\frac{P_2V_2}{\nu R}} = 1 - \frac{P_4V_4}{P_2V_2} = 1 - \frac{P_0V_0}{4P_0 \cdot 6V_0} = 1 - \frac 1{4 \cdot 6}  = \frac{23}{24} \approx 0{,}958.
    \end{align*}
}
\solutionspace{360pt}

\tasknumber{2}%
\task{%
    Порция идеального одноатомного газа перешла из состояния 1 в состояние 2: $P_1 = 4\,\text{МПа}$, $V_1 = 3\,\text{л}$, $P_2 = 2{,}5\,\text{МПа}$, $V_2 = 4\,\text{л}$.
    Определите, какую работу при этом совершил газ, чему равно изменение внутренней энергии газа, сколько теплоты подвели к нему в этом процессе?
    При решении обратите внимание на знаки искомых величин.
    Известно, что в PV-координатах график процесса 12 представляет собой отрезок прямой.
}
\answer{%
    \begin{align*}
    P_1V_1 &= \nu R T_1, P_2V_2 = \nu R T_2, \\
    \Delta U &= U_2-U_1 = \frac 32 \nu R T_2- \frac 32 \nu R T_1 = \frac 32 P_2 V_2 - \frac 32 P_1 V_1= \frac 32 \cdot \cbr{2{,}5\,\text{МПа} \cdot4\,\text{л} - 4\,\text{МПа} \cdot3\,\text{л}} = -3{,}000\,\text{кДж}.
    \\
    A_\text{газа} &= \frac{P_2 + P_1} 2 \cdot (V_2 - V_1) = \frac{2{,}5\,\text{МПа} + 4\,\text{МПа}} 2 \cdot (4\,\text{л} - 3\,\text{л}) = 3{,}25\,\text{кДж}, \\
    Q &= A_\text{газа} + \Delta U = \frac 32 (P_2 V_2 - P_1 V_1) + \frac{P_2 + P_1} 2 \cdot (V_2 - V_1) = -3{,}000\,\text{кДж} + 3{,}25\,\text{кДж} = 0{,}25\,\text{кДж}.
    \end{align*}
}
\solutionspace{150pt}

\tasknumber{3}%
\task{%
    Запишите формулы и рядом с каждой физичической величиной укажите её название и единицы изменения в СИ:
    \begin{enumerate}
        \item первое начало термодинамики,
        \item внутренняя энергия идеального одноатомного газа.
    \end{enumerate}
}

\variantsplitter

\addpersonalvariant{Герман Говоров}

\tasknumber{1}%
\task{%
    Определите КПД цикла 12341, рабочим телом которого является идеальный одноатомный газ, если
    12 — изобарическое расширение газа в два раза,
    23 — изохорическое охлаждение газа, при котором температура уменьшается в пять раз,
    34 — изобара, 41 — изохора.
    % Для этого:
    % \begin{enumerate}
    %     \item сделайте рисунок в PV-координатах,
    %     \item выберите удобные обозначения, чтобы не запутаться в множестве температур, давлений и объёмов,
    %     \item вычислите необходимые соотнощения между температурами, давлениями и объёмами
    %     (некоторые сразу видны по рисунку, некоторые — надо считать),
    %     \item определите для каждого участка поглощается или отдаётся тепло (и сколько именно:
    %     потребуется первое начало термодинамики, отдельный расчёт работ на участках через площади фигур и изменений внутренней энергии),
    %     \item вычислите полную работу газа в цикле,
    %     \item подставьте всё в формулу для КПД, упростите и доведите до ответа.
    % \end{enumerate}
    Определите КПД цикла Карно, температура нагревателя которого равна максимальной температуре в цикле 12341, а холодильника — минимальной.
    Ответы в обоих случаях оставьте точными в виде нескоратимой дроби, никаких округлений.
}
\answer{%
    \begin{align*}
    A_{12} &> 0, \Delta U_{12} > 0, \implies Q_{12} = A_{12} + \Delta U_{12} > 0, \\
    A_{23} &= 0, \Delta U_{23} < 0, \implies Q_{23} = A_{23} + \Delta U_{23} < 0, \\
    A_{34} &< 0, \Delta U_{34} < 0, \implies Q_{34} = A_{34} + \Delta U_{34} < 0, \\
    A_{41} &= 0, \Delta U_{41} > 0, \implies Q_{41} = A_{41} + \Delta U_{41} > 0.
    \\
    P_1V_1 &= \nu R T_1, P_2V_2 = \nu R T_2, P_3V_3 = \nu R T_3, P_4V_4 = \nu R T_4 \text{ — уравнения состояния идеального газа}, \\
    &\text{Пусть $P_0$, $V_0$, $T_0$ — давление, объём и температура в точке 4 (минимальные во всём цикле):} \\
    P_1 &= P_2, P_3 = P_4 = P_0, V_1 = V_4 = V_0, V_2 = V_3 = 2 V_1 = 2 V_0,, \text{остальные соотношения между объёмами и давлениями не даны, нужно считать} \\
    T_3 &= \frac{T_2}5 \text{(по условию)} \implies \frac{P_2}{P_3} = \frac{P_2 V_2}{P_3 V_3}= \frac{\nu R T_2}{\nu R T_3} = \frac{T_2}{T_3} = 5 \implies P_1 = P_2 = 5 P_0 \\
    A_\text{цикл} &= (5P_0 - P_0)(2V_0 - V_0) = 4P_0V_0, \\
    A_{12} &= 5P_0 \cdot (2V_0 - V_0) = 5P_0V_0, \\
    \Delta U_{12} &= \frac 32 \nu R T_2 - \frac 32 \nu R T_1 = \frac 32 P_2 V_2 - \frac 32 P_1 V_1 = \frac 32 \cdot 5 P_0 \cdot 2 V_0 -  \frac 32 \cdot 5 P_0 \cdot V_0 = \frac 32 \cdot 5 \cdot P_0V_0, \\
    \Delta U_{41} &= \frac 32 \nu R T_1 - \frac 32 \nu R T_4 = \frac 32 P_1 V_1 - \frac 32 P_4 V_4 = \frac 32 \cdot 5 P_0 V_0 - \frac 32 P_0 V_0 = \frac 32 \cdot 4 \cdot P_0V_0.
    \\
    \eta &= \frac{A_\text{цикл}}{Q_+} = \frac{A_\text{цикл}}{Q_{12} + Q_{41}}  = \frac{A_\text{цикл}}{A_{12} + \Delta U_{12} + A_{41} + \Delta U_{41}} =  \\
     &= \frac{4P_0V_0}{5P_0V_0 + \frac 32 \cdot 5 \cdot P_0V_0 + 0 + \frac 32 \cdot 4 \cdot P_0V_0} = \frac{4}{5 + \frac 32 \cdot 5 + \frac 32 \cdot 4} = \frac{8}{37} \approx 0{,}216.
     \\
    \eta_\text{Карно} &= 1 - \frac{T_\text{х}}{T_\text{н}} = 1 - \frac{T_\text{4}}{T_\text{2}} = 1 - \frac{\frac{P_4V_4}{\nu R}}{\frac{P_2V_2}{\nu R}} = 1 - \frac{P_4V_4}{P_2V_2} = 1 - \frac{P_0V_0}{5P_0 \cdot 2V_0} = 1 - \frac 1{5 \cdot 2}  = \frac{9}{10} \approx 0{,}900.
    \end{align*}
}
\solutionspace{360pt}

\tasknumber{2}%
\task{%
    Порция идеального одноатомного газа перешла из состояния 1 в состояние 2: $P_1 = 2\,\text{МПа}$, $V_1 = 7\,\text{л}$, $P_2 = 2{,}5\,\text{МПа}$, $V_2 = 6\,\text{л}$.
    Определите, какую работу при этом совершил газ, чему равно изменение внутренней энергии газа, сколько теплоты подвели к нему в этом процессе?
    При решении обратите внимание на знаки искомых величин.
    Известно, что в PV-координатах график процесса 12 представляет собой отрезок прямой.
}
\answer{%
    \begin{align*}
    P_1V_1 &= \nu R T_1, P_2V_2 = \nu R T_2, \\
    \Delta U &= U_2-U_1 = \frac 32 \nu R T_2- \frac 32 \nu R T_1 = \frac 32 P_2 V_2 - \frac 32 P_1 V_1= \frac 32 \cdot \cbr{2{,}5\,\text{МПа} \cdot6\,\text{л} - 2\,\text{МПа} \cdot7\,\text{л}} = 1{,}50\,\text{кДж}.
    \\
    A_\text{газа} &= \frac{P_2 + P_1} 2 \cdot (V_2 - V_1) = \frac{2{,}5\,\text{МПа} + 2\,\text{МПа}} 2 \cdot (6\,\text{л} - 7\,\text{л}) = -2{,}250\,\text{кДж}, \\
    Q &= A_\text{газа} + \Delta U = \frac 32 (P_2 V_2 - P_1 V_1) + \frac{P_2 + P_1} 2 \cdot (V_2 - V_1) = 1{,}50\,\text{кДж} -2{,}250\,\text{кДж} = -0{,}7500\,\text{кДж}.
    \end{align*}
}
\solutionspace{150pt}

\tasknumber{3}%
\task{%
    Запишите формулы и рядом с каждой физичической величиной укажите её название и единицы изменения в СИ:
    \begin{enumerate}
        \item первое начало термодинамики,
        \item внутренняя энергия идеального одноатомного газа.
    \end{enumerate}
}

\variantsplitter

\addpersonalvariant{София Журавлева}

\tasknumber{1}%
\task{%
    Определите КПД цикла 12341, рабочим телом которого является идеальный одноатомный газ, если
    12 — изобарическое расширение газа в три раза,
    23 — изохорическое охлаждение газа, при котором температура уменьшается в два раза,
    34 — изобара, 41 — изохора.
    % Для этого:
    % \begin{enumerate}
    %     \item сделайте рисунок в PV-координатах,
    %     \item выберите удобные обозначения, чтобы не запутаться в множестве температур, давлений и объёмов,
    %     \item вычислите необходимые соотнощения между температурами, давлениями и объёмами
    %     (некоторые сразу видны по рисунку, некоторые — надо считать),
    %     \item определите для каждого участка поглощается или отдаётся тепло (и сколько именно:
    %     потребуется первое начало термодинамики, отдельный расчёт работ на участках через площади фигур и изменений внутренней энергии),
    %     \item вычислите полную работу газа в цикле,
    %     \item подставьте всё в формулу для КПД, упростите и доведите до ответа.
    % \end{enumerate}
    Определите КПД цикла Карно, температура нагревателя которого равна максимальной температуре в цикле 12341, а холодильника — минимальной.
    Ответы в обоих случаях оставьте точными в виде нескоратимой дроби, никаких округлений.
}
\answer{%
    \begin{align*}
    A_{12} &> 0, \Delta U_{12} > 0, \implies Q_{12} = A_{12} + \Delta U_{12} > 0, \\
    A_{23} &= 0, \Delta U_{23} < 0, \implies Q_{23} = A_{23} + \Delta U_{23} < 0, \\
    A_{34} &< 0, \Delta U_{34} < 0, \implies Q_{34} = A_{34} + \Delta U_{34} < 0, \\
    A_{41} &= 0, \Delta U_{41} > 0, \implies Q_{41} = A_{41} + \Delta U_{41} > 0.
    \\
    P_1V_1 &= \nu R T_1, P_2V_2 = \nu R T_2, P_3V_3 = \nu R T_3, P_4V_4 = \nu R T_4 \text{ — уравнения состояния идеального газа}, \\
    &\text{Пусть $P_0$, $V_0$, $T_0$ — давление, объём и температура в точке 4 (минимальные во всём цикле):} \\
    P_1 &= P_2, P_3 = P_4 = P_0, V_1 = V_4 = V_0, V_2 = V_3 = 3 V_1 = 3 V_0,, \text{остальные соотношения между объёмами и давлениями не даны, нужно считать} \\
    T_3 &= \frac{T_2}2 \text{(по условию)} \implies \frac{P_2}{P_3} = \frac{P_2 V_2}{P_3 V_3}= \frac{\nu R T_2}{\nu R T_3} = \frac{T_2}{T_3} = 2 \implies P_1 = P_2 = 2 P_0 \\
    A_\text{цикл} &= (2P_0 - P_0)(3V_0 - V_0) = 2P_0V_0, \\
    A_{12} &= 2P_0 \cdot (3V_0 - V_0) = 4P_0V_0, \\
    \Delta U_{12} &= \frac 32 \nu R T_2 - \frac 32 \nu R T_1 = \frac 32 P_2 V_2 - \frac 32 P_1 V_1 = \frac 32 \cdot 2 P_0 \cdot 3 V_0 -  \frac 32 \cdot 2 P_0 \cdot V_0 = \frac 32 \cdot 4 \cdot P_0V_0, \\
    \Delta U_{41} &= \frac 32 \nu R T_1 - \frac 32 \nu R T_4 = \frac 32 P_1 V_1 - \frac 32 P_4 V_4 = \frac 32 \cdot 2 P_0 V_0 - \frac 32 P_0 V_0 = \frac 32 \cdot 1 \cdot P_0V_0.
    \\
    \eta &= \frac{A_\text{цикл}}{Q_+} = \frac{A_\text{цикл}}{Q_{12} + Q_{41}}  = \frac{A_\text{цикл}}{A_{12} + \Delta U_{12} + A_{41} + \Delta U_{41}} =  \\
     &= \frac{2P_0V_0}{4P_0V_0 + \frac 32 \cdot 4 \cdot P_0V_0 + 0 + \frac 32 \cdot 1 \cdot P_0V_0} = \frac{2}{4 + \frac 32 \cdot 4 + \frac 32 \cdot 1} = \frac{4}{23} \approx 0{,}174.
     \\
    \eta_\text{Карно} &= 1 - \frac{T_\text{х}}{T_\text{н}} = 1 - \frac{T_\text{4}}{T_\text{2}} = 1 - \frac{\frac{P_4V_4}{\nu R}}{\frac{P_2V_2}{\nu R}} = 1 - \frac{P_4V_4}{P_2V_2} = 1 - \frac{P_0V_0}{2P_0 \cdot 3V_0} = 1 - \frac 1{2 \cdot 3}  = \frac{5}{6} \approx 0{,}833.
    \end{align*}
}
\solutionspace{360pt}

\tasknumber{2}%
\task{%
    Порция идеального одноатомного газа перешла из состояния 1 в состояние 2: $P_1 = 4\,\text{МПа}$, $V_1 = 5\,\text{л}$, $P_2 = 2{,}5\,\text{МПа}$, $V_2 = 6\,\text{л}$.
    Определите, какую работу при этом совершил газ, чему равно изменение внутренней энергии газа, сколько теплоты подвели к нему в этом процессе?
    При решении обратите внимание на знаки искомых величин.
    Известно, что в PV-координатах график процесса 12 представляет собой отрезок прямой.
}
\answer{%
    \begin{align*}
    P_1V_1 &= \nu R T_1, P_2V_2 = \nu R T_2, \\
    \Delta U &= U_2-U_1 = \frac 32 \nu R T_2- \frac 32 \nu R T_1 = \frac 32 P_2 V_2 - \frac 32 P_1 V_1= \frac 32 \cdot \cbr{2{,}5\,\text{МПа} \cdot6\,\text{л} - 4\,\text{МПа} \cdot5\,\text{л}} = -7{,}500\,\text{кДж}.
    \\
    A_\text{газа} &= \frac{P_2 + P_1} 2 \cdot (V_2 - V_1) = \frac{2{,}5\,\text{МПа} + 4\,\text{МПа}} 2 \cdot (6\,\text{л} - 5\,\text{л}) = 3{,}25\,\text{кДж}, \\
    Q &= A_\text{газа} + \Delta U = \frac 32 (P_2 V_2 - P_1 V_1) + \frac{P_2 + P_1} 2 \cdot (V_2 - V_1) = -7{,}500\,\text{кДж} + 3{,}25\,\text{кДж} = -4{,}250\,\text{кДж}.
    \end{align*}
}
\solutionspace{150pt}

\tasknumber{3}%
\task{%
    Запишите формулы и рядом с каждой физичической величиной укажите её название и единицы изменения в СИ:
    \begin{enumerate}
        \item первое начало термодинамики,
        \item внутренняя энергия идеального одноатомного газа.
    \end{enumerate}
}

\variantsplitter

\addpersonalvariant{Константин Козлов}

\tasknumber{1}%
\task{%
    Определите КПД цикла 12341, рабочим телом которого является идеальный одноатомный газ, если
    12 — изобарическое расширение газа в шесть раз,
    23 — изохорическое охлаждение газа, при котором температура уменьшается в шесть раз,
    34 — изобара, 41 — изохора.
    % Для этого:
    % \begin{enumerate}
    %     \item сделайте рисунок в PV-координатах,
    %     \item выберите удобные обозначения, чтобы не запутаться в множестве температур, давлений и объёмов,
    %     \item вычислите необходимые соотнощения между температурами, давлениями и объёмами
    %     (некоторые сразу видны по рисунку, некоторые — надо считать),
    %     \item определите для каждого участка поглощается или отдаётся тепло (и сколько именно:
    %     потребуется первое начало термодинамики, отдельный расчёт работ на участках через площади фигур и изменений внутренней энергии),
    %     \item вычислите полную работу газа в цикле,
    %     \item подставьте всё в формулу для КПД, упростите и доведите до ответа.
    % \end{enumerate}
    Определите КПД цикла Карно, температура нагревателя которого равна максимальной температуре в цикле 12341, а холодильника — минимальной.
    Ответы в обоих случаях оставьте точными в виде нескоратимой дроби, никаких округлений.
}
\answer{%
    \begin{align*}
    A_{12} &> 0, \Delta U_{12} > 0, \implies Q_{12} = A_{12} + \Delta U_{12} > 0, \\
    A_{23} &= 0, \Delta U_{23} < 0, \implies Q_{23} = A_{23} + \Delta U_{23} < 0, \\
    A_{34} &< 0, \Delta U_{34} < 0, \implies Q_{34} = A_{34} + \Delta U_{34} < 0, \\
    A_{41} &= 0, \Delta U_{41} > 0, \implies Q_{41} = A_{41} + \Delta U_{41} > 0.
    \\
    P_1V_1 &= \nu R T_1, P_2V_2 = \nu R T_2, P_3V_3 = \nu R T_3, P_4V_4 = \nu R T_4 \text{ — уравнения состояния идеального газа}, \\
    &\text{Пусть $P_0$, $V_0$, $T_0$ — давление, объём и температура в точке 4 (минимальные во всём цикле):} \\
    P_1 &= P_2, P_3 = P_4 = P_0, V_1 = V_4 = V_0, V_2 = V_3 = 6 V_1 = 6 V_0,, \text{остальные соотношения между объёмами и давлениями не даны, нужно считать} \\
    T_3 &= \frac{T_2}6 \text{(по условию)} \implies \frac{P_2}{P_3} = \frac{P_2 V_2}{P_3 V_3}= \frac{\nu R T_2}{\nu R T_3} = \frac{T_2}{T_3} = 6 \implies P_1 = P_2 = 6 P_0 \\
    A_\text{цикл} &= (6P_0 - P_0)(6V_0 - V_0) = 25P_0V_0, \\
    A_{12} &= 6P_0 \cdot (6V_0 - V_0) = 30P_0V_0, \\
    \Delta U_{12} &= \frac 32 \nu R T_2 - \frac 32 \nu R T_1 = \frac 32 P_2 V_2 - \frac 32 P_1 V_1 = \frac 32 \cdot 6 P_0 \cdot 6 V_0 -  \frac 32 \cdot 6 P_0 \cdot V_0 = \frac 32 \cdot 30 \cdot P_0V_0, \\
    \Delta U_{41} &= \frac 32 \nu R T_1 - \frac 32 \nu R T_4 = \frac 32 P_1 V_1 - \frac 32 P_4 V_4 = \frac 32 \cdot 6 P_0 V_0 - \frac 32 P_0 V_0 = \frac 32 \cdot 5 \cdot P_0V_0.
    \\
    \eta &= \frac{A_\text{цикл}}{Q_+} = \frac{A_\text{цикл}}{Q_{12} + Q_{41}}  = \frac{A_\text{цикл}}{A_{12} + \Delta U_{12} + A_{41} + \Delta U_{41}} =  \\
     &= \frac{25P_0V_0}{30P_0V_0 + \frac 32 \cdot 30 \cdot P_0V_0 + 0 + \frac 32 \cdot 5 \cdot P_0V_0} = \frac{25}{30 + \frac 32 \cdot 30 + \frac 32 \cdot 5} = \frac{10}{33} \approx 0{,}303.
     \\
    \eta_\text{Карно} &= 1 - \frac{T_\text{х}}{T_\text{н}} = 1 - \frac{T_\text{4}}{T_\text{2}} = 1 - \frac{\frac{P_4V_4}{\nu R}}{\frac{P_2V_2}{\nu R}} = 1 - \frac{P_4V_4}{P_2V_2} = 1 - \frac{P_0V_0}{6P_0 \cdot 6V_0} = 1 - \frac 1{6 \cdot 6}  = \frac{35}{36} \approx 0{,}972.
    \end{align*}
}
\solutionspace{360pt}

\tasknumber{2}%
\task{%
    Порция идеального одноатомного газа перешла из состояния 1 в состояние 2: $P_1 = 2\,\text{МПа}$, $V_1 = 5\,\text{л}$, $P_2 = 1{,}5\,\text{МПа}$, $V_2 = 8\,\text{л}$.
    Определите, какую работу при этом совершил газ, чему равно изменение внутренней энергии газа, сколько теплоты подвели к нему в этом процессе?
    При решении обратите внимание на знаки искомых величин.
    Известно, что в PV-координатах график процесса 12 представляет собой отрезок прямой.
}
\answer{%
    \begin{align*}
    P_1V_1 &= \nu R T_1, P_2V_2 = \nu R T_2, \\
    \Delta U &= U_2-U_1 = \frac 32 \nu R T_2- \frac 32 \nu R T_1 = \frac 32 P_2 V_2 - \frac 32 P_1 V_1= \frac 32 \cdot \cbr{1{,}5\,\text{МПа} \cdot8\,\text{л} - 2\,\text{МПа} \cdot5\,\text{л}} = 3{,}00\,\text{кДж}.
    \\
    A_\text{газа} &= \frac{P_2 + P_1} 2 \cdot (V_2 - V_1) = \frac{1{,}5\,\text{МПа} + 2\,\text{МПа}} 2 \cdot (8\,\text{л} - 5\,\text{л}) = 5{,}25\,\text{кДж}, \\
    Q &= A_\text{газа} + \Delta U = \frac 32 (P_2 V_2 - P_1 V_1) + \frac{P_2 + P_1} 2 \cdot (V_2 - V_1) = 3{,}00\,\text{кДж} + 5{,}25\,\text{кДж} = 8{,}25\,\text{кДж}.
    \end{align*}
}
\solutionspace{150pt}

\tasknumber{3}%
\task{%
    Запишите формулы и рядом с каждой физичической величиной укажите её название и единицы изменения в СИ:
    \begin{enumerate}
        \item первое начало термодинамики,
        \item внутренняя энергия идеального одноатомного газа.
    \end{enumerate}
}

\variantsplitter

\addpersonalvariant{Наталья Кравченко}

\tasknumber{1}%
\task{%
    Определите КПД цикла 12341, рабочим телом которого является идеальный одноатомный газ, если
    12 — изобарическое расширение газа в два раза,
    23 — изохорическое охлаждение газа, при котором температура уменьшается в шесть раз,
    34 — изобара, 41 — изохора.
    % Для этого:
    % \begin{enumerate}
    %     \item сделайте рисунок в PV-координатах,
    %     \item выберите удобные обозначения, чтобы не запутаться в множестве температур, давлений и объёмов,
    %     \item вычислите необходимые соотнощения между температурами, давлениями и объёмами
    %     (некоторые сразу видны по рисунку, некоторые — надо считать),
    %     \item определите для каждого участка поглощается или отдаётся тепло (и сколько именно:
    %     потребуется первое начало термодинамики, отдельный расчёт работ на участках через площади фигур и изменений внутренней энергии),
    %     \item вычислите полную работу газа в цикле,
    %     \item подставьте всё в формулу для КПД, упростите и доведите до ответа.
    % \end{enumerate}
    Определите КПД цикла Карно, температура нагревателя которого равна максимальной температуре в цикле 12341, а холодильника — минимальной.
    Ответы в обоих случаях оставьте точными в виде нескоратимой дроби, никаких округлений.
}
\answer{%
    \begin{align*}
    A_{12} &> 0, \Delta U_{12} > 0, \implies Q_{12} = A_{12} + \Delta U_{12} > 0, \\
    A_{23} &= 0, \Delta U_{23} < 0, \implies Q_{23} = A_{23} + \Delta U_{23} < 0, \\
    A_{34} &< 0, \Delta U_{34} < 0, \implies Q_{34} = A_{34} + \Delta U_{34} < 0, \\
    A_{41} &= 0, \Delta U_{41} > 0, \implies Q_{41} = A_{41} + \Delta U_{41} > 0.
    \\
    P_1V_1 &= \nu R T_1, P_2V_2 = \nu R T_2, P_3V_3 = \nu R T_3, P_4V_4 = \nu R T_4 \text{ — уравнения состояния идеального газа}, \\
    &\text{Пусть $P_0$, $V_0$, $T_0$ — давление, объём и температура в точке 4 (минимальные во всём цикле):} \\
    P_1 &= P_2, P_3 = P_4 = P_0, V_1 = V_4 = V_0, V_2 = V_3 = 2 V_1 = 2 V_0,, \text{остальные соотношения между объёмами и давлениями не даны, нужно считать} \\
    T_3 &= \frac{T_2}6 \text{(по условию)} \implies \frac{P_2}{P_3} = \frac{P_2 V_2}{P_3 V_3}= \frac{\nu R T_2}{\nu R T_3} = \frac{T_2}{T_3} = 6 \implies P_1 = P_2 = 6 P_0 \\
    A_\text{цикл} &= (6P_0 - P_0)(2V_0 - V_0) = 5P_0V_0, \\
    A_{12} &= 6P_0 \cdot (2V_0 - V_0) = 6P_0V_0, \\
    \Delta U_{12} &= \frac 32 \nu R T_2 - \frac 32 \nu R T_1 = \frac 32 P_2 V_2 - \frac 32 P_1 V_1 = \frac 32 \cdot 6 P_0 \cdot 2 V_0 -  \frac 32 \cdot 6 P_0 \cdot V_0 = \frac 32 \cdot 6 \cdot P_0V_0, \\
    \Delta U_{41} &= \frac 32 \nu R T_1 - \frac 32 \nu R T_4 = \frac 32 P_1 V_1 - \frac 32 P_4 V_4 = \frac 32 \cdot 6 P_0 V_0 - \frac 32 P_0 V_0 = \frac 32 \cdot 5 \cdot P_0V_0.
    \\
    \eta &= \frac{A_\text{цикл}}{Q_+} = \frac{A_\text{цикл}}{Q_{12} + Q_{41}}  = \frac{A_\text{цикл}}{A_{12} + \Delta U_{12} + A_{41} + \Delta U_{41}} =  \\
     &= \frac{5P_0V_0}{6P_0V_0 + \frac 32 \cdot 6 \cdot P_0V_0 + 0 + \frac 32 \cdot 5 \cdot P_0V_0} = \frac{5}{6 + \frac 32 \cdot 6 + \frac 32 \cdot 5} = \frac{2}{9} \approx 0{,}222.
     \\
    \eta_\text{Карно} &= 1 - \frac{T_\text{х}}{T_\text{н}} = 1 - \frac{T_\text{4}}{T_\text{2}} = 1 - \frac{\frac{P_4V_4}{\nu R}}{\frac{P_2V_2}{\nu R}} = 1 - \frac{P_4V_4}{P_2V_2} = 1 - \frac{P_0V_0}{6P_0 \cdot 2V_0} = 1 - \frac 1{6 \cdot 2}  = \frac{11}{12} \approx 0{,}917.
    \end{align*}
}
\solutionspace{360pt}

\tasknumber{2}%
\task{%
    Порция идеального одноатомного газа перешла из состояния 1 в состояние 2: $P_1 = 2\,\text{МПа}$, $V_1 = 7\,\text{л}$, $P_2 = 2{,}5\,\text{МПа}$, $V_2 = 6\,\text{л}$.
    Определите, какую работу при этом совершил газ, чему равно изменение внутренней энергии газа, сколько теплоты подвели к нему в этом процессе?
    При решении обратите внимание на знаки искомых величин.
    Известно, что в PV-координатах график процесса 12 представляет собой отрезок прямой.
}
\answer{%
    \begin{align*}
    P_1V_1 &= \nu R T_1, P_2V_2 = \nu R T_2, \\
    \Delta U &= U_2-U_1 = \frac 32 \nu R T_2- \frac 32 \nu R T_1 = \frac 32 P_2 V_2 - \frac 32 P_1 V_1= \frac 32 \cdot \cbr{2{,}5\,\text{МПа} \cdot6\,\text{л} - 2\,\text{МПа} \cdot7\,\text{л}} = 1{,}50\,\text{кДж}.
    \\
    A_\text{газа} &= \frac{P_2 + P_1} 2 \cdot (V_2 - V_1) = \frac{2{,}5\,\text{МПа} + 2\,\text{МПа}} 2 \cdot (6\,\text{л} - 7\,\text{л}) = -2{,}250\,\text{кДж}, \\
    Q &= A_\text{газа} + \Delta U = \frac 32 (P_2 V_2 - P_1 V_1) + \frac{P_2 + P_1} 2 \cdot (V_2 - V_1) = 1{,}50\,\text{кДж} -2{,}250\,\text{кДж} = -0{,}7500\,\text{кДж}.
    \end{align*}
}
\solutionspace{150pt}

\tasknumber{3}%
\task{%
    Запишите формулы и рядом с каждой физичической величиной укажите её название и единицы изменения в СИ:
    \begin{enumerate}
        \item первое начало термодинамики,
        \item внутренняя энергия идеального одноатомного газа.
    \end{enumerate}
}

\variantsplitter

\addpersonalvariant{Матвей Кузьмин}

\tasknumber{1}%
\task{%
    Определите КПД цикла 12341, рабочим телом которого является идеальный одноатомный газ, если
    12 — изобарическое расширение газа в два раза,
    23 — изохорическое охлаждение газа, при котором температура уменьшается в три раза,
    34 — изобара, 41 — изохора.
    % Для этого:
    % \begin{enumerate}
    %     \item сделайте рисунок в PV-координатах,
    %     \item выберите удобные обозначения, чтобы не запутаться в множестве температур, давлений и объёмов,
    %     \item вычислите необходимые соотнощения между температурами, давлениями и объёмами
    %     (некоторые сразу видны по рисунку, некоторые — надо считать),
    %     \item определите для каждого участка поглощается или отдаётся тепло (и сколько именно:
    %     потребуется первое начало термодинамики, отдельный расчёт работ на участках через площади фигур и изменений внутренней энергии),
    %     \item вычислите полную работу газа в цикле,
    %     \item подставьте всё в формулу для КПД, упростите и доведите до ответа.
    % \end{enumerate}
    Определите КПД цикла Карно, температура нагревателя которого равна максимальной температуре в цикле 12341, а холодильника — минимальной.
    Ответы в обоих случаях оставьте точными в виде нескоратимой дроби, никаких округлений.
}
\answer{%
    \begin{align*}
    A_{12} &> 0, \Delta U_{12} > 0, \implies Q_{12} = A_{12} + \Delta U_{12} > 0, \\
    A_{23} &= 0, \Delta U_{23} < 0, \implies Q_{23} = A_{23} + \Delta U_{23} < 0, \\
    A_{34} &< 0, \Delta U_{34} < 0, \implies Q_{34} = A_{34} + \Delta U_{34} < 0, \\
    A_{41} &= 0, \Delta U_{41} > 0, \implies Q_{41} = A_{41} + \Delta U_{41} > 0.
    \\
    P_1V_1 &= \nu R T_1, P_2V_2 = \nu R T_2, P_3V_3 = \nu R T_3, P_4V_4 = \nu R T_4 \text{ — уравнения состояния идеального газа}, \\
    &\text{Пусть $P_0$, $V_0$, $T_0$ — давление, объём и температура в точке 4 (минимальные во всём цикле):} \\
    P_1 &= P_2, P_3 = P_4 = P_0, V_1 = V_4 = V_0, V_2 = V_3 = 2 V_1 = 2 V_0,, \text{остальные соотношения между объёмами и давлениями не даны, нужно считать} \\
    T_3 &= \frac{T_2}3 \text{(по условию)} \implies \frac{P_2}{P_3} = \frac{P_2 V_2}{P_3 V_3}= \frac{\nu R T_2}{\nu R T_3} = \frac{T_2}{T_3} = 3 \implies P_1 = P_2 = 3 P_0 \\
    A_\text{цикл} &= (3P_0 - P_0)(2V_0 - V_0) = 2P_0V_0, \\
    A_{12} &= 3P_0 \cdot (2V_0 - V_0) = 3P_0V_0, \\
    \Delta U_{12} &= \frac 32 \nu R T_2 - \frac 32 \nu R T_1 = \frac 32 P_2 V_2 - \frac 32 P_1 V_1 = \frac 32 \cdot 3 P_0 \cdot 2 V_0 -  \frac 32 \cdot 3 P_0 \cdot V_0 = \frac 32 \cdot 3 \cdot P_0V_0, \\
    \Delta U_{41} &= \frac 32 \nu R T_1 - \frac 32 \nu R T_4 = \frac 32 P_1 V_1 - \frac 32 P_4 V_4 = \frac 32 \cdot 3 P_0 V_0 - \frac 32 P_0 V_0 = \frac 32 \cdot 2 \cdot P_0V_0.
    \\
    \eta &= \frac{A_\text{цикл}}{Q_+} = \frac{A_\text{цикл}}{Q_{12} + Q_{41}}  = \frac{A_\text{цикл}}{A_{12} + \Delta U_{12} + A_{41} + \Delta U_{41}} =  \\
     &= \frac{2P_0V_0}{3P_0V_0 + \frac 32 \cdot 3 \cdot P_0V_0 + 0 + \frac 32 \cdot 2 \cdot P_0V_0} = \frac{2}{3 + \frac 32 \cdot 3 + \frac 32 \cdot 2} = \frac{4}{21} \approx 0{,}190.
     \\
    \eta_\text{Карно} &= 1 - \frac{T_\text{х}}{T_\text{н}} = 1 - \frac{T_\text{4}}{T_\text{2}} = 1 - \frac{\frac{P_4V_4}{\nu R}}{\frac{P_2V_2}{\nu R}} = 1 - \frac{P_4V_4}{P_2V_2} = 1 - \frac{P_0V_0}{3P_0 \cdot 2V_0} = 1 - \frac 1{3 \cdot 2}  = \frac{5}{6} \approx 0{,}833.
    \end{align*}
}
\solutionspace{360pt}

\tasknumber{2}%
\task{%
    Порция идеального одноатомного газа перешла из состояния 1 в состояние 2: $P_1 = 2\,\text{МПа}$, $V_1 = 3\,\text{л}$, $P_2 = 2{,}5\,\text{МПа}$, $V_2 = 8\,\text{л}$.
    Определите, какую работу при этом совершил газ, чему равно изменение внутренней энергии газа, сколько теплоты подвели к нему в этом процессе?
    При решении обратите внимание на знаки искомых величин.
    Известно, что в PV-координатах график процесса 12 представляет собой отрезок прямой.
}
\answer{%
    \begin{align*}
    P_1V_1 &= \nu R T_1, P_2V_2 = \nu R T_2, \\
    \Delta U &= U_2-U_1 = \frac 32 \nu R T_2- \frac 32 \nu R T_1 = \frac 32 P_2 V_2 - \frac 32 P_1 V_1= \frac 32 \cdot \cbr{2{,}5\,\text{МПа} \cdot8\,\text{л} - 2\,\text{МПа} \cdot3\,\text{л}} = 21{,}00\,\text{кДж}.
    \\
    A_\text{газа} &= \frac{P_2 + P_1} 2 \cdot (V_2 - V_1) = \frac{2{,}5\,\text{МПа} + 2\,\text{МПа}} 2 \cdot (8\,\text{л} - 3\,\text{л}) = 11{,}25\,\text{кДж}, \\
    Q &= A_\text{газа} + \Delta U = \frac 32 (P_2 V_2 - P_1 V_1) + \frac{P_2 + P_1} 2 \cdot (V_2 - V_1) = 21{,}00\,\text{кДж} + 11{,}25\,\text{кДж} = 32{,}25\,\text{кДж}.
    \end{align*}
}
\solutionspace{150pt}

\tasknumber{3}%
\task{%
    Запишите формулы и рядом с каждой физичической величиной укажите её название и единицы изменения в СИ:
    \begin{enumerate}
        \item первое начало термодинамики,
        \item внутренняя энергия идеального одноатомного газа.
    \end{enumerate}
}

\variantsplitter

\addpersonalvariant{Сергей Малышев}

\tasknumber{1}%
\task{%
    Определите КПД цикла 12341, рабочим телом которого является идеальный одноатомный газ, если
    12 — изобарическое расширение газа в три раза,
    23 — изохорическое охлаждение газа, при котором температура уменьшается в два раза,
    34 — изобара, 41 — изохора.
    % Для этого:
    % \begin{enumerate}
    %     \item сделайте рисунок в PV-координатах,
    %     \item выберите удобные обозначения, чтобы не запутаться в множестве температур, давлений и объёмов,
    %     \item вычислите необходимые соотнощения между температурами, давлениями и объёмами
    %     (некоторые сразу видны по рисунку, некоторые — надо считать),
    %     \item определите для каждого участка поглощается или отдаётся тепло (и сколько именно:
    %     потребуется первое начало термодинамики, отдельный расчёт работ на участках через площади фигур и изменений внутренней энергии),
    %     \item вычислите полную работу газа в цикле,
    %     \item подставьте всё в формулу для КПД, упростите и доведите до ответа.
    % \end{enumerate}
    Определите КПД цикла Карно, температура нагревателя которого равна максимальной температуре в цикле 12341, а холодильника — минимальной.
    Ответы в обоих случаях оставьте точными в виде нескоратимой дроби, никаких округлений.
}
\answer{%
    \begin{align*}
    A_{12} &> 0, \Delta U_{12} > 0, \implies Q_{12} = A_{12} + \Delta U_{12} > 0, \\
    A_{23} &= 0, \Delta U_{23} < 0, \implies Q_{23} = A_{23} + \Delta U_{23} < 0, \\
    A_{34} &< 0, \Delta U_{34} < 0, \implies Q_{34} = A_{34} + \Delta U_{34} < 0, \\
    A_{41} &= 0, \Delta U_{41} > 0, \implies Q_{41} = A_{41} + \Delta U_{41} > 0.
    \\
    P_1V_1 &= \nu R T_1, P_2V_2 = \nu R T_2, P_3V_3 = \nu R T_3, P_4V_4 = \nu R T_4 \text{ — уравнения состояния идеального газа}, \\
    &\text{Пусть $P_0$, $V_0$, $T_0$ — давление, объём и температура в точке 4 (минимальные во всём цикле):} \\
    P_1 &= P_2, P_3 = P_4 = P_0, V_1 = V_4 = V_0, V_2 = V_3 = 3 V_1 = 3 V_0,, \text{остальные соотношения между объёмами и давлениями не даны, нужно считать} \\
    T_3 &= \frac{T_2}2 \text{(по условию)} \implies \frac{P_2}{P_3} = \frac{P_2 V_2}{P_3 V_3}= \frac{\nu R T_2}{\nu R T_3} = \frac{T_2}{T_3} = 2 \implies P_1 = P_2 = 2 P_0 \\
    A_\text{цикл} &= (2P_0 - P_0)(3V_0 - V_0) = 2P_0V_0, \\
    A_{12} &= 2P_0 \cdot (3V_0 - V_0) = 4P_0V_0, \\
    \Delta U_{12} &= \frac 32 \nu R T_2 - \frac 32 \nu R T_1 = \frac 32 P_2 V_2 - \frac 32 P_1 V_1 = \frac 32 \cdot 2 P_0 \cdot 3 V_0 -  \frac 32 \cdot 2 P_0 \cdot V_0 = \frac 32 \cdot 4 \cdot P_0V_0, \\
    \Delta U_{41} &= \frac 32 \nu R T_1 - \frac 32 \nu R T_4 = \frac 32 P_1 V_1 - \frac 32 P_4 V_4 = \frac 32 \cdot 2 P_0 V_0 - \frac 32 P_0 V_0 = \frac 32 \cdot 1 \cdot P_0V_0.
    \\
    \eta &= \frac{A_\text{цикл}}{Q_+} = \frac{A_\text{цикл}}{Q_{12} + Q_{41}}  = \frac{A_\text{цикл}}{A_{12} + \Delta U_{12} + A_{41} + \Delta U_{41}} =  \\
     &= \frac{2P_0V_0}{4P_0V_0 + \frac 32 \cdot 4 \cdot P_0V_0 + 0 + \frac 32 \cdot 1 \cdot P_0V_0} = \frac{2}{4 + \frac 32 \cdot 4 + \frac 32 \cdot 1} = \frac{4}{23} \approx 0{,}174.
     \\
    \eta_\text{Карно} &= 1 - \frac{T_\text{х}}{T_\text{н}} = 1 - \frac{T_\text{4}}{T_\text{2}} = 1 - \frac{\frac{P_4V_4}{\nu R}}{\frac{P_2V_2}{\nu R}} = 1 - \frac{P_4V_4}{P_2V_2} = 1 - \frac{P_0V_0}{2P_0 \cdot 3V_0} = 1 - \frac 1{2 \cdot 3}  = \frac{5}{6} \approx 0{,}833.
    \end{align*}
}
\solutionspace{360pt}

\tasknumber{2}%
\task{%
    Порция идеального одноатомного газа перешла из состояния 1 в состояние 2: $P_1 = 2\,\text{МПа}$, $V_1 = 5\,\text{л}$, $P_2 = 4{,}5\,\text{МПа}$, $V_2 = 4\,\text{л}$.
    Определите, какую работу при этом совершил газ, чему равно изменение внутренней энергии газа, сколько теплоты подвели к нему в этом процессе?
    При решении обратите внимание на знаки искомых величин.
    Известно, что в PV-координатах график процесса 12 представляет собой отрезок прямой.
}
\answer{%
    \begin{align*}
    P_1V_1 &= \nu R T_1, P_2V_2 = \nu R T_2, \\
    \Delta U &= U_2-U_1 = \frac 32 \nu R T_2- \frac 32 \nu R T_1 = \frac 32 P_2 V_2 - \frac 32 P_1 V_1= \frac 32 \cdot \cbr{4{,}5\,\text{МПа} \cdot4\,\text{л} - 2\,\text{МПа} \cdot5\,\text{л}} = 12{,}00\,\text{кДж}.
    \\
    A_\text{газа} &= \frac{P_2 + P_1} 2 \cdot (V_2 - V_1) = \frac{4{,}5\,\text{МПа} + 2\,\text{МПа}} 2 \cdot (4\,\text{л} - 5\,\text{л}) = -3{,}250\,\text{кДж}, \\
    Q &= A_\text{газа} + \Delta U = \frac 32 (P_2 V_2 - P_1 V_1) + \frac{P_2 + P_1} 2 \cdot (V_2 - V_1) = 12{,}00\,\text{кДж} -3{,}250\,\text{кДж} = 8{,}75\,\text{кДж}.
    \end{align*}
}
\solutionspace{150pt}

\tasknumber{3}%
\task{%
    Запишите формулы и рядом с каждой физичической величиной укажите её название и единицы изменения в СИ:
    \begin{enumerate}
        \item первое начало термодинамики,
        \item внутренняя энергия идеального одноатомного газа.
    \end{enumerate}
}

\variantsplitter

\addpersonalvariant{Алина Полканова}

\tasknumber{1}%
\task{%
    Определите КПД цикла 12341, рабочим телом которого является идеальный одноатомный газ, если
    12 — изобарическое расширение газа в шесть раз,
    23 — изохорическое охлаждение газа, при котором температура уменьшается в четыре раза,
    34 — изобара, 41 — изохора.
    % Для этого:
    % \begin{enumerate}
    %     \item сделайте рисунок в PV-координатах,
    %     \item выберите удобные обозначения, чтобы не запутаться в множестве температур, давлений и объёмов,
    %     \item вычислите необходимые соотнощения между температурами, давлениями и объёмами
    %     (некоторые сразу видны по рисунку, некоторые — надо считать),
    %     \item определите для каждого участка поглощается или отдаётся тепло (и сколько именно:
    %     потребуется первое начало термодинамики, отдельный расчёт работ на участках через площади фигур и изменений внутренней энергии),
    %     \item вычислите полную работу газа в цикле,
    %     \item подставьте всё в формулу для КПД, упростите и доведите до ответа.
    % \end{enumerate}
    Определите КПД цикла Карно, температура нагревателя которого равна максимальной температуре в цикле 12341, а холодильника — минимальной.
    Ответы в обоих случаях оставьте точными в виде нескоратимой дроби, никаких округлений.
}
\answer{%
    \begin{align*}
    A_{12} &> 0, \Delta U_{12} > 0, \implies Q_{12} = A_{12} + \Delta U_{12} > 0, \\
    A_{23} &= 0, \Delta U_{23} < 0, \implies Q_{23} = A_{23} + \Delta U_{23} < 0, \\
    A_{34} &< 0, \Delta U_{34} < 0, \implies Q_{34} = A_{34} + \Delta U_{34} < 0, \\
    A_{41} &= 0, \Delta U_{41} > 0, \implies Q_{41} = A_{41} + \Delta U_{41} > 0.
    \\
    P_1V_1 &= \nu R T_1, P_2V_2 = \nu R T_2, P_3V_3 = \nu R T_3, P_4V_4 = \nu R T_4 \text{ — уравнения состояния идеального газа}, \\
    &\text{Пусть $P_0$, $V_0$, $T_0$ — давление, объём и температура в точке 4 (минимальные во всём цикле):} \\
    P_1 &= P_2, P_3 = P_4 = P_0, V_1 = V_4 = V_0, V_2 = V_3 = 6 V_1 = 6 V_0,, \text{остальные соотношения между объёмами и давлениями не даны, нужно считать} \\
    T_3 &= \frac{T_2}4 \text{(по условию)} \implies \frac{P_2}{P_3} = \frac{P_2 V_2}{P_3 V_3}= \frac{\nu R T_2}{\nu R T_3} = \frac{T_2}{T_3} = 4 \implies P_1 = P_2 = 4 P_0 \\
    A_\text{цикл} &= (4P_0 - P_0)(6V_0 - V_0) = 15P_0V_0, \\
    A_{12} &= 4P_0 \cdot (6V_0 - V_0) = 20P_0V_0, \\
    \Delta U_{12} &= \frac 32 \nu R T_2 - \frac 32 \nu R T_1 = \frac 32 P_2 V_2 - \frac 32 P_1 V_1 = \frac 32 \cdot 4 P_0 \cdot 6 V_0 -  \frac 32 \cdot 4 P_0 \cdot V_0 = \frac 32 \cdot 20 \cdot P_0V_0, \\
    \Delta U_{41} &= \frac 32 \nu R T_1 - \frac 32 \nu R T_4 = \frac 32 P_1 V_1 - \frac 32 P_4 V_4 = \frac 32 \cdot 4 P_0 V_0 - \frac 32 P_0 V_0 = \frac 32 \cdot 3 \cdot P_0V_0.
    \\
    \eta &= \frac{A_\text{цикл}}{Q_+} = \frac{A_\text{цикл}}{Q_{12} + Q_{41}}  = \frac{A_\text{цикл}}{A_{12} + \Delta U_{12} + A_{41} + \Delta U_{41}} =  \\
     &= \frac{15P_0V_0}{20P_0V_0 + \frac 32 \cdot 20 \cdot P_0V_0 + 0 + \frac 32 \cdot 3 \cdot P_0V_0} = \frac{15}{20 + \frac 32 \cdot 20 + \frac 32 \cdot 3} = \frac{30}{109} \approx 0{,}275.
     \\
    \eta_\text{Карно} &= 1 - \frac{T_\text{х}}{T_\text{н}} = 1 - \frac{T_\text{4}}{T_\text{2}} = 1 - \frac{\frac{P_4V_4}{\nu R}}{\frac{P_2V_2}{\nu R}} = 1 - \frac{P_4V_4}{P_2V_2} = 1 - \frac{P_0V_0}{4P_0 \cdot 6V_0} = 1 - \frac 1{4 \cdot 6}  = \frac{23}{24} \approx 0{,}958.
    \end{align*}
}
\solutionspace{360pt}

\tasknumber{2}%
\task{%
    Порция идеального одноатомного газа перешла из состояния 1 в состояние 2: $P_1 = 3\,\text{МПа}$, $V_1 = 7\,\text{л}$, $P_2 = 2{,}5\,\text{МПа}$, $V_2 = 6\,\text{л}$.
    Определите, какую работу при этом совершил газ, чему равно изменение внутренней энергии газа, сколько теплоты подвели к нему в этом процессе?
    При решении обратите внимание на знаки искомых величин.
    Известно, что в PV-координатах график процесса 12 представляет собой отрезок прямой.
}
\answer{%
    \begin{align*}
    P_1V_1 &= \nu R T_1, P_2V_2 = \nu R T_2, \\
    \Delta U &= U_2-U_1 = \frac 32 \nu R T_2- \frac 32 \nu R T_1 = \frac 32 P_2 V_2 - \frac 32 P_1 V_1= \frac 32 \cdot \cbr{2{,}5\,\text{МПа} \cdot6\,\text{л} - 3\,\text{МПа} \cdot7\,\text{л}} = -9{,}000\,\text{кДж}.
    \\
    A_\text{газа} &= \frac{P_2 + P_1} 2 \cdot (V_2 - V_1) = \frac{2{,}5\,\text{МПа} + 3\,\text{МПа}} 2 \cdot (6\,\text{л} - 7\,\text{л}) = -2{,}750\,\text{кДж}, \\
    Q &= A_\text{газа} + \Delta U = \frac 32 (P_2 V_2 - P_1 V_1) + \frac{P_2 + P_1} 2 \cdot (V_2 - V_1) = -9{,}000\,\text{кДж} -2{,}750\,\text{кДж} = -11{,}7500\,\text{кДж}.
    \end{align*}
}
\solutionspace{150pt}

\tasknumber{3}%
\task{%
    Запишите формулы и рядом с каждой физичической величиной укажите её название и единицы изменения в СИ:
    \begin{enumerate}
        \item первое начало термодинамики,
        \item внутренняя энергия идеального одноатомного газа.
    \end{enumerate}
}

\variantsplitter

\addpersonalvariant{Сергей Пономарёв}

\tasknumber{1}%
\task{%
    Определите КПД цикла 12341, рабочим телом которого является идеальный одноатомный газ, если
    12 — изобарическое расширение газа в четыре раза,
    23 — изохорическое охлаждение газа, при котором температура уменьшается в три раза,
    34 — изобара, 41 — изохора.
    % Для этого:
    % \begin{enumerate}
    %     \item сделайте рисунок в PV-координатах,
    %     \item выберите удобные обозначения, чтобы не запутаться в множестве температур, давлений и объёмов,
    %     \item вычислите необходимые соотнощения между температурами, давлениями и объёмами
    %     (некоторые сразу видны по рисунку, некоторые — надо считать),
    %     \item определите для каждого участка поглощается или отдаётся тепло (и сколько именно:
    %     потребуется первое начало термодинамики, отдельный расчёт работ на участках через площади фигур и изменений внутренней энергии),
    %     \item вычислите полную работу газа в цикле,
    %     \item подставьте всё в формулу для КПД, упростите и доведите до ответа.
    % \end{enumerate}
    Определите КПД цикла Карно, температура нагревателя которого равна максимальной температуре в цикле 12341, а холодильника — минимальной.
    Ответы в обоих случаях оставьте точными в виде нескоратимой дроби, никаких округлений.
}
\answer{%
    \begin{align*}
    A_{12} &> 0, \Delta U_{12} > 0, \implies Q_{12} = A_{12} + \Delta U_{12} > 0, \\
    A_{23} &= 0, \Delta U_{23} < 0, \implies Q_{23} = A_{23} + \Delta U_{23} < 0, \\
    A_{34} &< 0, \Delta U_{34} < 0, \implies Q_{34} = A_{34} + \Delta U_{34} < 0, \\
    A_{41} &= 0, \Delta U_{41} > 0, \implies Q_{41} = A_{41} + \Delta U_{41} > 0.
    \\
    P_1V_1 &= \nu R T_1, P_2V_2 = \nu R T_2, P_3V_3 = \nu R T_3, P_4V_4 = \nu R T_4 \text{ — уравнения состояния идеального газа}, \\
    &\text{Пусть $P_0$, $V_0$, $T_0$ — давление, объём и температура в точке 4 (минимальные во всём цикле):} \\
    P_1 &= P_2, P_3 = P_4 = P_0, V_1 = V_4 = V_0, V_2 = V_3 = 4 V_1 = 4 V_0,, \text{остальные соотношения между объёмами и давлениями не даны, нужно считать} \\
    T_3 &= \frac{T_2}3 \text{(по условию)} \implies \frac{P_2}{P_3} = \frac{P_2 V_2}{P_3 V_3}= \frac{\nu R T_2}{\nu R T_3} = \frac{T_2}{T_3} = 3 \implies P_1 = P_2 = 3 P_0 \\
    A_\text{цикл} &= (3P_0 - P_0)(4V_0 - V_0) = 6P_0V_0, \\
    A_{12} &= 3P_0 \cdot (4V_0 - V_0) = 9P_0V_0, \\
    \Delta U_{12} &= \frac 32 \nu R T_2 - \frac 32 \nu R T_1 = \frac 32 P_2 V_2 - \frac 32 P_1 V_1 = \frac 32 \cdot 3 P_0 \cdot 4 V_0 -  \frac 32 \cdot 3 P_0 \cdot V_0 = \frac 32 \cdot 9 \cdot P_0V_0, \\
    \Delta U_{41} &= \frac 32 \nu R T_1 - \frac 32 \nu R T_4 = \frac 32 P_1 V_1 - \frac 32 P_4 V_4 = \frac 32 \cdot 3 P_0 V_0 - \frac 32 P_0 V_0 = \frac 32 \cdot 2 \cdot P_0V_0.
    \\
    \eta &= \frac{A_\text{цикл}}{Q_+} = \frac{A_\text{цикл}}{Q_{12} + Q_{41}}  = \frac{A_\text{цикл}}{A_{12} + \Delta U_{12} + A_{41} + \Delta U_{41}} =  \\
     &= \frac{6P_0V_0}{9P_0V_0 + \frac 32 \cdot 9 \cdot P_0V_0 + 0 + \frac 32 \cdot 2 \cdot P_0V_0} = \frac{6}{9 + \frac 32 \cdot 9 + \frac 32 \cdot 2} = \frac{4}{17} \approx 0{,}235.
     \\
    \eta_\text{Карно} &= 1 - \frac{T_\text{х}}{T_\text{н}} = 1 - \frac{T_\text{4}}{T_\text{2}} = 1 - \frac{\frac{P_4V_4}{\nu R}}{\frac{P_2V_2}{\nu R}} = 1 - \frac{P_4V_4}{P_2V_2} = 1 - \frac{P_0V_0}{3P_0 \cdot 4V_0} = 1 - \frac 1{3 \cdot 4}  = \frac{11}{12} \approx 0{,}917.
    \end{align*}
}
\solutionspace{360pt}

\tasknumber{2}%
\task{%
    Порция идеального одноатомного газа перешла из состояния 1 в состояние 2: $P_1 = 3\,\text{МПа}$, $V_1 = 7\,\text{л}$, $P_2 = 2{,}5\,\text{МПа}$, $V_2 = 8\,\text{л}$.
    Определите, какую работу при этом совершил газ, чему равно изменение внутренней энергии газа, сколько теплоты подвели к нему в этом процессе?
    При решении обратите внимание на знаки искомых величин.
    Известно, что в PV-координатах график процесса 12 представляет собой отрезок прямой.
}
\answer{%
    \begin{align*}
    P_1V_1 &= \nu R T_1, P_2V_2 = \nu R T_2, \\
    \Delta U &= U_2-U_1 = \frac 32 \nu R T_2- \frac 32 \nu R T_1 = \frac 32 P_2 V_2 - \frac 32 P_1 V_1= \frac 32 \cdot \cbr{2{,}5\,\text{МПа} \cdot8\,\text{л} - 3\,\text{МПа} \cdot7\,\text{л}} = -1{,}5000\,\text{кДж}.
    \\
    A_\text{газа} &= \frac{P_2 + P_1} 2 \cdot (V_2 - V_1) = \frac{2{,}5\,\text{МПа} + 3\,\text{МПа}} 2 \cdot (8\,\text{л} - 7\,\text{л}) = 2{,}75\,\text{кДж}, \\
    Q &= A_\text{газа} + \Delta U = \frac 32 (P_2 V_2 - P_1 V_1) + \frac{P_2 + P_1} 2 \cdot (V_2 - V_1) = -1{,}5000\,\text{кДж} + 2{,}75\,\text{кДж} = 1{,}25\,\text{кДж}.
    \end{align*}
}
\solutionspace{150pt}

\tasknumber{3}%
\task{%
    Запишите формулы и рядом с каждой физичической величиной укажите её название и единицы изменения в СИ:
    \begin{enumerate}
        \item первое начало термодинамики,
        \item внутренняя энергия идеального одноатомного газа.
    \end{enumerate}
}

\variantsplitter

\addpersonalvariant{Егор Свистушкин}

\tasknumber{1}%
\task{%
    Определите КПД цикла 12341, рабочим телом которого является идеальный одноатомный газ, если
    12 — изобарическое расширение газа в два раза,
    23 — изохорическое охлаждение газа, при котором температура уменьшается в пять раз,
    34 — изобара, 41 — изохора.
    % Для этого:
    % \begin{enumerate}
    %     \item сделайте рисунок в PV-координатах,
    %     \item выберите удобные обозначения, чтобы не запутаться в множестве температур, давлений и объёмов,
    %     \item вычислите необходимые соотнощения между температурами, давлениями и объёмами
    %     (некоторые сразу видны по рисунку, некоторые — надо считать),
    %     \item определите для каждого участка поглощается или отдаётся тепло (и сколько именно:
    %     потребуется первое начало термодинамики, отдельный расчёт работ на участках через площади фигур и изменений внутренней энергии),
    %     \item вычислите полную работу газа в цикле,
    %     \item подставьте всё в формулу для КПД, упростите и доведите до ответа.
    % \end{enumerate}
    Определите КПД цикла Карно, температура нагревателя которого равна максимальной температуре в цикле 12341, а холодильника — минимальной.
    Ответы в обоих случаях оставьте точными в виде нескоратимой дроби, никаких округлений.
}
\answer{%
    \begin{align*}
    A_{12} &> 0, \Delta U_{12} > 0, \implies Q_{12} = A_{12} + \Delta U_{12} > 0, \\
    A_{23} &= 0, \Delta U_{23} < 0, \implies Q_{23} = A_{23} + \Delta U_{23} < 0, \\
    A_{34} &< 0, \Delta U_{34} < 0, \implies Q_{34} = A_{34} + \Delta U_{34} < 0, \\
    A_{41} &= 0, \Delta U_{41} > 0, \implies Q_{41} = A_{41} + \Delta U_{41} > 0.
    \\
    P_1V_1 &= \nu R T_1, P_2V_2 = \nu R T_2, P_3V_3 = \nu R T_3, P_4V_4 = \nu R T_4 \text{ — уравнения состояния идеального газа}, \\
    &\text{Пусть $P_0$, $V_0$, $T_0$ — давление, объём и температура в точке 4 (минимальные во всём цикле):} \\
    P_1 &= P_2, P_3 = P_4 = P_0, V_1 = V_4 = V_0, V_2 = V_3 = 2 V_1 = 2 V_0,, \text{остальные соотношения между объёмами и давлениями не даны, нужно считать} \\
    T_3 &= \frac{T_2}5 \text{(по условию)} \implies \frac{P_2}{P_3} = \frac{P_2 V_2}{P_3 V_3}= \frac{\nu R T_2}{\nu R T_3} = \frac{T_2}{T_3} = 5 \implies P_1 = P_2 = 5 P_0 \\
    A_\text{цикл} &= (5P_0 - P_0)(2V_0 - V_0) = 4P_0V_0, \\
    A_{12} &= 5P_0 \cdot (2V_0 - V_0) = 5P_0V_0, \\
    \Delta U_{12} &= \frac 32 \nu R T_2 - \frac 32 \nu R T_1 = \frac 32 P_2 V_2 - \frac 32 P_1 V_1 = \frac 32 \cdot 5 P_0 \cdot 2 V_0 -  \frac 32 \cdot 5 P_0 \cdot V_0 = \frac 32 \cdot 5 \cdot P_0V_0, \\
    \Delta U_{41} &= \frac 32 \nu R T_1 - \frac 32 \nu R T_4 = \frac 32 P_1 V_1 - \frac 32 P_4 V_4 = \frac 32 \cdot 5 P_0 V_0 - \frac 32 P_0 V_0 = \frac 32 \cdot 4 \cdot P_0V_0.
    \\
    \eta &= \frac{A_\text{цикл}}{Q_+} = \frac{A_\text{цикл}}{Q_{12} + Q_{41}}  = \frac{A_\text{цикл}}{A_{12} + \Delta U_{12} + A_{41} + \Delta U_{41}} =  \\
     &= \frac{4P_0V_0}{5P_0V_0 + \frac 32 \cdot 5 \cdot P_0V_0 + 0 + \frac 32 \cdot 4 \cdot P_0V_0} = \frac{4}{5 + \frac 32 \cdot 5 + \frac 32 \cdot 4} = \frac{8}{37} \approx 0{,}216.
     \\
    \eta_\text{Карно} &= 1 - \frac{T_\text{х}}{T_\text{н}} = 1 - \frac{T_\text{4}}{T_\text{2}} = 1 - \frac{\frac{P_4V_4}{\nu R}}{\frac{P_2V_2}{\nu R}} = 1 - \frac{P_4V_4}{P_2V_2} = 1 - \frac{P_0V_0}{5P_0 \cdot 2V_0} = 1 - \frac 1{5 \cdot 2}  = \frac{9}{10} \approx 0{,}900.
    \end{align*}
}
\solutionspace{360pt}

\tasknumber{2}%
\task{%
    Порция идеального одноатомного газа перешла из состояния 1 в состояние 2: $P_1 = 3\,\text{МПа}$, $V_1 = 5\,\text{л}$, $P_2 = 2{,}5\,\text{МПа}$, $V_2 = 8\,\text{л}$.
    Определите, какую работу при этом совершил газ, чему равно изменение внутренней энергии газа, сколько теплоты подвели к нему в этом процессе?
    При решении обратите внимание на знаки искомых величин.
    Известно, что в PV-координатах график процесса 12 представляет собой отрезок прямой.
}
\answer{%
    \begin{align*}
    P_1V_1 &= \nu R T_1, P_2V_2 = \nu R T_2, \\
    \Delta U &= U_2-U_1 = \frac 32 \nu R T_2- \frac 32 \nu R T_1 = \frac 32 P_2 V_2 - \frac 32 P_1 V_1= \frac 32 \cdot \cbr{2{,}5\,\text{МПа} \cdot8\,\text{л} - 3\,\text{МПа} \cdot5\,\text{л}} = 7{,}50\,\text{кДж}.
    \\
    A_\text{газа} &= \frac{P_2 + P_1} 2 \cdot (V_2 - V_1) = \frac{2{,}5\,\text{МПа} + 3\,\text{МПа}} 2 \cdot (8\,\text{л} - 5\,\text{л}) = 8{,}25\,\text{кДж}, \\
    Q &= A_\text{газа} + \Delta U = \frac 32 (P_2 V_2 - P_1 V_1) + \frac{P_2 + P_1} 2 \cdot (V_2 - V_1) = 7{,}50\,\text{кДж} + 8{,}25\,\text{кДж} = 15{,}75\,\text{кДж}.
    \end{align*}
}
\solutionspace{150pt}

\tasknumber{3}%
\task{%
    Запишите формулы и рядом с каждой физичической величиной укажите её название и единицы изменения в СИ:
    \begin{enumerate}
        \item первое начало термодинамики,
        \item внутренняя энергия идеального одноатомного газа.
    \end{enumerate}
}

\variantsplitter

\addpersonalvariant{Дмитрий Соколов}

\tasknumber{1}%
\task{%
    Определите КПД цикла 12341, рабочим телом которого является идеальный одноатомный газ, если
    12 — изобарическое расширение газа в шесть раз,
    23 — изохорическое охлаждение газа, при котором температура уменьшается в два раза,
    34 — изобара, 41 — изохора.
    % Для этого:
    % \begin{enumerate}
    %     \item сделайте рисунок в PV-координатах,
    %     \item выберите удобные обозначения, чтобы не запутаться в множестве температур, давлений и объёмов,
    %     \item вычислите необходимые соотнощения между температурами, давлениями и объёмами
    %     (некоторые сразу видны по рисунку, некоторые — надо считать),
    %     \item определите для каждого участка поглощается или отдаётся тепло (и сколько именно:
    %     потребуется первое начало термодинамики, отдельный расчёт работ на участках через площади фигур и изменений внутренней энергии),
    %     \item вычислите полную работу газа в цикле,
    %     \item подставьте всё в формулу для КПД, упростите и доведите до ответа.
    % \end{enumerate}
    Определите КПД цикла Карно, температура нагревателя которого равна максимальной температуре в цикле 12341, а холодильника — минимальной.
    Ответы в обоих случаях оставьте точными в виде нескоратимой дроби, никаких округлений.
}
\answer{%
    \begin{align*}
    A_{12} &> 0, \Delta U_{12} > 0, \implies Q_{12} = A_{12} + \Delta U_{12} > 0, \\
    A_{23} &= 0, \Delta U_{23} < 0, \implies Q_{23} = A_{23} + \Delta U_{23} < 0, \\
    A_{34} &< 0, \Delta U_{34} < 0, \implies Q_{34} = A_{34} + \Delta U_{34} < 0, \\
    A_{41} &= 0, \Delta U_{41} > 0, \implies Q_{41} = A_{41} + \Delta U_{41} > 0.
    \\
    P_1V_1 &= \nu R T_1, P_2V_2 = \nu R T_2, P_3V_3 = \nu R T_3, P_4V_4 = \nu R T_4 \text{ — уравнения состояния идеального газа}, \\
    &\text{Пусть $P_0$, $V_0$, $T_0$ — давление, объём и температура в точке 4 (минимальные во всём цикле):} \\
    P_1 &= P_2, P_3 = P_4 = P_0, V_1 = V_4 = V_0, V_2 = V_3 = 6 V_1 = 6 V_0,, \text{остальные соотношения между объёмами и давлениями не даны, нужно считать} \\
    T_3 &= \frac{T_2}2 \text{(по условию)} \implies \frac{P_2}{P_3} = \frac{P_2 V_2}{P_3 V_3}= \frac{\nu R T_2}{\nu R T_3} = \frac{T_2}{T_3} = 2 \implies P_1 = P_2 = 2 P_0 \\
    A_\text{цикл} &= (2P_0 - P_0)(6V_0 - V_0) = 5P_0V_0, \\
    A_{12} &= 2P_0 \cdot (6V_0 - V_0) = 10P_0V_0, \\
    \Delta U_{12} &= \frac 32 \nu R T_2 - \frac 32 \nu R T_1 = \frac 32 P_2 V_2 - \frac 32 P_1 V_1 = \frac 32 \cdot 2 P_0 \cdot 6 V_0 -  \frac 32 \cdot 2 P_0 \cdot V_0 = \frac 32 \cdot 10 \cdot P_0V_0, \\
    \Delta U_{41} &= \frac 32 \nu R T_1 - \frac 32 \nu R T_4 = \frac 32 P_1 V_1 - \frac 32 P_4 V_4 = \frac 32 \cdot 2 P_0 V_0 - \frac 32 P_0 V_0 = \frac 32 \cdot 1 \cdot P_0V_0.
    \\
    \eta &= \frac{A_\text{цикл}}{Q_+} = \frac{A_\text{цикл}}{Q_{12} + Q_{41}}  = \frac{A_\text{цикл}}{A_{12} + \Delta U_{12} + A_{41} + \Delta U_{41}} =  \\
     &= \frac{5P_0V_0}{10P_0V_0 + \frac 32 \cdot 10 \cdot P_0V_0 + 0 + \frac 32 \cdot 1 \cdot P_0V_0} = \frac{5}{10 + \frac 32 \cdot 10 + \frac 32 \cdot 1} = \frac{10}{53} \approx 0{,}189.
     \\
    \eta_\text{Карно} &= 1 - \frac{T_\text{х}}{T_\text{н}} = 1 - \frac{T_\text{4}}{T_\text{2}} = 1 - \frac{\frac{P_4V_4}{\nu R}}{\frac{P_2V_2}{\nu R}} = 1 - \frac{P_4V_4}{P_2V_2} = 1 - \frac{P_0V_0}{2P_0 \cdot 6V_0} = 1 - \frac 1{2 \cdot 6}  = \frac{11}{12} \approx 0{,}917.
    \end{align*}
}
\solutionspace{360pt}

\tasknumber{2}%
\task{%
    Порция идеального одноатомного газа перешла из состояния 1 в состояние 2: $P_1 = 3\,\text{МПа}$, $V_1 = 3\,\text{л}$, $P_2 = 3{,}5\,\text{МПа}$, $V_2 = 4\,\text{л}$.
    Определите, какую работу при этом совершил газ, чему равно изменение внутренней энергии газа, сколько теплоты подвели к нему в этом процессе?
    При решении обратите внимание на знаки искомых величин.
    Известно, что в PV-координатах график процесса 12 представляет собой отрезок прямой.
}
\answer{%
    \begin{align*}
    P_1V_1 &= \nu R T_1, P_2V_2 = \nu R T_2, \\
    \Delta U &= U_2-U_1 = \frac 32 \nu R T_2- \frac 32 \nu R T_1 = \frac 32 P_2 V_2 - \frac 32 P_1 V_1= \frac 32 \cdot \cbr{3{,}5\,\text{МПа} \cdot4\,\text{л} - 3\,\text{МПа} \cdot3\,\text{л}} = 7{,}50\,\text{кДж}.
    \\
    A_\text{газа} &= \frac{P_2 + P_1} 2 \cdot (V_2 - V_1) = \frac{3{,}5\,\text{МПа} + 3\,\text{МПа}} 2 \cdot (4\,\text{л} - 3\,\text{л}) = 3{,}25\,\text{кДж}, \\
    Q &= A_\text{газа} + \Delta U = \frac 32 (P_2 V_2 - P_1 V_1) + \frac{P_2 + P_1} 2 \cdot (V_2 - V_1) = 7{,}50\,\text{кДж} + 3{,}25\,\text{кДж} = 10{,}75\,\text{кДж}.
    \end{align*}
}
\solutionspace{150pt}

\tasknumber{3}%
\task{%
    Запишите формулы и рядом с каждой физичической величиной укажите её название и единицы изменения в СИ:
    \begin{enumerate}
        \item первое начало термодинамики,
        \item внутренняя энергия идеального одноатомного газа.
    \end{enumerate}
}

\end{document}
% autogenerated
