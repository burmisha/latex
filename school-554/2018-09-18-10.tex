\adddate{10 класс. 18 сентября 2018}
Ускорение свободного падения положить $g = 10\,\frac{\text{м}}{\text{с}^2}$.

\task 1
Тело начинает двигаться вдоль прямой с постоянным ускорением $a = 2\,\frac{\text{м}}{\text{с}^2}$. Определите, какой путь оно пройдёт за время $\tau = 4\units{c}$, чему будут равны его  скорость в этот момент времени и ускорение?

\task 2
Материальная точка начинает двигаться вдоль прямой с постоянным ускорением $a = 4\,\frac{\text{м}}{\text{с}^2}$.
В какой момент времени она пройдёт путь равный $l = 98\units{м}$? Чему будут равны её  скорость в этот момент времени и ускорение?

\task 3
Тело бросают с поверхности земли вертикально вверх со скоростью $v = 30\,\frac{\text{м}}{\text{с}}$. Определите, через какое время тело достигнет наивысшей точки своего полёта? Чему будет равна скорость в этот момент времени, а ускорение, куда они направлены?

\task 4
С какой минимальной скоростью и в каком направлении необходимо бросить рюкзак, чтобы он поднялся до высоты $H = 18\units{м}$?

\task 5
Мяч бросают под углом $\alpha=30^\circ=\frac\pi{6}$ к горизонту со скоростью $v = 20\,\frac{\text{м}}{\text{с}}$. Через какое время он упадёт обратно на землю? Какова будет дальность полёта?

\task 5
Тело бросают под углом $\alpha$ к горизонту со скоростью $v = 20\,\frac{\text{м}}{\text{с}}$. Каким следует выбрать угол $\alpha$, чтобы дальность полёта была максимальна?

\task 6
Камень бросают с утёса высотой $H=40\units{м}$ под углом $\alpha$ к горизонту со скоростью $v = 20\,\frac{\text{м}}{\text{с}}$. На какое максимальное расстояние можно забросить камень?