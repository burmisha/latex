\setdate{24~декабря~2019}
\setclass{9«А»}

\addpersonalvariant{Михаил Бурмистров}

\tasknumber{1}%
\task{%
    Укажите, верны ли утверждения:
    \begin{itemize}
        \item механические волны переносят вещество,
        \item механические волны переносят энергию,
        \item источником механических волны служат колеблющиеся тела,
        \item продольные волны могут распространяться только в твёрдых телах,
        \item в твёрдых телах могут распространяться поперечные волны,
        \item скорость распространения волны равна произведению её длины на её частоту,
        \item звуковая волна~--- поперечная волна,
        \item волна на поверхности озера~--- поперечная волна (или же поверхностная).
    \end{itemize}
    (4 из 8~--- это «–»)
}
\answer{%
    нет, да, да, нет, да, да, нет, да
}
\solutionspace{20pt}

\tasknumber{2}%
\task{%
    Определите расстояние между третьим и восьмым гребнями волн,
    если длина волны равна $6\,\text{м}$.
    Сколько между ними ещё уместилось гребней?
}
\answer{%
    $
        l = (n_2 - n_1) \cdot \lambda = \cbr{8 - 3} \cdot 6\,\text{м} = 30\,\text{м},
        \quad n = n_2 - n_1 - 1 = 8 - 3 - 1 = 4
    $
}
\solutionspace{120pt}

\tasknumber{3}%
\task{%
    Определите скорость звука в среде, если источник звука,
    колеблющийся с периодом $3\,\text{мc}$, возбуждает волны длиной
    $1{,}2\,\text{м}$.
}
\answer{%
    $\lambda = vT \implies v = \frac{\lambda}T = \frac{ 1{,}2\,\text{м} }{ 3\,\text{мc} } = { 400{,}0\,\frac{\text{м}}{\text{c}} }$
}
\solutionspace{120pt}

\tasknumber{4}%
\task{%
    Мимо неподвижного наблюдателя прошло $5$ гребней волн за $5\,\text{c}$,
    начиная с первого.
    Каковы длина, период и частота волны,
    если скорость распространения волн $3\,\frac{\text{м}}{\text{с}}$?
}
\answer{%
    \begin{align*}
    \lambda &= \frac L{N-1} = \frac {vt}{N-1} = \frac {3\,\frac{\text{м}}{\text{с}} \cdot 5\,\text{c}}{5 - 1} = 3{,}75\,\text{м},  \\
    T &= \frac {\lambda}{v} = \frac {vt}{\cbr{N-1}v} = \frac {t}{N-1} =  \frac { 5\,\text{c} }{5 - 1} = 1{,}25\,\text{с},  \\
    \nu &= \frac 1T = \frac {N-1}{t} = \frac {5 - 1}{ 5\,\text{c} } = 0{,}80\,\text{Гц}.
    \\
    &\text{Если же считать гребни целиком, т.е.
    не вычитать единицу:}  \\
    \lambda' &= \frac L{N} = \frac {vt}{N} = \frac {3\,\frac{\text{м}}{\text{с}} \cdot 5\,\text{c}}{ 5 } = 3{,}00\,\text{м},  \\
    T' &= \frac {\lambda'}{v} = \frac {vt}{Nv} = \frac tN =  \frac { 5\,\text{c} }{ 5 } = 1{,}00\,\text{с},  \\
    \nu' &= \frac 1{T'} = \frac {N}{t} = \frac { 5 }{ 5\,\text{c} } = 1{,}00\,\text{Гц}.
    \end{align*}
}

\variantsplitter

\addpersonalvariant{Ирина Ан}

\tasknumber{1}%
\task{%
    Укажите, верны ли утверждения:
    \begin{itemize}
        \item механические волны переносят вещество,
        \item механические волны переносят энергию,
        \item источником механических волны служат колеблющиеся тела,
        \item продольные волны могут распространяться только в твёрдых телах,
        \item в твёрдых телах могут распространяться поперечные волны,
        \item скорость распространения волны равна произведению её длины на её частоту,
        \item звуковая волна~--- поперечная волна,
        \item волна на поверхности озера~--- поперечная волна (или же поверхностная).
    \end{itemize}
    (4 из 8~--- это «–»)
}
\answer{%
    нет, да, да, нет, да, да, нет, да
}
\solutionspace{20pt}

\tasknumber{2}%
\task{%
    Определите расстояние между вторым и девятым гребнями волн,
    если длина волны равна $6\,\text{м}$.
    Сколько между ними ещё уместилось гребней?
}
\answer{%
    $
        l = (n_2 - n_1) \cdot \lambda = \cbr{9 - 2} \cdot 6\,\text{м} = 42\,\text{м},
        \quad n = n_2 - n_1 - 1 = 9 - 2 - 1 = 6
    $
}
\solutionspace{120pt}

\tasknumber{3}%
\task{%
    Определите скорость звука в среде, если источник звука,
    колеблющийся с периодом $5\,\text{мc}$, возбуждает волны длиной
    $1{,}2\,\text{м}$.
}
\answer{%
    $\lambda = vT \implies v = \frac{\lambda}T = \frac{ 1{,}2\,\text{м} }{ 5\,\text{мc} } = { 240{,}0\,\frac{\text{м}}{\text{c}} }$
}
\solutionspace{120pt}

\tasknumber{4}%
\task{%
    Мимо неподвижного наблюдателя прошло $6$ гребней волн за $8\,\text{c}$,
    начиная с первого.
    Каковы длина, период и частота волны,
    если скорость распространения волн $2\,\frac{\text{м}}{\text{с}}$?
}
\answer{%
    \begin{align*}
    \lambda &= \frac L{N-1} = \frac {vt}{N-1} = \frac {2\,\frac{\text{м}}{\text{с}} \cdot 8\,\text{c}}{6 - 1} = 3{,}20\,\text{м},  \\
    T &= \frac {\lambda}{v} = \frac {vt}{\cbr{N-1}v} = \frac {t}{N-1} =  \frac { 8\,\text{c} }{6 - 1} = 1{,}60\,\text{с},  \\
    \nu &= \frac 1T = \frac {N-1}{t} = \frac {6 - 1}{ 8\,\text{c} } = 0{,}62\,\text{Гц}.
    \\
    &\text{Если же считать гребни целиком, т.е.
    не вычитать единицу:}  \\
    \lambda' &= \frac L{N} = \frac {vt}{N} = \frac {2\,\frac{\text{м}}{\text{с}} \cdot 8\,\text{c}}{ 6 } = 2{,}67\,\text{м},  \\
    T' &= \frac {\lambda'}{v} = \frac {vt}{Nv} = \frac tN =  \frac { 8\,\text{c} }{ 6 } = 1{,}33\,\text{с},  \\
    \nu' &= \frac 1{T'} = \frac {N}{t} = \frac { 6 }{ 8\,\text{c} } = 0{,}75\,\text{Гц}.
    \end{align*}
}

\variantsplitter

\addpersonalvariant{Софья Андрианова}

\tasknumber{1}%
\task{%
    Укажите, верны ли утверждения:
    \begin{itemize}
        \item механические волны переносят вещество,
        \item механические волны переносят энергию,
        \item источником механических волны служат колеблющиеся тела,
        \item продольные волны могут распространяться только в твёрдых телах,
        \item в твёрдых телах могут распространяться поперечные волны,
        \item скорость распространения волны равна произведению её длины на её частоту,
        \item звуковая волна~--- поперечная волна,
        \item волна на поверхности озера~--- поперечная волна (или же поверхностная).
    \end{itemize}
    (4 из 8~--- это «–»)
}
\answer{%
    нет, да, да, нет, да, да, нет, да
}
\solutionspace{20pt}

\tasknumber{2}%
\task{%
    Определите расстояние между вторым и десятым гребнями волн,
    если длина волны равна $4\,\text{м}$.
    Сколько между ними ещё уместилось гребней?
}
\answer{%
    $
        l = (n_2 - n_1) \cdot \lambda = \cbr{10 - 2} \cdot 4\,\text{м} = 32\,\text{м},
        \quad n = n_2 - n_1 - 1 = 10 - 2 - 1 = 7
    $
}
\solutionspace{120pt}

\tasknumber{3}%
\task{%
    Определите скорость звука в среде, если источник звука,
    колеблющийся с периодом $6\,\text{мc}$, возбуждает волны длиной
    $1{,}5\,\text{м}$.
}
\answer{%
    $\lambda = vT \implies v = \frac{\lambda}T = \frac{ 1{,}5\,\text{м} }{ 6\,\text{мc} } = { 250{,}0\,\frac{\text{м}}{\text{c}} }$
}
\solutionspace{120pt}

\tasknumber{4}%
\task{%
    Мимо неподвижного наблюдателя прошло $5$ гребней волн за $6\,\text{c}$,
    начиная с первого.
    Каковы длина, период и частота волны,
    если скорость распространения волн $3\,\frac{\text{м}}{\text{с}}$?
}
\answer{%
    \begin{align*}
    \lambda &= \frac L{N-1} = \frac {vt}{N-1} = \frac {3\,\frac{\text{м}}{\text{с}} \cdot 6\,\text{c}}{5 - 1} = 4{,}50\,\text{м},  \\
    T &= \frac {\lambda}{v} = \frac {vt}{\cbr{N-1}v} = \frac {t}{N-1} =  \frac { 6\,\text{c} }{5 - 1} = 1{,}50\,\text{с},  \\
    \nu &= \frac 1T = \frac {N-1}{t} = \frac {5 - 1}{ 6\,\text{c} } = 0{,}67\,\text{Гц}.
    \\
    &\text{Если же считать гребни целиком, т.е.
    не вычитать единицу:}  \\
    \lambda' &= \frac L{N} = \frac {vt}{N} = \frac {3\,\frac{\text{м}}{\text{с}} \cdot 6\,\text{c}}{ 5 } = 3{,}60\,\text{м},  \\
    T' &= \frac {\lambda'}{v} = \frac {vt}{Nv} = \frac tN =  \frac { 6\,\text{c} }{ 5 } = 1{,}20\,\text{с},  \\
    \nu' &= \frac 1{T'} = \frac {N}{t} = \frac { 5 }{ 6\,\text{c} } = 0{,}83\,\text{Гц}.
    \end{align*}
}

\variantsplitter

\addpersonalvariant{Владимир Артемчук}

\tasknumber{1}%
\task{%
    Укажите, верны ли утверждения:
    \begin{itemize}
        \item механические волны переносят вещество,
        \item механические волны переносят энергию,
        \item источником механических волны служат колеблющиеся тела,
        \item продольные волны могут распространяться только в твёрдых телах,
        \item в твёрдых телах могут распространяться поперечные волны,
        \item скорость распространения волны равна произведению её длины на её частоту,
        \item звуковая волна~--- поперечная волна,
        \item волна на поверхности озера~--- поперечная волна (или же поверхностная).
    \end{itemize}
    (4 из 8~--- это «–»)
}
\answer{%
    нет, да, да, нет, да, да, нет, да
}
\solutionspace{20pt}

\tasknumber{2}%
\task{%
    Определите расстояние между вторым и седьмым гребнями волн,
    если длина волны равна $3\,\text{м}$.
    Сколько между ними ещё уместилось гребней?
}
\answer{%
    $
        l = (n_2 - n_1) \cdot \lambda = \cbr{7 - 2} \cdot 3\,\text{м} = 15\,\text{м},
        \quad n = n_2 - n_1 - 1 = 7 - 2 - 1 = 4
    $
}
\solutionspace{120pt}

\tasknumber{3}%
\task{%
    Определите скорость звука в среде, если источник звука,
    колеблющийся с периодом $3\,\text{мc}$, возбуждает волны длиной
    $2{,}1\,\text{м}$.
}
\answer{%
    $\lambda = vT \implies v = \frac{\lambda}T = \frac{ 2{,}1\,\text{м} }{ 3\,\text{мc} } = { 700{,}0\,\frac{\text{м}}{\text{c}} }$
}
\solutionspace{120pt}

\tasknumber{4}%
\task{%
    Мимо неподвижного наблюдателя прошло $6$ гребней волн за $10\,\text{c}$,
    начиная с первого.
    Каковы длина, период и частота волны,
    если скорость распространения волн $2\,\frac{\text{м}}{\text{с}}$?
}
\answer{%
    \begin{align*}
    \lambda &= \frac L{N-1} = \frac {vt}{N-1} = \frac {2\,\frac{\text{м}}{\text{с}} \cdot 10\,\text{c}}{6 - 1} = 4{,}00\,\text{м},  \\
    T &= \frac {\lambda}{v} = \frac {vt}{\cbr{N-1}v} = \frac {t}{N-1} =  \frac { 10\,\text{c} }{6 - 1} = 2{,}00\,\text{с},  \\
    \nu &= \frac 1T = \frac {N-1}{t} = \frac {6 - 1}{ 10\,\text{c} } = 0{,}50\,\text{Гц}.
    \\
    &\text{Если же считать гребни целиком, т.е.
    не вычитать единицу:}  \\
    \lambda' &= \frac L{N} = \frac {vt}{N} = \frac {2\,\frac{\text{м}}{\text{с}} \cdot 10\,\text{c}}{ 6 } = 3{,}33\,\text{м},  \\
    T' &= \frac {\lambda'}{v} = \frac {vt}{Nv} = \frac tN =  \frac { 10\,\text{c} }{ 6 } = 1{,}67\,\text{с},  \\
    \nu' &= \frac 1{T'} = \frac {N}{t} = \frac { 6 }{ 10\,\text{c} } = 0{,}60\,\text{Гц}.
    \end{align*}
}

\variantsplitter

\addpersonalvariant{Софья Белянкина}

\tasknumber{1}%
\task{%
    Укажите, верны ли утверждения:
    \begin{itemize}
        \item механические волны переносят вещество,
        \item механические волны переносят энергию,
        \item источником механических волны служат колеблющиеся тела,
        \item продольные волны могут распространяться только в твёрдых телах,
        \item в твёрдых телах могут распространяться поперечные волны,
        \item скорость распространения волны равна произведению её длины на её частоту,
        \item звуковая волна~--- поперечная волна,
        \item волна на поверхности озера~--- поперечная волна (или же поверхностная).
    \end{itemize}
    (4 из 8~--- это «–»)
}
\answer{%
    нет, да, да, нет, да, да, нет, да
}
\solutionspace{20pt}

\tasknumber{2}%
\task{%
    Определите расстояние между вторым и девятым гребнями волн,
    если длина волны равна $6\,\text{м}$.
    Сколько между ними ещё уместилось гребней?
}
\answer{%
    $
        l = (n_2 - n_1) \cdot \lambda = \cbr{9 - 2} \cdot 6\,\text{м} = 42\,\text{м},
        \quad n = n_2 - n_1 - 1 = 9 - 2 - 1 = 6
    $
}
\solutionspace{120pt}

\tasknumber{3}%
\task{%
    Определите скорость звука в среде, если источник звука,
    колеблющийся с периодом $5\,\text{мc}$, возбуждает волны длиной
    $1{,}2\,\text{м}$.
}
\answer{%
    $\lambda = vT \implies v = \frac{\lambda}T = \frac{ 1{,}2\,\text{м} }{ 5\,\text{мc} } = { 240{,}0\,\frac{\text{м}}{\text{c}} }$
}
\solutionspace{120pt}

\tasknumber{4}%
\task{%
    Мимо неподвижного наблюдателя прошло $4$ гребней волн за $8\,\text{c}$,
    начиная с первого.
    Каковы длина, период и частота волны,
    если скорость распространения волн $3\,\frac{\text{м}}{\text{с}}$?
}
\answer{%
    \begin{align*}
    \lambda &= \frac L{N-1} = \frac {vt}{N-1} = \frac {3\,\frac{\text{м}}{\text{с}} \cdot 8\,\text{c}}{4 - 1} = 8{,}00\,\text{м},  \\
    T &= \frac {\lambda}{v} = \frac {vt}{\cbr{N-1}v} = \frac {t}{N-1} =  \frac { 8\,\text{c} }{4 - 1} = 2{,}67\,\text{с},  \\
    \nu &= \frac 1T = \frac {N-1}{t} = \frac {4 - 1}{ 8\,\text{c} } = 0{,}38\,\text{Гц}.
    \\
    &\text{Если же считать гребни целиком, т.е.
    не вычитать единицу:}  \\
    \lambda' &= \frac L{N} = \frac {vt}{N} = \frac {3\,\frac{\text{м}}{\text{с}} \cdot 8\,\text{c}}{ 4 } = 6{,}00\,\text{м},  \\
    T' &= \frac {\lambda'}{v} = \frac {vt}{Nv} = \frac tN =  \frac { 8\,\text{c} }{ 4 } = 2{,}00\,\text{с},  \\
    \nu' &= \frac 1{T'} = \frac {N}{t} = \frac { 4 }{ 8\,\text{c} } = 0{,}50\,\text{Гц}.
    \end{align*}
}

\variantsplitter

\addpersonalvariant{Варвара Егиазарян}

\tasknumber{1}%
\task{%
    Укажите, верны ли утверждения:
    \begin{itemize}
        \item механические волны переносят вещество,
        \item механические волны переносят энергию,
        \item источником механических волны служат колеблющиеся тела,
        \item продольные волны могут распространяться только в твёрдых телах,
        \item в твёрдых телах могут распространяться поперечные волны,
        \item скорость распространения волны равна произведению её длины на её частоту,
        \item звуковая волна~--- поперечная волна,
        \item волна на поверхности озера~--- поперечная волна (или же поверхностная).
    \end{itemize}
    (4 из 8~--- это «–»)
}
\answer{%
    нет, да, да, нет, да, да, нет, да
}
\solutionspace{20pt}

\tasknumber{2}%
\task{%
    Определите расстояние между вторым и десятым гребнями волн,
    если длина волны равна $5\,\text{м}$.
    Сколько между ними ещё уместилось гребней?
}
\answer{%
    $
        l = (n_2 - n_1) \cdot \lambda = \cbr{10 - 2} \cdot 5\,\text{м} = 40\,\text{м},
        \quad n = n_2 - n_1 - 1 = 10 - 2 - 1 = 7
    $
}
\solutionspace{120pt}

\tasknumber{3}%
\task{%
    Определите скорость звука в среде, если источник звука,
    колеблющийся с периодом $4\,\text{мc}$, возбуждает волны длиной
    $2{,}1\,\text{м}$.
}
\answer{%
    $\lambda = vT \implies v = \frac{\lambda}T = \frac{ 2{,}1\,\text{м} }{ 4\,\text{мc} } = { 525{,}0\,\frac{\text{м}}{\text{c}} }$
}
\solutionspace{120pt}

\tasknumber{4}%
\task{%
    Мимо неподвижного наблюдателя прошло $6$ гребней волн за $10\,\text{c}$,
    начиная с первого.
    Каковы длина, период и частота волны,
    если скорость распространения волн $5\,\frac{\text{м}}{\text{с}}$?
}
\answer{%
    \begin{align*}
    \lambda &= \frac L{N-1} = \frac {vt}{N-1} = \frac {5\,\frac{\text{м}}{\text{с}} \cdot 10\,\text{c}}{6 - 1} = 10{,}00\,\text{м},  \\
    T &= \frac {\lambda}{v} = \frac {vt}{\cbr{N-1}v} = \frac {t}{N-1} =  \frac { 10\,\text{c} }{6 - 1} = 2{,}00\,\text{с},  \\
    \nu &= \frac 1T = \frac {N-1}{t} = \frac {6 - 1}{ 10\,\text{c} } = 0{,}50\,\text{Гц}.
    \\
    &\text{Если же считать гребни целиком, т.е.
    не вычитать единицу:}  \\
    \lambda' &= \frac L{N} = \frac {vt}{N} = \frac {5\,\frac{\text{м}}{\text{с}} \cdot 10\,\text{c}}{ 6 } = 8{,}33\,\text{м},  \\
    T' &= \frac {\lambda'}{v} = \frac {vt}{Nv} = \frac tN =  \frac { 10\,\text{c} }{ 6 } = 1{,}67\,\text{с},  \\
    \nu' &= \frac 1{T'} = \frac {N}{t} = \frac { 6 }{ 10\,\text{c} } = 0{,}60\,\text{Гц}.
    \end{align*}
}

\variantsplitter

\addpersonalvariant{Владислав Емелин}

\tasknumber{1}%
\task{%
    Укажите, верны ли утверждения:
    \begin{itemize}
        \item механические волны переносят вещество,
        \item механические волны переносят энергию,
        \item источником механических волны служат колеблющиеся тела,
        \item продольные волны могут распространяться только в твёрдых телах,
        \item в твёрдых телах могут распространяться поперечные волны,
        \item скорость распространения волны равна произведению её длины на её частоту,
        \item звуковая волна~--- поперечная волна,
        \item волна на поверхности озера~--- поперечная волна (или же поверхностная).
    \end{itemize}
    (4 из 8~--- это «–»)
}
\answer{%
    нет, да, да, нет, да, да, нет, да
}
\solutionspace{20pt}

\tasknumber{2}%
\task{%
    Определите расстояние между третьим и девятым гребнями волн,
    если длина волны равна $4\,\text{м}$.
    Сколько между ними ещё уместилось гребней?
}
\answer{%
    $
        l = (n_2 - n_1) \cdot \lambda = \cbr{9 - 3} \cdot 4\,\text{м} = 24\,\text{м},
        \quad n = n_2 - n_1 - 1 = 9 - 3 - 1 = 5
    $
}
\solutionspace{120pt}

\tasknumber{3}%
\task{%
    Определите скорость звука в среде, если источник звука,
    колеблющийся с периодом $5\,\text{мc}$, возбуждает волны длиной
    $2{,}4\,\text{м}$.
}
\answer{%
    $\lambda = vT \implies v = \frac{\lambda}T = \frac{ 2{,}4\,\text{м} }{ 5\,\text{мc} } = { 480{,}0\,\frac{\text{м}}{\text{c}} }$
}
\solutionspace{120pt}

\tasknumber{4}%
\task{%
    Мимо неподвижного наблюдателя прошло $5$ гребней волн за $6\,\text{c}$,
    начиная с первого.
    Каковы длина, период и частота волны,
    если скорость распространения волн $1\,\frac{\text{м}}{\text{с}}$?
}
\answer{%
    \begin{align*}
    \lambda &= \frac L{N-1} = \frac {vt}{N-1} = \frac {1\,\frac{\text{м}}{\text{с}} \cdot 6\,\text{c}}{5 - 1} = 1{,}50\,\text{м},  \\
    T &= \frac {\lambda}{v} = \frac {vt}{\cbr{N-1}v} = \frac {t}{N-1} =  \frac { 6\,\text{c} }{5 - 1} = 1{,}50\,\text{с},  \\
    \nu &= \frac 1T = \frac {N-1}{t} = \frac {5 - 1}{ 6\,\text{c} } = 0{,}67\,\text{Гц}.
    \\
    &\text{Если же считать гребни целиком, т.е.
    не вычитать единицу:}  \\
    \lambda' &= \frac L{N} = \frac {vt}{N} = \frac {1\,\frac{\text{м}}{\text{с}} \cdot 6\,\text{c}}{ 5 } = 1{,}20\,\text{м},  \\
    T' &= \frac {\lambda'}{v} = \frac {vt}{Nv} = \frac tN =  \frac { 6\,\text{c} }{ 5 } = 1{,}20\,\text{с},  \\
    \nu' &= \frac 1{T'} = \frac {N}{t} = \frac { 5 }{ 6\,\text{c} } = 0{,}83\,\text{Гц}.
    \end{align*}
}

\variantsplitter

\addpersonalvariant{Артём Жичин}

\tasknumber{1}%
\task{%
    Укажите, верны ли утверждения:
    \begin{itemize}
        \item механические волны переносят вещество,
        \item механические волны переносят энергию,
        \item источником механических волны служат колеблющиеся тела,
        \item продольные волны могут распространяться только в твёрдых телах,
        \item в твёрдых телах могут распространяться поперечные волны,
        \item скорость распространения волны равна произведению её длины на её частоту,
        \item звуковая волна~--- поперечная волна,
        \item волна на поверхности озера~--- поперечная волна (или же поверхностная).
    \end{itemize}
    (4 из 8~--- это «–»)
}
\answer{%
    нет, да, да, нет, да, да, нет, да
}
\solutionspace{20pt}

\tasknumber{2}%
\task{%
    Определите расстояние между вторым и восьмым гребнями волн,
    если длина волны равна $4\,\text{м}$.
    Сколько между ними ещё уместилось гребней?
}
\answer{%
    $
        l = (n_2 - n_1) \cdot \lambda = \cbr{8 - 2} \cdot 4\,\text{м} = 24\,\text{м},
        \quad n = n_2 - n_1 - 1 = 8 - 2 - 1 = 5
    $
}
\solutionspace{120pt}

\tasknumber{3}%
\task{%
    Определите скорость звука в среде, если источник звука,
    колеблющийся с периодом $4\,\text{мc}$, возбуждает волны длиной
    $1{,}2\,\text{м}$.
}
\answer{%
    $\lambda = vT \implies v = \frac{\lambda}T = \frac{ 1{,}2\,\text{м} }{ 4\,\text{мc} } = { 300{,}0\,\frac{\text{м}}{\text{c}} }$
}
\solutionspace{120pt}

\tasknumber{4}%
\task{%
    Мимо неподвижного наблюдателя прошло $6$ гребней волн за $6\,\text{c}$,
    начиная с первого.
    Каковы длина, период и частота волны,
    если скорость распространения волн $3\,\frac{\text{м}}{\text{с}}$?
}
\answer{%
    \begin{align*}
    \lambda &= \frac L{N-1} = \frac {vt}{N-1} = \frac {3\,\frac{\text{м}}{\text{с}} \cdot 6\,\text{c}}{6 - 1} = 3{,}60\,\text{м},  \\
    T &= \frac {\lambda}{v} = \frac {vt}{\cbr{N-1}v} = \frac {t}{N-1} =  \frac { 6\,\text{c} }{6 - 1} = 1{,}20\,\text{с},  \\
    \nu &= \frac 1T = \frac {N-1}{t} = \frac {6 - 1}{ 6\,\text{c} } = 0{,}83\,\text{Гц}.
    \\
    &\text{Если же считать гребни целиком, т.е.
    не вычитать единицу:}  \\
    \lambda' &= \frac L{N} = \frac {vt}{N} = \frac {3\,\frac{\text{м}}{\text{с}} \cdot 6\,\text{c}}{ 6 } = 3{,}00\,\text{м},  \\
    T' &= \frac {\lambda'}{v} = \frac {vt}{Nv} = \frac tN =  \frac { 6\,\text{c} }{ 6 } = 1{,}00\,\text{с},  \\
    \nu' &= \frac 1{T'} = \frac {N}{t} = \frac { 6 }{ 6\,\text{c} } = 1{,}00\,\text{Гц}.
    \end{align*}
}

\variantsplitter

\addpersonalvariant{Елизавета Карманова}

\tasknumber{1}%
\task{%
    Укажите, верны ли утверждения:
    \begin{itemize}
        \item механические волны переносят вещество,
        \item механические волны переносят энергию,
        \item источником механических волны служат колеблющиеся тела,
        \item продольные волны могут распространяться только в твёрдых телах,
        \item в твёрдых телах могут распространяться поперечные волны,
        \item скорость распространения волны равна произведению её длины на её частоту,
        \item звуковая волна~--- поперечная волна,
        \item волна на поверхности озера~--- поперечная волна (или же поверхностная).
    \end{itemize}
    (4 из 8~--- это «–»)
}
\answer{%
    нет, да, да, нет, да, да, нет, да
}
\solutionspace{20pt}

\tasknumber{2}%
\task{%
    Определите расстояние между первым и седьмым гребнями волн,
    если длина волны равна $6\,\text{м}$.
    Сколько между ними ещё уместилось гребней?
}
\answer{%
    $
        l = (n_2 - n_1) \cdot \lambda = \cbr{7 - 1} \cdot 6\,\text{м} = 36\,\text{м},
        \quad n = n_2 - n_1 - 1 = 7 - 1 - 1 = 5
    $
}
\solutionspace{120pt}

\tasknumber{3}%
\task{%
    Определите скорость звука в среде, если источник звука,
    колеблющийся с периодом $5\,\text{мc}$, возбуждает волны длиной
    $2{,}4\,\text{м}$.
}
\answer{%
    $\lambda = vT \implies v = \frac{\lambda}T = \frac{ 2{,}4\,\text{м} }{ 5\,\text{мc} } = { 480{,}0\,\frac{\text{м}}{\text{c}} }$
}
\solutionspace{120pt}

\tasknumber{4}%
\task{%
    Мимо неподвижного наблюдателя прошло $4$ гребней волн за $8\,\text{c}$,
    начиная с первого.
    Каковы длина, период и частота волны,
    если скорость распространения волн $5\,\frac{\text{м}}{\text{с}}$?
}
\answer{%
    \begin{align*}
    \lambda &= \frac L{N-1} = \frac {vt}{N-1} = \frac {5\,\frac{\text{м}}{\text{с}} \cdot 8\,\text{c}}{4 - 1} = 13{,}33\,\text{м},  \\
    T &= \frac {\lambda}{v} = \frac {vt}{\cbr{N-1}v} = \frac {t}{N-1} =  \frac { 8\,\text{c} }{4 - 1} = 2{,}67\,\text{с},  \\
    \nu &= \frac 1T = \frac {N-1}{t} = \frac {4 - 1}{ 8\,\text{c} } = 0{,}38\,\text{Гц}.
    \\
    &\text{Если же считать гребни целиком, т.е.
    не вычитать единицу:}  \\
    \lambda' &= \frac L{N} = \frac {vt}{N} = \frac {5\,\frac{\text{м}}{\text{с}} \cdot 8\,\text{c}}{ 4 } = 10{,}00\,\text{м},  \\
    T' &= \frac {\lambda'}{v} = \frac {vt}{Nv} = \frac tN =  \frac { 8\,\text{c} }{ 4 } = 2{,}00\,\text{с},  \\
    \nu' &= \frac 1{T'} = \frac {N}{t} = \frac { 4 }{ 8\,\text{c} } = 0{,}50\,\text{Гц}.
    \end{align*}
}

\variantsplitter

\addpersonalvariant{Анна Кузьмичёва}

\tasknumber{1}%
\task{%
    Укажите, верны ли утверждения:
    \begin{itemize}
        \item механические волны переносят вещество,
        \item механические волны переносят энергию,
        \item источником механических волны служат колеблющиеся тела,
        \item продольные волны могут распространяться только в твёрдых телах,
        \item в твёрдых телах могут распространяться поперечные волны,
        \item скорость распространения волны равна произведению её длины на её частоту,
        \item звуковая волна~--- поперечная волна,
        \item волна на поверхности озера~--- поперечная волна (или же поверхностная).
    \end{itemize}
    (4 из 8~--- это «–»)
}
\answer{%
    нет, да, да, нет, да, да, нет, да
}
\solutionspace{20pt}

\tasknumber{2}%
\task{%
    Определите расстояние между вторым и девятым гребнями волн,
    если длина волны равна $6\,\text{м}$.
    Сколько между ними ещё уместилось гребней?
}
\answer{%
    $
        l = (n_2 - n_1) \cdot \lambda = \cbr{9 - 2} \cdot 6\,\text{м} = 42\,\text{м},
        \quad n = n_2 - n_1 - 1 = 9 - 2 - 1 = 6
    $
}
\solutionspace{120pt}

\tasknumber{3}%
\task{%
    Определите скорость звука в среде, если источник звука,
    колеблющийся с периодом $2\,\text{мc}$, возбуждает волны длиной
    $2{,}1\,\text{м}$.
}
\answer{%
    $\lambda = vT \implies v = \frac{\lambda}T = \frac{ 2{,}1\,\text{м} }{ 2\,\text{мc} } = { 1050{,}0\,\frac{\text{м}}{\text{c}} }$
}
\solutionspace{120pt}

\tasknumber{4}%
\task{%
    Мимо неподвижного наблюдателя прошло $6$ гребней волн за $5\,\text{c}$,
    начиная с первого.
    Каковы длина, период и частота волны,
    если скорость распространения волн $1\,\frac{\text{м}}{\text{с}}$?
}
\answer{%
    \begin{align*}
    \lambda &= \frac L{N-1} = \frac {vt}{N-1} = \frac {1\,\frac{\text{м}}{\text{с}} \cdot 5\,\text{c}}{6 - 1} = 1{,}00\,\text{м},  \\
    T &= \frac {\lambda}{v} = \frac {vt}{\cbr{N-1}v} = \frac {t}{N-1} =  \frac { 5\,\text{c} }{6 - 1} = 1{,}00\,\text{с},  \\
    \nu &= \frac 1T = \frac {N-1}{t} = \frac {6 - 1}{ 5\,\text{c} } = 1{,}00\,\text{Гц}.
    \\
    &\text{Если же считать гребни целиком, т.е.
    не вычитать единицу:}  \\
    \lambda' &= \frac L{N} = \frac {vt}{N} = \frac {1\,\frac{\text{м}}{\text{с}} \cdot 5\,\text{c}}{ 6 } = 0{,}83\,\text{м},  \\
    T' &= \frac {\lambda'}{v} = \frac {vt}{Nv} = \frac tN =  \frac { 5\,\text{c} }{ 6 } = 0{,}83\,\text{с},  \\
    \nu' &= \frac 1{T'} = \frac {N}{t} = \frac { 6 }{ 5\,\text{c} } = 1{,}20\,\text{Гц}.
    \end{align*}
}

\variantsplitter

\addpersonalvariant{Алёна Куприянова}

\tasknumber{1}%
\task{%
    Укажите, верны ли утверждения:
    \begin{itemize}
        \item механические волны переносят вещество,
        \item механические волны переносят энергию,
        \item источником механических волны служат колеблющиеся тела,
        \item продольные волны могут распространяться только в твёрдых телах,
        \item в твёрдых телах могут распространяться поперечные волны,
        \item скорость распространения волны равна произведению её длины на её частоту,
        \item звуковая волна~--- поперечная волна,
        \item волна на поверхности озера~--- поперечная волна (или же поверхностная).
    \end{itemize}
    (4 из 8~--- это «–»)
}
\answer{%
    нет, да, да, нет, да, да, нет, да
}
\solutionspace{20pt}

\tasknumber{2}%
\task{%
    Определите расстояние между первым и восьмым гребнями волн,
    если длина волны равна $4\,\text{м}$.
    Сколько между ними ещё уместилось гребней?
}
\answer{%
    $
        l = (n_2 - n_1) \cdot \lambda = \cbr{8 - 1} \cdot 4\,\text{м} = 28\,\text{м},
        \quad n = n_2 - n_1 - 1 = 8 - 1 - 1 = 6
    $
}
\solutionspace{120pt}

\tasknumber{3}%
\task{%
    Определите скорость звука в среде, если источник звука,
    колеблющийся с периодом $5\,\text{мc}$, возбуждает волны длиной
    $2{,}4\,\text{м}$.
}
\answer{%
    $\lambda = vT \implies v = \frac{\lambda}T = \frac{ 2{,}4\,\text{м} }{ 5\,\text{мc} } = { 480{,}0\,\frac{\text{м}}{\text{c}} }$
}
\solutionspace{120pt}

\tasknumber{4}%
\task{%
    Мимо неподвижного наблюдателя прошло $6$ гребней волн за $8\,\text{c}$,
    начиная с первого.
    Каковы длина, период и частота волны,
    если скорость распространения волн $5\,\frac{\text{м}}{\text{с}}$?
}
\answer{%
    \begin{align*}
    \lambda &= \frac L{N-1} = \frac {vt}{N-1} = \frac {5\,\frac{\text{м}}{\text{с}} \cdot 8\,\text{c}}{6 - 1} = 8{,}00\,\text{м},  \\
    T &= \frac {\lambda}{v} = \frac {vt}{\cbr{N-1}v} = \frac {t}{N-1} =  \frac { 8\,\text{c} }{6 - 1} = 1{,}60\,\text{с},  \\
    \nu &= \frac 1T = \frac {N-1}{t} = \frac {6 - 1}{ 8\,\text{c} } = 0{,}62\,\text{Гц}.
    \\
    &\text{Если же считать гребни целиком, т.е.
    не вычитать единицу:}  \\
    \lambda' &= \frac L{N} = \frac {vt}{N} = \frac {5\,\frac{\text{м}}{\text{с}} \cdot 8\,\text{c}}{ 6 } = 6{,}67\,\text{м},  \\
    T' &= \frac {\lambda'}{v} = \frac {vt}{Nv} = \frac tN =  \frac { 8\,\text{c} }{ 6 } = 1{,}33\,\text{с},  \\
    \nu' &= \frac 1{T'} = \frac {N}{t} = \frac { 6 }{ 8\,\text{c} } = 0{,}75\,\text{Гц}.
    \end{align*}
}

\variantsplitter

\addpersonalvariant{Анастасия Ламанова}

\tasknumber{1}%
\task{%
    Укажите, верны ли утверждения:
    \begin{itemize}
        \item механические волны переносят вещество,
        \item механические волны переносят энергию,
        \item источником механических волны служат колеблющиеся тела,
        \item продольные волны могут распространяться только в твёрдых телах,
        \item в твёрдых телах могут распространяться поперечные волны,
        \item скорость распространения волны равна произведению её длины на её частоту,
        \item звуковая волна~--- поперечная волна,
        \item волна на поверхности озера~--- поперечная волна (или же поверхностная).
    \end{itemize}
    (4 из 8~--- это «–»)
}
\answer{%
    нет, да, да, нет, да, да, нет, да
}
\solutionspace{20pt}

\tasknumber{2}%
\task{%
    Определите расстояние между вторым и шестым гребнями волн,
    если длина волны равна $4\,\text{м}$.
    Сколько между ними ещё уместилось гребней?
}
\answer{%
    $
        l = (n_2 - n_1) \cdot \lambda = \cbr{6 - 2} \cdot 4\,\text{м} = 16\,\text{м},
        \quad n = n_2 - n_1 - 1 = 6 - 2 - 1 = 3
    $
}
\solutionspace{120pt}

\tasknumber{3}%
\task{%
    Определите скорость звука в среде, если источник звука,
    колеблющийся с периодом $3\,\text{мc}$, возбуждает волны длиной
    $2{,}4\,\text{м}$.
}
\answer{%
    $\lambda = vT \implies v = \frac{\lambda}T = \frac{ 2{,}4\,\text{м} }{ 3\,\text{мc} } = { 800{,}0\,\frac{\text{м}}{\text{c}} }$
}
\solutionspace{120pt}

\tasknumber{4}%
\task{%
    Мимо неподвижного наблюдателя прошло $5$ гребней волн за $5\,\text{c}$,
    начиная с первого.
    Каковы длина, период и частота волны,
    если скорость распространения волн $4\,\frac{\text{м}}{\text{с}}$?
}
\answer{%
    \begin{align*}
    \lambda &= \frac L{N-1} = \frac {vt}{N-1} = \frac {4\,\frac{\text{м}}{\text{с}} \cdot 5\,\text{c}}{5 - 1} = 5{,}00\,\text{м},  \\
    T &= \frac {\lambda}{v} = \frac {vt}{\cbr{N-1}v} = \frac {t}{N-1} =  \frac { 5\,\text{c} }{5 - 1} = 1{,}25\,\text{с},  \\
    \nu &= \frac 1T = \frac {N-1}{t} = \frac {5 - 1}{ 5\,\text{c} } = 0{,}80\,\text{Гц}.
    \\
    &\text{Если же считать гребни целиком, т.е.
    не вычитать единицу:}  \\
    \lambda' &= \frac L{N} = \frac {vt}{N} = \frac {4\,\frac{\text{м}}{\text{с}} \cdot 5\,\text{c}}{ 5 } = 4{,}00\,\text{м},  \\
    T' &= \frac {\lambda'}{v} = \frac {vt}{Nv} = \frac tN =  \frac { 5\,\text{c} }{ 5 } = 1{,}00\,\text{с},  \\
    \nu' &= \frac 1{T'} = \frac {N}{t} = \frac { 5 }{ 5\,\text{c} } = 1{,}00\,\text{Гц}.
    \end{align*}
}

\variantsplitter

\addpersonalvariant{Виктория Легонькова}

\tasknumber{1}%
\task{%
    Укажите, верны ли утверждения:
    \begin{itemize}
        \item механические волны переносят вещество,
        \item механические волны переносят энергию,
        \item источником механических волны служат колеблющиеся тела,
        \item продольные волны могут распространяться только в твёрдых телах,
        \item в твёрдых телах могут распространяться поперечные волны,
        \item скорость распространения волны равна произведению её длины на её частоту,
        \item звуковая волна~--- поперечная волна,
        \item волна на поверхности озера~--- поперечная волна (или же поверхностная).
    \end{itemize}
    (4 из 8~--- это «–»)
}
\answer{%
    нет, да, да, нет, да, да, нет, да
}
\solutionspace{20pt}

\tasknumber{2}%
\task{%
    Определите расстояние между вторым и восьмым гребнями волн,
    если длина волны равна $3\,\text{м}$.
    Сколько между ними ещё уместилось гребней?
}
\answer{%
    $
        l = (n_2 - n_1) \cdot \lambda = \cbr{8 - 2} \cdot 3\,\text{м} = 18\,\text{м},
        \quad n = n_2 - n_1 - 1 = 8 - 2 - 1 = 5
    $
}
\solutionspace{120pt}

\tasknumber{3}%
\task{%
    Определите скорость звука в среде, если источник звука,
    колеблющийся с периодом $2\,\text{мc}$, возбуждает волны длиной
    $2{,}1\,\text{м}$.
}
\answer{%
    $\lambda = vT \implies v = \frac{\lambda}T = \frac{ 2{,}1\,\text{м} }{ 2\,\text{мc} } = { 1050{,}0\,\frac{\text{м}}{\text{c}} }$
}
\solutionspace{120pt}

\tasknumber{4}%
\task{%
    Мимо неподвижного наблюдателя прошло $5$ гребней волн за $5\,\text{c}$,
    начиная с первого.
    Каковы длина, период и частота волны,
    если скорость распространения волн $2\,\frac{\text{м}}{\text{с}}$?
}
\answer{%
    \begin{align*}
    \lambda &= \frac L{N-1} = \frac {vt}{N-1} = \frac {2\,\frac{\text{м}}{\text{с}} \cdot 5\,\text{c}}{5 - 1} = 2{,}50\,\text{м},  \\
    T &= \frac {\lambda}{v} = \frac {vt}{\cbr{N-1}v} = \frac {t}{N-1} =  \frac { 5\,\text{c} }{5 - 1} = 1{,}25\,\text{с},  \\
    \nu &= \frac 1T = \frac {N-1}{t} = \frac {5 - 1}{ 5\,\text{c} } = 0{,}80\,\text{Гц}.
    \\
    &\text{Если же считать гребни целиком, т.е.
    не вычитать единицу:}  \\
    \lambda' &= \frac L{N} = \frac {vt}{N} = \frac {2\,\frac{\text{м}}{\text{с}} \cdot 5\,\text{c}}{ 5 } = 2{,}00\,\text{м},  \\
    T' &= \frac {\lambda'}{v} = \frac {vt}{Nv} = \frac tN =  \frac { 5\,\text{c} }{ 5 } = 1{,}00\,\text{с},  \\
    \nu' &= \frac 1{T'} = \frac {N}{t} = \frac { 5 }{ 5\,\text{c} } = 1{,}00\,\text{Гц}.
    \end{align*}
}

\variantsplitter

\addpersonalvariant{Семён Мартынов}

\tasknumber{1}%
\task{%
    Укажите, верны ли утверждения:
    \begin{itemize}
        \item механические волны переносят вещество,
        \item механические волны переносят энергию,
        \item источником механических волны служат колеблющиеся тела,
        \item продольные волны могут распространяться только в твёрдых телах,
        \item в твёрдых телах могут распространяться поперечные волны,
        \item скорость распространения волны равна произведению её длины на её частоту,
        \item звуковая волна~--- поперечная волна,
        \item волна на поверхности озера~--- поперечная волна (или же поверхностная).
    \end{itemize}
    (4 из 8~--- это «–»)
}
\answer{%
    нет, да, да, нет, да, да, нет, да
}
\solutionspace{20pt}

\tasknumber{2}%
\task{%
    Определите расстояние между вторым и шестым гребнями волн,
    если длина волны равна $6\,\text{м}$.
    Сколько между ними ещё уместилось гребней?
}
\answer{%
    $
        l = (n_2 - n_1) \cdot \lambda = \cbr{6 - 2} \cdot 6\,\text{м} = 24\,\text{м},
        \quad n = n_2 - n_1 - 1 = 6 - 2 - 1 = 3
    $
}
\solutionspace{120pt}

\tasknumber{3}%
\task{%
    Определите скорость звука в среде, если источник звука,
    колеблющийся с периодом $3\,\text{мc}$, возбуждает волны длиной
    $2{,}1\,\text{м}$.
}
\answer{%
    $\lambda = vT \implies v = \frac{\lambda}T = \frac{ 2{,}1\,\text{м} }{ 3\,\text{мc} } = { 700{,}0\,\frac{\text{м}}{\text{c}} }$
}
\solutionspace{120pt}

\tasknumber{4}%
\task{%
    Мимо неподвижного наблюдателя прошло $4$ гребней волн за $8\,\text{c}$,
    начиная с первого.
    Каковы длина, период и частота волны,
    если скорость распространения волн $5\,\frac{\text{м}}{\text{с}}$?
}
\answer{%
    \begin{align*}
    \lambda &= \frac L{N-1} = \frac {vt}{N-1} = \frac {5\,\frac{\text{м}}{\text{с}} \cdot 8\,\text{c}}{4 - 1} = 13{,}33\,\text{м},  \\
    T &= \frac {\lambda}{v} = \frac {vt}{\cbr{N-1}v} = \frac {t}{N-1} =  \frac { 8\,\text{c} }{4 - 1} = 2{,}67\,\text{с},  \\
    \nu &= \frac 1T = \frac {N-1}{t} = \frac {4 - 1}{ 8\,\text{c} } = 0{,}38\,\text{Гц}.
    \\
    &\text{Если же считать гребни целиком, т.е.
    не вычитать единицу:}  \\
    \lambda' &= \frac L{N} = \frac {vt}{N} = \frac {5\,\frac{\text{м}}{\text{с}} \cdot 8\,\text{c}}{ 4 } = 10{,}00\,\text{м},  \\
    T' &= \frac {\lambda'}{v} = \frac {vt}{Nv} = \frac tN =  \frac { 8\,\text{c} }{ 4 } = 2{,}00\,\text{с},  \\
    \nu' &= \frac 1{T'} = \frac {N}{t} = \frac { 4 }{ 8\,\text{c} } = 0{,}50\,\text{Гц}.
    \end{align*}
}

\variantsplitter

\addpersonalvariant{Варвара Минаева}

\tasknumber{1}%
\task{%
    Укажите, верны ли утверждения:
    \begin{itemize}
        \item механические волны переносят вещество,
        \item механические волны переносят энергию,
        \item источником механических волны служат колеблющиеся тела,
        \item продольные волны могут распространяться только в твёрдых телах,
        \item в твёрдых телах могут распространяться поперечные волны,
        \item скорость распространения волны равна произведению её длины на её частоту,
        \item звуковая волна~--- поперечная волна,
        \item волна на поверхности озера~--- поперечная волна (или же поверхностная).
    \end{itemize}
    (4 из 8~--- это «–»)
}
\answer{%
    нет, да, да, нет, да, да, нет, да
}
\solutionspace{20pt}

\tasknumber{2}%
\task{%
    Определите расстояние между первым и десятым гребнями волн,
    если длина волны равна $4\,\text{м}$.
    Сколько между ними ещё уместилось гребней?
}
\answer{%
    $
        l = (n_2 - n_1) \cdot \lambda = \cbr{10 - 1} \cdot 4\,\text{м} = 36\,\text{м},
        \quad n = n_2 - n_1 - 1 = 10 - 1 - 1 = 8
    $
}
\solutionspace{120pt}

\tasknumber{3}%
\task{%
    Определите скорость звука в среде, если источник звука,
    колеблющийся с периодом $5\,\text{мc}$, возбуждает волны длиной
    $1{,}2\,\text{м}$.
}
\answer{%
    $\lambda = vT \implies v = \frac{\lambda}T = \frac{ 1{,}2\,\text{м} }{ 5\,\text{мc} } = { 240{,}0\,\frac{\text{м}}{\text{c}} }$
}
\solutionspace{120pt}

\tasknumber{4}%
\task{%
    Мимо неподвижного наблюдателя прошло $4$ гребней волн за $6\,\text{c}$,
    начиная с первого.
    Каковы длина, период и частота волны,
    если скорость распространения волн $5\,\frac{\text{м}}{\text{с}}$?
}
\answer{%
    \begin{align*}
    \lambda &= \frac L{N-1} = \frac {vt}{N-1} = \frac {5\,\frac{\text{м}}{\text{с}} \cdot 6\,\text{c}}{4 - 1} = 10{,}00\,\text{м},  \\
    T &= \frac {\lambda}{v} = \frac {vt}{\cbr{N-1}v} = \frac {t}{N-1} =  \frac { 6\,\text{c} }{4 - 1} = 2{,}00\,\text{с},  \\
    \nu &= \frac 1T = \frac {N-1}{t} = \frac {4 - 1}{ 6\,\text{c} } = 0{,}50\,\text{Гц}.
    \\
    &\text{Если же считать гребни целиком, т.е.
    не вычитать единицу:}  \\
    \lambda' &= \frac L{N} = \frac {vt}{N} = \frac {5\,\frac{\text{м}}{\text{с}} \cdot 6\,\text{c}}{ 4 } = 7{,}50\,\text{м},  \\
    T' &= \frac {\lambda'}{v} = \frac {vt}{Nv} = \frac tN =  \frac { 6\,\text{c} }{ 4 } = 1{,}50\,\text{с},  \\
    \nu' &= \frac 1{T'} = \frac {N}{t} = \frac { 4 }{ 6\,\text{c} } = 0{,}67\,\text{Гц}.
    \end{align*}
}

\variantsplitter

\addpersonalvariant{Тимофей Полетаев}

\tasknumber{1}%
\task{%
    Укажите, верны ли утверждения:
    \begin{itemize}
        \item механические волны переносят вещество,
        \item механические волны переносят энергию,
        \item источником механических волны служат колеблющиеся тела,
        \item продольные волны могут распространяться только в твёрдых телах,
        \item в твёрдых телах могут распространяться поперечные волны,
        \item скорость распространения волны равна произведению её длины на её частоту,
        \item звуковая волна~--- поперечная волна,
        \item волна на поверхности озера~--- поперечная волна (или же поверхностная).
    \end{itemize}
    (4 из 8~--- это «–»)
}
\answer{%
    нет, да, да, нет, да, да, нет, да
}
\solutionspace{20pt}

\tasknumber{2}%
\task{%
    Определите расстояние между первым и девятым гребнями волн,
    если длина волны равна $5\,\text{м}$.
    Сколько между ними ещё уместилось гребней?
}
\answer{%
    $
        l = (n_2 - n_1) \cdot \lambda = \cbr{9 - 1} \cdot 5\,\text{м} = 40\,\text{м},
        \quad n = n_2 - n_1 - 1 = 9 - 1 - 1 = 7
    $
}
\solutionspace{120pt}

\tasknumber{3}%
\task{%
    Определите скорость звука в среде, если источник звука,
    колеблющийся с периодом $3\,\text{мc}$, возбуждает волны длиной
    $2{,}4\,\text{м}$.
}
\answer{%
    $\lambda = vT \implies v = \frac{\lambda}T = \frac{ 2{,}4\,\text{м} }{ 3\,\text{мc} } = { 800{,}0\,\frac{\text{м}}{\text{c}} }$
}
\solutionspace{120pt}

\tasknumber{4}%
\task{%
    Мимо неподвижного наблюдателя прошло $5$ гребней волн за $6\,\text{c}$,
    начиная с первого.
    Каковы длина, период и частота волны,
    если скорость распространения волн $3\,\frac{\text{м}}{\text{с}}$?
}
\answer{%
    \begin{align*}
    \lambda &= \frac L{N-1} = \frac {vt}{N-1} = \frac {3\,\frac{\text{м}}{\text{с}} \cdot 6\,\text{c}}{5 - 1} = 4{,}50\,\text{м},  \\
    T &= \frac {\lambda}{v} = \frac {vt}{\cbr{N-1}v} = \frac {t}{N-1} =  \frac { 6\,\text{c} }{5 - 1} = 1{,}50\,\text{с},  \\
    \nu &= \frac 1T = \frac {N-1}{t} = \frac {5 - 1}{ 6\,\text{c} } = 0{,}67\,\text{Гц}.
    \\
    &\text{Если же считать гребни целиком, т.е.
    не вычитать единицу:}  \\
    \lambda' &= \frac L{N} = \frac {vt}{N} = \frac {3\,\frac{\text{м}}{\text{с}} \cdot 6\,\text{c}}{ 5 } = 3{,}60\,\text{м},  \\
    T' &= \frac {\lambda'}{v} = \frac {vt}{Nv} = \frac tN =  \frac { 6\,\text{c} }{ 5 } = 1{,}20\,\text{с},  \\
    \nu' &= \frac 1{T'} = \frac {N}{t} = \frac { 5 }{ 6\,\text{c} } = 0{,}83\,\text{Гц}.
    \end{align*}
}

\variantsplitter

\addpersonalvariant{Андрей Рожков}

\tasknumber{1}%
\task{%
    Укажите, верны ли утверждения:
    \begin{itemize}
        \item механические волны переносят вещество,
        \item механические волны переносят энергию,
        \item источником механических волны служат колеблющиеся тела,
        \item продольные волны могут распространяться только в твёрдых телах,
        \item в твёрдых телах могут распространяться поперечные волны,
        \item скорость распространения волны равна произведению её длины на её частоту,
        \item звуковая волна~--- поперечная волна,
        \item волна на поверхности озера~--- поперечная волна (или же поверхностная).
    \end{itemize}
    (4 из 8~--- это «–»)
}
\answer{%
    нет, да, да, нет, да, да, нет, да
}
\solutionspace{20pt}

\tasknumber{2}%
\task{%
    Определите расстояние между третьим и шестым гребнями волн,
    если длина волны равна $3\,\text{м}$.
    Сколько между ними ещё уместилось гребней?
}
\answer{%
    $
        l = (n_2 - n_1) \cdot \lambda = \cbr{6 - 3} \cdot 3\,\text{м} = 9\,\text{м},
        \quad n = n_2 - n_1 - 1 = 6 - 3 - 1 = 2
    $
}
\solutionspace{120pt}

\tasknumber{3}%
\task{%
    Определите скорость звука в среде, если источник звука,
    колеблющийся с периодом $2\,\text{мc}$, возбуждает волны длиной
    $1{,}2\,\text{м}$.
}
\answer{%
    $\lambda = vT \implies v = \frac{\lambda}T = \frac{ 1{,}2\,\text{м} }{ 2\,\text{мc} } = { 600{,}0\,\frac{\text{м}}{\text{c}} }$
}
\solutionspace{120pt}

\tasknumber{4}%
\task{%
    Мимо неподвижного наблюдателя прошло $6$ гребней волн за $5\,\text{c}$,
    начиная с первого.
    Каковы длина, период и частота волны,
    если скорость распространения волн $2\,\frac{\text{м}}{\text{с}}$?
}
\answer{%
    \begin{align*}
    \lambda &= \frac L{N-1} = \frac {vt}{N-1} = \frac {2\,\frac{\text{м}}{\text{с}} \cdot 5\,\text{c}}{6 - 1} = 2{,}00\,\text{м},  \\
    T &= \frac {\lambda}{v} = \frac {vt}{\cbr{N-1}v} = \frac {t}{N-1} =  \frac { 5\,\text{c} }{6 - 1} = 1{,}00\,\text{с},  \\
    \nu &= \frac 1T = \frac {N-1}{t} = \frac {6 - 1}{ 5\,\text{c} } = 1{,}00\,\text{Гц}.
    \\
    &\text{Если же считать гребни целиком, т.е.
    не вычитать единицу:}  \\
    \lambda' &= \frac L{N} = \frac {vt}{N} = \frac {2\,\frac{\text{м}}{\text{с}} \cdot 5\,\text{c}}{ 6 } = 1{,}67\,\text{м},  \\
    T' &= \frac {\lambda'}{v} = \frac {vt}{Nv} = \frac tN =  \frac { 5\,\text{c} }{ 6 } = 0{,}83\,\text{с},  \\
    \nu' &= \frac 1{T'} = \frac {N}{t} = \frac { 6 }{ 5\,\text{c} } = 1{,}20\,\text{Гц}.
    \end{align*}
}

\variantsplitter

\addpersonalvariant{Тимур Сидиков}

\tasknumber{1}%
\task{%
    Укажите, верны ли утверждения:
    \begin{itemize}
        \item механические волны переносят вещество,
        \item механические волны переносят энергию,
        \item источником механических волны служат колеблющиеся тела,
        \item продольные волны могут распространяться только в твёрдых телах,
        \item в твёрдых телах могут распространяться поперечные волны,
        \item скорость распространения волны равна произведению её длины на её частоту,
        \item звуковая волна~--- поперечная волна,
        \item волна на поверхности озера~--- поперечная волна (или же поверхностная).
    \end{itemize}
    (4 из 8~--- это «–»)
}
\answer{%
    нет, да, да, нет, да, да, нет, да
}
\solutionspace{20pt}

\tasknumber{2}%
\task{%
    Определите расстояние между первым и десятым гребнями волн,
    если длина волны равна $6\,\text{м}$.
    Сколько между ними ещё уместилось гребней?
}
\answer{%
    $
        l = (n_2 - n_1) \cdot \lambda = \cbr{10 - 1} \cdot 6\,\text{м} = 54\,\text{м},
        \quad n = n_2 - n_1 - 1 = 10 - 1 - 1 = 8
    $
}
\solutionspace{120pt}

\tasknumber{3}%
\task{%
    Определите скорость звука в среде, если источник звука,
    колеблющийся с периодом $5\,\text{мc}$, возбуждает волны длиной
    $2{,}1\,\text{м}$.
}
\answer{%
    $\lambda = vT \implies v = \frac{\lambda}T = \frac{ 2{,}1\,\text{м} }{ 5\,\text{мc} } = { 420{,}0\,\frac{\text{м}}{\text{c}} }$
}
\solutionspace{120pt}

\tasknumber{4}%
\task{%
    Мимо неподвижного наблюдателя прошло $5$ гребней волн за $5\,\text{c}$,
    начиная с первого.
    Каковы длина, период и частота волны,
    если скорость распространения волн $5\,\frac{\text{м}}{\text{с}}$?
}
\answer{%
    \begin{align*}
    \lambda &= \frac L{N-1} = \frac {vt}{N-1} = \frac {5\,\frac{\text{м}}{\text{с}} \cdot 5\,\text{c}}{5 - 1} = 6{,}25\,\text{м},  \\
    T &= \frac {\lambda}{v} = \frac {vt}{\cbr{N-1}v} = \frac {t}{N-1} =  \frac { 5\,\text{c} }{5 - 1} = 1{,}25\,\text{с},  \\
    \nu &= \frac 1T = \frac {N-1}{t} = \frac {5 - 1}{ 5\,\text{c} } = 0{,}80\,\text{Гц}.
    \\
    &\text{Если же считать гребни целиком, т.е.
    не вычитать единицу:}  \\
    \lambda' &= \frac L{N} = \frac {vt}{N} = \frac {5\,\frac{\text{м}}{\text{с}} \cdot 5\,\text{c}}{ 5 } = 5{,}00\,\text{м},  \\
    T' &= \frac {\lambda'}{v} = \frac {vt}{Nv} = \frac tN =  \frac { 5\,\text{c} }{ 5 } = 1{,}00\,\text{с},  \\
    \nu' &= \frac 1{T'} = \frac {N}{t} = \frac { 5 }{ 5\,\text{c} } = 1{,}00\,\text{Гц}.
    \end{align*}
}

\variantsplitter

\addpersonalvariant{Рената Таржиманова}

\tasknumber{1}%
\task{%
    Укажите, верны ли утверждения:
    \begin{itemize}
        \item механические волны переносят вещество,
        \item механические волны переносят энергию,
        \item источником механических волны служат колеблющиеся тела,
        \item продольные волны могут распространяться только в твёрдых телах,
        \item в твёрдых телах могут распространяться поперечные волны,
        \item скорость распространения волны равна произведению её длины на её частоту,
        \item звуковая волна~--- поперечная волна,
        \item волна на поверхности озера~--- поперечная волна (или же поверхностная).
    \end{itemize}
    (4 из 8~--- это «–»)
}
\answer{%
    нет, да, да, нет, да, да, нет, да
}
\solutionspace{20pt}

\tasknumber{2}%
\task{%
    Определите расстояние между вторым и девятым гребнями волн,
    если длина волны равна $6\,\text{м}$.
    Сколько между ними ещё уместилось гребней?
}
\answer{%
    $
        l = (n_2 - n_1) \cdot \lambda = \cbr{9 - 2} \cdot 6\,\text{м} = 42\,\text{м},
        \quad n = n_2 - n_1 - 1 = 9 - 2 - 1 = 6
    $
}
\solutionspace{120pt}

\tasknumber{3}%
\task{%
    Определите скорость звука в среде, если источник звука,
    колеблющийся с периодом $4\,\text{мc}$, возбуждает волны длиной
    $1{,}5\,\text{м}$.
}
\answer{%
    $\lambda = vT \implies v = \frac{\lambda}T = \frac{ 1{,}5\,\text{м} }{ 4\,\text{мc} } = { 375{,}0\,\frac{\text{м}}{\text{c}} }$
}
\solutionspace{120pt}

\tasknumber{4}%
\task{%
    Мимо неподвижного наблюдателя прошло $5$ гребней волн за $5\,\text{c}$,
    начиная с первого.
    Каковы длина, период и частота волны,
    если скорость распространения волн $5\,\frac{\text{м}}{\text{с}}$?
}
\answer{%
    \begin{align*}
    \lambda &= \frac L{N-1} = \frac {vt}{N-1} = \frac {5\,\frac{\text{м}}{\text{с}} \cdot 5\,\text{c}}{5 - 1} = 6{,}25\,\text{м},  \\
    T &= \frac {\lambda}{v} = \frac {vt}{\cbr{N-1}v} = \frac {t}{N-1} =  \frac { 5\,\text{c} }{5 - 1} = 1{,}25\,\text{с},  \\
    \nu &= \frac 1T = \frac {N-1}{t} = \frac {5 - 1}{ 5\,\text{c} } = 0{,}80\,\text{Гц}.
    \\
    &\text{Если же считать гребни целиком, т.е.
    не вычитать единицу:}  \\
    \lambda' &= \frac L{N} = \frac {vt}{N} = \frac {5\,\frac{\text{м}}{\text{с}} \cdot 5\,\text{c}}{ 5 } = 5{,}00\,\text{м},  \\
    T' &= \frac {\lambda'}{v} = \frac {vt}{Nv} = \frac tN =  \frac { 5\,\text{c} }{ 5 } = 1{,}00\,\text{с},  \\
    \nu' &= \frac 1{T'} = \frac {N}{t} = \frac { 5 }{ 5\,\text{c} } = 1{,}00\,\text{Гц}.
    \end{align*}
}

\variantsplitter

\addpersonalvariant{Глеб Урбанский}

\tasknumber{1}%
\task{%
    Укажите, верны ли утверждения:
    \begin{itemize}
        \item механические волны переносят вещество,
        \item механические волны переносят энергию,
        \item источником механических волны служат колеблющиеся тела,
        \item продольные волны могут распространяться только в твёрдых телах,
        \item в твёрдых телах могут распространяться поперечные волны,
        \item скорость распространения волны равна произведению её длины на её частоту,
        \item звуковая волна~--- поперечная волна,
        \item волна на поверхности озера~--- поперечная волна (или же поверхностная).
    \end{itemize}
    (4 из 8~--- это «–»)
}
\answer{%
    нет, да, да, нет, да, да, нет, да
}
\solutionspace{20pt}

\tasknumber{2}%
\task{%
    Определите расстояние между первым и седьмым гребнями волн,
    если длина волны равна $5\,\text{м}$.
    Сколько между ними ещё уместилось гребней?
}
\answer{%
    $
        l = (n_2 - n_1) \cdot \lambda = \cbr{7 - 1} \cdot 5\,\text{м} = 30\,\text{м},
        \quad n = n_2 - n_1 - 1 = 7 - 1 - 1 = 5
    $
}
\solutionspace{120pt}

\tasknumber{3}%
\task{%
    Определите скорость звука в среде, если источник звука,
    колеблющийся с периодом $5\,\text{мc}$, возбуждает волны длиной
    $2{,}1\,\text{м}$.
}
\answer{%
    $\lambda = vT \implies v = \frac{\lambda}T = \frac{ 2{,}1\,\text{м} }{ 5\,\text{мc} } = { 420{,}0\,\frac{\text{м}}{\text{c}} }$
}
\solutionspace{120pt}

\tasknumber{4}%
\task{%
    Мимо неподвижного наблюдателя прошло $4$ гребней волн за $5\,\text{c}$,
    начиная с первого.
    Каковы длина, период и частота волны,
    если скорость распространения волн $3\,\frac{\text{м}}{\text{с}}$?
}
\answer{%
    \begin{align*}
    \lambda &= \frac L{N-1} = \frac {vt}{N-1} = \frac {3\,\frac{\text{м}}{\text{с}} \cdot 5\,\text{c}}{4 - 1} = 5{,}00\,\text{м},  \\
    T &= \frac {\lambda}{v} = \frac {vt}{\cbr{N-1}v} = \frac {t}{N-1} =  \frac { 5\,\text{c} }{4 - 1} = 1{,}67\,\text{с},  \\
    \nu &= \frac 1T = \frac {N-1}{t} = \frac {4 - 1}{ 5\,\text{c} } = 0{,}60\,\text{Гц}.
    \\
    &\text{Если же считать гребни целиком, т.е.
    не вычитать единицу:}  \\
    \lambda' &= \frac L{N} = \frac {vt}{N} = \frac {3\,\frac{\text{м}}{\text{с}} \cdot 5\,\text{c}}{ 4 } = 3{,}75\,\text{м},  \\
    T' &= \frac {\lambda'}{v} = \frac {vt}{Nv} = \frac tN =  \frac { 5\,\text{c} }{ 4 } = 1{,}25\,\text{с},  \\
    \nu' &= \frac 1{T'} = \frac {N}{t} = \frac { 4 }{ 5\,\text{c} } = 0{,}80\,\text{Гц}.
    \end{align*}
}

\variantsplitter

\addpersonalvariant{Кирилл Швец}

\tasknumber{1}%
\task{%
    Укажите, верны ли утверждения:
    \begin{itemize}
        \item механические волны переносят вещество,
        \item механические волны переносят энергию,
        \item источником механических волны служат колеблющиеся тела,
        \item продольные волны могут распространяться только в твёрдых телах,
        \item в твёрдых телах могут распространяться поперечные волны,
        \item скорость распространения волны равна произведению её длины на её частоту,
        \item звуковая волна~--- поперечная волна,
        \item волна на поверхности озера~--- поперечная волна (или же поверхностная).
    \end{itemize}
    (4 из 8~--- это «–»)
}
\answer{%
    нет, да, да, нет, да, да, нет, да
}
\solutionspace{20pt}

\tasknumber{2}%
\task{%
    Определите расстояние между вторым и девятым гребнями волн,
    если длина волны равна $5\,\text{м}$.
    Сколько между ними ещё уместилось гребней?
}
\answer{%
    $
        l = (n_2 - n_1) \cdot \lambda = \cbr{9 - 2} \cdot 5\,\text{м} = 35\,\text{м},
        \quad n = n_2 - n_1 - 1 = 9 - 2 - 1 = 6
    $
}
\solutionspace{120pt}

\tasknumber{3}%
\task{%
    Определите скорость звука в среде, если источник звука,
    колеблющийся с периодом $4\,\text{мc}$, возбуждает волны длиной
    $2{,}4\,\text{м}$.
}
\answer{%
    $\lambda = vT \implies v = \frac{\lambda}T = \frac{ 2{,}4\,\text{м} }{ 4\,\text{мc} } = { 600{,}0\,\frac{\text{м}}{\text{c}} }$
}
\solutionspace{120pt}

\tasknumber{4}%
\task{%
    Мимо неподвижного наблюдателя прошло $5$ гребней волн за $5\,\text{c}$,
    начиная с первого.
    Каковы длина, период и частота волны,
    если скорость распространения волн $5\,\frac{\text{м}}{\text{с}}$?
}
\answer{%
    \begin{align*}
    \lambda &= \frac L{N-1} = \frac {vt}{N-1} = \frac {5\,\frac{\text{м}}{\text{с}} \cdot 5\,\text{c}}{5 - 1} = 6{,}25\,\text{м},  \\
    T &= \frac {\lambda}{v} = \frac {vt}{\cbr{N-1}v} = \frac {t}{N-1} =  \frac { 5\,\text{c} }{5 - 1} = 1{,}25\,\text{с},  \\
    \nu &= \frac 1T = \frac {N-1}{t} = \frac {5 - 1}{ 5\,\text{c} } = 0{,}80\,\text{Гц}.
    \\
    &\text{Если же считать гребни целиком, т.е.
    не вычитать единицу:}  \\
    \lambda' &= \frac L{N} = \frac {vt}{N} = \frac {5\,\frac{\text{м}}{\text{с}} \cdot 5\,\text{c}}{ 5 } = 5{,}00\,\text{м},  \\
    T' &= \frac {\lambda'}{v} = \frac {vt}{Nv} = \frac tN =  \frac { 5\,\text{c} }{ 5 } = 1{,}00\,\text{с},  \\
    \nu' &= \frac 1{T'} = \frac {N}{t} = \frac { 5 }{ 5\,\text{c} } = 1{,}00\,\text{Гц}.
    \end{align*}
}

\variantsplitter

\addpersonalvariant{Андрей Щербаков}

\tasknumber{1}%
\task{%
    Укажите, верны ли утверждения:
    \begin{itemize}
        \item механические волны переносят вещество,
        \item механические волны переносят энергию,
        \item источником механических волны служат колеблющиеся тела,
        \item продольные волны могут распространяться только в твёрдых телах,
        \item в твёрдых телах могут распространяться поперечные волны,
        \item скорость распространения волны равна произведению её длины на её частоту,
        \item звуковая волна~--- поперечная волна,
        \item волна на поверхности озера~--- поперечная волна (или же поверхностная).
    \end{itemize}
    (4 из 8~--- это «–»)
}
\answer{%
    нет, да, да, нет, да, да, нет, да
}
\solutionspace{20pt}

\tasknumber{2}%
\task{%
    Определите расстояние между вторым и седьмым гребнями волн,
    если длина волны равна $4\,\text{м}$.
    Сколько между ними ещё уместилось гребней?
}
\answer{%
    $
        l = (n_2 - n_1) \cdot \lambda = \cbr{7 - 2} \cdot 4\,\text{м} = 20\,\text{м},
        \quad n = n_2 - n_1 - 1 = 7 - 2 - 1 = 4
    $
}
\solutionspace{120pt}

\tasknumber{3}%
\task{%
    Определите скорость звука в среде, если источник звука,
    колеблющийся с периодом $3\,\text{мc}$, возбуждает волны длиной
    $1{,}5\,\text{м}$.
}
\answer{%
    $\lambda = vT \implies v = \frac{\lambda}T = \frac{ 1{,}5\,\text{м} }{ 3\,\text{мc} } = { 500{,}0\,\frac{\text{м}}{\text{c}} }$
}
\solutionspace{120pt}

\tasknumber{4}%
\task{%
    Мимо неподвижного наблюдателя прошло $4$ гребней волн за $6\,\text{c}$,
    начиная с первого.
    Каковы длина, период и частота волны,
    если скорость распространения волн $4\,\frac{\text{м}}{\text{с}}$?
}
\answer{%
    \begin{align*}
    \lambda &= \frac L{N-1} = \frac {vt}{N-1} = \frac {4\,\frac{\text{м}}{\text{с}} \cdot 6\,\text{c}}{4 - 1} = 8{,}00\,\text{м},  \\
    T &= \frac {\lambda}{v} = \frac {vt}{\cbr{N-1}v} = \frac {t}{N-1} =  \frac { 6\,\text{c} }{4 - 1} = 2{,}00\,\text{с},  \\
    \nu &= \frac 1T = \frac {N-1}{t} = \frac {4 - 1}{ 6\,\text{c} } = 0{,}50\,\text{Гц}.
    \\
    &\text{Если же считать гребни целиком, т.е.
    не вычитать единицу:}  \\
    \lambda' &= \frac L{N} = \frac {vt}{N} = \frac {4\,\frac{\text{м}}{\text{с}} \cdot 6\,\text{c}}{ 4 } = 6{,}00\,\text{м},  \\
    T' &= \frac {\lambda'}{v} = \frac {vt}{Nv} = \frac tN =  \frac { 6\,\text{c} }{ 4 } = 1{,}50\,\text{с},  \\
    \nu' &= \frac 1{T'} = \frac {N}{t} = \frac { 4 }{ 6\,\text{c} } = 0{,}67\,\text{Гц}.
    \end{align*}
}

\variantsplitter

\addpersonalvariant{Михаил Ярошевский}

\tasknumber{1}%
\task{%
    Укажите, верны ли утверждения:
    \begin{itemize}
        \item механические волны переносят вещество,
        \item механические волны переносят энергию,
        \item источником механических волны служат колеблющиеся тела,
        \item продольные волны могут распространяться только в твёрдых телах,
        \item в твёрдых телах могут распространяться поперечные волны,
        \item скорость распространения волны равна произведению её длины на её частоту,
        \item звуковая волна~--- поперечная волна,
        \item волна на поверхности озера~--- поперечная волна (или же поверхностная).
    \end{itemize}
    (4 из 8~--- это «–»)
}
\answer{%
    нет, да, да, нет, да, да, нет, да
}
\solutionspace{20pt}

\tasknumber{2}%
\task{%
    Определите расстояние между вторым и восьмым гребнями волн,
    если длина волны равна $4\,\text{м}$.
    Сколько между ними ещё уместилось гребней?
}
\answer{%
    $
        l = (n_2 - n_1) \cdot \lambda = \cbr{8 - 2} \cdot 4\,\text{м} = 24\,\text{м},
        \quad n = n_2 - n_1 - 1 = 8 - 2 - 1 = 5
    $
}
\solutionspace{120pt}

\tasknumber{3}%
\task{%
    Определите скорость звука в среде, если источник звука,
    колеблющийся с периодом $4\,\text{мc}$, возбуждает волны длиной
    $1{,}2\,\text{м}$.
}
\answer{%
    $\lambda = vT \implies v = \frac{\lambda}T = \frac{ 1{,}2\,\text{м} }{ 4\,\text{мc} } = { 300{,}0\,\frac{\text{м}}{\text{c}} }$
}
\solutionspace{120pt}

\tasknumber{4}%
\task{%
    Мимо неподвижного наблюдателя прошло $5$ гребней волн за $6\,\text{c}$,
    начиная с первого.
    Каковы длина, период и частота волны,
    если скорость распространения волн $1\,\frac{\text{м}}{\text{с}}$?
}
\answer{%
    \begin{align*}
    \lambda &= \frac L{N-1} = \frac {vt}{N-1} = \frac {1\,\frac{\text{м}}{\text{с}} \cdot 6\,\text{c}}{5 - 1} = 1{,}50\,\text{м},  \\
    T &= \frac {\lambda}{v} = \frac {vt}{\cbr{N-1}v} = \frac {t}{N-1} =  \frac { 6\,\text{c} }{5 - 1} = 1{,}50\,\text{с},  \\
    \nu &= \frac 1T = \frac {N-1}{t} = \frac {5 - 1}{ 6\,\text{c} } = 0{,}67\,\text{Гц}.
    \\
    &\text{Если же считать гребни целиком, т.е.
    не вычитать единицу:}  \\
    \lambda' &= \frac L{N} = \frac {vt}{N} = \frac {1\,\frac{\text{м}}{\text{с}} \cdot 6\,\text{c}}{ 5 } = 1{,}20\,\text{м},  \\
    T' &= \frac {\lambda'}{v} = \frac {vt}{Nv} = \frac tN =  \frac { 6\,\text{c} }{ 5 } = 1{,}20\,\text{с},  \\
    \nu' &= \frac 1{T'} = \frac {N}{t} = \frac { 5 }{ 6\,\text{c} } = 0{,}83\,\text{Гц}.
    \end{align*}
}
% autogenerated
