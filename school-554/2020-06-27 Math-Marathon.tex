\documentclass[12pt,a4paper]{amsart}%DVI-mode.
\usepackage{graphics,graphicx,epsfig}%DVI-mode.
% \documentclass[pdftex,12pt]{amsart} %PDF-mode.
% \usepackage[pdftex]{graphicx}       %PDF-mode.
% \usepackage[babel=true]{microtype}
% \usepackage[T1]{fontenc}
% \usepackage{lmodern}

\usepackage{cmap}
%\usepackage{a4wide}                 % Fit the text to A4 page tightly.
% \usepackage[utf8]{inputenc}
\usepackage[T2A]{fontenc}
\usepackage[english,russian]{babel} % Download Russian fonts.
\usepackage{amsmath,amsfonts,amssymb,amsthm,amscd,mathrsfs} % Use AMS symbols.
\usepackage{tikz}
\usetikzlibrary{circuits.ee.IEC}
\usetikzlibrary{shapes.geometric}
\usetikzlibrary{decorations.markings}
%\usetikzlibrary{dashs}
%\usetikzlibrary{info}


\textheight=28cm % высота текста
\textwidth=18cm % ширина текста
\topmargin=-2.5cm % отступ от верхнего края
\parskip=2pt % интервал между абзацами
\oddsidemargin=-1.5cm
\evensidemargin=-1.5cm 

\parindent=0pt % абзацный отступ
\tolerance=500 % терпимость к "жидким" строкам
\binoppenalty=10000 % штраф за перенос формул - 10000 - абсолютный запрет
\relpenalty=10000
\flushbottom % выравнивание высоты страниц
\pagenumbering{gobble}

\newcommand\bivec[2]{\begin{pmatrix} #1 \\ #2 \end{pmatrix}}

\newcommand\ol[1]{\overline{#1}}

\newcommand\p[1]{\Prob\!\left(#1\right)}
\newcommand\e[1]{\mathsf{E}\!\left(#1\right)}
\newcommand\disp[1]{\mathsf{D}\!\left(#1\right)}
%\newcommand\norm[2]{\mathcal{N}\!\cbr{#1,#2}}
\newcommand\sign{\text{ sign }}

\newcommand\al[1]{\begin{align*} #1 \end{align*}}
\newcommand\begcas[1]{\begin{cases}#1\end{cases}}
\newcommand\tab[2]{	\vspace{-#1pt}
						\begin{tabbing} 
						#2
						\end{tabbing}
					\vspace{-#1pt}
					}

\newcommand\maintext[1]{{\bfseries\sffamily{#1}}}
\newcommand\skipped[1]{ {\ensuremath{\text{\small{\sffamily{Пропущено:} #1} } } } }
\newcommand\simpletitle[1]{\begin{center} \maintext{#1} \end{center}}

\def\le{\leqslant}
\def\ge{\geqslant}
\def\Ell{\mathcal{L}}
\def\eps{{\varepsilon}}
\def\Rn{\mathbb{R}^n}
\def\RSS{\mathsf{RSS}}

\newcommand\foral[1]{\forall\,#1\:}
\newcommand\exist[1]{\exists\,#1\:\colon}

\newcommand\cbr[1]{\left(#1\right)} %circled brackets
\newcommand\fbr[1]{\left\{#1\right\}} %figure brackets
\newcommand\sbr[1]{\left[#1\right]} %square brackets
\newcommand\modul[1]{\left|#1\right|}

\newcommand\sqr[1]{\cbr{#1}^2}
\newcommand\inv[1]{\cbr{#1}^{-1}}

\newcommand\cdf[2]{\cdot\frac{#1}{#2}}
\newcommand\dd[2]{\frac{\partial#1}{\partial#2}}

\newcommand\integr[2]{\int\limits_{#1}^{#2}}
\newcommand\suml[2]{\sum\limits_{#1}^{#2}}
\newcommand\isum[2]{\sum\limits_{#1=#2}^{+\infty}}
\newcommand\idots[3]{#1_{#2},\ldots,#1_{#3}}
\newcommand\fdots[5]{#4{#1_{#2}}#5\ldots#5#4{#1_{#3}}}

\newcommand\obol[1]{O\!\cbr{#1}}
\newcommand\omal[1]{o\!\cbr{#1}}

\newcommand\addeps[2]{
	\begin{figure} [!ht] %lrp
		\centering
		\includegraphics[height=320px]{#1.eps}
		\vspace{-10pt}
		\caption{#2}
		\label{eps:#1}
	\end{figure}
}

\newcommand\addepssize[3]{
	\begin{figure} [!ht] %lrp hp
		\centering
		\includegraphics[height=#3px]{#1.eps}
		\vspace{-10pt}
		\caption{#2}
		\label{eps:#1}
	\end{figure}
}


\newcommand\norm[1]{\ensuremath{\left\|{#1}\right\|}}
\newcommand\ort{\bot}
\newcommand\theorem[1]{{\sffamily Теорема #1\ }}
\newcommand\lemma[1]{{\sffamily Лемма #1\ }}
\newcommand\difflim[2]{\frac{#1\cbr{#2 + \Delta#2} - #1\cbr{#2}}{\Delta #2}}
\renewcommand\proof[1]{\par\noindent$\square$ #1 \hfill$\blacksquare$\par}
\newcommand\defenition[1]{{\sffamilyОпределение #1\ }}

% \begin{document}
% %\raggedright
% \addclassdate{7}{20 апреля 2018}

\task 1
Площадь большого поршня гидравлического домкрата $S_1 = 20\units{см}^2$, а малого $S_2 = 0{,}5\units{см}^2.$ Груз какой максимальной массы можно поднять этим домкратом, если на малый поршень давить с силой не более $F=200\units{Н}?$ Силой трения от стенки цилиндров пренебречь.

\task 2
В сосуд налита вода. Расстояние от поверхности воды до дна $H = 0{,}5\units{м},$ площадь дна $S = 0{,}1\units{м}^2.$ Найти гидростатическое давление $P_1$ и полное давление $P_2$ вблизи дна. Найти силу давления воды на дно. Плотность воды \rhowater

\task 3
На лёгкий поршень площадью $S=900\units{см}^2,$ касающийся поверхности воды, поставили гирю массы $m=3\units{кг}$. Высота слоя воды в сосуде с вертикальными стенками $H = 20\units{см}$. Определить давление жидкости вблизи дна, если плотность воды \rhowater

\task 4
Давление газов в конце сгорания в цилиндре дизельного двигателя трактора $P = 9\units{МПа}.$ Диаметр цилиндра $d = 130\units{мм}.$ С какой силой газы давят на поршень в цилиндре? Площадь круга диаметром $D$ равна $S = \cfrac{\pi D^2}4.$

\task 5
Площадь малого поршня гидравлического подъёмника $S_1 = 0{,}8\units{см}^2$, а большого $S_2 = 40\units{см}^2.$ Какую силу $F$ надо приложить к малому поршню, чтобы поднять груз весом $P = 8\units{кН}?$

\task 6
Герметичный сосуд полностью заполнен водой и стоит на столе. На небольшой поршень площадью $S$ давят рукой с силой $F$. Поршень находится ниже крышки сосуда на $H_1$, выше дна на $H_2$ и может свободно перемещаться. Плотность воды $\rho$, атмосферное давление $P_A$. Найти давления $P_1$ и $P_2$ в воде вблизи крышки и дна сосуда.
\\ \\
\addclassdate{7}{20 апреля 2018}

\task 1
Площадь большого поршня гидравлического домкрата $S_1 = 20\units{см}^2$, а малого $S_2 = 0{,}5\units{см}^2.$ Груз какой максимальной массы можно поднять этим домкратом, если на малый поршень давить с силой не более $F=200\units{Н}?$ Силой трения от стенки цилиндров пренебречь.

\task 2
В сосуд налита вода. Расстояние от поверхности воды до дна $H = 0{,}5\units{м},$ площадь дна $S = 0{,}1\units{м}^2.$ Найти гидростатическое давление $P_1$ и полное давление $P_2$ вблизи дна. Найти силу давления воды на дно. Плотность воды \rhowater

\task 3
На лёгкий поршень площадью $S=900\units{см}^2,$ касающийся поверхности воды, поставили гирю массы $m=3\units{кг}$. Высота слоя воды в сосуде с вертикальными стенками $H = 20\units{см}$. Определить давление жидкости вблизи дна, если плотность воды \rhowater

\task 4
Давление газов в конце сгорания в цилиндре дизельного двигателя трактора $P = 9\units{МПа}.$ Диаметр цилиндра $d = 130\units{мм}.$ С какой силой газы давят на поршень в цилиндре? Площадь круга диаметром $D$ равна $S = \cfrac{\pi D^2}4.$

\task 5
Площадь малого поршня гидравлического подъёмника $S_1 = 0{,}8\units{см}^2$, а большого $S_2 = 40\units{см}^2.$ Какую силу $F$ надо приложить к малому поршню, чтобы поднять груз весом $P = 8\units{кН}?$

\task 6
Герметичный сосуд полностью заполнен водой и стоит на столе. На небольшой поршень площадью $S$ давят рукой с силой $F$. Поршень находится ниже крышки сосуда на $H_1$, выше дна на $H_2$ и может свободно перемещаться. Плотность воды $\rho$, атмосферное давление $P_A$. Найти давления $P_1$ и $P_2$ в воде вблизи крышки и дна сосуда.

\newpage

\adddate{8 класс. 20 апреля 2018}

\task 1
Между точками $A$ и $B$ электрической цепи подключены последовательно резисторы $R_1 = 10\units{Ом}$ и $R_2 = 20\units{Ом}$ и параллельно им $R_3 = 30\units{Ом}.$ Найдите эквивалентное сопротивление $R_{AB}$ этого участка цепи.

\task 2
Электрическая цепь состоит из последовательности $N$ одинаковых звеньев, в которых каждый резистор имеет сопротивление $r$. Последнее звено замкнуто резистором сопротивлением $R$. При каком соотношении $\cfrac{R}{r}$ сопротивление цепи не зависит от числа звеньев?

\task 3
Для измерения сопротивления $R$ проводника собрана электрическая цепь. Вольтметр $V$ показывает напряжение $U_V = 5\units{В},$ показание амперметра $A$ равно $I_A = 25\units{мА}.$ Найдите величину $R$ сопротивления проводника. Внутреннее сопротивление вольтметра $R_V = 1{,}0\units{кОм},$ внутреннее сопротивление амперметра $R_A = 2{,}0\units{Ом}.$

\task 4
Шкала гальванометра имеет $N=100$ делений, цена деления $\delta = 1\units{мкА}$. Внутреннее сопротивление гальванометра $R_G = 1{,}0\units{кОм}.$ Как из этого прибора сделать вольтметр для измерения напряжений до $U = 100\units{В}$ или амперметр для измерения токов силой до $I = 1\units{А}?$

\\ \\ \\ \\ \\ \\ \\ \\
\adddate{8 класс. 20 апреля 2018}

\task 1
Между точками $A$ и $B$ электрической цепи подключены последовательно резисторы $R_1 = 10\units{Ом}$ и $R_2 = 20\units{Ом}$ и параллельно им $R_3 = 30\units{Ом}.$ Найдите эквивалентное сопротивление $R_{AB}$ этого участка цепи.

\task 2
Электрическая цепь состоит из последовательности $N$ одинаковых звеньев, в которых каждый резистор имеет сопротивление $r$. Последнее звено замкнуто резистором сопротивлением $R$. При каком соотношении $\cfrac{R}{r}$ сопротивление цепи не зависит от числа звеньев?

\task 3
Для измерения сопротивления $R$ проводника собрана электрическая цепь. Вольтметр $V$ показывает напряжение $U_V = 5\units{В},$ показание амперметра $A$ равно $I_A = 25\units{мА}.$ Найдите величину $R$ сопротивления проводника. Внутреннее сопротивление вольтметра $R_V = 1{,}0\units{кОм},$ внутреннее сопротивление амперметра $R_A = 2{,}0\units{Ом}.$

\task 4
Шкала гальванометра имеет $N=100$ делений, цена деления $\delta = 1\units{мкА}$. Внутреннее сопротивление гальванометра $R_G = 1{,}0\units{кОм}.$ Как из этого прибора сделать вольтметр для измерения напряжений до $U = 100\units{В}$ или амперметр для измерения токов силой до $I = 1\units{А}?$


% % \begin{flushright}
\textsc{ГБОУ школа №554, 20 ноября 2018\,г.}
\end{flushright}

\begin{center}
\LARGE \textsc{Математический бой, 8 класс}
\end{center}

\problem{1} Есть тридцать карточек, на каждой написано по одному числу: на десяти карточках~–~$a$,  на десяти других~–~$b$ и на десяти оставшихся~–~$c$ (числа  различны). Известно, что к любым пяти карточкам можно подобрать ещё пять так, что сумма чисел на этих десяти карточках будет равна нулю. Докажите, что~одно из~чисел~$a, b, c$ равно нулю.

\problem{2} Вокруг стола стола пустили пакет с орешками. Первый взял один орешек, второй — 2, третий — 3 и так далее: каждый следующий брал на 1 орешек больше. Известно, что на втором круге было взято в сумме на 100 орешков больше, чем на первом. Сколько человек сидело за столом?

% \problem{2} Натуральное число разрешено увеличить на любое целое число процентов от 1 до 100, если при этом получаем натуральное число. Найдите наименьшее натуральное число, которое нельзя при помощи таких операций получить из~числа 1.

% \problem{3} Найти сумму $1^2 - 2^2 + 3^2 - 4^2 + 5^2 + \ldots - 2018^2$.

\problem{3} В кружке рукоделия, где занимается Валя, более 93\% участников~—~девочки. Какое наименьшее число участников может быть в таком кружке?

\problem{4} Произведение 2018 целых чисел равно 1. Может ли их сумма оказаться равной~0?

% \problem{4} Можно ли все натуральные числа от~1 до~9 записать в~клетки таблицы~$3\times3$ так, чтобы сумма в~любых двух соседних (по~вертикали или горизонтали) клетках равнялось простому числу?

\problem{5} На доске написано 2018 нулей и 2019 единиц. Женя стирает 2 числа и, если они были одинаковы, дописывает к оставшимся один ноль, а~если разные — единицу. Потом Женя повторяет эту операцию снова, потом ещё и~так далее. В~результате на~доске останется только одно число. Что это за~число?

\problem{6} Докажите, что в~любой компании людей найдутся 2~человека, имеющие равное число знакомых в этой компании (если $A$~знаком с~$B$, то~и $B$~знаком с~$A$).

\problem{7} Три колокола начинают бить одновременно. Интервалы между ударами колоколов соответственно составляют $\cfrac43$~секунды, $\cfrac53$~секунды и $2$~секунды. Совпавшие по времени удары воспринимаются за~один. Сколько ударов будет услышано за 1~минуту, включая первый и последний удары?

\problem{8} Восемь одинаковых момент расположены по кругу. Известно, что три из~них~— фальшивые, и они расположены рядом друг с~другом. Вес фальшивой монеты отличается от~веса настоящей. Все фальшивые монеты весят одинаково, но неизвестно, тяжелее или легче фальшивая монета настоящей. Покажите, что за~3~взвешивания на~чашечных весах без~гирь можно определить все фальшивые монеты.

% \end{document}

\narrow

\renewcommand{\theenumi}{\asbuk{enumi}}
\renewcommand{\labelenumi}{\asbuk{enumi})}

\begin{document}
\begin{flushright}
\textsc{Летняя научная школа \\ ГБОУ школа №554 \\ 27~июня 2020\,г.}
\end{flushright}

\begin{center}
\LARGE \textsc{Математический марафон}
\LARGE \textsc{8–11~классы}
\end{center}

Правила участия: \href{https://www.notion.so/Math-Marathon-8acf3ff3b2874cefabbfa78d2db4f07e}{https://notion.so/Math-Marathon-8acf3ff3b2874cefabbfa78d2db4f07e}.

% Максимчик
\problem{1.1}%
Найдите все значения переметра~$a$, при~которых уравнение 
$$\sqrt{3ax + 5a} = 3x + 5$$
имеет только одно решение.
% \answer{$a=0$}

\problem{1.2}%
Найдите все значения переметра~$a$, при~которых уравнение
$$\cbr{ax^2 - \cbr{a^2+12}x + 12a}\sqrt{x+5} = 0$$
имеет только два решения.
% \answer{$a\in(-\infty; -5]\cup\{-2\sqrt 3; 2\sqrt 3\}\cup[-2{,}4; 0]$}

\problem{1.3}%
Найдите все значения переметра~$a$, при~которых уравнение 
$$\sqrt{x^2 + 8x} - x = \frac a2$$
имеет единственное решение.
% \answer{$a\in[0;8)\cup[16;+\infty)$}

\problem{1.4}%
Найдите все значения переметра~$a$, при~которых уравнение 
$$\sqrt{5x^2 + 6ax - 27a^2} = x + 3a$$
имеет только два решения.
% \answer{$a>0$}

\problem{1.5}%
Найдите все значения переметра~$a$, при~которых уравнение 
$$\sqrt{25-x^2} = x - a$$
имеет единственное решение.
% \answer{$a\in(-5;5]\cup\{-5\sqrt 2\}$}

\problem{1.6}%
Сколько корней имеет уравнение 
$$\sqrt{2-x} + 3 = ax^2$$
в~зависимости от значений параметра~$a$?
% \answer{$a\in(-\infty; 0] \implies 0; a\in(0, 0{,}75) \implies 1; a\in [0{,}75; +\infty) \implies 2$}

\problem{1.7}%
При каких значениях параметра~$a$, где 
$$a=\sqrt{\frac{x-3}{x+1}}\cdot\frac{1+(x-1)\sqrt{x^2-2x-3} - x^2}{x^2-(x+3)\sqrt{x^2 - 2x - 3} - 9},$$
абсолютная величина~$\abs{a-0{,}66}$ будет наименьшей для $x\in\mathbb{N}$?
% \answer{$a=\frac23$}

\problem{1.8}%
При каких значениях параметра~$a$ функция 
$$y=g(x) = \sqrt{\sqr{x-a} - (x-a) + 4}$$
является чётной?
% \answer{$a=-\frac 12$}

% Жичина
\problem{2.1}%
Из цистерны в~бассейн сначала перелили 50\% имеющейся в~цистерне воды, затем ещё 100 литров, затем ещё 5\% от остатка. 
При этом количество воды в~бассейне возросло на~31\%. 
Сколько литров воды было в~цистерне, если в~бассейне первоначально было 2000 литров воды?
% \answer{1000}

\problem{2.2}%
Брокерская фирма приобрела два пакета акций, а затем их продала на~общую сумму 7680 тысяч рублей. Получив при~этом 28\% прибыли. 
За какую сумму фирма приобрела каждый из~пакетов акций, если при~продаже первого пакета прибыль составила 40\%, а при~продаже второго -20\%? 
Ответ дайте в~тысячах рублей.
% \answer{2400 и 3600}

\problem{2.3}%
На факультете Х отличники составляют 10\% от общего количества студентов этого факультета, на~факультете~$Y$ — 20\%, а на~факультете~$Z$ — 4\%. 
Найдите средний процент отличников по всем трем факультетам, если известно, что на~факультете~$Y$ учится на~50\% больше студентов, 
чем на~факультете Х, а на~факультете~$Z$~— вдвое меньше, чем на~факультете Х.
% \answer{14}

\problem{2.4}%
Магазин закупил некоторое количество товара и начал его реализацию по цене на~25\% выше цены, назначенной производителем, 
чтобы покрыть затраты, связанные с~транспортировкой, и другие дополнительные расходы. 
Оставшуюся после реализации часть товара магазин уценил на~16\% с~тем, чтобы покрыть только затраты 
на закупку этой части товара у производителя и его транспортировку. 
Сколько процентов от цены, назначенной производителем, составляла стоимость транспортировки товара?
% \answer{5}

\problem{2.5}%
Банк ежегодно начисляет 10\% от суммы вклада. Через сколько лет вклад увеличится хотя бы вдвое?
% \answer{8}

\problem{2.6}%
15 июля планируется взять кредит на~сумму 900\,000~рублей. Условия его возврата таковы: 
\begin{itemize}
    \item 1-го числа каждого месяца долг возрастает на~2\% по~сравнению с~концом предыдущего месяца;
    \item со 2-го по 14-е число каждого месяца необходимо выплатить некоторую часть долга.
\end{itemize}
На какое минимальное количество месяцев можно взять кредит при~условии, что ежемесячные выплаты не должны превышать 180\,000~рублей? 
% \answer{6}

\problem{2.7}%
31 декабря 2014 года Пётр взял в~банке некоторую сумму в~кредит под~некоторый процент годовых. 
Схема выплаты кредита следующая: 31 декабря каждого следующего года банк начисляет проценты на~оставшуюся сумму долга 
(т.е. увеличивает долг на~определенное число процентов), затем Пётр переводит очередной транш. 
Если он будет платить каждый год по 2\,592\,000~рублей, то выплатит долг за~4~года, а если по 4\,392\,000~рублей, то за~2~года. 
Под какой процент Пётр взял деньги в~банке?
% \answer{20}

\problem{2.8}%
Вклад в~размере 10 млн рублей планируется открыть на~четыре года. В~конце каждого года вклад увеличивается 
на 10\% по сравнению с~его размером в~начале года, а кроме этого, в~начале третьего и четвёртого годов вклад 
ежегодно пополняется на~одну и ту же фиксированную сумму, равную целому числу миллионов рублей. 
Найдите наименьший возможный размер такой суммы, при~котором через четыре года 
вклад станет не меньше 30 млн рублей. Ответ дайте в~млн рублей.
% \answer{7}

\problem{2.9}%
31 декабря 2013 года Сергей взял в~банке 9\,930\,000~рублей в~кредит под~10\% годовых. 
Схема выплаты кредита следующая: 31 декабря каждого следующего года банк начисляет проценты на~оставшуюся сумму долга 
(то есть увеличивает долг на~10\%), затем Сергей переводит в~банк определённую сумму ежегодного платежа. 
Какой должна быть сумма ежегодного платежа, чтобы Сергей выплатил долг тремя равными ежегодными платежами? 
% \answer{3993000}

\problem{2.10}%
В~июле 2020 года планируется взять кредит в~банке на~сумму 300\,000~рублей. Условия его возврата таковы:
\begin{itemize}
    \item каждый январь долг увеличивается на~$r\%$ по сравнению с~концом предыдущего года;
    \item с~февраля по июнь каждого года необходимо выплатить одним платежом часть долга.
\end{itemize}
Найдите~$r$, если известно, что кредит будет полностью погашен за~два года, причём в~первый год будет выплачено 260\,000~рублей, а во~второй год — 169\,000~рублей
% \answer{30}

\problem{2.11}%
В~июле планируется взять кредит в~банке на~сумму 5~млн рублей на~некоторый срок (целое число лет). Условия его возврата таковы: 
\begin{itemize}
    \item каждый январь долг возрастает на~20\% по сравнению с~концом предыдущего года; 
    \item с~февраля по июнь каждого года необходимо выплатить часть долга; 
    \item в~июле каждого года долг должен быть на~одну и ту же сумму меньше долга на~июль предыдущего года. 
\end{itemize}
На сколько лет планируется взять кредит, если известно, что общая сумма выплат после его полного погашения составит 7{,}5 млн рублей?
% \answer{4}

\problem{2.12}%
Решите уравнение:
$$(3-2x)(2x+3) - (4-2x)(5+2x) = 4.$$
% \answer{7,5}

\problem{2.13}%
Решите уравнение:
$$\sqr{6x+1}(1-x) + \sqr{5-6x}(x+1) = 14.$$
% \answer{0,5}

\problem{2.14}%
Решите уравнение:
$$2x + 1 + \frac{2x-1}6 = \frac{7x-13}4.$$
% \answer{-7}

\problem{2.15}%
Решите уравнение:
$$(3x+2)(3x-2) - \sqr{3x-4} = 28.$$
% \answer{2}

\problem{2.16}%
Решите уравнение:
$$(2x-1)\cbr{1+2x+4x^2} - 4x\cbr{2x^2-3} = 23.$$
% \answer{2}

\problem{2.17}%
Найдите значение выражения при~$a=\cfrac{4}{11}, b=\cfrac{3}{5}$:
$$(8a+3b)(3a-8b) - (3a + 8b)(8a - 3b).$$
% \answer{-24}

\problem{2.18}%
Найдите значение выражения при~$a = -1, b = 0{,}25$:
$$\sqr{3a-b} - (9a+5b)(a-3b).$$
% \answer{-3}

% Шелаков
\problem{3.1}%
Из вершины прямого угла треугольника~$\triangle ABC$ проведена медиана~$CM$. Окружность, вписанная в~треугольник~$CAM$, касается~$CM$ в~её середине. Найдите угол~$\angle BAC$.
% \answer{$60\degrees$}

\problem{3.2}%
Прямая, перпендикулярная гипотенузе~$AB$ прямоугольного треугольника~$\triangle ABC$, пересекает прямые~$AC$ и $BC$ в~точках~$E$ и $D$ соответственно. Найдите угол между прямыми~$AD$ и $BE$.
% \answer{90}

\problem{3.3}%
В~треугольнике~$\triangle ABC$: $AC = 8, BC = 5$. Прямая, параллельная биссектрисе внешнего угла~$C$, проходит через середину стороны~$AB$ и точку~$E$ на~стороне~$AC$. Найдите~$AE$.
% \answer{1,5}

\problem{3.4}%
В~квадрате $ABCD$ со стороной 1 точка~$F$~— середина стороны BC, $E$~— основание перпендикуляра, опущенного из~вершины~$A$ на~$DF$. Найдите длину~$BE$.
% \answer{1}

\problem{3.5}%
Пусть~$CH$~— высота остроугольного треугольника~$\triangle ABC$, $O$~— центр описанной около него окружности. Точка~$T$~— проекция вершины~$C$ на~прямую~$AO$. В~каком отношении прямая~$TH$ делит сторону~$BC$?
% \answer{1:1}

\problem{3.6}%
Найдите среднюю линию равнобокой трапеции, если её диагональ равна 25, а высота равна 15.
% \answer{20}

\problem{3.7}%
В~прямоугольном треугольнике~$ABC$ $M$~— середина гипотенузы~$AB$, $N$~— середина катета AC. Окружность, проходящая через точки~$M$ и $N$, касается катета~$BC$ в~точке~$K$. В~каком отношении точка~$K$ делит этот катет?
% \answer{1:3}

\problem{3.8}%
В~выпуклом четырехугольнике~$ABCD$ угол~$\angle A = 60\degrees,$ угол~$\angle B = 150\degrees,$ угол~$\angle C = 45\degrees$ и $AB = BC$. Найдите угол~$\angle ABD$.
% \answer{$60\degrees$}

\end{document}