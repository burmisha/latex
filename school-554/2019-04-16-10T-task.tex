\documentclass[12pt,a4paper]{amsart}%DVI-mode.
\usepackage{graphics,graphicx,epsfig}%DVI-mode.
% \documentclass[pdftex,12pt]{amsart} %PDF-mode.
% \usepackage[pdftex]{graphicx}       %PDF-mode.
% \usepackage[babel=true]{microtype}
% \usepackage[T1]{fontenc}
% \usepackage{lmodern}

\usepackage{cmap}
%\usepackage{a4wide}                 % Fit the text to A4 page tightly.
% \usepackage[utf8]{inputenc}
\usepackage[T2A]{fontenc}
\usepackage[english,russian]{babel} % Download Russian fonts.
\usepackage{amsmath,amsfonts,amssymb,amsthm,amscd,mathrsfs} % Use AMS symbols.
\usepackage{tikz}
\usetikzlibrary{circuits.ee.IEC}
\usetikzlibrary{shapes.geometric}
\usetikzlibrary{decorations.markings}
%\usetikzlibrary{dashs}
%\usetikzlibrary{info}


\textheight=28cm % высота текста
\textwidth=18cm % ширина текста
\topmargin=-2.5cm % отступ от верхнего края
\parskip=2pt % интервал между абзацами
\oddsidemargin=-1.5cm
\evensidemargin=-1.5cm 

\parindent=0pt % абзацный отступ
\tolerance=500 % терпимость к "жидким" строкам
\binoppenalty=10000 % штраф за перенос формул - 10000 - абсолютный запрет
\relpenalty=10000
\flushbottom % выравнивание высоты страниц
\pagenumbering{gobble}

\newcommand\bivec[2]{\begin{pmatrix} #1 \\ #2 \end{pmatrix}}

\newcommand\ol[1]{\overline{#1}}

\newcommand\p[1]{\Prob\!\left(#1\right)}
\newcommand\e[1]{\mathsf{E}\!\left(#1\right)}
\newcommand\disp[1]{\mathsf{D}\!\left(#1\right)}
%\newcommand\norm[2]{\mathcal{N}\!\cbr{#1,#2}}
\newcommand\sign{\text{ sign }}

\newcommand\al[1]{\begin{align*} #1 \end{align*}}
\newcommand\begcas[1]{\begin{cases}#1\end{cases}}
\newcommand\tab[2]{	\vspace{-#1pt}
						\begin{tabbing} 
						#2
						\end{tabbing}
					\vspace{-#1pt}
					}

\newcommand\maintext[1]{{\bfseries\sffamily{#1}}}
\newcommand\skipped[1]{ {\ensuremath{\text{\small{\sffamily{Пропущено:} #1} } } } }
\newcommand\simpletitle[1]{\begin{center} \maintext{#1} \end{center}}

\def\le{\leqslant}
\def\ge{\geqslant}
\def\Ell{\mathcal{L}}
\def\eps{{\varepsilon}}
\def\Rn{\mathbb{R}^n}
\def\RSS{\mathsf{RSS}}

\newcommand\foral[1]{\forall\,#1\:}
\newcommand\exist[1]{\exists\,#1\:\colon}

\newcommand\cbr[1]{\left(#1\right)} %circled brackets
\newcommand\fbr[1]{\left\{#1\right\}} %figure brackets
\newcommand\sbr[1]{\left[#1\right]} %square brackets
\newcommand\modul[1]{\left|#1\right|}

\newcommand\sqr[1]{\cbr{#1}^2}
\newcommand\inv[1]{\cbr{#1}^{-1}}

\newcommand\cdf[2]{\cdot\frac{#1}{#2}}
\newcommand\dd[2]{\frac{\partial#1}{\partial#2}}

\newcommand\integr[2]{\int\limits_{#1}^{#2}}
\newcommand\suml[2]{\sum\limits_{#1}^{#2}}
\newcommand\isum[2]{\sum\limits_{#1=#2}^{+\infty}}
\newcommand\idots[3]{#1_{#2},\ldots,#1_{#3}}
\newcommand\fdots[5]{#4{#1_{#2}}#5\ldots#5#4{#1_{#3}}}

\newcommand\obol[1]{O\!\cbr{#1}}
\newcommand\omal[1]{o\!\cbr{#1}}

\newcommand\addeps[2]{
	\begin{figure} [!ht] %lrp
		\centering
		\includegraphics[height=320px]{#1.eps}
		\vspace{-10pt}
		\caption{#2}
		\label{eps:#1}
	\end{figure}
}

\newcommand\addepssize[3]{
	\begin{figure} [!ht] %lrp hp
		\centering
		\includegraphics[height=#3px]{#1.eps}
		\vspace{-10pt}
		\caption{#2}
		\label{eps:#1}
	\end{figure}
}


\newcommand\norm[1]{\ensuremath{\left\|{#1}\right\|}}
\newcommand\ort{\bot}
\newcommand\theorem[1]{{\sffamily Теорема #1\ }}
\newcommand\lemma[1]{{\sffamily Лемма #1\ }}
\newcommand\difflim[2]{\frac{#1\cbr{#2 + \Delta#2} - #1\cbr{#2}}{\Delta #2}}
\renewcommand\proof[1]{\par\noindent$\square$ #1 \hfill$\blacksquare$\par}
\newcommand\defenition[1]{{\sffamilyОпределение #1\ }}

% \begin{document}
% %\raggedright
% \addclassdate{7}{20 апреля 2018}

\task 1
Площадь большого поршня гидравлического домкрата $S_1 = 20\units{см}^2$, а малого $S_2 = 0{,}5\units{см}^2.$ Груз какой максимальной массы можно поднять этим домкратом, если на малый поршень давить с силой не более $F=200\units{Н}?$ Силой трения от стенки цилиндров пренебречь.

\task 2
В сосуд налита вода. Расстояние от поверхности воды до дна $H = 0{,}5\units{м},$ площадь дна $S = 0{,}1\units{м}^2.$ Найти гидростатическое давление $P_1$ и полное давление $P_2$ вблизи дна. Найти силу давления воды на дно. Плотность воды \rhowater

\task 3
На лёгкий поршень площадью $S=900\units{см}^2,$ касающийся поверхности воды, поставили гирю массы $m=3\units{кг}$. Высота слоя воды в сосуде с вертикальными стенками $H = 20\units{см}$. Определить давление жидкости вблизи дна, если плотность воды \rhowater

\task 4
Давление газов в конце сгорания в цилиндре дизельного двигателя трактора $P = 9\units{МПа}.$ Диаметр цилиндра $d = 130\units{мм}.$ С какой силой газы давят на поршень в цилиндре? Площадь круга диаметром $D$ равна $S = \cfrac{\pi D^2}4.$

\task 5
Площадь малого поршня гидравлического подъёмника $S_1 = 0{,}8\units{см}^2$, а большого $S_2 = 40\units{см}^2.$ Какую силу $F$ надо приложить к малому поршню, чтобы поднять груз весом $P = 8\units{кН}?$

\task 6
Герметичный сосуд полностью заполнен водой и стоит на столе. На небольшой поршень площадью $S$ давят рукой с силой $F$. Поршень находится ниже крышки сосуда на $H_1$, выше дна на $H_2$ и может свободно перемещаться. Плотность воды $\rho$, атмосферное давление $P_A$. Найти давления $P_1$ и $P_2$ в воде вблизи крышки и дна сосуда.
\\ \\
\addclassdate{7}{20 апреля 2018}

\task 1
Площадь большого поршня гидравлического домкрата $S_1 = 20\units{см}^2$, а малого $S_2 = 0{,}5\units{см}^2.$ Груз какой максимальной массы можно поднять этим домкратом, если на малый поршень давить с силой не более $F=200\units{Н}?$ Силой трения от стенки цилиндров пренебречь.

\task 2
В сосуд налита вода. Расстояние от поверхности воды до дна $H = 0{,}5\units{м},$ площадь дна $S = 0{,}1\units{м}^2.$ Найти гидростатическое давление $P_1$ и полное давление $P_2$ вблизи дна. Найти силу давления воды на дно. Плотность воды \rhowater

\task 3
На лёгкий поршень площадью $S=900\units{см}^2,$ касающийся поверхности воды, поставили гирю массы $m=3\units{кг}$. Высота слоя воды в сосуде с вертикальными стенками $H = 20\units{см}$. Определить давление жидкости вблизи дна, если плотность воды \rhowater

\task 4
Давление газов в конце сгорания в цилиндре дизельного двигателя трактора $P = 9\units{МПа}.$ Диаметр цилиндра $d = 130\units{мм}.$ С какой силой газы давят на поршень в цилиндре? Площадь круга диаметром $D$ равна $S = \cfrac{\pi D^2}4.$

\task 5
Площадь малого поршня гидравлического подъёмника $S_1 = 0{,}8\units{см}^2$, а большого $S_2 = 40\units{см}^2.$ Какую силу $F$ надо приложить к малому поршню, чтобы поднять груз весом $P = 8\units{кН}?$

\task 6
Герметичный сосуд полностью заполнен водой и стоит на столе. На небольшой поршень площадью $S$ давят рукой с силой $F$. Поршень находится ниже крышки сосуда на $H_1$, выше дна на $H_2$ и может свободно перемещаться. Плотность воды $\rho$, атмосферное давление $P_A$. Найти давления $P_1$ и $P_2$ в воде вблизи крышки и дна сосуда.

\newpage

\adddate{8 класс. 20 апреля 2018}

\task 1
Между точками $A$ и $B$ электрической цепи подключены последовательно резисторы $R_1 = 10\units{Ом}$ и $R_2 = 20\units{Ом}$ и параллельно им $R_3 = 30\units{Ом}.$ Найдите эквивалентное сопротивление $R_{AB}$ этого участка цепи.

\task 2
Электрическая цепь состоит из последовательности $N$ одинаковых звеньев, в которых каждый резистор имеет сопротивление $r$. Последнее звено замкнуто резистором сопротивлением $R$. При каком соотношении $\cfrac{R}{r}$ сопротивление цепи не зависит от числа звеньев?

\task 3
Для измерения сопротивления $R$ проводника собрана электрическая цепь. Вольтметр $V$ показывает напряжение $U_V = 5\units{В},$ показание амперметра $A$ равно $I_A = 25\units{мА}.$ Найдите величину $R$ сопротивления проводника. Внутреннее сопротивление вольтметра $R_V = 1{,}0\units{кОм},$ внутреннее сопротивление амперметра $R_A = 2{,}0\units{Ом}.$

\task 4
Шкала гальванометра имеет $N=100$ делений, цена деления $\delta = 1\units{мкА}$. Внутреннее сопротивление гальванометра $R_G = 1{,}0\units{кОм}.$ Как из этого прибора сделать вольтметр для измерения напряжений до $U = 100\units{В}$ или амперметр для измерения токов силой до $I = 1\units{А}?$

\\ \\ \\ \\ \\ \\ \\ \\
\adddate{8 класс. 20 апреля 2018}

\task 1
Между точками $A$ и $B$ электрической цепи подключены последовательно резисторы $R_1 = 10\units{Ом}$ и $R_2 = 20\units{Ом}$ и параллельно им $R_3 = 30\units{Ом}.$ Найдите эквивалентное сопротивление $R_{AB}$ этого участка цепи.

\task 2
Электрическая цепь состоит из последовательности $N$ одинаковых звеньев, в которых каждый резистор имеет сопротивление $r$. Последнее звено замкнуто резистором сопротивлением $R$. При каком соотношении $\cfrac{R}{r}$ сопротивление цепи не зависит от числа звеньев?

\task 3
Для измерения сопротивления $R$ проводника собрана электрическая цепь. Вольтметр $V$ показывает напряжение $U_V = 5\units{В},$ показание амперметра $A$ равно $I_A = 25\units{мА}.$ Найдите величину $R$ сопротивления проводника. Внутреннее сопротивление вольтметра $R_V = 1{,}0\units{кОм},$ внутреннее сопротивление амперметра $R_A = 2{,}0\units{Ом}.$

\task 4
Шкала гальванометра имеет $N=100$ делений, цена деления $\delta = 1\units{мкА}$. Внутреннее сопротивление гальванометра $R_G = 1{,}0\units{кОм}.$ Как из этого прибора сделать вольтметр для измерения напряжений до $U = 100\units{В}$ или амперметр для измерения токов силой до $I = 1\units{А}?$


% % \begin{flushright}
\textsc{ГБОУ школа №554, 20 ноября 2018\,г.}
\end{flushright}

\begin{center}
\LARGE \textsc{Математический бой, 8 класс}
\end{center}

\problem{1} Есть тридцать карточек, на каждой написано по одному числу: на десяти карточках~–~$a$,  на десяти других~–~$b$ и на десяти оставшихся~–~$c$ (числа  различны). Известно, что к любым пяти карточкам можно подобрать ещё пять так, что сумма чисел на этих десяти карточках будет равна нулю. Докажите, что~одно из~чисел~$a, b, c$ равно нулю.

\problem{2} Вокруг стола стола пустили пакет с орешками. Первый взял один орешек, второй — 2, третий — 3 и так далее: каждый следующий брал на 1 орешек больше. Известно, что на втором круге было взято в сумме на 100 орешков больше, чем на первом. Сколько человек сидело за столом?

% \problem{2} Натуральное число разрешено увеличить на любое целое число процентов от 1 до 100, если при этом получаем натуральное число. Найдите наименьшее натуральное число, которое нельзя при помощи таких операций получить из~числа 1.

% \problem{3} Найти сумму $1^2 - 2^2 + 3^2 - 4^2 + 5^2 + \ldots - 2018^2$.

\problem{3} В кружке рукоделия, где занимается Валя, более 93\% участников~—~девочки. Какое наименьшее число участников может быть в таком кружке?

\problem{4} Произведение 2018 целых чисел равно 1. Может ли их сумма оказаться равной~0?

% \problem{4} Можно ли все натуральные числа от~1 до~9 записать в~клетки таблицы~$3\times3$ так, чтобы сумма в~любых двух соседних (по~вертикали или горизонтали) клетках равнялось простому числу?

\problem{5} На доске написано 2018 нулей и 2019 единиц. Женя стирает 2 числа и, если они были одинаковы, дописывает к оставшимся один ноль, а~если разные — единицу. Потом Женя повторяет эту операцию снова, потом ещё и~так далее. В~результате на~доске останется только одно число. Что это за~число?

\problem{6} Докажите, что в~любой компании людей найдутся 2~человека, имеющие равное число знакомых в этой компании (если $A$~знаком с~$B$, то~и $B$~знаком с~$A$).

\problem{7} Три колокола начинают бить одновременно. Интервалы между ударами колоколов соответственно составляют $\cfrac43$~секунды, $\cfrac53$~секунды и $2$~секунды. Совпавшие по времени удары воспринимаются за~один. Сколько ударов будет услышано за 1~минуту, включая первый и последний удары?

\problem{8} Восемь одинаковых момент расположены по кругу. Известно, что три из~них~— фальшивые, и они расположены рядом друг с~другом. Вес фальшивой монеты отличается от~веса настоящей. Все фальшивые монеты весят одинаково, но неизвестно, тяжелее или легче фальшивая монета настоящей. Покажите, что за~3~взвешивания на~чашечных весах без~гирь можно определить все фальшивые монеты.

% \end{document}

\begin{document}
\noanswers

\setdate{16~апреля~2019}
\setclass{10«Т»}

\addpersonalvariant{Михаил Бурмистров}


\tasknumber{1}\task{
    С какой силой взаимодействуют 2 точечных заряда $q_1=3\units{нКл}$ и $q_2=2\units{нКл}$,
    находящиеся на расстоянии $r=2\units{см}$?
}
\answer{
    $
        F
            = k\frac{q_1q_2}{r^2}
            = 9 \cdot 10^9 \funits{Н $\cdot$ м$^2$}{Кл$^2$} \cdot \frac{
                3\cdot 10^{-9}\units{Кл}
                \cdot
                2\cdot 10^{-9}\units{Кл}
            }{
                \left(2 \cdot 10^{-2}\units{м}\right)^2
            }
            = \frac{27}{2}\cdot10^{-5}\units{Н}
              \approx {13.50}\cdot10^{-5}\units{Н}
    $
}
\vspace{120pt}

\tasknumber{2}\task{
    Два одинаковых маленьких проводящих заряженных шарика находятся
    на расстоянии~$d$ друг от друга.
    Заряд первого равен~$+4q$, второго~---$-3q$.
    Шарики приводят в соприкосновение, а после опять разводят на то же самое расстояние~$d$.
    Каким стал заряд каждого из шариков?
    Определите характер (притяжение или отталкивание)
    и силу взаимодействия шариков до и после соприкосновения.
}
\answer{
    \begin{align*}
        F   &= k\frac{q_1q_2}{d^2} = k\frac{(+4q)\cdot(-3q)}{d^2},
        \text{отталкивание};
        \\
        q'_1 = q'_2 = \frac{q_1 + q_2}2 = \frac{(+4q) + (-3q)}2 \implies
        F'  &= k\frac{q'_1q'_2}{d^2}
            = k\frac{
                    \left(\frac{(+4q) + (-3q)}2\right)^2
                }{
                    d^2
                },
        \text{отталкивание}.
    \end{align*}
}
\vspace{120pt}

\tasknumber{3}\task{
    На координатной плоскости в точках $(-r; 0)$ и $(r; 0)$
    находятся заряды, соответственно, $+q$ и $-q$.
    Сделайте рисунок, определите величину напряжённости электрического поля
    в точках $(0; -r)$ и $(-2r; 0)$ и укажите её направление.
}
\vspace{120pt}

\tasknumber{4}\task{
    Заряд $q_1$ создает в точке $A$ электрическое поле
    по величине равное~$E_1=200\funits{В}{м}$,
    а $q_2$~---$E_2=200\funits{В}{м}$.
    Угол между векторами $\vect{E_1}$ и $\vect{E_2}$ равен $\alpha$.
    Определите величину суммарного электрического поля в точке $A$,
    создаваемого обоими зарядами $q_1$ и $q_2$.
    Сделайте рисунок и вычислите её значение для двух значений угла $\alpha$:
    $\alpha_1=0^\circ$ и $\alpha_2=60^\circ$.
}
\newpage

\addpersonalvariant{Гагик Аракелян}


\tasknumber{1}\task{
    С какой силой взаимодействуют 2 точечных заряда $q_1=2\units{нКл}$ и $q_2=4\units{нКл}$,
    находящиеся на расстоянии $r=3\units{см}$?
}
\answer{
    $
        F
            = k\frac{q_1q_2}{r^2}
            = 9 \cdot 10^9 \funits{Н $\cdot$ м$^2$}{Кл$^2$} \cdot \frac{
                2\cdot 10^{-9}\units{Кл}
                \cdot
                4\cdot 10^{-9}\units{Кл}
            }{
                \left(3 \cdot 10^{-2}\units{м}\right)^2
            }
            = \frac{8}{1}\cdot10^{-5}\units{Н}
              \approx {8.00}\cdot10^{-5}\units{Н}
    $
}
\vspace{120pt}

\tasknumber{2}\task{
    Два одинаковых маленьких проводящих заряженных шарика находятся
    на расстоянии~$l$ друг от друга.
    Заряд первого равен~$-3q$, второго~---$+5q$.
    Шарики приводят в соприкосновение, а после опять разводят на то же самое расстояние~$l$.
    Каким стал заряд каждого из шариков?
    Определите характер (притяжение или отталкивание)
    и силу взаимодействия шариков до и после соприкосновения.
}
\answer{
    \begin{align*}
        F   &= k\frac{q_1q_2}{l^2} = k\frac{(-3q)\cdot(+5q)}{l^2},
        \text{отталкивание};
        \\
        q'_1 = q'_2 = \frac{q_1 + q_2}2 = \frac{(-3q) + (+5q)}2 \implies
        F'  &= k\frac{q'_1q'_2}{l^2}
            = k\frac{
                    \left(\frac{(-3q) + (+5q)}2\right)^2
                }{
                    l^2
                },
        \text{отталкивание}.
    \end{align*}
}
\vspace{120pt}

\tasknumber{3}\task{
    На координатной плоскости в точках $(-l; 0)$ и $(l; 0)$
    находятся заряды, соответственно, $-Q$ и $+Q$.
    Сделайте рисунок, определите величину напряжённости электрического поля
    в точках $(0; l)$ и $(-2l; 0)$ и укажите её направление.
}
\vspace{120pt}

\tasknumber{4}\task{
    Заряд $q_1$ создает в точке $A$ электрическое поле
    по величине равное~$E_1=24\funits{В}{м}$,
    а $q_2$~---$E_2=7\funits{В}{м}$.
    Угол между векторами $\vect{E_1}$ и $\vect{E_2}$ равен $\varphi$.
    Определите величину суммарного электрического поля в точке $A$,
    создаваемого обоими зарядами $q_1$ и $q_2$.
    Сделайте рисунок и вычислите её значение для двух значений угла $\varphi$:
    $\varphi_1=90^\circ$ и $\varphi_2=180^\circ$.
}
\newpage

\addpersonalvariant{Ирен Аракелян}


\tasknumber{1}\task{
    С какой силой взаимодействуют 2 точечных заряда $q_1=2\units{нКл}$ и $q_2=4\units{нКл}$,
    находящиеся на расстоянии $d=2\units{см}$?
}
\answer{
    $
        F
            = k\frac{q_1q_2}{d^2}
            = 9 \cdot 10^9 \funits{Н $\cdot$ м$^2$}{Кл$^2$} \cdot \frac{
                2\cdot 10^{-9}\units{Кл}
                \cdot
                4\cdot 10^{-9}\units{Кл}
            }{
                \left(2 \cdot 10^{-2}\units{м}\right)^2
            }
            = \frac{18}{1}\cdot10^{-5}\units{Н}
              \approx {18.00}\cdot10^{-5}\units{Н}
    $
}
\vspace{120pt}

\tasknumber{2}\task{
    Два одинаковых маленьких проводящих заряженных шарика находятся
    на расстоянии~$d$ друг от друга.
    Заряд первого равен~$-4Q$, второго~---$+3Q$.
    Шарики приводят в соприкосновение, а после опять разводят на то же самое расстояние~$d$.
    Каким стал заряд каждого из шариков?
    Определите характер (притяжение или отталкивание)
    и силу взаимодействия шариков до и после соприкосновения.
}
\answer{
    \begin{align*}
        F   &= k\frac{q_1q_2}{d^2} = k\frac{(-4Q)\cdot(+3Q)}{d^2},
        \text{отталкивание};
        \\
        q'_1 = q'_2 = \frac{q_1 + q_2}2 = \frac{(-4Q) + (+3Q)}2 \implies
        F'  &= k\frac{q'_1q'_2}{d^2}
            = k\frac{
                    \left(\frac{(-4Q) + (+3Q)}2\right)^2
                }{
                    d^2
                },
        \text{отталкивание}.
    \end{align*}
}
\vspace{120pt}

\tasknumber{3}\task{
    На координатной плоскости в точках $(-l; 0)$ и $(l; 0)$
    находятся заряды, соответственно, $-q$ и $-q$.
    Сделайте рисунок, определите величину напряжённости электрического поля
    в точках $(0; -l)$ и $(-2l; 0)$ и укажите её направление.
}
\vspace{120pt}

\tasknumber{4}\task{
    Заряд $q_1$ создает в точке $A$ электрическое поле
    по величине равное~$E_1=72\funits{В}{м}$,
    а $q_2$~---$E_2=72\funits{В}{м}$.
    Угол между векторами $\vect{E_1}$ и $\vect{E_2}$ равен $\varphi$.
    Определите величину суммарного электрического поля в точке $A$,
    создаваемого обоими зарядами $q_1$ и $q_2$.
    Сделайте рисунок и вычислите её значение для двух значений угла $\varphi$:
    $\varphi_1=0^\circ$ и $\varphi_2=120^\circ$.
}
\newpage

\addpersonalvariant{Сабина Асадуллаева}


\tasknumber{1}\task{
    С какой силой взаимодействуют 2 точечных заряда $q_1=4\units{нКл}$ и $q_2=2\units{нКл}$,
    находящиеся на расстоянии $r=6\units{см}$?
}
\answer{
    $
        F
            = k\frac{q_1q_2}{r^2}
            = 9 \cdot 10^9 \funits{Н $\cdot$ м$^2$}{Кл$^2$} \cdot \frac{
                4\cdot 10^{-9}\units{Кл}
                \cdot
                2\cdot 10^{-9}\units{Кл}
            }{
                \left(6 \cdot 10^{-2}\units{м}\right)^2
            }
            = \frac{2}{1}\cdot10^{-5}\units{Н}
              \approx {2.00}\cdot10^{-5}\units{Н}
    $
}
\vspace{120pt}

\tasknumber{2}\task{
    Два одинаковых маленьких проводящих заряженных шарика находятся
    на расстоянии~$d$ друг от друга.
    Заряд первого равен~$+3q$, второго~---$-4q$.
    Шарики приводят в соприкосновение, а после опять разводят на то же самое расстояние~$d$.
    Каким стал заряд каждого из шариков?
    Определите характер (притяжение или отталкивание)
    и силу взаимодействия шариков до и после соприкосновения.
}
\answer{
    \begin{align*}
        F   &= k\frac{q_1q_2}{d^2} = k\frac{(+3q)\cdot(-4q)}{d^2},
        \text{отталкивание};
        \\
        q'_1 = q'_2 = \frac{q_1 + q_2}2 = \frac{(+3q) + (-4q)}2 \implies
        F'  &= k\frac{q'_1q'_2}{d^2}
            = k\frac{
                    \left(\frac{(+3q) + (-4q)}2\right)^2
                }{
                    d^2
                },
        \text{отталкивание}.
    \end{align*}
}
\vspace{120pt}

\tasknumber{3}\task{
    На координатной плоскости в точках $(-r; 0)$ и $(r; 0)$
    находятся заряды, соответственно, $+Q$ и $+Q$.
    Сделайте рисунок, определите величину напряжённости электрического поля
    в точках $(0; r)$ и $(2r; 0)$ и укажите её направление.
}
\vspace{120pt}

\tasknumber{4}\task{
    Заряд $q_1$ создает в точке $A$ электрическое поле
    по величине равное~$E_1=7\funits{В}{м}$,
    а $q_2$~---$E_2=24\funits{В}{м}$.
    Угол между векторами $\vect{E_1}$ и $\vect{E_2}$ равен $\alpha$.
    Определите величину суммарного электрического поля в точке $A$,
    создаваемого обоими зарядами $q_1$ и $q_2$.
    Сделайте рисунок и вычислите её значение для двух значений угла $\alpha$:
    $\alpha_1=0^\circ$ и $\alpha_2=90^\circ$.
}
\newpage

\addpersonalvariant{Вероника Битерякова}


\tasknumber{1}\task{
    С какой силой взаимодействуют 2 точечных заряда $q_1=2\units{нКл}$ и $q_2=4\units{нКл}$,
    находящиеся на расстоянии $d=5\units{см}$?
}
\answer{
    $
        F
            = k\frac{q_1q_2}{d^2}
            = 9 \cdot 10^9 \funits{Н $\cdot$ м$^2$}{Кл$^2$} \cdot \frac{
                2\cdot 10^{-9}\units{Кл}
                \cdot
                4\cdot 10^{-9}\units{Кл}
            }{
                \left(5 \cdot 10^{-2}\units{м}\right)^2
            }
            = \frac{72}{25}\cdot10^{-5}\units{Н}
              \approx {2.88}\cdot10^{-5}\units{Н}
    $
}
\vspace{120pt}

\tasknumber{2}\task{
    Два одинаковых маленьких проводящих заряженных шарика находятся
    на расстоянии~$l$ друг от друга.
    Заряд первого равен~$-5q$, второго~---$+3q$.
    Шарики приводят в соприкосновение, а после опять разводят на то же самое расстояние~$l$.
    Каким стал заряд каждого из шариков?
    Определите характер (притяжение или отталкивание)
    и силу взаимодействия шариков до и после соприкосновения.
}
\answer{
    \begin{align*}
        F   &= k\frac{q_1q_2}{l^2} = k\frac{(-5q)\cdot(+3q)}{l^2},
        \text{отталкивание};
        \\
        q'_1 = q'_2 = \frac{q_1 + q_2}2 = \frac{(-5q) + (+3q)}2 \implies
        F'  &= k\frac{q'_1q'_2}{l^2}
            = k\frac{
                    \left(\frac{(-5q) + (+3q)}2\right)^2
                }{
                    l^2
                },
        \text{отталкивание}.
    \end{align*}
}
\vspace{120pt}

\tasknumber{3}\task{
    На координатной плоскости в точках $(-l; 0)$ и $(l; 0)$
    находятся заряды, соответственно, $+Q$ и $+Q$.
    Сделайте рисунок, определите величину напряжённости электрического поля
    в точках $(0; -l)$ и $(2l; 0)$ и укажите её направление.
}
\vspace{120pt}

\tasknumber{4}\task{
    Заряд $q_1$ создает в точке $A$ электрическое поле
    по величине равное~$E_1=72\funits{В}{м}$,
    а $q_2$~---$E_2=72\funits{В}{м}$.
    Угол между векторами $\vect{E_1}$ и $\vect{E_2}$ равен $\varphi$.
    Определите величину суммарного электрического поля в точке $A$,
    создаваемого обоими зарядами $q_1$ и $q_2$.
    Сделайте рисунок и вычислите её значение для двух значений угла $\varphi$:
    $\varphi_1=0^\circ$ и $\varphi_2=120^\circ$.
}
\newpage

\addpersonalvariant{Юлия Буянова}


\tasknumber{1}\task{
    С какой силой взаимодействуют 2 точечных заряда $q_1=4\units{нКл}$ и $q_2=2\units{нКл}$,
    находящиеся на расстоянии $d=5\units{см}$?
}
\answer{
    $
        F
            = k\frac{q_1q_2}{d^2}
            = 9 \cdot 10^9 \funits{Н $\cdot$ м$^2$}{Кл$^2$} \cdot \frac{
                4\cdot 10^{-9}\units{Кл}
                \cdot
                2\cdot 10^{-9}\units{Кл}
            }{
                \left(5 \cdot 10^{-2}\units{м}\right)^2
            }
            = \frac{72}{25}\cdot10^{-5}\units{Н}
              \approx {2.88}\cdot10^{-5}\units{Н}
    $
}
\vspace{120pt}

\tasknumber{2}\task{
    Два одинаковых маленьких проводящих заряженных шарика находятся
    на расстоянии~$d$ друг от друга.
    Заряд первого равен~$+2q$, второго~---$-3q$.
    Шарики приводят в соприкосновение, а после опять разводят на то же самое расстояние~$d$.
    Каким стал заряд каждого из шариков?
    Определите характер (притяжение или отталкивание)
    и силу взаимодействия шариков до и после соприкосновения.
}
\answer{
    \begin{align*}
        F   &= k\frac{q_1q_2}{d^2} = k\frac{(+2q)\cdot(-3q)}{d^2},
        \text{отталкивание};
        \\
        q'_1 = q'_2 = \frac{q_1 + q_2}2 = \frac{(+2q) + (-3q)}2 \implies
        F'  &= k\frac{q'_1q'_2}{d^2}
            = k\frac{
                    \left(\frac{(+2q) + (-3q)}2\right)^2
                }{
                    d^2
                },
        \text{отталкивание}.
    \end{align*}
}
\vspace{120pt}

\tasknumber{3}\task{
    На координатной плоскости в точках $(-r; 0)$ и $(r; 0)$
    находятся заряды, соответственно, $-q$ и $-q$.
    Сделайте рисунок, определите величину напряжённости электрического поля
    в точках $(0; r)$ и $(2r; 0)$ и укажите её направление.
}
\vspace{120pt}

\tasknumber{4}\task{
    Заряд $q_1$ создает в точке $A$ электрическое поле
    по величине равное~$E_1=7\funits{В}{м}$,
    а $q_2$~---$E_2=24\funits{В}{м}$.
    Угол между векторами $\vect{E_1}$ и $\vect{E_2}$ равен $\alpha$.
    Определите величину суммарного электрического поля в точке $A$,
    создаваемого обоими зарядами $q_1$ и $q_2$.
    Сделайте рисунок и вычислите её значение для двух значений угла $\alpha$:
    $\alpha_1=0^\circ$ и $\alpha_2=90^\circ$.
}
\newpage

\addpersonalvariant{Пелагея Вдовина}


\tasknumber{1}\task{
    С какой силой взаимодействуют 2 точечных заряда $q_1=3\units{нКл}$ и $q_2=4\units{нКл}$,
    находящиеся на расстоянии $l=5\units{см}$?
}
\answer{
    $
        F
            = k\frac{q_1q_2}{l^2}
            = 9 \cdot 10^9 \funits{Н $\cdot$ м$^2$}{Кл$^2$} \cdot \frac{
                3\cdot 10^{-9}\units{Кл}
                \cdot
                4\cdot 10^{-9}\units{Кл}
            }{
                \left(5 \cdot 10^{-2}\units{м}\right)^2
            }
            = \frac{108}{25}\cdot10^{-5}\units{Н}
              \approx {4.32}\cdot10^{-5}\units{Н}
    $
}
\vspace{120pt}

\tasknumber{2}\task{
    Два одинаковых маленьких проводящих заряженных шарика находятся
    на расстоянии~$l$ друг от друга.
    Заряд первого равен~$+5Q$, второго~---$-4Q$.
    Шарики приводят в соприкосновение, а после опять разводят на то же самое расстояние~$l$.
    Каким стал заряд каждого из шариков?
    Определите характер (притяжение или отталкивание)
    и силу взаимодействия шариков до и после соприкосновения.
}
\answer{
    \begin{align*}
        F   &= k\frac{q_1q_2}{l^2} = k\frac{(+5Q)\cdot(-4Q)}{l^2},
        \text{отталкивание};
        \\
        q'_1 = q'_2 = \frac{q_1 + q_2}2 = \frac{(+5Q) + (-4Q)}2 \implies
        F'  &= k\frac{q'_1q'_2}{l^2}
            = k\frac{
                    \left(\frac{(+5Q) + (-4Q)}2\right)^2
                }{
                    l^2
                },
        \text{отталкивание}.
    \end{align*}
}
\vspace{120pt}

\tasknumber{3}\task{
    На координатной плоскости в точках $(-r; 0)$ и $(r; 0)$
    находятся заряды, соответственно, $-Q$ и $+Q$.
    Сделайте рисунок, определите величину напряжённости электрического поля
    в точках $(0; -r)$ и $(-2r; 0)$ и укажите её направление.
}
\vspace{120pt}

\tasknumber{4}\task{
    Заряд $q_1$ создает в точке $A$ электрическое поле
    по величине равное~$E_1=300\funits{В}{м}$,
    а $q_2$~---$E_2=400\funits{В}{м}$.
    Угол между векторами $\vect{E_1}$ и $\vect{E_2}$ равен $\alpha$.
    Определите величину суммарного электрического поля в точке $A$,
    создаваемого обоими зарядами $q_1$ и $q_2$.
    Сделайте рисунок и вычислите её значение для двух значений угла $\alpha$:
    $\alpha_1=0^\circ$ и $\alpha_2=90^\circ$.
}
\newpage

\addpersonalvariant{Леонид Викторов}


\tasknumber{1}\task{
    С какой силой взаимодействуют 2 точечных заряда $q_1=3\units{нКл}$ и $q_2=2\units{нКл}$,
    находящиеся на расстоянии $r=2\units{см}$?
}
\answer{
    $
        F
            = k\frac{q_1q_2}{r^2}
            = 9 \cdot 10^9 \funits{Н $\cdot$ м$^2$}{Кл$^2$} \cdot \frac{
                3\cdot 10^{-9}\units{Кл}
                \cdot
                2\cdot 10^{-9}\units{Кл}
            }{
                \left(2 \cdot 10^{-2}\units{м}\right)^2
            }
            = \frac{27}{2}\cdot10^{-5}\units{Н}
              \approx {13.50}\cdot10^{-5}\units{Н}
    $
}
\vspace{120pt}

\tasknumber{2}\task{
    Два одинаковых маленьких проводящих заряженных шарика находятся
    на расстоянии~$l$ друг от друга.
    Заряд первого равен~$+2q$, второго~---$-4q$.
    Шарики приводят в соприкосновение, а после опять разводят на то же самое расстояние~$l$.
    Каким стал заряд каждого из шариков?
    Определите характер (притяжение или отталкивание)
    и силу взаимодействия шариков до и после соприкосновения.
}
\answer{
    \begin{align*}
        F   &= k\frac{q_1q_2}{l^2} = k\frac{(+2q)\cdot(-4q)}{l^2},
        \text{отталкивание};
        \\
        q'_1 = q'_2 = \frac{q_1 + q_2}2 = \frac{(+2q) + (-4q)}2 \implies
        F'  &= k\frac{q'_1q'_2}{l^2}
            = k\frac{
                    \left(\frac{(+2q) + (-4q)}2\right)^2
                }{
                    l^2
                },
        \text{отталкивание}.
    \end{align*}
}
\vspace{120pt}

\tasknumber{3}\task{
    На координатной плоскости в точках $(-r; 0)$ и $(r; 0)$
    находятся заряды, соответственно, $+Q$ и $+Q$.
    Сделайте рисунок, определите величину напряжённости электрического поля
    в точках $(0; r)$ и $(-2r; 0)$ и укажите её направление.
}
\vspace{120pt}

\tasknumber{4}\task{
    Заряд $q_1$ создает в точке $A$ электрическое поле
    по величине равное~$E_1=250\funits{В}{м}$,
    а $q_2$~---$E_2=250\funits{В}{м}$.
    Угол между векторами $\vect{E_1}$ и $\vect{E_2}$ равен $\alpha$.
    Определите величину суммарного электрического поля в точке $A$,
    создаваемого обоими зарядами $q_1$ и $q_2$.
    Сделайте рисунок и вычислите её значение для двух значений угла $\alpha$:
    $\alpha_1=0^\circ$ и $\alpha_2=60^\circ$.
}
\newpage

\addpersonalvariant{Фёдор Гнутов}


\tasknumber{1}\task{
    С какой силой взаимодействуют 2 точечных заряда $q_1=4\units{нКл}$ и $q_2=2\units{нКл}$,
    находящиеся на расстоянии $l=5\units{см}$?
}
\answer{
    $
        F
            = k\frac{q_1q_2}{l^2}
            = 9 \cdot 10^9 \funits{Н $\cdot$ м$^2$}{Кл$^2$} \cdot \frac{
                4\cdot 10^{-9}\units{Кл}
                \cdot
                2\cdot 10^{-9}\units{Кл}
            }{
                \left(5 \cdot 10^{-2}\units{м}\right)^2
            }
            = \frac{72}{25}\cdot10^{-5}\units{Н}
              \approx {2.88}\cdot10^{-5}\units{Н}
    $
}
\vspace{120pt}

\tasknumber{2}\task{
    Два одинаковых маленьких проводящих заряженных шарика находятся
    на расстоянии~$d$ друг от друга.
    Заряд первого равен~$-2q$, второго~---$+3q$.
    Шарики приводят в соприкосновение, а после опять разводят на то же самое расстояние~$d$.
    Каким стал заряд каждого из шариков?
    Определите характер (притяжение или отталкивание)
    и силу взаимодействия шариков до и после соприкосновения.
}
\answer{
    \begin{align*}
        F   &= k\frac{q_1q_2}{d^2} = k\frac{(-2q)\cdot(+3q)}{d^2},
        \text{отталкивание};
        \\
        q'_1 = q'_2 = \frac{q_1 + q_2}2 = \frac{(-2q) + (+3q)}2 \implies
        F'  &= k\frac{q'_1q'_2}{d^2}
            = k\frac{
                    \left(\frac{(-2q) + (+3q)}2\right)^2
                }{
                    d^2
                },
        \text{отталкивание}.
    \end{align*}
}
\vspace{120pt}

\tasknumber{3}\task{
    На координатной плоскости в точках $(-l; 0)$ и $(l; 0)$
    находятся заряды, соответственно, $-q$ и $-q$.
    Сделайте рисунок, определите величину напряжённости электрического поля
    в точках $(0; l)$ и $(2l; 0)$ и укажите её направление.
}
\vspace{120pt}

\tasknumber{4}\task{
    Заряд $q_1$ создает в точке $A$ электрическое поле
    по величине равное~$E_1=72\funits{В}{м}$,
    а $q_2$~---$E_2=72\funits{В}{м}$.
    Угол между векторами $\vect{E_1}$ и $\vect{E_2}$ равен $\varphi$.
    Определите величину суммарного электрического поля в точке $A$,
    создаваемого обоими зарядами $q_1$ и $q_2$.
    Сделайте рисунок и вычислите её значение для двух значений угла $\varphi$:
    $\varphi_1=0^\circ$ и $\varphi_2=120^\circ$.
}
\newpage

\addpersonalvariant{Илья Гримберг}


\tasknumber{1}\task{
    С какой силой взаимодействуют 2 точечных заряда $q_1=2\units{нКл}$ и $q_2=4\units{нКл}$,
    находящиеся на расстоянии $l=3\units{см}$?
}
\answer{
    $
        F
            = k\frac{q_1q_2}{l^2}
            = 9 \cdot 10^9 \funits{Н $\cdot$ м$^2$}{Кл$^2$} \cdot \frac{
                2\cdot 10^{-9}\units{Кл}
                \cdot
                4\cdot 10^{-9}\units{Кл}
            }{
                \left(3 \cdot 10^{-2}\units{м}\right)^2
            }
            = \frac{8}{1}\cdot10^{-5}\units{Н}
              \approx {8.00}\cdot10^{-5}\units{Н}
    $
}
\vspace{120pt}

\tasknumber{2}\task{
    Два одинаковых маленьких проводящих заряженных шарика находятся
    на расстоянии~$d$ друг от друга.
    Заряд первого равен~$+5Q$, второго~---$-4Q$.
    Шарики приводят в соприкосновение, а после опять разводят на то же самое расстояние~$d$.
    Каким стал заряд каждого из шариков?
    Определите характер (притяжение или отталкивание)
    и силу взаимодействия шариков до и после соприкосновения.
}
\answer{
    \begin{align*}
        F   &= k\frac{q_1q_2}{d^2} = k\frac{(+5Q)\cdot(-4Q)}{d^2},
        \text{отталкивание};
        \\
        q'_1 = q'_2 = \frac{q_1 + q_2}2 = \frac{(+5Q) + (-4Q)}2 \implies
        F'  &= k\frac{q'_1q'_2}{d^2}
            = k\frac{
                    \left(\frac{(+5Q) + (-4Q)}2\right)^2
                }{
                    d^2
                },
        \text{отталкивание}.
    \end{align*}
}
\vspace{120pt}

\tasknumber{3}\task{
    На координатной плоскости в точках $(-r; 0)$ и $(r; 0)$
    находятся заряды, соответственно, $+q$ и $-q$.
    Сделайте рисунок, определите величину напряжённости электрического поля
    в точках $(0; -r)$ и $(-2r; 0)$ и укажите её направление.
}
\vspace{120pt}

\tasknumber{4}\task{
    Заряд $q_1$ создает в точке $A$ электрическое поле
    по величине равное~$E_1=250\funits{В}{м}$,
    а $q_2$~---$E_2=250\funits{В}{м}$.
    Угол между векторами $\vect{E_1}$ и $\vect{E_2}$ равен $\alpha$.
    Определите величину суммарного электрического поля в точке $A$,
    создаваемого обоими зарядами $q_1$ и $q_2$.
    Сделайте рисунок и вычислите её значение для двух значений угла $\alpha$:
    $\alpha_1=0^\circ$ и $\alpha_2=60^\circ$.
}
\newpage

\addpersonalvariant{Иван Гурьянов}


\tasknumber{1}\task{
    С какой силой взаимодействуют 2 точечных заряда $q_1=3\units{нКл}$ и $q_2=4\units{нКл}$,
    находящиеся на расстоянии $l=3\units{см}$?
}
\answer{
    $
        F
            = k\frac{q_1q_2}{l^2}
            = 9 \cdot 10^9 \funits{Н $\cdot$ м$^2$}{Кл$^2$} \cdot \frac{
                3\cdot 10^{-9}\units{Кл}
                \cdot
                4\cdot 10^{-9}\units{Кл}
            }{
                \left(3 \cdot 10^{-2}\units{м}\right)^2
            }
            = \frac{12}{1}\cdot10^{-5}\units{Н}
              \approx {12.00}\cdot10^{-5}\units{Н}
    $
}
\vspace{120pt}

\tasknumber{2}\task{
    Два одинаковых маленьких проводящих заряженных шарика находятся
    на расстоянии~$l$ друг от друга.
    Заряд первого равен~$+4q$, второго~---$+3q$.
    Шарики приводят в соприкосновение, а после опять разводят на то же самое расстояние~$l$.
    Каким стал заряд каждого из шариков?
    Определите характер (притяжение или отталкивание)
    и силу взаимодействия шариков до и после соприкосновения.
}
\answer{
    \begin{align*}
        F   &= k\frac{q_1q_2}{l^2} = k\frac{(+4q)\cdot(+3q)}{l^2},
        \text{отталкивание};
        \\
        q'_1 = q'_2 = \frac{q_1 + q_2}2 = \frac{(+4q) + (+3q)}2 \implies
        F'  &= k\frac{q'_1q'_2}{l^2}
            = k\frac{
                    \left(\frac{(+4q) + (+3q)}2\right)^2
                }{
                    l^2
                },
        \text{отталкивание}.
    \end{align*}
}
\vspace{120pt}

\tasknumber{3}\task{
    На координатной плоскости в точках $(-a; 0)$ и $(a; 0)$
    находятся заряды, соответственно, $-Q$ и $+Q$.
    Сделайте рисунок, определите величину напряжённости электрического поля
    в точках $(0; -a)$ и $(-2a; 0)$ и укажите её направление.
}
\vspace{120pt}

\tasknumber{4}\task{
    Заряд $q_1$ создает в точке $A$ электрическое поле
    по величине равное~$E_1=300\funits{В}{м}$,
    а $q_2$~---$E_2=400\funits{В}{м}$.
    Угол между векторами $\vect{E_1}$ и $\vect{E_2}$ равен $\alpha$.
    Определите величину суммарного электрического поля в точке $A$,
    создаваемого обоими зарядами $q_1$ и $q_2$.
    Сделайте рисунок и вычислите её значение для двух значений угла $\alpha$:
    $\alpha_1=90^\circ$ и $\alpha_2=180^\circ$.
}
\newpage

\addpersonalvariant{Артём Денежкин}


\tasknumber{1}\task{
    С какой силой взаимодействуют 2 точечных заряда $q_1=3\units{нКл}$ и $q_2=2\units{нКл}$,
    находящиеся на расстоянии $l=2\units{см}$?
}
\answer{
    $
        F
            = k\frac{q_1q_2}{l^2}
            = 9 \cdot 10^9 \funits{Н $\cdot$ м$^2$}{Кл$^2$} \cdot \frac{
                3\cdot 10^{-9}\units{Кл}
                \cdot
                2\cdot 10^{-9}\units{Кл}
            }{
                \left(2 \cdot 10^{-2}\units{м}\right)^2
            }
            = \frac{27}{2}\cdot10^{-5}\units{Н}
              \approx {13.50}\cdot10^{-5}\units{Н}
    $
}
\vspace{120pt}

\tasknumber{2}\task{
    Два одинаковых маленьких проводящих заряженных шарика находятся
    на расстоянии~$l$ друг от друга.
    Заряд первого равен~$-2q$, второго~---$-4q$.
    Шарики приводят в соприкосновение, а после опять разводят на то же самое расстояние~$l$.
    Каким стал заряд каждого из шариков?
    Определите характер (притяжение или отталкивание)
    и силу взаимодействия шариков до и после соприкосновения.
}
\answer{
    \begin{align*}
        F   &= k\frac{q_1q_2}{l^2} = k\frac{(-2q)\cdot(-4q)}{l^2},
        \text{отталкивание};
        \\
        q'_1 = q'_2 = \frac{q_1 + q_2}2 = \frac{(-2q) + (-4q)}2 \implies
        F'  &= k\frac{q'_1q'_2}{l^2}
            = k\frac{
                    \left(\frac{(-2q) + (-4q)}2\right)^2
                }{
                    l^2
                },
        \text{отталкивание}.
    \end{align*}
}
\vspace{120pt}

\tasknumber{3}\task{
    На координатной плоскости в точках $(-d; 0)$ и $(d; 0)$
    находятся заряды, соответственно, $+Q$ и $+Q$.
    Сделайте рисунок, определите величину напряжённости электрического поля
    в точках $(0; d)$ и $(-2d; 0)$ и укажите её направление.
}
\vspace{120pt}

\tasknumber{4}\task{
    Заряд $q_1$ создает в точке $A$ электрическое поле
    по величине равное~$E_1=200\funits{В}{м}$,
    а $q_2$~---$E_2=200\funits{В}{м}$.
    Угол между векторами $\vect{E_1}$ и $\vect{E_2}$ равен $\varphi$.
    Определите величину суммарного электрического поля в точке $A$,
    создаваемого обоими зарядами $q_1$ и $q_2$.
    Сделайте рисунок и вычислите её значение для двух значений угла $\varphi$:
    $\varphi_1=0^\circ$ и $\varphi_2=60^\circ$.
}
\newpage

\addpersonalvariant{Виктор Жилин}


\tasknumber{1}\task{
    С какой силой взаимодействуют 2 точечных заряда $q_1=2\units{нКл}$ и $q_2=4\units{нКл}$,
    находящиеся на расстоянии $l=6\units{см}$?
}
\answer{
    $
        F
            = k\frac{q_1q_2}{l^2}
            = 9 \cdot 10^9 \funits{Н $\cdot$ м$^2$}{Кл$^2$} \cdot \frac{
                2\cdot 10^{-9}\units{Кл}
                \cdot
                4\cdot 10^{-9}\units{Кл}
            }{
                \left(6 \cdot 10^{-2}\units{м}\right)^2
            }
            = \frac{2}{1}\cdot10^{-5}\units{Н}
              \approx {2.00}\cdot10^{-5}\units{Н}
    $
}
\vspace{120pt}

\tasknumber{2}\task{
    Два одинаковых маленьких проводящих заряженных шарика находятся
    на расстоянии~$d$ друг от друга.
    Заряд первого равен~$+2Q$, второго~---$-5Q$.
    Шарики приводят в соприкосновение, а после опять разводят на то же самое расстояние~$d$.
    Каким стал заряд каждого из шариков?
    Определите характер (притяжение или отталкивание)
    и силу взаимодействия шариков до и после соприкосновения.
}
\answer{
    \begin{align*}
        F   &= k\frac{q_1q_2}{d^2} = k\frac{(+2Q)\cdot(-5Q)}{d^2},
        \text{отталкивание};
        \\
        q'_1 = q'_2 = \frac{q_1 + q_2}2 = \frac{(+2Q) + (-5Q)}2 \implies
        F'  &= k\frac{q'_1q'_2}{d^2}
            = k\frac{
                    \left(\frac{(+2Q) + (-5Q)}2\right)^2
                }{
                    d^2
                },
        \text{отталкивание}.
    \end{align*}
}
\vspace{120pt}

\tasknumber{3}\task{
    На координатной плоскости в точках $(-r; 0)$ и $(r; 0)$
    находятся заряды, соответственно, $+q$ и $-q$.
    Сделайте рисунок, определите величину напряжённости электрического поля
    в точках $(0; -r)$ и $(2r; 0)$ и укажите её направление.
}
\vspace{120pt}

\tasknumber{4}\task{
    Заряд $q_1$ создает в точке $A$ электрическое поле
    по величине равное~$E_1=50\funits{В}{м}$,
    а $q_2$~---$E_2=120\funits{В}{м}$.
    Угол между векторами $\vect{E_1}$ и $\vect{E_2}$ равен $\alpha$.
    Определите величину суммарного электрического поля в точке $A$,
    создаваемого обоими зарядами $q_1$ и $q_2$.
    Сделайте рисунок и вычислите её значение для двух значений угла $\alpha$:
    $\alpha_1=0^\circ$ и $\alpha_2=90^\circ$.
}
\newpage

\addpersonalvariant{Дмитрий Иванов}


\tasknumber{1}\task{
    С какой силой взаимодействуют 2 точечных заряда $q_1=3\units{нКл}$ и $q_2=4\units{нКл}$,
    находящиеся на расстоянии $l=3\units{см}$?
}
\answer{
    $
        F
            = k\frac{q_1q_2}{l^2}
            = 9 \cdot 10^9 \funits{Н $\cdot$ м$^2$}{Кл$^2$} \cdot \frac{
                3\cdot 10^{-9}\units{Кл}
                \cdot
                4\cdot 10^{-9}\units{Кл}
            }{
                \left(3 \cdot 10^{-2}\units{м}\right)^2
            }
            = \frac{12}{1}\cdot10^{-5}\units{Н}
              \approx {12.00}\cdot10^{-5}\units{Н}
    $
}
\vspace{120pt}

\tasknumber{2}\task{
    Два одинаковых маленьких проводящих заряженных шарика находятся
    на расстоянии~$r$ друг от друга.
    Заряд первого равен~$+2Q$, второго~---$+4Q$.
    Шарики приводят в соприкосновение, а после опять разводят на то же самое расстояние~$r$.
    Каким стал заряд каждого из шариков?
    Определите характер (притяжение или отталкивание)
    и силу взаимодействия шариков до и после соприкосновения.
}
\answer{
    \begin{align*}
        F   &= k\frac{q_1q_2}{r^2} = k\frac{(+2Q)\cdot(+4Q)}{r^2},
        \text{отталкивание};
        \\
        q'_1 = q'_2 = \frac{q_1 + q_2}2 = \frac{(+2Q) + (+4Q)}2 \implies
        F'  &= k\frac{q'_1q'_2}{r^2}
            = k\frac{
                    \left(\frac{(+2Q) + (+4Q)}2\right)^2
                }{
                    r^2
                },
        \text{отталкивание}.
    \end{align*}
}
\vspace{120pt}

\tasknumber{3}\task{
    На координатной плоскости в точках $(-a; 0)$ и $(a; 0)$
    находятся заряды, соответственно, $-Q$ и $+Q$.
    Сделайте рисунок, определите величину напряжённости электрического поля
    в точках $(0; a)$ и $(-2a; 0)$ и укажите её направление.
}
\vspace{120pt}

\tasknumber{4}\task{
    Заряд $q_1$ создает в точке $A$ электрическое поле
    по величине равное~$E_1=72\funits{В}{м}$,
    а $q_2$~---$E_2=72\funits{В}{м}$.
    Угол между векторами $\vect{E_1}$ и $\vect{E_2}$ равен $\alpha$.
    Определите величину суммарного электрического поля в точке $A$,
    создаваемого обоими зарядами $q_1$ и $q_2$.
    Сделайте рисунок и вычислите её значение для двух значений угла $\alpha$:
    $\alpha_1=0^\circ$ и $\alpha_2=120^\circ$.
}
\newpage

\addpersonalvariant{Олег Климов}


\tasknumber{1}\task{
    С какой силой взаимодействуют 2 точечных заряда $q_1=3\units{нКл}$ и $q_2=4\units{нКл}$,
    находящиеся на расстоянии $r=6\units{см}$?
}
\answer{
    $
        F
            = k\frac{q_1q_2}{r^2}
            = 9 \cdot 10^9 \funits{Н $\cdot$ м$^2$}{Кл$^2$} \cdot \frac{
                3\cdot 10^{-9}\units{Кл}
                \cdot
                4\cdot 10^{-9}\units{Кл}
            }{
                \left(6 \cdot 10^{-2}\units{м}\right)^2
            }
            = \frac{3}{1}\cdot10^{-5}\units{Н}
              \approx {3.00}\cdot10^{-5}\units{Н}
    $
}
\vspace{120pt}

\tasknumber{2}\task{
    Два одинаковых маленьких проводящих заряженных шарика находятся
    на расстоянии~$r$ друг от друга.
    Заряд первого равен~$-5q$, второго~---$-2q$.
    Шарики приводят в соприкосновение, а после опять разводят на то же самое расстояние~$r$.
    Каким стал заряд каждого из шариков?
    Определите характер (притяжение или отталкивание)
    и силу взаимодействия шариков до и после соприкосновения.
}
\answer{
    \begin{align*}
        F   &= k\frac{q_1q_2}{r^2} = k\frac{(-5q)\cdot(-2q)}{r^2},
        \text{отталкивание};
        \\
        q'_1 = q'_2 = \frac{q_1 + q_2}2 = \frac{(-5q) + (-2q)}2 \implies
        F'  &= k\frac{q'_1q'_2}{r^2}
            = k\frac{
                    \left(\frac{(-5q) + (-2q)}2\right)^2
                }{
                    r^2
                },
        \text{отталкивание}.
    \end{align*}
}
\vspace{120pt}

\tasknumber{3}\task{
    На координатной плоскости в точках $(-d; 0)$ и $(d; 0)$
    находятся заряды, соответственно, $+Q$ и $+Q$.
    Сделайте рисунок, определите величину напряжённости электрического поля
    в точках $(0; d)$ и $(-2d; 0)$ и укажите её направление.
}
\vspace{120pt}

\tasknumber{4}\task{
    Заряд $q_1$ создает в точке $A$ электрическое поле
    по величине равное~$E_1=7\funits{В}{м}$,
    а $q_2$~---$E_2=24\funits{В}{м}$.
    Угол между векторами $\vect{E_1}$ и $\vect{E_2}$ равен $\varphi$.
    Определите величину суммарного электрического поля в точке $A$,
    создаваемого обоими зарядами $q_1$ и $q_2$.
    Сделайте рисунок и вычислите её значение для двух значений угла $\varphi$:
    $\varphi_1=0^\circ$ и $\varphi_2=90^\circ$.
}
\newpage

\addpersonalvariant{Анна Ковалева}


\tasknumber{1}\task{
    С какой силой взаимодействуют 2 точечных заряда $q_1=4\units{нКл}$ и $q_2=2\units{нКл}$,
    находящиеся на расстоянии $d=2\units{см}$?
}
\answer{
    $
        F
            = k\frac{q_1q_2}{d^2}
            = 9 \cdot 10^9 \funits{Н $\cdot$ м$^2$}{Кл$^2$} \cdot \frac{
                4\cdot 10^{-9}\units{Кл}
                \cdot
                2\cdot 10^{-9}\units{Кл}
            }{
                \left(2 \cdot 10^{-2}\units{м}\right)^2
            }
            = \frac{18}{1}\cdot10^{-5}\units{Н}
              \approx {18.00}\cdot10^{-5}\units{Н}
    $
}
\vspace{120pt}

\tasknumber{2}\task{
    Два одинаковых маленьких проводящих заряженных шарика находятся
    на расстоянии~$l$ друг от друга.
    Заряд первого равен~$+3q$, второго~---$-2q$.
    Шарики приводят в соприкосновение, а после опять разводят на то же самое расстояние~$l$.
    Каким стал заряд каждого из шариков?
    Определите характер (притяжение или отталкивание)
    и силу взаимодействия шариков до и после соприкосновения.
}
\answer{
    \begin{align*}
        F   &= k\frac{q_1q_2}{l^2} = k\frac{(+3q)\cdot(-2q)}{l^2},
        \text{отталкивание};
        \\
        q'_1 = q'_2 = \frac{q_1 + q_2}2 = \frac{(+3q) + (-2q)}2 \implies
        F'  &= k\frac{q'_1q'_2}{l^2}
            = k\frac{
                    \left(\frac{(+3q) + (-2q)}2\right)^2
                }{
                    l^2
                },
        \text{отталкивание}.
    \end{align*}
}
\vspace{120pt}

\tasknumber{3}\task{
    На координатной плоскости в точках $(-r; 0)$ и $(r; 0)$
    находятся заряды, соответственно, $+q$ и $-q$.
    Сделайте рисунок, определите величину напряжённости электрического поля
    в точках $(0; -r)$ и $(-2r; 0)$ и укажите её направление.
}
\vspace{120pt}

\tasknumber{4}\task{
    Заряд $q_1$ создает в точке $A$ электрическое поле
    по величине равное~$E_1=200\funits{В}{м}$,
    а $q_2$~---$E_2=200\funits{В}{м}$.
    Угол между векторами $\vect{E_1}$ и $\vect{E_2}$ равен $\alpha$.
    Определите величину суммарного электрического поля в точке $A$,
    создаваемого обоими зарядами $q_1$ и $q_2$.
    Сделайте рисунок и вычислите её значение для двух значений угла $\alpha$:
    $\alpha_1=0^\circ$ и $\alpha_2=60^\circ$.
}
\newpage

\addpersonalvariant{Глеб Ковылин}


\tasknumber{1}\task{
    С какой силой взаимодействуют 2 точечных заряда $q_1=3\units{нКл}$ и $q_2=4\units{нКл}$,
    находящиеся на расстоянии $r=5\units{см}$?
}
\answer{
    $
        F
            = k\frac{q_1q_2}{r^2}
            = 9 \cdot 10^9 \funits{Н $\cdot$ м$^2$}{Кл$^2$} \cdot \frac{
                3\cdot 10^{-9}\units{Кл}
                \cdot
                4\cdot 10^{-9}\units{Кл}
            }{
                \left(5 \cdot 10^{-2}\units{м}\right)^2
            }
            = \frac{108}{25}\cdot10^{-5}\units{Н}
              \approx {4.32}\cdot10^{-5}\units{Н}
    $
}
\vspace{120pt}

\tasknumber{2}\task{
    Два одинаковых маленьких проводящих заряженных шарика находятся
    на расстоянии~$l$ друг от друга.
    Заряд первого равен~$-5q$, второго~---$+4q$.
    Шарики приводят в соприкосновение, а после опять разводят на то же самое расстояние~$l$.
    Каким стал заряд каждого из шариков?
    Определите характер (притяжение или отталкивание)
    и силу взаимодействия шариков до и после соприкосновения.
}
\answer{
    \begin{align*}
        F   &= k\frac{q_1q_2}{l^2} = k\frac{(-5q)\cdot(+4q)}{l^2},
        \text{отталкивание};
        \\
        q'_1 = q'_2 = \frac{q_1 + q_2}2 = \frac{(-5q) + (+4q)}2 \implies
        F'  &= k\frac{q'_1q'_2}{l^2}
            = k\frac{
                    \left(\frac{(-5q) + (+4q)}2\right)^2
                }{
                    l^2
                },
        \text{отталкивание}.
    \end{align*}
}
\vspace{120pt}

\tasknumber{3}\task{
    На координатной плоскости в точках $(-l; 0)$ и $(l; 0)$
    находятся заряды, соответственно, $+q$ и $-q$.
    Сделайте рисунок, определите величину напряжённости электрического поля
    в точках $(0; -l)$ и $(-2l; 0)$ и укажите её направление.
}
\vspace{120pt}

\tasknumber{4}\task{
    Заряд $q_1$ создает в точке $A$ электрическое поле
    по величине равное~$E_1=120\funits{В}{м}$,
    а $q_2$~---$E_2=50\funits{В}{м}$.
    Угол между векторами $\vect{E_1}$ и $\vect{E_2}$ равен $\varphi$.
    Определите величину суммарного электрического поля в точке $A$,
    создаваемого обоими зарядами $q_1$ и $q_2$.
    Сделайте рисунок и вычислите её значение для двух значений угла $\varphi$:
    $\varphi_1=90^\circ$ и $\varphi_2=180^\circ$.
}
\newpage

\addpersonalvariant{Даниил Космынин}


\tasknumber{1}\task{
    С какой силой взаимодействуют 2 точечных заряда $q_1=4\units{нКл}$ и $q_2=2\units{нКл}$,
    находящиеся на расстоянии $r=2\units{см}$?
}
\answer{
    $
        F
            = k\frac{q_1q_2}{r^2}
            = 9 \cdot 10^9 \funits{Н $\cdot$ м$^2$}{Кл$^2$} \cdot \frac{
                4\cdot 10^{-9}\units{Кл}
                \cdot
                2\cdot 10^{-9}\units{Кл}
            }{
                \left(2 \cdot 10^{-2}\units{м}\right)^2
            }
            = \frac{18}{1}\cdot10^{-5}\units{Н}
              \approx {18.00}\cdot10^{-5}\units{Н}
    $
}
\vspace{120pt}

\tasknumber{2}\task{
    Два одинаковых маленьких проводящих заряженных шарика находятся
    на расстоянии~$d$ друг от друга.
    Заряд первого равен~$+5Q$, второго~---$+2Q$.
    Шарики приводят в соприкосновение, а после опять разводят на то же самое расстояние~$d$.
    Каким стал заряд каждого из шариков?
    Определите характер (притяжение или отталкивание)
    и силу взаимодействия шариков до и после соприкосновения.
}
\answer{
    \begin{align*}
        F   &= k\frac{q_1q_2}{d^2} = k\frac{(+5Q)\cdot(+2Q)}{d^2},
        \text{отталкивание};
        \\
        q'_1 = q'_2 = \frac{q_1 + q_2}2 = \frac{(+5Q) + (+2Q)}2 \implies
        F'  &= k\frac{q'_1q'_2}{d^2}
            = k\frac{
                    \left(\frac{(+5Q) + (+2Q)}2\right)^2
                }{
                    d^2
                },
        \text{отталкивание}.
    \end{align*}
}
\vspace{120pt}

\tasknumber{3}\task{
    На координатной плоскости в точках $(-a; 0)$ и $(a; 0)$
    находятся заряды, соответственно, $+Q$ и $+Q$.
    Сделайте рисунок, определите величину напряжённости электрического поля
    в точках $(0; a)$ и $(-2a; 0)$ и укажите её направление.
}
\vspace{120pt}

\tasknumber{4}\task{
    Заряд $q_1$ создает в точке $A$ электрическое поле
    по величине равное~$E_1=120\funits{В}{м}$,
    а $q_2$~---$E_2=50\funits{В}{м}$.
    Угол между векторами $\vect{E_1}$ и $\vect{E_2}$ равен $\alpha$.
    Определите величину суммарного электрического поля в точке $A$,
    создаваемого обоими зарядами $q_1$ и $q_2$.
    Сделайте рисунок и вычислите её значение для двух значений угла $\alpha$:
    $\alpha_1=90^\circ$ и $\alpha_2=180^\circ$.
}
\newpage

\addpersonalvariant{Алина Леоничева}


\tasknumber{1}\task{
    С какой силой взаимодействуют 2 точечных заряда $q_1=3\units{нКл}$ и $q_2=4\units{нКл}$,
    находящиеся на расстоянии $r=3\units{см}$?
}
\answer{
    $
        F
            = k\frac{q_1q_2}{r^2}
            = 9 \cdot 10^9 \funits{Н $\cdot$ м$^2$}{Кл$^2$} \cdot \frac{
                3\cdot 10^{-9}\units{Кл}
                \cdot
                4\cdot 10^{-9}\units{Кл}
            }{
                \left(3 \cdot 10^{-2}\units{м}\right)^2
            }
            = \frac{12}{1}\cdot10^{-5}\units{Н}
              \approx {12.00}\cdot10^{-5}\units{Н}
    $
}
\vspace{120pt}

\tasknumber{2}\task{
    Два одинаковых маленьких проводящих заряженных шарика находятся
    на расстоянии~$r$ друг от друга.
    Заряд первого равен~$-2q$, второго~---$-3q$.
    Шарики приводят в соприкосновение, а после опять разводят на то же самое расстояние~$r$.
    Каким стал заряд каждого из шариков?
    Определите характер (притяжение или отталкивание)
    и силу взаимодействия шариков до и после соприкосновения.
}
\answer{
    \begin{align*}
        F   &= k\frac{q_1q_2}{r^2} = k\frac{(-2q)\cdot(-3q)}{r^2},
        \text{отталкивание};
        \\
        q'_1 = q'_2 = \frac{q_1 + q_2}2 = \frac{(-2q) + (-3q)}2 \implies
        F'  &= k\frac{q'_1q'_2}{r^2}
            = k\frac{
                    \left(\frac{(-2q) + (-3q)}2\right)^2
                }{
                    r^2
                },
        \text{отталкивание}.
    \end{align*}
}
\vspace{120pt}

\tasknumber{3}\task{
    На координатной плоскости в точках $(-d; 0)$ и $(d; 0)$
    находятся заряды, соответственно, $+q$ и $-q$.
    Сделайте рисунок, определите величину напряжённости электрического поля
    в точках $(0; d)$ и $(2d; 0)$ и укажите её направление.
}
\vspace{120pt}

\tasknumber{4}\task{
    Заряд $q_1$ создает в точке $A$ электрическое поле
    по величине равное~$E_1=7\funits{В}{м}$,
    а $q_2$~---$E_2=24\funits{В}{м}$.
    Угол между векторами $\vect{E_1}$ и $\vect{E_2}$ равен $\varphi$.
    Определите величину суммарного электрического поля в точке $A$,
    создаваемого обоими зарядами $q_1$ и $q_2$.
    Сделайте рисунок и вычислите её значение для двух значений угла $\varphi$:
    $\varphi_1=0^\circ$ и $\varphi_2=90^\circ$.
}
\newpage

\addpersonalvariant{Ирина Лин}


\tasknumber{1}\task{
    С какой силой взаимодействуют 2 точечных заряда $q_1=4\units{нКл}$ и $q_2=3\units{нКл}$,
    находящиеся на расстоянии $d=3\units{см}$?
}
\answer{
    $
        F
            = k\frac{q_1q_2}{d^2}
            = 9 \cdot 10^9 \funits{Н $\cdot$ м$^2$}{Кл$^2$} \cdot \frac{
                4\cdot 10^{-9}\units{Кл}
                \cdot
                3\cdot 10^{-9}\units{Кл}
            }{
                \left(3 \cdot 10^{-2}\units{м}\right)^2
            }
            = \frac{12}{1}\cdot10^{-5}\units{Н}
              \approx {12.00}\cdot10^{-5}\units{Н}
    $
}
\vspace{120pt}

\tasknumber{2}\task{
    Два одинаковых маленьких проводящих заряженных шарика находятся
    на расстоянии~$r$ друг от друга.
    Заряд первого равен~$-2q$, второго~---$-5q$.
    Шарики приводят в соприкосновение, а после опять разводят на то же самое расстояние~$r$.
    Каким стал заряд каждого из шариков?
    Определите характер (притяжение или отталкивание)
    и силу взаимодействия шариков до и после соприкосновения.
}
\answer{
    \begin{align*}
        F   &= k\frac{q_1q_2}{r^2} = k\frac{(-2q)\cdot(-5q)}{r^2},
        \text{отталкивание};
        \\
        q'_1 = q'_2 = \frac{q_1 + q_2}2 = \frac{(-2q) + (-5q)}2 \implies
        F'  &= k\frac{q'_1q'_2}{r^2}
            = k\frac{
                    \left(\frac{(-2q) + (-5q)}2\right)^2
                }{
                    r^2
                },
        \text{отталкивание}.
    \end{align*}
}
\vspace{120pt}

\tasknumber{3}\task{
    На координатной плоскости в точках $(-d; 0)$ и $(d; 0)$
    находятся заряды, соответственно, $+q$ и $-q$.
    Сделайте рисунок, определите величину напряжённости электрического поля
    в точках $(0; -d)$ и $(2d; 0)$ и укажите её направление.
}
\vspace{120pt}

\tasknumber{4}\task{
    Заряд $q_1$ создает в точке $A$ электрическое поле
    по величине равное~$E_1=7\funits{В}{м}$,
    а $q_2$~---$E_2=24\funits{В}{м}$.
    Угол между векторами $\vect{E_1}$ и $\vect{E_2}$ равен $\varphi$.
    Определите величину суммарного электрического поля в точке $A$,
    создаваемого обоими зарядами $q_1$ и $q_2$.
    Сделайте рисунок и вычислите её значение для двух значений угла $\varphi$:
    $\varphi_1=0^\circ$ и $\varphi_2=90^\circ$.
}
\newpage

\addpersonalvariant{Олег Мальцев}


\tasknumber{1}\task{
    С какой силой взаимодействуют 2 точечных заряда $q_1=4\units{нКл}$ и $q_2=3\units{нКл}$,
    находящиеся на расстоянии $r=2\units{см}$?
}
\answer{
    $
        F
            = k\frac{q_1q_2}{r^2}
            = 9 \cdot 10^9 \funits{Н $\cdot$ м$^2$}{Кл$^2$} \cdot \frac{
                4\cdot 10^{-9}\units{Кл}
                \cdot
                3\cdot 10^{-9}\units{Кл}
            }{
                \left(2 \cdot 10^{-2}\units{м}\right)^2
            }
            = \frac{27}{1}\cdot10^{-5}\units{Н}
              \approx {27.00}\cdot10^{-5}\units{Н}
    $
}
\vspace{120pt}

\tasknumber{2}\task{
    Два одинаковых маленьких проводящих заряженных шарика находятся
    на расстоянии~$d$ друг от друга.
    Заряд первого равен~$-5Q$, второго~---$-2Q$.
    Шарики приводят в соприкосновение, а после опять разводят на то же самое расстояние~$d$.
    Каким стал заряд каждого из шариков?
    Определите характер (притяжение или отталкивание)
    и силу взаимодействия шариков до и после соприкосновения.
}
\answer{
    \begin{align*}
        F   &= k\frac{q_1q_2}{d^2} = k\frac{(-5Q)\cdot(-2Q)}{d^2},
        \text{отталкивание};
        \\
        q'_1 = q'_2 = \frac{q_1 + q_2}2 = \frac{(-5Q) + (-2Q)}2 \implies
        F'  &= k\frac{q'_1q'_2}{d^2}
            = k\frac{
                    \left(\frac{(-5Q) + (-2Q)}2\right)^2
                }{
                    d^2
                },
        \text{отталкивание}.
    \end{align*}
}
\vspace{120pt}

\tasknumber{3}\task{
    На координатной плоскости в точках $(-d; 0)$ и $(d; 0)$
    находятся заряды, соответственно, $+Q$ и $+Q$.
    Сделайте рисунок, определите величину напряжённости электрического поля
    в точках $(0; d)$ и $(-2d; 0)$ и укажите её направление.
}
\vspace{120pt}

\tasknumber{4}\task{
    Заряд $q_1$ создает в точке $A$ электрическое поле
    по величине равное~$E_1=7\funits{В}{м}$,
    а $q_2$~---$E_2=24\funits{В}{м}$.
    Угол между векторами $\vect{E_1}$ и $\vect{E_2}$ равен $\varphi$.
    Определите величину суммарного электрического поля в точке $A$,
    создаваемого обоими зарядами $q_1$ и $q_2$.
    Сделайте рисунок и вычислите её значение для двух значений угла $\varphi$:
    $\varphi_1=0^\circ$ и $\varphi_2=90^\circ$.
}
\newpage

\addpersonalvariant{Ислам Мунаев}


\tasknumber{1}\task{
    С какой силой взаимодействуют 2 точечных заряда $q_1=4\units{нКл}$ и $q_2=2\units{нКл}$,
    находящиеся на расстоянии $l=2\units{см}$?
}
\answer{
    $
        F
            = k\frac{q_1q_2}{l^2}
            = 9 \cdot 10^9 \funits{Н $\cdot$ м$^2$}{Кл$^2$} \cdot \frac{
                4\cdot 10^{-9}\units{Кл}
                \cdot
                2\cdot 10^{-9}\units{Кл}
            }{
                \left(2 \cdot 10^{-2}\units{м}\right)^2
            }
            = \frac{18}{1}\cdot10^{-5}\units{Н}
              \approx {18.00}\cdot10^{-5}\units{Н}
    $
}
\vspace{120pt}

\tasknumber{2}\task{
    Два одинаковых маленьких проводящих заряженных шарика находятся
    на расстоянии~$d$ друг от друга.
    Заряд первого равен~$-5q$, второго~---$+2q$.
    Шарики приводят в соприкосновение, а после опять разводят на то же самое расстояние~$d$.
    Каким стал заряд каждого из шариков?
    Определите характер (притяжение или отталкивание)
    и силу взаимодействия шариков до и после соприкосновения.
}
\answer{
    \begin{align*}
        F   &= k\frac{q_1q_2}{d^2} = k\frac{(-5q)\cdot(+2q)}{d^2},
        \text{отталкивание};
        \\
        q'_1 = q'_2 = \frac{q_1 + q_2}2 = \frac{(-5q) + (+2q)}2 \implies
        F'  &= k\frac{q'_1q'_2}{d^2}
            = k\frac{
                    \left(\frac{(-5q) + (+2q)}2\right)^2
                }{
                    d^2
                },
        \text{отталкивание}.
    \end{align*}
}
\vspace{120pt}

\tasknumber{3}\task{
    На координатной плоскости в точках $(-l; 0)$ и $(l; 0)$
    находятся заряды, соответственно, $-Q$ и $+Q$.
    Сделайте рисунок, определите величину напряжённости электрического поля
    в точках $(0; l)$ и $(-2l; 0)$ и укажите её направление.
}
\vspace{120pt}

\tasknumber{4}\task{
    Заряд $q_1$ создает в точке $A$ электрическое поле
    по величине равное~$E_1=72\funits{В}{м}$,
    а $q_2$~---$E_2=72\funits{В}{м}$.
    Угол между векторами $\vect{E_1}$ и $\vect{E_2}$ равен $\varphi$.
    Определите величину суммарного электрического поля в точке $A$,
    создаваемого обоими зарядами $q_1$ и $q_2$.
    Сделайте рисунок и вычислите её значение для двух значений угла $\varphi$:
    $\varphi_1=0^\circ$ и $\varphi_2=120^\circ$.
}
\newpage

\addpersonalvariant{Александр Наумов}


\tasknumber{1}\task{
    С какой силой взаимодействуют 2 точечных заряда $q_1=4\units{нКл}$ и $q_2=2\units{нКл}$,
    находящиеся на расстоянии $d=3\units{см}$?
}
\answer{
    $
        F
            = k\frac{q_1q_2}{d^2}
            = 9 \cdot 10^9 \funits{Н $\cdot$ м$^2$}{Кл$^2$} \cdot \frac{
                4\cdot 10^{-9}\units{Кл}
                \cdot
                2\cdot 10^{-9}\units{Кл}
            }{
                \left(3 \cdot 10^{-2}\units{м}\right)^2
            }
            = \frac{8}{1}\cdot10^{-5}\units{Н}
              \approx {8.00}\cdot10^{-5}\units{Н}
    $
}
\vspace{120pt}

\tasknumber{2}\task{
    Два одинаковых маленьких проводящих заряженных шарика находятся
    на расстоянии~$l$ друг от друга.
    Заряд первого равен~$+4q$, второго~---$+5q$.
    Шарики приводят в соприкосновение, а после опять разводят на то же самое расстояние~$l$.
    Каким стал заряд каждого из шариков?
    Определите характер (притяжение или отталкивание)
    и силу взаимодействия шариков до и после соприкосновения.
}
\answer{
    \begin{align*}
        F   &= k\frac{q_1q_2}{l^2} = k\frac{(+4q)\cdot(+5q)}{l^2},
        \text{отталкивание};
        \\
        q'_1 = q'_2 = \frac{q_1 + q_2}2 = \frac{(+4q) + (+5q)}2 \implies
        F'  &= k\frac{q'_1q'_2}{l^2}
            = k\frac{
                    \left(\frac{(+4q) + (+5q)}2\right)^2
                }{
                    l^2
                },
        \text{отталкивание}.
    \end{align*}
}
\vspace{120pt}

\tasknumber{3}\task{
    На координатной плоскости в точках $(-a; 0)$ и $(a; 0)$
    находятся заряды, соответственно, $+q$ и $-q$.
    Сделайте рисунок, определите величину напряжённости электрического поля
    в точках $(0; a)$ и $(2a; 0)$ и укажите её направление.
}
\vspace{120pt}

\tasknumber{4}\task{
    Заряд $q_1$ создает в точке $A$ электрическое поле
    по величине равное~$E_1=120\funits{В}{м}$,
    а $q_2$~---$E_2=50\funits{В}{м}$.
    Угол между векторами $\vect{E_1}$ и $\vect{E_2}$ равен $\alpha$.
    Определите величину суммарного электрического поля в точке $A$,
    создаваемого обоими зарядами $q_1$ и $q_2$.
    Сделайте рисунок и вычислите её значение для двух значений угла $\alpha$:
    $\alpha_1=90^\circ$ и $\alpha_2=180^\circ$.
}
\newpage

\addpersonalvariant{Георгий Новиков}


\tasknumber{1}\task{
    С какой силой взаимодействуют 2 точечных заряда $q_1=4\units{нКл}$ и $q_2=3\units{нКл}$,
    находящиеся на расстоянии $l=3\units{см}$?
}
\answer{
    $
        F
            = k\frac{q_1q_2}{l^2}
            = 9 \cdot 10^9 \funits{Н $\cdot$ м$^2$}{Кл$^2$} \cdot \frac{
                4\cdot 10^{-9}\units{Кл}
                \cdot
                3\cdot 10^{-9}\units{Кл}
            }{
                \left(3 \cdot 10^{-2}\units{м}\right)^2
            }
            = \frac{12}{1}\cdot10^{-5}\units{Н}
              \approx {12.00}\cdot10^{-5}\units{Н}
    $
}
\vspace{120pt}

\tasknumber{2}\task{
    Два одинаковых маленьких проводящих заряженных шарика находятся
    на расстоянии~$r$ друг от друга.
    Заряд первого равен~$+2Q$, второго~---$-5Q$.
    Шарики приводят в соприкосновение, а после опять разводят на то же самое расстояние~$r$.
    Каким стал заряд каждого из шариков?
    Определите характер (притяжение или отталкивание)
    и силу взаимодействия шариков до и после соприкосновения.
}
\answer{
    \begin{align*}
        F   &= k\frac{q_1q_2}{r^2} = k\frac{(+2Q)\cdot(-5Q)}{r^2},
        \text{отталкивание};
        \\
        q'_1 = q'_2 = \frac{q_1 + q_2}2 = \frac{(+2Q) + (-5Q)}2 \implies
        F'  &= k\frac{q'_1q'_2}{r^2}
            = k\frac{
                    \left(\frac{(+2Q) + (-5Q)}2\right)^2
                }{
                    r^2
                },
        \text{отталкивание}.
    \end{align*}
}
\vspace{120pt}

\tasknumber{3}\task{
    На координатной плоскости в точках $(-r; 0)$ и $(r; 0)$
    находятся заряды, соответственно, $-Q$ и $+Q$.
    Сделайте рисунок, определите величину напряжённости электрического поля
    в точках $(0; -r)$ и $(2r; 0)$ и укажите её направление.
}
\vspace{120pt}

\tasknumber{4}\task{
    Заряд $q_1$ создает в точке $A$ электрическое поле
    по величине равное~$E_1=50\funits{В}{м}$,
    а $q_2$~---$E_2=120\funits{В}{м}$.
    Угол между векторами $\vect{E_1}$ и $\vect{E_2}$ равен $\alpha$.
    Определите величину суммарного электрического поля в точке $A$,
    создаваемого обоими зарядами $q_1$ и $q_2$.
    Сделайте рисунок и вычислите её значение для двух значений угла $\alpha$:
    $\alpha_1=0^\circ$ и $\alpha_2=90^\circ$.
}
\newpage

\addpersonalvariant{Егор Осипов}


\tasknumber{1}\task{
    С какой силой взаимодействуют 2 точечных заряда $q_1=4\units{нКл}$ и $q_2=2\units{нКл}$,
    находящиеся на расстоянии $l=6\units{см}$?
}
\answer{
    $
        F
            = k\frac{q_1q_2}{l^2}
            = 9 \cdot 10^9 \funits{Н $\cdot$ м$^2$}{Кл$^2$} \cdot \frac{
                4\cdot 10^{-9}\units{Кл}
                \cdot
                2\cdot 10^{-9}\units{Кл}
            }{
                \left(6 \cdot 10^{-2}\units{м}\right)^2
            }
            = \frac{2}{1}\cdot10^{-5}\units{Н}
              \approx {2.00}\cdot10^{-5}\units{Н}
    $
}
\vspace{120pt}

\tasknumber{2}\task{
    Два одинаковых маленьких проводящих заряженных шарика находятся
    на расстоянии~$r$ друг от друга.
    Заряд первого равен~$-5Q$, второго~---$-3Q$.
    Шарики приводят в соприкосновение, а после опять разводят на то же самое расстояние~$r$.
    Каким стал заряд каждого из шариков?
    Определите характер (притяжение или отталкивание)
    и силу взаимодействия шариков до и после соприкосновения.
}
\answer{
    \begin{align*}
        F   &= k\frac{q_1q_2}{r^2} = k\frac{(-5Q)\cdot(-3Q)}{r^2},
        \text{отталкивание};
        \\
        q'_1 = q'_2 = \frac{q_1 + q_2}2 = \frac{(-5Q) + (-3Q)}2 \implies
        F'  &= k\frac{q'_1q'_2}{r^2}
            = k\frac{
                    \left(\frac{(-5Q) + (-3Q)}2\right)^2
                }{
                    r^2
                },
        \text{отталкивание}.
    \end{align*}
}
\vspace{120pt}

\tasknumber{3}\task{
    На координатной плоскости в точках $(-d; 0)$ и $(d; 0)$
    находятся заряды, соответственно, $-q$ и $-q$.
    Сделайте рисунок, определите величину напряжённости электрического поля
    в точках $(0; -d)$ и $(2d; 0)$ и укажите её направление.
}
\vspace{120pt}

\tasknumber{4}\task{
    Заряд $q_1$ создает в точке $A$ электрическое поле
    по величине равное~$E_1=300\funits{В}{м}$,
    а $q_2$~---$E_2=400\funits{В}{м}$.
    Угол между векторами $\vect{E_1}$ и $\vect{E_2}$ равен $\varphi$.
    Определите величину суммарного электрического поля в точке $A$,
    создаваемого обоими зарядами $q_1$ и $q_2$.
    Сделайте рисунок и вычислите её значение для двух значений угла $\varphi$:
    $\varphi_1=0^\circ$ и $\varphi_2=90^\circ$.
}
\newpage

\addpersonalvariant{Руслан Перепелица}


\tasknumber{1}\task{
    С какой силой взаимодействуют 2 точечных заряда $q_1=2\units{нКл}$ и $q_2=3\units{нКл}$,
    находящиеся на расстоянии $d=2\units{см}$?
}
\answer{
    $
        F
            = k\frac{q_1q_2}{d^2}
            = 9 \cdot 10^9 \funits{Н $\cdot$ м$^2$}{Кл$^2$} \cdot \frac{
                2\cdot 10^{-9}\units{Кл}
                \cdot
                3\cdot 10^{-9}\units{Кл}
            }{
                \left(2 \cdot 10^{-2}\units{м}\right)^2
            }
            = \frac{27}{2}\cdot10^{-5}\units{Н}
              \approx {13.50}\cdot10^{-5}\units{Н}
    $
}
\vspace{120pt}

\tasknumber{2}\task{
    Два одинаковых маленьких проводящих заряженных шарика находятся
    на расстоянии~$l$ друг от друга.
    Заряд первого равен~$+2Q$, второго~---$-3Q$.
    Шарики приводят в соприкосновение, а после опять разводят на то же самое расстояние~$l$.
    Каким стал заряд каждого из шариков?
    Определите характер (притяжение или отталкивание)
    и силу взаимодействия шариков до и после соприкосновения.
}
\answer{
    \begin{align*}
        F   &= k\frac{q_1q_2}{l^2} = k\frac{(+2Q)\cdot(-3Q)}{l^2},
        \text{отталкивание};
        \\
        q'_1 = q'_2 = \frac{q_1 + q_2}2 = \frac{(+2Q) + (-3Q)}2 \implies
        F'  &= k\frac{q'_1q'_2}{l^2}
            = k\frac{
                    \left(\frac{(+2Q) + (-3Q)}2\right)^2
                }{
                    l^2
                },
        \text{отталкивание}.
    \end{align*}
}
\vspace{120pt}

\tasknumber{3}\task{
    На координатной плоскости в точках $(-r; 0)$ и $(r; 0)$
    находятся заряды, соответственно, $-Q$ и $+Q$.
    Сделайте рисунок, определите величину напряжённости электрического поля
    в точках $(0; r)$ и $(2r; 0)$ и укажите её направление.
}
\vspace{120pt}

\tasknumber{4}\task{
    Заряд $q_1$ создает в точке $A$ электрическое поле
    по величине равное~$E_1=50\funits{В}{м}$,
    а $q_2$~---$E_2=120\funits{В}{м}$.
    Угол между векторами $\vect{E_1}$ и $\vect{E_2}$ равен $\alpha$.
    Определите величину суммарного электрического поля в точке $A$,
    создаваемого обоими зарядами $q_1$ и $q_2$.
    Сделайте рисунок и вычислите её значение для двух значений угла $\alpha$:
    $\alpha_1=0^\circ$ и $\alpha_2=90^\circ$.
}
\newpage

\addpersonalvariant{Михаил Перин}


\tasknumber{1}\task{
    С какой силой взаимодействуют 2 точечных заряда $q_1=2\units{нКл}$ и $q_2=3\units{нКл}$,
    находящиеся на расстоянии $r=6\units{см}$?
}
\answer{
    $
        F
            = k\frac{q_1q_2}{r^2}
            = 9 \cdot 10^9 \funits{Н $\cdot$ м$^2$}{Кл$^2$} \cdot \frac{
                2\cdot 10^{-9}\units{Кл}
                \cdot
                3\cdot 10^{-9}\units{Кл}
            }{
                \left(6 \cdot 10^{-2}\units{м}\right)^2
            }
            = \frac{3}{2}\cdot10^{-5}\units{Н}
              \approx {1.50}\cdot10^{-5}\units{Н}
    $
}
\vspace{120pt}

\tasknumber{2}\task{
    Два одинаковых маленьких проводящих заряженных шарика находятся
    на расстоянии~$r$ друг от друга.
    Заряд первого равен~$+4q$, второго~---$-3q$.
    Шарики приводят в соприкосновение, а после опять разводят на то же самое расстояние~$r$.
    Каким стал заряд каждого из шариков?
    Определите характер (притяжение или отталкивание)
    и силу взаимодействия шариков до и после соприкосновения.
}
\answer{
    \begin{align*}
        F   &= k\frac{q_1q_2}{r^2} = k\frac{(+4q)\cdot(-3q)}{r^2},
        \text{отталкивание};
        \\
        q'_1 = q'_2 = \frac{q_1 + q_2}2 = \frac{(+4q) + (-3q)}2 \implies
        F'  &= k\frac{q'_1q'_2}{r^2}
            = k\frac{
                    \left(\frac{(+4q) + (-3q)}2\right)^2
                }{
                    r^2
                },
        \text{отталкивание}.
    \end{align*}
}
\vspace{120pt}

\tasknumber{3}\task{
    На координатной плоскости в точках $(-r; 0)$ и $(r; 0)$
    находятся заряды, соответственно, $-Q$ и $+Q$.
    Сделайте рисунок, определите величину напряжённости электрического поля
    в точках $(0; -r)$ и $(-2r; 0)$ и укажите её направление.
}
\vspace{120pt}

\tasknumber{4}\task{
    Заряд $q_1$ создает в точке $A$ электрическое поле
    по величине равное~$E_1=250\funits{В}{м}$,
    а $q_2$~---$E_2=250\funits{В}{м}$.
    Угол между векторами $\vect{E_1}$ и $\vect{E_2}$ равен $\alpha$.
    Определите величину суммарного электрического поля в точке $A$,
    создаваемого обоими зарядами $q_1$ и $q_2$.
    Сделайте рисунок и вычислите её значение для двух значений угла $\alpha$:
    $\alpha_1=0^\circ$ и $\alpha_2=60^\circ$.
}
\newpage

\addpersonalvariant{Егор Подуровский}


\tasknumber{1}\task{
    С какой силой взаимодействуют 2 точечных заряда $q_1=4\units{нКл}$ и $q_2=2\units{нКл}$,
    находящиеся на расстоянии $r=6\units{см}$?
}
\answer{
    $
        F
            = k\frac{q_1q_2}{r^2}
            = 9 \cdot 10^9 \funits{Н $\cdot$ м$^2$}{Кл$^2$} \cdot \frac{
                4\cdot 10^{-9}\units{Кл}
                \cdot
                2\cdot 10^{-9}\units{Кл}
            }{
                \left(6 \cdot 10^{-2}\units{м}\right)^2
            }
            = \frac{2}{1}\cdot10^{-5}\units{Н}
              \approx {2.00}\cdot10^{-5}\units{Н}
    $
}
\vspace{120pt}

\tasknumber{2}\task{
    Два одинаковых маленьких проводящих заряженных шарика находятся
    на расстоянии~$r$ друг от друга.
    Заряд первого равен~$+5Q$, второго~---$-3Q$.
    Шарики приводят в соприкосновение, а после опять разводят на то же самое расстояние~$r$.
    Каким стал заряд каждого из шариков?
    Определите характер (притяжение или отталкивание)
    и силу взаимодействия шариков до и после соприкосновения.
}
\answer{
    \begin{align*}
        F   &= k\frac{q_1q_2}{r^2} = k\frac{(+5Q)\cdot(-3Q)}{r^2},
        \text{отталкивание};
        \\
        q'_1 = q'_2 = \frac{q_1 + q_2}2 = \frac{(+5Q) + (-3Q)}2 \implies
        F'  &= k\frac{q'_1q'_2}{r^2}
            = k\frac{
                    \left(\frac{(+5Q) + (-3Q)}2\right)^2
                }{
                    r^2
                },
        \text{отталкивание}.
    \end{align*}
}
\vspace{120pt}

\tasknumber{3}\task{
    На координатной плоскости в точках $(-r; 0)$ и $(r; 0)$
    находятся заряды, соответственно, $-q$ и $-q$.
    Сделайте рисунок, определите величину напряжённости электрического поля
    в точках $(0; -r)$ и $(2r; 0)$ и укажите её направление.
}
\vspace{120pt}

\tasknumber{4}\task{
    Заряд $q_1$ создает в точке $A$ электрическое поле
    по величине равное~$E_1=50\funits{В}{м}$,
    а $q_2$~---$E_2=120\funits{В}{м}$.
    Угол между векторами $\vect{E_1}$ и $\vect{E_2}$ равен $\alpha$.
    Определите величину суммарного электрического поля в точке $A$,
    создаваемого обоими зарядами $q_1$ и $q_2$.
    Сделайте рисунок и вычислите её значение для двух значений угла $\alpha$:
    $\alpha_1=0^\circ$ и $\alpha_2=90^\circ$.
}
\newpage

\addpersonalvariant{Роман Прибылов}


\tasknumber{1}\task{
    С какой силой взаимодействуют 2 точечных заряда $q_1=2\units{нКл}$ и $q_2=3\units{нКл}$,
    находящиеся на расстоянии $d=5\units{см}$?
}
\answer{
    $
        F
            = k\frac{q_1q_2}{d^2}
            = 9 \cdot 10^9 \funits{Н $\cdot$ м$^2$}{Кл$^2$} \cdot \frac{
                2\cdot 10^{-9}\units{Кл}
                \cdot
                3\cdot 10^{-9}\units{Кл}
            }{
                \left(5 \cdot 10^{-2}\units{м}\right)^2
            }
            = \frac{54}{25}\cdot10^{-5}\units{Н}
              \approx {2.16}\cdot10^{-5}\units{Н}
    $
}
\vspace{120pt}

\tasknumber{2}\task{
    Два одинаковых маленьких проводящих заряженных шарика находятся
    на расстоянии~$r$ друг от друга.
    Заряд первого равен~$+3Q$, второго~---$-2Q$.
    Шарики приводят в соприкосновение, а после опять разводят на то же самое расстояние~$r$.
    Каким стал заряд каждого из шариков?
    Определите характер (притяжение или отталкивание)
    и силу взаимодействия шариков до и после соприкосновения.
}
\answer{
    \begin{align*}
        F   &= k\frac{q_1q_2}{r^2} = k\frac{(+3Q)\cdot(-2Q)}{r^2},
        \text{отталкивание};
        \\
        q'_1 = q'_2 = \frac{q_1 + q_2}2 = \frac{(+3Q) + (-2Q)}2 \implies
        F'  &= k\frac{q'_1q'_2}{r^2}
            = k\frac{
                    \left(\frac{(+3Q) + (-2Q)}2\right)^2
                }{
                    r^2
                },
        \text{отталкивание}.
    \end{align*}
}
\vspace{120pt}

\tasknumber{3}\task{
    На координатной плоскости в точках $(-r; 0)$ и $(r; 0)$
    находятся заряды, соответственно, $-q$ и $-q$.
    Сделайте рисунок, определите величину напряжённости электрического поля
    в точках $(0; -r)$ и $(-2r; 0)$ и укажите её направление.
}
\vspace{120pt}

\tasknumber{4}\task{
    Заряд $q_1$ создает в точке $A$ электрическое поле
    по величине равное~$E_1=200\funits{В}{м}$,
    а $q_2$~---$E_2=200\funits{В}{м}$.
    Угол между векторами $\vect{E_1}$ и $\vect{E_2}$ равен $\alpha$.
    Определите величину суммарного электрического поля в точке $A$,
    создаваемого обоими зарядами $q_1$ и $q_2$.
    Сделайте рисунок и вычислите её значение для двух значений угла $\alpha$:
    $\alpha_1=0^\circ$ и $\alpha_2=60^\circ$.
}
\newpage

\addpersonalvariant{Александр Селехметьев}


\tasknumber{1}\task{
    С какой силой взаимодействуют 2 точечных заряда $q_1=3\units{нКл}$ и $q_2=4\units{нКл}$,
    находящиеся на расстоянии $d=5\units{см}$?
}
\answer{
    $
        F
            = k\frac{q_1q_2}{d^2}
            = 9 \cdot 10^9 \funits{Н $\cdot$ м$^2$}{Кл$^2$} \cdot \frac{
                3\cdot 10^{-9}\units{Кл}
                \cdot
                4\cdot 10^{-9}\units{Кл}
            }{
                \left(5 \cdot 10^{-2}\units{м}\right)^2
            }
            = \frac{108}{25}\cdot10^{-5}\units{Н}
              \approx {4.32}\cdot10^{-5}\units{Н}
    $
}
\vspace{120pt}

\tasknumber{2}\task{
    Два одинаковых маленьких проводящих заряженных шарика находятся
    на расстоянии~$l$ друг от друга.
    Заряд первого равен~$-5q$, второго~---$-4q$.
    Шарики приводят в соприкосновение, а после опять разводят на то же самое расстояние~$l$.
    Каким стал заряд каждого из шариков?
    Определите характер (притяжение или отталкивание)
    и силу взаимодействия шариков до и после соприкосновения.
}
\answer{
    \begin{align*}
        F   &= k\frac{q_1q_2}{l^2} = k\frac{(-5q)\cdot(-4q)}{l^2},
        \text{отталкивание};
        \\
        q'_1 = q'_2 = \frac{q_1 + q_2}2 = \frac{(-5q) + (-4q)}2 \implies
        F'  &= k\frac{q'_1q'_2}{l^2}
            = k\frac{
                    \left(\frac{(-5q) + (-4q)}2\right)^2
                }{
                    l^2
                },
        \text{отталкивание}.
    \end{align*}
}
\vspace{120pt}

\tasknumber{3}\task{
    На координатной плоскости в точках $(-d; 0)$ и $(d; 0)$
    находятся заряды, соответственно, $+Q$ и $+Q$.
    Сделайте рисунок, определите величину напряжённости электрического поля
    в точках $(0; -d)$ и $(-2d; 0)$ и укажите её направление.
}
\vspace{120pt}

\tasknumber{4}\task{
    Заряд $q_1$ создает в точке $A$ электрическое поле
    по величине равное~$E_1=300\funits{В}{м}$,
    а $q_2$~---$E_2=400\funits{В}{м}$.
    Угол между векторами $\vect{E_1}$ и $\vect{E_2}$ равен $\varphi$.
    Определите величину суммарного электрического поля в точке $A$,
    создаваемого обоими зарядами $q_1$ и $q_2$.
    Сделайте рисунок и вычислите её значение для двух значений угла $\varphi$:
    $\varphi_1=0^\circ$ и $\varphi_2=90^\circ$.
}
\newpage

\addpersonalvariant{Алексей Тихонов}


\tasknumber{1}\task{
    С какой силой взаимодействуют 2 точечных заряда $q_1=3\units{нКл}$ и $q_2=2\units{нКл}$,
    находящиеся на расстоянии $r=3\units{см}$?
}
\answer{
    $
        F
            = k\frac{q_1q_2}{r^2}
            = 9 \cdot 10^9 \funits{Н $\cdot$ м$^2$}{Кл$^2$} \cdot \frac{
                3\cdot 10^{-9}\units{Кл}
                \cdot
                2\cdot 10^{-9}\units{Кл}
            }{
                \left(3 \cdot 10^{-2}\units{м}\right)^2
            }
            = \frac{6}{1}\cdot10^{-5}\units{Н}
              \approx {6.00}\cdot10^{-5}\units{Н}
    $
}
\vspace{120pt}

\tasknumber{2}\task{
    Два одинаковых маленьких проводящих заряженных шарика находятся
    на расстоянии~$l$ друг от друга.
    Заряд первого равен~$+4q$, второго~---$+2q$.
    Шарики приводят в соприкосновение, а после опять разводят на то же самое расстояние~$l$.
    Каким стал заряд каждого из шариков?
    Определите характер (притяжение или отталкивание)
    и силу взаимодействия шариков до и после соприкосновения.
}
\answer{
    \begin{align*}
        F   &= k\frac{q_1q_2}{l^2} = k\frac{(+4q)\cdot(+2q)}{l^2},
        \text{отталкивание};
        \\
        q'_1 = q'_2 = \frac{q_1 + q_2}2 = \frac{(+4q) + (+2q)}2 \implies
        F'  &= k\frac{q'_1q'_2}{l^2}
            = k\frac{
                    \left(\frac{(+4q) + (+2q)}2\right)^2
                }{
                    l^2
                },
        \text{отталкивание}.
    \end{align*}
}
\vspace{120pt}

\tasknumber{3}\task{
    На координатной плоскости в точках $(-a; 0)$ и $(a; 0)$
    находятся заряды, соответственно, $-q$ и $-q$.
    Сделайте рисунок, определите величину напряжённости электрического поля
    в точках $(0; -a)$ и $(2a; 0)$ и укажите её направление.
}
\vspace{120pt}

\tasknumber{4}\task{
    Заряд $q_1$ создает в точке $A$ электрическое поле
    по величине равное~$E_1=500\funits{В}{м}$,
    а $q_2$~---$E_2=500\funits{В}{м}$.
    Угол между векторами $\vect{E_1}$ и $\vect{E_2}$ равен $\alpha$.
    Определите величину суммарного электрического поля в точке $A$,
    создаваемого обоими зарядами $q_1$ и $q_2$.
    Сделайте рисунок и вычислите её значение для двух значений угла $\alpha$:
    $\alpha_1=0^\circ$ и $\alpha_2=120^\circ$.
}
\newpage

\addpersonalvariant{Алина Филиппова}


\tasknumber{1}\task{
    С какой силой взаимодействуют 2 точечных заряда $q_1=2\units{нКл}$ и $q_2=3\units{нКл}$,
    находящиеся на расстоянии $r=2\units{см}$?
}
\answer{
    $
        F
            = k\frac{q_1q_2}{r^2}
            = 9 \cdot 10^9 \funits{Н $\cdot$ м$^2$}{Кл$^2$} \cdot \frac{
                2\cdot 10^{-9}\units{Кл}
                \cdot
                3\cdot 10^{-9}\units{Кл}
            }{
                \left(2 \cdot 10^{-2}\units{м}\right)^2
            }
            = \frac{27}{2}\cdot10^{-5}\units{Н}
              \approx {13.50}\cdot10^{-5}\units{Н}
    $
}
\vspace{120pt}

\tasknumber{2}\task{
    Два одинаковых маленьких проводящих заряженных шарика находятся
    на расстоянии~$d$ друг от друга.
    Заряд первого равен~$+4q$, второго~---$+2q$.
    Шарики приводят в соприкосновение, а после опять разводят на то же самое расстояние~$d$.
    Каким стал заряд каждого из шариков?
    Определите характер (притяжение или отталкивание)
    и силу взаимодействия шариков до и после соприкосновения.
}
\answer{
    \begin{align*}
        F   &= k\frac{q_1q_2}{d^2} = k\frac{(+4q)\cdot(+2q)}{d^2},
        \text{отталкивание};
        \\
        q'_1 = q'_2 = \frac{q_1 + q_2}2 = \frac{(+4q) + (+2q)}2 \implies
        F'  &= k\frac{q'_1q'_2}{d^2}
            = k\frac{
                    \left(\frac{(+4q) + (+2q)}2\right)^2
                }{
                    d^2
                },
        \text{отталкивание}.
    \end{align*}
}
\vspace{120pt}

\tasknumber{3}\task{
    На координатной плоскости в точках $(-a; 0)$ и $(a; 0)$
    находятся заряды, соответственно, $+Q$ и $+Q$.
    Сделайте рисунок, определите величину напряжённости электрического поля
    в точках $(0; -a)$ и $(2a; 0)$ и укажите её направление.
}
\vspace{120pt}

\tasknumber{4}\task{
    Заряд $q_1$ создает в точке $A$ электрическое поле
    по величине равное~$E_1=120\funits{В}{м}$,
    а $q_2$~---$E_2=50\funits{В}{м}$.
    Угол между векторами $\vect{E_1}$ и $\vect{E_2}$ равен $\alpha$.
    Определите величину суммарного электрического поля в точке $A$,
    создаваемого обоими зарядами $q_1$ и $q_2$.
    Сделайте рисунок и вычислите её значение для двух значений угла $\alpha$:
    $\alpha_1=90^\circ$ и $\alpha_2=180^\circ$.
}
\newpage

\addpersonalvariant{Алина Яшина}


\tasknumber{1}\task{
    С какой силой взаимодействуют 2 точечных заряда $q_1=2\units{нКл}$ и $q_2=4\units{нКл}$,
    находящиеся на расстоянии $l=3\units{см}$?
}
\answer{
    $
        F
            = k\frac{q_1q_2}{l^2}
            = 9 \cdot 10^9 \funits{Н $\cdot$ м$^2$}{Кл$^2$} \cdot \frac{
                2\cdot 10^{-9}\units{Кл}
                \cdot
                4\cdot 10^{-9}\units{Кл}
            }{
                \left(3 \cdot 10^{-2}\units{м}\right)^2
            }
            = \frac{8}{1}\cdot10^{-5}\units{Н}
              \approx {8.00}\cdot10^{-5}\units{Н}
    $
}
\vspace{120pt}

\tasknumber{2}\task{
    Два одинаковых маленьких проводящих заряженных шарика находятся
    на расстоянии~$l$ друг от друга.
    Заряд первого равен~$+3q$, второго~---$-5q$.
    Шарики приводят в соприкосновение, а после опять разводят на то же самое расстояние~$l$.
    Каким стал заряд каждого из шариков?
    Определите характер (притяжение или отталкивание)
    и силу взаимодействия шариков до и после соприкосновения.
}
\answer{
    \begin{align*}
        F   &= k\frac{q_1q_2}{l^2} = k\frac{(+3q)\cdot(-5q)}{l^2},
        \text{отталкивание};
        \\
        q'_1 = q'_2 = \frac{q_1 + q_2}2 = \frac{(+3q) + (-5q)}2 \implies
        F'  &= k\frac{q'_1q'_2}{l^2}
            = k\frac{
                    \left(\frac{(+3q) + (-5q)}2\right)^2
                }{
                    l^2
                },
        \text{отталкивание}.
    \end{align*}
}
\vspace{120pt}

\tasknumber{3}\task{
    На координатной плоскости в точках $(-r; 0)$ и $(r; 0)$
    находятся заряды, соответственно, $-q$ и $-q$.
    Сделайте рисунок, определите величину напряжённости электрического поля
    в точках $(0; r)$ и $(-2r; 0)$ и укажите её направление.
}
\vspace{120pt}

\tasknumber{4}\task{
    Заряд $q_1$ создает в точке $A$ электрическое поле
    по величине равное~$E_1=50\funits{В}{м}$,
    а $q_2$~---$E_2=120\funits{В}{м}$.
    Угол между векторами $\vect{E_1}$ и $\vect{E_2}$ равен $\alpha$.
    Определите величину суммарного электрического поля в точке $A$,
    создаваемого обоими зарядами $q_1$ и $q_2$.
    Сделайте рисунок и вычислите её значение для двух значений угла $\alpha$:
    $\alpha_1=0^\circ$ и $\alpha_2=90^\circ$.
}

\end{document}
% autogenerated
