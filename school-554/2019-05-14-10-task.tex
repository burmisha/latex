\documentclass[12pt,a4paper]{amsart}%DVI-mode.
\usepackage{graphics,graphicx,epsfig}%DVI-mode.
% \documentclass[pdftex,12pt]{amsart} %PDF-mode.
% \usepackage[pdftex]{graphicx}       %PDF-mode.
% \usepackage[babel=true]{microtype}
% \usepackage[T1]{fontenc}
% \usepackage{lmodern}

\usepackage{cmap}
%\usepackage{a4wide}                 % Fit the text to A4 page tightly.
% \usepackage[utf8]{inputenc}
\usepackage[T2A]{fontenc}
\usepackage[english,russian]{babel} % Download Russian fonts.
\usepackage{amsmath,amsfonts,amssymb,amsthm,amscd,mathrsfs} % Use AMS symbols.
\usepackage{tikz}
\usetikzlibrary{circuits.ee.IEC}
\usetikzlibrary{shapes.geometric}
\usetikzlibrary{decorations.markings}
%\usetikzlibrary{dashs}
%\usetikzlibrary{info}


\textheight=28cm % высота текста
\textwidth=18cm % ширина текста
\topmargin=-2.5cm % отступ от верхнего края
\parskip=2pt % интервал между абзацами
\oddsidemargin=-1.5cm
\evensidemargin=-1.5cm 

\parindent=0pt % абзацный отступ
\tolerance=500 % терпимость к "жидким" строкам
\binoppenalty=10000 % штраф за перенос формул - 10000 - абсолютный запрет
\relpenalty=10000
\flushbottom % выравнивание высоты страниц
\pagenumbering{gobble}

\newcommand\bivec[2]{\begin{pmatrix} #1 \\ #2 \end{pmatrix}}

\newcommand\ol[1]{\overline{#1}}

\newcommand\p[1]{\Prob\!\left(#1\right)}
\newcommand\e[1]{\mathsf{E}\!\left(#1\right)}
\newcommand\disp[1]{\mathsf{D}\!\left(#1\right)}
%\newcommand\norm[2]{\mathcal{N}\!\cbr{#1,#2}}
\newcommand\sign{\text{ sign }}

\newcommand\al[1]{\begin{align*} #1 \end{align*}}
\newcommand\begcas[1]{\begin{cases}#1\end{cases}}
\newcommand\tab[2]{	\vspace{-#1pt}
						\begin{tabbing} 
						#2
						\end{tabbing}
					\vspace{-#1pt}
					}

\newcommand\maintext[1]{{\bfseries\sffamily{#1}}}
\newcommand\skipped[1]{ {\ensuremath{\text{\small{\sffamily{Пропущено:} #1} } } } }
\newcommand\simpletitle[1]{\begin{center} \maintext{#1} \end{center}}

\def\le{\leqslant}
\def\ge{\geqslant}
\def\Ell{\mathcal{L}}
\def\eps{{\varepsilon}}
\def\Rn{\mathbb{R}^n}
\def\RSS{\mathsf{RSS}}

\newcommand\foral[1]{\forall\,#1\:}
\newcommand\exist[1]{\exists\,#1\:\colon}

\newcommand\cbr[1]{\left(#1\right)} %circled brackets
\newcommand\fbr[1]{\left\{#1\right\}} %figure brackets
\newcommand\sbr[1]{\left[#1\right]} %square brackets
\newcommand\modul[1]{\left|#1\right|}

\newcommand\sqr[1]{\cbr{#1}^2}
\newcommand\inv[1]{\cbr{#1}^{-1}}

\newcommand\cdf[2]{\cdot\frac{#1}{#2}}
\newcommand\dd[2]{\frac{\partial#1}{\partial#2}}

\newcommand\integr[2]{\int\limits_{#1}^{#2}}
\newcommand\suml[2]{\sum\limits_{#1}^{#2}}
\newcommand\isum[2]{\sum\limits_{#1=#2}^{+\infty}}
\newcommand\idots[3]{#1_{#2},\ldots,#1_{#3}}
\newcommand\fdots[5]{#4{#1_{#2}}#5\ldots#5#4{#1_{#3}}}

\newcommand\obol[1]{O\!\cbr{#1}}
\newcommand\omal[1]{o\!\cbr{#1}}

\newcommand\addeps[2]{
	\begin{figure} [!ht] %lrp
		\centering
		\includegraphics[height=320px]{#1.eps}
		\vspace{-10pt}
		\caption{#2}
		\label{eps:#1}
	\end{figure}
}

\newcommand\addepssize[3]{
	\begin{figure} [!ht] %lrp hp
		\centering
		\includegraphics[height=#3px]{#1.eps}
		\vspace{-10pt}
		\caption{#2}
		\label{eps:#1}
	\end{figure}
}


\newcommand\norm[1]{\ensuremath{\left\|{#1}\right\|}}
\newcommand\ort{\bot}
\newcommand\theorem[1]{{\sffamily Теорема #1\ }}
\newcommand\lemma[1]{{\sffamily Лемма #1\ }}
\newcommand\difflim[2]{\frac{#1\cbr{#2 + \Delta#2} - #1\cbr{#2}}{\Delta #2}}
\renewcommand\proof[1]{\par\noindent$\square$ #1 \hfill$\blacksquare$\par}
\newcommand\defenition[1]{{\sffamilyОпределение #1\ }}

% \begin{document}
% %\raggedright
% \addclassdate{7}{20 апреля 2018}

\task 1
Площадь большого поршня гидравлического домкрата $S_1 = 20\units{см}^2$, а малого $S_2 = 0{,}5\units{см}^2.$ Груз какой максимальной массы можно поднять этим домкратом, если на малый поршень давить с силой не более $F=200\units{Н}?$ Силой трения от стенки цилиндров пренебречь.

\task 2
В сосуд налита вода. Расстояние от поверхности воды до дна $H = 0{,}5\units{м},$ площадь дна $S = 0{,}1\units{м}^2.$ Найти гидростатическое давление $P_1$ и полное давление $P_2$ вблизи дна. Найти силу давления воды на дно. Плотность воды \rhowater

\task 3
На лёгкий поршень площадью $S=900\units{см}^2,$ касающийся поверхности воды, поставили гирю массы $m=3\units{кг}$. Высота слоя воды в сосуде с вертикальными стенками $H = 20\units{см}$. Определить давление жидкости вблизи дна, если плотность воды \rhowater

\task 4
Давление газов в конце сгорания в цилиндре дизельного двигателя трактора $P = 9\units{МПа}.$ Диаметр цилиндра $d = 130\units{мм}.$ С какой силой газы давят на поршень в цилиндре? Площадь круга диаметром $D$ равна $S = \cfrac{\pi D^2}4.$

\task 5
Площадь малого поршня гидравлического подъёмника $S_1 = 0{,}8\units{см}^2$, а большого $S_2 = 40\units{см}^2.$ Какую силу $F$ надо приложить к малому поршню, чтобы поднять груз весом $P = 8\units{кН}?$

\task 6
Герметичный сосуд полностью заполнен водой и стоит на столе. На небольшой поршень площадью $S$ давят рукой с силой $F$. Поршень находится ниже крышки сосуда на $H_1$, выше дна на $H_2$ и может свободно перемещаться. Плотность воды $\rho$, атмосферное давление $P_A$. Найти давления $P_1$ и $P_2$ в воде вблизи крышки и дна сосуда.
\\ \\
\addclassdate{7}{20 апреля 2018}

\task 1
Площадь большого поршня гидравлического домкрата $S_1 = 20\units{см}^2$, а малого $S_2 = 0{,}5\units{см}^2.$ Груз какой максимальной массы можно поднять этим домкратом, если на малый поршень давить с силой не более $F=200\units{Н}?$ Силой трения от стенки цилиндров пренебречь.

\task 2
В сосуд налита вода. Расстояние от поверхности воды до дна $H = 0{,}5\units{м},$ площадь дна $S = 0{,}1\units{м}^2.$ Найти гидростатическое давление $P_1$ и полное давление $P_2$ вблизи дна. Найти силу давления воды на дно. Плотность воды \rhowater

\task 3
На лёгкий поршень площадью $S=900\units{см}^2,$ касающийся поверхности воды, поставили гирю массы $m=3\units{кг}$. Высота слоя воды в сосуде с вертикальными стенками $H = 20\units{см}$. Определить давление жидкости вблизи дна, если плотность воды \rhowater

\task 4
Давление газов в конце сгорания в цилиндре дизельного двигателя трактора $P = 9\units{МПа}.$ Диаметр цилиндра $d = 130\units{мм}.$ С какой силой газы давят на поршень в цилиндре? Площадь круга диаметром $D$ равна $S = \cfrac{\pi D^2}4.$

\task 5
Площадь малого поршня гидравлического подъёмника $S_1 = 0{,}8\units{см}^2$, а большого $S_2 = 40\units{см}^2.$ Какую силу $F$ надо приложить к малому поршню, чтобы поднять груз весом $P = 8\units{кН}?$

\task 6
Герметичный сосуд полностью заполнен водой и стоит на столе. На небольшой поршень площадью $S$ давят рукой с силой $F$. Поршень находится ниже крышки сосуда на $H_1$, выше дна на $H_2$ и может свободно перемещаться. Плотность воды $\rho$, атмосферное давление $P_A$. Найти давления $P_1$ и $P_2$ в воде вблизи крышки и дна сосуда.

\newpage

\adddate{8 класс. 20 апреля 2018}

\task 1
Между точками $A$ и $B$ электрической цепи подключены последовательно резисторы $R_1 = 10\units{Ом}$ и $R_2 = 20\units{Ом}$ и параллельно им $R_3 = 30\units{Ом}.$ Найдите эквивалентное сопротивление $R_{AB}$ этого участка цепи.

\task 2
Электрическая цепь состоит из последовательности $N$ одинаковых звеньев, в которых каждый резистор имеет сопротивление $r$. Последнее звено замкнуто резистором сопротивлением $R$. При каком соотношении $\cfrac{R}{r}$ сопротивление цепи не зависит от числа звеньев?

\task 3
Для измерения сопротивления $R$ проводника собрана электрическая цепь. Вольтметр $V$ показывает напряжение $U_V = 5\units{В},$ показание амперметра $A$ равно $I_A = 25\units{мА}.$ Найдите величину $R$ сопротивления проводника. Внутреннее сопротивление вольтметра $R_V = 1{,}0\units{кОм},$ внутреннее сопротивление амперметра $R_A = 2{,}0\units{Ом}.$

\task 4
Шкала гальванометра имеет $N=100$ делений, цена деления $\delta = 1\units{мкА}$. Внутреннее сопротивление гальванометра $R_G = 1{,}0\units{кОм}.$ Как из этого прибора сделать вольтметр для измерения напряжений до $U = 100\units{В}$ или амперметр для измерения токов силой до $I = 1\units{А}?$

\\ \\ \\ \\ \\ \\ \\ \\
\adddate{8 класс. 20 апреля 2018}

\task 1
Между точками $A$ и $B$ электрической цепи подключены последовательно резисторы $R_1 = 10\units{Ом}$ и $R_2 = 20\units{Ом}$ и параллельно им $R_3 = 30\units{Ом}.$ Найдите эквивалентное сопротивление $R_{AB}$ этого участка цепи.

\task 2
Электрическая цепь состоит из последовательности $N$ одинаковых звеньев, в которых каждый резистор имеет сопротивление $r$. Последнее звено замкнуто резистором сопротивлением $R$. При каком соотношении $\cfrac{R}{r}$ сопротивление цепи не зависит от числа звеньев?

\task 3
Для измерения сопротивления $R$ проводника собрана электрическая цепь. Вольтметр $V$ показывает напряжение $U_V = 5\units{В},$ показание амперметра $A$ равно $I_A = 25\units{мА}.$ Найдите величину $R$ сопротивления проводника. Внутреннее сопротивление вольтметра $R_V = 1{,}0\units{кОм},$ внутреннее сопротивление амперметра $R_A = 2{,}0\units{Ом}.$

\task 4
Шкала гальванометра имеет $N=100$ делений, цена деления $\delta = 1\units{мкА}$. Внутреннее сопротивление гальванометра $R_G = 1{,}0\units{кОм}.$ Как из этого прибора сделать вольтметр для измерения напряжений до $U = 100\units{В}$ или амперметр для измерения токов силой до $I = 1\units{А}?$


% % \begin{flushright}
\textsc{ГБОУ школа №554, 20 ноября 2018\,г.}
\end{flushright}

\begin{center}
\LARGE \textsc{Математический бой, 8 класс}
\end{center}

\problem{1} Есть тридцать карточек, на каждой написано по одному числу: на десяти карточках~–~$a$,  на десяти других~–~$b$ и на десяти оставшихся~–~$c$ (числа  различны). Известно, что к любым пяти карточкам можно подобрать ещё пять так, что сумма чисел на этих десяти карточках будет равна нулю. Докажите, что~одно из~чисел~$a, b, c$ равно нулю.

\problem{2} Вокруг стола стола пустили пакет с орешками. Первый взял один орешек, второй — 2, третий — 3 и так далее: каждый следующий брал на 1 орешек больше. Известно, что на втором круге было взято в сумме на 100 орешков больше, чем на первом. Сколько человек сидело за столом?

% \problem{2} Натуральное число разрешено увеличить на любое целое число процентов от 1 до 100, если при этом получаем натуральное число. Найдите наименьшее натуральное число, которое нельзя при помощи таких операций получить из~числа 1.

% \problem{3} Найти сумму $1^2 - 2^2 + 3^2 - 4^2 + 5^2 + \ldots - 2018^2$.

\problem{3} В кружке рукоделия, где занимается Валя, более 93\% участников~—~девочки. Какое наименьшее число участников может быть в таком кружке?

\problem{4} Произведение 2018 целых чисел равно 1. Может ли их сумма оказаться равной~0?

% \problem{4} Можно ли все натуральные числа от~1 до~9 записать в~клетки таблицы~$3\times3$ так, чтобы сумма в~любых двух соседних (по~вертикали или горизонтали) клетках равнялось простому числу?

\problem{5} На доске написано 2018 нулей и 2019 единиц. Женя стирает 2 числа и, если они были одинаковы, дописывает к оставшимся один ноль, а~если разные — единицу. Потом Женя повторяет эту операцию снова, потом ещё и~так далее. В~результате на~доске останется только одно число. Что это за~число?

\problem{6} Докажите, что в~любой компании людей найдутся 2~человека, имеющие равное число знакомых в этой компании (если $A$~знаком с~$B$, то~и $B$~знаком с~$A$).

\problem{7} Три колокола начинают бить одновременно. Интервалы между ударами колоколов соответственно составляют $\cfrac43$~секунды, $\cfrac53$~секунды и $2$~секунды. Совпавшие по времени удары воспринимаются за~один. Сколько ударов будет услышано за 1~минуту, включая первый и последний удары?

\problem{8} Восемь одинаковых момент расположены по кругу. Известно, что три из~них~— фальшивые, и они расположены рядом друг с~другом. Вес фальшивой монеты отличается от~веса настоящей. Все фальшивые монеты весят одинаково, но неизвестно, тяжелее или легче фальшивая монета настоящей. Покажите, что за~3~взвешивания на~чашечных весах без~гирь можно определить все фальшивые монеты.

% \end{document}

\begin{document}
\noanswers

\setdate{14~мая~2019}
\setclass{10}

\addpersonalvariant{Михаил Бурмистров}
\tasknumber{1}\task{
    На резистор сопротивлением $R=5\units{Ом}$ подали напряжение $V=150\units{В}$.
    Определите ток, который потечёт через резистор, и мощность, выделяющуюся на нём.
}
\answer{
    \begin{align*}
    \mathcal{I} &= \frac{ V }{ R } = \frac{ 150 \units{В} }{ 5 \units{Ом} } = 30{,}00 \units{А}, \\
    P &= \frac{ V^2 }{ R } = \frac{ \sqr{ 150 \units{В} } }{ 5 \units{Ом} } = 4500{,}00 \units{Вт}
    \end{align*}
}
\vspace{150pt}

\tasknumber{2}\task{
    Замкнутая электрическая цепь состоит из ЭДС $\mathcal{E}=20\units{В}$ и сопротивлением $r$
    и резистора $R=30\units{Ом}$.
    Определите ток, протекающий в цепи.
    Какая тепловая энергия выделится на резисторе за время
    $\tau=4\units{с}$? Какая работа будет совершена ЭДС за это время? Каков знак этой работы? Чему равен КПД цепи? Вычислите значения для 2 случаев:
    $r=0$ и $r=4\units{Ом}$.
}
\answer{
    \begin{align*}
    \mathcal{I}_1
                    &= \frac{ \mathcal{E} }{ R }
                    = \frac{ 20 \units{В} }{ 30 \units{Ом} }
                    = 0{,}67 \units{А}, \\
    \mathcal{I}_2
                    &= \frac{ \mathcal{E} }{ R + r }
                    = \frac{ 20 \units{В} }{ 30 \units{Ом} + 4 \units{Ом} }
                    = 0{,}59 \units{А}, \\
    Q_1
                    &= \mathcal{I}_1^2R\tau
                    = \cbr{ \frac{ \mathcal{E} }{ R } }^2 R \tau
                    = \cbr{ \frac{ 20 \units{В} }{ 30 \units{Ом} } }^2 \cdot 30 \units{Ом} \cdot 4 \units{с}
                    = 53{,}33 \units{Дж}, \\
    Q_2
                    &= \mathcal{I}_2^2R\tau
                    = \cbr{\frac{ \mathcal{E} }{ R + r } }^2 R \tau
                    = \cbr{\frac{ 20 \units{В} }{ 30 \units{Ом} + 4 \units{Ом} } }^2 \cdot 30 \units{Ом} \cdot 4 \units{с}
                    = 41{,}52 \units{Дж}, \\
    A_1
                    &= \mathcal{I}_1\tau\mathcal{E}
                    = \frac{ \mathcal{E} }{ R } \tau \mathcal{E}
                    = \frac{ \mathcal{E}^2 \tau }{ R }
                    = \frac{ \sqr{ 20 \units{В} } \cdot 4 \units{с} }{ 30 \units{Ом} }
                    = 53{,}33 \units{Дж}, \text{положительна}, \\
    A_2
                    &= \mathcal{I}_2\tau\mathcal{E}
                    = \frac{ \mathcal{E} }{ R + r } \tau \mathcal{E}
                    = \frac{ \mathcal{E}^2 \tau }{ R + r }
                    = \frac{ \sqr{ 20 \units{В} } \cdot 4 \units{с} }{ 30 \units{Ом} + 4 \units{Ом} }
                    = 47{,}06 \units{Дж}, \text{положительна}, \\
    \eta_1
                    &= \frac{ Q_1 }{ A_1 }
                    = \ldots
                    = \frac{ R }{ R }
                    = 1, \\
    \eta_2
                    &= \frac{ Q_2 }{ A_2 }
                    = \ldots
                    = \frac{ R }{ R + r }
                    = 0{,}882352941176
    \end{align*}
}
\vspace{150pt}

\tasknumber{3}\task{
    Лампочки, сопротивления которых $R_1=1\units{Ом}$ и $R_2=4\units{Ом}$, поочерёдно подключённные к некоторому источнику тока,
    потребляют одинаковую мощность.
    Найти внутреннее сопротивление источника и КПД цепи в каждом случае.
}
\answer{
    \begin{align*}
    P_1 &= \sqr{ \frac{ \mathcal{E} }{ R_1 + r } } R_1,
    P_2  = \sqr{ \frac{ \mathcal{E} }{ R_2 + r } } R_2,
    P_1 = P_2 \implies \\
    &\implies R_1 \sqr{ R_2 + r } = R_2 \sqr{ R_1 + r } \implies \\
    &\implies R_1 R_2^2 + 2 R_1 R_2 r + R_1 r^2 =
                R_2 R_1^2 + 2 R_2 R_1 r + R_2 r^2  \implies \\
    &\implies r^2 (R_2 - R_1) = R_2^2 R_2 - R_1^2 R_2 \implies \\
    &\implies r
        = \sqrt{ R_1 R_2 \frac{ R_2 - R_1 }{ R_2 - R_1 } }
        = \sqrt{ R_1 R_2 }
        = \sqrt{ 1 \units{Ом} \cdot 4 \units{Ом} }
        = 2{,}00 \units{Ом}.
        \\
     \eta_1
        &= \frac{ R_1 }{ R_1 + r }
        = \frac{ \sqrt{ R_1 } }{ \sqrt{ R_1 } + \sqrt{ R_2 } }
        = 0{,}333333333333, \\
     \eta_2
        &= \frac{ R_2 }{ R_2 + r }
        = \frac{ \sqrt{ R_2 } }{ \sqrt{ R_2 } + \sqrt{ R_1 } }
        = 0{,}666666666667
    \end{align*}
}
\vspace{150pt}

\tasknumber{4}\task{
    Определите ток, протекающий через резистор $R = 10\,\text{Ом}$ и разность потенциалов на нём (см.
    рис.
    на доске),
    если $r_1 = 2\,\text{Ом}$, $r_2 = 1\,\text{Ом}$, $\mathcal{E}_1 = 60\,\text{В}$, $\mathcal{E}_2 = 20\,\text{В}$
}
\newpage

\addpersonalvariant{Гагик Аракелян}
\tasknumber{1}\task{
    На резистор сопротивлением $R=12\units{Ом}$ подали напряжение $U=240\units{В}$.
    Определите ток, который потечёт через резистор, и мощность, выделяющуюся на нём.
}
\answer{
    \begin{align*}
    \mathcal{I} &= \frac{ U }{ R } = \frac{ 240 \units{В} }{ 12 \units{Ом} } = 20{,}00 \units{А}, \\
    P &= \frac{ U^2 }{ R } = \frac{ \sqr{ 240 \units{В} } }{ 12 \units{Ом} } = 4800{,}00 \units{Вт}
    \end{align*}
}
\vspace{150pt}

\tasknumber{2}\task{
    Замкнутая электрическая цепь состоит из ЭДС $\mathcal{E}=10\units{В}$ и сопротивлением $r$
    и резистора $R=10\units{Ом}$.
    Определите ток, протекающий в цепи.
    Какая тепловая энергия выделится на резисторе за время
    $\tau=1\units{с}$? Какая работа будет совершена ЭДС за это время? Каков знак этой работы? Чему равен КПД цепи? Вычислите значения для 2 случаев:
    $r=0$ и $r=1\units{Ом}$.
}
\answer{
    \begin{align*}
    \mathcal{I}_1
                    &= \frac{ \mathcal{E} }{ R }
                    = \frac{ 10 \units{В} }{ 10 \units{Ом} }
                    = 1{,}00 \units{А}, \\
    \mathcal{I}_2
                    &= \frac{ \mathcal{E} }{ R + r }
                    = \frac{ 10 \units{В} }{ 10 \units{Ом} + 1 \units{Ом} }
                    = 0{,}91 \units{А}, \\
    Q_1
                    &= \mathcal{I}_1^2R\tau
                    = \cbr{ \frac{ \mathcal{E} }{ R } }^2 R \tau
                    = \cbr{ \frac{ 10 \units{В} }{ 10 \units{Ом} } }^2 \cdot 10 \units{Ом} \cdot 1 \units{с}
                    = 10{,}00 \units{Дж}, \\
    Q_2
                    &= \mathcal{I}_2^2R\tau
                    = \cbr{\frac{ \mathcal{E} }{ R + r } }^2 R \tau
                    = \cbr{\frac{ 10 \units{В} }{ 10 \units{Ом} + 1 \units{Ом} } }^2 \cdot 10 \units{Ом} \cdot 1 \units{с}
                    = 8{,}26 \units{Дж}, \\
    A_1
                    &= \mathcal{I}_1\tau\mathcal{E}
                    = \frac{ \mathcal{E} }{ R } \tau \mathcal{E}
                    = \frac{ \mathcal{E}^2 \tau }{ R }
                    = \frac{ \sqr{ 10 \units{В} } \cdot 1 \units{с} }{ 10 \units{Ом} }
                    = 10{,}00 \units{Дж}, \text{положительна}, \\
    A_2
                    &= \mathcal{I}_2\tau\mathcal{E}
                    = \frac{ \mathcal{E} }{ R + r } \tau \mathcal{E}
                    = \frac{ \mathcal{E}^2 \tau }{ R + r }
                    = \frac{ \sqr{ 10 \units{В} } \cdot 1 \units{с} }{ 10 \units{Ом} + 1 \units{Ом} }
                    = 9{,}09 \units{Дж}, \text{положительна}, \\
    \eta_1
                    &= \frac{ Q_1 }{ A_1 }
                    = \ldots
                    = \frac{ R }{ R }
                    = 1, \\
    \eta_2
                    &= \frac{ Q_2 }{ A_2 }
                    = \ldots
                    = \frac{ R }{ R + r }
                    = 0{,}909090909091
    \end{align*}
}
\vspace{150pt}

\tasknumber{3}\task{
    Лампочки, сопротивления которых $R_1=0.5\units{Ом}$ и $R_2=4.5\units{Ом}$, поочерёдно подключённные к некоторому источнику тока,
    потребляют одинаковую мощность.
    Найти внутреннее сопротивление источника и КПД цепи в каждом случае.
}
\answer{
    \begin{align*}
    P_1 &= \sqr{ \frac{ \mathcal{E} }{ R_1 + r } } R_1,
    P_2  = \sqr{ \frac{ \mathcal{E} }{ R_2 + r } } R_2,
    P_1 = P_2 \implies \\
    &\implies R_1 \sqr{ R_2 + r } = R_2 \sqr{ R_1 + r } \implies \\
    &\implies R_1 R_2^2 + 2 R_1 R_2 r + R_1 r^2 =
                R_2 R_1^2 + 2 R_2 R_1 r + R_2 r^2  \implies \\
    &\implies r^2 (R_2 - R_1) = R_2^2 R_2 - R_1^2 R_2 \implies \\
    &\implies r
        = \sqrt{ R_1 R_2 \frac{ R_2 - R_1 }{ R_2 - R_1 } }
        = \sqrt{ R_1 R_2 }
        = \sqrt{ 0{,}50 \units{Ом} \cdot 4{,}50 \units{Ом} }
        = 1{,}50 \units{Ом}.
        \\
     \eta_1
        &= \frac{ R_1 }{ R_1 + r }
        = \frac{ \sqrt{ R_1 } }{ \sqrt{ R_1 } + \sqrt{ R_2 } }
        = 0{,}25, \\
     \eta_2
        &= \frac{ R_2 }{ R_2 + r }
        = \frac{ \sqrt{ R_2 } }{ \sqrt{ R_2 } + \sqrt{ R_1 } }
        = 0{,}75
    \end{align*}
}
\vspace{150pt}

\tasknumber{4}\task{
    Определите ток, протекающий через резистор $R = 15\,\text{Ом}$ и разность потенциалов на нём (см.
    рис.
    на доске),
    если $r_1 = 1\,\text{Ом}$, $r_2 = 3\,\text{Ом}$, $\mathcal{E}_1 = 30\,\text{В}$, $\mathcal{E}_2 = 60\,\text{В}$
}
\newpage

\addpersonalvariant{Ирен Аракелян}
\tasknumber{1}\task{
    На резистор сопротивлением $R=5\units{Ом}$ подали напряжение $V=180\units{В}$.
    Определите ток, который потечёт через резистор, и мощность, выделяющуюся на нём.
}
\answer{
    \begin{align*}
    \mathcal{I} &= \frac{ V }{ R } = \frac{ 180 \units{В} }{ 5 \units{Ом} } = 36{,}00 \units{А}, \\
    P &= \frac{ V^2 }{ R } = \frac{ \sqr{ 180 \units{В} } }{ 5 \units{Ом} } = 6480{,}00 \units{Вт}
    \end{align*}
}
\vspace{150pt}

\tasknumber{2}\task{
    Замкнутая электрическая цепь состоит из ЭДС $\mathcal{E}=20\units{В}$ и сопротивлением $r$
    и резистора $R=15\units{Ом}$.
    Определите ток, протекающий в цепи.
    Какая тепловая энергия выделится на резисторе за время
    $\tau=1\units{с}$? Какая работа будет совершена ЭДС за это время? Каков знак этой работы? Чему равен КПД цепи? Вычислите значения для 2 случаев:
    $r=0$ и $r=1\units{Ом}$.
}
\answer{
    \begin{align*}
    \mathcal{I}_1
                    &= \frac{ \mathcal{E} }{ R }
                    = \frac{ 20 \units{В} }{ 15 \units{Ом} }
                    = 1{,}33 \units{А}, \\
    \mathcal{I}_2
                    &= \frac{ \mathcal{E} }{ R + r }
                    = \frac{ 20 \units{В} }{ 15 \units{Ом} + 1 \units{Ом} }
                    = 1{,}25 \units{А}, \\
    Q_1
                    &= \mathcal{I}_1^2R\tau
                    = \cbr{ \frac{ \mathcal{E} }{ R } }^2 R \tau
                    = \cbr{ \frac{ 20 \units{В} }{ 15 \units{Ом} } }^2 \cdot 15 \units{Ом} \cdot 1 \units{с}
                    = 26{,}67 \units{Дж}, \\
    Q_2
                    &= \mathcal{I}_2^2R\tau
                    = \cbr{\frac{ \mathcal{E} }{ R + r } }^2 R \tau
                    = \cbr{\frac{ 20 \units{В} }{ 15 \units{Ом} + 1 \units{Ом} } }^2 \cdot 15 \units{Ом} \cdot 1 \units{с}
                    = 23{,}44 \units{Дж}, \\
    A_1
                    &= \mathcal{I}_1\tau\mathcal{E}
                    = \frac{ \mathcal{E} }{ R } \tau \mathcal{E}
                    = \frac{ \mathcal{E}^2 \tau }{ R }
                    = \frac{ \sqr{ 20 \units{В} } \cdot 1 \units{с} }{ 15 \units{Ом} }
                    = 26{,}67 \units{Дж}, \text{положительна}, \\
    A_2
                    &= \mathcal{I}_2\tau\mathcal{E}
                    = \frac{ \mathcal{E} }{ R + r } \tau \mathcal{E}
                    = \frac{ \mathcal{E}^2 \tau }{ R + r }
                    = \frac{ \sqr{ 20 \units{В} } \cdot 1 \units{с} }{ 15 \units{Ом} + 1 \units{Ом} }
                    = 25{,}00 \units{Дж}, \text{положительна}, \\
    \eta_1
                    &= \frac{ Q_1 }{ A_1 }
                    = \ldots
                    = \frac{ R }{ R }
                    = 1, \\
    \eta_2
                    &= \frac{ Q_2 }{ A_2 }
                    = \ldots
                    = \frac{ R }{ R + r }
                    = 0{,}9375
    \end{align*}
}
\vspace{150pt}

\tasknumber{3}\task{
    Лампочки, сопротивления которых $R_1=1\units{Ом}$ и $R_2=49\units{Ом}$, поочерёдно подключённные к некоторому источнику тока,
    потребляют одинаковую мощность.
    Найти внутреннее сопротивление источника и КПД цепи в каждом случае.
}
\answer{
    \begin{align*}
    P_1 &= \sqr{ \frac{ \mathcal{E} }{ R_1 + r } } R_1,
    P_2  = \sqr{ \frac{ \mathcal{E} }{ R_2 + r } } R_2,
    P_1 = P_2 \implies \\
    &\implies R_1 \sqr{ R_2 + r } = R_2 \sqr{ R_1 + r } \implies \\
    &\implies R_1 R_2^2 + 2 R_1 R_2 r + R_1 r^2 =
                R_2 R_1^2 + 2 R_2 R_1 r + R_2 r^2  \implies \\
    &\implies r^2 (R_2 - R_1) = R_2^2 R_2 - R_1^2 R_2 \implies \\
    &\implies r
        = \sqrt{ R_1 R_2 \frac{ R_2 - R_1 }{ R_2 - R_1 } }
        = \sqrt{ R_1 R_2 }
        = \sqrt{ 1 \units{Ом} \cdot 49 \units{Ом} }
        = 7{,}00 \units{Ом}.
        \\
     \eta_1
        &= \frac{ R_1 }{ R_1 + r }
        = \frac{ \sqrt{ R_1 } }{ \sqrt{ R_1 } + \sqrt{ R_2 } }
        = 0{,}125, \\
     \eta_2
        &= \frac{ R_2 }{ R_2 + r }
        = \frac{ \sqrt{ R_2 } }{ \sqrt{ R_2 } + \sqrt{ R_1 } }
        = 0{,}875
    \end{align*}
}
\vspace{150pt}

\tasknumber{4}\task{
    Определите ток, протекающий через резистор $R = 20\,\text{Ом}$ и разность потенциалов на нём (см.
    рис.
    на доске),
    если $r_1 = 3\,\text{Ом}$, $r_2 = 1\,\text{Ом}$, $\mathcal{E}_1 = 60\,\text{В}$, $\mathcal{E}_2 = 30\,\text{В}$
}
\newpage

\addpersonalvariant{Сабина Асадуллаева}
\tasknumber{1}\task{
    Через резистор сопротивлением $r=12\units{Ом}$ протекает электрический ток $\mathcal{I}=6\units{А}$.
    Определите, чему равны напряжение на резисторе и мощность, выделяющаяся на нём.
}
\answer{
    \begin{align*}
    U &= \mathcal{I}r = 6 \units{А} \cdot 12 \units{Ом} = 72 \units{В}, \\
    P &= \mathcal{I}^2r = \sqr{ 6 \units{А} } \cdot 12 \units{Ом} = 432 \units{Вт}
    \end{align*}
}
\vspace{150pt}

\tasknumber{2}\task{
    Замкнутая электрическая цепь состоит из ЭДС $\mathcal{E}=20\units{В}$ и сопротивлением $r$
    и резистора $R=15\units{Ом}$.
    Определите ток, протекающий в цепи.
    Какая тепловая энергия выделится на резисторе за время
    $\tau=4\units{с}$? Какая работа будет совершена ЭДС за это время? Каков знак этой работы? Чему равен КПД цепи? Вычислите значения для 2 случаев:
    $r=0$ и $r=4\units{Ом}$.
}
\answer{
    \begin{align*}
    \mathcal{I}_1
                    &= \frac{ \mathcal{E} }{ R }
                    = \frac{ 20 \units{В} }{ 15 \units{Ом} }
                    = 1{,}33 \units{А}, \\
    \mathcal{I}_2
                    &= \frac{ \mathcal{E} }{ R + r }
                    = \frac{ 20 \units{В} }{ 15 \units{Ом} + 4 \units{Ом} }
                    = 1{,}05 \units{А}, \\
    Q_1
                    &= \mathcal{I}_1^2R\tau
                    = \cbr{ \frac{ \mathcal{E} }{ R } }^2 R \tau
                    = \cbr{ \frac{ 20 \units{В} }{ 15 \units{Ом} } }^2 \cdot 15 \units{Ом} \cdot 4 \units{с}
                    = 106{,}67 \units{Дж}, \\
    Q_2
                    &= \mathcal{I}_2^2R\tau
                    = \cbr{\frac{ \mathcal{E} }{ R + r } }^2 R \tau
                    = \cbr{\frac{ 20 \units{В} }{ 15 \units{Ом} + 4 \units{Ом} } }^2 \cdot 15 \units{Ом} \cdot 4 \units{с}
                    = 66{,}48 \units{Дж}, \\
    A_1
                    &= \mathcal{I}_1\tau\mathcal{E}
                    = \frac{ \mathcal{E} }{ R } \tau \mathcal{E}
                    = \frac{ \mathcal{E}^2 \tau }{ R }
                    = \frac{ \sqr{ 20 \units{В} } \cdot 4 \units{с} }{ 15 \units{Ом} }
                    = 106{,}67 \units{Дж}, \text{положительна}, \\
    A_2
                    &= \mathcal{I}_2\tau\mathcal{E}
                    = \frac{ \mathcal{E} }{ R + r } \tau \mathcal{E}
                    = \frac{ \mathcal{E}^2 \tau }{ R + r }
                    = \frac{ \sqr{ 20 \units{В} } \cdot 4 \units{с} }{ 15 \units{Ом} + 4 \units{Ом} }
                    = 84{,}21 \units{Дж}, \text{положительна}, \\
    \eta_1
                    &= \frac{ Q_1 }{ A_1 }
                    = \ldots
                    = \frac{ R }{ R }
                    = 1, \\
    \eta_2
                    &= \frac{ Q_2 }{ A_2 }
                    = \ldots
                    = \frac{ R }{ R + r }
                    = 0{,}789473684211
    \end{align*}
}
\vspace{150pt}

\tasknumber{3}\task{
    Лампочки, сопротивления которых $R_1=1\units{Ом}$ и $R_2=4\units{Ом}$, поочерёдно подключённные к некоторому источнику тока,
    потребляют одинаковую мощность.
    Найти внутреннее сопротивление источника и КПД цепи в каждом случае.
}
\answer{
    \begin{align*}
    P_1 &= \sqr{ \frac{ \mathcal{E} }{ R_1 + r } } R_1,
    P_2  = \sqr{ \frac{ \mathcal{E} }{ R_2 + r } } R_2,
    P_1 = P_2 \implies \\
    &\implies R_1 \sqr{ R_2 + r } = R_2 \sqr{ R_1 + r } \implies \\
    &\implies R_1 R_2^2 + 2 R_1 R_2 r + R_1 r^2 =
                R_2 R_1^2 + 2 R_2 R_1 r + R_2 r^2  \implies \\
    &\implies r^2 (R_2 - R_1) = R_2^2 R_2 - R_1^2 R_2 \implies \\
    &\implies r
        = \sqrt{ R_1 R_2 \frac{ R_2 - R_1 }{ R_2 - R_1 } }
        = \sqrt{ R_1 R_2 }
        = \sqrt{ 1 \units{Ом} \cdot 4 \units{Ом} }
        = 2{,}00 \units{Ом}.
        \\
     \eta_1
        &= \frac{ R_1 }{ R_1 + r }
        = \frac{ \sqrt{ R_1 } }{ \sqrt{ R_1 } + \sqrt{ R_2 } }
        = 0{,}333333333333, \\
     \eta_2
        &= \frac{ R_2 }{ R_2 + r }
        = \frac{ \sqrt{ R_2 } }{ \sqrt{ R_2 } + \sqrt{ R_1 } }
        = 0{,}666666666667
    \end{align*}
}
\vspace{150pt}

\tasknumber{4}\task{
    Определите ток, протекающий через резистор $R = 18\,\text{Ом}$ и разность потенциалов на нём (см.
    рис.
    на доске),
    если $r_1 = 1\,\text{Ом}$, $r_2 = 3\,\text{Ом}$, $\mathcal{E}_1 = 20\,\text{В}$, $\mathcal{E}_2 = 40\,\text{В}$
}
\newpage

\addpersonalvariant{Вероника Битерякова}
\tasknumber{1}\task{
    На резистор сопротивлением $R=18\units{Ом}$ подали напряжение $U=240\units{В}$.
    Определите ток, который потечёт через резистор, и мощность, выделяющуюся на нём.
}
\answer{
    \begin{align*}
    \mathcal{I} &= \frac{ U }{ R } = \frac{ 240 \units{В} }{ 18 \units{Ом} } = 13{,}33 \units{А}, \\
    P &= \frac{ U^2 }{ R } = \frac{ \sqr{ 240 \units{В} } }{ 18 \units{Ом} } = 3200{,}00 \units{Вт}
    \end{align*}
}
\vspace{150pt}

\tasknumber{2}\task{
    Замкнутая электрическая цепь состоит из ЭДС $\mathcal{E}=10\units{В}$ и сопротивлением $r$
    и резистора $R=24\units{Ом}$.
    Определите ток, протекающий в цепи.
    Какая тепловая энергия выделится на резисторе за время
    $\tau=4\units{с}$? Какая работа будет совершена ЭДС за это время? Каков знак этой работы? Чему равен КПД цепи? Вычислите значения для 2 случаев:
    $r=0$ и $r=4\units{Ом}$.
}
\answer{
    \begin{align*}
    \mathcal{I}_1
                    &= \frac{ \mathcal{E} }{ R }
                    = \frac{ 10 \units{В} }{ 24 \units{Ом} }
                    = 0{,}42 \units{А}, \\
    \mathcal{I}_2
                    &= \frac{ \mathcal{E} }{ R + r }
                    = \frac{ 10 \units{В} }{ 24 \units{Ом} + 4 \units{Ом} }
                    = 0{,}36 \units{А}, \\
    Q_1
                    &= \mathcal{I}_1^2R\tau
                    = \cbr{ \frac{ \mathcal{E} }{ R } }^2 R \tau
                    = \cbr{ \frac{ 10 \units{В} }{ 24 \units{Ом} } }^2 \cdot 24 \units{Ом} \cdot 4 \units{с}
                    = 16{,}67 \units{Дж}, \\
    Q_2
                    &= \mathcal{I}_2^2R\tau
                    = \cbr{\frac{ \mathcal{E} }{ R + r } }^2 R \tau
                    = \cbr{\frac{ 10 \units{В} }{ 24 \units{Ом} + 4 \units{Ом} } }^2 \cdot 24 \units{Ом} \cdot 4 \units{с}
                    = 12{,}24 \units{Дж}, \\
    A_1
                    &= \mathcal{I}_1\tau\mathcal{E}
                    = \frac{ \mathcal{E} }{ R } \tau \mathcal{E}
                    = \frac{ \mathcal{E}^2 \tau }{ R }
                    = \frac{ \sqr{ 10 \units{В} } \cdot 4 \units{с} }{ 24 \units{Ом} }
                    = 16{,}67 \units{Дж}, \text{положительна}, \\
    A_2
                    &= \mathcal{I}_2\tau\mathcal{E}
                    = \frac{ \mathcal{E} }{ R + r } \tau \mathcal{E}
                    = \frac{ \mathcal{E}^2 \tau }{ R + r }
                    = \frac{ \sqr{ 10 \units{В} } \cdot 4 \units{с} }{ 24 \units{Ом} + 4 \units{Ом} }
                    = 14{,}29 \units{Дж}, \text{положительна}, \\
    \eta_1
                    &= \frac{ Q_1 }{ A_1 }
                    = \ldots
                    = \frac{ R }{ R }
                    = 1, \\
    \eta_2
                    &= \frac{ Q_2 }{ A_2 }
                    = \ldots
                    = \frac{ R }{ R + r }
                    = 0{,}857142857143
    \end{align*}
}
\vspace{150pt}

\tasknumber{3}\task{
    Лампочки, сопротивления которых $R_1=0.5\units{Ом}$ и $R_2=2\units{Ом}$, поочерёдно подключённные к некоторому источнику тока,
    потребляют одинаковую мощность.
    Найти внутреннее сопротивление источника и КПД цепи в каждом случае.
}
\answer{
    \begin{align*}
    P_1 &= \sqr{ \frac{ \mathcal{E} }{ R_1 + r } } R_1,
    P_2  = \sqr{ \frac{ \mathcal{E} }{ R_2 + r } } R_2,
    P_1 = P_2 \implies \\
    &\implies R_1 \sqr{ R_2 + r } = R_2 \sqr{ R_1 + r } \implies \\
    &\implies R_1 R_2^2 + 2 R_1 R_2 r + R_1 r^2 =
                R_2 R_1^2 + 2 R_2 R_1 r + R_2 r^2  \implies \\
    &\implies r^2 (R_2 - R_1) = R_2^2 R_2 - R_1^2 R_2 \implies \\
    &\implies r
        = \sqrt{ R_1 R_2 \frac{ R_2 - R_1 }{ R_2 - R_1 } }
        = \sqrt{ R_1 R_2 }
        = \sqrt{ 0{,}50 \units{Ом} \cdot 2 \units{Ом} }
        = 1{,}00 \units{Ом}.
        \\
     \eta_1
        &= \frac{ R_1 }{ R_1 + r }
        = \frac{ \sqrt{ R_1 } }{ \sqrt{ R_1 } + \sqrt{ R_2 } }
        = 0{,}333333333333, \\
     \eta_2
        &= \frac{ R_2 }{ R_2 + r }
        = \frac{ \sqrt{ R_2 } }{ \sqrt{ R_2 } + \sqrt{ R_1 } }
        = 0{,}666666666667
    \end{align*}
}
\vspace{150pt}

\tasknumber{4}\task{
    Определите ток, протекающий через резистор $R = 15\,\text{Ом}$ и разность потенциалов на нём (см.
    рис.
    на доске),
    если $r_1 = 2\,\text{Ом}$, $r_2 = 3\,\text{Ом}$, $\mathcal{E}_1 = 40\,\text{В}$, $\mathcal{E}_2 = 30\,\text{В}$
}
\newpage

\addpersonalvariant{Юлия Буянова}
\tasknumber{1}\task{
    Через резистор сопротивлением $R=30\units{Ом}$ протекает электрический ток $\mathcal{I}=5\units{А}$.
    Определите, чему равны напряжение на резисторе и мощность, выделяющаяся на нём.
}
\answer{
    \begin{align*}
    U &= \mathcal{I}R = 5 \units{А} \cdot 30 \units{Ом} = 150 \units{В}, \\
    P &= \mathcal{I}^2R = \sqr{ 5 \units{А} } \cdot 30 \units{Ом} = 750 \units{Вт}
    \end{align*}
}
\vspace{150pt}

\tasknumber{2}\task{
    Замкнутая электрическая цепь состоит из ЭДС $\mathcal{E}=20\units{В}$ и сопротивлением $r$
    и резистора $R=24\units{Ом}$.
    Определите ток, протекающий в цепи.
    Какая тепловая энергия выделится на резисторе за время
    $\tau=3\units{с}$? Какая работа будет совершена ЭДС за это время? Каков знак этой работы? Чему равен КПД цепи? Вычислите значения для 2 случаев:
    $r=0$ и $r=3\units{Ом}$.
}
\answer{
    \begin{align*}
    \mathcal{I}_1
                    &= \frac{ \mathcal{E} }{ R }
                    = \frac{ 20 \units{В} }{ 24 \units{Ом} }
                    = 0{,}83 \units{А}, \\
    \mathcal{I}_2
                    &= \frac{ \mathcal{E} }{ R + r }
                    = \frac{ 20 \units{В} }{ 24 \units{Ом} + 3 \units{Ом} }
                    = 0{,}74 \units{А}, \\
    Q_1
                    &= \mathcal{I}_1^2R\tau
                    = \cbr{ \frac{ \mathcal{E} }{ R } }^2 R \tau
                    = \cbr{ \frac{ 20 \units{В} }{ 24 \units{Ом} } }^2 \cdot 24 \units{Ом} \cdot 3 \units{с}
                    = 50{,}00 \units{Дж}, \\
    Q_2
                    &= \mathcal{I}_2^2R\tau
                    = \cbr{\frac{ \mathcal{E} }{ R + r } }^2 R \tau
                    = \cbr{\frac{ 20 \units{В} }{ 24 \units{Ом} + 3 \units{Ом} } }^2 \cdot 24 \units{Ом} \cdot 3 \units{с}
                    = 39{,}51 \units{Дж}, \\
    A_1
                    &= \mathcal{I}_1\tau\mathcal{E}
                    = \frac{ \mathcal{E} }{ R } \tau \mathcal{E}
                    = \frac{ \mathcal{E}^2 \tau }{ R }
                    = \frac{ \sqr{ 20 \units{В} } \cdot 3 \units{с} }{ 24 \units{Ом} }
                    = 50{,}00 \units{Дж}, \text{положительна}, \\
    A_2
                    &= \mathcal{I}_2\tau\mathcal{E}
                    = \frac{ \mathcal{E} }{ R + r } \tau \mathcal{E}
                    = \frac{ \mathcal{E}^2 \tau }{ R + r }
                    = \frac{ \sqr{ 20 \units{В} } \cdot 3 \units{с} }{ 24 \units{Ом} + 3 \units{Ом} }
                    = 44{,}44 \units{Дж}, \text{положительна}, \\
    \eta_1
                    &= \frac{ Q_1 }{ A_1 }
                    = \ldots
                    = \frac{ R }{ R }
                    = 1, \\
    \eta_2
                    &= \frac{ Q_2 }{ A_2 }
                    = \ldots
                    = \frac{ R }{ R + r }
                    = 0{,}888888888889
    \end{align*}
}
\vspace{150pt}

\tasknumber{3}\task{
    Лампочки, сопротивления которых $R_1=3\units{Ом}$ и $R_2=12\units{Ом}$, поочерёдно подключённные к некоторому источнику тока,
    потребляют одинаковую мощность.
    Найти внутреннее сопротивление источника и КПД цепи в каждом случае.
}
\answer{
    \begin{align*}
    P_1 &= \sqr{ \frac{ \mathcal{E} }{ R_1 + r } } R_1,
    P_2  = \sqr{ \frac{ \mathcal{E} }{ R_2 + r } } R_2,
    P_1 = P_2 \implies \\
    &\implies R_1 \sqr{ R_2 + r } = R_2 \sqr{ R_1 + r } \implies \\
    &\implies R_1 R_2^2 + 2 R_1 R_2 r + R_1 r^2 =
                R_2 R_1^2 + 2 R_2 R_1 r + R_2 r^2  \implies \\
    &\implies r^2 (R_2 - R_1) = R_2^2 R_2 - R_1^2 R_2 \implies \\
    &\implies r
        = \sqrt{ R_1 R_2 \frac{ R_2 - R_1 }{ R_2 - R_1 } }
        = \sqrt{ R_1 R_2 }
        = \sqrt{ 3 \units{Ом} \cdot 12 \units{Ом} }
        = 6{,}00 \units{Ом}.
        \\
     \eta_1
        &= \frac{ R_1 }{ R_1 + r }
        = \frac{ \sqrt{ R_1 } }{ \sqrt{ R_1 } + \sqrt{ R_2 } }
        = 0{,}333333333333, \\
     \eta_2
        &= \frac{ R_2 }{ R_2 + r }
        = \frac{ \sqrt{ R_2 } }{ \sqrt{ R_2 } + \sqrt{ R_1 } }
        = 0{,}666666666667
    \end{align*}
}
\vspace{150pt}

\tasknumber{4}\task{
    Определите ток, протекающий через резистор $R = 10\,\text{Ом}$ и разность потенциалов на нём (см.
    рис.
    на доске),
    если $r_1 = 3\,\text{Ом}$, $r_2 = 1\,\text{Ом}$, $\mathcal{E}_1 = 60\,\text{В}$, $\mathcal{E}_2 = 40\,\text{В}$
}
\newpage

\addpersonalvariant{Пелагея Вдовина}
\tasknumber{1}\task{
    Через резистор сопротивлением $r=12\units{Ом}$ протекает электрический ток $\mathcal{I}=5\units{А}$.
    Определите, чему равны напряжение на резисторе и мощность, выделяющаяся на нём.
}
\answer{
    \begin{align*}
    U &= \mathcal{I}r = 5 \units{А} \cdot 12 \units{Ом} = 60 \units{В}, \\
    P &= \mathcal{I}^2r = \sqr{ 5 \units{А} } \cdot 12 \units{Ом} = 300 \units{Вт}
    \end{align*}
}
\vspace{150pt}

\tasknumber{2}\task{
    Замкнутая электрическая цепь состоит из ЭДС $\mathcal{E}=30\units{В}$ и сопротивлением $r$
    и резистора $R=15\units{Ом}$.
    Определите ток, протекающий в цепи.
    Какая тепловая энергия выделится на резисторе за время
    $\tau=1\units{с}$? Какая работа будет совершена ЭДС за это время? Каков знак этой работы? Чему равен КПД цепи? Вычислите значения для 2 случаев:
    $r=0$ и $r=1\units{Ом}$.
}
\answer{
    \begin{align*}
    \mathcal{I}_1
                    &= \frac{ \mathcal{E} }{ R }
                    = \frac{ 30 \units{В} }{ 15 \units{Ом} }
                    = 2{,}00 \units{А}, \\
    \mathcal{I}_2
                    &= \frac{ \mathcal{E} }{ R + r }
                    = \frac{ 30 \units{В} }{ 15 \units{Ом} + 1 \units{Ом} }
                    = 1{,}88 \units{А}, \\
    Q_1
                    &= \mathcal{I}_1^2R\tau
                    = \cbr{ \frac{ \mathcal{E} }{ R } }^2 R \tau
                    = \cbr{ \frac{ 30 \units{В} }{ 15 \units{Ом} } }^2 \cdot 15 \units{Ом} \cdot 1 \units{с}
                    = 60{,}00 \units{Дж}, \\
    Q_2
                    &= \mathcal{I}_2^2R\tau
                    = \cbr{\frac{ \mathcal{E} }{ R + r } }^2 R \tau
                    = \cbr{\frac{ 30 \units{В} }{ 15 \units{Ом} + 1 \units{Ом} } }^2 \cdot 15 \units{Ом} \cdot 1 \units{с}
                    = 52{,}73 \units{Дж}, \\
    A_1
                    &= \mathcal{I}_1\tau\mathcal{E}
                    = \frac{ \mathcal{E} }{ R } \tau \mathcal{E}
                    = \frac{ \mathcal{E}^2 \tau }{ R }
                    = \frac{ \sqr{ 30 \units{В} } \cdot 1 \units{с} }{ 15 \units{Ом} }
                    = 60{,}00 \units{Дж}, \text{положительна}, \\
    A_2
                    &= \mathcal{I}_2\tau\mathcal{E}
                    = \frac{ \mathcal{E} }{ R + r } \tau \mathcal{E}
                    = \frac{ \mathcal{E}^2 \tau }{ R + r }
                    = \frac{ \sqr{ 30 \units{В} } \cdot 1 \units{с} }{ 15 \units{Ом} + 1 \units{Ом} }
                    = 56{,}25 \units{Дж}, \text{положительна}, \\
    \eta_1
                    &= \frac{ Q_1 }{ A_1 }
                    = \ldots
                    = \frac{ R }{ R }
                    = 1, \\
    \eta_2
                    &= \frac{ Q_2 }{ A_2 }
                    = \ldots
                    = \frac{ R }{ R + r }
                    = 0{,}9375
    \end{align*}
}
\vspace{150pt}

\tasknumber{3}\task{
    Лампочки, сопротивления которых $R_1=6\units{Ом}$ и $R_2=24\units{Ом}$, поочерёдно подключённные к некоторому источнику тока,
    потребляют одинаковую мощность.
    Найти внутреннее сопротивление источника и КПД цепи в каждом случае.
}
\answer{
    \begin{align*}
    P_1 &= \sqr{ \frac{ \mathcal{E} }{ R_1 + r } } R_1,
    P_2  = \sqr{ \frac{ \mathcal{E} }{ R_2 + r } } R_2,
    P_1 = P_2 \implies \\
    &\implies R_1 \sqr{ R_2 + r } = R_2 \sqr{ R_1 + r } \implies \\
    &\implies R_1 R_2^2 + 2 R_1 R_2 r + R_1 r^2 =
                R_2 R_1^2 + 2 R_2 R_1 r + R_2 r^2  \implies \\
    &\implies r^2 (R_2 - R_1) = R_2^2 R_2 - R_1^2 R_2 \implies \\
    &\implies r
        = \sqrt{ R_1 R_2 \frac{ R_2 - R_1 }{ R_2 - R_1 } }
        = \sqrt{ R_1 R_2 }
        = \sqrt{ 6 \units{Ом} \cdot 24 \units{Ом} }
        = 12{,}00 \units{Ом}.
        \\
     \eta_1
        &= \frac{ R_1 }{ R_1 + r }
        = \frac{ \sqrt{ R_1 } }{ \sqrt{ R_1 } + \sqrt{ R_2 } }
        = 0{,}333333333333, \\
     \eta_2
        &= \frac{ R_2 }{ R_2 + r }
        = \frac{ \sqrt{ R_2 } }{ \sqrt{ R_2 } + \sqrt{ R_1 } }
        = 0{,}666666666667
    \end{align*}
}
\vspace{150pt}

\tasknumber{4}\task{
    Определите ток, протекающий через резистор $R = 10\,\text{Ом}$ и разность потенциалов на нём (см.
    рис.
    на доске),
    если $r_1 = 3\,\text{Ом}$, $r_2 = 1\,\text{Ом}$, $\mathcal{E}_1 = 20\,\text{В}$, $\mathcal{E}_2 = 40\,\text{В}$
}
\newpage

\addpersonalvariant{Леонид Викторов}
\tasknumber{1}\task{
    На резистор сопротивлением $R=18\units{Ом}$ подали напряжение $V=180\units{В}$.
    Определите ток, который потечёт через резистор, и мощность, выделяющуюся на нём.
}
\answer{
    \begin{align*}
    \mathcal{I} &= \frac{ V }{ R } = \frac{ 180 \units{В} }{ 18 \units{Ом} } = 10{,}00 \units{А}, \\
    P &= \frac{ V^2 }{ R } = \frac{ \sqr{ 180 \units{В} } }{ 18 \units{Ом} } = 1800{,}00 \units{Вт}
    \end{align*}
}
\vspace{150pt}

\tasknumber{2}\task{
    Замкнутая электрическая цепь состоит из ЭДС $\mathcal{E}=20\units{В}$ и сопротивлением $r$
    и резистора $R=15\units{Ом}$.
    Определите ток, протекающий в цепи.
    Какая тепловая энергия выделится на резисторе за время
    $\tau=3\units{с}$? Какая работа будет совершена ЭДС за это время? Каков знак этой работы? Чему равен КПД цепи? Вычислите значения для 2 случаев:
    $r=0$ и $r=3\units{Ом}$.
}
\answer{
    \begin{align*}
    \mathcal{I}_1
                    &= \frac{ \mathcal{E} }{ R }
                    = \frac{ 20 \units{В} }{ 15 \units{Ом} }
                    = 1{,}33 \units{А}, \\
    \mathcal{I}_2
                    &= \frac{ \mathcal{E} }{ R + r }
                    = \frac{ 20 \units{В} }{ 15 \units{Ом} + 3 \units{Ом} }
                    = 1{,}11 \units{А}, \\
    Q_1
                    &= \mathcal{I}_1^2R\tau
                    = \cbr{ \frac{ \mathcal{E} }{ R } }^2 R \tau
                    = \cbr{ \frac{ 20 \units{В} }{ 15 \units{Ом} } }^2 \cdot 15 \units{Ом} \cdot 3 \units{с}
                    = 80{,}00 \units{Дж}, \\
    Q_2
                    &= \mathcal{I}_2^2R\tau
                    = \cbr{\frac{ \mathcal{E} }{ R + r } }^2 R \tau
                    = \cbr{\frac{ 20 \units{В} }{ 15 \units{Ом} + 3 \units{Ом} } }^2 \cdot 15 \units{Ом} \cdot 3 \units{с}
                    = 55{,}56 \units{Дж}, \\
    A_1
                    &= \mathcal{I}_1\tau\mathcal{E}
                    = \frac{ \mathcal{E} }{ R } \tau \mathcal{E}
                    = \frac{ \mathcal{E}^2 \tau }{ R }
                    = \frac{ \sqr{ 20 \units{В} } \cdot 3 \units{с} }{ 15 \units{Ом} }
                    = 80{,}00 \units{Дж}, \text{положительна}, \\
    A_2
                    &= \mathcal{I}_2\tau\mathcal{E}
                    = \frac{ \mathcal{E} }{ R + r } \tau \mathcal{E}
                    = \frac{ \mathcal{E}^2 \tau }{ R + r }
                    = \frac{ \sqr{ 20 \units{В} } \cdot 3 \units{с} }{ 15 \units{Ом} + 3 \units{Ом} }
                    = 66{,}67 \units{Дж}, \text{положительна}, \\
    \eta_1
                    &= \frac{ Q_1 }{ A_1 }
                    = \ldots
                    = \frac{ R }{ R }
                    = 1, \\
    \eta_2
                    &= \frac{ Q_2 }{ A_2 }
                    = \ldots
                    = \frac{ R }{ R + r }
                    = 0{,}833333333333
    \end{align*}
}
\vspace{150pt}

\tasknumber{3}\task{
    Лампочки, сопротивления которых $R_1=5\units{Ом}$ и $R_2=80\units{Ом}$, поочерёдно подключённные к некоторому источнику тока,
    потребляют одинаковую мощность.
    Найти внутреннее сопротивление источника и КПД цепи в каждом случае.
}
\answer{
    \begin{align*}
    P_1 &= \sqr{ \frac{ \mathcal{E} }{ R_1 + r } } R_1,
    P_2  = \sqr{ \frac{ \mathcal{E} }{ R_2 + r } } R_2,
    P_1 = P_2 \implies \\
    &\implies R_1 \sqr{ R_2 + r } = R_2 \sqr{ R_1 + r } \implies \\
    &\implies R_1 R_2^2 + 2 R_1 R_2 r + R_1 r^2 =
                R_2 R_1^2 + 2 R_2 R_1 r + R_2 r^2  \implies \\
    &\implies r^2 (R_2 - R_1) = R_2^2 R_2 - R_1^2 R_2 \implies \\
    &\implies r
        = \sqrt{ R_1 R_2 \frac{ R_2 - R_1 }{ R_2 - R_1 } }
        = \sqrt{ R_1 R_2 }
        = \sqrt{ 5 \units{Ом} \cdot 80 \units{Ом} }
        = 20{,}00 \units{Ом}.
        \\
     \eta_1
        &= \frac{ R_1 }{ R_1 + r }
        = \frac{ \sqrt{ R_1 } }{ \sqrt{ R_1 } + \sqrt{ R_2 } }
        = 0{,}2, \\
     \eta_2
        &= \frac{ R_2 }{ R_2 + r }
        = \frac{ \sqrt{ R_2 } }{ \sqrt{ R_2 } + \sqrt{ R_1 } }
        = 0{,}8
    \end{align*}
}
\vspace{150pt}

\tasknumber{4}\task{
    Определите ток, протекающий через резистор $R = 20\,\text{Ом}$ и разность потенциалов на нём (см.
    рис.
    на доске),
    если $r_1 = 1\,\text{Ом}$, $r_2 = 2\,\text{Ом}$, $\mathcal{E}_1 = 60\,\text{В}$, $\mathcal{E}_2 = 30\,\text{В}$
}
\newpage

\addpersonalvariant{Фёдор Гнутов}
\tasknumber{1}\task{
    На резистор сопротивлением $r=30\units{Ом}$ подали напряжение $V=120\units{В}$.
    Определите ток, который потечёт через резистор, и мощность, выделяющуюся на нём.
}
\answer{
    \begin{align*}
    \mathcal{I} &= \frac{ V }{ r } = \frac{ 120 \units{В} }{ 30 \units{Ом} } = 4{,}00 \units{А}, \\
    P &= \frac{ V^2 }{ r } = \frac{ \sqr{ 120 \units{В} } }{ 30 \units{Ом} } = 480{,}00 \units{Вт}
    \end{align*}
}
\vspace{150pt}

\tasknumber{2}\task{
    Замкнутая электрическая цепь состоит из ЭДС $\mathcal{E}=30\units{В}$ и сопротивлением $r$
    и резистора $R=24\units{Ом}$.
    Определите ток, протекающий в цепи.
    Какая тепловая энергия выделится на резисторе за время
    $\tau=4\units{с}$? Какая работа будет совершена ЭДС за это время? Каков знак этой работы? Чему равен КПД цепи? Вычислите значения для 2 случаев:
    $r=0$ и $r=4\units{Ом}$.
}
\answer{
    \begin{align*}
    \mathcal{I}_1
                    &= \frac{ \mathcal{E} }{ R }
                    = \frac{ 30 \units{В} }{ 24 \units{Ом} }
                    = 1{,}25 \units{А}, \\
    \mathcal{I}_2
                    &= \frac{ \mathcal{E} }{ R + r }
                    = \frac{ 30 \units{В} }{ 24 \units{Ом} + 4 \units{Ом} }
                    = 1{,}07 \units{А}, \\
    Q_1
                    &= \mathcal{I}_1^2R\tau
                    = \cbr{ \frac{ \mathcal{E} }{ R } }^2 R \tau
                    = \cbr{ \frac{ 30 \units{В} }{ 24 \units{Ом} } }^2 \cdot 24 \units{Ом} \cdot 4 \units{с}
                    = 150{,}00 \units{Дж}, \\
    Q_2
                    &= \mathcal{I}_2^2R\tau
                    = \cbr{\frac{ \mathcal{E} }{ R + r } }^2 R \tau
                    = \cbr{\frac{ 30 \units{В} }{ 24 \units{Ом} + 4 \units{Ом} } }^2 \cdot 24 \units{Ом} \cdot 4 \units{с}
                    = 110{,}20 \units{Дж}, \\
    A_1
                    &= \mathcal{I}_1\tau\mathcal{E}
                    = \frac{ \mathcal{E} }{ R } \tau \mathcal{E}
                    = \frac{ \mathcal{E}^2 \tau }{ R }
                    = \frac{ \sqr{ 30 \units{В} } \cdot 4 \units{с} }{ 24 \units{Ом} }
                    = 150{,}00 \units{Дж}, \text{положительна}, \\
    A_2
                    &= \mathcal{I}_2\tau\mathcal{E}
                    = \frac{ \mathcal{E} }{ R + r } \tau \mathcal{E}
                    = \frac{ \mathcal{E}^2 \tau }{ R + r }
                    = \frac{ \sqr{ 30 \units{В} } \cdot 4 \units{с} }{ 24 \units{Ом} + 4 \units{Ом} }
                    = 128{,}57 \units{Дж}, \text{положительна}, \\
    \eta_1
                    &= \frac{ Q_1 }{ A_1 }
                    = \ldots
                    = \frac{ R }{ R }
                    = 1, \\
    \eta_2
                    &= \frac{ Q_2 }{ A_2 }
                    = \ldots
                    = \frac{ R }{ R + r }
                    = 0{,}857142857143
    \end{align*}
}
\vspace{150pt}

\tasknumber{3}\task{
    Лампочки, сопротивления которых $R_1=1\units{Ом}$ и $R_2=49\units{Ом}$, поочерёдно подключённные к некоторому источнику тока,
    потребляют одинаковую мощность.
    Найти внутреннее сопротивление источника и КПД цепи в каждом случае.
}
\answer{
    \begin{align*}
    P_1 &= \sqr{ \frac{ \mathcal{E} }{ R_1 + r } } R_1,
    P_2  = \sqr{ \frac{ \mathcal{E} }{ R_2 + r } } R_2,
    P_1 = P_2 \implies \\
    &\implies R_1 \sqr{ R_2 + r } = R_2 \sqr{ R_1 + r } \implies \\
    &\implies R_1 R_2^2 + 2 R_1 R_2 r + R_1 r^2 =
                R_2 R_1^2 + 2 R_2 R_1 r + R_2 r^2  \implies \\
    &\implies r^2 (R_2 - R_1) = R_2^2 R_2 - R_1^2 R_2 \implies \\
    &\implies r
        = \sqrt{ R_1 R_2 \frac{ R_2 - R_1 }{ R_2 - R_1 } }
        = \sqrt{ R_1 R_2 }
        = \sqrt{ 1 \units{Ом} \cdot 49 \units{Ом} }
        = 7{,}00 \units{Ом}.
        \\
     \eta_1
        &= \frac{ R_1 }{ R_1 + r }
        = \frac{ \sqrt{ R_1 } }{ \sqrt{ R_1 } + \sqrt{ R_2 } }
        = 0{,}125, \\
     \eta_2
        &= \frac{ R_2 }{ R_2 + r }
        = \frac{ \sqrt{ R_2 } }{ \sqrt{ R_2 } + \sqrt{ R_1 } }
        = 0{,}875
    \end{align*}
}
\vspace{150pt}

\tasknumber{4}\task{
    Определите ток, протекающий через резистор $R = 18\,\text{Ом}$ и разность потенциалов на нём (см.
    рис.
    на доске),
    если $r_1 = 2\,\text{Ом}$, $r_2 = 3\,\text{Ом}$, $\mathcal{E}_1 = 60\,\text{В}$, $\mathcal{E}_2 = 20\,\text{В}$
}
\newpage

\addpersonalvariant{Илья Гримберг}
\tasknumber{1}\task{
    На резистор сопротивлением $R=12\units{Ом}$ подали напряжение $U=180\units{В}$.
    Определите ток, который потечёт через резистор, и мощность, выделяющуюся на нём.
}
\answer{
    \begin{align*}
    \mathcal{I} &= \frac{ U }{ R } = \frac{ 180 \units{В} }{ 12 \units{Ом} } = 15{,}00 \units{А}, \\
    P &= \frac{ U^2 }{ R } = \frac{ \sqr{ 180 \units{В} } }{ 12 \units{Ом} } = 2700{,}00 \units{Вт}
    \end{align*}
}
\vspace{150pt}

\tasknumber{2}\task{
    Замкнутая электрическая цепь состоит из ЭДС $\mathcal{E}=30\units{В}$ и сопротивлением $r$
    и резистора $R=15\units{Ом}$.
    Определите ток, протекающий в цепи.
    Какая тепловая энергия выделится на резисторе за время
    $\tau=1\units{с}$? Какая работа будет совершена ЭДС за это время? Каков знак этой работы? Чему равен КПД цепи? Вычислите значения для 2 случаев:
    $r=0$ и $r=1\units{Ом}$.
}
\answer{
    \begin{align*}
    \mathcal{I}_1
                    &= \frac{ \mathcal{E} }{ R }
                    = \frac{ 30 \units{В} }{ 15 \units{Ом} }
                    = 2{,}00 \units{А}, \\
    \mathcal{I}_2
                    &= \frac{ \mathcal{E} }{ R + r }
                    = \frac{ 30 \units{В} }{ 15 \units{Ом} + 1 \units{Ом} }
                    = 1{,}88 \units{А}, \\
    Q_1
                    &= \mathcal{I}_1^2R\tau
                    = \cbr{ \frac{ \mathcal{E} }{ R } }^2 R \tau
                    = \cbr{ \frac{ 30 \units{В} }{ 15 \units{Ом} } }^2 \cdot 15 \units{Ом} \cdot 1 \units{с}
                    = 60{,}00 \units{Дж}, \\
    Q_2
                    &= \mathcal{I}_2^2R\tau
                    = \cbr{\frac{ \mathcal{E} }{ R + r } }^2 R \tau
                    = \cbr{\frac{ 30 \units{В} }{ 15 \units{Ом} + 1 \units{Ом} } }^2 \cdot 15 \units{Ом} \cdot 1 \units{с}
                    = 52{,}73 \units{Дж}, \\
    A_1
                    &= \mathcal{I}_1\tau\mathcal{E}
                    = \frac{ \mathcal{E} }{ R } \tau \mathcal{E}
                    = \frac{ \mathcal{E}^2 \tau }{ R }
                    = \frac{ \sqr{ 30 \units{В} } \cdot 1 \units{с} }{ 15 \units{Ом} }
                    = 60{,}00 \units{Дж}, \text{положительна}, \\
    A_2
                    &= \mathcal{I}_2\tau\mathcal{E}
                    = \frac{ \mathcal{E} }{ R + r } \tau \mathcal{E}
                    = \frac{ \mathcal{E}^2 \tau }{ R + r }
                    = \frac{ \sqr{ 30 \units{В} } \cdot 1 \units{с} }{ 15 \units{Ом} + 1 \units{Ом} }
                    = 56{,}25 \units{Дж}, \text{положительна}, \\
    \eta_1
                    &= \frac{ Q_1 }{ A_1 }
                    = \ldots
                    = \frac{ R }{ R }
                    = 1, \\
    \eta_2
                    &= \frac{ Q_2 }{ A_2 }
                    = \ldots
                    = \frac{ R }{ R + r }
                    = 0{,}9375
    \end{align*}
}
\vspace{150pt}

\tasknumber{3}\task{
    Лампочки, сопротивления которых $R_1=0.25\units{Ом}$ и $R_2=16\units{Ом}$, поочерёдно подключённные к некоторому источнику тока,
    потребляют одинаковую мощность.
    Найти внутреннее сопротивление источника и КПД цепи в каждом случае.
}
\answer{
    \begin{align*}
    P_1 &= \sqr{ \frac{ \mathcal{E} }{ R_1 + r } } R_1,
    P_2  = \sqr{ \frac{ \mathcal{E} }{ R_2 + r } } R_2,
    P_1 = P_2 \implies \\
    &\implies R_1 \sqr{ R_2 + r } = R_2 \sqr{ R_1 + r } \implies \\
    &\implies R_1 R_2^2 + 2 R_1 R_2 r + R_1 r^2 =
                R_2 R_1^2 + 2 R_2 R_1 r + R_2 r^2  \implies \\
    &\implies r^2 (R_2 - R_1) = R_2^2 R_2 - R_1^2 R_2 \implies \\
    &\implies r
        = \sqrt{ R_1 R_2 \frac{ R_2 - R_1 }{ R_2 - R_1 } }
        = \sqrt{ R_1 R_2 }
        = \sqrt{ 0{,}25 \units{Ом} \cdot 16 \units{Ом} }
        = 2{,}00 \units{Ом}.
        \\
     \eta_1
        &= \frac{ R_1 }{ R_1 + r }
        = \frac{ \sqrt{ R_1 } }{ \sqrt{ R_1 } + \sqrt{ R_2 } }
        = 0{,}111111111111, \\
     \eta_2
        &= \frac{ R_2 }{ R_2 + r }
        = \frac{ \sqrt{ R_2 } }{ \sqrt{ R_2 } + \sqrt{ R_1 } }
        = 0{,}888888888889
    \end{align*}
}
\vspace{150pt}

\tasknumber{4}\task{
    Определите ток, протекающий через резистор $R = 15\,\text{Ом}$ и разность потенциалов на нём (см.
    рис.
    на доске),
    если $r_1 = 1\,\text{Ом}$, $r_2 = 2\,\text{Ом}$, $\mathcal{E}_1 = 40\,\text{В}$, $\mathcal{E}_2 = 60\,\text{В}$
}
\newpage

\addpersonalvariant{Иван Гурьянов}
\tasknumber{1}\task{
    Через резистор сопротивлением $R=12\units{Ом}$ протекает электрический ток $\mathcal{I}=2\units{А}$.
    Определите, чему равны напряжение на резисторе и мощность, выделяющаяся на нём.
}
\answer{
    \begin{align*}
    U &= \mathcal{I}R = 2 \units{А} \cdot 12 \units{Ом} = 24 \units{В}, \\
    P &= \mathcal{I}^2R = \sqr{ 2 \units{А} } \cdot 12 \units{Ом} = 48 \units{Вт}
    \end{align*}
}
\vspace{150pt}

\tasknumber{2}\task{
    Замкнутая электрическая цепь состоит из ЭДС $\mathcal{E}=30\units{В}$ и сопротивлением $r$
    и резистора $R=30\units{Ом}$.
    Определите ток, протекающий в цепи.
    Какая тепловая энергия выделится на резисторе за время
    $\tau=4\units{с}$? Какая работа будет совершена ЭДС за это время? Каков знак этой работы? Чему равен КПД цепи? Вычислите значения для 2 случаев:
    $r=0$ и $r=4\units{Ом}$.
}
\answer{
    \begin{align*}
    \mathcal{I}_1
                    &= \frac{ \mathcal{E} }{ R }
                    = \frac{ 30 \units{В} }{ 30 \units{Ом} }
                    = 1{,}00 \units{А}, \\
    \mathcal{I}_2
                    &= \frac{ \mathcal{E} }{ R + r }
                    = \frac{ 30 \units{В} }{ 30 \units{Ом} + 4 \units{Ом} }
                    = 0{,}88 \units{А}, \\
    Q_1
                    &= \mathcal{I}_1^2R\tau
                    = \cbr{ \frac{ \mathcal{E} }{ R } }^2 R \tau
                    = \cbr{ \frac{ 30 \units{В} }{ 30 \units{Ом} } }^2 \cdot 30 \units{Ом} \cdot 4 \units{с}
                    = 120{,}00 \units{Дж}, \\
    Q_2
                    &= \mathcal{I}_2^2R\tau
                    = \cbr{\frac{ \mathcal{E} }{ R + r } }^2 R \tau
                    = \cbr{\frac{ 30 \units{В} }{ 30 \units{Ом} + 4 \units{Ом} } }^2 \cdot 30 \units{Ом} \cdot 4 \units{с}
                    = 93{,}43 \units{Дж}, \\
    A_1
                    &= \mathcal{I}_1\tau\mathcal{E}
                    = \frac{ \mathcal{E} }{ R } \tau \mathcal{E}
                    = \frac{ \mathcal{E}^2 \tau }{ R }
                    = \frac{ \sqr{ 30 \units{В} } \cdot 4 \units{с} }{ 30 \units{Ом} }
                    = 120{,}00 \units{Дж}, \text{положительна}, \\
    A_2
                    &= \mathcal{I}_2\tau\mathcal{E}
                    = \frac{ \mathcal{E} }{ R + r } \tau \mathcal{E}
                    = \frac{ \mathcal{E}^2 \tau }{ R + r }
                    = \frac{ \sqr{ 30 \units{В} } \cdot 4 \units{с} }{ 30 \units{Ом} + 4 \units{Ом} }
                    = 105{,}88 \units{Дж}, \text{положительна}, \\
    \eta_1
                    &= \frac{ Q_1 }{ A_1 }
                    = \ldots
                    = \frac{ R }{ R }
                    = 1, \\
    \eta_2
                    &= \frac{ Q_2 }{ A_2 }
                    = \ldots
                    = \frac{ R }{ R + r }
                    = 0{,}882352941176
    \end{align*}
}
\vspace{150pt}

\tasknumber{3}\task{
    Лампочки, сопротивления которых $R_1=5\units{Ом}$ и $R_2=45\units{Ом}$, поочерёдно подключённные к некоторому источнику тока,
    потребляют одинаковую мощность.
    Найти внутреннее сопротивление источника и КПД цепи в каждом случае.
}
\answer{
    \begin{align*}
    P_1 &= \sqr{ \frac{ \mathcal{E} }{ R_1 + r } } R_1,
    P_2  = \sqr{ \frac{ \mathcal{E} }{ R_2 + r } } R_2,
    P_1 = P_2 \implies \\
    &\implies R_1 \sqr{ R_2 + r } = R_2 \sqr{ R_1 + r } \implies \\
    &\implies R_1 R_2^2 + 2 R_1 R_2 r + R_1 r^2 =
                R_2 R_1^2 + 2 R_2 R_1 r + R_2 r^2  \implies \\
    &\implies r^2 (R_2 - R_1) = R_2^2 R_2 - R_1^2 R_2 \implies \\
    &\implies r
        = \sqrt{ R_1 R_2 \frac{ R_2 - R_1 }{ R_2 - R_1 } }
        = \sqrt{ R_1 R_2 }
        = \sqrt{ 5 \units{Ом} \cdot 45 \units{Ом} }
        = 15{,}00 \units{Ом}.
        \\
     \eta_1
        &= \frac{ R_1 }{ R_1 + r }
        = \frac{ \sqrt{ R_1 } }{ \sqrt{ R_1 } + \sqrt{ R_2 } }
        = 0{,}25, \\
     \eta_2
        &= \frac{ R_2 }{ R_2 + r }
        = \frac{ \sqrt{ R_2 } }{ \sqrt{ R_2 } + \sqrt{ R_1 } }
        = 0{,}75
    \end{align*}
}
\vspace{150pt}

\tasknumber{4}\task{
    Определите ток, протекающий через резистор $R = 15\,\text{Ом}$ и разность потенциалов на нём (см.
    рис.
    на доске),
    если $r_1 = 2\,\text{Ом}$, $r_2 = 3\,\text{Ом}$, $\mathcal{E}_1 = 30\,\text{В}$, $\mathcal{E}_2 = 60\,\text{В}$
}
\newpage

\addpersonalvariant{Артём Денежкин}
\tasknumber{1}\task{
    Через резистор сопротивлением $r=18\units{Ом}$ протекает электрический ток $\mathcal{I}=10\units{А}$.
    Определите, чему равны напряжение на резисторе и мощность, выделяющаяся на нём.
}
\answer{
    \begin{align*}
    U &= \mathcal{I}r = 10 \units{А} \cdot 18 \units{Ом} = 180 \units{В}, \\
    P &= \mathcal{I}^2r = \sqr{ 10 \units{А} } \cdot 18 \units{Ом} = 1800 \units{Вт}
    \end{align*}
}
\vspace{150pt}

\tasknumber{2}\task{
    Замкнутая электрическая цепь состоит из ЭДС $\mathcal{E}=10\units{В}$ и сопротивлением $r$
    и резистора $R=10\units{Ом}$.
    Определите ток, протекающий в цепи.
    Какая тепловая энергия выделится на резисторе за время
    $\tau=3\units{с}$? Какая работа будет совершена ЭДС за это время? Каков знак этой работы? Чему равен КПД цепи? Вычислите значения для 2 случаев:
    $r=0$ и $r=3\units{Ом}$.
}
\answer{
    \begin{align*}
    \mathcal{I}_1
                    &= \frac{ \mathcal{E} }{ R }
                    = \frac{ 10 \units{В} }{ 10 \units{Ом} }
                    = 1{,}00 \units{А}, \\
    \mathcal{I}_2
                    &= \frac{ \mathcal{E} }{ R + r }
                    = \frac{ 10 \units{В} }{ 10 \units{Ом} + 3 \units{Ом} }
                    = 0{,}77 \units{А}, \\
    Q_1
                    &= \mathcal{I}_1^2R\tau
                    = \cbr{ \frac{ \mathcal{E} }{ R } }^2 R \tau
                    = \cbr{ \frac{ 10 \units{В} }{ 10 \units{Ом} } }^2 \cdot 10 \units{Ом} \cdot 3 \units{с}
                    = 30{,}00 \units{Дж}, \\
    Q_2
                    &= \mathcal{I}_2^2R\tau
                    = \cbr{\frac{ \mathcal{E} }{ R + r } }^2 R \tau
                    = \cbr{\frac{ 10 \units{В} }{ 10 \units{Ом} + 3 \units{Ом} } }^2 \cdot 10 \units{Ом} \cdot 3 \units{с}
                    = 17{,}75 \units{Дж}, \\
    A_1
                    &= \mathcal{I}_1\tau\mathcal{E}
                    = \frac{ \mathcal{E} }{ R } \tau \mathcal{E}
                    = \frac{ \mathcal{E}^2 \tau }{ R }
                    = \frac{ \sqr{ 10 \units{В} } \cdot 3 \units{с} }{ 10 \units{Ом} }
                    = 30{,}00 \units{Дж}, \text{положительна}, \\
    A_2
                    &= \mathcal{I}_2\tau\mathcal{E}
                    = \frac{ \mathcal{E} }{ R + r } \tau \mathcal{E}
                    = \frac{ \mathcal{E}^2 \tau }{ R + r }
                    = \frac{ \sqr{ 10 \units{В} } \cdot 3 \units{с} }{ 10 \units{Ом} + 3 \units{Ом} }
                    = 23{,}08 \units{Дж}, \text{положительна}, \\
    \eta_1
                    &= \frac{ Q_1 }{ A_1 }
                    = \ldots
                    = \frac{ R }{ R }
                    = 1, \\
    \eta_2
                    &= \frac{ Q_2 }{ A_2 }
                    = \ldots
                    = \frac{ R }{ R + r }
                    = 0{,}769230769231
    \end{align*}
}
\vspace{150pt}

\tasknumber{3}\task{
    Лампочки, сопротивления которых $R_1=4\units{Ом}$ и $R_2=36\units{Ом}$, поочерёдно подключённные к некоторому источнику тока,
    потребляют одинаковую мощность.
    Найти внутреннее сопротивление источника и КПД цепи в каждом случае.
}
\answer{
    \begin{align*}
    P_1 &= \sqr{ \frac{ \mathcal{E} }{ R_1 + r } } R_1,
    P_2  = \sqr{ \frac{ \mathcal{E} }{ R_2 + r } } R_2,
    P_1 = P_2 \implies \\
    &\implies R_1 \sqr{ R_2 + r } = R_2 \sqr{ R_1 + r } \implies \\
    &\implies R_1 R_2^2 + 2 R_1 R_2 r + R_1 r^2 =
                R_2 R_1^2 + 2 R_2 R_1 r + R_2 r^2  \implies \\
    &\implies r^2 (R_2 - R_1) = R_2^2 R_2 - R_1^2 R_2 \implies \\
    &\implies r
        = \sqrt{ R_1 R_2 \frac{ R_2 - R_1 }{ R_2 - R_1 } }
        = \sqrt{ R_1 R_2 }
        = \sqrt{ 4 \units{Ом} \cdot 36 \units{Ом} }
        = 12{,}00 \units{Ом}.
        \\
     \eta_1
        &= \frac{ R_1 }{ R_1 + r }
        = \frac{ \sqrt{ R_1 } }{ \sqrt{ R_1 } + \sqrt{ R_2 } }
        = 0{,}25, \\
     \eta_2
        &= \frac{ R_2 }{ R_2 + r }
        = \frac{ \sqrt{ R_2 } }{ \sqrt{ R_2 } + \sqrt{ R_1 } }
        = 0{,}75
    \end{align*}
}
\vspace{150pt}

\tasknumber{4}\task{
    Определите ток, протекающий через резистор $R = 10\,\text{Ом}$ и разность потенциалов на нём (см.
    рис.
    на доске),
    если $r_1 = 2\,\text{Ом}$, $r_2 = 3\,\text{Ом}$, $\mathcal{E}_1 = 40\,\text{В}$, $\mathcal{E}_2 = 30\,\text{В}$
}
\newpage

\addpersonalvariant{Виктор Жилин}
\tasknumber{1}\task{
    Через резистор сопротивлением $r=5\units{Ом}$ протекает электрический ток $\mathcal{I}=4\units{А}$.
    Определите, чему равны напряжение на резисторе и мощность, выделяющаяся на нём.
}
\answer{
    \begin{align*}
    U &= \mathcal{I}r = 4 \units{А} \cdot 5 \units{Ом} = 20 \units{В}, \\
    P &= \mathcal{I}^2r = \sqr{ 4 \units{А} } \cdot 5 \units{Ом} = 80 \units{Вт}
    \end{align*}
}
\vspace{150pt}

\tasknumber{2}\task{
    Замкнутая электрическая цепь состоит из ЭДС $\mathcal{E}=20\units{В}$ и сопротивлением $r$
    и резистора $R=10\units{Ом}$.
    Определите ток, протекающий в цепи.
    Какая тепловая энергия выделится на резисторе за время
    $\tau=3\units{с}$? Какая работа будет совершена ЭДС за это время? Каков знак этой работы? Чему равен КПД цепи? Вычислите значения для 2 случаев:
    $r=0$ и $r=3\units{Ом}$.
}
\answer{
    \begin{align*}
    \mathcal{I}_1
                    &= \frac{ \mathcal{E} }{ R }
                    = \frac{ 20 \units{В} }{ 10 \units{Ом} }
                    = 2{,}00 \units{А}, \\
    \mathcal{I}_2
                    &= \frac{ \mathcal{E} }{ R + r }
                    = \frac{ 20 \units{В} }{ 10 \units{Ом} + 3 \units{Ом} }
                    = 1{,}54 \units{А}, \\
    Q_1
                    &= \mathcal{I}_1^2R\tau
                    = \cbr{ \frac{ \mathcal{E} }{ R } }^2 R \tau
                    = \cbr{ \frac{ 20 \units{В} }{ 10 \units{Ом} } }^2 \cdot 10 \units{Ом} \cdot 3 \units{с}
                    = 120{,}00 \units{Дж}, \\
    Q_2
                    &= \mathcal{I}_2^2R\tau
                    = \cbr{\frac{ \mathcal{E} }{ R + r } }^2 R \tau
                    = \cbr{\frac{ 20 \units{В} }{ 10 \units{Ом} + 3 \units{Ом} } }^2 \cdot 10 \units{Ом} \cdot 3 \units{с}
                    = 71{,}01 \units{Дж}, \\
    A_1
                    &= \mathcal{I}_1\tau\mathcal{E}
                    = \frac{ \mathcal{E} }{ R } \tau \mathcal{E}
                    = \frac{ \mathcal{E}^2 \tau }{ R }
                    = \frac{ \sqr{ 20 \units{В} } \cdot 3 \units{с} }{ 10 \units{Ом} }
                    = 120{,}00 \units{Дж}, \text{положительна}, \\
    A_2
                    &= \mathcal{I}_2\tau\mathcal{E}
                    = \frac{ \mathcal{E} }{ R + r } \tau \mathcal{E}
                    = \frac{ \mathcal{E}^2 \tau }{ R + r }
                    = \frac{ \sqr{ 20 \units{В} } \cdot 3 \units{с} }{ 10 \units{Ом} + 3 \units{Ом} }
                    = 92{,}31 \units{Дж}, \text{положительна}, \\
    \eta_1
                    &= \frac{ Q_1 }{ A_1 }
                    = \ldots
                    = \frac{ R }{ R }
                    = 1, \\
    \eta_2
                    &= \frac{ Q_2 }{ A_2 }
                    = \ldots
                    = \frac{ R }{ R + r }
                    = 0{,}769230769231
    \end{align*}
}
\vspace{150pt}

\tasknumber{3}\task{
    Лампочки, сопротивления которых $R_1=0.25\units{Ом}$ и $R_2=16\units{Ом}$, поочерёдно подключённные к некоторому источнику тока,
    потребляют одинаковую мощность.
    Найти внутреннее сопротивление источника и КПД цепи в каждом случае.
}
\answer{
    \begin{align*}
    P_1 &= \sqr{ \frac{ \mathcal{E} }{ R_1 + r } } R_1,
    P_2  = \sqr{ \frac{ \mathcal{E} }{ R_2 + r } } R_2,
    P_1 = P_2 \implies \\
    &\implies R_1 \sqr{ R_2 + r } = R_2 \sqr{ R_1 + r } \implies \\
    &\implies R_1 R_2^2 + 2 R_1 R_2 r + R_1 r^2 =
                R_2 R_1^2 + 2 R_2 R_1 r + R_2 r^2  \implies \\
    &\implies r^2 (R_2 - R_1) = R_2^2 R_2 - R_1^2 R_2 \implies \\
    &\implies r
        = \sqrt{ R_1 R_2 \frac{ R_2 - R_1 }{ R_2 - R_1 } }
        = \sqrt{ R_1 R_2 }
        = \sqrt{ 0{,}25 \units{Ом} \cdot 16 \units{Ом} }
        = 2{,}00 \units{Ом}.
        \\
     \eta_1
        &= \frac{ R_1 }{ R_1 + r }
        = \frac{ \sqrt{ R_1 } }{ \sqrt{ R_1 } + \sqrt{ R_2 } }
        = 0{,}111111111111, \\
     \eta_2
        &= \frac{ R_2 }{ R_2 + r }
        = \frac{ \sqrt{ R_2 } }{ \sqrt{ R_2 } + \sqrt{ R_1 } }
        = 0{,}888888888889
    \end{align*}
}
\vspace{150pt}

\tasknumber{4}\task{
    Определите ток, протекающий через резистор $R = 18\,\text{Ом}$ и разность потенциалов на нём (см.
    рис.
    на доске),
    если $r_1 = 2\,\text{Ом}$, $r_2 = 3\,\text{Ом}$, $\mathcal{E}_1 = 40\,\text{В}$, $\mathcal{E}_2 = 30\,\text{В}$
}
\newpage

\addpersonalvariant{Дмитрий Иванов}
\tasknumber{1}\task{
    На резистор сопротивлением $r=12\units{Ом}$ подали напряжение $U=240\units{В}$.
    Определите ток, который потечёт через резистор, и мощность, выделяющуюся на нём.
}
\answer{
    \begin{align*}
    \mathcal{I} &= \frac{ U }{ r } = \frac{ 240 \units{В} }{ 12 \units{Ом} } = 20{,}00 \units{А}, \\
    P &= \frac{ U^2 }{ r } = \frac{ \sqr{ 240 \units{В} } }{ 12 \units{Ом} } = 4800{,}00 \units{Вт}
    \end{align*}
}
\vspace{150pt}

\tasknumber{2}\task{
    Замкнутая электрическая цепь состоит из ЭДС $\mathcal{E}=30\units{В}$ и сопротивлением $r$
    и резистора $R=15\units{Ом}$.
    Определите ток, протекающий в цепи.
    Какая тепловая энергия выделится на резисторе за время
    $\tau=4\units{с}$? Какая работа будет совершена ЭДС за это время? Каков знак этой работы? Чему равен КПД цепи? Вычислите значения для 2 случаев:
    $r=0$ и $r=4\units{Ом}$.
}
\answer{
    \begin{align*}
    \mathcal{I}_1
                    &= \frac{ \mathcal{E} }{ R }
                    = \frac{ 30 \units{В} }{ 15 \units{Ом} }
                    = 2{,}00 \units{А}, \\
    \mathcal{I}_2
                    &= \frac{ \mathcal{E} }{ R + r }
                    = \frac{ 30 \units{В} }{ 15 \units{Ом} + 4 \units{Ом} }
                    = 1{,}58 \units{А}, \\
    Q_1
                    &= \mathcal{I}_1^2R\tau
                    = \cbr{ \frac{ \mathcal{E} }{ R } }^2 R \tau
                    = \cbr{ \frac{ 30 \units{В} }{ 15 \units{Ом} } }^2 \cdot 15 \units{Ом} \cdot 4 \units{с}
                    = 240{,}00 \units{Дж}, \\
    Q_2
                    &= \mathcal{I}_2^2R\tau
                    = \cbr{\frac{ \mathcal{E} }{ R + r } }^2 R \tau
                    = \cbr{\frac{ 30 \units{В} }{ 15 \units{Ом} + 4 \units{Ом} } }^2 \cdot 15 \units{Ом} \cdot 4 \units{с}
                    = 149{,}58 \units{Дж}, \\
    A_1
                    &= \mathcal{I}_1\tau\mathcal{E}
                    = \frac{ \mathcal{E} }{ R } \tau \mathcal{E}
                    = \frac{ \mathcal{E}^2 \tau }{ R }
                    = \frac{ \sqr{ 30 \units{В} } \cdot 4 \units{с} }{ 15 \units{Ом} }
                    = 240{,}00 \units{Дж}, \text{положительна}, \\
    A_2
                    &= \mathcal{I}_2\tau\mathcal{E}
                    = \frac{ \mathcal{E} }{ R + r } \tau \mathcal{E}
                    = \frac{ \mathcal{E}^2 \tau }{ R + r }
                    = \frac{ \sqr{ 30 \units{В} } \cdot 4 \units{с} }{ 15 \units{Ом} + 4 \units{Ом} }
                    = 189{,}47 \units{Дж}, \text{положительна}, \\
    \eta_1
                    &= \frac{ Q_1 }{ A_1 }
                    = \ldots
                    = \frac{ R }{ R }
                    = 1, \\
    \eta_2
                    &= \frac{ Q_2 }{ A_2 }
                    = \ldots
                    = \frac{ R }{ R + r }
                    = 0{,}789473684211
    \end{align*}
}
\vspace{150pt}

\tasknumber{3}\task{
    Лампочки, сопротивления которых $R_1=3\units{Ом}$ и $R_2=48\units{Ом}$, поочерёдно подключённные к некоторому источнику тока,
    потребляют одинаковую мощность.
    Найти внутреннее сопротивление источника и КПД цепи в каждом случае.
}
\answer{
    \begin{align*}
    P_1 &= \sqr{ \frac{ \mathcal{E} }{ R_1 + r } } R_1,
    P_2  = \sqr{ \frac{ \mathcal{E} }{ R_2 + r } } R_2,
    P_1 = P_2 \implies \\
    &\implies R_1 \sqr{ R_2 + r } = R_2 \sqr{ R_1 + r } \implies \\
    &\implies R_1 R_2^2 + 2 R_1 R_2 r + R_1 r^2 =
                R_2 R_1^2 + 2 R_2 R_1 r + R_2 r^2  \implies \\
    &\implies r^2 (R_2 - R_1) = R_2^2 R_2 - R_1^2 R_2 \implies \\
    &\implies r
        = \sqrt{ R_1 R_2 \frac{ R_2 - R_1 }{ R_2 - R_1 } }
        = \sqrt{ R_1 R_2 }
        = \sqrt{ 3 \units{Ом} \cdot 48 \units{Ом} }
        = 12{,}00 \units{Ом}.
        \\
     \eta_1
        &= \frac{ R_1 }{ R_1 + r }
        = \frac{ \sqrt{ R_1 } }{ \sqrt{ R_1 } + \sqrt{ R_2 } }
        = 0{,}2, \\
     \eta_2
        &= \frac{ R_2 }{ R_2 + r }
        = \frac{ \sqrt{ R_2 } }{ \sqrt{ R_2 } + \sqrt{ R_1 } }
        = 0{,}8
    \end{align*}
}
\vspace{150pt}

\tasknumber{4}\task{
    Определите ток, протекающий через резистор $R = 10\,\text{Ом}$ и разность потенциалов на нём (см.
    рис.
    на доске),
    если $r_1 = 1\,\text{Ом}$, $r_2 = 2\,\text{Ом}$, $\mathcal{E}_1 = 30\,\text{В}$, $\mathcal{E}_2 = 20\,\text{В}$
}
\newpage

\addpersonalvariant{Олег Климов}
\tasknumber{1}\task{
    На резистор сопротивлением $r=30\units{Ом}$ подали напряжение $U=180\units{В}$.
    Определите ток, который потечёт через резистор, и мощность, выделяющуюся на нём.
}
\answer{
    \begin{align*}
    \mathcal{I} &= \frac{ U }{ r } = \frac{ 180 \units{В} }{ 30 \units{Ом} } = 6{,}00 \units{А}, \\
    P &= \frac{ U^2 }{ r } = \frac{ \sqr{ 180 \units{В} } }{ 30 \units{Ом} } = 1080{,}00 \units{Вт}
    \end{align*}
}
\vspace{150pt}

\tasknumber{2}\task{
    Замкнутая электрическая цепь состоит из ЭДС $\mathcal{E}=10\units{В}$ и сопротивлением $r$
    и резистора $R=30\units{Ом}$.
    Определите ток, протекающий в цепи.
    Какая тепловая энергия выделится на резисторе за время
    $\tau=4\units{с}$? Какая работа будет совершена ЭДС за это время? Каков знак этой работы? Чему равен КПД цепи? Вычислите значения для 2 случаев:
    $r=0$ и $r=4\units{Ом}$.
}
\answer{
    \begin{align*}
    \mathcal{I}_1
                    &= \frac{ \mathcal{E} }{ R }
                    = \frac{ 10 \units{В} }{ 30 \units{Ом} }
                    = 0{,}33 \units{А}, \\
    \mathcal{I}_2
                    &= \frac{ \mathcal{E} }{ R + r }
                    = \frac{ 10 \units{В} }{ 30 \units{Ом} + 4 \units{Ом} }
                    = 0{,}29 \units{А}, \\
    Q_1
                    &= \mathcal{I}_1^2R\tau
                    = \cbr{ \frac{ \mathcal{E} }{ R } }^2 R \tau
                    = \cbr{ \frac{ 10 \units{В} }{ 30 \units{Ом} } }^2 \cdot 30 \units{Ом} \cdot 4 \units{с}
                    = 13{,}33 \units{Дж}, \\
    Q_2
                    &= \mathcal{I}_2^2R\tau
                    = \cbr{\frac{ \mathcal{E} }{ R + r } }^2 R \tau
                    = \cbr{\frac{ 10 \units{В} }{ 30 \units{Ом} + 4 \units{Ом} } }^2 \cdot 30 \units{Ом} \cdot 4 \units{с}
                    = 10{,}38 \units{Дж}, \\
    A_1
                    &= \mathcal{I}_1\tau\mathcal{E}
                    = \frac{ \mathcal{E} }{ R } \tau \mathcal{E}
                    = \frac{ \mathcal{E}^2 \tau }{ R }
                    = \frac{ \sqr{ 10 \units{В} } \cdot 4 \units{с} }{ 30 \units{Ом} }
                    = 13{,}33 \units{Дж}, \text{положительна}, \\
    A_2
                    &= \mathcal{I}_2\tau\mathcal{E}
                    = \frac{ \mathcal{E} }{ R + r } \tau \mathcal{E}
                    = \frac{ \mathcal{E}^2 \tau }{ R + r }
                    = \frac{ \sqr{ 10 \units{В} } \cdot 4 \units{с} }{ 30 \units{Ом} + 4 \units{Ом} }
                    = 11{,}76 \units{Дж}, \text{положительна}, \\
    \eta_1
                    &= \frac{ Q_1 }{ A_1 }
                    = \ldots
                    = \frac{ R }{ R }
                    = 1, \\
    \eta_2
                    &= \frac{ Q_2 }{ A_2 }
                    = \ldots
                    = \frac{ R }{ R + r }
                    = 0{,}882352941176
    \end{align*}
}
\vspace{150pt}

\tasknumber{3}\task{
    Лампочки, сопротивления которых $R_1=4\units{Ом}$ и $R_2=100\units{Ом}$, поочерёдно подключённные к некоторому источнику тока,
    потребляют одинаковую мощность.
    Найти внутреннее сопротивление источника и КПД цепи в каждом случае.
}
\answer{
    \begin{align*}
    P_1 &= \sqr{ \frac{ \mathcal{E} }{ R_1 + r } } R_1,
    P_2  = \sqr{ \frac{ \mathcal{E} }{ R_2 + r } } R_2,
    P_1 = P_2 \implies \\
    &\implies R_1 \sqr{ R_2 + r } = R_2 \sqr{ R_1 + r } \implies \\
    &\implies R_1 R_2^2 + 2 R_1 R_2 r + R_1 r^2 =
                R_2 R_1^2 + 2 R_2 R_1 r + R_2 r^2  \implies \\
    &\implies r^2 (R_2 - R_1) = R_2^2 R_2 - R_1^2 R_2 \implies \\
    &\implies r
        = \sqrt{ R_1 R_2 \frac{ R_2 - R_1 }{ R_2 - R_1 } }
        = \sqrt{ R_1 R_2 }
        = \sqrt{ 4 \units{Ом} \cdot 100 \units{Ом} }
        = 20{,}00 \units{Ом}.
        \\
     \eta_1
        &= \frac{ R_1 }{ R_1 + r }
        = \frac{ \sqrt{ R_1 } }{ \sqrt{ R_1 } + \sqrt{ R_2 } }
        = 0{,}166666666667, \\
     \eta_2
        &= \frac{ R_2 }{ R_2 + r }
        = \frac{ \sqrt{ R_2 } }{ \sqrt{ R_2 } + \sqrt{ R_1 } }
        = 0{,}833333333333
    \end{align*}
}
\vspace{150pt}

\tasknumber{4}\task{
    Определите ток, протекающий через резистор $R = 20\,\text{Ом}$ и разность потенциалов на нём (см.
    рис.
    на доске),
    если $r_1 = 1\,\text{Ом}$, $r_2 = 3\,\text{Ом}$, $\mathcal{E}_1 = 60\,\text{В}$, $\mathcal{E}_2 = 30\,\text{В}$
}
\newpage

\addpersonalvariant{Анна Ковалева}
\tasknumber{1}\task{
    На резистор сопротивлением $r=5\units{Ом}$ подали напряжение $U=120\units{В}$.
    Определите ток, который потечёт через резистор, и мощность, выделяющуюся на нём.
}
\answer{
    \begin{align*}
    \mathcal{I} &= \frac{ U }{ r } = \frac{ 120 \units{В} }{ 5 \units{Ом} } = 24{,}00 \units{А}, \\
    P &= \frac{ U^2 }{ r } = \frac{ \sqr{ 120 \units{В} } }{ 5 \units{Ом} } = 2880{,}00 \units{Вт}
    \end{align*}
}
\vspace{150pt}

\tasknumber{2}\task{
    Замкнутая электрическая цепь состоит из ЭДС $\mathcal{E}=20\units{В}$ и сопротивлением $r$
    и резистора $R=15\units{Ом}$.
    Определите ток, протекающий в цепи.
    Какая тепловая энергия выделится на резисторе за время
    $\tau=2\units{с}$? Какая работа будет совершена ЭДС за это время? Каков знак этой работы? Чему равен КПД цепи? Вычислите значения для 2 случаев:
    $r=0$ и $r=2\units{Ом}$.
}
\answer{
    \begin{align*}
    \mathcal{I}_1
                    &= \frac{ \mathcal{E} }{ R }
                    = \frac{ 20 \units{В} }{ 15 \units{Ом} }
                    = 1{,}33 \units{А}, \\
    \mathcal{I}_2
                    &= \frac{ \mathcal{E} }{ R + r }
                    = \frac{ 20 \units{В} }{ 15 \units{Ом} + 2 \units{Ом} }
                    = 1{,}18 \units{А}, \\
    Q_1
                    &= \mathcal{I}_1^2R\tau
                    = \cbr{ \frac{ \mathcal{E} }{ R } }^2 R \tau
                    = \cbr{ \frac{ 20 \units{В} }{ 15 \units{Ом} } }^2 \cdot 15 \units{Ом} \cdot 2 \units{с}
                    = 53{,}33 \units{Дж}, \\
    Q_2
                    &= \mathcal{I}_2^2R\tau
                    = \cbr{\frac{ \mathcal{E} }{ R + r } }^2 R \tau
                    = \cbr{\frac{ 20 \units{В} }{ 15 \units{Ом} + 2 \units{Ом} } }^2 \cdot 15 \units{Ом} \cdot 2 \units{с}
                    = 41{,}52 \units{Дж}, \\
    A_1
                    &= \mathcal{I}_1\tau\mathcal{E}
                    = \frac{ \mathcal{E} }{ R } \tau \mathcal{E}
                    = \frac{ \mathcal{E}^2 \tau }{ R }
                    = \frac{ \sqr{ 20 \units{В} } \cdot 2 \units{с} }{ 15 \units{Ом} }
                    = 53{,}33 \units{Дж}, \text{положительна}, \\
    A_2
                    &= \mathcal{I}_2\tau\mathcal{E}
                    = \frac{ \mathcal{E} }{ R + r } \tau \mathcal{E}
                    = \frac{ \mathcal{E}^2 \tau }{ R + r }
                    = \frac{ \sqr{ 20 \units{В} } \cdot 2 \units{с} }{ 15 \units{Ом} + 2 \units{Ом} }
                    = 47{,}06 \units{Дж}, \text{положительна}, \\
    \eta_1
                    &= \frac{ Q_1 }{ A_1 }
                    = \ldots
                    = \frac{ R }{ R }
                    = 1, \\
    \eta_2
                    &= \frac{ Q_2 }{ A_2 }
                    = \ldots
                    = \frac{ R }{ R + r }
                    = 0{,}882352941176
    \end{align*}
}
\vspace{150pt}

\tasknumber{3}\task{
    Лампочки, сопротивления которых $R_1=3\units{Ом}$ и $R_2=12\units{Ом}$, поочерёдно подключённные к некоторому источнику тока,
    потребляют одинаковую мощность.
    Найти внутреннее сопротивление источника и КПД цепи в каждом случае.
}
\answer{
    \begin{align*}
    P_1 &= \sqr{ \frac{ \mathcal{E} }{ R_1 + r } } R_1,
    P_2  = \sqr{ \frac{ \mathcal{E} }{ R_2 + r } } R_2,
    P_1 = P_2 \implies \\
    &\implies R_1 \sqr{ R_2 + r } = R_2 \sqr{ R_1 + r } \implies \\
    &\implies R_1 R_2^2 + 2 R_1 R_2 r + R_1 r^2 =
                R_2 R_1^2 + 2 R_2 R_1 r + R_2 r^2  \implies \\
    &\implies r^2 (R_2 - R_1) = R_2^2 R_2 - R_1^2 R_2 \implies \\
    &\implies r
        = \sqrt{ R_1 R_2 \frac{ R_2 - R_1 }{ R_2 - R_1 } }
        = \sqrt{ R_1 R_2 }
        = \sqrt{ 3 \units{Ом} \cdot 12 \units{Ом} }
        = 6{,}00 \units{Ом}.
        \\
     \eta_1
        &= \frac{ R_1 }{ R_1 + r }
        = \frac{ \sqrt{ R_1 } }{ \sqrt{ R_1 } + \sqrt{ R_2 } }
        = 0{,}333333333333, \\
     \eta_2
        &= \frac{ R_2 }{ R_2 + r }
        = \frac{ \sqrt{ R_2 } }{ \sqrt{ R_2 } + \sqrt{ R_1 } }
        = 0{,}666666666667
    \end{align*}
}
\vspace{150pt}

\tasknumber{4}\task{
    Определите ток, протекающий через резистор $R = 18\,\text{Ом}$ и разность потенциалов на нём (см.
    рис.
    на доске),
    если $r_1 = 2\,\text{Ом}$, $r_2 = 3\,\text{Ом}$, $\mathcal{E}_1 = 30\,\text{В}$, $\mathcal{E}_2 = 20\,\text{В}$
}
\newpage

\addpersonalvariant{Глеб Ковылин}
\tasknumber{1}\task{
    Через резистор сопротивлением $r=18\units{Ом}$ протекает электрический ток $\mathcal{I}=4\units{А}$.
    Определите, чему равны напряжение на резисторе и мощность, выделяющаяся на нём.
}
\answer{
    \begin{align*}
    U &= \mathcal{I}r = 4 \units{А} \cdot 18 \units{Ом} = 72 \units{В}, \\
    P &= \mathcal{I}^2r = \sqr{ 4 \units{А} } \cdot 18 \units{Ом} = 288 \units{Вт}
    \end{align*}
}
\vspace{150pt}

\tasknumber{2}\task{
    Замкнутая электрическая цепь состоит из ЭДС $\mathcal{E}=30\units{В}$ и сопротивлением $r$
    и резистора $R=10\units{Ом}$.
    Определите ток, протекающий в цепи.
    Какая тепловая энергия выделится на резисторе за время
    $\tau=2\units{с}$? Какая работа будет совершена ЭДС за это время? Каков знак этой работы? Чему равен КПД цепи? Вычислите значения для 2 случаев:
    $r=0$ и $r=2\units{Ом}$.
}
\answer{
    \begin{align*}
    \mathcal{I}_1
                    &= \frac{ \mathcal{E} }{ R }
                    = \frac{ 30 \units{В} }{ 10 \units{Ом} }
                    = 3{,}00 \units{А}, \\
    \mathcal{I}_2
                    &= \frac{ \mathcal{E} }{ R + r }
                    = \frac{ 30 \units{В} }{ 10 \units{Ом} + 2 \units{Ом} }
                    = 2{,}50 \units{А}, \\
    Q_1
                    &= \mathcal{I}_1^2R\tau
                    = \cbr{ \frac{ \mathcal{E} }{ R } }^2 R \tau
                    = \cbr{ \frac{ 30 \units{В} }{ 10 \units{Ом} } }^2 \cdot 10 \units{Ом} \cdot 2 \units{с}
                    = 180{,}00 \units{Дж}, \\
    Q_2
                    &= \mathcal{I}_2^2R\tau
                    = \cbr{\frac{ \mathcal{E} }{ R + r } }^2 R \tau
                    = \cbr{\frac{ 30 \units{В} }{ 10 \units{Ом} + 2 \units{Ом} } }^2 \cdot 10 \units{Ом} \cdot 2 \units{с}
                    = 125{,}00 \units{Дж}, \\
    A_1
                    &= \mathcal{I}_1\tau\mathcal{E}
                    = \frac{ \mathcal{E} }{ R } \tau \mathcal{E}
                    = \frac{ \mathcal{E}^2 \tau }{ R }
                    = \frac{ \sqr{ 30 \units{В} } \cdot 2 \units{с} }{ 10 \units{Ом} }
                    = 180{,}00 \units{Дж}, \text{положительна}, \\
    A_2
                    &= \mathcal{I}_2\tau\mathcal{E}
                    = \frac{ \mathcal{E} }{ R + r } \tau \mathcal{E}
                    = \frac{ \mathcal{E}^2 \tau }{ R + r }
                    = \frac{ \sqr{ 30 \units{В} } \cdot 2 \units{с} }{ 10 \units{Ом} + 2 \units{Ом} }
                    = 150{,}00 \units{Дж}, \text{положительна}, \\
    \eta_1
                    &= \frac{ Q_1 }{ A_1 }
                    = \ldots
                    = \frac{ R }{ R }
                    = 1, \\
    \eta_2
                    &= \frac{ Q_2 }{ A_2 }
                    = \ldots
                    = \frac{ R }{ R + r }
                    = 0{,}833333333333
    \end{align*}
}
\vspace{150pt}

\tasknumber{3}\task{
    Лампочки, сопротивления которых $R_1=0.5\units{Ом}$ и $R_2=18\units{Ом}$, поочерёдно подключённные к некоторому источнику тока,
    потребляют одинаковую мощность.
    Найти внутреннее сопротивление источника и КПД цепи в каждом случае.
}
\answer{
    \begin{align*}
    P_1 &= \sqr{ \frac{ \mathcal{E} }{ R_1 + r } } R_1,
    P_2  = \sqr{ \frac{ \mathcal{E} }{ R_2 + r } } R_2,
    P_1 = P_2 \implies \\
    &\implies R_1 \sqr{ R_2 + r } = R_2 \sqr{ R_1 + r } \implies \\
    &\implies R_1 R_2^2 + 2 R_1 R_2 r + R_1 r^2 =
                R_2 R_1^2 + 2 R_2 R_1 r + R_2 r^2  \implies \\
    &\implies r^2 (R_2 - R_1) = R_2^2 R_2 - R_1^2 R_2 \implies \\
    &\implies r
        = \sqrt{ R_1 R_2 \frac{ R_2 - R_1 }{ R_2 - R_1 } }
        = \sqrt{ R_1 R_2 }
        = \sqrt{ 0{,}50 \units{Ом} \cdot 18 \units{Ом} }
        = 3{,}00 \units{Ом}.
        \\
     \eta_1
        &= \frac{ R_1 }{ R_1 + r }
        = \frac{ \sqrt{ R_1 } }{ \sqrt{ R_1 } + \sqrt{ R_2 } }
        = 0{,}142857142857, \\
     \eta_2
        &= \frac{ R_2 }{ R_2 + r }
        = \frac{ \sqrt{ R_2 } }{ \sqrt{ R_2 } + \sqrt{ R_1 } }
        = 0{,}857142857143
    \end{align*}
}
\vspace{150pt}

\tasknumber{4}\task{
    Определите ток, протекающий через резистор $R = 15\,\text{Ом}$ и разность потенциалов на нём (см.
    рис.
    на доске),
    если $r_1 = 2\,\text{Ом}$, $r_2 = 3\,\text{Ом}$, $\mathcal{E}_1 = 40\,\text{В}$, $\mathcal{E}_2 = 60\,\text{В}$
}
\newpage

\addpersonalvariant{Даниил Космынин}
\tasknumber{1}\task{
    Через резистор сопротивлением $r=5\units{Ом}$ протекает электрический ток $\mathcal{I}=5\units{А}$.
    Определите, чему равны напряжение на резисторе и мощность, выделяющаяся на нём.
}
\answer{
    \begin{align*}
    U &= \mathcal{I}r = 5 \units{А} \cdot 5 \units{Ом} = 25 \units{В}, \\
    P &= \mathcal{I}^2r = \sqr{ 5 \units{А} } \cdot 5 \units{Ом} = 125 \units{Вт}
    \end{align*}
}
\vspace{150pt}

\tasknumber{2}\task{
    Замкнутая электрическая цепь состоит из ЭДС $\mathcal{E}=30\units{В}$ и сопротивлением $r$
    и резистора $R=10\units{Ом}$.
    Определите ток, протекающий в цепи.
    Какая тепловая энергия выделится на резисторе за время
    $\tau=3\units{с}$? Какая работа будет совершена ЭДС за это время? Каков знак этой работы? Чему равен КПД цепи? Вычислите значения для 2 случаев:
    $r=0$ и $r=3\units{Ом}$.
}
\answer{
    \begin{align*}
    \mathcal{I}_1
                    &= \frac{ \mathcal{E} }{ R }
                    = \frac{ 30 \units{В} }{ 10 \units{Ом} }
                    = 3{,}00 \units{А}, \\
    \mathcal{I}_2
                    &= \frac{ \mathcal{E} }{ R + r }
                    = \frac{ 30 \units{В} }{ 10 \units{Ом} + 3 \units{Ом} }
                    = 2{,}31 \units{А}, \\
    Q_1
                    &= \mathcal{I}_1^2R\tau
                    = \cbr{ \frac{ \mathcal{E} }{ R } }^2 R \tau
                    = \cbr{ \frac{ 30 \units{В} }{ 10 \units{Ом} } }^2 \cdot 10 \units{Ом} \cdot 3 \units{с}
                    = 270{,}00 \units{Дж}, \\
    Q_2
                    &= \mathcal{I}_2^2R\tau
                    = \cbr{\frac{ \mathcal{E} }{ R + r } }^2 R \tau
                    = \cbr{\frac{ 30 \units{В} }{ 10 \units{Ом} + 3 \units{Ом} } }^2 \cdot 10 \units{Ом} \cdot 3 \units{с}
                    = 159{,}76 \units{Дж}, \\
    A_1
                    &= \mathcal{I}_1\tau\mathcal{E}
                    = \frac{ \mathcal{E} }{ R } \tau \mathcal{E}
                    = \frac{ \mathcal{E}^2 \tau }{ R }
                    = \frac{ \sqr{ 30 \units{В} } \cdot 3 \units{с} }{ 10 \units{Ом} }
                    = 270{,}00 \units{Дж}, \text{положительна}, \\
    A_2
                    &= \mathcal{I}_2\tau\mathcal{E}
                    = \frac{ \mathcal{E} }{ R + r } \tau \mathcal{E}
                    = \frac{ \mathcal{E}^2 \tau }{ R + r }
                    = \frac{ \sqr{ 30 \units{В} } \cdot 3 \units{с} }{ 10 \units{Ом} + 3 \units{Ом} }
                    = 207{,}69 \units{Дж}, \text{положительна}, \\
    \eta_1
                    &= \frac{ Q_1 }{ A_1 }
                    = \ldots
                    = \frac{ R }{ R }
                    = 1, \\
    \eta_2
                    &= \frac{ Q_2 }{ A_2 }
                    = \ldots
                    = \frac{ R }{ R + r }
                    = 0{,}769230769231
    \end{align*}
}
\vspace{150pt}

\tasknumber{3}\task{
    Лампочки, сопротивления которых $R_1=6\units{Ом}$ и $R_2=24\units{Ом}$, поочерёдно подключённные к некоторому источнику тока,
    потребляют одинаковую мощность.
    Найти внутреннее сопротивление источника и КПД цепи в каждом случае.
}
\answer{
    \begin{align*}
    P_1 &= \sqr{ \frac{ \mathcal{E} }{ R_1 + r } } R_1,
    P_2  = \sqr{ \frac{ \mathcal{E} }{ R_2 + r } } R_2,
    P_1 = P_2 \implies \\
    &\implies R_1 \sqr{ R_2 + r } = R_2 \sqr{ R_1 + r } \implies \\
    &\implies R_1 R_2^2 + 2 R_1 R_2 r + R_1 r^2 =
                R_2 R_1^2 + 2 R_2 R_1 r + R_2 r^2  \implies \\
    &\implies r^2 (R_2 - R_1) = R_2^2 R_2 - R_1^2 R_2 \implies \\
    &\implies r
        = \sqrt{ R_1 R_2 \frac{ R_2 - R_1 }{ R_2 - R_1 } }
        = \sqrt{ R_1 R_2 }
        = \sqrt{ 6 \units{Ом} \cdot 24 \units{Ом} }
        = 12{,}00 \units{Ом}.
        \\
     \eta_1
        &= \frac{ R_1 }{ R_1 + r }
        = \frac{ \sqrt{ R_1 } }{ \sqrt{ R_1 } + \sqrt{ R_2 } }
        = 0{,}333333333333, \\
     \eta_2
        &= \frac{ R_2 }{ R_2 + r }
        = \frac{ \sqrt{ R_2 } }{ \sqrt{ R_2 } + \sqrt{ R_1 } }
        = 0{,}666666666667
    \end{align*}
}
\vspace{150pt}

\tasknumber{4}\task{
    Определите ток, протекающий через резистор $R = 18\,\text{Ом}$ и разность потенциалов на нём (см.
    рис.
    на доске),
    если $r_1 = 1\,\text{Ом}$, $r_2 = 3\,\text{Ом}$, $\mathcal{E}_1 = 30\,\text{В}$, $\mathcal{E}_2 = 60\,\text{В}$
}
\newpage

\addpersonalvariant{Алина Леоничева}
\tasknumber{1}\task{
    Через резистор сопротивлением $r=12\units{Ом}$ протекает электрический ток $\mathcal{I}=15\units{А}$.
    Определите, чему равны напряжение на резисторе и мощность, выделяющаяся на нём.
}
\answer{
    \begin{align*}
    U &= \mathcal{I}r = 15 \units{А} \cdot 12 \units{Ом} = 180 \units{В}, \\
    P &= \mathcal{I}^2r = \sqr{ 15 \units{А} } \cdot 12 \units{Ом} = 2700 \units{Вт}
    \end{align*}
}
\vspace{150pt}

\tasknumber{2}\task{
    Замкнутая электрическая цепь состоит из ЭДС $\mathcal{E}=60\units{В}$ и сопротивлением $r$
    и резистора $R=24\units{Ом}$.
    Определите ток, протекающий в цепи.
    Какая тепловая энергия выделится на резисторе за время
    $\tau=1\units{с}$? Какая работа будет совершена ЭДС за это время? Каков знак этой работы? Чему равен КПД цепи? Вычислите значения для 2 случаев:
    $r=0$ и $r=1\units{Ом}$.
}
\answer{
    \begin{align*}
    \mathcal{I}_1
                    &= \frac{ \mathcal{E} }{ R }
                    = \frac{ 60 \units{В} }{ 24 \units{Ом} }
                    = 2{,}50 \units{А}, \\
    \mathcal{I}_2
                    &= \frac{ \mathcal{E} }{ R + r }
                    = \frac{ 60 \units{В} }{ 24 \units{Ом} + 1 \units{Ом} }
                    = 2{,}40 \units{А}, \\
    Q_1
                    &= \mathcal{I}_1^2R\tau
                    = \cbr{ \frac{ \mathcal{E} }{ R } }^2 R \tau
                    = \cbr{ \frac{ 60 \units{В} }{ 24 \units{Ом} } }^2 \cdot 24 \units{Ом} \cdot 1 \units{с}
                    = 150{,}00 \units{Дж}, \\
    Q_2
                    &= \mathcal{I}_2^2R\tau
                    = \cbr{\frac{ \mathcal{E} }{ R + r } }^2 R \tau
                    = \cbr{\frac{ 60 \units{В} }{ 24 \units{Ом} + 1 \units{Ом} } }^2 \cdot 24 \units{Ом} \cdot 1 \units{с}
                    = 138{,}24 \units{Дж}, \\
    A_1
                    &= \mathcal{I}_1\tau\mathcal{E}
                    = \frac{ \mathcal{E} }{ R } \tau \mathcal{E}
                    = \frac{ \mathcal{E}^2 \tau }{ R }
                    = \frac{ \sqr{ 60 \units{В} } \cdot 1 \units{с} }{ 24 \units{Ом} }
                    = 150{,}00 \units{Дж}, \text{положительна}, \\
    A_2
                    &= \mathcal{I}_2\tau\mathcal{E}
                    = \frac{ \mathcal{E} }{ R + r } \tau \mathcal{E}
                    = \frac{ \mathcal{E}^2 \tau }{ R + r }
                    = \frac{ \sqr{ 60 \units{В} } \cdot 1 \units{с} }{ 24 \units{Ом} + 1 \units{Ом} }
                    = 144{,}00 \units{Дж}, \text{положительна}, \\
    \eta_1
                    &= \frac{ Q_1 }{ A_1 }
                    = \ldots
                    = \frac{ R }{ R }
                    = 1, \\
    \eta_2
                    &= \frac{ Q_2 }{ A_2 }
                    = \ldots
                    = \frac{ R }{ R + r }
                    = 0{,}96
    \end{align*}
}
\vspace{150pt}

\tasknumber{3}\task{
    Лампочки, сопротивления которых $R_1=0.25\units{Ом}$ и $R_2=64\units{Ом}$, поочерёдно подключённные к некоторому источнику тока,
    потребляют одинаковую мощность.
    Найти внутреннее сопротивление источника и КПД цепи в каждом случае.
}
\answer{
    \begin{align*}
    P_1 &= \sqr{ \frac{ \mathcal{E} }{ R_1 + r } } R_1,
    P_2  = \sqr{ \frac{ \mathcal{E} }{ R_2 + r } } R_2,
    P_1 = P_2 \implies \\
    &\implies R_1 \sqr{ R_2 + r } = R_2 \sqr{ R_1 + r } \implies \\
    &\implies R_1 R_2^2 + 2 R_1 R_2 r + R_1 r^2 =
                R_2 R_1^2 + 2 R_2 R_1 r + R_2 r^2  \implies \\
    &\implies r^2 (R_2 - R_1) = R_2^2 R_2 - R_1^2 R_2 \implies \\
    &\implies r
        = \sqrt{ R_1 R_2 \frac{ R_2 - R_1 }{ R_2 - R_1 } }
        = \sqrt{ R_1 R_2 }
        = \sqrt{ 0{,}25 \units{Ом} \cdot 64 \units{Ом} }
        = 4{,}00 \units{Ом}.
        \\
     \eta_1
        &= \frac{ R_1 }{ R_1 + r }
        = \frac{ \sqrt{ R_1 } }{ \sqrt{ R_1 } + \sqrt{ R_2 } }
        = 0{,}0588235294118, \\
     \eta_2
        &= \frac{ R_2 }{ R_2 + r }
        = \frac{ \sqrt{ R_2 } }{ \sqrt{ R_2 } + \sqrt{ R_1 } }
        = 0{,}941176470588
    \end{align*}
}
\vspace{150pt}

\tasknumber{4}\task{
    Определите ток, протекающий через резистор $R = 12\,\text{Ом}$ и разность потенциалов на нём (см.
    рис.
    на доске),
    если $r_1 = 1\,\text{Ом}$, $r_2 = 3\,\text{Ом}$, $\mathcal{E}_1 = 30\,\text{В}$, $\mathcal{E}_2 = 20\,\text{В}$
}
\newpage

\addpersonalvariant{Ирина Лин}
\tasknumber{1}\task{
    Через резистор сопротивлением $r=18\units{Ом}$ протекает электрический ток $\mathcal{I}=15\units{А}$.
    Определите, чему равны напряжение на резисторе и мощность, выделяющаяся на нём.
}
\answer{
    \begin{align*}
    U &= \mathcal{I}r = 15 \units{А} \cdot 18 \units{Ом} = 270 \units{В}, \\
    P &= \mathcal{I}^2r = \sqr{ 15 \units{А} } \cdot 18 \units{Ом} = 4050 \units{Вт}
    \end{align*}
}
\vspace{150pt}

\tasknumber{2}\task{
    Замкнутая электрическая цепь состоит из ЭДС $\mathcal{E}=20\units{В}$ и сопротивлением $r$
    и резистора $R=30\units{Ом}$.
    Определите ток, протекающий в цепи.
    Какая тепловая энергия выделится на резисторе за время
    $\tau=1\units{с}$? Какая работа будет совершена ЭДС за это время? Каков знак этой работы? Чему равен КПД цепи? Вычислите значения для 2 случаев:
    $r=0$ и $r=1\units{Ом}$.
}
\answer{
    \begin{align*}
    \mathcal{I}_1
                    &= \frac{ \mathcal{E} }{ R }
                    = \frac{ 20 \units{В} }{ 30 \units{Ом} }
                    = 0{,}67 \units{А}, \\
    \mathcal{I}_2
                    &= \frac{ \mathcal{E} }{ R + r }
                    = \frac{ 20 \units{В} }{ 30 \units{Ом} + 1 \units{Ом} }
                    = 0{,}65 \units{А}, \\
    Q_1
                    &= \mathcal{I}_1^2R\tau
                    = \cbr{ \frac{ \mathcal{E} }{ R } }^2 R \tau
                    = \cbr{ \frac{ 20 \units{В} }{ 30 \units{Ом} } }^2 \cdot 30 \units{Ом} \cdot 1 \units{с}
                    = 13{,}33 \units{Дж}, \\
    Q_2
                    &= \mathcal{I}_2^2R\tau
                    = \cbr{\frac{ \mathcal{E} }{ R + r } }^2 R \tau
                    = \cbr{\frac{ 20 \units{В} }{ 30 \units{Ом} + 1 \units{Ом} } }^2 \cdot 30 \units{Ом} \cdot 1 \units{с}
                    = 12{,}49 \units{Дж}, \\
    A_1
                    &= \mathcal{I}_1\tau\mathcal{E}
                    = \frac{ \mathcal{E} }{ R } \tau \mathcal{E}
                    = \frac{ \mathcal{E}^2 \tau }{ R }
                    = \frac{ \sqr{ 20 \units{В} } \cdot 1 \units{с} }{ 30 \units{Ом} }
                    = 13{,}33 \units{Дж}, \text{положительна}, \\
    A_2
                    &= \mathcal{I}_2\tau\mathcal{E}
                    = \frac{ \mathcal{E} }{ R + r } \tau \mathcal{E}
                    = \frac{ \mathcal{E}^2 \tau }{ R + r }
                    = \frac{ \sqr{ 20 \units{В} } \cdot 1 \units{с} }{ 30 \units{Ом} + 1 \units{Ом} }
                    = 12{,}90 \units{Дж}, \text{положительна}, \\
    \eta_1
                    &= \frac{ Q_1 }{ A_1 }
                    = \ldots
                    = \frac{ R }{ R }
                    = 1, \\
    \eta_2
                    &= \frac{ Q_2 }{ A_2 }
                    = \ldots
                    = \frac{ R }{ R + r }
                    = 0{,}967741935484
    \end{align*}
}
\vspace{150pt}

\tasknumber{3}\task{
    Лампочки, сопротивления которых $R_1=5\units{Ом}$ и $R_2=80\units{Ом}$, поочерёдно подключённные к некоторому источнику тока,
    потребляют одинаковую мощность.
    Найти внутреннее сопротивление источника и КПД цепи в каждом случае.
}
\answer{
    \begin{align*}
    P_1 &= \sqr{ \frac{ \mathcal{E} }{ R_1 + r } } R_1,
    P_2  = \sqr{ \frac{ \mathcal{E} }{ R_2 + r } } R_2,
    P_1 = P_2 \implies \\
    &\implies R_1 \sqr{ R_2 + r } = R_2 \sqr{ R_1 + r } \implies \\
    &\implies R_1 R_2^2 + 2 R_1 R_2 r + R_1 r^2 =
                R_2 R_1^2 + 2 R_2 R_1 r + R_2 r^2  \implies \\
    &\implies r^2 (R_2 - R_1) = R_2^2 R_2 - R_1^2 R_2 \implies \\
    &\implies r
        = \sqrt{ R_1 R_2 \frac{ R_2 - R_1 }{ R_2 - R_1 } }
        = \sqrt{ R_1 R_2 }
        = \sqrt{ 5 \units{Ом} \cdot 80 \units{Ом} }
        = 20{,}00 \units{Ом}.
        \\
     \eta_1
        &= \frac{ R_1 }{ R_1 + r }
        = \frac{ \sqrt{ R_1 } }{ \sqrt{ R_1 } + \sqrt{ R_2 } }
        = 0{,}2, \\
     \eta_2
        &= \frac{ R_2 }{ R_2 + r }
        = \frac{ \sqrt{ R_2 } }{ \sqrt{ R_2 } + \sqrt{ R_1 } }
        = 0{,}8
    \end{align*}
}
\vspace{150pt}

\tasknumber{4}\task{
    Определите ток, протекающий через резистор $R = 15\,\text{Ом}$ и разность потенциалов на нём (см.
    рис.
    на доске),
    если $r_1 = 1\,\text{Ом}$, $r_2 = 3\,\text{Ом}$, $\mathcal{E}_1 = 40\,\text{В}$, $\mathcal{E}_2 = 60\,\text{В}$
}
\newpage

\addpersonalvariant{Олег Мальцев}
\tasknumber{1}\task{
    На резистор сопротивлением $R=30\units{Ом}$ подали напряжение $V=150\units{В}$.
    Определите ток, который потечёт через резистор, и мощность, выделяющуюся на нём.
}
\answer{
    \begin{align*}
    \mathcal{I} &= \frac{ V }{ R } = \frac{ 150 \units{В} }{ 30 \units{Ом} } = 5{,}00 \units{А}, \\
    P &= \frac{ V^2 }{ R } = \frac{ \sqr{ 150 \units{В} } }{ 30 \units{Ом} } = 750{,}00 \units{Вт}
    \end{align*}
}
\vspace{150pt}

\tasknumber{2}\task{
    Замкнутая электрическая цепь состоит из ЭДС $\mathcal{E}=60\units{В}$ и сопротивлением $r$
    и резистора $R=10\units{Ом}$.
    Определите ток, протекающий в цепи.
    Какая тепловая энергия выделится на резисторе за время
    $\tau=3\units{с}$? Какая работа будет совершена ЭДС за это время? Каков знак этой работы? Чему равен КПД цепи? Вычислите значения для 2 случаев:
    $r=0$ и $r=3\units{Ом}$.
}
\answer{
    \begin{align*}
    \mathcal{I}_1
                    &= \frac{ \mathcal{E} }{ R }
                    = \frac{ 60 \units{В} }{ 10 \units{Ом} }
                    = 6{,}00 \units{А}, \\
    \mathcal{I}_2
                    &= \frac{ \mathcal{E} }{ R + r }
                    = \frac{ 60 \units{В} }{ 10 \units{Ом} + 3 \units{Ом} }
                    = 4{,}62 \units{А}, \\
    Q_1
                    &= \mathcal{I}_1^2R\tau
                    = \cbr{ \frac{ \mathcal{E} }{ R } }^2 R \tau
                    = \cbr{ \frac{ 60 \units{В} }{ 10 \units{Ом} } }^2 \cdot 10 \units{Ом} \cdot 3 \units{с}
                    = 1080{,}00 \units{Дж}, \\
    Q_2
                    &= \mathcal{I}_2^2R\tau
                    = \cbr{\frac{ \mathcal{E} }{ R + r } }^2 R \tau
                    = \cbr{\frac{ 60 \units{В} }{ 10 \units{Ом} + 3 \units{Ом} } }^2 \cdot 10 \units{Ом} \cdot 3 \units{с}
                    = 639{,}05 \units{Дж}, \\
    A_1
                    &= \mathcal{I}_1\tau\mathcal{E}
                    = \frac{ \mathcal{E} }{ R } \tau \mathcal{E}
                    = \frac{ \mathcal{E}^2 \tau }{ R }
                    = \frac{ \sqr{ 60 \units{В} } \cdot 3 \units{с} }{ 10 \units{Ом} }
                    = 1080{,}00 \units{Дж}, \text{положительна}, \\
    A_2
                    &= \mathcal{I}_2\tau\mathcal{E}
                    = \frac{ \mathcal{E} }{ R + r } \tau \mathcal{E}
                    = \frac{ \mathcal{E}^2 \tau }{ R + r }
                    = \frac{ \sqr{ 60 \units{В} } \cdot 3 \units{с} }{ 10 \units{Ом} + 3 \units{Ом} }
                    = 830{,}77 \units{Дж}, \text{положительна}, \\
    \eta_1
                    &= \frac{ Q_1 }{ A_1 }
                    = \ldots
                    = \frac{ R }{ R }
                    = 1, \\
    \eta_2
                    &= \frac{ Q_2 }{ A_2 }
                    = \ldots
                    = \frac{ R }{ R + r }
                    = 0{,}769230769231
    \end{align*}
}
\vspace{150pt}

\tasknumber{3}\task{
    Лампочки, сопротивления которых $R_1=6\units{Ом}$ и $R_2=54\units{Ом}$, поочерёдно подключённные к некоторому источнику тока,
    потребляют одинаковую мощность.
    Найти внутреннее сопротивление источника и КПД цепи в каждом случае.
}
\answer{
    \begin{align*}
    P_1 &= \sqr{ \frac{ \mathcal{E} }{ R_1 + r } } R_1,
    P_2  = \sqr{ \frac{ \mathcal{E} }{ R_2 + r } } R_2,
    P_1 = P_2 \implies \\
    &\implies R_1 \sqr{ R_2 + r } = R_2 \sqr{ R_1 + r } \implies \\
    &\implies R_1 R_2^2 + 2 R_1 R_2 r + R_1 r^2 =
                R_2 R_1^2 + 2 R_2 R_1 r + R_2 r^2  \implies \\
    &\implies r^2 (R_2 - R_1) = R_2^2 R_2 - R_1^2 R_2 \implies \\
    &\implies r
        = \sqrt{ R_1 R_2 \frac{ R_2 - R_1 }{ R_2 - R_1 } }
        = \sqrt{ R_1 R_2 }
        = \sqrt{ 6 \units{Ом} \cdot 54 \units{Ом} }
        = 18{,}00 \units{Ом}.
        \\
     \eta_1
        &= \frac{ R_1 }{ R_1 + r }
        = \frac{ \sqrt{ R_1 } }{ \sqrt{ R_1 } + \sqrt{ R_2 } }
        = 0{,}25, \\
     \eta_2
        &= \frac{ R_2 }{ R_2 + r }
        = \frac{ \sqrt{ R_2 } }{ \sqrt{ R_2 } + \sqrt{ R_1 } }
        = 0{,}75
    \end{align*}
}
\vspace{150pt}

\tasknumber{4}\task{
    Определите ток, протекающий через резистор $R = 15\,\text{Ом}$ и разность потенциалов на нём (см.
    рис.
    на доске),
    если $r_1 = 1\,\text{Ом}$, $r_2 = 3\,\text{Ом}$, $\mathcal{E}_1 = 60\,\text{В}$, $\mathcal{E}_2 = 30\,\text{В}$
}
\newpage

\addpersonalvariant{Ислам Мунаев}
\tasknumber{1}\task{
    На резистор сопротивлением $r=18\units{Ом}$ подали напряжение $V=120\units{В}$.
    Определите ток, который потечёт через резистор, и мощность, выделяющуюся на нём.
}
\answer{
    \begin{align*}
    \mathcal{I} &= \frac{ V }{ r } = \frac{ 120 \units{В} }{ 18 \units{Ом} } = 6{,}67 \units{А}, \\
    P &= \frac{ V^2 }{ r } = \frac{ \sqr{ 120 \units{В} } }{ 18 \units{Ом} } = 800{,}00 \units{Вт}
    \end{align*}
}
\vspace{150pt}

\tasknumber{2}\task{
    Замкнутая электрическая цепь состоит из ЭДС $\mathcal{E}=30\units{В}$ и сопротивлением $r$
    и резистора $R=15\units{Ом}$.
    Определите ток, протекающий в цепи.
    Какая тепловая энергия выделится на резисторе за время
    $\tau=1\units{с}$? Какая работа будет совершена ЭДС за это время? Каков знак этой работы? Чему равен КПД цепи? Вычислите значения для 2 случаев:
    $r=0$ и $r=1\units{Ом}$.
}
\answer{
    \begin{align*}
    \mathcal{I}_1
                    &= \frac{ \mathcal{E} }{ R }
                    = \frac{ 30 \units{В} }{ 15 \units{Ом} }
                    = 2{,}00 \units{А}, \\
    \mathcal{I}_2
                    &= \frac{ \mathcal{E} }{ R + r }
                    = \frac{ 30 \units{В} }{ 15 \units{Ом} + 1 \units{Ом} }
                    = 1{,}88 \units{А}, \\
    Q_1
                    &= \mathcal{I}_1^2R\tau
                    = \cbr{ \frac{ \mathcal{E} }{ R } }^2 R \tau
                    = \cbr{ \frac{ 30 \units{В} }{ 15 \units{Ом} } }^2 \cdot 15 \units{Ом} \cdot 1 \units{с}
                    = 60{,}00 \units{Дж}, \\
    Q_2
                    &= \mathcal{I}_2^2R\tau
                    = \cbr{\frac{ \mathcal{E} }{ R + r } }^2 R \tau
                    = \cbr{\frac{ 30 \units{В} }{ 15 \units{Ом} + 1 \units{Ом} } }^2 \cdot 15 \units{Ом} \cdot 1 \units{с}
                    = 52{,}73 \units{Дж}, \\
    A_1
                    &= \mathcal{I}_1\tau\mathcal{E}
                    = \frac{ \mathcal{E} }{ R } \tau \mathcal{E}
                    = \frac{ \mathcal{E}^2 \tau }{ R }
                    = \frac{ \sqr{ 30 \units{В} } \cdot 1 \units{с} }{ 15 \units{Ом} }
                    = 60{,}00 \units{Дж}, \text{положительна}, \\
    A_2
                    &= \mathcal{I}_2\tau\mathcal{E}
                    = \frac{ \mathcal{E} }{ R + r } \tau \mathcal{E}
                    = \frac{ \mathcal{E}^2 \tau }{ R + r }
                    = \frac{ \sqr{ 30 \units{В} } \cdot 1 \units{с} }{ 15 \units{Ом} + 1 \units{Ом} }
                    = 56{,}25 \units{Дж}, \text{положительна}, \\
    \eta_1
                    &= \frac{ Q_1 }{ A_1 }
                    = \ldots
                    = \frac{ R }{ R }
                    = 1, \\
    \eta_2
                    &= \frac{ Q_2 }{ A_2 }
                    = \ldots
                    = \frac{ R }{ R + r }
                    = 0{,}9375
    \end{align*}
}
\vspace{150pt}

\tasknumber{3}\task{
    Лампочки, сопротивления которых $R_1=4\units{Ом}$ и $R_2=36\units{Ом}$, поочерёдно подключённные к некоторому источнику тока,
    потребляют одинаковую мощность.
    Найти внутреннее сопротивление источника и КПД цепи в каждом случае.
}
\answer{
    \begin{align*}
    P_1 &= \sqr{ \frac{ \mathcal{E} }{ R_1 + r } } R_1,
    P_2  = \sqr{ \frac{ \mathcal{E} }{ R_2 + r } } R_2,
    P_1 = P_2 \implies \\
    &\implies R_1 \sqr{ R_2 + r } = R_2 \sqr{ R_1 + r } \implies \\
    &\implies R_1 R_2^2 + 2 R_1 R_2 r + R_1 r^2 =
                R_2 R_1^2 + 2 R_2 R_1 r + R_2 r^2  \implies \\
    &\implies r^2 (R_2 - R_1) = R_2^2 R_2 - R_1^2 R_2 \implies \\
    &\implies r
        = \sqrt{ R_1 R_2 \frac{ R_2 - R_1 }{ R_2 - R_1 } }
        = \sqrt{ R_1 R_2 }
        = \sqrt{ 4 \units{Ом} \cdot 36 \units{Ом} }
        = 12{,}00 \units{Ом}.
        \\
     \eta_1
        &= \frac{ R_1 }{ R_1 + r }
        = \frac{ \sqrt{ R_1 } }{ \sqrt{ R_1 } + \sqrt{ R_2 } }
        = 0{,}25, \\
     \eta_2
        &= \frac{ R_2 }{ R_2 + r }
        = \frac{ \sqrt{ R_2 } }{ \sqrt{ R_2 } + \sqrt{ R_1 } }
        = 0{,}75
    \end{align*}
}
\vspace{150pt}

\tasknumber{4}\task{
    Определите ток, протекающий через резистор $R = 12\,\text{Ом}$ и разность потенциалов на нём (см.
    рис.
    на доске),
    если $r_1 = 3\,\text{Ом}$, $r_2 = 1\,\text{Ом}$, $\mathcal{E}_1 = 20\,\text{В}$, $\mathcal{E}_2 = 30\,\text{В}$
}
\newpage

\addpersonalvariant{Александр Наумов}
\tasknumber{1}\task{
    На резистор сопротивлением $R=18\units{Ом}$ подали напряжение $V=120\units{В}$.
    Определите ток, который потечёт через резистор, и мощность, выделяющуюся на нём.
}
\answer{
    \begin{align*}
    \mathcal{I} &= \frac{ V }{ R } = \frac{ 120 \units{В} }{ 18 \units{Ом} } = 6{,}67 \units{А}, \\
    P &= \frac{ V^2 }{ R } = \frac{ \sqr{ 120 \units{В} } }{ 18 \units{Ом} } = 800{,}00 \units{Вт}
    \end{align*}
}
\vspace{150pt}

\tasknumber{2}\task{
    Замкнутая электрическая цепь состоит из ЭДС $\mathcal{E}=20\units{В}$ и сопротивлением $r$
    и резистора $R=30\units{Ом}$.
    Определите ток, протекающий в цепи.
    Какая тепловая энергия выделится на резисторе за время
    $\tau=2\units{с}$? Какая работа будет совершена ЭДС за это время? Каков знак этой работы? Чему равен КПД цепи? Вычислите значения для 2 случаев:
    $r=0$ и $r=2\units{Ом}$.
}
\answer{
    \begin{align*}
    \mathcal{I}_1
                    &= \frac{ \mathcal{E} }{ R }
                    = \frac{ 20 \units{В} }{ 30 \units{Ом} }
                    = 0{,}67 \units{А}, \\
    \mathcal{I}_2
                    &= \frac{ \mathcal{E} }{ R + r }
                    = \frac{ 20 \units{В} }{ 30 \units{Ом} + 2 \units{Ом} }
                    = 0{,}62 \units{А}, \\
    Q_1
                    &= \mathcal{I}_1^2R\tau
                    = \cbr{ \frac{ \mathcal{E} }{ R } }^2 R \tau
                    = \cbr{ \frac{ 20 \units{В} }{ 30 \units{Ом} } }^2 \cdot 30 \units{Ом} \cdot 2 \units{с}
                    = 26{,}67 \units{Дж}, \\
    Q_2
                    &= \mathcal{I}_2^2R\tau
                    = \cbr{\frac{ \mathcal{E} }{ R + r } }^2 R \tau
                    = \cbr{\frac{ 20 \units{В} }{ 30 \units{Ом} + 2 \units{Ом} } }^2 \cdot 30 \units{Ом} \cdot 2 \units{с}
                    = 23{,}44 \units{Дж}, \\
    A_1
                    &= \mathcal{I}_1\tau\mathcal{E}
                    = \frac{ \mathcal{E} }{ R } \tau \mathcal{E}
                    = \frac{ \mathcal{E}^2 \tau }{ R }
                    = \frac{ \sqr{ 20 \units{В} } \cdot 2 \units{с} }{ 30 \units{Ом} }
                    = 26{,}67 \units{Дж}, \text{положительна}, \\
    A_2
                    &= \mathcal{I}_2\tau\mathcal{E}
                    = \frac{ \mathcal{E} }{ R + r } \tau \mathcal{E}
                    = \frac{ \mathcal{E}^2 \tau }{ R + r }
                    = \frac{ \sqr{ 20 \units{В} } \cdot 2 \units{с} }{ 30 \units{Ом} + 2 \units{Ом} }
                    = 25{,}00 \units{Дж}, \text{положительна}, \\
    \eta_1
                    &= \frac{ Q_1 }{ A_1 }
                    = \ldots
                    = \frac{ R }{ R }
                    = 1, \\
    \eta_2
                    &= \frac{ Q_2 }{ A_2 }
                    = \ldots
                    = \frac{ R }{ R + r }
                    = 0{,}9375
    \end{align*}
}
\vspace{150pt}

\tasknumber{3}\task{
    Лампочки, сопротивления которых $R_1=5\units{Ом}$ и $R_2=45\units{Ом}$, поочерёдно подключённные к некоторому источнику тока,
    потребляют одинаковую мощность.
    Найти внутреннее сопротивление источника и КПД цепи в каждом случае.
}
\answer{
    \begin{align*}
    P_1 &= \sqr{ \frac{ \mathcal{E} }{ R_1 + r } } R_1,
    P_2  = \sqr{ \frac{ \mathcal{E} }{ R_2 + r } } R_2,
    P_1 = P_2 \implies \\
    &\implies R_1 \sqr{ R_2 + r } = R_2 \sqr{ R_1 + r } \implies \\
    &\implies R_1 R_2^2 + 2 R_1 R_2 r + R_1 r^2 =
                R_2 R_1^2 + 2 R_2 R_1 r + R_2 r^2  \implies \\
    &\implies r^2 (R_2 - R_1) = R_2^2 R_2 - R_1^2 R_2 \implies \\
    &\implies r
        = \sqrt{ R_1 R_2 \frac{ R_2 - R_1 }{ R_2 - R_1 } }
        = \sqrt{ R_1 R_2 }
        = \sqrt{ 5 \units{Ом} \cdot 45 \units{Ом} }
        = 15{,}00 \units{Ом}.
        \\
     \eta_1
        &= \frac{ R_1 }{ R_1 + r }
        = \frac{ \sqrt{ R_1 } }{ \sqrt{ R_1 } + \sqrt{ R_2 } }
        = 0{,}25, \\
     \eta_2
        &= \frac{ R_2 }{ R_2 + r }
        = \frac{ \sqrt{ R_2 } }{ \sqrt{ R_2 } + \sqrt{ R_1 } }
        = 0{,}75
    \end{align*}
}
\vspace{150pt}

\tasknumber{4}\task{
    Определите ток, протекающий через резистор $R = 12\,\text{Ом}$ и разность потенциалов на нём (см.
    рис.
    на доске),
    если $r_1 = 2\,\text{Ом}$, $r_2 = 3\,\text{Ом}$, $\mathcal{E}_1 = 40\,\text{В}$, $\mathcal{E}_2 = 30\,\text{В}$
}
\newpage

\addpersonalvariant{Георгий Новиков}
\tasknumber{1}\task{
    Через резистор сопротивлением $r=30\units{Ом}$ протекает электрический ток $\mathcal{I}=6\units{А}$.
    Определите, чему равны напряжение на резисторе и мощность, выделяющаяся на нём.
}
\answer{
    \begin{align*}
    U &= \mathcal{I}r = 6 \units{А} \cdot 30 \units{Ом} = 180 \units{В}, \\
    P &= \mathcal{I}^2r = \sqr{ 6 \units{А} } \cdot 30 \units{Ом} = 1080 \units{Вт}
    \end{align*}
}
\vspace{150pt}

\tasknumber{2}\task{
    Замкнутая электрическая цепь состоит из ЭДС $\mathcal{E}=30\units{В}$ и сопротивлением $r$
    и резистора $R=10\units{Ом}$.
    Определите ток, протекающий в цепи.
    Какая тепловая энергия выделится на резисторе за время
    $\tau=1\units{с}$? Какая работа будет совершена ЭДС за это время? Каков знак этой работы? Чему равен КПД цепи? Вычислите значения для 2 случаев:
    $r=0$ и $r=1\units{Ом}$.
}
\answer{
    \begin{align*}
    \mathcal{I}_1
                    &= \frac{ \mathcal{E} }{ R }
                    = \frac{ 30 \units{В} }{ 10 \units{Ом} }
                    = 3{,}00 \units{А}, \\
    \mathcal{I}_2
                    &= \frac{ \mathcal{E} }{ R + r }
                    = \frac{ 30 \units{В} }{ 10 \units{Ом} + 1 \units{Ом} }
                    = 2{,}73 \units{А}, \\
    Q_1
                    &= \mathcal{I}_1^2R\tau
                    = \cbr{ \frac{ \mathcal{E} }{ R } }^2 R \tau
                    = \cbr{ \frac{ 30 \units{В} }{ 10 \units{Ом} } }^2 \cdot 10 \units{Ом} \cdot 1 \units{с}
                    = 90{,}00 \units{Дж}, \\
    Q_2
                    &= \mathcal{I}_2^2R\tau
                    = \cbr{\frac{ \mathcal{E} }{ R + r } }^2 R \tau
                    = \cbr{\frac{ 30 \units{В} }{ 10 \units{Ом} + 1 \units{Ом} } }^2 \cdot 10 \units{Ом} \cdot 1 \units{с}
                    = 74{,}38 \units{Дж}, \\
    A_1
                    &= \mathcal{I}_1\tau\mathcal{E}
                    = \frac{ \mathcal{E} }{ R } \tau \mathcal{E}
                    = \frac{ \mathcal{E}^2 \tau }{ R }
                    = \frac{ \sqr{ 30 \units{В} } \cdot 1 \units{с} }{ 10 \units{Ом} }
                    = 90{,}00 \units{Дж}, \text{положительна}, \\
    A_2
                    &= \mathcal{I}_2\tau\mathcal{E}
                    = \frac{ \mathcal{E} }{ R + r } \tau \mathcal{E}
                    = \frac{ \mathcal{E}^2 \tau }{ R + r }
                    = \frac{ \sqr{ 30 \units{В} } \cdot 1 \units{с} }{ 10 \units{Ом} + 1 \units{Ом} }
                    = 81{,}82 \units{Дж}, \text{положительна}, \\
    \eta_1
                    &= \frac{ Q_1 }{ A_1 }
                    = \ldots
                    = \frac{ R }{ R }
                    = 1, \\
    \eta_2
                    &= \frac{ Q_2 }{ A_2 }
                    = \ldots
                    = \frac{ R }{ R + r }
                    = 0{,}909090909091
    \end{align*}
}
\vspace{150pt}

\tasknumber{3}\task{
    Лампочки, сопротивления которых $R_1=0.5\units{Ом}$ и $R_2=4.5\units{Ом}$, поочерёдно подключённные к некоторому источнику тока,
    потребляют одинаковую мощность.
    Найти внутреннее сопротивление источника и КПД цепи в каждом случае.
}
\answer{
    \begin{align*}
    P_1 &= \sqr{ \frac{ \mathcal{E} }{ R_1 + r } } R_1,
    P_2  = \sqr{ \frac{ \mathcal{E} }{ R_2 + r } } R_2,
    P_1 = P_2 \implies \\
    &\implies R_1 \sqr{ R_2 + r } = R_2 \sqr{ R_1 + r } \implies \\
    &\implies R_1 R_2^2 + 2 R_1 R_2 r + R_1 r^2 =
                R_2 R_1^2 + 2 R_2 R_1 r + R_2 r^2  \implies \\
    &\implies r^2 (R_2 - R_1) = R_2^2 R_2 - R_1^2 R_2 \implies \\
    &\implies r
        = \sqrt{ R_1 R_2 \frac{ R_2 - R_1 }{ R_2 - R_1 } }
        = \sqrt{ R_1 R_2 }
        = \sqrt{ 0{,}50 \units{Ом} \cdot 4{,}50 \units{Ом} }
        = 1{,}50 \units{Ом}.
        \\
     \eta_1
        &= \frac{ R_1 }{ R_1 + r }
        = \frac{ \sqrt{ R_1 } }{ \sqrt{ R_1 } + \sqrt{ R_2 } }
        = 0{,}25, \\
     \eta_2
        &= \frac{ R_2 }{ R_2 + r }
        = \frac{ \sqrt{ R_2 } }{ \sqrt{ R_2 } + \sqrt{ R_1 } }
        = 0{,}75
    \end{align*}
}
\vspace{150pt}

\tasknumber{4}\task{
    Определите ток, протекающий через резистор $R = 12\,\text{Ом}$ и разность потенциалов на нём (см.
    рис.
    на доске),
    если $r_1 = 2\,\text{Ом}$, $r_2 = 3\,\text{Ом}$, $\mathcal{E}_1 = 60\,\text{В}$, $\mathcal{E}_2 = 30\,\text{В}$
}
\newpage

\addpersonalvariant{Егор Осипов}
\tasknumber{1}\task{
    Через резистор сопротивлением $r=30\units{Ом}$ протекает электрический ток $\mathcal{I}=3\units{А}$.
    Определите, чему равны напряжение на резисторе и мощность, выделяющаяся на нём.
}
\answer{
    \begin{align*}
    U &= \mathcal{I}r = 3 \units{А} \cdot 30 \units{Ом} = 90 \units{В}, \\
    P &= \mathcal{I}^2r = \sqr{ 3 \units{А} } \cdot 30 \units{Ом} = 270 \units{Вт}
    \end{align*}
}
\vspace{150pt}

\tasknumber{2}\task{
    Замкнутая электрическая цепь состоит из ЭДС $\mathcal{E}=10\units{В}$ и сопротивлением $r$
    и резистора $R=30\units{Ом}$.
    Определите ток, протекающий в цепи.
    Какая тепловая энергия выделится на резисторе за время
    $\tau=3\units{с}$? Какая работа будет совершена ЭДС за это время? Каков знак этой работы? Чему равен КПД цепи? Вычислите значения для 2 случаев:
    $r=0$ и $r=3\units{Ом}$.
}
\answer{
    \begin{align*}
    \mathcal{I}_1
                    &= \frac{ \mathcal{E} }{ R }
                    = \frac{ 10 \units{В} }{ 30 \units{Ом} }
                    = 0{,}33 \units{А}, \\
    \mathcal{I}_2
                    &= \frac{ \mathcal{E} }{ R + r }
                    = \frac{ 10 \units{В} }{ 30 \units{Ом} + 3 \units{Ом} }
                    = 0{,}30 \units{А}, \\
    Q_1
                    &= \mathcal{I}_1^2R\tau
                    = \cbr{ \frac{ \mathcal{E} }{ R } }^2 R \tau
                    = \cbr{ \frac{ 10 \units{В} }{ 30 \units{Ом} } }^2 \cdot 30 \units{Ом} \cdot 3 \units{с}
                    = 10{,}00 \units{Дж}, \\
    Q_2
                    &= \mathcal{I}_2^2R\tau
                    = \cbr{\frac{ \mathcal{E} }{ R + r } }^2 R \tau
                    = \cbr{\frac{ 10 \units{В} }{ 30 \units{Ом} + 3 \units{Ом} } }^2 \cdot 30 \units{Ом} \cdot 3 \units{с}
                    = 8{,}26 \units{Дж}, \\
    A_1
                    &= \mathcal{I}_1\tau\mathcal{E}
                    = \frac{ \mathcal{E} }{ R } \tau \mathcal{E}
                    = \frac{ \mathcal{E}^2 \tau }{ R }
                    = \frac{ \sqr{ 10 \units{В} } \cdot 3 \units{с} }{ 30 \units{Ом} }
                    = 10{,}00 \units{Дж}, \text{положительна}, \\
    A_2
                    &= \mathcal{I}_2\tau\mathcal{E}
                    = \frac{ \mathcal{E} }{ R + r } \tau \mathcal{E}
                    = \frac{ \mathcal{E}^2 \tau }{ R + r }
                    = \frac{ \sqr{ 10 \units{В} } \cdot 3 \units{с} }{ 30 \units{Ом} + 3 \units{Ом} }
                    = 9{,}09 \units{Дж}, \text{положительна}, \\
    \eta_1
                    &= \frac{ Q_1 }{ A_1 }
                    = \ldots
                    = \frac{ R }{ R }
                    = 1, \\
    \eta_2
                    &= \frac{ Q_2 }{ A_2 }
                    = \ldots
                    = \frac{ R }{ R + r }
                    = 0{,}909090909091
    \end{align*}
}
\vspace{150pt}

\tasknumber{3}\task{
    Лампочки, сопротивления которых $R_1=6\units{Ом}$ и $R_2=54\units{Ом}$, поочерёдно подключённные к некоторому источнику тока,
    потребляют одинаковую мощность.
    Найти внутреннее сопротивление источника и КПД цепи в каждом случае.
}
\answer{
    \begin{align*}
    P_1 &= \sqr{ \frac{ \mathcal{E} }{ R_1 + r } } R_1,
    P_2  = \sqr{ \frac{ \mathcal{E} }{ R_2 + r } } R_2,
    P_1 = P_2 \implies \\
    &\implies R_1 \sqr{ R_2 + r } = R_2 \sqr{ R_1 + r } \implies \\
    &\implies R_1 R_2^2 + 2 R_1 R_2 r + R_1 r^2 =
                R_2 R_1^2 + 2 R_2 R_1 r + R_2 r^2  \implies \\
    &\implies r^2 (R_2 - R_1) = R_2^2 R_2 - R_1^2 R_2 \implies \\
    &\implies r
        = \sqrt{ R_1 R_2 \frac{ R_2 - R_1 }{ R_2 - R_1 } }
        = \sqrt{ R_1 R_2 }
        = \sqrt{ 6 \units{Ом} \cdot 54 \units{Ом} }
        = 18{,}00 \units{Ом}.
        \\
     \eta_1
        &= \frac{ R_1 }{ R_1 + r }
        = \frac{ \sqrt{ R_1 } }{ \sqrt{ R_1 } + \sqrt{ R_2 } }
        = 0{,}25, \\
     \eta_2
        &= \frac{ R_2 }{ R_2 + r }
        = \frac{ \sqrt{ R_2 } }{ \sqrt{ R_2 } + \sqrt{ R_1 } }
        = 0{,}75
    \end{align*}
}
\vspace{150pt}

\tasknumber{4}\task{
    Определите ток, протекающий через резистор $R = 12\,\text{Ом}$ и разность потенциалов на нём (см.
    рис.
    на доске),
    если $r_1 = 1\,\text{Ом}$, $r_2 = 3\,\text{Ом}$, $\mathcal{E}_1 = 40\,\text{В}$, $\mathcal{E}_2 = 60\,\text{В}$
}
\newpage

\addpersonalvariant{Руслан Перепелица}
\tasknumber{1}\task{
    На резистор сопротивлением $R=30\units{Ом}$ подали напряжение $U=240\units{В}$.
    Определите ток, который потечёт через резистор, и мощность, выделяющуюся на нём.
}
\answer{
    \begin{align*}
    \mathcal{I} &= \frac{ U }{ R } = \frac{ 240 \units{В} }{ 30 \units{Ом} } = 8{,}00 \units{А}, \\
    P &= \frac{ U^2 }{ R } = \frac{ \sqr{ 240 \units{В} } }{ 30 \units{Ом} } = 1920{,}00 \units{Вт}
    \end{align*}
}
\vspace{150pt}

\tasknumber{2}\task{
    Замкнутая электрическая цепь состоит из ЭДС $\mathcal{E}=20\units{В}$ и сопротивлением $r$
    и резистора $R=24\units{Ом}$.
    Определите ток, протекающий в цепи.
    Какая тепловая энергия выделится на резисторе за время
    $\tau=1\units{с}$? Какая работа будет совершена ЭДС за это время? Каков знак этой работы? Чему равен КПД цепи? Вычислите значения для 2 случаев:
    $r=0$ и $r=1\units{Ом}$.
}
\answer{
    \begin{align*}
    \mathcal{I}_1
                    &= \frac{ \mathcal{E} }{ R }
                    = \frac{ 20 \units{В} }{ 24 \units{Ом} }
                    = 0{,}83 \units{А}, \\
    \mathcal{I}_2
                    &= \frac{ \mathcal{E} }{ R + r }
                    = \frac{ 20 \units{В} }{ 24 \units{Ом} + 1 \units{Ом} }
                    = 0{,}80 \units{А}, \\
    Q_1
                    &= \mathcal{I}_1^2R\tau
                    = \cbr{ \frac{ \mathcal{E} }{ R } }^2 R \tau
                    = \cbr{ \frac{ 20 \units{В} }{ 24 \units{Ом} } }^2 \cdot 24 \units{Ом} \cdot 1 \units{с}
                    = 16{,}67 \units{Дж}, \\
    Q_2
                    &= \mathcal{I}_2^2R\tau
                    = \cbr{\frac{ \mathcal{E} }{ R + r } }^2 R \tau
                    = \cbr{\frac{ 20 \units{В} }{ 24 \units{Ом} + 1 \units{Ом} } }^2 \cdot 24 \units{Ом} \cdot 1 \units{с}
                    = 15{,}36 \units{Дж}, \\
    A_1
                    &= \mathcal{I}_1\tau\mathcal{E}
                    = \frac{ \mathcal{E} }{ R } \tau \mathcal{E}
                    = \frac{ \mathcal{E}^2 \tau }{ R }
                    = \frac{ \sqr{ 20 \units{В} } \cdot 1 \units{с} }{ 24 \units{Ом} }
                    = 16{,}67 \units{Дж}, \text{положительна}, \\
    A_2
                    &= \mathcal{I}_2\tau\mathcal{E}
                    = \frac{ \mathcal{E} }{ R + r } \tau \mathcal{E}
                    = \frac{ \mathcal{E}^2 \tau }{ R + r }
                    = \frac{ \sqr{ 20 \units{В} } \cdot 1 \units{с} }{ 24 \units{Ом} + 1 \units{Ом} }
                    = 16{,}00 \units{Дж}, \text{положительна}, \\
    \eta_1
                    &= \frac{ Q_1 }{ A_1 }
                    = \ldots
                    = \frac{ R }{ R }
                    = 1, \\
    \eta_2
                    &= \frac{ Q_2 }{ A_2 }
                    = \ldots
                    = \frac{ R }{ R + r }
                    = 0{,}96
    \end{align*}
}
\vspace{150pt}

\tasknumber{3}\task{
    Лампочки, сопротивления которых $R_1=1\units{Ом}$ и $R_2=9\units{Ом}$, поочерёдно подключённные к некоторому источнику тока,
    потребляют одинаковую мощность.
    Найти внутреннее сопротивление источника и КПД цепи в каждом случае.
}
\answer{
    \begin{align*}
    P_1 &= \sqr{ \frac{ \mathcal{E} }{ R_1 + r } } R_1,
    P_2  = \sqr{ \frac{ \mathcal{E} }{ R_2 + r } } R_2,
    P_1 = P_2 \implies \\
    &\implies R_1 \sqr{ R_2 + r } = R_2 \sqr{ R_1 + r } \implies \\
    &\implies R_1 R_2^2 + 2 R_1 R_2 r + R_1 r^2 =
                R_2 R_1^2 + 2 R_2 R_1 r + R_2 r^2  \implies \\
    &\implies r^2 (R_2 - R_1) = R_2^2 R_2 - R_1^2 R_2 \implies \\
    &\implies r
        = \sqrt{ R_1 R_2 \frac{ R_2 - R_1 }{ R_2 - R_1 } }
        = \sqrt{ R_1 R_2 }
        = \sqrt{ 1 \units{Ом} \cdot 9 \units{Ом} }
        = 3{,}00 \units{Ом}.
        \\
     \eta_1
        &= \frac{ R_1 }{ R_1 + r }
        = \frac{ \sqrt{ R_1 } }{ \sqrt{ R_1 } + \sqrt{ R_2 } }
        = 0{,}25, \\
     \eta_2
        &= \frac{ R_2 }{ R_2 + r }
        = \frac{ \sqrt{ R_2 } }{ \sqrt{ R_2 } + \sqrt{ R_1 } }
        = 0{,}75
    \end{align*}
}
\vspace{150pt}

\tasknumber{4}\task{
    Определите ток, протекающий через резистор $R = 10\,\text{Ом}$ и разность потенциалов на нём (см.
    рис.
    на доске),
    если $r_1 = 2\,\text{Ом}$, $r_2 = 1\,\text{Ом}$, $\mathcal{E}_1 = 30\,\text{В}$, $\mathcal{E}_2 = 60\,\text{В}$
}
\newpage

\addpersonalvariant{Михаил Перин}
\tasknumber{1}\task{
    На резистор сопротивлением $R=12\units{Ом}$ подали напряжение $V=120\units{В}$.
    Определите ток, который потечёт через резистор, и мощность, выделяющуюся на нём.
}
\answer{
    \begin{align*}
    \mathcal{I} &= \frac{ V }{ R } = \frac{ 120 \units{В} }{ 12 \units{Ом} } = 10{,}00 \units{А}, \\
    P &= \frac{ V^2 }{ R } = \frac{ \sqr{ 120 \units{В} } }{ 12 \units{Ом} } = 1200{,}00 \units{Вт}
    \end{align*}
}
\vspace{150pt}

\tasknumber{2}\task{
    Замкнутая электрическая цепь состоит из ЭДС $\mathcal{E}=10\units{В}$ и сопротивлением $r$
    и резистора $R=24\units{Ом}$.
    Определите ток, протекающий в цепи.
    Какая тепловая энергия выделится на резисторе за время
    $\tau=3\units{с}$? Какая работа будет совершена ЭДС за это время? Каков знак этой работы? Чему равен КПД цепи? Вычислите значения для 2 случаев:
    $r=0$ и $r=3\units{Ом}$.
}
\answer{
    \begin{align*}
    \mathcal{I}_1
                    &= \frac{ \mathcal{E} }{ R }
                    = \frac{ 10 \units{В} }{ 24 \units{Ом} }
                    = 0{,}42 \units{А}, \\
    \mathcal{I}_2
                    &= \frac{ \mathcal{E} }{ R + r }
                    = \frac{ 10 \units{В} }{ 24 \units{Ом} + 3 \units{Ом} }
                    = 0{,}37 \units{А}, \\
    Q_1
                    &= \mathcal{I}_1^2R\tau
                    = \cbr{ \frac{ \mathcal{E} }{ R } }^2 R \tau
                    = \cbr{ \frac{ 10 \units{В} }{ 24 \units{Ом} } }^2 \cdot 24 \units{Ом} \cdot 3 \units{с}
                    = 12{,}50 \units{Дж}, \\
    Q_2
                    &= \mathcal{I}_2^2R\tau
                    = \cbr{\frac{ \mathcal{E} }{ R + r } }^2 R \tau
                    = \cbr{\frac{ 10 \units{В} }{ 24 \units{Ом} + 3 \units{Ом} } }^2 \cdot 24 \units{Ом} \cdot 3 \units{с}
                    = 9{,}88 \units{Дж}, \\
    A_1
                    &= \mathcal{I}_1\tau\mathcal{E}
                    = \frac{ \mathcal{E} }{ R } \tau \mathcal{E}
                    = \frac{ \mathcal{E}^2 \tau }{ R }
                    = \frac{ \sqr{ 10 \units{В} } \cdot 3 \units{с} }{ 24 \units{Ом} }
                    = 12{,}50 \units{Дж}, \text{положительна}, \\
    A_2
                    &= \mathcal{I}_2\tau\mathcal{E}
                    = \frac{ \mathcal{E} }{ R + r } \tau \mathcal{E}
                    = \frac{ \mathcal{E}^2 \tau }{ R + r }
                    = \frac{ \sqr{ 10 \units{В} } \cdot 3 \units{с} }{ 24 \units{Ом} + 3 \units{Ом} }
                    = 11{,}11 \units{Дж}, \text{положительна}, \\
    \eta_1
                    &= \frac{ Q_1 }{ A_1 }
                    = \ldots
                    = \frac{ R }{ R }
                    = 1, \\
    \eta_2
                    &= \frac{ Q_2 }{ A_2 }
                    = \ldots
                    = \frac{ R }{ R + r }
                    = 0{,}888888888889
    \end{align*}
}
\vspace{150pt}

\tasknumber{3}\task{
    Лампочки, сопротивления которых $R_1=0.5\units{Ом}$ и $R_2=18\units{Ом}$, поочерёдно подключённные к некоторому источнику тока,
    потребляют одинаковую мощность.
    Найти внутреннее сопротивление источника и КПД цепи в каждом случае.
}
\answer{
    \begin{align*}
    P_1 &= \sqr{ \frac{ \mathcal{E} }{ R_1 + r } } R_1,
    P_2  = \sqr{ \frac{ \mathcal{E} }{ R_2 + r } } R_2,
    P_1 = P_2 \implies \\
    &\implies R_1 \sqr{ R_2 + r } = R_2 \sqr{ R_1 + r } \implies \\
    &\implies R_1 R_2^2 + 2 R_1 R_2 r + R_1 r^2 =
                R_2 R_1^2 + 2 R_2 R_1 r + R_2 r^2  \implies \\
    &\implies r^2 (R_2 - R_1) = R_2^2 R_2 - R_1^2 R_2 \implies \\
    &\implies r
        = \sqrt{ R_1 R_2 \frac{ R_2 - R_1 }{ R_2 - R_1 } }
        = \sqrt{ R_1 R_2 }
        = \sqrt{ 0{,}50 \units{Ом} \cdot 18 \units{Ом} }
        = 3{,}00 \units{Ом}.
        \\
     \eta_1
        &= \frac{ R_1 }{ R_1 + r }
        = \frac{ \sqrt{ R_1 } }{ \sqrt{ R_1 } + \sqrt{ R_2 } }
        = 0{,}142857142857, \\
     \eta_2
        &= \frac{ R_2 }{ R_2 + r }
        = \frac{ \sqrt{ R_2 } }{ \sqrt{ R_2 } + \sqrt{ R_1 } }
        = 0{,}857142857143
    \end{align*}
}
\vspace{150pt}

\tasknumber{4}\task{
    Определите ток, протекающий через резистор $R = 18\,\text{Ом}$ и разность потенциалов на нём (см.
    рис.
    на доске),
    если $r_1 = 2\,\text{Ом}$, $r_2 = 1\,\text{Ом}$, $\mathcal{E}_1 = 30\,\text{В}$, $\mathcal{E}_2 = 40\,\text{В}$
}
\newpage

\addpersonalvariant{Егор Подуровский}
\tasknumber{1}\task{
    На резистор сопротивлением $R=30\units{Ом}$ подали напряжение $V=240\units{В}$.
    Определите ток, который потечёт через резистор, и мощность, выделяющуюся на нём.
}
\answer{
    \begin{align*}
    \mathcal{I} &= \frac{ V }{ R } = \frac{ 240 \units{В} }{ 30 \units{Ом} } = 8{,}00 \units{А}, \\
    P &= \frac{ V^2 }{ R } = \frac{ \sqr{ 240 \units{В} } }{ 30 \units{Ом} } = 1920{,}00 \units{Вт}
    \end{align*}
}
\vspace{150pt}

\tasknumber{2}\task{
    Замкнутая электрическая цепь состоит из ЭДС $\mathcal{E}=30\units{В}$ и сопротивлением $r$
    и резистора $R=24\units{Ом}$.
    Определите ток, протекающий в цепи.
    Какая тепловая энергия выделится на резисторе за время
    $\tau=1\units{с}$? Какая работа будет совершена ЭДС за это время? Каков знак этой работы? Чему равен КПД цепи? Вычислите значения для 2 случаев:
    $r=0$ и $r=1\units{Ом}$.
}
\answer{
    \begin{align*}
    \mathcal{I}_1
                    &= \frac{ \mathcal{E} }{ R }
                    = \frac{ 30 \units{В} }{ 24 \units{Ом} }
                    = 1{,}25 \units{А}, \\
    \mathcal{I}_2
                    &= \frac{ \mathcal{E} }{ R + r }
                    = \frac{ 30 \units{В} }{ 24 \units{Ом} + 1 \units{Ом} }
                    = 1{,}20 \units{А}, \\
    Q_1
                    &= \mathcal{I}_1^2R\tau
                    = \cbr{ \frac{ \mathcal{E} }{ R } }^2 R \tau
                    = \cbr{ \frac{ 30 \units{В} }{ 24 \units{Ом} } }^2 \cdot 24 \units{Ом} \cdot 1 \units{с}
                    = 37{,}50 \units{Дж}, \\
    Q_2
                    &= \mathcal{I}_2^2R\tau
                    = \cbr{\frac{ \mathcal{E} }{ R + r } }^2 R \tau
                    = \cbr{\frac{ 30 \units{В} }{ 24 \units{Ом} + 1 \units{Ом} } }^2 \cdot 24 \units{Ом} \cdot 1 \units{с}
                    = 34{,}56 \units{Дж}, \\
    A_1
                    &= \mathcal{I}_1\tau\mathcal{E}
                    = \frac{ \mathcal{E} }{ R } \tau \mathcal{E}
                    = \frac{ \mathcal{E}^2 \tau }{ R }
                    = \frac{ \sqr{ 30 \units{В} } \cdot 1 \units{с} }{ 24 \units{Ом} }
                    = 37{,}50 \units{Дж}, \text{положительна}, \\
    A_2
                    &= \mathcal{I}_2\tau\mathcal{E}
                    = \frac{ \mathcal{E} }{ R + r } \tau \mathcal{E}
                    = \frac{ \mathcal{E}^2 \tau }{ R + r }
                    = \frac{ \sqr{ 30 \units{В} } \cdot 1 \units{с} }{ 24 \units{Ом} + 1 \units{Ом} }
                    = 36{,}00 \units{Дж}, \text{положительна}, \\
    \eta_1
                    &= \frac{ Q_1 }{ A_1 }
                    = \ldots
                    = \frac{ R }{ R }
                    = 1, \\
    \eta_2
                    &= \frac{ Q_2 }{ A_2 }
                    = \ldots
                    = \frac{ R }{ R + r }
                    = 0{,}96
    \end{align*}
}
\vspace{150pt}

\tasknumber{3}\task{
    Лампочки, сопротивления которых $R_1=0.5\units{Ом}$ и $R_2=2\units{Ом}$, поочерёдно подключённные к некоторому источнику тока,
    потребляют одинаковую мощность.
    Найти внутреннее сопротивление источника и КПД цепи в каждом случае.
}
\answer{
    \begin{align*}
    P_1 &= \sqr{ \frac{ \mathcal{E} }{ R_1 + r } } R_1,
    P_2  = \sqr{ \frac{ \mathcal{E} }{ R_2 + r } } R_2,
    P_1 = P_2 \implies \\
    &\implies R_1 \sqr{ R_2 + r } = R_2 \sqr{ R_1 + r } \implies \\
    &\implies R_1 R_2^2 + 2 R_1 R_2 r + R_1 r^2 =
                R_2 R_1^2 + 2 R_2 R_1 r + R_2 r^2  \implies \\
    &\implies r^2 (R_2 - R_1) = R_2^2 R_2 - R_1^2 R_2 \implies \\
    &\implies r
        = \sqrt{ R_1 R_2 \frac{ R_2 - R_1 }{ R_2 - R_1 } }
        = \sqrt{ R_1 R_2 }
        = \sqrt{ 0{,}50 \units{Ом} \cdot 2 \units{Ом} }
        = 1{,}00 \units{Ом}.
        \\
     \eta_1
        &= \frac{ R_1 }{ R_1 + r }
        = \frac{ \sqrt{ R_1 } }{ \sqrt{ R_1 } + \sqrt{ R_2 } }
        = 0{,}333333333333, \\
     \eta_2
        &= \frac{ R_2 }{ R_2 + r }
        = \frac{ \sqrt{ R_2 } }{ \sqrt{ R_2 } + \sqrt{ R_1 } }
        = 0{,}666666666667
    \end{align*}
}
\vspace{150pt}

\tasknumber{4}\task{
    Определите ток, протекающий через резистор $R = 10\,\text{Ом}$ и разность потенциалов на нём (см.
    рис.
    на доске),
    если $r_1 = 2\,\text{Ом}$, $r_2 = 3\,\text{Ом}$, $\mathcal{E}_1 = 40\,\text{В}$, $\mathcal{E}_2 = 60\,\text{В}$
}
\newpage

\addpersonalvariant{Роман Прибылов}
\tasknumber{1}\task{
    Через резистор сопротивлением $R=30\units{Ом}$ протекает электрический ток $\mathcal{I}=6\units{А}$.
    Определите, чему равны напряжение на резисторе и мощность, выделяющаяся на нём.
}
\answer{
    \begin{align*}
    U &= \mathcal{I}R = 6 \units{А} \cdot 30 \units{Ом} = 180 \units{В}, \\
    P &= \mathcal{I}^2R = \sqr{ 6 \units{А} } \cdot 30 \units{Ом} = 1080 \units{Вт}
    \end{align*}
}
\vspace{150pt}

\tasknumber{2}\task{
    Замкнутая электрическая цепь состоит из ЭДС $\mathcal{E}=30\units{В}$ и сопротивлением $r$
    и резистора $R=15\units{Ом}$.
    Определите ток, протекающий в цепи.
    Какая тепловая энергия выделится на резисторе за время
    $\tau=2\units{с}$? Какая работа будет совершена ЭДС за это время? Каков знак этой работы? Чему равен КПД цепи? Вычислите значения для 2 случаев:
    $r=0$ и $r=2\units{Ом}$.
}
\answer{
    \begin{align*}
    \mathcal{I}_1
                    &= \frac{ \mathcal{E} }{ R }
                    = \frac{ 30 \units{В} }{ 15 \units{Ом} }
                    = 2{,}00 \units{А}, \\
    \mathcal{I}_2
                    &= \frac{ \mathcal{E} }{ R + r }
                    = \frac{ 30 \units{В} }{ 15 \units{Ом} + 2 \units{Ом} }
                    = 1{,}76 \units{А}, \\
    Q_1
                    &= \mathcal{I}_1^2R\tau
                    = \cbr{ \frac{ \mathcal{E} }{ R } }^2 R \tau
                    = \cbr{ \frac{ 30 \units{В} }{ 15 \units{Ом} } }^2 \cdot 15 \units{Ом} \cdot 2 \units{с}
                    = 120{,}00 \units{Дж}, \\
    Q_2
                    &= \mathcal{I}_2^2R\tau
                    = \cbr{\frac{ \mathcal{E} }{ R + r } }^2 R \tau
                    = \cbr{\frac{ 30 \units{В} }{ 15 \units{Ом} + 2 \units{Ом} } }^2 \cdot 15 \units{Ом} \cdot 2 \units{с}
                    = 93{,}43 \units{Дж}, \\
    A_1
                    &= \mathcal{I}_1\tau\mathcal{E}
                    = \frac{ \mathcal{E} }{ R } \tau \mathcal{E}
                    = \frac{ \mathcal{E}^2 \tau }{ R }
                    = \frac{ \sqr{ 30 \units{В} } \cdot 2 \units{с} }{ 15 \units{Ом} }
                    = 120{,}00 \units{Дж}, \text{положительна}, \\
    A_2
                    &= \mathcal{I}_2\tau\mathcal{E}
                    = \frac{ \mathcal{E} }{ R + r } \tau \mathcal{E}
                    = \frac{ \mathcal{E}^2 \tau }{ R + r }
                    = \frac{ \sqr{ 30 \units{В} } \cdot 2 \units{с} }{ 15 \units{Ом} + 2 \units{Ом} }
                    = 105{,}88 \units{Дж}, \text{положительна}, \\
    \eta_1
                    &= \frac{ Q_1 }{ A_1 }
                    = \ldots
                    = \frac{ R }{ R }
                    = 1, \\
    \eta_2
                    &= \frac{ Q_2 }{ A_2 }
                    = \ldots
                    = \frac{ R }{ R + r }
                    = 0{,}882352941176
    \end{align*}
}
\vspace{150pt}

\tasknumber{3}\task{
    Лампочки, сопротивления которых $R_1=1\units{Ом}$ и $R_2=9\units{Ом}$, поочерёдно подключённные к некоторому источнику тока,
    потребляют одинаковую мощность.
    Найти внутреннее сопротивление источника и КПД цепи в каждом случае.
}
\answer{
    \begin{align*}
    P_1 &= \sqr{ \frac{ \mathcal{E} }{ R_1 + r } } R_1,
    P_2  = \sqr{ \frac{ \mathcal{E} }{ R_2 + r } } R_2,
    P_1 = P_2 \implies \\
    &\implies R_1 \sqr{ R_2 + r } = R_2 \sqr{ R_1 + r } \implies \\
    &\implies R_1 R_2^2 + 2 R_1 R_2 r + R_1 r^2 =
                R_2 R_1^2 + 2 R_2 R_1 r + R_2 r^2  \implies \\
    &\implies r^2 (R_2 - R_1) = R_2^2 R_2 - R_1^2 R_2 \implies \\
    &\implies r
        = \sqrt{ R_1 R_2 \frac{ R_2 - R_1 }{ R_2 - R_1 } }
        = \sqrt{ R_1 R_2 }
        = \sqrt{ 1 \units{Ом} \cdot 9 \units{Ом} }
        = 3{,}00 \units{Ом}.
        \\
     \eta_1
        &= \frac{ R_1 }{ R_1 + r }
        = \frac{ \sqrt{ R_1 } }{ \sqrt{ R_1 } + \sqrt{ R_2 } }
        = 0{,}25, \\
     \eta_2
        &= \frac{ R_2 }{ R_2 + r }
        = \frac{ \sqrt{ R_2 } }{ \sqrt{ R_2 } + \sqrt{ R_1 } }
        = 0{,}75
    \end{align*}
}
\vspace{150pt}

\tasknumber{4}\task{
    Определите ток, протекающий через резистор $R = 10\,\text{Ом}$ и разность потенциалов на нём (см.
    рис.
    на доске),
    если $r_1 = 3\,\text{Ом}$, $r_2 = 1\,\text{Ом}$, $\mathcal{E}_1 = 60\,\text{В}$, $\mathcal{E}_2 = 30\,\text{В}$
}
\newpage

\addpersonalvariant{Александр Селехметьев}
\tasknumber{1}\task{
    Через резистор сопротивлением $R=30\units{Ом}$ протекает электрический ток $\mathcal{I}=4\units{А}$.
    Определите, чему равны напряжение на резисторе и мощность, выделяющаяся на нём.
}
\answer{
    \begin{align*}
    U &= \mathcal{I}R = 4 \units{А} \cdot 30 \units{Ом} = 120 \units{В}, \\
    P &= \mathcal{I}^2R = \sqr{ 4 \units{А} } \cdot 30 \units{Ом} = 480 \units{Вт}
    \end{align*}
}
\vspace{150pt}

\tasknumber{2}\task{
    Замкнутая электрическая цепь состоит из ЭДС $\mathcal{E}=20\units{В}$ и сопротивлением $r$
    и резистора $R=15\units{Ом}$.
    Определите ток, протекающий в цепи.
    Какая тепловая энергия выделится на резисторе за время
    $\tau=3\units{с}$? Какая работа будет совершена ЭДС за это время? Каков знак этой работы? Чему равен КПД цепи? Вычислите значения для 2 случаев:
    $r=0$ и $r=3\units{Ом}$.
}
\answer{
    \begin{align*}
    \mathcal{I}_1
                    &= \frac{ \mathcal{E} }{ R }
                    = \frac{ 20 \units{В} }{ 15 \units{Ом} }
                    = 1{,}33 \units{А}, \\
    \mathcal{I}_2
                    &= \frac{ \mathcal{E} }{ R + r }
                    = \frac{ 20 \units{В} }{ 15 \units{Ом} + 3 \units{Ом} }
                    = 1{,}11 \units{А}, \\
    Q_1
                    &= \mathcal{I}_1^2R\tau
                    = \cbr{ \frac{ \mathcal{E} }{ R } }^2 R \tau
                    = \cbr{ \frac{ 20 \units{В} }{ 15 \units{Ом} } }^2 \cdot 15 \units{Ом} \cdot 3 \units{с}
                    = 80{,}00 \units{Дж}, \\
    Q_2
                    &= \mathcal{I}_2^2R\tau
                    = \cbr{\frac{ \mathcal{E} }{ R + r } }^2 R \tau
                    = \cbr{\frac{ 20 \units{В} }{ 15 \units{Ом} + 3 \units{Ом} } }^2 \cdot 15 \units{Ом} \cdot 3 \units{с}
                    = 55{,}56 \units{Дж}, \\
    A_1
                    &= \mathcal{I}_1\tau\mathcal{E}
                    = \frac{ \mathcal{E} }{ R } \tau \mathcal{E}
                    = \frac{ \mathcal{E}^2 \tau }{ R }
                    = \frac{ \sqr{ 20 \units{В} } \cdot 3 \units{с} }{ 15 \units{Ом} }
                    = 80{,}00 \units{Дж}, \text{положительна}, \\
    A_2
                    &= \mathcal{I}_2\tau\mathcal{E}
                    = \frac{ \mathcal{E} }{ R + r } \tau \mathcal{E}
                    = \frac{ \mathcal{E}^2 \tau }{ R + r }
                    = \frac{ \sqr{ 20 \units{В} } \cdot 3 \units{с} }{ 15 \units{Ом} + 3 \units{Ом} }
                    = 66{,}67 \units{Дж}, \text{положительна}, \\
    \eta_1
                    &= \frac{ Q_1 }{ A_1 }
                    = \ldots
                    = \frac{ R }{ R }
                    = 1, \\
    \eta_2
                    &= \frac{ Q_2 }{ A_2 }
                    = \ldots
                    = \frac{ R }{ R + r }
                    = 0{,}833333333333
    \end{align*}
}
\vspace{150pt}

\tasknumber{3}\task{
    Лампочки, сопротивления которых $R_1=0.25\units{Ом}$ и $R_2=4\units{Ом}$, поочерёдно подключённные к некоторому источнику тока,
    потребляют одинаковую мощность.
    Найти внутреннее сопротивление источника и КПД цепи в каждом случае.
}
\answer{
    \begin{align*}
    P_1 &= \sqr{ \frac{ \mathcal{E} }{ R_1 + r } } R_1,
    P_2  = \sqr{ \frac{ \mathcal{E} }{ R_2 + r } } R_2,
    P_1 = P_2 \implies \\
    &\implies R_1 \sqr{ R_2 + r } = R_2 \sqr{ R_1 + r } \implies \\
    &\implies R_1 R_2^2 + 2 R_1 R_2 r + R_1 r^2 =
                R_2 R_1^2 + 2 R_2 R_1 r + R_2 r^2  \implies \\
    &\implies r^2 (R_2 - R_1) = R_2^2 R_2 - R_1^2 R_2 \implies \\
    &\implies r
        = \sqrt{ R_1 R_2 \frac{ R_2 - R_1 }{ R_2 - R_1 } }
        = \sqrt{ R_1 R_2 }
        = \sqrt{ 0{,}25 \units{Ом} \cdot 4 \units{Ом} }
        = 1{,}00 \units{Ом}.
        \\
     \eta_1
        &= \frac{ R_1 }{ R_1 + r }
        = \frac{ \sqrt{ R_1 } }{ \sqrt{ R_1 } + \sqrt{ R_2 } }
        = 0{,}2, \\
     \eta_2
        &= \frac{ R_2 }{ R_2 + r }
        = \frac{ \sqrt{ R_2 } }{ \sqrt{ R_2 } + \sqrt{ R_1 } }
        = 0{,}8
    \end{align*}
}
\vspace{150pt}

\tasknumber{4}\task{
    Определите ток, протекающий через резистор $R = 18\,\text{Ом}$ и разность потенциалов на нём (см.
    рис.
    на доске),
    если $r_1 = 2\,\text{Ом}$, $r_2 = 3\,\text{Ом}$, $\mathcal{E}_1 = 40\,\text{В}$, $\mathcal{E}_2 = 60\,\text{В}$
}
\newpage

\addpersonalvariant{Алексей Тихонов}
\tasknumber{1}\task{
    На резистор сопротивлением $r=18\units{Ом}$ подали напряжение $V=240\units{В}$.
    Определите ток, который потечёт через резистор, и мощность, выделяющуюся на нём.
}
\answer{
    \begin{align*}
    \mathcal{I} &= \frac{ V }{ r } = \frac{ 240 \units{В} }{ 18 \units{Ом} } = 13{,}33 \units{А}, \\
    P &= \frac{ V^2 }{ r } = \frac{ \sqr{ 240 \units{В} } }{ 18 \units{Ом} } = 3200{,}00 \units{Вт}
    \end{align*}
}
\vspace{150pt}

\tasknumber{2}\task{
    Замкнутая электрическая цепь состоит из ЭДС $\mathcal{E}=30\units{В}$ и сопротивлением $r$
    и резистора $R=30\units{Ом}$.
    Определите ток, протекающий в цепи.
    Какая тепловая энергия выделится на резисторе за время
    $\tau=3\units{с}$? Какая работа будет совершена ЭДС за это время? Каков знак этой работы? Чему равен КПД цепи? Вычислите значения для 2 случаев:
    $r=0$ и $r=3\units{Ом}$.
}
\answer{
    \begin{align*}
    \mathcal{I}_1
                    &= \frac{ \mathcal{E} }{ R }
                    = \frac{ 30 \units{В} }{ 30 \units{Ом} }
                    = 1{,}00 \units{А}, \\
    \mathcal{I}_2
                    &= \frac{ \mathcal{E} }{ R + r }
                    = \frac{ 30 \units{В} }{ 30 \units{Ом} + 3 \units{Ом} }
                    = 0{,}91 \units{А}, \\
    Q_1
                    &= \mathcal{I}_1^2R\tau
                    = \cbr{ \frac{ \mathcal{E} }{ R } }^2 R \tau
                    = \cbr{ \frac{ 30 \units{В} }{ 30 \units{Ом} } }^2 \cdot 30 \units{Ом} \cdot 3 \units{с}
                    = 90{,}00 \units{Дж}, \\
    Q_2
                    &= \mathcal{I}_2^2R\tau
                    = \cbr{\frac{ \mathcal{E} }{ R + r } }^2 R \tau
                    = \cbr{\frac{ 30 \units{В} }{ 30 \units{Ом} + 3 \units{Ом} } }^2 \cdot 30 \units{Ом} \cdot 3 \units{с}
                    = 74{,}38 \units{Дж}, \\
    A_1
                    &= \mathcal{I}_1\tau\mathcal{E}
                    = \frac{ \mathcal{E} }{ R } \tau \mathcal{E}
                    = \frac{ \mathcal{E}^2 \tau }{ R }
                    = \frac{ \sqr{ 30 \units{В} } \cdot 3 \units{с} }{ 30 \units{Ом} }
                    = 90{,}00 \units{Дж}, \text{положительна}, \\
    A_2
                    &= \mathcal{I}_2\tau\mathcal{E}
                    = \frac{ \mathcal{E} }{ R + r } \tau \mathcal{E}
                    = \frac{ \mathcal{E}^2 \tau }{ R + r }
                    = \frac{ \sqr{ 30 \units{В} } \cdot 3 \units{с} }{ 30 \units{Ом} + 3 \units{Ом} }
                    = 81{,}82 \units{Дж}, \text{положительна}, \\
    \eta_1
                    &= \frac{ Q_1 }{ A_1 }
                    = \ldots
                    = \frac{ R }{ R }
                    = 1, \\
    \eta_2
                    &= \frac{ Q_2 }{ A_2 }
                    = \ldots
                    = \frac{ R }{ R + r }
                    = 0{,}909090909091
    \end{align*}
}
\vspace{150pt}

\tasknumber{3}\task{
    Лампочки, сопротивления которых $R_1=4\units{Ом}$ и $R_2=100\units{Ом}$, поочерёдно подключённные к некоторому источнику тока,
    потребляют одинаковую мощность.
    Найти внутреннее сопротивление источника и КПД цепи в каждом случае.
}
\answer{
    \begin{align*}
    P_1 &= \sqr{ \frac{ \mathcal{E} }{ R_1 + r } } R_1,
    P_2  = \sqr{ \frac{ \mathcal{E} }{ R_2 + r } } R_2,
    P_1 = P_2 \implies \\
    &\implies R_1 \sqr{ R_2 + r } = R_2 \sqr{ R_1 + r } \implies \\
    &\implies R_1 R_2^2 + 2 R_1 R_2 r + R_1 r^2 =
                R_2 R_1^2 + 2 R_2 R_1 r + R_2 r^2  \implies \\
    &\implies r^2 (R_2 - R_1) = R_2^2 R_2 - R_1^2 R_2 \implies \\
    &\implies r
        = \sqrt{ R_1 R_2 \frac{ R_2 - R_1 }{ R_2 - R_1 } }
        = \sqrt{ R_1 R_2 }
        = \sqrt{ 4 \units{Ом} \cdot 100 \units{Ом} }
        = 20{,}00 \units{Ом}.
        \\
     \eta_1
        &= \frac{ R_1 }{ R_1 + r }
        = \frac{ \sqrt{ R_1 } }{ \sqrt{ R_1 } + \sqrt{ R_2 } }
        = 0{,}166666666667, \\
     \eta_2
        &= \frac{ R_2 }{ R_2 + r }
        = \frac{ \sqrt{ R_2 } }{ \sqrt{ R_2 } + \sqrt{ R_1 } }
        = 0{,}833333333333
    \end{align*}
}
\vspace{150pt}

\tasknumber{4}\task{
    Определите ток, протекающий через резистор $R = 15\,\text{Ом}$ и разность потенциалов на нём (см.
    рис.
    на доске),
    если $r_1 = 3\,\text{Ом}$, $r_2 = 2\,\text{Ом}$, $\mathcal{E}_1 = 40\,\text{В}$, $\mathcal{E}_2 = 30\,\text{В}$
}
\newpage

\addpersonalvariant{Алина Филиппова}
\tasknumber{1}\task{
    На резистор сопротивлением $R=18\units{Ом}$ подали напряжение $U=180\units{В}$.
    Определите ток, который потечёт через резистор, и мощность, выделяющуюся на нём.
}
\answer{
    \begin{align*}
    \mathcal{I} &= \frac{ U }{ R } = \frac{ 180 \units{В} }{ 18 \units{Ом} } = 10{,}00 \units{А}, \\
    P &= \frac{ U^2 }{ R } = \frac{ \sqr{ 180 \units{В} } }{ 18 \units{Ом} } = 1800{,}00 \units{Вт}
    \end{align*}
}
\vspace{150pt}

\tasknumber{2}\task{
    Замкнутая электрическая цепь состоит из ЭДС $\mathcal{E}=60\units{В}$ и сопротивлением $r$
    и резистора $R=10\units{Ом}$.
    Определите ток, протекающий в цепи.
    Какая тепловая энергия выделится на резисторе за время
    $\tau=1\units{с}$? Какая работа будет совершена ЭДС за это время? Каков знак этой работы? Чему равен КПД цепи? Вычислите значения для 2 случаев:
    $r=0$ и $r=1\units{Ом}$.
}
\answer{
    \begin{align*}
    \mathcal{I}_1
                    &= \frac{ \mathcal{E} }{ R }
                    = \frac{ 60 \units{В} }{ 10 \units{Ом} }
                    = 6{,}00 \units{А}, \\
    \mathcal{I}_2
                    &= \frac{ \mathcal{E} }{ R + r }
                    = \frac{ 60 \units{В} }{ 10 \units{Ом} + 1 \units{Ом} }
                    = 5{,}45 \units{А}, \\
    Q_1
                    &= \mathcal{I}_1^2R\tau
                    = \cbr{ \frac{ \mathcal{E} }{ R } }^2 R \tau
                    = \cbr{ \frac{ 60 \units{В} }{ 10 \units{Ом} } }^2 \cdot 10 \units{Ом} \cdot 1 \units{с}
                    = 360{,}00 \units{Дж}, \\
    Q_2
                    &= \mathcal{I}_2^2R\tau
                    = \cbr{\frac{ \mathcal{E} }{ R + r } }^2 R \tau
                    = \cbr{\frac{ 60 \units{В} }{ 10 \units{Ом} + 1 \units{Ом} } }^2 \cdot 10 \units{Ом} \cdot 1 \units{с}
                    = 297{,}52 \units{Дж}, \\
    A_1
                    &= \mathcal{I}_1\tau\mathcal{E}
                    = \frac{ \mathcal{E} }{ R } \tau \mathcal{E}
                    = \frac{ \mathcal{E}^2 \tau }{ R }
                    = \frac{ \sqr{ 60 \units{В} } \cdot 1 \units{с} }{ 10 \units{Ом} }
                    = 360{,}00 \units{Дж}, \text{положительна}, \\
    A_2
                    &= \mathcal{I}_2\tau\mathcal{E}
                    = \frac{ \mathcal{E} }{ R + r } \tau \mathcal{E}
                    = \frac{ \mathcal{E}^2 \tau }{ R + r }
                    = \frac{ \sqr{ 60 \units{В} } \cdot 1 \units{с} }{ 10 \units{Ом} + 1 \units{Ом} }
                    = 327{,}27 \units{Дж}, \text{положительна}, \\
    \eta_1
                    &= \frac{ Q_1 }{ A_1 }
                    = \ldots
                    = \frac{ R }{ R }
                    = 1, \\
    \eta_2
                    &= \frac{ Q_2 }{ A_2 }
                    = \ldots
                    = \frac{ R }{ R + r }
                    = 0{,}909090909091
    \end{align*}
}
\vspace{150pt}

\tasknumber{3}\task{
    Лампочки, сопротивления которых $R_1=3\units{Ом}$ и $R_2=48\units{Ом}$, поочерёдно подключённные к некоторому источнику тока,
    потребляют одинаковую мощность.
    Найти внутреннее сопротивление источника и КПД цепи в каждом случае.
}
\answer{
    \begin{align*}
    P_1 &= \sqr{ \frac{ \mathcal{E} }{ R_1 + r } } R_1,
    P_2  = \sqr{ \frac{ \mathcal{E} }{ R_2 + r } } R_2,
    P_1 = P_2 \implies \\
    &\implies R_1 \sqr{ R_2 + r } = R_2 \sqr{ R_1 + r } \implies \\
    &\implies R_1 R_2^2 + 2 R_1 R_2 r + R_1 r^2 =
                R_2 R_1^2 + 2 R_2 R_1 r + R_2 r^2  \implies \\
    &\implies r^2 (R_2 - R_1) = R_2^2 R_2 - R_1^2 R_2 \implies \\
    &\implies r
        = \sqrt{ R_1 R_2 \frac{ R_2 - R_1 }{ R_2 - R_1 } }
        = \sqrt{ R_1 R_2 }
        = \sqrt{ 3 \units{Ом} \cdot 48 \units{Ом} }
        = 12{,}00 \units{Ом}.
        \\
     \eta_1
        &= \frac{ R_1 }{ R_1 + r }
        = \frac{ \sqrt{ R_1 } }{ \sqrt{ R_1 } + \sqrt{ R_2 } }
        = 0{,}2, \\
     \eta_2
        &= \frac{ R_2 }{ R_2 + r }
        = \frac{ \sqrt{ R_2 } }{ \sqrt{ R_2 } + \sqrt{ R_1 } }
        = 0{,}8
    \end{align*}
}
\vspace{150pt}

\tasknumber{4}\task{
    Определите ток, протекающий через резистор $R = 10\,\text{Ом}$ и разность потенциалов на нём (см.
    рис.
    на доске),
    если $r_1 = 1\,\text{Ом}$, $r_2 = 2\,\text{Ом}$, $\mathcal{E}_1 = 30\,\text{В}$, $\mathcal{E}_2 = 40\,\text{В}$
}
\newpage

\addpersonalvariant{Алина Яшина}
\tasknumber{1}\task{
    Через резистор сопротивлением $R=5\units{Ом}$ протекает электрический ток $\mathcal{I}=5\units{А}$.
    Определите, чему равны напряжение на резисторе и мощность, выделяющаяся на нём.
}
\answer{
    \begin{align*}
    U &= \mathcal{I}R = 5 \units{А} \cdot 5 \units{Ом} = 25 \units{В}, \\
    P &= \mathcal{I}^2R = \sqr{ 5 \units{А} } \cdot 5 \units{Ом} = 125 \units{Вт}
    \end{align*}
}
\vspace{150pt}

\tasknumber{2}\task{
    Замкнутая электрическая цепь состоит из ЭДС $\mathcal{E}=60\units{В}$ и сопротивлением $r$
    и резистора $R=15\units{Ом}$.
    Определите ток, протекающий в цепи.
    Какая тепловая энергия выделится на резисторе за время
    $\tau=4\units{с}$? Какая работа будет совершена ЭДС за это время? Каков знак этой работы? Чему равен КПД цепи? Вычислите значения для 2 случаев:
    $r=0$ и $r=4\units{Ом}$.
}
\answer{
    \begin{align*}
    \mathcal{I}_1
                    &= \frac{ \mathcal{E} }{ R }
                    = \frac{ 60 \units{В} }{ 15 \units{Ом} }
                    = 4{,}00 \units{А}, \\
    \mathcal{I}_2
                    &= \frac{ \mathcal{E} }{ R + r }
                    = \frac{ 60 \units{В} }{ 15 \units{Ом} + 4 \units{Ом} }
                    = 3{,}16 \units{А}, \\
    Q_1
                    &= \mathcal{I}_1^2R\tau
                    = \cbr{ \frac{ \mathcal{E} }{ R } }^2 R \tau
                    = \cbr{ \frac{ 60 \units{В} }{ 15 \units{Ом} } }^2 \cdot 15 \units{Ом} \cdot 4 \units{с}
                    = 960{,}00 \units{Дж}, \\
    Q_2
                    &= \mathcal{I}_2^2R\tau
                    = \cbr{\frac{ \mathcal{E} }{ R + r } }^2 R \tau
                    = \cbr{\frac{ 60 \units{В} }{ 15 \units{Ом} + 4 \units{Ом} } }^2 \cdot 15 \units{Ом} \cdot 4 \units{с}
                    = 598{,}34 \units{Дж}, \\
    A_1
                    &= \mathcal{I}_1\tau\mathcal{E}
                    = \frac{ \mathcal{E} }{ R } \tau \mathcal{E}
                    = \frac{ \mathcal{E}^2 \tau }{ R }
                    = \frac{ \sqr{ 60 \units{В} } \cdot 4 \units{с} }{ 15 \units{Ом} }
                    = 960{,}00 \units{Дж}, \text{положительна}, \\
    A_2
                    &= \mathcal{I}_2\tau\mathcal{E}
                    = \frac{ \mathcal{E} }{ R + r } \tau \mathcal{E}
                    = \frac{ \mathcal{E}^2 \tau }{ R + r }
                    = \frac{ \sqr{ 60 \units{В} } \cdot 4 \units{с} }{ 15 \units{Ом} + 4 \units{Ом} }
                    = 757{,}89 \units{Дж}, \text{положительна}, \\
    \eta_1
                    &= \frac{ Q_1 }{ A_1 }
                    = \ldots
                    = \frac{ R }{ R }
                    = 1, \\
    \eta_2
                    &= \frac{ Q_2 }{ A_2 }
                    = \ldots
                    = \frac{ R }{ R + r }
                    = 0{,}789473684211
    \end{align*}
}
\vspace{150pt}

\tasknumber{3}\task{
    Лампочки, сопротивления которых $R_1=0.25\units{Ом}$ и $R_2=64\units{Ом}$, поочерёдно подключённные к некоторому источнику тока,
    потребляют одинаковую мощность.
    Найти внутреннее сопротивление источника и КПД цепи в каждом случае.
}
\answer{
    \begin{align*}
    P_1 &= \sqr{ \frac{ \mathcal{E} }{ R_1 + r } } R_1,
    P_2  = \sqr{ \frac{ \mathcal{E} }{ R_2 + r } } R_2,
    P_1 = P_2 \implies \\
    &\implies R_1 \sqr{ R_2 + r } = R_2 \sqr{ R_1 + r } \implies \\
    &\implies R_1 R_2^2 + 2 R_1 R_2 r + R_1 r^2 =
                R_2 R_1^2 + 2 R_2 R_1 r + R_2 r^2  \implies \\
    &\implies r^2 (R_2 - R_1) = R_2^2 R_2 - R_1^2 R_2 \implies \\
    &\implies r
        = \sqrt{ R_1 R_2 \frac{ R_2 - R_1 }{ R_2 - R_1 } }
        = \sqrt{ R_1 R_2 }
        = \sqrt{ 0{,}25 \units{Ом} \cdot 64 \units{Ом} }
        = 4{,}00 \units{Ом}.
        \\
     \eta_1
        &= \frac{ R_1 }{ R_1 + r }
        = \frac{ \sqrt{ R_1 } }{ \sqrt{ R_1 } + \sqrt{ R_2 } }
        = 0{,}0588235294118, \\
     \eta_2
        &= \frac{ R_2 }{ R_2 + r }
        = \frac{ \sqrt{ R_2 } }{ \sqrt{ R_2 } + \sqrt{ R_1 } }
        = 0{,}941176470588
    \end{align*}
}
\vspace{150pt}

\tasknumber{4}\task{
    Определите ток, протекающий через резистор $R = 15\,\text{Ом}$ и разность потенциалов на нём (см.
    рис.
    на доске),
    если $r_1 = 2\,\text{Ом}$, $r_2 = 1\,\text{Ом}$, $\mathcal{E}_1 = 20\,\text{В}$, $\mathcal{E}_2 = 40\,\text{В}$
}

\end{document}
% autogenerated
