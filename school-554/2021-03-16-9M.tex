\setdate{16~марта~2021}
\setclass{9«М»}

\addpersonalvariant{Михаил Бурмистров}

\tasknumber{1}%
\task{%
    Определите ёмкость конденсатора, если при его зарядке до напряжения
    $U = 5\,\text{кВ}$ он приобретает заряд $q = 18\,\text{нКл}$.
    Чему при этом равны заряды обкладок конденсатора (сделайте рисунок и укажите их)?
}
\answer{%
    $
        q = CU \implies
        C = \frac{ q }{ U } = \frac{ 18\,\text{нКл} }{ 5\,\text{кВ} } = 3{,}60\,\text{пФ}.
        \text{Заряды обкладок: $q$ и $-q$}
    $
}
\solutionspace{120pt}

\tasknumber{2}%
\task{%
    На конденсаторе указано: $C = 120\,\text{пФ}$, $V = 400\,\text{В}$.
    Удастся ли его использовать для накопления заряда $Q = 50\,\text{нКл}$?
}
\answer{%
    $
        Q_{ \text{ max } } = CV = 120\,\text{пФ} \cdot 400\,\text{В} = 48\,\text{нКл}
        \implies Q_{ \text{ max } }  <  Q \implies \text{не удастся}
    $
}
\solutionspace{80pt}

\tasknumber{3}%
\task{%
    Как и во сколько раз изменится ёмкость плоского конденсатора при уменьшении площади пластин в 3 раз
    и уменьшении расстояния между ними в 4 раз?
}
\answer{%
    $
        \frac{C'}{C}
            = \frac{\eps_0\eps \frac S3}{\frac d4} \Big/ \frac{\eps_0\eps S}{d}
            = \frac{4}{3} > 1 \implies \text{увеличится в $\frac43$ раз}
    $
}
\solutionspace{80pt}

\tasknumber{4}%
\task{%
    Электрическая ёмкость конденсатора равна $C = 750\,\text{пФ}$,
    при этом ему сообщён заряд $q = 300\,\text{нКл}$.
    Какова энергия заряженного конденсатора?
}
\answer{%
    $
        W
        = \frac{q^2}{2C}
        = \frac{\sqr{ 300\,\text{нКл} }}{2 \cdot 750\,\text{пФ}}
        = 60{,}00\,\text{мкДж}
    $
}
\solutionspace{80pt}

\tasknumber{5}%
\task{%
    Напротив физических величин укажите их обозначения и единицы измерения в СИ:
    \begin{enumerate}
        \item ёмкость конденсатора,
        \item индуктивность.
    \end{enumerate}
}
\solutionspace{40pt}

\tasknumber{6}%
\task{%
    Запишите формулы, выражающие:
    \begin{enumerate}
        \item заряд кондесатора через его ёмкость и поданное напряжение,
        \item энергию кондесатора через его ёмкость и заряд,
        \item частоту колебаний в электромагнитном контуре, состоящем из конденсатора и катушки индуктивности,
    \end{enumerate}
}

\variantsplitter

\addpersonalvariant{Артём Глембо}

\tasknumber{1}%
\task{%
    Определите ёмкость конденсатора, если при его зарядке до напряжения
    $V = 3\,\text{кВ}$ он приобретает заряд $q = 4\,\text{нКл}$.
    Чему при этом равны заряды обкладок конденсатора (сделайте рисунок и укажите их)?
}
\answer{%
    $
        q = CV \implies
        C = \frac{ q }{ V } = \frac{ 4\,\text{нКл} }{ 3\,\text{кВ} } = 1{,}33\,\text{пФ}.
        \text{Заряды обкладок: $q$ и $-q$}
    $
}
\solutionspace{120pt}

\tasknumber{2}%
\task{%
    На конденсаторе указано: $C = 80\,\text{пФ}$, $U = 450\,\text{В}$.
    Удастся ли его использовать для накопления заряда $q = 60\,\text{нКл}$?
}
\answer{%
    $
        q_{ \text{ max } } = CU = 80\,\text{пФ} \cdot 450\,\text{В} = 36\,\text{нКл}
        \implies q_{ \text{ max } }  <  q \implies \text{не удастся}
    $
}
\solutionspace{80pt}

\tasknumber{3}%
\task{%
    Как и во сколько раз изменится ёмкость плоского конденсатора при уменьшении площади пластин в 2 раз
    и уменьшении расстояния между ними в 2 раз?
}
\answer{%
    $
        \frac{C'}{C}
            = \frac{\eps_0\eps \frac S2}{\frac d2} \Big/ \frac{\eps_0\eps S}{d}
            = \frac{2}{2} = 1 \implies \text{не изменится}
    $
}
\solutionspace{80pt}

\tasknumber{4}%
\task{%
    Электрическая ёмкость конденсатора равна $C = 750\,\text{пФ}$,
    при этом ему сообщён заряд $Q = 500\,\text{нКл}$.
    Какова энергия заряженного конденсатора?
}
\answer{%
    $
        W
        = \frac{Q^2}{2C}
        = \frac{\sqr{ 500\,\text{нКл} }}{2 \cdot 750\,\text{пФ}}
        = 166{,}67\,\text{мкДж}
    $
}
\solutionspace{80pt}

\tasknumber{5}%
\task{%
    Напротив физических величин укажите их обозначения и единицы измерения в СИ:
    \begin{enumerate}
        \item ёмкость конденсатора,
        \item индуктивность.
    \end{enumerate}
}
\solutionspace{40pt}

\tasknumber{6}%
\task{%
    Запишите формулы, выражающие:
    \begin{enumerate}
        \item заряд кондесатора через его ёмкость и поданное напряжение,
        \item энергию кондесатора через его ёмкость и заряд,
        \item период колебаний в электромагнитном контуре, состоящем из конденсатора и катушки индуктивности,
    \end{enumerate}
}

\variantsplitter

\addpersonalvariant{Наталья Гончарова}

\tasknumber{1}%
\task{%
    Определите ёмкость конденсатора, если при его зарядке до напряжения
    $V = 6\,\text{кВ}$ он приобретает заряд $q = 25\,\text{нКл}$.
    Чему при этом равны заряды обкладок конденсатора (сделайте рисунок и укажите их)?
}
\answer{%
    $
        q = CV \implies
        C = \frac{ q }{ V } = \frac{ 25\,\text{нКл} }{ 6\,\text{кВ} } = 4{,}17\,\text{пФ}.
        \text{Заряды обкладок: $q$ и $-q$}
    $
}
\solutionspace{120pt}

\tasknumber{2}%
\task{%
    На конденсаторе указано: $C = 50\,\text{пФ}$, $V = 200\,\text{В}$.
    Удастся ли его использовать для накопления заряда $q = 60\,\text{нКл}$?
}
\answer{%
    $
        q_{ \text{ max } } = CV = 50\,\text{пФ} \cdot 200\,\text{В} = 10\,\text{нКл}
        \implies q_{ \text{ max } }  <  q \implies \text{не удастся}
    $
}
\solutionspace{80pt}

\tasknumber{3}%
\task{%
    Как и во сколько раз изменится ёмкость плоского конденсатора при уменьшении площади пластин в 2 раз
    и уменьшении расстояния между ними в 7 раз?
}
\answer{%
    $
        \frac{C'}{C}
            = \frac{\eps_0\eps \frac S2}{\frac d7} \Big/ \frac{\eps_0\eps S}{d}
            = \frac{7}{2} > 1 \implies \text{увеличится в $\frac72$ раз}
    $
}
\solutionspace{80pt}

\tasknumber{4}%
\task{%
    Электрическая ёмкость конденсатора равна $C = 200\,\text{пФ}$,
    при этом ему сообщён заряд $q = 800\,\text{нКл}$.
    Какова энергия заряженного конденсатора?
}
\answer{%
    $
        W
        = \frac{q^2}{2C}
        = \frac{\sqr{ 800\,\text{нКл} }}{2 \cdot 200\,\text{пФ}}
        = 1600{,}00\,\text{мкДж}
    $
}
\solutionspace{80pt}

\tasknumber{5}%
\task{%
    Напротив физических величин укажите их обозначения и единицы измерения в СИ:
    \begin{enumerate}
        \item ёмкость конденсатора,
        \item индуктивность.
    \end{enumerate}
}
\solutionspace{40pt}

\tasknumber{6}%
\task{%
    Запишите формулы, выражающие:
    \begin{enumerate}
        \item заряд кондесатора через его ёмкость и поданное напряжение,
        \item энергию кондесатора через его ёмкость и заряд,
        \item период колебаний в электромагнитном контуре, состоящем из конденсатора и катушки индуктивности,
    \end{enumerate}
}

\variantsplitter

\addpersonalvariant{Файёзбек Касымов}

\tasknumber{1}%
\task{%
    Определите ёмкость конденсатора, если при его зарядке до напряжения
    $U = 3\,\text{кВ}$ он приобретает заряд $Q = 4\,\text{нКл}$.
    Чему при этом равны заряды обкладок конденсатора (сделайте рисунок и укажите их)?
}
\answer{%
    $
        Q = CU \implies
        C = \frac{ Q }{ U } = \frac{ 4\,\text{нКл} }{ 3\,\text{кВ} } = 1{,}33\,\text{пФ}.
        \text{Заряды обкладок: $Q$ и $-Q$}
    $
}
\solutionspace{120pt}

\tasknumber{2}%
\task{%
    На конденсаторе указано: $C = 80\,\text{пФ}$, $U = 450\,\text{В}$.
    Удастся ли его использовать для накопления заряда $q = 30\,\text{нКл}$?
}
\answer{%
    $
        q_{ \text{ max } } = CU = 80\,\text{пФ} \cdot 450\,\text{В} = 36\,\text{нКл}
        \implies q_{ \text{ max } } \ge q \implies \text{удастся}
    $
}
\solutionspace{80pt}

\tasknumber{3}%
\task{%
    Как и во сколько раз изменится ёмкость плоского конденсатора при уменьшении площади пластин в 7 раз
    и уменьшении расстояния между ними в 2 раз?
}
\answer{%
    $
        \frac{C'}{C}
            = \frac{\eps_0\eps \frac S7}{\frac d2} \Big/ \frac{\eps_0\eps S}{d}
            = \frac{2}{7} < 1 \implies \text{уменьшится в $\frac72$ раз}
    $
}
\solutionspace{80pt}

\tasknumber{4}%
\task{%
    Электрическая ёмкость конденсатора равна $C = 200\,\text{пФ}$,
    при этом ему сообщён заряд $Q = 300\,\text{нКл}$.
    Какова энергия заряженного конденсатора?
}
\answer{%
    $
        W
        = \frac{Q^2}{2C}
        = \frac{\sqr{ 300\,\text{нКл} }}{2 \cdot 200\,\text{пФ}}
        = 225{,}00\,\text{мкДж}
    $
}
\solutionspace{80pt}

\tasknumber{5}%
\task{%
    Напротив физических величин укажите их обозначения и единицы измерения в СИ:
    \begin{enumerate}
        \item ёмкость конденсатора,
        \item индуктивность.
    \end{enumerate}
}
\solutionspace{40pt}

\tasknumber{6}%
\task{%
    Запишите формулы, выражающие:
    \begin{enumerate}
        \item заряд кондесатора через его ёмкость и поданное напряжение,
        \item энергию кондесатора через его ёмкость и поданное напряжение,
        \item частоту колебаний в электромагнитном контуре, состоящем из конденсатора и катушки индуктивности,
    \end{enumerate}
}

\variantsplitter

\addpersonalvariant{Александр Козинец}

\tasknumber{1}%
\task{%
    Определите ёмкость конденсатора, если при его зарядке до напряжения
    $V = 15\,\text{кВ}$ он приобретает заряд $Q = 24\,\text{нКл}$.
    Чему при этом равны заряды обкладок конденсатора (сделайте рисунок и укажите их)?
}
\answer{%
    $
        Q = CV \implies
        C = \frac{ Q }{ V } = \frac{ 24\,\text{нКл} }{ 15\,\text{кВ} } = 1{,}60\,\text{пФ}.
        \text{Заряды обкладок: $Q$ и $-Q$}
    $
}
\solutionspace{120pt}

\tasknumber{2}%
\task{%
    На конденсаторе указано: $C = 50\,\text{пФ}$, $V = 400\,\text{В}$.
    Удастся ли его использовать для накопления заряда $q = 50\,\text{нКл}$?
}
\answer{%
    $
        q_{ \text{ max } } = CV = 50\,\text{пФ} \cdot 400\,\text{В} = 20\,\text{нКл}
        \implies q_{ \text{ max } }  <  q \implies \text{не удастся}
    $
}
\solutionspace{80pt}

\tasknumber{3}%
\task{%
    Как и во сколько раз изменится ёмкость плоского конденсатора при уменьшении площади пластин в 4 раз
    и уменьшении расстояния между ними в 6 раз?
}
\answer{%
    $
        \frac{C'}{C}
            = \frac{\eps_0\eps \frac S4}{\frac d6} \Big/ \frac{\eps_0\eps S}{d}
            = \frac{6}{4} > 1 \implies \text{увеличится в $\frac32$ раз}
    $
}
\solutionspace{80pt}

\tasknumber{4}%
\task{%
    Электрическая ёмкость конденсатора равна $C = 400\,\text{пФ}$,
    при этом ему сообщён заряд $q = 500\,\text{нКл}$.
    Какова энергия заряженного конденсатора?
}
\answer{%
    $
        W
        = \frac{q^2}{2C}
        = \frac{\sqr{ 500\,\text{нКл} }}{2 \cdot 400\,\text{пФ}}
        = 312{,}50\,\text{мкДж}
    $
}
\solutionspace{80pt}

\tasknumber{5}%
\task{%
    Напротив физических величин укажите их обозначения и единицы измерения в СИ:
    \begin{enumerate}
        \item ёмкость конденсатора,
        \item индуктивность.
    \end{enumerate}
}
\solutionspace{40pt}

\tasknumber{6}%
\task{%
    Запишите формулы, выражающие:
    \begin{enumerate}
        \item заряд кондесатора через его ёмкость и поданное напряжение,
        \item энергию кондесатора через его ёмкость и поданное напряжение,
        \item период колебаний в электромагнитном контуре, состоящем из конденсатора и катушки индуктивности,
    \end{enumerate}
}

\variantsplitter

\addpersonalvariant{Андрей Куликовский}

\tasknumber{1}%
\task{%
    Определите ёмкость конденсатора, если при его зарядке до напряжения
    $V = 3\,\text{кВ}$ он приобретает заряд $q = 25\,\text{нКл}$.
    Чему при этом равны заряды обкладок конденсатора (сделайте рисунок и укажите их)?
}
\answer{%
    $
        q = CV \implies
        C = \frac{ q }{ V } = \frac{ 25\,\text{нКл} }{ 3\,\text{кВ} } = 8{,}33\,\text{пФ}.
        \text{Заряды обкладок: $q$ и $-q$}
    $
}
\solutionspace{120pt}

\tasknumber{2}%
\task{%
    На конденсаторе указано: $C = 120\,\text{пФ}$, $V = 400\,\text{В}$.
    Удастся ли его использовать для накопления заряда $Q = 50\,\text{нКл}$?
}
\answer{%
    $
        Q_{ \text{ max } } = CV = 120\,\text{пФ} \cdot 400\,\text{В} = 48\,\text{нКл}
        \implies Q_{ \text{ max } }  <  Q \implies \text{не удастся}
    $
}
\solutionspace{80pt}

\tasknumber{3}%
\task{%
    Как и во сколько раз изменится ёмкость плоского конденсатора при уменьшении площади пластин в 4 раз
    и уменьшении расстояния между ними в 2 раз?
}
\answer{%
    $
        \frac{C'}{C}
            = \frac{\eps_0\eps \frac S4}{\frac d2} \Big/ \frac{\eps_0\eps S}{d}
            = \frac{2}{4} < 1 \implies \text{уменьшится в $\frac21$ раз}
    $
}
\solutionspace{80pt}

\tasknumber{4}%
\task{%
    Электрическая ёмкость конденсатора равна $C = 400\,\text{пФ}$,
    при этом ему сообщён заряд $Q = 800\,\text{нКл}$.
    Какова энергия заряженного конденсатора?
}
\answer{%
    $
        W
        = \frac{Q^2}{2C}
        = \frac{\sqr{ 800\,\text{нКл} }}{2 \cdot 400\,\text{пФ}}
        = 800{,}00\,\text{мкДж}
    $
}
\solutionspace{80pt}

\tasknumber{5}%
\task{%
    Напротив физических величин укажите их обозначения и единицы измерения в СИ:
    \begin{enumerate}
        \item ёмкость конденсатора,
        \item индуктивность.
    \end{enumerate}
}
\solutionspace{40pt}

\tasknumber{6}%
\task{%
    Запишите формулы, выражающие:
    \begin{enumerate}
        \item заряд кондесатора через его ёмкость и поданное напряжение,
        \item энергию кондесатора через его ёмкость и поданное напряжение,
        \item частоту колебаний в электромагнитном контуре, состоящем из конденсатора и катушки индуктивности,
    \end{enumerate}
}

\variantsplitter

\addpersonalvariant{Полина Лоткова}

\tasknumber{1}%
\task{%
    Определите ёмкость конденсатора, если при его зарядке до напряжения
    $V = 20\,\text{кВ}$ он приобретает заряд $q = 6\,\text{нКл}$.
    Чему при этом равны заряды обкладок конденсатора (сделайте рисунок и укажите их)?
}
\answer{%
    $
        q = CV \implies
        C = \frac{ q }{ V } = \frac{ 6\,\text{нКл} }{ 20\,\text{кВ} } = 0{,}30\,\text{пФ}.
        \text{Заряды обкладок: $q$ и $-q$}
    $
}
\solutionspace{120pt}

\tasknumber{2}%
\task{%
    На конденсаторе указано: $C = 80\,\text{пФ}$, $U = 450\,\text{В}$.
    Удастся ли его использовать для накопления заряда $q = 50\,\text{нКл}$?
}
\answer{%
    $
        q_{ \text{ max } } = CU = 80\,\text{пФ} \cdot 450\,\text{В} = 36\,\text{нКл}
        \implies q_{ \text{ max } }  <  q \implies \text{не удастся}
    $
}
\solutionspace{80pt}

\tasknumber{3}%
\task{%
    Как и во сколько раз изменится ёмкость плоского конденсатора при уменьшении площади пластин в 3 раз
    и уменьшении расстояния между ними в 2 раз?
}
\answer{%
    $
        \frac{C'}{C}
            = \frac{\eps_0\eps \frac S3}{\frac d2} \Big/ \frac{\eps_0\eps S}{d}
            = \frac{2}{3} < 1 \implies \text{уменьшится в $\frac32$ раз}
    $
}
\solutionspace{80pt}

\tasknumber{4}%
\task{%
    Электрическая ёмкость конденсатора равна $C = 200\,\text{пФ}$,
    при этом ему сообщён заряд $q = 300\,\text{нКл}$.
    Какова энергия заряженного конденсатора?
}
\answer{%
    $
        W
        = \frac{q^2}{2C}
        = \frac{\sqr{ 300\,\text{нКл} }}{2 \cdot 200\,\text{пФ}}
        = 225{,}00\,\text{мкДж}
    $
}
\solutionspace{80pt}

\tasknumber{5}%
\task{%
    Напротив физических величин укажите их обозначения и единицы измерения в СИ:
    \begin{enumerate}
        \item ёмкость конденсатора,
        \item индуктивность.
    \end{enumerate}
}
\solutionspace{40pt}

\tasknumber{6}%
\task{%
    Запишите формулы, выражающие:
    \begin{enumerate}
        \item заряд кондесатора через его ёмкость и поданное напряжение,
        \item энергию кондесатора через его заряд и поданное напряжение,
        \item частоту колебаний в электромагнитном контуре, состоящем из конденсатора и катушки индуктивности,
    \end{enumerate}
}

\variantsplitter

\addpersonalvariant{Екатерина Медведева}

\tasknumber{1}%
\task{%
    Определите ёмкость конденсатора, если при его зарядке до напряжения
    $V = 5\,\text{кВ}$ он приобретает заряд $q = 15\,\text{нКл}$.
    Чему при этом равны заряды обкладок конденсатора (сделайте рисунок и укажите их)?
}
\answer{%
    $
        q = CV \implies
        C = \frac{ q }{ V } = \frac{ 15\,\text{нКл} }{ 5\,\text{кВ} } = 3{,}00\,\text{пФ}.
        \text{Заряды обкладок: $q$ и $-q$}
    $
}
\solutionspace{120pt}

\tasknumber{2}%
\task{%
    На конденсаторе указано: $C = 50\,\text{пФ}$, $U = 400\,\text{В}$.
    Удастся ли его использовать для накопления заряда $Q = 30\,\text{нКл}$?
}
\answer{%
    $
        Q_{ \text{ max } } = CU = 50\,\text{пФ} \cdot 400\,\text{В} = 20\,\text{нКл}
        \implies Q_{ \text{ max } }  <  Q \implies \text{не удастся}
    $
}
\solutionspace{80pt}

\tasknumber{3}%
\task{%
    Как и во сколько раз изменится ёмкость плоского конденсатора при уменьшении площади пластин в 4 раз
    и уменьшении расстояния между ними в 3 раз?
}
\answer{%
    $
        \frac{C'}{C}
            = \frac{\eps_0\eps \frac S4}{\frac d3} \Big/ \frac{\eps_0\eps S}{d}
            = \frac{3}{4} < 1 \implies \text{уменьшится в $\frac43$ раз}
    $
}
\solutionspace{80pt}

\tasknumber{4}%
\task{%
    Электрическая ёмкость конденсатора равна $C = 400\,\text{пФ}$,
    при этом ему сообщён заряд $Q = 800\,\text{нКл}$.
    Какова энергия заряженного конденсатора?
}
\answer{%
    $
        W
        = \frac{Q^2}{2C}
        = \frac{\sqr{ 800\,\text{нКл} }}{2 \cdot 400\,\text{пФ}}
        = 800{,}00\,\text{мкДж}
    $
}
\solutionspace{80pt}

\tasknumber{5}%
\task{%
    Напротив физических величин укажите их обозначения и единицы измерения в СИ:
    \begin{enumerate}
        \item ёмкость конденсатора,
        \item индуктивность.
    \end{enumerate}
}
\solutionspace{40pt}

\tasknumber{6}%
\task{%
    Запишите формулы, выражающие:
    \begin{enumerate}
        \item заряд кондесатора через его ёмкость и поданное напряжение,
        \item энергию кондесатора через его ёмкость и поданное напряжение,
        \item частоту колебаний в электромагнитном контуре, состоящем из конденсатора и катушки индуктивности,
    \end{enumerate}
}

\variantsplitter

\addpersonalvariant{Константин Мельник}

\tasknumber{1}%
\task{%
    Определите ёмкость конденсатора, если при его зарядке до напряжения
    $V = 15\,\text{кВ}$ он приобретает заряд $q = 4\,\text{нКл}$.
    Чему при этом равны заряды обкладок конденсатора (сделайте рисунок и укажите их)?
}
\answer{%
    $
        q = CV \implies
        C = \frac{ q }{ V } = \frac{ 4\,\text{нКл} }{ 15\,\text{кВ} } = 0{,}27\,\text{пФ}.
        \text{Заряды обкладок: $q$ и $-q$}
    $
}
\solutionspace{120pt}

\tasknumber{2}%
\task{%
    На конденсаторе указано: $C = 80\,\text{пФ}$, $V = 300\,\text{В}$.
    Удастся ли его использовать для накопления заряда $Q = 50\,\text{нКл}$?
}
\answer{%
    $
        Q_{ \text{ max } } = CV = 80\,\text{пФ} \cdot 300\,\text{В} = 24\,\text{нКл}
        \implies Q_{ \text{ max } }  <  Q \implies \text{не удастся}
    $
}
\solutionspace{80pt}

\tasknumber{3}%
\task{%
    Как и во сколько раз изменится ёмкость плоского конденсатора при уменьшении площади пластин в 2 раз
    и уменьшении расстояния между ними в 6 раз?
}
\answer{%
    $
        \frac{C'}{C}
            = \frac{\eps_0\eps \frac S2}{\frac d6} \Big/ \frac{\eps_0\eps S}{d}
            = \frac{6}{2} > 1 \implies \text{увеличится в $\frac31$ раз}
    $
}
\solutionspace{80pt}

\tasknumber{4}%
\task{%
    Электрическая ёмкость конденсатора равна $C = 750\,\text{пФ}$,
    при этом ему сообщён заряд $q = 900\,\text{нКл}$.
    Какова энергия заряженного конденсатора?
}
\answer{%
    $
        W
        = \frac{q^2}{2C}
        = \frac{\sqr{ 900\,\text{нКл} }}{2 \cdot 750\,\text{пФ}}
        = 540{,}00\,\text{мкДж}
    $
}
\solutionspace{80pt}

\tasknumber{5}%
\task{%
    Напротив физических величин укажите их обозначения и единицы измерения в СИ:
    \begin{enumerate}
        \item ёмкость конденсатора,
        \item индуктивность.
    \end{enumerate}
}
\solutionspace{40pt}

\tasknumber{6}%
\task{%
    Запишите формулы, выражающие:
    \begin{enumerate}
        \item заряд кондесатора через его ёмкость и поданное напряжение,
        \item энергию кондесатора через его ёмкость и заряд,
        \item период колебаний в электромагнитном контуре, состоящем из конденсатора и катушки индуктивности,
    \end{enumerate}
}

\variantsplitter

\addpersonalvariant{Степан Небоваренков}

\tasknumber{1}%
\task{%
    Определите ёмкость конденсатора, если при его зарядке до напряжения
    $V = 20\,\text{кВ}$ он приобретает заряд $Q = 15\,\text{нКл}$.
    Чему при этом равны заряды обкладок конденсатора (сделайте рисунок и укажите их)?
}
\answer{%
    $
        Q = CV \implies
        C = \frac{ Q }{ V } = \frac{ 15\,\text{нКл} }{ 20\,\text{кВ} } = 0{,}75\,\text{пФ}.
        \text{Заряды обкладок: $Q$ и $-Q$}
    $
}
\solutionspace{120pt}

\tasknumber{2}%
\task{%
    На конденсаторе указано: $C = 50\,\text{пФ}$, $V = 450\,\text{В}$.
    Удастся ли его использовать для накопления заряда $Q = 60\,\text{нКл}$?
}
\answer{%
    $
        Q_{ \text{ max } } = CV = 50\,\text{пФ} \cdot 450\,\text{В} = 22\,\text{нКл}
        \implies Q_{ \text{ max } }  <  Q \implies \text{не удастся}
    $
}
\solutionspace{80pt}

\tasknumber{3}%
\task{%
    Как и во сколько раз изменится ёмкость плоского конденсатора при уменьшении площади пластин в 6 раз
    и уменьшении расстояния между ними в 2 раз?
}
\answer{%
    $
        \frac{C'}{C}
            = \frac{\eps_0\eps \frac S6}{\frac d2} \Big/ \frac{\eps_0\eps S}{d}
            = \frac{2}{6} < 1 \implies \text{уменьшится в $\frac31$ раз}
    $
}
\solutionspace{80pt}

\tasknumber{4}%
\task{%
    Электрическая ёмкость конденсатора равна $C = 600\,\text{пФ}$,
    при этом ему сообщён заряд $q = 900\,\text{нКл}$.
    Какова энергия заряженного конденсатора?
}
\answer{%
    $
        W
        = \frac{q^2}{2C}
        = \frac{\sqr{ 900\,\text{нКл} }}{2 \cdot 600\,\text{пФ}}
        = 675{,}00\,\text{мкДж}
    $
}
\solutionspace{80pt}

\tasknumber{5}%
\task{%
    Напротив физических величин укажите их обозначения и единицы измерения в СИ:
    \begin{enumerate}
        \item ёмкость конденсатора,
        \item индуктивность.
    \end{enumerate}
}
\solutionspace{40pt}

\tasknumber{6}%
\task{%
    Запишите формулы, выражающие:
    \begin{enumerate}
        \item заряд кондесатора через его ёмкость и поданное напряжение,
        \item энергию кондесатора через его ёмкость и заряд,
        \item период колебаний в электромагнитном контуре, состоящем из конденсатора и катушки индуктивности,
    \end{enumerate}
}

\variantsplitter

\addpersonalvariant{Матвей Неретин}

\tasknumber{1}%
\task{%
    Определите ёмкость конденсатора, если при его зарядке до напряжения
    $V = 6\,\text{кВ}$ он приобретает заряд $Q = 6\,\text{нКл}$.
    Чему при этом равны заряды обкладок конденсатора (сделайте рисунок и укажите их)?
}
\answer{%
    $
        Q = CV \implies
        C = \frac{ Q }{ V } = \frac{ 6\,\text{нКл} }{ 6\,\text{кВ} } = 1{,}00\,\text{пФ}.
        \text{Заряды обкладок: $Q$ и $-Q$}
    $
}
\solutionspace{120pt}

\tasknumber{2}%
\task{%
    На конденсаторе указано: $C = 100\,\text{пФ}$, $V = 300\,\text{В}$.
    Удастся ли его использовать для накопления заряда $q = 50\,\text{нКл}$?
}
\answer{%
    $
        q_{ \text{ max } } = CV = 100\,\text{пФ} \cdot 300\,\text{В} = 30\,\text{нКл}
        \implies q_{ \text{ max } }  <  q \implies \text{не удастся}
    $
}
\solutionspace{80pt}

\tasknumber{3}%
\task{%
    Как и во сколько раз изменится ёмкость плоского конденсатора при уменьшении площади пластин в 3 раз
    и уменьшении расстояния между ними в 3 раз?
}
\answer{%
    $
        \frac{C'}{C}
            = \frac{\eps_0\eps \frac S3}{\frac d3} \Big/ \frac{\eps_0\eps S}{d}
            = \frac{3}{3} = 1 \implies \text{не изменится}
    $
}
\solutionspace{80pt}

\tasknumber{4}%
\task{%
    Электрическая ёмкость конденсатора равна $C = 750\,\text{пФ}$,
    при этом ему сообщён заряд $q = 900\,\text{нКл}$.
    Какова энергия заряженного конденсатора?
}
\answer{%
    $
        W
        = \frac{q^2}{2C}
        = \frac{\sqr{ 900\,\text{нКл} }}{2 \cdot 750\,\text{пФ}}
        = 540{,}00\,\text{мкДж}
    $
}
\solutionspace{80pt}

\tasknumber{5}%
\task{%
    Напротив физических величин укажите их обозначения и единицы измерения в СИ:
    \begin{enumerate}
        \item ёмкость конденсатора,
        \item индуктивность.
    \end{enumerate}
}
\solutionspace{40pt}

\tasknumber{6}%
\task{%
    Запишите формулы, выражающие:
    \begin{enumerate}
        \item заряд кондесатора через его ёмкость и поданное напряжение,
        \item энергию кондесатора через его ёмкость и поданное напряжение,
        \item частоту колебаний в электромагнитном контуре, состоящем из конденсатора и катушки индуктивности,
    \end{enumerate}
}

\variantsplitter

\addpersonalvariant{Мария Никонова}

\tasknumber{1}%
\task{%
    Определите ёмкость конденсатора, если при его зарядке до напряжения
    $V = 3\,\text{кВ}$ он приобретает заряд $Q = 4\,\text{нКл}$.
    Чему при этом равны заряды обкладок конденсатора (сделайте рисунок и укажите их)?
}
\answer{%
    $
        Q = CV \implies
        C = \frac{ Q }{ V } = \frac{ 4\,\text{нКл} }{ 3\,\text{кВ} } = 1{,}33\,\text{пФ}.
        \text{Заряды обкладок: $Q$ и $-Q$}
    $
}
\solutionspace{120pt}

\tasknumber{2}%
\task{%
    На конденсаторе указано: $C = 100\,\text{пФ}$, $V = 400\,\text{В}$.
    Удастся ли его использовать для накопления заряда $Q = 60\,\text{нКл}$?
}
\answer{%
    $
        Q_{ \text{ max } } = CV = 100\,\text{пФ} \cdot 400\,\text{В} = 40\,\text{нКл}
        \implies Q_{ \text{ max } }  <  Q \implies \text{не удастся}
    $
}
\solutionspace{80pt}

\tasknumber{3}%
\task{%
    Как и во сколько раз изменится ёмкость плоского конденсатора при уменьшении площади пластин в 7 раз
    и уменьшении расстояния между ними в 2 раз?
}
\answer{%
    $
        \frac{C'}{C}
            = \frac{\eps_0\eps \frac S7}{\frac d2} \Big/ \frac{\eps_0\eps S}{d}
            = \frac{2}{7} < 1 \implies \text{уменьшится в $\frac72$ раз}
    $
}
\solutionspace{80pt}

\tasknumber{4}%
\task{%
    Электрическая ёмкость конденсатора равна $C = 750\,\text{пФ}$,
    при этом ему сообщён заряд $Q = 900\,\text{нКл}$.
    Какова энергия заряженного конденсатора?
}
\answer{%
    $
        W
        = \frac{Q^2}{2C}
        = \frac{\sqr{ 900\,\text{нКл} }}{2 \cdot 750\,\text{пФ}}
        = 540{,}00\,\text{мкДж}
    $
}
\solutionspace{80pt}

\tasknumber{5}%
\task{%
    Напротив физических величин укажите их обозначения и единицы измерения в СИ:
    \begin{enumerate}
        \item ёмкость конденсатора,
        \item индуктивность.
    \end{enumerate}
}
\solutionspace{40pt}

\tasknumber{6}%
\task{%
    Запишите формулы, выражающие:
    \begin{enumerate}
        \item заряд кондесатора через его ёмкость и поданное напряжение,
        \item энергию кондесатора через его ёмкость и поданное напряжение,
        \item период колебаний в электромагнитном контуре, состоящем из конденсатора и катушки индуктивности,
    \end{enumerate}
}

\variantsplitter

\addpersonalvariant{Даниил Палаткин}

\tasknumber{1}%
\task{%
    Определите ёмкость конденсатора, если при его зарядке до напряжения
    $U = 2\,\text{кВ}$ он приобретает заряд $Q = 6\,\text{нКл}$.
    Чему при этом равны заряды обкладок конденсатора (сделайте рисунок и укажите их)?
}
\answer{%
    $
        Q = CU \implies
        C = \frac{ Q }{ U } = \frac{ 6\,\text{нКл} }{ 2\,\text{кВ} } = 3{,}00\,\text{пФ}.
        \text{Заряды обкладок: $Q$ и $-Q$}
    $
}
\solutionspace{120pt}

\tasknumber{2}%
\task{%
    На конденсаторе указано: $C = 150\,\text{пФ}$, $U = 400\,\text{В}$.
    Удастся ли его использовать для накопления заряда $q = 60\,\text{нКл}$?
}
\answer{%
    $
        q_{ \text{ max } } = CU = 150\,\text{пФ} \cdot 400\,\text{В} = 60\,\text{нКл}
        \implies q_{ \text{ max } } \ge q \implies \text{удастся}
    $
}
\solutionspace{80pt}

\tasknumber{3}%
\task{%
    Как и во сколько раз изменится ёмкость плоского конденсатора при уменьшении площади пластин в 8 раз
    и уменьшении расстояния между ними в 5 раз?
}
\answer{%
    $
        \frac{C'}{C}
            = \frac{\eps_0\eps \frac S8}{\frac d5} \Big/ \frac{\eps_0\eps S}{d}
            = \frac{5}{8} < 1 \implies \text{уменьшится в $\frac85$ раз}
    $
}
\solutionspace{80pt}

\tasknumber{4}%
\task{%
    Электрическая ёмкость конденсатора равна $C = 600\,\text{пФ}$,
    при этом ему сообщён заряд $Q = 900\,\text{нКл}$.
    Какова энергия заряженного конденсатора?
}
\answer{%
    $
        W
        = \frac{Q^2}{2C}
        = \frac{\sqr{ 900\,\text{нКл} }}{2 \cdot 600\,\text{пФ}}
        = 675{,}00\,\text{мкДж}
    $
}
\solutionspace{80pt}

\tasknumber{5}%
\task{%
    Напротив физических величин укажите их обозначения и единицы измерения в СИ:
    \begin{enumerate}
        \item ёмкость конденсатора,
        \item индуктивность.
    \end{enumerate}
}
\solutionspace{40pt}

\tasknumber{6}%
\task{%
    Запишите формулы, выражающие:
    \begin{enumerate}
        \item заряд кондесатора через его ёмкость и поданное напряжение,
        \item энергию кондесатора через его заряд и поданное напряжение,
        \item период колебаний в электромагнитном контуре, состоящем из конденсатора и катушки индуктивности,
    \end{enumerate}
}

\variantsplitter

\addpersonalvariant{Станислав Пикун}

\tasknumber{1}%
\task{%
    Определите ёмкость конденсатора, если при его зарядке до напряжения
    $V = 3\,\text{кВ}$ он приобретает заряд $Q = 25\,\text{нКл}$.
    Чему при этом равны заряды обкладок конденсатора (сделайте рисунок и укажите их)?
}
\answer{%
    $
        Q = CV \implies
        C = \frac{ Q }{ V } = \frac{ 25\,\text{нКл} }{ 3\,\text{кВ} } = 8{,}33\,\text{пФ}.
        \text{Заряды обкладок: $Q$ и $-Q$}
    $
}
\solutionspace{120pt}

\tasknumber{2}%
\task{%
    На конденсаторе указано: $C = 80\,\text{пФ}$, $U = 300\,\text{В}$.
    Удастся ли его использовать для накопления заряда $Q = 30\,\text{нКл}$?
}
\answer{%
    $
        Q_{ \text{ max } } = CU = 80\,\text{пФ} \cdot 300\,\text{В} = 24\,\text{нКл}
        \implies Q_{ \text{ max } }  <  Q \implies \text{не удастся}
    $
}
\solutionspace{80pt}

\tasknumber{3}%
\task{%
    Как и во сколько раз изменится ёмкость плоского конденсатора при уменьшении площади пластин в 8 раз
    и уменьшении расстояния между ними в 4 раз?
}
\answer{%
    $
        \frac{C'}{C}
            = \frac{\eps_0\eps \frac S8}{\frac d4} \Big/ \frac{\eps_0\eps S}{d}
            = \frac{4}{8} < 1 \implies \text{уменьшится в $\frac21$ раз}
    $
}
\solutionspace{80pt}

\tasknumber{4}%
\task{%
    Электрическая ёмкость конденсатора равна $C = 750\,\text{пФ}$,
    при этом ему сообщён заряд $Q = 900\,\text{нКл}$.
    Какова энергия заряженного конденсатора?
}
\answer{%
    $
        W
        = \frac{Q^2}{2C}
        = \frac{\sqr{ 900\,\text{нКл} }}{2 \cdot 750\,\text{пФ}}
        = 540{,}00\,\text{мкДж}
    $
}
\solutionspace{80pt}

\tasknumber{5}%
\task{%
    Напротив физических величин укажите их обозначения и единицы измерения в СИ:
    \begin{enumerate}
        \item ёмкость конденсатора,
        \item индуктивность.
    \end{enumerate}
}
\solutionspace{40pt}

\tasknumber{6}%
\task{%
    Запишите формулы, выражающие:
    \begin{enumerate}
        \item заряд кондесатора через его ёмкость и поданное напряжение,
        \item энергию кондесатора через его ёмкость и заряд,
        \item период колебаний в электромагнитном контуре, состоящем из конденсатора и катушки индуктивности,
    \end{enumerate}
}

\variantsplitter

\addpersonalvariant{Илья Пичугин}

\tasknumber{1}%
\task{%
    Определите ёмкость конденсатора, если при его зарядке до напряжения
    $V = 20\,\text{кВ}$ он приобретает заряд $q = 15\,\text{нКл}$.
    Чему при этом равны заряды обкладок конденсатора (сделайте рисунок и укажите их)?
}
\answer{%
    $
        q = CV \implies
        C = \frac{ q }{ V } = \frac{ 15\,\text{нКл} }{ 20\,\text{кВ} } = 0{,}75\,\text{пФ}.
        \text{Заряды обкладок: $q$ и $-q$}
    $
}
\solutionspace{120pt}

\tasknumber{2}%
\task{%
    На конденсаторе указано: $C = 50\,\text{пФ}$, $V = 300\,\text{В}$.
    Удастся ли его использовать для накопления заряда $Q = 50\,\text{нКл}$?
}
\answer{%
    $
        Q_{ \text{ max } } = CV = 50\,\text{пФ} \cdot 300\,\text{В} = 15\,\text{нКл}
        \implies Q_{ \text{ max } }  <  Q \implies \text{не удастся}
    $
}
\solutionspace{80pt}

\tasknumber{3}%
\task{%
    Как и во сколько раз изменится ёмкость плоского конденсатора при уменьшении площади пластин в 5 раз
    и уменьшении расстояния между ними в 3 раз?
}
\answer{%
    $
        \frac{C'}{C}
            = \frac{\eps_0\eps \frac S5}{\frac d3} \Big/ \frac{\eps_0\eps S}{d}
            = \frac{3}{5} < 1 \implies \text{уменьшится в $\frac53$ раз}
    $
}
\solutionspace{80pt}

\tasknumber{4}%
\task{%
    Электрическая ёмкость конденсатора равна $C = 600\,\text{пФ}$,
    при этом ему сообщён заряд $q = 300\,\text{нКл}$.
    Какова энергия заряженного конденсатора?
}
\answer{%
    $
        W
        = \frac{q^2}{2C}
        = \frac{\sqr{ 300\,\text{нКл} }}{2 \cdot 600\,\text{пФ}}
        = 75{,}00\,\text{мкДж}
    $
}
\solutionspace{80pt}

\tasknumber{5}%
\task{%
    Напротив физических величин укажите их обозначения и единицы измерения в СИ:
    \begin{enumerate}
        \item ёмкость конденсатора,
        \item индуктивность.
    \end{enumerate}
}
\solutionspace{40pt}

\tasknumber{6}%
\task{%
    Запишите формулы, выражающие:
    \begin{enumerate}
        \item заряд кондесатора через его ёмкость и поданное напряжение,
        \item энергию кондесатора через его ёмкость и заряд,
        \item период колебаний в электромагнитном контуре, состоящем из конденсатора и катушки индуктивности,
    \end{enumerate}
}

\variantsplitter

\addpersonalvariant{Кирилл Севрюгин}

\tasknumber{1}%
\task{%
    Определите ёмкость конденсатора, если при его зарядке до напряжения
    $U = 12\,\text{кВ}$ он приобретает заряд $Q = 4\,\text{нКл}$.
    Чему при этом равны заряды обкладок конденсатора (сделайте рисунок и укажите их)?
}
\answer{%
    $
        Q = CU \implies
        C = \frac{ Q }{ U } = \frac{ 4\,\text{нКл} }{ 12\,\text{кВ} } = 0{,}33\,\text{пФ}.
        \text{Заряды обкладок: $Q$ и $-Q$}
    $
}
\solutionspace{120pt}

\tasknumber{2}%
\task{%
    На конденсаторе указано: $C = 150\,\text{пФ}$, $V = 450\,\text{В}$.
    Удастся ли его использовать для накопления заряда $Q = 50\,\text{нКл}$?
}
\answer{%
    $
        Q_{ \text{ max } } = CV = 150\,\text{пФ} \cdot 450\,\text{В} = 67\,\text{нКл}
        \implies Q_{ \text{ max } } \ge Q \implies \text{удастся}
    $
}
\solutionspace{80pt}

\tasknumber{3}%
\task{%
    Как и во сколько раз изменится ёмкость плоского конденсатора при уменьшении площади пластин в 4 раз
    и уменьшении расстояния между ними в 3 раз?
}
\answer{%
    $
        \frac{C'}{C}
            = \frac{\eps_0\eps \frac S4}{\frac d3} \Big/ \frac{\eps_0\eps S}{d}
            = \frac{3}{4} < 1 \implies \text{уменьшится в $\frac43$ раз}
    $
}
\solutionspace{80pt}

\tasknumber{4}%
\task{%
    Электрическая ёмкость конденсатора равна $C = 750\,\text{пФ}$,
    при этом ему сообщён заряд $Q = 900\,\text{нКл}$.
    Какова энергия заряженного конденсатора?
}
\answer{%
    $
        W
        = \frac{Q^2}{2C}
        = \frac{\sqr{ 900\,\text{нКл} }}{2 \cdot 750\,\text{пФ}}
        = 540{,}00\,\text{мкДж}
    $
}
\solutionspace{80pt}

\tasknumber{5}%
\task{%
    Напротив физических величин укажите их обозначения и единицы измерения в СИ:
    \begin{enumerate}
        \item ёмкость конденсатора,
        \item индуктивность.
    \end{enumerate}
}
\solutionspace{40pt}

\tasknumber{6}%
\task{%
    Запишите формулы, выражающие:
    \begin{enumerate}
        \item заряд кондесатора через его ёмкость и поданное напряжение,
        \item энергию кондесатора через его заряд и поданное напряжение,
        \item частоту колебаний в электромагнитном контуре, состоящем из конденсатора и катушки индуктивности,
    \end{enumerate}
}

\variantsplitter

\addpersonalvariant{Илья Стратонников}

\tasknumber{1}%
\task{%
    Определите ёмкость конденсатора, если при его зарядке до напряжения
    $U = 12\,\text{кВ}$ он приобретает заряд $Q = 6\,\text{нКл}$.
    Чему при этом равны заряды обкладок конденсатора (сделайте рисунок и укажите их)?
}
\answer{%
    $
        Q = CU \implies
        C = \frac{ Q }{ U } = \frac{ 6\,\text{нКл} }{ 12\,\text{кВ} } = 0{,}50\,\text{пФ}.
        \text{Заряды обкладок: $Q$ и $-Q$}
    $
}
\solutionspace{120pt}

\tasknumber{2}%
\task{%
    На конденсаторе указано: $C = 100\,\text{пФ}$, $V = 300\,\text{В}$.
    Удастся ли его использовать для накопления заряда $q = 30\,\text{нКл}$?
}
\answer{%
    $
        q_{ \text{ max } } = CV = 100\,\text{пФ} \cdot 300\,\text{В} = 30\,\text{нКл}
        \implies q_{ \text{ max } } \ge q \implies \text{удастся}
    $
}
\solutionspace{80pt}

\tasknumber{3}%
\task{%
    Как и во сколько раз изменится ёмкость плоского конденсатора при уменьшении площади пластин в 2 раз
    и уменьшении расстояния между ними в 5 раз?
}
\answer{%
    $
        \frac{C'}{C}
            = \frac{\eps_0\eps \frac S2}{\frac d5} \Big/ \frac{\eps_0\eps S}{d}
            = \frac{5}{2} > 1 \implies \text{увеличится в $\frac52$ раз}
    $
}
\solutionspace{80pt}

\tasknumber{4}%
\task{%
    Электрическая ёмкость конденсатора равна $C = 400\,\text{пФ}$,
    при этом ему сообщён заряд $q = 900\,\text{нКл}$.
    Какова энергия заряженного конденсатора?
}
\answer{%
    $
        W
        = \frac{q^2}{2C}
        = \frac{\sqr{ 900\,\text{нКл} }}{2 \cdot 400\,\text{пФ}}
        = 1012{,}50\,\text{мкДж}
    $
}
\solutionspace{80pt}

\tasknumber{5}%
\task{%
    Напротив физических величин укажите их обозначения и единицы измерения в СИ:
    \begin{enumerate}
        \item ёмкость конденсатора,
        \item индуктивность.
    \end{enumerate}
}
\solutionspace{40pt}

\tasknumber{6}%
\task{%
    Запишите формулы, выражающие:
    \begin{enumerate}
        \item заряд кондесатора через его ёмкость и поданное напряжение,
        \item энергию кондесатора через его ёмкость и поданное напряжение,
        \item частоту колебаний в электромагнитном контуре, состоящем из конденсатора и катушки индуктивности,
    \end{enumerate}
}

\variantsplitter

\addpersonalvariant{Иван Шустов}

\tasknumber{1}%
\task{%
    Определите ёмкость конденсатора, если при его зарядке до напряжения
    $U = 12\,\text{кВ}$ он приобретает заряд $q = 18\,\text{нКл}$.
    Чему при этом равны заряды обкладок конденсатора (сделайте рисунок и укажите их)?
}
\answer{%
    $
        q = CU \implies
        C = \frac{ q }{ U } = \frac{ 18\,\text{нКл} }{ 12\,\text{кВ} } = 1{,}50\,\text{пФ}.
        \text{Заряды обкладок: $q$ и $-q$}
    $
}
\solutionspace{120pt}

\tasknumber{2}%
\task{%
    На конденсаторе указано: $C = 120\,\text{пФ}$, $U = 300\,\text{В}$.
    Удастся ли его использовать для накопления заряда $q = 30\,\text{нКл}$?
}
\answer{%
    $
        q_{ \text{ max } } = CU = 120\,\text{пФ} \cdot 300\,\text{В} = 36\,\text{нКл}
        \implies q_{ \text{ max } } \ge q \implies \text{удастся}
    $
}
\solutionspace{80pt}

\tasknumber{3}%
\task{%
    Как и во сколько раз изменится ёмкость плоского конденсатора при уменьшении площади пластин в 7 раз
    и уменьшении расстояния между ними в 2 раз?
}
\answer{%
    $
        \frac{C'}{C}
            = \frac{\eps_0\eps \frac S7}{\frac d2} \Big/ \frac{\eps_0\eps S}{d}
            = \frac{2}{7} < 1 \implies \text{уменьшится в $\frac72$ раз}
    $
}
\solutionspace{80pt}

\tasknumber{4}%
\task{%
    Электрическая ёмкость конденсатора равна $C = 400\,\text{пФ}$,
    при этом ему сообщён заряд $q = 800\,\text{нКл}$.
    Какова энергия заряженного конденсатора?
}
\answer{%
    $
        W
        = \frac{q^2}{2C}
        = \frac{\sqr{ 800\,\text{нКл} }}{2 \cdot 400\,\text{пФ}}
        = 800{,}00\,\text{мкДж}
    $
}
\solutionspace{80pt}

\tasknumber{5}%
\task{%
    Напротив физических величин укажите их обозначения и единицы измерения в СИ:
    \begin{enumerate}
        \item ёмкость конденсатора,
        \item индуктивность.
    \end{enumerate}
}
\solutionspace{40pt}

\tasknumber{6}%
\task{%
    Запишите формулы, выражающие:
    \begin{enumerate}
        \item заряд кондесатора через его ёмкость и поданное напряжение,
        \item энергию кондесатора через его ёмкость и поданное напряжение,
        \item период колебаний в электромагнитном контуре, состоящем из конденсатора и катушки индуктивности,
    \end{enumerate}
}
% autogenerated
