\documentclass[12pt,a4paper]{amsart}%DVI-mode.
\usepackage{graphics,graphicx,epsfig}%DVI-mode.
% \documentclass[pdftex,12pt]{amsart} %PDF-mode.
% \usepackage[pdftex]{graphicx}       %PDF-mode.
% \usepackage[babel=true]{microtype}
% \usepackage[T1]{fontenc}
% \usepackage{lmodern}

\usepackage{cmap}
%\usepackage{a4wide}                 % Fit the text to A4 page tightly.
% \usepackage[utf8]{inputenc}
\usepackage[T2A]{fontenc}
\usepackage[english,russian]{babel} % Download Russian fonts.
\usepackage{amsmath,amsfonts,amssymb,amsthm,amscd,mathrsfs} % Use AMS symbols.
\usepackage{tikz}
\usetikzlibrary{circuits.ee.IEC}
\usetikzlibrary{shapes.geometric}
\usetikzlibrary{decorations.markings}
%\usetikzlibrary{dashs}
%\usetikzlibrary{info}


\textheight=28cm % высота текста
\textwidth=18cm % ширина текста
\topmargin=-2.5cm % отступ от верхнего края
\parskip=2pt % интервал между абзацами
\oddsidemargin=-1.5cm
\evensidemargin=-1.5cm 

\parindent=0pt % абзацный отступ
\tolerance=500 % терпимость к "жидким" строкам
\binoppenalty=10000 % штраф за перенос формул - 10000 - абсолютный запрет
\relpenalty=10000
\flushbottom % выравнивание высоты страниц
\pagenumbering{gobble}

\newcommand\bivec[2]{\begin{pmatrix} #1 \\ #2 \end{pmatrix}}

\newcommand\ol[1]{\overline{#1}}

\newcommand\p[1]{\Prob\!\left(#1\right)}
\newcommand\e[1]{\mathsf{E}\!\left(#1\right)}
\newcommand\disp[1]{\mathsf{D}\!\left(#1\right)}
%\newcommand\norm[2]{\mathcal{N}\!\cbr{#1,#2}}
\newcommand\sign{\text{ sign }}

\newcommand\al[1]{\begin{align*} #1 \end{align*}}
\newcommand\begcas[1]{\begin{cases}#1\end{cases}}
\newcommand\tab[2]{	\vspace{-#1pt}
						\begin{tabbing} 
						#2
						\end{tabbing}
					\vspace{-#1pt}
					}

\newcommand\maintext[1]{{\bfseries\sffamily{#1}}}
\newcommand\skipped[1]{ {\ensuremath{\text{\small{\sffamily{Пропущено:} #1} } } } }
\newcommand\simpletitle[1]{\begin{center} \maintext{#1} \end{center}}

\def\le{\leqslant}
\def\ge{\geqslant}
\def\Ell{\mathcal{L}}
\def\eps{{\varepsilon}}
\def\Rn{\mathbb{R}^n}
\def\RSS{\mathsf{RSS}}

\newcommand\foral[1]{\forall\,#1\:}
\newcommand\exist[1]{\exists\,#1\:\colon}

\newcommand\cbr[1]{\left(#1\right)} %circled brackets
\newcommand\fbr[1]{\left\{#1\right\}} %figure brackets
\newcommand\sbr[1]{\left[#1\right]} %square brackets
\newcommand\modul[1]{\left|#1\right|}

\newcommand\sqr[1]{\cbr{#1}^2}
\newcommand\inv[1]{\cbr{#1}^{-1}}

\newcommand\cdf[2]{\cdot\frac{#1}{#2}}
\newcommand\dd[2]{\frac{\partial#1}{\partial#2}}

\newcommand\integr[2]{\int\limits_{#1}^{#2}}
\newcommand\suml[2]{\sum\limits_{#1}^{#2}}
\newcommand\isum[2]{\sum\limits_{#1=#2}^{+\infty}}
\newcommand\idots[3]{#1_{#2},\ldots,#1_{#3}}
\newcommand\fdots[5]{#4{#1_{#2}}#5\ldots#5#4{#1_{#3}}}

\newcommand\obol[1]{O\!\cbr{#1}}
\newcommand\omal[1]{o\!\cbr{#1}}

\newcommand\addeps[2]{
	\begin{figure} [!ht] %lrp
		\centering
		\includegraphics[height=320px]{#1.eps}
		\vspace{-10pt}
		\caption{#2}
		\label{eps:#1}
	\end{figure}
}

\newcommand\addepssize[3]{
	\begin{figure} [!ht] %lrp hp
		\centering
		\includegraphics[height=#3px]{#1.eps}
		\vspace{-10pt}
		\caption{#2}
		\label{eps:#1}
	\end{figure}
}


\newcommand\norm[1]{\ensuremath{\left\|{#1}\right\|}}
\newcommand\ort{\bot}
\newcommand\theorem[1]{{\sffamily Теорема #1\ }}
\newcommand\lemma[1]{{\sffamily Лемма #1\ }}
\newcommand\difflim[2]{\frac{#1\cbr{#2 + \Delta#2} - #1\cbr{#2}}{\Delta #2}}
\renewcommand\proof[1]{\par\noindent$\square$ #1 \hfill$\blacksquare$\par}
\newcommand\defenition[1]{{\sffamilyОпределение #1\ }}

% \begin{document}
% %\raggedright
% \addclassdate{7}{20 апреля 2018}

\task 1
Площадь большого поршня гидравлического домкрата $S_1 = 20\units{см}^2$, а малого $S_2 = 0{,}5\units{см}^2.$ Груз какой максимальной массы можно поднять этим домкратом, если на малый поршень давить с силой не более $F=200\units{Н}?$ Силой трения от стенки цилиндров пренебречь.

\task 2
В сосуд налита вода. Расстояние от поверхности воды до дна $H = 0{,}5\units{м},$ площадь дна $S = 0{,}1\units{м}^2.$ Найти гидростатическое давление $P_1$ и полное давление $P_2$ вблизи дна. Найти силу давления воды на дно. Плотность воды \rhowater

\task 3
На лёгкий поршень площадью $S=900\units{см}^2,$ касающийся поверхности воды, поставили гирю массы $m=3\units{кг}$. Высота слоя воды в сосуде с вертикальными стенками $H = 20\units{см}$. Определить давление жидкости вблизи дна, если плотность воды \rhowater

\task 4
Давление газов в конце сгорания в цилиндре дизельного двигателя трактора $P = 9\units{МПа}.$ Диаметр цилиндра $d = 130\units{мм}.$ С какой силой газы давят на поршень в цилиндре? Площадь круга диаметром $D$ равна $S = \cfrac{\pi D^2}4.$

\task 5
Площадь малого поршня гидравлического подъёмника $S_1 = 0{,}8\units{см}^2$, а большого $S_2 = 40\units{см}^2.$ Какую силу $F$ надо приложить к малому поршню, чтобы поднять груз весом $P = 8\units{кН}?$

\task 6
Герметичный сосуд полностью заполнен водой и стоит на столе. На небольшой поршень площадью $S$ давят рукой с силой $F$. Поршень находится ниже крышки сосуда на $H_1$, выше дна на $H_2$ и может свободно перемещаться. Плотность воды $\rho$, атмосферное давление $P_A$. Найти давления $P_1$ и $P_2$ в воде вблизи крышки и дна сосуда.
\\ \\
\addclassdate{7}{20 апреля 2018}

\task 1
Площадь большого поршня гидравлического домкрата $S_1 = 20\units{см}^2$, а малого $S_2 = 0{,}5\units{см}^2.$ Груз какой максимальной массы можно поднять этим домкратом, если на малый поршень давить с силой не более $F=200\units{Н}?$ Силой трения от стенки цилиндров пренебречь.

\task 2
В сосуд налита вода. Расстояние от поверхности воды до дна $H = 0{,}5\units{м},$ площадь дна $S = 0{,}1\units{м}^2.$ Найти гидростатическое давление $P_1$ и полное давление $P_2$ вблизи дна. Найти силу давления воды на дно. Плотность воды \rhowater

\task 3
На лёгкий поршень площадью $S=900\units{см}^2,$ касающийся поверхности воды, поставили гирю массы $m=3\units{кг}$. Высота слоя воды в сосуде с вертикальными стенками $H = 20\units{см}$. Определить давление жидкости вблизи дна, если плотность воды \rhowater

\task 4
Давление газов в конце сгорания в цилиндре дизельного двигателя трактора $P = 9\units{МПа}.$ Диаметр цилиндра $d = 130\units{мм}.$ С какой силой газы давят на поршень в цилиндре? Площадь круга диаметром $D$ равна $S = \cfrac{\pi D^2}4.$

\task 5
Площадь малого поршня гидравлического подъёмника $S_1 = 0{,}8\units{см}^2$, а большого $S_2 = 40\units{см}^2.$ Какую силу $F$ надо приложить к малому поршню, чтобы поднять груз весом $P = 8\units{кН}?$

\task 6
Герметичный сосуд полностью заполнен водой и стоит на столе. На небольшой поршень площадью $S$ давят рукой с силой $F$. Поршень находится ниже крышки сосуда на $H_1$, выше дна на $H_2$ и может свободно перемещаться. Плотность воды $\rho$, атмосферное давление $P_A$. Найти давления $P_1$ и $P_2$ в воде вблизи крышки и дна сосуда.

\newpage

\adddate{8 класс. 20 апреля 2018}

\task 1
Между точками $A$ и $B$ электрической цепи подключены последовательно резисторы $R_1 = 10\units{Ом}$ и $R_2 = 20\units{Ом}$ и параллельно им $R_3 = 30\units{Ом}.$ Найдите эквивалентное сопротивление $R_{AB}$ этого участка цепи.

\task 2
Электрическая цепь состоит из последовательности $N$ одинаковых звеньев, в которых каждый резистор имеет сопротивление $r$. Последнее звено замкнуто резистором сопротивлением $R$. При каком соотношении $\cfrac{R}{r}$ сопротивление цепи не зависит от числа звеньев?

\task 3
Для измерения сопротивления $R$ проводника собрана электрическая цепь. Вольтметр $V$ показывает напряжение $U_V = 5\units{В},$ показание амперметра $A$ равно $I_A = 25\units{мА}.$ Найдите величину $R$ сопротивления проводника. Внутреннее сопротивление вольтметра $R_V = 1{,}0\units{кОм},$ внутреннее сопротивление амперметра $R_A = 2{,}0\units{Ом}.$

\task 4
Шкала гальванометра имеет $N=100$ делений, цена деления $\delta = 1\units{мкА}$. Внутреннее сопротивление гальванометра $R_G = 1{,}0\units{кОм}.$ Как из этого прибора сделать вольтметр для измерения напряжений до $U = 100\units{В}$ или амперметр для измерения токов силой до $I = 1\units{А}?$

\\ \\ \\ \\ \\ \\ \\ \\
\adddate{8 класс. 20 апреля 2018}

\task 1
Между точками $A$ и $B$ электрической цепи подключены последовательно резисторы $R_1 = 10\units{Ом}$ и $R_2 = 20\units{Ом}$ и параллельно им $R_3 = 30\units{Ом}.$ Найдите эквивалентное сопротивление $R_{AB}$ этого участка цепи.

\task 2
Электрическая цепь состоит из последовательности $N$ одинаковых звеньев, в которых каждый резистор имеет сопротивление $r$. Последнее звено замкнуто резистором сопротивлением $R$. При каком соотношении $\cfrac{R}{r}$ сопротивление цепи не зависит от числа звеньев?

\task 3
Для измерения сопротивления $R$ проводника собрана электрическая цепь. Вольтметр $V$ показывает напряжение $U_V = 5\units{В},$ показание амперметра $A$ равно $I_A = 25\units{мА}.$ Найдите величину $R$ сопротивления проводника. Внутреннее сопротивление вольтметра $R_V = 1{,}0\units{кОм},$ внутреннее сопротивление амперметра $R_A = 2{,}0\units{Ом}.$

\task 4
Шкала гальванометра имеет $N=100$ делений, цена деления $\delta = 1\units{мкА}$. Внутреннее сопротивление гальванометра $R_G = 1{,}0\units{кОм}.$ Как из этого прибора сделать вольтметр для измерения напряжений до $U = 100\units{В}$ или амперметр для измерения токов силой до $I = 1\units{А}?$


% % \begin{flushright}
\textsc{ГБОУ школа №554, 20 ноября 2018\,г.}
\end{flushright}

\begin{center}
\LARGE \textsc{Математический бой, 8 класс}
\end{center}

\problem{1} Есть тридцать карточек, на каждой написано по одному числу: на десяти карточках~–~$a$,  на десяти других~–~$b$ и на десяти оставшихся~–~$c$ (числа  различны). Известно, что к любым пяти карточкам можно подобрать ещё пять так, что сумма чисел на этих десяти карточках будет равна нулю. Докажите, что~одно из~чисел~$a, b, c$ равно нулю.

\problem{2} Вокруг стола стола пустили пакет с орешками. Первый взял один орешек, второй — 2, третий — 3 и так далее: каждый следующий брал на 1 орешек больше. Известно, что на втором круге было взято в сумме на 100 орешков больше, чем на первом. Сколько человек сидело за столом?

% \problem{2} Натуральное число разрешено увеличить на любое целое число процентов от 1 до 100, если при этом получаем натуральное число. Найдите наименьшее натуральное число, которое нельзя при помощи таких операций получить из~числа 1.

% \problem{3} Найти сумму $1^2 - 2^2 + 3^2 - 4^2 + 5^2 + \ldots - 2018^2$.

\problem{3} В кружке рукоделия, где занимается Валя, более 93\% участников~—~девочки. Какое наименьшее число участников может быть в таком кружке?

\problem{4} Произведение 2018 целых чисел равно 1. Может ли их сумма оказаться равной~0?

% \problem{4} Можно ли все натуральные числа от~1 до~9 записать в~клетки таблицы~$3\times3$ так, чтобы сумма в~любых двух соседних (по~вертикали или горизонтали) клетках равнялось простому числу?

\problem{5} На доске написано 2018 нулей и 2019 единиц. Женя стирает 2 числа и, если они были одинаковы, дописывает к оставшимся один ноль, а~если разные — единицу. Потом Женя повторяет эту операцию снова, потом ещё и~так далее. В~результате на~доске останется только одно число. Что это за~число?

\problem{6} Докажите, что в~любой компании людей найдутся 2~человека, имеющие равное число знакомых в этой компании (если $A$~знаком с~$B$, то~и $B$~знаком с~$A$).

\problem{7} Три колокола начинают бить одновременно. Интервалы между ударами колоколов соответственно составляют $\cfrac43$~секунды, $\cfrac53$~секунды и $2$~секунды. Совпавшие по времени удары воспринимаются за~один. Сколько ударов будет услышано за 1~минуту, включая первый и последний удары?

\problem{8} Восемь одинаковых момент расположены по кругу. Известно, что три из~них~— фальшивые, и они расположены рядом друг с~другом. Вес фальшивой монеты отличается от~веса настоящей. Все фальшивые монеты весят одинаково, но неизвестно, тяжелее или легче фальшивая монета настоящей. Покажите, что за~3~взвешивания на~чашечных весах без~гирь можно определить все фальшивые монеты.

% \end{document}

\begin{document}
\noanswers

\setdate{30~апреля~2019}
\setclass{10«Т»}

\addpersonalvariant{Михаил Бурмистров}

\tasknumber{1}%
\task{%
    В однородном электрическом поле напряжённостью $E = 4\,\frac{\text{кВ}}{\text{м}}$
    переместили заряд $q = 25\,\text{нКл}$ в направлении силовой линии
    на $d = 4\,\text{см}$.
    Определите работу поля, изменение потенциальной энергии заряда,
    напряжение между начальной и конечной точками перемещения.
}
\answer{%
    $
        A   = Eqd
            = 4\,\frac{\text{кВ}}{\text{м}} \cdot 25\,\text{нКл} \cdot 4\,\text{см}
            = 400{,}00 \cdot 10^{-7} \units{Дж}
    $
}
\solutionspace{120pt}

\tasknumber{2}%
\task{%
    Напряжение между двумя точками, лежащими на одной линии напряжённости однородного электрического поля,
    равно $U = 3\,\text{кВ}$.
    Расстояние между точками $l = 30\,\text{см}$.
    Какова напряжённость этого поля?
}
\solutionspace{120pt}

\tasknumber{3}%
\task{%
    Найти напряжение между точками $A$ и $B$ в однородном электрическом поле (см.
    рис.
    на доске), если
    $AB=l = 8\,\text{см}$,
    $\alpha=45^\circ$,
    $E = 60\,\frac{\text{В}}{\text{м}}$$.
    Потенциал какой из точек $A$ и $B$ больше?
}
\solutionspace{120pt}

\tasknumber{4}%
\task{%
    При какой скорости электрона его кинетическая энергия равна $E_\text{к} = 20\,\text{эВ}?$
}
\solutionspace{120pt}

\tasknumber{5}%
\task{%
    Электрон $e^-$ вылетает из точки, потенциал которой $\varphi = 1000\,\text{В}$,
    со скоростью $v = 12000000\,\frac{\text{м}}{\text{с}}$ в направлении линий напряжённости поля.
    Будет поле ускорять или тормозить электрон?
    Каков потенциал точки, дойдя до которой электрон остановится?
}

\variantsplitter

\addpersonalvariant{Гагик Аракелян}

\tasknumber{1}%
\task{%
    В однородном электрическом поле напряжённостью $E = 20\,\frac{\text{кВ}}{\text{м}}$
    переместили заряд $Q = -40\,\text{нКл}$ в направлении силовой линии
    на $l = 10\,\text{см}$.
    Определите работу поля, изменение потенциальной энергии заряда,
    напряжение между начальной и конечной точками перемещения.
}
\answer{%
    $
        A   = EQl
            = 20\,\frac{\text{кВ}}{\text{м}} \cdot -40\,\text{нКл} \cdot 10\,\text{см}
            = -8000{,}00 \cdot 10^{-7} \units{Дж}
    $
}
\solutionspace{120pt}

\tasknumber{2}%
\task{%
    Напряжение между двумя точками, лежащими на одной линии напряжённости однородного электрического поля,
    равно $V = 4\,\text{кВ}$.
    Расстояние между точками $d = 40\,\text{см}$.
    Какова напряжённость этого поля?
}
\solutionspace{120pt}

\tasknumber{3}%
\task{%
    Найти напряжение между точками $A$ и $B$ в однородном электрическом поле (см.
    рис.
    на доске), если
    $AB=d = 4\,\text{см}$,
    $\alpha=30^\circ$,
    $E = 50\,\frac{\text{В}}{\text{м}}$$.
    Потенциал какой из точек $A$ и $B$ больше?
}
\solutionspace{120pt}

\tasknumber{4}%
\task{%
    При какой скорости электрона его кинетическая энергия равна $E_\text{к} = 40\,\text{эВ}?$
}
\solutionspace{120pt}

\tasknumber{5}%
\task{%
    Электрон $e^-$ вылетает из точки, потенциал которой $\varphi = 800\,\text{В}$,
    со скоростью $v = 3000000\,\frac{\text{м}}{\text{с}}$ в направлении линий напряжённости поля.
    Будет поле ускорять или тормозить электрон?
    Каков потенциал точки, дойдя до которой электрон остановится?
}

\variantsplitter

\addpersonalvariant{Ирен Аракелян}

\tasknumber{1}%
\task{%
    В однородном электрическом поле напряжённостью $E = 2\,\frac{\text{кВ}}{\text{м}}$
    переместили заряд $Q = 25\,\text{нКл}$ в направлении силовой линии
    на $d = 4\,\text{см}$.
    Определите работу поля, изменение потенциальной энергии заряда,
    напряжение между начальной и конечной точками перемещения.
}
\answer{%
    $
        A   = EQd
            = 2\,\frac{\text{кВ}}{\text{м}} \cdot 25\,\text{нКл} \cdot 4\,\text{см}
            = 200{,}00 \cdot 10^{-7} \units{Дж}
    $
}
\solutionspace{120pt}

\tasknumber{2}%
\task{%
    Напряжение между двумя точками, лежащими на одной линии напряжённости однородного электрического поля,
    равно $U = 3\,\text{кВ}$.
    Расстояние между точками $r = 40\,\text{см}$.
    Какова напряжённость этого поля?
}
\solutionspace{120pt}

\tasknumber{3}%
\task{%
    Найти напряжение между точками $A$ и $B$ в однородном электрическом поле (см.
    рис.
    на доске), если
    $AB=r = 8\,\text{см}$,
    $\alpha=30^\circ$,
    $E = 100\,\frac{\text{В}}{\text{м}}$$.
    Потенциал какой из точек $A$ и $B$ больше?
}
\solutionspace{120pt}

\tasknumber{4}%
\task{%
    При какой скорости электрона его кинетическая энергия равна $E_\text{к} = 1000\,\text{эВ}?$
}
\solutionspace{120pt}

\tasknumber{5}%
\task{%
    Электрон $e^-$ вылетает из точки, потенциал которой $\varphi = 800\,\text{В}$,
    со скоростью $v = 3000000\,\frac{\text{м}}{\text{с}}$ в направлении линий напряжённости поля.
    Будет поле ускорять или тормозить электрон?
    Каков потенциал точки, дойдя до которой электрон остановится?
}

\variantsplitter

\addpersonalvariant{Сабина Асадуллаева}

\tasknumber{1}%
\task{%
    В однородном электрическом поле напряжённостью $E = 2\,\frac{\text{кВ}}{\text{м}}$
    переместили заряд $Q = -10\,\text{нКл}$ в направлении силовой линии
    на $r = 2\,\text{см}$.
    Определите работу поля, изменение потенциальной энергии заряда,
    напряжение между начальной и конечной точками перемещения.
}
\answer{%
    $
        A   = EQr
            = 2\,\frac{\text{кВ}}{\text{м}} \cdot -10\,\text{нКл} \cdot 2\,\text{см}
            = -40{,}00 \cdot 10^{-7} \units{Дж}
    $
}
\solutionspace{120pt}

\tasknumber{2}%
\task{%
    Напряжение между двумя точками, лежащими на одной линии напряжённости однородного электрического поля,
    равно $U = 6\,\text{кВ}$.
    Расстояние между точками $d = 10\,\text{см}$.
    Какова напряжённость этого поля?
}
\solutionspace{120pt}

\tasknumber{3}%
\task{%
    Найти напряжение между точками $A$ и $B$ в однородном электрическом поле (см.
    рис.
    на доске), если
    $AB=r = 4\,\text{см}$,
    $\alpha=30^\circ$,
    $E = 60\,\frac{\text{В}}{\text{м}}$$.
    Потенциал какой из точек $A$ и $B$ больше?
}
\solutionspace{120pt}

\tasknumber{4}%
\task{%
    При какой скорости электрона его кинетическая энергия равна $E_\text{к} = 50\,\text{эВ}?$
}
\solutionspace{120pt}

\tasknumber{5}%
\task{%
    Электрон $e^-$ вылетает из точки, потенциал которой $\varphi = 200\,\text{В}$,
    со скоростью $v = 4000000\,\frac{\text{м}}{\text{с}}$ в направлении линий напряжённости поля.
    Будет поле ускорять или тормозить электрон?
    Каков потенциал точки, дойдя до которой электрон остановится?
}

\variantsplitter

\addpersonalvariant{Вероника Битерякова}

\tasknumber{1}%
\task{%
    В однородном электрическом поле напряжённостью $E = 20\,\frac{\text{кВ}}{\text{м}}$
    переместили заряд $q = 40\,\text{нКл}$ в направлении силовой линии
    на $r = 4\,\text{см}$.
    Определите работу поля, изменение потенциальной энергии заряда,
    напряжение между начальной и конечной точками перемещения.
}
\answer{%
    $
        A   = Eqr
            = 20\,\frac{\text{кВ}}{\text{м}} \cdot 40\,\text{нКл} \cdot 4\,\text{см}
            = 3200{,}00 \cdot 10^{-7} \units{Дж}
    $
}
\solutionspace{120pt}

\tasknumber{2}%
\task{%
    Напряжение между двумя точками, лежащими на одной линии напряжённости однородного электрического поля,
    равно $U = 2\,\text{кВ}$.
    Расстояние между точками $r = 20\,\text{см}$.
    Какова напряжённость этого поля?
}
\solutionspace{120pt}

\tasknumber{3}%
\task{%
    Найти напряжение между точками $A$ и $B$ в однородном электрическом поле (см.
    рис.
    на доске), если
    $AB=l = 6\,\text{см}$,
    $\varphi=30^\circ$,
    $E = 50\,\frac{\text{В}}{\text{м}}$$.
    Потенциал какой из точек $A$ и $B$ больше?
}
\solutionspace{120pt}

\tasknumber{4}%
\task{%
    При какой скорости электрона его кинетическая энергия равна $E_\text{к} = 20\,\text{эВ}?$
}
\solutionspace{120pt}

\tasknumber{5}%
\task{%
    Электрон $e^-$ вылетает из точки, потенциал которой $\varphi = 1000\,\text{В}$,
    со скоростью $v = 4000000\,\frac{\text{м}}{\text{с}}$ в направлении линий напряжённости поля.
    Будет поле ускорять или тормозить электрон?
    Каков потенциал точки, дойдя до которой электрон остановится?
}

\variantsplitter

\addpersonalvariant{Юлия Буянова}

\tasknumber{1}%
\task{%
    В однородном электрическом поле напряжённостью $E = 2\,\frac{\text{кВ}}{\text{м}}$
    переместили заряд $q = -40\,\text{нКл}$ в направлении силовой линии
    на $r = 5\,\text{см}$.
    Определите работу поля, изменение потенциальной энергии заряда,
    напряжение между начальной и конечной точками перемещения.
}
\answer{%
    $
        A   = Eqr
            = 2\,\frac{\text{кВ}}{\text{м}} \cdot -40\,\text{нКл} \cdot 5\,\text{см}
            = -400{,}00 \cdot 10^{-7} \units{Дж}
    $
}
\solutionspace{120pt}

\tasknumber{2}%
\task{%
    Напряжение между двумя точками, лежащими на одной линии напряжённости однородного электрического поля,
    равно $U = 2\,\text{кВ}$.
    Расстояние между точками $d = 20\,\text{см}$.
    Какова напряжённость этого поля?
}
\solutionspace{120pt}

\tasknumber{3}%
\task{%
    Найти напряжение между точками $A$ и $B$ в однородном электрическом поле (см.
    рис.
    на доске), если
    $AB=d = 4\,\text{см}$,
    $\alpha=45^\circ$,
    $E = 30\,\frac{\text{В}}{\text{м}}$$.
    Потенциал какой из точек $A$ и $B$ больше?
}
\solutionspace{120pt}

\tasknumber{4}%
\task{%
    При какой скорости электрона его кинетическая энергия равна $E_\text{к} = 30\,\text{эВ}?$
}
\solutionspace{120pt}

\tasknumber{5}%
\task{%
    Электрон $e^-$ вылетает из точки, потенциал которой $\varphi = 600\,\text{В}$,
    со скоростью $v = 10000000\,\frac{\text{м}}{\text{с}}$ в направлении линий напряжённости поля.
    Будет поле ускорять или тормозить электрон?
    Каков потенциал точки, дойдя до которой электрон остановится?
}

\variantsplitter

\addpersonalvariant{Пелагея Вдовина}

\tasknumber{1}%
\task{%
    В однородном электрическом поле напряжённостью $E = 2\,\frac{\text{кВ}}{\text{м}}$
    переместили заряд $q = -25\,\text{нКл}$ в направлении силовой линии
    на $d = 2\,\text{см}$.
    Определите работу поля, изменение потенциальной энергии заряда,
    напряжение между начальной и конечной точками перемещения.
}
\answer{%
    $
        A   = Eqd
            = 2\,\frac{\text{кВ}}{\text{м}} \cdot -25\,\text{нКл} \cdot 2\,\text{см}
            = -100{,}00 \cdot 10^{-7} \units{Дж}
    $
}
\solutionspace{120pt}

\tasknumber{2}%
\task{%
    Напряжение между двумя точками, лежащими на одной линии напряжённости однородного электрического поля,
    равно $U = 4\,\text{кВ}$.
    Расстояние между точками $d = 30\,\text{см}$.
    Какова напряжённость этого поля?
}
\solutionspace{120pt}

\tasknumber{3}%
\task{%
    Найти напряжение между точками $A$ и $B$ в однородном электрическом поле (см.
    рис.
    на доске), если
    $AB=r = 4\,\text{см}$,
    $\varphi=30^\circ$,
    $E = 100\,\frac{\text{В}}{\text{м}}$$.
    Потенциал какой из точек $A$ и $B$ больше?
}
\solutionspace{120pt}

\tasknumber{4}%
\task{%
    При какой скорости электрона его кинетическая энергия равна $E_\text{к} = 40\,\text{эВ}?$
}
\solutionspace{120pt}

\tasknumber{5}%
\task{%
    Электрон $e^-$ вылетает из точки, потенциал которой $\varphi = 600\,\text{В}$,
    со скоростью $v = 10000000\,\frac{\text{м}}{\text{с}}$ в направлении линий напряжённости поля.
    Будет поле ускорять или тормозить электрон?
    Каков потенциал точки, дойдя до которой электрон остановится?
}

\variantsplitter

\addpersonalvariant{Леонид Викторов}

\tasknumber{1}%
\task{%
    В однородном электрическом поле напряжённостью $E = 2\,\frac{\text{кВ}}{\text{м}}$
    переместили заряд $Q = 25\,\text{нКл}$ в направлении силовой линии
    на $l = 2\,\text{см}$.
    Определите работу поля, изменение потенциальной энергии заряда,
    напряжение между начальной и конечной точками перемещения.
}
\answer{%
    $
        A   = EQl
            = 2\,\frac{\text{кВ}}{\text{м}} \cdot 25\,\text{нКл} \cdot 2\,\text{см}
            = 100{,}00 \cdot 10^{-7} \units{Дж}
    $
}
\solutionspace{120pt}

\tasknumber{2}%
\task{%
    Напряжение между двумя точками, лежащими на одной линии напряжённости однородного электрического поля,
    равно $U = 4\,\text{кВ}$.
    Расстояние между точками $r = 20\,\text{см}$.
    Какова напряжённость этого поля?
}
\solutionspace{120pt}

\tasknumber{3}%
\task{%
    Найти напряжение между точками $A$ и $B$ в однородном электрическом поле (см.
    рис.
    на доске), если
    $AB=r = 6\,\text{см}$,
    $\alpha=45^\circ$,
    $E = 100\,\frac{\text{В}}{\text{м}}$$.
    Потенциал какой из точек $A$ и $B$ больше?
}
\solutionspace{120pt}

\tasknumber{4}%
\task{%
    При какой скорости электрона его кинетическая энергия равна $E_\text{к} = 30\,\text{эВ}?$
}
\solutionspace{120pt}

\tasknumber{5}%
\task{%
    Электрон $e^-$ вылетает из точки, потенциал которой $\varphi = 400\,\text{В}$,
    со скоростью $v = 6000000\,\frac{\text{м}}{\text{с}}$ в направлении линий напряжённости поля.
    Будет поле ускорять или тормозить электрон?
    Каков потенциал точки, дойдя до которой электрон остановится?
}

\variantsplitter

\addpersonalvariant{Фёдор Гнутов}

\tasknumber{1}%
\task{%
    В однородном электрическом поле напряжённостью $E = 20\,\frac{\text{кВ}}{\text{м}}$
    переместили заряд $q = 10\,\text{нКл}$ в направлении силовой линии
    на $d = 2\,\text{см}$.
    Определите работу поля, изменение потенциальной энергии заряда,
    напряжение между начальной и конечной точками перемещения.
}
\answer{%
    $
        A   = Eqd
            = 20\,\frac{\text{кВ}}{\text{м}} \cdot 10\,\text{нКл} \cdot 2\,\text{см}
            = 400{,}00 \cdot 10^{-7} \units{Дж}
    $
}
\solutionspace{120pt}

\tasknumber{2}%
\task{%
    Напряжение между двумя точками, лежащими на одной линии напряжённости однородного электрического поля,
    равно $U = 4\,\text{кВ}$.
    Расстояние между точками $l = 30\,\text{см}$.
    Какова напряжённость этого поля?
}
\solutionspace{120pt}

\tasknumber{3}%
\task{%
    Найти напряжение между точками $A$ и $B$ в однородном электрическом поле (см.
    рис.
    на доске), если
    $AB=l = 10\,\text{см}$,
    $\alpha=45^\circ$,
    $E = 50\,\frac{\text{В}}{\text{м}}$$.
    Потенциал какой из точек $A$ и $B$ больше?
}
\solutionspace{120pt}

\tasknumber{4}%
\task{%
    При какой скорости электрона его кинетическая энергия равна $E_\text{к} = 4\,\text{эВ}?$
}
\solutionspace{120pt}

\tasknumber{5}%
\task{%
    Электрон $e^-$ вылетает из точки, потенциал которой $\varphi = 800\,\text{В}$,
    со скоростью $v = 10000000\,\frac{\text{м}}{\text{с}}$ в направлении линий напряжённости поля.
    Будет поле ускорять или тормозить электрон?
    Каков потенциал точки, дойдя до которой электрон остановится?
}

\variantsplitter

\addpersonalvariant{Илья Гримберг}

\tasknumber{1}%
\task{%
    В однородном электрическом поле напряжённостью $E = 4\,\frac{\text{кВ}}{\text{м}}$
    переместили заряд $q = -10\,\text{нКл}$ в направлении силовой линии
    на $r = 5\,\text{см}$.
    Определите работу поля, изменение потенциальной энергии заряда,
    напряжение между начальной и конечной точками перемещения.
}
\answer{%
    $
        A   = Eqr
            = 4\,\frac{\text{кВ}}{\text{м}} \cdot -10\,\text{нКл} \cdot 5\,\text{см}
            = -200{,}00 \cdot 10^{-7} \units{Дж}
    $
}
\solutionspace{120pt}

\tasknumber{2}%
\task{%
    Напряжение между двумя точками, лежащими на одной линии напряжённости однородного электрического поля,
    равно $V = 2\,\text{кВ}$.
    Расстояние между точками $r = 40\,\text{см}$.
    Какова напряжённость этого поля?
}
\solutionspace{120pt}

\tasknumber{3}%
\task{%
    Найти напряжение между точками $A$ и $B$ в однородном электрическом поле (см.
    рис.
    на доске), если
    $AB=l = 8\,\text{см}$,
    $\alpha=60^\circ$,
    $E = 30\,\frac{\text{В}}{\text{м}}$$.
    Потенциал какой из точек $A$ и $B$ больше?
}
\solutionspace{120pt}

\tasknumber{4}%
\task{%
    При какой скорости электрона его кинетическая энергия равна $E_\text{к} = 20\,\text{эВ}?$
}
\solutionspace{120pt}

\tasknumber{5}%
\task{%
    Электрон $e^-$ вылетает из точки, потенциал которой $\varphi = 200\,\text{В}$,
    со скоростью $v = 4000000\,\frac{\text{м}}{\text{с}}$ в направлении линий напряжённости поля.
    Будет поле ускорять или тормозить электрон?
    Каков потенциал точки, дойдя до которой электрон остановится?
}

\variantsplitter

\addpersonalvariant{Иван Гурьянов}

\tasknumber{1}%
\task{%
    В однородном электрическом поле напряжённостью $E = 4\,\frac{\text{кВ}}{\text{м}}$
    переместили заряд $q = -25\,\text{нКл}$ в направлении силовой линии
    на $r = 10\,\text{см}$.
    Определите работу поля, изменение потенциальной энергии заряда,
    напряжение между начальной и конечной точками перемещения.
}
\answer{%
    $
        A   = Eqr
            = 4\,\frac{\text{кВ}}{\text{м}} \cdot -25\,\text{нКл} \cdot 10\,\text{см}
            = -1000{,}00 \cdot 10^{-7} \units{Дж}
    $
}
\solutionspace{120pt}

\tasknumber{2}%
\task{%
    Напряжение между двумя точками, лежащими на одной линии напряжённости однородного электрического поля,
    равно $U = 3\,\text{кВ}$.
    Расстояние между точками $d = 30\,\text{см}$.
    Какова напряжённость этого поля?
}
\solutionspace{120pt}

\tasknumber{3}%
\task{%
    Найти напряжение между точками $A$ и $B$ в однородном электрическом поле (см.
    рис.
    на доске), если
    $AB=d = 8\,\text{см}$,
    $\alpha=60^\circ$,
    $E = 60\,\frac{\text{В}}{\text{м}}$$.
    Потенциал какой из точек $A$ и $B$ больше?
}
\solutionspace{120pt}

\tasknumber{4}%
\task{%
    При какой скорости электрона его кинетическая энергия равна $E_\text{к} = 400\,\text{эВ}?$
}
\solutionspace{120pt}

\tasknumber{5}%
\task{%
    Электрон $e^-$ вылетает из точки, потенциал которой $\varphi = 800\,\text{В}$,
    со скоростью $v = 10000000\,\frac{\text{м}}{\text{с}}$ в направлении линий напряжённости поля.
    Будет поле ускорять или тормозить электрон?
    Каков потенциал точки, дойдя до которой электрон остановится?
}

\variantsplitter

\addpersonalvariant{Артём Денежкин}

\tasknumber{1}%
\task{%
    В однородном электрическом поле напряжённостью $E = 4\,\frac{\text{кВ}}{\text{м}}$
    переместили заряд $q = 40\,\text{нКл}$ в направлении силовой линии
    на $r = 4\,\text{см}$.
    Определите работу поля, изменение потенциальной энергии заряда,
    напряжение между начальной и конечной точками перемещения.
}
\answer{%
    $
        A   = Eqr
            = 4\,\frac{\text{кВ}}{\text{м}} \cdot 40\,\text{нКл} \cdot 4\,\text{см}
            = 640{,}00 \cdot 10^{-7} \units{Дж}
    $
}
\solutionspace{120pt}

\tasknumber{2}%
\task{%
    Напряжение между двумя точками, лежащими на одной линии напряжённости однородного электрического поля,
    равно $U = 2\,\text{кВ}$.
    Расстояние между точками $l = 10\,\text{см}$.
    Какова напряжённость этого поля?
}
\solutionspace{120pt}

\tasknumber{3}%
\task{%
    Найти напряжение между точками $A$ и $B$ в однородном электрическом поле (см.
    рис.
    на доске), если
    $AB=l = 6\,\text{см}$,
    $\varphi=45^\circ$,
    $E = 100\,\frac{\text{В}}{\text{м}}$$.
    Потенциал какой из точек $A$ и $B$ больше?
}
\solutionspace{120pt}

\tasknumber{4}%
\task{%
    При какой скорости электрона его кинетическая энергия равна $E_\text{к} = 1000\,\text{эВ}?$
}
\solutionspace{120pt}

\tasknumber{5}%
\task{%
    Электрон $e^-$ вылетает из точки, потенциал которой $\varphi = 800\,\text{В}$,
    со скоростью $v = 4000000\,\frac{\text{м}}{\text{с}}$ в направлении линий напряжённости поля.
    Будет поле ускорять или тормозить электрон?
    Каков потенциал точки, дойдя до которой электрон остановится?
}

\variantsplitter

\addpersonalvariant{Виктор Жилин}

\tasknumber{1}%
\task{%
    В однородном электрическом поле напряжённостью $E = 4\,\frac{\text{кВ}}{\text{м}}$
    переместили заряд $q = 40\,\text{нКл}$ в направлении силовой линии
    на $d = 10\,\text{см}$.
    Определите работу поля, изменение потенциальной энергии заряда,
    напряжение между начальной и конечной точками перемещения.
}
\answer{%
    $
        A   = Eqd
            = 4\,\frac{\text{кВ}}{\text{м}} \cdot 40\,\text{нКл} \cdot 10\,\text{см}
            = 1600{,}00 \cdot 10^{-7} \units{Дж}
    $
}
\solutionspace{120pt}

\tasknumber{2}%
\task{%
    Напряжение между двумя точками, лежащими на одной линии напряжённости однородного электрического поля,
    равно $U = 2\,\text{кВ}$.
    Расстояние между точками $l = 30\,\text{см}$.
    Какова напряжённость этого поля?
}
\solutionspace{120pt}

\tasknumber{3}%
\task{%
    Найти напряжение между точками $A$ и $B$ в однородном электрическом поле (см.
    рис.
    на доске), если
    $AB=r = 6\,\text{см}$,
    $\alpha=45^\circ$,
    $E = 120\,\frac{\text{В}}{\text{м}}$$.
    Потенциал какой из точек $A$ и $B$ больше?
}
\solutionspace{120pt}

\tasknumber{4}%
\task{%
    При какой скорости электрона его кинетическая энергия равна $E_\text{к} = 1000\,\text{эВ}?$
}
\solutionspace{120pt}

\tasknumber{5}%
\task{%
    Электрон $e^-$ вылетает из точки, потенциал которой $\varphi = 600\,\text{В}$,
    со скоростью $v = 12000000\,\frac{\text{м}}{\text{с}}$ в направлении линий напряжённости поля.
    Будет поле ускорять или тормозить электрон?
    Каков потенциал точки, дойдя до которой электрон остановится?
}

\variantsplitter

\addpersonalvariant{Дмитрий Иванов}

\tasknumber{1}%
\task{%
    В однородном электрическом поле напряжённостью $E = 4\,\frac{\text{кВ}}{\text{м}}$
    переместили заряд $Q = 40\,\text{нКл}$ в направлении силовой линии
    на $r = 10\,\text{см}$.
    Определите работу поля, изменение потенциальной энергии заряда,
    напряжение между начальной и конечной точками перемещения.
}
\answer{%
    $
        A   = EQr
            = 4\,\frac{\text{кВ}}{\text{м}} \cdot 40\,\text{нКл} \cdot 10\,\text{см}
            = 1600{,}00 \cdot 10^{-7} \units{Дж}
    $
}
\solutionspace{120pt}

\tasknumber{2}%
\task{%
    Напряжение между двумя точками, лежащими на одной линии напряжённости однородного электрического поля,
    равно $U = 6\,\text{кВ}$.
    Расстояние между точками $d = 20\,\text{см}$.
    Какова напряжённость этого поля?
}
\solutionspace{120pt}

\tasknumber{3}%
\task{%
    Найти напряжение между точками $A$ и $B$ в однородном электрическом поле (см.
    рис.
    на доске), если
    $AB=r = 8\,\text{см}$,
    $\alpha=30^\circ$,
    $E = 60\,\frac{\text{В}}{\text{м}}$$.
    Потенциал какой из точек $A$ и $B$ больше?
}
\solutionspace{120pt}

\tasknumber{4}%
\task{%
    При какой скорости электрона его кинетическая энергия равна $E_\text{к} = 8\,\text{эВ}?$
}
\solutionspace{120pt}

\tasknumber{5}%
\task{%
    Электрон $e^-$ вылетает из точки, потенциал которой $\varphi = 800\,\text{В}$,
    со скоростью $v = 4000000\,\frac{\text{м}}{\text{с}}$ в направлении линий напряжённости поля.
    Будет поле ускорять или тормозить электрон?
    Каков потенциал точки, дойдя до которой электрон остановится?
}

\variantsplitter

\addpersonalvariant{Олег Климов}

\tasknumber{1}%
\task{%
    В однородном электрическом поле напряжённостью $E = 2\,\frac{\text{кВ}}{\text{м}}$
    переместили заряд $Q = 10\,\text{нКл}$ в направлении силовой линии
    на $d = 10\,\text{см}$.
    Определите работу поля, изменение потенциальной энергии заряда,
    напряжение между начальной и конечной точками перемещения.
}
\answer{%
    $
        A   = EQd
            = 2\,\frac{\text{кВ}}{\text{м}} \cdot 10\,\text{нКл} \cdot 10\,\text{см}
            = 200{,}00 \cdot 10^{-7} \units{Дж}
    $
}
\solutionspace{120pt}

\tasknumber{2}%
\task{%
    Напряжение между двумя точками, лежащими на одной линии напряжённости однородного электрического поля,
    равно $V = 4\,\text{кВ}$.
    Расстояние между точками $d = 10\,\text{см}$.
    Какова напряжённость этого поля?
}
\solutionspace{120pt}

\tasknumber{3}%
\task{%
    Найти напряжение между точками $A$ и $B$ в однородном электрическом поле (см.
    рис.
    на доске), если
    $AB=l = 4\,\text{см}$,
    $\alpha=45^\circ$,
    $E = 60\,\frac{\text{В}}{\text{м}}$$.
    Потенциал какой из точек $A$ и $B$ больше?
}
\solutionspace{120pt}

\tasknumber{4}%
\task{%
    При какой скорости электрона его кинетическая энергия равна $E_\text{к} = 30\,\text{эВ}?$
}
\solutionspace{120pt}

\tasknumber{5}%
\task{%
    Электрон $e^-$ вылетает из точки, потенциал которой $\varphi = 200\,\text{В}$,
    со скоростью $v = 10000000\,\frac{\text{м}}{\text{с}}$ в направлении линий напряжённости поля.
    Будет поле ускорять или тормозить электрон?
    Каков потенциал точки, дойдя до которой электрон остановится?
}

\variantsplitter

\addpersonalvariant{Анна Ковалева}

\tasknumber{1}%
\task{%
    В однородном электрическом поле напряжённостью $E = 2\,\frac{\text{кВ}}{\text{м}}$
    переместили заряд $Q = 10\,\text{нКл}$ в направлении силовой линии
    на $d = 4\,\text{см}$.
    Определите работу поля, изменение потенциальной энергии заряда,
    напряжение между начальной и конечной точками перемещения.
}
\answer{%
    $
        A   = EQd
            = 2\,\frac{\text{кВ}}{\text{м}} \cdot 10\,\text{нКл} \cdot 4\,\text{см}
            = 80{,}00 \cdot 10^{-7} \units{Дж}
    $
}
\solutionspace{120pt}

\tasknumber{2}%
\task{%
    Напряжение между двумя точками, лежащими на одной линии напряжённости однородного электрического поля,
    равно $U = 3\,\text{кВ}$.
    Расстояние между точками $r = 40\,\text{см}$.
    Какова напряжённость этого поля?
}
\solutionspace{120pt}

\tasknumber{3}%
\task{%
    Найти напряжение между точками $A$ и $B$ в однородном электрическом поле (см.
    рис.
    на доске), если
    $AB=r = 8\,\text{см}$,
    $\varphi=30^\circ$,
    $E = 120\,\frac{\text{В}}{\text{м}}$$.
    Потенциал какой из точек $A$ и $B$ больше?
}
\solutionspace{120pt}

\tasknumber{4}%
\task{%
    При какой скорости электрона его кинетическая энергия равна $E_\text{к} = 20\,\text{эВ}?$
}
\solutionspace{120pt}

\tasknumber{5}%
\task{%
    Электрон $e^-$ вылетает из точки, потенциал которой $\varphi = 400\,\text{В}$,
    со скоростью $v = 4000000\,\frac{\text{м}}{\text{с}}$ в направлении линий напряжённости поля.
    Будет поле ускорять или тормозить электрон?
    Каков потенциал точки, дойдя до которой электрон остановится?
}

\variantsplitter

\addpersonalvariant{Глеб Ковылин}

\tasknumber{1}%
\task{%
    В однородном электрическом поле напряжённостью $E = 2\,\frac{\text{кВ}}{\text{м}}$
    переместили заряд $Q = 10\,\text{нКл}$ в направлении силовой линии
    на $d = 10\,\text{см}$.
    Определите работу поля, изменение потенциальной энергии заряда,
    напряжение между начальной и конечной точками перемещения.
}
\answer{%
    $
        A   = EQd
            = 2\,\frac{\text{кВ}}{\text{м}} \cdot 10\,\text{нКл} \cdot 10\,\text{см}
            = 200{,}00 \cdot 10^{-7} \units{Дж}
    $
}
\solutionspace{120pt}

\tasknumber{2}%
\task{%
    Напряжение между двумя точками, лежащими на одной линии напряжённости однородного электрического поля,
    равно $U = 3\,\text{кВ}$.
    Расстояние между точками $d = 40\,\text{см}$.
    Какова напряжённость этого поля?
}
\solutionspace{120pt}

\tasknumber{3}%
\task{%
    Найти напряжение между точками $A$ и $B$ в однородном электрическом поле (см.
    рис.
    на доске), если
    $AB=l = 4\,\text{см}$,
    $\varphi=30^\circ$,
    $E = 30\,\frac{\text{В}}{\text{м}}$$.
    Потенциал какой из точек $A$ и $B$ больше?
}
\solutionspace{120pt}

\tasknumber{4}%
\task{%
    При какой скорости электрона его кинетическая энергия равна $E_\text{к} = 40\,\text{эВ}?$
}
\solutionspace{120pt}

\tasknumber{5}%
\task{%
    Электрон $e^-$ вылетает из точки, потенциал которой $\varphi = 200\,\text{В}$,
    со скоростью $v = 3000000\,\frac{\text{м}}{\text{с}}$ в направлении линий напряжённости поля.
    Будет поле ускорять или тормозить электрон?
    Каков потенциал точки, дойдя до которой электрон остановится?
}

\variantsplitter

\addpersonalvariant{Даниил Космынин}

\tasknumber{1}%
\task{%
    В однородном электрическом поле напряжённостью $E = 4\,\frac{\text{кВ}}{\text{м}}$
    переместили заряд $Q = 40\,\text{нКл}$ в направлении силовой линии
    на $d = 4\,\text{см}$.
    Определите работу поля, изменение потенциальной энергии заряда,
    напряжение между начальной и конечной точками перемещения.
}
\answer{%
    $
        A   = EQd
            = 4\,\frac{\text{кВ}}{\text{м}} \cdot 40\,\text{нКл} \cdot 4\,\text{см}
            = 640{,}00 \cdot 10^{-7} \units{Дж}
    $
}
\solutionspace{120pt}

\tasknumber{2}%
\task{%
    Напряжение между двумя точками, лежащими на одной линии напряжённости однородного электрического поля,
    равно $U = 3\,\text{кВ}$.
    Расстояние между точками $l = 40\,\text{см}$.
    Какова напряжённость этого поля?
}
\solutionspace{120pt}

\tasknumber{3}%
\task{%
    Найти напряжение между точками $A$ и $B$ в однородном электрическом поле (см.
    рис.
    на доске), если
    $AB=d = 10\,\text{см}$,
    $\alpha=60^\circ$,
    $E = 120\,\frac{\text{В}}{\text{м}}$$.
    Потенциал какой из точек $A$ и $B$ больше?
}
\solutionspace{120pt}

\tasknumber{4}%
\task{%
    При какой скорости электрона его кинетическая энергия равна $E_\text{к} = 4\,\text{эВ}?$
}
\solutionspace{120pt}

\tasknumber{5}%
\task{%
    Электрон $e^-$ вылетает из точки, потенциал которой $\varphi = 400\,\text{В}$,
    со скоростью $v = 10000000\,\frac{\text{м}}{\text{с}}$ в направлении линий напряжённости поля.
    Будет поле ускорять или тормозить электрон?
    Каков потенциал точки, дойдя до которой электрон остановится?
}

\variantsplitter

\addpersonalvariant{Алина Леоничева}

\tasknumber{1}%
\task{%
    В однородном электрическом поле напряжённостью $E = 20\,\frac{\text{кВ}}{\text{м}}$
    переместили заряд $q = -25\,\text{нКл}$ в направлении силовой линии
    на $l = 4\,\text{см}$.
    Определите работу поля, изменение потенциальной энергии заряда,
    напряжение между начальной и конечной точками перемещения.
}
\answer{%
    $
        A   = Eql
            = 20\,\frac{\text{кВ}}{\text{м}} \cdot -25\,\text{нКл} \cdot 4\,\text{см}
            = -2000{,}00 \cdot 10^{-7} \units{Дж}
    $
}
\solutionspace{120pt}

\tasknumber{2}%
\task{%
    Напряжение между двумя точками, лежащими на одной линии напряжённости однородного электрического поля,
    равно $V = 4\,\text{кВ}$.
    Расстояние между точками $d = 10\,\text{см}$.
    Какова напряжённость этого поля?
}
\solutionspace{120pt}

\tasknumber{3}%
\task{%
    Найти напряжение между точками $A$ и $B$ в однородном электрическом поле (см.
    рис.
    на доске), если
    $AB=r = 10\,\text{см}$,
    $\alpha=30^\circ$,
    $E = 100\,\frac{\text{В}}{\text{м}}$$.
    Потенциал какой из точек $A$ и $B$ больше?
}
\solutionspace{120pt}

\tasknumber{4}%
\task{%
    При какой скорости электрона его кинетическая энергия равна $E_\text{к} = 600\,\text{эВ}?$
}
\solutionspace{120pt}

\tasknumber{5}%
\task{%
    Электрон $e^-$ вылетает из точки, потенциал которой $\varphi = 600\,\text{В}$,
    со скоростью $v = 10000000\,\frac{\text{м}}{\text{с}}$ в направлении линий напряжённости поля.
    Будет поле ускорять или тормозить электрон?
    Каков потенциал точки, дойдя до которой электрон остановится?
}

\variantsplitter

\addpersonalvariant{Ирина Лин}

\tasknumber{1}%
\task{%
    В однородном электрическом поле напряжённостью $E = 20\,\frac{\text{кВ}}{\text{м}}$
    переместили заряд $q = -25\,\text{нКл}$ в направлении силовой линии
    на $l = 10\,\text{см}$.
    Определите работу поля, изменение потенциальной энергии заряда,
    напряжение между начальной и конечной точками перемещения.
}
\answer{%
    $
        A   = Eql
            = 20\,\frac{\text{кВ}}{\text{м}} \cdot -25\,\text{нКл} \cdot 10\,\text{см}
            = -5000{,}00 \cdot 10^{-7} \units{Дж}
    $
}
\solutionspace{120pt}

\tasknumber{2}%
\task{%
    Напряжение между двумя точками, лежащими на одной линии напряжённости однородного электрического поля,
    равно $U = 2\,\text{кВ}$.
    Расстояние между точками $d = 30\,\text{см}$.
    Какова напряжённость этого поля?
}
\solutionspace{120pt}

\tasknumber{3}%
\task{%
    Найти напряжение между точками $A$ и $B$ в однородном электрическом поле (см.
    рис.
    на доске), если
    $AB=l = 4\,\text{см}$,
    $\alpha=30^\circ$,
    $E = 100\,\frac{\text{В}}{\text{м}}$$.
    Потенциал какой из точек $A$ и $B$ больше?
}
\solutionspace{120pt}

\tasknumber{4}%
\task{%
    При какой скорости электрона его кинетическая энергия равна $E_\text{к} = 40\,\text{эВ}?$
}
\solutionspace{120pt}

\tasknumber{5}%
\task{%
    Электрон $e^-$ вылетает из точки, потенциал которой $\varphi = 600\,\text{В}$,
    со скоростью $v = 4000000\,\frac{\text{м}}{\text{с}}$ в направлении линий напряжённости поля.
    Будет поле ускорять или тормозить электрон?
    Каков потенциал точки, дойдя до которой электрон остановится?
}

\variantsplitter

\addpersonalvariant{Олег Мальцев}

\tasknumber{1}%
\task{%
    В однородном электрическом поле напряжённостью $E = 4\,\frac{\text{кВ}}{\text{м}}$
    переместили заряд $q = 10\,\text{нКл}$ в направлении силовой линии
    на $d = 10\,\text{см}$.
    Определите работу поля, изменение потенциальной энергии заряда,
    напряжение между начальной и конечной точками перемещения.
}
\answer{%
    $
        A   = Eqd
            = 4\,\frac{\text{кВ}}{\text{м}} \cdot 10\,\text{нКл} \cdot 10\,\text{см}
            = 400{,}00 \cdot 10^{-7} \units{Дж}
    $
}
\solutionspace{120pt}

\tasknumber{2}%
\task{%
    Напряжение между двумя точками, лежащими на одной линии напряжённости однородного электрического поля,
    равно $U = 3\,\text{кВ}$.
    Расстояние между точками $d = 40\,\text{см}$.
    Какова напряжённость этого поля?
}
\solutionspace{120pt}

\tasknumber{3}%
\task{%
    Найти напряжение между точками $A$ и $B$ в однородном электрическом поле (см.
    рис.
    на доске), если
    $AB=l = 4\,\text{см}$,
    $\alpha=60^\circ$,
    $E = 60\,\frac{\text{В}}{\text{м}}$$.
    Потенциал какой из точек $A$ и $B$ больше?
}
\solutionspace{120pt}

\tasknumber{4}%
\task{%
    При какой скорости электрона его кинетическая энергия равна $E_\text{к} = 8\,\text{эВ}?$
}
\solutionspace{120pt}

\tasknumber{5}%
\task{%
    Электрон $e^-$ вылетает из точки, потенциал которой $\varphi = 400\,\text{В}$,
    со скоростью $v = 4000000\,\frac{\text{м}}{\text{с}}$ в направлении линий напряжённости поля.
    Будет поле ускорять или тормозить электрон?
    Каков потенциал точки, дойдя до которой электрон остановится?
}

\variantsplitter

\addpersonalvariant{Ислам Мунаев}

\tasknumber{1}%
\task{%
    В однородном электрическом поле напряжённостью $E = 2\,\frac{\text{кВ}}{\text{м}}$
    переместили заряд $Q = -40\,\text{нКл}$ в направлении силовой линии
    на $r = 2\,\text{см}$.
    Определите работу поля, изменение потенциальной энергии заряда,
    напряжение между начальной и конечной точками перемещения.
}
\answer{%
    $
        A   = EQr
            = 2\,\frac{\text{кВ}}{\text{м}} \cdot -40\,\text{нКл} \cdot 2\,\text{см}
            = -160{,}00 \cdot 10^{-7} \units{Дж}
    $
}
\solutionspace{120pt}

\tasknumber{2}%
\task{%
    Напряжение между двумя точками, лежащими на одной линии напряжённости однородного электрического поля,
    равно $V = 4\,\text{кВ}$.
    Расстояние между точками $l = 10\,\text{см}$.
    Какова напряжённость этого поля?
}
\solutionspace{120pt}

\tasknumber{3}%
\task{%
    Найти напряжение между точками $A$ и $B$ в однородном электрическом поле (см.
    рис.
    на доске), если
    $AB=r = 6\,\text{см}$,
    $\alpha=30^\circ$,
    $E = 100\,\frac{\text{В}}{\text{м}}$$.
    Потенциал какой из точек $A$ и $B$ больше?
}
\solutionspace{120pt}

\tasknumber{4}%
\task{%
    При какой скорости электрона его кинетическая энергия равна $E_\text{к} = 20\,\text{эВ}?$
}
\solutionspace{120pt}

\tasknumber{5}%
\task{%
    Электрон $e^-$ вылетает из точки, потенциал которой $\varphi = 600\,\text{В}$,
    со скоростью $v = 12000000\,\frac{\text{м}}{\text{с}}$ в направлении линий напряжённости поля.
    Будет поле ускорять или тормозить электрон?
    Каков потенциал точки, дойдя до которой электрон остановится?
}

\variantsplitter

\addpersonalvariant{Александр Наумов}

\tasknumber{1}%
\task{%
    В однородном электрическом поле напряжённостью $E = 2\,\frac{\text{кВ}}{\text{м}}$
    переместили заряд $Q = -40\,\text{нКл}$ в направлении силовой линии
    на $d = 2\,\text{см}$.
    Определите работу поля, изменение потенциальной энергии заряда,
    напряжение между начальной и конечной точками перемещения.
}
\answer{%
    $
        A   = EQd
            = 2\,\frac{\text{кВ}}{\text{м}} \cdot -40\,\text{нКл} \cdot 2\,\text{см}
            = -160{,}00 \cdot 10^{-7} \units{Дж}
    $
}
\solutionspace{120pt}

\tasknumber{2}%
\task{%
    Напряжение между двумя точками, лежащими на одной линии напряжённости однородного электрического поля,
    равно $U = 2\,\text{кВ}$.
    Расстояние между точками $r = 20\,\text{см}$.
    Какова напряжённость этого поля?
}
\solutionspace{120pt}

\tasknumber{3}%
\task{%
    Найти напряжение между точками $A$ и $B$ в однородном электрическом поле (см.
    рис.
    на доске), если
    $AB=r = 10\,\text{см}$,
    $\alpha=60^\circ$,
    $E = 60\,\frac{\text{В}}{\text{м}}$$.
    Потенциал какой из точек $A$ и $B$ больше?
}
\solutionspace{120pt}

\tasknumber{4}%
\task{%
    При какой скорости электрона его кинетическая энергия равна $E_\text{к} = 1000\,\text{эВ}?$
}
\solutionspace{120pt}

\tasknumber{5}%
\task{%
    Электрон $e^-$ вылетает из точки, потенциал которой $\varphi = 1000\,\text{В}$,
    со скоростью $v = 10000000\,\frac{\text{м}}{\text{с}}$ в направлении линий напряжённости поля.
    Будет поле ускорять или тормозить электрон?
    Каков потенциал точки, дойдя до которой электрон остановится?
}

\variantsplitter

\addpersonalvariant{Георгий Новиков}

\tasknumber{1}%
\task{%
    В однородном электрическом поле напряжённостью $E = 20\,\frac{\text{кВ}}{\text{м}}$
    переместили заряд $Q = 10\,\text{нКл}$ в направлении силовой линии
    на $l = 4\,\text{см}$.
    Определите работу поля, изменение потенциальной энергии заряда,
    напряжение между начальной и конечной точками перемещения.
}
\answer{%
    $
        A   = EQl
            = 20\,\frac{\text{кВ}}{\text{м}} \cdot 10\,\text{нКл} \cdot 4\,\text{см}
            = 800{,}00 \cdot 10^{-7} \units{Дж}
    $
}
\solutionspace{120pt}

\tasknumber{2}%
\task{%
    Напряжение между двумя точками, лежащими на одной линии напряжённости однородного электрического поля,
    равно $V = 6\,\text{кВ}$.
    Расстояние между точками $l = 20\,\text{см}$.
    Какова напряжённость этого поля?
}
\solutionspace{120pt}

\tasknumber{3}%
\task{%
    Найти напряжение между точками $A$ и $B$ в однородном электрическом поле (см.
    рис.
    на доске), если
    $AB=r = 4\,\text{см}$,
    $\varphi=30^\circ$,
    $E = 50\,\frac{\text{В}}{\text{м}}$$.
    Потенциал какой из точек $A$ и $B$ больше?
}
\solutionspace{120pt}

\tasknumber{4}%
\task{%
    При какой скорости электрона его кинетическая энергия равна $E_\text{к} = 40\,\text{эВ}?$
}
\solutionspace{120pt}

\tasknumber{5}%
\task{%
    Электрон $e^-$ вылетает из точки, потенциал которой $\varphi = 600\,\text{В}$,
    со скоростью $v = 6000000\,\frac{\text{м}}{\text{с}}$ в направлении линий напряжённости поля.
    Будет поле ускорять или тормозить электрон?
    Каков потенциал точки, дойдя до которой электрон остановится?
}

\variantsplitter

\addpersonalvariant{Егор Осипов}

\tasknumber{1}%
\task{%
    В однородном электрическом поле напряжённостью $E = 2\,\frac{\text{кВ}}{\text{м}}$
    переместили заряд $q = -25\,\text{нКл}$ в направлении силовой линии
    на $r = 2\,\text{см}$.
    Определите работу поля, изменение потенциальной энергии заряда,
    напряжение между начальной и конечной точками перемещения.
}
\answer{%
    $
        A   = Eqr
            = 2\,\frac{\text{кВ}}{\text{м}} \cdot -25\,\text{нКл} \cdot 2\,\text{см}
            = -100{,}00 \cdot 10^{-7} \units{Дж}
    $
}
\solutionspace{120pt}

\tasknumber{2}%
\task{%
    Напряжение между двумя точками, лежащими на одной линии напряжённости однородного электрического поля,
    равно $U = 4\,\text{кВ}$.
    Расстояние между точками $r = 10\,\text{см}$.
    Какова напряжённость этого поля?
}
\solutionspace{120pt}

\tasknumber{3}%
\task{%
    Найти напряжение между точками $A$ и $B$ в однородном электрическом поле (см.
    рис.
    на доске), если
    $AB=d = 6\,\text{см}$,
    $\alpha=45^\circ$,
    $E = 60\,\frac{\text{В}}{\text{м}}$$.
    Потенциал какой из точек $A$ и $B$ больше?
}
\solutionspace{120pt}

\tasknumber{4}%
\task{%
    При какой скорости электрона его кинетическая энергия равна $E_\text{к} = 30\,\text{эВ}?$
}
\solutionspace{120pt}

\tasknumber{5}%
\task{%
    Электрон $e^-$ вылетает из точки, потенциал которой $\varphi = 800\,\text{В}$,
    со скоростью $v = 10000000\,\frac{\text{м}}{\text{с}}$ в направлении линий напряжённости поля.
    Будет поле ускорять или тормозить электрон?
    Каков потенциал точки, дойдя до которой электрон остановится?
}

\variantsplitter

\addpersonalvariant{Руслан Перепелица}

\tasknumber{1}%
\task{%
    В однородном электрическом поле напряжённостью $E = 20\,\frac{\text{кВ}}{\text{м}}$
    переместили заряд $Q = 25\,\text{нКл}$ в направлении силовой линии
    на $r = 10\,\text{см}$.
    Определите работу поля, изменение потенциальной энергии заряда,
    напряжение между начальной и конечной точками перемещения.
}
\answer{%
    $
        A   = EQr
            = 20\,\frac{\text{кВ}}{\text{м}} \cdot 25\,\text{нКл} \cdot 10\,\text{см}
            = 5000{,}00 \cdot 10^{-7} \units{Дж}
    $
}
\solutionspace{120pt}

\tasknumber{2}%
\task{%
    Напряжение между двумя точками, лежащими на одной линии напряжённости однородного электрического поля,
    равно $V = 3\,\text{кВ}$.
    Расстояние между точками $l = 40\,\text{см}$.
    Какова напряжённость этого поля?
}
\solutionspace{120pt}

\tasknumber{3}%
\task{%
    Найти напряжение между точками $A$ и $B$ в однородном электрическом поле (см.
    рис.
    на доске), если
    $AB=l = 12\,\text{см}$,
    $\varphi=45^\circ$,
    $E = 100\,\frac{\text{В}}{\text{м}}$$.
    Потенциал какой из точек $A$ и $B$ больше?
}
\solutionspace{120pt}

\tasknumber{4}%
\task{%
    При какой скорости электрона его кинетическая энергия равна $E_\text{к} = 50\,\text{эВ}?$
}
\solutionspace{120pt}

\tasknumber{5}%
\task{%
    Электрон $e^-$ вылетает из точки, потенциал которой $\varphi = 600\,\text{В}$,
    со скоростью $v = 3000000\,\frac{\text{м}}{\text{с}}$ в направлении линий напряжённости поля.
    Будет поле ускорять или тормозить электрон?
    Каков потенциал точки, дойдя до которой электрон остановится?
}

\variantsplitter

\addpersonalvariant{Михаил Перин}

\tasknumber{1}%
\task{%
    В однородном электрическом поле напряжённостью $E = 20\,\frac{\text{кВ}}{\text{м}}$
    переместили заряд $q = 40\,\text{нКл}$ в направлении силовой линии
    на $r = 2\,\text{см}$.
    Определите работу поля, изменение потенциальной энергии заряда,
    напряжение между начальной и конечной точками перемещения.
}
\answer{%
    $
        A   = Eqr
            = 20\,\frac{\text{кВ}}{\text{м}} \cdot 40\,\text{нКл} \cdot 2\,\text{см}
            = 1600{,}00 \cdot 10^{-7} \units{Дж}
    $
}
\solutionspace{120pt}

\tasknumber{2}%
\task{%
    Напряжение между двумя точками, лежащими на одной линии напряжённости однородного электрического поля,
    равно $V = 4\,\text{кВ}$.
    Расстояние между точками $r = 40\,\text{см}$.
    Какова напряжённость этого поля?
}
\solutionspace{120pt}

\tasknumber{3}%
\task{%
    Найти напряжение между точками $A$ и $B$ в однородном электрическом поле (см.
    рис.
    на доске), если
    $AB=d = 4\,\text{см}$,
    $\alpha=45^\circ$,
    $E = 50\,\frac{\text{В}}{\text{м}}$$.
    Потенциал какой из точек $A$ и $B$ больше?
}
\solutionspace{120pt}

\tasknumber{4}%
\task{%
    При какой скорости электрона его кинетическая энергия равна $E_\text{к} = 200\,\text{эВ}?$
}
\solutionspace{120pt}

\tasknumber{5}%
\task{%
    Электрон $e^-$ вылетает из точки, потенциал которой $\varphi = 400\,\text{В}$,
    со скоростью $v = 6000000\,\frac{\text{м}}{\text{с}}$ в направлении линий напряжённости поля.
    Будет поле ускорять или тормозить электрон?
    Каков потенциал точки, дойдя до которой электрон остановится?
}

\variantsplitter

\addpersonalvariant{Егор Подуровский}

\tasknumber{1}%
\task{%
    В однородном электрическом поле напряжённостью $E = 2\,\frac{\text{кВ}}{\text{м}}$
    переместили заряд $Q = -40\,\text{нКл}$ в направлении силовой линии
    на $r = 2\,\text{см}$.
    Определите работу поля, изменение потенциальной энергии заряда,
    напряжение между начальной и конечной точками перемещения.
}
\answer{%
    $
        A   = EQr
            = 2\,\frac{\text{кВ}}{\text{м}} \cdot -40\,\text{нКл} \cdot 2\,\text{см}
            = -160{,}00 \cdot 10^{-7} \units{Дж}
    $
}
\solutionspace{120pt}

\tasknumber{2}%
\task{%
    Напряжение между двумя точками, лежащими на одной линии напряжённости однородного электрического поля,
    равно $U = 3\,\text{кВ}$.
    Расстояние между точками $d = 20\,\text{см}$.
    Какова напряжённость этого поля?
}
\solutionspace{120pt}

\tasknumber{3}%
\task{%
    Найти напряжение между точками $A$ и $B$ в однородном электрическом поле (см.
    рис.
    на доске), если
    $AB=d = 4\,\text{см}$,
    $\varphi=45^\circ$,
    $E = 100\,\frac{\text{В}}{\text{м}}$$.
    Потенциал какой из точек $A$ и $B$ больше?
}
\solutionspace{120pt}

\tasknumber{4}%
\task{%
    При какой скорости электрона его кинетическая энергия равна $E_\text{к} = 1000\,\text{эВ}?$
}
\solutionspace{120pt}

\tasknumber{5}%
\task{%
    Электрон $e^-$ вылетает из точки, потенциал которой $\varphi = 400\,\text{В}$,
    со скоростью $v = 4000000\,\frac{\text{м}}{\text{с}}$ в направлении линий напряжённости поля.
    Будет поле ускорять или тормозить электрон?
    Каков потенциал точки, дойдя до которой электрон остановится?
}

\variantsplitter

\addpersonalvariant{Роман Прибылов}

\tasknumber{1}%
\task{%
    В однородном электрическом поле напряжённостью $E = 2\,\frac{\text{кВ}}{\text{м}}$
    переместили заряд $Q = -40\,\text{нКл}$ в направлении силовой линии
    на $l = 10\,\text{см}$.
    Определите работу поля, изменение потенциальной энергии заряда,
    напряжение между начальной и конечной точками перемещения.
}
\answer{%
    $
        A   = EQl
            = 2\,\frac{\text{кВ}}{\text{м}} \cdot -40\,\text{нКл} \cdot 10\,\text{см}
            = -800{,}00 \cdot 10^{-7} \units{Дж}
    $
}
\solutionspace{120pt}

\tasknumber{2}%
\task{%
    Напряжение между двумя точками, лежащими на одной линии напряжённости однородного электрического поля,
    равно $U = 6\,\text{кВ}$.
    Расстояние между точками $l = 20\,\text{см}$.
    Какова напряжённость этого поля?
}
\solutionspace{120pt}

\tasknumber{3}%
\task{%
    Найти напряжение между точками $A$ и $B$ в однородном электрическом поле (см.
    рис.
    на доске), если
    $AB=r = 10\,\text{см}$,
    $\alpha=30^\circ$,
    $E = 50\,\frac{\text{В}}{\text{м}}$$.
    Потенциал какой из точек $A$ и $B$ больше?
}
\solutionspace{120pt}

\tasknumber{4}%
\task{%
    При какой скорости электрона его кинетическая энергия равна $E_\text{к} = 8\,\text{эВ}?$
}
\solutionspace{120pt}

\tasknumber{5}%
\task{%
    Электрон $e^-$ вылетает из точки, потенциал которой $\varphi = 400\,\text{В}$,
    со скоростью $v = 4000000\,\frac{\text{м}}{\text{с}}$ в направлении линий напряжённости поля.
    Будет поле ускорять или тормозить электрон?
    Каков потенциал точки, дойдя до которой электрон остановится?
}

\variantsplitter

\addpersonalvariant{Александр Селехметьев}

\tasknumber{1}%
\task{%
    В однородном электрическом поле напряжённостью $E = 20\,\frac{\text{кВ}}{\text{м}}$
    переместили заряд $q = -10\,\text{нКл}$ в направлении силовой линии
    на $l = 5\,\text{см}$.
    Определите работу поля, изменение потенциальной энергии заряда,
    напряжение между начальной и конечной точками перемещения.
}
\answer{%
    $
        A   = Eql
            = 20\,\frac{\text{кВ}}{\text{м}} \cdot -10\,\text{нКл} \cdot 5\,\text{см}
            = -1000{,}00 \cdot 10^{-7} \units{Дж}
    $
}
\solutionspace{120pt}

\tasknumber{2}%
\task{%
    Напряжение между двумя точками, лежащими на одной линии напряжённости однородного электрического поля,
    равно $V = 3\,\text{кВ}$.
    Расстояние между точками $l = 40\,\text{см}$.
    Какова напряжённость этого поля?
}
\solutionspace{120pt}

\tasknumber{3}%
\task{%
    Найти напряжение между точками $A$ и $B$ в однородном электрическом поле (см.
    рис.
    на доске), если
    $AB=r = 6\,\text{см}$,
    $\varphi=45^\circ$,
    $E = 50\,\frac{\text{В}}{\text{м}}$$.
    Потенциал какой из точек $A$ и $B$ больше?
}
\solutionspace{120pt}

\tasknumber{4}%
\task{%
    При какой скорости электрона его кинетическая энергия равна $E_\text{к} = 8\,\text{эВ}?$
}
\solutionspace{120pt}

\tasknumber{5}%
\task{%
    Электрон $e^-$ вылетает из точки, потенциал которой $\varphi = 600\,\text{В}$,
    со скоростью $v = 6000000\,\frac{\text{м}}{\text{с}}$ в направлении линий напряжённости поля.
    Будет поле ускорять или тормозить электрон?
    Каков потенциал точки, дойдя до которой электрон остановится?
}

\variantsplitter

\addpersonalvariant{Алексей Тихонов}

\tasknumber{1}%
\task{%
    В однородном электрическом поле напряжённостью $E = 2\,\frac{\text{кВ}}{\text{м}}$
    переместили заряд $q = 10\,\text{нКл}$ в направлении силовой линии
    на $l = 2\,\text{см}$.
    Определите работу поля, изменение потенциальной энергии заряда,
    напряжение между начальной и конечной точками перемещения.
}
\answer{%
    $
        A   = Eql
            = 2\,\frac{\text{кВ}}{\text{м}} \cdot 10\,\text{нКл} \cdot 2\,\text{см}
            = 40{,}00 \cdot 10^{-7} \units{Дж}
    $
}
\solutionspace{120pt}

\tasknumber{2}%
\task{%
    Напряжение между двумя точками, лежащими на одной линии напряжённости однородного электрического поля,
    равно $U = 6\,\text{кВ}$.
    Расстояние между точками $d = 20\,\text{см}$.
    Какова напряжённость этого поля?
}
\solutionspace{120pt}

\tasknumber{3}%
\task{%
    Найти напряжение между точками $A$ и $B$ в однородном электрическом поле (см.
    рис.
    на доске), если
    $AB=r = 12\,\text{см}$,
    $\alpha=30^\circ$,
    $E = 50\,\frac{\text{В}}{\text{м}}$$.
    Потенциал какой из точек $A$ и $B$ больше?
}
\solutionspace{120pt}

\tasknumber{4}%
\task{%
    При какой скорости электрона его кинетическая энергия равна $E_\text{к} = 1000\,\text{эВ}?$
}
\solutionspace{120pt}

\tasknumber{5}%
\task{%
    Электрон $e^-$ вылетает из точки, потенциал которой $\varphi = 600\,\text{В}$,
    со скоростью $v = 6000000\,\frac{\text{м}}{\text{с}}$ в направлении линий напряжённости поля.
    Будет поле ускорять или тормозить электрон?
    Каков потенциал точки, дойдя до которой электрон остановится?
}

\variantsplitter

\addpersonalvariant{Алина Филиппова}

\tasknumber{1}%
\task{%
    В однородном электрическом поле напряжённостью $E = 2\,\frac{\text{кВ}}{\text{м}}$
    переместили заряд $q = 40\,\text{нКл}$ в направлении силовой линии
    на $l = 5\,\text{см}$.
    Определите работу поля, изменение потенциальной энергии заряда,
    напряжение между начальной и конечной точками перемещения.
}
\answer{%
    $
        A   = Eql
            = 2\,\frac{\text{кВ}}{\text{м}} \cdot 40\,\text{нКл} \cdot 5\,\text{см}
            = 400{,}00 \cdot 10^{-7} \units{Дж}
    $
}
\solutionspace{120pt}

\tasknumber{2}%
\task{%
    Напряжение между двумя точками, лежащими на одной линии напряжённости однородного электрического поля,
    равно $U = 2\,\text{кВ}$.
    Расстояние между точками $d = 30\,\text{см}$.
    Какова напряжённость этого поля?
}
\solutionspace{120pt}

\tasknumber{3}%
\task{%
    Найти напряжение между точками $A$ и $B$ в однородном электрическом поле (см.
    рис.
    на доске), если
    $AB=d = 12\,\text{см}$,
    $\varphi=30^\circ$,
    $E = 50\,\frac{\text{В}}{\text{м}}$$.
    Потенциал какой из точек $A$ и $B$ больше?
}
\solutionspace{120pt}

\tasknumber{4}%
\task{%
    При какой скорости электрона его кинетическая энергия равна $E_\text{к} = 50\,\text{эВ}?$
}
\solutionspace{120pt}

\tasknumber{5}%
\task{%
    Электрон $e^-$ вылетает из точки, потенциал которой $\varphi = 400\,\text{В}$,
    со скоростью $v = 6000000\,\frac{\text{м}}{\text{с}}$ в направлении линий напряжённости поля.
    Будет поле ускорять или тормозить электрон?
    Каков потенциал точки, дойдя до которой электрон остановится?
}

\variantsplitter

\addpersonalvariant{Алина Яшина}

\tasknumber{1}%
\task{%
    В однородном электрическом поле напряжённостью $E = 2\,\frac{\text{кВ}}{\text{м}}$
    переместили заряд $q = 10\,\text{нКл}$ в направлении силовой линии
    на $d = 2\,\text{см}$.
    Определите работу поля, изменение потенциальной энергии заряда,
    напряжение между начальной и конечной точками перемещения.
}
\answer{%
    $
        A   = Eqd
            = 2\,\frac{\text{кВ}}{\text{м}} \cdot 10\,\text{нКл} \cdot 2\,\text{см}
            = 40{,}00 \cdot 10^{-7} \units{Дж}
    $
}
\solutionspace{120pt}

\tasknumber{2}%
\task{%
    Напряжение между двумя точками, лежащими на одной линии напряжённости однородного электрического поля,
    равно $V = 4\,\text{кВ}$.
    Расстояние между точками $d = 20\,\text{см}$.
    Какова напряжённость этого поля?
}
\solutionspace{120pt}

\tasknumber{3}%
\task{%
    Найти напряжение между точками $A$ и $B$ в однородном электрическом поле (см.
    рис.
    на доске), если
    $AB=d = 6\,\text{см}$,
    $\varphi=30^\circ$,
    $E = 120\,\frac{\text{В}}{\text{м}}$$.
    Потенциал какой из точек $A$ и $B$ больше?
}
\solutionspace{120pt}

\tasknumber{4}%
\task{%
    При какой скорости электрона его кинетическая энергия равна $E_\text{к} = 600\,\text{эВ}?$
}
\solutionspace{120pt}

\tasknumber{5}%
\task{%
    Электрон $e^-$ вылетает из точки, потенциал которой $\varphi = 400\,\text{В}$,
    со скоростью $v = 10000000\,\frac{\text{м}}{\text{с}}$ в направлении линий напряжённости поля.
    Будет поле ускорять или тормозить электрон?
    Каков потенциал точки, дойдя до которой электрон остановится?
}

\end{document}
% autogenerated
