\documentclass[12pt,a4paper]{amsart}%DVI-mode.
\usepackage{graphics,graphicx,epsfig}%DVI-mode.
% \documentclass[pdftex,12pt]{amsart} %PDF-mode.
% \usepackage[pdftex]{graphicx}       %PDF-mode.
% \usepackage[babel=true]{microtype}
% \usepackage[T1]{fontenc}
% \usepackage{lmodern}

\usepackage{cmap}
%\usepackage{a4wide}                 % Fit the text to A4 page tightly.
% \usepackage[utf8]{inputenc}
\usepackage[T2A]{fontenc}
\usepackage[english,russian]{babel} % Download Russian fonts.
\usepackage{amsmath,amsfonts,amssymb,amsthm,amscd,mathrsfs} % Use AMS symbols.
\usepackage{tikz}
\usetikzlibrary{circuits.ee.IEC}
\usetikzlibrary{shapes.geometric}
\usetikzlibrary{decorations.markings}
%\usetikzlibrary{dashs}
%\usetikzlibrary{info}


\textheight=28cm % высота текста
\textwidth=18cm % ширина текста
\topmargin=-2.5cm % отступ от верхнего края
\parskip=2pt % интервал между абзацами
\oddsidemargin=-1.5cm
\evensidemargin=-1.5cm 

\parindent=0pt % абзацный отступ
\tolerance=500 % терпимость к "жидким" строкам
\binoppenalty=10000 % штраф за перенос формул - 10000 - абсолютный запрет
\relpenalty=10000
\flushbottom % выравнивание высоты страниц
\pagenumbering{gobble}

\newcommand\bivec[2]{\begin{pmatrix} #1 \\ #2 \end{pmatrix}}

\newcommand\ol[1]{\overline{#1}}

\newcommand\p[1]{\Prob\!\left(#1\right)}
\newcommand\e[1]{\mathsf{E}\!\left(#1\right)}
\newcommand\disp[1]{\mathsf{D}\!\left(#1\right)}
%\newcommand\norm[2]{\mathcal{N}\!\cbr{#1,#2}}
\newcommand\sign{\text{ sign }}

\newcommand\al[1]{\begin{align*} #1 \end{align*}}
\newcommand\begcas[1]{\begin{cases}#1\end{cases}}
\newcommand\tab[2]{	\vspace{-#1pt}
						\begin{tabbing} 
						#2
						\end{tabbing}
					\vspace{-#1pt}
					}

\newcommand\maintext[1]{{\bfseries\sffamily{#1}}}
\newcommand\skipped[1]{ {\ensuremath{\text{\small{\sffamily{Пропущено:} #1} } } } }
\newcommand\simpletitle[1]{\begin{center} \maintext{#1} \end{center}}

\def\le{\leqslant}
\def\ge{\geqslant}
\def\Ell{\mathcal{L}}
\def\eps{{\varepsilon}}
\def\Rn{\mathbb{R}^n}
\def\RSS{\mathsf{RSS}}

\newcommand\foral[1]{\forall\,#1\:}
\newcommand\exist[1]{\exists\,#1\:\colon}

\newcommand\cbr[1]{\left(#1\right)} %circled brackets
\newcommand\fbr[1]{\left\{#1\right\}} %figure brackets
\newcommand\sbr[1]{\left[#1\right]} %square brackets
\newcommand\modul[1]{\left|#1\right|}

\newcommand\sqr[1]{\cbr{#1}^2}
\newcommand\inv[1]{\cbr{#1}^{-1}}

\newcommand\cdf[2]{\cdot\frac{#1}{#2}}
\newcommand\dd[2]{\frac{\partial#1}{\partial#2}}

\newcommand\integr[2]{\int\limits_{#1}^{#2}}
\newcommand\suml[2]{\sum\limits_{#1}^{#2}}
\newcommand\isum[2]{\sum\limits_{#1=#2}^{+\infty}}
\newcommand\idots[3]{#1_{#2},\ldots,#1_{#3}}
\newcommand\fdots[5]{#4{#1_{#2}}#5\ldots#5#4{#1_{#3}}}

\newcommand\obol[1]{O\!\cbr{#1}}
\newcommand\omal[1]{o\!\cbr{#1}}

\newcommand\addeps[2]{
	\begin{figure} [!ht] %lrp
		\centering
		\includegraphics[height=320px]{#1.eps}
		\vspace{-10pt}
		\caption{#2}
		\label{eps:#1}
	\end{figure}
}

\newcommand\addepssize[3]{
	\begin{figure} [!ht] %lrp hp
		\centering
		\includegraphics[height=#3px]{#1.eps}
		\vspace{-10pt}
		\caption{#2}
		\label{eps:#1}
	\end{figure}
}


\newcommand\norm[1]{\ensuremath{\left\|{#1}\right\|}}
\newcommand\ort{\bot}
\newcommand\theorem[1]{{\sffamily Теорема #1\ }}
\newcommand\lemma[1]{{\sffamily Лемма #1\ }}
\newcommand\difflim[2]{\frac{#1\cbr{#2 + \Delta#2} - #1\cbr{#2}}{\Delta #2}}
\renewcommand\proof[1]{\par\noindent$\square$ #1 \hfill$\blacksquare$\par}
\newcommand\defenition[1]{{\sffamilyОпределение #1\ }}

% \begin{document}
% %\raggedright
% \addclassdate{7}{20 апреля 2018}

\task 1
Площадь большого поршня гидравлического домкрата $S_1 = 20\units{см}^2$, а малого $S_2 = 0{,}5\units{см}^2.$ Груз какой максимальной массы можно поднять этим домкратом, если на малый поршень давить с силой не более $F=200\units{Н}?$ Силой трения от стенки цилиндров пренебречь.

\task 2
В сосуд налита вода. Расстояние от поверхности воды до дна $H = 0{,}5\units{м},$ площадь дна $S = 0{,}1\units{м}^2.$ Найти гидростатическое давление $P_1$ и полное давление $P_2$ вблизи дна. Найти силу давления воды на дно. Плотность воды \rhowater

\task 3
На лёгкий поршень площадью $S=900\units{см}^2,$ касающийся поверхности воды, поставили гирю массы $m=3\units{кг}$. Высота слоя воды в сосуде с вертикальными стенками $H = 20\units{см}$. Определить давление жидкости вблизи дна, если плотность воды \rhowater

\task 4
Давление газов в конце сгорания в цилиндре дизельного двигателя трактора $P = 9\units{МПа}.$ Диаметр цилиндра $d = 130\units{мм}.$ С какой силой газы давят на поршень в цилиндре? Площадь круга диаметром $D$ равна $S = \cfrac{\pi D^2}4.$

\task 5
Площадь малого поршня гидравлического подъёмника $S_1 = 0{,}8\units{см}^2$, а большого $S_2 = 40\units{см}^2.$ Какую силу $F$ надо приложить к малому поршню, чтобы поднять груз весом $P = 8\units{кН}?$

\task 6
Герметичный сосуд полностью заполнен водой и стоит на столе. На небольшой поршень площадью $S$ давят рукой с силой $F$. Поршень находится ниже крышки сосуда на $H_1$, выше дна на $H_2$ и может свободно перемещаться. Плотность воды $\rho$, атмосферное давление $P_A$. Найти давления $P_1$ и $P_2$ в воде вблизи крышки и дна сосуда.
\\ \\
\addclassdate{7}{20 апреля 2018}

\task 1
Площадь большого поршня гидравлического домкрата $S_1 = 20\units{см}^2$, а малого $S_2 = 0{,}5\units{см}^2.$ Груз какой максимальной массы можно поднять этим домкратом, если на малый поршень давить с силой не более $F=200\units{Н}?$ Силой трения от стенки цилиндров пренебречь.

\task 2
В сосуд налита вода. Расстояние от поверхности воды до дна $H = 0{,}5\units{м},$ площадь дна $S = 0{,}1\units{м}^2.$ Найти гидростатическое давление $P_1$ и полное давление $P_2$ вблизи дна. Найти силу давления воды на дно. Плотность воды \rhowater

\task 3
На лёгкий поршень площадью $S=900\units{см}^2,$ касающийся поверхности воды, поставили гирю массы $m=3\units{кг}$. Высота слоя воды в сосуде с вертикальными стенками $H = 20\units{см}$. Определить давление жидкости вблизи дна, если плотность воды \rhowater

\task 4
Давление газов в конце сгорания в цилиндре дизельного двигателя трактора $P = 9\units{МПа}.$ Диаметр цилиндра $d = 130\units{мм}.$ С какой силой газы давят на поршень в цилиндре? Площадь круга диаметром $D$ равна $S = \cfrac{\pi D^2}4.$

\task 5
Площадь малого поршня гидравлического подъёмника $S_1 = 0{,}8\units{см}^2$, а большого $S_2 = 40\units{см}^2.$ Какую силу $F$ надо приложить к малому поршню, чтобы поднять груз весом $P = 8\units{кН}?$

\task 6
Герметичный сосуд полностью заполнен водой и стоит на столе. На небольшой поршень площадью $S$ давят рукой с силой $F$. Поршень находится ниже крышки сосуда на $H_1$, выше дна на $H_2$ и может свободно перемещаться. Плотность воды $\rho$, атмосферное давление $P_A$. Найти давления $P_1$ и $P_2$ в воде вблизи крышки и дна сосуда.

\newpage

\adddate{8 класс. 20 апреля 2018}

\task 1
Между точками $A$ и $B$ электрической цепи подключены последовательно резисторы $R_1 = 10\units{Ом}$ и $R_2 = 20\units{Ом}$ и параллельно им $R_3 = 30\units{Ом}.$ Найдите эквивалентное сопротивление $R_{AB}$ этого участка цепи.

\task 2
Электрическая цепь состоит из последовательности $N$ одинаковых звеньев, в которых каждый резистор имеет сопротивление $r$. Последнее звено замкнуто резистором сопротивлением $R$. При каком соотношении $\cfrac{R}{r}$ сопротивление цепи не зависит от числа звеньев?

\task 3
Для измерения сопротивления $R$ проводника собрана электрическая цепь. Вольтметр $V$ показывает напряжение $U_V = 5\units{В},$ показание амперметра $A$ равно $I_A = 25\units{мА}.$ Найдите величину $R$ сопротивления проводника. Внутреннее сопротивление вольтметра $R_V = 1{,}0\units{кОм},$ внутреннее сопротивление амперметра $R_A = 2{,}0\units{Ом}.$

\task 4
Шкала гальванометра имеет $N=100$ делений, цена деления $\delta = 1\units{мкА}$. Внутреннее сопротивление гальванометра $R_G = 1{,}0\units{кОм}.$ Как из этого прибора сделать вольтметр для измерения напряжений до $U = 100\units{В}$ или амперметр для измерения токов силой до $I = 1\units{А}?$

\\ \\ \\ \\ \\ \\ \\ \\
\adddate{8 класс. 20 апреля 2018}

\task 1
Между точками $A$ и $B$ электрической цепи подключены последовательно резисторы $R_1 = 10\units{Ом}$ и $R_2 = 20\units{Ом}$ и параллельно им $R_3 = 30\units{Ом}.$ Найдите эквивалентное сопротивление $R_{AB}$ этого участка цепи.

\task 2
Электрическая цепь состоит из последовательности $N$ одинаковых звеньев, в которых каждый резистор имеет сопротивление $r$. Последнее звено замкнуто резистором сопротивлением $R$. При каком соотношении $\cfrac{R}{r}$ сопротивление цепи не зависит от числа звеньев?

\task 3
Для измерения сопротивления $R$ проводника собрана электрическая цепь. Вольтметр $V$ показывает напряжение $U_V = 5\units{В},$ показание амперметра $A$ равно $I_A = 25\units{мА}.$ Найдите величину $R$ сопротивления проводника. Внутреннее сопротивление вольтметра $R_V = 1{,}0\units{кОм},$ внутреннее сопротивление амперметра $R_A = 2{,}0\units{Ом}.$

\task 4
Шкала гальванометра имеет $N=100$ делений, цена деления $\delta = 1\units{мкА}$. Внутреннее сопротивление гальванометра $R_G = 1{,}0\units{кОм}.$ Как из этого прибора сделать вольтметр для измерения напряжений до $U = 100\units{В}$ или амперметр для измерения токов силой до $I = 1\units{А}?$


% % \begin{flushright}
\textsc{ГБОУ школа №554, 20 ноября 2018\,г.}
\end{flushright}

\begin{center}
\LARGE \textsc{Математический бой, 8 класс}
\end{center}

\problem{1} Есть тридцать карточек, на каждой написано по одному числу: на десяти карточках~–~$a$,  на десяти других~–~$b$ и на десяти оставшихся~–~$c$ (числа  различны). Известно, что к любым пяти карточкам можно подобрать ещё пять так, что сумма чисел на этих десяти карточках будет равна нулю. Докажите, что~одно из~чисел~$a, b, c$ равно нулю.

\problem{2} Вокруг стола стола пустили пакет с орешками. Первый взял один орешек, второй — 2, третий — 3 и так далее: каждый следующий брал на 1 орешек больше. Известно, что на втором круге было взято в сумме на 100 орешков больше, чем на первом. Сколько человек сидело за столом?

% \problem{2} Натуральное число разрешено увеличить на любое целое число процентов от 1 до 100, если при этом получаем натуральное число. Найдите наименьшее натуральное число, которое нельзя при помощи таких операций получить из~числа 1.

% \problem{3} Найти сумму $1^2 - 2^2 + 3^2 - 4^2 + 5^2 + \ldots - 2018^2$.

\problem{3} В кружке рукоделия, где занимается Валя, более 93\% участников~—~девочки. Какое наименьшее число участников может быть в таком кружке?

\problem{4} Произведение 2018 целых чисел равно 1. Может ли их сумма оказаться равной~0?

% \problem{4} Можно ли все натуральные числа от~1 до~9 записать в~клетки таблицы~$3\times3$ так, чтобы сумма в~любых двух соседних (по~вертикали или горизонтали) клетках равнялось простому числу?

\problem{5} На доске написано 2018 нулей и 2019 единиц. Женя стирает 2 числа и, если они были одинаковы, дописывает к оставшимся один ноль, а~если разные — единицу. Потом Женя повторяет эту операцию снова, потом ещё и~так далее. В~результате на~доске останется только одно число. Что это за~число?

\problem{6} Докажите, что в~любой компании людей найдутся 2~человека, имеющие равное число знакомых в этой компании (если $A$~знаком с~$B$, то~и $B$~знаком с~$A$).

\problem{7} Три колокола начинают бить одновременно. Интервалы между ударами колоколов соответственно составляют $\cfrac43$~секунды, $\cfrac53$~секунды и $2$~секунды. Совпавшие по времени удары воспринимаются за~один. Сколько ударов будет услышано за 1~минуту, включая первый и последний удары?

\problem{8} Восемь одинаковых момент расположены по кругу. Известно, что три из~них~— фальшивые, и они расположены рядом друг с~другом. Вес фальшивой монеты отличается от~веса настоящей. Все фальшивые монеты весят одинаково, но неизвестно, тяжелее или легче фальшивая монета настоящей. Покажите, что за~3~взвешивания на~чашечных весах без~гирь можно определить все фальшивые монеты.

% \end{document}

\begin{document}
\noanswers

\setdate{29~января~2021}
\setclass{10«АБ»}

\addpersonalvariant{Михаил Бурмистров}

\tasknumber{1}%
\task{%
    Укажите, верны ли утверждения («да» или «нет» слева от каждого утверждения):
    \begin{enumerate}
        \item Увеличение температуры на 3 градуса цельсия всегда соответствует увеличению на 3 градуса кельвина.
        \item Температуру тела всегда можно понизить на 30 кельвин (пусть при этом и может произойти фазовый переход).
        % \item Температуру тела всегда можно повысить на 30 кельвин (пусть при этом и может произойти фазовый переход).
        \item Температуру тела всегда можно понизить на 30 градусов Цельсия (пусть при этом и может произойти фазовый переход).
        % \item Температуру тела всегда можно повысить на 30 градусов Цельсия (пусть при этом и может произойти фазовый переход).

        \item У шкалы температур Кельвина есть минимальное значение (пусть и недостижимое): 0 кельвин, а у шкалы Цельсия такого значения нет вовсе и возможны температуры меньше 0 градусов цельсия.
        \item Если бы стекло, из которого изготовлен термометр, расширялось при нагревании сильнее жидкости внутри, то мы бы наблюдали, как столбик жидкости укорачивается при нагревании.
        \item Давление газа на окружающий его сосуд вызвано ударами молекул газа о его стенки: при этом изменяется импульс молекул, а значит кто-то (стенка) действовала с некоторой силой, а тогда по 3 закону Ньютона и газ действовал на стенку.
        \item В модели идеального газа невозможен теплообмен: например, если смешать две порции кислорода и азота разной температуры, то их молекулы не будут сталкиваться и обмениваться энергиями.
        Диффузия при этом произойдет.
        \item Основное уравнение МКТ идеального газа применимо к газам сколь угодно малой плотности.

        \item Основное уравнение МКТ способно описать даже плазму: состояние вещества, при котором молекулы от ударов друг об друга начинают расщепляется на ионы и электроны.
        \item Основное уравнение МКТ ИГ может быть получено теоретически из модели идеального газа, однако в нем присутствуют микропараметры, поэтому оно не допускает непосредственной экспериментальной проверки.

        \item Все процессы: изохорный, изобарный, изотермный по умолчанию предполагают, что количество вещества в них не изменяется.
        \item При горении, например, водорода в кислороде (2H2+O2-2H2O), не изменяется, ни масса вещества участвующего в реакции, ни его количество.
        Также при этом не изменяется и количество протонов, нейтронов и электронов.
        \item Каждый набор макропараметров идеального газа (P, V и T) задаёт точку в трехмерном пространстве.
        При их изменении образуется линия в этом пространстве.
        Строя графики изопроцессов в координатах PV, VT, PT мы строим проекцию этой линии на одну из плоскостей.
    \end{enumerate}
}

\tasknumber{2}%
\task{%
    Выразите:
    \begin{enumerate}
        \item плотность тела через его массу и объём,
        \item количество вещества через число частиц и число Авогадро,
        \item основное уравнение МКТ идеального газа через концентрацию и среднюю кинетическую энергию поступательного движения его молекул.
    \end{enumerate}
}

\tasknumber{3}%
\task{%
    Напротив каждой физической величины укажите её обозначение и единицы измерения в СИ:
    \begin{enumerate}
        \item температура в Цельсиях,
        \item плотность,
        \item число Авогадро.
    \end{enumerate}
}

\tasknumber{4}%
\task{%
    Запишите, как бы вы обозначили...
    \begin{enumerate}
        \item увеличение давления в сосуде с газом,
        \item число частиц: до и после утечки газа из баллона,
        \item массы газа: до и после утечки газа из баллона,
        \item объемы газа: до и после сжатия в 4 раза.
    \end{enumerate}
}

\tasknumber{5}%
\task{%
    Запишите, какие физические величины соответствуют следующим единицам измерения (указать название и обозначение),
    \begin{enumerate}
        \item градус Цельсия,
        \item мПа,
        \item мДж,
        \item $\text{ л }$,
        \item $\units{ г }$.
    \end{enumerate}
}

\tasknumber{6}%
\task{%
    Выразите одну величину через остальные, используя при необходимости постоянную Больцмана, число Авогадро или универсальную газовую постоянную:
    \begin{enumerate}
        \item температуру газа через его давление, объем, число частиц,
        \item плотность газа через концентрацию молекул и массу молекулы,
        \item среднюю кинетическую энергию поступательного движения молекул через температуру,
        \item средний квадрат скорости молекул через скорости отдельных молекул,
    \end{enumerate}
}

\variantsplitter

\addpersonalvariant{Ирина Ан}

\tasknumber{1}%
\task{%
    Укажите, верны ли утверждения («да» или «нет» слева от каждого утверждения):
    \begin{enumerate}
        \item Увеличение температуры на 3 градуса цельсия всегда соответствует увеличению на 3 градуса кельвина.
        \item Температуру тела всегда можно понизить на 30 кельвин (пусть при этом и может произойти фазовый переход).
        % \item Температуру тела всегда можно повысить на 30 кельвин (пусть при этом и может произойти фазовый переход).
        \item Температуру тела всегда можно понизить на 30 градусов Цельсия (пусть при этом и может произойти фазовый переход).
        % \item Температуру тела всегда можно повысить на 30 градусов Цельсия (пусть при этом и может произойти фазовый переход).

        \item У шкалы температур Кельвина есть минимальное значение (пусть и недостижимое): 0 кельвин, а у шкалы Цельсия такого значения нет вовсе и возможны температуры меньше 0 градусов цельсия.
        \item Если бы стекло, из которого изготовлен термометр, расширялось при нагревании сильнее жидкости внутри, то мы бы наблюдали, как столбик жидкости укорачивается при нагревании.
        \item Давление газа на окружающий его сосуд вызвано ударами молекул газа о его стенки: при этом изменяется импульс молекул, а значит кто-то (стенка) действовала с некоторой силой, а тогда по 3 закону Ньютона и газ действовал на стенку.
        \item В модели идеального газа невозможен теплообмен: например, если смешать две порции кислорода и азота разной температуры, то их молекулы не будут сталкиваться и обмениваться энергиями.
        Диффузия при этом произойдет.
        \item Основное уравнение МКТ идеального газа применимо к газам сколь угодно малой плотности.

        \item Основное уравнение МКТ способно описать даже плазму: состояние вещества, при котором молекулы от ударов друг об друга начинают расщепляется на ионы и электроны.
        \item Основное уравнение МКТ ИГ может быть получено теоретически из модели идеального газа, однако в нем присутствуют микропараметры, поэтому оно не допускает непосредственной экспериментальной проверки.

        \item Все процессы: изохорный, изобарный, изотермный по умолчанию предполагают, что количество вещества в них не изменяется.
        \item При горении, например, водорода в кислороде (2H2+O2-2H2O), не изменяется, ни масса вещества участвующего в реакции, ни его количество.
        Также при этом не изменяется и количество протонов, нейтронов и электронов.
        \item Каждый набор макропараметров идеального газа (P, V и T) задаёт точку в трехмерном пространстве.
        При их изменении образуется линия в этом пространстве.
        Строя графики изопроцессов в координатах PV, VT, PT мы строим проекцию этой линии на одну из плоскостей.
    \end{enumerate}
}

\tasknumber{2}%
\task{%
    Выразите:
    \begin{enumerate}
        \item объём тела через его массу и плотность,
        \item количество вещества через число частиц и число Авогадро,
        \item основное уравнение МКТ идеального газа через концентрацию и среднюю кинетическую энергию поступательного движения его молекул.
    \end{enumerate}
}

\tasknumber{3}%
\task{%
    Напротив каждой физической величины укажите её обозначение и единицы измерения в СИ:
    \begin{enumerate}
        \item температура в Цельсиях,
        \item плотность,
        \item число Авогадро.
    \end{enumerate}
}

\tasknumber{4}%
\task{%
    Запишите, как бы вы обозначили...
    \begin{enumerate}
        \item увеличение давления в сосуде с газом,
        \item количество вещества: до и после утечки газа из баллона,
        \item массы газа: до и после утечки газа из баллона,
        \item температуры газа: до и после нагрева в 3 раза.
    \end{enumerate}
}

\tasknumber{5}%
\task{%
    Запишите, какие физические величины соответствуют следующим единицам измерения (указать название и обозначение),
    \begin{enumerate}
        \item кельвин,
        \item мПа,
        \item мкДж,
        \item $\text{ л }$,
        \item $\units{ моль }$.
    \end{enumerate}
}

\tasknumber{6}%
\task{%
    Выразите одну величину через остальные, используя при необходимости постоянную Больцмана, число Авогадро или универсальную газовую постоянную:
    \begin{enumerate}
        \item температуру газа через его давление, объем, число частиц,
        \item плотность газа через концентрацию молекул и массу молекулы,
        \item число молекул через температуру, давление и объём,
        \item средний квадрат скорости молекул через скорости отдельных молекул,
    \end{enumerate}
}

\variantsplitter

\addpersonalvariant{Софья Андрианова}

\tasknumber{1}%
\task{%
    Укажите, верны ли утверждения («да» или «нет» слева от каждого утверждения):
    \begin{enumerate}
        \item Увеличение температуры на 3 градуса цельсия всегда соответствует увеличению на 3 градуса кельвина.
        \item Температуру тела всегда можно понизить на 30 кельвин (пусть при этом и может произойти фазовый переход).
        % \item Температуру тела всегда можно повысить на 30 кельвин (пусть при этом и может произойти фазовый переход).
        \item Температуру тела всегда можно понизить на 30 градусов Цельсия (пусть при этом и может произойти фазовый переход).
        % \item Температуру тела всегда можно повысить на 30 градусов Цельсия (пусть при этом и может произойти фазовый переход).

        \item У шкалы температур Кельвина есть минимальное значение (пусть и недостижимое): 0 кельвин, а у шкалы Цельсия такого значения нет вовсе и возможны температуры меньше 0 градусов цельсия.
        \item Если бы стекло, из которого изготовлен термометр, расширялось при нагревании сильнее жидкости внутри, то мы бы наблюдали, как столбик жидкости укорачивается при нагревании.
        \item Давление газа на окружающий его сосуд вызвано ударами молекул газа о его стенки: при этом изменяется импульс молекул, а значит кто-то (стенка) действовала с некоторой силой, а тогда по 3 закону Ньютона и газ действовал на стенку.
        \item В модели идеального газа невозможен теплообмен: например, если смешать две порции кислорода и азота разной температуры, то их молекулы не будут сталкиваться и обмениваться энергиями.
        Диффузия при этом произойдет.
        \item Основное уравнение МКТ идеального газа применимо к газам сколь угодно малой плотности.

        \item Основное уравнение МКТ способно описать даже плазму: состояние вещества, при котором молекулы от ударов друг об друга начинают расщепляется на ионы и электроны.
        \item Основное уравнение МКТ ИГ может быть получено теоретически из модели идеального газа, однако в нем присутствуют микропараметры, поэтому оно не допускает непосредственной экспериментальной проверки.

        \item Все процессы: изохорный, изобарный, изотермный по умолчанию предполагают, что количество вещества в них не изменяется.
        \item При горении, например, водорода в кислороде (2H2+O2-2H2O), не изменяется, ни масса вещества участвующего в реакции, ни его количество.
        Также при этом не изменяется и количество протонов, нейтронов и электронов.
        \item Каждый набор макропараметров идеального газа (P, V и T) задаёт точку в трехмерном пространстве.
        При их изменении образуется линия в этом пространстве.
        Строя графики изопроцессов в координатах PV, VT, PT мы строим проекцию этой линии на одну из плоскостей.
    \end{enumerate}
}

\tasknumber{2}%
\task{%
    Выразите:
    \begin{enumerate}
        \item плотность тела через его массу и объём,
        \item количество вещества через массу и молярную массу,
        \item основное уравнение МКТ идеального газа через концентрацию и среднюю кинетическую энергию поступательного движения его молекул.
    \end{enumerate}
}

\tasknumber{3}%
\task{%
    Напротив каждой физической величины укажите её обозначение и единицы измерения в СИ:
    \begin{enumerate}
        \item температура в Кельвинах,
        \item число Авогадро,
        \item постоянная Больцмана.
    \end{enumerate}
}

\tasknumber{4}%
\task{%
    Запишите, как бы вы обозначили...
    \begin{enumerate}
        \item два объема газа: до и после его расширения,
        \item число частиц: до и после утечки газа из баллона,
        \item концентрацию молекул газа в сосуде после сжатия и нагрева,
        \item объемы газа: до и после сжатия в 4 раза.
    \end{enumerate}
}

\tasknumber{5}%
\task{%
    Запишите, какие физические величины соответствуют следующим единицам измерения (указать название и обозначение),
    \begin{enumerate}
        \item градус Цельсия,
        \item мПа,
        \item эВ,
        \item $\text{ м }^3$,
        \item $\units{ моль }$.
    \end{enumerate}
}

\tasknumber{6}%
\task{%
    Выразите одну величину через остальные, используя при необходимости постоянную Больцмана, число Авогадро или универсальную газовую постоянную:
    \begin{enumerate}
        \item концентрацию молекул через давление и температуру,
        \item массу молекулы через молярную массу вещества,
        \item число молекул через температуру, давление и объём,
        \item среднеквадратичную скорость молекул через средний квадрат скорости,
    \end{enumerate}
}

\variantsplitter

\addpersonalvariant{Владимир Артемчук}

\tasknumber{1}%
\task{%
    Укажите, верны ли утверждения («да» или «нет» слева от каждого утверждения):
    \begin{enumerate}
        \item Увеличение температуры на 3 градуса цельсия всегда соответствует увеличению на 3 градуса кельвина.
        \item Температуру тела всегда можно понизить на 30 кельвин (пусть при этом и может произойти фазовый переход).
        % \item Температуру тела всегда можно повысить на 30 кельвин (пусть при этом и может произойти фазовый переход).
        \item Температуру тела всегда можно понизить на 30 градусов Цельсия (пусть при этом и может произойти фазовый переход).
        % \item Температуру тела всегда можно повысить на 30 градусов Цельсия (пусть при этом и может произойти фазовый переход).

        \item У шкалы температур Кельвина есть минимальное значение (пусть и недостижимое): 0 кельвин, а у шкалы Цельсия такого значения нет вовсе и возможны температуры меньше 0 градусов цельсия.
        \item Если бы стекло, из которого изготовлен термометр, расширялось при нагревании сильнее жидкости внутри, то мы бы наблюдали, как столбик жидкости укорачивается при нагревании.
        \item Давление газа на окружающий его сосуд вызвано ударами молекул газа о его стенки: при этом изменяется импульс молекул, а значит кто-то (стенка) действовала с некоторой силой, а тогда по 3 закону Ньютона и газ действовал на стенку.
        \item В модели идеального газа невозможен теплообмен: например, если смешать две порции кислорода и азота разной температуры, то их молекулы не будут сталкиваться и обмениваться энергиями.
        Диффузия при этом произойдет.
        \item Основное уравнение МКТ идеального газа применимо к газам сколь угодно малой плотности.

        \item Основное уравнение МКТ способно описать даже плазму: состояние вещества, при котором молекулы от ударов друг об друга начинают расщепляется на ионы и электроны.
        \item Основное уравнение МКТ ИГ может быть получено теоретически из модели идеального газа, однако в нем присутствуют микропараметры, поэтому оно не допускает непосредственной экспериментальной проверки.

        \item Все процессы: изохорный, изобарный, изотермный по умолчанию предполагают, что количество вещества в них не изменяется.
        \item При горении, например, водорода в кислороде (2H2+O2-2H2O), не изменяется, ни масса вещества участвующего в реакции, ни его количество.
        Также при этом не изменяется и количество протонов, нейтронов и электронов.
        \item Каждый набор макропараметров идеального газа (P, V и T) задаёт точку в трехмерном пространстве.
        При их изменении образуется линия в этом пространстве.
        Строя графики изопроцессов в координатах PV, VT, PT мы строим проекцию этой линии на одну из плоскостей.
    \end{enumerate}
}

\tasknumber{2}%
\task{%
    Выразите:
    \begin{enumerate}
        \item объём тела через его массу и плотность,
        \item количество вещества через число частиц и число Авогадро,
        \item основное уравнение МКТ идеального газа через концентрацию и среднюю кинетическую энергию поступательного движения его молекул.
    \end{enumerate}
}

\tasknumber{3}%
\task{%
    Напротив каждой физической величины укажите её обозначение и единицы измерения в СИ:
    \begin{enumerate}
        \item температура в Цельсиях,
        \item плотность,
        \item постоянная Больцмана.
    \end{enumerate}
}

\tasknumber{4}%
\task{%
    Запишите, как бы вы обозначили...
    \begin{enumerate}
        \item два объема газа: до и после его расширения,
        \item число частиц: до и после утечки газа из баллона,
        \item концентрацию молекул газа в сосуде после сжатия и нагрева,
        \item объемы газа: до и после сжатия в 4 раза.
    \end{enumerate}
}

\tasknumber{5}%
\task{%
    Запишите, какие физические величины соответствуют следующим единицам измерения (указать название и обозначение),
    \begin{enumerate}
        \item градус Цельсия,
        \item МПа,
        \item мкДж,
        \item $\frac{ 1 }{ \text{ м }^3 }$,
        \item $\funits{ кг }{ моль }$.
    \end{enumerate}
}

\tasknumber{6}%
\task{%
    Выразите одну величину через остальные, используя при необходимости постоянную Больцмана, число Авогадро или универсальную газовую постоянную:
    \begin{enumerate}
        \item концентрацию молекул через давление и температуру,
        \item массу молекулы через молярную массу вещества,
        \item число молекул через температуру, давление и объём,
        \item среднеквадратичную скорость молекул через средний квадрат скорости,
    \end{enumerate}
}

\variantsplitter

\addpersonalvariant{Софья Белянкина}

\tasknumber{1}%
\task{%
    Укажите, верны ли утверждения («да» или «нет» слева от каждого утверждения):
    \begin{enumerate}
        \item Увеличение температуры на 3 градуса цельсия всегда соответствует увеличению на 3 градуса кельвина.
        \item Температуру тела всегда можно понизить на 30 кельвин (пусть при этом и может произойти фазовый переход).
        % \item Температуру тела всегда можно повысить на 30 кельвин (пусть при этом и может произойти фазовый переход).
        \item Температуру тела всегда можно понизить на 30 градусов Цельсия (пусть при этом и может произойти фазовый переход).
        % \item Температуру тела всегда можно повысить на 30 градусов Цельсия (пусть при этом и может произойти фазовый переход).

        \item У шкалы температур Кельвина есть минимальное значение (пусть и недостижимое): 0 кельвин, а у шкалы Цельсия такого значения нет вовсе и возможны температуры меньше 0 градусов цельсия.
        \item Если бы стекло, из которого изготовлен термометр, расширялось при нагревании сильнее жидкости внутри, то мы бы наблюдали, как столбик жидкости укорачивается при нагревании.
        \item Давление газа на окружающий его сосуд вызвано ударами молекул газа о его стенки: при этом изменяется импульс молекул, а значит кто-то (стенка) действовала с некоторой силой, а тогда по 3 закону Ньютона и газ действовал на стенку.
        \item В модели идеального газа невозможен теплообмен: например, если смешать две порции кислорода и азота разной температуры, то их молекулы не будут сталкиваться и обмениваться энергиями.
        Диффузия при этом произойдет.
        \item Основное уравнение МКТ идеального газа применимо к газам сколь угодно малой плотности.

        \item Основное уравнение МКТ способно описать даже плазму: состояние вещества, при котором молекулы от ударов друг об друга начинают расщепляется на ионы и электроны.
        \item Основное уравнение МКТ ИГ может быть получено теоретически из модели идеального газа, однако в нем присутствуют микропараметры, поэтому оно не допускает непосредственной экспериментальной проверки.

        \item Все процессы: изохорный, изобарный, изотермный по умолчанию предполагают, что количество вещества в них не изменяется.
        \item При горении, например, водорода в кислороде (2H2+O2-2H2O), не изменяется, ни масса вещества участвующего в реакции, ни его количество.
        Также при этом не изменяется и количество протонов, нейтронов и электронов.
        \item Каждый набор макропараметров идеального газа (P, V и T) задаёт точку в трехмерном пространстве.
        При их изменении образуется линия в этом пространстве.
        Строя графики изопроцессов в координатах PV, VT, PT мы строим проекцию этой линии на одну из плоскостей.
    \end{enumerate}
}

\tasknumber{2}%
\task{%
    Выразите:
    \begin{enumerate}
        \item плотность тела через его массу и объём,
        \item количество вещества через число частиц и число Авогадро,
        \item основное уравнение МКТ идеального газа через концентрацию и среднюю кинетическую энергию поступательного движения его молекул.
    \end{enumerate}
}

\tasknumber{3}%
\task{%
    Напротив каждой физической величины укажите её обозначение и единицы измерения в СИ:
    \begin{enumerate}
        \item температура в Кельвинах,
        \item количество вещества,
        \item постоянная Больцмана.
    \end{enumerate}
}

\tasknumber{4}%
\task{%
    Запишите, как бы вы обозначили...
    \begin{enumerate}
        \item два объема газа: до и после его расширения,
        \item количество вещества: до и после утечки газа из баллона,
        \item концентрацию молекул газа в сосуде после сжатия и нагрева,
        \item температуры газа: до и после нагрева в 3 раза.
    \end{enumerate}
}

\tasknumber{5}%
\task{%
    Запишите, какие физические величины соответствуют следующим единицам измерения (указать название и обозначение),
    \begin{enumerate}
        \item градус Цельсия,
        \item МПа,
        \item мкДж,
        \item $\frac{ \text{ кг } }{ \text{ м }^3 }$,
        \item $\funits{ кг }{ моль }$.
    \end{enumerate}
}

\tasknumber{6}%
\task{%
    Выразите одну величину через остальные, используя при необходимости постоянную Больцмана, число Авогадро или универсальную газовую постоянную:
    \begin{enumerate}
        \item концентрацию молекул через давление и температуру,
        \item массу молекулы через молярную массу вещества,
        \item среднюю кинетическую энергию поступательного движения молекул через температуру,
        \item среднеквадратичную скорость поступательного движения молекул через температуру и массу молекулы,
    \end{enumerate}
}

\variantsplitter

\addpersonalvariant{Варвара Егиазарян}

\tasknumber{1}%
\task{%
    Укажите, верны ли утверждения («да» или «нет» слева от каждого утверждения):
    \begin{enumerate}
        \item Увеличение температуры на 3 градуса цельсия всегда соответствует увеличению на 3 градуса кельвина.
        \item Температуру тела всегда можно понизить на 30 кельвин (пусть при этом и может произойти фазовый переход).
        % \item Температуру тела всегда можно повысить на 30 кельвин (пусть при этом и может произойти фазовый переход).
        \item Температуру тела всегда можно понизить на 30 градусов Цельсия (пусть при этом и может произойти фазовый переход).
        % \item Температуру тела всегда можно повысить на 30 градусов Цельсия (пусть при этом и может произойти фазовый переход).

        \item У шкалы температур Кельвина есть минимальное значение (пусть и недостижимое): 0 кельвин, а у шкалы Цельсия такого значения нет вовсе и возможны температуры меньше 0 градусов цельсия.
        \item Если бы стекло, из которого изготовлен термометр, расширялось при нагревании сильнее жидкости внутри, то мы бы наблюдали, как столбик жидкости укорачивается при нагревании.
        \item Давление газа на окружающий его сосуд вызвано ударами молекул газа о его стенки: при этом изменяется импульс молекул, а значит кто-то (стенка) действовала с некоторой силой, а тогда по 3 закону Ньютона и газ действовал на стенку.
        \item В модели идеального газа невозможен теплообмен: например, если смешать две порции кислорода и азота разной температуры, то их молекулы не будут сталкиваться и обмениваться энергиями.
        Диффузия при этом произойдет.
        \item Основное уравнение МКТ идеального газа применимо к газам сколь угодно малой плотности.

        \item Основное уравнение МКТ способно описать даже плазму: состояние вещества, при котором молекулы от ударов друг об друга начинают расщепляется на ионы и электроны.
        \item Основное уравнение МКТ ИГ может быть получено теоретически из модели идеального газа, однако в нем присутствуют микропараметры, поэтому оно не допускает непосредственной экспериментальной проверки.

        \item Все процессы: изохорный, изобарный, изотермный по умолчанию предполагают, что количество вещества в них не изменяется.
        \item При горении, например, водорода в кислороде (2H2+O2-2H2O), не изменяется, ни масса вещества участвующего в реакции, ни его количество.
        Также при этом не изменяется и количество протонов, нейтронов и электронов.
        \item Каждый набор макропараметров идеального газа (P, V и T) задаёт точку в трехмерном пространстве.
        При их изменении образуется линия в этом пространстве.
        Строя графики изопроцессов в координатах PV, VT, PT мы строим проекцию этой линии на одну из плоскостей.
    \end{enumerate}
}

\tasknumber{2}%
\task{%
    Выразите:
    \begin{enumerate}
        \item плотность тела через его массу и объём,
        \item количество вещества через массу и молярную массу,
        \item основное уравнение МКТ идеального газа через концентрацию и среднюю кинетическую энергию поступательного движения его молекул.
    \end{enumerate}
}

\tasknumber{3}%
\task{%
    Напротив каждой физической величины укажите её обозначение и единицы измерения в СИ:
    \begin{enumerate}
        \item температура в Цельсиях,
        \item число частиц,
        \item постоянная Больцмана.
    \end{enumerate}
}

\tasknumber{4}%
\task{%
    Запишите, как бы вы обозначили...
    \begin{enumerate}
        \item два объема газа: до и после его расширения,
        \item число частиц: до и после утечки газа из баллона,
        \item массы газа: до и после утечки газа из баллона,
        \item температуры газа: до и после нагрева в 3 раза.
    \end{enumerate}
}

\tasknumber{5}%
\task{%
    Запишите, какие физические величины соответствуют следующим единицам измерения (указать название и обозначение),
    \begin{enumerate}
        \item кельвин,
        \item МПа,
        \item эВ,
        \item $\frac{ \text{ г } }{ \text{ л } }$,
        \item $\units{ моль }$.
    \end{enumerate}
}

\tasknumber{6}%
\task{%
    Выразите одну величину через остальные, используя при необходимости постоянную Больцмана, число Авогадро или универсальную газовую постоянную:
    \begin{enumerate}
        \item концентрацию молекул через давление и температуру,
        \item массу молекулы через молярную массу вещества,
        \item число молекул через температуру, давление и объём,
        \item среднеквадратичную скорость поступательного движения молекул через температуру и массу молекулы,
    \end{enumerate}
}

\variantsplitter

\addpersonalvariant{Владислав Емелин}

\tasknumber{1}%
\task{%
    Укажите, верны ли утверждения («да» или «нет» слева от каждого утверждения):
    \begin{enumerate}
        \item Увеличение температуры на 3 градуса цельсия всегда соответствует увеличению на 3 градуса кельвина.
        \item Температуру тела всегда можно понизить на 30 кельвин (пусть при этом и может произойти фазовый переход).
        % \item Температуру тела всегда можно повысить на 30 кельвин (пусть при этом и может произойти фазовый переход).
        \item Температуру тела всегда можно понизить на 30 градусов Цельсия (пусть при этом и может произойти фазовый переход).
        % \item Температуру тела всегда можно повысить на 30 градусов Цельсия (пусть при этом и может произойти фазовый переход).

        \item У шкалы температур Кельвина есть минимальное значение (пусть и недостижимое): 0 кельвин, а у шкалы Цельсия такого значения нет вовсе и возможны температуры меньше 0 градусов цельсия.
        \item Если бы стекло, из которого изготовлен термометр, расширялось при нагревании сильнее жидкости внутри, то мы бы наблюдали, как столбик жидкости укорачивается при нагревании.
        \item Давление газа на окружающий его сосуд вызвано ударами молекул газа о его стенки: при этом изменяется импульс молекул, а значит кто-то (стенка) действовала с некоторой силой, а тогда по 3 закону Ньютона и газ действовал на стенку.
        \item В модели идеального газа невозможен теплообмен: например, если смешать две порции кислорода и азота разной температуры, то их молекулы не будут сталкиваться и обмениваться энергиями.
        Диффузия при этом произойдет.
        \item Основное уравнение МКТ идеального газа применимо к газам сколь угодно малой плотности.

        \item Основное уравнение МКТ способно описать даже плазму: состояние вещества, при котором молекулы от ударов друг об друга начинают расщепляется на ионы и электроны.
        \item Основное уравнение МКТ ИГ может быть получено теоретически из модели идеального газа, однако в нем присутствуют микропараметры, поэтому оно не допускает непосредственной экспериментальной проверки.

        \item Все процессы: изохорный, изобарный, изотермный по умолчанию предполагают, что количество вещества в них не изменяется.
        \item При горении, например, водорода в кислороде (2H2+O2-2H2O), не изменяется, ни масса вещества участвующего в реакции, ни его количество.
        Также при этом не изменяется и количество протонов, нейтронов и электронов.
        \item Каждый набор макропараметров идеального газа (P, V и T) задаёт точку в трехмерном пространстве.
        При их изменении образуется линия в этом пространстве.
        Строя графики изопроцессов в координатах PV, VT, PT мы строим проекцию этой линии на одну из плоскостей.
    \end{enumerate}
}

\tasknumber{2}%
\task{%
    Выразите:
    \begin{enumerate}
        \item плотность тела через его массу и объём,
        \item количество вещества через массу и молярную массу,
        \item концентрацию молекул через их число и объём.
    \end{enumerate}
}

\tasknumber{3}%
\task{%
    Напротив каждой физической величины укажите её обозначение и единицы измерения в СИ:
    \begin{enumerate}
        \item температура в Цельсиях,
        \item молярная масса,
        \item число Авогадро.
    \end{enumerate}
}

\tasknumber{4}%
\task{%
    Запишите, как бы вы обозначили...
    \begin{enumerate}
        \item увеличение давления в сосуде с газом,
        \item число частиц: до и после утечки газа из баллона,
        \item концентрацию молекул газа в сосуде после сжатия и нагрева,
        \item объемы газа: до и после сжатия в 4 раза.
    \end{enumerate}
}

\tasknumber{5}%
\task{%
    Запишите, какие физические величины соответствуют следующим единицам измерения (указать название и обозначение),
    \begin{enumerate}
        \item градус Цельсия,
        \item мПа,
        \item мкДж,
        \item $\text{ л }$,
        \item $\funits{ кг }{ моль }$.
    \end{enumerate}
}

\tasknumber{6}%
\task{%
    Выразите одну величину через остальные, используя при необходимости постоянную Больцмана, число Авогадро или универсальную газовую постоянную:
    \begin{enumerate}
        \item температуру газа через его давление, объем, число частиц,
        \item массу молекулы через молярную массу вещества,
        \item число молекул через температуру, давление и объём,
        \item средний квадрат проекции скорости молекул на ось $Ox$ через средний квадрат скорости молекул,
    \end{enumerate}
}

\variantsplitter

\addpersonalvariant{Артём Жичин}

\tasknumber{1}%
\task{%
    Укажите, верны ли утверждения («да» или «нет» слева от каждого утверждения):
    \begin{enumerate}
        \item Увеличение температуры на 3 градуса цельсия всегда соответствует увеличению на 3 градуса кельвина.
        \item Температуру тела всегда можно понизить на 30 кельвин (пусть при этом и может произойти фазовый переход).
        % \item Температуру тела всегда можно повысить на 30 кельвин (пусть при этом и может произойти фазовый переход).
        \item Температуру тела всегда можно понизить на 30 градусов Цельсия (пусть при этом и может произойти фазовый переход).
        % \item Температуру тела всегда можно повысить на 30 градусов Цельсия (пусть при этом и может произойти фазовый переход).

        \item У шкалы температур Кельвина есть минимальное значение (пусть и недостижимое): 0 кельвин, а у шкалы Цельсия такого значения нет вовсе и возможны температуры меньше 0 градусов цельсия.
        \item Если бы стекло, из которого изготовлен термометр, расширялось при нагревании сильнее жидкости внутри, то мы бы наблюдали, как столбик жидкости укорачивается при нагревании.
        \item Давление газа на окружающий его сосуд вызвано ударами молекул газа о его стенки: при этом изменяется импульс молекул, а значит кто-то (стенка) действовала с некоторой силой, а тогда по 3 закону Ньютона и газ действовал на стенку.
        \item В модели идеального газа невозможен теплообмен: например, если смешать две порции кислорода и азота разной температуры, то их молекулы не будут сталкиваться и обмениваться энергиями.
        Диффузия при этом произойдет.
        \item Основное уравнение МКТ идеального газа применимо к газам сколь угодно малой плотности.

        \item Основное уравнение МКТ способно описать даже плазму: состояние вещества, при котором молекулы от ударов друг об друга начинают расщепляется на ионы и электроны.
        \item Основное уравнение МКТ ИГ может быть получено теоретически из модели идеального газа, однако в нем присутствуют микропараметры, поэтому оно не допускает непосредственной экспериментальной проверки.

        \item Все процессы: изохорный, изобарный, изотермный по умолчанию предполагают, что количество вещества в них не изменяется.
        \item При горении, например, водорода в кислороде (2H2+O2-2H2O), не изменяется, ни масса вещества участвующего в реакции, ни его количество.
        Также при этом не изменяется и количество протонов, нейтронов и электронов.
        \item Каждый набор макропараметров идеального газа (P, V и T) задаёт точку в трехмерном пространстве.
        При их изменении образуется линия в этом пространстве.
        Строя графики изопроцессов в координатах PV, VT, PT мы строим проекцию этой линии на одну из плоскостей.
    \end{enumerate}
}

\tasknumber{2}%
\task{%
    Выразите:
    \begin{enumerate}
        \item объём тела через его массу и плотность,
        \item количество вещества через массу и молярную массу,
        \item основное уравнение МКТ идеального газа через концентрацию и среднюю кинетическую энергию поступательного движения его молекул.
    \end{enumerate}
}

\tasknumber{3}%
\task{%
    Напротив каждой физической величины укажите её обозначение и единицы измерения в СИ:
    \begin{enumerate}
        \item температура в Цельсиях,
        \item молярная масса,
        \item постоянная Больцмана.
    \end{enumerate}
}

\tasknumber{4}%
\task{%
    Запишите, как бы вы обозначили...
    \begin{enumerate}
        \item увеличение давления в сосуде с газом,
        \item число частиц: до и после утечки газа из баллона,
        \item массы газа: до и после утечки газа из баллона,
        \item объемы газа: до и после сжатия в 4 раза.
    \end{enumerate}
}

\tasknumber{5}%
\task{%
    Запишите, какие физические величины соответствуют следующим единицам измерения (указать название и обозначение),
    \begin{enumerate}
        \item градус Цельсия,
        \item МПа,
        \item мДж,
        \item $\frac{ 1 }{ \text{ м }^3 }$,
        \item $\funits{ г }{ моль }$.
    \end{enumerate}
}

\tasknumber{6}%
\task{%
    Выразите одну величину через остальные, используя при необходимости постоянную Больцмана, число Авогадро или универсальную газовую постоянную:
    \begin{enumerate}
        \item температуру газа через его давление, объем, число частиц,
        \item плотность газа через концентрацию молекул и массу молекулы,
        \item среднюю кинетическую энергию поступательного движения молекул через температуру,
        \item среднеквадратичную скорость поступательного движения молекул через температуру и массу молекулы,
    \end{enumerate}
}

\variantsplitter

\addpersonalvariant{Дарья Кошман}

\tasknumber{1}%
\task{%
    Укажите, верны ли утверждения («да» или «нет» слева от каждого утверждения):
    \begin{enumerate}
        \item Увеличение температуры на 3 градуса цельсия всегда соответствует увеличению на 3 градуса кельвина.
        \item Температуру тела всегда можно понизить на 30 кельвин (пусть при этом и может произойти фазовый переход).
        % \item Температуру тела всегда можно повысить на 30 кельвин (пусть при этом и может произойти фазовый переход).
        \item Температуру тела всегда можно понизить на 30 градусов Цельсия (пусть при этом и может произойти фазовый переход).
        % \item Температуру тела всегда можно повысить на 30 градусов Цельсия (пусть при этом и может произойти фазовый переход).

        \item У шкалы температур Кельвина есть минимальное значение (пусть и недостижимое): 0 кельвин, а у шкалы Цельсия такого значения нет вовсе и возможны температуры меньше 0 градусов цельсия.
        \item Если бы стекло, из которого изготовлен термометр, расширялось при нагревании сильнее жидкости внутри, то мы бы наблюдали, как столбик жидкости укорачивается при нагревании.
        \item Давление газа на окружающий его сосуд вызвано ударами молекул газа о его стенки: при этом изменяется импульс молекул, а значит кто-то (стенка) действовала с некоторой силой, а тогда по 3 закону Ньютона и газ действовал на стенку.
        \item В модели идеального газа невозможен теплообмен: например, если смешать две порции кислорода и азота разной температуры, то их молекулы не будут сталкиваться и обмениваться энергиями.
        Диффузия при этом произойдет.
        \item Основное уравнение МКТ идеального газа применимо к газам сколь угодно малой плотности.

        \item Основное уравнение МКТ способно описать даже плазму: состояние вещества, при котором молекулы от ударов друг об друга начинают расщепляется на ионы и электроны.
        \item Основное уравнение МКТ ИГ может быть получено теоретически из модели идеального газа, однако в нем присутствуют микропараметры, поэтому оно не допускает непосредственной экспериментальной проверки.

        \item Все процессы: изохорный, изобарный, изотермный по умолчанию предполагают, что количество вещества в них не изменяется.
        \item При горении, например, водорода в кислороде (2H2+O2-2H2O), не изменяется, ни масса вещества участвующего в реакции, ни его количество.
        Также при этом не изменяется и количество протонов, нейтронов и электронов.
        \item Каждый набор макропараметров идеального газа (P, V и T) задаёт точку в трехмерном пространстве.
        При их изменении образуется линия в этом пространстве.
        Строя графики изопроцессов в координатах PV, VT, PT мы строим проекцию этой линии на одну из плоскостей.
    \end{enumerate}
}

\tasknumber{2}%
\task{%
    Выразите:
    \begin{enumerate}
        \item массу тела через его плотность и объём,
        \item количество вещества через число частиц и число Авогадро,
        \item основное уравнение МКТ идеального газа через концентрацию и среднюю кинетическую энергию поступательного движения его молекул.
    \end{enumerate}
}

\tasknumber{3}%
\task{%
    Напротив каждой физической величины укажите её обозначение и единицы измерения в СИ:
    \begin{enumerate}
        \item температура в Кельвинах,
        \item число Авогадро,
        \item число Авогадро.
    \end{enumerate}
}

\tasknumber{4}%
\task{%
    Запишите, как бы вы обозначили...
    \begin{enumerate}
        \item увеличение давления в сосуде с газом,
        \item количество вещества: до и после утечки газа из баллона,
        \item концентрацию молекул газа в сосуде после сжатия и нагрева,
        \item давления газа: до и после его увеличения в 5 раз.
    \end{enumerate}
}

\tasknumber{5}%
\task{%
    Запишите, какие физические величины соответствуют следующим единицам измерения (указать название и обозначение),
    \begin{enumerate}
        \item кельвин,
        \item МПа,
        \item мДж,
        \item $\text{ м }^3$,
        \item $\units{ моль }$.
    \end{enumerate}
}

\tasknumber{6}%
\task{%
    Выразите одну величину через остальные, используя при необходимости постоянную Больцмана, число Авогадро или универсальную газовую постоянную:
    \begin{enumerate}
        \item температуру газа через его давление, объем, число частиц,
        \item плотность газа через его молярную массу и концентрацию молекул,
        \item среднюю кинетическую энергию поступательного движения молекул через температуру,
        \item средний квадрат проекции скорости молекул на ось $Ox$ через средний квадрат скорости молекул,
    \end{enumerate}
}

\variantsplitter

\addpersonalvariant{Анна Кузьмичёва}

\tasknumber{1}%
\task{%
    Укажите, верны ли утверждения («да» или «нет» слева от каждого утверждения):
    \begin{enumerate}
        \item Увеличение температуры на 3 градуса цельсия всегда соответствует увеличению на 3 градуса кельвина.
        \item Температуру тела всегда можно понизить на 30 кельвин (пусть при этом и может произойти фазовый переход).
        % \item Температуру тела всегда можно повысить на 30 кельвин (пусть при этом и может произойти фазовый переход).
        \item Температуру тела всегда можно понизить на 30 градусов Цельсия (пусть при этом и может произойти фазовый переход).
        % \item Температуру тела всегда можно повысить на 30 градусов Цельсия (пусть при этом и может произойти фазовый переход).

        \item У шкалы температур Кельвина есть минимальное значение (пусть и недостижимое): 0 кельвин, а у шкалы Цельсия такого значения нет вовсе и возможны температуры меньше 0 градусов цельсия.
        \item Если бы стекло, из которого изготовлен термометр, расширялось при нагревании сильнее жидкости внутри, то мы бы наблюдали, как столбик жидкости укорачивается при нагревании.
        \item Давление газа на окружающий его сосуд вызвано ударами молекул газа о его стенки: при этом изменяется импульс молекул, а значит кто-то (стенка) действовала с некоторой силой, а тогда по 3 закону Ньютона и газ действовал на стенку.
        \item В модели идеального газа невозможен теплообмен: например, если смешать две порции кислорода и азота разной температуры, то их молекулы не будут сталкиваться и обмениваться энергиями.
        Диффузия при этом произойдет.
        \item Основное уравнение МКТ идеального газа применимо к газам сколь угодно малой плотности.

        \item Основное уравнение МКТ способно описать даже плазму: состояние вещества, при котором молекулы от ударов друг об друга начинают расщепляется на ионы и электроны.
        \item Основное уравнение МКТ ИГ может быть получено теоретически из модели идеального газа, однако в нем присутствуют микропараметры, поэтому оно не допускает непосредственной экспериментальной проверки.

        \item Все процессы: изохорный, изобарный, изотермный по умолчанию предполагают, что количество вещества в них не изменяется.
        \item При горении, например, водорода в кислороде (2H2+O2-2H2O), не изменяется, ни масса вещества участвующего в реакции, ни его количество.
        Также при этом не изменяется и количество протонов, нейтронов и электронов.
        \item Каждый набор макропараметров идеального газа (P, V и T) задаёт точку в трехмерном пространстве.
        При их изменении образуется линия в этом пространстве.
        Строя графики изопроцессов в координатах PV, VT, PT мы строим проекцию этой линии на одну из плоскостей.
    \end{enumerate}
}

\tasknumber{2}%
\task{%
    Выразите:
    \begin{enumerate}
        \item плотность тела через его массу и объём,
        \item количество вещества через число частиц и число Авогадро,
        \item концентрацию молекул через их число и объём.
    \end{enumerate}
}

\tasknumber{3}%
\task{%
    Напротив каждой физической величины укажите её обозначение и единицы измерения в СИ:
    \begin{enumerate}
        \item температура в Цельсиях,
        \item число частиц,
        \item число Авогадро.
    \end{enumerate}
}

\tasknumber{4}%
\task{%
    Запишите, как бы вы обозначили...
    \begin{enumerate}
        \item два объема газа: до и после его расширения,
        \item число частиц: до и после утечки газа из баллона,
        \item массы газа: до и после утечки газа из баллона,
        \item давления газа: до и после его увеличения в 5 раз.
    \end{enumerate}
}

\tasknumber{5}%
\task{%
    Запишите, какие физические величины соответствуют следующим единицам измерения (указать название и обозначение),
    \begin{enumerate}
        \item кельвин,
        \item МПа,
        \item мкДж,
        \item $\frac{ 1 }{ \text{ л } }$,
        \item $\funits{ г }{ моль }$.
    \end{enumerate}
}

\tasknumber{6}%
\task{%
    Выразите одну величину через остальные, используя при необходимости постоянную Больцмана, число Авогадро или универсальную газовую постоянную:
    \begin{enumerate}
        \item концентрацию молекул через давление и температуру,
        \item плотность газа через его молярную массу и концентрацию молекул,
        \item число молекул через температуру, давление и объём,
        \item средний квадрат проекции скорости молекул на ось $Ox$ через средний квадрат скорости молекул,
    \end{enumerate}
}

\variantsplitter

\addpersonalvariant{Алёна Куприянова}

\tasknumber{1}%
\task{%
    Укажите, верны ли утверждения («да» или «нет» слева от каждого утверждения):
    \begin{enumerate}
        \item Увеличение температуры на 3 градуса цельсия всегда соответствует увеличению на 3 градуса кельвина.
        \item Температуру тела всегда можно понизить на 30 кельвин (пусть при этом и может произойти фазовый переход).
        % \item Температуру тела всегда можно повысить на 30 кельвин (пусть при этом и может произойти фазовый переход).
        \item Температуру тела всегда можно понизить на 30 градусов Цельсия (пусть при этом и может произойти фазовый переход).
        % \item Температуру тела всегда можно повысить на 30 градусов Цельсия (пусть при этом и может произойти фазовый переход).

        \item У шкалы температур Кельвина есть минимальное значение (пусть и недостижимое): 0 кельвин, а у шкалы Цельсия такого значения нет вовсе и возможны температуры меньше 0 градусов цельсия.
        \item Если бы стекло, из которого изготовлен термометр, расширялось при нагревании сильнее жидкости внутри, то мы бы наблюдали, как столбик жидкости укорачивается при нагревании.
        \item Давление газа на окружающий его сосуд вызвано ударами молекул газа о его стенки: при этом изменяется импульс молекул, а значит кто-то (стенка) действовала с некоторой силой, а тогда по 3 закону Ньютона и газ действовал на стенку.
        \item В модели идеального газа невозможен теплообмен: например, если смешать две порции кислорода и азота разной температуры, то их молекулы не будут сталкиваться и обмениваться энергиями.
        Диффузия при этом произойдет.
        \item Основное уравнение МКТ идеального газа применимо к газам сколь угодно малой плотности.

        \item Основное уравнение МКТ способно описать даже плазму: состояние вещества, при котором молекулы от ударов друг об друга начинают расщепляется на ионы и электроны.
        \item Основное уравнение МКТ ИГ может быть получено теоретически из модели идеального газа, однако в нем присутствуют микропараметры, поэтому оно не допускает непосредственной экспериментальной проверки.

        \item Все процессы: изохорный, изобарный, изотермный по умолчанию предполагают, что количество вещества в них не изменяется.
        \item При горении, например, водорода в кислороде (2H2+O2-2H2O), не изменяется, ни масса вещества участвующего в реакции, ни его количество.
        Также при этом не изменяется и количество протонов, нейтронов и электронов.
        \item Каждый набор макропараметров идеального газа (P, V и T) задаёт точку в трехмерном пространстве.
        При их изменении образуется линия в этом пространстве.
        Строя графики изопроцессов в координатах PV, VT, PT мы строим проекцию этой линии на одну из плоскостей.
    \end{enumerate}
}

\tasknumber{2}%
\task{%
    Выразите:
    \begin{enumerate}
        \item плотность тела через его массу и объём,
        \item количество вещества через массу и молярную массу,
        \item основное уравнение МКТ идеального газа через концентрацию и среднюю кинетическую энергию поступательного движения его молекул.
    \end{enumerate}
}

\tasknumber{3}%
\task{%
    Напротив каждой физической величины укажите её обозначение и единицы измерения в СИ:
    \begin{enumerate}
        \item температура в Цельсиях,
        \item количество вещества,
        \item число Авогадро.
    \end{enumerate}
}

\tasknumber{4}%
\task{%
    Запишите, как бы вы обозначили...
    \begin{enumerate}
        \item два объема газа: до и после его расширения,
        \item число частиц: до и после утечки газа из баллона,
        \item концентрацию молекул газа в сосуде после сжатия и нагрева,
        \item давления газа: до и после его увеличения в 5 раз.
    \end{enumerate}
}

\tasknumber{5}%
\task{%
    Запишите, какие физические величины соответствуют следующим единицам измерения (указать название и обозначение),
    \begin{enumerate}
        \item градус Цельсия,
        \item мПа,
        \item мкДж,
        \item $\frac{ \text{ кг } }{ \text{ м }^3 }$,
        \item $\units{ моль }$.
    \end{enumerate}
}

\tasknumber{6}%
\task{%
    Выразите одну величину через остальные, используя при необходимости постоянную Больцмана, число Авогадро или универсальную газовую постоянную:
    \begin{enumerate}
        \item концентрацию молекул через давление и температуру,
        \item плотность газа через концентрацию молекул и массу молекулы,
        \item среднюю кинетическую энергию поступательного движения молекул через температуру,
        \item среднеквадратичную скорость молекул через средний квадрат скорости,
    \end{enumerate}
}

\variantsplitter

\addpersonalvariant{Ярослав Лавровский}

\tasknumber{1}%
\task{%
    Укажите, верны ли утверждения («да» или «нет» слева от каждого утверждения):
    \begin{enumerate}
        \item Увеличение температуры на 3 градуса цельсия всегда соответствует увеличению на 3 градуса кельвина.
        \item Температуру тела всегда можно понизить на 30 кельвин (пусть при этом и может произойти фазовый переход).
        % \item Температуру тела всегда можно повысить на 30 кельвин (пусть при этом и может произойти фазовый переход).
        \item Температуру тела всегда можно понизить на 30 градусов Цельсия (пусть при этом и может произойти фазовый переход).
        % \item Температуру тела всегда можно повысить на 30 градусов Цельсия (пусть при этом и может произойти фазовый переход).

        \item У шкалы температур Кельвина есть минимальное значение (пусть и недостижимое): 0 кельвин, а у шкалы Цельсия такого значения нет вовсе и возможны температуры меньше 0 градусов цельсия.
        \item Если бы стекло, из которого изготовлен термометр, расширялось при нагревании сильнее жидкости внутри, то мы бы наблюдали, как столбик жидкости укорачивается при нагревании.
        \item Давление газа на окружающий его сосуд вызвано ударами молекул газа о его стенки: при этом изменяется импульс молекул, а значит кто-то (стенка) действовала с некоторой силой, а тогда по 3 закону Ньютона и газ действовал на стенку.
        \item В модели идеального газа невозможен теплообмен: например, если смешать две порции кислорода и азота разной температуры, то их молекулы не будут сталкиваться и обмениваться энергиями.
        Диффузия при этом произойдет.
        \item Основное уравнение МКТ идеального газа применимо к газам сколь угодно малой плотности.

        \item Основное уравнение МКТ способно описать даже плазму: состояние вещества, при котором молекулы от ударов друг об друга начинают расщепляется на ионы и электроны.
        \item Основное уравнение МКТ ИГ может быть получено теоретически из модели идеального газа, однако в нем присутствуют микропараметры, поэтому оно не допускает непосредственной экспериментальной проверки.

        \item Все процессы: изохорный, изобарный, изотермный по умолчанию предполагают, что количество вещества в них не изменяется.
        \item При горении, например, водорода в кислороде (2H2+O2-2H2O), не изменяется, ни масса вещества участвующего в реакции, ни его количество.
        Также при этом не изменяется и количество протонов, нейтронов и электронов.
        \item Каждый набор макропараметров идеального газа (P, V и T) задаёт точку в трехмерном пространстве.
        При их изменении образуется линия в этом пространстве.
        Строя графики изопроцессов в координатах PV, VT, PT мы строим проекцию этой линии на одну из плоскостей.
    \end{enumerate}
}

\tasknumber{2}%
\task{%
    Выразите:
    \begin{enumerate}
        \item объём тела через его массу и плотность,
        \item количество вещества через массу и молярную массу,
        \item концентрацию молекул через их число и объём.
    \end{enumerate}
}

\tasknumber{3}%
\task{%
    Напротив каждой физической величины укажите её обозначение и единицы измерения в СИ:
    \begin{enumerate}
        \item температура в Цельсиях,
        \item число частиц,
        \item постоянная Больцмана.
    \end{enumerate}
}

\tasknumber{4}%
\task{%
    Запишите, как бы вы обозначили...
    \begin{enumerate}
        \item увеличение давления в сосуде с газом,
        \item число частиц: до и после утечки газа из баллона,
        \item концентрацию молекул газа в сосуде после сжатия и нагрева,
        \item объемы газа: до и после сжатия в 4 раза.
    \end{enumerate}
}

\tasknumber{5}%
\task{%
    Запишите, какие физические величины соответствуют следующим единицам измерения (указать название и обозначение),
    \begin{enumerate}
        \item градус Цельсия,
        \item МПа,
        \item эВ,
        \item $\text{ м }^3$,
        \item $\funits{ г }{ моль }$.
    \end{enumerate}
}

\tasknumber{6}%
\task{%
    Выразите одну величину через остальные, используя при необходимости постоянную Больцмана, число Авогадро или универсальную газовую постоянную:
    \begin{enumerate}
        \item температуру газа через его давление, объем, число частиц,
        \item массу молекулы через молярную массу вещества,
        \item число молекул через температуру, давление и объём,
        \item средний квадрат проекции скорости молекул на ось $Ox$ через средний квадрат скорости молекул,
    \end{enumerate}
}

\variantsplitter

\addpersonalvariant{Анастасия Ламанова}

\tasknumber{1}%
\task{%
    Укажите, верны ли утверждения («да» или «нет» слева от каждого утверждения):
    \begin{enumerate}
        \item Увеличение температуры на 3 градуса цельсия всегда соответствует увеличению на 3 градуса кельвина.
        \item Температуру тела всегда можно понизить на 30 кельвин (пусть при этом и может произойти фазовый переход).
        % \item Температуру тела всегда можно повысить на 30 кельвин (пусть при этом и может произойти фазовый переход).
        \item Температуру тела всегда можно понизить на 30 градусов Цельсия (пусть при этом и может произойти фазовый переход).
        % \item Температуру тела всегда можно повысить на 30 градусов Цельсия (пусть при этом и может произойти фазовый переход).

        \item У шкалы температур Кельвина есть минимальное значение (пусть и недостижимое): 0 кельвин, а у шкалы Цельсия такого значения нет вовсе и возможны температуры меньше 0 градусов цельсия.
        \item Если бы стекло, из которого изготовлен термометр, расширялось при нагревании сильнее жидкости внутри, то мы бы наблюдали, как столбик жидкости укорачивается при нагревании.
        \item Давление газа на окружающий его сосуд вызвано ударами молекул газа о его стенки: при этом изменяется импульс молекул, а значит кто-то (стенка) действовала с некоторой силой, а тогда по 3 закону Ньютона и газ действовал на стенку.
        \item В модели идеального газа невозможен теплообмен: например, если смешать две порции кислорода и азота разной температуры, то их молекулы не будут сталкиваться и обмениваться энергиями.
        Диффузия при этом произойдет.
        \item Основное уравнение МКТ идеального газа применимо к газам сколь угодно малой плотности.

        \item Основное уравнение МКТ способно описать даже плазму: состояние вещества, при котором молекулы от ударов друг об друга начинают расщепляется на ионы и электроны.
        \item Основное уравнение МКТ ИГ может быть получено теоретически из модели идеального газа, однако в нем присутствуют микропараметры, поэтому оно не допускает непосредственной экспериментальной проверки.

        \item Все процессы: изохорный, изобарный, изотермный по умолчанию предполагают, что количество вещества в них не изменяется.
        \item При горении, например, водорода в кислороде (2H2+O2-2H2O), не изменяется, ни масса вещества участвующего в реакции, ни его количество.
        Также при этом не изменяется и количество протонов, нейтронов и электронов.
        \item Каждый набор макропараметров идеального газа (P, V и T) задаёт точку в трехмерном пространстве.
        При их изменении образуется линия в этом пространстве.
        Строя графики изопроцессов в координатах PV, VT, PT мы строим проекцию этой линии на одну из плоскостей.
    \end{enumerate}
}

\tasknumber{2}%
\task{%
    Выразите:
    \begin{enumerate}
        \item объём тела через его массу и плотность,
        \item количество вещества через массу и молярную массу,
        \item концентрацию молекул через их число и объём.
    \end{enumerate}
}

\tasknumber{3}%
\task{%
    Напротив каждой физической величины укажите её обозначение и единицы измерения в СИ:
    \begin{enumerate}
        \item температура в Кельвинах,
        \item количество вещества,
        \item число Авогадро.
    \end{enumerate}
}

\tasknumber{4}%
\task{%
    Запишите, как бы вы обозначили...
    \begin{enumerate}
        \item увеличение давления в сосуде с газом,
        \item количество вещества: до и после утечки газа из баллона,
        \item концентрацию молекул газа в сосуде после сжатия и нагрева,
        \item температуры газа: до и после нагрева в 3 раза.
    \end{enumerate}
}

\tasknumber{5}%
\task{%
    Запишите, какие физические величины соответствуют следующим единицам измерения (указать название и обозначение),
    \begin{enumerate}
        \item кельвин,
        \item МПа,
        \item эВ,
        \item $\text{ м }^3$,
        \item $\units{ г }$.
    \end{enumerate}
}

\tasknumber{6}%
\task{%
    Выразите одну величину через остальные, используя при необходимости постоянную Больцмана, число Авогадро или универсальную газовую постоянную:
    \begin{enumerate}
        \item температуру газа через его давление, объем, число частиц,
        \item массу молекулы через молярную массу вещества,
        \item среднюю кинетическую энергию поступательного движения молекул через температуру,
        \item средний квадрат скорости молекул через скорости отдельных молекул,
    \end{enumerate}
}

\variantsplitter

\addpersonalvariant{Виктория Легонькова}

\tasknumber{1}%
\task{%
    Укажите, верны ли утверждения («да» или «нет» слева от каждого утверждения):
    \begin{enumerate}
        \item Увеличение температуры на 3 градуса цельсия всегда соответствует увеличению на 3 градуса кельвина.
        \item Температуру тела всегда можно понизить на 30 кельвин (пусть при этом и может произойти фазовый переход).
        % \item Температуру тела всегда можно повысить на 30 кельвин (пусть при этом и может произойти фазовый переход).
        \item Температуру тела всегда можно понизить на 30 градусов Цельсия (пусть при этом и может произойти фазовый переход).
        % \item Температуру тела всегда можно повысить на 30 градусов Цельсия (пусть при этом и может произойти фазовый переход).

        \item У шкалы температур Кельвина есть минимальное значение (пусть и недостижимое): 0 кельвин, а у шкалы Цельсия такого значения нет вовсе и возможны температуры меньше 0 градусов цельсия.
        \item Если бы стекло, из которого изготовлен термометр, расширялось при нагревании сильнее жидкости внутри, то мы бы наблюдали, как столбик жидкости укорачивается при нагревании.
        \item Давление газа на окружающий его сосуд вызвано ударами молекул газа о его стенки: при этом изменяется импульс молекул, а значит кто-то (стенка) действовала с некоторой силой, а тогда по 3 закону Ньютона и газ действовал на стенку.
        \item В модели идеального газа невозможен теплообмен: например, если смешать две порции кислорода и азота разной температуры, то их молекулы не будут сталкиваться и обмениваться энергиями.
        Диффузия при этом произойдет.
        \item Основное уравнение МКТ идеального газа применимо к газам сколь угодно малой плотности.

        \item Основное уравнение МКТ способно описать даже плазму: состояние вещества, при котором молекулы от ударов друг об друга начинают расщепляется на ионы и электроны.
        \item Основное уравнение МКТ ИГ может быть получено теоретически из модели идеального газа, однако в нем присутствуют микропараметры, поэтому оно не допускает непосредственной экспериментальной проверки.

        \item Все процессы: изохорный, изобарный, изотермный по умолчанию предполагают, что количество вещества в них не изменяется.
        \item При горении, например, водорода в кислороде (2H2+O2-2H2O), не изменяется, ни масса вещества участвующего в реакции, ни его количество.
        Также при этом не изменяется и количество протонов, нейтронов и электронов.
        \item Каждый набор макропараметров идеального газа (P, V и T) задаёт точку в трехмерном пространстве.
        При их изменении образуется линия в этом пространстве.
        Строя графики изопроцессов в координатах PV, VT, PT мы строим проекцию этой линии на одну из плоскостей.
    \end{enumerate}
}

\tasknumber{2}%
\task{%
    Выразите:
    \begin{enumerate}
        \item плотность тела через его массу и объём,
        \item количество вещества через число частиц и число Авогадро,
        \item основное уравнение МКТ идеального газа через концентрацию и среднюю кинетическую энергию поступательного движения его молекул.
    \end{enumerate}
}

\tasknumber{3}%
\task{%
    Напротив каждой физической величины укажите её обозначение и единицы измерения в СИ:
    \begin{enumerate}
        \item температура в Цельсиях,
        \item число частиц,
        \item число Авогадро.
    \end{enumerate}
}

\tasknumber{4}%
\task{%
    Запишите, как бы вы обозначили...
    \begin{enumerate}
        \item два объема газа: до и после его расширения,
        \item количество вещества: до и после утечки газа из баллона,
        \item концентрацию молекул газа в сосуде после сжатия и нагрева,
        \item давления газа: до и после его увеличения в 5 раз.
    \end{enumerate}
}

\tasknumber{5}%
\task{%
    Запишите, какие физические величины соответствуют следующим единицам измерения (указать название и обозначение),
    \begin{enumerate}
        \item градус Цельсия,
        \item МПа,
        \item мДж,
        \item $\text{ м }^3$,
        \item $\funits{ кг }{ моль }$.
    \end{enumerate}
}

\tasknumber{6}%
\task{%
    Выразите одну величину через остальные, используя при необходимости постоянную Больцмана, число Авогадро или универсальную газовую постоянную:
    \begin{enumerate}
        \item концентрацию молекул через давление и температуру,
        \item плотность газа через его молярную массу и концентрацию молекул,
        \item среднюю кинетическую энергию поступательного движения молекул через температуру,
        \item среднеквадратичную скорость молекул через средний квадрат скорости,
    \end{enumerate}
}

\variantsplitter

\addpersonalvariant{Семён Мартынов}

\tasknumber{1}%
\task{%
    Укажите, верны ли утверждения («да» или «нет» слева от каждого утверждения):
    \begin{enumerate}
        \item Увеличение температуры на 3 градуса цельсия всегда соответствует увеличению на 3 градуса кельвина.
        \item Температуру тела всегда можно понизить на 30 кельвин (пусть при этом и может произойти фазовый переход).
        % \item Температуру тела всегда можно повысить на 30 кельвин (пусть при этом и может произойти фазовый переход).
        \item Температуру тела всегда можно понизить на 30 градусов Цельсия (пусть при этом и может произойти фазовый переход).
        % \item Температуру тела всегда можно повысить на 30 градусов Цельсия (пусть при этом и может произойти фазовый переход).

        \item У шкалы температур Кельвина есть минимальное значение (пусть и недостижимое): 0 кельвин, а у шкалы Цельсия такого значения нет вовсе и возможны температуры меньше 0 градусов цельсия.
        \item Если бы стекло, из которого изготовлен термометр, расширялось при нагревании сильнее жидкости внутри, то мы бы наблюдали, как столбик жидкости укорачивается при нагревании.
        \item Давление газа на окружающий его сосуд вызвано ударами молекул газа о его стенки: при этом изменяется импульс молекул, а значит кто-то (стенка) действовала с некоторой силой, а тогда по 3 закону Ньютона и газ действовал на стенку.
        \item В модели идеального газа невозможен теплообмен: например, если смешать две порции кислорода и азота разной температуры, то их молекулы не будут сталкиваться и обмениваться энергиями.
        Диффузия при этом произойдет.
        \item Основное уравнение МКТ идеального газа применимо к газам сколь угодно малой плотности.

        \item Основное уравнение МКТ способно описать даже плазму: состояние вещества, при котором молекулы от ударов друг об друга начинают расщепляется на ионы и электроны.
        \item Основное уравнение МКТ ИГ может быть получено теоретически из модели идеального газа, однако в нем присутствуют микропараметры, поэтому оно не допускает непосредственной экспериментальной проверки.

        \item Все процессы: изохорный, изобарный, изотермный по умолчанию предполагают, что количество вещества в них не изменяется.
        \item При горении, например, водорода в кислороде (2H2+O2-2H2O), не изменяется, ни масса вещества участвующего в реакции, ни его количество.
        Также при этом не изменяется и количество протонов, нейтронов и электронов.
        \item Каждый набор макропараметров идеального газа (P, V и T) задаёт точку в трехмерном пространстве.
        При их изменении образуется линия в этом пространстве.
        Строя графики изопроцессов в координатах PV, VT, PT мы строим проекцию этой линии на одну из плоскостей.
    \end{enumerate}
}

\tasknumber{2}%
\task{%
    Выразите:
    \begin{enumerate}
        \item плотность тела через его массу и объём,
        \item количество вещества через массу и молярную массу,
        \item основное уравнение МКТ идеального газа через концентрацию и среднюю кинетическую энергию поступательного движения его молекул.
    \end{enumerate}
}

\tasknumber{3}%
\task{%
    Напротив каждой физической величины укажите её обозначение и единицы измерения в СИ:
    \begin{enumerate}
        \item температура в Кельвинах,
        \item плотность,
        \item число Авогадро.
    \end{enumerate}
}

\tasknumber{4}%
\task{%
    Запишите, как бы вы обозначили...
    \begin{enumerate}
        \item два объема газа: до и после его расширения,
        \item число частиц: до и после утечки газа из баллона,
        \item концентрацию молекул газа в сосуде после сжатия и нагрева,
        \item температуры газа: до и после нагрева в 3 раза.
    \end{enumerate}
}

\tasknumber{5}%
\task{%
    Запишите, какие физические величины соответствуют следующим единицам измерения (указать название и обозначение),
    \begin{enumerate}
        \item кельвин,
        \item мПа,
        \item мДж,
        \item $\frac{ \text{ кг } }{ \text{ м }^3 }$,
        \item $\funits{ кг }{ моль }$.
    \end{enumerate}
}

\tasknumber{6}%
\task{%
    Выразите одну величину через остальные, используя при необходимости постоянную Больцмана, число Авогадро или универсальную газовую постоянную:
    \begin{enumerate}
        \item концентрацию молекул через давление и температуру,
        \item плотность газа через его молярную массу и концентрацию молекул,
        \item число молекул через температуру, давление и объём,
        \item средний квадрат скорости молекул через скорости отдельных молекул,
    \end{enumerate}
}

\variantsplitter

\addpersonalvariant{Варвара Минаева}

\tasknumber{1}%
\task{%
    Укажите, верны ли утверждения («да» или «нет» слева от каждого утверждения):
    \begin{enumerate}
        \item Увеличение температуры на 3 градуса цельсия всегда соответствует увеличению на 3 градуса кельвина.
        \item Температуру тела всегда можно понизить на 30 кельвин (пусть при этом и может произойти фазовый переход).
        % \item Температуру тела всегда можно повысить на 30 кельвин (пусть при этом и может произойти фазовый переход).
        \item Температуру тела всегда можно понизить на 30 градусов Цельсия (пусть при этом и может произойти фазовый переход).
        % \item Температуру тела всегда можно повысить на 30 градусов Цельсия (пусть при этом и может произойти фазовый переход).

        \item У шкалы температур Кельвина есть минимальное значение (пусть и недостижимое): 0 кельвин, а у шкалы Цельсия такого значения нет вовсе и возможны температуры меньше 0 градусов цельсия.
        \item Если бы стекло, из которого изготовлен термометр, расширялось при нагревании сильнее жидкости внутри, то мы бы наблюдали, как столбик жидкости укорачивается при нагревании.
        \item Давление газа на окружающий его сосуд вызвано ударами молекул газа о его стенки: при этом изменяется импульс молекул, а значит кто-то (стенка) действовала с некоторой силой, а тогда по 3 закону Ньютона и газ действовал на стенку.
        \item В модели идеального газа невозможен теплообмен: например, если смешать две порции кислорода и азота разной температуры, то их молекулы не будут сталкиваться и обмениваться энергиями.
        Диффузия при этом произойдет.
        \item Основное уравнение МКТ идеального газа применимо к газам сколь угодно малой плотности.

        \item Основное уравнение МКТ способно описать даже плазму: состояние вещества, при котором молекулы от ударов друг об друга начинают расщепляется на ионы и электроны.
        \item Основное уравнение МКТ ИГ может быть получено теоретически из модели идеального газа, однако в нем присутствуют микропараметры, поэтому оно не допускает непосредственной экспериментальной проверки.

        \item Все процессы: изохорный, изобарный, изотермный по умолчанию предполагают, что количество вещества в них не изменяется.
        \item При горении, например, водорода в кислороде (2H2+O2-2H2O), не изменяется, ни масса вещества участвующего в реакции, ни его количество.
        Также при этом не изменяется и количество протонов, нейтронов и электронов.
        \item Каждый набор макропараметров идеального газа (P, V и T) задаёт точку в трехмерном пространстве.
        При их изменении образуется линия в этом пространстве.
        Строя графики изопроцессов в координатах PV, VT, PT мы строим проекцию этой линии на одну из плоскостей.
    \end{enumerate}
}

\tasknumber{2}%
\task{%
    Выразите:
    \begin{enumerate}
        \item массу тела через его плотность и объём,
        \item количество вещества через массу и молярную массу,
        \item основное уравнение МКТ идеального газа через концентрацию и среднюю кинетическую энергию поступательного движения его молекул.
    \end{enumerate}
}

\tasknumber{3}%
\task{%
    Напротив каждой физической величины укажите её обозначение и единицы измерения в СИ:
    \begin{enumerate}
        \item температура в Кельвинах,
        \item плотность,
        \item число Авогадро.
    \end{enumerate}
}

\tasknumber{4}%
\task{%
    Запишите, как бы вы обозначили...
    \begin{enumerate}
        \item увеличение давления в сосуде с газом,
        \item число частиц: до и после утечки газа из баллона,
        \item концентрацию молекул газа в сосуде после сжатия и нагрева,
        \item объемы газа: до и после сжатия в 4 раза.
    \end{enumerate}
}

\tasknumber{5}%
\task{%
    Запишите, какие физические величины соответствуют следующим единицам измерения (указать название и обозначение),
    \begin{enumerate}
        \item градус Цельсия,
        \item МПа,
        \item мДж,
        \item $\frac{ 1 }{ \text{ м }^3 }$,
        \item $\funits{ кг }{ моль }$.
    \end{enumerate}
}

\tasknumber{6}%
\task{%
    Выразите одну величину через остальные, используя при необходимости постоянную Больцмана, число Авогадро или универсальную газовую постоянную:
    \begin{enumerate}
        \item температуру газа через его давление, объем, число частиц,
        \item массу молекулы через молярную массу вещества,
        \item число молекул через температуру, давление и объём,
        \item среднеквадратичную скорость молекул через средний квадрат скорости,
    \end{enumerate}
}

\variantsplitter

\addpersonalvariant{Леонид Никитин}

\tasknumber{1}%
\task{%
    Укажите, верны ли утверждения («да» или «нет» слева от каждого утверждения):
    \begin{enumerate}
        \item Увеличение температуры на 3 градуса цельсия всегда соответствует увеличению на 3 градуса кельвина.
        \item Температуру тела всегда можно понизить на 30 кельвин (пусть при этом и может произойти фазовый переход).
        % \item Температуру тела всегда можно повысить на 30 кельвин (пусть при этом и может произойти фазовый переход).
        \item Температуру тела всегда можно понизить на 30 градусов Цельсия (пусть при этом и может произойти фазовый переход).
        % \item Температуру тела всегда можно повысить на 30 градусов Цельсия (пусть при этом и может произойти фазовый переход).

        \item У шкалы температур Кельвина есть минимальное значение (пусть и недостижимое): 0 кельвин, а у шкалы Цельсия такого значения нет вовсе и возможны температуры меньше 0 градусов цельсия.
        \item Если бы стекло, из которого изготовлен термометр, расширялось при нагревании сильнее жидкости внутри, то мы бы наблюдали, как столбик жидкости укорачивается при нагревании.
        \item Давление газа на окружающий его сосуд вызвано ударами молекул газа о его стенки: при этом изменяется импульс молекул, а значит кто-то (стенка) действовала с некоторой силой, а тогда по 3 закону Ньютона и газ действовал на стенку.
        \item В модели идеального газа невозможен теплообмен: например, если смешать две порции кислорода и азота разной температуры, то их молекулы не будут сталкиваться и обмениваться энергиями.
        Диффузия при этом произойдет.
        \item Основное уравнение МКТ идеального газа применимо к газам сколь угодно малой плотности.

        \item Основное уравнение МКТ способно описать даже плазму: состояние вещества, при котором молекулы от ударов друг об друга начинают расщепляется на ионы и электроны.
        \item Основное уравнение МКТ ИГ может быть получено теоретически из модели идеального газа, однако в нем присутствуют микропараметры, поэтому оно не допускает непосредственной экспериментальной проверки.

        \item Все процессы: изохорный, изобарный, изотермный по умолчанию предполагают, что количество вещества в них не изменяется.
        \item При горении, например, водорода в кислороде (2H2+O2-2H2O), не изменяется, ни масса вещества участвующего в реакции, ни его количество.
        Также при этом не изменяется и количество протонов, нейтронов и электронов.
        \item Каждый набор макропараметров идеального газа (P, V и T) задаёт точку в трехмерном пространстве.
        При их изменении образуется линия в этом пространстве.
        Строя графики изопроцессов в координатах PV, VT, PT мы строим проекцию этой линии на одну из плоскостей.
    \end{enumerate}
}

\tasknumber{2}%
\task{%
    Выразите:
    \begin{enumerate}
        \item объём тела через его массу и плотность,
        \item количество вещества через массу и молярную массу,
        \item концентрацию молекул через их число и объём.
    \end{enumerate}
}

\tasknumber{3}%
\task{%
    Напротив каждой физической величины укажите её обозначение и единицы измерения в СИ:
    \begin{enumerate}
        \item температура в Цельсиях,
        \item число Авогадро,
        \item постоянная Больцмана.
    \end{enumerate}
}

\tasknumber{4}%
\task{%
    Запишите, как бы вы обозначили...
    \begin{enumerate}
        \item два объема газа: до и после его расширения,
        \item число частиц: до и после утечки газа из баллона,
        \item концентрацию молекул газа в сосуде после сжатия и нагрева,
        \item объемы газа: до и после сжатия в 4 раза.
    \end{enumerate}
}

\tasknumber{5}%
\task{%
    Запишите, какие физические величины соответствуют следующим единицам измерения (указать название и обозначение),
    \begin{enumerate}
        \item кельвин,
        \item мПа,
        \item эВ,
        \item $\frac{ 1 }{ \text{ л } }$,
        \item $\funits{ кг }{ моль }$.
    \end{enumerate}
}

\tasknumber{6}%
\task{%
    Выразите одну величину через остальные, используя при необходимости постоянную Больцмана, число Авогадро или универсальную газовую постоянную:
    \begin{enumerate}
        \item концентрацию молекул через давление и температуру,
        \item массу молекулы через молярную массу вещества,
        \item число молекул через температуру, давление и объём,
        \item средний квадрат проекции скорости молекул на ось $Ox$ через средний квадрат скорости молекул,
    \end{enumerate}
}

\variantsplitter

\addpersonalvariant{Тимофей Полетаев}

\tasknumber{1}%
\task{%
    Укажите, верны ли утверждения («да» или «нет» слева от каждого утверждения):
    \begin{enumerate}
        \item Увеличение температуры на 3 градуса цельсия всегда соответствует увеличению на 3 градуса кельвина.
        \item Температуру тела всегда можно понизить на 30 кельвин (пусть при этом и может произойти фазовый переход).
        % \item Температуру тела всегда можно повысить на 30 кельвин (пусть при этом и может произойти фазовый переход).
        \item Температуру тела всегда можно понизить на 30 градусов Цельсия (пусть при этом и может произойти фазовый переход).
        % \item Температуру тела всегда можно повысить на 30 градусов Цельсия (пусть при этом и может произойти фазовый переход).

        \item У шкалы температур Кельвина есть минимальное значение (пусть и недостижимое): 0 кельвин, а у шкалы Цельсия такого значения нет вовсе и возможны температуры меньше 0 градусов цельсия.
        \item Если бы стекло, из которого изготовлен термометр, расширялось при нагревании сильнее жидкости внутри, то мы бы наблюдали, как столбик жидкости укорачивается при нагревании.
        \item Давление газа на окружающий его сосуд вызвано ударами молекул газа о его стенки: при этом изменяется импульс молекул, а значит кто-то (стенка) действовала с некоторой силой, а тогда по 3 закону Ньютона и газ действовал на стенку.
        \item В модели идеального газа невозможен теплообмен: например, если смешать две порции кислорода и азота разной температуры, то их молекулы не будут сталкиваться и обмениваться энергиями.
        Диффузия при этом произойдет.
        \item Основное уравнение МКТ идеального газа применимо к газам сколь угодно малой плотности.

        \item Основное уравнение МКТ способно описать даже плазму: состояние вещества, при котором молекулы от ударов друг об друга начинают расщепляется на ионы и электроны.
        \item Основное уравнение МКТ ИГ может быть получено теоретически из модели идеального газа, однако в нем присутствуют микропараметры, поэтому оно не допускает непосредственной экспериментальной проверки.

        \item Все процессы: изохорный, изобарный, изотермный по умолчанию предполагают, что количество вещества в них не изменяется.
        \item При горении, например, водорода в кислороде (2H2+O2-2H2O), не изменяется, ни масса вещества участвующего в реакции, ни его количество.
        Также при этом не изменяется и количество протонов, нейтронов и электронов.
        \item Каждый набор макропараметров идеального газа (P, V и T) задаёт точку в трехмерном пространстве.
        При их изменении образуется линия в этом пространстве.
        Строя графики изопроцессов в координатах PV, VT, PT мы строим проекцию этой линии на одну из плоскостей.
    \end{enumerate}
}

\tasknumber{2}%
\task{%
    Выразите:
    \begin{enumerate}
        \item объём тела через его массу и плотность,
        \item количество вещества через число частиц и число Авогадро,
        \item концентрацию молекул через их число и объём.
    \end{enumerate}
}

\tasknumber{3}%
\task{%
    Напротив каждой физической величины укажите её обозначение и единицы измерения в СИ:
    \begin{enumerate}
        \item температура в Цельсиях,
        \item молярная масса,
        \item число Авогадро.
    \end{enumerate}
}

\tasknumber{4}%
\task{%
    Запишите, как бы вы обозначили...
    \begin{enumerate}
        \item два объема газа: до и после его расширения,
        \item число частиц: до и после утечки газа из баллона,
        \item концентрацию молекул газа в сосуде после сжатия и нагрева,
        \item давления газа: до и после его увеличения в 5 раз.
    \end{enumerate}
}

\tasknumber{5}%
\task{%
    Запишите, какие физические величины соответствуют следующим единицам измерения (указать название и обозначение),
    \begin{enumerate}
        \item кельвин,
        \item мПа,
        \item эВ,
        \item $\frac{ \text{ г } }{ \text{ л } }$,
        \item $\units{ г }$.
    \end{enumerate}
}

\tasknumber{6}%
\task{%
    Выразите одну величину через остальные, используя при необходимости постоянную Больцмана, число Авогадро или универсальную газовую постоянную:
    \begin{enumerate}
        \item концентрацию молекул через давление и температуру,
        \item плотность газа через концентрацию молекул и массу молекулы,
        \item среднюю кинетическую энергию поступательного движения молекул через температуру,
        \item средний квадрат проекции скорости молекул на ось $Ox$ через средний квадрат скорости молекул,
    \end{enumerate}
}

\variantsplitter

\addpersonalvariant{Андрей Рожков}

\tasknumber{1}%
\task{%
    Укажите, верны ли утверждения («да» или «нет» слева от каждого утверждения):
    \begin{enumerate}
        \item Увеличение температуры на 3 градуса цельсия всегда соответствует увеличению на 3 градуса кельвина.
        \item Температуру тела всегда можно понизить на 30 кельвин (пусть при этом и может произойти фазовый переход).
        % \item Температуру тела всегда можно повысить на 30 кельвин (пусть при этом и может произойти фазовый переход).
        \item Температуру тела всегда можно понизить на 30 градусов Цельсия (пусть при этом и может произойти фазовый переход).
        % \item Температуру тела всегда можно повысить на 30 градусов Цельсия (пусть при этом и может произойти фазовый переход).

        \item У шкалы температур Кельвина есть минимальное значение (пусть и недостижимое): 0 кельвин, а у шкалы Цельсия такого значения нет вовсе и возможны температуры меньше 0 градусов цельсия.
        \item Если бы стекло, из которого изготовлен термометр, расширялось при нагревании сильнее жидкости внутри, то мы бы наблюдали, как столбик жидкости укорачивается при нагревании.
        \item Давление газа на окружающий его сосуд вызвано ударами молекул газа о его стенки: при этом изменяется импульс молекул, а значит кто-то (стенка) действовала с некоторой силой, а тогда по 3 закону Ньютона и газ действовал на стенку.
        \item В модели идеального газа невозможен теплообмен: например, если смешать две порции кислорода и азота разной температуры, то их молекулы не будут сталкиваться и обмениваться энергиями.
        Диффузия при этом произойдет.
        \item Основное уравнение МКТ идеального газа применимо к газам сколь угодно малой плотности.

        \item Основное уравнение МКТ способно описать даже плазму: состояние вещества, при котором молекулы от ударов друг об друга начинают расщепляется на ионы и электроны.
        \item Основное уравнение МКТ ИГ может быть получено теоретически из модели идеального газа, однако в нем присутствуют микропараметры, поэтому оно не допускает непосредственной экспериментальной проверки.

        \item Все процессы: изохорный, изобарный, изотермный по умолчанию предполагают, что количество вещества в них не изменяется.
        \item При горении, например, водорода в кислороде (2H2+O2-2H2O), не изменяется, ни масса вещества участвующего в реакции, ни его количество.
        Также при этом не изменяется и количество протонов, нейтронов и электронов.
        \item Каждый набор макропараметров идеального газа (P, V и T) задаёт точку в трехмерном пространстве.
        При их изменении образуется линия в этом пространстве.
        Строя графики изопроцессов в координатах PV, VT, PT мы строим проекцию этой линии на одну из плоскостей.
    \end{enumerate}
}

\tasknumber{2}%
\task{%
    Выразите:
    \begin{enumerate}
        \item плотность тела через его массу и объём,
        \item количество вещества через массу и молярную массу,
        \item концентрацию молекул через их число и объём.
    \end{enumerate}
}

\tasknumber{3}%
\task{%
    Напротив каждой физической величины укажите её обозначение и единицы измерения в СИ:
    \begin{enumerate}
        \item температура в Кельвинах,
        \item плотность,
        \item постоянная Больцмана.
    \end{enumerate}
}

\tasknumber{4}%
\task{%
    Запишите, как бы вы обозначили...
    \begin{enumerate}
        \item увеличение давления в сосуде с газом,
        \item число частиц: до и после утечки газа из баллона,
        \item концентрацию молекул газа в сосуде после сжатия и нагрева,
        \item температуры газа: до и после нагрева в 3 раза.
    \end{enumerate}
}

\tasknumber{5}%
\task{%
    Запишите, какие физические величины соответствуют следующим единицам измерения (указать название и обозначение),
    \begin{enumerate}
        \item градус Цельсия,
        \item мПа,
        \item эВ,
        \item $\frac{ 1 }{ \text{ л } }$,
        \item $\funits{ кг }{ моль }$.
    \end{enumerate}
}

\tasknumber{6}%
\task{%
    Выразите одну величину через остальные, используя при необходимости постоянную Больцмана, число Авогадро или универсальную газовую постоянную:
    \begin{enumerate}
        \item температуру газа через его давление, объем, число частиц,
        \item плотность газа через его молярную массу и концентрацию молекул,
        \item число молекул через температуру, давление и объём,
        \item средний квадрат скорости молекул через скорости отдельных молекул,
    \end{enumerate}
}

\variantsplitter

\addpersonalvariant{Рената Таржиманова}

\tasknumber{1}%
\task{%
    Укажите, верны ли утверждения («да» или «нет» слева от каждого утверждения):
    \begin{enumerate}
        \item Увеличение температуры на 3 градуса цельсия всегда соответствует увеличению на 3 градуса кельвина.
        \item Температуру тела всегда можно понизить на 30 кельвин (пусть при этом и может произойти фазовый переход).
        % \item Температуру тела всегда можно повысить на 30 кельвин (пусть при этом и может произойти фазовый переход).
        \item Температуру тела всегда можно понизить на 30 градусов Цельсия (пусть при этом и может произойти фазовый переход).
        % \item Температуру тела всегда можно повысить на 30 градусов Цельсия (пусть при этом и может произойти фазовый переход).

        \item У шкалы температур Кельвина есть минимальное значение (пусть и недостижимое): 0 кельвин, а у шкалы Цельсия такого значения нет вовсе и возможны температуры меньше 0 градусов цельсия.
        \item Если бы стекло, из которого изготовлен термометр, расширялось при нагревании сильнее жидкости внутри, то мы бы наблюдали, как столбик жидкости укорачивается при нагревании.
        \item Давление газа на окружающий его сосуд вызвано ударами молекул газа о его стенки: при этом изменяется импульс молекул, а значит кто-то (стенка) действовала с некоторой силой, а тогда по 3 закону Ньютона и газ действовал на стенку.
        \item В модели идеального газа невозможен теплообмен: например, если смешать две порции кислорода и азота разной температуры, то их молекулы не будут сталкиваться и обмениваться энергиями.
        Диффузия при этом произойдет.
        \item Основное уравнение МКТ идеального газа применимо к газам сколь угодно малой плотности.

        \item Основное уравнение МКТ способно описать даже плазму: состояние вещества, при котором молекулы от ударов друг об друга начинают расщепляется на ионы и электроны.
        \item Основное уравнение МКТ ИГ может быть получено теоретически из модели идеального газа, однако в нем присутствуют микропараметры, поэтому оно не допускает непосредственной экспериментальной проверки.

        \item Все процессы: изохорный, изобарный, изотермный по умолчанию предполагают, что количество вещества в них не изменяется.
        \item При горении, например, водорода в кислороде (2H2+O2-2H2O), не изменяется, ни масса вещества участвующего в реакции, ни его количество.
        Также при этом не изменяется и количество протонов, нейтронов и электронов.
        \item Каждый набор макропараметров идеального газа (P, V и T) задаёт точку в трехмерном пространстве.
        При их изменении образуется линия в этом пространстве.
        Строя графики изопроцессов в координатах PV, VT, PT мы строим проекцию этой линии на одну из плоскостей.
    \end{enumerate}
}

\tasknumber{2}%
\task{%
    Выразите:
    \begin{enumerate}
        \item массу тела через его плотность и объём,
        \item количество вещества через массу и молярную массу,
        \item концентрацию молекул через их число и объём.
    \end{enumerate}
}

\tasknumber{3}%
\task{%
    Напротив каждой физической величины укажите её обозначение и единицы измерения в СИ:
    \begin{enumerate}
        \item температура в Кельвинах,
        \item объем,
        \item число Авогадро.
    \end{enumerate}
}

\tasknumber{4}%
\task{%
    Запишите, как бы вы обозначили...
    \begin{enumerate}
        \item два объема газа: до и после его расширения,
        \item количество вещества: до и после утечки газа из баллона,
        \item концентрацию молекул газа в сосуде после сжатия и нагрева,
        \item объемы газа: до и после сжатия в 4 раза.
    \end{enumerate}
}

\tasknumber{5}%
\task{%
    Запишите, какие физические величины соответствуют следующим единицам измерения (указать название и обозначение),
    \begin{enumerate}
        \item кельвин,
        \item мПа,
        \item эВ,
        \item $\text{ л }$,
        \item $\units{ моль }$.
    \end{enumerate}
}

\tasknumber{6}%
\task{%
    Выразите одну величину через остальные, используя при необходимости постоянную Больцмана, число Авогадро или универсальную газовую постоянную:
    \begin{enumerate}
        \item концентрацию молекул через давление и температуру,
        \item плотность газа через концентрацию молекул и массу молекулы,
        \item число молекул через температуру, давление и объём,
        \item среднеквадратичную скорость молекул через средний квадрат скорости,
    \end{enumerate}
}

\variantsplitter

\addpersonalvariant{Арсений Трофимов}

\tasknumber{1}%
\task{%
    Укажите, верны ли утверждения («да» или «нет» слева от каждого утверждения):
    \begin{enumerate}
        \item Увеличение температуры на 3 градуса цельсия всегда соответствует увеличению на 3 градуса кельвина.
        \item Температуру тела всегда можно понизить на 30 кельвин (пусть при этом и может произойти фазовый переход).
        % \item Температуру тела всегда можно повысить на 30 кельвин (пусть при этом и может произойти фазовый переход).
        \item Температуру тела всегда можно понизить на 30 градусов Цельсия (пусть при этом и может произойти фазовый переход).
        % \item Температуру тела всегда можно повысить на 30 градусов Цельсия (пусть при этом и может произойти фазовый переход).

        \item У шкалы температур Кельвина есть минимальное значение (пусть и недостижимое): 0 кельвин, а у шкалы Цельсия такого значения нет вовсе и возможны температуры меньше 0 градусов цельсия.
        \item Если бы стекло, из которого изготовлен термометр, расширялось при нагревании сильнее жидкости внутри, то мы бы наблюдали, как столбик жидкости укорачивается при нагревании.
        \item Давление газа на окружающий его сосуд вызвано ударами молекул газа о его стенки: при этом изменяется импульс молекул, а значит кто-то (стенка) действовала с некоторой силой, а тогда по 3 закону Ньютона и газ действовал на стенку.
        \item В модели идеального газа невозможен теплообмен: например, если смешать две порции кислорода и азота разной температуры, то их молекулы не будут сталкиваться и обмениваться энергиями.
        Диффузия при этом произойдет.
        \item Основное уравнение МКТ идеального газа применимо к газам сколь угодно малой плотности.

        \item Основное уравнение МКТ способно описать даже плазму: состояние вещества, при котором молекулы от ударов друг об друга начинают расщепляется на ионы и электроны.
        \item Основное уравнение МКТ ИГ может быть получено теоретически из модели идеального газа, однако в нем присутствуют микропараметры, поэтому оно не допускает непосредственной экспериментальной проверки.

        \item Все процессы: изохорный, изобарный, изотермный по умолчанию предполагают, что количество вещества в них не изменяется.
        \item При горении, например, водорода в кислороде (2H2+O2-2H2O), не изменяется, ни масса вещества участвующего в реакции, ни его количество.
        Также при этом не изменяется и количество протонов, нейтронов и электронов.
        \item Каждый набор макропараметров идеального газа (P, V и T) задаёт точку в трехмерном пространстве.
        При их изменении образуется линия в этом пространстве.
        Строя графики изопроцессов в координатах PV, VT, PT мы строим проекцию этой линии на одну из плоскостей.
    \end{enumerate}
}

\tasknumber{2}%
\task{%
    Выразите:
    \begin{enumerate}
        \item плотность тела через его массу и объём,
        \item количество вещества через число частиц и число Авогадро,
        \item основное уравнение МКТ идеального газа через концентрацию и среднюю кинетическую энергию поступательного движения его молекул.
    \end{enumerate}
}

\tasknumber{3}%
\task{%
    Напротив каждой физической величины укажите её обозначение и единицы измерения в СИ:
    \begin{enumerate}
        \item температура в Цельсиях,
        \item число Авогадро,
        \item число Авогадро.
    \end{enumerate}
}

\tasknumber{4}%
\task{%
    Запишите, как бы вы обозначили...
    \begin{enumerate}
        \item увеличение давления в сосуде с газом,
        \item число частиц: до и после утечки газа из баллона,
        \item массы газа: до и после утечки газа из баллона,
        \item давления газа: до и после его увеличения в 5 раз.
    \end{enumerate}
}

\tasknumber{5}%
\task{%
    Запишите, какие физические величины соответствуют следующим единицам измерения (указать название и обозначение),
    \begin{enumerate}
        \item градус Цельсия,
        \item МПа,
        \item эВ,
        \item $\frac{ \text{ г } }{ \text{ л } }$,
        \item $\funits{ г }{ моль }$.
    \end{enumerate}
}

\tasknumber{6}%
\task{%
    Выразите одну величину через остальные, используя при необходимости постоянную Больцмана, число Авогадро или универсальную газовую постоянную:
    \begin{enumerate}
        \item температуру газа через его давление, объем, число частиц,
        \item плотность газа через его молярную массу и концентрацию молекул,
        \item число молекул через температуру, давление и объём,
        \item среднеквадратичную скорость молекул через средний квадрат скорости,
    \end{enumerate}
}

\variantsplitter

\addpersonalvariant{Андрей Щербаков}

\tasknumber{1}%
\task{%
    Укажите, верны ли утверждения («да» или «нет» слева от каждого утверждения):
    \begin{enumerate}
        \item Увеличение температуры на 3 градуса цельсия всегда соответствует увеличению на 3 градуса кельвина.
        \item Температуру тела всегда можно понизить на 30 кельвин (пусть при этом и может произойти фазовый переход).
        % \item Температуру тела всегда можно повысить на 30 кельвин (пусть при этом и может произойти фазовый переход).
        \item Температуру тела всегда можно понизить на 30 градусов Цельсия (пусть при этом и может произойти фазовый переход).
        % \item Температуру тела всегда можно повысить на 30 градусов Цельсия (пусть при этом и может произойти фазовый переход).

        \item У шкалы температур Кельвина есть минимальное значение (пусть и недостижимое): 0 кельвин, а у шкалы Цельсия такого значения нет вовсе и возможны температуры меньше 0 градусов цельсия.
        \item Если бы стекло, из которого изготовлен термометр, расширялось при нагревании сильнее жидкости внутри, то мы бы наблюдали, как столбик жидкости укорачивается при нагревании.
        \item Давление газа на окружающий его сосуд вызвано ударами молекул газа о его стенки: при этом изменяется импульс молекул, а значит кто-то (стенка) действовала с некоторой силой, а тогда по 3 закону Ньютона и газ действовал на стенку.
        \item В модели идеального газа невозможен теплообмен: например, если смешать две порции кислорода и азота разной температуры, то их молекулы не будут сталкиваться и обмениваться энергиями.
        Диффузия при этом произойдет.
        \item Основное уравнение МКТ идеального газа применимо к газам сколь угодно малой плотности.

        \item Основное уравнение МКТ способно описать даже плазму: состояние вещества, при котором молекулы от ударов друг об друга начинают расщепляется на ионы и электроны.
        \item Основное уравнение МКТ ИГ может быть получено теоретически из модели идеального газа, однако в нем присутствуют микропараметры, поэтому оно не допускает непосредственной экспериментальной проверки.

        \item Все процессы: изохорный, изобарный, изотермный по умолчанию предполагают, что количество вещества в них не изменяется.
        \item При горении, например, водорода в кислороде (2H2+O2-2H2O), не изменяется, ни масса вещества участвующего в реакции, ни его количество.
        Также при этом не изменяется и количество протонов, нейтронов и электронов.
        \item Каждый набор макропараметров идеального газа (P, V и T) задаёт точку в трехмерном пространстве.
        При их изменении образуется линия в этом пространстве.
        Строя графики изопроцессов в координатах PV, VT, PT мы строим проекцию этой линии на одну из плоскостей.
    \end{enumerate}
}

\tasknumber{2}%
\task{%
    Выразите:
    \begin{enumerate}
        \item плотность тела через его массу и объём,
        \item количество вещества через массу и молярную массу,
        \item основное уравнение МКТ идеального газа через концентрацию и среднюю кинетическую энергию поступательного движения его молекул.
    \end{enumerate}
}

\tasknumber{3}%
\task{%
    Напротив каждой физической величины укажите её обозначение и единицы измерения в СИ:
    \begin{enumerate}
        \item температура в Кельвинах,
        \item число частиц,
        \item число Авогадро.
    \end{enumerate}
}

\tasknumber{4}%
\task{%
    Запишите, как бы вы обозначили...
    \begin{enumerate}
        \item два объема газа: до и после его расширения,
        \item число частиц: до и после утечки газа из баллона,
        \item массы газа: до и после утечки газа из баллона,
        \item температуры газа: до и после нагрева в 3 раза.
    \end{enumerate}
}

\tasknumber{5}%
\task{%
    Запишите, какие физические величины соответствуют следующим единицам измерения (указать название и обозначение),
    \begin{enumerate}
        \item градус Цельсия,
        \item мПа,
        \item мкДж,
        \item $\frac{ 1 }{ \text{ м }^3 }$,
        \item $\units{ моль }$.
    \end{enumerate}
}

\tasknumber{6}%
\task{%
    Выразите одну величину через остальные, используя при необходимости постоянную Больцмана, число Авогадро или универсальную газовую постоянную:
    \begin{enumerate}
        \item концентрацию молекул через давление и температуру,
        \item массу молекулы через молярную массу вещества,
        \item число молекул через температуру, давление и объём,
        \item средний квадрат скорости молекул через скорости отдельных молекул,
    \end{enumerate}
}

\variantsplitter

\addpersonalvariant{Михаил Ярошевский}

\tasknumber{1}%
\task{%
    Укажите, верны ли утверждения («да» или «нет» слева от каждого утверждения):
    \begin{enumerate}
        \item Увеличение температуры на 3 градуса цельсия всегда соответствует увеличению на 3 градуса кельвина.
        \item Температуру тела всегда можно понизить на 30 кельвин (пусть при этом и может произойти фазовый переход).
        % \item Температуру тела всегда можно повысить на 30 кельвин (пусть при этом и может произойти фазовый переход).
        \item Температуру тела всегда можно понизить на 30 градусов Цельсия (пусть при этом и может произойти фазовый переход).
        % \item Температуру тела всегда можно повысить на 30 градусов Цельсия (пусть при этом и может произойти фазовый переход).

        \item У шкалы температур Кельвина есть минимальное значение (пусть и недостижимое): 0 кельвин, а у шкалы Цельсия такого значения нет вовсе и возможны температуры меньше 0 градусов цельсия.
        \item Если бы стекло, из которого изготовлен термометр, расширялось при нагревании сильнее жидкости внутри, то мы бы наблюдали, как столбик жидкости укорачивается при нагревании.
        \item Давление газа на окружающий его сосуд вызвано ударами молекул газа о его стенки: при этом изменяется импульс молекул, а значит кто-то (стенка) действовала с некоторой силой, а тогда по 3 закону Ньютона и газ действовал на стенку.
        \item В модели идеального газа невозможен теплообмен: например, если смешать две порции кислорода и азота разной температуры, то их молекулы не будут сталкиваться и обмениваться энергиями.
        Диффузия при этом произойдет.
        \item Основное уравнение МКТ идеального газа применимо к газам сколь угодно малой плотности.

        \item Основное уравнение МКТ способно описать даже плазму: состояние вещества, при котором молекулы от ударов друг об друга начинают расщепляется на ионы и электроны.
        \item Основное уравнение МКТ ИГ может быть получено теоретически из модели идеального газа, однако в нем присутствуют микропараметры, поэтому оно не допускает непосредственной экспериментальной проверки.

        \item Все процессы: изохорный, изобарный, изотермный по умолчанию предполагают, что количество вещества в них не изменяется.
        \item При горении, например, водорода в кислороде (2H2+O2-2H2O), не изменяется, ни масса вещества участвующего в реакции, ни его количество.
        Также при этом не изменяется и количество протонов, нейтронов и электронов.
        \item Каждый набор макропараметров идеального газа (P, V и T) задаёт точку в трехмерном пространстве.
        При их изменении образуется линия в этом пространстве.
        Строя графики изопроцессов в координатах PV, VT, PT мы строим проекцию этой линии на одну из плоскостей.
    \end{enumerate}
}

\tasknumber{2}%
\task{%
    Выразите:
    \begin{enumerate}
        \item объём тела через его массу и плотность,
        \item количество вещества через массу и молярную массу,
        \item основное уравнение МКТ идеального газа через концентрацию и среднюю кинетическую энергию поступательного движения его молекул.
    \end{enumerate}
}

\tasknumber{3}%
\task{%
    Напротив каждой физической величины укажите её обозначение и единицы измерения в СИ:
    \begin{enumerate}
        \item температура в Цельсиях,
        \item объем,
        \item постоянная Больцмана.
    \end{enumerate}
}

\tasknumber{4}%
\task{%
    Запишите, как бы вы обозначили...
    \begin{enumerate}
        \item два объема газа: до и после его расширения,
        \item число частиц: до и после утечки газа из баллона,
        \item концентрацию молекул газа в сосуде после сжатия и нагрева,
        \item объемы газа: до и после сжатия в 4 раза.
    \end{enumerate}
}

\tasknumber{5}%
\task{%
    Запишите, какие физические величины соответствуют следующим единицам измерения (указать название и обозначение),
    \begin{enumerate}
        \item градус Цельсия,
        \item мПа,
        \item мкДж,
        \item $\frac{ \text{ кг } }{ \text{ м }^3 }$,
        \item $\funits{ г }{ моль }$.
    \end{enumerate}
}

\tasknumber{6}%
\task{%
    Выразите одну величину через остальные, используя при необходимости постоянную Больцмана, число Авогадро или универсальную газовую постоянную:
    \begin{enumerate}
        \item концентрацию молекул через давление и температуру,
        \item массу молекулы через молярную массу вещества,
        \item число молекул через температуру, давление и объём,
        \item среднеквадратичную скорость молекул через средний квадрат скорости,
    \end{enumerate}
}

\variantsplitter

\addpersonalvariant{Алексей Алимпиев}

\tasknumber{1}%
\task{%
    Укажите, верны ли утверждения («да» или «нет» слева от каждого утверждения):
    \begin{enumerate}
        \item Увеличение температуры на 3 градуса цельсия всегда соответствует увеличению на 3 градуса кельвина.
        \item Температуру тела всегда можно понизить на 30 кельвин (пусть при этом и может произойти фазовый переход).
        % \item Температуру тела всегда можно повысить на 30 кельвин (пусть при этом и может произойти фазовый переход).
        \item Температуру тела всегда можно понизить на 30 градусов Цельсия (пусть при этом и может произойти фазовый переход).
        % \item Температуру тела всегда можно повысить на 30 градусов Цельсия (пусть при этом и может произойти фазовый переход).

        \item У шкалы температур Кельвина есть минимальное значение (пусть и недостижимое): 0 кельвин, а у шкалы Цельсия такого значения нет вовсе и возможны температуры меньше 0 градусов цельсия.
        \item Если бы стекло, из которого изготовлен термометр, расширялось при нагревании сильнее жидкости внутри, то мы бы наблюдали, как столбик жидкости укорачивается при нагревании.
        \item Давление газа на окружающий его сосуд вызвано ударами молекул газа о его стенки: при этом изменяется импульс молекул, а значит кто-то (стенка) действовала с некоторой силой, а тогда по 3 закону Ньютона и газ действовал на стенку.
        \item В модели идеального газа невозможен теплообмен: например, если смешать две порции кислорода и азота разной температуры, то их молекулы не будут сталкиваться и обмениваться энергиями.
        Диффузия при этом произойдет.
        \item Основное уравнение МКТ идеального газа применимо к газам сколь угодно малой плотности.

        \item Основное уравнение МКТ способно описать даже плазму: состояние вещества, при котором молекулы от ударов друг об друга начинают расщепляется на ионы и электроны.
        \item Основное уравнение МКТ ИГ может быть получено теоретически из модели идеального газа, однако в нем присутствуют микропараметры, поэтому оно не допускает непосредственной экспериментальной проверки.

        \item Все процессы: изохорный, изобарный, изотермный по умолчанию предполагают, что количество вещества в них не изменяется.
        \item При горении, например, водорода в кислороде (2H2+O2-2H2O), не изменяется, ни масса вещества участвующего в реакции, ни его количество.
        Также при этом не изменяется и количество протонов, нейтронов и электронов.
        \item Каждый набор макропараметров идеального газа (P, V и T) задаёт точку в трехмерном пространстве.
        При их изменении образуется линия в этом пространстве.
        Строя графики изопроцессов в координатах PV, VT, PT мы строим проекцию этой линии на одну из плоскостей.
    \end{enumerate}
}

\tasknumber{2}%
\task{%
    Выразите:
    \begin{enumerate}
        \item объём тела через его массу и плотность,
        \item количество вещества через число частиц и число Авогадро,
        \item основное уравнение МКТ идеального газа через концентрацию и среднюю кинетическую энергию поступательного движения его молекул.
    \end{enumerate}
}

\tasknumber{3}%
\task{%
    Напротив каждой физической величины укажите её обозначение и единицы измерения в СИ:
    \begin{enumerate}
        \item температура в Кельвинах,
        \item число частиц,
        \item число Авогадро.
    \end{enumerate}
}

\tasknumber{4}%
\task{%
    Запишите, как бы вы обозначили...
    \begin{enumerate}
        \item увеличение давления в сосуде с газом,
        \item число частиц: до и после утечки газа из баллона,
        \item массы газа: до и после утечки газа из баллона,
        \item объемы газа: до и после сжатия в 4 раза.
    \end{enumerate}
}

\tasknumber{5}%
\task{%
    Запишите, какие физические величины соответствуют следующим единицам измерения (указать название и обозначение),
    \begin{enumerate}
        \item градус Цельсия,
        \item МПа,
        \item эВ,
        \item $\text{ м }^3$,
        \item $\funits{ кг }{ моль }$.
    \end{enumerate}
}

\tasknumber{6}%
\task{%
    Выразите одну величину через остальные, используя при необходимости постоянную Больцмана, число Авогадро или универсальную газовую постоянную:
    \begin{enumerate}
        \item температуру газа через его давление, объем, число частиц,
        \item плотность газа через концентрацию молекул и массу молекулы,
        \item среднюю кинетическую энергию поступательного движения молекул через температуру,
        \item среднеквадратичную скорость поступательного движения молекул через температуру и массу молекулы,
    \end{enumerate}
}

\variantsplitter

\addpersonalvariant{Евгений Васин}

\tasknumber{1}%
\task{%
    Укажите, верны ли утверждения («да» или «нет» слева от каждого утверждения):
    \begin{enumerate}
        \item Увеличение температуры на 3 градуса цельсия всегда соответствует увеличению на 3 градуса кельвина.
        \item Температуру тела всегда можно понизить на 30 кельвин (пусть при этом и может произойти фазовый переход).
        % \item Температуру тела всегда можно повысить на 30 кельвин (пусть при этом и может произойти фазовый переход).
        \item Температуру тела всегда можно понизить на 30 градусов Цельсия (пусть при этом и может произойти фазовый переход).
        % \item Температуру тела всегда можно повысить на 30 градусов Цельсия (пусть при этом и может произойти фазовый переход).

        \item У шкалы температур Кельвина есть минимальное значение (пусть и недостижимое): 0 кельвин, а у шкалы Цельсия такого значения нет вовсе и возможны температуры меньше 0 градусов цельсия.
        \item Если бы стекло, из которого изготовлен термометр, расширялось при нагревании сильнее жидкости внутри, то мы бы наблюдали, как столбик жидкости укорачивается при нагревании.
        \item Давление газа на окружающий его сосуд вызвано ударами молекул газа о его стенки: при этом изменяется импульс молекул, а значит кто-то (стенка) действовала с некоторой силой, а тогда по 3 закону Ньютона и газ действовал на стенку.
        \item В модели идеального газа невозможен теплообмен: например, если смешать две порции кислорода и азота разной температуры, то их молекулы не будут сталкиваться и обмениваться энергиями.
        Диффузия при этом произойдет.
        \item Основное уравнение МКТ идеального газа применимо к газам сколь угодно малой плотности.

        \item Основное уравнение МКТ способно описать даже плазму: состояние вещества, при котором молекулы от ударов друг об друга начинают расщепляется на ионы и электроны.
        \item Основное уравнение МКТ ИГ может быть получено теоретически из модели идеального газа, однако в нем присутствуют микропараметры, поэтому оно не допускает непосредственной экспериментальной проверки.

        \item Все процессы: изохорный, изобарный, изотермный по умолчанию предполагают, что количество вещества в них не изменяется.
        \item При горении, например, водорода в кислороде (2H2+O2-2H2O), не изменяется, ни масса вещества участвующего в реакции, ни его количество.
        Также при этом не изменяется и количество протонов, нейтронов и электронов.
        \item Каждый набор макропараметров идеального газа (P, V и T) задаёт точку в трехмерном пространстве.
        При их изменении образуется линия в этом пространстве.
        Строя графики изопроцессов в координатах PV, VT, PT мы строим проекцию этой линии на одну из плоскостей.
    \end{enumerate}
}

\tasknumber{2}%
\task{%
    Выразите:
    \begin{enumerate}
        \item массу тела через его плотность и объём,
        \item количество вещества через число частиц и число Авогадро,
        \item концентрацию молекул через их число и объём.
    \end{enumerate}
}

\tasknumber{3}%
\task{%
    Напротив каждой физической величины укажите её обозначение и единицы измерения в СИ:
    \begin{enumerate}
        \item температура в Кельвинах,
        \item масса,
        \item постоянная Больцмана.
    \end{enumerate}
}

\tasknumber{4}%
\task{%
    Запишите, как бы вы обозначили...
    \begin{enumerate}
        \item увеличение давления в сосуде с газом,
        \item число частиц: до и после утечки газа из баллона,
        \item массы газа: до и после утечки газа из баллона,
        \item температуры газа: до и после нагрева в 3 раза.
    \end{enumerate}
}

\tasknumber{5}%
\task{%
    Запишите, какие физические величины соответствуют следующим единицам измерения (указать название и обозначение),
    \begin{enumerate}
        \item кельвин,
        \item МПа,
        \item эВ,
        \item $\frac{ 1 }{ \text{ л } }$,
        \item $\units{ г }$.
    \end{enumerate}
}

\tasknumber{6}%
\task{%
    Выразите одну величину через остальные, используя при необходимости постоянную Больцмана, число Авогадро или универсальную газовую постоянную:
    \begin{enumerate}
        \item температуру газа через его давление, объем, число частиц,
        \item массу молекулы через молярную массу вещества,
        \item число молекул через температуру, давление и объём,
        \item средний квадрат скорости молекул через скорости отдельных молекул,
    \end{enumerate}
}

\variantsplitter

\addpersonalvariant{Волохов Вячеслав}

\tasknumber{1}%
\task{%
    Укажите, верны ли утверждения («да» или «нет» слева от каждого утверждения):
    \begin{enumerate}
        \item Увеличение температуры на 3 градуса цельсия всегда соответствует увеличению на 3 градуса кельвина.
        \item Температуру тела всегда можно понизить на 30 кельвин (пусть при этом и может произойти фазовый переход).
        % \item Температуру тела всегда можно повысить на 30 кельвин (пусть при этом и может произойти фазовый переход).
        \item Температуру тела всегда можно понизить на 30 градусов Цельсия (пусть при этом и может произойти фазовый переход).
        % \item Температуру тела всегда можно повысить на 30 градусов Цельсия (пусть при этом и может произойти фазовый переход).

        \item У шкалы температур Кельвина есть минимальное значение (пусть и недостижимое): 0 кельвин, а у шкалы Цельсия такого значения нет вовсе и возможны температуры меньше 0 градусов цельсия.
        \item Если бы стекло, из которого изготовлен термометр, расширялось при нагревании сильнее жидкости внутри, то мы бы наблюдали, как столбик жидкости укорачивается при нагревании.
        \item Давление газа на окружающий его сосуд вызвано ударами молекул газа о его стенки: при этом изменяется импульс молекул, а значит кто-то (стенка) действовала с некоторой силой, а тогда по 3 закону Ньютона и газ действовал на стенку.
        \item В модели идеального газа невозможен теплообмен: например, если смешать две порции кислорода и азота разной температуры, то их молекулы не будут сталкиваться и обмениваться энергиями.
        Диффузия при этом произойдет.
        \item Основное уравнение МКТ идеального газа применимо к газам сколь угодно малой плотности.

        \item Основное уравнение МКТ способно описать даже плазму: состояние вещества, при котором молекулы от ударов друг об друга начинают расщепляется на ионы и электроны.
        \item Основное уравнение МКТ ИГ может быть получено теоретически из модели идеального газа, однако в нем присутствуют микропараметры, поэтому оно не допускает непосредственной экспериментальной проверки.

        \item Все процессы: изохорный, изобарный, изотермный по умолчанию предполагают, что количество вещества в них не изменяется.
        \item При горении, например, водорода в кислороде (2H2+O2-2H2O), не изменяется, ни масса вещества участвующего в реакции, ни его количество.
        Также при этом не изменяется и количество протонов, нейтронов и электронов.
        \item Каждый набор макропараметров идеального газа (P, V и T) задаёт точку в трехмерном пространстве.
        При их изменении образуется линия в этом пространстве.
        Строя графики изопроцессов в координатах PV, VT, PT мы строим проекцию этой линии на одну из плоскостей.
    \end{enumerate}
}

\tasknumber{2}%
\task{%
    Выразите:
    \begin{enumerate}
        \item плотность тела через его массу и объём,
        \item количество вещества через число частиц и число Авогадро,
        \item основное уравнение МКТ идеального газа через концентрацию и среднюю кинетическую энергию поступательного движения его молекул.
    \end{enumerate}
}

\tasknumber{3}%
\task{%
    Напротив каждой физической величины укажите её обозначение и единицы измерения в СИ:
    \begin{enumerate}
        \item температура в Кельвинах,
        \item плотность,
        \item число Авогадро.
    \end{enumerate}
}

\tasknumber{4}%
\task{%
    Запишите, как бы вы обозначили...
    \begin{enumerate}
        \item два объема газа: до и после его расширения,
        \item количество вещества: до и после утечки газа из баллона,
        \item концентрацию молекул газа в сосуде после сжатия и нагрева,
        \item давления газа: до и после его увеличения в 5 раз.
    \end{enumerate}
}

\tasknumber{5}%
\task{%
    Запишите, какие физические величины соответствуют следующим единицам измерения (указать название и обозначение),
    \begin{enumerate}
        \item градус Цельсия,
        \item МПа,
        \item эВ,
        \item $\frac{ \text{ г } }{ \text{ л } }$,
        \item $\units{ моль }$.
    \end{enumerate}
}

\tasknumber{6}%
\task{%
    Выразите одну величину через остальные, используя при необходимости постоянную Больцмана, число Авогадро или универсальную газовую постоянную:
    \begin{enumerate}
        \item концентрацию молекул через давление и температуру,
        \item плотность газа через его молярную массу и концентрацию молекул,
        \item среднюю кинетическую энергию поступательного движения молекул через температуру,
        \item средний квадрат проекции скорости молекул на ось $Ox$ через средний квадрат скорости молекул,
    \end{enumerate}
}

\variantsplitter

\addpersonalvariant{Герман Говоров}

\tasknumber{1}%
\task{%
    Укажите, верны ли утверждения («да» или «нет» слева от каждого утверждения):
    \begin{enumerate}
        \item Увеличение температуры на 3 градуса цельсия всегда соответствует увеличению на 3 градуса кельвина.
        \item Температуру тела всегда можно понизить на 30 кельвин (пусть при этом и может произойти фазовый переход).
        % \item Температуру тела всегда можно повысить на 30 кельвин (пусть при этом и может произойти фазовый переход).
        \item Температуру тела всегда можно понизить на 30 градусов Цельсия (пусть при этом и может произойти фазовый переход).
        % \item Температуру тела всегда можно повысить на 30 градусов Цельсия (пусть при этом и может произойти фазовый переход).

        \item У шкалы температур Кельвина есть минимальное значение (пусть и недостижимое): 0 кельвин, а у шкалы Цельсия такого значения нет вовсе и возможны температуры меньше 0 градусов цельсия.
        \item Если бы стекло, из которого изготовлен термометр, расширялось при нагревании сильнее жидкости внутри, то мы бы наблюдали, как столбик жидкости укорачивается при нагревании.
        \item Давление газа на окружающий его сосуд вызвано ударами молекул газа о его стенки: при этом изменяется импульс молекул, а значит кто-то (стенка) действовала с некоторой силой, а тогда по 3 закону Ньютона и газ действовал на стенку.
        \item В модели идеального газа невозможен теплообмен: например, если смешать две порции кислорода и азота разной температуры, то их молекулы не будут сталкиваться и обмениваться энергиями.
        Диффузия при этом произойдет.
        \item Основное уравнение МКТ идеального газа применимо к газам сколь угодно малой плотности.

        \item Основное уравнение МКТ способно описать даже плазму: состояние вещества, при котором молекулы от ударов друг об друга начинают расщепляется на ионы и электроны.
        \item Основное уравнение МКТ ИГ может быть получено теоретически из модели идеального газа, однако в нем присутствуют микропараметры, поэтому оно не допускает непосредственной экспериментальной проверки.

        \item Все процессы: изохорный, изобарный, изотермный по умолчанию предполагают, что количество вещества в них не изменяется.
        \item При горении, например, водорода в кислороде (2H2+O2-2H2O), не изменяется, ни масса вещества участвующего в реакции, ни его количество.
        Также при этом не изменяется и количество протонов, нейтронов и электронов.
        \item Каждый набор макропараметров идеального газа (P, V и T) задаёт точку в трехмерном пространстве.
        При их изменении образуется линия в этом пространстве.
        Строя графики изопроцессов в координатах PV, VT, PT мы строим проекцию этой линии на одну из плоскостей.
    \end{enumerate}
}

\tasknumber{2}%
\task{%
    Выразите:
    \begin{enumerate}
        \item объём тела через его массу и плотность,
        \item количество вещества через число частиц и число Авогадро,
        \item концентрацию молекул через их число и объём.
    \end{enumerate}
}

\tasknumber{3}%
\task{%
    Напротив каждой физической величины укажите её обозначение и единицы измерения в СИ:
    \begin{enumerate}
        \item температура в Кельвинах,
        \item число Авогадро,
        \item постоянная Больцмана.
    \end{enumerate}
}

\tasknumber{4}%
\task{%
    Запишите, как бы вы обозначили...
    \begin{enumerate}
        \item два объема газа: до и после его расширения,
        \item число частиц: до и после утечки газа из баллона,
        \item массы газа: до и после утечки газа из баллона,
        \item объемы газа: до и после сжатия в 4 раза.
    \end{enumerate}
}

\tasknumber{5}%
\task{%
    Запишите, какие физические величины соответствуют следующим единицам измерения (указать название и обозначение),
    \begin{enumerate}
        \item градус Цельсия,
        \item МПа,
        \item эВ,
        \item $\frac{ 1 }{ \text{ л } }$,
        \item $\units{ моль }$.
    \end{enumerate}
}

\tasknumber{6}%
\task{%
    Выразите одну величину через остальные, используя при необходимости постоянную Больцмана, число Авогадро или универсальную газовую постоянную:
    \begin{enumerate}
        \item концентрацию молекул через давление и температуру,
        \item плотность газа через концентрацию молекул и массу молекулы,
        \item среднюю кинетическую энергию поступательного движения молекул через температуру,
        \item среднеквадратичную скорость поступательного движения молекул через температуру и массу молекулы,
    \end{enumerate}
}

\variantsplitter

\addpersonalvariant{София Журавлева}

\tasknumber{1}%
\task{%
    Укажите, верны ли утверждения («да» или «нет» слева от каждого утверждения):
    \begin{enumerate}
        \item Увеличение температуры на 3 градуса цельсия всегда соответствует увеличению на 3 градуса кельвина.
        \item Температуру тела всегда можно понизить на 30 кельвин (пусть при этом и может произойти фазовый переход).
        % \item Температуру тела всегда можно повысить на 30 кельвин (пусть при этом и может произойти фазовый переход).
        \item Температуру тела всегда можно понизить на 30 градусов Цельсия (пусть при этом и может произойти фазовый переход).
        % \item Температуру тела всегда можно повысить на 30 градусов Цельсия (пусть при этом и может произойти фазовый переход).

        \item У шкалы температур Кельвина есть минимальное значение (пусть и недостижимое): 0 кельвин, а у шкалы Цельсия такого значения нет вовсе и возможны температуры меньше 0 градусов цельсия.
        \item Если бы стекло, из которого изготовлен термометр, расширялось при нагревании сильнее жидкости внутри, то мы бы наблюдали, как столбик жидкости укорачивается при нагревании.
        \item Давление газа на окружающий его сосуд вызвано ударами молекул газа о его стенки: при этом изменяется импульс молекул, а значит кто-то (стенка) действовала с некоторой силой, а тогда по 3 закону Ньютона и газ действовал на стенку.
        \item В модели идеального газа невозможен теплообмен: например, если смешать две порции кислорода и азота разной температуры, то их молекулы не будут сталкиваться и обмениваться энергиями.
        Диффузия при этом произойдет.
        \item Основное уравнение МКТ идеального газа применимо к газам сколь угодно малой плотности.

        \item Основное уравнение МКТ способно описать даже плазму: состояние вещества, при котором молекулы от ударов друг об друга начинают расщепляется на ионы и электроны.
        \item Основное уравнение МКТ ИГ может быть получено теоретически из модели идеального газа, однако в нем присутствуют микропараметры, поэтому оно не допускает непосредственной экспериментальной проверки.

        \item Все процессы: изохорный, изобарный, изотермный по умолчанию предполагают, что количество вещества в них не изменяется.
        \item При горении, например, водорода в кислороде (2H2+O2-2H2O), не изменяется, ни масса вещества участвующего в реакции, ни его количество.
        Также при этом не изменяется и количество протонов, нейтронов и электронов.
        \item Каждый набор макропараметров идеального газа (P, V и T) задаёт точку в трехмерном пространстве.
        При их изменении образуется линия в этом пространстве.
        Строя графики изопроцессов в координатах PV, VT, PT мы строим проекцию этой линии на одну из плоскостей.
    \end{enumerate}
}

\tasknumber{2}%
\task{%
    Выразите:
    \begin{enumerate}
        \item объём тела через его массу и плотность,
        \item количество вещества через массу и молярную массу,
        \item основное уравнение МКТ идеального газа через концентрацию и среднюю кинетическую энергию поступательного движения его молекул.
    \end{enumerate}
}

\tasknumber{3}%
\task{%
    Напротив каждой физической величины укажите её обозначение и единицы измерения в СИ:
    \begin{enumerate}
        \item температура в Кельвинах,
        \item количество вещества,
        \item число Авогадро.
    \end{enumerate}
}

\tasknumber{4}%
\task{%
    Запишите, как бы вы обозначили...
    \begin{enumerate}
        \item два объема газа: до и после его расширения,
        \item количество вещества: до и после утечки газа из баллона,
        \item концентрацию молекул газа в сосуде после сжатия и нагрева,
        \item давления газа: до и после его увеличения в 5 раз.
    \end{enumerate}
}

\tasknumber{5}%
\task{%
    Запишите, какие физические величины соответствуют следующим единицам измерения (указать название и обозначение),
    \begin{enumerate}
        \item градус Цельсия,
        \item МПа,
        \item эВ,
        \item $\frac{ \text{ кг } }{ \text{ м }^3 }$,
        \item $\funits{ г }{ моль }$.
    \end{enumerate}
}

\tasknumber{6}%
\task{%
    Выразите одну величину через остальные, используя при необходимости постоянную Больцмана, число Авогадро или универсальную газовую постоянную:
    \begin{enumerate}
        \item концентрацию молекул через давление и температуру,
        \item плотность газа через его молярную массу и концентрацию молекул,
        \item среднюю кинетическую энергию поступательного движения молекул через температуру,
        \item среднеквадратичную скорость молекул через средний квадрат скорости,
    \end{enumerate}
}

\variantsplitter

\addpersonalvariant{Константин Козлов}

\tasknumber{1}%
\task{%
    Укажите, верны ли утверждения («да» или «нет» слева от каждого утверждения):
    \begin{enumerate}
        \item Увеличение температуры на 3 градуса цельсия всегда соответствует увеличению на 3 градуса кельвина.
        \item Температуру тела всегда можно понизить на 30 кельвин (пусть при этом и может произойти фазовый переход).
        % \item Температуру тела всегда можно повысить на 30 кельвин (пусть при этом и может произойти фазовый переход).
        \item Температуру тела всегда можно понизить на 30 градусов Цельсия (пусть при этом и может произойти фазовый переход).
        % \item Температуру тела всегда можно повысить на 30 градусов Цельсия (пусть при этом и может произойти фазовый переход).

        \item У шкалы температур Кельвина есть минимальное значение (пусть и недостижимое): 0 кельвин, а у шкалы Цельсия такого значения нет вовсе и возможны температуры меньше 0 градусов цельсия.
        \item Если бы стекло, из которого изготовлен термометр, расширялось при нагревании сильнее жидкости внутри, то мы бы наблюдали, как столбик жидкости укорачивается при нагревании.
        \item Давление газа на окружающий его сосуд вызвано ударами молекул газа о его стенки: при этом изменяется импульс молекул, а значит кто-то (стенка) действовала с некоторой силой, а тогда по 3 закону Ньютона и газ действовал на стенку.
        \item В модели идеального газа невозможен теплообмен: например, если смешать две порции кислорода и азота разной температуры, то их молекулы не будут сталкиваться и обмениваться энергиями.
        Диффузия при этом произойдет.
        \item Основное уравнение МКТ идеального газа применимо к газам сколь угодно малой плотности.

        \item Основное уравнение МКТ способно описать даже плазму: состояние вещества, при котором молекулы от ударов друг об друга начинают расщепляется на ионы и электроны.
        \item Основное уравнение МКТ ИГ может быть получено теоретически из модели идеального газа, однако в нем присутствуют микропараметры, поэтому оно не допускает непосредственной экспериментальной проверки.

        \item Все процессы: изохорный, изобарный, изотермный по умолчанию предполагают, что количество вещества в них не изменяется.
        \item При горении, например, водорода в кислороде (2H2+O2-2H2O), не изменяется, ни масса вещества участвующего в реакции, ни его количество.
        Также при этом не изменяется и количество протонов, нейтронов и электронов.
        \item Каждый набор макропараметров идеального газа (P, V и T) задаёт точку в трехмерном пространстве.
        При их изменении образуется линия в этом пространстве.
        Строя графики изопроцессов в координатах PV, VT, PT мы строим проекцию этой линии на одну из плоскостей.
    \end{enumerate}
}

\tasknumber{2}%
\task{%
    Выразите:
    \begin{enumerate}
        \item плотность тела через его массу и объём,
        \item количество вещества через массу и молярную массу,
        \item концентрацию молекул через их число и объём.
    \end{enumerate}
}

\tasknumber{3}%
\task{%
    Напротив каждой физической величины укажите её обозначение и единицы измерения в СИ:
    \begin{enumerate}
        \item температура в Цельсиях,
        \item молярная масса,
        \item число Авогадро.
    \end{enumerate}
}

\tasknumber{4}%
\task{%
    Запишите, как бы вы обозначили...
    \begin{enumerate}
        \item увеличение давления в сосуде с газом,
        \item число частиц: до и после утечки газа из баллона,
        \item концентрацию молекул газа в сосуде после сжатия и нагрева,
        \item температуры газа: до и после нагрева в 3 раза.
    \end{enumerate}
}

\tasknumber{5}%
\task{%
    Запишите, какие физические величины соответствуют следующим единицам измерения (указать название и обозначение),
    \begin{enumerate}
        \item градус Цельсия,
        \item МПа,
        \item мкДж,
        \item $\text{ л }$,
        \item $\funits{ г }{ моль }$.
    \end{enumerate}
}

\tasknumber{6}%
\task{%
    Выразите одну величину через остальные, используя при необходимости постоянную Больцмана, число Авогадро или универсальную газовую постоянную:
    \begin{enumerate}
        \item температуру газа через его давление, объем, число частиц,
        \item плотность газа через его молярную массу и концентрацию молекул,
        \item число молекул через температуру, давление и объём,
        \item средний квадрат скорости молекул через скорости отдельных молекул,
    \end{enumerate}
}

\variantsplitter

\addpersonalvariant{Наталья Кравченко}

\tasknumber{1}%
\task{%
    Укажите, верны ли утверждения («да» или «нет» слева от каждого утверждения):
    \begin{enumerate}
        \item Увеличение температуры на 3 градуса цельсия всегда соответствует увеличению на 3 градуса кельвина.
        \item Температуру тела всегда можно понизить на 30 кельвин (пусть при этом и может произойти фазовый переход).
        % \item Температуру тела всегда можно повысить на 30 кельвин (пусть при этом и может произойти фазовый переход).
        \item Температуру тела всегда можно понизить на 30 градусов Цельсия (пусть при этом и может произойти фазовый переход).
        % \item Температуру тела всегда можно повысить на 30 градусов Цельсия (пусть при этом и может произойти фазовый переход).

        \item У шкалы температур Кельвина есть минимальное значение (пусть и недостижимое): 0 кельвин, а у шкалы Цельсия такого значения нет вовсе и возможны температуры меньше 0 градусов цельсия.
        \item Если бы стекло, из которого изготовлен термометр, расширялось при нагревании сильнее жидкости внутри, то мы бы наблюдали, как столбик жидкости укорачивается при нагревании.
        \item Давление газа на окружающий его сосуд вызвано ударами молекул газа о его стенки: при этом изменяется импульс молекул, а значит кто-то (стенка) действовала с некоторой силой, а тогда по 3 закону Ньютона и газ действовал на стенку.
        \item В модели идеального газа невозможен теплообмен: например, если смешать две порции кислорода и азота разной температуры, то их молекулы не будут сталкиваться и обмениваться энергиями.
        Диффузия при этом произойдет.
        \item Основное уравнение МКТ идеального газа применимо к газам сколь угодно малой плотности.

        \item Основное уравнение МКТ способно описать даже плазму: состояние вещества, при котором молекулы от ударов друг об друга начинают расщепляется на ионы и электроны.
        \item Основное уравнение МКТ ИГ может быть получено теоретически из модели идеального газа, однако в нем присутствуют микропараметры, поэтому оно не допускает непосредственной экспериментальной проверки.

        \item Все процессы: изохорный, изобарный, изотермный по умолчанию предполагают, что количество вещества в них не изменяется.
        \item При горении, например, водорода в кислороде (2H2+O2-2H2O), не изменяется, ни масса вещества участвующего в реакции, ни его количество.
        Также при этом не изменяется и количество протонов, нейтронов и электронов.
        \item Каждый набор макропараметров идеального газа (P, V и T) задаёт точку в трехмерном пространстве.
        При их изменении образуется линия в этом пространстве.
        Строя графики изопроцессов в координатах PV, VT, PT мы строим проекцию этой линии на одну из плоскостей.
    \end{enumerate}
}

\tasknumber{2}%
\task{%
    Выразите:
    \begin{enumerate}
        \item объём тела через его массу и плотность,
        \item количество вещества через число частиц и число Авогадро,
        \item концентрацию молекул через их число и объём.
    \end{enumerate}
}

\tasknumber{3}%
\task{%
    Напротив каждой физической величины укажите её обозначение и единицы измерения в СИ:
    \begin{enumerate}
        \item температура в Цельсиях,
        \item масса,
        \item число Авогадро.
    \end{enumerate}
}

\tasknumber{4}%
\task{%
    Запишите, как бы вы обозначили...
    \begin{enumerate}
        \item увеличение давления в сосуде с газом,
        \item число частиц: до и после утечки газа из баллона,
        \item массы газа: до и после утечки газа из баллона,
        \item объемы газа: до и после сжатия в 4 раза.
    \end{enumerate}
}

\tasknumber{5}%
\task{%
    Запишите, какие физические величины соответствуют следующим единицам измерения (указать название и обозначение),
    \begin{enumerate}
        \item кельвин,
        \item МПа,
        \item эВ,
        \item $\frac{ 1 }{ \text{ м }^3 }$,
        \item $\units{ моль }$.
    \end{enumerate}
}

\tasknumber{6}%
\task{%
    Выразите одну величину через остальные, используя при необходимости постоянную Больцмана, число Авогадро или универсальную газовую постоянную:
    \begin{enumerate}
        \item температуру газа через его давление, объем, число частиц,
        \item плотность газа через концентрацию молекул и массу молекулы,
        \item среднюю кинетическую энергию поступательного движения молекул через температуру,
        \item средний квадрат скорости молекул через скорости отдельных молекул,
    \end{enumerate}
}

\variantsplitter

\addpersonalvariant{Матвей Кузьмин}

\tasknumber{1}%
\task{%
    Укажите, верны ли утверждения («да» или «нет» слева от каждого утверждения):
    \begin{enumerate}
        \item Увеличение температуры на 3 градуса цельсия всегда соответствует увеличению на 3 градуса кельвина.
        \item Температуру тела всегда можно понизить на 30 кельвин (пусть при этом и может произойти фазовый переход).
        % \item Температуру тела всегда можно повысить на 30 кельвин (пусть при этом и может произойти фазовый переход).
        \item Температуру тела всегда можно понизить на 30 градусов Цельсия (пусть при этом и может произойти фазовый переход).
        % \item Температуру тела всегда можно повысить на 30 градусов Цельсия (пусть при этом и может произойти фазовый переход).

        \item У шкалы температур Кельвина есть минимальное значение (пусть и недостижимое): 0 кельвин, а у шкалы Цельсия такого значения нет вовсе и возможны температуры меньше 0 градусов цельсия.
        \item Если бы стекло, из которого изготовлен термометр, расширялось при нагревании сильнее жидкости внутри, то мы бы наблюдали, как столбик жидкости укорачивается при нагревании.
        \item Давление газа на окружающий его сосуд вызвано ударами молекул газа о его стенки: при этом изменяется импульс молекул, а значит кто-то (стенка) действовала с некоторой силой, а тогда по 3 закону Ньютона и газ действовал на стенку.
        \item В модели идеального газа невозможен теплообмен: например, если смешать две порции кислорода и азота разной температуры, то их молекулы не будут сталкиваться и обмениваться энергиями.
        Диффузия при этом произойдет.
        \item Основное уравнение МКТ идеального газа применимо к газам сколь угодно малой плотности.

        \item Основное уравнение МКТ способно описать даже плазму: состояние вещества, при котором молекулы от ударов друг об друга начинают расщепляется на ионы и электроны.
        \item Основное уравнение МКТ ИГ может быть получено теоретически из модели идеального газа, однако в нем присутствуют микропараметры, поэтому оно не допускает непосредственной экспериментальной проверки.

        \item Все процессы: изохорный, изобарный, изотермный по умолчанию предполагают, что количество вещества в них не изменяется.
        \item При горении, например, водорода в кислороде (2H2+O2-2H2O), не изменяется, ни масса вещества участвующего в реакции, ни его количество.
        Также при этом не изменяется и количество протонов, нейтронов и электронов.
        \item Каждый набор макропараметров идеального газа (P, V и T) задаёт точку в трехмерном пространстве.
        При их изменении образуется линия в этом пространстве.
        Строя графики изопроцессов в координатах PV, VT, PT мы строим проекцию этой линии на одну из плоскостей.
    \end{enumerate}
}

\tasknumber{2}%
\task{%
    Выразите:
    \begin{enumerate}
        \item объём тела через его массу и плотность,
        \item количество вещества через число частиц и число Авогадро,
        \item концентрацию молекул через их число и объём.
    \end{enumerate}
}

\tasknumber{3}%
\task{%
    Напротив каждой физической величины укажите её обозначение и единицы измерения в СИ:
    \begin{enumerate}
        \item температура в Цельсиях,
        \item число частиц,
        \item число Авогадро.
    \end{enumerate}
}

\tasknumber{4}%
\task{%
    Запишите, как бы вы обозначили...
    \begin{enumerate}
        \item увеличение давления в сосуде с газом,
        \item число частиц: до и после утечки газа из баллона,
        \item массы газа: до и после утечки газа из баллона,
        \item температуры газа: до и после нагрева в 3 раза.
    \end{enumerate}
}

\tasknumber{5}%
\task{%
    Запишите, какие физические величины соответствуют следующим единицам измерения (указать название и обозначение),
    \begin{enumerate}
        \item градус Цельсия,
        \item мПа,
        \item мДж,
        \item $\frac{ \text{ г } }{ \text{ л } }$,
        \item $\funits{ кг }{ моль }$.
    \end{enumerate}
}

\tasknumber{6}%
\task{%
    Выразите одну величину через остальные, используя при необходимости постоянную Больцмана, число Авогадро или универсальную газовую постоянную:
    \begin{enumerate}
        \item температуру газа через его давление, объем, число частиц,
        \item массу молекулы через молярную массу вещества,
        \item число молекул через температуру, давление и объём,
        \item средний квадрат скорости молекул через скорости отдельных молекул,
    \end{enumerate}
}

\variantsplitter

\addpersonalvariant{Сергей Малышев}

\tasknumber{1}%
\task{%
    Укажите, верны ли утверждения («да» или «нет» слева от каждого утверждения):
    \begin{enumerate}
        \item Увеличение температуры на 3 градуса цельсия всегда соответствует увеличению на 3 градуса кельвина.
        \item Температуру тела всегда можно понизить на 30 кельвин (пусть при этом и может произойти фазовый переход).
        % \item Температуру тела всегда можно повысить на 30 кельвин (пусть при этом и может произойти фазовый переход).
        \item Температуру тела всегда можно понизить на 30 градусов Цельсия (пусть при этом и может произойти фазовый переход).
        % \item Температуру тела всегда можно повысить на 30 градусов Цельсия (пусть при этом и может произойти фазовый переход).

        \item У шкалы температур Кельвина есть минимальное значение (пусть и недостижимое): 0 кельвин, а у шкалы Цельсия такого значения нет вовсе и возможны температуры меньше 0 градусов цельсия.
        \item Если бы стекло, из которого изготовлен термометр, расширялось при нагревании сильнее жидкости внутри, то мы бы наблюдали, как столбик жидкости укорачивается при нагревании.
        \item Давление газа на окружающий его сосуд вызвано ударами молекул газа о его стенки: при этом изменяется импульс молекул, а значит кто-то (стенка) действовала с некоторой силой, а тогда по 3 закону Ньютона и газ действовал на стенку.
        \item В модели идеального газа невозможен теплообмен: например, если смешать две порции кислорода и азота разной температуры, то их молекулы не будут сталкиваться и обмениваться энергиями.
        Диффузия при этом произойдет.
        \item Основное уравнение МКТ идеального газа применимо к газам сколь угодно малой плотности.

        \item Основное уравнение МКТ способно описать даже плазму: состояние вещества, при котором молекулы от ударов друг об друга начинают расщепляется на ионы и электроны.
        \item Основное уравнение МКТ ИГ может быть получено теоретически из модели идеального газа, однако в нем присутствуют микропараметры, поэтому оно не допускает непосредственной экспериментальной проверки.

        \item Все процессы: изохорный, изобарный, изотермный по умолчанию предполагают, что количество вещества в них не изменяется.
        \item При горении, например, водорода в кислороде (2H2+O2-2H2O), не изменяется, ни масса вещества участвующего в реакции, ни его количество.
        Также при этом не изменяется и количество протонов, нейтронов и электронов.
        \item Каждый набор макропараметров идеального газа (P, V и T) задаёт точку в трехмерном пространстве.
        При их изменении образуется линия в этом пространстве.
        Строя графики изопроцессов в координатах PV, VT, PT мы строим проекцию этой линии на одну из плоскостей.
    \end{enumerate}
}

\tasknumber{2}%
\task{%
    Выразите:
    \begin{enumerate}
        \item массу тела через его плотность и объём,
        \item количество вещества через массу и молярную массу,
        \item концентрацию молекул через их число и объём.
    \end{enumerate}
}

\tasknumber{3}%
\task{%
    Напротив каждой физической величины укажите её обозначение и единицы измерения в СИ:
    \begin{enumerate}
        \item температура в Цельсиях,
        \item плотность,
        \item число Авогадро.
    \end{enumerate}
}

\tasknumber{4}%
\task{%
    Запишите, как бы вы обозначили...
    \begin{enumerate}
        \item два объема газа: до и после его расширения,
        \item количество вещества: до и после утечки газа из баллона,
        \item массы газа: до и после утечки газа из баллона,
        \item температуры газа: до и после нагрева в 3 раза.
    \end{enumerate}
}

\tasknumber{5}%
\task{%
    Запишите, какие физические величины соответствуют следующим единицам измерения (указать название и обозначение),
    \begin{enumerate}
        \item кельвин,
        \item МПа,
        \item мкДж,
        \item $\frac{ 1 }{ \text{ м }^3 }$,
        \item $\units{ г }$.
    \end{enumerate}
}

\tasknumber{6}%
\task{%
    Выразите одну величину через остальные, используя при необходимости постоянную Больцмана, число Авогадро или универсальную газовую постоянную:
    \begin{enumerate}
        \item концентрацию молекул через давление и температуру,
        \item плотность газа через концентрацию молекул и массу молекулы,
        \item число молекул через температуру, давление и объём,
        \item средний квадрат скорости молекул через скорости отдельных молекул,
    \end{enumerate}
}

\variantsplitter

\addpersonalvariant{Алина Полканова}

\tasknumber{1}%
\task{%
    Укажите, верны ли утверждения («да» или «нет» слева от каждого утверждения):
    \begin{enumerate}
        \item Увеличение температуры на 3 градуса цельсия всегда соответствует увеличению на 3 градуса кельвина.
        \item Температуру тела всегда можно понизить на 30 кельвин (пусть при этом и может произойти фазовый переход).
        % \item Температуру тела всегда можно повысить на 30 кельвин (пусть при этом и может произойти фазовый переход).
        \item Температуру тела всегда можно понизить на 30 градусов Цельсия (пусть при этом и может произойти фазовый переход).
        % \item Температуру тела всегда можно повысить на 30 градусов Цельсия (пусть при этом и может произойти фазовый переход).

        \item У шкалы температур Кельвина есть минимальное значение (пусть и недостижимое): 0 кельвин, а у шкалы Цельсия такого значения нет вовсе и возможны температуры меньше 0 градусов цельсия.
        \item Если бы стекло, из которого изготовлен термометр, расширялось при нагревании сильнее жидкости внутри, то мы бы наблюдали, как столбик жидкости укорачивается при нагревании.
        \item Давление газа на окружающий его сосуд вызвано ударами молекул газа о его стенки: при этом изменяется импульс молекул, а значит кто-то (стенка) действовала с некоторой силой, а тогда по 3 закону Ньютона и газ действовал на стенку.
        \item В модели идеального газа невозможен теплообмен: например, если смешать две порции кислорода и азота разной температуры, то их молекулы не будут сталкиваться и обмениваться энергиями.
        Диффузия при этом произойдет.
        \item Основное уравнение МКТ идеального газа применимо к газам сколь угодно малой плотности.

        \item Основное уравнение МКТ способно описать даже плазму: состояние вещества, при котором молекулы от ударов друг об друга начинают расщепляется на ионы и электроны.
        \item Основное уравнение МКТ ИГ может быть получено теоретически из модели идеального газа, однако в нем присутствуют микропараметры, поэтому оно не допускает непосредственной экспериментальной проверки.

        \item Все процессы: изохорный, изобарный, изотермный по умолчанию предполагают, что количество вещества в них не изменяется.
        \item При горении, например, водорода в кислороде (2H2+O2-2H2O), не изменяется, ни масса вещества участвующего в реакции, ни его количество.
        Также при этом не изменяется и количество протонов, нейтронов и электронов.
        \item Каждый набор макропараметров идеального газа (P, V и T) задаёт точку в трехмерном пространстве.
        При их изменении образуется линия в этом пространстве.
        Строя графики изопроцессов в координатах PV, VT, PT мы строим проекцию этой линии на одну из плоскостей.
    \end{enumerate}
}

\tasknumber{2}%
\task{%
    Выразите:
    \begin{enumerate}
        \item плотность тела через его массу и объём,
        \item количество вещества через число частиц и число Авогадро,
        \item концентрацию молекул через их число и объём.
    \end{enumerate}
}

\tasknumber{3}%
\task{%
    Напротив каждой физической величины укажите её обозначение и единицы измерения в СИ:
    \begin{enumerate}
        \item температура в Кельвинах,
        \item число частиц,
        \item число Авогадро.
    \end{enumerate}
}

\tasknumber{4}%
\task{%
    Запишите, как бы вы обозначили...
    \begin{enumerate}
        \item увеличение давления в сосуде с газом,
        \item количество вещества: до и после утечки газа из баллона,
        \item концентрацию молекул газа в сосуде после сжатия и нагрева,
        \item температуры газа: до и после нагрева в 3 раза.
    \end{enumerate}
}

\tasknumber{5}%
\task{%
    Запишите, какие физические величины соответствуют следующим единицам измерения (указать название и обозначение),
    \begin{enumerate}
        \item градус Цельсия,
        \item мПа,
        \item мкДж,
        \item $\text{ м }^3$,
        \item $\units{ моль }$.
    \end{enumerate}
}

\tasknumber{6}%
\task{%
    Выразите одну величину через остальные, используя при необходимости постоянную Больцмана, число Авогадро или универсальную газовую постоянную:
    \begin{enumerate}
        \item температуру газа через его давление, объем, число частиц,
        \item массу молекулы через молярную массу вещества,
        \item среднюю кинетическую энергию поступательного движения молекул через температуру,
        \item средний квадрат скорости молекул через скорости отдельных молекул,
    \end{enumerate}
}

\variantsplitter

\addpersonalvariant{Сергей Пономарёв}

\tasknumber{1}%
\task{%
    Укажите, верны ли утверждения («да» или «нет» слева от каждого утверждения):
    \begin{enumerate}
        \item Увеличение температуры на 3 градуса цельсия всегда соответствует увеличению на 3 градуса кельвина.
        \item Температуру тела всегда можно понизить на 30 кельвин (пусть при этом и может произойти фазовый переход).
        % \item Температуру тела всегда можно повысить на 30 кельвин (пусть при этом и может произойти фазовый переход).
        \item Температуру тела всегда можно понизить на 30 градусов Цельсия (пусть при этом и может произойти фазовый переход).
        % \item Температуру тела всегда можно повысить на 30 градусов Цельсия (пусть при этом и может произойти фазовый переход).

        \item У шкалы температур Кельвина есть минимальное значение (пусть и недостижимое): 0 кельвин, а у шкалы Цельсия такого значения нет вовсе и возможны температуры меньше 0 градусов цельсия.
        \item Если бы стекло, из которого изготовлен термометр, расширялось при нагревании сильнее жидкости внутри, то мы бы наблюдали, как столбик жидкости укорачивается при нагревании.
        \item Давление газа на окружающий его сосуд вызвано ударами молекул газа о его стенки: при этом изменяется импульс молекул, а значит кто-то (стенка) действовала с некоторой силой, а тогда по 3 закону Ньютона и газ действовал на стенку.
        \item В модели идеального газа невозможен теплообмен: например, если смешать две порции кислорода и азота разной температуры, то их молекулы не будут сталкиваться и обмениваться энергиями.
        Диффузия при этом произойдет.
        \item Основное уравнение МКТ идеального газа применимо к газам сколь угодно малой плотности.

        \item Основное уравнение МКТ способно описать даже плазму: состояние вещества, при котором молекулы от ударов друг об друга начинают расщепляется на ионы и электроны.
        \item Основное уравнение МКТ ИГ может быть получено теоретически из модели идеального газа, однако в нем присутствуют микропараметры, поэтому оно не допускает непосредственной экспериментальной проверки.

        \item Все процессы: изохорный, изобарный, изотермный по умолчанию предполагают, что количество вещества в них не изменяется.
        \item При горении, например, водорода в кислороде (2H2+O2-2H2O), не изменяется, ни масса вещества участвующего в реакции, ни его количество.
        Также при этом не изменяется и количество протонов, нейтронов и электронов.
        \item Каждый набор макропараметров идеального газа (P, V и T) задаёт точку в трехмерном пространстве.
        При их изменении образуется линия в этом пространстве.
        Строя графики изопроцессов в координатах PV, VT, PT мы строим проекцию этой линии на одну из плоскостей.
    \end{enumerate}
}

\tasknumber{2}%
\task{%
    Выразите:
    \begin{enumerate}
        \item плотность тела через его массу и объём,
        \item количество вещества через число частиц и число Авогадро,
        \item основное уравнение МКТ идеального газа через концентрацию и среднюю кинетическую энергию поступательного движения его молекул.
    \end{enumerate}
}

\tasknumber{3}%
\task{%
    Напротив каждой физической величины укажите её обозначение и единицы измерения в СИ:
    \begin{enumerate}
        \item температура в Цельсиях,
        \item плотность,
        \item постоянная Больцмана.
    \end{enumerate}
}

\tasknumber{4}%
\task{%
    Запишите, как бы вы обозначили...
    \begin{enumerate}
        \item два объема газа: до и после его расширения,
        \item количество вещества: до и после утечки газа из баллона,
        \item концентрацию молекул газа в сосуде после сжатия и нагрева,
        \item давления газа: до и после его увеличения в 5 раз.
    \end{enumerate}
}

\tasknumber{5}%
\task{%
    Запишите, какие физические величины соответствуют следующим единицам измерения (указать название и обозначение),
    \begin{enumerate}
        \item кельвин,
        \item мПа,
        \item мкДж,
        \item $\frac{ 1 }{ \text{ л } }$,
        \item $\units{ г }$.
    \end{enumerate}
}

\tasknumber{6}%
\task{%
    Выразите одну величину через остальные, используя при необходимости постоянную Больцмана, число Авогадро или универсальную газовую постоянную:
    \begin{enumerate}
        \item концентрацию молекул через давление и температуру,
        \item плотность газа через его молярную массу и концентрацию молекул,
        \item среднюю кинетическую энергию поступательного движения молекул через температуру,
        \item среднеквадратичную скорость молекул через средний квадрат скорости,
    \end{enumerate}
}

\variantsplitter

\addpersonalvariant{Егор Свистушкин}

\tasknumber{1}%
\task{%
    Укажите, верны ли утверждения («да» или «нет» слева от каждого утверждения):
    \begin{enumerate}
        \item Увеличение температуры на 3 градуса цельсия всегда соответствует увеличению на 3 градуса кельвина.
        \item Температуру тела всегда можно понизить на 30 кельвин (пусть при этом и может произойти фазовый переход).
        % \item Температуру тела всегда можно повысить на 30 кельвин (пусть при этом и может произойти фазовый переход).
        \item Температуру тела всегда можно понизить на 30 градусов Цельсия (пусть при этом и может произойти фазовый переход).
        % \item Температуру тела всегда можно повысить на 30 градусов Цельсия (пусть при этом и может произойти фазовый переход).

        \item У шкалы температур Кельвина есть минимальное значение (пусть и недостижимое): 0 кельвин, а у шкалы Цельсия такого значения нет вовсе и возможны температуры меньше 0 градусов цельсия.
        \item Если бы стекло, из которого изготовлен термометр, расширялось при нагревании сильнее жидкости внутри, то мы бы наблюдали, как столбик жидкости укорачивается при нагревании.
        \item Давление газа на окружающий его сосуд вызвано ударами молекул газа о его стенки: при этом изменяется импульс молекул, а значит кто-то (стенка) действовала с некоторой силой, а тогда по 3 закону Ньютона и газ действовал на стенку.
        \item В модели идеального газа невозможен теплообмен: например, если смешать две порции кислорода и азота разной температуры, то их молекулы не будут сталкиваться и обмениваться энергиями.
        Диффузия при этом произойдет.
        \item Основное уравнение МКТ идеального газа применимо к газам сколь угодно малой плотности.

        \item Основное уравнение МКТ способно описать даже плазму: состояние вещества, при котором молекулы от ударов друг об друга начинают расщепляется на ионы и электроны.
        \item Основное уравнение МКТ ИГ может быть получено теоретически из модели идеального газа, однако в нем присутствуют микропараметры, поэтому оно не допускает непосредственной экспериментальной проверки.

        \item Все процессы: изохорный, изобарный, изотермный по умолчанию предполагают, что количество вещества в них не изменяется.
        \item При горении, например, водорода в кислороде (2H2+O2-2H2O), не изменяется, ни масса вещества участвующего в реакции, ни его количество.
        Также при этом не изменяется и количество протонов, нейтронов и электронов.
        \item Каждый набор макропараметров идеального газа (P, V и T) задаёт точку в трехмерном пространстве.
        При их изменении образуется линия в этом пространстве.
        Строя графики изопроцессов в координатах PV, VT, PT мы строим проекцию этой линии на одну из плоскостей.
    \end{enumerate}
}

\tasknumber{2}%
\task{%
    Выразите:
    \begin{enumerate}
        \item массу тела через его плотность и объём,
        \item количество вещества через число частиц и число Авогадро,
        \item основное уравнение МКТ идеального газа через концентрацию и среднюю кинетическую энергию поступательного движения его молекул.
    \end{enumerate}
}

\tasknumber{3}%
\task{%
    Напротив каждой физической величины укажите её обозначение и единицы измерения в СИ:
    \begin{enumerate}
        \item температура в Кельвинах,
        \item плотность,
        \item число Авогадро.
    \end{enumerate}
}

\tasknumber{4}%
\task{%
    Запишите, как бы вы обозначили...
    \begin{enumerate}
        \item увеличение давления в сосуде с газом,
        \item число частиц: до и после утечки газа из баллона,
        \item концентрацию молекул газа в сосуде после сжатия и нагрева,
        \item объемы газа: до и после сжатия в 4 раза.
    \end{enumerate}
}

\tasknumber{5}%
\task{%
    Запишите, какие физические величины соответствуют следующим единицам измерения (указать название и обозначение),
    \begin{enumerate}
        \item кельвин,
        \item мПа,
        \item мкДж,
        \item $\text{ л }$,
        \item $\units{ г }$.
    \end{enumerate}
}

\tasknumber{6}%
\task{%
    Выразите одну величину через остальные, используя при необходимости постоянную Больцмана, число Авогадро или универсальную газовую постоянную:
    \begin{enumerate}
        \item температуру газа через его давление, объем, число частиц,
        \item массу молекулы через молярную массу вещества,
        \item число молекул через температуру, давление и объём,
        \item среднеквадратичную скорость молекул через средний квадрат скорости,
    \end{enumerate}
}

\variantsplitter

\addpersonalvariant{Дмитрий Соколов}

\tasknumber{1}%
\task{%
    Укажите, верны ли утверждения («да» или «нет» слева от каждого утверждения):
    \begin{enumerate}
        \item Увеличение температуры на 3 градуса цельсия всегда соответствует увеличению на 3 градуса кельвина.
        \item Температуру тела всегда можно понизить на 30 кельвин (пусть при этом и может произойти фазовый переход).
        % \item Температуру тела всегда можно повысить на 30 кельвин (пусть при этом и может произойти фазовый переход).
        \item Температуру тела всегда можно понизить на 30 градусов Цельсия (пусть при этом и может произойти фазовый переход).
        % \item Температуру тела всегда можно повысить на 30 градусов Цельсия (пусть при этом и может произойти фазовый переход).

        \item У шкалы температур Кельвина есть минимальное значение (пусть и недостижимое): 0 кельвин, а у шкалы Цельсия такого значения нет вовсе и возможны температуры меньше 0 градусов цельсия.
        \item Если бы стекло, из которого изготовлен термометр, расширялось при нагревании сильнее жидкости внутри, то мы бы наблюдали, как столбик жидкости укорачивается при нагревании.
        \item Давление газа на окружающий его сосуд вызвано ударами молекул газа о его стенки: при этом изменяется импульс молекул, а значит кто-то (стенка) действовала с некоторой силой, а тогда по 3 закону Ньютона и газ действовал на стенку.
        \item В модели идеального газа невозможен теплообмен: например, если смешать две порции кислорода и азота разной температуры, то их молекулы не будут сталкиваться и обмениваться энергиями.
        Диффузия при этом произойдет.
        \item Основное уравнение МКТ идеального газа применимо к газам сколь угодно малой плотности.

        \item Основное уравнение МКТ способно описать даже плазму: состояние вещества, при котором молекулы от ударов друг об друга начинают расщепляется на ионы и электроны.
        \item Основное уравнение МКТ ИГ может быть получено теоретически из модели идеального газа, однако в нем присутствуют микропараметры, поэтому оно не допускает непосредственной экспериментальной проверки.

        \item Все процессы: изохорный, изобарный, изотермный по умолчанию предполагают, что количество вещества в них не изменяется.
        \item При горении, например, водорода в кислороде (2H2+O2-2H2O), не изменяется, ни масса вещества участвующего в реакции, ни его количество.
        Также при этом не изменяется и количество протонов, нейтронов и электронов.
        \item Каждый набор макропараметров идеального газа (P, V и T) задаёт точку в трехмерном пространстве.
        При их изменении образуется линия в этом пространстве.
        Строя графики изопроцессов в координатах PV, VT, PT мы строим проекцию этой линии на одну из плоскостей.
    \end{enumerate}
}

\tasknumber{2}%
\task{%
    Выразите:
    \begin{enumerate}
        \item массу тела через его плотность и объём,
        \item количество вещества через число частиц и число Авогадро,
        \item концентрацию молекул через их число и объём.
    \end{enumerate}
}

\tasknumber{3}%
\task{%
    Напротив каждой физической величины укажите её обозначение и единицы измерения в СИ:
    \begin{enumerate}
        \item температура в Кельвинах,
        \item масса,
        \item постоянная Больцмана.
    \end{enumerate}
}

\tasknumber{4}%
\task{%
    Запишите, как бы вы обозначили...
    \begin{enumerate}
        \item два объема газа: до и после его расширения,
        \item число частиц: до и после утечки газа из баллона,
        \item концентрацию молекул газа в сосуде после сжатия и нагрева,
        \item температуры газа: до и после нагрева в 3 раза.
    \end{enumerate}
}

\tasknumber{5}%
\task{%
    Запишите, какие физические величины соответствуют следующим единицам измерения (указать название и обозначение),
    \begin{enumerate}
        \item кельвин,
        \item МПа,
        \item мДж,
        \item $\frac{ \text{ кг } }{ \text{ м }^3 }$,
        \item $\funits{ кг }{ моль }$.
    \end{enumerate}
}

\tasknumber{6}%
\task{%
    Выразите одну величину через остальные, используя при необходимости постоянную Больцмана, число Авогадро или универсальную газовую постоянную:
    \begin{enumerate}
        \item концентрацию молекул через давление и температуру,
        \item плотность газа через его молярную массу и концентрацию молекул,
        \item число молекул через температуру, давление и объём,
        \item средний квадрат скорости молекул через скорости отдельных молекул,
    \end{enumerate}
}

\end{document}
% autogenerated
