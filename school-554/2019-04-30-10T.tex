\setdate{30~апреля~2019}
\setclass{10«Т»}

\addpersonalvariant{Михаил Бурмистров}

\tasknumber{1}%
\task{%
    В однородном электрическом поле напряжённостью $E = 4\,\frac{\text{кВ}}{\text{м}}$
    переместили заряд $q = 25\,\text{нКл}$ в направлении силовой линии
    на $d = 4\,\text{см}$.
    Определите
    \begin{itemize}
        \item работу поля,
        \item изменение потенциальной энергии заряда.
        % \item напряжение между начальной и конечной точками перемещения.
    \end{itemize}
}
\answer{%
    $
        A   = Eqd
            = 4\,\frac{\text{кВ}}{\text{м}} \cdot 25\,\text{нКл} \cdot 4\,\text{см}
            = 400{,}00 \cdot 10^{-7} \units{Дж}
    $
}
\solutionspace{80pt}

\tasknumber{2}%
\task{%
    Напряжение между двумя точками, лежащими на одной линии напряжённости
    однородного электрического поля, равно $U = 3\,\text{кВ}$.
    Расстояние между точками $l = 30\,\text{см}$.
    Какова напряжённость этого поля?
}
\solutionspace{40pt}

\tasknumber{3}%
\task{%
    Найти напряжение между точками $A$ и $B$ в однородном электрическом поле
    (см.
    рис.
    на доске), если $AB=l = 8\,\text{см}$, $\alpha=45^\circ$,
    $E = 60\,\frac{\text{В}}{\text{м}}$.
    Потенциал какой из точек $A$ и $B$ больше?
}
\solutionspace{120pt}

\tasknumber{4}%
\task{%
    При какой скорости позитрона его кинетическая энергия равна $E_\text{ к } = 8\,\text{эВ}$?
}
\solutionspace{40pt}

\tasknumber{5}%
\task{%
    Электрон $e^-$ вылетает из точки, потенциал которой $\varphi = 1000\,\text{В}$,
    со скоростью $v = 12000000\,\frac{\text{м}}{\text{с}}$ в направлении линий напряжённости поля.
    Будет поле ускорять или тормозить электрон?
    Каков потенциал точки, дойдя до которой электрон остановится?
}

\variantsplitter

\addpersonalvariant{Гагик Аракелян}

\tasknumber{1}%
\task{%
    В однородном электрическом поле напряжённостью $E = 20\,\frac{\text{кВ}}{\text{м}}$
    переместили заряд $Q = -40\,\text{нКл}$ в направлении силовой линии
    на $l = 10\,\text{см}$.
    Определите
    \begin{itemize}
        \item работу поля,
        \item изменение потенциальной энергии заряда.
        % \item напряжение между начальной и конечной точками перемещения.
    \end{itemize}
}
\answer{%
    $
        A   = EQl
            = 20\,\frac{\text{кВ}}{\text{м}} \cdot -40\,\text{нКл} \cdot 10\,\text{см}
            = -8000{,}00 \cdot 10^{-7} \units{Дж}
    $
}
\solutionspace{80pt}

\tasknumber{2}%
\task{%
    Напряжение между двумя точками, лежащими на одной линии напряжённости
    однородного электрического поля, равно $V = 4\,\text{кВ}$.
    Расстояние между точками $d = 40\,\text{см}$.
    Какова напряжённость этого поля?
}
\solutionspace{40pt}

\tasknumber{3}%
\task{%
    Найти напряжение между точками $A$ и $B$ в однородном электрическом поле
    (см.
    рис.
    на доске), если $AB=d = 4\,\text{см}$, $\alpha=30^\circ$,
    $E = 50\,\frac{\text{В}}{\text{м}}$.
    Потенциал какой из точек $A$ и $B$ больше?
}
\solutionspace{120pt}

\tasknumber{4}%
\task{%
    При какой скорости позитрона его кинетическая энергия равна $E_\text{ к } = 20\,\text{эВ}$?
}
\solutionspace{40pt}

\tasknumber{5}%
\task{%
    Электрон $e^-$ вылетает из точки, потенциал которой $\varphi = 800\,\text{В}$,
    со скоростью $v = 3000000\,\frac{\text{м}}{\text{с}}$ в направлении линий напряжённости поля.
    Будет поле ускорять или тормозить электрон?
    Каков потенциал точки, дойдя до которой электрон остановится?
}

\variantsplitter

\addpersonalvariant{Ирен Аракелян}

\tasknumber{1}%
\task{%
    В однородном электрическом поле напряжённостью $E = 2\,\frac{\text{кВ}}{\text{м}}$
    переместили заряд $Q = 25\,\text{нКл}$ в направлении силовой линии
    на $d = 4\,\text{см}$.
    Определите
    \begin{itemize}
        \item работу поля,
        \item изменение потенциальной энергии заряда.
        % \item напряжение между начальной и конечной точками перемещения.
    \end{itemize}
}
\answer{%
    $
        A   = EQd
            = 2\,\frac{\text{кВ}}{\text{м}} \cdot 25\,\text{нКл} \cdot 4\,\text{см}
            = 200{,}00 \cdot 10^{-7} \units{Дж}
    $
}
\solutionspace{80pt}

\tasknumber{2}%
\task{%
    Напряжение между двумя точками, лежащими на одной линии напряжённости
    однородного электрического поля, равно $U = 3\,\text{кВ}$.
    Расстояние между точками $r = 40\,\text{см}$.
    Какова напряжённость этого поля?
}
\solutionspace{40pt}

\tasknumber{3}%
\task{%
    Найти напряжение между точками $A$ и $B$ в однородном электрическом поле
    (см.
    рис.
    на доске), если $AB=r = 8\,\text{см}$, $\alpha=30^\circ$,
    $E = 100\,\frac{\text{В}}{\text{м}}$.
    Потенциал какой из точек $A$ и $B$ больше?
}
\solutionspace{120pt}

\tasknumber{4}%
\task{%
    При какой скорости электрона его кинетическая энергия равна $E_\text{ к } = 200\,\text{эВ}$?
}
\solutionspace{40pt}

\tasknumber{5}%
\task{%
    Электрон $e^-$ вылетает из точки, потенциал которой $\varphi = 800\,\text{В}$,
    со скоростью $v = 3000000\,\frac{\text{м}}{\text{с}}$ в направлении линий напряжённости поля.
    Будет поле ускорять или тормозить электрон?
    Каков потенциал точки, дойдя до которой электрон остановится?
}

\variantsplitter

\addpersonalvariant{Сабина Асадуллаева}

\tasknumber{1}%
\task{%
    В однородном электрическом поле напряжённостью $E = 2\,\frac{\text{кВ}}{\text{м}}$
    переместили заряд $Q = -10\,\text{нКл}$ в направлении силовой линии
    на $r = 2\,\text{см}$.
    Определите
    \begin{itemize}
        \item работу поля,
        \item изменение потенциальной энергии заряда.
        % \item напряжение между начальной и конечной точками перемещения.
    \end{itemize}
}
\answer{%
    $
        A   = EQr
            = 2\,\frac{\text{кВ}}{\text{м}} \cdot -10\,\text{нКл} \cdot 2\,\text{см}
            = -40{,}00 \cdot 10^{-7} \units{Дж}
    $
}
\solutionspace{80pt}

\tasknumber{2}%
\task{%
    Напряжение между двумя точками, лежащими на одной линии напряжённости
    однородного электрического поля, равно $U = 6\,\text{кВ}$.
    Расстояние между точками $d = 10\,\text{см}$.
    Какова напряжённость этого поля?
}
\solutionspace{40pt}

\tasknumber{3}%
\task{%
    Найти напряжение между точками $A$ и $B$ в однородном электрическом поле
    (см.
    рис.
    на доске), если $AB=r = 4\,\text{см}$, $\alpha=30^\circ$,
    $E = 60\,\frac{\text{В}}{\text{м}}$.
    Потенциал какой из точек $A$ и $B$ больше?
}
\solutionspace{120pt}

\tasknumber{4}%
\task{%
    При какой скорости электрона его кинетическая энергия равна $E_\text{ к } = 600\,\text{эВ}$?
}
\solutionspace{40pt}

\tasknumber{5}%
\task{%
    Электрон $e^-$ вылетает из точки, потенциал которой $\varphi = 200\,\text{В}$,
    со скоростью $v = 4000000\,\frac{\text{м}}{\text{с}}$ в направлении линий напряжённости поля.
    Будет поле ускорять или тормозить электрон?
    Каков потенциал точки, дойдя до которой электрон остановится?
}

\variantsplitter

\addpersonalvariant{Вероника Битерякова}

\tasknumber{1}%
\task{%
    В однородном электрическом поле напряжённостью $E = 20\,\frac{\text{кВ}}{\text{м}}$
    переместили заряд $q = 40\,\text{нКл}$ в направлении силовой линии
    на $r = 4\,\text{см}$.
    Определите
    \begin{itemize}
        \item работу поля,
        \item изменение потенциальной энергии заряда.
        % \item напряжение между начальной и конечной точками перемещения.
    \end{itemize}
}
\answer{%
    $
        A   = Eqr
            = 20\,\frac{\text{кВ}}{\text{м}} \cdot 40\,\text{нКл} \cdot 4\,\text{см}
            = 3200{,}00 \cdot 10^{-7} \units{Дж}
    $
}
\solutionspace{80pt}

\tasknumber{2}%
\task{%
    Напряжение между двумя точками, лежащими на одной линии напряжённости
    однородного электрического поля, равно $U = 2\,\text{кВ}$.
    Расстояние между точками $r = 20\,\text{см}$.
    Какова напряжённость этого поля?
}
\solutionspace{40pt}

\tasknumber{3}%
\task{%
    Найти напряжение между точками $A$ и $B$ в однородном электрическом поле
    (см.
    рис.
    на доске), если $AB=l = 6\,\text{см}$, $\varphi=30^\circ$,
    $E = 50\,\frac{\text{В}}{\text{м}}$.
    Потенциал какой из точек $A$ и $B$ больше?
}
\solutionspace{120pt}

\tasknumber{4}%
\task{%
    При какой скорости электрона его кинетическая энергия равна $E_\text{ к } = 200\,\text{эВ}$?
}
\solutionspace{40pt}

\tasknumber{5}%
\task{%
    Электрон $e^-$ вылетает из точки, потенциал которой $\varphi = 1000\,\text{В}$,
    со скоростью $v = 4000000\,\frac{\text{м}}{\text{с}}$ в направлении линий напряжённости поля.
    Будет поле ускорять или тормозить электрон?
    Каков потенциал точки, дойдя до которой электрон остановится?
}

\variantsplitter

\addpersonalvariant{Юлия Буянова}

\tasknumber{1}%
\task{%
    В однородном электрическом поле напряжённостью $E = 2\,\frac{\text{кВ}}{\text{м}}$
    переместили заряд $q = -40\,\text{нКл}$ в направлении силовой линии
    на $r = 5\,\text{см}$.
    Определите
    \begin{itemize}
        \item работу поля,
        \item изменение потенциальной энергии заряда.
        % \item напряжение между начальной и конечной точками перемещения.
    \end{itemize}
}
\answer{%
    $
        A   = Eqr
            = 2\,\frac{\text{кВ}}{\text{м}} \cdot -40\,\text{нКл} \cdot 5\,\text{см}
            = -400{,}00 \cdot 10^{-7} \units{Дж}
    $
}
\solutionspace{80pt}

\tasknumber{2}%
\task{%
    Напряжение между двумя точками, лежащими на одной линии напряжённости
    однородного электрического поля, равно $U = 2\,\text{кВ}$.
    Расстояние между точками $d = 20\,\text{см}$.
    Какова напряжённость этого поля?
}
\solutionspace{40pt}

\tasknumber{3}%
\task{%
    Найти напряжение между точками $A$ и $B$ в однородном электрическом поле
    (см.
    рис.
    на доске), если $AB=d = 4\,\text{см}$, $\alpha=45^\circ$,
    $E = 30\,\frac{\text{В}}{\text{м}}$.
    Потенциал какой из точек $A$ и $B$ больше?
}
\solutionspace{120pt}

\tasknumber{4}%
\task{%
    При какой скорости электрона его кинетическая энергия равна $E_\text{ к } = 200\,\text{эВ}$?
}
\solutionspace{40pt}

\tasknumber{5}%
\task{%
    Электрон $e^-$ вылетает из точки, потенциал которой $\varphi = 600\,\text{В}$,
    со скоростью $v = 10000000\,\frac{\text{м}}{\text{с}}$ в направлении линий напряжённости поля.
    Будет поле ускорять или тормозить электрон?
    Каков потенциал точки, дойдя до которой электрон остановится?
}

\variantsplitter

\addpersonalvariant{Пелагея Вдовина}

\tasknumber{1}%
\task{%
    В однородном электрическом поле напряжённостью $E = 2\,\frac{\text{кВ}}{\text{м}}$
    переместили заряд $q = -25\,\text{нКл}$ в направлении силовой линии
    на $d = 2\,\text{см}$.
    Определите
    \begin{itemize}
        \item работу поля,
        \item изменение потенциальной энергии заряда.
        % \item напряжение между начальной и конечной точками перемещения.
    \end{itemize}
}
\answer{%
    $
        A   = Eqd
            = 2\,\frac{\text{кВ}}{\text{м}} \cdot -25\,\text{нКл} \cdot 2\,\text{см}
            = -100{,}00 \cdot 10^{-7} \units{Дж}
    $
}
\solutionspace{80pt}

\tasknumber{2}%
\task{%
    Напряжение между двумя точками, лежащими на одной линии напряжённости
    однородного электрического поля, равно $U = 4\,\text{кВ}$.
    Расстояние между точками $d = 30\,\text{см}$.
    Какова напряжённость этого поля?
}
\solutionspace{40pt}

\tasknumber{3}%
\task{%
    Найти напряжение между точками $A$ и $B$ в однородном электрическом поле
    (см.
    рис.
    на доске), если $AB=r = 4\,\text{см}$, $\varphi=30^\circ$,
    $E = 100\,\frac{\text{В}}{\text{м}}$.
    Потенциал какой из точек $A$ и $B$ больше?
}
\solutionspace{120pt}

\tasknumber{4}%
\task{%
    При какой скорости электрона его кинетическая энергия равна $E_\text{ к } = 600\,\text{эВ}$?
}
\solutionspace{40pt}

\tasknumber{5}%
\task{%
    Электрон $e^-$ вылетает из точки, потенциал которой $\varphi = 600\,\text{В}$,
    со скоростью $v = 10000000\,\frac{\text{м}}{\text{с}}$ в направлении линий напряжённости поля.
    Будет поле ускорять или тормозить электрон?
    Каков потенциал точки, дойдя до которой электрон остановится?
}

\variantsplitter

\addpersonalvariant{Леонид Викторов}

\tasknumber{1}%
\task{%
    В однородном электрическом поле напряжённостью $E = 2\,\frac{\text{кВ}}{\text{м}}$
    переместили заряд $Q = 25\,\text{нКл}$ в направлении силовой линии
    на $l = 2\,\text{см}$.
    Определите
    \begin{itemize}
        \item работу поля,
        \item изменение потенциальной энергии заряда.
        % \item напряжение между начальной и конечной точками перемещения.
    \end{itemize}
}
\answer{%
    $
        A   = EQl
            = 2\,\frac{\text{кВ}}{\text{м}} \cdot 25\,\text{нКл} \cdot 2\,\text{см}
            = 100{,}00 \cdot 10^{-7} \units{Дж}
    $
}
\solutionspace{80pt}

\tasknumber{2}%
\task{%
    Напряжение между двумя точками, лежащими на одной линии напряжённости
    однородного электрического поля, равно $U = 4\,\text{кВ}$.
    Расстояние между точками $r = 20\,\text{см}$.
    Какова напряжённость этого поля?
}
\solutionspace{40pt}

\tasknumber{3}%
\task{%
    Найти напряжение между точками $A$ и $B$ в однородном электрическом поле
    (см.
    рис.
    на доске), если $AB=r = 6\,\text{см}$, $\alpha=45^\circ$,
    $E = 100\,\frac{\text{В}}{\text{м}}$.
    Потенциал какой из точек $A$ и $B$ больше?
}
\solutionspace{120pt}

\tasknumber{4}%
\task{%
    При какой скорости электрона его кинетическая энергия равна $E_\text{ к } = 50\,\text{эВ}$?
}
\solutionspace{40pt}

\tasknumber{5}%
\task{%
    Электрон $e^-$ вылетает из точки, потенциал которой $\varphi = 400\,\text{В}$,
    со скоростью $v = 6000000\,\frac{\text{м}}{\text{с}}$ в направлении линий напряжённости поля.
    Будет поле ускорять или тормозить электрон?
    Каков потенциал точки, дойдя до которой электрон остановится?
}

\variantsplitter

\addpersonalvariant{Фёдор Гнутов}

\tasknumber{1}%
\task{%
    В однородном электрическом поле напряжённостью $E = 20\,\frac{\text{кВ}}{\text{м}}$
    переместили заряд $q = 10\,\text{нКл}$ в направлении силовой линии
    на $d = 2\,\text{см}$.
    Определите
    \begin{itemize}
        \item работу поля,
        \item изменение потенциальной энергии заряда.
        % \item напряжение между начальной и конечной точками перемещения.
    \end{itemize}
}
\answer{%
    $
        A   = Eqd
            = 20\,\frac{\text{кВ}}{\text{м}} \cdot 10\,\text{нКл} \cdot 2\,\text{см}
            = 400{,}00 \cdot 10^{-7} \units{Дж}
    $
}
\solutionspace{80pt}

\tasknumber{2}%
\task{%
    Напряжение между двумя точками, лежащими на одной линии напряжённости
    однородного электрического поля, равно $U = 4\,\text{кВ}$.
    Расстояние между точками $l = 30\,\text{см}$.
    Какова напряжённость этого поля?
}
\solutionspace{40pt}

\tasknumber{3}%
\task{%
    Найти напряжение между точками $A$ и $B$ в однородном электрическом поле
    (см.
    рис.
    на доске), если $AB=l = 10\,\text{см}$, $\alpha=45^\circ$,
    $E = 50\,\frac{\text{В}}{\text{м}}$.
    Потенциал какой из точек $A$ и $B$ больше?
}
\solutionspace{120pt}

\tasknumber{4}%
\task{%
    При какой скорости позитрона его кинетическая энергия равна $E_\text{ к } = 200\,\text{эВ}$?
}
\solutionspace{40pt}

\tasknumber{5}%
\task{%
    Электрон $e^-$ вылетает из точки, потенциал которой $\varphi = 800\,\text{В}$,
    со скоростью $v = 10000000\,\frac{\text{м}}{\text{с}}$ в направлении линий напряжённости поля.
    Будет поле ускорять или тормозить электрон?
    Каков потенциал точки, дойдя до которой электрон остановится?
}

\variantsplitter

\addpersonalvariant{Илья Гримберг}

\tasknumber{1}%
\task{%
    В однородном электрическом поле напряжённостью $E = 4\,\frac{\text{кВ}}{\text{м}}$
    переместили заряд $q = -10\,\text{нКл}$ в направлении силовой линии
    на $r = 5\,\text{см}$.
    Определите
    \begin{itemize}
        \item работу поля,
        \item изменение потенциальной энергии заряда.
        % \item напряжение между начальной и конечной точками перемещения.
    \end{itemize}
}
\answer{%
    $
        A   = Eqr
            = 4\,\frac{\text{кВ}}{\text{м}} \cdot -10\,\text{нКл} \cdot 5\,\text{см}
            = -200{,}00 \cdot 10^{-7} \units{Дж}
    $
}
\solutionspace{80pt}

\tasknumber{2}%
\task{%
    Напряжение между двумя точками, лежащими на одной линии напряжённости
    однородного электрического поля, равно $V = 2\,\text{кВ}$.
    Расстояние между точками $r = 40\,\text{см}$.
    Какова напряжённость этого поля?
}
\solutionspace{40pt}

\tasknumber{3}%
\task{%
    Найти напряжение между точками $A$ и $B$ в однородном электрическом поле
    (см.
    рис.
    на доске), если $AB=l = 8\,\text{см}$, $\alpha=60^\circ$,
    $E = 30\,\frac{\text{В}}{\text{м}}$.
    Потенциал какой из точек $A$ и $B$ больше?
}
\solutionspace{120pt}

\tasknumber{4}%
\task{%
    При какой скорости электрона его кинетическая энергия равна $E_\text{ к } = 400\,\text{эВ}$?
}
\solutionspace{40pt}

\tasknumber{5}%
\task{%
    Электрон $e^-$ вылетает из точки, потенциал которой $\varphi = 200\,\text{В}$,
    со скоростью $v = 4000000\,\frac{\text{м}}{\text{с}}$ в направлении линий напряжённости поля.
    Будет поле ускорять или тормозить электрон?
    Каков потенциал точки, дойдя до которой электрон остановится?
}

\variantsplitter

\addpersonalvariant{Иван Гурьянов}

\tasknumber{1}%
\task{%
    В однородном электрическом поле напряжённостью $E = 4\,\frac{\text{кВ}}{\text{м}}$
    переместили заряд $q = -25\,\text{нКл}$ в направлении силовой линии
    на $r = 10\,\text{см}$.
    Определите
    \begin{itemize}
        \item работу поля,
        \item изменение потенциальной энергии заряда.
        % \item напряжение между начальной и конечной точками перемещения.
    \end{itemize}
}
\answer{%
    $
        A   = Eqr
            = 4\,\frac{\text{кВ}}{\text{м}} \cdot -25\,\text{нКл} \cdot 10\,\text{см}
            = -1000{,}00 \cdot 10^{-7} \units{Дж}
    $
}
\solutionspace{80pt}

\tasknumber{2}%
\task{%
    Напряжение между двумя точками, лежащими на одной линии напряжённости
    однородного электрического поля, равно $U = 3\,\text{кВ}$.
    Расстояние между точками $d = 30\,\text{см}$.
    Какова напряжённость этого поля?
}
\solutionspace{40pt}

\tasknumber{3}%
\task{%
    Найти напряжение между точками $A$ и $B$ в однородном электрическом поле
    (см.
    рис.
    на доске), если $AB=d = 8\,\text{см}$, $\alpha=60^\circ$,
    $E = 60\,\frac{\text{В}}{\text{м}}$.
    Потенциал какой из точек $A$ и $B$ больше?
}
\solutionspace{120pt}

\tasknumber{4}%
\task{%
    При какой скорости позитрона его кинетическая энергия равна $E_\text{ к } = 20\,\text{эВ}$?
}
\solutionspace{40pt}

\tasknumber{5}%
\task{%
    Электрон $e^-$ вылетает из точки, потенциал которой $\varphi = 800\,\text{В}$,
    со скоростью $v = 10000000\,\frac{\text{м}}{\text{с}}$ в направлении линий напряжённости поля.
    Будет поле ускорять или тормозить электрон?
    Каков потенциал точки, дойдя до которой электрон остановится?
}

\variantsplitter

\addpersonalvariant{Артём Денежкин}

\tasknumber{1}%
\task{%
    В однородном электрическом поле напряжённостью $E = 4\,\frac{\text{кВ}}{\text{м}}$
    переместили заряд $q = 40\,\text{нКл}$ в направлении силовой линии
    на $r = 4\,\text{см}$.
    Определите
    \begin{itemize}
        \item работу поля,
        \item изменение потенциальной энергии заряда.
        % \item напряжение между начальной и конечной точками перемещения.
    \end{itemize}
}
\answer{%
    $
        A   = Eqr
            = 4\,\frac{\text{кВ}}{\text{м}} \cdot 40\,\text{нКл} \cdot 4\,\text{см}
            = 640{,}00 \cdot 10^{-7} \units{Дж}
    $
}
\solutionspace{80pt}

\tasknumber{2}%
\task{%
    Напряжение между двумя точками, лежащими на одной линии напряжённости
    однородного электрического поля, равно $U = 2\,\text{кВ}$.
    Расстояние между точками $l = 10\,\text{см}$.
    Какова напряжённость этого поля?
}
\solutionspace{40pt}

\tasknumber{3}%
\task{%
    Найти напряжение между точками $A$ и $B$ в однородном электрическом поле
    (см.
    рис.
    на доске), если $AB=l = 6\,\text{см}$, $\varphi=45^\circ$,
    $E = 100\,\frac{\text{В}}{\text{м}}$.
    Потенциал какой из точек $A$ и $B$ больше?
}
\solutionspace{120pt}

\tasknumber{4}%
\task{%
    При какой скорости позитрона его кинетическая энергия равна $E_\text{ к } = 600\,\text{эВ}$?
}
\solutionspace{40pt}

\tasknumber{5}%
\task{%
    Электрон $e^-$ вылетает из точки, потенциал которой $\varphi = 800\,\text{В}$,
    со скоростью $v = 4000000\,\frac{\text{м}}{\text{с}}$ в направлении линий напряжённости поля.
    Будет поле ускорять или тормозить электрон?
    Каков потенциал точки, дойдя до которой электрон остановится?
}

\variantsplitter

\addpersonalvariant{Виктор Жилин}

\tasknumber{1}%
\task{%
    В однородном электрическом поле напряжённостью $E = 4\,\frac{\text{кВ}}{\text{м}}$
    переместили заряд $q = 40\,\text{нКл}$ в направлении силовой линии
    на $d = 10\,\text{см}$.
    Определите
    \begin{itemize}
        \item работу поля,
        \item изменение потенциальной энергии заряда.
        % \item напряжение между начальной и конечной точками перемещения.
    \end{itemize}
}
\answer{%
    $
        A   = Eqd
            = 4\,\frac{\text{кВ}}{\text{м}} \cdot 40\,\text{нКл} \cdot 10\,\text{см}
            = 1600{,}00 \cdot 10^{-7} \units{Дж}
    $
}
\solutionspace{80pt}

\tasknumber{2}%
\task{%
    Напряжение между двумя точками, лежащими на одной линии напряжённости
    однородного электрического поля, равно $U = 2\,\text{кВ}$.
    Расстояние между точками $l = 30\,\text{см}$.
    Какова напряжённость этого поля?
}
\solutionspace{40pt}

\tasknumber{3}%
\task{%
    Найти напряжение между точками $A$ и $B$ в однородном электрическом поле
    (см.
    рис.
    на доске), если $AB=r = 6\,\text{см}$, $\alpha=45^\circ$,
    $E = 120\,\frac{\text{В}}{\text{м}}$.
    Потенциал какой из точек $A$ и $B$ больше?
}
\solutionspace{120pt}

\tasknumber{4}%
\task{%
    При какой скорости электрона его кинетическая энергия равна $E_\text{ к } = 600\,\text{эВ}$?
}
\solutionspace{40pt}

\tasknumber{5}%
\task{%
    Электрон $e^-$ вылетает из точки, потенциал которой $\varphi = 600\,\text{В}$,
    со скоростью $v = 12000000\,\frac{\text{м}}{\text{с}}$ в направлении линий напряжённости поля.
    Будет поле ускорять или тормозить электрон?
    Каков потенциал точки, дойдя до которой электрон остановится?
}

\variantsplitter

\addpersonalvariant{Дмитрий Иванов}

\tasknumber{1}%
\task{%
    В однородном электрическом поле напряжённостью $E = 4\,\frac{\text{кВ}}{\text{м}}$
    переместили заряд $Q = 40\,\text{нКл}$ в направлении силовой линии
    на $r = 10\,\text{см}$.
    Определите
    \begin{itemize}
        \item работу поля,
        \item изменение потенциальной энергии заряда.
        % \item напряжение между начальной и конечной точками перемещения.
    \end{itemize}
}
\answer{%
    $
        A   = EQr
            = 4\,\frac{\text{кВ}}{\text{м}} \cdot 40\,\text{нКл} \cdot 10\,\text{см}
            = 1600{,}00 \cdot 10^{-7} \units{Дж}
    $
}
\solutionspace{80pt}

\tasknumber{2}%
\task{%
    Напряжение между двумя точками, лежащими на одной линии напряжённости
    однородного электрического поля, равно $U = 6\,\text{кВ}$.
    Расстояние между точками $d = 20\,\text{см}$.
    Какова напряжённость этого поля?
}
\solutionspace{40pt}

\tasknumber{3}%
\task{%
    Найти напряжение между точками $A$ и $B$ в однородном электрическом поле
    (см.
    рис.
    на доске), если $AB=r = 8\,\text{см}$, $\alpha=30^\circ$,
    $E = 60\,\frac{\text{В}}{\text{м}}$.
    Потенциал какой из точек $A$ и $B$ больше?
}
\solutionspace{120pt}

\tasknumber{4}%
\task{%
    При какой скорости позитрона его кинетическая энергия равна $E_\text{ к } = 30\,\text{эВ}$?
}
\solutionspace{40pt}

\tasknumber{5}%
\task{%
    Электрон $e^-$ вылетает из точки, потенциал которой $\varphi = 800\,\text{В}$,
    со скоростью $v = 4000000\,\frac{\text{м}}{\text{с}}$ в направлении линий напряжённости поля.
    Будет поле ускорять или тормозить электрон?
    Каков потенциал точки, дойдя до которой электрон остановится?
}

\variantsplitter

\addpersonalvariant{Олег Климов}

\tasknumber{1}%
\task{%
    В однородном электрическом поле напряжённостью $E = 2\,\frac{\text{кВ}}{\text{м}}$
    переместили заряд $Q = 10\,\text{нКл}$ в направлении силовой линии
    на $d = 10\,\text{см}$.
    Определите
    \begin{itemize}
        \item работу поля,
        \item изменение потенциальной энергии заряда.
        % \item напряжение между начальной и конечной точками перемещения.
    \end{itemize}
}
\answer{%
    $
        A   = EQd
            = 2\,\frac{\text{кВ}}{\text{м}} \cdot 10\,\text{нКл} \cdot 10\,\text{см}
            = 200{,}00 \cdot 10^{-7} \units{Дж}
    $
}
\solutionspace{80pt}

\tasknumber{2}%
\task{%
    Напряжение между двумя точками, лежащими на одной линии напряжённости
    однородного электрического поля, равно $V = 4\,\text{кВ}$.
    Расстояние между точками $d = 10\,\text{см}$.
    Какова напряжённость этого поля?
}
\solutionspace{40pt}

\tasknumber{3}%
\task{%
    Найти напряжение между точками $A$ и $B$ в однородном электрическом поле
    (см.
    рис.
    на доске), если $AB=l = 4\,\text{см}$, $\alpha=45^\circ$,
    $E = 60\,\frac{\text{В}}{\text{м}}$.
    Потенциал какой из точек $A$ и $B$ больше?
}
\solutionspace{120pt}

\tasknumber{4}%
\task{%
    При какой скорости позитрона его кинетическая энергия равна $E_\text{ к } = 200\,\text{эВ}$?
}
\solutionspace{40pt}

\tasknumber{5}%
\task{%
    Электрон $e^-$ вылетает из точки, потенциал которой $\varphi = 200\,\text{В}$,
    со скоростью $v = 10000000\,\frac{\text{м}}{\text{с}}$ в направлении линий напряжённости поля.
    Будет поле ускорять или тормозить электрон?
    Каков потенциал точки, дойдя до которой электрон остановится?
}

\variantsplitter

\addpersonalvariant{Анна Ковалева}

\tasknumber{1}%
\task{%
    В однородном электрическом поле напряжённостью $E = 2\,\frac{\text{кВ}}{\text{м}}$
    переместили заряд $Q = 10\,\text{нКл}$ в направлении силовой линии
    на $d = 4\,\text{см}$.
    Определите
    \begin{itemize}
        \item работу поля,
        \item изменение потенциальной энергии заряда.
        % \item напряжение между начальной и конечной точками перемещения.
    \end{itemize}
}
\answer{%
    $
        A   = EQd
            = 2\,\frac{\text{кВ}}{\text{м}} \cdot 10\,\text{нКл} \cdot 4\,\text{см}
            = 80{,}00 \cdot 10^{-7} \units{Дж}
    $
}
\solutionspace{80pt}

\tasknumber{2}%
\task{%
    Напряжение между двумя точками, лежащими на одной линии напряжённости
    однородного электрического поля, равно $U = 3\,\text{кВ}$.
    Расстояние между точками $r = 40\,\text{см}$.
    Какова напряжённость этого поля?
}
\solutionspace{40pt}

\tasknumber{3}%
\task{%
    Найти напряжение между точками $A$ и $B$ в однородном электрическом поле
    (см.
    рис.
    на доске), если $AB=r = 8\,\text{см}$, $\varphi=30^\circ$,
    $E = 120\,\frac{\text{В}}{\text{м}}$.
    Потенциал какой из точек $A$ и $B$ больше?
}
\solutionspace{120pt}

\tasknumber{4}%
\task{%
    При какой скорости позитрона его кинетическая энергия равна $E_\text{ к } = 4\,\text{эВ}$?
}
\solutionspace{40pt}

\tasknumber{5}%
\task{%
    Электрон $e^-$ вылетает из точки, потенциал которой $\varphi = 400\,\text{В}$,
    со скоростью $v = 4000000\,\frac{\text{м}}{\text{с}}$ в направлении линий напряжённости поля.
    Будет поле ускорять или тормозить электрон?
    Каков потенциал точки, дойдя до которой электрон остановится?
}

\variantsplitter

\addpersonalvariant{Глеб Ковылин}

\tasknumber{1}%
\task{%
    В однородном электрическом поле напряжённостью $E = 2\,\frac{\text{кВ}}{\text{м}}$
    переместили заряд $Q = 10\,\text{нКл}$ в направлении силовой линии
    на $d = 10\,\text{см}$.
    Определите
    \begin{itemize}
        \item работу поля,
        \item изменение потенциальной энергии заряда.
        % \item напряжение между начальной и конечной точками перемещения.
    \end{itemize}
}
\answer{%
    $
        A   = EQd
            = 2\,\frac{\text{кВ}}{\text{м}} \cdot 10\,\text{нКл} \cdot 10\,\text{см}
            = 200{,}00 \cdot 10^{-7} \units{Дж}
    $
}
\solutionspace{80pt}

\tasknumber{2}%
\task{%
    Напряжение между двумя точками, лежащими на одной линии напряжённости
    однородного электрического поля, равно $U = 3\,\text{кВ}$.
    Расстояние между точками $d = 40\,\text{см}$.
    Какова напряжённость этого поля?
}
\solutionspace{40pt}

\tasknumber{3}%
\task{%
    Найти напряжение между точками $A$ и $B$ в однородном электрическом поле
    (см.
    рис.
    на доске), если $AB=l = 4\,\text{см}$, $\varphi=30^\circ$,
    $E = 30\,\frac{\text{В}}{\text{м}}$.
    Потенциал какой из точек $A$ и $B$ больше?
}
\solutionspace{120pt}

\tasknumber{4}%
\task{%
    При какой скорости электрона его кинетическая энергия равна $E_\text{ к } = 4\,\text{эВ}$?
}
\solutionspace{40pt}

\tasknumber{5}%
\task{%
    Электрон $e^-$ вылетает из точки, потенциал которой $\varphi = 200\,\text{В}$,
    со скоростью $v = 3000000\,\frac{\text{м}}{\text{с}}$ в направлении линий напряжённости поля.
    Будет поле ускорять или тормозить электрон?
    Каков потенциал точки, дойдя до которой электрон остановится?
}

\variantsplitter

\addpersonalvariant{Даниил Космынин}

\tasknumber{1}%
\task{%
    В однородном электрическом поле напряжённостью $E = 4\,\frac{\text{кВ}}{\text{м}}$
    переместили заряд $Q = 40\,\text{нКл}$ в направлении силовой линии
    на $d = 4\,\text{см}$.
    Определите
    \begin{itemize}
        \item работу поля,
        \item изменение потенциальной энергии заряда.
        % \item напряжение между начальной и конечной точками перемещения.
    \end{itemize}
}
\answer{%
    $
        A   = EQd
            = 4\,\frac{\text{кВ}}{\text{м}} \cdot 40\,\text{нКл} \cdot 4\,\text{см}
            = 640{,}00 \cdot 10^{-7} \units{Дж}
    $
}
\solutionspace{80pt}

\tasknumber{2}%
\task{%
    Напряжение между двумя точками, лежащими на одной линии напряжённости
    однородного электрического поля, равно $U = 3\,\text{кВ}$.
    Расстояние между точками $l = 40\,\text{см}$.
    Какова напряжённость этого поля?
}
\solutionspace{40pt}

\tasknumber{3}%
\task{%
    Найти напряжение между точками $A$ и $B$ в однородном электрическом поле
    (см.
    рис.
    на доске), если $AB=d = 10\,\text{см}$, $\alpha=60^\circ$,
    $E = 120\,\frac{\text{В}}{\text{м}}$.
    Потенциал какой из точек $A$ и $B$ больше?
}
\solutionspace{120pt}

\tasknumber{4}%
\task{%
    При какой скорости позитрона его кинетическая энергия равна $E_\text{ к } = 20\,\text{эВ}$?
}
\solutionspace{40pt}

\tasknumber{5}%
\task{%
    Электрон $e^-$ вылетает из точки, потенциал которой $\varphi = 400\,\text{В}$,
    со скоростью $v = 10000000\,\frac{\text{м}}{\text{с}}$ в направлении линий напряжённости поля.
    Будет поле ускорять или тормозить электрон?
    Каков потенциал точки, дойдя до которой электрон остановится?
}

\variantsplitter

\addpersonalvariant{Алина Леоничева}

\tasknumber{1}%
\task{%
    В однородном электрическом поле напряжённостью $E = 20\,\frac{\text{кВ}}{\text{м}}$
    переместили заряд $q = -25\,\text{нКл}$ в направлении силовой линии
    на $l = 4\,\text{см}$.
    Определите
    \begin{itemize}
        \item работу поля,
        \item изменение потенциальной энергии заряда.
        % \item напряжение между начальной и конечной точками перемещения.
    \end{itemize}
}
\answer{%
    $
        A   = Eql
            = 20\,\frac{\text{кВ}}{\text{м}} \cdot -25\,\text{нКл} \cdot 4\,\text{см}
            = -2000{,}00 \cdot 10^{-7} \units{Дж}
    $
}
\solutionspace{80pt}

\tasknumber{2}%
\task{%
    Напряжение между двумя точками, лежащими на одной линии напряжённости
    однородного электрического поля, равно $V = 4\,\text{кВ}$.
    Расстояние между точками $d = 10\,\text{см}$.
    Какова напряжённость этого поля?
}
\solutionspace{40pt}

\tasknumber{3}%
\task{%
    Найти напряжение между точками $A$ и $B$ в однородном электрическом поле
    (см.
    рис.
    на доске), если $AB=r = 10\,\text{см}$, $\alpha=30^\circ$,
    $E = 100\,\frac{\text{В}}{\text{м}}$.
    Потенциал какой из точек $A$ и $B$ больше?
}
\solutionspace{120pt}

\tasknumber{4}%
\task{%
    При какой скорости позитрона его кинетическая энергия равна $E_\text{ к } = 400\,\text{эВ}$?
}
\solutionspace{40pt}

\tasknumber{5}%
\task{%
    Электрон $e^-$ вылетает из точки, потенциал которой $\varphi = 600\,\text{В}$,
    со скоростью $v = 10000000\,\frac{\text{м}}{\text{с}}$ в направлении линий напряжённости поля.
    Будет поле ускорять или тормозить электрон?
    Каков потенциал точки, дойдя до которой электрон остановится?
}

\variantsplitter

\addpersonalvariant{Ирина Лин}

\tasknumber{1}%
\task{%
    В однородном электрическом поле напряжённостью $E = 20\,\frac{\text{кВ}}{\text{м}}$
    переместили заряд $q = -25\,\text{нКл}$ в направлении силовой линии
    на $l = 10\,\text{см}$.
    Определите
    \begin{itemize}
        \item работу поля,
        \item изменение потенциальной энергии заряда.
        % \item напряжение между начальной и конечной точками перемещения.
    \end{itemize}
}
\answer{%
    $
        A   = Eql
            = 20\,\frac{\text{кВ}}{\text{м}} \cdot -25\,\text{нКл} \cdot 10\,\text{см}
            = -5000{,}00 \cdot 10^{-7} \units{Дж}
    $
}
\solutionspace{80pt}

\tasknumber{2}%
\task{%
    Напряжение между двумя точками, лежащими на одной линии напряжённости
    однородного электрического поля, равно $U = 2\,\text{кВ}$.
    Расстояние между точками $d = 30\,\text{см}$.
    Какова напряжённость этого поля?
}
\solutionspace{40pt}

\tasknumber{3}%
\task{%
    Найти напряжение между точками $A$ и $B$ в однородном электрическом поле
    (см.
    рис.
    на доске), если $AB=l = 4\,\text{см}$, $\alpha=30^\circ$,
    $E = 100\,\frac{\text{В}}{\text{м}}$.
    Потенциал какой из точек $A$ и $B$ больше?
}
\solutionspace{120pt}

\tasknumber{4}%
\task{%
    При какой скорости электрона его кинетическая энергия равна $E_\text{ к } = 400\,\text{эВ}$?
}
\solutionspace{40pt}

\tasknumber{5}%
\task{%
    Электрон $e^-$ вылетает из точки, потенциал которой $\varphi = 600\,\text{В}$,
    со скоростью $v = 4000000\,\frac{\text{м}}{\text{с}}$ в направлении линий напряжённости поля.
    Будет поле ускорять или тормозить электрон?
    Каков потенциал точки, дойдя до которой электрон остановится?
}

\variantsplitter

\addpersonalvariant{Олег Мальцев}

\tasknumber{1}%
\task{%
    В однородном электрическом поле напряжённостью $E = 4\,\frac{\text{кВ}}{\text{м}}$
    переместили заряд $q = 10\,\text{нКл}$ в направлении силовой линии
    на $d = 10\,\text{см}$.
    Определите
    \begin{itemize}
        \item работу поля,
        \item изменение потенциальной энергии заряда.
        % \item напряжение между начальной и конечной точками перемещения.
    \end{itemize}
}
\answer{%
    $
        A   = Eqd
            = 4\,\frac{\text{кВ}}{\text{м}} \cdot 10\,\text{нКл} \cdot 10\,\text{см}
            = 400{,}00 \cdot 10^{-7} \units{Дж}
    $
}
\solutionspace{80pt}

\tasknumber{2}%
\task{%
    Напряжение между двумя точками, лежащими на одной линии напряжённости
    однородного электрического поля, равно $U = 3\,\text{кВ}$.
    Расстояние между точками $d = 40\,\text{см}$.
    Какова напряжённость этого поля?
}
\solutionspace{40pt}

\tasknumber{3}%
\task{%
    Найти напряжение между точками $A$ и $B$ в однородном электрическом поле
    (см.
    рис.
    на доске), если $AB=l = 4\,\text{см}$, $\alpha=60^\circ$,
    $E = 60\,\frac{\text{В}}{\text{м}}$.
    Потенциал какой из точек $A$ и $B$ больше?
}
\solutionspace{120pt}

\tasknumber{4}%
\task{%
    При какой скорости позитрона его кинетическая энергия равна $E_\text{ к } = 30\,\text{эВ}$?
}
\solutionspace{40pt}

\tasknumber{5}%
\task{%
    Электрон $e^-$ вылетает из точки, потенциал которой $\varphi = 400\,\text{В}$,
    со скоростью $v = 4000000\,\frac{\text{м}}{\text{с}}$ в направлении линий напряжённости поля.
    Будет поле ускорять или тормозить электрон?
    Каков потенциал точки, дойдя до которой электрон остановится?
}

\variantsplitter

\addpersonalvariant{Ислам Мунаев}

\tasknumber{1}%
\task{%
    В однородном электрическом поле напряжённостью $E = 2\,\frac{\text{кВ}}{\text{м}}$
    переместили заряд $Q = -40\,\text{нКл}$ в направлении силовой линии
    на $r = 2\,\text{см}$.
    Определите
    \begin{itemize}
        \item работу поля,
        \item изменение потенциальной энергии заряда.
        % \item напряжение между начальной и конечной точками перемещения.
    \end{itemize}
}
\answer{%
    $
        A   = EQr
            = 2\,\frac{\text{кВ}}{\text{м}} \cdot -40\,\text{нКл} \cdot 2\,\text{см}
            = -160{,}00 \cdot 10^{-7} \units{Дж}
    $
}
\solutionspace{80pt}

\tasknumber{2}%
\task{%
    Напряжение между двумя точками, лежащими на одной линии напряжённости
    однородного электрического поля, равно $V = 4\,\text{кВ}$.
    Расстояние между точками $l = 10\,\text{см}$.
    Какова напряжённость этого поля?
}
\solutionspace{40pt}

\tasknumber{3}%
\task{%
    Найти напряжение между точками $A$ и $B$ в однородном электрическом поле
    (см.
    рис.
    на доске), если $AB=r = 6\,\text{см}$, $\alpha=30^\circ$,
    $E = 100\,\frac{\text{В}}{\text{м}}$.
    Потенциал какой из точек $A$ и $B$ больше?
}
\solutionspace{120pt}

\tasknumber{4}%
\task{%
    При какой скорости электрона его кинетическая энергия равна $E_\text{ к } = 1000\,\text{эВ}$?
}
\solutionspace{40pt}

\tasknumber{5}%
\task{%
    Электрон $e^-$ вылетает из точки, потенциал которой $\varphi = 600\,\text{В}$,
    со скоростью $v = 12000000\,\frac{\text{м}}{\text{с}}$ в направлении линий напряжённости поля.
    Будет поле ускорять или тормозить электрон?
    Каков потенциал точки, дойдя до которой электрон остановится?
}

\variantsplitter

\addpersonalvariant{Александр Наумов}

\tasknumber{1}%
\task{%
    В однородном электрическом поле напряжённостью $E = 2\,\frac{\text{кВ}}{\text{м}}$
    переместили заряд $Q = -40\,\text{нКл}$ в направлении силовой линии
    на $d = 2\,\text{см}$.
    Определите
    \begin{itemize}
        \item работу поля,
        \item изменение потенциальной энергии заряда.
        % \item напряжение между начальной и конечной точками перемещения.
    \end{itemize}
}
\answer{%
    $
        A   = EQd
            = 2\,\frac{\text{кВ}}{\text{м}} \cdot -40\,\text{нКл} \cdot 2\,\text{см}
            = -160{,}00 \cdot 10^{-7} \units{Дж}
    $
}
\solutionspace{80pt}

\tasknumber{2}%
\task{%
    Напряжение между двумя точками, лежащими на одной линии напряжённости
    однородного электрического поля, равно $U = 2\,\text{кВ}$.
    Расстояние между точками $r = 20\,\text{см}$.
    Какова напряжённость этого поля?
}
\solutionspace{40pt}

\tasknumber{3}%
\task{%
    Найти напряжение между точками $A$ и $B$ в однородном электрическом поле
    (см.
    рис.
    на доске), если $AB=r = 10\,\text{см}$, $\alpha=60^\circ$,
    $E = 60\,\frac{\text{В}}{\text{м}}$.
    Потенциал какой из точек $A$ и $B$ больше?
}
\solutionspace{120pt}

\tasknumber{4}%
\task{%
    При какой скорости электрона его кинетическая энергия равна $E_\text{ к } = 1000\,\text{эВ}$?
}
\solutionspace{40pt}

\tasknumber{5}%
\task{%
    Электрон $e^-$ вылетает из точки, потенциал которой $\varphi = 1000\,\text{В}$,
    со скоростью $v = 10000000\,\frac{\text{м}}{\text{с}}$ в направлении линий напряжённости поля.
    Будет поле ускорять или тормозить электрон?
    Каков потенциал точки, дойдя до которой электрон остановится?
}

\variantsplitter

\addpersonalvariant{Георгий Новиков}

\tasknumber{1}%
\task{%
    В однородном электрическом поле напряжённостью $E = 20\,\frac{\text{кВ}}{\text{м}}$
    переместили заряд $Q = 10\,\text{нКл}$ в направлении силовой линии
    на $l = 4\,\text{см}$.
    Определите
    \begin{itemize}
        \item работу поля,
        \item изменение потенциальной энергии заряда.
        % \item напряжение между начальной и конечной точками перемещения.
    \end{itemize}
}
\answer{%
    $
        A   = EQl
            = 20\,\frac{\text{кВ}}{\text{м}} \cdot 10\,\text{нКл} \cdot 4\,\text{см}
            = 800{,}00 \cdot 10^{-7} \units{Дж}
    $
}
\solutionspace{80pt}

\tasknumber{2}%
\task{%
    Напряжение между двумя точками, лежащими на одной линии напряжённости
    однородного электрического поля, равно $V = 6\,\text{кВ}$.
    Расстояние между точками $l = 20\,\text{см}$.
    Какова напряжённость этого поля?
}
\solutionspace{40pt}

\tasknumber{3}%
\task{%
    Найти напряжение между точками $A$ и $B$ в однородном электрическом поле
    (см.
    рис.
    на доске), если $AB=r = 4\,\text{см}$, $\varphi=30^\circ$,
    $E = 50\,\frac{\text{В}}{\text{м}}$.
    Потенциал какой из точек $A$ и $B$ больше?
}
\solutionspace{120pt}

\tasknumber{4}%
\task{%
    При какой скорости электрона его кинетическая энергия равна $E_\text{ к } = 30\,\text{эВ}$?
}
\solutionspace{40pt}

\tasknumber{5}%
\task{%
    Электрон $e^-$ вылетает из точки, потенциал которой $\varphi = 600\,\text{В}$,
    со скоростью $v = 6000000\,\frac{\text{м}}{\text{с}}$ в направлении линий напряжённости поля.
    Будет поле ускорять или тормозить электрон?
    Каков потенциал точки, дойдя до которой электрон остановится?
}

\variantsplitter

\addpersonalvariant{Егор Осипов}

\tasknumber{1}%
\task{%
    В однородном электрическом поле напряжённостью $E = 2\,\frac{\text{кВ}}{\text{м}}$
    переместили заряд $q = -25\,\text{нКл}$ в направлении силовой линии
    на $r = 2\,\text{см}$.
    Определите
    \begin{itemize}
        \item работу поля,
        \item изменение потенциальной энергии заряда.
        % \item напряжение между начальной и конечной точками перемещения.
    \end{itemize}
}
\answer{%
    $
        A   = Eqr
            = 2\,\frac{\text{кВ}}{\text{м}} \cdot -25\,\text{нКл} \cdot 2\,\text{см}
            = -100{,}00 \cdot 10^{-7} \units{Дж}
    $
}
\solutionspace{80pt}

\tasknumber{2}%
\task{%
    Напряжение между двумя точками, лежащими на одной линии напряжённости
    однородного электрического поля, равно $U = 4\,\text{кВ}$.
    Расстояние между точками $r = 10\,\text{см}$.
    Какова напряжённость этого поля?
}
\solutionspace{40pt}

\tasknumber{3}%
\task{%
    Найти напряжение между точками $A$ и $B$ в однородном электрическом поле
    (см.
    рис.
    на доске), если $AB=d = 6\,\text{см}$, $\alpha=45^\circ$,
    $E = 60\,\frac{\text{В}}{\text{м}}$.
    Потенциал какой из точек $A$ и $B$ больше?
}
\solutionspace{120pt}

\tasknumber{4}%
\task{%
    При какой скорости электрона его кинетическая энергия равна $E_\text{ к } = 4\,\text{эВ}$?
}
\solutionspace{40pt}

\tasknumber{5}%
\task{%
    Электрон $e^-$ вылетает из точки, потенциал которой $\varphi = 800\,\text{В}$,
    со скоростью $v = 10000000\,\frac{\text{м}}{\text{с}}$ в направлении линий напряжённости поля.
    Будет поле ускорять или тормозить электрон?
    Каков потенциал точки, дойдя до которой электрон остановится?
}

\variantsplitter

\addpersonalvariant{Руслан Перепелица}

\tasknumber{1}%
\task{%
    В однородном электрическом поле напряжённостью $E = 20\,\frac{\text{кВ}}{\text{м}}$
    переместили заряд $Q = 25\,\text{нКл}$ в направлении силовой линии
    на $r = 10\,\text{см}$.
    Определите
    \begin{itemize}
        \item работу поля,
        \item изменение потенциальной энергии заряда.
        % \item напряжение между начальной и конечной точками перемещения.
    \end{itemize}
}
\answer{%
    $
        A   = EQr
            = 20\,\frac{\text{кВ}}{\text{м}} \cdot 25\,\text{нКл} \cdot 10\,\text{см}
            = 5000{,}00 \cdot 10^{-7} \units{Дж}
    $
}
\solutionspace{80pt}

\tasknumber{2}%
\task{%
    Напряжение между двумя точками, лежащими на одной линии напряжённости
    однородного электрического поля, равно $V = 3\,\text{кВ}$.
    Расстояние между точками $l = 40\,\text{см}$.
    Какова напряжённость этого поля?
}
\solutionspace{40pt}

\tasknumber{3}%
\task{%
    Найти напряжение между точками $A$ и $B$ в однородном электрическом поле
    (см.
    рис.
    на доске), если $AB=l = 12\,\text{см}$, $\varphi=45^\circ$,
    $E = 100\,\frac{\text{В}}{\text{м}}$.
    Потенциал какой из точек $A$ и $B$ больше?
}
\solutionspace{120pt}

\tasknumber{4}%
\task{%
    При какой скорости электрона его кинетическая энергия равна $E_\text{ к } = 20\,\text{эВ}$?
}
\solutionspace{40pt}

\tasknumber{5}%
\task{%
    Электрон $e^-$ вылетает из точки, потенциал которой $\varphi = 600\,\text{В}$,
    со скоростью $v = 3000000\,\frac{\text{м}}{\text{с}}$ в направлении линий напряжённости поля.
    Будет поле ускорять или тормозить электрон?
    Каков потенциал точки, дойдя до которой электрон остановится?
}

\variantsplitter

\addpersonalvariant{Михаил Перин}

\tasknumber{1}%
\task{%
    В однородном электрическом поле напряжённостью $E = 20\,\frac{\text{кВ}}{\text{м}}$
    переместили заряд $q = 40\,\text{нКл}$ в направлении силовой линии
    на $r = 2\,\text{см}$.
    Определите
    \begin{itemize}
        \item работу поля,
        \item изменение потенциальной энергии заряда.
        % \item напряжение между начальной и конечной точками перемещения.
    \end{itemize}
}
\answer{%
    $
        A   = Eqr
            = 20\,\frac{\text{кВ}}{\text{м}} \cdot 40\,\text{нКл} \cdot 2\,\text{см}
            = 1600{,}00 \cdot 10^{-7} \units{Дж}
    $
}
\solutionspace{80pt}

\tasknumber{2}%
\task{%
    Напряжение между двумя точками, лежащими на одной линии напряжённости
    однородного электрического поля, равно $V = 4\,\text{кВ}$.
    Расстояние между точками $r = 40\,\text{см}$.
    Какова напряжённость этого поля?
}
\solutionspace{40pt}

\tasknumber{3}%
\task{%
    Найти напряжение между точками $A$ и $B$ в однородном электрическом поле
    (см.
    рис.
    на доске), если $AB=d = 4\,\text{см}$, $\alpha=45^\circ$,
    $E = 50\,\frac{\text{В}}{\text{м}}$.
    Потенциал какой из точек $A$ и $B$ больше?
}
\solutionspace{120pt}

\tasknumber{4}%
\task{%
    При какой скорости позитрона его кинетическая энергия равна $E_\text{ к } = 400\,\text{эВ}$?
}
\solutionspace{40pt}

\tasknumber{5}%
\task{%
    Электрон $e^-$ вылетает из точки, потенциал которой $\varphi = 400\,\text{В}$,
    со скоростью $v = 6000000\,\frac{\text{м}}{\text{с}}$ в направлении линий напряжённости поля.
    Будет поле ускорять или тормозить электрон?
    Каков потенциал точки, дойдя до которой электрон остановится?
}

\variantsplitter

\addpersonalvariant{Егор Подуровский}

\tasknumber{1}%
\task{%
    В однородном электрическом поле напряжённостью $E = 2\,\frac{\text{кВ}}{\text{м}}$
    переместили заряд $Q = -40\,\text{нКл}$ в направлении силовой линии
    на $r = 2\,\text{см}$.
    Определите
    \begin{itemize}
        \item работу поля,
        \item изменение потенциальной энергии заряда.
        % \item напряжение между начальной и конечной точками перемещения.
    \end{itemize}
}
\answer{%
    $
        A   = EQr
            = 2\,\frac{\text{кВ}}{\text{м}} \cdot -40\,\text{нКл} \cdot 2\,\text{см}
            = -160{,}00 \cdot 10^{-7} \units{Дж}
    $
}
\solutionspace{80pt}

\tasknumber{2}%
\task{%
    Напряжение между двумя точками, лежащими на одной линии напряжённости
    однородного электрического поля, равно $U = 3\,\text{кВ}$.
    Расстояние между точками $d = 20\,\text{см}$.
    Какова напряжённость этого поля?
}
\solutionspace{40pt}

\tasknumber{3}%
\task{%
    Найти напряжение между точками $A$ и $B$ в однородном электрическом поле
    (см.
    рис.
    на доске), если $AB=d = 4\,\text{см}$, $\varphi=45^\circ$,
    $E = 100\,\frac{\text{В}}{\text{м}}$.
    Потенциал какой из точек $A$ и $B$ больше?
}
\solutionspace{120pt}

\tasknumber{4}%
\task{%
    При какой скорости электрона его кинетическая энергия равна $E_\text{ к } = 20\,\text{эВ}$?
}
\solutionspace{40pt}

\tasknumber{5}%
\task{%
    Электрон $e^-$ вылетает из точки, потенциал которой $\varphi = 400\,\text{В}$,
    со скоростью $v = 4000000\,\frac{\text{м}}{\text{с}}$ в направлении линий напряжённости поля.
    Будет поле ускорять или тормозить электрон?
    Каков потенциал точки, дойдя до которой электрон остановится?
}

\variantsplitter

\addpersonalvariant{Роман Прибылов}

\tasknumber{1}%
\task{%
    В однородном электрическом поле напряжённостью $E = 2\,\frac{\text{кВ}}{\text{м}}$
    переместили заряд $Q = -40\,\text{нКл}$ в направлении силовой линии
    на $l = 10\,\text{см}$.
    Определите
    \begin{itemize}
        \item работу поля,
        \item изменение потенциальной энергии заряда.
        % \item напряжение между начальной и конечной точками перемещения.
    \end{itemize}
}
\answer{%
    $
        A   = EQl
            = 2\,\frac{\text{кВ}}{\text{м}} \cdot -40\,\text{нКл} \cdot 10\,\text{см}
            = -800{,}00 \cdot 10^{-7} \units{Дж}
    $
}
\solutionspace{80pt}

\tasknumber{2}%
\task{%
    Напряжение между двумя точками, лежащими на одной линии напряжённости
    однородного электрического поля, равно $U = 6\,\text{кВ}$.
    Расстояние между точками $l = 20\,\text{см}$.
    Какова напряжённость этого поля?
}
\solutionspace{40pt}

\tasknumber{3}%
\task{%
    Найти напряжение между точками $A$ и $B$ в однородном электрическом поле
    (см.
    рис.
    на доске), если $AB=r = 10\,\text{см}$, $\alpha=30^\circ$,
    $E = 50\,\frac{\text{В}}{\text{м}}$.
    Потенциал какой из точек $A$ и $B$ больше?
}
\solutionspace{120pt}

\tasknumber{4}%
\task{%
    При какой скорости позитрона его кинетическая энергия равна $E_\text{ к } = 4\,\text{эВ}$?
}
\solutionspace{40pt}

\tasknumber{5}%
\task{%
    Электрон $e^-$ вылетает из точки, потенциал которой $\varphi = 400\,\text{В}$,
    со скоростью $v = 4000000\,\frac{\text{м}}{\text{с}}$ в направлении линий напряжённости поля.
    Будет поле ускорять или тормозить электрон?
    Каков потенциал точки, дойдя до которой электрон остановится?
}

\variantsplitter

\addpersonalvariant{Александр Селехметьев}

\tasknumber{1}%
\task{%
    В однородном электрическом поле напряжённостью $E = 20\,\frac{\text{кВ}}{\text{м}}$
    переместили заряд $q = -10\,\text{нКл}$ в направлении силовой линии
    на $l = 5\,\text{см}$.
    Определите
    \begin{itemize}
        \item работу поля,
        \item изменение потенциальной энергии заряда.
        % \item напряжение между начальной и конечной точками перемещения.
    \end{itemize}
}
\answer{%
    $
        A   = Eql
            = 20\,\frac{\text{кВ}}{\text{м}} \cdot -10\,\text{нКл} \cdot 5\,\text{см}
            = -1000{,}00 \cdot 10^{-7} \units{Дж}
    $
}
\solutionspace{80pt}

\tasknumber{2}%
\task{%
    Напряжение между двумя точками, лежащими на одной линии напряжённости
    однородного электрического поля, равно $V = 3\,\text{кВ}$.
    Расстояние между точками $l = 40\,\text{см}$.
    Какова напряжённость этого поля?
}
\solutionspace{40pt}

\tasknumber{3}%
\task{%
    Найти напряжение между точками $A$ и $B$ в однородном электрическом поле
    (см.
    рис.
    на доске), если $AB=r = 6\,\text{см}$, $\varphi=45^\circ$,
    $E = 50\,\frac{\text{В}}{\text{м}}$.
    Потенциал какой из точек $A$ и $B$ больше?
}
\solutionspace{120pt}

\tasknumber{4}%
\task{%
    При какой скорости электрона его кинетическая энергия равна $E_\text{ к } = 1000\,\text{эВ}$?
}
\solutionspace{40pt}

\tasknumber{5}%
\task{%
    Электрон $e^-$ вылетает из точки, потенциал которой $\varphi = 600\,\text{В}$,
    со скоростью $v = 6000000\,\frac{\text{м}}{\text{с}}$ в направлении линий напряжённости поля.
    Будет поле ускорять или тормозить электрон?
    Каков потенциал точки, дойдя до которой электрон остановится?
}

\variantsplitter

\addpersonalvariant{Алексей Тихонов}

\tasknumber{1}%
\task{%
    В однородном электрическом поле напряжённостью $E = 2\,\frac{\text{кВ}}{\text{м}}$
    переместили заряд $q = 10\,\text{нКл}$ в направлении силовой линии
    на $l = 2\,\text{см}$.
    Определите
    \begin{itemize}
        \item работу поля,
        \item изменение потенциальной энергии заряда.
        % \item напряжение между начальной и конечной точками перемещения.
    \end{itemize}
}
\answer{%
    $
        A   = Eql
            = 2\,\frac{\text{кВ}}{\text{м}} \cdot 10\,\text{нКл} \cdot 2\,\text{см}
            = 40{,}00 \cdot 10^{-7} \units{Дж}
    $
}
\solutionspace{80pt}

\tasknumber{2}%
\task{%
    Напряжение между двумя точками, лежащими на одной линии напряжённости
    однородного электрического поля, равно $U = 6\,\text{кВ}$.
    Расстояние между точками $d = 20\,\text{см}$.
    Какова напряжённость этого поля?
}
\solutionspace{40pt}

\tasknumber{3}%
\task{%
    Найти напряжение между точками $A$ и $B$ в однородном электрическом поле
    (см.
    рис.
    на доске), если $AB=r = 12\,\text{см}$, $\alpha=30^\circ$,
    $E = 50\,\frac{\text{В}}{\text{м}}$.
    Потенциал какой из точек $A$ и $B$ больше?
}
\solutionspace{120pt}

\tasknumber{4}%
\task{%
    При какой скорости электрона его кинетическая энергия равна $E_\text{ к } = 4\,\text{эВ}$?
}
\solutionspace{40pt}

\tasknumber{5}%
\task{%
    Электрон $e^-$ вылетает из точки, потенциал которой $\varphi = 600\,\text{В}$,
    со скоростью $v = 6000000\,\frac{\text{м}}{\text{с}}$ в направлении линий напряжённости поля.
    Будет поле ускорять или тормозить электрон?
    Каков потенциал точки, дойдя до которой электрон остановится?
}

\variantsplitter

\addpersonalvariant{Алина Филиппова}

\tasknumber{1}%
\task{%
    В однородном электрическом поле напряжённостью $E = 2\,\frac{\text{кВ}}{\text{м}}$
    переместили заряд $q = 40\,\text{нКл}$ в направлении силовой линии
    на $l = 5\,\text{см}$.
    Определите
    \begin{itemize}
        \item работу поля,
        \item изменение потенциальной энергии заряда.
        % \item напряжение между начальной и конечной точками перемещения.
    \end{itemize}
}
\answer{%
    $
        A   = Eql
            = 2\,\frac{\text{кВ}}{\text{м}} \cdot 40\,\text{нКл} \cdot 5\,\text{см}
            = 400{,}00 \cdot 10^{-7} \units{Дж}
    $
}
\solutionspace{80pt}

\tasknumber{2}%
\task{%
    Напряжение между двумя точками, лежащими на одной линии напряжённости
    однородного электрического поля, равно $U = 2\,\text{кВ}$.
    Расстояние между точками $d = 30\,\text{см}$.
    Какова напряжённость этого поля?
}
\solutionspace{40pt}

\tasknumber{3}%
\task{%
    Найти напряжение между точками $A$ и $B$ в однородном электрическом поле
    (см.
    рис.
    на доске), если $AB=d = 12\,\text{см}$, $\varphi=30^\circ$,
    $E = 50\,\frac{\text{В}}{\text{м}}$.
    Потенциал какой из точек $A$ и $B$ больше?
}
\solutionspace{120pt}

\tasknumber{4}%
\task{%
    При какой скорости позитрона его кинетическая энергия равна $E_\text{ к } = 50\,\text{эВ}$?
}
\solutionspace{40pt}

\tasknumber{5}%
\task{%
    Электрон $e^-$ вылетает из точки, потенциал которой $\varphi = 400\,\text{В}$,
    со скоростью $v = 6000000\,\frac{\text{м}}{\text{с}}$ в направлении линий напряжённости поля.
    Будет поле ускорять или тормозить электрон?
    Каков потенциал точки, дойдя до которой электрон остановится?
}

\variantsplitter

\addpersonalvariant{Алина Яшина}

\tasknumber{1}%
\task{%
    В однородном электрическом поле напряжённостью $E = 2\,\frac{\text{кВ}}{\text{м}}$
    переместили заряд $q = 10\,\text{нКл}$ в направлении силовой линии
    на $d = 2\,\text{см}$.
    Определите
    \begin{itemize}
        \item работу поля,
        \item изменение потенциальной энергии заряда.
        % \item напряжение между начальной и конечной точками перемещения.
    \end{itemize}
}
\answer{%
    $
        A   = Eqd
            = 2\,\frac{\text{кВ}}{\text{м}} \cdot 10\,\text{нКл} \cdot 2\,\text{см}
            = 40{,}00 \cdot 10^{-7} \units{Дж}
    $
}
\solutionspace{80pt}

\tasknumber{2}%
\task{%
    Напряжение между двумя точками, лежащими на одной линии напряжённости
    однородного электрического поля, равно $V = 4\,\text{кВ}$.
    Расстояние между точками $d = 20\,\text{см}$.
    Какова напряжённость этого поля?
}
\solutionspace{40pt}

\tasknumber{3}%
\task{%
    Найти напряжение между точками $A$ и $B$ в однородном электрическом поле
    (см.
    рис.
    на доске), если $AB=d = 6\,\text{см}$, $\varphi=30^\circ$,
    $E = 120\,\frac{\text{В}}{\text{м}}$.
    Потенциал какой из точек $A$ и $B$ больше?
}
\solutionspace{120pt}

\tasknumber{4}%
\task{%
    При какой скорости позитрона его кинетическая энергия равна $E_\text{ к } = 20\,\text{эВ}$?
}
\solutionspace{40pt}

\tasknumber{5}%
\task{%
    Электрон $e^-$ вылетает из точки, потенциал которой $\varphi = 400\,\text{В}$,
    со скоростью $v = 10000000\,\frac{\text{м}}{\text{с}}$ в направлении линий напряжённости поля.
    Будет поле ускорять или тормозить электрон?
    Каков потенциал точки, дойдя до которой электрон остановится?
}
% autogenerated
