\setdate{20~марта~2021}
\setclass{10«АБ»}

\addpersonalvariant{Михаил Бурмистров}

\tasknumber{1}%
\task{%
    Определите КПД (оставив ответ точным в виде нескоратимой дроби) цикла 1231, рабочим телом которого является идеальный одноатомный газ, если
    \begin{itemize}
        \item 12 — изохорический нагрев в шесть раз,
        \item 23 — изобарическое расширение, при котором температура растёт в три раза,
        \item 31 — процесс, график которого в $PV$-координатах является отрезком прямой.
    \end{itemize}
}
\answer{%
    \begin{align*}
    A_{12} &= 0, \Delta U_{12} > 0, \implies Q_{12} = A_{12} + \Delta U_{12} > 0.
    \\
    A_{23} &> 0, \Delta U_{23} > 0, \implies Q_{23} = A_{23} + \Delta U_{23} > 0, \\
    A_{31} &= 0, \Delta U_{31} < 0, \implies Q_{31} = A_{31} + \Delta U_{31} < 0.
    \\
    P_1V_1 &= \nu R T_1, P_2V_2 = \nu R T_2, P_3V_3 = \nu R T_3 \text{ — уравнения состояния идеального газа}, \\
    &\text{Пусть $P_0$, $V_0$, $T_0$ — давление, объём и температура в точке 1 (минимальные во всём цикле):} \\
    P_1 &= P_0, P_2 = P_3, V_1 = V_2 = V_0, \text{остальные соотношения нужно считать} \\
    T_2 &= 6T_1 = 6T_0 \text{(по условию)} \implies \frac{P_2}{P_1} = \frac{P_2V_0}{P_1V_0} = \frac{P_2 V_2}{P_1 V_1}= \frac{\nu R T_2}{\nu R T_1} = \frac{T_2}{T_1} = 6 \implies P_2 = 6 P_1 = 6 P_0, \\
    T_3 &= 3T_2 = 18T_0 \text{(по условию)} \implies \frac{V_3}{V_2} = \frac{P_3V_3}{P_2V_2}= \frac{\nu R T_3}{\nu R T_2} = \frac{T_3}{T_2} = 3 \implies V_3 = 3 V_2 = 3 V_0.
    \\
    A_\text{цикл} &= \frac 12 (3P_0 - P_0)(6V_0 - V_0) = \frac 12 \cdot 10 \cdot P_0V_0, \\
    A_{23} &= 6P_0 \cdot (3V_0 - V_0) = 12P_0V_0, \\
    \Delta U_{23} &= \frac 32 \nu R T_3 - \frac 32 \nu R T_3 = \frac 32 P_3 V_3 - \frac 32 P_2 V_2 = \frac 32 \cdot 6 P_0 \cdot 3 V_0 -  \frac 32 \cdot 6 P_0 \cdot V_0 = \frac 32 \cdot 12 \cdot P_0V_0, \\
    \Delta U_{12} &= \frac 32 \nu R T_2 - \frac 32 \nu R T_1 = \frac 32 P_2 V_2 - \frac 32 P_1 V_1 = \frac 32 \cdot 6 P_0V_0 - \frac 32 P_0V_0 = \frac 32 \cdot 5 \cdot P_0V_0.
    \\
    \eta &= \frac{A_\text{цикл}}{Q_+} = \frac{A_\text{цикл}}{Q_{12} + Q_{23}}  = \frac{A_\text{цикл}}{A_{12} + \Delta U_{12} + A_{23} + \Delta U_{23}} =  \\
     &= \frac{\frac 12 \cdot 10 \cdot P_0V_0}{0 + \frac 32 \cdot 5 \cdot P_0V_0 + 12P_0V_0 + \frac 32 \cdot 12 \cdot P_0V_0} = \frac{\frac 12 \cdot 10}{\frac 32 \cdot 5 + 12 + \frac 32 \cdot 12} = \frac{2}{15} \approx 0{,}133.
    \end{align*}
}
\solutionspace{360pt}

\tasknumber{2}%
\task{%
    Определите КПД (оставив ответ точным в виде нескоратимой дроби) цикла 1231, рабочим телом которого является идеальный одноатомный газ, если
    \begin{itemize}
        \item 12 — изобарическое расширение,
        \item 23 — процесс, график которого в $PV$-координатах является отрезком прямой, а объём уменьшается в три раза,
        \item 31 — изохорический нагрев с увеличением давления в шесть раз,
    \end{itemize}
}
\answer{%
    \begin{align*}
    A_{12} &> 0, \Delta U_{12} > 0, \implies Q_{12} = A_{12} + \Delta U_{12} > 0.
    \\
    A_{23} &< 0, \Delta U_{23} < 0, \implies Q_{23} = A_{23} + \Delta U_{23} < 0, \\
    A_{31} &= 0, \Delta U_{31} > 0, \implies Q_{31} = A_{31} + \Delta U_{31} > 0.
    \\
    P_1V_1 &= \nu R T_1, P_2V_2 = \nu R T_2, P_3V_3 = \nu R T_3 \text{ — уравнения состояния идеального газа}, \\
    &\text{Пусть $P_0$, $V_0$, $T_0$ — давление, объём и температура в точке 3 (минимальные во всём цикле):} \\
    P_3 &= P_0, P_1 = P_2 = 6P_0, V_1 = V_3 = V_0, V_2 = 3V_3 = 3V_0 \\
    A_\text{цикл} &= \frac 12 (P_2-P_1)(V_1-V_2) = \frac 12 (6P_0 - P_0)(3V_0 - V_0) = \frac 12 \cdot 10 \cdot P_0V_0, \\
    A_{12} &= 3P_0 \cdot (6V_0 - V_0) = 15P_0V_0, \\
    \Delta U_{12} &= \frac 32 \nu R T_2 - \frac 32 \nu R T_1 = \frac 32 P_2 V_2 - \frac 32 P_1 V_1 = \frac 32 \cdot 6 P_0 \cdot 3 V_0 -  \frac 32 \cdot 3 P_0 \cdot V_0 = \frac 32 \cdot 15 \cdot P_0V_0, \\
    \Delta U_{31} &= \frac 32 \nu R T_1 - \frac 32 \nu R T_3 = \frac 32 P_1 V_1 - \frac 32 P_3 V_3 = \frac 32 \cdot 6 P_0V_0 - \frac 32 P_0V_0 = \frac 32 \cdot 5 \cdot P_0V_0.
    \\
    \eta &= \frac{A_\text{цикл}}{Q_+} = \frac{A_\text{цикл}}{Q_{12} + Q_{31}}  = \frac{A_\text{цикл}}{A_{12} + \Delta U_{12} + A_{31} + \Delta U_{31}} =  \\
     &= \frac{\frac 12 \cdot 10 \cdot P_0V_0}{15P_0V_0 + \frac 32 \cdot 15 \cdot P_0V_0 + 0 + \frac 32 \cdot 5 \cdot P_0V_0} = \frac{\frac 12 \cdot 10}{15 + \frac 32 \cdot 15 + \frac 32 \cdot 5} = \frac{1}{9} \approx 0{,}111.
    \end{align*}
}

\variantsplitter

\addpersonalvariant{Ирина Ан}

\tasknumber{1}%
\task{%
    Определите КПД (оставив ответ точным в виде нескоратимой дроби) цикла 1231, рабочим телом которого является идеальный одноатомный газ, если
    \begin{itemize}
        \item 12 — изохорический нагрев в пять раз,
        \item 23 — изобарическое расширение, при котором температура растёт в три раза,
        \item 31 — процесс, график которого в $PV$-координатах является отрезком прямой.
    \end{itemize}
}
\answer{%
    \begin{align*}
    A_{12} &= 0, \Delta U_{12} > 0, \implies Q_{12} = A_{12} + \Delta U_{12} > 0.
    \\
    A_{23} &> 0, \Delta U_{23} > 0, \implies Q_{23} = A_{23} + \Delta U_{23} > 0, \\
    A_{31} &= 0, \Delta U_{31} < 0, \implies Q_{31} = A_{31} + \Delta U_{31} < 0.
    \\
    P_1V_1 &= \nu R T_1, P_2V_2 = \nu R T_2, P_3V_3 = \nu R T_3 \text{ — уравнения состояния идеального газа}, \\
    &\text{Пусть $P_0$, $V_0$, $T_0$ — давление, объём и температура в точке 1 (минимальные во всём цикле):} \\
    P_1 &= P_0, P_2 = P_3, V_1 = V_2 = V_0, \text{остальные соотношения нужно считать} \\
    T_2 &= 5T_1 = 5T_0 \text{(по условию)} \implies \frac{P_2}{P_1} = \frac{P_2V_0}{P_1V_0} = \frac{P_2 V_2}{P_1 V_1}= \frac{\nu R T_2}{\nu R T_1} = \frac{T_2}{T_1} = 5 \implies P_2 = 5 P_1 = 5 P_0, \\
    T_3 &= 3T_2 = 15T_0 \text{(по условию)} \implies \frac{V_3}{V_2} = \frac{P_3V_3}{P_2V_2}= \frac{\nu R T_3}{\nu R T_2} = \frac{T_3}{T_2} = 3 \implies V_3 = 3 V_2 = 3 V_0.
    \\
    A_\text{цикл} &= \frac 12 (3P_0 - P_0)(5V_0 - V_0) = \frac 12 \cdot 8 \cdot P_0V_0, \\
    A_{23} &= 5P_0 \cdot (3V_0 - V_0) = 10P_0V_0, \\
    \Delta U_{23} &= \frac 32 \nu R T_3 - \frac 32 \nu R T_3 = \frac 32 P_3 V_3 - \frac 32 P_2 V_2 = \frac 32 \cdot 5 P_0 \cdot 3 V_0 -  \frac 32 \cdot 5 P_0 \cdot V_0 = \frac 32 \cdot 10 \cdot P_0V_0, \\
    \Delta U_{12} &= \frac 32 \nu R T_2 - \frac 32 \nu R T_1 = \frac 32 P_2 V_2 - \frac 32 P_1 V_1 = \frac 32 \cdot 5 P_0V_0 - \frac 32 P_0V_0 = \frac 32 \cdot 4 \cdot P_0V_0.
    \\
    \eta &= \frac{A_\text{цикл}}{Q_+} = \frac{A_\text{цикл}}{Q_{12} + Q_{23}}  = \frac{A_\text{цикл}}{A_{12} + \Delta U_{12} + A_{23} + \Delta U_{23}} =  \\
     &= \frac{\frac 12 \cdot 8 \cdot P_0V_0}{0 + \frac 32 \cdot 4 \cdot P_0V_0 + 10P_0V_0 + \frac 32 \cdot 10 \cdot P_0V_0} = \frac{\frac 12 \cdot 8}{\frac 32 \cdot 4 + 10 + \frac 32 \cdot 10} = \frac{4}{31} \approx 0{,}129.
    \end{align*}
}
\solutionspace{360pt}

\tasknumber{2}%
\task{%
    Определите КПД (оставив ответ точным в виде нескоратимой дроби) цикла 1231, рабочим телом которого является идеальный одноатомный газ, если
    \begin{itemize}
        \item 12 — изобарическое расширение,
        \item 23 — процесс, график которого в $PV$-координатах является отрезком прямой, а объём уменьшается в три раза,
        \item 31 — изохорический нагрев с увеличением давления в пять раз,
    \end{itemize}
}
\answer{%
    \begin{align*}
    A_{12} &> 0, \Delta U_{12} > 0, \implies Q_{12} = A_{12} + \Delta U_{12} > 0.
    \\
    A_{23} &< 0, \Delta U_{23} < 0, \implies Q_{23} = A_{23} + \Delta U_{23} < 0, \\
    A_{31} &= 0, \Delta U_{31} > 0, \implies Q_{31} = A_{31} + \Delta U_{31} > 0.
    \\
    P_1V_1 &= \nu R T_1, P_2V_2 = \nu R T_2, P_3V_3 = \nu R T_3 \text{ — уравнения состояния идеального газа}, \\
    &\text{Пусть $P_0$, $V_0$, $T_0$ — давление, объём и температура в точке 3 (минимальные во всём цикле):} \\
    P_3 &= P_0, P_1 = P_2 = 5P_0, V_1 = V_3 = V_0, V_2 = 3V_3 = 3V_0 \\
    A_\text{цикл} &= \frac 12 (P_2-P_1)(V_1-V_2) = \frac 12 (5P_0 - P_0)(3V_0 - V_0) = \frac 12 \cdot 8 \cdot P_0V_0, \\
    A_{12} &= 3P_0 \cdot (5V_0 - V_0) = 12P_0V_0, \\
    \Delta U_{12} &= \frac 32 \nu R T_2 - \frac 32 \nu R T_1 = \frac 32 P_2 V_2 - \frac 32 P_1 V_1 = \frac 32 \cdot 5 P_0 \cdot 3 V_0 -  \frac 32 \cdot 3 P_0 \cdot V_0 = \frac 32 \cdot 12 \cdot P_0V_0, \\
    \Delta U_{31} &= \frac 32 \nu R T_1 - \frac 32 \nu R T_3 = \frac 32 P_1 V_1 - \frac 32 P_3 V_3 = \frac 32 \cdot 5 P_0V_0 - \frac 32 P_0V_0 = \frac 32 \cdot 4 \cdot P_0V_0.
    \\
    \eta &= \frac{A_\text{цикл}}{Q_+} = \frac{A_\text{цикл}}{Q_{12} + Q_{31}}  = \frac{A_\text{цикл}}{A_{12} + \Delta U_{12} + A_{31} + \Delta U_{31}} =  \\
     &= \frac{\frac 12 \cdot 8 \cdot P_0V_0}{12P_0V_0 + \frac 32 \cdot 12 \cdot P_0V_0 + 0 + \frac 32 \cdot 4 \cdot P_0V_0} = \frac{\frac 12 \cdot 8}{12 + \frac 32 \cdot 12 + \frac 32 \cdot 4} = \frac{1}{9} \approx 0{,}111.
    \end{align*}
}

\variantsplitter

\addpersonalvariant{Софья Андрианова}

\tasknumber{1}%
\task{%
    Определите КПД (оставив ответ точным в виде нескоратимой дроби) цикла 1231, рабочим телом которого является идеальный одноатомный газ, если
    \begin{itemize}
        \item 12 — изохорический нагрев в три раза,
        \item 23 — изобарическое расширение, при котором температура растёт в четыре раза,
        \item 31 — процесс, график которого в $PV$-координатах является отрезком прямой.
    \end{itemize}
}
\answer{%
    \begin{align*}
    A_{12} &= 0, \Delta U_{12} > 0, \implies Q_{12} = A_{12} + \Delta U_{12} > 0.
    \\
    A_{23} &> 0, \Delta U_{23} > 0, \implies Q_{23} = A_{23} + \Delta U_{23} > 0, \\
    A_{31} &= 0, \Delta U_{31} < 0, \implies Q_{31} = A_{31} + \Delta U_{31} < 0.
    \\
    P_1V_1 &= \nu R T_1, P_2V_2 = \nu R T_2, P_3V_3 = \nu R T_3 \text{ — уравнения состояния идеального газа}, \\
    &\text{Пусть $P_0$, $V_0$, $T_0$ — давление, объём и температура в точке 1 (минимальные во всём цикле):} \\
    P_1 &= P_0, P_2 = P_3, V_1 = V_2 = V_0, \text{остальные соотношения нужно считать} \\
    T_2 &= 3T_1 = 3T_0 \text{(по условию)} \implies \frac{P_2}{P_1} = \frac{P_2V_0}{P_1V_0} = \frac{P_2 V_2}{P_1 V_1}= \frac{\nu R T_2}{\nu R T_1} = \frac{T_2}{T_1} = 3 \implies P_2 = 3 P_1 = 3 P_0, \\
    T_3 &= 4T_2 = 12T_0 \text{(по условию)} \implies \frac{V_3}{V_2} = \frac{P_3V_3}{P_2V_2}= \frac{\nu R T_3}{\nu R T_2} = \frac{T_3}{T_2} = 4 \implies V_3 = 4 V_2 = 4 V_0.
    \\
    A_\text{цикл} &= \frac 12 (4P_0 - P_0)(3V_0 - V_0) = \frac 12 \cdot 6 \cdot P_0V_0, \\
    A_{23} &= 3P_0 \cdot (4V_0 - V_0) = 9P_0V_0, \\
    \Delta U_{23} &= \frac 32 \nu R T_3 - \frac 32 \nu R T_3 = \frac 32 P_3 V_3 - \frac 32 P_2 V_2 = \frac 32 \cdot 3 P_0 \cdot 4 V_0 -  \frac 32 \cdot 3 P_0 \cdot V_0 = \frac 32 \cdot 9 \cdot P_0V_0, \\
    \Delta U_{12} &= \frac 32 \nu R T_2 - \frac 32 \nu R T_1 = \frac 32 P_2 V_2 - \frac 32 P_1 V_1 = \frac 32 \cdot 3 P_0V_0 - \frac 32 P_0V_0 = \frac 32 \cdot 2 \cdot P_0V_0.
    \\
    \eta &= \frac{A_\text{цикл}}{Q_+} = \frac{A_\text{цикл}}{Q_{12} + Q_{23}}  = \frac{A_\text{цикл}}{A_{12} + \Delta U_{12} + A_{23} + \Delta U_{23}} =  \\
     &= \frac{\frac 12 \cdot 6 \cdot P_0V_0}{0 + \frac 32 \cdot 2 \cdot P_0V_0 + 9P_0V_0 + \frac 32 \cdot 9 \cdot P_0V_0} = \frac{\frac 12 \cdot 6}{\frac 32 \cdot 2 + 9 + \frac 32 \cdot 9} = \frac{2}{17} \approx 0{,}118.
    \end{align*}
}
\solutionspace{360pt}

\tasknumber{2}%
\task{%
    Определите КПД (оставив ответ точным в виде нескоратимой дроби) цикла 1231, рабочим телом которого является идеальный одноатомный газ, если
    \begin{itemize}
        \item 12 — изобарическое расширение,
        \item 23 — процесс, график которого в $PV$-координатах является отрезком прямой, а объём уменьшается в четыре раза,
        \item 31 — изохорический нагрев с увеличением давления в три раза,
    \end{itemize}
}
\answer{%
    \begin{align*}
    A_{12} &> 0, \Delta U_{12} > 0, \implies Q_{12} = A_{12} + \Delta U_{12} > 0.
    \\
    A_{23} &< 0, \Delta U_{23} < 0, \implies Q_{23} = A_{23} + \Delta U_{23} < 0, \\
    A_{31} &= 0, \Delta U_{31} > 0, \implies Q_{31} = A_{31} + \Delta U_{31} > 0.
    \\
    P_1V_1 &= \nu R T_1, P_2V_2 = \nu R T_2, P_3V_3 = \nu R T_3 \text{ — уравнения состояния идеального газа}, \\
    &\text{Пусть $P_0$, $V_0$, $T_0$ — давление, объём и температура в точке 3 (минимальные во всём цикле):} \\
    P_3 &= P_0, P_1 = P_2 = 3P_0, V_1 = V_3 = V_0, V_2 = 4V_3 = 4V_0 \\
    A_\text{цикл} &= \frac 12 (P_2-P_1)(V_1-V_2) = \frac 12 (3P_0 - P_0)(4V_0 - V_0) = \frac 12 \cdot 6 \cdot P_0V_0, \\
    A_{12} &= 4P_0 \cdot (3V_0 - V_0) = 8P_0V_0, \\
    \Delta U_{12} &= \frac 32 \nu R T_2 - \frac 32 \nu R T_1 = \frac 32 P_2 V_2 - \frac 32 P_1 V_1 = \frac 32 \cdot 3 P_0 \cdot 4 V_0 -  \frac 32 \cdot 4 P_0 \cdot V_0 = \frac 32 \cdot 8 \cdot P_0V_0, \\
    \Delta U_{31} &= \frac 32 \nu R T_1 - \frac 32 \nu R T_3 = \frac 32 P_1 V_1 - \frac 32 P_3 V_3 = \frac 32 \cdot 3 P_0V_0 - \frac 32 P_0V_0 = \frac 32 \cdot 2 \cdot P_0V_0.
    \\
    \eta &= \frac{A_\text{цикл}}{Q_+} = \frac{A_\text{цикл}}{Q_{12} + Q_{31}}  = \frac{A_\text{цикл}}{A_{12} + \Delta U_{12} + A_{31} + \Delta U_{31}} =  \\
     &= \frac{\frac 12 \cdot 6 \cdot P_0V_0}{8P_0V_0 + \frac 32 \cdot 8 \cdot P_0V_0 + 0 + \frac 32 \cdot 2 \cdot P_0V_0} = \frac{\frac 12 \cdot 6}{8 + \frac 32 \cdot 8 + \frac 32 \cdot 2} = \frac{3}{23} \approx 0{,}130.
    \end{align*}
}

\variantsplitter

\addpersonalvariant{Владимир Артемчук}

\tasknumber{1}%
\task{%
    Определите КПД (оставив ответ точным в виде нескоратимой дроби) цикла 1231, рабочим телом которого является идеальный одноатомный газ, если
    \begin{itemize}
        \item 12 — изохорический нагрев в четыре раза,
        \item 23 — изобарическое расширение, при котором температура растёт в четыре раза,
        \item 31 — процесс, график которого в $PV$-координатах является отрезком прямой.
    \end{itemize}
}
\answer{%
    \begin{align*}
    A_{12} &= 0, \Delta U_{12} > 0, \implies Q_{12} = A_{12} + \Delta U_{12} > 0.
    \\
    A_{23} &> 0, \Delta U_{23} > 0, \implies Q_{23} = A_{23} + \Delta U_{23} > 0, \\
    A_{31} &= 0, \Delta U_{31} < 0, \implies Q_{31} = A_{31} + \Delta U_{31} < 0.
    \\
    P_1V_1 &= \nu R T_1, P_2V_2 = \nu R T_2, P_3V_3 = \nu R T_3 \text{ — уравнения состояния идеального газа}, \\
    &\text{Пусть $P_0$, $V_0$, $T_0$ — давление, объём и температура в точке 1 (минимальные во всём цикле):} \\
    P_1 &= P_0, P_2 = P_3, V_1 = V_2 = V_0, \text{остальные соотношения нужно считать} \\
    T_2 &= 4T_1 = 4T_0 \text{(по условию)} \implies \frac{P_2}{P_1} = \frac{P_2V_0}{P_1V_0} = \frac{P_2 V_2}{P_1 V_1}= \frac{\nu R T_2}{\nu R T_1} = \frac{T_2}{T_1} = 4 \implies P_2 = 4 P_1 = 4 P_0, \\
    T_3 &= 4T_2 = 16T_0 \text{(по условию)} \implies \frac{V_3}{V_2} = \frac{P_3V_3}{P_2V_2}= \frac{\nu R T_3}{\nu R T_2} = \frac{T_3}{T_2} = 4 \implies V_3 = 4 V_2 = 4 V_0.
    \\
    A_\text{цикл} &= \frac 12 (4P_0 - P_0)(4V_0 - V_0) = \frac 12 \cdot 9 \cdot P_0V_0, \\
    A_{23} &= 4P_0 \cdot (4V_0 - V_0) = 12P_0V_0, \\
    \Delta U_{23} &= \frac 32 \nu R T_3 - \frac 32 \nu R T_3 = \frac 32 P_3 V_3 - \frac 32 P_2 V_2 = \frac 32 \cdot 4 P_0 \cdot 4 V_0 -  \frac 32 \cdot 4 P_0 \cdot V_0 = \frac 32 \cdot 12 \cdot P_0V_0, \\
    \Delta U_{12} &= \frac 32 \nu R T_2 - \frac 32 \nu R T_1 = \frac 32 P_2 V_2 - \frac 32 P_1 V_1 = \frac 32 \cdot 4 P_0V_0 - \frac 32 P_0V_0 = \frac 32 \cdot 3 \cdot P_0V_0.
    \\
    \eta &= \frac{A_\text{цикл}}{Q_+} = \frac{A_\text{цикл}}{Q_{12} + Q_{23}}  = \frac{A_\text{цикл}}{A_{12} + \Delta U_{12} + A_{23} + \Delta U_{23}} =  \\
     &= \frac{\frac 12 \cdot 9 \cdot P_0V_0}{0 + \frac 32 \cdot 3 \cdot P_0V_0 + 12P_0V_0 + \frac 32 \cdot 12 \cdot P_0V_0} = \frac{\frac 12 \cdot 9}{\frac 32 \cdot 3 + 12 + \frac 32 \cdot 12} = \frac{3}{23} \approx 0{,}130.
    \end{align*}
}
\solutionspace{360pt}

\tasknumber{2}%
\task{%
    Определите КПД (оставив ответ точным в виде нескоратимой дроби) цикла 1231, рабочим телом которого является идеальный одноатомный газ, если
    \begin{itemize}
        \item 12 — изобарическое расширение,
        \item 23 — процесс, график которого в $PV$-координатах является отрезком прямой, а объём уменьшается в четыре раза,
        \item 31 — изохорический нагрев с увеличением давления в четыре раза,
    \end{itemize}
}
\answer{%
    \begin{align*}
    A_{12} &> 0, \Delta U_{12} > 0, \implies Q_{12} = A_{12} + \Delta U_{12} > 0.
    \\
    A_{23} &< 0, \Delta U_{23} < 0, \implies Q_{23} = A_{23} + \Delta U_{23} < 0, \\
    A_{31} &= 0, \Delta U_{31} > 0, \implies Q_{31} = A_{31} + \Delta U_{31} > 0.
    \\
    P_1V_1 &= \nu R T_1, P_2V_2 = \nu R T_2, P_3V_3 = \nu R T_3 \text{ — уравнения состояния идеального газа}, \\
    &\text{Пусть $P_0$, $V_0$, $T_0$ — давление, объём и температура в точке 3 (минимальные во всём цикле):} \\
    P_3 &= P_0, P_1 = P_2 = 4P_0, V_1 = V_3 = V_0, V_2 = 4V_3 = 4V_0 \\
    A_\text{цикл} &= \frac 12 (P_2-P_1)(V_1-V_2) = \frac 12 (4P_0 - P_0)(4V_0 - V_0) = \frac 12 \cdot 9 \cdot P_0V_0, \\
    A_{12} &= 4P_0 \cdot (4V_0 - V_0) = 12P_0V_0, \\
    \Delta U_{12} &= \frac 32 \nu R T_2 - \frac 32 \nu R T_1 = \frac 32 P_2 V_2 - \frac 32 P_1 V_1 = \frac 32 \cdot 4 P_0 \cdot 4 V_0 -  \frac 32 \cdot 4 P_0 \cdot V_0 = \frac 32 \cdot 12 \cdot P_0V_0, \\
    \Delta U_{31} &= \frac 32 \nu R T_1 - \frac 32 \nu R T_3 = \frac 32 P_1 V_1 - \frac 32 P_3 V_3 = \frac 32 \cdot 4 P_0V_0 - \frac 32 P_0V_0 = \frac 32 \cdot 3 \cdot P_0V_0.
    \\
    \eta &= \frac{A_\text{цикл}}{Q_+} = \frac{A_\text{цикл}}{Q_{12} + Q_{31}}  = \frac{A_\text{цикл}}{A_{12} + \Delta U_{12} + A_{31} + \Delta U_{31}} =  \\
     &= \frac{\frac 12 \cdot 9 \cdot P_0V_0}{12P_0V_0 + \frac 32 \cdot 12 \cdot P_0V_0 + 0 + \frac 32 \cdot 3 \cdot P_0V_0} = \frac{\frac 12 \cdot 9}{12 + \frac 32 \cdot 12 + \frac 32 \cdot 3} = \frac{3}{23} \approx 0{,}130.
    \end{align*}
}

\variantsplitter

\addpersonalvariant{Софья Белянкина}

\tasknumber{1}%
\task{%
    Определите КПД (оставив ответ точным в виде нескоратимой дроби) цикла 1231, рабочим телом которого является идеальный одноатомный газ, если
    \begin{itemize}
        \item 12 — изохорический нагрев в пять раз,
        \item 23 — изобарическое расширение, при котором температура растёт в три раза,
        \item 31 — процесс, график которого в $PV$-координатах является отрезком прямой.
    \end{itemize}
}
\answer{%
    \begin{align*}
    A_{12} &= 0, \Delta U_{12} > 0, \implies Q_{12} = A_{12} + \Delta U_{12} > 0.
    \\
    A_{23} &> 0, \Delta U_{23} > 0, \implies Q_{23} = A_{23} + \Delta U_{23} > 0, \\
    A_{31} &= 0, \Delta U_{31} < 0, \implies Q_{31} = A_{31} + \Delta U_{31} < 0.
    \\
    P_1V_1 &= \nu R T_1, P_2V_2 = \nu R T_2, P_3V_3 = \nu R T_3 \text{ — уравнения состояния идеального газа}, \\
    &\text{Пусть $P_0$, $V_0$, $T_0$ — давление, объём и температура в точке 1 (минимальные во всём цикле):} \\
    P_1 &= P_0, P_2 = P_3, V_1 = V_2 = V_0, \text{остальные соотношения нужно считать} \\
    T_2 &= 5T_1 = 5T_0 \text{(по условию)} \implies \frac{P_2}{P_1} = \frac{P_2V_0}{P_1V_0} = \frac{P_2 V_2}{P_1 V_1}= \frac{\nu R T_2}{\nu R T_1} = \frac{T_2}{T_1} = 5 \implies P_2 = 5 P_1 = 5 P_0, \\
    T_3 &= 3T_2 = 15T_0 \text{(по условию)} \implies \frac{V_3}{V_2} = \frac{P_3V_3}{P_2V_2}= \frac{\nu R T_3}{\nu R T_2} = \frac{T_3}{T_2} = 3 \implies V_3 = 3 V_2 = 3 V_0.
    \\
    A_\text{цикл} &= \frac 12 (3P_0 - P_0)(5V_0 - V_0) = \frac 12 \cdot 8 \cdot P_0V_0, \\
    A_{23} &= 5P_0 \cdot (3V_0 - V_0) = 10P_0V_0, \\
    \Delta U_{23} &= \frac 32 \nu R T_3 - \frac 32 \nu R T_3 = \frac 32 P_3 V_3 - \frac 32 P_2 V_2 = \frac 32 \cdot 5 P_0 \cdot 3 V_0 -  \frac 32 \cdot 5 P_0 \cdot V_0 = \frac 32 \cdot 10 \cdot P_0V_0, \\
    \Delta U_{12} &= \frac 32 \nu R T_2 - \frac 32 \nu R T_1 = \frac 32 P_2 V_2 - \frac 32 P_1 V_1 = \frac 32 \cdot 5 P_0V_0 - \frac 32 P_0V_0 = \frac 32 \cdot 4 \cdot P_0V_0.
    \\
    \eta &= \frac{A_\text{цикл}}{Q_+} = \frac{A_\text{цикл}}{Q_{12} + Q_{23}}  = \frac{A_\text{цикл}}{A_{12} + \Delta U_{12} + A_{23} + \Delta U_{23}} =  \\
     &= \frac{\frac 12 \cdot 8 \cdot P_0V_0}{0 + \frac 32 \cdot 4 \cdot P_0V_0 + 10P_0V_0 + \frac 32 \cdot 10 \cdot P_0V_0} = \frac{\frac 12 \cdot 8}{\frac 32 \cdot 4 + 10 + \frac 32 \cdot 10} = \frac{4}{31} \approx 0{,}129.
    \end{align*}
}
\solutionspace{360pt}

\tasknumber{2}%
\task{%
    Определите КПД (оставив ответ точным в виде нескоратимой дроби) цикла 1231, рабочим телом которого является идеальный одноатомный газ, если
    \begin{itemize}
        \item 12 — изобарическое расширение,
        \item 23 — процесс, график которого в $PV$-координатах является отрезком прямой, а объём уменьшается в три раза,
        \item 31 — изохорический нагрев с увеличением давления в пять раз,
    \end{itemize}
}
\answer{%
    \begin{align*}
    A_{12} &> 0, \Delta U_{12} > 0, \implies Q_{12} = A_{12} + \Delta U_{12} > 0.
    \\
    A_{23} &< 0, \Delta U_{23} < 0, \implies Q_{23} = A_{23} + \Delta U_{23} < 0, \\
    A_{31} &= 0, \Delta U_{31} > 0, \implies Q_{31} = A_{31} + \Delta U_{31} > 0.
    \\
    P_1V_1 &= \nu R T_1, P_2V_2 = \nu R T_2, P_3V_3 = \nu R T_3 \text{ — уравнения состояния идеального газа}, \\
    &\text{Пусть $P_0$, $V_0$, $T_0$ — давление, объём и температура в точке 3 (минимальные во всём цикле):} \\
    P_3 &= P_0, P_1 = P_2 = 5P_0, V_1 = V_3 = V_0, V_2 = 3V_3 = 3V_0 \\
    A_\text{цикл} &= \frac 12 (P_2-P_1)(V_1-V_2) = \frac 12 (5P_0 - P_0)(3V_0 - V_0) = \frac 12 \cdot 8 \cdot P_0V_0, \\
    A_{12} &= 3P_0 \cdot (5V_0 - V_0) = 12P_0V_0, \\
    \Delta U_{12} &= \frac 32 \nu R T_2 - \frac 32 \nu R T_1 = \frac 32 P_2 V_2 - \frac 32 P_1 V_1 = \frac 32 \cdot 5 P_0 \cdot 3 V_0 -  \frac 32 \cdot 3 P_0 \cdot V_0 = \frac 32 \cdot 12 \cdot P_0V_0, \\
    \Delta U_{31} &= \frac 32 \nu R T_1 - \frac 32 \nu R T_3 = \frac 32 P_1 V_1 - \frac 32 P_3 V_3 = \frac 32 \cdot 5 P_0V_0 - \frac 32 P_0V_0 = \frac 32 \cdot 4 \cdot P_0V_0.
    \\
    \eta &= \frac{A_\text{цикл}}{Q_+} = \frac{A_\text{цикл}}{Q_{12} + Q_{31}}  = \frac{A_\text{цикл}}{A_{12} + \Delta U_{12} + A_{31} + \Delta U_{31}} =  \\
     &= \frac{\frac 12 \cdot 8 \cdot P_0V_0}{12P_0V_0 + \frac 32 \cdot 12 \cdot P_0V_0 + 0 + \frac 32 \cdot 4 \cdot P_0V_0} = \frac{\frac 12 \cdot 8}{12 + \frac 32 \cdot 12 + \frac 32 \cdot 4} = \frac{1}{9} \approx 0{,}111.
    \end{align*}
}

\variantsplitter

\addpersonalvariant{Варвара Егиазарян}

\tasknumber{1}%
\task{%
    Определите КПД (оставив ответ точным в виде нескоратимой дроби) цикла 1231, рабочим телом которого является идеальный одноатомный газ, если
    \begin{itemize}
        \item 12 — изохорический нагрев в два раза,
        \item 23 — изобарическое расширение, при котором температура растёт в три раза,
        \item 31 — процесс, график которого в $PV$-координатах является отрезком прямой.
    \end{itemize}
}
\answer{%
    \begin{align*}
    A_{12} &= 0, \Delta U_{12} > 0, \implies Q_{12} = A_{12} + \Delta U_{12} > 0.
    \\
    A_{23} &> 0, \Delta U_{23} > 0, \implies Q_{23} = A_{23} + \Delta U_{23} > 0, \\
    A_{31} &= 0, \Delta U_{31} < 0, \implies Q_{31} = A_{31} + \Delta U_{31} < 0.
    \\
    P_1V_1 &= \nu R T_1, P_2V_2 = \nu R T_2, P_3V_3 = \nu R T_3 \text{ — уравнения состояния идеального газа}, \\
    &\text{Пусть $P_0$, $V_0$, $T_0$ — давление, объём и температура в точке 1 (минимальные во всём цикле):} \\
    P_1 &= P_0, P_2 = P_3, V_1 = V_2 = V_0, \text{остальные соотношения нужно считать} \\
    T_2 &= 2T_1 = 2T_0 \text{(по условию)} \implies \frac{P_2}{P_1} = \frac{P_2V_0}{P_1V_0} = \frac{P_2 V_2}{P_1 V_1}= \frac{\nu R T_2}{\nu R T_1} = \frac{T_2}{T_1} = 2 \implies P_2 = 2 P_1 = 2 P_0, \\
    T_3 &= 3T_2 = 6T_0 \text{(по условию)} \implies \frac{V_3}{V_2} = \frac{P_3V_3}{P_2V_2}= \frac{\nu R T_3}{\nu R T_2} = \frac{T_3}{T_2} = 3 \implies V_3 = 3 V_2 = 3 V_0.
    \\
    A_\text{цикл} &= \frac 12 (3P_0 - P_0)(2V_0 - V_0) = \frac 12 \cdot 2 \cdot P_0V_0, \\
    A_{23} &= 2P_0 \cdot (3V_0 - V_0) = 4P_0V_0, \\
    \Delta U_{23} &= \frac 32 \nu R T_3 - \frac 32 \nu R T_3 = \frac 32 P_3 V_3 - \frac 32 P_2 V_2 = \frac 32 \cdot 2 P_0 \cdot 3 V_0 -  \frac 32 \cdot 2 P_0 \cdot V_0 = \frac 32 \cdot 4 \cdot P_0V_0, \\
    \Delta U_{12} &= \frac 32 \nu R T_2 - \frac 32 \nu R T_1 = \frac 32 P_2 V_2 - \frac 32 P_1 V_1 = \frac 32 \cdot 2 P_0V_0 - \frac 32 P_0V_0 = \frac 32 \cdot 1 \cdot P_0V_0.
    \\
    \eta &= \frac{A_\text{цикл}}{Q_+} = \frac{A_\text{цикл}}{Q_{12} + Q_{23}}  = \frac{A_\text{цикл}}{A_{12} + \Delta U_{12} + A_{23} + \Delta U_{23}} =  \\
     &= \frac{\frac 12 \cdot 2 \cdot P_0V_0}{0 + \frac 32 \cdot 1 \cdot P_0V_0 + 4P_0V_0 + \frac 32 \cdot 4 \cdot P_0V_0} = \frac{\frac 12 \cdot 2}{\frac 32 \cdot 1 + 4 + \frac 32 \cdot 4} = \frac{2}{23} \approx 0{,}087.
    \end{align*}
}
\solutionspace{360pt}

\tasknumber{2}%
\task{%
    Определите КПД (оставив ответ точным в виде нескоратимой дроби) цикла 1231, рабочим телом которого является идеальный одноатомный газ, если
    \begin{itemize}
        \item 12 — изобарическое расширение,
        \item 23 — процесс, график которого в $PV$-координатах является отрезком прямой, а объём уменьшается в три раза,
        \item 31 — изохорический нагрев с увеличением давления в два раза,
    \end{itemize}
}
\answer{%
    \begin{align*}
    A_{12} &> 0, \Delta U_{12} > 0, \implies Q_{12} = A_{12} + \Delta U_{12} > 0.
    \\
    A_{23} &< 0, \Delta U_{23} < 0, \implies Q_{23} = A_{23} + \Delta U_{23} < 0, \\
    A_{31} &= 0, \Delta U_{31} > 0, \implies Q_{31} = A_{31} + \Delta U_{31} > 0.
    \\
    P_1V_1 &= \nu R T_1, P_2V_2 = \nu R T_2, P_3V_3 = \nu R T_3 \text{ — уравнения состояния идеального газа}, \\
    &\text{Пусть $P_0$, $V_0$, $T_0$ — давление, объём и температура в точке 3 (минимальные во всём цикле):} \\
    P_3 &= P_0, P_1 = P_2 = 2P_0, V_1 = V_3 = V_0, V_2 = 3V_3 = 3V_0 \\
    A_\text{цикл} &= \frac 12 (P_2-P_1)(V_1-V_2) = \frac 12 (2P_0 - P_0)(3V_0 - V_0) = \frac 12 \cdot 2 \cdot P_0V_0, \\
    A_{12} &= 3P_0 \cdot (2V_0 - V_0) = 3P_0V_0, \\
    \Delta U_{12} &= \frac 32 \nu R T_2 - \frac 32 \nu R T_1 = \frac 32 P_2 V_2 - \frac 32 P_1 V_1 = \frac 32 \cdot 2 P_0 \cdot 3 V_0 -  \frac 32 \cdot 3 P_0 \cdot V_0 = \frac 32 \cdot 3 \cdot P_0V_0, \\
    \Delta U_{31} &= \frac 32 \nu R T_1 - \frac 32 \nu R T_3 = \frac 32 P_1 V_1 - \frac 32 P_3 V_3 = \frac 32 \cdot 2 P_0V_0 - \frac 32 P_0V_0 = \frac 32 \cdot 1 \cdot P_0V_0.
    \\
    \eta &= \frac{A_\text{цикл}}{Q_+} = \frac{A_\text{цикл}}{Q_{12} + Q_{31}}  = \frac{A_\text{цикл}}{A_{12} + \Delta U_{12} + A_{31} + \Delta U_{31}} =  \\
     &= \frac{\frac 12 \cdot 2 \cdot P_0V_0}{3P_0V_0 + \frac 32 \cdot 3 \cdot P_0V_0 + 0 + \frac 32 \cdot 1 \cdot P_0V_0} = \frac{\frac 12 \cdot 2}{3 + \frac 32 \cdot 3 + \frac 32 \cdot 1} = \frac{1}{9} \approx 0{,}111.
    \end{align*}
}

\variantsplitter

\addpersonalvariant{Владислав Емелин}

\tasknumber{1}%
\task{%
    Определите КПД (оставив ответ точным в виде нескоратимой дроби) цикла 1231, рабочим телом которого является идеальный одноатомный газ, если
    \begin{itemize}
        \item 12 — изохорический нагрев в два раза,
        \item 23 — изобарическое расширение, при котором температура растёт в два раза,
        \item 31 — процесс, график которого в $PV$-координатах является отрезком прямой.
    \end{itemize}
}
\answer{%
    \begin{align*}
    A_{12} &= 0, \Delta U_{12} > 0, \implies Q_{12} = A_{12} + \Delta U_{12} > 0.
    \\
    A_{23} &> 0, \Delta U_{23} > 0, \implies Q_{23} = A_{23} + \Delta U_{23} > 0, \\
    A_{31} &= 0, \Delta U_{31} < 0, \implies Q_{31} = A_{31} + \Delta U_{31} < 0.
    \\
    P_1V_1 &= \nu R T_1, P_2V_2 = \nu R T_2, P_3V_3 = \nu R T_3 \text{ — уравнения состояния идеального газа}, \\
    &\text{Пусть $P_0$, $V_0$, $T_0$ — давление, объём и температура в точке 1 (минимальные во всём цикле):} \\
    P_1 &= P_0, P_2 = P_3, V_1 = V_2 = V_0, \text{остальные соотношения нужно считать} \\
    T_2 &= 2T_1 = 2T_0 \text{(по условию)} \implies \frac{P_2}{P_1} = \frac{P_2V_0}{P_1V_0} = \frac{P_2 V_2}{P_1 V_1}= \frac{\nu R T_2}{\nu R T_1} = \frac{T_2}{T_1} = 2 \implies P_2 = 2 P_1 = 2 P_0, \\
    T_3 &= 2T_2 = 4T_0 \text{(по условию)} \implies \frac{V_3}{V_2} = \frac{P_3V_3}{P_2V_2}= \frac{\nu R T_3}{\nu R T_2} = \frac{T_3}{T_2} = 2 \implies V_3 = 2 V_2 = 2 V_0.
    \\
    A_\text{цикл} &= \frac 12 (2P_0 - P_0)(2V_0 - V_0) = \frac 12 \cdot 1 \cdot P_0V_0, \\
    A_{23} &= 2P_0 \cdot (2V_0 - V_0) = 2P_0V_0, \\
    \Delta U_{23} &= \frac 32 \nu R T_3 - \frac 32 \nu R T_3 = \frac 32 P_3 V_3 - \frac 32 P_2 V_2 = \frac 32 \cdot 2 P_0 \cdot 2 V_0 -  \frac 32 \cdot 2 P_0 \cdot V_0 = \frac 32 \cdot 2 \cdot P_0V_0, \\
    \Delta U_{12} &= \frac 32 \nu R T_2 - \frac 32 \nu R T_1 = \frac 32 P_2 V_2 - \frac 32 P_1 V_1 = \frac 32 \cdot 2 P_0V_0 - \frac 32 P_0V_0 = \frac 32 \cdot 1 \cdot P_0V_0.
    \\
    \eta &= \frac{A_\text{цикл}}{Q_+} = \frac{A_\text{цикл}}{Q_{12} + Q_{23}}  = \frac{A_\text{цикл}}{A_{12} + \Delta U_{12} + A_{23} + \Delta U_{23}} =  \\
     &= \frac{\frac 12 \cdot 1 \cdot P_0V_0}{0 + \frac 32 \cdot 1 \cdot P_0V_0 + 2P_0V_0 + \frac 32 \cdot 2 \cdot P_0V_0} = \frac{\frac 12 \cdot 1}{\frac 32 \cdot 1 + 2 + \frac 32 \cdot 2} = \frac{1}{13} \approx 0{,}077.
    \end{align*}
}
\solutionspace{360pt}

\tasknumber{2}%
\task{%
    Определите КПД (оставив ответ точным в виде нескоратимой дроби) цикла 1231, рабочим телом которого является идеальный одноатомный газ, если
    \begin{itemize}
        \item 12 — изобарическое расширение,
        \item 23 — процесс, график которого в $PV$-координатах является отрезком прямой, а объём уменьшается в два раза,
        \item 31 — изохорический нагрев с увеличением давления в два раза,
    \end{itemize}
}
\answer{%
    \begin{align*}
    A_{12} &> 0, \Delta U_{12} > 0, \implies Q_{12} = A_{12} + \Delta U_{12} > 0.
    \\
    A_{23} &< 0, \Delta U_{23} < 0, \implies Q_{23} = A_{23} + \Delta U_{23} < 0, \\
    A_{31} &= 0, \Delta U_{31} > 0, \implies Q_{31} = A_{31} + \Delta U_{31} > 0.
    \\
    P_1V_1 &= \nu R T_1, P_2V_2 = \nu R T_2, P_3V_3 = \nu R T_3 \text{ — уравнения состояния идеального газа}, \\
    &\text{Пусть $P_0$, $V_0$, $T_0$ — давление, объём и температура в точке 3 (минимальные во всём цикле):} \\
    P_3 &= P_0, P_1 = P_2 = 2P_0, V_1 = V_3 = V_0, V_2 = 2V_3 = 2V_0 \\
    A_\text{цикл} &= \frac 12 (P_2-P_1)(V_1-V_2) = \frac 12 (2P_0 - P_0)(2V_0 - V_0) = \frac 12 \cdot 1 \cdot P_0V_0, \\
    A_{12} &= 2P_0 \cdot (2V_0 - V_0) = 2P_0V_0, \\
    \Delta U_{12} &= \frac 32 \nu R T_2 - \frac 32 \nu R T_1 = \frac 32 P_2 V_2 - \frac 32 P_1 V_1 = \frac 32 \cdot 2 P_0 \cdot 2 V_0 -  \frac 32 \cdot 2 P_0 \cdot V_0 = \frac 32 \cdot 2 \cdot P_0V_0, \\
    \Delta U_{31} &= \frac 32 \nu R T_1 - \frac 32 \nu R T_3 = \frac 32 P_1 V_1 - \frac 32 P_3 V_3 = \frac 32 \cdot 2 P_0V_0 - \frac 32 P_0V_0 = \frac 32 \cdot 1 \cdot P_0V_0.
    \\
    \eta &= \frac{A_\text{цикл}}{Q_+} = \frac{A_\text{цикл}}{Q_{12} + Q_{31}}  = \frac{A_\text{цикл}}{A_{12} + \Delta U_{12} + A_{31} + \Delta U_{31}} =  \\
     &= \frac{\frac 12 \cdot 1 \cdot P_0V_0}{2P_0V_0 + \frac 32 \cdot 2 \cdot P_0V_0 + 0 + \frac 32 \cdot 1 \cdot P_0V_0} = \frac{\frac 12 \cdot 1}{2 + \frac 32 \cdot 2 + \frac 32 \cdot 1} = \frac{1}{13} \approx 0{,}077.
    \end{align*}
}

\variantsplitter

\addpersonalvariant{Артём Жичин}

\tasknumber{1}%
\task{%
    Определите КПД (оставив ответ точным в виде нескоратимой дроби) цикла 1231, рабочим телом которого является идеальный одноатомный газ, если
    \begin{itemize}
        \item 12 — изохорический нагрев в четыре раза,
        \item 23 — изобарическое расширение, при котором температура растёт в пять раз,
        \item 31 — процесс, график которого в $PV$-координатах является отрезком прямой.
    \end{itemize}
}
\answer{%
    \begin{align*}
    A_{12} &= 0, \Delta U_{12} > 0, \implies Q_{12} = A_{12} + \Delta U_{12} > 0.
    \\
    A_{23} &> 0, \Delta U_{23} > 0, \implies Q_{23} = A_{23} + \Delta U_{23} > 0, \\
    A_{31} &= 0, \Delta U_{31} < 0, \implies Q_{31} = A_{31} + \Delta U_{31} < 0.
    \\
    P_1V_1 &= \nu R T_1, P_2V_2 = \nu R T_2, P_3V_3 = \nu R T_3 \text{ — уравнения состояния идеального газа}, \\
    &\text{Пусть $P_0$, $V_0$, $T_0$ — давление, объём и температура в точке 1 (минимальные во всём цикле):} \\
    P_1 &= P_0, P_2 = P_3, V_1 = V_2 = V_0, \text{остальные соотношения нужно считать} \\
    T_2 &= 4T_1 = 4T_0 \text{(по условию)} \implies \frac{P_2}{P_1} = \frac{P_2V_0}{P_1V_0} = \frac{P_2 V_2}{P_1 V_1}= \frac{\nu R T_2}{\nu R T_1} = \frac{T_2}{T_1} = 4 \implies P_2 = 4 P_1 = 4 P_0, \\
    T_3 &= 5T_2 = 20T_0 \text{(по условию)} \implies \frac{V_3}{V_2} = \frac{P_3V_3}{P_2V_2}= \frac{\nu R T_3}{\nu R T_2} = \frac{T_3}{T_2} = 5 \implies V_3 = 5 V_2 = 5 V_0.
    \\
    A_\text{цикл} &= \frac 12 (5P_0 - P_0)(4V_0 - V_0) = \frac 12 \cdot 12 \cdot P_0V_0, \\
    A_{23} &= 4P_0 \cdot (5V_0 - V_0) = 16P_0V_0, \\
    \Delta U_{23} &= \frac 32 \nu R T_3 - \frac 32 \nu R T_3 = \frac 32 P_3 V_3 - \frac 32 P_2 V_2 = \frac 32 \cdot 4 P_0 \cdot 5 V_0 -  \frac 32 \cdot 4 P_0 \cdot V_0 = \frac 32 \cdot 16 \cdot P_0V_0, \\
    \Delta U_{12} &= \frac 32 \nu R T_2 - \frac 32 \nu R T_1 = \frac 32 P_2 V_2 - \frac 32 P_1 V_1 = \frac 32 \cdot 4 P_0V_0 - \frac 32 P_0V_0 = \frac 32 \cdot 3 \cdot P_0V_0.
    \\
    \eta &= \frac{A_\text{цикл}}{Q_+} = \frac{A_\text{цикл}}{Q_{12} + Q_{23}}  = \frac{A_\text{цикл}}{A_{12} + \Delta U_{12} + A_{23} + \Delta U_{23}} =  \\
     &= \frac{\frac 12 \cdot 12 \cdot P_0V_0}{0 + \frac 32 \cdot 3 \cdot P_0V_0 + 16P_0V_0 + \frac 32 \cdot 16 \cdot P_0V_0} = \frac{\frac 12 \cdot 12}{\frac 32 \cdot 3 + 16 + \frac 32 \cdot 16} = \frac{12}{89} \approx 0{,}135.
    \end{align*}
}
\solutionspace{360pt}

\tasknumber{2}%
\task{%
    Определите КПД (оставив ответ точным в виде нескоратимой дроби) цикла 1231, рабочим телом которого является идеальный одноатомный газ, если
    \begin{itemize}
        \item 12 — изобарическое расширение,
        \item 23 — процесс, график которого в $PV$-координатах является отрезком прямой, а объём уменьшается в пять раз,
        \item 31 — изохорический нагрев с увеличением давления в четыре раза,
    \end{itemize}
}
\answer{%
    \begin{align*}
    A_{12} &> 0, \Delta U_{12} > 0, \implies Q_{12} = A_{12} + \Delta U_{12} > 0.
    \\
    A_{23} &< 0, \Delta U_{23} < 0, \implies Q_{23} = A_{23} + \Delta U_{23} < 0, \\
    A_{31} &= 0, \Delta U_{31} > 0, \implies Q_{31} = A_{31} + \Delta U_{31} > 0.
    \\
    P_1V_1 &= \nu R T_1, P_2V_2 = \nu R T_2, P_3V_3 = \nu R T_3 \text{ — уравнения состояния идеального газа}, \\
    &\text{Пусть $P_0$, $V_0$, $T_0$ — давление, объём и температура в точке 3 (минимальные во всём цикле):} \\
    P_3 &= P_0, P_1 = P_2 = 4P_0, V_1 = V_3 = V_0, V_2 = 5V_3 = 5V_0 \\
    A_\text{цикл} &= \frac 12 (P_2-P_1)(V_1-V_2) = \frac 12 (4P_0 - P_0)(5V_0 - V_0) = \frac 12 \cdot 12 \cdot P_0V_0, \\
    A_{12} &= 5P_0 \cdot (4V_0 - V_0) = 15P_0V_0, \\
    \Delta U_{12} &= \frac 32 \nu R T_2 - \frac 32 \nu R T_1 = \frac 32 P_2 V_2 - \frac 32 P_1 V_1 = \frac 32 \cdot 4 P_0 \cdot 5 V_0 -  \frac 32 \cdot 5 P_0 \cdot V_0 = \frac 32 \cdot 15 \cdot P_0V_0, \\
    \Delta U_{31} &= \frac 32 \nu R T_1 - \frac 32 \nu R T_3 = \frac 32 P_1 V_1 - \frac 32 P_3 V_3 = \frac 32 \cdot 4 P_0V_0 - \frac 32 P_0V_0 = \frac 32 \cdot 3 \cdot P_0V_0.
    \\
    \eta &= \frac{A_\text{цикл}}{Q_+} = \frac{A_\text{цикл}}{Q_{12} + Q_{31}}  = \frac{A_\text{цикл}}{A_{12} + \Delta U_{12} + A_{31} + \Delta U_{31}} =  \\
     &= \frac{\frac 12 \cdot 12 \cdot P_0V_0}{15P_0V_0 + \frac 32 \cdot 15 \cdot P_0V_0 + 0 + \frac 32 \cdot 3 \cdot P_0V_0} = \frac{\frac 12 \cdot 12}{15 + \frac 32 \cdot 15 + \frac 32 \cdot 3} = \frac{1}{7} \approx 0{,}143.
    \end{align*}
}

\variantsplitter

\addpersonalvariant{Дарья Кошман}

\tasknumber{1}%
\task{%
    Определите КПД (оставив ответ точным в виде нескоратимой дроби) цикла 1231, рабочим телом которого является идеальный одноатомный газ, если
    \begin{itemize}
        \item 12 — изохорический нагрев в четыре раза,
        \item 23 — изобарическое расширение, при котором температура растёт в шесть раз,
        \item 31 — процесс, график которого в $PV$-координатах является отрезком прямой.
    \end{itemize}
}
\answer{%
    \begin{align*}
    A_{12} &= 0, \Delta U_{12} > 0, \implies Q_{12} = A_{12} + \Delta U_{12} > 0.
    \\
    A_{23} &> 0, \Delta U_{23} > 0, \implies Q_{23} = A_{23} + \Delta U_{23} > 0, \\
    A_{31} &= 0, \Delta U_{31} < 0, \implies Q_{31} = A_{31} + \Delta U_{31} < 0.
    \\
    P_1V_1 &= \nu R T_1, P_2V_2 = \nu R T_2, P_3V_3 = \nu R T_3 \text{ — уравнения состояния идеального газа}, \\
    &\text{Пусть $P_0$, $V_0$, $T_0$ — давление, объём и температура в точке 1 (минимальные во всём цикле):} \\
    P_1 &= P_0, P_2 = P_3, V_1 = V_2 = V_0, \text{остальные соотношения нужно считать} \\
    T_2 &= 4T_1 = 4T_0 \text{(по условию)} \implies \frac{P_2}{P_1} = \frac{P_2V_0}{P_1V_0} = \frac{P_2 V_2}{P_1 V_1}= \frac{\nu R T_2}{\nu R T_1} = \frac{T_2}{T_1} = 4 \implies P_2 = 4 P_1 = 4 P_0, \\
    T_3 &= 6T_2 = 24T_0 \text{(по условию)} \implies \frac{V_3}{V_2} = \frac{P_3V_3}{P_2V_2}= \frac{\nu R T_3}{\nu R T_2} = \frac{T_3}{T_2} = 6 \implies V_3 = 6 V_2 = 6 V_0.
    \\
    A_\text{цикл} &= \frac 12 (6P_0 - P_0)(4V_0 - V_0) = \frac 12 \cdot 15 \cdot P_0V_0, \\
    A_{23} &= 4P_0 \cdot (6V_0 - V_0) = 20P_0V_0, \\
    \Delta U_{23} &= \frac 32 \nu R T_3 - \frac 32 \nu R T_3 = \frac 32 P_3 V_3 - \frac 32 P_2 V_2 = \frac 32 \cdot 4 P_0 \cdot 6 V_0 -  \frac 32 \cdot 4 P_0 \cdot V_0 = \frac 32 \cdot 20 \cdot P_0V_0, \\
    \Delta U_{12} &= \frac 32 \nu R T_2 - \frac 32 \nu R T_1 = \frac 32 P_2 V_2 - \frac 32 P_1 V_1 = \frac 32 \cdot 4 P_0V_0 - \frac 32 P_0V_0 = \frac 32 \cdot 3 \cdot P_0V_0.
    \\
    \eta &= \frac{A_\text{цикл}}{Q_+} = \frac{A_\text{цикл}}{Q_{12} + Q_{23}}  = \frac{A_\text{цикл}}{A_{12} + \Delta U_{12} + A_{23} + \Delta U_{23}} =  \\
     &= \frac{\frac 12 \cdot 15 \cdot P_0V_0}{0 + \frac 32 \cdot 3 \cdot P_0V_0 + 20P_0V_0 + \frac 32 \cdot 20 \cdot P_0V_0} = \frac{\frac 12 \cdot 15}{\frac 32 \cdot 3 + 20 + \frac 32 \cdot 20} = \frac{15}{109} \approx 0{,}138.
    \end{align*}
}
\solutionspace{360pt}

\tasknumber{2}%
\task{%
    Определите КПД (оставив ответ точным в виде нескоратимой дроби) цикла 1231, рабочим телом которого является идеальный одноатомный газ, если
    \begin{itemize}
        \item 12 — изобарическое расширение,
        \item 23 — процесс, график которого в $PV$-координатах является отрезком прямой, а объём уменьшается в шесть раз,
        \item 31 — изохорический нагрев с увеличением давления в четыре раза,
    \end{itemize}
}
\answer{%
    \begin{align*}
    A_{12} &> 0, \Delta U_{12} > 0, \implies Q_{12} = A_{12} + \Delta U_{12} > 0.
    \\
    A_{23} &< 0, \Delta U_{23} < 0, \implies Q_{23} = A_{23} + \Delta U_{23} < 0, \\
    A_{31} &= 0, \Delta U_{31} > 0, \implies Q_{31} = A_{31} + \Delta U_{31} > 0.
    \\
    P_1V_1 &= \nu R T_1, P_2V_2 = \nu R T_2, P_3V_3 = \nu R T_3 \text{ — уравнения состояния идеального газа}, \\
    &\text{Пусть $P_0$, $V_0$, $T_0$ — давление, объём и температура в точке 3 (минимальные во всём цикле):} \\
    P_3 &= P_0, P_1 = P_2 = 4P_0, V_1 = V_3 = V_0, V_2 = 6V_3 = 6V_0 \\
    A_\text{цикл} &= \frac 12 (P_2-P_1)(V_1-V_2) = \frac 12 (4P_0 - P_0)(6V_0 - V_0) = \frac 12 \cdot 15 \cdot P_0V_0, \\
    A_{12} &= 6P_0 \cdot (4V_0 - V_0) = 18P_0V_0, \\
    \Delta U_{12} &= \frac 32 \nu R T_2 - \frac 32 \nu R T_1 = \frac 32 P_2 V_2 - \frac 32 P_1 V_1 = \frac 32 \cdot 4 P_0 \cdot 6 V_0 -  \frac 32 \cdot 6 P_0 \cdot V_0 = \frac 32 \cdot 18 \cdot P_0V_0, \\
    \Delta U_{31} &= \frac 32 \nu R T_1 - \frac 32 \nu R T_3 = \frac 32 P_1 V_1 - \frac 32 P_3 V_3 = \frac 32 \cdot 4 P_0V_0 - \frac 32 P_0V_0 = \frac 32 \cdot 3 \cdot P_0V_0.
    \\
    \eta &= \frac{A_\text{цикл}}{Q_+} = \frac{A_\text{цикл}}{Q_{12} + Q_{31}}  = \frac{A_\text{цикл}}{A_{12} + \Delta U_{12} + A_{31} + \Delta U_{31}} =  \\
     &= \frac{\frac 12 \cdot 15 \cdot P_0V_0}{18P_0V_0 + \frac 32 \cdot 18 \cdot P_0V_0 + 0 + \frac 32 \cdot 3 \cdot P_0V_0} = \frac{\frac 12 \cdot 15}{18 + \frac 32 \cdot 18 + \frac 32 \cdot 3} = \frac{5}{33} \approx 0{,}152.
    \end{align*}
}

\variantsplitter

\addpersonalvariant{Анна Кузьмичёва}

\tasknumber{1}%
\task{%
    Определите КПД (оставив ответ точным в виде нескоратимой дроби) цикла 1231, рабочим телом которого является идеальный одноатомный газ, если
    \begin{itemize}
        \item 12 — изохорический нагрев в два раза,
        \item 23 — изобарическое расширение, при котором температура растёт в шесть раз,
        \item 31 — процесс, график которого в $PV$-координатах является отрезком прямой.
    \end{itemize}
}
\answer{%
    \begin{align*}
    A_{12} &= 0, \Delta U_{12} > 0, \implies Q_{12} = A_{12} + \Delta U_{12} > 0.
    \\
    A_{23} &> 0, \Delta U_{23} > 0, \implies Q_{23} = A_{23} + \Delta U_{23} > 0, \\
    A_{31} &= 0, \Delta U_{31} < 0, \implies Q_{31} = A_{31} + \Delta U_{31} < 0.
    \\
    P_1V_1 &= \nu R T_1, P_2V_2 = \nu R T_2, P_3V_3 = \nu R T_3 \text{ — уравнения состояния идеального газа}, \\
    &\text{Пусть $P_0$, $V_0$, $T_0$ — давление, объём и температура в точке 1 (минимальные во всём цикле):} \\
    P_1 &= P_0, P_2 = P_3, V_1 = V_2 = V_0, \text{остальные соотношения нужно считать} \\
    T_2 &= 2T_1 = 2T_0 \text{(по условию)} \implies \frac{P_2}{P_1} = \frac{P_2V_0}{P_1V_0} = \frac{P_2 V_2}{P_1 V_1}= \frac{\nu R T_2}{\nu R T_1} = \frac{T_2}{T_1} = 2 \implies P_2 = 2 P_1 = 2 P_0, \\
    T_3 &= 6T_2 = 12T_0 \text{(по условию)} \implies \frac{V_3}{V_2} = \frac{P_3V_3}{P_2V_2}= \frac{\nu R T_3}{\nu R T_2} = \frac{T_3}{T_2} = 6 \implies V_3 = 6 V_2 = 6 V_0.
    \\
    A_\text{цикл} &= \frac 12 (6P_0 - P_0)(2V_0 - V_0) = \frac 12 \cdot 5 \cdot P_0V_0, \\
    A_{23} &= 2P_0 \cdot (6V_0 - V_0) = 10P_0V_0, \\
    \Delta U_{23} &= \frac 32 \nu R T_3 - \frac 32 \nu R T_3 = \frac 32 P_3 V_3 - \frac 32 P_2 V_2 = \frac 32 \cdot 2 P_0 \cdot 6 V_0 -  \frac 32 \cdot 2 P_0 \cdot V_0 = \frac 32 \cdot 10 \cdot P_0V_0, \\
    \Delta U_{12} &= \frac 32 \nu R T_2 - \frac 32 \nu R T_1 = \frac 32 P_2 V_2 - \frac 32 P_1 V_1 = \frac 32 \cdot 2 P_0V_0 - \frac 32 P_0V_0 = \frac 32 \cdot 1 \cdot P_0V_0.
    \\
    \eta &= \frac{A_\text{цикл}}{Q_+} = \frac{A_\text{цикл}}{Q_{12} + Q_{23}}  = \frac{A_\text{цикл}}{A_{12} + \Delta U_{12} + A_{23} + \Delta U_{23}} =  \\
     &= \frac{\frac 12 \cdot 5 \cdot P_0V_0}{0 + \frac 32 \cdot 1 \cdot P_0V_0 + 10P_0V_0 + \frac 32 \cdot 10 \cdot P_0V_0} = \frac{\frac 12 \cdot 5}{\frac 32 \cdot 1 + 10 + \frac 32 \cdot 10} = \frac{5}{53} \approx 0{,}094.
    \end{align*}
}
\solutionspace{360pt}

\tasknumber{2}%
\task{%
    Определите КПД (оставив ответ точным в виде нескоратимой дроби) цикла 1231, рабочим телом которого является идеальный одноатомный газ, если
    \begin{itemize}
        \item 12 — изобарическое расширение,
        \item 23 — процесс, график которого в $PV$-координатах является отрезком прямой, а объём уменьшается в шесть раз,
        \item 31 — изохорический нагрев с увеличением давления в два раза,
    \end{itemize}
}
\answer{%
    \begin{align*}
    A_{12} &> 0, \Delta U_{12} > 0, \implies Q_{12} = A_{12} + \Delta U_{12} > 0.
    \\
    A_{23} &< 0, \Delta U_{23} < 0, \implies Q_{23} = A_{23} + \Delta U_{23} < 0, \\
    A_{31} &= 0, \Delta U_{31} > 0, \implies Q_{31} = A_{31} + \Delta U_{31} > 0.
    \\
    P_1V_1 &= \nu R T_1, P_2V_2 = \nu R T_2, P_3V_3 = \nu R T_3 \text{ — уравнения состояния идеального газа}, \\
    &\text{Пусть $P_0$, $V_0$, $T_0$ — давление, объём и температура в точке 3 (минимальные во всём цикле):} \\
    P_3 &= P_0, P_1 = P_2 = 2P_0, V_1 = V_3 = V_0, V_2 = 6V_3 = 6V_0 \\
    A_\text{цикл} &= \frac 12 (P_2-P_1)(V_1-V_2) = \frac 12 (2P_0 - P_0)(6V_0 - V_0) = \frac 12 \cdot 5 \cdot P_0V_0, \\
    A_{12} &= 6P_0 \cdot (2V_0 - V_0) = 6P_0V_0, \\
    \Delta U_{12} &= \frac 32 \nu R T_2 - \frac 32 \nu R T_1 = \frac 32 P_2 V_2 - \frac 32 P_1 V_1 = \frac 32 \cdot 2 P_0 \cdot 6 V_0 -  \frac 32 \cdot 6 P_0 \cdot V_0 = \frac 32 \cdot 6 \cdot P_0V_0, \\
    \Delta U_{31} &= \frac 32 \nu R T_1 - \frac 32 \nu R T_3 = \frac 32 P_1 V_1 - \frac 32 P_3 V_3 = \frac 32 \cdot 2 P_0V_0 - \frac 32 P_0V_0 = \frac 32 \cdot 1 \cdot P_0V_0.
    \\
    \eta &= \frac{A_\text{цикл}}{Q_+} = \frac{A_\text{цикл}}{Q_{12} + Q_{31}}  = \frac{A_\text{цикл}}{A_{12} + \Delta U_{12} + A_{31} + \Delta U_{31}} =  \\
     &= \frac{\frac 12 \cdot 5 \cdot P_0V_0}{6P_0V_0 + \frac 32 \cdot 6 \cdot P_0V_0 + 0 + \frac 32 \cdot 1 \cdot P_0V_0} = \frac{\frac 12 \cdot 5}{6 + \frac 32 \cdot 6 + \frac 32 \cdot 1} = \frac{5}{33} \approx 0{,}152.
    \end{align*}
}

\variantsplitter

\addpersonalvariant{Алёна Куприянова}

\tasknumber{1}%
\task{%
    Определите КПД (оставив ответ точным в виде нескоратимой дроби) цикла 1231, рабочим телом которого является идеальный одноатомный газ, если
    \begin{itemize}
        \item 12 — изохорический нагрев в четыре раза,
        \item 23 — изобарическое расширение, при котором температура растёт в пять раз,
        \item 31 — процесс, график которого в $PV$-координатах является отрезком прямой.
    \end{itemize}
}
\answer{%
    \begin{align*}
    A_{12} &= 0, \Delta U_{12} > 0, \implies Q_{12} = A_{12} + \Delta U_{12} > 0.
    \\
    A_{23} &> 0, \Delta U_{23} > 0, \implies Q_{23} = A_{23} + \Delta U_{23} > 0, \\
    A_{31} &= 0, \Delta U_{31} < 0, \implies Q_{31} = A_{31} + \Delta U_{31} < 0.
    \\
    P_1V_1 &= \nu R T_1, P_2V_2 = \nu R T_2, P_3V_3 = \nu R T_3 \text{ — уравнения состояния идеального газа}, \\
    &\text{Пусть $P_0$, $V_0$, $T_0$ — давление, объём и температура в точке 1 (минимальные во всём цикле):} \\
    P_1 &= P_0, P_2 = P_3, V_1 = V_2 = V_0, \text{остальные соотношения нужно считать} \\
    T_2 &= 4T_1 = 4T_0 \text{(по условию)} \implies \frac{P_2}{P_1} = \frac{P_2V_0}{P_1V_0} = \frac{P_2 V_2}{P_1 V_1}= \frac{\nu R T_2}{\nu R T_1} = \frac{T_2}{T_1} = 4 \implies P_2 = 4 P_1 = 4 P_0, \\
    T_3 &= 5T_2 = 20T_0 \text{(по условию)} \implies \frac{V_3}{V_2} = \frac{P_3V_3}{P_2V_2}= \frac{\nu R T_3}{\nu R T_2} = \frac{T_3}{T_2} = 5 \implies V_3 = 5 V_2 = 5 V_0.
    \\
    A_\text{цикл} &= \frac 12 (5P_0 - P_0)(4V_0 - V_0) = \frac 12 \cdot 12 \cdot P_0V_0, \\
    A_{23} &= 4P_0 \cdot (5V_0 - V_0) = 16P_0V_0, \\
    \Delta U_{23} &= \frac 32 \nu R T_3 - \frac 32 \nu R T_3 = \frac 32 P_3 V_3 - \frac 32 P_2 V_2 = \frac 32 \cdot 4 P_0 \cdot 5 V_0 -  \frac 32 \cdot 4 P_0 \cdot V_0 = \frac 32 \cdot 16 \cdot P_0V_0, \\
    \Delta U_{12} &= \frac 32 \nu R T_2 - \frac 32 \nu R T_1 = \frac 32 P_2 V_2 - \frac 32 P_1 V_1 = \frac 32 \cdot 4 P_0V_0 - \frac 32 P_0V_0 = \frac 32 \cdot 3 \cdot P_0V_0.
    \\
    \eta &= \frac{A_\text{цикл}}{Q_+} = \frac{A_\text{цикл}}{Q_{12} + Q_{23}}  = \frac{A_\text{цикл}}{A_{12} + \Delta U_{12} + A_{23} + \Delta U_{23}} =  \\
     &= \frac{\frac 12 \cdot 12 \cdot P_0V_0}{0 + \frac 32 \cdot 3 \cdot P_0V_0 + 16P_0V_0 + \frac 32 \cdot 16 \cdot P_0V_0} = \frac{\frac 12 \cdot 12}{\frac 32 \cdot 3 + 16 + \frac 32 \cdot 16} = \frac{12}{89} \approx 0{,}135.
    \end{align*}
}
\solutionspace{360pt}

\tasknumber{2}%
\task{%
    Определите КПД (оставив ответ точным в виде нескоратимой дроби) цикла 1231, рабочим телом которого является идеальный одноатомный газ, если
    \begin{itemize}
        \item 12 — изобарическое расширение,
        \item 23 — процесс, график которого в $PV$-координатах является отрезком прямой, а объём уменьшается в пять раз,
        \item 31 — изохорический нагрев с увеличением давления в четыре раза,
    \end{itemize}
}
\answer{%
    \begin{align*}
    A_{12} &> 0, \Delta U_{12} > 0, \implies Q_{12} = A_{12} + \Delta U_{12} > 0.
    \\
    A_{23} &< 0, \Delta U_{23} < 0, \implies Q_{23} = A_{23} + \Delta U_{23} < 0, \\
    A_{31} &= 0, \Delta U_{31} > 0, \implies Q_{31} = A_{31} + \Delta U_{31} > 0.
    \\
    P_1V_1 &= \nu R T_1, P_2V_2 = \nu R T_2, P_3V_3 = \nu R T_3 \text{ — уравнения состояния идеального газа}, \\
    &\text{Пусть $P_0$, $V_0$, $T_0$ — давление, объём и температура в точке 3 (минимальные во всём цикле):} \\
    P_3 &= P_0, P_1 = P_2 = 4P_0, V_1 = V_3 = V_0, V_2 = 5V_3 = 5V_0 \\
    A_\text{цикл} &= \frac 12 (P_2-P_1)(V_1-V_2) = \frac 12 (4P_0 - P_0)(5V_0 - V_0) = \frac 12 \cdot 12 \cdot P_0V_0, \\
    A_{12} &= 5P_0 \cdot (4V_0 - V_0) = 15P_0V_0, \\
    \Delta U_{12} &= \frac 32 \nu R T_2 - \frac 32 \nu R T_1 = \frac 32 P_2 V_2 - \frac 32 P_1 V_1 = \frac 32 \cdot 4 P_0 \cdot 5 V_0 -  \frac 32 \cdot 5 P_0 \cdot V_0 = \frac 32 \cdot 15 \cdot P_0V_0, \\
    \Delta U_{31} &= \frac 32 \nu R T_1 - \frac 32 \nu R T_3 = \frac 32 P_1 V_1 - \frac 32 P_3 V_3 = \frac 32 \cdot 4 P_0V_0 - \frac 32 P_0V_0 = \frac 32 \cdot 3 \cdot P_0V_0.
    \\
    \eta &= \frac{A_\text{цикл}}{Q_+} = \frac{A_\text{цикл}}{Q_{12} + Q_{31}}  = \frac{A_\text{цикл}}{A_{12} + \Delta U_{12} + A_{31} + \Delta U_{31}} =  \\
     &= \frac{\frac 12 \cdot 12 \cdot P_0V_0}{15P_0V_0 + \frac 32 \cdot 15 \cdot P_0V_0 + 0 + \frac 32 \cdot 3 \cdot P_0V_0} = \frac{\frac 12 \cdot 12}{15 + \frac 32 \cdot 15 + \frac 32 \cdot 3} = \frac{1}{7} \approx 0{,}143.
    \end{align*}
}

\variantsplitter

\addpersonalvariant{Ярослав Лавровский}

\tasknumber{1}%
\task{%
    Определите КПД (оставив ответ точным в виде нескоратимой дроби) цикла 1231, рабочим телом которого является идеальный одноатомный газ, если
    \begin{itemize}
        \item 12 — изохорический нагрев в три раза,
        \item 23 — изобарическое расширение, при котором температура растёт в шесть раз,
        \item 31 — процесс, график которого в $PV$-координатах является отрезком прямой.
    \end{itemize}
}
\answer{%
    \begin{align*}
    A_{12} &= 0, \Delta U_{12} > 0, \implies Q_{12} = A_{12} + \Delta U_{12} > 0.
    \\
    A_{23} &> 0, \Delta U_{23} > 0, \implies Q_{23} = A_{23} + \Delta U_{23} > 0, \\
    A_{31} &= 0, \Delta U_{31} < 0, \implies Q_{31} = A_{31} + \Delta U_{31} < 0.
    \\
    P_1V_1 &= \nu R T_1, P_2V_2 = \nu R T_2, P_3V_3 = \nu R T_3 \text{ — уравнения состояния идеального газа}, \\
    &\text{Пусть $P_0$, $V_0$, $T_0$ — давление, объём и температура в точке 1 (минимальные во всём цикле):} \\
    P_1 &= P_0, P_2 = P_3, V_1 = V_2 = V_0, \text{остальные соотношения нужно считать} \\
    T_2 &= 3T_1 = 3T_0 \text{(по условию)} \implies \frac{P_2}{P_1} = \frac{P_2V_0}{P_1V_0} = \frac{P_2 V_2}{P_1 V_1}= \frac{\nu R T_2}{\nu R T_1} = \frac{T_2}{T_1} = 3 \implies P_2 = 3 P_1 = 3 P_0, \\
    T_3 &= 6T_2 = 18T_0 \text{(по условию)} \implies \frac{V_3}{V_2} = \frac{P_3V_3}{P_2V_2}= \frac{\nu R T_3}{\nu R T_2} = \frac{T_3}{T_2} = 6 \implies V_3 = 6 V_2 = 6 V_0.
    \\
    A_\text{цикл} &= \frac 12 (6P_0 - P_0)(3V_0 - V_0) = \frac 12 \cdot 10 \cdot P_0V_0, \\
    A_{23} &= 3P_0 \cdot (6V_0 - V_0) = 15P_0V_0, \\
    \Delta U_{23} &= \frac 32 \nu R T_3 - \frac 32 \nu R T_3 = \frac 32 P_3 V_3 - \frac 32 P_2 V_2 = \frac 32 \cdot 3 P_0 \cdot 6 V_0 -  \frac 32 \cdot 3 P_0 \cdot V_0 = \frac 32 \cdot 15 \cdot P_0V_0, \\
    \Delta U_{12} &= \frac 32 \nu R T_2 - \frac 32 \nu R T_1 = \frac 32 P_2 V_2 - \frac 32 P_1 V_1 = \frac 32 \cdot 3 P_0V_0 - \frac 32 P_0V_0 = \frac 32 \cdot 2 \cdot P_0V_0.
    \\
    \eta &= \frac{A_\text{цикл}}{Q_+} = \frac{A_\text{цикл}}{Q_{12} + Q_{23}}  = \frac{A_\text{цикл}}{A_{12} + \Delta U_{12} + A_{23} + \Delta U_{23}} =  \\
     &= \frac{\frac 12 \cdot 10 \cdot P_0V_0}{0 + \frac 32 \cdot 2 \cdot P_0V_0 + 15P_0V_0 + \frac 32 \cdot 15 \cdot P_0V_0} = \frac{\frac 12 \cdot 10}{\frac 32 \cdot 2 + 15 + \frac 32 \cdot 15} = \frac{10}{81} \approx 0{,}123.
    \end{align*}
}
\solutionspace{360pt}

\tasknumber{2}%
\task{%
    Определите КПД (оставив ответ точным в виде нескоратимой дроби) цикла 1231, рабочим телом которого является идеальный одноатомный газ, если
    \begin{itemize}
        \item 12 — изобарическое расширение,
        \item 23 — процесс, график которого в $PV$-координатах является отрезком прямой, а объём уменьшается в шесть раз,
        \item 31 — изохорический нагрев с увеличением давления в три раза,
    \end{itemize}
}
\answer{%
    \begin{align*}
    A_{12} &> 0, \Delta U_{12} > 0, \implies Q_{12} = A_{12} + \Delta U_{12} > 0.
    \\
    A_{23} &< 0, \Delta U_{23} < 0, \implies Q_{23} = A_{23} + \Delta U_{23} < 0, \\
    A_{31} &= 0, \Delta U_{31} > 0, \implies Q_{31} = A_{31} + \Delta U_{31} > 0.
    \\
    P_1V_1 &= \nu R T_1, P_2V_2 = \nu R T_2, P_3V_3 = \nu R T_3 \text{ — уравнения состояния идеального газа}, \\
    &\text{Пусть $P_0$, $V_0$, $T_0$ — давление, объём и температура в точке 3 (минимальные во всём цикле):} \\
    P_3 &= P_0, P_1 = P_2 = 3P_0, V_1 = V_3 = V_0, V_2 = 6V_3 = 6V_0 \\
    A_\text{цикл} &= \frac 12 (P_2-P_1)(V_1-V_2) = \frac 12 (3P_0 - P_0)(6V_0 - V_0) = \frac 12 \cdot 10 \cdot P_0V_0, \\
    A_{12} &= 6P_0 \cdot (3V_0 - V_0) = 12P_0V_0, \\
    \Delta U_{12} &= \frac 32 \nu R T_2 - \frac 32 \nu R T_1 = \frac 32 P_2 V_2 - \frac 32 P_1 V_1 = \frac 32 \cdot 3 P_0 \cdot 6 V_0 -  \frac 32 \cdot 6 P_0 \cdot V_0 = \frac 32 \cdot 12 \cdot P_0V_0, \\
    \Delta U_{31} &= \frac 32 \nu R T_1 - \frac 32 \nu R T_3 = \frac 32 P_1 V_1 - \frac 32 P_3 V_3 = \frac 32 \cdot 3 P_0V_0 - \frac 32 P_0V_0 = \frac 32 \cdot 2 \cdot P_0V_0.
    \\
    \eta &= \frac{A_\text{цикл}}{Q_+} = \frac{A_\text{цикл}}{Q_{12} + Q_{31}}  = \frac{A_\text{цикл}}{A_{12} + \Delta U_{12} + A_{31} + \Delta U_{31}} =  \\
     &= \frac{\frac 12 \cdot 10 \cdot P_0V_0}{12P_0V_0 + \frac 32 \cdot 12 \cdot P_0V_0 + 0 + \frac 32 \cdot 2 \cdot P_0V_0} = \frac{\frac 12 \cdot 10}{12 + \frac 32 \cdot 12 + \frac 32 \cdot 2} = \frac{5}{33} \approx 0{,}152.
    \end{align*}
}

\variantsplitter

\addpersonalvariant{Анастасия Ламанова}

\tasknumber{1}%
\task{%
    Определите КПД (оставив ответ точным в виде нескоратимой дроби) цикла 1231, рабочим телом которого является идеальный одноатомный газ, если
    \begin{itemize}
        \item 12 — изохорический нагрев в три раза,
        \item 23 — изобарическое расширение, при котором температура растёт в три раза,
        \item 31 — процесс, график которого в $PV$-координатах является отрезком прямой.
    \end{itemize}
}
\answer{%
    \begin{align*}
    A_{12} &= 0, \Delta U_{12} > 0, \implies Q_{12} = A_{12} + \Delta U_{12} > 0.
    \\
    A_{23} &> 0, \Delta U_{23} > 0, \implies Q_{23} = A_{23} + \Delta U_{23} > 0, \\
    A_{31} &= 0, \Delta U_{31} < 0, \implies Q_{31} = A_{31} + \Delta U_{31} < 0.
    \\
    P_1V_1 &= \nu R T_1, P_2V_2 = \nu R T_2, P_3V_3 = \nu R T_3 \text{ — уравнения состояния идеального газа}, \\
    &\text{Пусть $P_0$, $V_0$, $T_0$ — давление, объём и температура в точке 1 (минимальные во всём цикле):} \\
    P_1 &= P_0, P_2 = P_3, V_1 = V_2 = V_0, \text{остальные соотношения нужно считать} \\
    T_2 &= 3T_1 = 3T_0 \text{(по условию)} \implies \frac{P_2}{P_1} = \frac{P_2V_0}{P_1V_0} = \frac{P_2 V_2}{P_1 V_1}= \frac{\nu R T_2}{\nu R T_1} = \frac{T_2}{T_1} = 3 \implies P_2 = 3 P_1 = 3 P_0, \\
    T_3 &= 3T_2 = 9T_0 \text{(по условию)} \implies \frac{V_3}{V_2} = \frac{P_3V_3}{P_2V_2}= \frac{\nu R T_3}{\nu R T_2} = \frac{T_3}{T_2} = 3 \implies V_3 = 3 V_2 = 3 V_0.
    \\
    A_\text{цикл} &= \frac 12 (3P_0 - P_0)(3V_0 - V_0) = \frac 12 \cdot 4 \cdot P_0V_0, \\
    A_{23} &= 3P_0 \cdot (3V_0 - V_0) = 6P_0V_0, \\
    \Delta U_{23} &= \frac 32 \nu R T_3 - \frac 32 \nu R T_3 = \frac 32 P_3 V_3 - \frac 32 P_2 V_2 = \frac 32 \cdot 3 P_0 \cdot 3 V_0 -  \frac 32 \cdot 3 P_0 \cdot V_0 = \frac 32 \cdot 6 \cdot P_0V_0, \\
    \Delta U_{12} &= \frac 32 \nu R T_2 - \frac 32 \nu R T_1 = \frac 32 P_2 V_2 - \frac 32 P_1 V_1 = \frac 32 \cdot 3 P_0V_0 - \frac 32 P_0V_0 = \frac 32 \cdot 2 \cdot P_0V_0.
    \\
    \eta &= \frac{A_\text{цикл}}{Q_+} = \frac{A_\text{цикл}}{Q_{12} + Q_{23}}  = \frac{A_\text{цикл}}{A_{12} + \Delta U_{12} + A_{23} + \Delta U_{23}} =  \\
     &= \frac{\frac 12 \cdot 4 \cdot P_0V_0}{0 + \frac 32 \cdot 2 \cdot P_0V_0 + 6P_0V_0 + \frac 32 \cdot 6 \cdot P_0V_0} = \frac{\frac 12 \cdot 4}{\frac 32 \cdot 2 + 6 + \frac 32 \cdot 6} = \frac{1}{9} \approx 0{,}111.
    \end{align*}
}
\solutionspace{360pt}

\tasknumber{2}%
\task{%
    Определите КПД (оставив ответ точным в виде нескоратимой дроби) цикла 1231, рабочим телом которого является идеальный одноатомный газ, если
    \begin{itemize}
        \item 12 — изобарическое расширение,
        \item 23 — процесс, график которого в $PV$-координатах является отрезком прямой, а объём уменьшается в три раза,
        \item 31 — изохорический нагрев с увеличением давления в три раза,
    \end{itemize}
}
\answer{%
    \begin{align*}
    A_{12} &> 0, \Delta U_{12} > 0, \implies Q_{12} = A_{12} + \Delta U_{12} > 0.
    \\
    A_{23} &< 0, \Delta U_{23} < 0, \implies Q_{23} = A_{23} + \Delta U_{23} < 0, \\
    A_{31} &= 0, \Delta U_{31} > 0, \implies Q_{31} = A_{31} + \Delta U_{31} > 0.
    \\
    P_1V_1 &= \nu R T_1, P_2V_2 = \nu R T_2, P_3V_3 = \nu R T_3 \text{ — уравнения состояния идеального газа}, \\
    &\text{Пусть $P_0$, $V_0$, $T_0$ — давление, объём и температура в точке 3 (минимальные во всём цикле):} \\
    P_3 &= P_0, P_1 = P_2 = 3P_0, V_1 = V_3 = V_0, V_2 = 3V_3 = 3V_0 \\
    A_\text{цикл} &= \frac 12 (P_2-P_1)(V_1-V_2) = \frac 12 (3P_0 - P_0)(3V_0 - V_0) = \frac 12 \cdot 4 \cdot P_0V_0, \\
    A_{12} &= 3P_0 \cdot (3V_0 - V_0) = 6P_0V_0, \\
    \Delta U_{12} &= \frac 32 \nu R T_2 - \frac 32 \nu R T_1 = \frac 32 P_2 V_2 - \frac 32 P_1 V_1 = \frac 32 \cdot 3 P_0 \cdot 3 V_0 -  \frac 32 \cdot 3 P_0 \cdot V_0 = \frac 32 \cdot 6 \cdot P_0V_0, \\
    \Delta U_{31} &= \frac 32 \nu R T_1 - \frac 32 \nu R T_3 = \frac 32 P_1 V_1 - \frac 32 P_3 V_3 = \frac 32 \cdot 3 P_0V_0 - \frac 32 P_0V_0 = \frac 32 \cdot 2 \cdot P_0V_0.
    \\
    \eta &= \frac{A_\text{цикл}}{Q_+} = \frac{A_\text{цикл}}{Q_{12} + Q_{31}}  = \frac{A_\text{цикл}}{A_{12} + \Delta U_{12} + A_{31} + \Delta U_{31}} =  \\
     &= \frac{\frac 12 \cdot 4 \cdot P_0V_0}{6P_0V_0 + \frac 32 \cdot 6 \cdot P_0V_0 + 0 + \frac 32 \cdot 2 \cdot P_0V_0} = \frac{\frac 12 \cdot 4}{6 + \frac 32 \cdot 6 + \frac 32 \cdot 2} = \frac{1}{9} \approx 0{,}111.
    \end{align*}
}

\variantsplitter

\addpersonalvariant{Виктория Легонькова}

\tasknumber{1}%
\task{%
    Определите КПД (оставив ответ точным в виде нескоратимой дроби) цикла 1231, рабочим телом которого является идеальный одноатомный газ, если
    \begin{itemize}
        \item 12 — изохорический нагрев в пять раз,
        \item 23 — изобарическое расширение, при котором температура растёт в два раза,
        \item 31 — процесс, график которого в $PV$-координатах является отрезком прямой.
    \end{itemize}
}
\answer{%
    \begin{align*}
    A_{12} &= 0, \Delta U_{12} > 0, \implies Q_{12} = A_{12} + \Delta U_{12} > 0.
    \\
    A_{23} &> 0, \Delta U_{23} > 0, \implies Q_{23} = A_{23} + \Delta U_{23} > 0, \\
    A_{31} &= 0, \Delta U_{31} < 0, \implies Q_{31} = A_{31} + \Delta U_{31} < 0.
    \\
    P_1V_1 &= \nu R T_1, P_2V_2 = \nu R T_2, P_3V_3 = \nu R T_3 \text{ — уравнения состояния идеального газа}, \\
    &\text{Пусть $P_0$, $V_0$, $T_0$ — давление, объём и температура в точке 1 (минимальные во всём цикле):} \\
    P_1 &= P_0, P_2 = P_3, V_1 = V_2 = V_0, \text{остальные соотношения нужно считать} \\
    T_2 &= 5T_1 = 5T_0 \text{(по условию)} \implies \frac{P_2}{P_1} = \frac{P_2V_0}{P_1V_0} = \frac{P_2 V_2}{P_1 V_1}= \frac{\nu R T_2}{\nu R T_1} = \frac{T_2}{T_1} = 5 \implies P_2 = 5 P_1 = 5 P_0, \\
    T_3 &= 2T_2 = 10T_0 \text{(по условию)} \implies \frac{V_3}{V_2} = \frac{P_3V_3}{P_2V_2}= \frac{\nu R T_3}{\nu R T_2} = \frac{T_3}{T_2} = 2 \implies V_3 = 2 V_2 = 2 V_0.
    \\
    A_\text{цикл} &= \frac 12 (2P_0 - P_0)(5V_0 - V_0) = \frac 12 \cdot 4 \cdot P_0V_0, \\
    A_{23} &= 5P_0 \cdot (2V_0 - V_0) = 5P_0V_0, \\
    \Delta U_{23} &= \frac 32 \nu R T_3 - \frac 32 \nu R T_3 = \frac 32 P_3 V_3 - \frac 32 P_2 V_2 = \frac 32 \cdot 5 P_0 \cdot 2 V_0 -  \frac 32 \cdot 5 P_0 \cdot V_0 = \frac 32 \cdot 5 \cdot P_0V_0, \\
    \Delta U_{12} &= \frac 32 \nu R T_2 - \frac 32 \nu R T_1 = \frac 32 P_2 V_2 - \frac 32 P_1 V_1 = \frac 32 \cdot 5 P_0V_0 - \frac 32 P_0V_0 = \frac 32 \cdot 4 \cdot P_0V_0.
    \\
    \eta &= \frac{A_\text{цикл}}{Q_+} = \frac{A_\text{цикл}}{Q_{12} + Q_{23}}  = \frac{A_\text{цикл}}{A_{12} + \Delta U_{12} + A_{23} + \Delta U_{23}} =  \\
     &= \frac{\frac 12 \cdot 4 \cdot P_0V_0}{0 + \frac 32 \cdot 4 \cdot P_0V_0 + 5P_0V_0 + \frac 32 \cdot 5 \cdot P_0V_0} = \frac{\frac 12 \cdot 4}{\frac 32 \cdot 4 + 5 + \frac 32 \cdot 5} = \frac{4}{37} \approx 0{,}108.
    \end{align*}
}
\solutionspace{360pt}

\tasknumber{2}%
\task{%
    Определите КПД (оставив ответ точным в виде нескоратимой дроби) цикла 1231, рабочим телом которого является идеальный одноатомный газ, если
    \begin{itemize}
        \item 12 — изобарическое расширение,
        \item 23 — процесс, график которого в $PV$-координатах является отрезком прямой, а объём уменьшается в два раза,
        \item 31 — изохорический нагрев с увеличением давления в пять раз,
    \end{itemize}
}
\answer{%
    \begin{align*}
    A_{12} &> 0, \Delta U_{12} > 0, \implies Q_{12} = A_{12} + \Delta U_{12} > 0.
    \\
    A_{23} &< 0, \Delta U_{23} < 0, \implies Q_{23} = A_{23} + \Delta U_{23} < 0, \\
    A_{31} &= 0, \Delta U_{31} > 0, \implies Q_{31} = A_{31} + \Delta U_{31} > 0.
    \\
    P_1V_1 &= \nu R T_1, P_2V_2 = \nu R T_2, P_3V_3 = \nu R T_3 \text{ — уравнения состояния идеального газа}, \\
    &\text{Пусть $P_0$, $V_0$, $T_0$ — давление, объём и температура в точке 3 (минимальные во всём цикле):} \\
    P_3 &= P_0, P_1 = P_2 = 5P_0, V_1 = V_3 = V_0, V_2 = 2V_3 = 2V_0 \\
    A_\text{цикл} &= \frac 12 (P_2-P_1)(V_1-V_2) = \frac 12 (5P_0 - P_0)(2V_0 - V_0) = \frac 12 \cdot 4 \cdot P_0V_0, \\
    A_{12} &= 2P_0 \cdot (5V_0 - V_0) = 8P_0V_0, \\
    \Delta U_{12} &= \frac 32 \nu R T_2 - \frac 32 \nu R T_1 = \frac 32 P_2 V_2 - \frac 32 P_1 V_1 = \frac 32 \cdot 5 P_0 \cdot 2 V_0 -  \frac 32 \cdot 2 P_0 \cdot V_0 = \frac 32 \cdot 8 \cdot P_0V_0, \\
    \Delta U_{31} &= \frac 32 \nu R T_1 - \frac 32 \nu R T_3 = \frac 32 P_1 V_1 - \frac 32 P_3 V_3 = \frac 32 \cdot 5 P_0V_0 - \frac 32 P_0V_0 = \frac 32 \cdot 4 \cdot P_0V_0.
    \\
    \eta &= \frac{A_\text{цикл}}{Q_+} = \frac{A_\text{цикл}}{Q_{12} + Q_{31}}  = \frac{A_\text{цикл}}{A_{12} + \Delta U_{12} + A_{31} + \Delta U_{31}} =  \\
     &= \frac{\frac 12 \cdot 4 \cdot P_0V_0}{8P_0V_0 + \frac 32 \cdot 8 \cdot P_0V_0 + 0 + \frac 32 \cdot 4 \cdot P_0V_0} = \frac{\frac 12 \cdot 4}{8 + \frac 32 \cdot 8 + \frac 32 \cdot 4} = \frac{1}{13} \approx 0{,}077.
    \end{align*}
}

\variantsplitter

\addpersonalvariant{Семён Мартынов}

\tasknumber{1}%
\task{%
    Определите КПД (оставив ответ точным в виде нескоратимой дроби) цикла 1231, рабочим телом которого является идеальный одноатомный газ, если
    \begin{itemize}
        \item 12 — изохорический нагрев в шесть раз,
        \item 23 — изобарическое расширение, при котором температура растёт в четыре раза,
        \item 31 — процесс, график которого в $PV$-координатах является отрезком прямой.
    \end{itemize}
}
\answer{%
    \begin{align*}
    A_{12} &= 0, \Delta U_{12} > 0, \implies Q_{12} = A_{12} + \Delta U_{12} > 0.
    \\
    A_{23} &> 0, \Delta U_{23} > 0, \implies Q_{23} = A_{23} + \Delta U_{23} > 0, \\
    A_{31} &= 0, \Delta U_{31} < 0, \implies Q_{31} = A_{31} + \Delta U_{31} < 0.
    \\
    P_1V_1 &= \nu R T_1, P_2V_2 = \nu R T_2, P_3V_3 = \nu R T_3 \text{ — уравнения состояния идеального газа}, \\
    &\text{Пусть $P_0$, $V_0$, $T_0$ — давление, объём и температура в точке 1 (минимальные во всём цикле):} \\
    P_1 &= P_0, P_2 = P_3, V_1 = V_2 = V_0, \text{остальные соотношения нужно считать} \\
    T_2 &= 6T_1 = 6T_0 \text{(по условию)} \implies \frac{P_2}{P_1} = \frac{P_2V_0}{P_1V_0} = \frac{P_2 V_2}{P_1 V_1}= \frac{\nu R T_2}{\nu R T_1} = \frac{T_2}{T_1} = 6 \implies P_2 = 6 P_1 = 6 P_0, \\
    T_3 &= 4T_2 = 24T_0 \text{(по условию)} \implies \frac{V_3}{V_2} = \frac{P_3V_3}{P_2V_2}= \frac{\nu R T_3}{\nu R T_2} = \frac{T_3}{T_2} = 4 \implies V_3 = 4 V_2 = 4 V_0.
    \\
    A_\text{цикл} &= \frac 12 (4P_0 - P_0)(6V_0 - V_0) = \frac 12 \cdot 15 \cdot P_0V_0, \\
    A_{23} &= 6P_0 \cdot (4V_0 - V_0) = 18P_0V_0, \\
    \Delta U_{23} &= \frac 32 \nu R T_3 - \frac 32 \nu R T_3 = \frac 32 P_3 V_3 - \frac 32 P_2 V_2 = \frac 32 \cdot 6 P_0 \cdot 4 V_0 -  \frac 32 \cdot 6 P_0 \cdot V_0 = \frac 32 \cdot 18 \cdot P_0V_0, \\
    \Delta U_{12} &= \frac 32 \nu R T_2 - \frac 32 \nu R T_1 = \frac 32 P_2 V_2 - \frac 32 P_1 V_1 = \frac 32 \cdot 6 P_0V_0 - \frac 32 P_0V_0 = \frac 32 \cdot 5 \cdot P_0V_0.
    \\
    \eta &= \frac{A_\text{цикл}}{Q_+} = \frac{A_\text{цикл}}{Q_{12} + Q_{23}}  = \frac{A_\text{цикл}}{A_{12} + \Delta U_{12} + A_{23} + \Delta U_{23}} =  \\
     &= \frac{\frac 12 \cdot 15 \cdot P_0V_0}{0 + \frac 32 \cdot 5 \cdot P_0V_0 + 18P_0V_0 + \frac 32 \cdot 18 \cdot P_0V_0} = \frac{\frac 12 \cdot 15}{\frac 32 \cdot 5 + 18 + \frac 32 \cdot 18} = \frac{1}{7} \approx 0{,}143.
    \end{align*}
}
\solutionspace{360pt}

\tasknumber{2}%
\task{%
    Определите КПД (оставив ответ точным в виде нескоратимой дроби) цикла 1231, рабочим телом которого является идеальный одноатомный газ, если
    \begin{itemize}
        \item 12 — изобарическое расширение,
        \item 23 — процесс, график которого в $PV$-координатах является отрезком прямой, а объём уменьшается в четыре раза,
        \item 31 — изохорический нагрев с увеличением давления в шесть раз,
    \end{itemize}
}
\answer{%
    \begin{align*}
    A_{12} &> 0, \Delta U_{12} > 0, \implies Q_{12} = A_{12} + \Delta U_{12} > 0.
    \\
    A_{23} &< 0, \Delta U_{23} < 0, \implies Q_{23} = A_{23} + \Delta U_{23} < 0, \\
    A_{31} &= 0, \Delta U_{31} > 0, \implies Q_{31} = A_{31} + \Delta U_{31} > 0.
    \\
    P_1V_1 &= \nu R T_1, P_2V_2 = \nu R T_2, P_3V_3 = \nu R T_3 \text{ — уравнения состояния идеального газа}, \\
    &\text{Пусть $P_0$, $V_0$, $T_0$ — давление, объём и температура в точке 3 (минимальные во всём цикле):} \\
    P_3 &= P_0, P_1 = P_2 = 6P_0, V_1 = V_3 = V_0, V_2 = 4V_3 = 4V_0 \\
    A_\text{цикл} &= \frac 12 (P_2-P_1)(V_1-V_2) = \frac 12 (6P_0 - P_0)(4V_0 - V_0) = \frac 12 \cdot 15 \cdot P_0V_0, \\
    A_{12} &= 4P_0 \cdot (6V_0 - V_0) = 20P_0V_0, \\
    \Delta U_{12} &= \frac 32 \nu R T_2 - \frac 32 \nu R T_1 = \frac 32 P_2 V_2 - \frac 32 P_1 V_1 = \frac 32 \cdot 6 P_0 \cdot 4 V_0 -  \frac 32 \cdot 4 P_0 \cdot V_0 = \frac 32 \cdot 20 \cdot P_0V_0, \\
    \Delta U_{31} &= \frac 32 \nu R T_1 - \frac 32 \nu R T_3 = \frac 32 P_1 V_1 - \frac 32 P_3 V_3 = \frac 32 \cdot 6 P_0V_0 - \frac 32 P_0V_0 = \frac 32 \cdot 5 \cdot P_0V_0.
    \\
    \eta &= \frac{A_\text{цикл}}{Q_+} = \frac{A_\text{цикл}}{Q_{12} + Q_{31}}  = \frac{A_\text{цикл}}{A_{12} + \Delta U_{12} + A_{31} + \Delta U_{31}} =  \\
     &= \frac{\frac 12 \cdot 15 \cdot P_0V_0}{20P_0V_0 + \frac 32 \cdot 20 \cdot P_0V_0 + 0 + \frac 32 \cdot 5 \cdot P_0V_0} = \frac{\frac 12 \cdot 15}{20 + \frac 32 \cdot 20 + \frac 32 \cdot 5} = \frac{3}{23} \approx 0{,}130.
    \end{align*}
}

\variantsplitter

\addpersonalvariant{Варвара Минаева}

\tasknumber{1}%
\task{%
    Определите КПД (оставив ответ точным в виде нескоратимой дроби) цикла 1231, рабочим телом которого является идеальный одноатомный газ, если
    \begin{itemize}
        \item 12 — изохорический нагрев в два раза,
        \item 23 — изобарическое расширение, при котором температура растёт в четыре раза,
        \item 31 — процесс, график которого в $PV$-координатах является отрезком прямой.
    \end{itemize}
}
\answer{%
    \begin{align*}
    A_{12} &= 0, \Delta U_{12} > 0, \implies Q_{12} = A_{12} + \Delta U_{12} > 0.
    \\
    A_{23} &> 0, \Delta U_{23} > 0, \implies Q_{23} = A_{23} + \Delta U_{23} > 0, \\
    A_{31} &= 0, \Delta U_{31} < 0, \implies Q_{31} = A_{31} + \Delta U_{31} < 0.
    \\
    P_1V_1 &= \nu R T_1, P_2V_2 = \nu R T_2, P_3V_3 = \nu R T_3 \text{ — уравнения состояния идеального газа}, \\
    &\text{Пусть $P_0$, $V_0$, $T_0$ — давление, объём и температура в точке 1 (минимальные во всём цикле):} \\
    P_1 &= P_0, P_2 = P_3, V_1 = V_2 = V_0, \text{остальные соотношения нужно считать} \\
    T_2 &= 2T_1 = 2T_0 \text{(по условию)} \implies \frac{P_2}{P_1} = \frac{P_2V_0}{P_1V_0} = \frac{P_2 V_2}{P_1 V_1}= \frac{\nu R T_2}{\nu R T_1} = \frac{T_2}{T_1} = 2 \implies P_2 = 2 P_1 = 2 P_0, \\
    T_3 &= 4T_2 = 8T_0 \text{(по условию)} \implies \frac{V_3}{V_2} = \frac{P_3V_3}{P_2V_2}= \frac{\nu R T_3}{\nu R T_2} = \frac{T_3}{T_2} = 4 \implies V_3 = 4 V_2 = 4 V_0.
    \\
    A_\text{цикл} &= \frac 12 (4P_0 - P_0)(2V_0 - V_0) = \frac 12 \cdot 3 \cdot P_0V_0, \\
    A_{23} &= 2P_0 \cdot (4V_0 - V_0) = 6P_0V_0, \\
    \Delta U_{23} &= \frac 32 \nu R T_3 - \frac 32 \nu R T_3 = \frac 32 P_3 V_3 - \frac 32 P_2 V_2 = \frac 32 \cdot 2 P_0 \cdot 4 V_0 -  \frac 32 \cdot 2 P_0 \cdot V_0 = \frac 32 \cdot 6 \cdot P_0V_0, \\
    \Delta U_{12} &= \frac 32 \nu R T_2 - \frac 32 \nu R T_1 = \frac 32 P_2 V_2 - \frac 32 P_1 V_1 = \frac 32 \cdot 2 P_0V_0 - \frac 32 P_0V_0 = \frac 32 \cdot 1 \cdot P_0V_0.
    \\
    \eta &= \frac{A_\text{цикл}}{Q_+} = \frac{A_\text{цикл}}{Q_{12} + Q_{23}}  = \frac{A_\text{цикл}}{A_{12} + \Delta U_{12} + A_{23} + \Delta U_{23}} =  \\
     &= \frac{\frac 12 \cdot 3 \cdot P_0V_0}{0 + \frac 32 \cdot 1 \cdot P_0V_0 + 6P_0V_0 + \frac 32 \cdot 6 \cdot P_0V_0} = \frac{\frac 12 \cdot 3}{\frac 32 \cdot 1 + 6 + \frac 32 \cdot 6} = \frac{1}{11} \approx 0{,}091.
    \end{align*}
}
\solutionspace{360pt}

\tasknumber{2}%
\task{%
    Определите КПД (оставив ответ точным в виде нескоратимой дроби) цикла 1231, рабочим телом которого является идеальный одноатомный газ, если
    \begin{itemize}
        \item 12 — изобарическое расширение,
        \item 23 — процесс, график которого в $PV$-координатах является отрезком прямой, а объём уменьшается в четыре раза,
        \item 31 — изохорический нагрев с увеличением давления в два раза,
    \end{itemize}
}
\answer{%
    \begin{align*}
    A_{12} &> 0, \Delta U_{12} > 0, \implies Q_{12} = A_{12} + \Delta U_{12} > 0.
    \\
    A_{23} &< 0, \Delta U_{23} < 0, \implies Q_{23} = A_{23} + \Delta U_{23} < 0, \\
    A_{31} &= 0, \Delta U_{31} > 0, \implies Q_{31} = A_{31} + \Delta U_{31} > 0.
    \\
    P_1V_1 &= \nu R T_1, P_2V_2 = \nu R T_2, P_3V_3 = \nu R T_3 \text{ — уравнения состояния идеального газа}, \\
    &\text{Пусть $P_0$, $V_0$, $T_0$ — давление, объём и температура в точке 3 (минимальные во всём цикле):} \\
    P_3 &= P_0, P_1 = P_2 = 2P_0, V_1 = V_3 = V_0, V_2 = 4V_3 = 4V_0 \\
    A_\text{цикл} &= \frac 12 (P_2-P_1)(V_1-V_2) = \frac 12 (2P_0 - P_0)(4V_0 - V_0) = \frac 12 \cdot 3 \cdot P_0V_0, \\
    A_{12} &= 4P_0 \cdot (2V_0 - V_0) = 4P_0V_0, \\
    \Delta U_{12} &= \frac 32 \nu R T_2 - \frac 32 \nu R T_1 = \frac 32 P_2 V_2 - \frac 32 P_1 V_1 = \frac 32 \cdot 2 P_0 \cdot 4 V_0 -  \frac 32 \cdot 4 P_0 \cdot V_0 = \frac 32 \cdot 4 \cdot P_0V_0, \\
    \Delta U_{31} &= \frac 32 \nu R T_1 - \frac 32 \nu R T_3 = \frac 32 P_1 V_1 - \frac 32 P_3 V_3 = \frac 32 \cdot 2 P_0V_0 - \frac 32 P_0V_0 = \frac 32 \cdot 1 \cdot P_0V_0.
    \\
    \eta &= \frac{A_\text{цикл}}{Q_+} = \frac{A_\text{цикл}}{Q_{12} + Q_{31}}  = \frac{A_\text{цикл}}{A_{12} + \Delta U_{12} + A_{31} + \Delta U_{31}} =  \\
     &= \frac{\frac 12 \cdot 3 \cdot P_0V_0}{4P_0V_0 + \frac 32 \cdot 4 \cdot P_0V_0 + 0 + \frac 32 \cdot 1 \cdot P_0V_0} = \frac{\frac 12 \cdot 3}{4 + \frac 32 \cdot 4 + \frac 32 \cdot 1} = \frac{3}{23} \approx 0{,}130.
    \end{align*}
}

\variantsplitter

\addpersonalvariant{Леонид Никитин}

\tasknumber{1}%
\task{%
    Определите КПД (оставив ответ точным в виде нескоратимой дроби) цикла 1231, рабочим телом которого является идеальный одноатомный газ, если
    \begin{itemize}
        \item 12 — изохорический нагрев в два раза,
        \item 23 — изобарическое расширение, при котором температура растёт в пять раз,
        \item 31 — процесс, график которого в $PV$-координатах является отрезком прямой.
    \end{itemize}
}
\answer{%
    \begin{align*}
    A_{12} &= 0, \Delta U_{12} > 0, \implies Q_{12} = A_{12} + \Delta U_{12} > 0.
    \\
    A_{23} &> 0, \Delta U_{23} > 0, \implies Q_{23} = A_{23} + \Delta U_{23} > 0, \\
    A_{31} &= 0, \Delta U_{31} < 0, \implies Q_{31} = A_{31} + \Delta U_{31} < 0.
    \\
    P_1V_1 &= \nu R T_1, P_2V_2 = \nu R T_2, P_3V_3 = \nu R T_3 \text{ — уравнения состояния идеального газа}, \\
    &\text{Пусть $P_0$, $V_0$, $T_0$ — давление, объём и температура в точке 1 (минимальные во всём цикле):} \\
    P_1 &= P_0, P_2 = P_3, V_1 = V_2 = V_0, \text{остальные соотношения нужно считать} \\
    T_2 &= 2T_1 = 2T_0 \text{(по условию)} \implies \frac{P_2}{P_1} = \frac{P_2V_0}{P_1V_0} = \frac{P_2 V_2}{P_1 V_1}= \frac{\nu R T_2}{\nu R T_1} = \frac{T_2}{T_1} = 2 \implies P_2 = 2 P_1 = 2 P_0, \\
    T_3 &= 5T_2 = 10T_0 \text{(по условию)} \implies \frac{V_3}{V_2} = \frac{P_3V_3}{P_2V_2}= \frac{\nu R T_3}{\nu R T_2} = \frac{T_3}{T_2} = 5 \implies V_3 = 5 V_2 = 5 V_0.
    \\
    A_\text{цикл} &= \frac 12 (5P_0 - P_0)(2V_0 - V_0) = \frac 12 \cdot 4 \cdot P_0V_0, \\
    A_{23} &= 2P_0 \cdot (5V_0 - V_0) = 8P_0V_0, \\
    \Delta U_{23} &= \frac 32 \nu R T_3 - \frac 32 \nu R T_3 = \frac 32 P_3 V_3 - \frac 32 P_2 V_2 = \frac 32 \cdot 2 P_0 \cdot 5 V_0 -  \frac 32 \cdot 2 P_0 \cdot V_0 = \frac 32 \cdot 8 \cdot P_0V_0, \\
    \Delta U_{12} &= \frac 32 \nu R T_2 - \frac 32 \nu R T_1 = \frac 32 P_2 V_2 - \frac 32 P_1 V_1 = \frac 32 \cdot 2 P_0V_0 - \frac 32 P_0V_0 = \frac 32 \cdot 1 \cdot P_0V_0.
    \\
    \eta &= \frac{A_\text{цикл}}{Q_+} = \frac{A_\text{цикл}}{Q_{12} + Q_{23}}  = \frac{A_\text{цикл}}{A_{12} + \Delta U_{12} + A_{23} + \Delta U_{23}} =  \\
     &= \frac{\frac 12 \cdot 4 \cdot P_0V_0}{0 + \frac 32 \cdot 1 \cdot P_0V_0 + 8P_0V_0 + \frac 32 \cdot 8 \cdot P_0V_0} = \frac{\frac 12 \cdot 4}{\frac 32 \cdot 1 + 8 + \frac 32 \cdot 8} = \frac{4}{43} \approx 0{,}093.
    \end{align*}
}
\solutionspace{360pt}

\tasknumber{2}%
\task{%
    Определите КПД (оставив ответ точным в виде нескоратимой дроби) цикла 1231, рабочим телом которого является идеальный одноатомный газ, если
    \begin{itemize}
        \item 12 — изобарическое расширение,
        \item 23 — процесс, график которого в $PV$-координатах является отрезком прямой, а объём уменьшается в пять раз,
        \item 31 — изохорический нагрев с увеличением давления в два раза,
    \end{itemize}
}
\answer{%
    \begin{align*}
    A_{12} &> 0, \Delta U_{12} > 0, \implies Q_{12} = A_{12} + \Delta U_{12} > 0.
    \\
    A_{23} &< 0, \Delta U_{23} < 0, \implies Q_{23} = A_{23} + \Delta U_{23} < 0, \\
    A_{31} &= 0, \Delta U_{31} > 0, \implies Q_{31} = A_{31} + \Delta U_{31} > 0.
    \\
    P_1V_1 &= \nu R T_1, P_2V_2 = \nu R T_2, P_3V_3 = \nu R T_3 \text{ — уравнения состояния идеального газа}, \\
    &\text{Пусть $P_0$, $V_0$, $T_0$ — давление, объём и температура в точке 3 (минимальные во всём цикле):} \\
    P_3 &= P_0, P_1 = P_2 = 2P_0, V_1 = V_3 = V_0, V_2 = 5V_3 = 5V_0 \\
    A_\text{цикл} &= \frac 12 (P_2-P_1)(V_1-V_2) = \frac 12 (2P_0 - P_0)(5V_0 - V_0) = \frac 12 \cdot 4 \cdot P_0V_0, \\
    A_{12} &= 5P_0 \cdot (2V_0 - V_0) = 5P_0V_0, \\
    \Delta U_{12} &= \frac 32 \nu R T_2 - \frac 32 \nu R T_1 = \frac 32 P_2 V_2 - \frac 32 P_1 V_1 = \frac 32 \cdot 2 P_0 \cdot 5 V_0 -  \frac 32 \cdot 5 P_0 \cdot V_0 = \frac 32 \cdot 5 \cdot P_0V_0, \\
    \Delta U_{31} &= \frac 32 \nu R T_1 - \frac 32 \nu R T_3 = \frac 32 P_1 V_1 - \frac 32 P_3 V_3 = \frac 32 \cdot 2 P_0V_0 - \frac 32 P_0V_0 = \frac 32 \cdot 1 \cdot P_0V_0.
    \\
    \eta &= \frac{A_\text{цикл}}{Q_+} = \frac{A_\text{цикл}}{Q_{12} + Q_{31}}  = \frac{A_\text{цикл}}{A_{12} + \Delta U_{12} + A_{31} + \Delta U_{31}} =  \\
     &= \frac{\frac 12 \cdot 4 \cdot P_0V_0}{5P_0V_0 + \frac 32 \cdot 5 \cdot P_0V_0 + 0 + \frac 32 \cdot 1 \cdot P_0V_0} = \frac{\frac 12 \cdot 4}{5 + \frac 32 \cdot 5 + \frac 32 \cdot 1} = \frac{1}{7} \approx 0{,}143.
    \end{align*}
}

\variantsplitter

\addpersonalvariant{Тимофей Полетаев}

\tasknumber{1}%
\task{%
    Определите КПД (оставив ответ точным в виде нескоратимой дроби) цикла 1231, рабочим телом которого является идеальный одноатомный газ, если
    \begin{itemize}
        \item 12 — изохорический нагрев в четыре раза,
        \item 23 — изобарическое расширение, при котором температура растёт в три раза,
        \item 31 — процесс, график которого в $PV$-координатах является отрезком прямой.
    \end{itemize}
}
\answer{%
    \begin{align*}
    A_{12} &= 0, \Delta U_{12} > 0, \implies Q_{12} = A_{12} + \Delta U_{12} > 0.
    \\
    A_{23} &> 0, \Delta U_{23} > 0, \implies Q_{23} = A_{23} + \Delta U_{23} > 0, \\
    A_{31} &= 0, \Delta U_{31} < 0, \implies Q_{31} = A_{31} + \Delta U_{31} < 0.
    \\
    P_1V_1 &= \nu R T_1, P_2V_2 = \nu R T_2, P_3V_3 = \nu R T_3 \text{ — уравнения состояния идеального газа}, \\
    &\text{Пусть $P_0$, $V_0$, $T_0$ — давление, объём и температура в точке 1 (минимальные во всём цикле):} \\
    P_1 &= P_0, P_2 = P_3, V_1 = V_2 = V_0, \text{остальные соотношения нужно считать} \\
    T_2 &= 4T_1 = 4T_0 \text{(по условию)} \implies \frac{P_2}{P_1} = \frac{P_2V_0}{P_1V_0} = \frac{P_2 V_2}{P_1 V_1}= \frac{\nu R T_2}{\nu R T_1} = \frac{T_2}{T_1} = 4 \implies P_2 = 4 P_1 = 4 P_0, \\
    T_3 &= 3T_2 = 12T_0 \text{(по условию)} \implies \frac{V_3}{V_2} = \frac{P_3V_3}{P_2V_2}= \frac{\nu R T_3}{\nu R T_2} = \frac{T_3}{T_2} = 3 \implies V_3 = 3 V_2 = 3 V_0.
    \\
    A_\text{цикл} &= \frac 12 (3P_0 - P_0)(4V_0 - V_0) = \frac 12 \cdot 6 \cdot P_0V_0, \\
    A_{23} &= 4P_0 \cdot (3V_0 - V_0) = 8P_0V_0, \\
    \Delta U_{23} &= \frac 32 \nu R T_3 - \frac 32 \nu R T_3 = \frac 32 P_3 V_3 - \frac 32 P_2 V_2 = \frac 32 \cdot 4 P_0 \cdot 3 V_0 -  \frac 32 \cdot 4 P_0 \cdot V_0 = \frac 32 \cdot 8 \cdot P_0V_0, \\
    \Delta U_{12} &= \frac 32 \nu R T_2 - \frac 32 \nu R T_1 = \frac 32 P_2 V_2 - \frac 32 P_1 V_1 = \frac 32 \cdot 4 P_0V_0 - \frac 32 P_0V_0 = \frac 32 \cdot 3 \cdot P_0V_0.
    \\
    \eta &= \frac{A_\text{цикл}}{Q_+} = \frac{A_\text{цикл}}{Q_{12} + Q_{23}}  = \frac{A_\text{цикл}}{A_{12} + \Delta U_{12} + A_{23} + \Delta U_{23}} =  \\
     &= \frac{\frac 12 \cdot 6 \cdot P_0V_0}{0 + \frac 32 \cdot 3 \cdot P_0V_0 + 8P_0V_0 + \frac 32 \cdot 8 \cdot P_0V_0} = \frac{\frac 12 \cdot 6}{\frac 32 \cdot 3 + 8 + \frac 32 \cdot 8} = \frac{6}{49} \approx 0{,}122.
    \end{align*}
}
\solutionspace{360pt}

\tasknumber{2}%
\task{%
    Определите КПД (оставив ответ точным в виде нескоратимой дроби) цикла 1231, рабочим телом которого является идеальный одноатомный газ, если
    \begin{itemize}
        \item 12 — изобарическое расширение,
        \item 23 — процесс, график которого в $PV$-координатах является отрезком прямой, а объём уменьшается в три раза,
        \item 31 — изохорический нагрев с увеличением давления в четыре раза,
    \end{itemize}
}
\answer{%
    \begin{align*}
    A_{12} &> 0, \Delta U_{12} > 0, \implies Q_{12} = A_{12} + \Delta U_{12} > 0.
    \\
    A_{23} &< 0, \Delta U_{23} < 0, \implies Q_{23} = A_{23} + \Delta U_{23} < 0, \\
    A_{31} &= 0, \Delta U_{31} > 0, \implies Q_{31} = A_{31} + \Delta U_{31} > 0.
    \\
    P_1V_1 &= \nu R T_1, P_2V_2 = \nu R T_2, P_3V_3 = \nu R T_3 \text{ — уравнения состояния идеального газа}, \\
    &\text{Пусть $P_0$, $V_0$, $T_0$ — давление, объём и температура в точке 3 (минимальные во всём цикле):} \\
    P_3 &= P_0, P_1 = P_2 = 4P_0, V_1 = V_3 = V_0, V_2 = 3V_3 = 3V_0 \\
    A_\text{цикл} &= \frac 12 (P_2-P_1)(V_1-V_2) = \frac 12 (4P_0 - P_0)(3V_0 - V_0) = \frac 12 \cdot 6 \cdot P_0V_0, \\
    A_{12} &= 3P_0 \cdot (4V_0 - V_0) = 9P_0V_0, \\
    \Delta U_{12} &= \frac 32 \nu R T_2 - \frac 32 \nu R T_1 = \frac 32 P_2 V_2 - \frac 32 P_1 V_1 = \frac 32 \cdot 4 P_0 \cdot 3 V_0 -  \frac 32 \cdot 3 P_0 \cdot V_0 = \frac 32 \cdot 9 \cdot P_0V_0, \\
    \Delta U_{31} &= \frac 32 \nu R T_1 - \frac 32 \nu R T_3 = \frac 32 P_1 V_1 - \frac 32 P_3 V_3 = \frac 32 \cdot 4 P_0V_0 - \frac 32 P_0V_0 = \frac 32 \cdot 3 \cdot P_0V_0.
    \\
    \eta &= \frac{A_\text{цикл}}{Q_+} = \frac{A_\text{цикл}}{Q_{12} + Q_{31}}  = \frac{A_\text{цикл}}{A_{12} + \Delta U_{12} + A_{31} + \Delta U_{31}} =  \\
     &= \frac{\frac 12 \cdot 6 \cdot P_0V_0}{9P_0V_0 + \frac 32 \cdot 9 \cdot P_0V_0 + 0 + \frac 32 \cdot 3 \cdot P_0V_0} = \frac{\frac 12 \cdot 6}{9 + \frac 32 \cdot 9 + \frac 32 \cdot 3} = \frac{1}{9} \approx 0{,}111.
    \end{align*}
}

\variantsplitter

\addpersonalvariant{Андрей Рожков}

\tasknumber{1}%
\task{%
    Определите КПД (оставив ответ точным в виде нескоратимой дроби) цикла 1231, рабочим телом которого является идеальный одноатомный газ, если
    \begin{itemize}
        \item 12 — изохорический нагрев в четыре раза,
        \item 23 — изобарическое расширение, при котором температура растёт в три раза,
        \item 31 — процесс, график которого в $PV$-координатах является отрезком прямой.
    \end{itemize}
}
\answer{%
    \begin{align*}
    A_{12} &= 0, \Delta U_{12} > 0, \implies Q_{12} = A_{12} + \Delta U_{12} > 0.
    \\
    A_{23} &> 0, \Delta U_{23} > 0, \implies Q_{23} = A_{23} + \Delta U_{23} > 0, \\
    A_{31} &= 0, \Delta U_{31} < 0, \implies Q_{31} = A_{31} + \Delta U_{31} < 0.
    \\
    P_1V_1 &= \nu R T_1, P_2V_2 = \nu R T_2, P_3V_3 = \nu R T_3 \text{ — уравнения состояния идеального газа}, \\
    &\text{Пусть $P_0$, $V_0$, $T_0$ — давление, объём и температура в точке 1 (минимальные во всём цикле):} \\
    P_1 &= P_0, P_2 = P_3, V_1 = V_2 = V_0, \text{остальные соотношения нужно считать} \\
    T_2 &= 4T_1 = 4T_0 \text{(по условию)} \implies \frac{P_2}{P_1} = \frac{P_2V_0}{P_1V_0} = \frac{P_2 V_2}{P_1 V_1}= \frac{\nu R T_2}{\nu R T_1} = \frac{T_2}{T_1} = 4 \implies P_2 = 4 P_1 = 4 P_0, \\
    T_3 &= 3T_2 = 12T_0 \text{(по условию)} \implies \frac{V_3}{V_2} = \frac{P_3V_3}{P_2V_2}= \frac{\nu R T_3}{\nu R T_2} = \frac{T_3}{T_2} = 3 \implies V_3 = 3 V_2 = 3 V_0.
    \\
    A_\text{цикл} &= \frac 12 (3P_0 - P_0)(4V_0 - V_0) = \frac 12 \cdot 6 \cdot P_0V_0, \\
    A_{23} &= 4P_0 \cdot (3V_0 - V_0) = 8P_0V_0, \\
    \Delta U_{23} &= \frac 32 \nu R T_3 - \frac 32 \nu R T_3 = \frac 32 P_3 V_3 - \frac 32 P_2 V_2 = \frac 32 \cdot 4 P_0 \cdot 3 V_0 -  \frac 32 \cdot 4 P_0 \cdot V_0 = \frac 32 \cdot 8 \cdot P_0V_0, \\
    \Delta U_{12} &= \frac 32 \nu R T_2 - \frac 32 \nu R T_1 = \frac 32 P_2 V_2 - \frac 32 P_1 V_1 = \frac 32 \cdot 4 P_0V_0 - \frac 32 P_0V_0 = \frac 32 \cdot 3 \cdot P_0V_0.
    \\
    \eta &= \frac{A_\text{цикл}}{Q_+} = \frac{A_\text{цикл}}{Q_{12} + Q_{23}}  = \frac{A_\text{цикл}}{A_{12} + \Delta U_{12} + A_{23} + \Delta U_{23}} =  \\
     &= \frac{\frac 12 \cdot 6 \cdot P_0V_0}{0 + \frac 32 \cdot 3 \cdot P_0V_0 + 8P_0V_0 + \frac 32 \cdot 8 \cdot P_0V_0} = \frac{\frac 12 \cdot 6}{\frac 32 \cdot 3 + 8 + \frac 32 \cdot 8} = \frac{6}{49} \approx 0{,}122.
    \end{align*}
}
\solutionspace{360pt}

\tasknumber{2}%
\task{%
    Определите КПД (оставив ответ точным в виде нескоратимой дроби) цикла 1231, рабочим телом которого является идеальный одноатомный газ, если
    \begin{itemize}
        \item 12 — изобарическое расширение,
        \item 23 — процесс, график которого в $PV$-координатах является отрезком прямой, а объём уменьшается в три раза,
        \item 31 — изохорический нагрев с увеличением давления в четыре раза,
    \end{itemize}
}
\answer{%
    \begin{align*}
    A_{12} &> 0, \Delta U_{12} > 0, \implies Q_{12} = A_{12} + \Delta U_{12} > 0.
    \\
    A_{23} &< 0, \Delta U_{23} < 0, \implies Q_{23} = A_{23} + \Delta U_{23} < 0, \\
    A_{31} &= 0, \Delta U_{31} > 0, \implies Q_{31} = A_{31} + \Delta U_{31} > 0.
    \\
    P_1V_1 &= \nu R T_1, P_2V_2 = \nu R T_2, P_3V_3 = \nu R T_3 \text{ — уравнения состояния идеального газа}, \\
    &\text{Пусть $P_0$, $V_0$, $T_0$ — давление, объём и температура в точке 3 (минимальные во всём цикле):} \\
    P_3 &= P_0, P_1 = P_2 = 4P_0, V_1 = V_3 = V_0, V_2 = 3V_3 = 3V_0 \\
    A_\text{цикл} &= \frac 12 (P_2-P_1)(V_1-V_2) = \frac 12 (4P_0 - P_0)(3V_0 - V_0) = \frac 12 \cdot 6 \cdot P_0V_0, \\
    A_{12} &= 3P_0 \cdot (4V_0 - V_0) = 9P_0V_0, \\
    \Delta U_{12} &= \frac 32 \nu R T_2 - \frac 32 \nu R T_1 = \frac 32 P_2 V_2 - \frac 32 P_1 V_1 = \frac 32 \cdot 4 P_0 \cdot 3 V_0 -  \frac 32 \cdot 3 P_0 \cdot V_0 = \frac 32 \cdot 9 \cdot P_0V_0, \\
    \Delta U_{31} &= \frac 32 \nu R T_1 - \frac 32 \nu R T_3 = \frac 32 P_1 V_1 - \frac 32 P_3 V_3 = \frac 32 \cdot 4 P_0V_0 - \frac 32 P_0V_0 = \frac 32 \cdot 3 \cdot P_0V_0.
    \\
    \eta &= \frac{A_\text{цикл}}{Q_+} = \frac{A_\text{цикл}}{Q_{12} + Q_{31}}  = \frac{A_\text{цикл}}{A_{12} + \Delta U_{12} + A_{31} + \Delta U_{31}} =  \\
     &= \frac{\frac 12 \cdot 6 \cdot P_0V_0}{9P_0V_0 + \frac 32 \cdot 9 \cdot P_0V_0 + 0 + \frac 32 \cdot 3 \cdot P_0V_0} = \frac{\frac 12 \cdot 6}{9 + \frac 32 \cdot 9 + \frac 32 \cdot 3} = \frac{1}{9} \approx 0{,}111.
    \end{align*}
}

\variantsplitter

\addpersonalvariant{Рената Таржиманова}

\tasknumber{1}%
\task{%
    Определите КПД (оставив ответ точным в виде нескоратимой дроби) цикла 1231, рабочим телом которого является идеальный одноатомный газ, если
    \begin{itemize}
        \item 12 — изохорический нагрев в три раза,
        \item 23 — изобарическое расширение, при котором температура растёт в два раза,
        \item 31 — процесс, график которого в $PV$-координатах является отрезком прямой.
    \end{itemize}
}
\answer{%
    \begin{align*}
    A_{12} &= 0, \Delta U_{12} > 0, \implies Q_{12} = A_{12} + \Delta U_{12} > 0.
    \\
    A_{23} &> 0, \Delta U_{23} > 0, \implies Q_{23} = A_{23} + \Delta U_{23} > 0, \\
    A_{31} &= 0, \Delta U_{31} < 0, \implies Q_{31} = A_{31} + \Delta U_{31} < 0.
    \\
    P_1V_1 &= \nu R T_1, P_2V_2 = \nu R T_2, P_3V_3 = \nu R T_3 \text{ — уравнения состояния идеального газа}, \\
    &\text{Пусть $P_0$, $V_0$, $T_0$ — давление, объём и температура в точке 1 (минимальные во всём цикле):} \\
    P_1 &= P_0, P_2 = P_3, V_1 = V_2 = V_0, \text{остальные соотношения нужно считать} \\
    T_2 &= 3T_1 = 3T_0 \text{(по условию)} \implies \frac{P_2}{P_1} = \frac{P_2V_0}{P_1V_0} = \frac{P_2 V_2}{P_1 V_1}= \frac{\nu R T_2}{\nu R T_1} = \frac{T_2}{T_1} = 3 \implies P_2 = 3 P_1 = 3 P_0, \\
    T_3 &= 2T_2 = 6T_0 \text{(по условию)} \implies \frac{V_3}{V_2} = \frac{P_3V_3}{P_2V_2}= \frac{\nu R T_3}{\nu R T_2} = \frac{T_3}{T_2} = 2 \implies V_3 = 2 V_2 = 2 V_0.
    \\
    A_\text{цикл} &= \frac 12 (2P_0 - P_0)(3V_0 - V_0) = \frac 12 \cdot 2 \cdot P_0V_0, \\
    A_{23} &= 3P_0 \cdot (2V_0 - V_0) = 3P_0V_0, \\
    \Delta U_{23} &= \frac 32 \nu R T_3 - \frac 32 \nu R T_3 = \frac 32 P_3 V_3 - \frac 32 P_2 V_2 = \frac 32 \cdot 3 P_0 \cdot 2 V_0 -  \frac 32 \cdot 3 P_0 \cdot V_0 = \frac 32 \cdot 3 \cdot P_0V_0, \\
    \Delta U_{12} &= \frac 32 \nu R T_2 - \frac 32 \nu R T_1 = \frac 32 P_2 V_2 - \frac 32 P_1 V_1 = \frac 32 \cdot 3 P_0V_0 - \frac 32 P_0V_0 = \frac 32 \cdot 2 \cdot P_0V_0.
    \\
    \eta &= \frac{A_\text{цикл}}{Q_+} = \frac{A_\text{цикл}}{Q_{12} + Q_{23}}  = \frac{A_\text{цикл}}{A_{12} + \Delta U_{12} + A_{23} + \Delta U_{23}} =  \\
     &= \frac{\frac 12 \cdot 2 \cdot P_0V_0}{0 + \frac 32 \cdot 2 \cdot P_0V_0 + 3P_0V_0 + \frac 32 \cdot 3 \cdot P_0V_0} = \frac{\frac 12 \cdot 2}{\frac 32 \cdot 2 + 3 + \frac 32 \cdot 3} = \frac{2}{21} \approx 0{,}095.
    \end{align*}
}
\solutionspace{360pt}

\tasknumber{2}%
\task{%
    Определите КПД (оставив ответ точным в виде нескоратимой дроби) цикла 1231, рабочим телом которого является идеальный одноатомный газ, если
    \begin{itemize}
        \item 12 — изобарическое расширение,
        \item 23 — процесс, график которого в $PV$-координатах является отрезком прямой, а объём уменьшается в два раза,
        \item 31 — изохорический нагрев с увеличением давления в три раза,
    \end{itemize}
}
\answer{%
    \begin{align*}
    A_{12} &> 0, \Delta U_{12} > 0, \implies Q_{12} = A_{12} + \Delta U_{12} > 0.
    \\
    A_{23} &< 0, \Delta U_{23} < 0, \implies Q_{23} = A_{23} + \Delta U_{23} < 0, \\
    A_{31} &= 0, \Delta U_{31} > 0, \implies Q_{31} = A_{31} + \Delta U_{31} > 0.
    \\
    P_1V_1 &= \nu R T_1, P_2V_2 = \nu R T_2, P_3V_3 = \nu R T_3 \text{ — уравнения состояния идеального газа}, \\
    &\text{Пусть $P_0$, $V_0$, $T_0$ — давление, объём и температура в точке 3 (минимальные во всём цикле):} \\
    P_3 &= P_0, P_1 = P_2 = 3P_0, V_1 = V_3 = V_0, V_2 = 2V_3 = 2V_0 \\
    A_\text{цикл} &= \frac 12 (P_2-P_1)(V_1-V_2) = \frac 12 (3P_0 - P_0)(2V_0 - V_0) = \frac 12 \cdot 2 \cdot P_0V_0, \\
    A_{12} &= 2P_0 \cdot (3V_0 - V_0) = 4P_0V_0, \\
    \Delta U_{12} &= \frac 32 \nu R T_2 - \frac 32 \nu R T_1 = \frac 32 P_2 V_2 - \frac 32 P_1 V_1 = \frac 32 \cdot 3 P_0 \cdot 2 V_0 -  \frac 32 \cdot 2 P_0 \cdot V_0 = \frac 32 \cdot 4 \cdot P_0V_0, \\
    \Delta U_{31} &= \frac 32 \nu R T_1 - \frac 32 \nu R T_3 = \frac 32 P_1 V_1 - \frac 32 P_3 V_3 = \frac 32 \cdot 3 P_0V_0 - \frac 32 P_0V_0 = \frac 32 \cdot 2 \cdot P_0V_0.
    \\
    \eta &= \frac{A_\text{цикл}}{Q_+} = \frac{A_\text{цикл}}{Q_{12} + Q_{31}}  = \frac{A_\text{цикл}}{A_{12} + \Delta U_{12} + A_{31} + \Delta U_{31}} =  \\
     &= \frac{\frac 12 \cdot 2 \cdot P_0V_0}{4P_0V_0 + \frac 32 \cdot 4 \cdot P_0V_0 + 0 + \frac 32 \cdot 2 \cdot P_0V_0} = \frac{\frac 12 \cdot 2}{4 + \frac 32 \cdot 4 + \frac 32 \cdot 2} = \frac{1}{13} \approx 0{,}077.
    \end{align*}
}

\variantsplitter

\addpersonalvariant{Андрей Щербаков}

\tasknumber{1}%
\task{%
    Определите КПД (оставив ответ точным в виде нескоратимой дроби) цикла 1231, рабочим телом которого является идеальный одноатомный газ, если
    \begin{itemize}
        \item 12 — изохорический нагрев в четыре раза,
        \item 23 — изобарическое расширение, при котором температура растёт в шесть раз,
        \item 31 — процесс, график которого в $PV$-координатах является отрезком прямой.
    \end{itemize}
}
\answer{%
    \begin{align*}
    A_{12} &= 0, \Delta U_{12} > 0, \implies Q_{12} = A_{12} + \Delta U_{12} > 0.
    \\
    A_{23} &> 0, \Delta U_{23} > 0, \implies Q_{23} = A_{23} + \Delta U_{23} > 0, \\
    A_{31} &= 0, \Delta U_{31} < 0, \implies Q_{31} = A_{31} + \Delta U_{31} < 0.
    \\
    P_1V_1 &= \nu R T_1, P_2V_2 = \nu R T_2, P_3V_3 = \nu R T_3 \text{ — уравнения состояния идеального газа}, \\
    &\text{Пусть $P_0$, $V_0$, $T_0$ — давление, объём и температура в точке 1 (минимальные во всём цикле):} \\
    P_1 &= P_0, P_2 = P_3, V_1 = V_2 = V_0, \text{остальные соотношения нужно считать} \\
    T_2 &= 4T_1 = 4T_0 \text{(по условию)} \implies \frac{P_2}{P_1} = \frac{P_2V_0}{P_1V_0} = \frac{P_2 V_2}{P_1 V_1}= \frac{\nu R T_2}{\nu R T_1} = \frac{T_2}{T_1} = 4 \implies P_2 = 4 P_1 = 4 P_0, \\
    T_3 &= 6T_2 = 24T_0 \text{(по условию)} \implies \frac{V_3}{V_2} = \frac{P_3V_3}{P_2V_2}= \frac{\nu R T_3}{\nu R T_2} = \frac{T_3}{T_2} = 6 \implies V_3 = 6 V_2 = 6 V_0.
    \\
    A_\text{цикл} &= \frac 12 (6P_0 - P_0)(4V_0 - V_0) = \frac 12 \cdot 15 \cdot P_0V_0, \\
    A_{23} &= 4P_0 \cdot (6V_0 - V_0) = 20P_0V_0, \\
    \Delta U_{23} &= \frac 32 \nu R T_3 - \frac 32 \nu R T_3 = \frac 32 P_3 V_3 - \frac 32 P_2 V_2 = \frac 32 \cdot 4 P_0 \cdot 6 V_0 -  \frac 32 \cdot 4 P_0 \cdot V_0 = \frac 32 \cdot 20 \cdot P_0V_0, \\
    \Delta U_{12} &= \frac 32 \nu R T_2 - \frac 32 \nu R T_1 = \frac 32 P_2 V_2 - \frac 32 P_1 V_1 = \frac 32 \cdot 4 P_0V_0 - \frac 32 P_0V_0 = \frac 32 \cdot 3 \cdot P_0V_0.
    \\
    \eta &= \frac{A_\text{цикл}}{Q_+} = \frac{A_\text{цикл}}{Q_{12} + Q_{23}}  = \frac{A_\text{цикл}}{A_{12} + \Delta U_{12} + A_{23} + \Delta U_{23}} =  \\
     &= \frac{\frac 12 \cdot 15 \cdot P_0V_0}{0 + \frac 32 \cdot 3 \cdot P_0V_0 + 20P_0V_0 + \frac 32 \cdot 20 \cdot P_0V_0} = \frac{\frac 12 \cdot 15}{\frac 32 \cdot 3 + 20 + \frac 32 \cdot 20} = \frac{15}{109} \approx 0{,}138.
    \end{align*}
}
\solutionspace{360pt}

\tasknumber{2}%
\task{%
    Определите КПД (оставив ответ точным в виде нескоратимой дроби) цикла 1231, рабочим телом которого является идеальный одноатомный газ, если
    \begin{itemize}
        \item 12 — изобарическое расширение,
        \item 23 — процесс, график которого в $PV$-координатах является отрезком прямой, а объём уменьшается в шесть раз,
        \item 31 — изохорический нагрев с увеличением давления в четыре раза,
    \end{itemize}
}
\answer{%
    \begin{align*}
    A_{12} &> 0, \Delta U_{12} > 0, \implies Q_{12} = A_{12} + \Delta U_{12} > 0.
    \\
    A_{23} &< 0, \Delta U_{23} < 0, \implies Q_{23} = A_{23} + \Delta U_{23} < 0, \\
    A_{31} &= 0, \Delta U_{31} > 0, \implies Q_{31} = A_{31} + \Delta U_{31} > 0.
    \\
    P_1V_1 &= \nu R T_1, P_2V_2 = \nu R T_2, P_3V_3 = \nu R T_3 \text{ — уравнения состояния идеального газа}, \\
    &\text{Пусть $P_0$, $V_0$, $T_0$ — давление, объём и температура в точке 3 (минимальные во всём цикле):} \\
    P_3 &= P_0, P_1 = P_2 = 4P_0, V_1 = V_3 = V_0, V_2 = 6V_3 = 6V_0 \\
    A_\text{цикл} &= \frac 12 (P_2-P_1)(V_1-V_2) = \frac 12 (4P_0 - P_0)(6V_0 - V_0) = \frac 12 \cdot 15 \cdot P_0V_0, \\
    A_{12} &= 6P_0 \cdot (4V_0 - V_0) = 18P_0V_0, \\
    \Delta U_{12} &= \frac 32 \nu R T_2 - \frac 32 \nu R T_1 = \frac 32 P_2 V_2 - \frac 32 P_1 V_1 = \frac 32 \cdot 4 P_0 \cdot 6 V_0 -  \frac 32 \cdot 6 P_0 \cdot V_0 = \frac 32 \cdot 18 \cdot P_0V_0, \\
    \Delta U_{31} &= \frac 32 \nu R T_1 - \frac 32 \nu R T_3 = \frac 32 P_1 V_1 - \frac 32 P_3 V_3 = \frac 32 \cdot 4 P_0V_0 - \frac 32 P_0V_0 = \frac 32 \cdot 3 \cdot P_0V_0.
    \\
    \eta &= \frac{A_\text{цикл}}{Q_+} = \frac{A_\text{цикл}}{Q_{12} + Q_{31}}  = \frac{A_\text{цикл}}{A_{12} + \Delta U_{12} + A_{31} + \Delta U_{31}} =  \\
     &= \frac{\frac 12 \cdot 15 \cdot P_0V_0}{18P_0V_0 + \frac 32 \cdot 18 \cdot P_0V_0 + 0 + \frac 32 \cdot 3 \cdot P_0V_0} = \frac{\frac 12 \cdot 15}{18 + \frac 32 \cdot 18 + \frac 32 \cdot 3} = \frac{5}{33} \approx 0{,}152.
    \end{align*}
}

\variantsplitter

\addpersonalvariant{Михаил Ярошевский}

\tasknumber{1}%
\task{%
    Определите КПД (оставив ответ точным в виде нескоратимой дроби) цикла 1231, рабочим телом которого является идеальный одноатомный газ, если
    \begin{itemize}
        \item 12 — изохорический нагрев в пять раз,
        \item 23 — изобарическое расширение, при котором температура растёт в пять раз,
        \item 31 — процесс, график которого в $PV$-координатах является отрезком прямой.
    \end{itemize}
}
\answer{%
    \begin{align*}
    A_{12} &= 0, \Delta U_{12} > 0, \implies Q_{12} = A_{12} + \Delta U_{12} > 0.
    \\
    A_{23} &> 0, \Delta U_{23} > 0, \implies Q_{23} = A_{23} + \Delta U_{23} > 0, \\
    A_{31} &= 0, \Delta U_{31} < 0, \implies Q_{31} = A_{31} + \Delta U_{31} < 0.
    \\
    P_1V_1 &= \nu R T_1, P_2V_2 = \nu R T_2, P_3V_3 = \nu R T_3 \text{ — уравнения состояния идеального газа}, \\
    &\text{Пусть $P_0$, $V_0$, $T_0$ — давление, объём и температура в точке 1 (минимальные во всём цикле):} \\
    P_1 &= P_0, P_2 = P_3, V_1 = V_2 = V_0, \text{остальные соотношения нужно считать} \\
    T_2 &= 5T_1 = 5T_0 \text{(по условию)} \implies \frac{P_2}{P_1} = \frac{P_2V_0}{P_1V_0} = \frac{P_2 V_2}{P_1 V_1}= \frac{\nu R T_2}{\nu R T_1} = \frac{T_2}{T_1} = 5 \implies P_2 = 5 P_1 = 5 P_0, \\
    T_3 &= 5T_2 = 25T_0 \text{(по условию)} \implies \frac{V_3}{V_2} = \frac{P_3V_3}{P_2V_2}= \frac{\nu R T_3}{\nu R T_2} = \frac{T_3}{T_2} = 5 \implies V_3 = 5 V_2 = 5 V_0.
    \\
    A_\text{цикл} &= \frac 12 (5P_0 - P_0)(5V_0 - V_0) = \frac 12 \cdot 16 \cdot P_0V_0, \\
    A_{23} &= 5P_0 \cdot (5V_0 - V_0) = 20P_0V_0, \\
    \Delta U_{23} &= \frac 32 \nu R T_3 - \frac 32 \nu R T_3 = \frac 32 P_3 V_3 - \frac 32 P_2 V_2 = \frac 32 \cdot 5 P_0 \cdot 5 V_0 -  \frac 32 \cdot 5 P_0 \cdot V_0 = \frac 32 \cdot 20 \cdot P_0V_0, \\
    \Delta U_{12} &= \frac 32 \nu R T_2 - \frac 32 \nu R T_1 = \frac 32 P_2 V_2 - \frac 32 P_1 V_1 = \frac 32 \cdot 5 P_0V_0 - \frac 32 P_0V_0 = \frac 32 \cdot 4 \cdot P_0V_0.
    \\
    \eta &= \frac{A_\text{цикл}}{Q_+} = \frac{A_\text{цикл}}{Q_{12} + Q_{23}}  = \frac{A_\text{цикл}}{A_{12} + \Delta U_{12} + A_{23} + \Delta U_{23}} =  \\
     &= \frac{\frac 12 \cdot 16 \cdot P_0V_0}{0 + \frac 32 \cdot 4 \cdot P_0V_0 + 20P_0V_0 + \frac 32 \cdot 20 \cdot P_0V_0} = \frac{\frac 12 \cdot 16}{\frac 32 \cdot 4 + 20 + \frac 32 \cdot 20} = \frac{1}{7} \approx 0{,}143.
    \end{align*}
}
\solutionspace{360pt}

\tasknumber{2}%
\task{%
    Определите КПД (оставив ответ точным в виде нескоратимой дроби) цикла 1231, рабочим телом которого является идеальный одноатомный газ, если
    \begin{itemize}
        \item 12 — изобарическое расширение,
        \item 23 — процесс, график которого в $PV$-координатах является отрезком прямой, а объём уменьшается в пять раз,
        \item 31 — изохорический нагрев с увеличением давления в пять раз,
    \end{itemize}
}
\answer{%
    \begin{align*}
    A_{12} &> 0, \Delta U_{12} > 0, \implies Q_{12} = A_{12} + \Delta U_{12} > 0.
    \\
    A_{23} &< 0, \Delta U_{23} < 0, \implies Q_{23} = A_{23} + \Delta U_{23} < 0, \\
    A_{31} &= 0, \Delta U_{31} > 0, \implies Q_{31} = A_{31} + \Delta U_{31} > 0.
    \\
    P_1V_1 &= \nu R T_1, P_2V_2 = \nu R T_2, P_3V_3 = \nu R T_3 \text{ — уравнения состояния идеального газа}, \\
    &\text{Пусть $P_0$, $V_0$, $T_0$ — давление, объём и температура в точке 3 (минимальные во всём цикле):} \\
    P_3 &= P_0, P_1 = P_2 = 5P_0, V_1 = V_3 = V_0, V_2 = 5V_3 = 5V_0 \\
    A_\text{цикл} &= \frac 12 (P_2-P_1)(V_1-V_2) = \frac 12 (5P_0 - P_0)(5V_0 - V_0) = \frac 12 \cdot 16 \cdot P_0V_0, \\
    A_{12} &= 5P_0 \cdot (5V_0 - V_0) = 20P_0V_0, \\
    \Delta U_{12} &= \frac 32 \nu R T_2 - \frac 32 \nu R T_1 = \frac 32 P_2 V_2 - \frac 32 P_1 V_1 = \frac 32 \cdot 5 P_0 \cdot 5 V_0 -  \frac 32 \cdot 5 P_0 \cdot V_0 = \frac 32 \cdot 20 \cdot P_0V_0, \\
    \Delta U_{31} &= \frac 32 \nu R T_1 - \frac 32 \nu R T_3 = \frac 32 P_1 V_1 - \frac 32 P_3 V_3 = \frac 32 \cdot 5 P_0V_0 - \frac 32 P_0V_0 = \frac 32 \cdot 4 \cdot P_0V_0.
    \\
    \eta &= \frac{A_\text{цикл}}{Q_+} = \frac{A_\text{цикл}}{Q_{12} + Q_{31}}  = \frac{A_\text{цикл}}{A_{12} + \Delta U_{12} + A_{31} + \Delta U_{31}} =  \\
     &= \frac{\frac 12 \cdot 16 \cdot P_0V_0}{20P_0V_0 + \frac 32 \cdot 20 \cdot P_0V_0 + 0 + \frac 32 \cdot 4 \cdot P_0V_0} = \frac{\frac 12 \cdot 16}{20 + \frac 32 \cdot 20 + \frac 32 \cdot 4} = \frac{1}{7} \approx 0{,}143.
    \end{align*}
}

\variantsplitter

\addpersonalvariant{Алексей Алимпиев}

\tasknumber{1}%
\task{%
    Определите КПД (оставив ответ точным в виде нескоратимой дроби) цикла 1231, рабочим телом которого является идеальный одноатомный газ, если
    \begin{itemize}
        \item 12 — изохорический нагрев в четыре раза,
        \item 23 — изобарическое расширение, при котором температура растёт в шесть раз,
        \item 31 — процесс, график которого в $PV$-координатах является отрезком прямой.
    \end{itemize}
}
\answer{%
    \begin{align*}
    A_{12} &= 0, \Delta U_{12} > 0, \implies Q_{12} = A_{12} + \Delta U_{12} > 0.
    \\
    A_{23} &> 0, \Delta U_{23} > 0, \implies Q_{23} = A_{23} + \Delta U_{23} > 0, \\
    A_{31} &= 0, \Delta U_{31} < 0, \implies Q_{31} = A_{31} + \Delta U_{31} < 0.
    \\
    P_1V_1 &= \nu R T_1, P_2V_2 = \nu R T_2, P_3V_3 = \nu R T_3 \text{ — уравнения состояния идеального газа}, \\
    &\text{Пусть $P_0$, $V_0$, $T_0$ — давление, объём и температура в точке 1 (минимальные во всём цикле):} \\
    P_1 &= P_0, P_2 = P_3, V_1 = V_2 = V_0, \text{остальные соотношения нужно считать} \\
    T_2 &= 4T_1 = 4T_0 \text{(по условию)} \implies \frac{P_2}{P_1} = \frac{P_2V_0}{P_1V_0} = \frac{P_2 V_2}{P_1 V_1}= \frac{\nu R T_2}{\nu R T_1} = \frac{T_2}{T_1} = 4 \implies P_2 = 4 P_1 = 4 P_0, \\
    T_3 &= 6T_2 = 24T_0 \text{(по условию)} \implies \frac{V_3}{V_2} = \frac{P_3V_3}{P_2V_2}= \frac{\nu R T_3}{\nu R T_2} = \frac{T_3}{T_2} = 6 \implies V_3 = 6 V_2 = 6 V_0.
    \\
    A_\text{цикл} &= \frac 12 (6P_0 - P_0)(4V_0 - V_0) = \frac 12 \cdot 15 \cdot P_0V_0, \\
    A_{23} &= 4P_0 \cdot (6V_0 - V_0) = 20P_0V_0, \\
    \Delta U_{23} &= \frac 32 \nu R T_3 - \frac 32 \nu R T_3 = \frac 32 P_3 V_3 - \frac 32 P_2 V_2 = \frac 32 \cdot 4 P_0 \cdot 6 V_0 -  \frac 32 \cdot 4 P_0 \cdot V_0 = \frac 32 \cdot 20 \cdot P_0V_0, \\
    \Delta U_{12} &= \frac 32 \nu R T_2 - \frac 32 \nu R T_1 = \frac 32 P_2 V_2 - \frac 32 P_1 V_1 = \frac 32 \cdot 4 P_0V_0 - \frac 32 P_0V_0 = \frac 32 \cdot 3 \cdot P_0V_0.
    \\
    \eta &= \frac{A_\text{цикл}}{Q_+} = \frac{A_\text{цикл}}{Q_{12} + Q_{23}}  = \frac{A_\text{цикл}}{A_{12} + \Delta U_{12} + A_{23} + \Delta U_{23}} =  \\
     &= \frac{\frac 12 \cdot 15 \cdot P_0V_0}{0 + \frac 32 \cdot 3 \cdot P_0V_0 + 20P_0V_0 + \frac 32 \cdot 20 \cdot P_0V_0} = \frac{\frac 12 \cdot 15}{\frac 32 \cdot 3 + 20 + \frac 32 \cdot 20} = \frac{15}{109} \approx 0{,}138.
    \end{align*}
}
\solutionspace{360pt}

\tasknumber{2}%
\task{%
    Определите КПД (оставив ответ точным в виде нескоратимой дроби) цикла 1231, рабочим телом которого является идеальный одноатомный газ, если
    \begin{itemize}
        \item 12 — изобарическое расширение,
        \item 23 — процесс, график которого в $PV$-координатах является отрезком прямой, а объём уменьшается в шесть раз,
        \item 31 — изохорический нагрев с увеличением давления в четыре раза,
    \end{itemize}
}
\answer{%
    \begin{align*}
    A_{12} &> 0, \Delta U_{12} > 0, \implies Q_{12} = A_{12} + \Delta U_{12} > 0.
    \\
    A_{23} &< 0, \Delta U_{23} < 0, \implies Q_{23} = A_{23} + \Delta U_{23} < 0, \\
    A_{31} &= 0, \Delta U_{31} > 0, \implies Q_{31} = A_{31} + \Delta U_{31} > 0.
    \\
    P_1V_1 &= \nu R T_1, P_2V_2 = \nu R T_2, P_3V_3 = \nu R T_3 \text{ — уравнения состояния идеального газа}, \\
    &\text{Пусть $P_0$, $V_0$, $T_0$ — давление, объём и температура в точке 3 (минимальные во всём цикле):} \\
    P_3 &= P_0, P_1 = P_2 = 4P_0, V_1 = V_3 = V_0, V_2 = 6V_3 = 6V_0 \\
    A_\text{цикл} &= \frac 12 (P_2-P_1)(V_1-V_2) = \frac 12 (4P_0 - P_0)(6V_0 - V_0) = \frac 12 \cdot 15 \cdot P_0V_0, \\
    A_{12} &= 6P_0 \cdot (4V_0 - V_0) = 18P_0V_0, \\
    \Delta U_{12} &= \frac 32 \nu R T_2 - \frac 32 \nu R T_1 = \frac 32 P_2 V_2 - \frac 32 P_1 V_1 = \frac 32 \cdot 4 P_0 \cdot 6 V_0 -  \frac 32 \cdot 6 P_0 \cdot V_0 = \frac 32 \cdot 18 \cdot P_0V_0, \\
    \Delta U_{31} &= \frac 32 \nu R T_1 - \frac 32 \nu R T_3 = \frac 32 P_1 V_1 - \frac 32 P_3 V_3 = \frac 32 \cdot 4 P_0V_0 - \frac 32 P_0V_0 = \frac 32 \cdot 3 \cdot P_0V_0.
    \\
    \eta &= \frac{A_\text{цикл}}{Q_+} = \frac{A_\text{цикл}}{Q_{12} + Q_{31}}  = \frac{A_\text{цикл}}{A_{12} + \Delta U_{12} + A_{31} + \Delta U_{31}} =  \\
     &= \frac{\frac 12 \cdot 15 \cdot P_0V_0}{18P_0V_0 + \frac 32 \cdot 18 \cdot P_0V_0 + 0 + \frac 32 \cdot 3 \cdot P_0V_0} = \frac{\frac 12 \cdot 15}{18 + \frac 32 \cdot 18 + \frac 32 \cdot 3} = \frac{5}{33} \approx 0{,}152.
    \end{align*}
}

\variantsplitter

\addpersonalvariant{Евгений Васин}

\tasknumber{1}%
\task{%
    Определите КПД (оставив ответ точным в виде нескоратимой дроби) цикла 1231, рабочим телом которого является идеальный одноатомный газ, если
    \begin{itemize}
        \item 12 — изохорический нагрев в четыре раза,
        \item 23 — изобарическое расширение, при котором температура растёт в четыре раза,
        \item 31 — процесс, график которого в $PV$-координатах является отрезком прямой.
    \end{itemize}
}
\answer{%
    \begin{align*}
    A_{12} &= 0, \Delta U_{12} > 0, \implies Q_{12} = A_{12} + \Delta U_{12} > 0.
    \\
    A_{23} &> 0, \Delta U_{23} > 0, \implies Q_{23} = A_{23} + \Delta U_{23} > 0, \\
    A_{31} &= 0, \Delta U_{31} < 0, \implies Q_{31} = A_{31} + \Delta U_{31} < 0.
    \\
    P_1V_1 &= \nu R T_1, P_2V_2 = \nu R T_2, P_3V_3 = \nu R T_3 \text{ — уравнения состояния идеального газа}, \\
    &\text{Пусть $P_0$, $V_0$, $T_0$ — давление, объём и температура в точке 1 (минимальные во всём цикле):} \\
    P_1 &= P_0, P_2 = P_3, V_1 = V_2 = V_0, \text{остальные соотношения нужно считать} \\
    T_2 &= 4T_1 = 4T_0 \text{(по условию)} \implies \frac{P_2}{P_1} = \frac{P_2V_0}{P_1V_0} = \frac{P_2 V_2}{P_1 V_1}= \frac{\nu R T_2}{\nu R T_1} = \frac{T_2}{T_1} = 4 \implies P_2 = 4 P_1 = 4 P_0, \\
    T_3 &= 4T_2 = 16T_0 \text{(по условию)} \implies \frac{V_3}{V_2} = \frac{P_3V_3}{P_2V_2}= \frac{\nu R T_3}{\nu R T_2} = \frac{T_3}{T_2} = 4 \implies V_3 = 4 V_2 = 4 V_0.
    \\
    A_\text{цикл} &= \frac 12 (4P_0 - P_0)(4V_0 - V_0) = \frac 12 \cdot 9 \cdot P_0V_0, \\
    A_{23} &= 4P_0 \cdot (4V_0 - V_0) = 12P_0V_0, \\
    \Delta U_{23} &= \frac 32 \nu R T_3 - \frac 32 \nu R T_3 = \frac 32 P_3 V_3 - \frac 32 P_2 V_2 = \frac 32 \cdot 4 P_0 \cdot 4 V_0 -  \frac 32 \cdot 4 P_0 \cdot V_0 = \frac 32 \cdot 12 \cdot P_0V_0, \\
    \Delta U_{12} &= \frac 32 \nu R T_2 - \frac 32 \nu R T_1 = \frac 32 P_2 V_2 - \frac 32 P_1 V_1 = \frac 32 \cdot 4 P_0V_0 - \frac 32 P_0V_0 = \frac 32 \cdot 3 \cdot P_0V_0.
    \\
    \eta &= \frac{A_\text{цикл}}{Q_+} = \frac{A_\text{цикл}}{Q_{12} + Q_{23}}  = \frac{A_\text{цикл}}{A_{12} + \Delta U_{12} + A_{23} + \Delta U_{23}} =  \\
     &= \frac{\frac 12 \cdot 9 \cdot P_0V_0}{0 + \frac 32 \cdot 3 \cdot P_0V_0 + 12P_0V_0 + \frac 32 \cdot 12 \cdot P_0V_0} = \frac{\frac 12 \cdot 9}{\frac 32 \cdot 3 + 12 + \frac 32 \cdot 12} = \frac{3}{23} \approx 0{,}130.
    \end{align*}
}
\solutionspace{360pt}

\tasknumber{2}%
\task{%
    Определите КПД (оставив ответ точным в виде нескоратимой дроби) цикла 1231, рабочим телом которого является идеальный одноатомный газ, если
    \begin{itemize}
        \item 12 — изобарическое расширение,
        \item 23 — процесс, график которого в $PV$-координатах является отрезком прямой, а объём уменьшается в четыре раза,
        \item 31 — изохорический нагрев с увеличением давления в четыре раза,
    \end{itemize}
}
\answer{%
    \begin{align*}
    A_{12} &> 0, \Delta U_{12} > 0, \implies Q_{12} = A_{12} + \Delta U_{12} > 0.
    \\
    A_{23} &< 0, \Delta U_{23} < 0, \implies Q_{23} = A_{23} + \Delta U_{23} < 0, \\
    A_{31} &= 0, \Delta U_{31} > 0, \implies Q_{31} = A_{31} + \Delta U_{31} > 0.
    \\
    P_1V_1 &= \nu R T_1, P_2V_2 = \nu R T_2, P_3V_3 = \nu R T_3 \text{ — уравнения состояния идеального газа}, \\
    &\text{Пусть $P_0$, $V_0$, $T_0$ — давление, объём и температура в точке 3 (минимальные во всём цикле):} \\
    P_3 &= P_0, P_1 = P_2 = 4P_0, V_1 = V_3 = V_0, V_2 = 4V_3 = 4V_0 \\
    A_\text{цикл} &= \frac 12 (P_2-P_1)(V_1-V_2) = \frac 12 (4P_0 - P_0)(4V_0 - V_0) = \frac 12 \cdot 9 \cdot P_0V_0, \\
    A_{12} &= 4P_0 \cdot (4V_0 - V_0) = 12P_0V_0, \\
    \Delta U_{12} &= \frac 32 \nu R T_2 - \frac 32 \nu R T_1 = \frac 32 P_2 V_2 - \frac 32 P_1 V_1 = \frac 32 \cdot 4 P_0 \cdot 4 V_0 -  \frac 32 \cdot 4 P_0 \cdot V_0 = \frac 32 \cdot 12 \cdot P_0V_0, \\
    \Delta U_{31} &= \frac 32 \nu R T_1 - \frac 32 \nu R T_3 = \frac 32 P_1 V_1 - \frac 32 P_3 V_3 = \frac 32 \cdot 4 P_0V_0 - \frac 32 P_0V_0 = \frac 32 \cdot 3 \cdot P_0V_0.
    \\
    \eta &= \frac{A_\text{цикл}}{Q_+} = \frac{A_\text{цикл}}{Q_{12} + Q_{31}}  = \frac{A_\text{цикл}}{A_{12} + \Delta U_{12} + A_{31} + \Delta U_{31}} =  \\
     &= \frac{\frac 12 \cdot 9 \cdot P_0V_0}{12P_0V_0 + \frac 32 \cdot 12 \cdot P_0V_0 + 0 + \frac 32 \cdot 3 \cdot P_0V_0} = \frac{\frac 12 \cdot 9}{12 + \frac 32 \cdot 12 + \frac 32 \cdot 3} = \frac{3}{23} \approx 0{,}130.
    \end{align*}
}

\variantsplitter

\addpersonalvariant{Вячеслав Волохов}

\tasknumber{1}%
\task{%
    Определите КПД (оставив ответ точным в виде нескоратимой дроби) цикла 1231, рабочим телом которого является идеальный одноатомный газ, если
    \begin{itemize}
        \item 12 — изохорический нагрев в пять раз,
        \item 23 — изобарическое расширение, при котором температура растёт в четыре раза,
        \item 31 — процесс, график которого в $PV$-координатах является отрезком прямой.
    \end{itemize}
}
\answer{%
    \begin{align*}
    A_{12} &= 0, \Delta U_{12} > 0, \implies Q_{12} = A_{12} + \Delta U_{12} > 0.
    \\
    A_{23} &> 0, \Delta U_{23} > 0, \implies Q_{23} = A_{23} + \Delta U_{23} > 0, \\
    A_{31} &= 0, \Delta U_{31} < 0, \implies Q_{31} = A_{31} + \Delta U_{31} < 0.
    \\
    P_1V_1 &= \nu R T_1, P_2V_2 = \nu R T_2, P_3V_3 = \nu R T_3 \text{ — уравнения состояния идеального газа}, \\
    &\text{Пусть $P_0$, $V_0$, $T_0$ — давление, объём и температура в точке 1 (минимальные во всём цикле):} \\
    P_1 &= P_0, P_2 = P_3, V_1 = V_2 = V_0, \text{остальные соотношения нужно считать} \\
    T_2 &= 5T_1 = 5T_0 \text{(по условию)} \implies \frac{P_2}{P_1} = \frac{P_2V_0}{P_1V_0} = \frac{P_2 V_2}{P_1 V_1}= \frac{\nu R T_2}{\nu R T_1} = \frac{T_2}{T_1} = 5 \implies P_2 = 5 P_1 = 5 P_0, \\
    T_3 &= 4T_2 = 20T_0 \text{(по условию)} \implies \frac{V_3}{V_2} = \frac{P_3V_3}{P_2V_2}= \frac{\nu R T_3}{\nu R T_2} = \frac{T_3}{T_2} = 4 \implies V_3 = 4 V_2 = 4 V_0.
    \\
    A_\text{цикл} &= \frac 12 (4P_0 - P_0)(5V_0 - V_0) = \frac 12 \cdot 12 \cdot P_0V_0, \\
    A_{23} &= 5P_0 \cdot (4V_0 - V_0) = 15P_0V_0, \\
    \Delta U_{23} &= \frac 32 \nu R T_3 - \frac 32 \nu R T_3 = \frac 32 P_3 V_3 - \frac 32 P_2 V_2 = \frac 32 \cdot 5 P_0 \cdot 4 V_0 -  \frac 32 \cdot 5 P_0 \cdot V_0 = \frac 32 \cdot 15 \cdot P_0V_0, \\
    \Delta U_{12} &= \frac 32 \nu R T_2 - \frac 32 \nu R T_1 = \frac 32 P_2 V_2 - \frac 32 P_1 V_1 = \frac 32 \cdot 5 P_0V_0 - \frac 32 P_0V_0 = \frac 32 \cdot 4 \cdot P_0V_0.
    \\
    \eta &= \frac{A_\text{цикл}}{Q_+} = \frac{A_\text{цикл}}{Q_{12} + Q_{23}}  = \frac{A_\text{цикл}}{A_{12} + \Delta U_{12} + A_{23} + \Delta U_{23}} =  \\
     &= \frac{\frac 12 \cdot 12 \cdot P_0V_0}{0 + \frac 32 \cdot 4 \cdot P_0V_0 + 15P_0V_0 + \frac 32 \cdot 15 \cdot P_0V_0} = \frac{\frac 12 \cdot 12}{\frac 32 \cdot 4 + 15 + \frac 32 \cdot 15} = \frac{4}{29} \approx 0{,}138.
    \end{align*}
}
\solutionspace{360pt}

\tasknumber{2}%
\task{%
    Определите КПД (оставив ответ точным в виде нескоратимой дроби) цикла 1231, рабочим телом которого является идеальный одноатомный газ, если
    \begin{itemize}
        \item 12 — изобарическое расширение,
        \item 23 — процесс, график которого в $PV$-координатах является отрезком прямой, а объём уменьшается в четыре раза,
        \item 31 — изохорический нагрев с увеличением давления в пять раз,
    \end{itemize}
}
\answer{%
    \begin{align*}
    A_{12} &> 0, \Delta U_{12} > 0, \implies Q_{12} = A_{12} + \Delta U_{12} > 0.
    \\
    A_{23} &< 0, \Delta U_{23} < 0, \implies Q_{23} = A_{23} + \Delta U_{23} < 0, \\
    A_{31} &= 0, \Delta U_{31} > 0, \implies Q_{31} = A_{31} + \Delta U_{31} > 0.
    \\
    P_1V_1 &= \nu R T_1, P_2V_2 = \nu R T_2, P_3V_3 = \nu R T_3 \text{ — уравнения состояния идеального газа}, \\
    &\text{Пусть $P_0$, $V_0$, $T_0$ — давление, объём и температура в точке 3 (минимальные во всём цикле):} \\
    P_3 &= P_0, P_1 = P_2 = 5P_0, V_1 = V_3 = V_0, V_2 = 4V_3 = 4V_0 \\
    A_\text{цикл} &= \frac 12 (P_2-P_1)(V_1-V_2) = \frac 12 (5P_0 - P_0)(4V_0 - V_0) = \frac 12 \cdot 12 \cdot P_0V_0, \\
    A_{12} &= 4P_0 \cdot (5V_0 - V_0) = 16P_0V_0, \\
    \Delta U_{12} &= \frac 32 \nu R T_2 - \frac 32 \nu R T_1 = \frac 32 P_2 V_2 - \frac 32 P_1 V_1 = \frac 32 \cdot 5 P_0 \cdot 4 V_0 -  \frac 32 \cdot 4 P_0 \cdot V_0 = \frac 32 \cdot 16 \cdot P_0V_0, \\
    \Delta U_{31} &= \frac 32 \nu R T_1 - \frac 32 \nu R T_3 = \frac 32 P_1 V_1 - \frac 32 P_3 V_3 = \frac 32 \cdot 5 P_0V_0 - \frac 32 P_0V_0 = \frac 32 \cdot 4 \cdot P_0V_0.
    \\
    \eta &= \frac{A_\text{цикл}}{Q_+} = \frac{A_\text{цикл}}{Q_{12} + Q_{31}}  = \frac{A_\text{цикл}}{A_{12} + \Delta U_{12} + A_{31} + \Delta U_{31}} =  \\
     &= \frac{\frac 12 \cdot 12 \cdot P_0V_0}{16P_0V_0 + \frac 32 \cdot 16 \cdot P_0V_0 + 0 + \frac 32 \cdot 4 \cdot P_0V_0} = \frac{\frac 12 \cdot 12}{16 + \frac 32 \cdot 16 + \frac 32 \cdot 4} = \frac{3}{23} \approx 0{,}130.
    \end{align*}
}

\variantsplitter

\addpersonalvariant{Герман Говоров}

\tasknumber{1}%
\task{%
    Определите КПД (оставив ответ точным в виде нескоратимой дроби) цикла 1231, рабочим телом которого является идеальный одноатомный газ, если
    \begin{itemize}
        \item 12 — изохорический нагрев в четыре раза,
        \item 23 — изобарическое расширение, при котором температура растёт в пять раз,
        \item 31 — процесс, график которого в $PV$-координатах является отрезком прямой.
    \end{itemize}
}
\answer{%
    \begin{align*}
    A_{12} &= 0, \Delta U_{12} > 0, \implies Q_{12} = A_{12} + \Delta U_{12} > 0.
    \\
    A_{23} &> 0, \Delta U_{23} > 0, \implies Q_{23} = A_{23} + \Delta U_{23} > 0, \\
    A_{31} &= 0, \Delta U_{31} < 0, \implies Q_{31} = A_{31} + \Delta U_{31} < 0.
    \\
    P_1V_1 &= \nu R T_1, P_2V_2 = \nu R T_2, P_3V_3 = \nu R T_3 \text{ — уравнения состояния идеального газа}, \\
    &\text{Пусть $P_0$, $V_0$, $T_0$ — давление, объём и температура в точке 1 (минимальные во всём цикле):} \\
    P_1 &= P_0, P_2 = P_3, V_1 = V_2 = V_0, \text{остальные соотношения нужно считать} \\
    T_2 &= 4T_1 = 4T_0 \text{(по условию)} \implies \frac{P_2}{P_1} = \frac{P_2V_0}{P_1V_0} = \frac{P_2 V_2}{P_1 V_1}= \frac{\nu R T_2}{\nu R T_1} = \frac{T_2}{T_1} = 4 \implies P_2 = 4 P_1 = 4 P_0, \\
    T_3 &= 5T_2 = 20T_0 \text{(по условию)} \implies \frac{V_3}{V_2} = \frac{P_3V_3}{P_2V_2}= \frac{\nu R T_3}{\nu R T_2} = \frac{T_3}{T_2} = 5 \implies V_3 = 5 V_2 = 5 V_0.
    \\
    A_\text{цикл} &= \frac 12 (5P_0 - P_0)(4V_0 - V_0) = \frac 12 \cdot 12 \cdot P_0V_0, \\
    A_{23} &= 4P_0 \cdot (5V_0 - V_0) = 16P_0V_0, \\
    \Delta U_{23} &= \frac 32 \nu R T_3 - \frac 32 \nu R T_3 = \frac 32 P_3 V_3 - \frac 32 P_2 V_2 = \frac 32 \cdot 4 P_0 \cdot 5 V_0 -  \frac 32 \cdot 4 P_0 \cdot V_0 = \frac 32 \cdot 16 \cdot P_0V_0, \\
    \Delta U_{12} &= \frac 32 \nu R T_2 - \frac 32 \nu R T_1 = \frac 32 P_2 V_2 - \frac 32 P_1 V_1 = \frac 32 \cdot 4 P_0V_0 - \frac 32 P_0V_0 = \frac 32 \cdot 3 \cdot P_0V_0.
    \\
    \eta &= \frac{A_\text{цикл}}{Q_+} = \frac{A_\text{цикл}}{Q_{12} + Q_{23}}  = \frac{A_\text{цикл}}{A_{12} + \Delta U_{12} + A_{23} + \Delta U_{23}} =  \\
     &= \frac{\frac 12 \cdot 12 \cdot P_0V_0}{0 + \frac 32 \cdot 3 \cdot P_0V_0 + 16P_0V_0 + \frac 32 \cdot 16 \cdot P_0V_0} = \frac{\frac 12 \cdot 12}{\frac 32 \cdot 3 + 16 + \frac 32 \cdot 16} = \frac{12}{89} \approx 0{,}135.
    \end{align*}
}
\solutionspace{360pt}

\tasknumber{2}%
\task{%
    Определите КПД (оставив ответ точным в виде нескоратимой дроби) цикла 1231, рабочим телом которого является идеальный одноатомный газ, если
    \begin{itemize}
        \item 12 — изобарическое расширение,
        \item 23 — процесс, график которого в $PV$-координатах является отрезком прямой, а объём уменьшается в пять раз,
        \item 31 — изохорический нагрев с увеличением давления в четыре раза,
    \end{itemize}
}
\answer{%
    \begin{align*}
    A_{12} &> 0, \Delta U_{12} > 0, \implies Q_{12} = A_{12} + \Delta U_{12} > 0.
    \\
    A_{23} &< 0, \Delta U_{23} < 0, \implies Q_{23} = A_{23} + \Delta U_{23} < 0, \\
    A_{31} &= 0, \Delta U_{31} > 0, \implies Q_{31} = A_{31} + \Delta U_{31} > 0.
    \\
    P_1V_1 &= \nu R T_1, P_2V_2 = \nu R T_2, P_3V_3 = \nu R T_3 \text{ — уравнения состояния идеального газа}, \\
    &\text{Пусть $P_0$, $V_0$, $T_0$ — давление, объём и температура в точке 3 (минимальные во всём цикле):} \\
    P_3 &= P_0, P_1 = P_2 = 4P_0, V_1 = V_3 = V_0, V_2 = 5V_3 = 5V_0 \\
    A_\text{цикл} &= \frac 12 (P_2-P_1)(V_1-V_2) = \frac 12 (4P_0 - P_0)(5V_0 - V_0) = \frac 12 \cdot 12 \cdot P_0V_0, \\
    A_{12} &= 5P_0 \cdot (4V_0 - V_0) = 15P_0V_0, \\
    \Delta U_{12} &= \frac 32 \nu R T_2 - \frac 32 \nu R T_1 = \frac 32 P_2 V_2 - \frac 32 P_1 V_1 = \frac 32 \cdot 4 P_0 \cdot 5 V_0 -  \frac 32 \cdot 5 P_0 \cdot V_0 = \frac 32 \cdot 15 \cdot P_0V_0, \\
    \Delta U_{31} &= \frac 32 \nu R T_1 - \frac 32 \nu R T_3 = \frac 32 P_1 V_1 - \frac 32 P_3 V_3 = \frac 32 \cdot 4 P_0V_0 - \frac 32 P_0V_0 = \frac 32 \cdot 3 \cdot P_0V_0.
    \\
    \eta &= \frac{A_\text{цикл}}{Q_+} = \frac{A_\text{цикл}}{Q_{12} + Q_{31}}  = \frac{A_\text{цикл}}{A_{12} + \Delta U_{12} + A_{31} + \Delta U_{31}} =  \\
     &= \frac{\frac 12 \cdot 12 \cdot P_0V_0}{15P_0V_0 + \frac 32 \cdot 15 \cdot P_0V_0 + 0 + \frac 32 \cdot 3 \cdot P_0V_0} = \frac{\frac 12 \cdot 12}{15 + \frac 32 \cdot 15 + \frac 32 \cdot 3} = \frac{1}{7} \approx 0{,}143.
    \end{align*}
}

\variantsplitter

\addpersonalvariant{София Журавлёва}

\tasknumber{1}%
\task{%
    Определите КПД (оставив ответ точным в виде нескоратимой дроби) цикла 1231, рабочим телом которого является идеальный одноатомный газ, если
    \begin{itemize}
        \item 12 — изохорический нагрев в пять раз,
        \item 23 — изобарическое расширение, при котором температура растёт в пять раз,
        \item 31 — процесс, график которого в $PV$-координатах является отрезком прямой.
    \end{itemize}
}
\answer{%
    \begin{align*}
    A_{12} &= 0, \Delta U_{12} > 0, \implies Q_{12} = A_{12} + \Delta U_{12} > 0.
    \\
    A_{23} &> 0, \Delta U_{23} > 0, \implies Q_{23} = A_{23} + \Delta U_{23} > 0, \\
    A_{31} &= 0, \Delta U_{31} < 0, \implies Q_{31} = A_{31} + \Delta U_{31} < 0.
    \\
    P_1V_1 &= \nu R T_1, P_2V_2 = \nu R T_2, P_3V_3 = \nu R T_3 \text{ — уравнения состояния идеального газа}, \\
    &\text{Пусть $P_0$, $V_0$, $T_0$ — давление, объём и температура в точке 1 (минимальные во всём цикле):} \\
    P_1 &= P_0, P_2 = P_3, V_1 = V_2 = V_0, \text{остальные соотношения нужно считать} \\
    T_2 &= 5T_1 = 5T_0 \text{(по условию)} \implies \frac{P_2}{P_1} = \frac{P_2V_0}{P_1V_0} = \frac{P_2 V_2}{P_1 V_1}= \frac{\nu R T_2}{\nu R T_1} = \frac{T_2}{T_1} = 5 \implies P_2 = 5 P_1 = 5 P_0, \\
    T_3 &= 5T_2 = 25T_0 \text{(по условию)} \implies \frac{V_3}{V_2} = \frac{P_3V_3}{P_2V_2}= \frac{\nu R T_3}{\nu R T_2} = \frac{T_3}{T_2} = 5 \implies V_3 = 5 V_2 = 5 V_0.
    \\
    A_\text{цикл} &= \frac 12 (5P_0 - P_0)(5V_0 - V_0) = \frac 12 \cdot 16 \cdot P_0V_0, \\
    A_{23} &= 5P_0 \cdot (5V_0 - V_0) = 20P_0V_0, \\
    \Delta U_{23} &= \frac 32 \nu R T_3 - \frac 32 \nu R T_3 = \frac 32 P_3 V_3 - \frac 32 P_2 V_2 = \frac 32 \cdot 5 P_0 \cdot 5 V_0 -  \frac 32 \cdot 5 P_0 \cdot V_0 = \frac 32 \cdot 20 \cdot P_0V_0, \\
    \Delta U_{12} &= \frac 32 \nu R T_2 - \frac 32 \nu R T_1 = \frac 32 P_2 V_2 - \frac 32 P_1 V_1 = \frac 32 \cdot 5 P_0V_0 - \frac 32 P_0V_0 = \frac 32 \cdot 4 \cdot P_0V_0.
    \\
    \eta &= \frac{A_\text{цикл}}{Q_+} = \frac{A_\text{цикл}}{Q_{12} + Q_{23}}  = \frac{A_\text{цикл}}{A_{12} + \Delta U_{12} + A_{23} + \Delta U_{23}} =  \\
     &= \frac{\frac 12 \cdot 16 \cdot P_0V_0}{0 + \frac 32 \cdot 4 \cdot P_0V_0 + 20P_0V_0 + \frac 32 \cdot 20 \cdot P_0V_0} = \frac{\frac 12 \cdot 16}{\frac 32 \cdot 4 + 20 + \frac 32 \cdot 20} = \frac{1}{7} \approx 0{,}143.
    \end{align*}
}
\solutionspace{360pt}

\tasknumber{2}%
\task{%
    Определите КПД (оставив ответ точным в виде нескоратимой дроби) цикла 1231, рабочим телом которого является идеальный одноатомный газ, если
    \begin{itemize}
        \item 12 — изобарическое расширение,
        \item 23 — процесс, график которого в $PV$-координатах является отрезком прямой, а объём уменьшается в пять раз,
        \item 31 — изохорический нагрев с увеличением давления в пять раз,
    \end{itemize}
}
\answer{%
    \begin{align*}
    A_{12} &> 0, \Delta U_{12} > 0, \implies Q_{12} = A_{12} + \Delta U_{12} > 0.
    \\
    A_{23} &< 0, \Delta U_{23} < 0, \implies Q_{23} = A_{23} + \Delta U_{23} < 0, \\
    A_{31} &= 0, \Delta U_{31} > 0, \implies Q_{31} = A_{31} + \Delta U_{31} > 0.
    \\
    P_1V_1 &= \nu R T_1, P_2V_2 = \nu R T_2, P_3V_3 = \nu R T_3 \text{ — уравнения состояния идеального газа}, \\
    &\text{Пусть $P_0$, $V_0$, $T_0$ — давление, объём и температура в точке 3 (минимальные во всём цикле):} \\
    P_3 &= P_0, P_1 = P_2 = 5P_0, V_1 = V_3 = V_0, V_2 = 5V_3 = 5V_0 \\
    A_\text{цикл} &= \frac 12 (P_2-P_1)(V_1-V_2) = \frac 12 (5P_0 - P_0)(5V_0 - V_0) = \frac 12 \cdot 16 \cdot P_0V_0, \\
    A_{12} &= 5P_0 \cdot (5V_0 - V_0) = 20P_0V_0, \\
    \Delta U_{12} &= \frac 32 \nu R T_2 - \frac 32 \nu R T_1 = \frac 32 P_2 V_2 - \frac 32 P_1 V_1 = \frac 32 \cdot 5 P_0 \cdot 5 V_0 -  \frac 32 \cdot 5 P_0 \cdot V_0 = \frac 32 \cdot 20 \cdot P_0V_0, \\
    \Delta U_{31} &= \frac 32 \nu R T_1 - \frac 32 \nu R T_3 = \frac 32 P_1 V_1 - \frac 32 P_3 V_3 = \frac 32 \cdot 5 P_0V_0 - \frac 32 P_0V_0 = \frac 32 \cdot 4 \cdot P_0V_0.
    \\
    \eta &= \frac{A_\text{цикл}}{Q_+} = \frac{A_\text{цикл}}{Q_{12} + Q_{31}}  = \frac{A_\text{цикл}}{A_{12} + \Delta U_{12} + A_{31} + \Delta U_{31}} =  \\
     &= \frac{\frac 12 \cdot 16 \cdot P_0V_0}{20P_0V_0 + \frac 32 \cdot 20 \cdot P_0V_0 + 0 + \frac 32 \cdot 4 \cdot P_0V_0} = \frac{\frac 12 \cdot 16}{20 + \frac 32 \cdot 20 + \frac 32 \cdot 4} = \frac{1}{7} \approx 0{,}143.
    \end{align*}
}

\variantsplitter

\addpersonalvariant{Константин Козлов}

\tasknumber{1}%
\task{%
    Определите КПД (оставив ответ точным в виде нескоратимой дроби) цикла 1231, рабочим телом которого является идеальный одноатомный газ, если
    \begin{itemize}
        \item 12 — изохорический нагрев в шесть раз,
        \item 23 — изобарическое расширение, при котором температура растёт в шесть раз,
        \item 31 — процесс, график которого в $PV$-координатах является отрезком прямой.
    \end{itemize}
}
\answer{%
    \begin{align*}
    A_{12} &= 0, \Delta U_{12} > 0, \implies Q_{12} = A_{12} + \Delta U_{12} > 0.
    \\
    A_{23} &> 0, \Delta U_{23} > 0, \implies Q_{23} = A_{23} + \Delta U_{23} > 0, \\
    A_{31} &= 0, \Delta U_{31} < 0, \implies Q_{31} = A_{31} + \Delta U_{31} < 0.
    \\
    P_1V_1 &= \nu R T_1, P_2V_2 = \nu R T_2, P_3V_3 = \nu R T_3 \text{ — уравнения состояния идеального газа}, \\
    &\text{Пусть $P_0$, $V_0$, $T_0$ — давление, объём и температура в точке 1 (минимальные во всём цикле):} \\
    P_1 &= P_0, P_2 = P_3, V_1 = V_2 = V_0, \text{остальные соотношения нужно считать} \\
    T_2 &= 6T_1 = 6T_0 \text{(по условию)} \implies \frac{P_2}{P_1} = \frac{P_2V_0}{P_1V_0} = \frac{P_2 V_2}{P_1 V_1}= \frac{\nu R T_2}{\nu R T_1} = \frac{T_2}{T_1} = 6 \implies P_2 = 6 P_1 = 6 P_0, \\
    T_3 &= 6T_2 = 36T_0 \text{(по условию)} \implies \frac{V_3}{V_2} = \frac{P_3V_3}{P_2V_2}= \frac{\nu R T_3}{\nu R T_2} = \frac{T_3}{T_2} = 6 \implies V_3 = 6 V_2 = 6 V_0.
    \\
    A_\text{цикл} &= \frac 12 (6P_0 - P_0)(6V_0 - V_0) = \frac 12 \cdot 25 \cdot P_0V_0, \\
    A_{23} &= 6P_0 \cdot (6V_0 - V_0) = 30P_0V_0, \\
    \Delta U_{23} &= \frac 32 \nu R T_3 - \frac 32 \nu R T_3 = \frac 32 P_3 V_3 - \frac 32 P_2 V_2 = \frac 32 \cdot 6 P_0 \cdot 6 V_0 -  \frac 32 \cdot 6 P_0 \cdot V_0 = \frac 32 \cdot 30 \cdot P_0V_0, \\
    \Delta U_{12} &= \frac 32 \nu R T_2 - \frac 32 \nu R T_1 = \frac 32 P_2 V_2 - \frac 32 P_1 V_1 = \frac 32 \cdot 6 P_0V_0 - \frac 32 P_0V_0 = \frac 32 \cdot 5 \cdot P_0V_0.
    \\
    \eta &= \frac{A_\text{цикл}}{Q_+} = \frac{A_\text{цикл}}{Q_{12} + Q_{23}}  = \frac{A_\text{цикл}}{A_{12} + \Delta U_{12} + A_{23} + \Delta U_{23}} =  \\
     &= \frac{\frac 12 \cdot 25 \cdot P_0V_0}{0 + \frac 32 \cdot 5 \cdot P_0V_0 + 30P_0V_0 + \frac 32 \cdot 30 \cdot P_0V_0} = \frac{\frac 12 \cdot 25}{\frac 32 \cdot 5 + 30 + \frac 32 \cdot 30} = \frac{5}{33} \approx 0{,}152.
    \end{align*}
}
\solutionspace{360pt}

\tasknumber{2}%
\task{%
    Определите КПД (оставив ответ точным в виде нескоратимой дроби) цикла 1231, рабочим телом которого является идеальный одноатомный газ, если
    \begin{itemize}
        \item 12 — изобарическое расширение,
        \item 23 — процесс, график которого в $PV$-координатах является отрезком прямой, а объём уменьшается в шесть раз,
        \item 31 — изохорический нагрев с увеличением давления в шесть раз,
    \end{itemize}
}
\answer{%
    \begin{align*}
    A_{12} &> 0, \Delta U_{12} > 0, \implies Q_{12} = A_{12} + \Delta U_{12} > 0.
    \\
    A_{23} &< 0, \Delta U_{23} < 0, \implies Q_{23} = A_{23} + \Delta U_{23} < 0, \\
    A_{31} &= 0, \Delta U_{31} > 0, \implies Q_{31} = A_{31} + \Delta U_{31} > 0.
    \\
    P_1V_1 &= \nu R T_1, P_2V_2 = \nu R T_2, P_3V_3 = \nu R T_3 \text{ — уравнения состояния идеального газа}, \\
    &\text{Пусть $P_0$, $V_0$, $T_0$ — давление, объём и температура в точке 3 (минимальные во всём цикле):} \\
    P_3 &= P_0, P_1 = P_2 = 6P_0, V_1 = V_3 = V_0, V_2 = 6V_3 = 6V_0 \\
    A_\text{цикл} &= \frac 12 (P_2-P_1)(V_1-V_2) = \frac 12 (6P_0 - P_0)(6V_0 - V_0) = \frac 12 \cdot 25 \cdot P_0V_0, \\
    A_{12} &= 6P_0 \cdot (6V_0 - V_0) = 30P_0V_0, \\
    \Delta U_{12} &= \frac 32 \nu R T_2 - \frac 32 \nu R T_1 = \frac 32 P_2 V_2 - \frac 32 P_1 V_1 = \frac 32 \cdot 6 P_0 \cdot 6 V_0 -  \frac 32 \cdot 6 P_0 \cdot V_0 = \frac 32 \cdot 30 \cdot P_0V_0, \\
    \Delta U_{31} &= \frac 32 \nu R T_1 - \frac 32 \nu R T_3 = \frac 32 P_1 V_1 - \frac 32 P_3 V_3 = \frac 32 \cdot 6 P_0V_0 - \frac 32 P_0V_0 = \frac 32 \cdot 5 \cdot P_0V_0.
    \\
    \eta &= \frac{A_\text{цикл}}{Q_+} = \frac{A_\text{цикл}}{Q_{12} + Q_{31}}  = \frac{A_\text{цикл}}{A_{12} + \Delta U_{12} + A_{31} + \Delta U_{31}} =  \\
     &= \frac{\frac 12 \cdot 25 \cdot P_0V_0}{30P_0V_0 + \frac 32 \cdot 30 \cdot P_0V_0 + 0 + \frac 32 \cdot 5 \cdot P_0V_0} = \frac{\frac 12 \cdot 25}{30 + \frac 32 \cdot 30 + \frac 32 \cdot 5} = \frac{5}{33} \approx 0{,}152.
    \end{align*}
}

\variantsplitter

\addpersonalvariant{Наталья Кравченко}

\tasknumber{1}%
\task{%
    Определите КПД (оставив ответ точным в виде нескоратимой дроби) цикла 1231, рабочим телом которого является идеальный одноатомный газ, если
    \begin{itemize}
        \item 12 — изохорический нагрев в два раза,
        \item 23 — изобарическое расширение, при котором температура растёт в шесть раз,
        \item 31 — процесс, график которого в $PV$-координатах является отрезком прямой.
    \end{itemize}
}
\answer{%
    \begin{align*}
    A_{12} &= 0, \Delta U_{12} > 0, \implies Q_{12} = A_{12} + \Delta U_{12} > 0.
    \\
    A_{23} &> 0, \Delta U_{23} > 0, \implies Q_{23} = A_{23} + \Delta U_{23} > 0, \\
    A_{31} &= 0, \Delta U_{31} < 0, \implies Q_{31} = A_{31} + \Delta U_{31} < 0.
    \\
    P_1V_1 &= \nu R T_1, P_2V_2 = \nu R T_2, P_3V_3 = \nu R T_3 \text{ — уравнения состояния идеального газа}, \\
    &\text{Пусть $P_0$, $V_0$, $T_0$ — давление, объём и температура в точке 1 (минимальные во всём цикле):} \\
    P_1 &= P_0, P_2 = P_3, V_1 = V_2 = V_0, \text{остальные соотношения нужно считать} \\
    T_2 &= 2T_1 = 2T_0 \text{(по условию)} \implies \frac{P_2}{P_1} = \frac{P_2V_0}{P_1V_0} = \frac{P_2 V_2}{P_1 V_1}= \frac{\nu R T_2}{\nu R T_1} = \frac{T_2}{T_1} = 2 \implies P_2 = 2 P_1 = 2 P_0, \\
    T_3 &= 6T_2 = 12T_0 \text{(по условию)} \implies \frac{V_3}{V_2} = \frac{P_3V_3}{P_2V_2}= \frac{\nu R T_3}{\nu R T_2} = \frac{T_3}{T_2} = 6 \implies V_3 = 6 V_2 = 6 V_0.
    \\
    A_\text{цикл} &= \frac 12 (6P_0 - P_0)(2V_0 - V_0) = \frac 12 \cdot 5 \cdot P_0V_0, \\
    A_{23} &= 2P_0 \cdot (6V_0 - V_0) = 10P_0V_0, \\
    \Delta U_{23} &= \frac 32 \nu R T_3 - \frac 32 \nu R T_3 = \frac 32 P_3 V_3 - \frac 32 P_2 V_2 = \frac 32 \cdot 2 P_0 \cdot 6 V_0 -  \frac 32 \cdot 2 P_0 \cdot V_0 = \frac 32 \cdot 10 \cdot P_0V_0, \\
    \Delta U_{12} &= \frac 32 \nu R T_2 - \frac 32 \nu R T_1 = \frac 32 P_2 V_2 - \frac 32 P_1 V_1 = \frac 32 \cdot 2 P_0V_0 - \frac 32 P_0V_0 = \frac 32 \cdot 1 \cdot P_0V_0.
    \\
    \eta &= \frac{A_\text{цикл}}{Q_+} = \frac{A_\text{цикл}}{Q_{12} + Q_{23}}  = \frac{A_\text{цикл}}{A_{12} + \Delta U_{12} + A_{23} + \Delta U_{23}} =  \\
     &= \frac{\frac 12 \cdot 5 \cdot P_0V_0}{0 + \frac 32 \cdot 1 \cdot P_0V_0 + 10P_0V_0 + \frac 32 \cdot 10 \cdot P_0V_0} = \frac{\frac 12 \cdot 5}{\frac 32 \cdot 1 + 10 + \frac 32 \cdot 10} = \frac{5}{53} \approx 0{,}094.
    \end{align*}
}
\solutionspace{360pt}

\tasknumber{2}%
\task{%
    Определите КПД (оставив ответ точным в виде нескоратимой дроби) цикла 1231, рабочим телом которого является идеальный одноатомный газ, если
    \begin{itemize}
        \item 12 — изобарическое расширение,
        \item 23 — процесс, график которого в $PV$-координатах является отрезком прямой, а объём уменьшается в шесть раз,
        \item 31 — изохорический нагрев с увеличением давления в два раза,
    \end{itemize}
}
\answer{%
    \begin{align*}
    A_{12} &> 0, \Delta U_{12} > 0, \implies Q_{12} = A_{12} + \Delta U_{12} > 0.
    \\
    A_{23} &< 0, \Delta U_{23} < 0, \implies Q_{23} = A_{23} + \Delta U_{23} < 0, \\
    A_{31} &= 0, \Delta U_{31} > 0, \implies Q_{31} = A_{31} + \Delta U_{31} > 0.
    \\
    P_1V_1 &= \nu R T_1, P_2V_2 = \nu R T_2, P_3V_3 = \nu R T_3 \text{ — уравнения состояния идеального газа}, \\
    &\text{Пусть $P_0$, $V_0$, $T_0$ — давление, объём и температура в точке 3 (минимальные во всём цикле):} \\
    P_3 &= P_0, P_1 = P_2 = 2P_0, V_1 = V_3 = V_0, V_2 = 6V_3 = 6V_0 \\
    A_\text{цикл} &= \frac 12 (P_2-P_1)(V_1-V_2) = \frac 12 (2P_0 - P_0)(6V_0 - V_0) = \frac 12 \cdot 5 \cdot P_0V_0, \\
    A_{12} &= 6P_0 \cdot (2V_0 - V_0) = 6P_0V_0, \\
    \Delta U_{12} &= \frac 32 \nu R T_2 - \frac 32 \nu R T_1 = \frac 32 P_2 V_2 - \frac 32 P_1 V_1 = \frac 32 \cdot 2 P_0 \cdot 6 V_0 -  \frac 32 \cdot 6 P_0 \cdot V_0 = \frac 32 \cdot 6 \cdot P_0V_0, \\
    \Delta U_{31} &= \frac 32 \nu R T_1 - \frac 32 \nu R T_3 = \frac 32 P_1 V_1 - \frac 32 P_3 V_3 = \frac 32 \cdot 2 P_0V_0 - \frac 32 P_0V_0 = \frac 32 \cdot 1 \cdot P_0V_0.
    \\
    \eta &= \frac{A_\text{цикл}}{Q_+} = \frac{A_\text{цикл}}{Q_{12} + Q_{31}}  = \frac{A_\text{цикл}}{A_{12} + \Delta U_{12} + A_{31} + \Delta U_{31}} =  \\
     &= \frac{\frac 12 \cdot 5 \cdot P_0V_0}{6P_0V_0 + \frac 32 \cdot 6 \cdot P_0V_0 + 0 + \frac 32 \cdot 1 \cdot P_0V_0} = \frac{\frac 12 \cdot 5}{6 + \frac 32 \cdot 6 + \frac 32 \cdot 1} = \frac{5}{33} \approx 0{,}152.
    \end{align*}
}

\variantsplitter

\addpersonalvariant{Матвей Кузьмин}

\tasknumber{1}%
\task{%
    Определите КПД (оставив ответ точным в виде нескоратимой дроби) цикла 1231, рабочим телом которого является идеальный одноатомный газ, если
    \begin{itemize}
        \item 12 — изохорический нагрев в четыре раза,
        \item 23 — изобарическое расширение, при котором температура растёт в шесть раз,
        \item 31 — процесс, график которого в $PV$-координатах является отрезком прямой.
    \end{itemize}
}
\answer{%
    \begin{align*}
    A_{12} &= 0, \Delta U_{12} > 0, \implies Q_{12} = A_{12} + \Delta U_{12} > 0.
    \\
    A_{23} &> 0, \Delta U_{23} > 0, \implies Q_{23} = A_{23} + \Delta U_{23} > 0, \\
    A_{31} &= 0, \Delta U_{31} < 0, \implies Q_{31} = A_{31} + \Delta U_{31} < 0.
    \\
    P_1V_1 &= \nu R T_1, P_2V_2 = \nu R T_2, P_3V_3 = \nu R T_3 \text{ — уравнения состояния идеального газа}, \\
    &\text{Пусть $P_0$, $V_0$, $T_0$ — давление, объём и температура в точке 1 (минимальные во всём цикле):} \\
    P_1 &= P_0, P_2 = P_3, V_1 = V_2 = V_0, \text{остальные соотношения нужно считать} \\
    T_2 &= 4T_1 = 4T_0 \text{(по условию)} \implies \frac{P_2}{P_1} = \frac{P_2V_0}{P_1V_0} = \frac{P_2 V_2}{P_1 V_1}= \frac{\nu R T_2}{\nu R T_1} = \frac{T_2}{T_1} = 4 \implies P_2 = 4 P_1 = 4 P_0, \\
    T_3 &= 6T_2 = 24T_0 \text{(по условию)} \implies \frac{V_3}{V_2} = \frac{P_3V_3}{P_2V_2}= \frac{\nu R T_3}{\nu R T_2} = \frac{T_3}{T_2} = 6 \implies V_3 = 6 V_2 = 6 V_0.
    \\
    A_\text{цикл} &= \frac 12 (6P_0 - P_0)(4V_0 - V_0) = \frac 12 \cdot 15 \cdot P_0V_0, \\
    A_{23} &= 4P_0 \cdot (6V_0 - V_0) = 20P_0V_0, \\
    \Delta U_{23} &= \frac 32 \nu R T_3 - \frac 32 \nu R T_3 = \frac 32 P_3 V_3 - \frac 32 P_2 V_2 = \frac 32 \cdot 4 P_0 \cdot 6 V_0 -  \frac 32 \cdot 4 P_0 \cdot V_0 = \frac 32 \cdot 20 \cdot P_0V_0, \\
    \Delta U_{12} &= \frac 32 \nu R T_2 - \frac 32 \nu R T_1 = \frac 32 P_2 V_2 - \frac 32 P_1 V_1 = \frac 32 \cdot 4 P_0V_0 - \frac 32 P_0V_0 = \frac 32 \cdot 3 \cdot P_0V_0.
    \\
    \eta &= \frac{A_\text{цикл}}{Q_+} = \frac{A_\text{цикл}}{Q_{12} + Q_{23}}  = \frac{A_\text{цикл}}{A_{12} + \Delta U_{12} + A_{23} + \Delta U_{23}} =  \\
     &= \frac{\frac 12 \cdot 15 \cdot P_0V_0}{0 + \frac 32 \cdot 3 \cdot P_0V_0 + 20P_0V_0 + \frac 32 \cdot 20 \cdot P_0V_0} = \frac{\frac 12 \cdot 15}{\frac 32 \cdot 3 + 20 + \frac 32 \cdot 20} = \frac{15}{109} \approx 0{,}138.
    \end{align*}
}
\solutionspace{360pt}

\tasknumber{2}%
\task{%
    Определите КПД (оставив ответ точным в виде нескоратимой дроби) цикла 1231, рабочим телом которого является идеальный одноатомный газ, если
    \begin{itemize}
        \item 12 — изобарическое расширение,
        \item 23 — процесс, график которого в $PV$-координатах является отрезком прямой, а объём уменьшается в шесть раз,
        \item 31 — изохорический нагрев с увеличением давления в четыре раза,
    \end{itemize}
}
\answer{%
    \begin{align*}
    A_{12} &> 0, \Delta U_{12} > 0, \implies Q_{12} = A_{12} + \Delta U_{12} > 0.
    \\
    A_{23} &< 0, \Delta U_{23} < 0, \implies Q_{23} = A_{23} + \Delta U_{23} < 0, \\
    A_{31} &= 0, \Delta U_{31} > 0, \implies Q_{31} = A_{31} + \Delta U_{31} > 0.
    \\
    P_1V_1 &= \nu R T_1, P_2V_2 = \nu R T_2, P_3V_3 = \nu R T_3 \text{ — уравнения состояния идеального газа}, \\
    &\text{Пусть $P_0$, $V_0$, $T_0$ — давление, объём и температура в точке 3 (минимальные во всём цикле):} \\
    P_3 &= P_0, P_1 = P_2 = 4P_0, V_1 = V_3 = V_0, V_2 = 6V_3 = 6V_0 \\
    A_\text{цикл} &= \frac 12 (P_2-P_1)(V_1-V_2) = \frac 12 (4P_0 - P_0)(6V_0 - V_0) = \frac 12 \cdot 15 \cdot P_0V_0, \\
    A_{12} &= 6P_0 \cdot (4V_0 - V_0) = 18P_0V_0, \\
    \Delta U_{12} &= \frac 32 \nu R T_2 - \frac 32 \nu R T_1 = \frac 32 P_2 V_2 - \frac 32 P_1 V_1 = \frac 32 \cdot 4 P_0 \cdot 6 V_0 -  \frac 32 \cdot 6 P_0 \cdot V_0 = \frac 32 \cdot 18 \cdot P_0V_0, \\
    \Delta U_{31} &= \frac 32 \nu R T_1 - \frac 32 \nu R T_3 = \frac 32 P_1 V_1 - \frac 32 P_3 V_3 = \frac 32 \cdot 4 P_0V_0 - \frac 32 P_0V_0 = \frac 32 \cdot 3 \cdot P_0V_0.
    \\
    \eta &= \frac{A_\text{цикл}}{Q_+} = \frac{A_\text{цикл}}{Q_{12} + Q_{31}}  = \frac{A_\text{цикл}}{A_{12} + \Delta U_{12} + A_{31} + \Delta U_{31}} =  \\
     &= \frac{\frac 12 \cdot 15 \cdot P_0V_0}{18P_0V_0 + \frac 32 \cdot 18 \cdot P_0V_0 + 0 + \frac 32 \cdot 3 \cdot P_0V_0} = \frac{\frac 12 \cdot 15}{18 + \frac 32 \cdot 18 + \frac 32 \cdot 3} = \frac{5}{33} \approx 0{,}152.
    \end{align*}
}

\variantsplitter

\addpersonalvariant{Сергей Малышев}

\tasknumber{1}%
\task{%
    Определите КПД (оставив ответ точным в виде нескоратимой дроби) цикла 1231, рабочим телом которого является идеальный одноатомный газ, если
    \begin{itemize}
        \item 12 — изохорический нагрев в шесть раз,
        \item 23 — изобарическое расширение, при котором температура растёт в четыре раза,
        \item 31 — процесс, график которого в $PV$-координатах является отрезком прямой.
    \end{itemize}
}
\answer{%
    \begin{align*}
    A_{12} &= 0, \Delta U_{12} > 0, \implies Q_{12} = A_{12} + \Delta U_{12} > 0.
    \\
    A_{23} &> 0, \Delta U_{23} > 0, \implies Q_{23} = A_{23} + \Delta U_{23} > 0, \\
    A_{31} &= 0, \Delta U_{31} < 0, \implies Q_{31} = A_{31} + \Delta U_{31} < 0.
    \\
    P_1V_1 &= \nu R T_1, P_2V_2 = \nu R T_2, P_3V_3 = \nu R T_3 \text{ — уравнения состояния идеального газа}, \\
    &\text{Пусть $P_0$, $V_0$, $T_0$ — давление, объём и температура в точке 1 (минимальные во всём цикле):} \\
    P_1 &= P_0, P_2 = P_3, V_1 = V_2 = V_0, \text{остальные соотношения нужно считать} \\
    T_2 &= 6T_1 = 6T_0 \text{(по условию)} \implies \frac{P_2}{P_1} = \frac{P_2V_0}{P_1V_0} = \frac{P_2 V_2}{P_1 V_1}= \frac{\nu R T_2}{\nu R T_1} = \frac{T_2}{T_1} = 6 \implies P_2 = 6 P_1 = 6 P_0, \\
    T_3 &= 4T_2 = 24T_0 \text{(по условию)} \implies \frac{V_3}{V_2} = \frac{P_3V_3}{P_2V_2}= \frac{\nu R T_3}{\nu R T_2} = \frac{T_3}{T_2} = 4 \implies V_3 = 4 V_2 = 4 V_0.
    \\
    A_\text{цикл} &= \frac 12 (4P_0 - P_0)(6V_0 - V_0) = \frac 12 \cdot 15 \cdot P_0V_0, \\
    A_{23} &= 6P_0 \cdot (4V_0 - V_0) = 18P_0V_0, \\
    \Delta U_{23} &= \frac 32 \nu R T_3 - \frac 32 \nu R T_3 = \frac 32 P_3 V_3 - \frac 32 P_2 V_2 = \frac 32 \cdot 6 P_0 \cdot 4 V_0 -  \frac 32 \cdot 6 P_0 \cdot V_0 = \frac 32 \cdot 18 \cdot P_0V_0, \\
    \Delta U_{12} &= \frac 32 \nu R T_2 - \frac 32 \nu R T_1 = \frac 32 P_2 V_2 - \frac 32 P_1 V_1 = \frac 32 \cdot 6 P_0V_0 - \frac 32 P_0V_0 = \frac 32 \cdot 5 \cdot P_0V_0.
    \\
    \eta &= \frac{A_\text{цикл}}{Q_+} = \frac{A_\text{цикл}}{Q_{12} + Q_{23}}  = \frac{A_\text{цикл}}{A_{12} + \Delta U_{12} + A_{23} + \Delta U_{23}} =  \\
     &= \frac{\frac 12 \cdot 15 \cdot P_0V_0}{0 + \frac 32 \cdot 5 \cdot P_0V_0 + 18P_0V_0 + \frac 32 \cdot 18 \cdot P_0V_0} = \frac{\frac 12 \cdot 15}{\frac 32 \cdot 5 + 18 + \frac 32 \cdot 18} = \frac{1}{7} \approx 0{,}143.
    \end{align*}
}
\solutionspace{360pt}

\tasknumber{2}%
\task{%
    Определите КПД (оставив ответ точным в виде нескоратимой дроби) цикла 1231, рабочим телом которого является идеальный одноатомный газ, если
    \begin{itemize}
        \item 12 — изобарическое расширение,
        \item 23 — процесс, график которого в $PV$-координатах является отрезком прямой, а объём уменьшается в четыре раза,
        \item 31 — изохорический нагрев с увеличением давления в шесть раз,
    \end{itemize}
}
\answer{%
    \begin{align*}
    A_{12} &> 0, \Delta U_{12} > 0, \implies Q_{12} = A_{12} + \Delta U_{12} > 0.
    \\
    A_{23} &< 0, \Delta U_{23} < 0, \implies Q_{23} = A_{23} + \Delta U_{23} < 0, \\
    A_{31} &= 0, \Delta U_{31} > 0, \implies Q_{31} = A_{31} + \Delta U_{31} > 0.
    \\
    P_1V_1 &= \nu R T_1, P_2V_2 = \nu R T_2, P_3V_3 = \nu R T_3 \text{ — уравнения состояния идеального газа}, \\
    &\text{Пусть $P_0$, $V_0$, $T_0$ — давление, объём и температура в точке 3 (минимальные во всём цикле):} \\
    P_3 &= P_0, P_1 = P_2 = 6P_0, V_1 = V_3 = V_0, V_2 = 4V_3 = 4V_0 \\
    A_\text{цикл} &= \frac 12 (P_2-P_1)(V_1-V_2) = \frac 12 (6P_0 - P_0)(4V_0 - V_0) = \frac 12 \cdot 15 \cdot P_0V_0, \\
    A_{12} &= 4P_0 \cdot (6V_0 - V_0) = 20P_0V_0, \\
    \Delta U_{12} &= \frac 32 \nu R T_2 - \frac 32 \nu R T_1 = \frac 32 P_2 V_2 - \frac 32 P_1 V_1 = \frac 32 \cdot 6 P_0 \cdot 4 V_0 -  \frac 32 \cdot 4 P_0 \cdot V_0 = \frac 32 \cdot 20 \cdot P_0V_0, \\
    \Delta U_{31} &= \frac 32 \nu R T_1 - \frac 32 \nu R T_3 = \frac 32 P_1 V_1 - \frac 32 P_3 V_3 = \frac 32 \cdot 6 P_0V_0 - \frac 32 P_0V_0 = \frac 32 \cdot 5 \cdot P_0V_0.
    \\
    \eta &= \frac{A_\text{цикл}}{Q_+} = \frac{A_\text{цикл}}{Q_{12} + Q_{31}}  = \frac{A_\text{цикл}}{A_{12} + \Delta U_{12} + A_{31} + \Delta U_{31}} =  \\
     &= \frac{\frac 12 \cdot 15 \cdot P_0V_0}{20P_0V_0 + \frac 32 \cdot 20 \cdot P_0V_0 + 0 + \frac 32 \cdot 5 \cdot P_0V_0} = \frac{\frac 12 \cdot 15}{20 + \frac 32 \cdot 20 + \frac 32 \cdot 5} = \frac{3}{23} \approx 0{,}130.
    \end{align*}
}

\variantsplitter

\addpersonalvariant{Алина Полканова}

\tasknumber{1}%
\task{%
    Определите КПД (оставив ответ точным в виде нескоратимой дроби) цикла 1231, рабочим телом которого является идеальный одноатомный газ, если
    \begin{itemize}
        \item 12 — изохорический нагрев в пять раз,
        \item 23 — изобарическое расширение, при котором температура растёт в шесть раз,
        \item 31 — процесс, график которого в $PV$-координатах является отрезком прямой.
    \end{itemize}
}
\answer{%
    \begin{align*}
    A_{12} &= 0, \Delta U_{12} > 0, \implies Q_{12} = A_{12} + \Delta U_{12} > 0.
    \\
    A_{23} &> 0, \Delta U_{23} > 0, \implies Q_{23} = A_{23} + \Delta U_{23} > 0, \\
    A_{31} &= 0, \Delta U_{31} < 0, \implies Q_{31} = A_{31} + \Delta U_{31} < 0.
    \\
    P_1V_1 &= \nu R T_1, P_2V_2 = \nu R T_2, P_3V_3 = \nu R T_3 \text{ — уравнения состояния идеального газа}, \\
    &\text{Пусть $P_0$, $V_0$, $T_0$ — давление, объём и температура в точке 1 (минимальные во всём цикле):} \\
    P_1 &= P_0, P_2 = P_3, V_1 = V_2 = V_0, \text{остальные соотношения нужно считать} \\
    T_2 &= 5T_1 = 5T_0 \text{(по условию)} \implies \frac{P_2}{P_1} = \frac{P_2V_0}{P_1V_0} = \frac{P_2 V_2}{P_1 V_1}= \frac{\nu R T_2}{\nu R T_1} = \frac{T_2}{T_1} = 5 \implies P_2 = 5 P_1 = 5 P_0, \\
    T_3 &= 6T_2 = 30T_0 \text{(по условию)} \implies \frac{V_3}{V_2} = \frac{P_3V_3}{P_2V_2}= \frac{\nu R T_3}{\nu R T_2} = \frac{T_3}{T_2} = 6 \implies V_3 = 6 V_2 = 6 V_0.
    \\
    A_\text{цикл} &= \frac 12 (6P_0 - P_0)(5V_0 - V_0) = \frac 12 \cdot 20 \cdot P_0V_0, \\
    A_{23} &= 5P_0 \cdot (6V_0 - V_0) = 25P_0V_0, \\
    \Delta U_{23} &= \frac 32 \nu R T_3 - \frac 32 \nu R T_3 = \frac 32 P_3 V_3 - \frac 32 P_2 V_2 = \frac 32 \cdot 5 P_0 \cdot 6 V_0 -  \frac 32 \cdot 5 P_0 \cdot V_0 = \frac 32 \cdot 25 \cdot P_0V_0, \\
    \Delta U_{12} &= \frac 32 \nu R T_2 - \frac 32 \nu R T_1 = \frac 32 P_2 V_2 - \frac 32 P_1 V_1 = \frac 32 \cdot 5 P_0V_0 - \frac 32 P_0V_0 = \frac 32 \cdot 4 \cdot P_0V_0.
    \\
    \eta &= \frac{A_\text{цикл}}{Q_+} = \frac{A_\text{цикл}}{Q_{12} + Q_{23}}  = \frac{A_\text{цикл}}{A_{12} + \Delta U_{12} + A_{23} + \Delta U_{23}} =  \\
     &= \frac{\frac 12 \cdot 20 \cdot P_0V_0}{0 + \frac 32 \cdot 4 \cdot P_0V_0 + 25P_0V_0 + \frac 32 \cdot 25 \cdot P_0V_0} = \frac{\frac 12 \cdot 20}{\frac 32 \cdot 4 + 25 + \frac 32 \cdot 25} = \frac{20}{137} \approx 0{,}146.
    \end{align*}
}
\solutionspace{360pt}

\tasknumber{2}%
\task{%
    Определите КПД (оставив ответ точным в виде нескоратимой дроби) цикла 1231, рабочим телом которого является идеальный одноатомный газ, если
    \begin{itemize}
        \item 12 — изобарическое расширение,
        \item 23 — процесс, график которого в $PV$-координатах является отрезком прямой, а объём уменьшается в шесть раз,
        \item 31 — изохорический нагрев с увеличением давления в пять раз,
    \end{itemize}
}
\answer{%
    \begin{align*}
    A_{12} &> 0, \Delta U_{12} > 0, \implies Q_{12} = A_{12} + \Delta U_{12} > 0.
    \\
    A_{23} &< 0, \Delta U_{23} < 0, \implies Q_{23} = A_{23} + \Delta U_{23} < 0, \\
    A_{31} &= 0, \Delta U_{31} > 0, \implies Q_{31} = A_{31} + \Delta U_{31} > 0.
    \\
    P_1V_1 &= \nu R T_1, P_2V_2 = \nu R T_2, P_3V_3 = \nu R T_3 \text{ — уравнения состояния идеального газа}, \\
    &\text{Пусть $P_0$, $V_0$, $T_0$ — давление, объём и температура в точке 3 (минимальные во всём цикле):} \\
    P_3 &= P_0, P_1 = P_2 = 5P_0, V_1 = V_3 = V_0, V_2 = 6V_3 = 6V_0 \\
    A_\text{цикл} &= \frac 12 (P_2-P_1)(V_1-V_2) = \frac 12 (5P_0 - P_0)(6V_0 - V_0) = \frac 12 \cdot 20 \cdot P_0V_0, \\
    A_{12} &= 6P_0 \cdot (5V_0 - V_0) = 24P_0V_0, \\
    \Delta U_{12} &= \frac 32 \nu R T_2 - \frac 32 \nu R T_1 = \frac 32 P_2 V_2 - \frac 32 P_1 V_1 = \frac 32 \cdot 5 P_0 \cdot 6 V_0 -  \frac 32 \cdot 6 P_0 \cdot V_0 = \frac 32 \cdot 24 \cdot P_0V_0, \\
    \Delta U_{31} &= \frac 32 \nu R T_1 - \frac 32 \nu R T_3 = \frac 32 P_1 V_1 - \frac 32 P_3 V_3 = \frac 32 \cdot 5 P_0V_0 - \frac 32 P_0V_0 = \frac 32 \cdot 4 \cdot P_0V_0.
    \\
    \eta &= \frac{A_\text{цикл}}{Q_+} = \frac{A_\text{цикл}}{Q_{12} + Q_{31}}  = \frac{A_\text{цикл}}{A_{12} + \Delta U_{12} + A_{31} + \Delta U_{31}} =  \\
     &= \frac{\frac 12 \cdot 20 \cdot P_0V_0}{24P_0V_0 + \frac 32 \cdot 24 \cdot P_0V_0 + 0 + \frac 32 \cdot 4 \cdot P_0V_0} = \frac{\frac 12 \cdot 20}{24 + \frac 32 \cdot 24 + \frac 32 \cdot 4} = \frac{5}{33} \approx 0{,}152.
    \end{align*}
}

\variantsplitter

\addpersonalvariant{Сергей Пономарёв}

\tasknumber{1}%
\task{%
    Определите КПД (оставив ответ точным в виде нескоратимой дроби) цикла 1231, рабочим телом которого является идеальный одноатомный газ, если
    \begin{itemize}
        \item 12 — изохорический нагрев в два раза,
        \item 23 — изобарическое расширение, при котором температура растёт в три раза,
        \item 31 — процесс, график которого в $PV$-координатах является отрезком прямой.
    \end{itemize}
}
\answer{%
    \begin{align*}
    A_{12} &= 0, \Delta U_{12} > 0, \implies Q_{12} = A_{12} + \Delta U_{12} > 0.
    \\
    A_{23} &> 0, \Delta U_{23} > 0, \implies Q_{23} = A_{23} + \Delta U_{23} > 0, \\
    A_{31} &= 0, \Delta U_{31} < 0, \implies Q_{31} = A_{31} + \Delta U_{31} < 0.
    \\
    P_1V_1 &= \nu R T_1, P_2V_2 = \nu R T_2, P_3V_3 = \nu R T_3 \text{ — уравнения состояния идеального газа}, \\
    &\text{Пусть $P_0$, $V_0$, $T_0$ — давление, объём и температура в точке 1 (минимальные во всём цикле):} \\
    P_1 &= P_0, P_2 = P_3, V_1 = V_2 = V_0, \text{остальные соотношения нужно считать} \\
    T_2 &= 2T_1 = 2T_0 \text{(по условию)} \implies \frac{P_2}{P_1} = \frac{P_2V_0}{P_1V_0} = \frac{P_2 V_2}{P_1 V_1}= \frac{\nu R T_2}{\nu R T_1} = \frac{T_2}{T_1} = 2 \implies P_2 = 2 P_1 = 2 P_0, \\
    T_3 &= 3T_2 = 6T_0 \text{(по условию)} \implies \frac{V_3}{V_2} = \frac{P_3V_3}{P_2V_2}= \frac{\nu R T_3}{\nu R T_2} = \frac{T_3}{T_2} = 3 \implies V_3 = 3 V_2 = 3 V_0.
    \\
    A_\text{цикл} &= \frac 12 (3P_0 - P_0)(2V_0 - V_0) = \frac 12 \cdot 2 \cdot P_0V_0, \\
    A_{23} &= 2P_0 \cdot (3V_0 - V_0) = 4P_0V_0, \\
    \Delta U_{23} &= \frac 32 \nu R T_3 - \frac 32 \nu R T_3 = \frac 32 P_3 V_3 - \frac 32 P_2 V_2 = \frac 32 \cdot 2 P_0 \cdot 3 V_0 -  \frac 32 \cdot 2 P_0 \cdot V_0 = \frac 32 \cdot 4 \cdot P_0V_0, \\
    \Delta U_{12} &= \frac 32 \nu R T_2 - \frac 32 \nu R T_1 = \frac 32 P_2 V_2 - \frac 32 P_1 V_1 = \frac 32 \cdot 2 P_0V_0 - \frac 32 P_0V_0 = \frac 32 \cdot 1 \cdot P_0V_0.
    \\
    \eta &= \frac{A_\text{цикл}}{Q_+} = \frac{A_\text{цикл}}{Q_{12} + Q_{23}}  = \frac{A_\text{цикл}}{A_{12} + \Delta U_{12} + A_{23} + \Delta U_{23}} =  \\
     &= \frac{\frac 12 \cdot 2 \cdot P_0V_0}{0 + \frac 32 \cdot 1 \cdot P_0V_0 + 4P_0V_0 + \frac 32 \cdot 4 \cdot P_0V_0} = \frac{\frac 12 \cdot 2}{\frac 32 \cdot 1 + 4 + \frac 32 \cdot 4} = \frac{2}{23} \approx 0{,}087.
    \end{align*}
}
\solutionspace{360pt}

\tasknumber{2}%
\task{%
    Определите КПД (оставив ответ точным в виде нескоратимой дроби) цикла 1231, рабочим телом которого является идеальный одноатомный газ, если
    \begin{itemize}
        \item 12 — изобарическое расширение,
        \item 23 — процесс, график которого в $PV$-координатах является отрезком прямой, а объём уменьшается в три раза,
        \item 31 — изохорический нагрев с увеличением давления в два раза,
    \end{itemize}
}
\answer{%
    \begin{align*}
    A_{12} &> 0, \Delta U_{12} > 0, \implies Q_{12} = A_{12} + \Delta U_{12} > 0.
    \\
    A_{23} &< 0, \Delta U_{23} < 0, \implies Q_{23} = A_{23} + \Delta U_{23} < 0, \\
    A_{31} &= 0, \Delta U_{31} > 0, \implies Q_{31} = A_{31} + \Delta U_{31} > 0.
    \\
    P_1V_1 &= \nu R T_1, P_2V_2 = \nu R T_2, P_3V_3 = \nu R T_3 \text{ — уравнения состояния идеального газа}, \\
    &\text{Пусть $P_0$, $V_0$, $T_0$ — давление, объём и температура в точке 3 (минимальные во всём цикле):} \\
    P_3 &= P_0, P_1 = P_2 = 2P_0, V_1 = V_3 = V_0, V_2 = 3V_3 = 3V_0 \\
    A_\text{цикл} &= \frac 12 (P_2-P_1)(V_1-V_2) = \frac 12 (2P_0 - P_0)(3V_0 - V_0) = \frac 12 \cdot 2 \cdot P_0V_0, \\
    A_{12} &= 3P_0 \cdot (2V_0 - V_0) = 3P_0V_0, \\
    \Delta U_{12} &= \frac 32 \nu R T_2 - \frac 32 \nu R T_1 = \frac 32 P_2 V_2 - \frac 32 P_1 V_1 = \frac 32 \cdot 2 P_0 \cdot 3 V_0 -  \frac 32 \cdot 3 P_0 \cdot V_0 = \frac 32 \cdot 3 \cdot P_0V_0, \\
    \Delta U_{31} &= \frac 32 \nu R T_1 - \frac 32 \nu R T_3 = \frac 32 P_1 V_1 - \frac 32 P_3 V_3 = \frac 32 \cdot 2 P_0V_0 - \frac 32 P_0V_0 = \frac 32 \cdot 1 \cdot P_0V_0.
    \\
    \eta &= \frac{A_\text{цикл}}{Q_+} = \frac{A_\text{цикл}}{Q_{12} + Q_{31}}  = \frac{A_\text{цикл}}{A_{12} + \Delta U_{12} + A_{31} + \Delta U_{31}} =  \\
     &= \frac{\frac 12 \cdot 2 \cdot P_0V_0}{3P_0V_0 + \frac 32 \cdot 3 \cdot P_0V_0 + 0 + \frac 32 \cdot 1 \cdot P_0V_0} = \frac{\frac 12 \cdot 2}{3 + \frac 32 \cdot 3 + \frac 32 \cdot 1} = \frac{1}{9} \approx 0{,}111.
    \end{align*}
}

\variantsplitter

\addpersonalvariant{Егор Свистушкин}

\tasknumber{1}%
\task{%
    Определите КПД (оставив ответ точным в виде нескоратимой дроби) цикла 1231, рабочим телом которого является идеальный одноатомный газ, если
    \begin{itemize}
        \item 12 — изохорический нагрев в три раза,
        \item 23 — изобарическое расширение, при котором температура растёт в три раза,
        \item 31 — процесс, график которого в $PV$-координатах является отрезком прямой.
    \end{itemize}
}
\answer{%
    \begin{align*}
    A_{12} &= 0, \Delta U_{12} > 0, \implies Q_{12} = A_{12} + \Delta U_{12} > 0.
    \\
    A_{23} &> 0, \Delta U_{23} > 0, \implies Q_{23} = A_{23} + \Delta U_{23} > 0, \\
    A_{31} &= 0, \Delta U_{31} < 0, \implies Q_{31} = A_{31} + \Delta U_{31} < 0.
    \\
    P_1V_1 &= \nu R T_1, P_2V_2 = \nu R T_2, P_3V_3 = \nu R T_3 \text{ — уравнения состояния идеального газа}, \\
    &\text{Пусть $P_0$, $V_0$, $T_0$ — давление, объём и температура в точке 1 (минимальные во всём цикле):} \\
    P_1 &= P_0, P_2 = P_3, V_1 = V_2 = V_0, \text{остальные соотношения нужно считать} \\
    T_2 &= 3T_1 = 3T_0 \text{(по условию)} \implies \frac{P_2}{P_1} = \frac{P_2V_0}{P_1V_0} = \frac{P_2 V_2}{P_1 V_1}= \frac{\nu R T_2}{\nu R T_1} = \frac{T_2}{T_1} = 3 \implies P_2 = 3 P_1 = 3 P_0, \\
    T_3 &= 3T_2 = 9T_0 \text{(по условию)} \implies \frac{V_3}{V_2} = \frac{P_3V_3}{P_2V_2}= \frac{\nu R T_3}{\nu R T_2} = \frac{T_3}{T_2} = 3 \implies V_3 = 3 V_2 = 3 V_0.
    \\
    A_\text{цикл} &= \frac 12 (3P_0 - P_0)(3V_0 - V_0) = \frac 12 \cdot 4 \cdot P_0V_0, \\
    A_{23} &= 3P_0 \cdot (3V_0 - V_0) = 6P_0V_0, \\
    \Delta U_{23} &= \frac 32 \nu R T_3 - \frac 32 \nu R T_3 = \frac 32 P_3 V_3 - \frac 32 P_2 V_2 = \frac 32 \cdot 3 P_0 \cdot 3 V_0 -  \frac 32 \cdot 3 P_0 \cdot V_0 = \frac 32 \cdot 6 \cdot P_0V_0, \\
    \Delta U_{12} &= \frac 32 \nu R T_2 - \frac 32 \nu R T_1 = \frac 32 P_2 V_2 - \frac 32 P_1 V_1 = \frac 32 \cdot 3 P_0V_0 - \frac 32 P_0V_0 = \frac 32 \cdot 2 \cdot P_0V_0.
    \\
    \eta &= \frac{A_\text{цикл}}{Q_+} = \frac{A_\text{цикл}}{Q_{12} + Q_{23}}  = \frac{A_\text{цикл}}{A_{12} + \Delta U_{12} + A_{23} + \Delta U_{23}} =  \\
     &= \frac{\frac 12 \cdot 4 \cdot P_0V_0}{0 + \frac 32 \cdot 2 \cdot P_0V_0 + 6P_0V_0 + \frac 32 \cdot 6 \cdot P_0V_0} = \frac{\frac 12 \cdot 4}{\frac 32 \cdot 2 + 6 + \frac 32 \cdot 6} = \frac{1}{9} \approx 0{,}111.
    \end{align*}
}
\solutionspace{360pt}

\tasknumber{2}%
\task{%
    Определите КПД (оставив ответ точным в виде нескоратимой дроби) цикла 1231, рабочим телом которого является идеальный одноатомный газ, если
    \begin{itemize}
        \item 12 — изобарическое расширение,
        \item 23 — процесс, график которого в $PV$-координатах является отрезком прямой, а объём уменьшается в три раза,
        \item 31 — изохорический нагрев с увеличением давления в три раза,
    \end{itemize}
}
\answer{%
    \begin{align*}
    A_{12} &> 0, \Delta U_{12} > 0, \implies Q_{12} = A_{12} + \Delta U_{12} > 0.
    \\
    A_{23} &< 0, \Delta U_{23} < 0, \implies Q_{23} = A_{23} + \Delta U_{23} < 0, \\
    A_{31} &= 0, \Delta U_{31} > 0, \implies Q_{31} = A_{31} + \Delta U_{31} > 0.
    \\
    P_1V_1 &= \nu R T_1, P_2V_2 = \nu R T_2, P_3V_3 = \nu R T_3 \text{ — уравнения состояния идеального газа}, \\
    &\text{Пусть $P_0$, $V_0$, $T_0$ — давление, объём и температура в точке 3 (минимальные во всём цикле):} \\
    P_3 &= P_0, P_1 = P_2 = 3P_0, V_1 = V_3 = V_0, V_2 = 3V_3 = 3V_0 \\
    A_\text{цикл} &= \frac 12 (P_2-P_1)(V_1-V_2) = \frac 12 (3P_0 - P_0)(3V_0 - V_0) = \frac 12 \cdot 4 \cdot P_0V_0, \\
    A_{12} &= 3P_0 \cdot (3V_0 - V_0) = 6P_0V_0, \\
    \Delta U_{12} &= \frac 32 \nu R T_2 - \frac 32 \nu R T_1 = \frac 32 P_2 V_2 - \frac 32 P_1 V_1 = \frac 32 \cdot 3 P_0 \cdot 3 V_0 -  \frac 32 \cdot 3 P_0 \cdot V_0 = \frac 32 \cdot 6 \cdot P_0V_0, \\
    \Delta U_{31} &= \frac 32 \nu R T_1 - \frac 32 \nu R T_3 = \frac 32 P_1 V_1 - \frac 32 P_3 V_3 = \frac 32 \cdot 3 P_0V_0 - \frac 32 P_0V_0 = \frac 32 \cdot 2 \cdot P_0V_0.
    \\
    \eta &= \frac{A_\text{цикл}}{Q_+} = \frac{A_\text{цикл}}{Q_{12} + Q_{31}}  = \frac{A_\text{цикл}}{A_{12} + \Delta U_{12} + A_{31} + \Delta U_{31}} =  \\
     &= \frac{\frac 12 \cdot 4 \cdot P_0V_0}{6P_0V_0 + \frac 32 \cdot 6 \cdot P_0V_0 + 0 + \frac 32 \cdot 2 \cdot P_0V_0} = \frac{\frac 12 \cdot 4}{6 + \frac 32 \cdot 6 + \frac 32 \cdot 2} = \frac{1}{9} \approx 0{,}111.
    \end{align*}
}

\variantsplitter

\addpersonalvariant{Дмитрий Соколов}

\tasknumber{1}%
\task{%
    Определите КПД (оставив ответ точным в виде нескоратимой дроби) цикла 1231, рабочим телом которого является идеальный одноатомный газ, если
    \begin{itemize}
        \item 12 — изохорический нагрев в два раза,
        \item 23 — изобарическое расширение, при котором температура растёт в три раза,
        \item 31 — процесс, график которого в $PV$-координатах является отрезком прямой.
    \end{itemize}
}
\answer{%
    \begin{align*}
    A_{12} &= 0, \Delta U_{12} > 0, \implies Q_{12} = A_{12} + \Delta U_{12} > 0.
    \\
    A_{23} &> 0, \Delta U_{23} > 0, \implies Q_{23} = A_{23} + \Delta U_{23} > 0, \\
    A_{31} &= 0, \Delta U_{31} < 0, \implies Q_{31} = A_{31} + \Delta U_{31} < 0.
    \\
    P_1V_1 &= \nu R T_1, P_2V_2 = \nu R T_2, P_3V_3 = \nu R T_3 \text{ — уравнения состояния идеального газа}, \\
    &\text{Пусть $P_0$, $V_0$, $T_0$ — давление, объём и температура в точке 1 (минимальные во всём цикле):} \\
    P_1 &= P_0, P_2 = P_3, V_1 = V_2 = V_0, \text{остальные соотношения нужно считать} \\
    T_2 &= 2T_1 = 2T_0 \text{(по условию)} \implies \frac{P_2}{P_1} = \frac{P_2V_0}{P_1V_0} = \frac{P_2 V_2}{P_1 V_1}= \frac{\nu R T_2}{\nu R T_1} = \frac{T_2}{T_1} = 2 \implies P_2 = 2 P_1 = 2 P_0, \\
    T_3 &= 3T_2 = 6T_0 \text{(по условию)} \implies \frac{V_3}{V_2} = \frac{P_3V_3}{P_2V_2}= \frac{\nu R T_3}{\nu R T_2} = \frac{T_3}{T_2} = 3 \implies V_3 = 3 V_2 = 3 V_0.
    \\
    A_\text{цикл} &= \frac 12 (3P_0 - P_0)(2V_0 - V_0) = \frac 12 \cdot 2 \cdot P_0V_0, \\
    A_{23} &= 2P_0 \cdot (3V_0 - V_0) = 4P_0V_0, \\
    \Delta U_{23} &= \frac 32 \nu R T_3 - \frac 32 \nu R T_3 = \frac 32 P_3 V_3 - \frac 32 P_2 V_2 = \frac 32 \cdot 2 P_0 \cdot 3 V_0 -  \frac 32 \cdot 2 P_0 \cdot V_0 = \frac 32 \cdot 4 \cdot P_0V_0, \\
    \Delta U_{12} &= \frac 32 \nu R T_2 - \frac 32 \nu R T_1 = \frac 32 P_2 V_2 - \frac 32 P_1 V_1 = \frac 32 \cdot 2 P_0V_0 - \frac 32 P_0V_0 = \frac 32 \cdot 1 \cdot P_0V_0.
    \\
    \eta &= \frac{A_\text{цикл}}{Q_+} = \frac{A_\text{цикл}}{Q_{12} + Q_{23}}  = \frac{A_\text{цикл}}{A_{12} + \Delta U_{12} + A_{23} + \Delta U_{23}} =  \\
     &= \frac{\frac 12 \cdot 2 \cdot P_0V_0}{0 + \frac 32 \cdot 1 \cdot P_0V_0 + 4P_0V_0 + \frac 32 \cdot 4 \cdot P_0V_0} = \frac{\frac 12 \cdot 2}{\frac 32 \cdot 1 + 4 + \frac 32 \cdot 4} = \frac{2}{23} \approx 0{,}087.
    \end{align*}
}
\solutionspace{360pt}

\tasknumber{2}%
\task{%
    Определите КПД (оставив ответ точным в виде нескоратимой дроби) цикла 1231, рабочим телом которого является идеальный одноатомный газ, если
    \begin{itemize}
        \item 12 — изобарическое расширение,
        \item 23 — процесс, график которого в $PV$-координатах является отрезком прямой, а объём уменьшается в три раза,
        \item 31 — изохорический нагрев с увеличением давления в два раза,
    \end{itemize}
}
\answer{%
    \begin{align*}
    A_{12} &> 0, \Delta U_{12} > 0, \implies Q_{12} = A_{12} + \Delta U_{12} > 0.
    \\
    A_{23} &< 0, \Delta U_{23} < 0, \implies Q_{23} = A_{23} + \Delta U_{23} < 0, \\
    A_{31} &= 0, \Delta U_{31} > 0, \implies Q_{31} = A_{31} + \Delta U_{31} > 0.
    \\
    P_1V_1 &= \nu R T_1, P_2V_2 = \nu R T_2, P_3V_3 = \nu R T_3 \text{ — уравнения состояния идеального газа}, \\
    &\text{Пусть $P_0$, $V_0$, $T_0$ — давление, объём и температура в точке 3 (минимальные во всём цикле):} \\
    P_3 &= P_0, P_1 = P_2 = 2P_0, V_1 = V_3 = V_0, V_2 = 3V_3 = 3V_0 \\
    A_\text{цикл} &= \frac 12 (P_2-P_1)(V_1-V_2) = \frac 12 (2P_0 - P_0)(3V_0 - V_0) = \frac 12 \cdot 2 \cdot P_0V_0, \\
    A_{12} &= 3P_0 \cdot (2V_0 - V_0) = 3P_0V_0, \\
    \Delta U_{12} &= \frac 32 \nu R T_2 - \frac 32 \nu R T_1 = \frac 32 P_2 V_2 - \frac 32 P_1 V_1 = \frac 32 \cdot 2 P_0 \cdot 3 V_0 -  \frac 32 \cdot 3 P_0 \cdot V_0 = \frac 32 \cdot 3 \cdot P_0V_0, \\
    \Delta U_{31} &= \frac 32 \nu R T_1 - \frac 32 \nu R T_3 = \frac 32 P_1 V_1 - \frac 32 P_3 V_3 = \frac 32 \cdot 2 P_0V_0 - \frac 32 P_0V_0 = \frac 32 \cdot 1 \cdot P_0V_0.
    \\
    \eta &= \frac{A_\text{цикл}}{Q_+} = \frac{A_\text{цикл}}{Q_{12} + Q_{31}}  = \frac{A_\text{цикл}}{A_{12} + \Delta U_{12} + A_{31} + \Delta U_{31}} =  \\
     &= \frac{\frac 12 \cdot 2 \cdot P_0V_0}{3P_0V_0 + \frac 32 \cdot 3 \cdot P_0V_0 + 0 + \frac 32 \cdot 1 \cdot P_0V_0} = \frac{\frac 12 \cdot 2}{3 + \frac 32 \cdot 3 + \frac 32 \cdot 1} = \frac{1}{9} \approx 0{,}111.
    \end{align*}
}

\variantsplitter

\addpersonalvariant{Арсений Трофимов}

\tasknumber{1}%
\task{%
    Определите КПД (оставив ответ точным в виде нескоратимой дроби) цикла 1231, рабочим телом которого является идеальный одноатомный газ, если
    \begin{itemize}
        \item 12 — изохорический нагрев в шесть раз,
        \item 23 — изобарическое расширение, при котором температура растёт в пять раз,
        \item 31 — процесс, график которого в $PV$-координатах является отрезком прямой.
    \end{itemize}
}
\answer{%
    \begin{align*}
    A_{12} &= 0, \Delta U_{12} > 0, \implies Q_{12} = A_{12} + \Delta U_{12} > 0.
    \\
    A_{23} &> 0, \Delta U_{23} > 0, \implies Q_{23} = A_{23} + \Delta U_{23} > 0, \\
    A_{31} &= 0, \Delta U_{31} < 0, \implies Q_{31} = A_{31} + \Delta U_{31} < 0.
    \\
    P_1V_1 &= \nu R T_1, P_2V_2 = \nu R T_2, P_3V_3 = \nu R T_3 \text{ — уравнения состояния идеального газа}, \\
    &\text{Пусть $P_0$, $V_0$, $T_0$ — давление, объём и температура в точке 1 (минимальные во всём цикле):} \\
    P_1 &= P_0, P_2 = P_3, V_1 = V_2 = V_0, \text{остальные соотношения нужно считать} \\
    T_2 &= 6T_1 = 6T_0 \text{(по условию)} \implies \frac{P_2}{P_1} = \frac{P_2V_0}{P_1V_0} = \frac{P_2 V_2}{P_1 V_1}= \frac{\nu R T_2}{\nu R T_1} = \frac{T_2}{T_1} = 6 \implies P_2 = 6 P_1 = 6 P_0, \\
    T_3 &= 5T_2 = 30T_0 \text{(по условию)} \implies \frac{V_3}{V_2} = \frac{P_3V_3}{P_2V_2}= \frac{\nu R T_3}{\nu R T_2} = \frac{T_3}{T_2} = 5 \implies V_3 = 5 V_2 = 5 V_0.
    \\
    A_\text{цикл} &= \frac 12 (5P_0 - P_0)(6V_0 - V_0) = \frac 12 \cdot 20 \cdot P_0V_0, \\
    A_{23} &= 6P_0 \cdot (5V_0 - V_0) = 24P_0V_0, \\
    \Delta U_{23} &= \frac 32 \nu R T_3 - \frac 32 \nu R T_3 = \frac 32 P_3 V_3 - \frac 32 P_2 V_2 = \frac 32 \cdot 6 P_0 \cdot 5 V_0 -  \frac 32 \cdot 6 P_0 \cdot V_0 = \frac 32 \cdot 24 \cdot P_0V_0, \\
    \Delta U_{12} &= \frac 32 \nu R T_2 - \frac 32 \nu R T_1 = \frac 32 P_2 V_2 - \frac 32 P_1 V_1 = \frac 32 \cdot 6 P_0V_0 - \frac 32 P_0V_0 = \frac 32 \cdot 5 \cdot P_0V_0.
    \\
    \eta &= \frac{A_\text{цикл}}{Q_+} = \frac{A_\text{цикл}}{Q_{12} + Q_{23}}  = \frac{A_\text{цикл}}{A_{12} + \Delta U_{12} + A_{23} + \Delta U_{23}} =  \\
     &= \frac{\frac 12 \cdot 20 \cdot P_0V_0}{0 + \frac 32 \cdot 5 \cdot P_0V_0 + 24P_0V_0 + \frac 32 \cdot 24 \cdot P_0V_0} = \frac{\frac 12 \cdot 20}{\frac 32 \cdot 5 + 24 + \frac 32 \cdot 24} = \frac{4}{27} \approx 0{,}148.
    \end{align*}
}
\solutionspace{360pt}

\tasknumber{2}%
\task{%
    Определите КПД (оставив ответ точным в виде нескоратимой дроби) цикла 1231, рабочим телом которого является идеальный одноатомный газ, если
    \begin{itemize}
        \item 12 — изобарическое расширение,
        \item 23 — процесс, график которого в $PV$-координатах является отрезком прямой, а объём уменьшается в пять раз,
        \item 31 — изохорический нагрев с увеличением давления в шесть раз,
    \end{itemize}
}
\answer{%
    \begin{align*}
    A_{12} &> 0, \Delta U_{12} > 0, \implies Q_{12} = A_{12} + \Delta U_{12} > 0.
    \\
    A_{23} &< 0, \Delta U_{23} < 0, \implies Q_{23} = A_{23} + \Delta U_{23} < 0, \\
    A_{31} &= 0, \Delta U_{31} > 0, \implies Q_{31} = A_{31} + \Delta U_{31} > 0.
    \\
    P_1V_1 &= \nu R T_1, P_2V_2 = \nu R T_2, P_3V_3 = \nu R T_3 \text{ — уравнения состояния идеального газа}, \\
    &\text{Пусть $P_0$, $V_0$, $T_0$ — давление, объём и температура в точке 3 (минимальные во всём цикле):} \\
    P_3 &= P_0, P_1 = P_2 = 6P_0, V_1 = V_3 = V_0, V_2 = 5V_3 = 5V_0 \\
    A_\text{цикл} &= \frac 12 (P_2-P_1)(V_1-V_2) = \frac 12 (6P_0 - P_0)(5V_0 - V_0) = \frac 12 \cdot 20 \cdot P_0V_0, \\
    A_{12} &= 5P_0 \cdot (6V_0 - V_0) = 25P_0V_0, \\
    \Delta U_{12} &= \frac 32 \nu R T_2 - \frac 32 \nu R T_1 = \frac 32 P_2 V_2 - \frac 32 P_1 V_1 = \frac 32 \cdot 6 P_0 \cdot 5 V_0 -  \frac 32 \cdot 5 P_0 \cdot V_0 = \frac 32 \cdot 25 \cdot P_0V_0, \\
    \Delta U_{31} &= \frac 32 \nu R T_1 - \frac 32 \nu R T_3 = \frac 32 P_1 V_1 - \frac 32 P_3 V_3 = \frac 32 \cdot 6 P_0V_0 - \frac 32 P_0V_0 = \frac 32 \cdot 5 \cdot P_0V_0.
    \\
    \eta &= \frac{A_\text{цикл}}{Q_+} = \frac{A_\text{цикл}}{Q_{12} + Q_{31}}  = \frac{A_\text{цикл}}{A_{12} + \Delta U_{12} + A_{31} + \Delta U_{31}} =  \\
     &= \frac{\frac 12 \cdot 20 \cdot P_0V_0}{25P_0V_0 + \frac 32 \cdot 25 \cdot P_0V_0 + 0 + \frac 32 \cdot 5 \cdot P_0V_0} = \frac{\frac 12 \cdot 20}{25 + \frac 32 \cdot 25 + \frac 32 \cdot 5} = \frac{1}{7} \approx 0{,}143.
    \end{align*}
}
% autogenerated
