\documentclass[12pt,a4paper,oneside]{article}
\usepackage[T2A]{fontenc}
\usepackage[utf8]{inputenc}       
\usepackage[english,russian]{babel} % Download Russian fonts.
%\usepackage[colorlinks,urlcolor=blue,unicode]{hyperref}
%\usepackage{amsmath,amsfonts,amssymb,amsthm,amscd,mathrsfs} % Use AMS symbols.
%\usepackage{enumitem} \setlist{nolistsep}
%\textheight=26.5cm % высота текста
\textwidth=18cm % ширина текста
\oddsidemargin=-0.75cm 
\evensidemargin=-0.75cm 
%\topmargin=0cm % отступ от верхнего края
%\parindent=0pt % абзацный отступ
\parskip=3pt % интервал между абзацами
%\tolerance=100 % терпимость к "жидким" строкам
%\binoppenalty=1000 % штраф за перенос формул - 10000 - абсолютный запрет
%\relpenalty=10000
\pagestyle{empty} 

\begin{document}
\title{Рецензия на работу А.\,П.\,Мотренко \\ <<Оценка плотности совместного распределения>>}
\author{М.\,О.\,Бурмистров}
\date{январь 2013 г.}
\maketitle

В работе А.\,П.\,Мотренко исследовалась проблема оценки плотности совместного неоднородного распределения набора случайных величин, включающего в себя одновременно дискретные и непрерывные величины. 
Порождающие алгоритмы классификации зачастую приходится применять к данным, содержащим оба типа случайных величин, поэтому необходимо разработать общий подход к описанию плотности совместного распределения.

В работе рассмотрены два подхода к этой оценке: факторизация и непараметрическое оценивание. 
При исследовании факторизации получены явные выражения для совместной плотности смешанного распределения в предположении о том, что одна из случайных величин является смесью гауссовских, а другая бинарной. 
В случае, когда какие-либо предположения о природе распределений сделать быть не могут, применено непараметрическая оценка плотности совместного распределения. 
Изучен механизм подбора оптимального ядра сглаживания. 
Студенткой проведены эксперименты на синтетических и реальных данных, демонстрирующие эффективность подходов.

Достоинством работы является ???

Основным недостатком работы следует считать ???

Несмотря на отмеченные недостатки, работа является самостоятельным научным исследованием, и удовлетворяет всем предъявляемым требованиям.
\end{document}