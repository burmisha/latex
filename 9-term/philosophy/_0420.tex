Группы сочинений

	1. Воросы логики и методологии. Категории. Об истолковании. Топиаа. Первая и вторая аналитика... Они объединены общим названием - органон (греч. Орудие, инструмент).
	2. Метафизика. Отвлеченные и абстрактные вещи.  Размышления о начале бытия. Это понятие ввел другой философ, издавший труды Аристотеля - Эндронит Радосский. Тогда уже разделяли физику на логику, физику и эстетику. Метафизика - то что идет за физикой.
	3. Научные темы. Работы: Физика, о небе. О возникновении и уничтожения. О метеорлогике. Трактат о душе.
	4. Вопросы связанные с государством, историей, этикой и эстетикой. Никомахова этика - та, что передается от отца к сыну. Политика. Поэтика.

Первая классификация наук. Наука - всё к чему приложил руку человек, как символ рационального знания. Нужен критерий. Признаки классификации: телеологический (цели и задачи) и по предмету.

	1. Теоретические. Существуют ради самого знания. Физика изучает телесную природу. Математика - абстр природа вещей. Первая философия и логика - неизменные сущности и основы бытия. (Дальше - больше объем)
	2. Практические. Ради управления поведением. Этика - поведение человека. Экономика - управление в семье. Политика - управление в государстве.
	3. Творческие. Ради пользы и осуществления прекрасного. Поэтика. Риторика - нужно общение, 2 человека. Искусство, техне (греч. в т.ч. ремесла).

Аристотель изучал основные роды бытия. Категории - наиболее общие понятия, основные роды бытия. Аристотель относи их бытию, они описывают познание и язык. 
Аристотель выделил 10 категорий

	1. Сущность - главная
	2. Качество
	3. Количество
	4. Отношение - связывает 2 и 3
	5. Место - т.е. пространство
	6. Время
	7. Положение
	8. Обладание- есть-нет
	9. Действие - степень активности
	10. Страдание - степень пассивности

Сущность - то что ипсывает и характеризует вещь выражается идеей. Первые сущности и вторые сущности. Первая сущность для Аристотеля - конкретная реально существующая вещь, она главная. Вторая сущность - идеи, производные от вещей. Т.е. стул в первую очередь именно стул, а не идея. Аристотель в Метафизике критикует теорию идей Платона. Аргументы против упрощенной теории идей и е вполне кореректна

	1. Бесполезность. Ибо идеи лишь копии вещей, это лишь общие признаки всех конкретных людей. Кто бы говорил. Впрочем, Платона интересовали идменно идеи.
	2. Обособленность этих миров. Вещи вполне могут существовать
	3. Третий человек. Платон не объясняет, каким образом вещи причастны к идеям,где они там сидят. Для одной вещи нужно целых три идеи.
	4. Развитие вещей необъяснимо, ведь сами идеи неподвижны. Ноьведь Платон под конец сам отказался от первоначального взгляда о полной неподвижности идей.

Материя вечна, но мы не можем её схватить - она должна приобрести форму. Материя заключается в вещах. Первоматерия - чистая возможность, аморфное нечто, может стать всем чем угодно. Форма переводит её в вещь. Так материя получает определенность. Форма отделяет материю вещи от всей другой материи и вот уже эту структуру мы можем характеризовать. Пример: медный шар - форма меди и шар, медь - форма огня и земли и воздуха с водой, а они еще чего-то.... медь и форма и материя. Первоматерия лишена формы. Переход бытия из потенциального в актуальное - движение (для Платона это лишь свойство чувственных вещей, поэтому он его не изучал.) Движение вечно, как и материя.
Аристотель выделял в человеке душу как активную форму, а тело как материю.
Учение Аристотеля о причинности. Учение о 4 причинах. Материя и орма - главные виды ричин. Вещь можно характеризовать, испоьзуя материальную и формальную причины. Еще: действующая причина вещи и источник возникновения вещи (позволит отделить естественные и искусственные вещи). Ну и еще для полного ухвата сущности вещи нужна цель - 4 компонент. Целевая причина.
Пример из трудовой деятельности. О строительстве. Кирпич.
Движение рассмативалось как вечное. Нотутверждение о вечности мира риводит к мысли о существовании перводвигателя мира - можно интерпретировать как бога (не религиозный, а бог- ум, универсальная объяснительная причина). 5 доводов. Суть: чтобы найти причину движения нужно еайти причину, но у того есть еще причина... Тут то Аристотнль  и сказал, что необходимо остановиться: на границе мира есть что-то неподвижное и неимеющее причины.
Перводвигатель

	1. Неподвижен
	2. Нетелесн ибо 1, см. создание через движение.
	3. Форма форм. Вечно созерцает свою деятельность. И остается в стороне. Бог-теоретик. Бог-ум. Бог-нус. Его деятельность воплощается лишь опосредованно.
	4. Находится на перифирии мирв. Он не трансцендентен
	5. Лично не вмешивается в дела мира. Он не толкает, а лишь создал первотолчок - разбив материю.
	6. Он же и цель.
	7. Творит умом.

Шоза?
Аристотельвыдвигает принцип: пиирода ничего не делает напрасно. Какьорганизуется природа. Изтпервоматерии (гюле) происходит трансфлрмация. Обособляются две пары противоположных качеств - см. картинки в почте. Сухое влажное, теплое холодные. Их попарные комбинции - 4 элемента природы. Сначала возикает смесь элеметов и затем лробится (гомеомери) - образуются протоатомы, а уже из них появляются вещи.
Еще картинка изконцентрических кругов. Эфир движется по кругу - иллюзия вращения неба.
Виды движения

	1. Увеличение и уменьшения - категория количества
	2. Превращение - категория качества - куколка-бабочка причем сущность та же.
	3. Возникновения и уничтожение. Кстати нужен пример.
	4. Перемещение. Категория места и времени. Основной вид движения, остальные без него немыслимы. Круговое, прямолинейное, криволинейное. Движение происходит под действием силы (инерцию тогда не знали, м.б. кто-то подталкивает).

Учение о душе - естественнонаучных сочинение. Душа - форма тела. Энтелехия (то что имеет цель в самом себе). Ссылался на традицию считает душу бессмертной - разумной её части. Питающая, чувствующая, разумная части души. Но и разумную можно делить на активный  пассивный разум. И видимо, пассивная умирает.
 