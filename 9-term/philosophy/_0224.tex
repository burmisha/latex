\lectiondate{24 февраля}

Аристотель

Фалес

Анаксимандр. 
Апейрон~--- беспредельный. Перос~--- предел
Пытался построить космологию. Считал, что всё в мире возникает через борьбу и конфликт, а их путем обособляются противоположности, в первую очередь, тепло и холод. 
Разговоры про какое-то яйцо, оболочки и раcпад на огненные кольца, Земля~--- это цилиндр. 
Но мы не будем заниматься ерундой. 
Геоцентрическая модель мира. Первая географическая карта : Европа и Азия. Ввел использование песочных часов. 
Создал учение о происхождении людей : зародились во влажном иле, а потом все стало высыхать и они были вынуждены покинуть влажную среду. Вначале были рыбообразны, на суше оболочка высохла, лопнула и вышли люди. 

Анаксимен
Год рождения неизвестен даже энциклопедии. 
В то время был распространен и обычно указывался период акме (~28--32 лет) ~--- период расцвета мужчины. 
У Анаксимена~--- 546 год до н.\,э.. 
Первоначало~--- воздух. Он сгущается или рассеивается. 
Тоже интересовался космологией. Земля плоская, возникла из сгущения воздухе, висит в воздухе. 
Неподвижные звезды вбиты в небосвод, который вращается, а Луна и т.\,п. как листья в воздухе. 

Смысл философии Милетской школы. 
Искали единую первооснову для всех разнообразных процессов, пытались объяснить мир. 
Отказались от мифологических представлений, объясняли всё естественно. 
Интерес сосредоточен вокруг природы и окружающего мира, а не человека. Природа~--- фюзис, поэтому люди стали физиками :-) 

Героклит. Эфеский.
Жил в Эфесе. Родился около 540, умер около 480. Более серьёзен и глубок, чем та троица. 
Первый~--- натурфилософские представления. У него присутствует гносеологическая (учение о познании) тематика. 
Кстати, 
Онтология~--- учение о бытие. 
Антропология~--- учение о человеке (антропос).
Логика~--- о формах и видах знания
Социальная философия. Но этого всего еще нет. 

Имел прозвище Тёмный. Ибо его стиль. Он пытался подражать оракулу, формулировал мысли не ясно, загадочно. 
Ему якобы было доступно больше, чем остальным, он и рассказывал тайные знания. 
Дело философа~--- постижение мудрости. Высказывал отношении к остальным : многознание не ведет к мудрости. Как её постичь? 
Не так, как милетцы, которые слушали и наблюдали, основываясь на своих чувствах. Считал, что мудрость основана на другом, что чувственное знание не истинно, не может являться основой. 
У нас же есть внутреннее зрение! Умозрение! Внутренние чувства~--- самостоятельный независимый источник знаний, что и ляжет в основу западной философии. Настаивал, что это и есть единственный источник знаний. 
Главное содержание мудрости~--- логос. Гераклит его не придумал, но придал ему особое значение. Слово, несущее в себе высший смысл, слово, переходящее в дело. См. также Евангелие №3. 
Общество должно соответствовать логосу. Логос состоит в борьбе, и единстве, и тождестве противоположных начал. Оттуда происходит гармония, война и борьба с конфликтом~--- отец всему. 
Пример лука (оружие) и лиры (музинструмент). Второй момент~--- текучесть вещей."Нельзя дважды войти в одну и ту же реку, ибо...". Софисты потом его потроллили: даже один раз не удастся, ведь река меняется по мере вхождения. 

Диалектика. 
Начало~--- огонь. Причем принцип трансформации~--- неизменен. Пути вверх и вниз~--- тождественны (огонь-воздух-вода-земля~--- в две стороны). Начинает рисовать. Огонь в центре, а всё вокруг него вращается. Вот вам и гелиоцентрическая модель космоса. 

Выбор гелиоцентрической модели был лишь из-за социума. Ученые всё могли понять, это не сильно сложнее, но это было уже не важно. 
