Лейбниц. Годфрид Вильгельм Лейбниц. 1646-1716. Рационалист
Многое~--- лишь тенденция. Далеко не все признаки обязаны присутствовать. Так же и у Лейбница. Ведь он видел и слабости этого течения, а потому учел их. Нельзя считать его чистым рационалистом: воспринял черты эмпиризма. Создал синтетическую, компромиссную концепцию. Был, как и Локк, весьма дипломатичен. Есть мнение, что это элемент эклектизма.

Немец, родился в семье профессора лейпцигского универа, в 20 лет его окончил, затем служил при различных дворах и курфюрствах. Ведь Германия была раздроблена на множество мелких княжеств, не то что Англия и Франция. Этот факт сильно повлиял на немецкий менталитет и философию. О единстве мечтали множество философов и пытались его хотя бы представить. Вообще философы составляли славу Германии на фоне печальной социально-экономической жизни. Лейбниц от этих особ получал поручения: дипломатия, генеалогия (можно поездить по Европе ... PROFIT!!!), знакомится с Гюйгенсом, Герма, Ньютоном (с ним сперва сотрудничал, но потом соперничал за первенство в открытии дифференциального исчисления), Бойлем, Спинозой... Устанавливал личные и научные связи. Переписывался практически со всеми. Его наследие в философии, в основном, письма. У Лейбница всего 3 законченных работы.
\begin{enumerate}
	\item Новые опыты о человеческом разуме. Новые~--- в сравнении с Локком.
	\item Монадология.
	\item Теодицея. Но там скорее учение о иерархии, а не оправдание зла.
\end{enumerate}
	
Переписывался и встречался с Петром Первым в Голландии. Помог устроить проект РАН. Вообще жизнь, конечно, интересная, но поскольку и короли бывают гады, то и унижений было много. Лейбниц был дипломатичен и не сдавался.
Научные интересы Лейбница были очень широки: тяжело назвать, чем он не занимался. Даже языкознанием. Его считают сильным логиком, он обновил её со времен Аристотеля. Его научные интересы переплетались с философскими. Дифференциальное исчисление~--- способ решения проблемы устройства мира. (Это не он ли обижался, когда его называли математиком?)
Мир делится на 2 части: 
\begin{enumerate}
	\item Ноуменальный. Умопостигаемый. Сущностный, реальный, настоящий, действительный, истинный мир.
	\item Феноменальный. Мир явлений, физический (wtf).
\end{enumerate}

Но они как бы вместе. Лейбниц стремился к выработке метода. Правила метода познания (и научного и философского) соответствуют принципам его философской системы.

Лейбниц идет от логики к онтологии~--- учению о бытия. Принципы:
\begin{enumerate}
	\item Принцип всеобщих различий. 
	\item Тождество неразличимых. Полагать 2 вещи неразличимыми~--- полагать одну и ту же вещь под двумя именами. Копий нет. События на мировой линии различны. 
	\item Принцип непрерывности. Это и есть философское выражение математического дифференциала. На мировой линии событий нет пропусков. Отчетливо различные явления опосредуются переходными малозаметными событиями, нет пустот, всё перетекает, бесконечное мн-во. Нет рождения и смерти, есть развертывание и свертывание. 
	\item Принцип монадности. Точки получают названия. Линия состоит из монад. Монада~--- простая субстанция. У Лейбница речь идет о множестве монад~--- это множество субстанций~--- их бесконечно. Они самодостаточны, при этом взаимодействуют с другими за счет того, что они~--- зеркало вселенной. Монады~--- духовные существа, психически активны, одухотворены. У них есть внутренние способности, которые они реализуют (взаимодействуя тем самым со всем остальным). Не одна способность, а целый букет, растущий из духовной активности монады. Активность можно выражаться в 
	\begin{itemize}
		\item Аппетиция (стремления)
		\item Перцепция (ощущения)
		\item Апперцепция ( =(само)сознание)~--- высшая степень.
	\end{itemize}
	Причем эта активность может перетекать из одной формы в другую, непрерывно. 
	Всё существует в процессе. 
	Модель~--- мировая линия (см. рисунок). Аналогия с математической прямой~--- не полная. Но в качестве схемы~--- студентам~--- сойдет. Как соотносится дискретность с континуальностью. Все события присутствуют на мировой линии. Мировая линия состоит из метафизических точек (не математических), этот мир еще более умопостигаем.
	Как понимали силу:
	\begin{itemize}
		\item $mv=\text{const}$~--- Декарт
		\item $\frac{mv^2}2=\text{const}$~--- Лейбниц
	\end{itemize}
	Существует различные формы монад. Эдакая пирамида. Наверху~--- монада монад. Ну конечно же это бог, это чистая перцепция.
	\item Принцип предустановленной гармонии. Все монады согласованы в каждый отдельно взятый момент времени. Согласие души и тела, например. Решение психофизиологической проблемы (у Декарта~--- окказионализм, у Спинозы~--- параллелизм). Бог, создавая мир, выбрал самую лучшую модель. Оптимизм~--- лучше и быть не может, мир уже гармоничен. Натянуто, зато сильным мира сего нравилось. Многие иронизировали. Волтер <<Кандит или оптимизм>>~--- саркастическая история о провинциальном сынке помещика, втянутом в войну, плененным, захваченным разбойниками ... попавшим в рабство на корабль, кораблекрушения, спасшимся, попавшим в Лиссабон на землетрясение, причем его всё время сопровождает учитель~--- Панглос~--- и говорит <<всё к лучшему в этом лучшем из миров>>... болезни, потери частей тела... возвращается к себе в сельскую местность, решает, что ничего не понял о ходе истории, и начинает растить свой сад. 
	\item[...] Есть абсолютное пространство и время. Сущности~--- монады. ... бла-бла про бла-бла ...
\end{enumerate}

Внес вклад в логику. Объединенный закон тождества (с законом противоречия и исключенного третьего). По Декарту истинно всё то, что ясно и отчетливо мыслится (очень слабо, если по честному). Лейбниц предложил добавить: идея истина, если можно перечислить все её характеристики. По правде говоря, не сильно лучше.

Как трактовал Лейбниц врожденные идеи. Как рационалист он признавал их существование, но не признавал их законченность и развернутость. Он считал, что они присутствуют в виде задатков и предрасположенностей, а для их активизации необходим опыт.

Цитата из Лейбница. О сравнении врожденных идей с белым мрамором с прожилками. Скульптор выбирает, что же можно из такого рисунка сделать, к чему лучше подойдет. Из полемики с Локком
<<...Можно ли отрицать, что в нашем духе имеется много врожденного... тысячи других предметов наших интеллектуальных идей.... не всегда осознаем их... если бы душа походила на чистую доску... но если бы в глыбе имелись прожилки... хотя потребовался бы труд...>>

Лейбниц то, без вопросов, молодец. Но школы не создал, не культивировал свою личность. Но его последователей возглавил Вольф. А философия Лейбница была не оформлена, поэтому Вольф с товарищами вырезал всё спорное, добавил что-то, поправил. Эта система устраивала всех, в т.ч. немецкие власти, и её стали преподавать во всех немецких вузах. И так в течение лет 25-30, минимум. Кант, например, преподавал Лейбницевскую философию.
 