Сегодня разберём две школы италийской школы. Южная Италия и Сицилия. 
В отличие от наивной и простой ионийской, эта философия является спекулятивной и темной. 

Пифагорейская школа. 
Первая школа. 
Пифагор (акме~--- окло 531 до н. э.) родился на острове Самос, что недалеко от Милета. Поэтому он был знаком с той философией. 
Его становление началось с путешествия на Восток, где узнал астрономию и математику. Укрываясь от Поликрата, ушел в Кратон. Говорил о своей божественности. 
Проповедь Пифагора имела огромный успех, имела последователей: философы, политики, врачи и даже женщины (из богатого сословия). Не занимал общественный постов, но был уважаем. 
Пифагоризм просуществовал до 1 века до новой эры. Порой целый набор школы был убит или погибал в огне. 
Положения пифагореизма

\begin{enumerate}
	\item Учение о бессмертии души. Но не просто мифологическое, а метенпсихоз. Прототеория анамзесиса~--- воспоминания души о прошлом. Эту идею потом будет развивать Платон. 
	Также идея катарсиса: для тела~--- вегетарианство и не есть бобы, для души~--- её развитие в плане познания (музыкально-числовая структура космоса, "да не войдет сюда не знающий геометрии (математики) ) 
	\item Учение делится на 2 части: акусмата и математа. Акус~--- что-то связанное со звуком, а не аксиоматика. Акусмата~--- то, что рассказал учитель, знание воспринимаемое на слух, доступное всем, даже неспециалистам. Математа~--- существовал запрет на разглашение, её нельзя было рассказывать людям, не прошедшим специальную процедуру. 
	Математус универсалез. Запрет на разглашение коснулся, в первую очередь, на несогласованность математики. 
	Например, несоизмеримость отрезков $1$ и $\sqrt{2}$ (античная математика знала лишь натуральные числа, не знала нуля, обозначала цифры буквами). 
	\item Первоначало~--- число."Число есть всё, и всё есть число." Это ново: число~--- идеальный объект, безразличен к предмету. 
	Числами можно описывать людей, гражданский закон, он универсальны. Весьма высокий уровень абстракции. Самые распространненые интерпретациии
\end{enumerate}

\begin{itemize}
	\item Указывает на наличие закономерностей в мире
	\item Геометрический принцип. Формообразующий принцип. Наш мир предметен, форму задать числом. 
	\item Квадратные и треугольные числа. $1+3=4=2^2$. $6=2+4=2\cdot3$. $1+2+\ldots+n=\frac{n(n+1)}2$. 
	\item Но и точка не материальна. Парадоксы путания физических и математических идей. 
	\item Физический принцп. Числа характеризуют стихии, из которых складываются все вещи. Огонь имеют форму тетраэдра, земля~--- куб, воздух~--- икосаэдр. Числа упорядочивают и физический мир
	\item Теологический принцип. Увлекались мистикой чисел. Например, 7. Или 1. Да и 3. См реальные объекты~--- количество ворот или число гласных, нот. Да там все числа божественные. 
\end{itemize}

4. К вопросу о подходе к космологии. Вначале определяют общий порядок мирового устройства~--- взаимоотношение протовоположных начал. См. зорроавстрийцы. Выжеляют 10 основных пар противоположностей. Набор забавный) 
Предел~--- беспределие
Нечетное~--- четное
Единство~--- множество
Правое~--- левое
Мужское~--- женское
Покой~--- движение
Прямое~--- кривое
Свет~--- тьма
Добро~--- зло
Квадрат~--- прямоугольник. 
Это и дало повод к развитию бинарной композиции. 
В центре мира огонь. Вокруг него вращаются 10 божественных тел: небо, 5 планет, солнце, земля, луна, противоземля~--- антихтон. Это не геоцентрическая система. Несочетание с другой системой~--- не проблема :-) 

Элейская школа
Основатель~--- Ксенофан. Решил изучать то, что не изменятся. Что это? Земля. В мире всё постоянно, едино и неподвижно. 
Парменид. 515 до н. э. --- неизветсно, но, видать, довольно долго. Полный антипод Гераклита (у него все течет), у того всё стоит. Все неподвижно. Говорит о бытие, о сущем. 
Пытался объять мыслью всё, помыслить как целостное. Вообще всё. Всё-всё-всё. Что было, есть и будет, что возможно. Поразился, какой мощный объект он обнаружил. Решил, что мысль человеческая сильнее природы. 
В таком случае нет ничего кроме бытия, ведь оно включает в себя всё, что существует."О природе". Его какая-то богиня вела по путям мнения или же истины (путь бытия). 
\begin{enumerate}
	\item Бытие можно только мыслить. Мышление конституирует бытие. Мыслить о ничто~--- не мыслить. 
	\item Бытие вневременно, неизменяемо и тождествено самому себе. 
	\item Непрерывно, неделимо, не состоит из частей. 
	\item Оно единственно. 
	\item Бытие неподвижно. Ему же некуда двигаться. 
	\item Бытие законеченно, совершенно, в нём все уже есть. 
\end{enumerate}

Вывод: бытие похоже на шар. 
Небытия нет в онтологическом смысле. Более того его нет в физическом смысле: нет пустоты, а значит и нет движение. Нельзя описать движение непротиворечиво, то есть это иллюзия. 
Бытие шире природы. Еоо не понюхаешь. 
Существование слово «небытия»~--- несовершенство языка. 

В чем смысл? Разработка логики. Разработан новый объект~--- мышление. Да, его вывод гипертрофирован. Необходимо его согласовать. 

Идя по пути мнения мы можем построить любую версию. 
Версия. Огонь и земля шсяк-шмяк и образовался шар. 

Зенон. Акме: 464-461 
Распространял чужие теории
Диалектика. Умение вести спор и находить ошибки у оппонента. Написал около 40 апорий. Дошло 9. Известны 4. 

\begin{itemize}
	\item Дихотомия
	\item Ихиллес и черепаха. 
	\item Летящая стрела
	\item Стадион: про 2 колесницы, проходящие разные расстояния. 
\end{itemize}

Всё дело в противоречивости движения) Ведь мышление и физика различны. Но мышление важнее, поэтому оно правильное. 
