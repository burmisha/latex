\lectiondate{7 сентября}
Новый завет~--- универсальная (мировая) религия.
В библии содержится 4 евангелия (было больше). Они считаются каноническими. Их выбор обусловлен политическими мотивами и борьбой. Толковать библию помогали священники, ибо можно было понять по-разному, да и читать было в таком случае не обязательно.

Влияние христианства отчетливо заметно во 2--14 веках (т.\,е. средних веках). 

Особенности мировоззрения в средние века, и как себя ощущал человек.

\begin{enumerate}
	\item Теоцентризм. Это главная черта. В античности был космоцентризм. А здесь же~--- акт божественного творения. Потому мир целесообразен (его создал бог), пропитан скрытым (сакральным) смыслом. Все события происходят не просто так, а это борьба добра и зла, цель~--- спасение душ лучших людей.
	\item Универсализм. Мир един и закончен. Идея обозримости мира для движимого верой разума. Всё уже есть, нет ничего нового. Всё известно, но неохватимо одним человеком. Цумма. Философия~--- служанка теологии. А зачем она нужна? Так служить же почетно, а быть рабом~--- здорово, получаешь от этого удовольствие.
	\item Символизм. Есть же сакральный смысл. Он то и несет дух подвижности~--- это смысловая динамика.
	\item Статичность мира. Отношения иерархии и вертикальных связей: у каждой вещи есть прототип и <<начальник>>.
	\item Персонализм. Пер сен уно~--- единый в самом себе. Личность~--- рациональная неделимая сущность. Душа вечна и не переселяется. Противоречие: человек подобен богу, но мир для человека. Все, что существует: неживое (камень), либо живое, но не имеет ощущений (растение), либо имеет, но не разумно (собака), либо разумно (человек). Да, человек выделен, но и он раб божий. Полное пренебрежение телом. Отделяют душу и тело. И так до самого Возрождения. Это вам е <<в здоровом теле здоровый дух>>, как в античности.
\end{enumerate}

Христианство~--- паторналистская культура.
Человек ничего не значит, его надо направлять.
Социализация через обряд крещения.
Главная ценность: вера (в христианство) и верность (своей соц группе). 
Плохо относились к оригинальности. Могли признать в ереси, отлучить от церкви: это потеря всех связей. Причем отлучение было страшнее костра.

Особенности средневековой философии.
\begin{enumerate}
	\item Ретроспективность (смотрит назад, но не вперед) 
	\item Традиционализм. Чем древнее, тем подлиннее, чем подлиннее, тем истиннее. Нужно выявлять из библии скрытое. Экзегеза~--- истолкование (экзегез~--- толкование, см. Exec~--- исполнять). 
	\item 3 науки: химия (в виде алхимии), астрономия (астрология), медицина (педицина? \texttt{o\_O}) 
\end{enumerate}
Все тексты~--- компиляции старого и плагиат.
Канонический текст полон загадок. Надо искать, но не переходить границы допустимого (эти границы зависели от правителя и эпохи). 
Степени экзегезы:

\begin{itemize}
	\item Семантический
	\item Еще какой-то
	\item Концептуальный анализ (комментарии) 
	\item Системотворческая~--- поиск философской концепции.
\end{itemize}
Дидактизм, назидательство. Все философы~--- наставники и учителя. Зарождение университетов в современной форме.
Средневековая философия тесно связана с теологией. Но у неё есть своё узкое рациональное поле. Философия всегда помогает теологии. Помогает на поле способности человека сомневаться и призвана помочь преодолеть сомнения.
