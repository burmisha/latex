\lectiondate{28 сентября}
Разные точки зрения: выделять ли средневековую философию, или же это лишь переход к философии нового времени. Ибо там нет ни особой оригинальности идей, она эклектична, мало знаний в сравнении с античностью.

Формирование философии нового времени.
Бертран Рассел (англичанин, 20 век, создатель аналитической философии, автор <<История западной философии>>~--- это не слишком классный учебник для физтехов~--- это скорее его личное отношение) ~--- его мнение.
Выявил 2 тенденции 14--17 веков: падение авторитета церкви и рост авторитета науки. Авторитет церкви был непререкаем в течение многих веков. 

\begin{enumerate}
	\item Отсюда вытекает изменение духовного климата. Преобладание светских элементов над церковными: это лишь тенденция, присутствовало не везде и не сразу. 
	\item Речь идет о новых экономических силах и стимулах. Вместо земельной аристократии выходит торгово-промышленный слой держателей капитала.
	\item Изменения в политике: формирование наций и национальных государств
	\item Формирование либеральных (либеро~--- свобода) торговцев, не задавленных социальной иерархией. Свобода экономической свободы предполагает веротерпимости, законодательно закрепленное уважение прав граждан и равенство их перед законом. Раньше об этом и речи не было. Законодательное закрепление прав граждан было завоевано в тяжелой борьбе~--- реакции на религиозные войны.
\end{enumerate}
Всё это нашло отражение в философских концепциях. Отвержение церковного авторитета началось раньше утверждения авторитета науки. В первую очередь, вольнодумные идеи, направленные против церковного засилия и ханжества. Прежде всего тут Итальянское возрождение (прочитать текст), хотя науки то еще и не было нигде. Оппозиция церкви в итальянском возрождении отвергала церковь в пользу идеалов античных образцов. Но это же далекое прошлое. Первое научная нападка~--- книга Коперника в 1543 году. Но его идеи не сразу получили широкое распространение и признание, была усовершенствована в 16 века Галилеем и Ньютоном. 
На чем же держится авторитет науки (тогда, а не сейчас~--- сейчас особо вопросов нет, её положение изменилось) в конце 16~--- 17 веков, отличие от церковного.

\begin{enumerate}
	\item Признание истин науки~--- результат свободной воли человека, не было никакой кары
	\item Открыто заявляется об неполноте, неточности и открытости научных знаний. Можно корректировать с развитием, знания частичны.
	\item Критицизм научного знания. Критика и самокритика играют решающую роль в развитии.
	\item Практическая направленность. Наука доказывает свою ценность и свою правильность доводя теоретические разработки до практических изобретений и механизмов, внедряя их в жизнь. Наука существует не сама для себя. В ней не возможны дискуссии схоластического смысла, без прикладного значения. Именно благодаря этому, в итоге, и был признан авторитет науки. (В первую очередь, военное дело) 
\end{enumerate}
При всем при этом, нужно оценивать процесс осторожно, говоря не только о положительных моментах. Проблема: с падением авторитета церкви в первые века~--- рост индивидуализма (вплоть до анархизма), ослабление нравственности, изменения в морально-этической сфере. Значительный элемент безнравственности в т.ч. в политике. Дом Медичи~--- яркий пример. Николо Макиавелли.
Честность ведения бизнеса, капитализм (Маркс считал, что люди там не при чем, а лишь как винтики). Макс Вебер (конец 19~--- начало 20) ~--- оппонент Маркса, считал, что протестантская этика сыграла важную роль в строении капитализма (книга <<Протестантская этика и дух капитализма>>). 

Появляются новые проблемы, связанные с тем, кому принадлежит власть, как она распределяется, как формируется государство. Взаимоотношения власти и народа, кто суверенен, носит верховное право, имеет ли право народ свергать власть. Капиталистической производство связано с объединением людей (чем дальше~--- тем больше людей), как их организовать, как управлять. Большой массив социально-философских построений.

Эпоха возрождения (14 век~--- конец 16). Чем севернее~--- тем позднее началась (началось с Италии и других южных стран). Позже всего~--- на Британских островах.

Френсис Б'экон. 1561--1626
Идеолог нового времени. Но особо вклада в научные исследования не внес. Люди были полны надеждами на перемены, предреволюционная эпоха, вера в позитивные изменения. Происходил из богатой семьи, вращался в высших кругах английского общества (был лордом-канцлером при воре кого-там 1618--1621 годах, но его обвинили во взяточничестве и отстранили от должности, через год принесли извинения, предложили вернуться, но он, якобы, обиделся. Поэтому у него и появилось время на писательскую деятельность). Насчет взяточничества~--- не ясно, было ли оно, но человеком он был очень богатым, в отличие от подавляющего большинства философов. Идеи странно перекликаются с идеями Роджера Бэкона :-) 

Философия нацелена на установление царства человека, на поддержку реформ, стимулирующих экономический прогресс. Сделал ставку на естествознание, на науки, изучающие природу (вслед за античными науками). Экономический прогресс основан на техническом прогрессе. Ценность естествознание~--- его продолжение в технических изобретениях и усовершенствованиях. Сочинение: <<Новая Антлантида>>~--- Атлантида, ибо сознание ассоциировало его с высокотехнологичным и высокоразвитым обществом. Это сочинение~--- утопия (жанр введен Томасом Мором~--- см текст.) (топос~--- место, у~--- отрицание, т.\,е. место которого нет). Сам Томас Мор был папистом и католиком, возник идеологический конфликт с Генрихом 8-м (он~--- король~--- хотел развод с женой, но не давали~--- появилось англиканство), всё кончилось плохо~--- обезглавили за госизмену.

Новая Атлантида. Про остров Бенсалем. Плыли люди, потерялись, приплыли на Бенсалем, удивились, как там всё хорошо. Обществом правят самые умные, а не короли по крови. Дом Соломона (см. аллюзию на израильского царя) ~--- прообраз академии наук, куда попадают за заслуги, знания, разработки. Всё делают умные машины, мало физического труда. Бэкон делает вывод, что стоит набираться знаний и решать проблемы от эпохи феодализма. 

Бэкон~--- <<пионер>> индустриальной эры. Он верил, что это улучшит жизнь людей. Но мы то знаем, что это не так: не всем стало хорошо.
Говорил, что философия должна иметь практическую направленность. Критиковал схоластику за отсутствие пользы народу. Новая философия будет пронизана научным духом и будет идти с наукой, а не с религией и теологией. Но он и не примитивный утилитаризм (установка на получение быстрого полезного практического результата), называл научные исследования опытами. Делил опыты на плодоносные (по-настоящему полезные) и светоносные (не дают немедленного практического рез-то, но нацелены на исследование глубинных свойств природы, материи вещей, изучают <<формы вещей>>, важные для дальнейших плодоносных опытов). Тогда это звучало в первый раз. <<Знание~--- сила>>~--- это его девиз. 

Для воплощения в жизнь выдвигает научный проект <<Великое восстановление наук>>. Говорит, что нельзя просто восстанавливать античную науку, надо двигаться дальше. Средневековье еще сузило античность и создало застой в механических искусствах. Теперь же стали известны отдаленные уголки старого света, мир расширился, открыт новый свет, изобретен порох, книгопечатание, компас\ldots <<до бесконечности разрослась груда опытов>>.

Проект состоит из 2 ступеней-этапов
\begin{enumerate}
	\item Нужно упорядочить и классифицировать. Создание новой классификации наук. (Аристотель в своё время тоже занимался составлением критерия). Как бы не переписывать критерии при каждом новом открытии, хоть наше знание и постоянно меняется. Кладет в основу классификации человека~--- субъект. Его способность познавать. Философы думали, что мир вокруг меняется, а человек нет. Какая же способность задействована. 3 способности память, воображение, рассудок. Слово наука при этом понимается в широком, античном смысле слова. (Не обязательно соблюдение всяких процедур, а всё, к чему человек приложил руку.) Не стоит постоянно менять положение науки.
	\item Методы. Нужны универсальные инструменты. Новый метод, а не схоластический. Схоластические приёмы при исследовании природы не работают, не дают приращения содержания. Схоластический метод~--- ссилогизм. Нужен научный метод: эмпирический. Индукция~--- на ней Бэкон делает акцент, но не простая, а истинная. Эмпирический~--- значит (эмпирия~--- чувственный опыт, не духовный). Положено начало европейскому эмпиризму (от Декарта~--- противоположное~--- рационализм, но не противопоставление. Разница в виде опыта на первом месте: интуитивный~--- математический или же чувственный, оба используют рациональные суждения). 
\end{enumerate}

Точка зрения на истинную индукцию

\begin{enumerate}
	\item Плоский эмпиризм: Путь муравья. Идем и всё полезное собираем. Но тогда есть шанс пропустить более полезное. Результаты случайны.
	\item Путь паука. Паутина вокруг себя и в неё что-то попадает. Причем паутину производит сам. Это путь схоластики: попытка хоть что-то поймать. Тут еще больше элемент случайности: не ясно, что залетит
	\item Правильный путь: путь пчелы. Сочетает положительные стороны, не имеет отрицательных. Много трудится, перелетая со цветка на цветок и собирая нектар, как муравей, а потом перерабатывает его в мёд науки, сама. 
\end{enumerate}
Истинная индукция:

\begin{itemize}
	\item исправление недостатков разума. Поспешность (перескок через ступени в познании, имея недостаточное количество знаний). Тут играет роль короткость человеческой жизни. Наука~--- не дело одного человека, нельзя полагаться на случай и ограниченность. Необходимо непрерывное и постоянное обобщение. От частных фактов к средним аксиомам, от средних~--- к генеральным.
	\item Исправление недостатков чувств. Нужно учитывать не только подтверждающие факты, но и противоречащие. Мешает и слепая вера в разум. Идея эксперимента подчеркивала Бэконом. Не просто наблюдать, но и активно вмешиваться, если мы хотим что-то ценное, то важно организовать наблюдение и эксперимента. Нужен объективный результат, а не прислушивание к отдельным людям.
\end{itemize}
Нужно составить список случаев присутствия опыта (когда наблюдали) и отсутствия (когда предполагаем, что могло бы быть, а его нет), и список степеней проявления. Не меньше трёх списков, чтобы судить о явлении. Когда люди работают в науке~--- они не индивидуумы, личные особенности нежелательны.

Учение Бэкона об идолах. Есть много сложностей для человека, становящегося на путь познания. Прежде всего надо знать о препятствиях к постижению истины: идолы и призраки. Всего их 4 рода.

\begin{enumerate}
	\item Призрак (идол) рода (рода человеческого). Есть объективные особенности устройства человека. Поспешность (стремление сделать вывод поскорей) ~--- нужна преемственность и т.д. А еще чувственный аппарат: видеть в темноте, звуки, запахи\ldots Все эти моменты нужно скорректировать. Например, заюзать приборы.
	\item Идол пещеры. У каждого человека есть своя пещера, где он живет. <<Мы все попадаем в мир, которой не выбираем>>. Учет особенностей нравов: кому-то особенности, кто-то обобщает. Но цель должны быть общая
	\item Идол рынка или площади. В центре средневекового города всегда была площадь~--- квинтэссенция человеческого общения. Все знания реализуются в языке. + Базаааар!!11 Бывают потасовки. В одно слово можно вложить разный смысл. Естественный язык очень сомнителен для передачи научных знаний. Язык может и вводить в заблуждение. Естественный язык для науки непригоден из-за многозначности. Это надо учитывать. Значение терминов должно быть строго определено и принято всем причастными людьми.
	\item Призрак театра. Вера в авторитет. У нас полная иллюзия погружения в реальность, если актеры хорошо играют. Но это не так (см. фильм Начало). Беспрекословное подчинение авторитету вредит науке, хотя совсем без них нельзя.
\end{enumerate}