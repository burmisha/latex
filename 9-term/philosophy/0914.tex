\lectiondate{14 сентября}
Средние века.
Они связаны с наступлением новой эпохи доминирования христианской религии.

Канон принят в 325 году на никейском соборе. Сочинение с отчетливым подходом к христианству~--- второй век.

Было много язычников. Первые христианские мыслители~--- апологеты (защитники). Ранний период~--- апологетика. Надо было концентрировать внимание на том, в чем христианство превосходит.

Первый период. Патристика. (Патр~--- отец). Многие христианские мыслители отрицательно относились к обширному наследию античных философов. Многие считали, что стоит апеллировать лишь к библии (ранее~--- на разум, высшую и лучшую часть души).

Тертуллиан. Считал, что христианская философия опирается лишь на библию. Верую, ибо это абсурдно. Противопоставлял веру и разум. Разум ничего не стоит.
Это была не доминирующая точка зрения в патристике. Более общая точка зрения~--- можно примирить непротиворечащую христианству часть античной культуры. Так считал и Клемент Александрийский.
Настало время определиться~--- созыв никейского собора, где решалась догматика. Отношение к арианству (течение, возглавленное монахом Арием. Считал, что сын не подобен отцу, т.е. и человек богу не подобен~--- они различны). На соборе арианство осуждено, принята божественность всех трех ипостасей (отец, сын, дух). Уклон в сторону веры, а не разума. Рационально объяснить невозможно, поэтому появился большой простор для философии.

Августин Блаженный~--- главный в патристике. Узнать годы жизни.
Обрел покой души, приняв христианство. Родители его были различного вероисповедания. Написал исповедь.
Кто-то. Кто?!
Много чем занимался. Сторонник неоплатонизма, преподавал риторику, пришел к неоскептицизму. Увлечение манихейством. В 387 принимает христианство. <<О граде божием>>. Был очень образован и использовал все свои знания светской философии для насыщения их христианским содержанием. Главное его учение: учение о боге. Бог~--- главный нематериальный бесконечный абсолют. Но при этом бог~--- личность. В неоплатонизме была идея иманации. Она отвергается, а взамен~--- идея о креационизме (творение из ничего, не из идей). Богу виднее, каким должен быть мир, человек может и не понимать. Агностизм~--- непознаваемость мира. Мир полон чудес, бог наполнил их ими по своей воле, у него есть свой план~--- это провиденционализм, но понять до конца весь процесс и план человек не может. Бог творит мир в соответствии со своим великим замыслом (он всемогущ и всеблаг). Теодицея~--- оправдание бога за то, что допускает в мире наличие зла. Как же он так мог? Зло не есть нечто абсолютное, а лишь недостаток добра, и человек лишь демонстрирует свою волю. Нет абсолютного зла, а лишь есть абсолютное добро. Зло временно и служит назиданию человека делать добро. Грешники нужны, ибо иначе не было бы праведников.

Волюнтаризм. Руководство не является абсолютным. Человек наделен волей, есть пространство свободы, для испытания человека этой свободой. Не стоит винить бога в человеческих поступках. Свобода воли человека реализуется на фоне общего предопределения. Но! Человек от рождения грешен (его предки грешники, а он подобен). Разум не очень может помочь человеку, а лишь должен призывать делать добро. В человеке способность совершать поступки важнее способности думать. Превосходство (активная способность) волевой стороны человека над рациональной (пассивная). Ведь делать придется, даже если не познал.

Социально-историческая концепцию. <<О граде божьем>>. Важность церкви как социального института, оправдывает широкое присутствие церкви во всех делах. Эсхатологизм (эсхато~--- конец). Учение о конечных судьбах человека и человечества. История имеет начало и конец, страшный суд, после чего окончится человеческая история. Рассуждал о судьбе всего человечества. (Хотя только о средиземноморье, по сути). Смысл истории символичен. Смысл~--- это борьба града земного и града высшего небесного, где живут праведники. Критикует все виды светских государств, как порочные. Каин~--- первый основатель светского гос-ва. Разделяет историю на 6 эпох (см. 6 дней). Единственно богоугодное гос-во~--- теократическое. Лучше, если церковь по возможности сильно влияет на политику.

Августин был канонизирован в католичестве.

Заканчиваем первый период. Переходим к переходному периоду. 7-10 века.

Мало идей в Европе. Тяжелая жизнь. Вливание большого числа нецивилизованных народов. Этим занималась церковь, взяла на себя образование. 7 свободных искусств. Тривиум (грамматика, риторика, диалектика (логика и философия)~--- платоновская школа) и квадрилиум (арифметика, геометрия, астрономия, музыка~--- пифагорейская школа).

Предистинация. По трудам твоим тебе воздастся. Но церковь её не очень рекламирует, особенно на востоке. Лишь в протестантизме~--- вновь появилась.

В 7-10 века никто даже с Августином не может сравниться, хотя и тот не слишком крут был в плане значимости для философии.
Лишь Иоанн Скотт Эриугена (810-887) (ученый-монах, ирландец). Британия не так в то время пострадала. Было много книг. Французский король пригласил Эриугена, а то им скучно было. Ему было поручено перевести 4 произведения Псевдо-Дионисия Ариопагита. Они были о возможности богопознания: как человеку использовать разум, не нарушая канонов. А думать можно было обо всем чем угодно~--- о боге)
Два пути богопознания.

Катафотическая теология (+, положительная) и Апофотаческая теология (-, отрицательная)
Положительная теология~--- поиск аналогий между богом (хотим познать) и миром (можем ощущать). Путь начинается с приписывания богу различных св-в сотворенных им вещей, а потом бесконечно их расширяем. (ХXX хорош => бог бесконечно хорош). Экстраполировать можно лишь положительные качества. А что же делать с остальными? Может мы что-то упустили?

Апофатическая. Последовательное отрицание всех свойств реального мира в боге. Тогда бог непознаваем. Церковь признала именно этот путь. Пресекали прямо с головой, порой. Тут следовало бы и отрицать подобие бога и человека (т.е. есть лишь в одну сторону)

За вольнодумство его сослали обратно в Британию.

Для заполнения еще пара имен...

10 век. Схоластика.
Жизнь в Европе наладилась. Города становятся центрами формирования нового типа цивилизации. Новые формы организации людей и получения знаний. Создание университетов и возникновение корпораций ученых с привилегиями и своим уставом.

Что должно доминировать: теология и диалектика.
Доминиант: философия~--- служанка теологии.
Ансельм Кентерберийский: вера опережает знание (жил в 1033-1109). Автор онтологического бытия бога~--- первое. Оно идет от размышлений о разумности бога и мира. Человек знает сущность бога, а оно самое совершенное существо $\Rightarrow$ он и есть все, включает в себя
