В начале~--- тест. 

С прошлого раза остался кусочек софистов. 
Горгий 
Написал книгу о природе или о несуществующем. Использовал ту же форму, что и Зенон в своих апориях. Работы, конечно не дошли. 
Тезисы

	1. Ничего не существует. Доказывает, рассматривая противоречивость утверждений и бытии, небытии и о них вместе. 
	В самом деле, если несущее существует, то оно существует, противоречие. Если сущее есть, то нет несущего, поэтому нет. Ну и всякий такой бред. 
	2. Если нечто и существует, то оно непостижимо~--- его нельзя мыслить непротиворечиво. 
	3. Даже если сущее и постигается, то оно неизъяснимо другому. Ибо различное понимание людьми слов и даже одним человеком в разным обстоятельствах. 
	Язык~--- изобретенное средство, а не естественное средство описания. Слова "лгут", не являясь зеркалом реальности. ~дураку закон не писан, а если и писан, то не читан, а если и читан то непонят, а если и понят, то неверно. 

У Горгия любое утверждение ложно. 

Софистика дискредитировала философию, не предоставляя ничего взамен. ~--- да неужто?

Школа скептиков. Книги в духе " против поэтов", "против философов", "против физиков". 

Софисты. 
Связь процессов в природе и обществе. Социальная жизнь не является продолжением процессов природы. Законы природы неизменны, а человекьорганизует жизнь по своему усмотрению. 
Идея естественного права~--- равноправия людей. Антифонт, Фрозимах~--- договорная теория образования государства, т. е. гос-во создается не богами, а людьми, а значит люди могут и ошибаться или лукавить. Например, законы могут содержать ошибки. 
Гиппий противопоставляет законы общества и законы природы. Антифонт ставит вопрос о происхождении рабства. 
Он считает, что это противоречит природе. Говорит что варвары и эллины равны. 
Отпустил своих рабов, вступил в брак со своей бывшей рабыней => был объявлен сумасшедшим и лишен гражданства. 

После поражения в войне Афины вступили в мрачную полосу. Крах Афинской демократии, не представительной, а личной. Идеи софистов о рабстве шли в разрез с новым курсом Александра Македонского, который реализовывал свои крупные проекты. 

Аристотель защищал рабство и очень критически отзывался. Его уживляло, почему это побежденный не должен быть рабом. 

Сократ
469-399 до н. э. 
Известен не сочиненями, а собственной жизнью. Обладал известной харизмой. Он ничего не писал и это было его принципиальной позицией. Два источника: 

	1. воспоминания Ксенофонта (не Ксенофана!) , но он считается человеком не очень глубоким~--- описал лишь внешнюю сторону жизни Сократа, 
	2. диалоги Платона~--- ученика Сократа, вначале тошькоти пересказывал его

Отец~--- каменотес, мать~--- акушерка (повевальная бабка) , из Афин, из бедной семьи. Обладал по греческим меркам уродливой внешностью, отвислый живот, непропорционален, большая голова. 
Говорил, что помогает рождению мысли подобно акушеру. Продолжал линию софистики. Отговаривал изучать астрономию и природу в целом~--- на них все равно повлиять нельзя. А вот изучение людей, общества, нравов~--- стоит. 
Жил разнообразно. Занимался софистикой, хожил в военные попходы. Потом начал ходить по улицам Афин и беседовать со всеми. И так каждый день. 
Сложился постоянный круг слушателей, молодых юношей, в т. ч. элиты. Это насторожило, стали смотреть, чему он учит. 
Он призывал людей к самостоятельному мышлению, к разуму (он то и был тем самым единственным богом) . 
Двух посторонних людей подговорили обвинить его в богохульстве и развращении юношей (бисексуальность была нормой) , умственном, настрое против обычаев. 
Состоялся суд, осужден на смерть. Должен был выпить чашу с ядом. Ему предложили побег, но решил стать вечным укором своим современником. Потом то граждане одумались
