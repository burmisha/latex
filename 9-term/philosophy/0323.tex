\lectiondate{23 марта}
Причина проблем в Гегелевской школе --- политизация: разделения на левы и правых, которые стали вести полемика ну грани фола, что вызывала недоверие к гегельянству. 

Более известны левые гегелианцы («младогегелианцы»).
\begin{itemize}
	\item Маркс 
	\item Штирнер, 
	\item Ругге
	\item Энгельс
	\item Фейербах
\end{itemize}

Расхождение между правыми и левыми происходили по 2 вопросам. Одни трактовали смысл Гегеля в консервативном ключе. А левые рассматривали её как философию революции, призывающую к изменению общества. 
\begin{enumerate}
	\item трактовка «Все действительно разумно, а всё разумное действительно»
	\item соотношение религии и философии: правые считали, что религия должна быть сохранена, а левые --- что снята (есть же философия).
\end{enumerate}

Экзистенциально-антропологические направления.
Классическая философия всегда интересовалась сущностью: чем-то незыблемым.

Людвиг Фейербах. 1804--1872. Родился в Бававрии в семье видного юриста. Учился в Берлинском университете, где слушал Гегеля. Поначалу попал по его влияние, от которого так и не избавился, хотя очень старался. 
Провинциален. Самая известная работа: «Сущность христианства».
Главная цель жизни: борьба против религии.
Говорит, что церковники обманывают доверчивых людей. Но причина религии не в этом, а в самой сути человека: его противостоянии с природой и социалными институтами. Бессилие человека ищет выход и утешение в фантазиях. Для психологической компенсации придуманы Боги. Отгошение медлу человекм и придуманным им богом переворачиваются: неожиданно получается, что бог придумал человека и создал его. 
фейербах предлагает учредить культ человека. Делает это с таким пафосом, что поклонение человеку приобретает религиозную форму. Просто исправлен объект поклонения. Обоготворим человека и ему будем поклоняться. Отличный девиз: «Человек человеку бог». В теории познания Фейрбах продолжал линию сенсуализма, подчеркивая решающую роль опыта. Фейербаховская философия~--- антропологический материализм. Всё это обязательно привязано к человеку. Циатата: «Человек --- единственный универсальный и высший предмет философии». Человек --- это психофизиологическое существо. Система общественных отношений является «Я не может быть счастливым без Ты. Счастье для «я» недостижимо без счастья для «ты»». И вот имеем вопрос: как же жить человеку в безрелигоизной цивилизации, как обходиться без понятия «бог», откуда а брать силы и энергию, на чем оттачивать волю, каковы меры допустимого. Этот вопрос у Фейербаха представлен весьма поверхностно, но в 20-м веке, он рассмотрен подробно.