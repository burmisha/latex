\lectiondate{13 апреля}
% Есть рисунок

Часть материала изучить самостоятельно: чтобы закончить всю античность. 

В прошлый раз говорили о Платоне. В центре его учения стоит мир идей. Рассуждая как философ, ищет в мире нечто вечное, устойчивое, неизменное во времени. 
Идеи вечны, вещи нет, они изменяются. 
Идеи:

\begin{enumerate}
	\item Универсалии. 
	\item Идеи выступают в роли меры. 
	\item И в роли предельных понятий для чувственных вещей. 
	\item Идея~--- идеал для чувственных вещей. Вещь настолько являются самой собой, насколько близка к идеалу. 
	\item Идея~--- причина вещи
	\item Идея~--- цель вещи. Это очень важный аспект платоновской системы, ведь она телеологическая. 
	\item Принцип вещи
	\item Сущность вещи, определяет её действительность. 
	\item Воплощает бытие вещи. 
	\item Идея~--- форма вещи. 
\end{enumerate}

Кажущееся бытие~--- то что между материей и бытием (мир идей). 
Эманация~--- переход от бытия ступени более совершенным к менее совершенным. 
Платоническая любовь утверждает, что стремление к более высокому. 
Материя пассивна, олицетворяет небытие. Её надо оформлять, внедряясь в неё. 
Где же искать идеи в вещах? Как их ухватить?
Платон этого не разъяснял, да и не собирался: понимай как хочешь, он же мистическо-поэтически всё излагал. Платон использует образы, создав миф о пещере, который должен был пояснить положение чувственных вещей, и идей, и место человека. 
Вещи~--- лишь тень идей. Миф изложен в диалоге «Государство». Сравнение с узниками, прикованными к пещере без возможности повернуться. Тем самым мы судим о мире лишь по теням. 
А повернуться узники к миру, входу из пещеры они не могут. Душа то может, но не пока она в теле. 
Идеи присутствуют в вещах, но дальнейших пояснений нет. За эту фразу ухватятся комментаторы Платона, а за них~--- средневековые схоласты. 

Как совместить неподвижность идей с изменчивостью вещей. Есть ли идея движения и изменений. Ведь она бы противоречила самой себе) Платон решает, что существует. 

Мир идей так же иерархичен. Высшая идея~--- благо, красота, добро~--- (позже) бог-демиург. Мир создан богом без совершенства, чтобы отличаться от самого бога. Оправдание бога (теодицея) ~--- зло возникает спонтанно, должно же быть что-то еще. 
Времени в мире идей нет. В первую очередь демиург творит мировую душу~--- посредника. Человеческая душа подобна ей. 
После мировой души он пытался вставить сферу чисел как переход от сверхчувственному к чувственному: это лишь дань увлечению пифагореизмом и не более того. 

Душа~--- чувственная часть (поесть поспать) и волевая. Как два кгни и возница, объединяющая их~--- разумная часть. Но это лишь условные части, душа неделима и бессмертна, скорее это стороны, способности. 

Главная цель познания~--- постижение идей, на это способен только разум. Разум постигает идеи прямым видением, синоптически, припоминанием. 
Очищение души от отвлекающих моментов. Знание доступно в любой момент~--- см. диалог Менон~--- бывший раб сам сформулировал теорему Пифагора. 
Высший род знаний~--- интуитивный. Ниже~--- рассудочный, где человек сравнивает идеи и чувственный мир. Третий род знаний~--- здравый смысл, т.\,е. лишь мнение, чувственное знание. 

Греки считали, что высшее назначение человека~--- служение другим или государство. Человек~--- существо общественное. 
Платона интересовало идея государства~--- идеальное государство. Его понимание государства~--- продолжение понимания человека и его устройства. Главное в человеке~--- душа, обладающая тремя свойствами. 
Идеальное гос-во учитывает структуру человеческой души и также состоит из трех социальных групп. Высшее~--- философы~--- правители~--- сюда же жрецы. Средний слой~--- стратеги или воины~--- аналог волевой способности, заботятся о порядке. 
Низший слой~--- производители~--- земледельцы и ремесленники, никаких рабов не упомянуто (формально они и не граждане). 
В идеальном государстве деятелям искусства нет места~--- они делают тени теней, вводя в заблуждение остальных. Остались лишь хоровое пение и ритуальные танцы. Касты. 
У верхних~--- нет частной собственности и семьи (как же они не вымерли~--- это ж каста). Средние~--- есть частная собственность, семья~--- общая, дети~--- государственные, часть будут завербованы в философы. 
Изучают физические упражнение, не различая м и ж. 
У низших~--- все по-людски. 

Общая добродетель для всех граждан~--- справедливость. 
Платон считал, что всё в государстве должно контролироваться. Даже численность (фиксированное число): отселять граждан. 

Формы правления + деление на правильные и неправильные. 

\begin{enumerate}
	\item Идеальное гос-во
	\item Аристократическая республика.
\end{enumerate}

Регрессивные (ниже~--- хуже) :

\begin{enumerate}
	\item Тимократия~--- военная хунта. Власть нескольких путем военным. Спарта. 
	\item Олигархия. Власть немногих на ростовщичестве и торговле. 
	\item Демократия. 
	\item Тирания. Направлена против аристократов. 
\end{enumerate}

Платон (широкоплечий) -прозвище, его звали Аристокл. 

Аристотель
384-322 гг до н. э. 
Стагиры, Македония~--- место рождения. Отец~--- Никомах, придворный лекарь царя, образован. 
Изучал естествознание. Стал учеником Платона, провел 20 лет в академии. Стал постепенно критиком платонизмом, уехал, путешествовал. Приглашен во двор царя Македонии и становится наставником сына Филиппа II~--- Александра. 
Потом он его всегда сопровождал. Где-то лет 30 воспитывал, после чего вернулся в Афины и основал школу~--- Ликей. 
Название~--- по храму Апполона ликейского. Более демократический способ образования, они прогуливались и обсуждали. 
Назывались~--- перипатетики, школа~--- перипатетическая. В 323\,г. Александр умирает, антимакедонская партия у власти, Платон~--- персона нон-грата, уезжает и умирает. 
 