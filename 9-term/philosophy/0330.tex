\lectiondate{30 марта}

Карл Маркс. 1818--1883. Родился в городе Тлир на границе с Бельгией в семье адвоката, достаточно обеспеченной, закончил лицей, поступил в Боннский универ, изучал юриспруденцию. 
Вёл веселую студжизнь, отец перевёл его в Берлинский, славившийся строгостью нравов. Хорошо, что он там оказался: смог приобщиться к новейшим веяниям немецкой философией. 
Сперва ходил на лекции а потом и подружился с Бруно Бауэром. Познакомился с Женей, но её родственники воспротивились. Жениться ему удалось лишь в 1843.
Избрал публицистическую карьеру, но вскоре журнал, где он работал запретили. Поехал в Париж, познакомился там с революционерами (в хорошем смысле) того времени. Познакомился с Энгельсом, который был состоятелен. 

В 1844 выпустил первое сочинение «Экономическо-философские рукописи».

1845 --- тезисы о Фейрбахе.

1847 --- что-то ещё, узнать, что.

1848 --- совместно с Энгельсом --- Манифест коммунистической партии.

1867 --- первый том Капитала. Второй и третий - посмертно.

Акции. Скидки! Только сегодня, без СМС.

Маркс считал себя учеником и последователем Гегеля. Использовал его идеологию. Маркс принципиально разносторонний (может всё-таки посюсторонний?). Это привлекает ему много поклонников. Главная тема его философии --- понятие практики, деятельности, производства, труда. В этом схож с Ницше: прежнюю философию нужно выкинуть, а философствовать иначе.

История --- процесс смены общественно-экономической формации, каждая новая совершеннее. В основе её лежат экономические процессы, несущие социальный и обезличенный характер. В этом смысле Маркс гегельянец. 

Маркс мечтал о счастье.
Будет коммунизм.

Философская антропология. (тема номер раз в изучении Маркса).

Отталкивался от 2 точек зрения.

Человеку, чтобы стать человеком нужно избавиться от социального отчуждения. Здесь заключены мысль о социальный роли человека. Эта мысль полностью выражена в тезисах о Фейербахе: там Маркс говорит о сущности человека как об «ансамбле общественных отношений.» Каких именно отношений? Человек выразил свои сущностные стороны в идее бога. Согласно Фейербаху сам могуч, божественен. 

Маркс Штернер (Штирне??). Автор книги «Собственность». Считал, что моральное отчуждение ... как и социальное.  Предлагал избавиться не только от религии и морали.

Марс отталкивался от обоих суждений :-)
Ответ о сущности человека стоит искать не в религии и не в морали. В человеке всё человечно, ибо он существует в связи с другими людьми. Его поведения и мышление определяются историческими обычаями и стереотипами, формируется окружением, меняется со временем.
Нет в его сущности чего-либо, сопровождающего человека на протяжении всего существования.

Много пропущено, спросить у Светы.

Гуманизм --- желание человеку помочь, избавить от отчуждения. Так считал Маркс.
Более всего человека тяготит что-то там.

Предлагает всё преобразовать: семьи, гос-ва. Отменить деньги, создать всечеловеческое братство. Но произойти это должно одновременно во всём мире. Радикалистам понравилось.

Философия истории.

...Много пропущено...

Деньги --- форма социального отчуждения.
Третья ступень -- присвоение человеком всех накопленных сил и потенций. Человек теперь господин всего. Нет подчинения деньгам, гос-ву, религии или чему-либо ещё.

У Маркса есть незаконченность. 

И, наконец, как же совершить все эти трансформации. Только социальная революция, только хардкор (но это ничем не подкреплено).