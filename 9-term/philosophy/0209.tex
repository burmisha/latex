\lectiondate{9 февраля}
Эмиризм Европейский. Нас интересуют Беркли (называл себя ирландцем, ибо там прожил большую часть жизни) и Юнг (шотландец).
Очень интересен вопрос субстанции. 2 основные: материальная и духовная (мыслящая). Это сказал Декарт, а потом каждый в той или иной степени высказался. 

Рационалисты же~--- любители крупных объектов и им очень нравилось это понятие, ни от чего не зависящее. Эмпиристы делали ставку на чувственный опыт, на то, с чем мы встречаемся ежедневно и непосредственно ощущаем (а субстанцию мы не ощущаем, лишь предметы), поэтому «субстанцию» они либо не использовали, либо высказывались осторожно. 
Локк сомневался, что можно характеризовать чувственную субстанцию, ибо часть вещей имеют диспозициональные свойства, которыеы не принадлежат целиком предмету, а определяются субъектом. 
Локк, рассуждая о личности, так и не пришел к пониманию, что же лежит в основе психических (духовных) актов. Ну это всё можно не записывать. Полной критике понятие «субстанция» подвергли Беркли и Юнг.

Оба они работали после революции, и можно было порассуждать, ибо был некий откат назад, да и произошло примирение буржуазии и аристократии. Наступала эпоха некоего разочарования и реакции. Вообще британские философы очень чутко реагировали на духовный климат в своей стране (вспомнить Гоббса, который только это и делал). 

Джордж Беркли. Присутствует в схеме (9 отличий эмпиризма и рационализма). 1685--1753. Просходил из семьи среднего достатка. 
Детство провел в Ирландии и поступил в Дублинский университет, а после стал учителем теологии, пойдя не по философской, а теологической часть. Общался с клириками. 
Ему было поручено организовать англиканскую миссий в Новой Англии (севернее Нью-Йорка). Там то он и занимался миссионерством, прикупил себе земли. Но дела расстроились и он был отозван в Британии. Миссия была не очень успешной, но всё же он зато побывал в Америке. 
Забавно, что университет Беркли в Калифорнии имеет отношение именно к этому Беркли. Беркли-человек произносится на Баркли, а город~--- Бёркли. Универ довольно революционный, хиппи, захват администрации, понаехавшие. 

Беркли, вернувшись в Ирландию двинулся по религиозной стязи и впоследствии стал епископом. Bishop Barkely. (загуглить какое-то ещё название). Жил в достатке, было 6 детей.

Что касается философии. Первый предмет его интереса: вопрос чувственного опыта. «Опыт новой теории зрения» --- первая работа. Тбо зрение даёт больше всего представлений об образе мира. Следующий трактат: «О приницпах человеческого знания». «Три разговора между Гиласом и Филонусом» --- самая известная работа. 
Гилас (Гюле, Гиле... --- древнегреческое слово --- материя) --- материалист. Филонус (филио --- греческое --- люблю, нус --- греческое --- ум) --- тот, кто предпочитает следовать духу, идеалист. Спор идеалиста и материалиста. Симпатии Беркли на стороне Филонуса.
Более поздние работы Беркли критикуют современное естествознание и математику, возвышая религию и теологии.
Он во всем эмпирист, кроме признания бога и религиозных истин, что забавно. Критикуют материю и её признание (ибо это ведёт к атеизму). Имматериализм --- Беркли не сомневается в существовании материальных вещей, а кроме этого ничего в материальном мире не существует. Нет материальной субстанции, вместилища всего.
Беркли стремится показать, что первичные качества не имеют обоснования. Хотя и сам Локк сомневался в статусе вторичных качествах, но ему приписывают будто «он рассуждал о них, как о субъективных». Беркли же сказал что и масса, плотность и т.п. тоже субъективны. Мы бы никогда ничего не узнали о форме, если бы это не было получено из субъективного опыта, через вторичные качества. Например, треугольник как контраст светлого и тёмного на доске, т.е. воспринимаем первичные качества черех вторичные. Вывод очевиден.