Юм. Познанием научного типа далеко не всё можем назвать. 
Считает, что философия должна ориентироваться на критерии строгости и аргументированности, на которые опирается современная ему наука в духе Ньютона и всего такого. 
При этом всё завязано на ощущения. Жёсткая причинность, никакой свободы воли, всё предопределено устройством чувственного аппарата человека, ибо мы мир воспринимаем в комплексе: не отдельные сигналы, а всё и сразу, не разделяя.
«Предполагать, что будущее зависит от прошлого, заставляет нас лишь привичка.»