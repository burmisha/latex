\lectiondate{16 февраля}
Юм. Познанием научного типа далеко не всё можем назвать. 
Считает, что философия должна ориентироваться на критерии строгости и аргументированности, на которые опирается современная ему наука в духе Ньютона и всего такого. 
При этом всё завязано на ощущения. Жёсткая причинность, никакой свободы воли, всё предопределено устройством чувственного аппарата человека, ибо мы мир воспринимаем в комплексе: не отдельные сигналы, а всё и сразу, не разделяя.
Цитаты:
«Предполагать, что будущее зависит от прошлого, заставляет нас лишь привычка.»
«Руководителем жизни является не разум, а привычка.»  
Более того, есть шанс не заметить что-то новое, ошибочно приняв это за привычное.

Дальше (после привычки) Юм говорит о вере. Не о религиозной, а о «belief». Вера~--- особый вид представления, вызывающие чувство реальности и подлинности. 
Однако, если что-то вызывает такое ощущение, то это не обязано означать истинность. Цитаты «Вера не присоединяет новые идеи к представлению ...»

\term{Феноменализм} (феномен = явление, «вещь как таковая» = реальность). Вспоминаем Платона: мир истинный~--- мир идей, а мир вещей~--- лишь отблеск. 
Юм говорит, что когда философы говорят об чем-то истинном, они спекулируют. Может что-то и есть, но у нас всё равно нет способа определить это. 
Беркли делал упор на разрушение понятия материальной субстанции, и ему это удалось. 
Духовная же при это осталась, и Беркли так и считал, этого и добивался.
Чья-то цитата (узнать, чья): «Душа, постольку поскольку мы можем постичь её, есть не что иное как ряд перцепций... Декарт утверждал, что мысль есть сущность духа...».

В самом деле, что нас заставляет считать, что дух/душа является единой сущностью. Наше поведение различно в разном возрасте, тогда как мы можем говорить, что это всё одно? Наше «я» переменчиво. 
А что позволяет нам говорить о тождестве личности? Это не простой вопрос, а глубокий и философский.

«Человеческая личность (душа)~--- связка или пучок различных восприятий, следующих друг за другом». 
«Дух~--- нечто вроде театра, в котором играют перцепции, но у нас нет ни малейшего представления о том, что они из себя представляют.»

Раздвоение 17--18 века философии на эмпиризм и рационализм. Но в конце 18-го века пришел Кант и разрешил дилемму. Взял сильные стороны от обоих течений и отказался от слабых. 
Сохранил идею опытного происхождения человеческого знания. При этом эмпиризм не мог всё описать: не мог зацепиться за некоторые вещи. Такие свойства нужно искать с субъекте познания. 
Юм психологизировал, а Кант подошел с другой точки зрения. 

Кант. 1724--1804. Долгожитель, хотя не был особо здоровым. Похож на обезьянку: худощавый, сутулый с большой головой, ножки такие. Самодисциплина и воля помогли ему сделать столько и долго прожить.
Родился в Кёнигсберг (ныне Калининград), там почти всю жизнь и прожил.
Похоронен в стене кафедрального собора. 
Родился в семье ремесленника среднего достатка, а детей там было много. Дядя помог ему получить образование: гимназия, местный университет (не смог окончить, ибо ему приходилось работать домашним учителем: физика/математика), экстерном защитил диссертацию, став доктором, преподавал философию в университете. Когда почувствовал, что начал дряхлеть, сам ушёл. Никогда не был женат, но не был сексистом и любил пообщаться с образованными женщинами.

Кант подчинил свою жизнь строгому распорядку: по нему можно было сверять часы (например, по регулярным прогулкам). 

Его творчество делят на 2 периода: докритический (1746--1769) и критический (с 1770--1797 --- в конце жизни мало писал) (странное деление, не правда ли?).  Рубеж --- 1770 год --- выход работы «О форме и принципах ...» (работа, которая была его диссертацией). 
Лейбницианская философия (систематическая, метафизическая). Кант преподавал философию по учебнику \na.
Взгляды на пространство-время: Лейбниц --- феноменальный, Ньютон -- абсолютный.

Космогоническая гипотеза и переосмысление метафизики. 

Космогоническая гипотеза. Основана на физической монадологии. 
Мир состоит из материальных атомов, которые можно назвать монадами, причем они обладают сущностными силами: силой притяжения и силой отталкивания.
Можно объяснить движение планет в одну сторону, увеличение дистанций между по мере удаления от Солнца.

Создание Солнечной системы. Облако пыли и частиц. Холодное и разреженное. По мере притяжения образовывались сгустки и они закрутились: возникли вихри, нагревающиеся от трения. 
Эти шарообразные сгустки - прообразы Солнца и других планет. Кант подчеркивал конструктивную роль отрицательной силы отталкивания. 
Не путать логическое (ведет к абсурду) и физическое (описывает реальность) отрицание. 
Говорит, что так могли о образоваться другие системы. 
Он не говорил, что бога нет. Но говорил, что можно объяснить и без него. Канту достаточно от бога лишь материи.

Это было описано в работе «Всеобщая естественная история и теория неба». 1755. Тираж небольшой, сбыта не нашло, издатель обанкротился, всё пошло под нож.

Нам пришлют 4 страницы-шпаргалки по Канту. Их распечатать и носить с собой. И схему.

Переосмысление метафизики. 
%Справедливо ли ориентироваться на 
Насколько можно с помощью логики понять физическую реальность. Можно ли осуществлять переход от логики к реальному миру. (еще пару раз переставить слова в предложении) 
Логика носит формальный характер и не расширяет знания. Расширение знаний происходит через опыт и накопление знаний о реальном мире.
Это не беспроблемный акт. Здесь заключен корень рационализма: там не было проблемы. А вот Кант считает это невозможным. 
Что-то про онтологическое доказательство существования бога.

Работа «Опыт введения в философию понятия отрицательных величин». Год \na. Анализирует понятия противоречия и отрицания. В математике всё ОК --- нет противоречий. В логике --- противоречие. В реальности --- всё норм.

Работа «Грёзы духовица, поясненные грёзами метафизики». 1766. Кант высмеивает стремление к сверхчувственному, чем и является метафизика (сравнивает с духовидением). 
Тогда были распространены спиритические сеансы и общение с душами мертвых. Когда ученый отрывается от опыта, появляется вся эта дребедень.

Кант говорит также, что не всё в жизни есть познание: есть мораль и религия, например. Но эти области не могут иметь теоретического обоснование, поэтому не стоит оттуда выводить существование бога. 
Отсюда специфическое понятие критики у Канта. Критика --- дисциплина чистого разума, занимающаяся установлением границ компетенции различных познавательных способностей.
Почему мир соответствует тому, что мы думаем? (именно так, а не наоборот). Надо идти от субъекта к объекту, а не наоборот. Надо изучить свои познавательные возможности, а потом идти к объекту. 
Раньше они были разделены, эта иллюзия очень долго существовала (вспомнить зеркало Локка, в котором мир отражается и выглядит, как картина.) 
Раньше субъект-человек был пассивен, а это же неверно. Не стоит мыслить о субъекте, как о корзине, куда вся информация валится. Синтез двуз типов активности: и изнутри, и снаружи. Синтетическая философия.

Кант признавал, что знакомство со скептицизмом Юма пробудило его от догматической спячки. И тут же впал в другой сон: антропологический :-)
