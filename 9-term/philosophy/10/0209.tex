\lectiondate{9 февраля}
Эмпиризм Европейский. Нас интересуют Беркли (называл себя ирландцем, ибо там прожил большую часть жизни) и Юнг (шотландец).
Очень интересен вопрос субстанции. 2 основные: материальная и духовная (мыслящая). Это сказал Декарт, а потом каждый в той или иной степени высказался. 

Рационалисты же~--- любители крупных объектов и им очень нравилось это понятие, ни от чего не зависящее. Эмпиристы делали ставку на чувственный опыт, на то, с чем мы встречаемся ежедневно и непосредственно ощущаем (а субстанцию мы не ощущаем, лишь предметы), поэтому «субстанцию» они либо не использовали, либо высказывались осторожно. 
Локк сомневался, что можно характеризовать чувственную субстанцию, ибо часть вещей имеют диспозициональные свойства, которыеы не принадлежат целиком предмету, а определяются субъектом. 
Локк, рассуждая о личности, так и не пришел к пониманию, что же лежит в основе психических (духовных) актов. Ну это всё можно не записывать. Полной критике понятие «субстанция» подвергли Беркли и Юнг.

Оба они работали после революции, и можно было порассуждать, ибо был некий откат назад, да и произошло примирение буржуазии и аристократии. Наступала эпоха некоего разочарования и реакции. Вообще британские философы очень чутко реагировали на духовный климат в своей стране (вспомнить Гоббса, который только это и делал). 

Джордж Беркли. Присутствует в схеме (9 отличий эмпиризма и рационализма). 1685--1753. Просходил из семьи среднего достатка. 
Детство провел в Ирландии и поступил в Дублинский университет, а после стал учителем теологии, пойдя не по философской, а теологической часть. Общался с клириками. 
Ему было поручено организовать англиканскую миссий в Новой Англии (севернее Нью-Йорка). Там то он и занимался миссионерством, прикупил себе земли. Но дела расстроились и он был отозван в Британии. Миссия была не очень успешной, но всё же он зато побывал в Америке. 
Забавно, что университет Беркли в Калифорнии имеет отношение именно к этому Беркли. Беркли-человек произносится на Баркли, а город~--- Бёркли. Универ довольно революционный, хиппи, захват администрации, понаехавшие. 

Беркли, вернувшись в Ирландию двинулся по религиозной стязи и впоследствии стал епископом. Bishop Barkely. (загуглить какое-то ещё название). Жил в достатке, было 6 детей.

Что касается философии. Первый предмет его интереса: вопрос чувственного опыта. «Опыт новой теории зрения» --- первая работа. Тбо зрение даёт больше всего представлений об образе мира. Следующий трактат: «О приницпах человеческого знания». «Три разговора между Гиласом и Филонусом» --- самая известная работа. 
Гилас (Гюле, Гиле... --- древнегреческое слово --- материя) --- материалист. Филонус (филио --- греческое --- люблю, нус --- греческое --- ум) --- тот, кто предпочитает следовать духу, идеалист. Спор идеалиста и материалиста. Симпатии Беркли на стороне Филонуса.
Более поздние работы Беркли критикуют современное естествознание и математику, возвышая религию и теологии.
Он во всем эмпирист, кроме признания бога и религиозных истин, что забавно. Критикуют материю и её признание (ибо это ведёт к атеизму). Имматериализм --- Беркли не сомневается в существовании материальных вещей, а кроме этого ничего в материальном мире не существует. Нет материальной субстанции, вместилища всего.
Беркли стремится показать, что первичные качества не имеют обоснования. Хотя и сам Локк сомневался в статусе вторичных качествах, но ему приписывают будто «он рассуждал о них, как о субъективных». Беркли же сказал что и масса, плотность и т.п. тоже субъективны. Мы бы никогда ничего не узнали о форме, если бы это не было получено из субъективного опыта, через вторичные качества. Например, треугольник как контраст светлого и тёмного на доске, т.е. воспринимаем первичные качества через вторичные. Вывод очевиден.

Главная фишка Беркли: попытался последовательно вывести всё знание из ощущений. Предложил проверить всё через истоки идеи, а это будет именно ощущение. Беркли убирает опыт рефлексии и говорит: «любая идея связана с каким-либо ощущением». Если мы не можем проследить источник идеи, то она не имеет право на существование. Это помогает ему дискредитировать понятие материальной субстанции.
На этом этапе важно, что за конкретными предметами не стоит материальной основы. Это не отвергает идею «стола». Беркли не отрицает существование ничего, что можно реально увидеть и потрогать. Для материальных вещей существовать означает «можно воспринимать». Есть ли что-то кроме этого --- неизвестно, ваш КО.
У Беркли возникает солипсизм (солюс --- лат --- я, единственный). Существование всего субъективно. Мир состоит из духов. Для духовных существ существовать означает «воспринимать». 
Третий вид существования касается материальных вещей. Вот мы уйдем из аудитории --- пропадает ли она. Возникает вопрос непрерывности существования вещей. Беркли вводит «существование в воображении возможного восприятия». Если мы можем вспомнить, то эти предметы продолжает существовать. Но если вдруг все забыли, то как проверить. Если воспринимающий дух находится в пределах досягаемости предмета и может в любой момент приблизиться и начать воспринимать, то всё ок. Для этого и нужен сторож: чтобы институт ночью не пропал. 
Четвертый вид существования: а если вдруг сторож заснул. У нас есть универсальны контролер: «бытие вещей и всего мира в боге». Бог, как универсален, держит всего под контролем. Он не может быть обманщиком и всё ок. Итак, 4 вида существования.

Считает, что многие абстрактные понятия философов возникают из ума, вследствие злоупотребления словами. Однако существование слова ещё ничего не значит. Вот слово «человек»: кого вы себе представите: мужчину или женщину, ребенка, русского или китайца?.. 
Беркли номиналист: считает, что существуют лишь конкретные идеи, а не какие-то общие. Абстрактная идея работает как репрезентант: представитель выступает в роли целого класса, однако сохраняет свою индивидуальность. Представив ребенка китайца, это всё так же человек.
Беркли утверждает, что «..цитата..».

В поздних работах Беркли критикует Ньютона, существование абсолютных пространств и времени (еще бы: это же не материальные характеристики).  Критикует понятие материальной причинности. Есть лишь духовная причинность. Материальные же объекты друг на друга не влияют. 

Теория хорошая, но приходится положиться на существование бога.

Дэвид Юм (может Юм или Юн??). Был похож на Ломоносова: упитанный человек в парике. 1711--1776. 
Происходил из дворянской семьи, которая жила в Эдинбурге. Учился в Эдинбургском университете. Там изучал языки, логику, метафизику и т.д. По настоянию отца после университета занялся коммерческой деятельностью. 
3 года провёл во Франции. Там особо бизнесом не занимался, зато налаживал контакты и изучал картезианскую (Декарта) философию. 
Там он и написал свой главный «Трактат о человеческой природе», будучи довольно-таки молодым. Считал, что оно потрясёт основы философии. Говорил, что восприятие человеком мира накладывает отпечаток. Но стиль написания был печален, и его никто не читал: лишь единицы одолели. Потом он тоже согласился, что книга была мертворожденной. 
Дальше очень много работал над стилем и стал весьма читаем. Выпустил очень сокращенное изложение, что удобно особенно для нас. Юм пытался заняться академической деятельности, но не попал на кафедру в Эдинбурге (Шотландская (пресветорианская??) церковь не любила его работы). Зато работал в библиотеке и получил доступ к обширному историческому материалу. 
Написал 8-томную историю Англии, сделав акцент на психологические аспекты политики Английских королей. Реакция и тут была неоднозначна. В 1757 --- «Естественная история религии», что еще больше сгустило над ним тучи в плане мировоззренческих позиций, ион начал восприниматься как человек вредный для религии. 
Так он до конца жизни и занимался библиотекой. Дружил с Адамом Смитом, и именно он опубликовал в 1758 его биографию. Воспринимался как историк. Юмизм расцвел в начале 20-го века. Лишь Иммануил Кант оценил работы Юма (и это было очень полезно для Канта).

С именем Юма в философии ассоциируется агностицизм. «Достоверное знание о мире невозможно». Но у Юма это скорее всё же скептицизм: есть большие трудности и сомнения, но не полное отрицание. 
Перцепции --- впечатления. Идеи --- слабые.

Опыт:
\begin{itemize}
	\item впечатления (яркие, впервые переживаемые)
	\begin{itemize}
		\item впечатление ощущения (их изучает биология, а не философия)
		\item впечатление рефлексии (желание, надежда, отвращение, всякие эмоции)
	\end{itemize}
	\item идеи 
	\begin{itemize}
		\item простые 
		\item сложные
		\begin{itemize}
			\item модусы
			\item субстанции
			\item отношения
		\end{itemize}
	\end{itemize}
\end{itemize}
Оба (идеи и впечатлении) ясны и отчетливы. Юм просто описывает чувственный опыт. 

Философия должна стать чисто описательной, по мнению Юма, превратиться в философскую психологию. Что-то про принцип ассоциации. Юм самое интересное не раскрывает. Если бы у каждого был свой опыт, никакой науки бы не было, значит есть что-то общее, какие-то базовые структуры, у нас общее формирования идеи мира. Юм это называет это принципом ассоциации и считает тайной человеческой природы. А вот Кант нам поможет.

Юм говорит, что ежели все наши идеи зависят от человека, как носителя этих идей, то самые главное понятия в естествознании имеют психологическую природу. А физика ищет законы, связи (а не будущее ли??). Юм говорит, что причинности в природе нет: лишь только в области психике, это способ соединения перцепций. Юм отрицал свободу воли: она никак не влияет ни на что, мы полностью во власти нашей природы человеческой, жесткий фатализм и детерминизм. Нет гарантий, что ничего не изменится: «всегда будет всходить солнце». Доказать, что так будет всегда невозможно, ибо наши знания основаны лишь на вере в сохранение порядка и на прошлом опыте. 