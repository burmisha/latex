\lectiondate{13 апреля}
Экзистенциализм.
Исторически существует в 2 формах:
\begin{itemize}
	\item немецкий (Мартин Хайдегер, К. Ясерс) и французский (Жан Поль Сартр, Камю),
	\item атеистический и религиозный.
\end{itemize}

Существование предшествует сущности, поэтому существование~--- главнее. Их противопоставляют.

Противопоставление истины и смысла. Смысл --- чисто человеческая форма. Истина --- явление--сущность, этому противопостваляется «феномен».

Классическая философия рассматривает понятие свободы как необходимости. В экзистенционализме --- свобода как исходная ситуация, как допознавательная данность.

Гуссерль. 1859-1938.

Хайдегер. Заподозрил, что кризис связан с чем-то пропущенным.

Эйбетическая редукция. Выносим все знания о мире за скобки. Затем соершаем трансцендентальную редукцию и выносим за скобки все знания о сознании (это второй этап).

Сознание интенционально, активно и живо. К одному и тому же предметы моно относиться по разному (пример про здания).
Интенция --- стремление, направленность на объект. Сознание всё время в чём-то заинтересовано, никогда не бывает спокойным. Деятельность непросто утсроено: есть жихненный горизонт, океан возможностей иметь ввиду объект, но это именно возможности, они актуализируются по мере необходимости.

Мартин Хайдегер. 1889--1976. Родился в простой католической семье, начал изучать теолгию, но через 2 года предпочел философию. Впоследствии познакомился с Гусселем и стал его ассистенотом. С 1919 --- разрабатывает собственную философию.
1933 --- вступил в *** партию, признав фашизм, поэтому отношение к нему неоднозначно.
Его фраза: «Язык --- дом бытия».

Работа: «Бытие и время». Переведена на русский.

Тут бытие. Dasein.\\

