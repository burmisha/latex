\lectiondate{20 апреля}

Аналитическая философия выросла из позитивизма. 
Этапы позитивизма
\begin{itemize}
	\item I позитивизм
	\item II позитивизм
	\item неопозитивизм 
	\item логический атомизм (Рассел, ранний Витгенштейн)
	\item логический позитивизм (Венский кружок)
	\item $\qquad$ поздний Витгенштейн (стоит особняком)
	\item философия науки (Карл Поппер, Томас Кун, Фейера)
\end{itemize}

Создатели аналитической философии: Рассел и Витгенштейн.

Бертран Рассел. 1872--1970. 100\% англичанин. 1950~--- Нобелевка по литературе (по философии же нет).
Работал в Кембридже. Одно время интересовался Советским Союзом. В 20--21 году посетил Советский Союз и Китай.
Как крупный философ заявил от себе в 19 «Принципы математики» (сперва написал сам, а потом в латинском варианте~--- «Principia Mathematica»~--- в 3 объёмных томах совместно с Уайтледом в 1910--1913)
Уайтлед тоже работал в Кембридже, но потом переехал в США и там увлекся какой-то религией и космизом.

Что-то про символическую логику. Главное её отличие от Аристотелевской: она была логикой классов, а у Рассела --- логика высказываний.

В работе «Principia Mathematica» показано, что 
\begin{itemize}
	\item математика~--- раздел логики
	\item базовая структура любого естественного языка --- математична. При этом естественные языки непригодны для анализа в силу расплывчатости. 
	\item Рассел рассматривает базовую структуру предложений. Различает атомарное высказывание и молекулярное высказывание. Пример: «Джон --- человек» --- атомарно. «Джон и Мери собираются в кино» --- молекулярно, ибо состоит из «Джон собирается в кино» и «Мери собирается в кино». От Беркли унаследована необходимость проверять идею наличием её источника среди чувств (а здесь надо предложение бить на атомарные).
\end{itemize}
Варнинг: идёт обсуждение европейских языков. 
Атомарные высказывания всегда имеют субъектно-предикатную форму. «Джон (субъект) смертен (предикат: быть смертным).»

Атомарные высказывания соответствуют фактам.
В природе не существует молекулярных высказываний --- фактов. Они должны быть разложены на атомарные через логический связки.                                         

«Все люди смертны» --- этому никакого факта не соответствует. Мир состоит из фактов, а каждый факт --- отдельный предмет и его индивидуальные свойства.

Дескрипция.
Проблема. Иногда мы считаем утверждения истинными, хотя объектов не существует (например, про Гамлета или медузу Горгону). Или вообще «Медузы Горгоны нет».
«Нынешний король Франции мудр». Вывод целой группы философов: сущности «нынешний король Франции», «медуза Горгона» логически существуют в царстве теней.
«Бог существует» в адекватной форме должно быть записано «Нечто причем только одно является единственным, всемогущим и блаженным».

Людвиг Витгенштейн. Ученик Рассела. 1889-1951. Австриец. Увлекался всем подряд. Рассел притащил его в Кембридж. 1922 --- «Логико-философский трактат». 
Цитата: «То, о чем нельзя ясно сказать, следует молчать». Был весьма странным лектором. Потом уехал в сельскую местность, учил детей математике, а сам следил за их изучением языка.
В трактате изложена.

Философия является полноценной деятельностью. Но она не открывает новых фактов. Философия даёт представление о структуре самого мира.

Поздний Витгенштейн. Говорит о том, как мы определяем значение слов. 
Как ребенок понимает, что такое «красный». У него есть некий внутренний язык. 
Мы видим «что-то» через «как». Мы видим либо азйца, либо утку.
