\lectiondate{20 апреля}

Аналитическая философия выросла из позитивизма. 
Этапы позитивизма
\begin{itemize}
	\item I позитивизм
	\item II позитивизм
	\item неопозитивизм 
	\item логический атомизм (Рассел, ранний Витгенштейн)
	\item логический позитивизм (Венский кружок)
	\item $\qquad$ поздний Витгенштейн (стоит особняком)
	\item философия науки (Карл Поппер, Томас Кун, Фейера)
\end{itemize}

Создатели аналитической философии: Рассел и Витгенштейн.

Бертран Рассел. 1872--1970. 100\% англичанин. 1950~--- Нобелевка по литературе (по философии же нет).
Работал в Кембридже. Одно время интересовался Советским Союзом. В 20--21 году посетил Советский Союз и Китай.
Как крупный философ заявил от себе в 19 «Принципы математики» (сперва написал сам, а потом в латинском варианте~--- «Principia Mathematica»~--- в 3 объёмных томах совместно с Уайтледом в 1910--1913)
Уайтлед тоже работал в Кембридже, но потом переехал в США и там увлекся какой-то религией и космизом.
