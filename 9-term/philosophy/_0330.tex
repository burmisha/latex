Итак. Про убийство Сократа. По-видимому, беспокойство было вызвано идеями Сократа.
Жена Сократа - Ксантипа. Всё время пилила своего мужа, ведь Сократ ничего не далал и было тяжело сводить концы с концами. Её образ стал нарицательным. 
Сократ жил на пожертвования, относился к дентгам холодно, кормился на пирах. У греков пиры были ритуалом - это было интеллектуальное общение, а не тусево. Жен не приглашали.

Философия Скората сводится к методу обсуждения вопросов. Его достижение методолгочес5ое. 
Сократ был противником натурфилософии, считал что природой интересоваться природой, ведь бог столь всемогущ, что познать его нельзя. Бессмысленно изучать, что является апейроном - вода или воздух. 
Был противником мифологии в смысле основы жизни любого человека. Первый ярко выраженный рационалист. Его бог - разум. В ответ детерминизму выдвинул телеологию - учение о цели (телос - цель, греч.). 
Разум в этом конекчте очень логичен. Зачем и для чего всё это произошло и было создано. Сократ верит, что тут есть  рациональное зерно. Человек может через разум понять цель своего существования. В этом смысле он безбожник.

Процедуры Сократа не столь просты и требуют обучения. Путь обращения к интуиции. Но не к жизненной, а интеллектальной - способности ясно представлять вещи, которые в опыте не даны. 
Это сравнимо с процессом обучения - озарение в процессе обучения, когда ты наконец-то понял формулу/теорему. Сократ считал, что эта особенность сильно связана со всеми сторонами человека, в т.ч. нравственности. 
Это одновременно и совесть, и внутрений голос. Сравнить с дествиями, которые разум предлагает сделать, но внутренний голос против.

Сократ мог видеть некий путь, скрытый от большинства. Некий демон его напрвлял, действующий о  имени истины. Эта способность дана не каждому, а лишь обладающим неким даром, возвышенным, умным, прозорливым, им то бог и дает такого демона. 
Откуда, правда, он знал, что это демон от истины? А тогда не было противооставления богов. <img src="trollface.svg" >

Принципы укреплены в структуре бытия, человек выступает в роли медиума.

Изучал суть добродетели. Что же ближе людям, каковы их нравственные принципы. А потом уже можно перейти к более абстрактным вещам. 
Как может быть нравственным человек, который не знает что такое добродетель? Считал, что разум - высшая способность души. (Кстати, греки считали, что разум-душа - единый комплекс, не разделяя их существенно.) Это гиперрационализм. 
Для Сократа мораль сливается со знанием. Это было наивно, как показывает историю. Его ученик Платон начнет насильственно насаждать нравственность. Коммунистя, кстати, тоже.
Сдержанность, мужество, справедливость - основные добродетели. Приобретаются путем познания и самопознания. Но человеку нужно помочь.
Философский принцип Сократа предлагвет 4 стадии (см. пример):

	1. Ирония. Ведущий в беседе пытается обнаружить неточность у собеседника и создать соответствующий психологический настрой. 
	Сократ прикидываается простаяком ("язнаю, что я ничего не знаю", впрочем он был неискреннен, дурача простачков, за что его ругал Ницше)
	2. Майевтика. Повевальное искуство. Самая главная часть. Сократ утверждал, что унаследовал от матери способность помогать рождению скрытй мысли. 
	Раз он помогает родиться мысли, то мысль уже существовала в глубинах разума, а он лишь помогает выявить её и вывести на свет.
	3. Индукция. Указывает на то, что майквтика проводится путем исследования частных случаев с целью отыскать общее, что объединяет поведение людей.
	4. Дефиниция. Строгое определение через род и видовое отличие. Это самый строгий вид определения в логике. Ср. с человек - двуногое без перьев с плоскими ногтями. 
	Понятийная фиксация общего, полученного в ходе индукции. Любую дефиницию по Сократу можно подвергнуть новой иронии.

Вся европейская философия рациональна.
Заканчивая с Сократом, приведем цитату из диалога какого-то.
После казни Сократа ( он выпил яд самомстоятельно) через год афиняне раскаялись.

Платон
427-347 до н.э.  сын афинскомо гражданина, аристократа и рабовладельца. В юности увлекался любыми соревнованиями - поэтическое искусство, бег и метание диск. Писал гимны и поэмы. 
Идет он как-то на очередной конкурс любителей поэзии и услышал кружок Сократа. Он всё забросил и стал ходить туда, став одним из верных учеников. На него очень сильное впечатление произвела смерть Сократа. 
И он уехал много путешествовать. Наиболее на него повлияла Сицилия - оплот Пифагореизма. 
Он восринял их идеи и образ жизни, имея и свои. Идеи об идеальном обществе и идеальном гос-ве, общение с местными аристократами не удалось. 
Его разгромили и в наказание за зловредные идеи продали в рабство. К счастью, его сторонники быстро его выкупили. Этот опыт был поучительным. 
Потом Платон н когда ничегоне писал о рабстве, описывая устройство общества и государства. Вернулся в Афины, стал доброаор<дочным гражданином, много работал-философствовал. 
Создал академию. Ранее все собирались в саду. Находился рядом с храмом во имя Академа. Вот название и пошло. Его научная форма стала образцовой для философии даже в средневековье. 
Однако она была не слишком демократичной - учитель вещал с кафедры. Через некоторе время - второе путешествие в Сиракузы, пытается рассказать свои идеи, но не находит понимания и его высылают. Там он подавлен и в 80 лет умирает.

Оставил обширное философское наследие. Его поэтический дар давал о себе знать, хоть он пытался задавить свой дар, считая его запрещенным в идеальном государстве. Писал диалоги. Персонажи олицетворяли идеи.
