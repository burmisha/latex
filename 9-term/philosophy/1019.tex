\simpletitle{19 октября}
Продолжаем знакомиться с европейским рационализмом.

Дуализм Декарта. Существование 2 субстанций: духовной и телесной
Почему же Декарт так считает? Почему нельзя вывести материальную субстанцию из духовной? Бог же, будучи нетелесным, создал материальное. Тут всё не так просто. Субстанции противоположны по своим свойствам. Всякая активность, свободное движение вне пространственных рамок для духовной субстанции, так вот это не встречается в реальном мире, ибо всё от чего-то зависит. Вот Декарт и утверждает, что это 2 мира, живущих по своим законам. Творческая, активная, свободная~--- хар-ки душевной. Костная, материальная~--- телесная, её нужно всё время толкать (по механическим законам). Модель: разбитая глыба ротацией и вихрями преобразовалась в то, что имеем. Но сама материя не способна сама двигаться, нужен первоначальный толчок: это бог и сделал. 
Сложность Декартова дуализма в следующем: проблема взаимодействия субстанций. Ведь субстанция, по определению, ни от чего не зависит. Кровообращение и рефлексы изучались для определения границы телесных взаимодействий, и где же начинается духовное/душевное воздействие/импульс. Декарт вынужден был в силу своего утверждения о противоположных субстанция, говорил, что все живые существа~--- странное сочетание двух субстанций (это лишь для людей, а животные~--- только телесные субстанции). 

Декарт придумал следующее:
многие поведенческие акты~--- лишь иллюзия, наше тело весьма совершенно
признает лишь механическое движение $\Rightarrow$ должно быть непосредственное взаимодействие души и тела. Идея времен Декарта: утончение материи позволяет взаимодействовать. Декарт отдал этому дань: считал, что в шишковидной железе (гипофиз) вступает в взаимодействие дух и материя, но это носит скорее гипотетический характер. Главный путь решения психофизической проблемы <<душа-тела>>: когда душа и тело взаимодействуют~--- это бог причастен. Хотя сама душа недоступна для внешнего и материального воздействия, а только посредством бога существует в душе представление о материальных вещах. Окказиональное свойство (см. окказионализм, акциденция)~--- единство материи и духа, причем каждый остается сам по себе, это единство не их природы, и их построения, глубинного взаимодействия нет, это случайно и временно, структура не изменятся (WTF). Бог здесь действует от случая к случаю. Декарт относился к материи предвзято: считал её костной и т.п., хотя сам занимался физиологией. Этот вопрос требовал дальнейшей разработки (см. далее про Спинозу).

А пока еще немного о Декарте.
Разработал своеобразную этику, где душа отождествляется с умом. (что-то про Гильберта и деление людей на душевнобольных и не очень). Чувственность и воображение (да и все остальные проявления) Декарт не выделяет, а считает модусами ума. Аффекты и страсти (человеческие волнения) возникают от влияния на душу телесных движений: пока разум не сработал, возникает заблуждения ума. (о\_О Тело же не может влиять, так какого?! видимо, это завязано на рефлексы и индивидуальную психику). Результат: злые поступки и всякие другие негативные этические действия. Заблуждение есть грех (по Декарту). Источник этого~--- не разум (он не может ошибаться!), а свободная воля человека. Воля побуждает человека торопиться и высказывать неверные суждения, совершать недостойные поступки. Упрощенно Декарт трактовал психическую жизнь, не взирал на все тайны и сложности. Зато он следовал своему принципу: превознося разум и забивая на тело.

Правила и методы
Мир~--- это механизм, а главная цель~--- познание мира. Главная наука о природе~--- механика. Познание~--- конструирование машины мира из каких-то там начал. Мы уподобляемся творцу: находим основание для мыслительной конструкции, а из неё разворачиваем всю структуру. Кстати, это можно делать и без самого изучения природы :-) Но Декарт интересовался исследованиями и таких выводов не делал (а Гегель потом делал и весьма успешно)
Строим знания с начал, которые не вызывают сомнений. Ищем врожденные идеи, которые будут отчетливы для всех и приняты на веру. Нужен метод, который превратит научное познание из кустарного промысла в систематическое планомерное производство. Метод дедуктивный. Причем поначалу нужна еще и интуиция. Метод~--- это инструмент, расширяющий возможности человека, и обеспечивающий планомерное и систематическое получение знаний.
\begin{enumerate}
	\item Начинать с простого и очевидного. 
	\item Путем дедукции получать более сложные высказывания. 
	\item Не упустить ни одного звена~--- не упустить ни единого разрыва звена
\end{enumerate}

Декарт отрицал, что бог спланировал определенную последовательность событий, и что есть некий предрешенный конец. Церковь к его системе относилась очень негативно. Отрицание объективной телеологии было очень важным для вдохновления ученых. А иначе ниче не надо делать~--- молись и молись, всё равно всё закончится страшным судом, зачем учиться и страдать..."

На этом с Декартом можно и закончить. Но мы к нему будем порой возвращаться, ведь он тоже важен (как и Платон для Античности)

Спиноза 1632-1677 Голландец.
Предки его пришли из Испанской империи (с территории Португалии), бежали ибо иудеи. В Голландии все было довольно терпимо, поэтому туда бежала его семья. Сперва говорил на португальском, но изучил латынь~--- ваще молодец: умел писать и читать. Его отец сколотил состояние и Спиноза получил большой кусок наследство. Но уже к этому моменту его не особо любили, ибо он говорил, что наука и церковь несовместимы. В результате его выгнали из общины, родственники отобрали наследство. Спиноза уехал из Амтсредама, но подал в суд и ему вернули. Выиграл, демонстративно отказался и уехал в провинцию. Занимался шлифовкой линз, что было весьма прибыльно. Умер он от каких-то легочных проблем. (Да и вообще не был он сильным и спортивным, хотя жил и на природе, а пыль стекла~--- та еще гадость). Жил, как учил (live fast die young).  Был очень увлечен: много читал, вроде бы встречался с Декартом. Спиноза пытался создать единую картину природы: его не устраивало двойственность мира. Написал много произведений. Нам надо знать <<Этика, доказанная в геометрическом порядке>>. Как всегда ложь и обман: этике посвящена только последняя глава, зато много по делу: леммы. теоремы и т.п. Спиноза вдохновлялся идеями пантиезма. Природа воспринимается как <<бог растворен в природе>>, природа видится живой и одухотворенной. Спиноза очень близко принял эту идею: трактовать природу как живое образование, содержащие начало движения и творческий импульс. Декартовский дуализм преодолел очень просто: совместил их воедино. В основу положил тождество бога, природы и субстанции. (Deus sive natura sive substantia). Никаких других субстанций нет, никакого трансцендентного бога. Субстанция есть причина самой себя (Substantia est causa sui). Бесконечное многообразие~--- проявляние единой субстанции. Единство~--- не сплошная расчлененность, но и большое число модусов субстанции (мн-во отдельных предметов, мыслей, теорий). Весь мир единообразен и органичен. Единство не исключает множественность. Но модусы касаются формы существования.
Качественные характеристики~--- атрибуты~--- неотъемлемые свойства субстанции. Субстанция включает в себя всё, поэтому и число атрибутов бесконечно: мало ли что она в себя включает. Но слабому человеческому уму открывается только 2 неотъемлемых атрибута: протяжение и мышление.
Спиноза стоит на позиции детерминизма и фатализма: никакой воли и выбора.

natura naturaus
natura naturate     снизе~--- природа, как совокупность модусов

какой-то бред, я не понимаю. :-( но про то, что познание может заменить человеку все другие стремления.
Чем больше стремлений у человек~--- тем он несчастнее. Но есть у человека не только негативные аффекты, но и 1 позитвный: страсть к познанию. Причем он существует сам по себе и для себя: а не для того, чтобы продать его и купить виллу поесть. Можно вытеснять отрицательные аффект познанием. Спиноза так и жил: познавал с утра до вечера. Так и умер счастливым.

Модусы бывают 2 типов. Телесные/материальные и духовные (мысли, теории, образы...). Модусы: единичные (Василий!) и собирательные (студенты). (Есть еще и формулы, а есть таблица умножения) 
Всё в рамках единой субстанции, поэтому нет психофизической проблемы. Параллелизм за счет единства. <<Порядок и связь идей те же, что порядок и связь вещей>>. При такой формуле и ошибаться не в чем: познаем то, что есть, причем одним способом. Верили в построение в общей теории, которая всё опишет от начала до конца, причем окажется довольно простой. Спиноза был одним из тех, кто формировал эту идею.
2 бесконечных модуса: они связывают мир модусов. Модус протяжения~--- движения. Бесконечный модус мышления~--- бесконечный интеллект. Движения не является атрибутом, не является неотъемлемым: движение должно иметь начало. Это связно именно с механической трактовкой движения: надо толкнуть/качнуть/пихнуть/ударить/потянуть.
Бесконечный интеллект~--- какой-то бред. Спиноза не выделяет бога как носителя сверхразума.

Теперь вторая работа Спинозы. <<Богословско-политический трактат>>. Спиноза предпринял не просто критику религии, но и начал традицию исторического критического рассмотрения библии. Научный подход распространяет и на эту книгу, понимая, что её написали люди. Ищет противоречия и т.п. Считает, что в основе религии лежит страх. Недостойно человека жить в этом страхе: надо избавляться от религиозных предрассудков. Спиноза серьезно об этом пишет: что это заблуждение не просто так, а ибо есть государственно-политическая база. (В те времена уже не жгли на костре). Спиноза считался атеистом. (он и пантеист, и атеист~--- одно другому не мешает). Подчеркивает, что церковь инграла очень реакционную роль в истории. Больше го беспокоит, что церковь препятствовала познанию. Страх, невежество.

С рационализмом временно заканчиваем. Посмотрим, что в это время делали другие ребята.