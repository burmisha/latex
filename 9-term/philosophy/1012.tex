\lectiondate{12 октября}
Зачет = (число контрольных * 8) + (число занятий / 2) + 1

Обсуждение таблицы с двумя направлениями: рационализм и эмпиризм. Не факт, что все философы придерживались всех признаков. Но это тенденции.

Эмпиризм представлен только англичанами (лишь Дэвид из Шотландии).

Посмотрим на основоположника европейского рационализма. Рене Декарт
Во Франции формируется дедуктивно-рационалистический метод познания. Радикально отличается от средневекового рационализма. Бэкон и Декарт не были знакомы, философствовали независимо. Но есть общие черты. Оба сперва критикуют схоластику. Говорят, что метод не подходит для знаний нового типа. Еще общее: ориентация на науку. Хоть Бэкон и не занимался наукой. Декарт использовал метод, применяемый математикой

\begin{enumerate}
	\item Разрыв со схоластикой
	\item Ориентация на науку
	\item Сомневаются в прежних авторитетах и знаниях, не полученных научным образом
\end{enumerate}
Может мотив и не так выражен у Бэкона. Но надо вспомнить его отношение к идолам.

Теперь немного об особенностях рационалистического метода. На примере истолкования истины.
Истина полная, абсолютная, вечная, неизменная~--- это мнение рационалистов (в корне отличается от эмпиризма). Классическая от Аристотеля традиция. Более того ей присущ всеобщий обязательный характер, она объективна и необходима. Такое толкование ориентировалась на математику~--- идеал научного знания. Но в конце 18 века математика действительно достигла высоких успехов, не то, что эмпиристы. Развитие~--- из-за запросов мануфактур, мореплавания (астрономия). В начале 17 века арифметика, алгебра и геометрия почти достигли современного уровня. В это период сложились математические методы исследования. Таблицы логарифмов Неппера. Бернулли и компания~--- дифференциальное и интегральное исчисление. Математика явилась основой философского и научного мышления эпохи и вообще 18 века. Что же выбрать первейшим источником знаний. Не чувственный опыт, а что же тогда? Апеллируют к разуму. Преувеличивали значение дедукции и разума как источника знания. Рациональное зерно: акцент на всеобщий характер логической структуры знаний. Нужна воспроизводимость доказательств. 
Отождествление (пункт 3) реальных причинно-следственных связей с отношениями логического выведения. Природные связи полностью и до концы разложимы до связей логических. Взаимно-однозначное соответствие между логикой и реальным миром, физическими процессами. Мир доступен человеку для познания. Он откроет все свои тайны со временем. Этой мыслю питались все философы: Декарт, Лейбниц, Ньютон. Эмпирическая проверка не нужна. Нужна интеллектуальная интуиция. Это же чистейший Платонизм нового времени. Всё сводится к мышлению, а потому есть соблазн объявить, что всеобщее мышление и творит мир

Логическая интроспекция~--- единственны путь познания. Конечно, это преувеличение. Умственный труд. Новая европейская цивилизация поставила во главу науку.

Рене Декарт. 1596-1650.
Картезиус. (Латынь)
Картезианство~--- его направление. Картезианцы. Но такой школы не было, хот< сторонники были.
Родился с западного побережья, небольшого городка. Под Нантом. Средний достаток семьи. Ранние способности к науке => отдали в иезуитский колледж Ля Флеш (см. Плоть~--- фр.). Математику учили хорошо. Декарта разочаровывал религиозный схоластический уклон. Поэтому по окончании скептически отзывался о колледже. Дальше отравился путешествовать. Это было не очень дорого в те времена. Познакомился с голландскими учеными. Начал писать. <<Правила для руководства ума>>~--- первая работа. Классика~--- 1637 <<рассуждения о методе>>. Вышла на латинском и, внимание, французском, что было в новинку (и поднимало продажи). Работа из 4 частей. И лишь 4-ая посвящена метафизике, а первые 3~--- научным вопросам. <<Размышления о первой философии>>. Через 6 лет на французском~--- <<Метафизические размышления>> (продолжение). За ней закрепилось название <<Медитация>>. В метафизических размышлениях полно изложены философские рассуждения Декарта. Начинать научную работу нужно усомнившись в данных и показаниях органов чувств. Аргументы в пользу обмана чувств:

\begin{enumerate}
	\item Размеры тел в зависимости от их удаления
	\item Ложка в стакане с водой сломана
	\item Сон. Там происходят любые события, причем они кажутся реальным.
	\item Фантомные боли.
\end{enumerate}

Итак, нас обманывают. Но надо усомниться и в математических истинах. Может это тоже вариант сна. Надо искать нечто несомненное. В некоторый момент надо усомниться и в существовании бога. Осуждения церкви и, в первую очередь, протестантов (ибо работы выходили в Голландии). Декарт обвинен в атеизме. Декарт говорил, что он же все восстанавливает, но а вдруг человек умрет, не дочитав книжку, так и оставшись атеистом. Отношения были не очень.

Был приглашен наладить научную жизнь в Швеции. Но там другой климат. Декарт не обладал крепким здоровьем, а из-за статуса должен был сопровождать королеву на утренних прогулках. Занимался математикой, физикой, психологией, кровообращением, рефлексами... ему это было нужно
Как и Бэкон требовал пересмотра всей прошлой традиции. Был революционером философии. <<Стер губкой всё что было написано до него. И начал с чистого листа>>. При этом Декарт апеллировал к разуму и самосознанию. Клал в основу принцип очевидности. (Не вызывающие сомнения абсолютные истины). Откуда их брать~--- из естественного света разума (люмен натурале). На истину скорее натолкнется скорее отдельный человек, чем целый народ, ибо обычай предписывает многие суждения принимать на веру. Деятели церкви верно считали, что Декарт им вреден, ведь он считал, что человек должен опираться на собственный разум. Нужно не усваивать чужие мнения, но после некого обучения стоит отодвинуть чужие мысли и попытаться найти что-то своё.
Сомнение~--- первое методологическое требование. Во всем.
Научное знание должно быть построено как единая система. Раньше считали, что научное знание~--- случайное собрание истин. Теперь нужно что-то очень прочное положить в основу. Что-то, чему все поверят.
Чему мы не станем сомневаться. Кто-то сомневается. Есть носитель! Вот оно. Причем сомнение рационально, поэтому сомнение~--- интеллектуальная процедура. Поэтому можно сомневаться во всем кроме наличия сомневающегося. Мыслю~--- следовательно~--- существую. Итак есть вещь, она сомневается, а значит существует. Источник сомнения. Есть Я как мыслящая вещь. Cogito ergo sum (кожито эрго сум). Рационализм стремится к общезначимости, но начинает с осознания себя как мыслящей вещи. Речь идет не о психологическом субъекте, а о субъекте познания.

Есть некая норма функционирования мышления. Все несогласные с этим~--- не мыслят или мыслят как-то неправильно. Лишь в 17 веке появилась эта идея: об изоляции. Схема отделения была неясна. Психиатрия сложилась на базе существующей социальной практики. Сперва никакого научного подхода не было. Сперва это было и средством устрашения.

Я знаю что не всесилен. Я и повлиять не на всё могу. Я лишь мыслящая вещь. Не могу повернуть природу. Даже своё собственное тело не могу поменять (тогда нельзя было, то ли дело ныне творится). Ест что-то более могущее, всемогущее. С ним я и соотношу все свои способности. Идея бога как совершенного существа. Повторяет онтологическое док-во сущ-ия бога.
Человеческое сознание все еще пассивно. Оно не генерирует принципы и начала. (Эта идея появится лишь в конце 18 века. Мышление как механическое образование. Бог как гарант правильности. Именно бог вложил в нас естественный свет естественный мир, а не бред какой-то. Ибо бог есть всеправдивейшее существо и не может вводить нас в заблуждение. Бог~--- источник объективного ...... что-то там.
Все смутные идеи~--- от людей, все ясные~--- от бога, а потому истины. Мыслящее Я~--- субстанция.
Субстанция~--- вещь для существования, не нуждающаяся ни в чем, кроме самой себя. Некая основа всех процессов. В т.ч. интеллектуальных, материальных. В строгом смысле субстанцией можно считать только бога. Субъективное я~--- конечно и зависит от бога. Но тем не менее применяется и к людям. Ибо люди нуждаются лишь в обычном содействии бога, а остальные~--- не только богах, но еще чего-нибудь.

Атрибут~--- неотъемлимое свойство субстанции. Модус~--- способ существование, состояние, проявление, частный случай субстанции. Главный атрибут мыслящей СУБСТАНЦИИ~--- непротяженность и неделимость. Телесная субстанция~--- полная противоположность мыслящей субстанции. Имеет в качестве атрибутов делимость и протяженность. И вправду, полная противоположность. Все материальные образование имеют определенный размер, форму. Цвет, запах, вкус, температура~--- связаны с влиянием на человеческое тело материальные образования~--- эти качества являются модусами. Это вторичные качества. Первичные~--- фундаментальные: протяженность ...
Первичные модусы~--- ясны и отчетливы, истины.
Декарт обладал странным взглядом на природу, как на полную противоположность мышлению. Считал, что бог создал материю как глыбу, дал импульс, разломил её и оставил как есть. Считал, что пустоты не существует. Всё происходит через процессы, которые создают некие вихри. Вихревое движение, основанное на равенстве сил действия и противодействия. Но необходим первый толчок для всего этого, и бог его совершает. А потом работает закон инерции (Декарт его и сформулировал).

Декарт отождествил материю с пространством. Геометрию можно считать учением о самом телесном мире. Ранее она считалась в чем-то пригодной, но не везде, была сама по себе. А теперь изучение природы подобно конструированию геометрических объектов. Атомы не признавал~--- всё делимо. Декарту присуще понимание мира как машины~--- гигантская система тонко сконструированных машин. Это снимает кучу ответственности. Мир устроен единобразно. Растения и животные~--- механизм. Да и человек во многом: его поведение не так уж и сложно. Такие вот замечательные механизмы, что есть иллюзия целесообразности. А в животных нет души и мыслей. Познание~--- конструирование некой машины мира. Круг замкнулся: всё связалось. Это положительная сторона.
Негативный момент. Есть. Дуализм.

\begin{enumerate}
	\item Конец
	\item Ооо
	\item Ооо
	\item Ооо
\end{enumerate}
