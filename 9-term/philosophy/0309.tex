\lectiondate{9 марта}

Сегодня пришлют материалы по Гегелю.
23 марта --- контрольная работа. Войдёт всё пройденное. Будут все: Юм, Кант (основная трудность), Фихт и Шиллинг, Гегель.

Продолжаем Канта. Рассмотрели учение о чувственности «Трансцендентальная эстетика». Выяснили, что математика возможно как наука лишь потому, что чувственный синтез связан с априорными формами: пространством и временем.

Вспомнили про 12 категорий. Тут происходит синтез чувственного созерцания и априорных категорий. Всё это выливается в суждение, а после формулируются законы (законы теоретического естествознания, в частности, физики).

Кант хочет выяснить, способна ли метафизика на суждение. По Канту, разум --- способность формулировать умозаключение, производя операции с суждениями, порождает идеи. Рассудок же формулирует суждение, оперирует категориями.
В процессе познания, есть стремление человека к максимуму синтеза, к охвату как можно большего познания. Но сфера познаваемого ограничена опытом, а опыт неполон: он всё время расширяется.
Разум как познавательная способность испытывает неудовлетворенность от того, что он вынужден всё время натыкать на пределы того, куда ему двигаться. Он стремится выйти за рамку сегодняшнего опыта, стремится к безусловному синтезу.
Возникает идея трех видов.
\begin{enumerate}
	\item идея душа, как идея всего психического. Отсюда возникает идея души --- вечной и неподвластной времени.
	\item идея мира в целом, как универсального синтеза всех физических явлений.
	\item идея бога, как абсолютно необходимого существа. Связывает психическое и физическое.
\end{enumerate}
Во всех 3 случаях прослеживается идея метафизики.

Кант приходит к негативному выводу во всех 3 случаях. Метафизика не способна формулировать синтетические априорные суждения. Потому что деятельность разума ошибочна: разум использует идеи конститутивно, как категории, применяя их к опыту. Правильно же использовать регулятивно: как некие предельные понятия, как цели, направляющие деятельность, как ориентиры.

Как именно мир в целом устроен. Антиномия чистого разума. Есть тезисы и антитезисы.

Фихт, Шеллинг и кто-то еще боготворили диалектику, говоря, что это круто.

Кант же рассматривал негативную диалектику: столкновение тезиса и антитезиса заставляет наш разум блуждать, попадать в ловушки, совершить ошибки. Считает, что антиномии неразрешимы: нельзя снять противоречие этих двух суждений. Это связано с тем, что разум выходит за рамки опыта, в то время как теоретическая деятельность должна базироваться на опытных данных.
Кант говорит, то разуму свойственен подобный путь, что он всегда так делает, с самой античности. 

Ещё интересно посмотреть на тезисы и антитезисы вертикально: объединить все тезисы, например. Получим, что разум преследует некий интерес: если собрать все тезисы -- получим Платона: рационалистическую философию. Антитезисы: линии Эпикура, Демокрита, интересуют детали устройства мира, решение различных технических задач.

Порождение трансцендентальной иллюзии.

Ещё стоит отметить, что Кант резко критикует доказательства бытия бога. Существовало 6: 5 физико-телеологического и самое первое онтологическое. Кант показывает их ошибочность. Опровергает (все?) существовавшие подходы к доказательству бытия бога.

3 и 4 категории. Верны и тезисы, и антитезисы, но в разных плоскостях.
В 3 антиномии --- намек на возможность иного (нетеоретического) разума.

Кант понимает человека как существо двойственно: принадлежит миру природы, подчиняется естественным законам, но при этом стоит выше: формулирует законы и принципы, не имеющие под собой ничего физического. Т.е. одновременно и явление, и вещь в себе.

Метафизика наукой быть не может: она связана с верой: ученый философ выбирает философию либо Платона, либо Демокрита в силу своей собственной веры в возможность познания и своих целей.

Кант отрицает возможность рационального доказательства бога. И предлагает моральное (см. «Мастера и Маргариту»). Итак, 7-е доказательство.

Человек свободен, в этом и заключается мораль. Ответственность человека  за нравственную реализацию своего поступка и свою же безнравственность имеет силу лишь в том случае, если поступок может быть рассмотрен в глобальном масштабе. 
Если мораль связывать с миром явлений, то можно оправдать всё, что угодно («ну такая жизнь у него сложная была, что ж вы хотите», причем люди ведут себя по-разному в одних и тех же ситуациях: предать / не предать). Оценкой моральности выступает бог, нужно абсолютное мерило, ибо никакой человек корректной оценки дать не может.

Высшая нравственность 


