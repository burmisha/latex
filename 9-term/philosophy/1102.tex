\lectiondate{2 ноября}
Продолжаем знакомиться с эмпиризмом.
Джон Локк, в первую очередь.

В 17--18 веках доминировала парадигма механики в части объяснения всего и вся. Даже представления об отношениях людей и всего общества были основаны на механике. Гоббс считает, что проблемы общественного договора нет, что есть общие интересы людей, и они их преследуют. Но это же предельно упрощенная модель. Утверждение о простоте устройства мира и всего остального было очень сильным, это был оптимистический настрой: надо найти лишь правильный метод и применить его, решив все проблемы.
Переход в гражданское состояние сопровождается выпадением из некоторых натуральных связей. Люди рискуют и предоставлены сами себе: нет никаких гарантий, надо самим придумывать законы и т.\,п.
Мнение Гоббса: тип управления страной (монархия/республика\ldots и т.\,п.) зависит лишь от числа людей в верхушке, а <<неправильные>> формы (~тирания) - лишь ругательства да обзывательства неугодных правительств. Нет плохих форм: всяко лучше, чем без них.
Гоббс и Локк не признавали никакой природной справедливости, считали, что для этого нужна выработанная система отношений.

Эпистомология Локка. Что??!
Исследовал проблему познания в философско-психологическом ключе. Люди раньше не чувствовали особого давления традиций. Где моё я? Когда же я себя осознал, ведь я постоянно различен? Где я живу, на каком языке говорю и мыслю, чей это язык, кто меня научил, как выразить свой опыт? Нужен общий опыт. Очень сложно найти истоки формирования нашего образа мира, но в 18 веке философам казалось, что это не так уж сложно. Вот они и умерли всё, неудачники. Для эмпириста чувствующий опыт определяет всё. Гоббс считал, что мыслить~--- это складывать и вычитать. Кстати, философы 17--18 века под человеком подразумевали только просвященного европейца, т.\,е. были расистами, что не мешало им путешествовать и одновременно считать другие народы недоевропейцами или полными дикарями.

Учение познания. Сенсуализм. (Чувственный опыт~--- единственный источник наших знаний). 

Локк в своём сочинении пишет (название содержит что-то про understanding). <<Ваша задача знать не все, а лишь важное для поведения\ldots Если лезть в глубины, то это ни к чему не приводит, а лишь рождает ненужные споры>>. Это декартовцы хотели знать всё. А эти лишь желали получить максимум из постигаемого. Исследовать сам разум, функции, способности и пределы. Локк первым обратил внимание на пределы (его дело впоследствии продолжил Кант). 

После объяснения задач Локк посылает к позитивной концепции~--- критике. Критика теории врожденных идей и моральных принципов. Сама идея таких идей идет от Платона. Локк опровергает: показывает, как наши идеи естественно возникают. Кстати, а что такое понятие врожденности? Признаки врожденности (выдвинутые сторонниками) 
\begin{enumerate}
	\item Некоторые идеи первичны по отношению к остальным знаниям, вот они то и являются врожденными.
	\item Всеобщее согласие (но это легко опровергнуть даже в отношении логики, а тем более в морали, не все же согласны с нормами). Умолишенные и дети, например, массово нарушают законы логики.
	\item Эти знания скрыты в душе, мы их не понимаем. Ну это вообще несерьезный довод, если уж по-честному.
\end{enumerate}
	
Таким образом систему знаний нужно строить из индивидуального чувственного опыта.

Итак, к происхождению знаний по Локку. 
Нет ничего в разуме такого, что не было бы прежде в ощущениях. Разум Локк уподобляет чистой доске (или зеркалу). Пока чувственный опыт ничего не начнет печатать/писать там ничего не будет. Предметы оказывают механические воздействия на органы чувств. От взаимодействия возникают идеи, ибо имеем дело с не объектами, а с отпечатками. Локк говорит, что основные элементы чувственного опыта являются нам как некая данность, ибо они составляют базу нашего мира (однородная и нерасчлененная реальность). До опыта ничего нет и говорить тут не о чем. Но элементы нашего опыта гетерогенны: одни отчетливы и ясны, а другие тусклы и нейтральны.

Опыт бывает разным:
\begin{itemize}
	\item Внешний. Яркий, поступает принудительно. Опыт ощущений. Простые идеи.
	\item Внутренний. Опыт рефлексии. Есть результат действия ума над внутренним опытом на основе внешнего содержания. Можно компоновать, переставлять, формировать. О рефлексии зависит и отношение наше к соответствующим элементам опыта. Ибо рефлексия определяет эмоциональный фон и эмоциональную окраску опыта: что-то сомнительно, что-то вызывает доверие, нейтрально или же побуждает к действию.
\end{itemize}
	

Исходные элементы знаний~--- простые идеи. Разум волен изменять идеи, которые поступают
на это зеркало. Разум есть зеркало, и вся его деятельность~--- рефлексия. Нельзя что-то показать и надеяться, что ничего не изменится, ведь могут быть искажения. Рефлексия --- чисто механический процесс. Рефлексия много простых идей складывает и получает сложную идею некоторого объекта. Разум может сравнивать (Моттль тоже об этом говорил) простые идеи.
\begin{enumerate}
	\item Сумма простых идей.
	\item Идея отношений (сравнений) 
	\item Общие идеи. Это не свойство существования объектов общего типа. По Локку они возникают в результате абстрагирования: при отнимании от простых идей обстоятелств времени и места.
\end{enumerate}
	
Эта классификация не по способу возникновения, а по степени устойчивости. Ибо реальность динамична: что-то устойчиво, а что-то не слишком.
\begin{enumerate}
	\item Идеи~--- субстанции. Субстанции~--- устойчивые совокупности идей, существующие самостоятельно (в отношении которых так полагают, что они, якобы, устойчивы) 
	\item Модусы. Модусы~--- сложные идеи, которые мы считаем зависящими от самостоятельных образований
\end{enumerate}
	
Качества (см. Бойля) 
\begin{itemize}
 	\item Первичные. Плотность, протяженность форма, покой или движение, объем и число. Это "реальные' сущности. Идеи их сходны с телами. Локк считает их объективно существующими. Изучаются точными науками.
	\item Вторичные. Цвет, вкус, запах, звук, температура. <<Номинальные>> сущности. Вызывают идеи, не имеющие прямого сходства с телами. Не существуют объективно. Качества зависят от первичных и диспозициональны (относительны). Реализуются при наличии ряда условий: для восприятия цвета необходимы сам предмет, свет, нормальное зрение человека. Носят субъективный характер (хотя Локк этого явно не утверждал, но его последователи называли). 
\end{itemize} 

И рационалистам, и эмпиристам очень тяжело объяснить, откуда возникают ошибки в познании. Неоткуда в этих теориях взяться ошибкам. Ибо теории очень просты.

Локк ассоциирует простые идеи в сложные. Значения слов~--- идеи.

\ldots

Идея материальной и духовной субстанции. Декарт то их трактовал как носителей качеств и вместилище всех свойств. Локк же опасался столь широко использовать понятие субстанции, считал, что у нас нету его. Локк пользуется бритвой Оккама: говорит, что такой объект не нужен, ибо мы с ним не имеем дела вовсе. То, что есть такое слово, еще ничего не значит. (Так-так, а что он делал со словами <<бог>> и т.\,п.? Ответ: он был дипломатичен и не смешивал). 

Следующий этап. Уже совсем неинтересный. Разделение знаний на виды и степени достоверности. Локк тут был не шибко оригинален

\begin{enumerate}
	\item Чувственное знание отдельных вещей. Наименее достоверный вид знаний
	\item Демонстративный (доказательный) характер. Здесь лежит большинство знаний. Ум опосредованно (с помощью рассуждений) воспринимает соответствие и несоответствие идей. Непротиворечивость описаний
	\item Интуитивные знания. Самые достоверные. Скорее, это дань традиции. Локк упрощенно трактует интуицию (никакого религиозного экстаза). 3>2, белое не черное, образ группы изоморфен фактор группе по ядру гомоморфизма.
\end{enumerate}
	
Скептические моменты в эмпиризме далее нарастают.
