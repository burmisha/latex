\documentclass[12pt]{article}
\usepackage[utf8]{inputenc}
\usepackage[T2A]{fontenc}
\usepackage{graphics,graphicx,epsfig}
\usepackage{color}
\usepackage[english,russian]{babel}
\usepackage{amssymb,amsfonts,amsthm,mathtext,cite,enumerate,float}
\usepackage[sumlimits, intlimits]{amsmath}
\usepackage{enumitem} \input macro.tex
\usepackage{textcase} 

\definecolor{linkcolor}{RGB}{7,31,63}%{15,80,120}

\textheight=24cm		\textwidth=16cm
\oddsidemargin=0mm 		\evensidemargin=0mm
\topmargin=-2,5cm
\parindent=24pt 		\parskip=3pt 
\footnotesep=3ex
\raggedbottom %\flushbottom
\clubpenalty=10000		\widowpenalty=10000 	\tolerance=500
\renewcommand{\baselinestretch}{1.25}

\setlist{nolistsep}

\begin{document}
	\maintext{Фундаментальная научная проблема, на анализ и обобщение результатов по которой, направлен проект} \\
	Объектом исследования данного проекта является выявление закономерностей и взаимосвязей в реально существующих объектах или событиях между совокупностью их явных (наблюдаемых) характеристик-признаков, с одной стороны, и свойств, скрытых от непосредственного наблюдения, с другой. 
	В данном проекте изучается возможность определения принадлежности объектов некоторому множеству. 
	В таком случае возможно говорить о задаче, как о задаче обучения распознавания образов или задаче классификации, при этом найденная зависимость называется решающим правилом. 
	Традиционно различают обучение с учителем (по прецедентам) и без учителя. 
	В данной работе используется метод обучения с учителем, при котором существует конечное число объектов, для которых известны и наблюдаемые, и скрытые характеристики (т.\,е. принадлежность классу). 
	Множество таких объектов будем называть обучающей выборкой.

	Методы обучению распознаванию образов развиваются очень быстро, однако одновременно с этим процессом интенсивно растет область желаемого применения этих алгоритмов. 
	При этом происходит наложение 4 процессов:
	\begin{itemize}
		\item рост требований к результату в уже существующих областях,
		\item увеличение размеров обучающих выборок с ростом самих областей применения,
		\item применение в областях, где ранее подобный подход не применялся,
		\item появление совершенно новых областей, где вовсе нет классических подходов.
	\end{itemize}
	При этом необходимо как можно точнее понимать, в каких областях какие подходы применимы, а также какие требования необходимы для корректного использования методов.
	Одновременно с ростом размеров обучающих выборок происходит их усложнение. 
	Хорошим и практически важным примером задачи распознавания образов является классификация электронных писем и сообщений на предмет наличия в них спама. 
	При попытке решения этой задачи возникают 3 существенных затруднения. 
	Во-первых, постоянный рост использования электронных писем и всё более высокое проникновение в жизнь людей непрерывно увеличивают как количество изучаемых объектов, так и их разнородность.
	Во-вторых, технологии рассылки спама непрерывно совершенствуются с целью обхода уже существующих фильтров, что также увеличивает количество разнородных объектов.
	В-третьих, существенные, но тяжело формализуемые, отличия спам-писем, от полезных. 
	При этом шаблонность письма вовсе не означает его вредоносность, поскольку это может быть рассылкой от одного из используемых сервисов.
	В-четвертых, возникающие трудности в составлении обучающей выборки. Для традиционных алгоритмов обучения с учителем является очень важным достаточная широта представленных в обучении объектов, однако в данном случае классы существенно неравноправны, по следующим причинам:
	\begin{itemize}
	 	\item полное отсутствие характеристик, присущих всем полезным письмам, и их очень высокая разнородность,
	 	\item спам-письма не скрываются людьми, тогда как полезные охраняются законом и менее доступны,
	 	\item как следствие, доступные обучающие выборки содержат достаточное количество спам-писем, однако слишком бедны для описания полезных писем.
	\end{itemize}

	Аналогичными свойствами обладают задачи диагностики заболеваний, поиска отклонений в работе технических устройств, т.\,е. в задачах с существенно неравноправными классами.

	Поэтому возникает \MakeTextUppercase{фундаментальная научная проблема разработки алгоритмов и методов обучения распознаванию образов в задачах с существенно неравноправными классами.}
\end{document} 