\documentclass[12pt]{article}
\usepackage[utf8]{inputenc}
\usepackage[T2A]{fontenc}
\usepackage{graphics,graphicx,epsfig}
\usepackage{color}
\usepackage[english,russian]{babel}
\usepackage{amssymb,amsfonts,amsthm,mathtext,cite,enumerate,float}
\usepackage[sumlimits, intlimits]{amsmath}
\usepackage{enumitem} \input macro.tex

\definecolor{linkcolor}{RGB}{7,31,63}%{15,80,120}

\textheight=24cm		\textwidth=16cm
\oddsidemargin=0mm 		\evensidemargin=0mm
\topmargin=-2,5cm
\parindent=24pt 		\parskip=3pt 
\footnotesep=3ex
\raggedbottom %\flushbottom
\clubpenalty=10000		\widowpenalty=10000 	\tolerance=500
\renewcommand{\baselinestretch}{1.25}

\setlist{nolistsep}

\begin{document}
	\maintext{Фундаментальная научная проблема, на анализ и обобщение результатов по которой, направлен проект} \\
	Объектом исследования данного проекта является выявление закономерностей и взаимосвязей в реально существующих объектах или событиях между совокупностью их явных (наблюдаемых) характеристик-признаков, с одной стороны, и свойств, скрытых от непосредственного наблюдения, с другой. 
	В данном проекте изучается возможность определения принадлежности объектов некоторому множеству. 
	В таком случае возможно говорить о задаче, как о задаче обучения распознавания образов или задаче классификации, при этом найденная зависимость называется решающим правилом. 
	Традиционно различают обучение с учителем (по прецедентам) и без учителя. 
	В данной работе используется метод обучения с учителем, при котором существует конечное число объектов, для которых известны и наблюдаемые, и скрытые характеристики (т.\,е. принадлежность классу). 
	Множество таких объектов будем называть обучающей выборкой.
\end{document} 