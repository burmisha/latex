\documentclass[unicode,lefteqn,c,hyperref={pdfpagelabels=false}]{beamer}
\usepackage[utf8]{inputenc}
\usepackage{amssymb}
\usepackage{amsmath,mathrsfs}
\usepackage[russian]{babel}
\usepackage{ulem}\normalem
\usepackage{color}

\input macro.tex

\usetheme{Warsaw}
\usefonttheme[onlylarge]{structurebold}
\setbeamerfont*{frametitle}{size=\normalsize,series=\bfseries}
\setbeamertemplate{navigation symbols}{}
\setbeameroption{show notes}
\definecolor{beamer@blendedblue}{RGB}{30,54,102}%{15,80,120}
\let\Tiny=\tiny
\def\shortspace{\hspace{1.5pt}}

%%%%%%%%%%%%%%%%%%%%%%%%%%%%%%%%%%%%%%%%%%%%%%%%%%%%%%%%%%%%%%%%%%%%%%%%%%%%%%%
\title[\hbox to 56mm{Классификация поискового спама\hfill\insertframenumber\,/\,\inserttotalframenumber}]{Классификация поискового спама по временным характеристикам сайта}
\author{М.\shortspaceО.\shortspaceБурмистров, Л.\shortspaceН.\shortspaceСандуляну} % [М.\shortspaceО.\shortspaceБурмистров, Л.\shortspaceН.\shortspaceСандуляну]
\institute{
		%Научный руководитель с.н.с. ВЦ РАН д.ф.-м.н.
		%\vfill К.\,В.~Воронцов \vfill ~ 
		\vfill Московский физико-технический институт
		\vfill Факультет управления и прикладной математики
		\vfill Кафедра интеллектуальных систем}
\date{\today}
%%%%%%%%%%%%%%%%%%%%%%%%%%%%%%%%%%%%%%%%%%%%%%%%%%%%%%%%%%%%%%%%%%%%%%%%%%%%%%%
\begin{document}
\begin{frame}
    \titlepage
\end{frame}
%%%%%%%%%%%%%%%%%%%%%%%%%%%%%%%%%%%%%%%%%%%%%%%%%%%%%%%%%%%%%%%%%%%%%%%%%%%%%%%
%%%%%%%%%%%%%%%%%%%%%%%%%%%%%%%%%%%%%%%%%%%%%%%%%%%%%%%%%%%%%%%%%%%%%%%%%%%%%%%
\section{Введение}
%%%%%%%%%%%%%%%%%%%%%%%%%%%%%%%%%%%%%%%%%%%%%%%%%%%%%%%%%%%%%%%%%%%%%%%%%%%%%%%
\subsection{Данные и вероятностные гипотезы}
%%%%%%%%%%%%%%%%%%%%%%%%%%%%%%%%%%%%%%%%%%%%%%%%%%%%%%%%%%%%%%%%%%%%%%%%%%%%%%%
\begin{frame}{Исходные данные задачи}
    \textbf{Пространства объектов и признаков}
    \begin{itemize}
    	\item $S$~--- конечное множество объектов (сайтов);
    	\item $X$~--- конечное множество наблюдаемых характеристик;
   		\item $Y = \{-1, 1\}$~--- метка классов <<спам / не спам>>;% (мнение наблюдателя);
        % \item $Z = Н$~--- поведение владельца сайта (скрытая характеристика);
		\item $T =\fbr{t_i}_{i=1}^{\tau}$~--- конечное линейно упорядоченное множество моментов времени.
    \end{itemize}

    \smallskip
    \textbf{Исходные данные об объектах} \par
	Каждому объекту $\foral{s\in S}$ сопоставлены
    \begin{itemize}
        %\item множество $T_s\subset T$
        \item совокупность его наблюдаемых характеристик $\x_s \in X^{\tau}$ % плохие обозначения
        \item совокупность ответов $\y_s \in Y^{\tau}. $
    \end{itemize}

    \smallskip
    \textbf{Задача:} \par
	Построить классификатор
	\[
		a\colon X\times T \to Y.
	\] 	
\end{frame}

\begin{frame}{Вероятностные предположения и модель}
    \textbf{Гипотеза о существовании генеральной совокупности}
    \begin{itemize}
        \item изучаем пары $\cbr{\omega, t} \in \Omega\times T$;
        \item $\foral{\omega_1 \ne \omega_2 \in \Omega, t_1, t_2 \in T} \p{\omega_1, t_1} \text{ и }\p{\omega_1, t_1} \text{ независимы}$;
    \end{itemize}

    \textbf{Гипотеза о влиянии наблюдателя (и верности наблюдений)}
    \begin{itemize}
    	\item $\p{\x_s(t_i)\cond\omega} = \p{\x_s(t)=\chi\cond \x_s(t_{i-1}), \y_s(t_{i-1})}$;
        \item $X=\hat X \sqcup \tilde X\colon$ \[\p{\hat{\x}_s(t)=\chi\cond \x_s(t_{i-1}), \y_s(t_{i-1})} = \p{\hat{\x}_s(t)=\chi\cond \hat{\x}_s(t_{i-1})}.\]
    \end{itemize}
\end{frame}



\end{document}
