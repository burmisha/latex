И так для нахожения значений $\mb a$ и $R$ необходимо решить следующую задачу
\begin{equation}
			\begcas{
			&R^2 + C\suml_i\xi_i \to \min\limits_{\mb a, R,\mb{\xi}}, \\
			&||\mb x_i-\mb a||^2\le R^2 + \delta_i,\;\;\delta_i\ge0,\;\;i = 1,\ldots,N.
			} 
\end{equation}
Функция Лагранжа этой задачи имеет вид
\al{
L(\mb a, R,\mb{\xi},\mb{\alpha},\mb{\gamma})=
	R^2 + C\suml_i\xi_i-\suml_i\gamma_i\delta_i-\suml_i\alpha_i\fbr{R^2 
	+ \delta_i-\cbr{\mb x_i\cdot\mb x_i-2\mb a\cdot\mb x_i+\mb a\cdot\mb a}},
}
где $\alpha_i\ge 0$  и $\gamma_i\ge 0$ множетели Лагранжа.
Необходимым условием минимума является равенство нулю частных производных функции Лагранжа по всем переменным

\begin{equation}
\label{minLag}
\begin{array}{ll}
\dd LR = 0: & \suml_i \alpha_i = 1\\
\dd L{\mb a} = 0: & \mb a = \frac{\suml_i\alpha_i\mb x_i}{\suml_i \alpha_i}= \suml_i \alpha_i\mb x_i\\
\dd L{\delta_i} = 0: & \gamma_i = C - \alpha_i, \;\; i = 1,\ldots,N.\\
\end{array}
\end{equation}

Из последнего уравнения получаем, что $\alpha_i = C - \gamma_i$. Тем самым мы получаем новые ограничения на $\alpha_i$ 
$$0\le\alpha_i\le C, \;\; i = 1,\ldots,N.$$
Если это ограничение выполнено, то мы можем вычислить $\gamma_i$ по формуле $\gamma_i = C - \alpha_i$ и автоматическибудет выполнено условие $\gamma_i\ge 0$.
Тогда для функции Лагранжа получим
\al{
L(\mb a, R,\mb{\xi},\mb{\alpha},\mb{\gamma})
	&= 	R^2 - \suml_i\alpha_iR^2 + C\suml_i\xi_i - \suml_i\alpha_i\delta_i 
	+ 	\suml_i\alpha_i\mb x_i\cdot\mb x_i - \\
	&- 2\suml_i\alpha_i\mb a\cdot\mb x_i + \suml_i\alpha_i\mb a\cdot\mb a +\suml_i\gamma_i\delta_i =\\
	&= 	\suml_i\alpha_i\mb x_i\cdot\mb x_i - 2 \suml_i\alpha_i \suml_j \alpha_j \mb x_j \cdot \mb x_i
	+ 	\suml_{i,j}\alpha_i\alpha_j \mb x_j\cdot \mb x_i = \\
	&= 	\suml_i\alpha_i\mb x_i\cdot\mb x_i-\suml_{i,j}\alpha_i\alpha_j\mb x_j\cdot\mb x_i
}
Полученная формула является квадратичной формой. Ее минимум находится по известным алгоритмам решения задач квадратичного программирования. По оптимальным значениям $\mb{\alpha}$ мы сможем найти оптимальное значение центра гиперсферы $\mb a$ и отступов $\mb{\xi}$  формулам (\ref{minLag}). Опорные векторы $\mb x_i$ для которых $\alpha_i=0$ и $\gamma_i=C$ лежат внутри гиперсферы, те для которых $0<\alpha_i<C$ и $<0\gamma_i<C$ ~--- на границе, а те для которых $\alpha_i=C$ и $\gamma_i=0$ лежат вне гиперсферы и имеют ненулевой отступ $\mb\xi$. Радиус $R$ определяется как расстояние от центра гиперсферы $\mb a$ до опорных векторов лежащих на границе гиперсферы.
