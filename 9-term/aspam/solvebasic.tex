И так для нахожения значений $\mb a$ и $R$ необходимо решить следующую задачу
\begin{equation}
			\begcas{
			&R^2 + C\sum\limits_{i}\xi_i \to \min\limits_{\mb a, R,\mb{\xi}}, \\
			&||\mb x_i-\mb a||^2\leq R^2 + \delta_i,\;\;\delta_i\geq0,\;\;i = 1,\ldots,N.
			} 
\end{equation}
Функция Лагранжа этой задачи имеет вид
$$L(\mb a, R,\mb{\xi},\mb{\alpha},\mb{\gamma})=R^2 + C\sum\limits_{i}\xi_i-\sum\limits_{i}\gamma_i\delta_i-$$
$$-\sum\limits_{i}\alpha_i\left\{R^2 + \delta_i-\left(\mb x_i\cdot\mb x_i-2\mb a\cdot\mb x_i+\mb a\cdot\mb a\right)\right\},$$
где $\alpha_i\geq 0$  и $\gamma_i\geq 0$ множетели Лагранжа.
Необходимым условием минимума является равенство нулю частных производных функции Лагранжа по всем переменным

\begin{equation}
\label{minLag}
\begin{array}{ll}
\frac{\partial L}{\partial R} = 0: & \sum\limits_{i} \alpha_i = 1\\
\frac{\partial L}{\partial \mb a} = 0: & \mb a = \frac{\sum\limits_{i}\alpha_i\mb x_i}{\sum\limits_{i} \alpha_i}= \sum\limits_{i} \alpha_i\mb x_i\\
\frac{\partial L}{\partial \delta_i} = 0: & \gamma_i = C - \alpha_i, \;\; i = 1,\ldots,N.\\
\end{array}
\end{equation}

Из последнего уравнения получаем, что $\alpha_i = C - \gamma_i$. Тем самым мы получаем новые ограничения на $\alpha_i$ 
$$0\leq\alpha_i\leq C, \;\; i = 1,\ldots,N.$$
Если это ограничение выполнено, то мы можем вычислить $\gamma_i$ по формуле $\gamma_i = C - \alpha_i$ и автоматическибудет выполнено условие $\gamma_i\geq 0$.
Тогда для функции Лагранжа получим
$$L(\mb a, R,\mb{\xi},\mb{\alpha},\mb{\gamma})=R^2-\sum\limits_{i}\alpha_iR^2 + 
C\sum\limits_{i}\xi_i-\sum\limits_{i}\alpha_i\delta_i+$$
$$+\sum\limits_{i}\alpha_i\mb x_i\cdot\mb x_i-2\sum\limits_{i}\alpha_i\mb a\cdot\mb x_i+\sum\limits_{i}\alpha_i\mb a\cdot\mb a +\sum\limits_{i}\gamma_i\delta_i=$$ 
$$ = \sum\limits_{i}\alpha_i\mb x_i\cdot\mb x_i-
2 \sum\limits_{i}\alpha_i \sum\limits_{j} \alpha_j \mb x_j\cdot \mb x_i+\sum\limits_{i,j}\alpha_i\alpha_j \mb x_j\cdot \mb x_i=$$
$$=\sum\limits_{i}\alpha_i\mb x_i\cdot\mb x_i-\sum\limits_{i,j}\alpha_i\alpha_j\mb x_j\cdot\mb x_i$$
Полученная формула является квадратичной формой. Ее минимум находится по известным алгоритмам решения задач квадратичного программирования. По оптимальным значениям $\mb{\alpha}$ мы сможем найти оптимальное значение центра гиперсферы $\mb a$ и отступов $\mb{\xi}$  формулам (\ref{minLag}). Опорные векторы $\mb x_i$ для которых $\alpha_i=0$ и $\gamma_i=C$ лежат внутри гиперсферы, те для которых $0<\alpha_i<C$ и $<0\gamma_i<C$ ~--- на границе, а те для которых $\alpha_i=C$ и $\gamma_i=0$ лежат вне гиперсферы и имеют ненулевой отступ $\mb\xi$. Радиус $R$ определяется как расстояние от центра гиперсферы $\mb a$ до опорных векторов лежащих на границе гиперсферы.
