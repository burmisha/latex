\documentclass[12pt,a4paper]{amsart}
\usepackage[utf8]{inputenc}
\usepackage[T2A]{fontenc}
\usepackage{graphics,graphicx,epsfig}
\usepackage{amssymb,amsfonts,amsthm,amsmath,mathtext,cite,enumerate,float}
\usepackage[english,russian]{babel}

\newcommand\bivec[2]{\begin{pmatrix} #1 \\ #2 \end{pmatrix}}

\newcommand\ol[1]{\overline{#1}}

\newcommand\p[1]{\ensuremath{\Prob\!\left(#1\right)}}
\def\cond{\,|\,}
\newcommand\e[1]{\mathsf{E}\!\left(#1\right)}
\newcommand\disp[1]{\mathsf{D}\!\left(#1\right)}
%\newcommand\norm[2]{\mathcal{N}\!\cbr{#1,#2}}
\newcommand\sign{\text{ sign }}

\newcommand\al[1]{\begin{align*} #1 \end{align*}}
\newcommand\begcas[1]{\begin{cases}#1\end{cases}}
\newcommand\tab[2]{	\vspace{-#1pt}
						\begin{tabbing} 
						#2
						\end{tabbing}
					\vspace{-#1pt}
					}


\newcommand\maintext[1]{{\bfseries\sffamily{#1}}}
\newcommand\simpletitle[1]{\begin{center} \maintext{#1} \end{center}}

\def\le{\leqslant}
\def\ge{\geqslant}
\def\Ell{\mathcal{L}}
\def\eps{\varepsilon}
\def\x{\ensuremath{\textbf{x}}}
\def\y{\ensuremath{\textbf{y}}}
\def\Rn{\ensuremath{\mathbb{R}^n}}
\def\RSS{\mathsf{RSS}}

\newcommand\mb[1]{\ensuremath{\mathbf{#1}}}
\newcommand\argmax[1]{\arg\underset{#1}\max\,} % \operatornamewithlimits
%\newcommand{\prodl}{\mathop{\textstyle\prod}\limits}
\newcommand{\prodl}{\prod\limits}
\newcommand{\suml}{\sum\limits}
\newcommand\foral[1]{\forall\,#1\:}
\newcommand\exist[1]{\exists\,#1\:\colon}

\newcommand\cbr[1]{\left(#1\right)} %circled brackets
\newcommand\fbr[1]{\left\{#1\right\}} %figure brackets
\newcommand\sbr[1]{\left[#1\right]} %square brackets
\newcommand\modul[1]{\left|#1\right|}
\newcommand\cdf[2]{\cdot\frac{#1}{#2}}
\newcommand\integr[3]{\int\limits_{#1}^{#2}{#3}}
\newcommand\obol[1]{O\!\cbr{#1}}
\newcommand\norm[1]{\ensuremath{\left\|{#1}\right\|}}

\newcommand\dd[2]{\frac{\partial#1}{\partial#2}}

\newcommand\addeps[2]{
	\begin{figure} [!ht] %lrp
		\centering
		\includegraphics[height=240px]{#1.eps}
		\vspace{-10pt}
		\caption{#2}
		\label{eps:#1}
	\end{figure}
}

\newcommand\addtikz[4]{
	\begin{figure} [!ht] %lrp
		\centering
		\begin{tikzpicture}[x=#2cm,y=#2cm,#3]
			\input{#1.tikz}
		\end{tikzpicture}
		\vspace{-10pt}
		\caption{#4}
		\label{tikz:#1}	
	\end{figure}
}



\newcommand\addepssize[3]{
	\begin{figure} [!ht] %lrp hp
		\centering
		\includegraphics[height=#3px]{#1.eps}
		\vspace{-10pt}
		\caption{#2}
		\label{eps:#1}
	\end{figure}
}

\def\algorithmicrequire{\textbf{Вход:}}
\def\algorithmicensure{\textbf{Выход:}}
\def\algorithmicif{\textbf{если}}
\def\algorithmicthen{\textbf{то}}
\def\algorithmicelse{\textbf{иначе}}
\def\algorithmicelsif{\textbf{иначе если}}
\def\algorithmicfor{\textbf{для}}
\def\algorithmicforall{\textbf{для всех}}
\def\algorithmicdo{}
\def\algorithmicwhile{\textbf{пока}}
\def\algorithmicrepeat{\textbf{повторять}}
\def\algorithmicuntil{\textbf{пока}}
\def\algorithmicloop{\textbf{цикл}}
% переопределение стиля комментариев
\def\algorithmiccomment#1{\quad// {\sl #1}}

\begin{document}
\section{Решение оптимизационной задачи}

И так для нахожения значений $\mb a$ и $R$ необходимо решить следующую задачу
\begin{equation}
			\begcas{
			&R^2 + C\sum\limits_{i}\xi_i \to \min\limits_{\mb a, R,\mb{\xi}}, \\
			&||\mb x_i-\mb a||^2\leq R^2 + \delta_i,\;\;\delta_i\geq0,\;\;i = 1,\ldots,N.
			} 
\end{equation}
Функция Лагранжа этой задачи имеет вид
$$L(\mb a, R,\mb{\xi},\mb{\alpha},\mb{\gamma})=R^2 + C\sum\limits_{i}\xi_i-\sum\limits_{i}\gamma_i\delta_i-$$
$$-\sum\limits_{i}\alpha_i\left\{R^2 + \delta_i-\left(\mb x_i\cdot\mb x_i-2\mb a\cdot\mb x_i+\mb a\cdot\mb a\right)\right\},$$
где $\alpha_i\geq 0$  и $\gamma_i\geq 0$ множетели Лагранжа.
Необходимым условием минимума является равенство нулю частных производных функции Лагранжа по всем переменным

\begin{equation}
\label{minLag}
\begin{array}{ll}
\frac{\partial L}{\partial R} = 0: & \sum\limits_{i} \alpha_i = 1\\
\frac{\partial L}{\partial \mb a} = 0: & \mb a = \frac{\sum\limits_{i}\alpha_i\mb x_i}{\sum\limits_{i} \alpha_i}= \sum\limits_{i} \alpha_i\mb x_i\\
\frac{\partial L}{\partial \delta_i} = 0: & \gamma_i = C - \alpha_i, \;\; i = 1,\ldots,N.\\
\end{array}
\end{equation}

Из последнего уравнения получаем, что $\alpha_i = C - \gamma_i$. Тем самым мы получаем новые ограничения на $\alpha_i$ 
$$0\leq\alpha_i\leq C, \;\; i = 1,\ldots,N.$$
Если это ограничение выполнено, то мы можем вычислить $\gamma_i$ по формуле $\gamma_i = C - \alpha_i$ и автоматическибудет выполнено условие $\gamma_i\geq 0$.
Тогда для функции Лагранжа получим
$$L(\mb a, R,\mb{\xi},\mb{\alpha},\mb{\gamma})=R^2-\sum\limits_{i}\alpha_iR^2 + 
C\sum\limits_{i}\xi_i-\sum\limits_{i}\alpha_i\delta_i+$$
$$+\sum\limits_{i}\alpha_i\mb x_i\cdot\mb x_i-2\sum\limits_{i}\alpha_i\mb a\cdot\mb x_i+\sum\limits_{i}\alpha_i\mb a\cdot\mb a +\sum\limits_{i}\gamma_i\delta_i=$$ 
$$ = \sum\limits_{i}\alpha_i\mb x_i\cdot\mb x_i-
2 \sum\limits_{i}\alpha_i \sum\limits_{j} \alpha_j \mb x_j\cdot \mb x_i+\sum\limits_{i,j}\alpha_i\alpha_j \mb x_j\cdot \mb x_i=$$
$$=\sum\limits_{i}\alpha_i\mb x_i\cdot\mb x_i-\sum\limits_{i,j}\alpha_i\alpha_j\mb x_j\cdot\mb x_i$$
Полученная формула является квадратичной формой. Ее минимум находится по известным алгоритмам решения задач квадратичного программирования. По оптимальным значениям $\mb{\alpha}$ мы сможем найти оптимальное значение центра гиперсферы $\mb a$ и отступов $\mb{\xi}$  формулам (\ref{minLag}). Опорные векторы $\mb x_i$ для которых $\alpha_i=0$ и $\gamma_i=C$ лежат внутри гиперсферы, те для которых $0<\alpha_i<C$ и $<0\gamma_i<C$ ~--- на границе, а те для которых $\alpha_i=С$ и $\gamma_i=0$ лежат вне гиперсферы и имеют ненулевой отступ $\mb{\xi}$. Радиус $R$ определяется как расстояние от центра гиперсферы $\mb a$ до опорных векторов лежащих на границе гиперсферы.




\end{document}