\newcommand\bivec[2]{\begin{pmatrix} #1 \\ #2 \end{pmatrix}}

\newcommand\ol[1]{\overline{#1}}

\newcommand\p[1]{\ensuremath{\Prob\!\left(#1\right)}}
\def\cond{\,|\,}
\newcommand\e[1]{\mathsf{E}\!\left(#1\right)}
\newcommand\disp[1]{\mathsf{D}\!\left(#1\right)}
%\newcommand\norm[2]{\mathcal{N}\!\cbr{#1,#2}}
\newcommand\sign{\text{ sign }}

\newcommand\al[1]{\begin{align*} #1 \end{align*}}
\newcommand\begcas[1]{\begin{cases}#1\end{cases}}
\newcommand\tab[2]{	\vspace{-#1pt}
						\begin{tabbing} 
						#2
						\end{tabbing}
					\vspace{-#1pt}
					}


\newcommand\maintext[1]{{\bfseries\sffamily{#1}}}
\newcommand\simpletitle[1]{\begin{center} \maintext{#1} \end{center}}

\def\le{\leqslant}
\def\ge{\geqslant}
\def\Ell{\mathcal{L}}
\def\eps{\varepsilon}
\def\x{\ensuremath{\textbf{x}}}
\def\y{\ensuremath{\textbf{y}}}
\def\Rn{\ensuremath{\mathbb{R}^n}}
\def\RSS{\mathsf{RSS}}

\newcommand\mb[1]{\ensuremath{\boldsymbol{\mathbf{#1}}}}
\newcommand\argmax[1]{\arg\underset{#1}\max\,} % \operatornamewithlimits
%\newcommand{\prodl}{\mathop{\textstyle\prod}\limits}
\newcommand{\prodl}{\prod\limits}
\newcommand{\suml}{\sum\limits}
\newcommand\foral[1]{\forall\,#1\:}
\newcommand\exist[1]{\exists\,#1\:\colon}

\newcommand\cbr[1]{\left(#1\right)} %circled brackets
\newcommand\fbr[1]{\left\{#1\right\}} %figure brackets
\newcommand\sbr[1]{\left[#1\right]} %square brackets
\newcommand\modul[1]{\left|#1\right|}
\newcommand\cdf[2]{\cdot\frac{#1}{#2}}
\newcommand\integr[3]{\int\limits_{#1}^{#2}{#3}}
\newcommand\obol[1]{O\!\cbr{#1}}
\newcommand\norm[1]{\ensuremath{\left\|{#1}\right\|}}

\newcommand\dd[2]{\frac{\partial#1}{\partial#2}}

\newcommand\addeps[2]{
	\begin{figure} [!ht] %lrp
		\centering
		\includegraphics[height=240px]{#1.eps}
		\vspace{-10pt}
		\caption{#2}
		\label{eps:#1}
	\end{figure}
}

\newcommand\addtikz[4]{
	\begin{figure} [!ht] %lrp
		\centering
		\begin{tikzpicture}[x=#2cm,y=#2cm,#3]
			\input{#1.tikz}
		\end{tikzpicture}
		\vspace{-10pt}
		\caption{#4}
		\label{tikz:#1}	
	\end{figure}
}



\newcommand\addepssize[3]{
	\begin{figure} [!ht] %lrp hp
		\centering
		\includegraphics[height=#3px]{#1.eps}
		\vspace{-10pt}
		\caption{#2}
		\label{eps:#1}
	\end{figure}
}

\def\algorithmicrequire{\textbf{Вход:}}
\def\algorithmicensure{\textbf{Выход:}}
\def\algorithmicif{\textbf{если}}
\def\algorithmicthen{\textbf{то}}
\def\algorithmicelse{\textbf{иначе}}
\def\algorithmicelsif{\textbf{иначе если}}
\def\algorithmicfor{\textbf{для}}
\def\algorithmicforall{\textbf{для всех}}
\def\algorithmicdo{}
\def\algorithmicwhile{\textbf{пока}}
\def\algorithmicrepeat{\textbf{повторять}}
\def\algorithmicuntil{\textbf{пока}}
\def\algorithmicloop{\textbf{цикл}}
% переопределение стиля комментариев
\def\algorithmiccomment#1{\quad// {\sl #1}}