\begin{abstract}
Решается задача одноклассовой классификации электронных писем на предмет наличия в них спама. В работе вводится квазивероятностная модель для классической эмпирической постановки задачи одноклассовой классификации. Произведена модификация, позволяющая накладывать различные требования отбора признаков. Построенные методы классификации проверяются вычислительными экспериментами на модельных и реальных данных.
%Решается задача автоматического разделения текстов по тематикам. Рассмотрены два подхода к решению задачи: вероятностный латентный семантический анализ и алгоритм латентного размещения Дирихле, основанные на различных вероятностных предположениях о текстах, однако обладающие схожей вычислительной техникой. Произведено сведение алгоритмов и их модификаций к новой обобщающей вычислительной схеме и построен новый алгоритм. Проанализировано качество и скорость сходимости построенного алгоритма в зависимости от внутренних параметров и числа тем, ассоциируемых с каждым словом в тексте. Результаты подтверждены численным экспериментом на реальных текстах. 
\end{abstract}