\title{Вероятностная модель одноклассовой классификации}
\author{ Бурмистров М.\,О., Сандуляну Л.\,Н.}
\organization{Московский физико-технический институт, ФУПМ, каф. <<Интеллектуальные системы>>}
\thanks{Научный руководитель О.\,В.\,Красоткина}
\email{burmisha@gmail.com, liubov.sanduleanu@gmail.com}
\date{декабрь 2012\,г.}

\abstract{
	Решается задача одноклассовой классификации электронных писем на предмет наличия в них спама.
	В работе вводится квазивероятностная модель для классической эмпирической постановки задачи одноклассовой классификации и
	производится сведение классического подхода к новой модели.
	%Произведена модификация, позволяющая накладывать различные требования отбора признаков.
	Построенные методы классификации проверяются вычислительными экспериментами на модельных и реальных данных.

	\bigskip\textbf{Ключевые слова}:
	\emph{одноклассовая классификация, вероятностная модель, байесовский подход, ядерные функции}.
}

\titleEng{Probabilistic model for one-class classification problem}
\authorEng{Burmistrov M.\,O., Sanduleanu L.\,N.}
\organizationEng{Moscow Institute of Physics and Technology}

\abstractEng{
	One-class classification methods are used to test e-mails for spam.
	Quasi-probabilistic model is introduced for traditional empirical approach to problem.
	The old model is shown to be a reduction of the new one.
	Built approaches to classification are numerically tested on model and real data.

	\bigskip\textbf{Keywords}:
	\emph{one-class classification, probabilistic model, Bayesian approach, kernel functions}.
}
