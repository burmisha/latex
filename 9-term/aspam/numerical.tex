% В эксперименте, иллюстрацией к которому служит рисунок \ref{eps:modelData}, в качестве исходного набора точек выступала смесь из 200 точек из нормального распределения с нулевым математическим ожиданием и единичной матрицей ковариации и 10 точек из нормального распределения с математическим ожиданием $\bivec55$ и также единичной матрицей ковариации. Величина $C = 0{,}025.$

% \addeps{modelData}{Модельные данные}

При оценке качества работы алгоритма необходимо ввести некую метрику. 
Следуя работам  \cite{Romanenko2012}, будем измерять качество одноклассовой класификации в терминах точности и полноты. 
В нашем случае точность (precision)~--- доля верно классифицированных объектов тестовой выборки среди всех объектов, отнесенных алгоритмом к единственному классу. Полнота (recall)~--- доля верно классифицированных объектов тестовой выборки среди всех объектов, принадлежащих к единственному классу. Более высокие значения точности и полноты соответствуют лучшему качеству классификации. В качестве агрегированного показателя, объединияющего точность $P$ и полноту $R$ используем $F_1$-меру \cite{Rijsbergen1979}:
$$F_1 = \frac{2PR}{P+R}.$$

Для проведения вычислительного эксперимента сгенерируем $N=400$ случайных точек $\fbr{\mb x_i}_{i=1}^N$ из распределения (\ref{PhiXARC}) при размерности пространства $2$ (для наглядности), положив направления смещений случайными и придав параметрам значения $a = \cbr{1,2}\T, R = 3, c = 0{,}2.$
После этого проведем tq-fold кросс-валидацию с $t = 10, q = 3$, скользящим контролем подбирая параметр $C$ и вычисляя $F_1$-метрику при каждом его значении. В результате получим следующую зависимость значения метрики от $C$: см. рис. \ref{eps:CrValLong}.
Из графика видно, что при $C\to 0$ обобщающая способность также стремится к нулю, поскольку практически отсутствует штраф за непопадание в класс при обучении. При этом большие штрафы заставляют необоснованно увеличивать сферу, снижая точность.

\addeps{CrValLong}{Зависимость $F_1$-метрики от параметра регуляризации $C$} 
% CV_0.000-0.010-0.200_T10_Q3.eps