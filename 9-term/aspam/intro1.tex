Задача отсеивния поискового спама в наше время является весьма актуальна. Современный интернет обладет широкой, распределенной и сложной архитектурой, в которой зачатую затруднительно найти требуемую информацию. Помочь пользователю решить задачу поиска призваны поиковые системы, которые автоматически сканируют интернет и выявляют наиболее подходящие пользователю сайты по некоторым ключевым словам. Однако, злоумышленники пытаются вывести подконтрольные им сайты в топ поисковой выдачи, искуственно изменяя характеристики сайта, видимые для поисковой машины. Такие действия ухудшают качество поиска и опасны для пользователя, поэтому необходимо либо полностью блокировать такие сайты, либо существенно опускать их в выдаче.

%Алгоритм определения степени соответствия сайта запросу пользователя основан на множестве характеристик, а владельцы ресурсов заинтересованы, чтобы поискове системы как можно выше оценивали их сайт. Зачастую

Данная задача не может быть рассмотрена как задача бинарной классификации, так как знание поисковой машины характерисик сайта влияет на эти характеристики. Например, если сайт признан по какой-либо причине опасным и удален из выдачи, у него резко снижается посещаемость, при этом даже если сайт исправит свою проблему, но не разбаниться автоматически, посещаемость все равно остантся низкой. Одновременно с этим существуют признаки, такие как количество переходов, которые, по-видимому, отражают степень полезности сайта, однако слабо зависят от действий поисковой системы. В работе также рассмотрена задача нахождения признаков, не зависящих от поисковой машины.

Идея: пусть каждый сайтовладелец в каждый момент времени $t\in T$ (см. презентацию) думает: сделать ему сайт хуже или лучше (спамерское поведение или, напротив, добропорядочное --- это и есть на самом деле класс $y\in Y$). В зависимости от этого он меняет характеристики своего сайта (тут чуть веселее, на самом деле: он одинаково меняет характеристики {\it всех} сайтов, которыми владеет, а это можно узнавать по whois-данным. {\it но я думаю, что в работе с этим заморачиваться не будем}). В таком случае наблюдаемые характеристики сайта (те, которые зависят) есть, на самом деле, функция от его наблюдаемого класса (то, что люди видят глазами). Надо додумать. Статей по теме не находил (да и не искал: придумал поздно вечером).