С широким развитием сети интернет и её проникновением в большую часть всех сфер жизни, у людей появилась возможность свободно обмениваться информацией и получать доступ к разнообразным ресурсом. 
Одним из наиболее распространенных способов общения людей через интернет является использование электронной почты. 
В силу большой открытости этого канала связи с точки зрения возможности передачи любого сообщения произвольному пользователю он активно используется мошенниками, злоумышленниками и распространителями рекламных материалов. При этом создается не только повышенная нагрузка на техническую инфраструктуру, но и тратится время людей, которым приходится отделять полезную информацию, от всей остальной. 
Поэтому задача автоматизации фильтрации электронной почты будет оставаться актуальной в течение всего времени её существования.

Задача фильтрации спама уже решалась многими методами \cite{Islam2007, Sun2008}, однако они в большой степени являлись эвристическими и не имели под собой четкой вероятностной модели. 
Также проблемой является корректное составление обучающей выборки. 
Дело в том, что спам-письма зачастую шаблонны и имеют много общего в своей структуре, к тому же они широко доступны. 
Составить же обучающую выборку, содержащую письма, полезные для пользователей, гораздо сложнее по следующим причинам:
\begin{itemize}
	\item меньшая доступность,
	\item высокая разнородность,
	\item большое число шаблонных писем (разнообразные уведомления от сервисов).
\end{itemize}
По этим причинам предлагается использовать методы одноклассовой классификации \cite{Tax2001, Khan2006}, чтобы отказаться от требования к обучающей выборки содержать достаточно широкое множество разнообразных представителей обоих классов.

В работе будет предложена квазивероятностная постановка задачи одноклассовой классификации. 
За счет такого подхода становятся яснее области применимости построенной модели и предъявляемые требования к данным.
% Поскольку количество признаков, которые можно извлечь из текстов спам-писем, очень велико, то предлагается применить отбор признаков. 
На основе полученной вероятностной постановки задачи, строится новая вероятностная модель порождения объектов, в ходе оптимизации которой происходит построение классификатора. %требуемый отбор признаков.

Полученные методы построения одноклассовых классификаторов применяются к модельным и реальным данным.
