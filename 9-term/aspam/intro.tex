С широким развитием сети интернет и её проникновением в большую часть всех сфер жизни, у людей появилась возможность свободно обмениваться информацией и получать доступ к разнообразным ресурсом. 
Одним из наиболее распространенных способов общения людей через интернет является использование электронной почты \cite{}. 
В силу большой открытости этого канала связи с точки зрения возможности передачи любого сообщения произвольному пользователю он активно используется мошенниками, злоумышленниками и распространителями рекламных материалов. При этом создается не только повышенная нагрузка на техническую инфраструктуру, но и тратится время людей, которым приходится отделять полезную информацию, от всей остальной \cite{}. 
Поэтому задача автоматизации фильтрации электронной почты будет оставаться актуальной в течение всего времени её существования.

Задача фильтрации спама уже решалась многими методами \cite{}, однако они в большой степени являлись эвристическими и не имели под собой четкой вероятностной модели. 
Также проблемой является корректное составление обучающей выборки. 
Дело в том, что спам-письма зачастую шаблонны и имеют много общего в своей структуре, к тому же они широко доступны. 
Составить же обучающую выборку, содержащую письма, полезные для пользователей, гораздо сложнее по следующим причинам:
\begin{itemize}
	\item меньшая доступность,
	\item высокая разнородность,
	\item большое число шаблонных писем (разнообразные уведомления от сервисов).
\end{itemize}
По этим причинам предлагается использовать методы одноклассовой классификации \cite{}, чтобы отказаться от требования к обучающей выборки содержать достаточно широкое множество разнообразных представителей обоих классов.

В работе будет предложена квазивероятностная постановка задачи одноклассовой классификации. 
За счет такого подхода становятся яснее области применимости построенной модели и предъявляемые требования к данным.

Поскольку количество признаков, которые можно извлечь из текстов спам-писем, очень велико, то предлагается применить отбор признаков. 
На основе полученной вероятностной постановки задачи, строится новая вероятностная модель порождения объектов, в ходе оптимизации которой происходит требуемый отбор признаков.

Полученные методы построения одноклассовых классификаторов применяются к модельным и реальным данным.

%Современный интернет обладает широкой, распределенной и сложной архитектурой, в которой зачатую затруднительно найти новую требуемую информацию. Помочь пользователю решить задачу поиска призваны поисковые системы, которые автоматически сканируют интернет и выявляют наиболее подходящие пользователю по некоторым ключевым словам. 

%Алгоритм определения степени соответствия сайта запросу пользователя основан на множестве характеристик, а владельцы ресурсов заинтересованы, чтобы поисковые системы как можно выше оценивали их сайт. Зачастую

%Злоумышленники пытаются вывести подконтрольные им сайты в топ поисковой выдачи, искусственно изменяя характеристики сайта, видимые для поисковой машины. Такие действия ухудшают качество поиска и опасны для пользователя, поэтому необходимо либо полностью блокировать такие сайты, либо существенно опускать их в выдаче.

%Задача могла бы рассматривать как традиционная задача бинарной классификации, если бы знание поисковой машины характеристик сайта не влияло на эти характеристики. Например, если сайт признан по какой-либо причине опасным и удален из выдачи, у него резко (в разы) снижается посещаемость, при этом сайт мог исправить свою проблему, но не разбанится автоматически и посещаемость останется низкой. Одновременно с этим существуют признаки (количество переходов), которые, по-видимому, отражают степень полезность сайта, однако слабо зависят от действий поисковой системы. Задача выделения признаков, не зависящих от поисковой машины также представляет интерес в нашем исследовании.

%Идея: пусть каждый сайтовладелец в каждый момент времени $t\in T$ (см. презентацию) думает: сделать ему сайт хуже или лучше (спамерское поведение или, напротив, добропорядочное --- это и есть на самом деле класс $y\in Y$). В зависимости от этого он меняет характеристики своего сайта (тут чуть веселее, на самом деле: он одинаково меняет характеристики {\it всех} сайтов, которыми владеет, а это можно узнавать по whois-данным. {\it но я думаю, что в работе с этим заморачиваться не будем}). В таком случае наблюдаемые характеристики сайта (те, которые зависят) есть, на самом деле, функция от его наблюдаемого класса (то, что люди видят глазами). Надо додумать. Статей по теме не находил (да и не искал: придумал поздно вечером).