%Современный интернет обладет широкой, распределенной и сложной архитектурой, в которой зачатую затруднительно найти новую требуемую информацию. Помочь пользователю решить задачу поиска призваны поиковые системы, которые автоматически сканируют интернет и выявляют наиболее подходящие пользователю по некоторым ключевым словам. 

%Алгоритм определения степени соответствия сайта запросу пользователя основан на множестве характеристик, а владельцы ресурсов заинтересованы, чтобы поискове системы как можно выше оценивали их сайт. Зачастую



Злоумышленники пытаются вывести подконтрольные им сайты в топ поисковой выдачи, искуственно изменяя характеристики сайта, видимые для поисковой машины. Такие действия ухудшают качество поиса и опасны для пользователя, поэтому необходимо либо полностью блокировать такие сайты, либо существенно опускать их в выдаче.

Задача могла бы рассматриать как традиционая задача бинарной классификации, если бы знание поисковой машины характерисик сайта не влияло на эти характеристики. Например, если сайт признан по какой-либо причине опасным и удален из выдачи, у него резко (в разы) снижается посещаемость, при этом сайт мог исправить свою проблему, но не разбаниться автоматически и посещаемость остантся низкой. Одновременно с этим существуют признаки (количество переходов), которые, по-видимому, отражают степень полезность сайта, однако слабо зависят от действий поисковой системы. Задача выделения признаков, не зависящих от поисковой машины также представляет интерес в нашем исследовании.

%Идея: пусть каждый сайтовладелец в каждый момент времени $t\in T$ (см. презентацию) думает: сделать ему сайт хуже или лучше (спамерское поведение или, напротив, добропорядочное --- это и есть на самом деле класс $y\in Y$). В зависимости от этого он меняет характеристики своего сайта (тут чуть веселее, на самом деле: он одинаково меняет характеристики {\it всех} сайтов, которыми владеет, а это можно узнавать по whois-данным. {\it но я думаю, что в работе с этим заморачиваться не будем}). В таком случае наблюдаемые характеристики сайта (те, которые зависят) есть, на самом деле, функция от его наблюдаемого класса (то, что люди видят глазами). Надо додумать. Статей по теме не находил (да и не искал: придумал поздно вечером).