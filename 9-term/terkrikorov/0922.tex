\simpletitle{22 сентября}
\al{
	A&\in \Ell[B_1 \to B_2] \\
	f&: O(x_0) \to B_2, O(x_0)\in B_1
}
\al{
	f(x)-f(x_0) &= A(x-x_0)+\eps(x)\norm{x-x_0},\\
	A &= f'(x_0), \\
	f'(x-x_0)(x-x_0) &= df(x-0, dx), dx=x-x_0
}

Производная Гато (слабая проивзводная): $d_c f(x_0, h) = \lim_{t\to +0}\frac{f(x_0+th) - f(x_0)}{t} = Ah. A = f'_c(x_0).$ 

Может и не быть линейным. Пример: 
$f(x,y) = \begcas{\frac{x^2}{x^2 + y^2}, x^2 + y^2 > 0 \\ 0, x=y=0}. $ $d_c f(0,0) = \frac{h_1^2}{h_1^2 + h_2^2}$

Утверждение. Если $f(x)$ непрер и имеет слаб производную на выпуклом множестве $M \subset B_1$, то $\foral{x_1, x_2 \in M}$ выполнено 
\al{
	\norm{f(x_1)-f(x_2)} &\le \norm{f'_c(x_1 +\theta(x_2-x_1))} \norm{x_2-x_1}, 0\le\theta\le1 \\
	\norm{f(x_1)-f(x_2) - f'_c(x_1)(x_2-x_1))} &\le \norm{f'_c(x_1 +\theta(x_2-x_1)) - f'_c(x_1)} \norm{x_2-x_1}, 0\le\theta\le1
}

\proof{
	выпуклое, поэтому соединим отрезком.
	$\psi(t) = f(x_1 + t(x_2-x_1)), 0\le t \le 1$
	\al{
		\norm{\psi(1) - \psi(0)} &\le \norm{\psi'(\theta)}(1 - 0) \\
		\norm{f(x_2) - f(x_1)} &\le \norm{f'_c(x_1) + \theta(x-2-x_1)(x_2-x_1)} \le \norm{f_c'(x_1 + \theta(x_2-x_1))}\norm{x_2-x_1} \\
		\norm{\psi(1)  - \psi(0) - \psi'(0)(1-0)} &\le \norm{\psi'(\theta) - \psi'(0)} \text{ --- что это??}
	}
}

Когда слабая производная совпадает с сильной? 

\theorem{}Если слабая произв непрерывна, то она совпадает с сильной
\proof{
	\al{
		\omega(x) &= f(x_0 + h) - f(x_0) - f'_c(x_0)(x-x_0) \\
		\norm{\omega(x)} &\le \norm{f'_c(x_0 + h) - f'_c(x_0)}\norm{x-x_0} \\
		f(x) - f(x_0) &= f'_c(x_0) (x-x_0) + \omega(x), \norm{\omega(x)} \le \eps(x)\norm{x-x_0}.
	}
}

\al{
	&f(x,y), x\in[a,b], y\in R \\
	&y(x), C[a,b], y\in C[a,b]
	F(y) &= f(x, y(x)) \\
	F_c'(y) &=\lim_{t\to+0}\frac{f(x, y(x)+th)}t = \dd fy (x, y(x))\cdot h \\
	F_c'(y) &= \dd fy(x, y(x)) \\
	&\modul{\cbr{\dd fy(x, y(x) + h(x)) - \dd fy (x,y(x))} h(x) } \le \omega_{f'}(\delta)\modul{h(x)}. \\
	&\modul{h(x)} \le \delta \text{ --- равномерно непрерывна на отрезке} \le \omega(\delta)\norm{h} \\
	&\norm{\cbr{F_c'(y+h) - F_c'(y)}h} \le \omega(\delta)\norm h  \\
	\text{<еще строка>}
}

\theorem{Интегральная формула конечных приращений} Пусть $M$ --- выпуклое множество в банаховом пространстве $B_1$, функция $f(x)$ имеет производную (сильную) на $M$, тогда справдливы формулы конечных приращений:
\al{
	f(x_2) - f(x_1) &= \integr 01{f'(x_1 + t(x_2-x_1))(x_2-x_1)dt} \\
	f(x_2) - f(x_1) - f'(x_1)(x_2-x_1) &= \integr 01{\cbr{f'(x_1 + t(x_2-x_1)) - f'(x_1)}(x_2-x_1)dt}
}

\proof{
	соединим отрезком
	\al{
		\psi(t) &= f(x_1 + t(x_2 - x_1)) \\
		\psi(t) &= f'(x_1 + t(x_2 - x_1))(x_2-x_1) \\
		\psi(0) - \psi(1) &= \integr 01 {\psi'(t)dt}
	}
	и подставить
}

Следствие 1. $\norm{f'(x)} \le C, x\in M \Rightarrow \norm{f(x_2) - f(x_1)} \le c\norm{x_2-x_1}$ --- условие Липшица

Следствие 1. Если $f'(x)$ удовл условию Липщица: $\norm{f'(x_2) - f'(x_1)} \le L\norm{x_2-x_1}$, то 
$$\norm{f(x_2) - f(x_1) - f'(x_1)(x_2-x_1)} \le \frac L2 \norm{x_2-x_1}^2$$

\proof{
	\al{
		\norm{f(x_1) - f(x_1) - f'(x_1)(x_2-x_1)} 
		&\le \integr 01\norm{f'(x_1+t(x_2-x_1)) -f'(x_1)}\norm{x_2-x_1}dt \le \\
		&\le \integr 01{L\norm{t(x_2-x_1)} \norm{x_2 -x_1} dt} = \frac L2 \norm{x_2-x_1}
	}
}

Пусть имеем функции двух переменных.
$f(x,y), (x,y)\in B_1\times B_2$
\al{
	A&\in \Ell[B_1\times B_2 \to B_3] \\
	A(x,y) &= A_1x + A_2y, A_1\in \Ell[B_1 \to B_3], A_2\in \Ell[B_2 \to B_3]  \Leftarrow A(x,y) &= A(x,0) + A(0,y).
}

Если $A_1$ и $A_2$ ограничены, то и $A$ ограничен: $\norm{A(x,y)} \le \norm{A_1}\norm{x} + \norm{A_2}\norm{y} \le C\cbr{\norm{x} + \norm{y}}$

\al{
	f(x,y) - f(x_0, y_0) &= A(x-x_0, y-y_0) + \eps(x,y)\norm{(x-x_0, y-y_0)} \\
	f(x,y) - f(x_0, y_0) &= A_1(x-x_0, 0) + A_2(0, y-y_0)+ \eps(x,y)\cbr{\norm{x-x_0} + \norm{y-y_0}} \\
	A_1 &= \dd fx(x_0, y_0), A_2 = \dd fy(x_0, y_0) \text{ --- частные производные}
}
\al{
	\dd fx(x_0, y_0)dx &+ \dd fy(x_0, y_0)dy \\
	f(x_2, y_2) - f(x_1, y_1) &= \integr 01{\cbr{\dd fx(x_1 + t(x_2-x_1), y_1 + t(y_2 - y_1)}(x_2-x_1)  + \dd fy(\dot)(y_2-y_1) dt}\\
	{}&\text{\skipped{длинная формула для 2 переменных}}
}

\maintext{Производные и дифференциалы высших порядков.}

$f(x)$ задана в области. $f'(x)dx$
\al{
	&B\colon \Ell[B_1\to B_2] \to \Ell[B_2] \\
	&\delta x\cdot dx \\
	f'&\in\Ell{B_1 \to B_2} \\
	f'(x_2) -f'(x_1) &= A(x_2 - x_1) + o(x_2 - x_1) \\
	{}&\text{\skipped{!!!}} \\
	f''(x_0)dx\delta x &= f''(x)dx^2, d^{n+1}f(x_0) = d(d^nf(x))|_{x=x_0}
}

Обобщение. $k$-линейность. Степень порождается симметричным оператором.
\al{
	{} & f_k(x_1,\ldots, x_k), f_xx^k=f_k(x,x,\ldots,x), \\
	{} & B^k \to B_1, \norm{f_k(x_1,\ldots, x_k)} \le C \norm{x_1} \cdot\ldots\cdot\norm{x_k} \\
	{} & l\le k, f_kx^lh^{k-l} = f_k(x,x,\ldots x, h, \ldots, h) \\
	{} & F_n(x+h)^n = \suml{k=0}n C_n^k x^kh^{n-k} \\
	{} & F_n(x+h)^n = F_n(x+h, \ldots, x+h) = F_n(x,\ldots, x) + kF_nx^{n-1}h + C_n^2F_nx^{n-2}h^2 + \ldots F_nh^n
}

Формула Тейлора. Тоже $k$-линейная ф-ия.
\al{
f(x+h) &= \suml{k=0}n\frac{f^{(k)}(x)}{k!}h^k + \frac1{n!}\integr01{(1-u)^n}f^{(n+1)}(x+uh)u^{n+1}du \\
	\psi(t) &= f(x+th)\\
	f(x+h) &= \psi(1) \\ 
	{}&\text{след раз}
}