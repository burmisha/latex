\simpletitle{17 ноября}
Мы решали аналитическое уравнение $f(x,y) = 0$, когда $f$ --- аналитическая функция в точке $(x_0, y_0)$ (раскладывается по степеням).
Была теорема
\begin{itemize}
	\item $f(x_0, y_0) = 0$
	\item $f^{-1}_u(x_0,y_0)$
\end{itemize}
$y = y_0 + \isum k1(y-y_0)^k$
\al{
	Ay + f(x,y) = 0, Ay_0+f(x_0,y_0) = 0 \\
	T = A - f_y(x_0, y_0) \\
	A(y-y_0) +f(x,y) - f(x_0, y_0) = 0 \\
	(A-f_y(x_0,y_0)))(y-y_0) = - (f(x,y) - f(x_0,y_0) - f_y(x_0,y_0))(y-y_0) \\
	y = y_0 + T^{-1}(f(x,y) - f(x_0,y_0) - f_y(x_0,y_0)) (y-y_0) =\psi(x,y)\\
	\tilde f(x,y) = y - \psi(x,y)
}
Утвреждение: если
\begin{itemize}
	\item $Ay + f(x,y) = 0$
	\item $Ay_0 + f(x_0,y_0) = 0$
	\item $\inv{Ay_0 + f(x_0,y_0)}_y$
\end{itemize}
то решение ищется в виде ряда по степеням $x-x_0$

\newcommand\ort{}
Пусть фредгольмов оператор $T$
\begin{itemize}
	\item нормально разрешим $Tx = y, y \ort N(T^*)$
	\item $N(T^*)$ и $N(T)$ конечномерны и имеют конечную размерность :-)
\end{itemize}

$$Tx = 0$$

\al{
	Tx &= f \\
	x &= \sum_{i=1}^n x_ie_i + y, (e_i^*, x) = x_i, y\ort e_i^* \\
	\fbr{e_i} &\text{ --- базис} в N(T) \\
	\fbr{e_i^*}, &(e_i^*, e_j)=\delta_{ij} \\ 
	x &= \sum_{i=1}^n(e_i,x)e_i + y \\
	(e_i^*,x) &= (e_i^*,x)+(e_i^*,y) \\
	Ty &= f, f\ort N(T^*) \\
	x &\in B, y\in\tilde B = {}^{\ort}N(T^*) \\
	T&\colon \tilde B \to B^2 \\
	y &= \sum_{i=1}^n(e_i^*)e_i \\
	x &= \sum_{i=1}^n x_ie_i + \tilde T^{-1}f
}

Теперь рассмотрим уравнение в банаховом пространстве
$Ax + f(x,y) = 0$
\begin{itemize}
	\item $Ax_0 + f(x_0, y_0) = 0$ 
	\item оператор $T = A + f_y(x_0, y_0)$ --- фредгольмовский
\end{itemize}
Без ограничения общности, $x_0 = y_0 = 0$.
\begin{equation}
	Ty + \Phi(x,y) = 0, \Phi(0,0) = 0, \Phi(0,0) = 0
\end{equation}
\al{
	N(T) = \Ell\cbr{\idots e1n}, Ty + \Phi(x,y) = 0 \\
	(e_i^*,\Phi(x,y)) = 0 , i = \ol{1,n} \\
	y = \sum_{i=1}^n \alpha_i e_i+z, z \ort N(T), z \in {}^\ort N(T^*) \\
	Qf = f - \sum^{i=1}^n (e_i^*,f)e_i, Qf \ort N(T^*)
}
\al{
	\tilde Tz + Q\Phi = 0 \\
	(e_i^*, \Phi(x,y)) = 0, i = \ol{1,n}\\
	(e_i^*, \Phi(x, \sum_{i=1}^n\alpha i, e_i + z)) = 0 \\
	(e_i^*, \Phi(x, \sum_{i=1}^n)) \\
	\skipped{5 lines}
}
Это уравнения разветвления Ляпунова-Шмидта.

Пусть имеем параметрическое (по $\lambda$) уравнение, причем тривиальное решение есть всегда (можно сравнить с поиском собственных значений)
\al{
	Ax + f(x,\lambda) = 0, f(0,\lambda) \\
	T = A - f_(0,\lambda_0) \text { --- фредгольмовский оператор}
	N(T) = \alpha e, Ax 0 f_x(0,\lambda_0 x) = 0
}
$\lambda_0$ --- бифуркационное
\al{
	f(x,y) = \alpha_1 x_0 + \psi_1(\lambda) x + x^2\psi_2(\lambda) + \ldots \
	\psi_1(\lambda) = \isum k1 \gamma_k(\lambda - \lambda_0)^k \\
	x = \gamma e + z \\
	z \ort e^*\\
	(e, e^*) = 1, (e^*,z) = 0\\
	Tx = -\cbr{\psi_1(\lambda) x + \psi_2(\lambda)x^2 + \ldots} \\
	\tilde Tz = -\cbr{\psi_1(\gamma e + z) x + \psi_2(\lambda)(\gamma e + z)^2 + \ldots} = F(x,\lambda) \\
	\begcas{\tilde Tz = \psi_2(\lambda)Q(\gamma e + z)^2 + \ldots \\
			\gamma \psi_1(\lambda) + \psi_2(\lambda)\cbr{e^*, (\gamma e + z)^2} + \ldots = 0}
}

Имеем систему уравнений, причем первое уравнение самостоятельно.
\al{
	z &= z(\gamma, \lambda) \\
	\gamma &\psi_1(\lambda) + \psi_2(\lambda)\chi_2(\gamma \lambda) + \ldots = 0	\\
}
$$\box{\lambda = a\gamma^p(1+\lambda_1\gamma + \ldots)}$$
$$\lambda = a\gamma^p + \ldots \gamma = \pm \lambda^{\frac1p}$$
От четности $p$ зависит, есть ли решения при положительных значениях $\lambda$. 

Бифуркация первого рода: при положительных - 2 решения (горизонтальная парабола)
Бифуркация второго рода: всегда 1 решение.

Давим на вертикально стояющую металлическую линейку сверху.
$$y''(s) + \lambda y(x)\sqrt{1-y'^2(s)} = 0, y(0) = 0, y(\pi) = 0, \lambda = \frac{\pi P}{EJ}$$
При $\lambda < 1$ решений, кроме тривиального, нет.
\al{
	-\integr 0\pi y'^2(s)ds + \lambda \integr 0\pi y^2(s)\sqrt{1-y'^2(s)}ds = 0 \\
	\lambda \integr 0\pi y^2(s)ds \ge \integr 0\pi y'^2(s)ds \ge \integr 0\pi y^2(s)ds \Rightarrow y = 0.
}
Последнее неравенство: через разложение по $\fbr{\sin nx}$ в $L_2[0, \pi]$ и равенство Парсеваля (а для производной --- в коэфф появятся $n$-ки) $\frac1\pi\integr 0\pi y^2(s)ds = \isum n1 b_n^2$

\al{
	\begcas{
		y''(s)+\lambda y(s) = 0 \\y(0)=y(\pi) = 0
	} \\
	\lambda_n = n^2, y_n(s) =\sin nx
}
Если справа от 1, то задача имеет 3 решения (2 нетрив), одно из них, правда, не особо физично. Но строго --- 3.


Математический маятник
\al{
	y'(s) = \sin \theta \\
	\sqrt{1-y'^2} = \cos \theta \\
	\cos \theta \frac{d\theta}{ds} + \lambda y(s)\cos\theta = 0 \\
	\frac{d\theta}{ds} + \lambda y(s) = 0 \\
	\begcas{
		\frac{d^2\theta}{ds^2} + \lambda \sin \theta = 0\\
		\theta'(0) = \theta(\pi) = 0
	}
}
\al{
\begcas{
		\frac{d^2\theta}{dt^2} + \lambda \sin \theta = 0\\
		\theta'(0) = \theta(\pi) = 0
	}
	}