\simpletitle{13 апреля}
\al{
	y(0,t) &= 0, y(\pi,t) = 0\\
	\dd yt &= \ddn2yx + \lambda y-\frac43y^3 = 0 \\
	Pf &= \frac2\pi\integr0\pi f(x)\sin x dx, Qf = f -\sin x Pf\\
	\lambda &= 1, \sin x, \tau - 2t\eps,\\
	y &= \sqrt\eps(u\sin x +v), Lv = \ddn2vx + v \\
	u_0 &= a(\tau) = \frac1{\sqrt{1+e^{-\tau-\tau_0}}}, v_0 = a^3(\tau)\frac1{24}\sin 3x\\
	&\begcas{
		2\ddd u\tau &- u + u^3 + \eps P\phi(u,v) = 0 \\
		2\eps\ddd v\tau &- Lv - \frac13u^3\sin 3x+\eps Q\phi(u,v)\\ 
	}\\
	\phi(u,v) &= 4u^2v\sin^2x+4\eps uv^2\sin x + \eps^3v^3 
}

\al{
	&C(\ol{R})\colon |f| \le Ca^2(\tau), 
		\norm f = \sup_{\tau\in\mathbb R} a^{-2}(\tau)|f(\tau)| \\
	&\tilde C ([0,\pi]\times\ol R)\colon |f(x,y)|\le ca^{-2}(\tau)
		\norm f = \sup_{\tau\in\mathbb R, x\in[0,\pi]} a^{-2}(\tau)|f(\tau)| \\
	&A\omega - F\omega = 0, \omega=\bivec uv, \\
	A &= \bivec{\ddd{}\tau - 1}{2\eps\ddd v\tau - Lv}
	F = \bivec{-u^3 - \eps P\phi(u,v)}{\eps c + \frac13u^3\sin 3x + \eps Q \phi},
	\omega_0 = \bivec{u(\tau)}{v(\tau)}
}
\begin{itemize}
	\item $F'(\omega)$ удовлетворяет условию Липшица в ограниченной области,
	\item $A-F'(\omega_0)$ имеет ограниченный обратный.
\end{itemize}
\al{
	2\dd{h_1}\tau &- h_1 + 3a^2(\tau)h_1 + \eps B_1(\omega)\\
	&\skipped{2--3}
}
Докажем, что укороченные уравнения без $\eps B_1(\omega)\ldots$ имеют решения, тогда и они будут иметь: $S+\eps B = S(I + \eps S^{-1}B)$.

\al{
	2\dd{h_1}\tau 
		&- h_1 + 3a^2(\tau)h_1 = g_1\\
	a'(\tau) 
		&= a-a^3 = a(1-a^2) \\
	h_1 
		&= a'(\tau)\int_{-\infty}^\tau\frac{g_1(s)}{a'(s)}ds\\
	|h_1(\tau)| 
		&\le a'(\tau) \norm{g_1}\integr{-\infty}\tau\frac{a^2(s)a'(s)}{a'^2(s)}ds 
		= a'(\tau) \norm{g_1}\integr{-\infty}\tau\frac{a^2(s)a'(s)}{a^2(s)(1-a^2(s))}ds \le\\
		&\le a'(\tau) \norm{g_1}\integr{-\infty}\tau\frac{a'(s)}{1-a^2(s)}ds = a'(\tau) \norm{g_1}\frac{a(\tau)}{1-a(\tau)} =
		= \norm{g_1}\frac{a(1-a^2)a}{1-a(\tau)} \le 2a^2\norm{g_1}\tau.\\
		\text{ибо }&|h_1(\tau)|\le2a^2(\tau)\norm{g_1}, a < 1;
}

Изучаем функции на бесконечности. Правило Лопиталя:
\al{
	h_1
		&= \frac{\integr0\tau\frac{g_1(s)}{a'(s)}ds}{\frac1{a'(\tau)}} 
		\propto -\frac{g_1(\tau)}{a'(\tau)\frac1{a'^2(\tau)}} = \\
		&= \frac{g_1(\tau)}{a'\frac{1-3a^2}{\ldots}}=\frac12g(+\infty)
}

\renewcommand\iint{\integr {-\infty}\tau}
\al{
	2\eps\dd{h_2}\tau &- \ddn2{h_2}x h_2 = g_2(x,\tau), h_2(0,\tau) = 0, h_2(\pi,\tau) = 0. \\
	h_2(x,\tau) 
		&= \frac1{\pi\eps}\integr0\pi\integr{-\infty}\tau
		\cbr{\isum k2 e^{\frac{n^2-1}{2\eps}(s-\tau)}\sin kx\sin k\xi}g_2(\xi,s)ds d\xi \\
	|h_2(x,\tau)| 
		&\le \frac{\norm{g_2}}\eps\isum k2 \integr{-\infty}\tau a^2(s)e^{\frac{n^2-1}{2\eps}(s-\tau)}ds	 \\
	I_n 
		&= \frac1\eps\iint a^2(s)e^{\frac{n^2-1}{\eps}(s-\tau)}ds 
		= \iint a^2(s)d\frac{e^{\frac{n^2-1}{\eps}(s-\tau)}}{n^2-1} = \\
		&= \frac{a^2(\tau)}{n^2-1} - \frac2{n^2-1}\iint a(s)(a-a^3)e^{\frac{n^2-1}{\eps}(s-\tau)} ds < \frac{a^2(\tau)}{n^2-1} \\
	&\skipped{5--9}
}

Солитоны
\al{
	\ddn2ux + \ddn2uy + \lambda u + \frac83u^3 = 0.
}
В полосе $0\le u\le\pi$. Причем $u=0$ на границах полосы.
Тривиальное решение есть: 0 --- ламинарный поток.
Интересно, когда появляется нетривиальное.
\al{
	\dddn2uy &+ \lambda u = 0, u(0)=u(\pi) = 0;\\
	\eps x &= \xi \\
	\lambda_0 &= 1, \phi =\sin x, \lambda = 1-\eps^2. \\
	V &= V_0+\eps^2V_1, \ddn2{V_0}y+V_0 = 0, V_0(0)=0, V_0(\pi) = 0\\
	\eps^2\ddn2u\xi &+ \ddn2uy+(1-\eps^2)u + \frac83u^3 = 0 \\
	u=\eps V\\
	\eps^2\ddn2v\xi &+ \ddn2vy + (1-\eps^2)V +\frac83\eps^3V^3 = 0 \\
	\ddn2{V_1}y &+ V_1 +(C''(\xi) - C(\xi))\sin y + \frac83 C^3\sin^3y \\
	\ddn2{V_1}y &+ V_1 = -(C''(\xi)-C(\xi))\sin y -\frac83\sin^3y, V_1(0)=V_1(\pi) = 0\\
	C''&-C+2C^3 = 0,\\
	C'^2 &= C^2+C^4 + A, y = C^2 - C^4 -C^2(1-C^2) \ldots WTF \skipped{6--9}
}