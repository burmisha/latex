\documentclass[12pt]{article}
\usepackage[utf8]{inputenc}
\usepackage[russian]{babel} %comment it for english!
\usepackage{amsfonts,longtable,amssymb,amsmath,array}
\usepackage{listings}
%\usepackage{euler}

\input pagestyle.tex
\input macro.tex

\begin{document}

\simpletitle{Домашнее задание 3. \\ Михаил Бурмистров. i.like.spam@ya.ru}
\section{Задача 1}
burmisha $\to$ ta ti ta ti ti ti ta ti \\
\subsection{a}
Номера последовательных состояний: 0 0 1 4 1 2 2 3 4 --- не принимается
\subsection{b}
Необходимо добавить не менее 4 букв: например, ta ti ta ti ti ti [ta ti] ta ti.
\subsection{c} 
Получившаяся в пункте b строка прнимается, но, видимо, этот пункт не надо решать.

\section{Задача 2}
Будем считать, что слова по умолчанию состоят из маленьких английских букв.
\subsection{a}
\ttfamily{w[a-z]([b-d]|[f-h]|[j-n]|[p-t]|[v-x]|z)}
\subsection{b}
\ttfamily{(([b-d]|[f-h]|[j-n]|[p-t]|[v-x]|z)*e){3}([b-d]|[f-h]|[j-n]|[p-t]|[v-x]|z)}
\subsection{c}
\ttfamily{(0+(1+0+)*|1+(0+1+)*)}
\end{document} 