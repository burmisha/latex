\documentclass[12pt]{article}
\usepackage[utf8]{inputenc}
\usepackage[russian]{babel} %comment it for english!
\usepackage{amsfonts,longtable,amssymb,amsmath,array}
\usepackage{listings}
%\usepackage{euler}

\input pagestyle.tex
\input macro.tex

\begin{document}

\simpletitle{Домашнее задание 3. \\ Михаил Бурмистров. i.like.spam@ya.ru}
\section{Задача 1}
burmisha $\to$ ta ti ta ti ti ti ta ti \\
\subsection{a}
Номера последовательных состояний: 0 0 1 4 1 2 2 3 4 --- не принимается
\subsection{b}
Необходимо добавить не менее 4 букв: например, ta ti ta ti ti ti [ta ti] ta ti.
\subsection{c} 
Получившаяся в пункте b строка принимается, но, видимо, этот пункт не надо решать.

\section{Задача 2}
Будем считать, что слова по умолчанию состоят из маленьких английских букв.
\subsection{a}
{\ttfamily{w[a-z]([b-d]|[f-h]|[j-n]|[p-t]|[v-x]|z)}}
\subsection{b}
{\ttfamily{(([b-d]|[f-h]|[j-n]|[p-t]|[v-x]|z)*e)\{3\}([b-d]|[f-h]|[j-n]|[p-t]|[v-x]|z)}}
\subsection{c}
{\ttfamily{(0+(1+0+)*|1+(0+1+)*|)}}

\section{Задача 3}
\subsection{a}
{\ttfamily{au+a*}}
\subsection{b}
{\ttfamily{(|cl|ai)(tutor|professor])(i(n|um))?}}
\subsection{c}
{\ttfamily{(0[1-9]|1[0-3])([a-d]|e?)}}
\subsection{d}
{\ttfamily{(h(a|ae)\{1,2\})+(t+|z)sch(i\{3,\}|ue|u|y)+}}

\section{Задача 4}
{\ttfamily{(geschleimt|schleim(e(t|nd?|st)?|st))}}
\subsection{a}
geschleimt
schleime
schleimet
schleimen
schleimend
schleimest
schleimst
\subsection{b}
\subsection{c}
{\ttfamily (geschleimt|schleim(t?e(t|nd?|st)?|st))}
Правда, при этом появятся лишние слова, удовлетворяющие РВ, но условие будет выполнено.

\section{Задача 5}
\section{Задача 6}
\section{Задача 7}
\section{Задача 8}
$\Omega=\fbr{\omega: \omega \sim \text{\ttfamily(a|b)\{4\}}}$.
\subsection{a}
Энтропия будет минимальна, если вероятность одного из слов --- 1, а всех остальных 0. Например $p(\omega)=\begcas{1, &\omega=aaaa\\0, &\omega\ne aaaa}.$ При этом энтропия будет равна 0.

\subsection{b}
Энтропия будет максимальна, если все слова равновероятны: $p(\omega)=\frac1{2^4}=\frac1{16}.$ $S = 2^4\cdot\cbr{-\frac1{2^4}\log_2\frac1{2^4}} = \log_2 2^4 = 4.$

\section{Задача 9}
У моей монетки вероятность $p=\frac1\pi$. \\
Моё распределение: $p$, а у честной монетки: $q.$
\subsection{a}
$S = -\frac1\pi \log_2\frac1\pi - \cbr{1-\frac1\pi} \log_2\cbr{1 - \frac1\pi} \approx 0{.}90253404359986589$

\subsection{b}
$$KL(p,q) = \frac1\pi \ln\frac{\frac1\pi}{\frac12} + \cbr{1-\frac1\pi} \ln\frac{1 - \frac1\pi}{\frac12} \approx 0{.}067558252879331454,$$
$$KL(q,p) = \frac12 \ln\frac{\frac12}{\frac1\pi} + \frac12 \ln\frac{\frac12}{1-\frac1\pi} \approx 0{.}070807813849007667.$$

\section{Задача 10}
Во всех грамматиках S --- начальное состояние
\subsection{a}
\al{
    S &\to 0S1S | 1S0S | \eps
}
\subsection{b}
\al{
    S &\to S0S1S1S | S1S0S1S | S1S1S0S| \eps
}
\subsection{c}
Исключая пустое слово получим:
\al{
    S \to   &aTa | bTb |cTc |dTd |eTe |fTf |gTg |hTh |iTi \\
            &|jTj |kTk |lTl |mTm |nTn |oTo |pTp |qTq |rTr \\
            &|tTt |tTt |uTu |vTv |wTw |xTx |yTy |zTz | \\
            & a |b |c |d |e |f |g |h |i |j |k |l |m |n |o |p |q |r |s |t |u |v |w |x |y |z \\
    T \to   &aTa | bTb |cTc |dTd |eTe |fTf |gTg |hTh |iTi \\
            &|jTj |kTk |lTl |mTm |nTn |oTo |pTp |qTq |rTr \\
            &|tTt |tTt |uTu |vTv |wTw |xTx |yTy |zTz | \\
            & a |b |c |d |e |f |g |h |i |j |k |l |m |n |o |p |q |r |s |t |u |v |w |x |y |z |\eps 
}

\section{Задача 11}
При применении правил грамматики в нормальной форме Хомского возможны следующие варианты:
\begin{itemize}
    \item Число нетерминалов увеличивается на 1 (причем при этом не может появится начальный символ), число терминалов не изменяется. (Правила вида $A\to BC,\;B,C\ne S$.)
    \item Число нетерминалов уменьшается на 1, число терминалов увеличивается на 1. (Правила вида $A\to \alpha$.)
    \item Число нетерминалов уменьшается на 1, число терминалов не изменяется (Правило $S\to\eps$.)
\end{itemize}
Таким образом единственный способ получить пустое слово --- применить правило $S\to\eps$. Причем при этом, вообще говоря, доказываемое утверждение неверно: $0\ne 2\cdot 0 - 1$. Поэтому в дальнейшем будем рассматривать непустые слова.

В таком случае, придется отказаться от возможности применения правила $S\to\eps$, поскольку применить его можно ровно один раз и не более (поскольку $S$ не встречается в правых частях правил.) Теперь рассуждаем таким образом. Сперва у нас в строке есть ровно один символ: $S$, он нетерминал. Чтобы получить ровно $n$ терминалов надо ровно $n$ применить правила вида $A\to\alpha$. При каждом таком применении число нетерминалов уменьшается на 1, поэтому предварительно необходимо их получить: это можно сделать лишь правилами  вида $A\to BC$, применений которых потребуется на $n$ нетерминалов ровно $n-1$ штук (ибо один нетерминал у нас в строке уже есть). Всё: имеем $n+n-1=2n-1$ применений продукций, что и завершает доказательство.

\section{Задача 12}
\al{
    S&\to XY|\eps\\
    X&\to XY\\
    Y&\to AB\\
    A&\to a\\
    B&\to b
}

\section{Задача 13}

{
\newcommand\Ra{\!\Rightarrow\!}
\newcommand\T{\text{\textsf{Type}}}
\newcommand\V{\text{\textsf{T\!V}}}
\newcommand\To{\:\:\to\:\:}

\subsection{a}
    Положим, что $\T$ --- начальный символ.
    Покажем, что у цепочки $\alpha\Ra\beta$ есть 2 различных дерева вывода
    $$\T \To \T\Ra\T \To \V\Ra\T \To \alpha\Ra\T \To \alpha\Ra\V \To \alpha\Ra\beta,$$
    $$\T \To \T\Ra\T \To \T\Ra\V \To \T\Ra\beta \To \V\Ra\beta \To \alpha\Ra\beta.$$
\subsection{b}
    Теперь положим, что $S$ --- начальный символ.
    \al{
        S &\To L\Ra R | P\\
        R &\To L\Ra R\\
        L &\To P\\
        R &\To P\\
        P &\To A*B|F\\
        A &\To A*B|F\\
        B &\To F\\
        F &\To \alpha|\beta|\gamma|\ldots
    }


}

\end{document} 