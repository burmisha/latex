\input header.tex
\input pagestyle.tex
\input macro.tex

\begin{document}
\simpletitle{Параметрическое построение контура,\\ описывающего облако точек. \\ Михаил Бурмистров. \today}


\maintext{Дано:} $A = \fbr{\cbr{x_i, y_i}}_{i=1}^N$ --- набор точек на плоскости.

\maintext{Найти:} $\fbr{n_j}_{j=1}^k$ --- последовательность индексов точек из $A$, описывающее пространственную структуру их расположения.

\simpletitle{Решение}
Предлагается использовать модифицированный алгоритм Джарвиса для поиска выпуклой оболочки. 
Основная модификация заключается в введении структурнго параметра  $r$: теперь будет выбираться точка, образующая наименьший угол с предыдущим направлением, но не из всех точек множества $A$, а из точек, лежащих в окрестности предыдущей точки радиуса $r$. 
Очевидно, что при $r\to\infty$ алгоритм переходит в алгоритм Джарвиса, а при $r\to0$ совершенно не работает (окрестность пуста). 
Ясно, что чем меньше параметр $r$, тем более мелкие неоднородности способен заметить алгоритм.
Параметр $r$ в предложенном решении подбирается экспериментально (оптимальные значения для присланных облаков точек уже подобраны).

В такой форме алгоритм не способен обходить протяженные вогнутые границы, поэтому была предложена следующая эвристика: выбирать очередную точку таким образом, что она ближе к последней точке, чем к предпоследней (если такие точки нашлись если же нет~--- всё как обычно). 
Такая эвристика удивительным образом помогла избавиться на тестовых изображениях от множества проблем, не создав новых.

\end{document}
