\input header.tex
\input pagestyle.tex
\input macro.tex

\begin{document}
\simpletitle{Параметрическое построение контура,\\ описывающего облако точек. \\ Михаил Бурмистров. \today}


\maintext{Дано:} $A = \fbr{\cbr{x_i, y_i}}_{i=1}^N$ --- набор точек на плоскости.

\maintext{Найти:} $\fbr{n_j}_{j=1}^k$ --- последовательность индексов точек из $A$, описывающее пространственную структуру их расположения.

\simpletitle{Решение}
Предлагается использовать модифицированный алгоритм Джарвиса для поиска выпуклой оболочки. 
Основная модификация заключается в введении структурнго параметра  $r$: теперь будет выбираться точка, образующая наименьший угол с предыдущим направлением, но не из всех точек множества $A$, а из точек, лежащих в окрестности предыдущей точки радиуса $r$. 
Очевидно, что при $r\to\infty$ алгоритм переходит в алгоритм Джарвиса, а при $r\to0$ совершенно не работает (окрестность пуста). 
Ясно, что чем меньше параметр $r$, тем более мелкие неоднородности способен заметить алгоритм.
Параметр $r$ в предложенном решении подбирается экспериментально (оптимальные значения для присланных облаков точек уже подобраны).

В такой форме алгоритм не способен обходить протяженные вогнутые границы, поэтому была предложена следующая эвристика: выбирать очередную точку таким образом, что она ближе к последней точке, чем к предпоследней (если такие точки нашлись, если же таковых нет~--- всё как обычно). 
Эта эвристика удивительным образом помогла избавиться на обработке тестовых изображениях от множества проблем, не создав новых.

\simpletitle{Инструкции по запуску}
Технически программа является MATLAB скриптом. 
\definecolor{dkgreen}{rgb}{0,0.6,0}
\definecolor{gray}{rgb}{0.5,0.5,0.5}
\definecolor{mauve}{rgb}{0.58,0,0.82}
 
\lstset{ %
  language=Octave,                % the language of the code
  basicstyle=\footnotesize,           % the size of the fonts that are used for the code
                                  % will be numbered
  numbersep=5pt,                  % how far the line-numbers are from the code
  backgroundcolor=\color{white},      % choose the background color. You must add \usepackage{color}
  showspaces=false,               % show spaces adding particular underscores
  showstringspaces=false,         % underline spaces within strings
  showtabs=false,                 % show tabs within strings adding particular underscores
  frame=single,                   % adds a frame around the code
  rulecolor=\color{black},        % if not set, the frame-color may be changed on line-breaks within not-black text (e.g. comments (green here))
  tabsize=2,                      % sets default tabsize to 2 spaces
  captionpos=b,                   % sets the caption-position to bottom
  breaklines=true,                % sets automatic line breaking
  breakatwhitespace=false,        % sets if automatic breaks should only happen at whitespace
  %title=\lstname,                   % show the filename of files included with \lstinputlisting;
                                  % also try caption instead of title
  keywordstyle=\color{blue},          % keyword style
  commentstyle=\color{dkgreen},       % comment style
  stringstyle=\color{mauve},         % string literal style
  escapeinside={\%*}{*)},            % if you want to add LaTeX within your code
  morekeywords={*,...},              % if you want to add more keywords to the set
  deletekeywords={...}              % if you want to delete keywords from the given language
}

Для демонстрационного использования с набором присланных изображений следует изменять номер файла (от 2 до 5) в строке 
\begin{lstlisting}[language=matlab, frame=lines]
FileNumber = 5;
\end{lstlisting}

Изменяя следующие строки, можно получить результат работы алгоритма на любом файле с расширением \texttt{.txt}, содержащего в каждой строке пару чисел, разделенных пробелом, --- координаты точек облака (без общего числа точек в первой строке). При этом можно задать любой желаемый радиус рассматриваемых окрестностей. 
\begin{lstlisting}[language=matlab, frame=lines]
FileNamePrefix = sprintf('%d',FileNumber);
r = RadiusOptimal(FileNumber);
\end{lstlisting}
\maintext{Внимание:} алгоритм может зациклится при неверном выборе параметра $r$. Ответственность за это полностью лежит на пользователе скрипта.

По окончании работы скрипт выводит на экран полученную границу, площадь и периметр ограничиваемой фигуры, а также записывет всё в файл.
\newpage
\addeps{pic/2_r12}{Файл \texttt{2.txt}, $r=12$}
\addeps{pic/3_r13}{Файл \texttt{3.txt}, $r=13$}
\newpage
\addeps{pic/4_r13}{Файл \texttt{4.txt}, $r=13$}
\addeps{pic/5_r19}{Файл \texttt{5.txt}, $r=19$}
\lstset{ %
	basicstyle = \scriptsize,
	numbers=left,                   % where to put the line-numbers
  numberstyle=\tiny\color{gray},  % the style that is used for the line-numbers
  stepnumber=5,                   % the step between two line-numbers. If it's 1, each line 
	numberfirstline=true,
}
\newpage
\simpletitle{Листинг}
\begin{lstlisting}[language=matlab, frame=lines]
clear all;

FileNumber = 5;
FileNamePrefix = sprintf('%d',FileNumber);
RadiusOptimal = [500, 12, 13, 13, 19];
X = dlmread(strcat(FileNamePrefix,'.txt'));
%X = [X(:,1), X(:,3)];  % 2 -> 12 .. 10     % 3 -> 13 .. 12     % 4 -> 13 .. 11     % 5 -> 19 only
r = RadiusOptimal(FileNumber);

MinAngle = -179;

AngleIdxInit = @() ndgrid(1000, 0);
clangle = @(x) (1.25-4*(heaviside(x(:,1).*x(:,4)-x(:,2).*x(:,3))-0.75).^2) .*acos((x(:,1).*x(:,3)+x(:,2).*x(:,4))./sqrt((x(:,1).^2+ x(:,2).^2) .*(x(:,3).^2+ x(:,4).^2)))*180/pi;

h = plot(X(:,1), X(:,2), 'r.','LineWidth',1);
axis normal; hold on;

[~, idx] = max(X(:,1)==max(X(:,1)));
plot(X(idx,1), X(idx,2), 'b.','LineWidth',3);

Previous = X(idx,:);
nearest = find(((X(:,1) - Previous(1)).^2 + (X(:,2) - Previous(2)).^2) < r^2);
nearest(nearest == idx) = [];
XX = X(nearest,:);
NPrev = (XX(:,1) - Previous(1)).^2 + (XX(:,2) - Previous(2)).^2;
Angle = clangle([ones(length(nearest),1) * [0 -1] (XX - ones(length(nearest),1) * Previous)]);
[~, NewIdx] = min(Angle(Angle > MinAngle));
idx = [idx nearest(NewIdx)];
plot(X(idx(end),1), X(idx(end),2), 'b.','LineWidth',3);

while idx(end) ~= idx(1)
    Previous = X(idx(end),:);
    PrePrevious = X(idx(end-1),:);
    LastLine = Previous - PrePrevious;
    nearest = find(((X(:,1) - Previous(1)).^2 + (X(:,2) - Previous(2)).^2) < r^2);
    nearest(nearest == idx(end)) = [];
    XX = X(nearest,:);
    NPrev = (XX(:,1) - Previous(1)).^2 + (XX(:,2) - Previous(2)).^2;
    NPPrev = (XX(:,1) - PrePrevious(1)).^2 + (XX(:,2) - PrePrevious(2)).^2;
    Angle = clangle([ones(length(nearest),1) * LastLine (XX-ones(length(nearest),1)*Previous) ] );
    Positive = find((Angle > MinAngle) & (NPrev <= NPPrev));
    if ~isempty(Angle(Positive))
        [~, BestPos] = min(Angle(Positive));
        NewIdx = nearest(Positive(BestPos));
    else
        Negative = find((Angle > MinAngle) & (NPrev > NPPrev));
        [~, BestNeg] = min(Angle(Negative));
        NewIdx = nearest(Negative(BestNeg));
    end
    %plot(XX(Positive,1), XX(Positive,2), 'y.','LineWidth',3);
    idx = [idx NewIdx];
    %plot(XX(Positive,1), XX(Positive,2), 'r.','LineWidth',3);
    plot(X(idx(end),1), X(idx(end),2), 'g.','LineWidth',3);
    %length(idx)
end

Perimeter = sum(sqrt((X(idx(1:end-1),1)-X(idx(2:end),1)).^2 + (X(idx(1:end-1),2)-X(idx(2:end),2)).^2) )
Area = 1/2 * sum(X(idx(1:end-1),1).*X(idx(2:end),2) - X(idx(1:end-1),2).*X(idx(2:end),1))

plot(X(idx,1), X(idx,2), 'g-','LineWidth',2);
%axis('tight')
axis('square');
axis('xy');
text(   0.5*min(X(:,1)) + 0.5* max(X(:,1)), min(X(:,2)), ...
    strcat('$$P = ', num2str(Perimeter),', S = ',num2str(Area),'$$'), ...
    'Interpreter','latex', 'FontSize',12);
saveas(h, strcat(FileNamePrefix, sprintf('_r%d.eps',r)),        'eps2c');
hold off
\end{lstlisting}
\end{document}
