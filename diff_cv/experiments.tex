Для экспериментального сравнения предложенных техник проведём вычислительный эксперимент на синтетических данных.

\subsection{Сравнение оценок скользящего контроля} % (fold)
В этом эксперименте данные сгенерированно $N=100$ объектов, обладающие $n=5$ признаками, по следующей вероятностной модели:
	\begin{align*}
		x_{ij} &\sim Exp(j), \: i=1,\ldots,N, \: j=1,\ldots,n,\\
		\mb a &= (0, 2, 0, -3, 0)^T, \\
		y_{i} &= \mb x_i^T \mb a + \eps_i, \eps_i \sim \Norm0{0{.}01}, i=1,\ldots,100. 
	\end{align*}

Вычислим оценки скользящего контроля по соотношениям (\ref{diffLOO}, \ref{LOOfixed}, \ref{LOO}) для определения оптимальных параметров регуляризации. Соответствующие графики приведены на рисунках (\ref{pic:diffLOO}, \ref{pic:LOOfixed}, \ref{pic:LOO}).

\begin{figure}[H]
	\label{pic:LOOfixed}
	\centering
	\includegraphics[height=240px]{graph/LOO_I.png}
	\caption{$S_{LOO, \hat I_\lams}(\lams)$}
\end{figure}

\begin{figure}[H]
	\label{pic:diffLOO}
	\centering
	\includegraphics[height=240px]{graph/diff_CV.png}
	\caption{$S_{Diff}(\lams)$}
\end{figure}

\begin{figure}[H]
	\label{pic:LOO}
	\centering
	\includegraphics[height=240px]{graph/honest_CV.png}
	\caption{$S_{LOO}(\lams)$}
\end{figure}

При этом оптимальные значения оказались равными:
\begin{align*}
	\hat S_{LOO, \hat I} 	&=0.0136, &\qquad	\hat \beta&=0.156, &\qquad	\hat \mu&=47.75 \\
	\hat S_{Diff} 			&=0.0119, &\qquad	\hat \beta&=0.160, &\qquad	\hat \mu&=48.55 \\
	\hat S_{LOO} 			&=0.0119, &\qquad	\hat \beta&=0.159, &\qquad	\hat \mu&=48.70.
\end{align*}

		% \addcolumn{LOO_I}{}{
		% 	\hat S_{LOO, \hat I} &=0.0136, \\ \hat \beta &=0.156, \\ \hat \mu &=47.75
		% }
		% \addcolumn{diff_CV}{}{
		% 	\hat S_{Diff} &=0.0119, \\ \hat \beta &=0.160, \\ \hat \mu &=48.55
		% }
		% \addcolumn{honest_CV}{S_{LOO}(\lams)}{
		% 	\hat S_{LOO} &=0.0119, \\ \hat \beta &=0.159, \\ \hat \mu &=48.70
		% }

% subsection  (end)