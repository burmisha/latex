\documentclass[11pt,a4paper]{amsart}
\usepackage{graphics,graphicx}
\usepackage{epsfig}
\usepackage{a4wide}
\usepackage[utf8]{inputenc}
\usepackage[english,russian]{babel}
\usepackage[T2A]{fontenc}
\usepackage{verbatim}
\usepackage{amssymb,amsfonts,amsthm,amsmath,mathtext,cite,enumerate,float}
\renewcommand{\baselinestretch}{1}

\begin{document}
\begin{center}
\end{center}

\bigskip

\begin{center}
\textbf{Рецензия на магистерскую диссертацию}
\end{center}

\textbf{студента 6 курса Бурмистрова Михаила Олеговича.}

\smallskip
\textbf{Рецензент:}

\smallskip
\textbf{Тема: «Дифференциальный скользящий контроль при выборе параметров регуляризации в задаче регрессионного анализа с помощью Elastic Net.»}

\thispagestyle{empty}

В дипломной работе М.\,О. Бурмистрова предложена схема беспереборного скользящего контроля в задаче линейной регрессии с применением регуляризации Elastic Net.
Метод основан на присвоении всем объектам обучающей выборки вектора весов, которые впоследствии варьируются в окрестности единичного вектора, с целью определить, как малые изменения влияют на функционал качества в критерии скользящего контроля.
Производится сравнение с другим подходом к беспереборному скользящему контролю и теоретически и экспериментально показано, как они соотносятся.

\vspace{4pt}
В начале работы описывается постановка задачи линейной регрессии и применяемые подходы к выбору моделей и регуляризации, а также вводится обобщение кросс-валидации в терминах взвешенных объектов.
Далее изучается решение общей задачи оптимизации и его зависимость от весов, присвоенных объектам.
Рассмотривается метод беспереборной кросс-валидации на основе предположения об устойчивости вторичных структурных параметров модели, который приводит к оценке критерия скользящего контроля, которая может быть эффективно вычислена. 
Предлагается метод дифференциальной кросс-валидации, сводящийся к вычислению частных производных функции риска на отдельных объектах по весу соответствующих объектов, что также не требует многократного обучения модели при кросс-валидации.
При этом производится линейная экстраполяция критерия качества в область нулевых весов.
После проводится теоретическое и экспериментальное сравнение обоих подходов, которое показывает их равноправность, а также предлагается общий принцип дифференциальной кросс-валидации, обобщающий прошлый подход.

\vspace{4pt}
К недостаткам работы следует отнести неполноту описания вычислительного эксперимента, а также отсутствие формализованных критериев применимости дифференциальной кросс-валидации.

\vspace{4pt}
Несмотря на отмеченные недостатки, рецензируемая работа удовлетворяет требованиям, предъявляемым к магистерским диссертациям на ФУПМ МФТИ, и заслуживает оценки <<отлично>>, а М. О. Бурмистров~--- присвоения квалификации магистра.

Рецензент: 
\end{document}
