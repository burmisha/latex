\documentclass[unicode,lefteqn,c,hyperref={pdfpagelabels=false}]{beamer}
\usepackage[utf8]{inputenc}
% \usepackage[T2A]{fontenc}
\usepackage{amssymb}
\usepackage{amsmath,mathrsfs}
\usepackage[russian]{babel}
\usepackage{ulem}\normalem
\usepackage{color}
\usepackage[noend]{algorithmic}

\input ../macro.tex

\usetheme{Warsaw}
\usefonttheme[onlylarge]{structurebold}
\setbeamerfont*{frametitle}{size=\normalsize,series=\bfseries}
\setbeamertemplate{navigation symbols}{}
\setbeameroption{show notes}
\definecolor{beamer@blendedblue}{RGB}{15,80,120}
\let\Tiny=\tiny
\def\shortspace{\hspace{1.5pt}}

%%%%%%%%%%%%%%%%%%%%%%%%%%%%%%%%%%%%%%%%%%%%%%%%%%%%%%%%%%%%%%%%%%%%%%%%%%%%%%%
\title[\hbox to 56mm{Дифференциальный скользящий контроль \hfill\insertframenumber\,/\,\inserttotalframenumber}]{Дифференциальный скользящий контроль при выборе параметров регуляризации в задаче регрессионного анализа с помощью Elastic Net}
\author[М.\shortspaceО.\shortspaceБурмистров]{М.\shortspaceО.\shortspaceБурмистров}
\institute{Научный руководитель д.т.н.
	\vfill В.\,В.~Моттль \vfill ~ 
	\vfill Московский физико-технический институт
	\vfill Факультет управления и прикладной математики
	\vfill Кафедра интеллектуальных систем}
\date{10 июня 2014г.}
%%%%%%%%%%%%%%%%%%%%%%%%%%%%%%%%%%%%%%%%%%%%%%%%%%%%%%%%%%%%%%%%%%%%%%%%%%%%%%%

\begin{document}

\begin{frame}
	\titlepage
\end{frame}

%%%%%%%%%%%%%%%%%%%%%%%%%%%%%%%%%%%%%%%%%%%%%%%%%%%%%%%%%%%%%%%%%%%%%%%%%%%%%%%
%%%%%%%%%%%%%%%%%%%%%%%%%%%%%%%%%%%%%%%%%%%%%%%%%%%%%%%%%%%%%%%%%%%%%%%%%%%%%%%
\section{Постановка задачи}
%%%%%%%%%%%%%%%%%%%%%%%%%%%%%%%%%%%%%%%%%%%%%%%%%%%%%%%%%%%%%%%%%%%%%%%%%%%%%%%
\subsection{Данные и вероятностные гипотезы}
%%%%%%%%%%%%%%%%%%%%%%%%%%%%%%%%%%%%%%%%%%%%%%%%%%%%%%%%%%%%%%%%%%%%%%%%%%%%%%%
\begin{frame}{Данные и вероятностные гипотезы}
	\textbf{Изучаемые объекты:}
	\begin{itemize}
			\item $\Omega$~--- объекты реального мира;
			\item $\mb x \colon \Omega \to X = \mathbb R^n$~--- признаковое описание объектов, $\mb x(\omega) = (x_1, \ldots, x_n)^T$;
			\item $y \colon \Omega \to Y = \mathbb R$~--- скрытая характеристика объектов.
	\end{itemize}

	\smallskip
	% \rlap{\textbf{Основные гипотезы вероятностной тематической модели:}}
	\textbf{Исходные данные}
	\begin{itemize}
		\item $\Omega^* = \fbr{w_i\cond i=1,\ldots,N}\subset \Omega$~--- обучающая совокупность;
		\item $\mb X = (\mb x_1, \ldots, \mb x_N)^T$~--- матрица объекты-признаки;
		\item $\mb y = (y_1, \ldots, y_N)^T$~--- вектор ответов.
	\end{itemize}

	\smallskip
	\textbf{Задача линейной регрессии:}

	Функция $\hat y \colon X \to Y$ ищется в параметрическом семействе 
	\begin{equation*}
		\hat y(\mb x) = \mb a^T\mb x + b, \:\text{где $\mb a \in \mathbb R^n$ и $b\in \mathbb R.$}
	\end{equation*}
\end{frame}

\begin{frame}{Регуляризация}
	\textbf{Исходный функционал качества}
	\begin{equation*}
		\suml_{i=1}^N(y_i - \hat y(\mb x_i))^2.
	\end{equation*}

	\smallskip
	\textbf{Регуляризация Elastic Net}

	Для отбора признаков введём штрафную функцию
	\begin{equation*}
		\beta\suml_{i=1}^n a_i^2 + \mu\suml_{i=1}^n\abs{a_i}.
	\end{equation*}

	\smallskip
	\textbf{Требования нормировки}

	При использовании общих коэффициентов штрафа для компонент \mb a необходимо сделать замену, чтобы обучающая совокупность была центрирована и нормирована:
	\begin{equation*}
		\frac1N\suml_{i=1}^N\mb x_i = \mb 0, \:
		\frac1N\suml_{i=1}^N y_i = 0, \:
		\foral{j=1,\ldots, n} \: \frac1N\suml_{i=1}^N\mb x_{ij}^2 = 1. 
	\end{equation*}
\end{frame}

\begin{frame}{Кросс валидация}
	\textbf{Задача оптимизации}
	\begin{equation*}
		\begin{cases}
			\suml_{i=1}^n\cbr{\beta a_i^2 + \mu \modul{a_i}} 
			+ \suml_{j=1}^N \delta_j^2 \to \min_{\mb a}, \\
			\delta_j = y_j - \suml_{i=1}^nx_{ij}a_i, j=1,\ldots,N. 
		\end{cases}
	\end{equation*}
	Оптимальный вектор \mb a зависит от $\beta, \mu, \Omega^*.$

	\smallskip
	\textbf{Кросс валидация}

	\begin{equation*}
		LOO(\beta, \mu) = \frac1N\suml_{i=1}^N\cbr{y_i - \hat{\mb a}^T(\Omega \backslash \fbr{\omega_i})\mb x_i}^2
	\end{equation*}
\end{frame}

\section*{}
\subsection{Заключение}
\begin{frame}{Заключение, результаты и выводы}
\end{frame}

\end{document}
