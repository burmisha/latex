Функция Лагранжа:
\begin{equation}
	\label{LagrWEN}
	\begin{split}
		L(\mb a,\idots \delta1N, \idots \lambda1N) 
	&= \suml_{i=1}^n\cbr{\beta a_i^2 + \mu \modul{a_i}} + \suml_{j=1}^N p_j\delta_j^2 - \sum_{j=1}^N\lambda_j\cbr{\delta_j - y_j + \suml_{i=1}^nx_{ij}a_i} = \\
	&= \suml_{i=1}^n\cbr{\beta a_i^2 + \mu \modul{a_i} - \suml_{j=1}^N\lambda_jx_{ij}a_i} + \suml_{j=1}^N \cbr{p_j\delta_j^2  - \lambda_j\cbr{\delta_j - y_j}} \to \\
	&\to 
	\begin{cases}
		\min_{\mb a,\idots \delta1N}\\
		\max_{\idots \lambda1N}\\
	\end{cases}
	\end{split}
\end{equation}

Положим 
\begin{equation}
	\label{partialLagr}
	L_i(a_i,\idots \lambda1N) = \beta a_i^2 + \mu \modul{a_i} - \cbr{\suml_{j=1}^N\lambda_jx_{ij}}a_i \to \max{a_i}.
\end{equation}

Каждая из функций $L_i$ является кусочно заданной комбинаций двух квадратных трехчленов:

\begin{equation}
	\label{partialLagrParted}
	L_i(a_i,\idots \lambda1N) =
\begin{cases}
	\beta a_i^2 + \mu a_i - \cbr{\suml_{j=1}^N\lambda_jx_{ij}}a_i, &a_i > 0,\\
	0, &a_i = 0,\\
	\beta a_i^2 - \mu a_i - \cbr{\suml_{j=1}^N\lambda_jx_{ij}}a_i, &a_i < 0.\\
\end{cases}
\end{equation}
