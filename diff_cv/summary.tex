В данной работе была рассмотрена задача восстановления регрессии по обучающей совокупности. 
Использовалось признаковое описание объектов и линейная параметрическая модель регрессии. 
Для регуляризации полученной задачи поиска оптимального набора параметров модели использовалась регуляризация Elastic Net.
При введении регуляризации в задаче появились дополнительные структурные параметры, которые в свою очередь также требуют подбора, приводя к новой задаче оптимизации.

В работе предложена схема дифференциальной кросс-валидации, позволяющая вычислительно эффективно решать возникающую задачу.
Эта схема заключается в присвоении объектам весов и изучении влияния варьирования этих весов в окрестности единиц, когда все объекты равноправны, на значение оптимизируемого критерия.

Проведено теоретическое и экспериментальное сравнение дифференциальной кросс-валидации с известными методами скользящего контроля, показана их эквивалентность.