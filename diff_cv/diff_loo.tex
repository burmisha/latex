\begin{align}
	\hat S(\lambda_1, \lambda_2) &= \frac1N\sum_{j=1}^N \hat\delta_{j, \lambda_1, \lambda_2}^2, \\
	\hat \delta_{j, \lambda_1, \lambda_2} &= y_j - \mb x_j^T \hat {\mb a}_{\lambda_1, \lambda_2}.
\end{align}
В используемых обозначениях кросс-валидацию можно рассматривать как обучение с поочередным присвоением объектам нулевых весов ($p_j=0$).
Усреденные квадратичные остатки 


\begin{align}
	\hat{S}_{\text{LOO}}&=\frac1N\sum_{j=1}^N\sqr{\hat{\delta}_{j, \lambda_1, \lambda_2}^{(j)}}, \\
	\hat \delta_{j, \lambda_1, \lambda_2}^{(j)} &= y_j - \mb x_j^T \hat {\mb a}_{\lambda_1, \lambda_2}^{(j)}.
\end{align}

\begin{align}
	% \dd{}{\mb p}\hat{S}_{\lambda_1, \lambda_2}&= \frac2N\cbr{y_j - x_j^T \dd{}{\mb p}\hat {\mb a}_{\lambda_1, \lambda_2}}.
	\dd{}{p_k}\hat{S}_{\lambda_1, \lambda_2}&= \frac2N\sum_{j=1}^N\cbr{y_j - x_j^T \dd{}{p_k}\hat {\mb a}_{\lambda_1, \lambda_2}}
\end{align}
