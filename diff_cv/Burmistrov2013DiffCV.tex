\documentclass[12pt]{article}
\usepackage[utf8]{inputenc}
\usepackage[T2A]{fontenc}
\usepackage{graphics,graphicx,epsfig}
\usepackage{amssymb,amsfonts,amsthm,amsmath,mathtext,cite,enumerate,float}
\usepackage[english,russian]{babel}
\usepackage{color}
\usepackage{algorithm}
\usepackage[noend]{algorithmic}
\usepackage[vflt]{floatflt}
\definecolor{linkcolor}{RGB}{7,31,63}%{15,80,120}
\usepackage[colorlinks,unicode,pdfpagelabels=false, linkcolor = linkcolor]{hyperref}

\textheight=24cm		\textwidth=18cm
\oddsidemargin=-10mm 		\evensidemargin=-10mm
\topmargin=-2,5cm
\parindent=24pt 		\parskip=3pt 
\footnotesep=3ex
\raggedbottom %\flushbottom
\clubpenalty=10000		\widowpenalty=10000 	\tolerance=500
\renewcommand{\baselinestretch}{1}%{1.4}

\input macro.tex

\begin{document}
	% {
	\renewcommand{\baselinestretch}{1}
	\thispagestyle{empty}
	\begin{center}
		\sc
			Министерство образования и науки Российской Федерации\\
			Московский физико-технический институт {\rm(государственный университет)}\\
			Факультет управления и прикладной математики\\
			Вычислительный центр им. А. А. Дородницына РАН\\
			Кафедра <<Интеллектуальные системы>>\\[40mm]
		\rm\large
			Бурмистров Михаил Олегович\\[5mm] 
		\bf\Large
			Дифференциальный скользящий контроль \\
			при выборе параметров регуляризации \\
			в задаче регрессионного анализа \\
			с помощью Elastic Net \\ [5mm]
			% Методы оптимизации параметров\\
			% вероятностных тематических моделей
		\rm\normalsize
			{511656 --- Математические и информационные технологии}\\[5mm]
		\sc
		Магистерская диссертация
		%Выпускная квалификационная работа бакалавра
		\vspace{40mm}
	\end{center}
	\hfill \parbox{80mm} { 
		\begin{flushleft}
			\bf{Научный руководитель:}\\
			\rm д.т.н., профессор \\
			в.н.с. ВЦ РАН \\
			Моттль Вадим Вячеславович
			% \rm д.ф.--м.н. Воронцов Константин Вячеславович\\
		\end{flushleft}
	}
	\\ \vspace{2cm}
	\begin{center}
		Москва\\
		2014
	\end{center}
}
	\newpage \tableofcontents
	% \newpage \begin{abstract}
Решается задача одноклассовой классификации электронных писем на предмет наличия в них спама. В работе вводится квазивероятностная модель для классической эмпирической постановки задачи одноклассовой классификации. Произведена модификация, позволяющая накладывать различные требования отбора признаков. Построенные методы классификации проверяются вычислительными экспериментами на модельных и реальных данных.
%Решается задача автоматического разделения текстов по тематикам. Рассмотрены два подхода к решению задачи: вероятностный латентный семантический анализ и алгоритм латентного размещения Дирихле, основанные на различных вероятностных предположениях о текстах, однако обладающие схожей вычислительной техникой. Произведено сведение алгоритмов и их модификаций к новой обобщающей вычислительной схеме и построен новый алгоритм. Проанализировано качество и скорость сходимости построенного алгоритма в зависимости от внутренних параметров и числа тем, ассоциируемых с каждым словом в тексте. Результаты подтверждены численным экспериментом на реальных текстах. 
\end{abstract}	2
	% \newpage \section{Введение} С широким развитием сети интернет и её проникновением в большую часть всех сфер жизни, у людей появилась возможность свободно обмениваться информацией и получать доступ к разнообразным ресурсом. 
Одним из наиболее распространенных способов общения людей через интернет является использование электронной почты \cite{}. 
В силу большой открытости этого канала связи с точки зрения возможности передачи любого сообщения произвольному пользователю он активно используется мошенниками, злоумышленниками и распространителями рекламных материалов. При этом создается не только повышенная нагрузка на техническую инфраструктуру, но и тратится время людей, которым приходится отделять полезную информацию, от всей остальной \cite{}. 
Поэтому задача автоматизации фильтрации электронной почты будет оставаться актуальной в течение всего времени её существования.

Задача фильтрации спама уже решалась многими методами \cite{}, однако они в большой степени являлись эвристическими и не имели под собой четкой вероятностной модели. 
Также проблемой является корректное составление обучающей выборки. 
Дело в том, что спам-письма зачастую шаблонны и имеют много общего в своей структуре, к тому же они широко доступны. 
Составить же обучающую выборку, содержащую письма, полезные для пользователей, гораздо сложнее по следующим причинам:
\begin{itemize}
	\item меньшая доступность,
	\item высокая разнородность,
	\item большое число шаблонных писем (разнообразные уведомления от сервисов).
\end{itemize}
По этим причинам предлагается использовать методы одноклассовой классификации \cite{}, чтобы отказаться от требования к обучающей выборки содержать достаточно широкой множество разнообразных представителей обоих классов.

В работе будет предложена квазивероятностная постановка задачи одноклассовой классификации. 
За счет такого подхода становятся яснее области применимости построенной модели и предъявляемые требования к данным.

Поскольку количество признаков, которые можно извлечь из текстов спам-писем, очень велико, то предлагается применить отбор признаков. 
На основе полученной вероятностной постановки задачи, строится новая вероятностная модель порождения объектов, в ходе оптимизации которой происходит требуемый отбор признаков.

Полученные методы построения одноклассовых классификаторов применяются к модельным и реальным данным.

%Современный интернет обладает широкой, распределенной и сложной архитектурой, в которой зачатую затруднительно найти новую требуемую информацию. Помочь пользователю решить задачу поиска призваны поисковые системы, которые автоматически сканируют интернет и выявляют наиболее подходящие пользователю по некоторым ключевым словам. 

%Алгоритм определения степени соответствия сайта запросу пользователя основан на множестве характеристик, а владельцы ресурсов заинтересованы, чтобы поисковые системы как можно выше оценивали их сайт. Зачастую

%Злоумышленники пытаются вывести подконтрольные им сайты в топ поисковой выдачи, искусственно изменяя характеристики сайта, видимые для поисковой машины. Такие действия ухудшают качество поиска и опасны для пользователя, поэтому необходимо либо полностью блокировать такие сайты, либо существенно опускать их в выдаче.

%Задача могла бы рассматривать как традиционная задача бинарной классификации, если бы знание поисковой машины характеристик сайта не влияло на эти характеристики. Например, если сайт признан по какой-либо причине опасным и удален из выдачи, у него резко (в разы) снижается посещаемость, при этом сайт мог исправить свою проблему, но не разбанится автоматически и посещаемость останется низкой. Одновременно с этим существуют признаки (количество переходов), которые, по-видимому, отражают степень полезность сайта, однако слабо зависят от действий поисковой системы. Задача выделения признаков, не зависящих от поисковой машины также представляет интерес в нашем исследовании.

%Идея: пусть каждый сайтовладелец в каждый момент времени $t\in T$ (см. презентацию) думает: сделать ему сайт хуже или лучше (спамерское поведение или, напротив, добропорядочное --- это и есть на самом деле класс $y\in Y$). В зависимости от этого он меняет характеристики своего сайта (тут чуть веселее, на самом деле: он одинаково меняет характеристики {\it всех} сайтов, которыми владеет, а это можно узнавать по whois-данным. {\it но я думаю, что в работе с этим заморачиваться не будем}). В таком случае наблюдаемые характеристики сайта (те, которые зависят) есть, на самом деле, функция от его наблюдаемого класса (то, что люди видят глазами). Надо додумать. Статей по теме не находил (да и не искал: придумал поздно вечером).

	\section{Исходная постановка задачи Naive Elastic Net}
		\subsection{Вероятностные предположения}	\subsection{Постановка задачи}
Рассматривается задача восстановления числовой регрессии по обучающей выборке объектов, заданных своим признаковым описанием. 
Пусть $\Omega$~--- некоторое множество объектов реального мира, каждому элементу $\omega$ которого сопоставлено число $y(\omega)\in \mathbb Y \subset \mathbb R,$ 
а каждый объект $\omega \in \Omega$ представлен конечным множеством своих числовых признаков: $\mb x(\omega) = \cbr{x_1, \ldots, x_n}^T \in \mathbb X \subset \mathbb R^n.$
То есть предположим существование функций $y\colon \Omega \to \mathbb R$ и $\mb x \colon \Omega \to \mathbb R^n.$

Пусть наблюдателю известны значения этих функций в пределах некоторой конечной обучающей совокупности $\Omega^* = \fbr{\omega_j \cond j = 1,\ldots, N}$: $D = \fbr{\mb x(\omega), y(\omega) \cond \omega\in\Omega^*}.$
Ставится задача построить функцию $\hat y: \mathbb X \to \mathbb Y$ по известному $D$ такую, 
что функция $\mb x \circ \hat y \colon \Omega \to \mathbb Y$ будет как можно точнее описывать функцию $y.$

\subsection{Задача линейной регрессии}
Такая постановка задачи является весьма общей, и для получения конкретного решения требуется введение дополнительных ограничений. 
Будем искать решение в классе линейных функций, параметризованных следующим образом:
\begin{equation*}
	\hat y(\mb x) = \mb a^T \mb x + b, 
	\:\text{где $\mb a = (a_1, \ldots, a_n)^T \in \mathbb R^n$ и $b\in \mathbb R.$}
\end{equation*}

Введём обозначения:
\begin{align*}
	\mb x(\omega_j)	&= \mb x_j, j=1, \ldots, N  \text{ --- векторы признаков объектов,}\\
	\mb X 			&= \norm{x_{ij}}_{i=1, j=1}^{n,N} = \cbr{\mb x_1, \ldots, \mb x_N}^T \text{ --- матрица объекты-признаки,} \\
	y(\omega_j) 	&= y_j, j=1, \ldots, N, \\
	\hat y(x_j) 	&= \hat y_j, j=1, \ldots, N, \\
\end{align*}

В качестве меры близости функций $\mb x \circ \hat y$ и $y$ предлагается использовать средний квадрат их разницы на обучающей совокупности:
\begin{equation*}
	\frac1N\suml_{j=1}^N \cbr{y_j - \hat y_j}^2.
\end{equation*}

Тогда мы имеем следующую задачу оптимизации 
\begin{equation}
	\label{bestAB}
	\cbr{\hat{\mb a}, \hat b} = \argmin \suml_{j=1}^N\cbr{y_j - \mb a^T \mb x_j - b}^2, 
\end{equation}
однако такая постановка задачи может вызвать трудности при непосредственном поиске оптимальных параметров $\cbr{\hat{\mb a}, \hat b}$ в случае мультиколлинеарности используемых признаков, 
выражающееся в численной неустойчивости ответа и невозможности найти единственный оптимум, т.е. задача оказывается нерегулярной.

\subsection{Регуляризация Elastic Net}
С целью регуляризации задачи (\ref{bestAB}) в [] предлагается ввести штраф, называемый регуляризацией Elastic Net:
\begin{equation}
	\label{ENregularization}
	\beta \norm{\mb a}_{\mathbb R^2}^2 + \mu \norm{\mb a}_{\mathbb R} 
	= \beta \suml_{i=1}^n a_i^2 + \mu \suml_{i=1}^n \abs{a_i}.
\end{equation}

Что приводит к задаче оптимизации: 
\begin{equation}
	\label{basicEN}
	\sum_{i=1}^n\cbr{\beta a_i^2 + \mu \modul{a_i}} 
	+ \sum_{j=1}^N\cbr{y_j - \mb a^T \mb x_j - b}^2 
	\to \min_{\mb a, b}
\end{equation}
Эта задача не является квадратичной при $\mu > 0,$ и её решение мы обсудим позднее.

Отметим, что этот подход обобщает в себе регуляризации, используемые в методе лассо (LASSO, []) и методе опорных вектором (SVM) в задачах регрессии ([]).

Регуляризация (\ref{ENregularization}) в равной степени штрафует все компоненты вектора $\mb a\in \mathbb R^n,$ 
поэтому необходимо убедиться в том, что все признаки нормированы. 
Этого можно добиться линейным преобразованием признаков. 
Одновременно с этим предлагается центрировать признаки, чтобы избавиться от параметра $b$.

Тогда по обучающей совокупности $D$ линейным преобразованием построим совокупность
\begin{equation*}
	D^* = \fbr{\mb x^*_i, y^*_i \cond i = 1, \ldots, N}, 
\end{equation*}
удовлетворяющую следующим условиям:
\begin{align}
	\label{normalization-start}
	\mb 0 	&= \frac1N\suml_{i=1}^N\mb x^*_i, \\
	0 		&= \frac1N\suml_{i=1}^N y^*_i, \\
	\label{normalization-end}
	1 		&= \frac1N\suml_{i=1}^N\cbr{\mb x^*_{ij}}^2, j=1,\ldots, n.
\end{align}

Подходящее линейное преобразование задаётся соотношениями:
\begin{align*}
	x^*_{ij} 	&= \frac{x_{ij} - \bar x_j}{d_j}, \oi iN, \oi jn;\\
	\bar x_j 	&= \frac1N\suml_{i=1}^N x_{ij}, \oi jn;\\
	d_j 		&= \sqrt{\frac1N\suml_{i=1}^N \cbr{x_{ij}-\bar x_j}^2}, \oi jn;\\
	y^*_i 		&= y_i - \bar y, \oi iN;\\
	\bar y 		&= \frac1N\suml_{i=1}^N y_i. \\
\end{align*}

В дальнейшем для упрощения изложения мы будем везде считать, что это преобразование уже произведено, и условия (\ref{normalization-start}--\ref{normalization-end}) выполнены для совокупности $D.$

В таком случае задача (\ref{basicEN}) принимает вид
\begin{equation}
	\label{mainEN}
	\sum_{i=1}^n\cbr{\beta a_i^2 + \mu \modul{a_i}} 
	+ \sum_{j=1}^N\cbr{y_j - \mb a^T \mb x_j}^2 
	\to \min_{\mb a}.
\end{equation}

\subsection{Скользящий контроль для поиска оптимальных параметров регуляризации} % (fold)
\label{sub:intro:LOO}
Для подбора оптимальных значений параметров $\beta, \mu$ предлагается воспользоваться критерием скользящего контроля leave-one-out (LOO, []).
Для этого построим совокупность множеств обучения $D^{(k)}, \oi kN$ из $D$ путем поочерёдного исключения из него каждого из объектов: 
\begin{equation}
	D^{(k)}=\fbr{(\mb x_i, y_i) \cond \oi iN, i\ne k}.
\end{equation}
Для каждого полученного обучающего множества решим задачу (\ref{mainEN}) и получим оптимальный вектор 
\begin{equation*}
	\hat{\mb a}^{(k)} 
	= \argmin \suml_{i=1}^n\cbr{\beta a_i^2 + \mu \modul{a_i}} 
	+ \suml_{\oi jN, j\ne k}\cbr{y_j - \mb a^T \mb x_j}^2.
\end{equation*}

Критерием качества будет являться среднее ошибок алгоритмов на объектах исключённых из исходного множества обучения:
\begin{equation*}
	% \label{LOO}
	S_{LOO}(\beta, \mu) = \frac1N\suml_{k=1}^N \cbr{y_k - \mb x_k^T\hat{\mb a}^{(k)}}^2.
\end{equation*}

Таким образом критерий традиционной кросс-валидации принимает вид:
\begin{equation}
	\label{LOOcriteria}
	\cbr{\hat \beta, \hat \mu} = \argmin \frac1N\suml_{k=1}^N \cbr{y_k - \mb x_k^T\hat{\mb a}^{(k)}}^2.
\end{equation}

В такой форме явно видно, что получение оценки $S_{LOO}(\beta, \mu)$ требует построение всех векторов $\hat{\mb a}^{(k)}, \oi kN,$ то есть придётся обучаться $N$ раз. 
% subsection subsection_name (end)

% \subsection{Скользящий контроль для поиска оптимальных параметров регуляризации}

\subsection{Взвешивание объектов обучающей совокупности}
\label{sub:intro:weight}
В работе [] была предложена идея дифференциального скользящего контроля. 
Она заключается в присвоении каждому объекту $\omega_j, \oi jN$ из обучающей совокупности некоторого числа $p_j, \oi jN,$ которое означает вес объекта при обучении. 
Задача оптимизации (\ref{mainEN}) при этом принимает вид: 
\begin{equation}
	\label{weighedEN}
	\sum_{i=1}^n\cbr{\beta a_i^2 + \mu \modul{a_i}} 
	+ \sum_{j=1}^Np_j\cbr{y_j - \mb a^T \mb x_j}^2 
	\to \min_{\mb a, b}.
\end{equation}

Теперь оптимальный вектор $\hat{\mb a}$ зависит не только от параметров $\mu, \beta$, но и от вектора весов $\mb p = (p_1, \ldots, p_N).$
Тогда критерий (\ref{LOOcriteria}) можно переписать в виде:
\begin{align*}
	\cbr{\hat \beta, \hat \mu} &= \argmin \frac1N\suml_{k=1}^N \cbr{y_k - \mb x_k^T\hat{\mb a}(\beta, \mu, \mb e^{(i)})}^2,\\
	\text{ где } \mb e^{(i)} &= (e^{(i)}_1, \ldots, e^{(i)}_N)^T, e^{(i)}_k=\begcas{1, i\ne k, \\ 0, i = k.}
\end{align*}

Задача в форме (\ref{weighedEN}) обобщает ранее рассмотренные (\ref{basicEN}, \ref{mainEN}), поэтому далее мы будем решать именно её.

		\subsection{Постановка задачи оптимизации} 	\begin{equation}
	\label{basicENdelta}
	\begin{cases}
		\suml_{i=1}^n\cbr{\beta a_i^2 + \mu \modul{a_i}} 
	+ \suml_{j=1}^N \delta_j^2 \to \min_{\mb a}, \\
	\delta_j = y_j - \suml_{i=1}^nx_{ij}a_i, j=1,\ldots,N. 
	\end{cases}
\end{equation}
	\section{Использование взвешенных объектов в обучающей совокупности}
		\subsection{Постановка задачи оптимизации} 	Обобщим задачу \ref{basicENdelta}, поставив в соответствие каждому объекту $x_j$ некоторый <<вес>> $p_j\in \mathbb R$.

\begin{equation}
	\label{weighedENdelta}
	\begin{cases}
		\suml_{i=1}^n\cbr{\beta a_i^2 + \mu \modul{a_u}} 
	+ \suml_{j=1}^N p_j\delta_j^2 \to \min_{\mb a}, \\
	\delta_j = y_j - \suml_{i=1}^nx_{ij}a_i, j=1,\ldots,N. 
	\end{cases}
\end{equation}
		\subsection{Двойственная задача}			Рассмотрим задачу (\ref{weighedEN}) и перепишем её в виде
\begin{equation}
	\label{weighedENdelta}
	\begin{cases}
		\suml_{i=1}^n\cbr{\beta a_i^2 + \mu \modul{a_i}} 
	+ \suml_{j=1}^N p_j\delta_j^2 \to \min_{\mb a}, \\
	\delta_j = y_j - \suml_{i=1}^nx_{ij}a_i, j=1,\ldots,N,
	\end{cases}
\end{equation}
где $\delta = \cbr{\delta_1, \ldots, \delta_N}^T$ --- вектор регрессионных остатков.

Функция Лагранжа:
\begin{equation}
	\label{LagrWEN}
	\begin{split}
		L(\mb a,\idots \delta1N, \idots \lambda1N) 
	&= \suml_{i=1}^n\cbr{\beta a_i^2 + \mu \modul{a_i}} + \suml_{j=1}^N p_j\delta_j^2 - \sum_{j=1}^N\lambda_j\cbr{\delta_j - y_j + \suml_{i=1}^nx_{ij}a_i} = \\
	&= \suml_{i=1}^n\cbr{\beta a_i^2 + \mu \modul{a_i} - \suml_{j=1}^N\lambda_jx_{ij}a_i} + \suml_{j=1}^N \cbr{p_j\delta_j^2  - \lambda_j\cbr{\delta_j - y_j}} \to \\
	&\to 
	\begin{cases}
		\min_{\mb a,\idots \delta1N}\\
		\max_{\idots \lambda1N}\\
	\end{cases}
	\end{split}
\end{equation}



\def\LiPart{L_i(a_i,\idots \lambda1N)}
\def\sumLambdaX{\suml_{j=1}^N\lambda_jx_{ij}}
\newcommand\diag[1]{\text{diag}\cbr{#1}}
Положим 
\begin{equation}
	\label{partialLagr}
	\LiPart = \beta a_i^2 + \mu \modul{a_i} - \cbr{\sumLambdaX}a_i \to \min_{a_i}.
\end{equation}

Каждая из функций $L_i$ является кусочно заданной комбинаций двух квадратных трехчленов:
\begin{equation}
	\label{partialLagrParted}
	\begin{split}
		\LiPart 
		&= \begin{cases}
			\beta a_i^2 + \mu a_i - \cbr{\sumLambdaX}a_i, &a_i > 0,\\
			0, &a_i = 0,\\
			\beta a_i^2 - \mu a_i - \cbr{\sumLambdaX}a_i, &a_i < 0,\\
		\end{cases} \quad= \\
		&= \begin{cases}
			\beta a_i^2 - \cbr{\sumLambdaX-\mu}a_i, &a_i > 0,\\
			0, &a_i = 0,\\
			\beta a_i^2 - \cbr{\sumLambdaX+\mu}a_i, &a_i < 0.\\
		\end{cases}	
	\end{split}
\end{equation}

Исходная постановка задачи Elastic Net \ref{basicEN} предполагает, что $\beta\ge0, \mu\ge0.$ Изучим случай $\beta > 0:$ 
\begin{equation}
	\label{hatAi}
	\begin{split}
		\hat a_i &= \argmin_{a_i} \LiPart = \\
				 &=	\argmin_{a_i} \fbr{
				 	\begin{aligned}
						\beta a_i^2 - \cbr{\sumLambdaX-\mu}a_i, &a_i > 0,\\
						0, \qquad\qquad &a_i = 0,\\
						\beta a_i^2 - \cbr{\sumLambdaX+\mu}a_i, &a_i < 0.\\
					\end{aligned}	
					}
	\end{split}
\end{equation}

Для поиска точки минимума рассмотрим 3 случая:
\begin{align}
	\label{hatAicases}
	         &\sumLambdaX \le -\mu 	&\Rightarrow \hat a_i &= \frac{\sumLambdaX+\mu}{2\beta}, \\
	-\mu <   &\sumLambdaX < \mu 	&\Rightarrow \hat a_i &= 0, \\
	\mu \le &\sumLambdaX 			&\Rightarrow \hat a_i &= \frac{\sumLambdaX-\mu}{2\beta},
\end{align}

% a x^2 + b x + c = a(x-b/{2a})^2 + c - b^2/{4a} = a(x-x_0)^2 + c - a x_0^2

Подставляя значения из \ref{hatAicases} в \ref{partialLagrParted}, получим решение задачи минимизации по $a_i$:
\begin{align}
	\label{hatLiSquared}
	\begin{split}
		\hat L_i(\idots \lambda1N) &= \min_{a_i}L_i(a_i, \idots \lambda1N) =
		\begin{cases}
			-\cfrac1{4\beta}\cbr{\sumLambdaX+\mu}^2, &\sumLambdaX \le -\mu,\\
			0, 										&-\mu < \sumLambdaX < \mu,\\
			-\cfrac1{4\beta}\cbr{\sumLambdaX-\mu}^2, &\mu\le\sumLambdaX,
		\end{cases} = \\
		&= -\frac1{4\beta}
		\begin{cases}
			\cbr{-\sumLambdaX-\mu}^2, &\sumLambdaX \le -\mu,\\
			0, 	 					&-\mu < \sumLambdaX < \mu,\\
			\cbr{\sumLambdaX-\mu}^2, &\mu\le\sumLambdaX.
		\end{cases} 	\\
		&= -\frac1{4\beta} \cbr{\min\fbr{\mu + \sumLambdaX, 0, \mu-\sumLambdaX}}^2
	\end{split}
\end{align}
(Здесь не сходится с результатами doc-файла)

Теперь перейдём к минимизации по $\cbr{\delta_1, \ldots, \delta_N}:$
\begin{equation}
	0 = \dd{}{\delta_j}L(\mb a,\idots \delta1N, \idots \lambda1N) = 2p_j\delta_j - \lambda_j 
\end{equation}
\begin{equation}
	\label{deltaJ}
	\delta_j = \frac{\lambda_j}{2p_j}
\end{equation}

Подставим результаты (\ref{hatLiSquared}) в (\ref{LagrWEN}):
\begin{align} 
	\hat L&\cbr{\mb a,\idots \delta1N, \idots \lambda1N} 
		= \suml_{i=1}^n \hat L_i(a_i,\idots \lambda1N)
		+ \suml_{j=1}^N \cbr{p_j\delta_j^2  - \lambda_j\cbr{\delta_j - y_j}} = \\
		&= -\frac1{4\beta} \suml_{i=1}^n \cbr{\min\fbr{\mu + \sumLambdaX, 0, \mu-\sumLambdaX}}^2
		+ \suml_{j=1}^N \cbr{p_j\cbr{\frac{\lambda_j}{2p_j}}^2  - \lambda_j\cbr{\frac{\lambda_j}{2p_j} - y_j}} = \\
		&= -\frac1{4\beta} \suml_{i=1}^n \cbr{\min\fbr{\mu + \sumLambdaX, 0, \mu - \sumLambdaX}}^2
		- \suml_{j=1}^N \cbr{ \frac{\lambda_j^2}{4p_j}  - \lambda_j y_j} 
\end{align}

Таким образом, задача (\ref{LagrWEN}) свелась к 
\begin{align}
-\frac1{4\beta} \suml_{i=1}^n \cbr{\min\fbr{\mu + \sumLambdaX, 0, \mu - \sumLambdaX}}^2
		- \suml_{j=1}^N \cbr{ \frac{\lambda_j^2}{4p_j}  - \lambda_j y_j} \to \max_{\idots \lambda1N}
\end{align}
Используя соотношение (\ref{deltaJ}), получим эквивалентную задачу:
\begin{align} 
	% \hat L&\cbr{\mb a,\idots \delta1N, \idots \lambda1N} 
	% 	= \suml_{i=1}^n \hat L_i(a_i,\idots \lambda1N)
	% 	+ \suml_{j=1}^N % \cbr{p_j\cbr{\frac{\lambda_j}{2p_j}}^2  - \lambda_j\cbr{\frac{\lambda_j}{2p_j} - y_j}} = \\
	% 		\cbr{p_j\delta_j^2  - \lambda_j\cbr{\delta_j - y_j}} = \\
	% 	&= -\frac1{4\beta} \suml_{i=1}^n \cbr{\min\fbr{\mu + \sumLambdaX, 0, \mu-\sumLambdaX}}^2
	% 	- \suml_{j=1}^N \cbr{p_j\delta_j^2  -  2 \delta_j p_j (\delta_j - y_j)} = \\
	% 	&= 
	-\frac1{4\beta} \suml_{i=1}^n \cbr{\min\fbr{\mu + \suml_{j=1}^N2 \delta_j p_jx_{ij}, 0, \mu-\suml_{j=1}^N2 \delta_j p_jx_{ij}}}^2
		- \suml_{j=1}^N \cbr{ \delta_j^2 p_j  - 2\delta_j p_j y_j} \to \max_{\idots \delta1N}
\end{align}
Отметим, что задача осталась задачей максимизации.

Положим 
\begin{align}
	W(\idots \delta1N) 
	= \frac1{4\beta} \suml_{i=1}^n \cbr{\min\fbr{\mu + \suml_{j=1}^N2 \delta_j p_jx_{ij}, 0, \mu-\suml_{j=1}^N2 \delta_j p_jx_{ij}}}^2
		+ \suml_{j=1}^N \cbr{ \delta_j^2 p_j  - 2\delta_j p_j y_j}
\end{align}

Итак, достаточно решить задачу 
\begin{align}
	W(\idots \delta1N) \to \min_{\idots \delta1N},
\end{align}
или же, используя матричные обозначения $\mb \delta = \cbr{\idots \delta1N}^T \in \mathbb R^N, \mb x_i = \cbr{x_{i1}, \ldots, , x_{iN}} \in \mathbb R^N, \mb y = \cbr{\idots y1N}^T \in \mathbb R^N, P = P^T= \diag{\sqrt{p_1}, \ldots, \sqrt{p_N}},$

\begin{align}
	W(\mb \delta) = \frac1{2\beta} \suml_{i=1}^n \cbr{\min\fbr{\frac\mu2 + \mb \delta^T P^2\mb x_i, 0, \frac\mu2- \mb \delta^T P^2\mb x_i}}^2
		+ (\mb \delta - \mb y)^T P (\mb \delta - \mb y) \to \min_{\mb \delta}.
\end{align}

Пусть $\hat{\mb \delta}$~--- решение этой задачи. Оно единственно в силу выпуклости функции $W(\mb \delta)$ по $\mb \delta$. Тогда оптимальное $\hat{\mb a}$ можно определить из соотношений  (\ref{deltaJ}) и (\ref{hatAicases}).
	\section{Метод дифференциальной кросс-валидации}
		\begin{align}
	\hat S(\lambda_1, \lambda_2) &= \frac1N\sum_{j=1}^N \hat\delta_{j, \lambda_1, \lambda_2}^2, \\
	\hat \delta_{j, \lambda_1, \lambda_2} &= y_j - \mb x_j^T \hat {\mb a}_{\lambda_1, \lambda_2}.
\end{align}
В используемых обозначениях кросс-валидацию можно рассматривать как обучение с поочередным присвоением объектам нулевых весов ($p_j=0$).
Усреденные квадратичные остатки 


\begin{align}
	\hat{S}_{\text{LOO}}&=\frac1N\sum_{j=1}^N\sqr{\hat{\delta}_{j, \lambda_1, \lambda_2}^{(j)}}, \\
	\hat \delta_{j, \lambda_1, \lambda_2}^{(j)} &= y_j - \mb x_j^T \hat {\mb a}_{\lambda_1, \lambda_2}^{(j)}.
\end{align}

\begin{align}
	% \dd{}{\mb p}\hat{S}_{\lambda_1, \lambda_2}&= \frac2N\cbr{y_j - x_j^T \dd{}{\mb p}\hat {\mb a}_{\lambda_1, \lambda_2}}.
	\dd{}{p_k}\hat{S}_{\lambda_1, \lambda_2}&= \frac2N\sum_{j=1}^N\cbr{y_j - x_j^T \dd{}{p_k}\hat {\mb a}_{\lambda_1, \lambda_2}}
\end{align}

		
У Жени всё в терминах $c$
$$
\hat y(x) = \sum_{j=1}^N c_j x_{ij}
$$

\begin{equation}
	\begin{cases}
		\suml_{i=1}^n c_i^2 + \suml_{j=1}^N \delta_j^2 \to \min_{\mb c, \mb \delta}, \\
	\delta_j = y_j - \suml_{i=1}^n c_ix_{ij}, j=1,\ldots,N. 
	\end{cases}
\end{equation}

\begin{equation}
	\begin{cases}
		\suml_{i=1}^n c_i^2 + \suml_{j=1}^N p_j\delta_j^2 \to \min_{\mb c, \mb \delta}, \\
	\delta_j = y_j - \suml_{i=1}^n c_ix_{ij}, j=1,\ldots,N. 
	\end{cases}
\end{equation}



\begin{equation}
L(\mb c,\idots \delta1N, \idots \lambda1N) 
	= \frac12 \mb c^T\mb c +  \frac12 \suml_{j=1}^N p_j\delta_j^2 + \sum_{j=1}^N\lambda_j\cbr{y_j  - \suml_{i=1}^n c_i x_{ij} - \delta_j }
\end{equation}

\begin{equation}
L(\beta) = - \left.\sum_{i=1}^N \dd{\delta_j^2(\mb p\cond \beta)}{p_j}\right|_{\mb p = (1, \ldots, 1)} \to \min_{\beta}
\end{equation}

\begin{equation}
\dd{\delta_j^2(\mb p)}{p_j} = 2 \delta_j(\mb p\cond \beta)\dd{\delta_j(\mb p)}{p_j} 
\end{equation}


		
Задача: найти выражание, связывающее $\hat{S}_{\text{LOO}}$ и $\hat S(\lambda_1, \lambda_2) - \mb e^T\dd{\hat{S}_{\lambda_1, \lambda_2}}{\mb p},$ где $\mb e = (1, \ldots, 1) \in \mathbb R^N$, при $\mb p = (1, \ldots, 1) \in \mathbb R^N$


Добавить:
- про природу, и почему мы используем эмпирическю функцию ошибки
- почему нет b: мы центрируем и нормируем
- написать то же самое, без b
- про индексы и их стабильность

\end{document} 