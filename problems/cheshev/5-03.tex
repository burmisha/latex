\task{%
    Катод вакуумного фотоэлемента облучается световым пучком с длиной волны $\lambda = 0{,}5\units{мкм}$ и мощностью $W=1 \units{Вт}$.
    При больших ускоряющих напряжениях между катодом и анодом фототок достигает насыщения
    (все электроны, выбитые из катода в единицу времени, достигают анода) $\mathcal{I}_{\text{н}} = 4\units{мА}$.
    Какое количество $n$ фотонов приходится на один электрон, выбиваемый из катода?
    Постоянная Планка $h = 6{,}63 \cdot 10^{-34} \units{Дж} \cdot \text{с}$, элементарный заряд $e = 1{,}6 \cdot 10^{-19}\units{Кл}$.
}
\answer{%
    $I = \frac q{\Delta t} = \frac{N_e e}{\Delta t}, E = W\Delta t = N_\nu\frac{hc}\lambda \implies n = \frac{N_\nu}{N_e} = \frac{W\lambda e}{Ihc}\approx 100{,}55$
}
% autogenerated
