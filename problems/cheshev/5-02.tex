\task{
    Находящийся в возбуждённом состоянии неподвижный атом водорода излучает фотон, а сам атом после этого начинает двигаться.
    Найти величину скорости атома после излучения фотона.
    Энергия возбуждения атома водорода $W = 1{,}63 \cdot 10^{-18}\units{Дж}$,
    энергия покоя атома водорода $E=mc^2 = 1{,}49\cdot 10^{-10}\units{Дж},$ где $m$~--- масса покоя атома, а $c$~--- скорость света.
    При расчёте движения атома можно использовать нерелятивистские формулы.
}
\answer{
    $v = \frac pm = \frac 1m \cdot mc\cbr{\sqrt{1 + \frac{2W}{mc^2}} - 1} \approx c \frac{W}{mc^2}\approx 3{,}28\funits{м}{c}$
}
% autogenerated
