\task{
    Неподвижный невозбуждённый атом водорода поглощает фотон.
    В результате атом переходит в возбуждённое состояние и начинает двигаться.
    Найти величину скорости, с которой стал двигаться атом после поглощения фотона.
    Энергия возбуждения атома $W = 1{,}63 \cdot 10^{-18}\units{Дж}$,
    энергия покоя атома водорода $E=mc^2 = 1{,}49\cdot 10^{-10}\units{Дж},$ где $m$~--- масса покоя атома, а $c$~--- скорость света.
}
\answer{
    $v = \frac pm = c\cbr{1 \pm \sqrt{1 - \frac{2W}{mc^2}}} \approx c \frac{W}{mc^2}\approx 3{,}28\funits{м}{c}$
}
% autogenerated
