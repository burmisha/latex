\task{%
    Монохроматический свет с длиной волны в вакууме 720 нм проходит через тонкую прозрачную плёнку с показателем преломления 1,8.
    Толщина плёнки 3,8 : 105 м.
    Световая волна падает на плёнку перпендикулярно её поверхности.
    Сколько раз длина волны света в плёнке укладывается в её толщине?
}
% autogenerated
