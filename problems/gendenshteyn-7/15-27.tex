\task{
    На тело действуют три силы $\vec F_1$, $\vec F_2$ и $\vec F_3$,
    направленные вдоль одной прямой, причём $F_1 = 3\units{H}$, $F_2 = 5\units{H}$.
    Чему равна $F_3$, если равнодействующая всех трёх сил равна  $10\units{Н}$.
    Сколько решений имеет эта задача?
    Сделайте в тетради схематические рисунки, соответствующие каждому из решений.
}
% autogenerated
